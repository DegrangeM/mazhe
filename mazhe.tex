% This is part of Giulietta
% Copyright (c) 2010-2020
%   Laurent Claessens
    % See the file fdl-1.3.txt for copying conditions.

% SCRIPT MARK -- DECLARATIVE PART
\documentclass[a4paper,twoside,11pt]{book}

\usepackage[utf8]{inputenc}
\usepackage[T1]{fontenc}

\usepackage{ifthen}
\usepackage{etoolbox}           % Ceci devrait remplacer ifthen.


% The following booleans serve to know in what type of compilation we are.
% They are set to 'true' by the pre-compilation scripts.
% http://ftp.oleane.net/pub/CTAN/macros/latex/contrib/etoolbox/etoolbox.pdf
% They are set quite early because some parts of the headings are changed
% by some plugins (ex: the colors in 'configuration.tex')
\newbool{isFrido}
\newbool{isGiulietta}
\newbool{isEnVolume}
\newbool{isBook}
\boolfalse{isFrido}
\boolfalse{isGiulietta}
\boolfalse{isEnVolume}      % This list is hard-coded in 'gardeVolume.tex'.
\boolfalse{isBook}

% The following line is changed by pytex. You can change the title
% but not the exact string ``\newcommand{\pdftitle}''.
\newcommand{\pdftitle}{Giulietta}

% This is part of Mes notes de mathématique
% Copyright (c) 2011-2017, 2019
%   Laurent Claessens
% See the file fdl-1.3.txt for copying conditions.

% TODO : mettre à jour pour suivre les recommandations
%http://openclassrooms.com/courses/guide-des-bonnes-pratiques-en-latex


\usepackage{etex}
\usepackage{latexsym}
\usepackage{amsfonts}
\usepackage{amsmath}
\usepackage{amsthm}
\usepackage{amssymb}
\usepackage{mathrsfs}
\usepackage{mathabx}           % For \divides et \widehat.
\usepackage{bbm}

\usepackage{enumitem}
\setlist[enumerate,1]{label=(\arabic*),ref=(\arabic*)}
\setlist[enumerate,2]{label=(\arabic{enumi}\alph{enumii}), ref=(\arabic{enumi}\alph{enumii})}


\usepackage{wrapfig}
\usepackage{framed}
\let\Sun\undefined
\let\Moon\undefined
\let\Venus\undefined
\let\Mars\undefined
\let\Jupiter\undefined
\let\Saturn\undefined
\let\Uranus\undefined
\let\Mercury\undefined
\let\Venus\undefined
\let\Mars\undefined
\let\Jupiter\undefined
\let\Saturn\undefined
\let\Uranus\undefined
\let\Neptune\undefined
\let\Pluto\undefined
\let\Earth\undefined
\let\Aries\undefined
\let\Taurus\undefined
\let\Gemini\undefined
\let\Leo\undefined
\let\Libra\undefined
\let\Scorpio\undefined
\usepackage{marvosym}       % marvosym redefines the previous symbols
\usepackage{tikz}           % Configuration at 1829426939
\usepackage{calc}
\usetikzlibrary{calc}
\usetikzlibrary{patterns}
\usepackage{color}
\usepackage{graphicx}                   % Pour l'inclusion d'image en pfd.


% Increase the size to the box containing the section number in the TOC. If not, ``10.20.12`` is too long
\usepackage[subfigure]{tocloft}
\renewcommand\cftsubsecnumwidth{1.5cm}
\renewcommand\cftsecnumwidth{1cm}



\usepackage{subfigure}

\usepackage{fancyvrb}
\usepackage{stmaryrd}       % Pour le \obslash
\usepackage{xstring}        % Utilisé pour les références vers wikipédia
\usepackage{cases}
\usepackage{lscape}         % pour l'environnement landscape, utilisé dans la correction corr0076.tex
\usepackage{multicol}
%\usepackage{xspace}
\usepackage[normalem]{ulem}		% Pour le barré, commande \sout
\usepackage[all]{xy}
\let\second\undefined      % le paquet mathabx définit \second
\let\degree\undefined       % le paquet mathabx définit \degree

\usepackage[cdot,thinqspace,amssymb]{SIunits}    %   1410612643

\usepackage{textcomp}
\usepackage{lmodern}
\usepackage[a4paper,margin=2cm,left=2.6cm]{geometry}

\usepackage{hyperref}
\usepackage{makeidx}
%\usepackage{minitoc}
\usepackage[nottoc]{tocbibind}      % Biblio inclue  dans la table des matières.
\usepackage[numbers]{natbib}        % le champ URL dans le fichier bibtex
\usepackage[refpage]{nomencl}       % Some configuration at   1338719836
\usepackage{array}

\usepackage{scalerel,stackengine}   % reallywidehat

% We do not use 'exocorr' for Frido.
%\ifbool{isFrido}{}{\usepackage[fr]{exocorr} }

\input{exocorr}


\usepackage[english,french]{babel}
\providecommand\frenchbsetup[1]{}
\providecommand\frenchsetup[1]{\frenchbsetup{#1}}
\frenchsetup{
  og=«,fg=»
}


% The macro ``pdftitle'' is changed by 'pytex' and so depends on 
% the context.
\makeatletter
%\@ifundefined{hypersetup}{}{%
  \hypersetup{%
    pdfauthor={Claessens, Laurent and al.},
    pdftitle={\pdftitle},
    pdfsubject={giulietta},
    pdfkeywords={agrégation, frido, giulietta},
    pdfmenubar=true,
    colorlinks=true,
    pdfpagemode=UseNone,
    pdfstartview=FitH,
  }%
%}
\makeatother

\usepackage{listingsutf8}   % Has to be called after babel


% Some  'configuration' for tikz       1829426939
\newcounter{defHatch}
\newcounter{defPattern}
\setcounter{defHatch}{0}
\setcounter{defPattern}{0}
\newcommand{\utilde}[1]{\underline{#1}}

\input{configuration}

%%%%%%%%%%%%%%%%%%%%%%%%%%
%
%   Trucs mathématiques
%
%%%%%%%%%%%%%%%%%%%%%%%%

% ENSEMBLES DE NOMBRES
\newcommand{\eA}{\mathbbm{A}}
\newcommand{\eB}{\mathbbm{B}}
\newcommand{\eC}{\mathbbm{C}}
\newcommand{\eD}{\mathbbm{D}}
\newcommand{\eE}{\mathbbm{E}}
\newcommand{\eF}{\mathbbm{F}}
\newcommand{\eG}{\mathbbm{G}}
\newcommand{\eH}{\mathbbm{H}}
\newcommand{\eK}{\mathbbm{K}}
\newcommand{\eL}{\mathbbm{L}}
\newcommand{\eM}{\mathbbm{M}}
\newcommand{\eN}{\mathbbm{N}}
\newcommand{\eP}{\mathbbm{P}}
\newcommand{\eQ}{\mathbbm{Q}}
\newcommand{\eR}{\mathbbm{R}}
\newcommand{\eS}{\mathbbm{S}}
\newcommand{\eT}{\mathbbm{S}}
\newcommand{\eZ}{\mathbbm{Z}}


% ENSEMBLES de fonctions
\newcommand{\aL}{\mathcal{L}}       % Les applications linéaires
\newcommand{\cL}{L}       % Les applications linéaires continues
\newcommand{\aC}{\mathcal{C}}       % Les fonctions C^1, C^2 etc
\newcommand{\swS}{\mathscr{S}}          % L'ensemble des fonctions Schwartz
\newcommand{\swD}{\mathscr{D}}          % L'ensemble des fonctions Cinfinie à support compact.
\newcommand{\swE}{\mathscr{E}}          % L'espace des fonctions qu'on peut déformer (le grand epsilon)
                                  % Les espaces de distributions correspondants sont les mêmes avec un prime.
\newcommand{\comC}{\mathcal{C}}       % Le commutant d'un endomorphisme.

\DeclareMathOperator{\Pol}{Pol}
\DeclareMathOperator{\Poly}{\mathcal{P}}        % Space of polynomials
\newcommand{\sdS}{\mathcal{S}}      % L'ensemble des subdivisions d'un intervalle.
\newcommand{\TF}{\mathcal{F}} %% Transformée de Fourier.
\newcommand{\mtu}{\mathbbm{1}}              % La matrice unité
\newcommand{\caract}{\mathbbm{1}}    % Characteristic function of a set
\newcommand{\catC}{\mathscr{C}}     % \catX is for the categories
\newcommand{\catD}{\mathscr{D}}
\newcommand{\catM}{\mathscr{M}}
\newcommand{\oB}{\mathfrak{B}}          % The space of bounded operators
\newcommand{\oK}{\mathcal{K}}           % L'espace des opérateurs compacts
\newcommand{\oL}{\mathscr{L}}           % Le L pour l'idéal de Schatten-von Neumann. Cela est aussi l'ensemble des opérateurs linéaires sur des espaces vectoriels.
\newcommand{\oP}{\mathscr{P}}           % Le P est pour l'ensemble des projections dans une algèbre de VN.

\newcommand{\euler}{\mbox{\rm e}}
\newcommand{\dist}{\operatorname {dist}}
\newcommand{\sii}{\mbox{\rm \scriptsize i}}

% LES NEWCOMMAND UN PEU ACTIFS

\newcommand*{\conclusion}{\emph{Conclusion~:~}}

% DECLARE MATH OPERATORS
\DeclareMathOperator{\fl}{fl}
\DeclareMathOperator{\SimplePrec}{sp}
\DeclareMathOperator{\NaN}{NaN}
\DeclareMathOperator{\supp}{supp}
\DeclareMathOperator{\Iso}{Iso}
\DeclareMathOperator{\Isom}{Isom}       % The group of isometries
\DeclareMathOperator{\Aut}{Aut}
\DeclareMathOperator{\Ob}{Ob}           % The ``set'' of object of a category
\DeclareMathOperator{\val}{val}     % valuation d'un polynôme
\DeclareMathOperator{\res}{res}     % Le résultant de deux polynômes
\DeclareMathOperator{\Inv}{Inv}     % L'application inverse
\DeclareMathOperator{\SP}{SP}
\DeclareMathOperator{\Conf}{Conf}
\DeclareMathOperator{\gsl}{\mathfrak{sl}}
\DeclareMathOperator{\go}{\mathfrak{o}}
\DeclareMathOperator{\gsu}{\mathfrak{su}}
\DeclareMathOperator{\gsp}{\mathfrak{sp}}
\DeclareMathOperator{\so}{\mathfrak{so}}
\DeclareMathOperator{\Spin}{Spin}
\DeclareMathOperator{\mSpin}{Spin}      % La commande \mSpin dénote l'application Spin qui va de SL(2,C) vers L^+ flèche.
\DeclareMathOperator{\spin}{\mathfrak{spin}}
\DeclareMathOperator{\Cl}{Cl}
\DeclareMathOperator{\Cliff}{Cl}
\DeclareMathOperator{\CCliff}{\Cliff^{\eC}}     % Changement de notation par rapport à avant.
\DeclareMathOperator{\volume}{vol}
\DeclareMathOperator{\esssup}{ess-\sup}
\DeclareMathOperator{\gpAff}{Aff}
\DeclareMathOperator{\Vect}{Vect}
\DeclareMathOperator{\eae}{eae}         % espace affine engendré
\DeclareMathOperator{\gpSymp}{Symp}
\DeclareMathOperator{\horsp}{hor}
\DeclareMathOperator{\Dim}{Dim}
\DeclareMathOperator{\Harm}{Harm}       % The space of harmonic forms
\DeclareMathOperator{\Sign}{Sign}       % The sign function.
\DeclareMathOperator{\Rank}{Rank}
\DeclareMathOperator{\Res}{Res}
\DeclareMathOperator{\ResW}{\Res_W}
\DeclareMathOperator{\sgrad}{sgrad}     % symplectic gradient
\DeclareMathOperator{\stab}{\mathfrak{Stab}}
\DeclareMathOperator{\ad}{ad}
\DeclareMathOperator{\Ad}{Ad}
\DeclareMathOperator{\AD}{\textbf{Ad}}
\DeclareMathOperator{\Der}{\texttt{Der}}
\DeclareMathOperator{\Inn}{Inn}
\DeclareMathOperator{\Out}{Out}
\DeclareMathOperator{\Diff}{Diff}
\DeclareMathOperator{\biDiff}{bi-Diff}
\DeclareMathOperator{\Hol}{Hol}
\DeclareMathOperator{\Ray}{Ray}
\DeclareMathOperator{\mfsp}{\mathfrak{sp}}
\DeclareMathOperator{\Fr}{Fr}
\DeclareMathOperator{\Rad}{Rad}
\DeclareMathOperator{\niv}{Level}
\DeclareMathOperator{\Supp}{Supp}
\DeclareMathOperator{\sech}{sech}
\DeclareMathOperator{\Prim}{Prim}
\DeclareMathOperator{\Trans}{Trans}
\DeclareMathOperator{\Verm}{Verm}       % Pour le module de Verma
\DeclareMathOperator{\Irr}{Irr}
\DeclareMathOperator{\vol}{Vol}
\DeclareMathOperator{\Op}{Op}           % Le truc de la quantification de Weyl
\DeclareMathOperator{\rDi}{Di}
\DeclareMathOperator{\rRac}{Rac}
\DeclareMathOperator{\Maxpprod}{\texttt{pprod}}
\DeclareMathOperator{\Maxproj}{\texttt{proj}}
\DeclareMathOperator{\Maxcom}{\texttt{com}}
\DeclareMathOperator{\Maxcombis}{\texttt{combi6}}
\DeclareMathOperator{\Maxtables}{\texttt{table6}}
\DeclareMathOperator{\Maxtablesc}{\texttt{table6c}}
\DeclareMathOperator{\Maxdecomps}{\texttt{decomp6}} % Ces commandes sont pour Maxima.
\DeclareMathOperator{\Maxdecompsc}{\texttt{decomp6c}}
\DeclareMathOperator{\Maxtableqc}{\texttt{table4c}}
\DeclareMathOperator{\Maxtableq}{\texttt{table4}}
\DeclareMathOperator{\Maxdecompq}{\texttt{decomp4}}
\DeclareMathOperator{\Maxdecompqc}{\texttt{decomp4c}}
\DeclareMathOperator{\Maxomega}{\texttt{omega}}
\DeclareMathOperator{\Maxsymple}{\texttt{symple}}
\DeclareMathOperator{\Maxcycle}{\texttt{cycle}}
\DeclareMathOperator{\Maxsolve}{\texttt{solve}}
\DeclareMathOperator{\Maxdelxistar}{\texttt{delxistar}}
\DeclareMathOperator{\Maxxistar}{\texttt{xistar}}


\DeclareMathOperator{\signe}{sgn}
\DeclareMathOperator{\Vol}{Vol}
\DeclareMathOperator{\Int}{Int}     % Intérieur d'un ensemble.
\DeclareMathOperator{\Adh}{Adh}     % Adhérence d'un ensemble.
\DeclareMathOperator{\Ind}{Ind}     % l'indice d'un chemin en analyse complexe
\DeclareMathOperator{\Turn}{Turn} % Le nombre de tours d'une courbe fermée
\DeclareMathOperator{\IR}{Ind}       % indice de rotation
\DeclareMathOperator{\Cond}{Cond}   % conditionnement d'une matrice
\DeclareMathOperator{\Diam}{Diam}
\DeclareMathOperator{\id}{Id}
\DeclareMathOperator{\Graph}{Graph}
\DeclareMathOperator{\Conv}{\mathcal{C}onv}
\DeclareMathOperator{\pr}{\texttt{proj}}
\DeclareMathOperator{\dom}{dom}
\DeclareMathOperator{\Graphe}{Gr}
\DeclareMathOperator{\Spec}{Spec}   % spectre d'un opérateur
\DeclareMathOperator{\arctg}{arctg}
\DeclareMathOperator{\cotg}{cotg}
\DeclareMathOperator{\cosec}{cosec}
\DeclareMathOperator{\arcsinh}{arcsinh}
\DeclareMathOperator{\sinc}{sinc}   % Le sinus cardinal
\DeclareMathOperator{\PGL}{PGL}   % le groupe projectif
\DeclareMathOperator{\SO}{SO}
\DeclareMathOperator{\SL}{SL}
\DeclareMathOperator{\PSL}{PSL}   % Le groupe modulaire SL(2,Z)/Z2
\DeclareMathOperator{\gS}{S}        % le groupe des matrices symétriques et aussi les nombres complexes de norme 1.
\DeclareMathOperator{\SU}{SU}
\DeclareMathOperator{\su}{\mathfrak{su}}
\DeclareMathOperator{\gU}{U}
\DeclareMathOperator{\gu}{\mathfrak{u}}
\DeclareMathOperator{\gO}{O}            % On mets un g devant les groupes dont le nom est juste une lettre, ou est ambigu.
\DeclareMathOperator{\diag}{diag}
\DeclareMathOperator{\Hom}{Hom}
\DeclareMathOperator{\Domain}{Dom}      % Domaine of an operator
\DeclareMathOperator{\Domaine}{Dom}
\DeclareMathOperator{\Dom}{Domaine}

\DeclareMathOperator{\real}{Re}        % Real and imaginary part of a complex number.
\DeclareMathOperator{\imag}{Im}        % These names are from Sage.

\DeclareMathOperator{\Image}{Image}        % ... avec \Image qui donne l'image d'une fonction ou d'un opérateur.
\DeclareMathOperator{\rang}{rang}
\DeclareMathOperator{\Kernel}{Ker}
\DeclareMathOperator{\Span}{Span}
\DeclareMathOperator{\End}{End}     % L'ensemble des endomorphismes
\DeclareMathOperator{\Cyl}{Cyl}
\DeclareMathOperator{\tr}{Tr}       % la trace
\DeclareMathOperator{\Tr}{Tr}       % la trace
\DeclareMathOperator{\trace}{Tr}       % la trace
\renewcommand{\det}{\mathop{\mathrm{det}}\nolimits}           % le déterminant
\DeclareMathOperator{\Majorant}{Maj}
\DeclareMathOperator{\codim}{codim} % pour la codimension.
\DeclareMathOperator{\diam}{diam} % le diamètre d'un ensemble.
\DeclareMathOperator{\Var}{Var}     % Variance d'une variable aléatoire.
\DeclareMathOperator{\Fun}{Fun}     % Ensemble des applications d'un ensemble vers l'autre.
\DeclareMathOperator{\Cov}{Cov}     % la covariance.
\DeclareMathOperator{\gr}{gr}     % le groupe engendré
\DeclareMathOperator{\pgcd}{pgcd}
\DeclareMathOperator{\ppcm}{ppcm}
\DeclareMathOperator{\Frob}{Frob}
\DeclareMathOperator{\Card}{Card}       % Le cardinal d'un ensemble.
\DeclareMathOperator{\Stab}{Stab}       % Le stabilisateur d'un point sous l'action d'un groupe.
\DeclareMathOperator{\Frac}{Frac}       % le corps des fractions d'un anneau
\DeclareMathOperator{\Aff}{Aff}         %  l'espace affine engendré


\newcommand{\dpt}[3]{#1\colon #2\to #3}
\newcommand{\subdem}[1]{\par\noindent {\it #1.} }       % TODO : use subproof instead

\newcommand{\slim}{\mathrm{s\lim}}
\newcommand{\uwlim}{\mathrm{uw\lim}}

\newcommand{\sod}{\mathfrak{so}(2)}
\newcommand{\SOdn}{\SO(2,n)}
\newcommand{\sodn}{  {\mathfrak{so}}(2,n)   }
\newcommand{\soun}{\mathfrak{so}(1,n)}
\newcommand{\SLdc}{\SL(2,\eC)}
\newcommand{\sldr}{\mathfrak{sl}(2,\eR)}
\newcommand{\SOun}{\SO(1,n)}
\newcommand{\gud}{\Gamma_{(2)}}
\newcommand{\Sput}{\Spin(1,3)}
\newcommand{\Sppq}{\Spin(p,q)}
\newcommand{\sppq}{\mathfrak{spin}(p,q)}
\newcommand{\Sopq}{\SO(p,q)}
\newcommand{\sopq}{\mathfrak{so}(p,q)}
\newcommand{\gl}{\mathfrak{gl}}
\newcommand{\Fix}{\operatorname{Fix}}
\newcommand{\cat}{\operatorname{cat}}

\newcommand{\ecarts}{ecarts}
\newcommand{\angl}{\quext{Anglais ?}}
\newcommand{\intr}[1]{\mathaccent 23 {#1}}
\newcommand{\UU}{\intr{U}}

\newcommand{\donc}{\Rightarrow}

\newcommand{\ecart}{ecart} % quand tu trouveras une meilleure traduction...

%--------- Alphabets math

\newcommand{\mfa}{\mathfrak{a}}
\newcommand{\mfb}{\mathfrak{b}}
\newcommand{\mfg}{\mathfrak{g}}
\newcommand{\mfs}{\mathfrak{s}}
\newcommand{\mfM}{\mathfrak{M}}

\newcommand{\scrC}{\mathscr{C}}
\newcommand{\scrD}{\mathscr{D}}         %Demande le paquetage mathrsfs.
\newcommand{\scrE}{\mathscr{E}}
\newcommand{\scrM}{\mathscr{M}}
\newcommand{\scrS}{\mathscr{S}}
\newcommand{\hS}{\mathscr{S}}           % C'est lui qui donne la singularité
\newcommand{\hF}{\mathscr{F}}           % \hF donne la partie libre de l'espace

\newcommand{\cA}{\mathfrak{A}}          % Pour les C^* algebres; comme ça je peux choisir.
\newcommand{\cB}{\mathcal{B}}           % Le mathfrak{B} est pour l'ensemble des operateurs bornés.
\newcommand{\cun}{\mtu}             % L'unite dans les $C^*$-algèbres.
\newcommand{\cI}{\mathfrak{I}}

\newcommand{\Cinf}{C^{\infty}}
\newcommand{\vnM}{\mathfrak{M}}         % Le M des algèbres de von Neumann
\newcommand{\hodge}{\star}         % the Hodge dual


%\newcommand{\bmodE}{\mathcal{\bar E}}      % Pour le module conjugué
\newcommand{\modM}{\mathfrak{M}}
\newcommand{\modN}{\mathfrak{N}}


\newcommand{\pH}{\mathscr{H}}           % L'espace de Hilbert pour la physique
\newcommand{\rR}{\mathcal{R}}           % Les \r? sont les lettres pour les rayons de l'espace de Hilbert.
\newcommand{\rC}{\mathcal{C}}

\newcommand{\cdA}{\mathscr{A}}
\newcommand{\cdE}{\mathscr{E}}          % Les ensembles de fonctions continuement d\'erivables
\newcommand{\cdD}{\mathscr{D}}          % L'ensemble des fonctions à support compact.

%-------Overline,underline, hat,tilde
\newcommand{\uw}{\underline{w}}
\newcommand{\uv}{\underline{v}}
\newcommand{\uW}{\underline{W}}
\newcommand{\uvH}{\underline{H}}
\newcommand{\ovH}{\overline{H}}
\newcommand{\ovN}{\overline{N}}
\newcommand{\ovx}{\overline{x}}
\newcommand{\ovy}{\overline{y}}
\newcommand{\ovj}{\overline{j}}
\newcommand{\ova}{\overline{a}}
\newcommand{\os}{\overline{s}}
\newcommand{\oJ}{\overline{J}}
\newcommand{\oX}{\overline{X}}
\newcommand{\oY}{\overline{Y}}
\newcommand{\ovR}{\overline{R}}
\newcommand{\olG}{\overline{\lG}}
\newcommand{\olR}{\overline{\lR}}
\newcommand{\olS}{\overline{\lS}}

\newcommand{\uG}{\underline{G}}
\newcommand{\uX}{\underline{X}}
\newcommand{\uAN}{\underline{AN}}

\newcommand{\oui}{\overline{1}_i}
\newcommand{\oalpha}{\overline{\alpha}}
\newcommand{\obeta}{\overline{\beta}}
\newcommand{\oeta}{\overline{\eta}}
\newcommand{\oxi}{\overline{\xi}}
\newcommand{\ogamma}{\overline{\gamma}}
\newcommand{\odelta}{\overline{\delta}}
\newcommand{\tA}{\widetilde{A}}
\newcommand{\tK}{\widetilde{K}}
\newcommand{\tN}{\widetilde{N}}
\newcommand{\tR}{\widetilde{R}}
\newcommand{\tilr}{\widetilde{r}}
\newcommand{\tx}{\tilde{x}}
\newcommand{\ty}{\tilde{y}}
\newcommand{\tE}{\tilde{E}}
\newcommand{\tF}{\tilde{F}}
\newcommand{\tH}{\tilde{H}}
\newcommand{\tX}{\tilde{X}}
\newcommand{\tY}{\tilde{Y}}
\newcommand{\utX}{\underline{X}}                %Il faut encore trouver comment souligner avec un tilde.
\newcommand{\utE}{\underline{E}}
\newcommand{\utH}{\underline{H}}
\newcommand{\talpha}{\tilde{\alpha}}
\newcommand{\tomega}{\tilde{\omega}}
\newcommand{\tgamma}{\tilde{\gamma}}
\newcommand{\tnab}{\widetilde{\nabla}}
\newcommand{\expotilde}{\widetilde{\hphantom{X}}}       % Waiting to know how to create a suitable tilde in twists_general.tex

\newcommand{\hpsi}{\hat{\psi}}
\newcommand{\ha}{\hat{a}}
\newcommand{\hg}{\hat{g}}
\newcommand{\hs}{\hat{s}}
\newcommand{\hu}{\hat{u}}
\newcommand{\hv}{\hat{v}}
\newcommand{\hw}{\hat{w}}
\newcommand{\hx}{\hat{x}}
\newcommand{\hy}{\hat{y}}
\newcommand{\hB}{\hat{B}}
\newcommand{\hX}{\hat{X}}
\newcommand{\hY}{\hat{Y}}


%https://tex.stackexchange.com/questions/100574/really-wide-hat-symbol
\stackMath
\newcommand\reallywidehat[1]{%
\savestack{\tmpbox}{\stretchto{%
  \scaleto{%
    \scalerel*[\widthof{\ensuremath{#1}}]{\kern-.6pt\bigwedge\kern-.6pt}%
    {\rule[-\textheight/2]{1ex}{\textheight}}%WIDTH-LIMITED BIG WEDGE
  }{\textheight}% 
}{0.5ex}}%
\stackon[1pt]{#1}{\tmpbox}%
}

%\Overline,underline, hat,tilde----------

%------Produits star
\newcommand{\stG}{\star^{G}}
\newcommand{\stW}{\star^{W}}
\newcommand{\stWt}{\star^{W}_{\theta}}
\newcommand{\stWh}{\star^{W}_{\hbar}}
\newcommand{\stX}{\star^{X}}
\newcommand{\stM}{\ast_M}
\newcommand{\stt}{\star_{\theta}}
%\newcommand{\st}{\ast}
%\Produits star-------------
\newcommand{\us}[1]{\frac{1}{#1}}
\newcommand{\dsd}[2]{\frac{\partial #1}{\partial #2}}
\newcommand{\me}[1]{(-1)^{#1}}
\newcommand{\xdp}[2]{#1\to #2}
\newcommand{\brak}[2]{\langle #1,#2\rangle}
\newcommand{\dsdd}[3]{\left.\frac{d}{d#2}#1\right|_{#2=#3}}
\newcommand{\Dsddb}[4]{\frac{d}{d#2}\Big[#1\Big]_{#3=#4}}
\newcommand{\Dsdd}[3]{ \Dsddb{#1}{#2}{#2}{#3}   }
\newcommand{\Dsddc}[3]{\frac{d}{d#2}\Big(#1\Big)_{#2=#3}}
\newcommand{\Dsddp}[3]{\frac{d}{d#2}\Big(#1\Big)_{#2=#3}}
\newcommand{\dDsdd}[5]{\frac{d}{d#2}\frac{d}{d#4}
           \Big[#1\Big]_{ \begin{subarray}{l}#4=#5\\#2=#3\end{subarray} }}

\newcommand{\DDsdd}[5]{\frac{d}{d#2}\frac{d}{d#4}
           \Big[#1\Big]_{ \begin{subarray}{l}#4=#5\\#2=#3\end{subarray} }}

\newcommand{\mfo}{\vartheta}

\newcommand{\hbeta}{^{\beta}}
\newcommand{\hkappa}{^{\kappa}}
\newcommand{\bgamma}{_{\gamma}}
\newcommand{\bdelta}{_{\delta}}
\newcommand{\hmu}{^{\mu}}
\newcommand{\hnu}{^{\nu}}
\newcommand{\bab}{_{\alpha\beta}}

\newcommand{\heta}{^{\eta}}
\newcommand{\bxi}{_{\xi}}
\newcommand{\hsigma}{^{\sigma}}

%Extension de l'alphabet grec------------


\newcommand{\lA}{\mathfrak{a}}
\newcommand{\lB}{\mathfrak{b}}
\newcommand{\lF}{\mathfrak{f}}
\newcommand{\lG}{\mathfrak{g}}      % Pour les algèbres de Lie en général; comme ça je peux choisir,
\newcommand{\lH}{\mathfrak{h}}      % mais le mal du \mG est déjà loin !
\newcommand{\lI}{\mathfrak{i}}
\newcommand{\lJ}{\mathfrak{j}}
\newcommand{\lK}{\mathfrak{k}}
\newcommand{\lL}{\mathfrak{L}}
\newcommand{\lM}{\mathfrak{m}}
\newcommand{\lN}{\mathfrak{n}}
\newcommand{\lP}{\mathfrak{p}}
\newcommand{\lQ}{\mathfrak{q}}
\newcommand{\lR}{\mathfrak{r}}
\newcommand{\lS}{\mathfrak{s}}
\newcommand{\lU}{\mathfrak{u}}
\newcommand{\lX}{\mathfrak{x}}
\newcommand{\lZ}{\mathfrak{z}}

\newcommand{\lW}{\mathcal{W}}

\newcommand{\iA}{\mathcal{A}}
\newcommand{\iK}{\mathcal{K}}
\newcommand{\iN}{\mathcal{N}}           % Pour les éléments de décomposition d'Iwasawa
\newcommand{\iP}{\mathcal{P}}
\newcommand{\iR}{\mathcal{R}}
\newcommand{\iAH}{\mathcal{A_H}}
\newcommand{\iKH}{\mathcal{K_H}}
\newcommand{\iKQ}{\iK_{\sQ}}
\newcommand{\iNH}{\mathcal{N_H}}
\newcommand{\iPH}{\mathcal{P_H}}
\newcommand{\iRH}{\mathcal{R_H}}

\newcommand{\curR}{\mathrm{R}}      % La courbure scalaire d'un triple spectral

\newcommand{\SUR}{\mathrm{R}}
\newcommand{\SUA}{\mathrm{A}}       % les SUx sont pour les parties de SU(1,n).
\newcommand{\SUN}{\mathrm{N}}

\newcommand{\suqA}{\mathcal{A}}   % The algebra of SU_q(n)

\newcommand{\sA}{\mathcal{A}}
\newcommand{\sG}{\mathcal{G}}
\newcommand{\sH}{\mathcal{H}}           % Pour les morceaux de SO(2,n) et SO(1,n)
\newcommand{\sK}{\mathcal{K}}           % En fait, les morceaux de AdS_l par Iwasawa vont aussi êres notes avec des \sX
\newcommand{\sN}{\mathcal{N}}
\newcommand{\sP}{\mathcal{P}}
\newcommand{\sQ}{\mathcal{Q}}
\newcommand{\sR}{\mathcal{R}}
\newcommand{\sS}{\mathcal{S}}
\newcommand{\sZ}{\mathcal{Z}}

\newcommand{\etS}{\mathcal{S}}      % L'ensemble des etats sur une $C^*$-algebre.
\newcommand{\etP}{\mathcal{P}}      % Les états purs

\newcommand{\tsA}{\widetilde{\mathcal{A}}}
\newcommand{\tsN}{\widetilde{\mathcal{N}}}
\newcommand{\tsR}{\widetilde{\mathcal{R}}}

\newcommand{\ovf}{\overline{ f }}
\newcommand{\cvec}{\mathfrak{X}}
\newcommand{\Wedge}{\bigwedge}
\newcommand{\LogOu}{\vee}
\newcommand{\LogEt}{\wedge}
\newcommand{\cuppr}{\sharp}     % En attendant de trouver mieux.
\newcommand{\svec}{\mathcal{B}}     % Les vecteurs C^{\infty} d'une action
\newcommand{\Dir}{\mathcal{D}}

\newcommand{\yG}{\mathcal{G}}  % L'algèbre de Lie dans le truc sur YM

%---------Constructions d'un besoin passager

\newcommand{\nomscript}[1]{\emph{#1}}
\newcommand{\dtau}{\partial_{\tau}}
\newcommand{\du}{\partial_{u}}
\newcommand{\dphi}{\partial_{\phi}}
\newcommand{\frZ}[2]{   \frac{2(#1,#2)}{(#1,#1)}     }
\newcommand{\heC}{^{\eC}}
\newcommand{\beC}{_{\eC}}
\newcommand{\beR}{_{\eR}}
\newcommand{\heR}{^{\eR}}
\newcommand{\blF}{_{\lF}}
\newcommand{\etalH}{\eta_{\lH}}
\newcommand{\lHeR}{\lH_{\eR}}
\newcommand{\lGeR}{\lG_{\eR}}
\newcommand{\lFeC}{\lF^{\eC}}
\newcommand{\lGeC}{\lG^{\eC}}
\newcommand{\lHeC}{\lH^{\eC}}
\newcommand{\lbha}{\beta^{\alpha}}
\newcommand{\lbba}{\beta_{\alpha}}
\newcommand{\aba}{a_{\alpha}}
\newcommand{\abb}{a_{\beta}}
\newcommand{\abg}{a_{\gamma}}
\newcommand{\abd}{a_{\delta}}
\newcommand{\abmb}{a_{-\beta}}
\newcommand{\abmg}{a_{-\gamma}}
\newcommand{\abmd}{a_{-\delta}}
\newcommand{\abab}{a_{\alpha+\beta}}
\newcommand{\abma}{a_{-\alpha}}
\newcommand{\abmr}{a_{-\rho}}
\newcommand{\abr}{a_{\rho}}
\newcommand{\abbp}{a_{\beta'}}
\newcommand{\xbg}{x_{\gamma}}
\newcommand{\xbd}{x_{\delta}}
\newcommand{\hbb}{h_{\beta}}
\newcommand{\xbma}{x_{-\alpha}}
\newcommand{\xbb}{x_{\beta}}
\newcommand{\xbmb}{x_{-\beta}}
\newcommand{\xbmab}{x_{\alpha-\beta}}
\newcommand{\xbmamb}{x_{-\alpha-\beta}}
\newcommand{\rmg}[1]{ \big( \rho(#1)-\gamma(#1) \big) }
\newcommand{\lRlR}{[\lR,\lR]}
\newcommand{\cloi}{\overline{i}}
\newcommand{\cloj}{\overline{j}}
\newcommand{\cloip}{\overline{i'}}
\newcommand{\dD}{\scrD}

\newcommand{\AutA}{\Aut(\lA)}
\newcommand{\IntA}{\Int(\lA)}
\newcommand{\AutB}{\Aut(\lB)}
\newcommand{\IntB}{\Int(\lB)}
\newcommand{\RM}{\pr_{\sQ}\sR}
\newcommand{\Oexp}[3]{ \Omega_2\Big(  e^{\ad#1}#2,e^{\ad#1}#3  \Big)  }
\newcommand{\Xrnz}{X\times(\eR^N\setminus\{o\})}
\newcommand{\DxaDxb}{D_x^{\alpha} D\bxi\hbeta}
\newcommand{\abxi}{|\xi|}
\newcommand{\baz}[2]{\{#1e_i\}_{#2}}
\newcommand{\decompss}[3]{%
\begin{equation}
\begin{split}
\mfs_1&=\{#1\}\\
\mfs_2&=\{#2\}#3
\end{split}
\end{equation}
}
\newcommand{\delE}[2]{\delta_{#1}E_{#2}}
\newcommand{\dcr}[1]{[[#1]]}
\newcommand{\dga}[2]{\gamma_{#1}\gamma_{#2}}
\newcommand{\tga}[3]{\gamma_{#1}\gamma_{#2}\gamma_{#3}}
\newcommand{\qga}[4]{\gamma_{#1}\gamma_{#2}\gamma_{#3}\gamma_{#4}}
\newcommand{\rhoM}{\rho^M}
\newcommand{\hperp}{^{\perp}}
   % Mes produits scalaires                 % Je crois que je vais unifier sous \braket pour le produit < x | y > et sous \scal pour < x , y >.
\newcommand{\braket}[2]{ \langle #1|#2\rangle }
\newcommand{\ket}[1]{ | #1\rangle }
\newcommand{\bra}[1]{ \langle #1| }
\newcommand{\scalp}[2]{  (#1|#2) }
\newcommand{\scald}[2]{ \scal{#1}{#2} }
\newcommand{\scalh}[2]{ \braket{#1}{#2} }
\newcommand{\ketbra}[2]{|#1\rangle\,\langle #2|}


\newcommand{\dptvb}[3]{#1\stackrel{#2}{\longrightarrow}#3}
\newcommand{\ovv}{\overline{v}}
\newcommand{\ovX}{\overline{X}}
\newcommand{\ovS}{\overline{S}}

\newcommand{\bghd}[3]{#1_{#2}^{\phantom{#2}#3}}

% These commands were \mathbf instead of overline but I have the ``Too many math alphabets used in version normal`` error.
\newcommand{\bE}{\overline{ E }}
\newcommand{\bA}{\overline{ A }}
\newcommand{\bB}{\overline{ B }}
\newcommand{\BX}{\overline{ X }}

\newcommand{\gab}{g_{\alpha\beta}}
\newcommand{\sbeta}{\sigma_{\beta}}
\newcommand{\salpha}{\sigma_{\alpha}}

\newcommand{\quextproj}{\quext{In project\ldots}}
%\newcommand{\tb}{\tilde{b}}
\newcommand{\gamsai}{\gamma_{\alpha j}}
\newcommand{\bsa}{{}_{(\alpha)}{}}
\newcommand{\gamaj}{\gamma_{\alpha j}}
\newcommand{\gamai}{\gamma_{\alpha i}}
\newcommand{\psisa}{\psi\bsa}
\newcommand{\opK}{\mathfrak{K}}     % Compact operators
\newcommand{\opB}{\mathfrak{B}}     % Bounded operators
\newcommand{\invtible}{^{\times}}   % Cette commande est en attendant de trouver un symbole plus spécifique à mettre sur les ensembles pour désigner leur partir inversible.
\newcommand{\osint}{\widetilde{\int}}
\newcommand{\Lie}[1]{\mathfrak{Lie}(#1)}
\newcommand{\qvect}[4]{(#1,#2,#3,#4)}
\newcommand{\osiint}{\widetilde{\iint}}
\newcommand{\osiiint}{\widetilde{\iiint}}
% La commande suivante est tirée de symbols-letter.pdf pour écrire une intégrale avec une barre dedans
\def\Xint#1{\mathchoice
   {\XXint\displaystyle\textstyle{#1}}%
   {\XXint\textstyle\scriptstyle{#1}}%
   {\XXint\scriptstyle\scriptscriptstyle{#1}}%
   {\XXint\scriptscriptstyle\scriptscriptstyle{#1}}%
   \!\int}
\def\XXint#1#2#3{{\setbox0=\hbox{$#1{#2#3}{\int}$}
     \vcenter{\hbox{$#2#3$}}\kern-.5\wd0}}
\newcommand{\ddashint}{\Xint=}
\newcommand{\dashint}{\Xint-}


% Le premier argument est optionnel, c'est pour ajouter un [math.QA] par exemple pour la nouvelle numérotation de arXiv. Comme tu le vois, la valeur par défaut est vide.
\newcommand{\arxiv}[2][]{%
\newline
\ifthenelse{\equal{#1}{}}{%                     Tester si un argument optionnel est passé ou non.
    \href{http://arxiv.org/abs/#2}{{\tt arXiv:#2}}%     Si tu ne mets pas ce %, il y a un problème d'espace.
            }
            {%
    \href{http://arxiv.org/abs/#2}{{\tt arXiv:#2}[#1]}%     Si tu ne mets pas ce %, il y a un problème d'espace.
}%                                  Ce %-ci aussi est indispensable pour un espace à éviter avant le . ajouté par bibtex.
}               % Fin de la commande \arxiv


% -- L'environement suivant est taxé de la classe article.cls, sauf que j'ai enlevé la possibilité que ce soit sur une page de titre.
\newcommand\abstractname{Abstract}
\makeatletter
  \newenvironment{abstract}{%
      \if@twocolumn
        \section*{\abstractname}%
      \else
        \small
        \begin{center}%
          {\bfseries \abstractname\vspace{-.5em}\vspace{\z@}}%
        \end{center}%
        \quotation
      \fi}
      {\if@twocolumn\else\endquotation\fi}
\makeatother

\newcommand{\PB}[2]{\left\{#1,#2\right\}}
                    % Les opérateurs définis pour Maxima
\newcommand{\LoL}{\mathscr{L}}      % Lorentz group
\newcommand{\LoP}{\mathscr{P}}      % Poincaré group
\newcommand{\f}{\frac}


%%%%%%%%%%%%%%%%%%%%%%%%%%
%
%   Les théorèmes et choses attenantes
%
%%%%%%%%%%%%%%%%%%%%%%%%



\setcounter{tocdepth}{2}        % Profondeur de la table des matières
\setcounter{secnumdepth}{3}     % Profondeur dans le texte

\newcounter{numtho}
\newcounter{numprob}
\newcounter{numTheme}           % For the thematic index, see InternalLinks

\makeatletter
\@addtoreset{numtho}{chapter}
\@addtoreset{equation}{chapter}
%\ifbool{isFrido}{}
%{
%    \@addtoreset{CountExercice}{chapter}
%}

\makeatother

\newlength{\EnvSpace}
\setlength{\EnvSpace}{9pt}      % C'est la distance que je veux mettre avant et après les théorèmes, remarques, \ldots

\usepackage[inline]{showlabels}
\newtheoremstyle{MyTheorems}%
        {\EnvSpace}{\EnvSpace}%
        {\itshape}%
        {}%
        {\bfseries}{.}%
        {\newline}%
        {}%
\newtheoremstyle{MyExamples}%
        {\EnvSpace}{\EnvSpace}%
        {}%
        {}%
        {\bfseries}{.}%
        {\newline}%
        {}%
\newtheoremstyle{MyRemarks}%
        {\EnvSpace}{\EnvSpace}%
        {}%
        {}%
        {\bfseries}{.}%
        {\newline}%
        {}%


\newcounter{numloiphyz}

\newcounter{CounterExample}
\renewcommand{\theCounterExample}{\thechapter.\arabic{CounterExample}}
\renewcommand{\thenumtho}{\thechapter.\arabic{numtho}}


% The 'example' environment is by William Babonnaud.
\theoremstyle{MyExamples}
    \newtheorem{example}[numtho]{Exemple}
\DeclareRobustCommand{\trig}{%
   \ifmmode \quad\hbox{\triangle}
   \else
      \leavevmode\unskip\penalty9999 \hbox{}\nobreak\hfill
      \quad\hbox{$\triangle$}
   \fi
}

\let\exold\example
\let\endexold\endexample
\renewenvironment{example}{\pushQED{\trig}\exold}{\popQED\endexold}


\newenvironment{Aretenir}{\refstepcounter{numtho}\begin{oframed}\noindent{\bf À retenir \thenumtho}\newline}{\end{oframed}\vspace{\EnvSpace}}

\theoremstyle{MyRemarks}    \newtheorem{remark}[numtho]{Remarque}

                \newtheorem{amusement}[numtho]{Amusement}
                \newtheorem{erreur}[numtho]{Error}
                \newtheorem{normaltext}[numtho]{}

\theoremstyle{MyTheorems}
            \newtheorem{lemma}[numtho]{Lemme}
            \newtheorem{corollary}[numtho]{Corolaire}
            \newtheorem{theorem}[numtho]{Théorème}
            \newtheorem{definition}[numtho]{Définition}
            \newtheorem{proposition}[numtho]{Proposition}
            \newtheorem{theoremDef}[numtho]{Théorème-définition}
            \newtheorem{corollaryDef}[numtho]{Corolaire-définition}
            \newtheorem{propositionDef}[numtho]{Proposition-définition}
            \newtheorem{lemmaDef}[numtho]{Lemme-définition}
			\newtheorem{loiphyz}[numloiphyz]{Loi numéro}

            %\newtheorem{exo}[CountExercice]{Exercice}       % C'est provisoire, pour Chafaï

% La numérotation des équations change dans les corrigés
\renewcommand{\theequation}{\thechapter.\arabic{equation}}

% This counter is defined in SystemeCorr.sty.
%\ifbool{isFrido}{}{
%    \renewcommand{\theCountExercice}{\arabic{CountExercice}}       
%}


\newcommand{\defe}[2]{\textbf{#1}\index{#2}}

%\renewcommand{\theenumi}{(\arabic{enumi})}
%\renewcommand{\labelenumi}{\theenumi}
%\renewcommand{\theenumii}{\arabic{enumi}\alph{enumii}}
%\renewcommand{\labelenumii}{(\theenumii)}



%%%%%%%%%%%%%%%%%%%%%%%%%%
%
%   Les macros qui font des choses
%
%%%%%%%%%%%%%%%%%%%%%%%%

\newcommand{\mA}{\mathcal{A}}
\newcommand{\mB}{\mathcal{B}}
\newcommand{\mC}{\mathcal{C}}
\newcommand{\mCC}{\mathcal{CC}}
\newcommand{\mD}{\mathcal{D}}
\newcommand{\mE}{\mathcal{E}}
\newcommand{\mF}{\mathcal{F}}
\newcommand{\mG}{\mathcal{G}}
\newcommand{\mH}{\mathcal{H}}
\newcommand{\mI}{\mathcal{I}}
\newcommand{\mJ}{\mathcal{J}}
\newcommand{\mK}{\mathcal{K}}
\newcommand{\mL}{\mathcal{L}}
\newcommand{\mM}{\mathcal{M}}
\newcommand{\mN}{\mathcal{N}}
\newcommand{\mO}{\mathcal{O}}
\newcommand{\mP}{\mathcal{P}}
\newcommand{\mQ}{\mathcal{Q}}
\newcommand{\mR}{\mathcal{R}}
\newcommand{\mS}{\mathcal{S}}
\newcommand{\mT}{\mathcal{T}}
\newcommand{\mU}{\mathcal{U}}
\newcommand{\mV}{\mathcal{V}}
\newcommand{\mW}{\mathcal{W}}
\newcommand{\mZ}{\mathcal{Z}}


\newcommand{\scal}[2]{ \langle #1,#2\rangle }
\newcommand{\vect}[1]{\overrightarrow{#1}}

\newcommand{\tq}{\text{ tel que }}          %TODO : il y a un paquet qui doit être changé en \st
\newcommand{\st}{\text{ such that }}
\newcommand{\tqs}{\text{ tels que }}
\newcommand{\quext}[1]{\footnote{\textsf{#1}}}
\newcommand{\info}[1]{\texttt{#1}}

\newcommand{\normal}{\lhd}  % Cette notation n'est plus censée être utilisée.

\newcommand{\Borelien}{\mathcal{B}\text{or}}       % Les boréliens
\newcommand{\Lebesgue}{\mathcal{L}\text{eb}}       % La tribu de Lebesgue
\newcommand{\Baire}{\mathcal{B}\text{a}}       % La tribu de Baire
\newcommand{\tribA}{\mathcal{A}}            % Une tribu A
\newcommand{\tribB}{\mathcal{B}}
\newcommand{\tribC}{\mathcal{C}}
\newcommand{\tribD}{\mathcal{D}}
\newcommand{\tribE}{\mathcal{E}}            % Une tribu E
\newcommand{\tribF}{\mathcal{F}}            % Une tribu F
\newcommand{\tribM}{\mathcal{M}}            % Une tribu M
\newcommand{\tribN}{\mathcal{N}}
\newcommand{\tribT}{\mathcal{T}}

\newcommand{\affE}{\mathcal{E}}            % Un espace affine E
\newcommand{\affF}{\mathcal{F}}
\newcommand{\affG}{\mathcal{G}}

\newcommand{\statS}{\mathcal{S}}            % Un modèle statistique
\newcommand{\partP}{\mathcal{P}}            % L'ensemble des parties d'un ensemble
\newcommand{\pP}{\mathcal{P}}            % L'ensemble des nombres premiers

\newcommand{\polyP}{\mathcal{P}}            % L'ensemble des polynômes

\newcommand{\dB}{\mathscr{B}}       % la distribution de Bernoulli
\newcommand{\dE}{\mathscr{E}}       % la distribution exponentielle
\newcommand{\dirE}{\mathcal{E}}           % Dirichlet form
\newcommand{\dG}{\mathscr{G}}       % la distribution géométrique.
\newcommand{\dM}{\mathscr{M}}       % la distribution multinomiale
\newcommand{\dN}{\mathscr{N}}       % la distribution normale.
\newcommand{\dP}{\mathscr{P}}       % la distribution de Poisson.
\newcommand{\dT}{\mathscr{T}}       % la distribution de Student
\newcommand{\dU}{\mathscr{U}}       % la distribution uniforme

\newcommand{\hL}{\mathscr{L}}
%\newcommand{\cL}{\hL}           % Pour la partie Chafai

\newcommand{\modE}{\mathcal{E}}         % Le E des modules
\newcommand{\modF}{\mathcal{F}}         % Le F des modules
\newcommand{\hH}{\mathscr{H}}           % Le H des espaces de Hilbert

\newcommand{\ellE}{\mathcal{E}}         % Le E des ellipsoïde
\newcommand{\ellF}{\mathcal{F}}         % Le F des ellipsoïde

% Configuration for nomencl        1338719836

\makenomenclature
\renewcommand{\nomname}{Liste des notations}
%
% La syntaxe est facile, par exemple
%       $\SL(2,\eR)$\nomenclature[G]{$\SL(2,\eR)$}{Le groupe de matrices deux par deux de déterminant 1.}
\renewcommand{\nomgroup}[1]{%
    \ifthenelse{\equal{#1}{A}}{\item[\textbf{Algèbre}]}{}%
    \ifthenelse{\equal{#1}{B}}{\item[\textbf{Ensembles de matrices}]}{}%
    \ifthenelse{\equal{#1}{G}}{\item[\textbf{Géométrie}]}{}%
    \ifthenelse{\equal{#1}{M}}{\item[\textbf{Chaînes de Markov}]}{}%
    \ifthenelse{\equal{#1}{R}}{\item[\textbf{Théorie des groupes}]}{}%
    \ifthenelse{\equal{#1}{P}}{\item[\textbf{Probabilités et statistique}]}{}%
    \ifthenelse{\equal{#1}{Y}}{\item[\textbf{Analyse}]}{}%
    \ifthenelse{\equal{#1}{T}}{\item[\textbf{Topologie et théorie des ensembles}]}{}%
}
% Note pour moi-même : si cette liste est changée, il faut changer mon raccourcis dans Vim.
\newcommand*{\Sp}{\textup{Sp}}
\newcommand*{\GL}{\textup{GL}}      % Le groupe linéaire; je crois qu'ontologiquement c'est mieux que le DeclareMathOperator


\newcommand{\subprooflabel}[1]{\underline{\bf #1}}

\newenvironment{subproof}{\let\Olddescriptionlabel\descriptionlabel\let\descriptionlabel\subprooflabel \begin{description}}{\end{description}\let\descriptionlabel\Olddescriptionlabel}


% See also the file 'src_front_back_matter/157_thematique.tex' in which we
% open and close the intermediate file 'theme.toc'
\newcommand{\InternalLinks}[1]
{
    \refstepcounter{numTheme} \paragraph{Thème \arabic{numTheme} : #1} \label{THTOC\arabic{numTheme}}
    \immediate\write\themetoc{%
        \unexpanded{\ref}{THTOC\arabic{numTheme}} %
        : \unexpanded{#1\\}%
    }
}


\newenvironment{probleme}{\refstepcounter{numprob}\tiny\fbox{\bf Problèmes et choses à faire}\\}{\normalsize}

\newcommand{\TextePourISBN}{
    \begin{center}
ISBN : Y-Y-YYYYYYY-Y-Y
    \end{center}
}


% If one change something here, one should also change
% in python/generic.tex
\newcommand{\LicenceFDL}{
\begin{center}

            \includegraphics[width=1cm]{pictures_bitmap/gfdl-logo-small.png}

Copyright 2011-2019

Permission is granted to copy, distribute and/or modify this document under the terms of the \href{http://www.gnu.org/licenses/fdl-1.3.html}{GNU Free Documentation License}, Version 1.3 or any later version published by the Free Software Foundation; with no Invariant Sections, no Front-Cover Texts, and no Back-Cover Texts. A copy of the license is included in the chapter entitled ``GNU Free Documentation~License''.

\end{center}
}

\newcommand{\LogoEtLicence}{
\notbool{isFrido}{
\LicenceFDL
}
{
    %\LicenceCC
    \LicenceFDL
    \TextePourISBN
}
}


% FIN DE MES CHOSES %%%%%%%

\newcounter{exoNico}
\setcounter{exoNico}{1}
\newcommand{\exerNico}{\stepcounter{exoNico}{\bf Exercice }\arabic{exoNico}. }



\newtheorem{theo}{Th{\'e}or{\`e}me}[section]
\newtheorem{defn}{D{\'e}finition}
\newtheorem{prop}{Proposition}     % redef encore dans Chafaï
%\newtheorem{lem}{Lemme}[section]


%\newtheorem{rem}{Remarque}[section]
%\newcommand{\R}{\mathbb{R}}
%\newcommand{\Rn}{\eR}
%\newcommand{\Nn}{\eN}
\newcommand{\dem}{\textbf{D{\'e}monstration.}}
\newcommand{\vc}[1]{\boldsymbol{#1}}
\newcommand{\p}{\textrm{P}}
%\newcommand{\e}{\textrm{E}}
\newcommand{\mbt}{arbre binaire markovien}
\newcommand{\mbts}{arbres binaires markoviens}

%\newcommand{\ea}{\end{array}}


%%%%%%%%%%%%%%%%%%%%%%%%%%%%%%%%%%%%%%
%
% les petis yeux
%
%%%%%%%%%%%%%%%%%%%%%%%%%%%%%%%%%%%%%%%%%%%%%

\newcommand{\coolexo}{$\circledast\circledast$}
\newcommand{\boringexo}{$\circleddash\circleddash$}
\newcommand{\minsyndical}{$\odot\odot$}
\newcommand{\mortelexo}{$\obslash\oslash$}



%%%%%%%%%%%%%% TRUCS DE PIERRE %%%%%%%%%%%%%%%%%%%%%%


% Le paquet array est là pour faire fonctionner l'environement arrowcases dans les trucs de Pierre.

% À régler par l'utilisateur
\newlength{\arrowsep}\setlength{\arrowsep}{3pt}
\newlength{\arrowlength}\setlength{\arrowlength}{1cm}

\newenvironment{arrowcases}%
{\begin{cases}}
{\end{cases}}



\makeatletter %% \limite[condition]x x_0
\newcommand*{\limite}[3][\@empty]{\lim_{\substack{#2\rightarrow#3\\#1}}}
\makeatother

\newcommand*\sev{<} %

\let\ssi\iff
\newcommand*{\ideal}[1]{\{#1\}}
\newcommand*{\fleche}[1]{\stackrel{#1}\longrightarrow}

%\setcounter{CountExercice}{0}

\newcommand{\Acplx}{A_\cdot}
\newcommand{\Bcplx}{B_\cdot}
\newcommand{\toisom}{\fleche\simeq}
\newcommand{\D}{\partial}
\newcommand{\lied}{\mathcal L}
\newcommand*{\nom}[1]{\textsc{#1}}
\newcommand*{\inner}{\imath}
\newcommand*{\newexo}{}
\newcommand*{\principe}{}
\newcommand*{\etape}{}
\newcommand*{\preuve}{}
\newcommand*{\exr}{\item}

\newcommand*{\crochets}[1]{\Bigl[ #1 \Bigr]}
\newcommand*{\llbrack}[1]{\left\lbrack #1 \right\lbrack}
\newcommand*{\rlbrack}[1]{\left\rbrack #1 \right\lbrack}
\newcommand*{\lrbrack}[1]{\left\lbrack #1 \right\rbrack}
\newcommand*{\rrbrack}[1]{\left\rbrack #1 \right\rbrack}


\newcommand*{\alg}[1]{\mathcal{#1}} % Algèbre
\newcommand*{\TT}{\ens T}% Tore !
\newcommand*{\topologie}{\mathscr{T}}
\newcommand*{\Topologie}{\textcursive{T}}
\newcommand{\LL}{\text{\textup{L}}} %% Espace de Lebesgue droit
\newcommand{\Ll}{\mathcal{L}} %% Lebesgue ronde
\newcommand{\sigmaalgebre}[1]{\mathcal{#1}} %% Une sigma algèbre...
\DeclareMathOperator{\SymMatrix}{Sym}
\DeclareMathOperator{\ASymMatrix}{ASym}
\newcommand{\Sym}{\SymMatrix}
\newcommand{\ASym}{\ASymMatrix}
%\newcommand{\transpose}[1]{{\vphantom{#1}}^{\mathit t}{\/#1}}
\newcommand*{\dprime}{{\prime\prime}}

%% Maths : Symboles divers
\newcommand{\surj}{\vers}
\newcommand{\isom}{\simeq}
\newcommand*{\Tau}{\alg T}
\newcommand{\cdv}{\mathfrak{X}} % Champs de vecteurs



\newcommand*{\abs}[1]{\left\vert#1\right\vert} % Valeur absolue.
\newcommand*{\module}[1]{\left\vert#1\right\vert} % Valeur absolue.
\newcommand*{\norme}[1]{\left\Vert#1\right\Vert} % norme
\newcommand*{\ordre}[1]{\left\vert#1\right\vert} % L'ordre d'un élément.
\newcommand*{\scalprod}[2]{\left\langle #1,#2\right\rangle}
\let\dual\ast

\newcommand*{\pardef}{\stackrel{\text{def}}{=}} % Par définition.
\newcommand*{\iffdefn}{\stackrel{\text{def}}{\iff}} % Par définition.
\newcommand*{\Defn}[1]{\emph{#1}} %
\newcommand*{\tensor}{\otimes}
\newcommand*{\pder}[2]{\frac{\partial #1}{\partial #2}}

\DeclareRobustCommand{\sfrac}[3][\mathrm]{\hspace{0.1em}%
  \raisebox{0.4ex}{$#1{\scriptstyle
#2}$}\hspace{-0.1em}/\hspace{-0.07em}%
  \mbox{$#1{\scriptstyle #3}$}}



% The following uses xstring in order to replace specific characters by %xx codes.
% See the table http://www.utf8-chartable.de/
\providecommand{\MakeUTFPerCent}[1]{% 
   \StrSubstitute{#1}({\%28}[\result]% 
   \expandafter\StrSubstitute\expandafter{\result}){\%29}[\result]% 
   \expandafter\StrSubstitute\expandafter{\result}{à}{\%C3\%A0}[\result]% 
   \expandafter\StrSubstitute\expandafter{\result}{â}{\%C3\%A2}[\result]% 
   \expandafter\StrSubstitute\expandafter{\result}{ç}{\%C3\%A7}[\result]% 
   \expandafter\StrSubstitute\expandafter{\result}{è}{\%C3\%A8}[\result]% 
   \expandafter\StrSubstitute\expandafter{\result}{é}{\%C3\%A9}[\result]% 
   \expandafter\StrSubstitute\expandafter{\result}{ê}{\%C3\%AA}[\result]% 
   \expandafter\StrSubstitute\expandafter{\result}{ù}{\%C3\%B9}[\result]% 
   \expandafter\StrSubstitute\expandafter{\result}{û}{\%C3\%BB}[\result]% 
   \expandafter\StrSubstitute\expandafter{\result}{ô}{\%C3\%B4}[\result]% 
   \expandafter\StrSubstitute\expandafter{\result}_{\_}[\result]% 
} 

   %\expandafter\StrSubstitute\expandafter{\result}{#}{\%23}[\result]%       If you want a # in the URL you still have to write \# in the source.

%------------------------
% Links to wikipedia.
%------------------------
% Typical use is
% \wikipedia{fr}{Norme_(mathématiques)}{Norme}
% It creates the link \href to the right page on wikipedia, replacing special characters by their respective %xx codes.
\providecommand{\wikipedia}[3]{% 
   \saveexpandmode\noexpandarg 
   \MakeUTFPerCent{#2}% 
   \restoreexpandmode 
   \href{http://#1.wikipedia.org/wiki/\result}{#3}% 
} 

\providecommand{\wikiversity}[3]{% 
   \saveexpandmode\noexpandarg 
   \MakeUTFPerCent{#2}% 
   \restoreexpandmode 
   \href{http://#1.wikiversity.org/wiki/\result}{#3}% 
} 


% 1410612643  :   L'option amssymb sert à éviter un conflit avec la commande \square de amssymb. Note que cette dernière n'est plus accessible.
%The change from SIunits to siunitx causes a too many math alphabet error.
% ! LaTeX Error: Too many math alphabets used in version normal.
% Thus we stick on the old SIunits



% 1 on utilise external et 0 on ne l'utilise pas
\newcounter{useexternal} \setcounter{useexternal}{1}
\ifthenelse{\value{useexternal}=1}{ \usetikzlibrary{external} \tikzexternalize[prefix=auto/pictures_tikz/] }{ \newcommand{\tikzsetnextfilename}[1]{} }

\makeindex
\makenomenclature


\begin{document}


\newcommand{\GitCommitHexsha}{\info{missing information}}       % Ceci est modifié par un plugin de lst_agreg.py

\emptyInputPath
\addInputPath{tex/front_back_matter}

% SCRIPT MARK -- GARDE MES NOTES
\selectlanguage{french}
% This is part of Mes notes de mathématique
% Copyright (c) 2011-2017
%   Laurent Claessens
% See the file fdl-1.3.txt for copying conditions.



\begin{center}
    Infos COVID19
\end{center}

Vu qu'il est indispendable d'être informé correctement, je propose ici une liste de vulgarisateurs scientifiques que je pense être de qualité.

\begin{itemize}
    \item e-penser explique le concept de «applatir la courbe»:  \url{https://www.youtube.com/watch?v=dp1thcnPbiM}
    \item science étonnante explique des généralités sur les épidémies \url{https://sciencetonnante.wordpress.com/2020/03/12/epidemie-nuage-radioactif-et-distanciation-sociale/}
    \item scienc4all fait un peu de mathématique \url{https://www.youtube.com/watch?v=mWkE7GXqOMY}
    \item Une vidéo commune à plein de youtubeurs scientifiques  \url{https://www.youtube.com/watch?v=tf3Z_rixaAM}
    \item Et bien entendu Wikipédia :
        \begin{itemize}
            \item Le virus : \url{https://fr.wikipedia.org/wiki/SARS-CoV-2}
            \item La maladie provoquée par le virus : \url{https://fr.wikipedia.org/wiki/Maladie_à_coronavirus_2019}
            \item La pandémie : \url{https://fr.wikipedia.org/wiki/Pandémie_de_maladie_à_coronavirus_de_2019-2020}
            \item La France : \url{https://fr.wikipedia.org/wiki/Pandémie_de_maladie_à_coronavirus_de_2020_en_France}
            \item La Belgique : \url{https://fr.wikipedia.org/wiki/Pandémie_de_maladie_à_coronavirus_de_2020_en_Belgique}
        \end{itemize}
\end{itemize}

Et maintenant \ldots place à la mathématique.

\newpage



\thispagestyle{empty}
\begin{center}
  \begin{minipage}{15cm}
    \hrule\par
    \vspace{2mm}
    \begin{center}
        \Huge \bfseries Le Frido \\  {\small Les quelques premières années de mathématiques}
    \normalsize
    \ifbool{isFrido}{%
        \url{http://laurent.claessens-donadello.eu/pdf/lefrido.pdf}
    }{}%
    \end{center}
    \hrule\par
  \end{minipage}
\end{center}

\vspace{2cm}

\begin{center}
    Contributeurs :\\
    Laurent \textsc{Claessens} \\
    Lilian \textsc{Besson} \\
    Bertrand \textsc{Desmons} \\

    \vspace{1cm}

    \today\\
    \texttt{\GitCommitHexsha}


\end{center}

\vfill

 \LogoEtLicence

 \newpage

% Créer une nouvelle branche git
% Copier tout dans un nouveau répertoire
% Supprimer la mise à jour automatique dans le scipt de mise en ligne.
% Supprimer le «Une version de ces notes est disponible dans la bibliothèque de l'agrégation»
% Ajouter ici l'ISBN. Pour les révisions, mettre un nouvel ISBN et indiquer que c'est une révision.
% Pour l'ISBN:
% Coder en dur la date (càd enlever \today)
% Comme c'est pour imprimer, regarder si c'est pas mieux d'enlever l'option ``oneside'' de la classe book.
% Pour l'URL donnée juste en dessous du titre, ajouter l'année : mes_notes.pdf --> mes_notes-2015.pdf
% supprimer la liste des développements possibles.

% http://www.bnf.fr/fr/professionnels/s_informer_obtenir_isbn/s.qu_est_ce_que_isbn.html

% Il faut écrire l'ISBN au verso de la page de titre, au bas de la dernière page de couverture et au bas de la dernière page de la jaquette des livres ;

% Imprimer quelques pages d'essai pour voir si les couleurs des liens passent bien. En particulier :
% - équation
% - théorèmes
% - notes en bas de page
% - bibliographie
% - URL
% - href
% Il y a le fichier test_couleur pour ça.


% De temps en temps, il faut renvoyer une nouvelle version à
% http://megamaths.perso.neuf.fr/

%\clearpage

% Pour avoir l'ISBN en dos de couverture, ajouter ceci :
%\clearpage
%\vfill
%\LogoEtLicence
%\clearpage


\notbool{isBook}{
    % Voir aussi le fichier 205_version_description.tex pour la version
% en anglais.

\thispagestyle{empty}

Plusieurs versions et extensions de ce document.
\begin{description}

    \item[La version courante]

        Vous trouverez une version dédiée à l'agrégation régulièrement mise à jour à l'adresse suivante :
        \begin{center}
            \url{http://laurent.claessens-donadello.eu/pdf/lefrido.pdf}
        \end{center}

    \item[Aux oraux d'agrégation]

        Une version est en vente en 4 volumes, voir la page dédiée
        \begin{center}
            \url{http://laurent.claessens-donadello.eu/frido.html}
        \end{center}
        ainsi que l'erratum :
        \begin{center}
            \url{https://github.com/LaurentClaessens/mazhe/blob/master/erratum.md}
        \end{center}

    \item[La version la plus complète]

        Une version plus complète, comprenant le Frido, des exercices ainsi que de la mathématique de niveau recherche :
        \begin{center}
            \url{http://laurent.claessens-donadello.eu/pdf/giulietta.pdf}
        \end{center}

    \item[Pour recompiler soi-même]
        Pour savoir comment recompiler ce document à l'identique, il faut lire
        \begin{center}
            \url{https://github.com/LaurentClaessens/mazhe/blob/master/COMPILATION_frido.md}
            \url{https://github.com/LaurentClaessens/mazhe/blob/master/COMPILATION_giulietta.md}
        \end{center}

\end{description}

    \newpage
}{}
\ifbool{isFrido}{
    \input{EtAgreg}
}{}

\newpage

% SCRIPT MARK -- GARDE MAZHE
\selectlanguage{english}
% This is part of Mes notes de mathématique
% Copyright (c) 2011-2013,2015, 2019
%   Laurent Claessens
% See the file fdl-1.3.txt for copying conditions.

\thispagestyle{empty}
\begin{center}
  \begin{minipage}{15cm}
    \hrule\par
    \vspace{2mm}
    \begin{center}
    \Huge \bfseries  Giulietta \\
    \large
    The mathematics book which wants to be collaborative
    \par
    \normalsize
    \url{http://laurent.claessens-donadello.eu/pdf/giulietta.pdf}
    \end{center}
    \hrule\par
  \end{minipage}
\end{center}

\vspace{2cm}

\begin{center}
    Laurent \textsc{Claessens}\\
    \today\\
    \texttt{\GitCommitHexsha}
\end{center}

\vfill

\LogoEtLicence
\clearpage

\input{205_version_description}


% SCRIPT MARK -- GARDE ENSEIGNEMENT
\selectlanguage{french}
\input{gardeEnseignement}

% SCRIPT MARK -- TOC

\newpage
\addcontentsline{toc}{chapter}{Thématique}
\setcounter{chapter}{-2}
% This is part of le Frido
% Copyright (c) 2016-2020
%   Laurent Claessens
% See the file fdl-1.3.txt for copying conditions.

%+++++++++++++++++++++++++++++++++++++++++++++++++++++++++++++++++++++++++++++++++++++++++++++++++++++++++++++++++++++++++++
\section*{Thèmes}
%+++++++++++++++++++++++++++++++++++++++++++++++++++++++++++++++++++++++++++++++++++++++++++++++++++++++++++++++++++++++++++

Ceci est une sorte d'index thématique.

% The macro '\InternalLink' writes in 'themestoc.tex' the lines like
% ~\ref {THTOC4} : méthode de Newton\\

% We open the file after the input (if before, we erase it) and
% we close at the end of this file.

\begin{multicols}{2}
\noindent
\ref {THTOC1} : Cardinalité\\
\ref {THTOC2} : tribu, algèbre de parties, \( \lambda \)-systèmes et co.\\
\ref {THTOC3} : théorie de la mesure\\
\ref {THTOC4} : intégration\\
\ref {THTOC5} : suites et séries\\
\ref {THTOC6} : polynôme de Taylor\\
\ref {THTOC7} : normes\\
\ref {THTOC8} : caractérisations séquentielles\\
\ref {THTOC9} : topologie produit\\
\ref {THTOC10} : espaces métriques, normés\\
\ref {THTOC11} : gaussienne\\
\ref {THTOC12} : compacts\\
\ref {THTOC13} : densité\\
\ref {THTOC14} : espaces de fonctions\\
\ref {THTOC15} : fonctions Lipschitz\\
\ref {THTOC16} : formule des accroissements finis\\
\ref {THTOC17} : limite et continuité\\
\ref {THTOC18} : différentiabilité\\
\ref {THTOC19} : points fixes\\
\ref {THTOC20} : théorèmes de Stokes, Green et compagnie\\
\ref {THTOC21} : permuter des limites\\
\ref {THTOC22} : applications continues et bornées\\
\ref {THTOC23} : inégalités\\
\ref {THTOC24} : connexité\\
\ref {THTOC25} : suite de Cauchy, espace complet\\
\ref {THTOC26} : application réciproque\\
\ref {THTOC27} : déduire la nullité d'une fonction depuis son intégrale\\
\ref {THTOC28} : équations différentielles\\
\ref {THTOC29} : injections\\
\ref {THTOC30} : logarithme\\
\ref {THTOC31} : inversion locale, fonction implicite\\
\ref {THTOC32} : convexité\\
\ref {THTOC33} : fonction puissance\\
\ref {THTOC34} : dualité\\
\ref {THTOC35} : opérations sur les distributions\\
\ref {THTOC36} : convolution\\
\ref {THTOC37} : séries de Fourier\\
\ref {THTOC38} : transformée de Fourier\\
\ref {THTOC39} : méthode de Newton\\
\ref {THTOC40} : méthodes de calcul\\
\ref {THTOC41} : espaces vectoriels\\
\ref {THTOC42} : définie positive\\
\ref {THTOC43} : norme matricielle, norme opérateur et rayon spectral\\
\ref {THTOC44} : série de matrices\\
\ref {THTOC45} : rang\\
\ref {THTOC46} : extension de corps et polynômes\\
\ref {THTOC47} : décomposition de matrices\\
\ref {THTOC48} : systèmes d'équations linéaires\\
\ref {THTOC49} : formes bilinéaires et quadratiques\\
\ref {THTOC50} : arithmétique modulo, théorème de Bézout\\
\ref {THTOC51} : polynômes\\
\ref {THTOC52} : zoologie de l'algèbre\\
\ref {THTOC53} : invariants de similitude\\
\ref {THTOC54} : réduction, diagonalisation\\
\ref {THTOC55} : endomorphismes cycliques\\
\ref {THTOC56} : déterminant\\
\ref {THTOC57} : polynôme d'endomorphismes\\
\ref {THTOC58} : exponentielle\\
\ref {THTOC59} : types d'anneaux\\
\ref {THTOC60} : sous-groupes\\
\ref {THTOC61} : groupe symétrique\\
\ref {THTOC62} : action de groupe\\
\ref {THTOC63} : classification de groupes\\
\ref {THTOC64} : produit semi-direct de groupes\\
\ref {THTOC65} : théorie des représentations\\
\ref {THTOC66} : isométries\\
\ref {THTOC67} : caractérisation de distributions en probabilités\\
\ref {THTOC68} : théorème central limite\\
\ref {THTOC69} : lemme de transfert\\
\ref {THTOC70} : probabilités et espérances conditionnelles\\
\ref {THTOC71} : dénombrements\\
\ref {THTOC72} : enveloppes\\
\ref {THTOC73} : équations diophantiennes\\
\ref {THTOC74} : techniques de calcul\\

\end{multicols}

\newwrite\themetoc
\immediate\openout\themetoc=themestoc.tex


% ATTENTION : il est très important que le titre «Thèmes» soit au haut d'une page et que le premier thème commence sur cette même page. Donc pas de texte trop long ici.
% La raison :
% Pour la division en volume, je prend le bloc 'thèmes' comme commençant à la page du premier. Cela est dû au fait que le titre n'est pas dans la TOC.

% Convention : les titres ne commencent pas par une majuscule
%  La raison : cela facilite les recherches dans le pdf (oui, je sais : on peut faire des recherches sans tenir compte des majuscules)

\InternalLinks{Cardinalité}
Le Frido ne définit pas la notion de nombre cardinal; ça nous mènerait trop loin. Au lieu de cela, nous allons nous contenter des notions d'équipotence, surpotence et subpotence, et démonter un certain nombre de résultats en utilisant sans retenue le lemme de Zorn.
\begin{enumerate}
    \item
        Définition d'équipotence, surpotence et subpotence, notations \( A\succ B\) et \( A\approx B\), définition \ref{DEFooXGXZooIgcBCg}.
    \item
    Toute partie d'une ensemble fini est finie, lemme \ref{LEMooTUIRooEXjfDY}.
\item
Si \( A\) est un ensemble fini ou dénombrable, alors il existe une surjection \( \eN\to A\), lemme \ref{LEMooSRZWooASgEfy}.
\item
    Si \( A\) est un ensemble infini et si \( f\colon A\to B\) est une application injective, alors \( f(A)\) est infini, lemme \ref{LEMooXPSQooRaSrxv}
\item
    Toute partie infinie de \( \eN\) est dénombrable, proposition \ref{PROPooOBKMooWEGCvM}
\item
    Une bijection \( \eN\to \eN\times \eN\), proposition \ref{PROPooLPKUooAlsYJg}.
\item
    Une décomposition de \( \eN\) en une infinité de parties équipotentes à \( \eN\), corollaire \ref{CORooNRPIooZPSmqa}.
\item
    Si il existe une surjection \( \eN\to A\), alors \( A\) est fini ou dénombrable, lemme \ref{LEMooDLWFooNAJbbq}.
\item
    Une union dénombrable d'ensembles finis ou dénombrables est finie ou dénombrable, proposition \ref{PROPooENTPooSPpmhY}
\item
    Tout ensemble infini contient une partie en bijection avec \( \eN\), proposition \ref{PROPooUIPAooCUEFme}.
\item
    Toute partie d'un ensemble fini est finie, et toute partie d'un ensemble dénombrable est finie ou dénombrable, proposition \ref{PropQEPoozLqOQ}.
\item
    Si \( A\succeq B\) et \( B\succeq A\), alors \( A\approx B\), théorème de Cantor-Schröder-Bernstein \ref{THOooRYZJooQcjlcl}
\item
    Le théorème de Cantor \ref{THOooJPNFooWSxUhd} dit qu'il n'existe pas de surjection d'un ensemble vers son ensemble des parties. On en déduit qu'il n'existe pas d'ensemble contenant tous les ensembles (corolaire \ref{CORooZMAOooPfJosM}).
\item 
    Si \( A\) est infini et si \( A\succeq B\), alors \( A\approx A\cup B\) par le lemme \ref{LEMooXMVDooIWLWis}.
\item
    Si \( S\) est un ensemble infini alors il existe une bijection \( \varphi\colon \{ 0,1 \}\times S\to S\), proposition \ref{PropVCSooMzmIX}.
\item
    Si \( A\) est infini, alors \( A\times \eN\approx A\), proposition \ref{PROPooFKBEooKXqujV}.
\item 
    Si \( A\) est infini et si \( B\prec A\), alors \( A\setminus B\approx A\), lemme \ref{LEMooIVCBooHWQiZB}.
\item
    Si \( A\) est infini, alors \( A\approx A\times A\), théorème \ref{THOooDGOVooRdURVi}.
\end{enumerate}

Il y a aussi des résultats de cardinalité autour des extensions de corps.
\begin{enumerate}
    \item
        Si \( \eK\) est un corps infini, alors \( \eK[X]\approx \eK\).
    \item
        Le théorème de Steinitz \ref{THOooEDQKooLEGlDv} affirme que tout corps admet une unique clôture algébrique. La preuve utilise pas mal de cardinalité ainsi que le lemme de Zorn.
\end{enumerate}

\InternalLinks{tribu, algèbre de parties, \( \lambda\)-systèmes et co.}  \label{INTooVDSCooHXLLKp}
    Il existe des centaines de notions de mesures et de classes de parties.
\begin{enumerate}
        \item
            Le plus souvent lorsque nous parlons de mesure est que nous parlons de mesure positive, définition~\ref{DefBTsgznn} sur un espace mesuré avec une tribu, définition~\ref{DefjRsGSy}.
        \item
            Une mesure extérieure est la définition~\ref{DefUMWoolmMaf}
        \item
            Une algèbre de partie : définition~\ref{DefTCUoogGDud}. Une mesure sur une algèbre de parties : définition~\ref{DefWUPHooEklLmR}. L'intérêt est que si on connait une mesure sur une algèbre de parties, elle se prolonge en une mesure sur la tribu engendrée par le théorème de prolongement de Hahn~\ref{ThoLCQoojiFfZ}.
        \item
            Un \( \lambda\)-système : définition~\ref{DefRECXooWwYgej}.
        \item
            Une mesure complexe : définition~\ref{DefGKHLooYjocEt}.
\end{enumerate}

En théorie de l'intégration, si \( X\) est une partie de \( \eR^n\), la convention est de considérer des fonctions
\begin{equation*}
    f\colon \big( X,\Lebesgue(X) \big)\to \big( \eR,\Borelien(\eR) \big).
\end{equation*}
Voir les points \ref{NORMooNFOMooYnaflN} et \ref{NORMooFZEDooIxSgLe} pour les conventions à ce propos.

À propos d'applications mesurables :
\begin{enumerate}
    \item
        Définition d'une application mesurable, définition \ref{DefQKjDSeC}.
    \item
        Une fonction continue est borélienne, théorème \ref{ThoJDOKooKaaiJh}.
\end{enumerate}


À propos de tribu induite:
\begin{enumerate}
    \item
        Définition \ref{DefDHTTooWNoKDP}.
    \item
        Les boréliens induits sont bien les boréliens de la topologie induite : \( \Borelien(Y)=\Borelien(X)_Y\), théorème \ref{ThoSVTHooChgvYa}.
\end{enumerate}

\InternalLinks{théorie de la mesure}       \label{THEMEooKLVRooEqecQk}
\begin{description}
    \item[Mesure] 
    À propos de mesure.
\begin{enumerate}
    \item
        Mesure positive, mesure finie et \( \sigma\)-finie, c'est la définition \ref{DefBTsgznn}.

    \item Le produit de tribus est donné par la définition~\ref{DefTribProfGfYTuR},     % Cette référence doit être vers le haut.
    \item
        Produit d'une mesure par une fonction, définition \ref{PropooVXPMooGSkyBo}.
    \item le produit d'espaces mesurés est donné par la définition~\ref{DefUMlBCAO}.     % Cette référence doit être vers le haut.
        \item
            Mesure de Lebesgue sur \( \eR\), définition~\ref{DefooYZSQooSOcyYN}.
        \item
            Une partie de \( \eR\) non mesurable au sens de Lebesgue : l'exemple \ref{EXooCZCFooRPgKjj}.
        \item
            Mesure de Lebesgue sur \( \eR^N\), définition~\ref{DEFooSWJNooCSFeTF}.
        \item
            Mesure à densité, définition~\ref{PropooVXPMooGSkyBo}.
\end{enumerate}
\item[Théorèmes d'approximation]
    Il est important de pouvoir approcher des fonctions continues ou \( L^p\) par des fonctions étagées, sinon on ne parvient pas à faire tourner la machine de l'intégration de Lebesgue.
    \begin{enumerate}
        \item
            Si \( (S,\tribA, \mu)\) est un espace mesuré et si \( f\colon S\to \mathopen[ 0 , +\infty \mathclose]\) est une fonction mesurable, le théorème fondamental d'approximation \ref{THOooXHIVooKUddLi} dit qu'il existe une suite croissante de fonctions étagées qui converge vers \( f\).
        \item
            Les fonctions simples sont denses dans \( L^p\), proposition \ref{PROPooUQUBooAWgNhm}.
        \item
            Encadrement d'un borélien \( A\) par un fermé \( F\) et un ouvert \( V\) par le lemme \ref{LEMooCGKXooYWjRwk} : \( F\subset A\subset V\) avec \( \mu(V\setminus F)<\epsilon\).
        \item
            Approximation \( L^p\) de la fonction caractéristique d'un borélien par une fonction continue par le théorème \ref{ThoAFXXcVa}.
    \end{enumerate}
\end{description}


\InternalLinks{intégration}     \label{THEMEooHINHooJaSYQW}

À propos d'intégration.
\begin{description}

\item[L'ordre dans lequel les choses sont faites]
\begin{itemize}
    \item 
        Nous commençons par considérer des fonctions \( f\colon \Omega\to \mathopen[ 0 , +\infty \mathclose]\) dans la définition \ref{DefTVOooleEst}.
    \item
        Nous donnerons ensuite quelques propriétés restreintes au fonctions à valeurs positives, par exemple
        \begin{enumerate}
            \item
                La convergence monotone \ref{ThoRRDooFUvEAN},
            \item
                Lemme de Fatou \ref{LemFatouUOQqyk}.
            \item 
                (presque) linéarité pour les fonctions positives, théorème \ref{ThoooCZCXooVvNcFD}.
        \end{enumerate}
        \item
            La définition pour les fonctions à valeurs dans \( \eR\) puis \( \eC\) est \ref{DefTCXooAstMYl}.
        \item
            Pour les fonctions à valeurs dans un espace vectoriel, c'est la définition \ref{PROPooOFSMooLhqOsc}.
\end{itemize}
    \item[Quelque résultats] 
\begin{enumerate}
    \item
        Intégrale associée à une mesure, définition~\ref{DefTVOooleEst}
\item
    L'existence d'une primitive pour toute fonction continue est le théorème~\ref{ThoEXXyooCLwgQg}.
\item
    La définition d'une primitive est la définition~\ref{DefXVMVooWhsfuI}.
\item
    Primitive et intégrale, proposition~\ref{PropEZFRsMj}.
\item
    Intégrale impropre, définition~\ref{DEFooINPOooWWObEz}.
\end{enumerate}
\item[Intégrale et mesure]
    \begin{enumerate}
        \item
            L'intégrale de la fonction \( 1\) donne la mesure : \( \int_B1d\mu=\mu(B)\), c'est le lemme \ref{LemooPJLNooVKrBhN}.
        \item
            Le théorème de Radon-Nikodym \ref{THOooEFVUooGKApaV} donne une densité pour certaines mesures.
        \item
            Le produit d'une mesure par une fonction donné par la définition \ref{PropooVXPMooGSkyBo} introduit aussi une densité : \( (w\cdot \mu)(A)=\int_Awd\mu\).
    \end{enumerate}

\item[Autre résultats]
\begin{enumerate}
    \item
        Si \( A,B\subset \Omega\) sont des parties disjointes, alors $\int_{A\cup B}f=\int_Af+\int_Bf$, proposition \ref{PropOPSCooVpzaBt}.
    \item
        La \( \sigma\)-additivité dénombrable, $\int_{\bigcup_iA_i}fd\mu=\sum_{i=0}^{\infty}\int_{A_i}fd\mu$ est dans les propositions \ref{PROPooTFOAooJBwmCV} et \ref{PROPooDWYNooWKJmEV}.
\end{enumerate}
\end{description}


\InternalLinks{suites et séries}

\begin{description}
    \item[Suites] 
        Les suites réelles sont en général dans la proposition \ref{PropLimiteSuiteNum} et ce qui s'ensuit. Cette proposition est souvent prise comme définition lorsque seules les suites réelles ne sont considérées.
        \begin{enumerate}
    \item
        Les suites adjacentes, c'est la définition \ref{DEFooDMZLooDtNPmu}. 
    \item
        Les séries alternées, théorème \ref{THOooOHANooHYfkII}. Il s'agit de dire que \( \sum_{k=0}^{\infty}(-1)^ka_k\) converge quand \( a_k\) est décroissante et tend vers zéro.
    \item
        Le concept de suite adjacente sert à étudier la série de Taylor de \( \ln(x+1)\), voir le lemme \ref{LEMooWMGGooRpAxBa} et ce qui l'entoure.
    \item
        La définition de la convergence absolue est la définition~\ref{DefVFUIXwU}.
            \item
                Une suite réelle croissante et majorée converge, proposition \ref{LemSuiteCrBorncv}.
            \item
                Toute suite dans un compact admet une sous-suite convergente, théorème \ref{THOooRDYOooJHLfGq}.
            \item
                Pour tout réel, il existe une suite croissante de rationnels qui y converge, proposition \ref{PropSLCUooUFgiSR}.
        \end{enumerate}
    \item[Calcul de suites]
        \begin{enumerate}
            \item
                Somme : \( x_n+y_n\to x+y\) est la proposition \ref{PROPooICZMooGfLdPc}.
        \end{enumerate}
    \item[Série] 
        Les séries sont en général dans la section \ref{SECooYCQBooSZNXhd}.
        \begin{enumerate}
    \item
        Quelques séries usuelles en \ref{SUBSECooDTYHooZjXXJf} : série harmonique, géométrique, de Riemann, et la mythique arithmético-géométrique.
        \begin{enumerate}
            \item
                La série est associative : \( \sum_k(a_k+b_k)=\sum_ka_k+\sum_kb_k\). C'est la proposition \ref{PROPooUEBWooUQBQvP}.
            \item
                La série harmonique diverge : \( \sum_k\frac{1}{ k }=\infty\), exemple \ref{EXooDVQZooEZGoiG}.
            \item
                La série géométrique : \( \sum_{k=0}^Nq^k=\frac{ 1-q^{N+1} }{ 1-q }\), exemple \ref{ExZMhWtJS}.
            \item
                Une autre cool série : \( \sum_{k=1}^N\frac{ 1 }{ k(k+1) }=\frac{ N }{ N+1 }\), lemme \ref{LEMooKDHPooPlFTIT}.
        \end{enumerate}
    \item
        Critère des séries alternées, théorème \ref{THOooOHANooHYfkII}.
    \item
        Convergence d'une série implique convergence vers zéro du terme général, proposition~\ref{PROPooYDFUooTGnYQg}.
        \end{enumerate}

    \item[Sommes infinies]
        En ce qui concerne les sommes finies, la notation \( \sum_{i=1}^N\) est définie en \ref{DEFooNEVNooJlmJOC}. Pour permuter les termes d'une somme avec un élément du groupe symétrique, nous avons la proposition \ref{PROPooJBQVooNqWErk}.

        Voici quelques résultats à propos de sommes infinies.% cette phrase est là pour le mot-clef ``somme infinie''.
        \begin{enumerate}
            \item
Une somme indexée par un ensemble quelconque est la définition~\ref{DefHYgkkA}.
    \item
        La définition de la somme d'une infinité de termes est donnée par la définition~\ref{DefGFHAaOL}.
    \item
        Une somme de termes positifs indexée par un ensemble indénombrable est toujours infinie par le lemme \ref{LEMooQIMGooOUpZjk}.
  \item
      si la série converge, on peut regrouper ses termes sans modifier la convergence ni la somme (associativité);
    Pour les sommes infinies l'associativité et la commutativité dans une série sont perdues. Néanmoins, il subsiste que
  \begin{enumerate}
  \item
      si la série converge absolument, on peut modifier l'ordre des termes sans modifier la convergence ni la somme (commutativité, proposition~\ref{PopriXWvIY}).
  \end{enumerate}
  \item Permuter une somme infinie avec une application linéaire : \( f(\sum_{i\in I}v_i)=\sum_{i\in I}f(v_i)\), c'est la proposition \ref{PROPooWLEDooJogXpQ}.
        \end{enumerate}
    \item[Fonction analytique]
        La fonction \( \sum_{k=0}^{\infty}a_nz^n\) est holomorphe dans son disque de convergence par la proposition \ref{PropSNMEooVgNqBP}.
\end{description}


\InternalLinks{polynôme de Taylor}

\begin{description}
    \item[Énoncés] 

        Il existe de nombreux énoncés du théorème de Taylor, et en particulier beaucoup de formules pour le reste.

    \begin{enumerate}
    \item
        Énoncé : théorème~\ref{ThoTaylor}.
    \item
        Une majoration du reste est dans le théorème \ref{THOooEUVEooXZJTRL}
    \item
        De classe \( C^2\) sur \( \eR^n\), proposition~\ref{PROPooTOXIooMMlghF}.
    \item
    Avec un reste donné par un point dans \( \mathopen] x , a \mathclose[\), proposition~\ref{PropResteTaylorc}.
        \item
            Avec reste intégral, proposition~\ref{PropAXaSClx} et théorème \ref{THOooDGCJooXKmFTT} pour le cas simple \( \eR\to \eR\).
        \item
            Le polynôme de Taylor généralise à l'utilisation de toutes les dérivées disponibles le résultat de développement limité donné par la proposition~\ref{PropUTenzfQ}.
        \item
            Pour les fonctions holomorphes, il y a le théorème~\ref{THOooSULFooHTLRPE} qui donne une série de Taylor sur un disque de convergence.
        \end{enumerate}

    \item[Utilisation]

        Des polynômes de Taylor sont utilisés pour démontrer des théorèmes par-ci par-là.

\begin{enumerate}
        \item
            Il est utilisé pour justifier la méthode de Newton autour de l'équation \eqref{EQooOPUBooYaznay}.
    \item
        On utilise pas mal de Taylor dans les résultats liant extrémum et différentielle/hessienne. Par exemple la proposition~\ref{PropoExtreRn}.
\end{enumerate}

\item[Quelques développements]

Voici quelques développements limités à savoir. Ils sont calculables en utilisant la formule de Taylor-Young (proposition~\ref{PropVDGooCexFwy}).
\begin{subequations}
    \begin{align*}
        e^x&=\sum_{k=0}^n\frac{ x^k }{ k! }+x^n\alpha(x)&\text{ordre } n, \text{proposition \ref{PROPooQBRGooAhGrvP}}\\
        \cos(x)&=\sum_{k=0}^p\frac{ (-1)^kx^{2k} }{ (2k)! }+x^{2p+1}\alpha(x)&\text{ordre } 2p+1,\text{proposition \ref{PROPooNPYXooTuwAHP}}\\
        \sin(x)&=\sum_{k=0}^p\frac{ (-1)^kx^{2k+1} }{ (2k+1)! }+x^{2p+2}\alpha(x)&\text{ordre } 2p+1,\text{proposition \ref{PROPooNPYXooTuwAHP}}\\
        \ln(1+x)&=\sum_{k=1}^n\frac{ (-1)^{k+1} }{ k }x^k+\alpha(x)x^n&\text{ordre }n,\text{proposition \ref{PROPooWCUEooJudkCV}}\\
        \ln(1+x)&=\sum_{k=1}^{\infty}\frac{ (-1)^{k+1} }{ k }x^k&\text{exact }\text{proposition \ref{PROPooKPBIooJdNsqX}}\\
        \ln(2)&=\sum_{k=1}^{\infty}\frac{ (-1)^{k+1} }{ k }&\text{exact }\text{proposition \ref{PROPooKPBIooJdNsqX}}\\
      (1+x)^l&=\sum_{k=0}^l\binom{ l }{ k }x^k&\text{exact si } l\text{ est entier.}\\
      (1+x)^{\alpha}&=1+\sum_{k=1}^n\frac{ \alpha(\alpha-1)\ldots(\alpha-k+1) }{ k! }x^k+x^n\alpha(x)&\text{ordre } n.
    \end{align*}
\end{subequations}
  Dans toutes ces formules, la fonction \( \alpha\colon \eR\to \eR\) vérifie \( \lim_{t\to 0} \alpha(t)=0\).

Le développement limité en $0$ d'une fonction paire ne contient que les puissances de $x$ d'exposant paire. Voir comme exemple le développement de la fonction cosinus.

\end{description}

\InternalLinks{normes}      \label{THEMEooUJVXooZdlmHj}

\begin{description}
    \item[Définition] Espace vectotiel normé : définition~\ref{DefNorme}.
    \item[Équivalence de norme]

        \begin{enumerate}
        \item
            Définition de l'équivalence de norme~\ref{DefEquivNorm}.
\item
    La proposition~\ref{PropLJEJooMOWPNi} sur l'équivalence des normes \( \| . \|_2\), \( \| . \|_1\) et \( \| . \|_{\infty}\)  dans \( \eR^n\).
\item
     En général pour les normes \( \| . \|_p\), il y a des inégalités dans \ref{THOooPPDPooJxTYIy} et \ref{CORooMBQMooWBAIIH}; voir aussi le thème \ref{THEMEooUJVXooZdlmHj}.
 \item
     La proposition \ref{PROPooQZTNooGACMlQ} donne l'inégalité \( \| x \|_q\leq n^{\frac{1}{ q }-\frac{1}{ p }}\| x \|_p\) dès que \( 0<q<p\).
\item
    Toutes les normes sur un espace vectoriel de dimension finie sont équivalentes par le théorème \ref{ThoNormesEquiv}.
\item
    Montrer que le problème \( a-b\) est stable dans l'exemple~\ref{ExooXJONooTAYZVc}.
\item
    La proposition~\ref{PROPooWZJBooTPLSZp} donnant \( \rho(A)\leq \| A \|\) utilise l'équivalence de toutes les normes sur un espace vectoriel de dimension finie (théorème \ref{ThoNormesEquiv}.).

        \end{enumerate}

    \item[Norme opérateur et d'algèbre] voir le thème~\ref{THEMEooOJJFooWMSAtL}.

\end{description}

\InternalLinks{caractérisations séquentielles}  
    \begin{enumerate}
        \item
            Fonction séquentiellement continue, définition \ref{DefENioICV}.
        \item
            La continuité implique la continuité séquentielle, proposition \ref{fContEstSeqCont} et corollaire \ref{PropFnContParSuite}.
        \item
            Pour des espaces métriques, la continuité séquentielle d'une fonction est équivalente à la continuité, proposition \ref{PropXIAQSXr}. Une version spéciale pour \( \eR^m\) est donnée par le théorème \ref{ThoLimSuite}.
    \end{enumerate}


\InternalLinks{topologie produit}       \label{THEMEooYRIWooDXZnhX}
    \begin{enumerate}
        \item
            La définition de la topologie produit est~\ref{DefIINHooAAjTdY}.
        \item
            Pour les espaces vectoriels normés, le produit est donné par la définition~\ref{DefFAJgTCE}.
        \item
            L'équivalence entre la topologie de la norme produit et la topologie produit est le lemme~\ref{DefFAJgTCE}.
        \item
            Quand \( V\) et \( W\) sont des espaces métriques, la topologie considérée sur \( V\times W\) est celle de la définition \ref{DefFAJgTCE}. C'est à la fois la topologie de la norme produit et la topologie produit.
        \item
            La convergence dans un espace vectoriel est si et seulement si il y a convergence composante par composante, proposition \ref{PROPooNRRIooCPesgO}.
        \end{enumerate}

\InternalLinks{espaces métriques, normés}
\begin{enumerate}
    \item
        Un espace métrique est un espace muni d'une distance, définition \ref{DefMVNVFsX}.
    \item
        La distance entre un point et un ensemble est la définition \ref{DEFooGNNUooFUZINs}.
    \item
        Le théorème-définition~\ref{ThoORdLYUu} donne la topologie sur un espace métrique en disant que les boules ouvertes sont une base de la topologie (définition \ref{DefQELfbBEyiB}).
    \item
        La définition de la convergence d'une suite est la définition~\ref{DefXSnbhZX}.
    \item
        Dans un espace vectoriel normé, une application est continue si et seulement si elle est bornée, proposition~\ref{PROPooQZYVooYJVlBd}.
    \item
        Un espace vectoriel topologique qui possède en tout point une base dénombrable de topologie accepte une distance, théorème \ref{THOooAGBXooZnvQLK}.
\end{enumerate}


\InternalLinks{gaussienne}
\begin{enumerate}
    \item
        Le calcul de l'intégrale
        \begin{equation*}
            \int_{\eR} e^{-x^2}dx=\sqrt{\pi }
        \end{equation*}
        est fait de deux façons dans l'exemple~\ref{EXooLUFAooGcxFUW}. Dans les deux cas, le théorème de Fubini~\ref{ThoFubinioYLtPI} est utilisé.
    \item
        Le lemme~\ref{LEMooPAAJooCsoyAJ} calcule la transformée de Fourier de $ g_{\epsilon}(x)=  e^{-\epsilon\| x \|^2}$ qui donne $\hat g_{\epsilon}(\xi)=\left( \frac{ \pi }{ \epsilon } \right)^{d/2} e^{-\| \xi \|^2/4\epsilon}$.
    \item
        Le lemme~\ref{LEMooTDWSooSBJXdv} donne une suite régularisante à base de gaussienne.
    \item
        Elle est utilisée pour régulariser une intégrale dans la preuve de la formule d'inversion de Fourier~\ref{PROPooLWTJooReGlaN}
\end{enumerate}


\InternalLinks{compacts}        \label{THEMEooQQBHooLcqoKB}
    \begin{description}

        \item[Propriétés générales]

            Quelques propriétés de compacts.

                \begin{enumerate}
    \item
        La définition d'un ensemble compact est la définition~\ref{DefJJVsEqs}.
    \item
        Les compacts sont les fermés bornés par le théorème~\ref{ThoXTEooxFmdI}.
    \item
        Le théorème de Borel-Lebesgue \ref{ThoBOrelLebesgue} dit qu'un intervalle\footnote{Définition \ref{DefEYAooMYYTz}.} de \( \eR\) est compact si et seulement si il est de la forme \( \mathopen[ a , b \mathclose]\).
    \item
        Théorème des fermés emboîtés dans le cas compact, corolaire \ref{CORooQABLooMPSUBf}. À ne pas confondre avec celui dans le cas des espaces métrique, théorème \ref{ThoCQAcZxX}.
    \item
        L'image d'un compact par une fonction continue est un compact, théorème~\ref{ThoImCompCotComp}.
    \item
        Suites dans un compact
        \begin{enumerate}
            \item
                Toute suite dans un compact admet une sous-suite convergente, théorème \ref{THOooRDYOooJHLfGq}.
            \item
                Dans \( \eR^n\), toute suite dans un compact admet une sous-suite convergente, théorème \ref{ThoBolzanoWeierstrassRn}. La démonstration de ce théorèma est non seulement plus compliquée que le cas général, mais utilise en plus le cas dans \( \eR\); lequel cas n'est pas démontré de façon directe dans le Frido.
            \item
                Un espace métrique est compact si et seulement si toute suite contient une sous-suite convergente. C'est le théorème de Bolzano-Weierstrass~\ref{ThoBWFTXAZNH}. La démonstration de ce théorème est indépendante.
        \end{enumerate}
    \item
        Une fonction continue sur un compact est bornée et atteint ses bornes, théorème~\ref{ThoWeirstrassRn}.
    \item
        Une fonction continue sur un compact y est uniformément continue, théorème de Heine \ref{PROPooBWUFooYhMlDp}.
                \end{enumerate}

        \item[Produits de compacts]
            À propos de produits de compacts. C'est un compact dans tous les cas métriques\quext{Si vous connaissez des exemples non métriques de produits de compacts qui ne sont pas compacts, écrivez-moi.}.
    \begin{enumerate}
    \item
        Les produits d'espaces métriques compacts sont compacts. Il s'agit du théorème de Tykhonov que nous verrons ce résultat dans les cas suivants.
        \begin{itemize}
    \item
         \( \eR\), lemme~\ref{LemCKBooXkwkte}.
    \item
        Produit fini d'espaces métriques compacts, théorème~\ref{THOIYmxXuu}.
    \item
        Produit dénombrable d'espaces métriques compacts, théorème~\ref{ThoKKBooNaZgoO}.
        \end{itemize}
    \end{enumerate}
    \end{description}

\InternalLinks{densité}         \label{THEooPUIIooLDPUuq}
\begin{enumerate}
    \item
        Densité de \( \eQ\) dans \( \eR\), proposition \ref{PropooUHNZooOUYIkn}.
    \item
        Densité des polynômes dans \( \Big( C^0\big( \mathopen[ 0 , 1 \mathclose] \big),\| . \|_{\infty} \Big)\), théorème de Bernstein~\ref{ThoDJIvrty}.
    \item
        Densité des polynômes dans \( \big( C^0(I),\| . \|_{\infty} \big)\) lorsque \( I=\mathopen[ a , b \mathclose]\), corolaire \ref{CORooCWLMooWwCOAP}.
    \item
        Densité de \( \swD(\eR^d)\) dans \( L^p(\eR^d)\) pour \( 1\leq p<\infty\), théorème~\ref{ThoILGYXhX}.
    \item
        Densité de \( \swS(\eR^d)\) dans l'espace de Sobolev \( H^s(\eR^d)\), proposition~\ref{PROPooMKAFooKDNTbO}.

    \item
        Densité de \( \swD(\eR^d)\) dans l'espace de Sobolev \( H^s(\eR^d)\), proposition~\ref{PROPooLIQJooKpWtnV}.

        Cela est utilisé pour le théorème de trace~\ref{THOooXEJZooBKtXBW}.
    \item
        Les applications étagées dans les applications mesurables (qui plus est avec limite croissante), théorème fondamental d'approximation~\ref{LempTBaUw}.
    \item
        Les fonctions continues à support compact dans \( L^2(I)\), théorème~\ref{ThoJsBKir}.
    \item
        Les polynômes trigonométriques sont denses dans \( L^p(S^1)\) pour \( 1\leq p<\infty\). Deux démonstrations indépendantes par le théorème~\ref{ThoDPTwimI} et le théorème~\ref{ThoQGPSSJq}.
\end{enumerate}
Les densités sont bien entendu utilisées pour prouver des formules sur un espace en sachant qu'elles sont vraies sur une partie dense. Mais également pour étendre une application définie seulement sur une partie dense. C'est par exemple ce qui est fait pour définir la trace \( \gamma_0\) sur les espaces de Sobolev \( H^s(\eR^d)\) en utilisant le théorème d'extension~\ref{PropTTiRgAq}.

Comme presque tous les théorèmes importants, le théorème de Stone-Weierstrass possède de nombreuses formulations à divers degrés de généralité.
\begin{itemize}
    \item Le lemme~\ref{LemYdYLXb} le donne pour la racine carré.
    \item Le théorème~\ref{ThoGddfas} donne la densité des polynômes dans les fonctions continues sur un compact.
    \item Le théorème~\ref{THOooMDILooGPXbTW} est une généralisation qui donne la densité uniforme d'une sous-algèbre de \( C(X,\eR)\) dès que \( X\) sépare les points.
    \item Le théorème \ref{ThoWmAzSMF} donne le même résultat pour la densité dans \( C(X,\eC)\).
    \item Le lemme~\ref{LemXGYaRlC} est une version pour les polynômes trigonométriques.
    \item
        Le lemme~\ref{LemYdYLXb} est un cas particulier du
        théorème~\ref{ThoGddfas}, mais nous en donnons une démonstration indépendante afin d'isoler la preuve
de la généralisation~\ref{ThoWmAzSMF}.
Une version pour les polynômes trigonométriques sera donnée dans le lemme~\ref{LemXGYaRlC}.
\end{itemize}
Le théorème de Stone-Weierstrass est utilisé, entre autres nombreuses choses, pour prouver la densité des polynômes trigonométriques dans les fonctions continues sur \( S^1\), voir la proposition \ref{PROPooTGBHooXGhdPR}.


\InternalLinks{espaces de fonctions}                \label{THEMooNMYKooVVeGTU}

En ce qui concerne les densités, voir le thème~\ref{THEooPUIIooLDPUuq}.


\begin{description}
    \item[Topologie]

        Les espaces de fonctions sont souvent munis de topologies définies par des semi-normes.

        \begin{enumerate}
            \item
                La topologie des semi-normes est la définition~\ref{DefPNXlwmi}.
            \item
                La définition~\ref{DefFGGCooTYgmYf} donne les topologies sur \(  C^{\infty}(\Omega)\), \( \swD(K)\) et \( \swD(\Omega)\).
            \item
                La topologie \( *\)-faible sur \( \swD'(\Omega)\) est donnée par la définition~\ref{DefASmjVaT}.
        \end{enumerate}

    \item[L'espace \( { L^2\big( \mathopen[ 0 , 2\pi \mathclose] \big) } \)]

        C'est un espace très important, entre autres parce qu'il est de Hilbert et est bien adapté à la transformée de Fourier.

        \begin{enumerate}
        \item
            Un rappel de la construction en \ref{NORMooUEIEooYtlFse}.
            \item
                Le produit scalaire \( \langle f, g\rangle \) est donné en \eqref{EQooBFKDooMkCZOt} et la base trigonométrique est \eqref{EQooKMYOooLZCNap}.
            \item
                La densité des polynômes trigonométriques dans \( L^p(S^2)\) est le théorème~\ref{ThoQGPSSJq} ou le théorème~\ref{ThoDPTwimI}, au choix.
            \item
                Une conséquence de cette densité est que le système trigonométrique est une base hilbertienne de \( L^2\) par le lemme~\ref{LEMooBJDQooLVPczR}.
        \end{enumerate}

            L'espace \( L^2\) est discuté en analyse fonctionnelle, dans la section \ref{SECooEVZSooLtLhUm} et les suivantes parce que l'étude de \( L^2\) utilise entre autres l'inégalité de Hölder~\ref{ProptYqspT}.

        Le fait que \( L^2\) soit une espace de Hilbert est utilisé dans la preuve du théorème de représentation de Riesz~\ref{PropOAVooYZSodR}.

\end{description}


\InternalLinks{fonctions Lipschitz}
    \begin{enumerate}
    \item
        Définition :~\ref{DEFooQHVEooDbYKmz}.
    \item
        La notion de Lipschitz est utilisée pour définir la stabilité d'un problème, définition~\ref{DEFooYIFAooSJbMkC}.
    \end{enumerate}


\InternalLinks{formule des accroissements finis}
    Il en existe plusieurs formes :
    \begin{enumerate}
        \item
            Une version adaptée aux espaces de dimension finie est le théorème~\ref{val_medio_2}.
        \item
        L'existence de \( c\in \mathopen] a , b \mathclose[\) tel que
            \begin{equation}
                f'(c)=\frac{ f(b)-f(a) }{ b-a }
            \end{equation}
            est le théorème des accroissements finis proprement dit. C'est le théorème \ref{ThoAccFinis}.
        \item
            Au premier ordre, proposition \ref{PropUTenzfQ}.
        \item
            Pour les fonctions \( \eR\to \eR\) en le théorème~\ref{ThoAccFinis}.
        \item
            Une généralisation pour les intervalles non bornés : théorème~\ref{THOooRIIBooOjkzMa}.
        \item
            Espaces vectoriels normés, théorème~\ref{ThoNAKKght}
    \end{enumerate}

\InternalLinks{limite et continuité}    \label{THEMEooGVCCooHBrNNd}

\begin{enumerate}
    \item
        Limite d'une fonction en un point : définition \ref{DefYNVoWBx}. Il n'y a pas unicité en général comme le montre l'exemple \ref{EXooSHKAooZQEVLB} dans un espace non séparé.
    \item
        La proposition~\ref{PropRBCiHbz} donne l'unicité de la limite dans le cas des espaces duaux pour la topologie \( *\)-faible. La proposition~\ref{PropFObayrf} nous dira qu'il y a unicité dès que l'espace d'arrivée est séparé.
    \item
        Définition de la continuité d'une fonction en un point et sur une partie de l'espace de départ : définiton~\ref{DefOLNtrxB}.
    \item
        Continuité sur une partie si et seulement si continue en chaque point, c'est le théorème~\ref{ThoESCaraB}.
    \item
        Voir l'exemple~\ref{EXooKREUooLeuIlv} traité en détail.
    \item
        La fonction \( f(x,y)=x+y\) est continue, lemme \ref{LEMooGKIPooWgpFTB}.
\end{enumerate}

\InternalLinks{différentiabilité}
\begin{enumerate}
    \item
        La différentielle est définie en général pour des espaces vectoriels normés par la proposition \ref{DefDifferentiellePta}
    \item
        Nous parlons de différentielle en dimension finie et donnons une interprétation géométrique en~\ref{SEBSECooLPRQooJRQCFL}.
    \item
        La recherche d'extrémums d'une fonction sur \( \eR^n\) passe par la seconde différentielle, proposition~\ref{PropoExtreRn}.
    \item
        Lien entre différentielle seconde (hessienne) et convexité en la proposition~\ref{PROPooBMIRooFkQSAb} et le corolaire~\ref{CORooMBQMooWBAIIH}.
    \item
        Une fonction est de classe \( C^1\) si et seulement si ses dérivées partielles sont continues, théorème \ref{THOooBEAOooBdvOdr}.
    \item
        Une fonction est \( C^n\) si et seulement si ses dérivées partielles sont \( C^{n-1}\), c'est le théorème \ref{THOooPZTAooTASBhZ}.
    \item
        La différentielle est liée aux dérivées partielles par les formules données au lemme~\ref{LemdfaSurLesPartielles}
	\begin{equation*}
        df_a(u)=\frac{ \partial f }{ \partial u }(a)=\Dsdd{ f(a+tu) }{t}{0}=\sum_{i=1}^mu_i\frac{ \partial f }{ \partial x_i }(a)=\nabla f(a)\cdot u.
	\end{equation*}
\end{enumerate}

\InternalLinks{points fixes}        \label{THEMEooWAYJooUSnmMh}
    \begin{enumerate}
\item
    Il y a plusieurs théorèmes de points fixes.
    \begin{description}
        \item[Théorème de Picard]~\ref{ThoEPVkCL} donne un point fixe comme limite d'itérés d'une fonction Lipschitz. Il aura pour conséquence le théorème de Cauchy-Lipschitz~\ref{ThokUUlgU}, l'équation de Fredholm, théorème~\ref{ThoagJPZJ} et le théorème d'inversion locale dans le cas des espaces de Banach~\ref{ThoXWpzqCn}.
    \item[Théorème de Brouwer] qui donne un point fixe pour une application d'une boule vers elle-même. Nous allons donner plusieurs versions et preuves.
            \begin{enumerate}
                \item
                    Dans \( \eR^n\) en version \( C^{\infty}\) via le théorème de Stokes, proposition~\ref{PropDRpYwv}.
                \item
                    Dans \( \eR^n\) en version continue, en s'appuyant sur le cas \( C^{\infty}\) et en faisant un passage à la limite, théorème~\ref{ThoRGjGdO}.
                \item
                    Dans \( \eR^2\) via l'homotopie, théorème~\ref{ThoLVViheK}. Oui, c'est très loin. Et c'est normal parce que ça va utiliser la formule de l'indice qui est de l'analyse complexe\footnote{On aime bien parce que ça ne demande pas Stokes, mais quand même hein, c'est pas gratos non plus.}.
            \end{enumerate}
        \item[Théorème de Markov-Kakutani]\ref{ThoeJCdMP} qui donne un point fixe à une application continue d'un convexe fermé borné dans lui-même. Ce théorème donnera la mesure de Haar~\ref{ThoBZBooOTxqcI} sur les groupes compacts.
        \item[Théorème de Schauder] C'est une version valable en dimension infinie du théorème de Brouwer. Théorème \ref{ThovHJXIU} 
    \end{description}

\item Pour les équations différentielles
    \begin{enumerate}
        \item
            Le théorème de Schauder a pour conséquence le théorème de Cauchy-Arzela~\ref{ThoHNBooUipgPX} pour les équations différentielles.
        \item
            Le théorème de Schauder~\ref{ThovHJXIU} permet de démontrer une version du théorème de Cauchy-Lipschitz (théorème~\ref{ThokUUlgU}) sans la condition Lipschitz, mais alors sans unicité de la solution. Notons que de ce point de vue nous sommes dans la même situation que la différence entre le théorème de Brouwer et celui de Picard : hors hypothèse de type «contraction», point d'unicité.
    \end{enumerate}
\item
    En calcul numérique
    \begin{itemize}
        \item
            La convergence d'une méthode de point fixe est donnée par la proposition~\ref{PROPooRPHKooLnPCVJ}.
        \item
            La convergence quadratique de la méthode de Newton est donnée par le théorème~\ref{THOooDOVSooWsAFkx}.
        \item
            En calcul numérique, section~\ref{SECooWUVTooMhmvaW}
        \item
            Méthode de Newton comme méthode de point fixe, sous-section~\ref{SUBSECooIBLNooTujslO}.
    \end{itemize}

\item
    D'autres utilisations de points fixes.
\begin{itemize}
    \item
        Processus de Galton-Watson, théorème~\ref{ThoJZnAOA}.
    \item
        Dans le théorème de Max-Milgram~\ref{THOooLLUXooHyqmVL}, le théorème de Picard est utilisé.
\end{itemize}
\end{enumerate}


\InternalLinks{théorèmes de Stokes, Green et compagnie}
    \begin{enumerate}
        \item
            Forme générale, théorème~\ref{ThoATsPuzF}.
        \item
            Rotationnel et circulation, théorème~\ref{THOooIRYTooFEyxif}.
        \end{enumerate}
        Le théorème de Stokes peut être utilisé pour montrer le théorème de Brower, proposition~\ref{PropDRpYwv}.



\InternalLinks{permuter des limites}
\begin{description}
    \item[Fonctions définies par une intégrale]
        Les théorèmes sur les fonctions définies par une intégrale, section~\ref{SecCHwnBDj}. Nous avons entre autres
        \begin{enumerate}
            \item
                \( \partial_i\int_Bf=\int_B\partial_if\), avec \( B\) compact, proposition~\ref{PropDerrSSIntegraleDSD}.
            \item
                Si \( f\) est majorée par une fonction ne dépendant pas de \( x\), nous avons le théorème~\ref{ThoKnuSNd} pour la continuité de \( x\mapsto \int_{\Omega}f(x,\omega)d\mu(\omega)\).
            \item
                Pour la fonction $F(x)=\int_{\Omega}f(x,\omega)d\mu(\omega)$, nous avons la dérivation sous l'intégrale par la formule de Leibnitz
                \begin{equation}
                    F'(a)=\int_{\Omega}\frac{ \partial f }{ \partial x }(a,\omega)d\mu(\omega)
                \end{equation}
                démontrée en le théorème \ref{ThoMWpRKYp}.
            \item
                Si l'intégrale est uniformément convergente, nous avons le théorème~\ref{ThotexmgE} qui donne la continuité de $F(x)=\int_{\Omega}f(x,\omega)d\mu(\omega)$.
            \item
                Pour dériver \( \int_Bg(t,z)dt\) avec \( B\) compact dans \( \eR\) et \( g\colon \eR\times \eC\to \eC\), il faut aller voir la proposition~\ref{PROPooZCLYooUaSMWA}.
            \item
                En ce qui concerne le \( x\) dans la borne, le théorème \ref{PropEZFRsMj} lie primitive et intégrale en montrant que \( F(x)=\int_a^xf(t)dt\) est une primitive de \( f\) (sous certaines conditions). Le théorème fondamental de l'analyse \ref{ThoRWXooTqHGbC} en est une conséquence.
        \end{enumerate}
    \item[Convergence monotone]
        Théorème de la convergence monotone, théorème~\ref{ThoRRDooFUvEAN}.
    \item[Fubini]
        Le théorème de Fubini permet non seulement de permuter des intégrales, mais également des sommes parce que ces dernières peuvent être vues comme des intégrales sur \( \eN\) muni de la tribu des parties et de la mesure de comptage\footnote{Mesure de comptage, définition \ref{DEFooILJRooByDzhs}.}. Nous utilisons cette technique pour permute une somme et une intégrale dans l'équation \eqref{EQooWOLOooFHSrsx}.
    \item
        L'utilisation de Fubini pour permuter des intégrales (sur deux variables différentes) ou deux sommes est expliquée dans \ref{NORMooKIRJooPvyPWQ}. 

        C'est par exemple utilisé pour permuter deux sommes dans le cadre des chaines de Markov en \ref{LEMooZIEPooXHGnvy}.
\begin{itemize}
    \item
        le théorème de Fubini-Tonelli~\ref{ThoWTMSthY} demande que la fonction soit mesurable et positive;
    \item
        le théorème de Fubini~\ref{ThoFubinioYLtPI} demande que la fonction soit intégrable (mais pas spécialement positive);
    \item
        le corolaire~\ref{CorTKZKwP} demande l'intégrabilité de la valeur absolue des intégrales partielles pour déduire que la fonction elle-même est intégrable.
\end{itemize}

\item[Limite et dérivées, différentielle]
    \begin{enumerate}
        \item
            Permuter limite et dérivée, théorème \ref{THOooXZQCooSRteSr}.
        \item
 Permuter limite et dérivées partielles, théorème \ref{ThoSerUnifDerr}.
        \item
            Permuter limite et différentielle, théorème \ref{ThoLDpRmXQ}.
    \end{enumerate}
    Quelques remarques sur les techniques de démonstration.
    \begin{enumerate}
        \item
            Le résultat fondamental \ref{THOooXZQCooSRteSr} est démontré sans recourir à des intégrales. Une preuve alternative, plus courte, avec des intégrales est donnée en \ref{NORMALooGYUEooKrYjyz}.
        \item
            Les résultats un peu plus élaborés \ref{ThoSerUnifDerr} et \ref{ThoLDpRmXQ} sont prouvés avec des intégrales, mais devraient pouvoir être adaptés.
    \end{enumerate}
\item[Somme et dérivée]
    Permuter somme et différentielle, théorème \ref{ThoLDpRmXQ}.
\item[Limite et mesure]
    Une mesure n'est pas toujours une limite, mais la définition d'une mesure positive sur un espace mesurable parle de permuter limite et mesure : définition \ref{DefBTsgznn}\ref{ItemQFjtOjXiii}.

\end{description}


\InternalLinks{applications continues et bornées}       \label{THEMEooYCBUooEnFdUg}
\begin{enumerate}
    \item
        Une application linéaire non continue : exemple~\ref{ExHKsIelG} de \( e_k\mapsto ke_k\). Les dérivées partielles sont calculées en \eqref{EQooWNLOooJNRUMQ}.
    \item
        La dérivation sur les polynômes (exemple \ref{EXooDMVJooAJywMU}) donne un autre exemple d'application linéaire non continue.
    \item
        Une application linéaire est bornée si et seulement si elle est continue, proposition~\ref{PROPooQZYVooYJVlBd}.
\end{enumerate}


\InternalLinks{inégalités}
\begin{description}
    \item[Inégalité de Jensen]
        \begin{enumerate}
            \item
                Une version discrète pour \( f\big( \sum_i\lambda_ix_i \big)\), la proposition~\ref{PropXIBooLxTkhU}.
            \item
                Une version intégrale pour \( f\big( \int \alpha d\mu \big)\), la proposition~\ref{PropXISooBxdaLk}.
            \item
                Une version pour l'espérance conditionnelle, la proposition~\ref{PropABtKbBo}.
        \end{enumerate}
    \item[Inégalité pour les normes $ \ell^p$]
        \begin{enumerate}
            \item
                Hölder pour \( L^p\): \( \| fg \|_1\leq \| f \|_p\| g \|_q\), proposition \ref{ProptYqspT}.
            \item
                Hölder pour \( \ell^p\): \( \| x \|_q\leq n^{\frac{1}{ q }-\frac{1}{ p }}\| x \|_p\), proposition \ref{PROPooQZTNooGACMlQ}.
        \end{enumerate}
    \item[Inégalité de Minkowsky]
        \begin{enumerate}
            \item
                Pour une forme quadratique\footnote{Définition \ref{DefBSIoouvuKR}.} \( q\) sur \( \eR^n\) nous avons $\sqrt{q(x+y)}\leq\sqrt{q(x)}+\sqrt{q(y)}$. Proposition~\ref{PropACHooLtsMUL}.
            \item
                Si \( 1\leq p<\infty\) et si \( f,g\in L^p(\Omega,\tribA,\mu)\) alors \(  \| f+g \|_p\leq \| f \|_p+\| g \|_p\). Proposition~\ref{PropInegMinkKUpRHg}.
            \item
                L'inégalité de Minkowsky sous forme intégrale s'écrit sous forme déballée
                \begin{equation*}
                    \left[ \int_X\Big( \int_Y| f(x,y) |d\nu(y) \Big)^pd\mu(x) \right]^{1/p}\leq \int_Y\Big( \int_X| f(x,y) |^pd\mu(x) \Big)^{1/p}d\nu(y).
                \end{equation*}
                ou sous forme compacte
                \begin{equation*}
                    \left\|   x\mapsto\int_Y f(x,y)d\nu(y)   \right\|_p\leq \int_Y  \| f_y \|_pd\nu(y)
                \end{equation*}
        \end{enumerate}
    \item[Transformée de Fourier]
                Pour tout \( f\in L^1(\eR^n)\) nous avons \( \| \hat f \|_{\infty}\leq \| f \|_1\), lemme~\ref{LEMooCBPTooYlcbrR}.
    \item[Inégalité des normes]
        Inégalité de normes : si \( f\in L^p\) et \( g\in L^1\), alors \( \| f*g \|_p\leq \| f \|_p\| g \|_1\), proposition~\ref{PROPooDMMCooPTuQuS}.

\end{description}


\InternalLinks{connexité}
    \begin{enumerate}
        \item
            Définition~\ref{DefIRKNooJJlmiD}
        \item
            L'image d'un connexe par une fonction continue est connexe, proposition \ref{PropGWMVzqb}.
        \item
            Connexité par arcs, définition \ref{DEFooOXVCooBizpgK}.
        \item
            Une partie de \( \eR^2\) qui est connexe, mais pas connexe par arcs, proposition \ref{PROPooVXDNooPZYKPr}.
        \item
            Une partie de \( \eR\) est connexe si et seulement si elle est un intervalle, proposition \ref{PropInterssiConn}.
        \item
            Le groupe \( \SL(n,\eK)\) est connexe par arcs : proposition~\ref{PROPooALQCooLZCKrH}.
        \item
            Le groupe \( \GL(n,\eC)\) est connexe par arcs : proposition~\ref{PROPooVJNIooMByUJQ}.
        \item
            Le groupe \( \GL(n,\eC)\) est connexe par arcs, proposition~\ref{PROPooVJNIooMByUJQ}.
        \item
            Le groupe \( \GL(n,\eR)\) a exactement deux composantes connexes par arcs, proposition~\ref{PROPooBIYQooWLndSW}.
        \item
            Le groupe \( \gO(n,\eR)\) n'est pas connexe, lemme~\ref{LEMooIPOVooZJyNoH}.
        \item
            Les groupes \( \gU(n)\) et \( \SU(n)\) sont connexes par arcs, lemme~\ref{LEMooQMXHooZQozMK}.
        \item
            Pour tout \( n\geq 2\), le groupe \( \SO(n)\) est connexe, le groupe \( \gO(n)\) a deux composantes connexes, proposition \ref{THOooYQFNooPaYmaP}.
        \item
            Connexité des formes quadratiques de signature donnée, proposition~\ref{PropNPbnsMd}.
        \end{enumerate}

\InternalLinks{suite de Cauchy, espace complet}     \label{THMooOCXTooWenIJE}

Nous parlons d'espaces topologiques complets. À ne pas confondre avec un espace mesuré complet, définition~\ref{DefBWAoomQZcI}.

\begin{enumerate}
    \item
        Corps complet : définition~\ref{DefKCGBooLRNdJf}\ref{ITEMooKZZYooDaidGU}, espace métrique complet : définition~\ref{DEFooHBAVooKmqerL}.
    \item
        La définition~\ref{DEFooXOYSooSPTRTn} donne la notion de suite de Cauchy dans un espace métrique.
    \item
        La définition~\ref{DefZSnlbPc} donne la notion de suite de \( \tau\)-Cauchy dans un espace vectoriel topologique.
    \item
        Deux espaces métriques (avec une distance) peuvent être isomorphes en tant qu'espaces topologiques, mais ne pas avoir les mêmes suites de Cauchy, exemple~\ref{EXooNMNVooXyJSDm}.
    \item
        La proposition~\ref{PropooUEEOooLeIImr} donne l'équivalence entre les suites de Cauchy et les suites \( \tau\)-Cauchy dans le cas des espaces vectoriels topologiques \emph{normés}.
    \item
        L'exemple~\ref{EXooNMNVooXyJSDm} est un exemple pire que simplement une suite de Cauchy qui ne converge pas. Le problème de convergence de cette suite n'est pas simplement que la limite n'est pas dans l'espace; c'est que la suite de Cauchy donnée ne convergerait même pas dans \( \eR\).
    \item
        Le théorème~\ref{ThoKHTQJXZ} est un théorème de complétion d'un espace métrique.
    \item
        Dans \( \eR\), une suite est convergente si et seulement si elle est de Cauchy, théorème \ref{THOooNULFooYUqQYo}\ref{ITEMooUUFCooIVtGgz}.
    \item
        Toute suite convergente dans un espace métrique est de Cauchy, proposition \ref{PROPooZZNWooHghltd}.
\end{enumerate}

Quelques espaces qui sont complets sont listés ci-dessous. Attention : la complétude est bien une propriété de la métrique; le même ensemble peut être complet pour une distance et pas pour une autre. Souvent, cependant la distance à considérer est donnée par  le contexte.
\begin{multicols}{2}
    \begin{enumerate}
        \item
            Les réels \( \eR\), théorème~\ref{THOooNULFooYUqQYo}.
        \item
            Un espace vectoriel normé sur un corps complet est complet, proposition~\ref{PROPooGJDTooXOoYfw}.
        \item
            La proposition~\ref{PropSYMEZGU} donne quelques espaces complets. Soit \( X\) un espace topologique métrique \( (Y,d)\) un espace espace métrique complet. Alors les espaces
    \begin{enumerate}
        \item
            \( \big( C^0_b(X,Y),\| . \|_{\infty} \big)\)
        \item
            \( \big( C^0_0(X,Y),\| . \|_{\infty} \big)\)
        \item
            \( \big( C^k_0(X,Y),\| . \|_{\infty} \big)\)
    \end{enumerate}
    sont complets.
\item
    Le lemme~\ref{LemdLKKnd} dit que \( \big( C^0(A,B),\| . \|_{\infty}\big)\) est complet dès que \( A\) est compact et \( B\) est complet.

\item
    L'espace \( \swD(K)\) est complet tant pour la topologie des semi-normes que pour la topologie métrique (qui sont les mêmes). C'est la proposition~\ref{PropQAEVcTi}.
\item
    L'espace \( \swS(\Omega)\) est complet et métrisable par la proposition~\ref{PropIIAcyDq}.
\item
    L'espace \( L^p(\Omega,\tribA,\mu)\) par le théorème \ref{ThoUYBDWQX}.
    \end{enumerate}
\end{multicols}

    La limite uniforme d'une suite de fonctions dérivables n'est pas spécialement dérivable. Même si les fonctions sont de classe \(  C^{\infty}\), la limite n'est pas spécialement mieux que continue. En effet, le théorème de Stone-Weierstrass~\ref{ThoGddfas} nous dit que les polynômes (qui sont \(  C^{\infty}\)) sont denses dans les fonctions continues sur un compact pour la norme uniforme. Vous ne pouvez donc pas espérer que \( \big( C^p(X,Y),\| . \|_{\infty} \big)\) soit complet en général.



\InternalLinks{application réciproque}
\begin{enumerate}
    \item
        Définition~\ref{DEFooTRGYooRxORpY}.
    \item
        Dans le cas des réels, des exemples sont donnés en~\ref{EXooCWYHooLEciVj}.
    \item
        Continuité, proposition~\ref{PropIntContMOnIvCont}.
    \item
        Théorème de la bijection~\ref{ThoKBRooQKXThd} (qui contient aussi de la continuité).
    \item
        Dérivabilité, proposition~\ref{PropMRBooXnnDLq}.
\end{enumerate}


\InternalLinks{déduire la nullité d'une fonction depuis son intégrale}
Des résultats qui disent que si \( \int f=0\) c'est que \( f=0\) dans un sens ou dans un autre.
\begin{enumerate}
    \item
        Il y a le lemme~\ref{Lemfobnwt} qui dit ça.
    \item
        Un lemme du genre dans \( L^2\) existe aussi pour \( \int f\varphi=0\) pour tout \( \varphi\). C'est le lemme~\ref{LemDQEKNNf}.
    \item
        Et encore un pour \( L^p\) dans la proposition~\ref{PropUKLZZZh}.
    \item
        Si \( \int f\chi=0\) pour tout \( \chi\) à support compact alors \( f=0\) presque partout, proposition~\ref{PropAAjSURG}.
    \item
        La proposition~\ref{PropRERZooYcEchc} donne \( f=0\) dans \( L^p\) lorsque \( \int fg=0\) pour tout \( g\in L^q\).
    \item
        Une fonction \( h\in C^{\infty}_c(I)\) admet une primitive dans \(  C^{\infty}_c(I)\) si et seulement si \( \int_Ih=0\). Théorème~\ref{PropHFWNpRb}.
\end{enumerate}


\InternalLinks{équations différentielles}
L'utilisation des théorèmes de point fixe pour l'existence de solutions à des équations différentielles est fait dans le chapitre sur les points fixes.
\begin{enumerate}
    \item
            Le théorème de Schauder a pour conséquence le théorème de Cauchy-Arzela~\ref{ThoHNBooUipgPX} pour les équations différentielles.
        \item
            Le théorème de Schauder~\ref{ThovHJXIU} permet de démontrer une version du théorème de Cauchy-Lipschitz (théorème~\ref{ThokUUlgU}) sans la condition Lipschitz
        \item
            Le théorème de Cauchy-Lipschitz~\ref{ThokUUlgU} est utilisé à plusieurs endroits :
            \begin{itemize}
                \item
                    Pour calculer la transformée de Fourier de \(  e^{-x^2/2}\) dans le lemme~\ref{LEMooPAAJooCsoyAJ}.
            \end{itemize}
    \item
        Théorème de stabilité de Lyapunov~\ref{ThoBSEJooIcdHYp}.
    \item
        Le système proie-prédateur de Lotka-Volterra~\ref{ThoJHCLooHjeCvT}
    \item
        Équation de Schrödinger, théorème~\ref{ThoLDmNnBR}.
    \item
        L'équation \( (x-x_0)^{\alpha}u=0\) pour \( u\in\swD'(\eR)\), théorème~\ref{ThoRDUXooQBlLNb}.
    \item
        La proposition~\ref{PropMYskGa} donne un résultat sur \( y''+qy=0\) à partir d'une hypothèse de croissance.
    \item
        Équation de Hill \( y''+qy=0\), proposition~\ref{PropGJCZcjR}.
\end{enumerate}


\InternalLinks{injections}
\begin{enumerate}
        \item
            L'espace de Sobolev \( H^1(I)\) s'injecte de façon compacte dans \( C^0(\bar I)\), proposition~\ref{ThoESIyxfU}.
        \item
            L'espace de Sobolev \( H^1(I)\) s'injecte de façon continue dans \( L^2(I)\), proposition~\ref{ThoESIyxfU}.
        \item
            L'espace \( L^2(\Omega)\) s'injecte continument dans \( \swD'(\Omega)\) (les distributions), proposition~\ref{PROPooYAJSooMSwVOm}.
\end{enumerate}

\InternalLinks{logarithme}
\begin{enumerate}
    \item
    Le logarithme pour les réels strictement positifs \( \ln\colon \mathopen] 0 , \infty \mathclose[\to \eR\) est donné en la définition~\ref{DEFooELGOooGiZQjt}.
    \item
        Les principales propriétés sont dans la proposition \ref{PROPooLAOWooEYvXmI} : \( \ln(xy)=\ln(x)+\ln(y)\) etc.
    \item
        La proposition \ref{PROPooKPBIooJdNsqX} donne la série
        \begin{equation}
            \ln(1+x)=\sum_{k=1}^{\infty}\frac{ (-1)^{k+1} }{ k }x^k.
        \end{equation}
    \item
        L'exemple \ref{EXooYMEEooMGpUNM} donne l'encadrement \( 0.644\leq \ln(2)\leq 0.846\).
    \item
        La proposition~\ref{PropKKdmnkD} dit que toute matrice complexe admet un logarithme. En particulier une série explicite est donnée pour le logarithme d'un bloc de Jordan.
    \item
        Sur les complexes, le logarithme \( \ln \colon \eC^*\to \eC\) est la définition~\ref{DEFooWDYNooYIXVMC}. Attention : ce n'est pas la seule définition possible.
    \item
        La série harmonique diverge à vitesse logarithmique, et la série des inverses des nombres premiers, c'est encore plus lent : théorème~\ref{ThonfVruT}.
\end{enumerate}

\InternalLinks{inversion locale, fonction implicite}
       \begin{description}
           \item[Des énoncés]
               Il existe plusieurs énoncés à différent niveaux de généralité.
               \begin{enumerate}
    \item Inversion locale dans \( \eR^n\) : théorème~\ref{THOooQGGWooPBRNEX}. Pour un Banach c'est le théorème~\ref{ThoXWpzqCn}.
    \item
        Fonction implicite dans un Banach : théorème~\ref{ThoAcaWho}.
               \end{enumerate}
           \item[Des utilisations]
               \begin{enumerate}

    \item
        Utilisé pour montrer que le flot d'une équation différentielle est un \( C^p\)-difféomorphisme local, voir~\ref{NORMooWEWVooXbGmfE}. % position 1051229132
    \item
        Pour le théorème de Von Neumann~\ref{ThoOBriEoe}.
               \end{enumerate}
       \end{description}

\InternalLinks{convexité}

L'essentiel des résultats sur les fonctions convexes sont dans la section~\ref{SECooVZWWooUjxXYi}. On a surtout :
\begin{enumerate}
    \item
        Définition des fonctions convexes :~\ref{DefVQXRJQz} et~\ref{DEFooKCFPooLwKAsS} en dimension supérieure.
    \item
        En termes de différentielles,~\ref{PROPooYNNHooSHLvHp} pour la différentielle première et~\ref{CORooMBQMooWBAIIH} pour la hessienne.
    \item
        Une courbe paramétrée convexe est la définition~\ref{DEFooVQODooJSNYLw}.
    \item
        L'enveloppe convexe d'une courbe fermée simple et convexe :~\ref{PROPooWZITooTFiWsi}.
    \item
        Courbure et convexité d'une courbe paramétrée : section~\ref{SUBSECooNJOLooYuGRjA}.
    \item
        Une courbe paramétrée convexe est localement le graphe d'une fonction convexe par le lemme~\ref{LEMooGEVEooHxPTMO}.
    \item
        La convexité est utilisée dans la méthode du gradient à pas optimal de la proposition~\ref{PropSOOooGoMOxG}.
\end{enumerate}

En termes de parties convexes, on : 
\begin{enumerate}
    \item
        Définition \ref{DEFooQQEOooAFKbcQ} d'une partie convexe d'un espace vectoriel.
    \item
        Une boule est convexe, proposition \ref{PROPooUQLUooDQfYLT}.
\end{enumerate}


\InternalLinks{fonction puissance}      \label{THEMEooBSBLooWcaQnR}

Il y a beaucoup de choses à dire\ldots

\begin{description}
    \item[Définition] 
        Nous considérons, pour \( a>0\), la fonction \( g_a\colon \eR\to \eR\) donnée par \( g_a(x)=a^x\). La définition de cette fonction se fait en de nombreuses étapes.
\begin{enumerate}
    \item
        \( a^n\) pour \( n\in \eN\) en la définition \ref{DEFooGVSFooFVLtNo}.
    \item 
        \( a^n\) pour \( n\in \eZ\) en la définition \ref{DEFooKEBIooZtPkac}.
    \item
        \( a^{1/n}\) pour \( n\in \eZ\) en la définition \ref{DEFooJWQLooWkOBxQ}.
    \item
        \( a^q\) pour \( q\in \eQ\) en la définition \ref{DEFooJWQLooWkOBxQ}.
    \item
        \( \sqrt[n]{ x }\) en la définition \ref{DEFooPOELooPouwtD}.
    \item
        La fonction \( g_a\) est Cauchy-continue sur \( \eQ\), c'est la proposition \ref{PROPooQRFSooVzYdJM}.
    \item
        \( a^x\) pour \( a>0\) et \( x\in \eR\) en la définition \ref{DEFooOJMKooJgcCtq}.
    \item
        \( a^z\) pour \( a>0\) et \( z\in \eC\) en la définition \ref{DEFooRBTDooNLcWGj}.
\end{enumerate}

\item[Quelques propriétés]
\begin{enumerate}
    \item
        Pour tout \( q\in \eQ\), il y a un \( \sqrt{ q }\) dans \( \eR\), proposition \ref{PROPooUHKFooVKmpte}.
    \item
        Pour \( a>0\) et \( x,y\in \eR\) nous avons $a^xa^y=a^{x+y}$, proposition \ref{PROPooVADRooLCLOzP}.
    \item
        Si \( a>0\) et \( x,y\in \eR\) nous avons \( (a^x)^y=(a^y)^x=a^{xy}\) par la proposition \ref{PROPooDWZKooNwXsdV}.
    \item
        La fonction puissance est strictement croissante (en ses deux arguments), proposition \ref{PROPooRXLNooWkPGsO}.
    \item
        La formule \( a^{-x}=1/a^x\) est la proposition \ref{PROPooVADRooLCLOzP}\ref{ITEMooSCJBooNVJZah}.
\end{enumerate}

\item[Dérivation]
    Comme toutes les choses sur la fonction puissance, les preuves sont assez différentes selon que l'on parle de \( a^x\) ou de \( x^a\).
\begin{enumerate}
    \item
        La fonction puissance est strictement croissante, proposition \ref{PROPooRXLNooWkPGsO}
    \item
        La fonction \( a^x\) est dérivable et sa dérivée vérifie \( g_a'(x)=g_a(x)g_a'(0)\), proposition \ref{PROPooMXCDooBffXbl}.
    \item
        La formule de dérivation pour \( x\mapsto x^q\) avec \( q\in \eQ\) est la proposition \ref{PROPooSGLGooIgzque}. 
    \item
        La dérivation de \( x\mapsto x^{\alpha}\) avec \( \alpha\in \eR\) est la proposition \ref{PROPooKIASooGngEDh}. Si elle est tellement loin, c'est parce qu'elle nécessite de permuter une limite de fonctions avec une dérivée.
    \item
        Pour la formule générale de dérivation de \( x\mapsto a^x\) demande de savoir les logarithmes (proposition \ref{PROPooKUULooKSEULJ}).
\end{enumerate}

\item[L'équation fonctionnelle]
    L'exponentielle et plus généralement la fonction puissance \( g_a(x)=a^x\) peut être introduite au moyen d'une équation fonctionnelle au lieu de l'équation différentielle usuelle. Cette fameuse équation fonctionnelle est
    \begin{equation}
        f(x+y)=f(x)f(y)
    \end{equation}
    en la définition \ref{DEFooPJKMooOfZzgy}.
\begin{enumerate}
    \item
        L'équivalence entre l'équation fonctionnelle et l'équation différentielle est donnée par la proposition \ref{PROPooGBUPooWtWaFI}.
    \item
        La fonction \( g_a(x)=a^x\) vérifie l'équation fonctionnelle \( g_a(x+y)=g_a(x)g_a(y)\) et les conséquences. C'est la définition \ref{DEFooPJKMooOfZzgy} et les choses qui suivent.
    \item
        L'équation fonctionnelle pour une fonction continue \( f\colon \eR\to S^1\) est traitée dans la proposition \ref{PROPooVJLYooOzfWCd}.
\end{enumerate}
\end{description}

Une définition alternative de la fonction puissance serait de poser directement
\begin{equation*}
    a^x=e^{x\ln(a)}.
\end{equation*}
De là les propriétés se déduisent facilement. Dans cette approche, les choses se mettent dans l'ordre suivant :
\begin{itemize}
    \item Définir \( \exp(x)\) par sa série pour tout \( x\).
    \item Démontrer que \( \exp(q)=\exp(1)^q\) pour tout rationnel \( q\) (première partie de la proposition \ref{PropCELWooLBSYmS}).
    \item Définir \( e=\exp(1)\).
    \item Définir, pour \( x\) irrationnel, \( a^x=\exp(x\ln(a))\).
    \item Prouver que \( e^x=\exp(x)\) pour tout \( x\).
\end{itemize}


\InternalLinks{dualité}     \label{THEMEooULGFooPscFJC}

Ne pas confondre dual algébrique et dual topologique d'un espace vectoriel.

\begin{enumerate}
    \item
        Définition de la base duale \ref{DEFooTMSEooZFtsqa}.
    \item
        Base préduale (existencec, unicité) : proposition \ref{PROPooDBPGooPagbEB}.
\end{enumerate}

\begin{description}
    \item[Dual topologique et algébrique]
        Ils sont définis par~\ref{DefJPGSHpn}. Le dual algébrique est l'ensemble des formes linéaires, et le dual topologique ne considère que les formes linéaires continues (en dimension infinie, les applications linéaires ne sont pas toutes continues).
    \item[Topologie]
        Une topologie possible sur le dual d'un espace vectoriel topologique est celle \( *\)-faible de la définition~\ref{DefHUelCDD}.

        Nous comparons les topologies faibles et de la norme en la section~\ref{SECooKOJNooQVawFY}.
    \item[Théorèmes de dualité]
        Quelques théorèmes établissent des dualités entre des espaces courants.
\begin{enumerate}
    \item
        Le théorème de représentation de Riesz~\ref{ThoQgTovL} pour les espaces de Hilbert.
    \item
        La proposition~\ref{PropOAVooYZSodR} pour les espaces \( L^p\big( \mathopen[ 0 , 1 \mathclose] \big)\) avec \( 1<p<2\).
    \item
        Le théorème de représentation de Riesz~\ref{ThoLPQPooPWBXuv} pour les espaces \( L^p\) en général.
\end{enumerate}
Tous ces théorèmes donnent la dualité par l'application \( \Phi_x=\langle x, .\rangle \).

\end{description}


\InternalLinks{opérations sur les distributions}
\begin{enumerate}
    \item
        Convolution d'une distribution par une fonction, définition par l'équation \eqref{EQooOUXKooGHDSzL}.
    \item
        Dérivation d'une distribution, proposition-définition~\ref{PropKJLrfSX}.
    \item
        Produit d'une distribution par une fonction, définition~\ref{DefZVRNooDXAoTU}.
\end{enumerate}


\InternalLinks{convolution}
\begin{enumerate}
    \item
        Définition \ref{DEFooHHCMooHzfStu}, et principales propriétés sur \( L^1(\eR)\) dans le théorème \ref{THOooMLNMooQfksn}.
    \item
        Inégalité de normes : si \( f\in L^p\) et \( g\in L^1\), alors \( \| f*g \|_p\leq \| f \|_p\| g \|_1\), proposition~\ref{PROPooDMMCooPTuQuS}.
    \item
        \( \varphi\in L^1(\eR)\) et \( \psi\in\swS(\eR)\), alors \( \varphi * \psi\in \swS(\eR)\), proposition~\ref{PROPooUNFYooYdbSbJ}.
    \item
        Les suites régularisantes : \( \lim_{n\to \infty} \rho_n*f=f\) dans la proposition~\ref{PROPooYUVUooMiOktf}.
    \item
        Convolution d'une distribution par une fonction, définition par l'équation \eqref{EQooOUXKooGHDSzL}.
\end{enumerate}

\InternalLinks{séries de Fourier}       \label{THMooHWEBooTMInve}
\begin{itemize}
    \item Formule sommatoire de Poisson, proposition~\ref{ProprPbkoQ}.
    \item Inégalité isopérimétrique, théorème~\ref{ThoIXyctPo}.
    \item Fonction continue et périodique dont la série de Fourier ne converge pas, proposition~\ref{PropREkHdol}.

    \item
Nous allons montrer la convergence de \( \sum_{k\in \eZ}c_k(f) e^{inx}\) vers \( f(x)\) dans divers cas :
\begin{enumerate}
    \item
        Si \( f\) est continue et périodique, convergence au sens de Cesaro, théorème de Fejèr~\ref{ThoJFqczow}.
    \item
        Convergence au sens \( L^2\Big( \mathopen[ 0 , 2\pi \mathclose] \Big)\) dans le théorème~\ref{ThoYDKZLyv}.
    \item
        Si \( f\) est continue, périodique et si sa série de Fourier converge uniformément, théorème \ref{PropmrLfGt}.
    \item
        Si \( f\) est périodique et la série des coefficients converge absolument pour tout \( x\), proposition~\ref{PropSgvPab}.
    \item
        Si \( f\) est périodique et de classe \( C^1\), théorème~\ref{ThozHXraQ}.
\end{enumerate}
Il est cependant faux de croire que la continuité et la périodicité suffisent à obtenir une convergence, comme le montre dans la proposition~\ref{PropREkHdol}.
\end{itemize}


\InternalLinks{transformée de Fourier}
\begin{enumerate}
    \item
        Définition sur \( L^1\), définition~\ref{DEFooRIXGooECoIbx}.
    \item
        La transformée de Fourier d'une fonction \( L^1(\eR^d)\) est continue, proposition~\ref{PropJvNfj}.
    \item
    L'espace de Schwartz est stable par transformée de Fourier. L'application $\TF\colon \swS(\eR^d)\to \swS(\eR^d)$ est continue. Proposition ~\ref{PropKPsjyzT}
\item
    L'application \( \TF\colon \swS(\eR^d)\to \swS(\eR^d)\) est une bijection. Formule d'inversion, proposition \ref{PROPooLWTJooReGlaN}.
\end{enumerate}


\InternalLinks{méthode de Newton}
    \begin{enumerate}
        \item
            Nous parlons un petit peu de méthode de Newton en dimension \( 1\) dans~\ref{SECooIKXNooACLljs}.
        \item
            La méthode de Newton fonctionne bien avec les fonctions convexes par la proposition~\ref{PROPooVTSAooAtSLeI}.
        \item
            La méthode de Newton en dimension $n$ est le théorème~\ref{ThoHGpGwXk}.
       \item
            Un intervalle de convergence autour de \( \alpha\) s'obtient par majoration de \( | g' |\), proposition~\ref{PROPooRPHKooLnPCVJ}.
       \item
           Un intervalle de convergence quadratique s'obtient par majoration de \( | g'' |\), théorème~\ref{THOooDOVSooWsAFkx}.
       \item
           En calcul numérique, section~\ref{SECooIKXNooACLljs}.
       \end{enumerate}


\InternalLinks{méthodes de calcul}
\begin{enumerate}
    \item
        Théorème de Rothstein-Trager~\ref{ThoXJFatfu}.
    \item
        Algorithme des facteurs invariants~\ref{PropPDfCqee}.
    \item
        Méthode de Newton, théorème~\ref{ThoHGpGwXk}
    \item
        Calcul d'intégrale par suite équirépartie~\ref{PropDMvPDc}.
\end{enumerate}

\InternalLinks{espaces vectoriels} 

\begin{enumerate}
    \item
        Existence d'une base. Pour un espace vectoriel quelconque, proposition \ref{PROPooHDCEooMhDjPi}.
    \item
        Théorème de la base incomplète. Pour un espace vectoriel quelconque, théorème \ref{THOooOQLQooHqEeDK}.
\end{enumerate}

\InternalLinks{définie positive}        \label{THEMEooYEVLooWotqMY}
\begin{enumerate}
    \item
        Une application bilinéaire est définie positive lorsque \( g(u,u)\geq 0\) et \( g(u,u)=0\) si et seulement si \( u=0\) est la définition~\ref{DEFooJIAQooZkBtTy}.
    \item
        Un opérateur ou une matrice est défini positif si toutes ses valeurs propres sont positives, c'est la définition~\ref{DefAWAooCMPuVM}.
    \item
        Pour une matrice symétrique, définie positive si et seulement si \( \langle Ax, x\rangle >0\) pour tout \( x\). C'est le lemme~\ref{LemWZFSooYvksjw}.
    \item
        Une application linéaire est définie positive si et seulement si sa matrice associée l'est. C'est la proposition~\ref{PROPooUAAFooEGVDRC}.
\end{enumerate}
Remarque : nous ne définissons pas la notion de matrice définie positive dans le cas d'une matrice non symétrique.

\InternalLinks{norme matricielle, norme opérateur et rayon spectral}     \label{THEMEooOJJFooWMSAtL}

Quelques définitions
\begin{enumerate}
    \item
        Définition de la norme opérateur : définition \ref{DefNFYUooBZCPTr}.
    \item
        Définition du rayon spectral~\ref{DEFooEAUKooSsjqaL}.
\end{enumerate}

    La norme matricielle n'est rien d'autre que la norme opérateur de l'application linéaire donnée par la matrice.

    \begin{enumerate}
        \item
            Lien entre norme matricielle et rayon spectral, le théorème~\ref{THOooNDQSooOUWQrK} assure que $\|A\|_2=\sqrt{\rho(A{^t}A)}$.
        \item
            Lien entre valeurs propres et norme opérateur : le lemme~\ref{LEMooNESTooVvUEOv} pour les matrices symétriques strictement définies positives donne \( \| A \|_2=\lambda_{max}\).
        \item
            Pour toute norme algébrique nous avons \( \rho(A)\leq \| A \|\), proposition~\ref{PROPooWZJBooTPLSZp}.
        \item
            Dans le cadre du conditionnement de matrice. Voir en particulier la proposition~\ref{PROPooNUAUooIbVgcN} qui utilise le théorème~\ref{THOooNDQSooOUWQrK}.
        \item
            Rayon spectral et convergence de méthode itérative, proposition~\ref{PROPooAQSWooSTXDCO}.
    \end{enumerate}

    Pour la norme opérateur nous avons les résultats suivants.

    \begin{enumerate}
        \item
            La majoration \( \| Au \|\leq \| A \|\| u \|\) est le lemme \ref{LEMooIBLEooLJczmu}.
        \item
            Définition d'une algèbre :~\ref{DefAEbnJqI} et pour une norme d'algèbre :~\ref{DefJWRWQue}.
        \item
            La norme opérateur est une norme d'algèbre, lemme \ref{LEMooFITMooBBBWGI}.
        \item
            Pour des espaces vectoriels normés, être borné est équivalent à être continu : proposition~\ref{PROPooQZYVooYJVlBd}.
        \item
            Le lemme à propos d'exponentielle de matrice~\ref{LemQEARooLRXEef} donne :
            \begin{equation*}
                \|  e^{tA} \|\leq P\big( | t | \big)\sum_{i=1}^r e^{t\real(\lambda_i)}.
            \end{equation*}
    \end{enumerate}

    La norme opérateur est utilisée pour donner une norme sur les produits tensoriels, définition \ref{DEFooEXXNooMgIpSV}.

    Une norme matricielle donne une topologie. Il y a donc également des liens entre rayon spectral et convergence de série. Dans cette optique, pour les séries de matrices, voir le thème~\ref{THEMEooPQKDooTAVKFH}.


\InternalLinks{série de matrices}       \label{THEMEooPQKDooTAVKFH}

\begin{enumerate}
    \item
        Rayon spectral et norme opérateur : thème~\ref{THEMEooOJJFooWMSAtL}.
    \item
        Exponentielle de matrices : thème~\ref{THEMEooKXSGooCsQNoY}.
    \item
        Série entière de matrices : section~\ref{secEVnZXgf}.
    \item
        Pour la série \( \sum_kA^k=(1-A)^{-1}\).
        \begin{itemize}
            \item Pour un espace de Banach : proposition~\ref{PropQAjqUNp}.
            \item Pour les matrices nilpotentes : proposition~\ref{PROPooWTFWooXHlmhp}.
            \item En lien avec le rayon spectral (si et seulement si \( \rho(A)<1\)) dans la proposition~\ref{THOooMNLGooKETwhh}.
            \item Le lemme~\ref{LemPQFDooGUPBvF} parle de la série entière \( \sum_{n\in \eN}z^{nk}=(1-z^k)^{-1}\).
        \end{itemize}
        Cette série est utilisée entre autres dans la proposition~\ref{PROPooZDMQooIZAbKK} pour prouver qu'une M-matrice irréductible vérifie \( A^{-1}>0\).
\end{enumerate}


\InternalLinks{rang}
    \begin{enumerate}
        \item Définition pour une application linéaire : \ref{DefALUAooSPcmyK}, pour une matrice : \ref{DEFooCSGXooFRzLRj}. L'équivalence est la proposition \ref{PROPooCINLooFGNtwS}.
        \item Le théorème du rang, théorème~\ref{ThoGkkffA}
        \item Pour une applicaton liénaire entre deux espaces vectoriels de même dimension finie, il est équivalent d'être injectif, surjectif ou bijectif, c'est le corollaire \ref{CORooCCXHooALmxKk}.
        \item Pour prouver que des matrices sont équivalentes et pour les mettre sous des formes canoniques, nous avons le lemme \ref{LemZMxxnfM} et son corolaire \ref{CorGOUYooErfOIe}.
        \item Tout hyperplan de \( \eM(n,\eK)\) coupe \( \GL(n,\eK)\), corolaire~\ref{CorGOUYooErfOIe}. Cela utilise la forme canonique sus-mentionnée.
        \item Le lien entre application duale et orthogonal de la proposition~\ref{PropWOPIooBHFDdP} utilise la notion de rang.
        \item Le lemme \ref{LEMooDFFDooJTQkRu} parle de commutant et utilise la notion de rang. Ce lemme sert à prouver diverses conditions équivalentes à être un endomorphisme cyclique dans le théorème \ref{THOooGLMSooYewNxW}.
        \end{enumerate}


\InternalLinks{extension de corps et polynômes} \label{THEMEooZYKFooQXhiPD}
    \begin{enumerate}
        \item
            Définition d'une extension de corps~\ref{DEFooFLJJooGJYDOe}.
        \item
            Pour l'extension du corps de base d'un espace vectoriel et les propriétés d'extension des applications linéaires, voir la section~\ref{SECooAUOWooNdYTZf}.
        \item
            Extension de corps de base et similitude d'application linéaire (ou de matrices, c'est la même chose), théorème~\ref{THOooHUFBooReKZWG}.
        \item
            Extension de corps de base et cyclicité des applications linéaires, corolaire~\ref{CORooAKQEooSliXPp}.
        \item
            À propos d'extensions de \( \eQ\), le lemme~\ref{LemSoXCQH}.
        \item
            Corps de rupture : définition~\ref{DEFooVALTooDJJmJv} existence par la proposition~\ref{PROPooUBIIooGZQyeE}. Il n'y a pas unicité.
        \item
            Corps de décomposition : définition~\ref{DEFooEKGZooSkvbum}. Attention : le plus souvent nous parlons de corps de décomposition d'un seul polynôme. Cette définition est un peu surfaite. Existence par la proposition~\ref{PROPooDPOYooFHcqkU} qui le donne même comme extension par toutes les racines, et unicité à isomorphisme près par le théorème~\ref{THOooQVKWooZAAYxK}, énoncé de façon plus simple (mais pas indépendante !) en la proposition~\ref{PropTMkfyM}.
        \item
            Si \( \eK\) est algébrique clos et si \( \alpha\colon \eK\to \eL\) est une extension algébrique, alors \( \alpha(\eK)=\eL\) par lemme \ref{LEMooYVHKooWhewKp}.
    \end{enumerate}

Un trio de résultats d'enfer est :
\begin{enumerate}
    \item
        Dans un anneau principal qui n'est pas un corps, un idéal est maximal si et seulement si il est engendré par un irréductible (proposition~\ref{PropomqcGe}).
    \item
        Dans un anneau, un idéal \( I\) est maximal si et seulement si \( A/I\) est un corps (proposition~\ref{PROPooSHHWooCyZPPw})
    \item
        Si \( \eK\) est un corps, \( \eK[X]\) est principal (lemme~\ref{LEMooIDSKooQfkeKp}).
\end{enumerate}


\InternalLinks{décomposition de matrices}   \label{DECooWTAIooNkZAFg}
\begin{enumerate}
    \item
        Décomposition de Bruhat, théorème~\ref{ThoizlYJO}.
    \item
        Décomposition de Dunford, théorème~\ref{ThoRURcpW}.
    \item
        Décomposition polaire~\ref{ThoLHebUAU} et la proposition~\ref{PropWCXAooDuFMjn} pour la régularité.
\end{enumerate}


\InternalLinks{systèmes d'équations linéaires}
\begin{itemize}
    \item Algorithme des facteurs invariants~\ref{PropPDfCqee}.
    \item La méthode du gradient à pas optimal permet de résoudre par itérations \( Ax=b\) lorsque \( A\) est symétrique strictement définie positive. Il s'agit de minimiser une fonction bien choisit. Propositions~\ref{PROPooYRLDooTwzfWU} pour l'existence et~\ref{PropSOOooGoMOxG} pour la méthode.
\end{itemize}


\InternalLinks{formes bilinéaires et quadratiques}      \label{THEMEooOAJKooEvcCVn}
    \begin{enumerate}
\item
    Les formes bilinéaires sont définies en~\ref{DEFooEEQGooNiPjHz}.
\item
    Forme quadratique, définition~\ref{DefBSIoouvuKR}. Sa matrice, définition \ref{DEFooAOGPooXWXUcN}.
\item
    Équivalence de forme quadratiques, définition \ref{DEFooOLWYooMwhMJp}. Deux formes quadratiques sont équivalentes si et seulement si elles ont même signature, proposition \ref{PROPooBWXMooLsgyKm}.
\item
    Une isométrie d'une forme bilinéaire est affine ou linéaire, théorème \ref{ThoDsFErq}.
\item
    Forme bilinéaire dégénérée, définition \ref{DEFooNUBFooLfCqaK}.
\item
    Une forme bilinéaire est non-dégénérée si et seulement si sa matrice associée est inversible, c'est la proposition \ref{PROPooQHHPooSqpgcb}.
\item
    Une isométrie d'une forme bilinéaire est linéaire ou affine par le théorème \ref{ThoDsFErq}.
\item
    Toute forme quadratique admet des bases orthogonales, théorème \ref{THOooIDMPooIMwkqB} pour le cas général; proposition \ref{PROPooUKRUooGRIDHt} pour le cas de \( \eR^n\), en se basant sur le théorème spectral.
\item
    Théorème de Sylvester à propos de signature (définition \ref{DEFooWDCLooDkRYLK}) de forme quadratique réelle : \ref{ThoQFVsBCk}.
\item
    Base \( q\)-orthogonale pour une forme quadratique, théorème \ref{THOooIDMPooIMwkqB}.
\end{enumerate}

\InternalLinks{arithmétique modulo, théorème de Bézout} \label{THEMEooNRZHooYuuHyt}
    \begin{enumerate}
        \item
            Pour \( \eZ^*\) c'est le théorème~\ref{ThoBuNjam}.
        \item
            Théorème de Bézout dans un anneau principal : corolaire~\ref{CorimHyXy}.
        \item
            Théorème de Bézout dans un anneau de polynômes : théorème~\ref{ThoBezoutOuGmLB}.
        \item
            En parlant des racines de l'unité et des générateurs du groupe unitaire dans le lemme~\ref{LemcFTNMa}. Au passage nous y parlerons de solfège.
        \item
            La proposition~\ref{PropLAbRSE} qui donne tout entier assez grand comme combinaisons de \( a \) et \( b\) à coefficients positifs est utilisée en chaines de Markov, voir la définition~\ref{DefCxvOaT} et ce qui suit.
        \item
            PGCD et PPCM sont dans la définition \ref{DefrYwbct}.
        \item
            Calcul effectif du PGCD puis des coefficients de Bézout : sous-sections~\ref{SUBSECooAEBLooFGJRkg} et~\ref{SUBSECooRHSQooEuBWbd}.
        \end{enumerate}

\InternalLinks{polynômes}

\begin{description}
    \item[Définitions]
        Soient un anneau \( A\), un corps \( \eK\), une extension \( \eL\) de \( \eK\) et un élément \( \alpha\in \eL\).
        \begin{enumerate}
            \item
                La définition la plus formelle est en tant que module produit \( A^{(\eN)}\), définition \ref{DEFooFYZRooMikwEL}. Le produit et l'évaluation sont définis en \ref{DEFooNXKUooLrGeuh} et la formule \( (PQ)(x)=P(x)Q(x)\) dans \ref{PROPooGDQCooHziCPH}.
            \item
                \( A[X]\), définition~\ref{DEFooFYZRooMikwEL}; l'anneau \( \eK[X]\) a même définition parce que c'est un cas particulier. L'évaluation d'un polynôme en un élément de l'anneau, \( P(\alpha)\) est définie en~\ref{DEFooNXKUooLrGeuh}.
            \item
                Liens entre \( \eK[\alpha]\), \( \eK[X]\), \( \eK(\alpha)\) et \( \eK(X)\) lorsque \( \alpha\) est transcendant, proposition~\ref{PROPooSYQWooFbfQtm}.

                Et la proposition~\ref{PropURZooVtwNXE} pour le cas où \( \alpha\) est algébrique\footnote{Définition \ref{LEMooLVPLooEkWYDN}.}.
            \item
                Si \( A\) est un anneau et si \( \alpha\) est un élément d'une extension de \( A\) (comme anneau), nous écrivons \( A[\alpha]\) pour le plus petit sous-anneau de \( B\) contenant \( A\) et \( \alpha\). C'est la définition~\ref{DEFooRFBFooKCXQsv}.
            \item
                \( \eK(X)\), le corps des fractions de \( \eK[X]\), définition~\ref{DEFooQPZIooQYiNVh}. Si \( R=P/Q\) dans \( \eK(X)\), l'évaluation est \( R(\alpha)=P(\alpha)Q(\alpha)^{-1}\), définition~\ref{DEFooZHBZooKlNfGZ}.
            \item
                \( \eK(\alpha)_{\eL}\) est le plus petit corps de \( \eL\) contenant \( \eK\) et \( \alpha\), définition~\ref{DEFooVSKGooMyeGel}.
            \item
                À propos de polynômes à plusieurs variables.
                \begin{itemize}
                    \item Anneau de polynômes : \( A[X_1,\ldots, X_n]\) est la définition~\ref{DEFooZNHOooCruuwI}. 
                    \item Corps engendré : \( \eK(\alpha_1,\ldots, \alpha_n)\) est la définition~\ref{DEFooRHRKooPqLNOp}. 
                    \item Corps des fractions rationnelles : \( \eK(X_1,\ldots, X_n)\) est la définition~\ref{DEFooOCPHooXneutp}.
                \end{itemize}
        \end{enumerate}

    \item[Coefficients dans un anneau commutatif]

        \begin{enumerate}
            \item
Les polynômes à coefficients dans un anneau commutatif  sont à la section~\ref{SECooVMABooVdhbPo}.
        \end{enumerate}


    \item[Coefficients dans un corps]
Les polynômes à coefficients dans un corps sont à la section~\ref{SECooFYOGooQHitgE}.
        \begin{enumerate}

\item
Nous parlons de l'idéal des polynômes annulateurs dans le théorème~\ref{ThoCCHkoU}.
            \item
                Le théorème~\ref{ThoCCHkoU} dit que \( \eK[X]\) est un anneau principal et que tous ses idéaux sont engendrés par un unique polynôme unitaire.
            \item
                Le polynôme minimal est irréductible, proposition~\ref{PropRARooKavaIT}.
            \item
                Quelques formules sur le \( \pgcd\), lemme~\ref{LemUELTuwK}.
        \end{enumerate}
    \item[Polynôme primitif]

        \begin{enumerate}
            \item
                Un polynôme est irréductible sur \( A\) si et seulement si irréductible et primitif sur le corps des fractions, corolaire~\ref{CORooZCSOooHQVAOV}.
        \end{enumerate}

    \item[Polynôme d'endomorphisme]
        C'est la section~\ref{SECooUEQVooLBrRiE}.

    \item[Racines et factorisation]

    \begin{enumerate}
        \item
            Si \( \eA\) est un anneau, la proposition~\ref{PropHSQooASRbeA} factorise une racine.
        \item
            Si \( \eA\) est un anneau, la proposition~\ref{PropahQQpA} factorise une racine avec sa multiplicité.
        \item
            Si \( \eA\) est un anneau, le théorème~\ref{ThoSVZooMpNANi} factorise plusieurs racines avec leurs multiplicités.
        \item
            Le théorème~\ref{ThoSVZooMpNANi} nous indique que lorsqu'on a autant de racines (multiplicité comprise) que le degré, alors nous avons toutes les racines.
        \item
            Si \( \eK\) est un corps et \( \alpha\) une racine dans une extension, le polynôme minimal de \( \alpha\) divise tout polynôme annulateur par la proposition~\ref{PropXULooPCusvE}.
        \item
            Le théorème~\ref{ThoLXTooNaUAKR} annule un polynôme de degré \( n\) ayant \( n+1\) racines distinctes.
        \item
            La proposition~\ref{PropTETooGuBYQf} nous annule un polynôme à plusieurs variables lorsqu'il a trop de racines.
        \item
            En analyse complexe, le principe des zéros isolés~\ref{ThoukDPBX} annule en gros toute série entière possédant un zéro non isolé.
        \item
            Polynômes irréductibles sur \( \eF_q\).
        \end{enumerate}

\end{description}

\InternalLinks{zoologie de l'algèbre}      \label{THEMEooVIQIooOcFAQS}

Nous listons ici un peu tous les termes qui arrivent en algèbre.

\begin{enumerate}
    \item[Éléments]
        Pour les éléments, nous avons :
        \begin{enumerate}
            \item
                Élément irréductible en \ref{DeirredBDhQfA}.
            \item
                Élément premier en \ref{DEFooZCRQooWXRalw}.
        \end{enumerate}
    \item[Anneaux]
        Pour les anneaux, nous avons :
        \begin{enumerate}
            \item
                Anneau factoriel en \ref{DEFooVCATooPJGWKq}.
            \item
                Anneau principal en \ref{DEFooGWOZooXzUlhK}.
            \item
                Anneau intègre en \ref{DEFooTAOPooWDPYmd}.
            \item
                Anneau noetherien en \ref{DEFooPWMHooCnrQuJ}.
        \end{enumerate}
    \item[Idéaux]
        Pour les idéaux, nous avons :
        \begin{enumerate}
            \item
                Idéal principal en \ref{DEFooMZRKooBPLAWH}.
            \item
                Idéal premier en \ref{DEFooAQSZooVhvQWv}.
        \end{enumerate}
\end{enumerate}



\InternalLinks{invariants de similitude}
    \begin{enumerate}
        \item
            Théorème~\ref{THOooDOWUooOzxzxm}.
        \item
            Pour prouver que la similitude d'applications linéaires résiste à l'extension du corps de base, théorème~\ref{THOooHUFBooReKZWG}.
        \item
            Pour prouver que la dimension du commutant d'un endomorphisme de \( E\) est de dimension au moins \( \dim(E)\), lemme~\ref{LEMooDFFDooJTQkRu}.
        \item
            Nous verrons dans la remarque~\ref{REMooPVLEooYDRXQI} à propos des invariants de similitude que toute matrice est semblable à la matrice bloc-diagonale constituées des matrices compagnon (définition~\ref{DEFooOSVAooGevsda}) de la suite des polynômes minimals.
        \end{enumerate}


\InternalLinks{réduction, diagonalisation}
    Des résultats qui parlent diagonalisation
    \begin{enumerate}
        \item
            Définition d'un endomorphisme diagonalisable :~\ref{DefCNJqsmo}.
        \item
            Conditions équivalentes au fait d'être diagonalisable en termes de polynôme minimal, y compris la décomposition en espaces propres : théorème~\ref{ThoDigLEQEXR}.
        \item
            Diagonalisation simultanée~\ref{PropGqhAMei}, pseudo-diagonalisation simultanée~\ref{CorNHKnLVA}.
        \item
            Diagonalisation d'exponentielle~\ref{PropCOMNooIErskN} utilisant Dunford.
        \item
            Décomposition polaire théorème~\ref{ThoLHebUAU}. \( M=SQ\), \( S\) est symétrique, réelle, définie positive, \( Q\) est orthogonale.
        \item
            Décomposition de Dunford~\ref{ThoRURcpW}. \( u=s+n\) où \( s\) est diagonalisable et \( n\) est nilpotent, \( [s,n]=0\).
        \item
            Réduction de Jordan (bloc-diagonale)~\ref{ThoGGMYooPzMVpe}.
        \item
            L'algorithme des facteurs invariants~\ref{PropPDfCqee} donne \( U=PDQ\) avec \( P\) et \( Q\) inversibles, \( D\) diagonale, sans hypothèse sur \( U\). De plus les éléments de \( D\) forment une chaine d'éléments qui se divisent l'un l'autre.
        \end{enumerate}
        Le théorème spectral et ses variantes :
        \begin{enumerate}
            \item
                Théorème spectral, matrice symétrique, théorème~\ref{ThoeTMXla}. Via le lemme de Schur complexe \ref{LemSchurComplHAftTq}.
            \item
                Théorème spectral autoadjoint (c'est le même, mais vu sans matrices), théorème~\ref{ThoRSBahHH}
            \item
                Théorème spectral hermitien, lemme~\ref{LEMooVCEOooIXnTpp}.
            \item
                Théorème spectral, matrice normales, théorème~\ref{ThogammwA}.
            \end{enumerate}
        Pour les résultats de décomposition dont une partie est diagonale, voir le thème~\ref{DECooWTAIooNkZAFg} sur les décompositions.
        Réduction de quadriques : 
        \begin{enumerate}
            \item
                Réduction de Gauss, théorème \ref{THOooOMMFooKxqICS}.
        \end{enumerate}


\InternalLinks{endomorphismes cycliques}
    \begin{enumerate}
        \item
            Définition~\ref{DEFooFEIFooNSGhQE}.
        \item
            Son lien avec le commutant donné dans la proposition~\ref{PropooQALUooTluDif} et le théorème~\ref{THOooGLMSooYewNxW}.
        \item
            Utilisation dans le théorème de Frobenius (invariants de similitude), théorème~\ref{THOooDOWUooOzxzxm}.
        \end{enumerate}


\InternalLinks{déterminant}     \label{THMooUXJMooOroxbI}
    \begin{enumerate}
        \item
            Déterminant d'une matrice : définition \ref{DEFooYCKRooTrajdP}.
    \item
        Déterminant d'un endomorphisme~\ref{DefCOZEooGhRfxA}. 
    \item
        Principales propriétés algébriques du déterminant : la proposition \ref{PropYQNMooZjlYlA}.
        \item
            Déterminant et manipulations de lignes et colonnes, section \ref{SUBSECooKMSVooBBHwkH} et les propositions qui précèdent à partir du lemme \ref{LEMooCEQYooYAbctZ} qui dit que \( \det(A)=\det(A^t)\).
    \item
        Les \( n\)-formes alternées forment un espace de dimension \( 1\), proposition~\ref{ProprbjihK}.
    \item
        Déterminant d'une famille de vecteurs~\ref{DEFooODDFooSNahPb}.
    \item
        Calcul d'un déterminant de taille \( 2\times 2\) : équation \eqref{EQooQRGVooChwRMd}.
    \item
        Interprétations géométriques
            \begin{enumerate}
        \item
            À propos d'orthogonalité, le déterminant est très lié au produit vectoriel en dimension \( 3\). Et il le généralise en dimension supérieure.
            \begin{enumerate}
                \item
            Liaison au produit vectoriel (orthogonalité) dans la proposition~\ref{PROPooTUVKooOQXKKl}.
        \item
            En particulier le lemme~\ref{LEMooFRWKooVloCSM} nous dit comment un déterminant donne un vecteur orthogonal à une famille donnée de vecteurs.
            \end{enumerate}
        \item
            Déterminant et aires, volumes
            \begin{enumerate}
                \item
            Déterminant et mesure de Lebesgue : théorème~\ref{ThoBVIJooMkifod}.
                \item
            Aire du parallélogramme : proposition~\ref{PropNormeProdVectoabsint}.
        \item
            Volume du parallélépipède avec le produit mixte et le déterminant \( 3\times 3\),~\ref{NORMooWWOKooWzScnZ}.
            \end{enumerate}
        \end{enumerate}
        Tant que nous en sommes dans les interprétations géométriques, il faut lier déterminant, produit vectoriel, orthogonalité et mesure en notant que l'élément de volume lors de l'intégration en dimension \( 3\) est donné par \eqref{EQooNYWSooZuvcPe} : \( dS=\| T_u\times T_v \|\) qui est la norme du produit vectoriel des vecteurs tangents au paramétrage.

        Nous voyons dans l'équation \eqref{EQooARMAooQPhQAL} que l'élément de volume pour une partie de dimension \( n\) dans \( \eR^m\) (à l'occasion d'y intégrer une fonction) est donné par un déterminant mettant en jeu les vecteurs tangents du paramétrage.
        \item
            Le déterminant de Vandermonde est à la proposition~\ref{PropnuUvtj}. Il est utilisé à divers endroits :
\begin{enumerate}
    \item
        Pour prouver que \( u\) est nilpotente si et seulement si \( \tr(u^p)=0\) pour tout \( p\) (lemme \ref{LemzgNOjY})
    \item
        Pour prouver qu'un endomorphisme possédant \( \dim(E)\) valeurs propres distinctes est cyclique (proposition~\ref{PropooQALUooTluDif}).
\end{enumerate}

   \end{enumerate}


\InternalLinks{polynôme d'endomorphismes}
    \begin{enumerate}
    \item Endomorphismes cycliques et commutant dans le cas diagonalisable, proposition~\ref{PropooQALUooTluDif}.
    \item Racine carrée d'une matrice hermitienne positive, proposition~\ref{PropVZvCWn}.
    \item Théorème de Burnside sur les sous-groupes d'exposant fini de \( \GL(n,\eC)\), théorème~\ref{ThooJLTit}.
    \item Décomposition de Dunford, théorème~\ref{ThoRURcpW}.
    \item Algorithme des facteurs invariants~\ref{PropPDfCqee}.
    \end{enumerate}

\InternalLinks{exponentielle}        \label{THEMEooKXSGooCsQNoY}

Toutes les exponentielles sont définies par la série
\begin{equation*}
    \exp(x)=\sum_{k=0}^{\infty}\frac{ x^k }{ k! },
\end{equation*}
tant que la somme a un sens.

\begin{description}
    \item[Réels]

Voici le plan que nous suivons dans le Frido :
\begin{itemize}
    \item L'exponentielle est définie par sa série en~\ref{DEFooSFDUooMNsgZY}.
    \item Nous démontrons qu'elle vérifie l'équation différentielle \( y'=y\), \( y(0)=1\) (théorème \ref{ThoKRYAooAcnTut}).
    \item Nous démontrons l'unicité de la solution à cette équation différentielle.
    \item Nous démontrons qu'elle est égale à \( x\mapsto y(1)^x\). Cela donne la définition du nombre \( e\) comme valant \( y(1)\).
    \item Nous définissons le logarithme comme l'application réciproque de l'exponentielle (définition~\ref{DEFooELGOooGiZQjt}).
    \item Les fonctions trigonométriques (sinus et cosinus) sont définies par leurs séries. Il est alors montré que \(  e^{ix}=\cos(x)+i\sin(x)\).
\end{itemize}

    \item[Propriétés]
        \begin{itemize}
            \item 
        La formule \( a^{-x}=1/a^x\) est la proposition \ref{PROPooVADRooLCLOzP}\ref{ITEMooSCJBooNVJZah}.
        \end{itemize}

    \item[Complexes]

        \begin{enumerate}
            \item
                Le fait que \(  e^{i\theta}\) donne tous les nombres complexes de norme \( 1\) est la proposition~\ref{PROPooZEFEooEKMOPT}.
            \item
                Le groupe des racines de l'unité est donné par l'équation \eqref{EqIEAXooIpvFPe}.
        \end{enumerate}

    \item[Algèbre normée commutative]

        Pour la définition c'est la proposition~\ref{DEFooSFDUooMNsgZY} et pour la régularité \(  C^{\infty}\) c'est la proposition~\ref{PROPooTBDAooQouzSk}.

    \item[Idem non commutatif]

        Il y a une tentative de théorème~\ref{THOooFGTQooZPiVLO}, mais c'est principalement pour les matrices qu'il y a des résultats.

    \item[Matrices]

        De nombreux résultats sont disponibles pour les exponentielles de matrices.

\begin{enumerate}
    \item
        Les sections \ref{secAOnIwQM} et \ref{SECooBYQBooZifJsg} parlent d'exponentielle de matrices.
    \item
        L'exponentielle donne lieu à une fonction de classe \(  C^{\infty}\), proposition~\ref{PropXFfOiOb}.
    \item
            Le lemme à propos d'exponentielle de matrice~\ref{LemQEARooLRXEef} donne :
            \begin{equation*}
                \|  e^{tA} \|\leq P\big( | t | \big)\sum_{i=1}^r e^{t\real(\lambda_i)}.
            \end{equation*}
        \item
            La proposition~\ref{PropCOMNooIErskN} : si \( A\in \eM(n,\eR)\) a un polynôme caractéristique scindé, alors \( A\) est diagonalisable si et seulement si \( e^A\) est diagonalisable.
\item
    La section~\ref{subsecXNcaQfZ} parle des fonctions exponentielle et logarithme pour les matrices. Entre autres la dérivation et les séries.
\item
    Pour résoudre des équations différentielles linéaires : sous-section~\ref{SUBSECooMDKIooKaaKlZ}.
\item
    La proposition~\ref{PropKKdmnkD} dit que l'exponentielle est surjective sur \( \GL(n,\eC)\).
\item

La proposition~\ref{PropFMqsIE} : si \( u\) est un endomorphisme, alors \( \exp(u)\) est un polynôme en \( u\).
\item
    Calcul effectif : sous-section~\ref{SUBSECooGAHVooBRUFub}.
\item Proposition~\ref{PROPooZUHOooQBwfZq} : si \( A\in\eM(n,\eC)\) alors $ e^{\tr(A)}=\det( e^{A}).$
    \item
        Les séries entières de matrices sont traitées autour de la proposition~\ref{PropFIPooSSmJDQ}.
\end{enumerate}


\end{description}

\InternalLinks{types d'anneaux}

\begin{enumerate}
    \item
        Définition d'anneau : définition \ref{DefHXJUooKoovob}.
    \item
        La définition d'anneau principal est \ref{DEFooGWOZooXzUlhK}; pour un idéal principal, c'est \ref{DEFooMZRKooBPLAWH}.
    \item
        \( \eZ\) est intègre, exemple~\ref{EXooLDXRooSxUAXs}, principal et euclidien (proposition~\ref{PROPooPJGLooQSrJTU}).
    \item
        \( \eZ[X]\) n'est pas principal (voir~\ref{ITEMooNQQMooSnuKvW}).
    \item   \label{ITEMooNQQMooSnuKvW}
        Si \( A\) est un anneau intègre qui n'est pas un corps, alors \( A[X]\) n'est pas principal, lemme~\ref{LEMooDJSUooJWyxCL}.
    \item
        L'anneau des fonctions holomorphes sur un compact donné est principal, proposition~\ref{PROPooVWRPooGQMenV}.
    \item
        L'anneau \( \eZ[i\sqrt{ 3 }]\) n'est pas factoriel, exemple~\ref{EXooCWJUooCDJqkr}.
    \item
        L'anneau \( \eZ[i\sqrt{ 5 }]\) n'est ni factoriel ni principal, exemple~\ref{EXooYCTDooGXAjGC}.
    \item
        Tous les idéaux de \( \eZ/6\eZ\) sont principaux, mais \( \eZ/6\eZ\) n'est pas principal. Exemple~\ref{EXooCJRPooYkWdyr}.
\end{enumerate}


\InternalLinks{sous-groupes}
\begin{enumerate}
    \item
        Théorème de Burnside sur les sous-groupes d'exposant fini de \( \GL(n,\eC)\), théorème~\ref{ThooJLTit}.
    \item
        Sous-groupes compacts de \( \GL(n,\eR)\), lemme~\ref{LemOCtdiaE} ou proposition~\ref{PropQZkeHeG}.
\end{enumerate}

\InternalLinks{groupe symétrique}       \label{THEMEooQEEWooXDhvhv}

\begin{enumerate}
    \item
        Définition~\ref{DEFooJNPIooMuzIXd}.
    \item
        La signature \( \epsilon\colon S_n\to \{ -1,1 \}\) est l'unique homomorphisme surjectif de \( S_n\) sur \( \{ -1,1 \}\), proposition \ref{ProphIuJrC}\ref{ITEMooBQKUooFTkvSu}.
    \item
        La table des caractères du groupe symétrique \( S_4\) est donné dans la section~\ref{SecUMIgTmO}.
    \item
        Le groupe symétrique \( S_4\) est le groupe des symétries affines du tétraèdre régulier, proposition~\ref{PROPooVNLKooOjQzCj}.
    \item
        Le groupe alterné \( A_5\) est l'unique groupe simple d'ordre \( 60\), proposition~\ref{PROPooUBIWooTrfCat}.
    \item
        La proposition~\ref{PROPooCPXOooVxPAij} donne la position du groupe alterné dans le groupe symétrique : \( A_n\) est un sous-groupe caractéristique de \( S_n\) et l'unique sous-groupe d'indice \( 2\).
\end{enumerate}

\InternalLinks{action de groupe}    \label{THEMEooKZHBooRCULcr}
    \begin{enumerate}
        \item Définition d'une action de groupe sur un ensemble :~\ref{DefActionGroupe}.
    \item Action du groupe modulaire sur le demi-plan de Poincaré, théorème~\ref{ThoItqXCm}.
    \item
        La formule de Burnside (théorème~\ref{THOooEFDMooDfosOw}) parle du nombre d'orbites pour l'action d'un groupe fini sur un ensemble fini.
    \item Des applications de la formule de Burnside : le jeu de la roulette et l'affaire du collier,~\ref{pTqJLY} et~\ref{siOQlG}.
    \item
        
        Le groupe symétrique  \( S_n\) agit sur l'anneau  \( \eK[T_1,\ldots, T_n]\), lemme \ref{LEMooIRVQooHvoNBq}.

    \end{enumerate}


\InternalLinks{classification de groupes}
\begin{enumerate}
    \item Structure des groupes d'ordre \( pq\), théorème~\ref{ThoLnTMBy}.
    \item Le groupe alterné est simple, théorème~\ref{ThoURfSUXP}.
    \item Définition~\ref{DEFooPRCHooVZdwST} d'un \( p\)-groupe.
    \item Théorème de Sylow~\ref{ThoUkPDXf}.
    \item Théorème de Burnside sur les sous-groupes d'exposant fini de \( \GL(n,\eC)\), théorème~\ref{ThooJLTit}.
    \item \( (\eZ/p\eZ)^*\simeq \eZ/(p-1)\eZ\), corolaire~\ref{CorpRUndR}.
\end{enumerate}



\InternalLinks{produit semi-direct de groupes}
    \begin{enumerate}
        \item
            Définition~\ref{DEFooKWEHooISNQzi}.         % Ce commentaire sert à rendre cette ligne unique ooQLCCooQfibcp
        \item
            Le corolaire~\ref{CoroGohOZ} donne un critère pour prouver qu'un produit \( NH\) est un produit semi-direct.
        \item
            L'exemple~\ref{EXooHNYYooUDsKnm} donne le groupe des isométries du carré comme un produit semi-direct.
        \item
            Le théorème~\ref{ThoLnTMBy} classifie les groupes d'ordre \( pq\) (\( p\), \( q\) premiers distincts) à grands coups de produit semi-directs.
        \item
            Le théorème~\ref{THOooQJSRooMrqQct} donne les isométries de \( \eR^n\) par \( \Isom(\eR^n)=T(n)\times_{\rho} O(n)\) où \( T(n)\) est le groupe des translations.
        \item
            La proposition~\ref{PROPooDHYWooXxEXvl} donne une décomposition du groupe orthogonal \( \gO(n)=\SO(n)\times_{\rho} C_2\) où \( C_2=\{ \id,R \}\) où \( R\) est de déterminant \( -1\).
        \item
            La proposition~\ref{PROPooTPFZooKtFxhg} donne \( \Aff(\eR^n)=T(n)\times_{\rho}\GL(n,\eR)\) où \( \Aff(\eR^n)\) est le groupe des applications affines bijectives de \( \eR^n\).
        \end{enumerate}


\InternalLinks{théorie des représentations}
\begin{enumerate}
    \item Définition \ref{DEFooXVMSooXDIfZV}.
    \item Table des caractères du groupe diédral, section~\ref{SecWMzheKf}.
    \item Table des caractères du groupe symétrique \( S_4\), section~\ref{SecUMIgTmO}.
\end{enumerate}


\InternalLinks{isométries}      \label{THMooVUCLooCrdbxm}

Il y a \( (\eR^n,\| . \|)\) et \( \eR^n,d\).

Les isométries de \( \| . \|\) sont linéaires, tandis que les isométries de la distance contiennent aussi les translations et les rotations de centre différent de l'origine.

Ne pas confondre une isométrie d'un espace affine avec une isométrie d'un espace euclidien. Les isométries d'un espace euclidien préservent le produit scalaire et fixent donc l'origine (lemme~\ref{LEMooYXJZooWKRFRu}). Les isométries des espaces affines par contre conservent les distances (définition~\ref{DEFooZGKBooGgjkgs}) et peuvent donc déplacer l'origine de l'espace vectoriel sur lequel il est modelé; typiquement les translations sont des isométries de l'espace affine mais pas de l'espace euclidien.

Parfois, lorsqu'on coupe les cheveux en quatre, il faut faire attention en parlant de \( \eR^n\) : soit on en parle comme d'un espace métrique (muni de la distance), soit on en parle comme d'un espace normé (muni de la norme ou du produit scalaire).

\begin{description}
    \item[Général] 
        Quelques résultats généraux et en vrac à propos d'isométries.
\begin{enumerate}
    \item
        Définition d'une isométrie pour une forme bilinéaire,~\ref{DEFooIQURooMeQuqX}. Pour une forme quadratique : définition~\ref{DEFooECTUooRxBhHf}.
    \item
        Définition du groupe orthogonal~\ref{DEFooUHANooLVBVID}, et le spécial orthogonal \( \SO(n)\) en la définition~\ref{DEFooJLNQooBKTYUy}. Le groupe \( \SO(2)\) est le groupe des rotations, par corolaire~\ref{CORooVYUJooDbkIFY}.
    \item
        La rotation \( R_A(\theta)\) d'un angle \( \theta\) autour du point \( A\in \eR^2\) est donnée par la définition \ref{DEFooADTDooKIZbrw}.
    \item
        La proposition \ref{PROPooOTIVooZpvLnb} donne à toute rotation \( R_0(\theta)\) une matrice de la forme connue. C'est autour de cela que nous définissons les angles, définition \ref{DEFooUPUUooKAPFrh}.
    \item
        Le groupe orthogonal est le groupe des isométries de \( \eR^n\), proposition~\ref{PropKBCXooOuEZcS}.
    \item
        Les isométries de l'espace euclidien sont affines,~\ref{ThoDsFErq}.
    \item
        Les isométries de l'espace euclidien comme produit semi-direct : $\Isom(\eR^n)\simeq T(n)\times_{\rho}\gO(n)$, théorème~\ref{THOooQJSRooMrqQct}.
    \item
        Isométries du cube, section~\ref{SecPVCmkxM}.
    \item
        Nous parlons des isométries affines du tétraèdre régulier dans la proposition~\ref{PROPooVNLKooOjQzCj}.
\end{enumerate}

    \item[Groupe diédral]
        Le groupe diédral est un peu central dans la théorie des isométries de \( (\eR^2,d)\) parce que beaucoup de sous-groupes finis des isométries de \( (\eR^2,d)\) sont en fait isomorphes au groupe diédral.
        \begin{enumerate}
    \item
        Générateurs du groupe diédral, proposition~\ref{PropLDIPoZ}.
    \item
        Un sous-groupe fini des isométries de \( (\eR^2,d)\) contenant au moins une réflexion est isomorphe au groupe diédral par le théorème \ref{THOooKDMUooUxQqbB}.
    \item
        Le théorème \ref{THOooAYZVooPmCiWI} dit que le groupe des isométries propres d'une partie quelconque de \( (\eR^2,d)\) est soit cyclique soit isomorphe au groupe diédral.
        \end{enumerate}
        
    \item[Isométries et réflexions]
        Dans un espace euclidien, toute isométrie peut être décomposée en réflexions autour d'hyperplans. Voici quelques énoncés à ce propos.
\begin{enumerate}
    \item
        Définition d'une réflexion dans \( \eR^2\) \ref{LEMooIJELooLWqBfE}.
    \item
        La caractérisation en termes de projection orthogonale est le lemme \ref{LEMooZSDRooUkNYer}; en termes de médiatrice c'est le lemme \ref{LEMooTCIEooXdyuHu}.
    \item
        Définition d'un hyperplan \ref{DEFooEWDTooQbUQws}.
    \item
        En dimension \( 2\), une rotation est définie comme composée de deux réflexions en la définition \ref{DEFooFUBYooHGXphm}.
    \item
        En dimension \( 2\), les réflexions ont un déterminant \( -1\) par le lemme \ref{LEMooSYZYooWDFScw}.
    \item
        Les isométries du plan \( (\eR^2,d)\) sont données dans le théorème \ref{THOooVRNOooAgaVRN}, et sont au plus 3 réflexions par le théorème \ref{THOooRORQooTDWFdv}.
    \item
        Décomposition des isométries de \( \eR^n\) en réflexions par le lemme \ref{LEMooJCDRooGAmlwp}.
    \item
        En particulier, les éléments de \( \SO(3)\) sont des compositions de deux réflexions par le corolaire \ref{CORooJCURooSRzSFb}.
    \item
        Une isométrie de \( \eR^n\) préserve l'orientation si et seulement si est elle composition d'un nombre pair de réflexions. C'est le théorème \ref{THOooWBIYooCtWoSq}.
\end{enumerate}
\item[Sous-groupe fini]
    \begin{enumerate}
        \item
            Les sous-groupes finis des isométries de \( (\eR^2,d)\) sont cycliques, théorème \ref{THOooKDMUooUxQqbB}.
        \item
            Les sous-groupes finis de \( \SO(3)\) sont listés dans \ref{PROPooBHPNooHPlgwH}.
        \item
            Les sous-groupes finis de \( \SO(2)\) sont cycliques, lemme \ref{LEMooUKEVooAEWvlM}.
    \end{enumerate}
\end{description}


\InternalLinks{caractérisation de distributions en probabilités}
\begin{enumerate}
    \item
        La probabilité conjointe est la définition~\ref{DefFonrepConj}.
    \item
        La fonction de répartition est la définition~\ref{DefooYAZVooNdxDCx}.
    \item
        La fonction caractéristique est la définition~\ref{DefooEIVXooNtHLQQ}.
\end{enumerate}


\InternalLinks{théorème central limite}
\begin{enumerate}
    \item
        Pour les processus de Poisson, théorème~\ref{ThoCSuLLo}.
\end{enumerate}

\InternalLinks{lemme de transfert}      \label{THEMEooJREIooKEdMOl}

Il y a deux résultats qui portent ce nom. Le premier est dans la théorie de Fourier, le résultat \( \hat{f'}=i\xi \hat{f}\).
\begin{enumerate}
    \item
        Lemme~\ref{LemQPVQjCx} sur \( \swS(\eR^d)\)
    \item
        Lemme~\ref{LEMooAGBZooWCbPDd} pour \( L^2\).
\end{enumerate}

L'autre lemme de transfert est en théorie des tribus, le résultat $\sigma\big( f^{-1}(\tribC) \big)=f^{-1}\big( \sigma(\tribC) \big)$ du lemme \ref{LemOQTBooWGYuDU}. Celui-ci est d'ailleurs plutôt nommé «lemme de transport».


\InternalLinks{probabilités et espérances conditionnelles}

    Les deux définitions de base, sur lesquelles se basent toutes les choses conditionnelles sont :
    \begin{itemize}
        \item L'espérance conditionnelle d'une variable aléatoire sachant une tribu : \( E(X|\tribF)\) de la définition~\ref{ThoMWfDPQ}.
    \end{itemize}

    Les autres sont listées ci-dessous.
\begin{description}

    \item[Probabilité conditionnelle]. 

        Plusieurs probabilités conditionnelles.
        \begin{itemize}
        \item D'un événement en sachant un autre : la définition \ref{DEFooGJVHooVbhVYv} donne
            \begin{equation*}
                P(A|B)=\frac{ P(A\cap B) }{ P(B) }
            \end{equation*}
            Cela est la définition de base. L'autre est une définition dérivée.
        \item D'un événement vis-à-vis d'une variable aléatoire discrète. C'est par la définition~\ref{DEFooFRLFooNvXuPK} qui définit la variable aléatoire
\begin{equation*}
    P(A|X)(\omega)=P(A|X=X(\omega)).
\end{equation*}
Dans le cas continu, c'est la définition~\ref{DEFooIUJMooBAVtMW} :
\begin{equation*}
    P(A|X)=P(A|\sigma(X))=E(\mtu_A|\sigma(X)).
\end{equation*}

    \item D'un événement par rapport à une tribu. C'est la variable aléatoire
\begin{equation*}
    P(A|\tribF)=E(\mtu_A|\tribF).
\end{equation*}

        \end{itemize}
    \item[Espérances conditionnelles] 
        Plusieurs espérances conditionnelles.
        \begin{itemize}
            \item
                D'une variable aléatoire par rapport à un évélement, définition \ref{DEFooOMLCooJgrbpx} :
                \begin{equation}
                    E(X|A)=\frac{ E(X\mtu_A) }{ P(A) }.
                \end{equation}
            \item d'une variable aléatoire par rapport à une tribu. La variable aléatoire \( E(X|\tribF)\) est la variable aléatoire \( \tribF\)-mesurable telle que
\begin{equation*}
    \int_BE(X|\tribF)=\int_BX
\end{equation*}
pour tout \( X\in \tribF\). Si \( X\in L^2(\Omega,\tribA,P)\) alors \( E(X|\tribF)=\pr_K(X)\) où \( K\) est le sous-ensemble de \( L^2(\Omega,\tribA,P)\) des fonctions \( \tribF\)-mesurables (théorème~\ref{ThoMWfDPQ}). Cela au sens des projections orthogonales.


    \item d'une variable aléatoire par rapport à une autre. La définition~\ref{DefooKIHPooMhvirn} est une variation sur le même thème :
\begin{equation*}
    E(X|Y)=E(X|\sigma(Y)),
\end{equation*}

%TODO : mettre cette définition à côté de celle du conditionnement par rapport à la tribu.
        \end{itemize}

\end{description}

Notons que partout, si \( X\) est une variable aléatoire, la notation «sachant \( X\)» est un raccourci pour dire «sachant la tribu engendrée par \( X\)».

Quelque formules.
\begin{enumerate}
    \item
        Pour l'espérance conditionnelle d'une variable aléatoire prenant seulement une quantité dénombrable de valeurs :
        $E(X|A)=\sum_{k=0}^{\infty}y_kP(X=y_k|A)$ par le lemme \ref{LEMooRTVBooCEeIxL}.
    \item
        La probabilité conditionnelle se factorise par rapport à l'union disjointe par le lemme \ref{LEMooRDXRooQLMRGF} : $P\big( \bigcup_{i=0}^{\infty}A_i|B \big)=\sum_{i=0}^{\infty}P(A_i|B)$.
\end{enumerate}

\InternalLinks{dénombrements}
\begin{itemize}
    \item Coloriage de roulette (\ref{pTqJLY}) et composition de colliers (\ref{siOQlG}).
    \item Nombres de Bell, théorème~\ref{ThoYFAzwSg}.
    \item Le dénombrement des solutions de l'équation \( \alpha_1 n_1+\ldots \alpha_pn_p=n\) utilise des séries entières et des décompositions de fractions en éléments simples, théorème~\ref{ThoDIDNooUrFFei}.
\end{itemize}


\InternalLinks{enveloppes}
    \begin{enumerate}
        \item
            L'ellipse de John-Loewner donne un ellipsoïde de volume minimum autour d'un compact dans \( \eR^n\), théorème~\ref{PropJYVooRMaPok}.
        \item
            Le cercle circonscrit à une courbe donne un cercle de rayon minimal contenant une courbe fermée simple, proposition~\ref{PROPDEFooCWESooVbDven}.
    \item Enveloppe convexe du groupe orthogonal~\ref{ThoVBzqUpy}.
    \item Enveloppe convexe d'une courbe fermée plane comme intersection des demi-plans tangents, proposition~\ref{PROPooOORPooCXrIQi}.
        \end{enumerate}


\InternalLinks{équations diophantiennes}
    \begin{enumerate}
        \item
            Équation \( ax+by=c\) dans \( \eN\), équation \eqref{EqTOVSooJbxlIq}.
        \item Dans~\ref{subsecZVKNooXNjPSf}, nous résolvons \( ax+by=c\) en utilisant Bézout (théorème~\ref{ThoBuNjam}).
        \item L'exemple~\ref{ExZPVFooPpdKJc} donne une application de la pure notion de modulo pour \( x^2=3y^2+8\). Pas de solutions.
        \item L'exemple~\ref{ExmuQisZU} résout l'équation \( x^2+2=y^3\) en parlant de l'extension \( \eZ[i\sqrt{2}]\) et de stathme.
        \item Les propositions~\ref{PropXHMLooRnJKRi} et~\ref{propFKKKooFYQcxE} parlent de triplets pythagoriciens.
        \item Le dénombrement des solutions de l'équation \( \alpha_1 n_1+\ldots \alpha_pn_p=n\) utilise des séries entières et des décompositions de fractions en éléments simples, théorème~\ref{ThoDIDNooUrFFei}.
        \item La proposition~\ref{PROPooLPKUooAlsYJg} donne une bijection \( \eN\times \eN\to \eN\) en résolvant dans \( \eN\) (entre autres) l'équation \( k=y^2+x\) pour \( k\) fixé.
        \end{enumerate}

\InternalLinks{techniques de calcul}        \label{THEMEooLTCIooGDIPnF}

Il y en a pour tous les gouts.

\begin{description}
    \item[Primitives et intégrales]
        Toute la section~\ref{SECooKSOFooEVKDLh} donne des trucs et astuces pour trouver des primitives et des intégrales.
    \item[Limite à deux variables]

        Les exemples de limites à plusieurs variables font souvent intervenir des coordonnées polaires (du théorème \ref{THOooBETSooXSQhdX}) ou autres fonctions trigonométriques. Ils sont donc placés beaucoup plus bas que la théorie.
        \begin{itemize}
            \item Méthode du développement asymptotique, sous-section~\ref{SUBSECooRAKKooAnpvkE}.
            \item Méthode des coordonnées polaires, sous-section~\ref{SUBSECooWCGMooPrXSpt}.
            \item Utilisation du théorème de la fonction implicite, dans l'exemple~\ref{EXooSDHDooJzDioW}.
        \end{itemize}

\end{description}

\immediate\closeout\themetoc

%+++++++++++++++++++++++++++++++++++++++++++++++++++++++++++++++++++++++++++++++++++++++++++++++++++++++++++++++++++++++++++
\section{Conventions sur les matrices et changement de bases}
%+++++++++++++++++++++++++++++++++++++++++++++++++++++++++++++++++++++++++++++++++++++++++++++++++++++++++++++++++++++++++++
\label{SECooBTTTooZZABWA}

Le lien entre matrice et application linéaire est donné par la définition \ref{DEFooJVOAooUgGKme}. L'application d'une matrice à un vecteur est \eqref{EQooQFVTooMFfzol}. Le lien le plus simple entre l'application linéaire et les éléments de matrice est donné par la proposition \ref{PROPooGXDBooHfKRrv}. Voici les relations :
\begin{subequations}
    \begin{align}
    T_{\alpha i}&=T(e_i)_{\alpha}\\
    T(e_i)&=\sum_{\alpha}T_{\alpha i}f_{\alpha}\\
    T(x)&=\sum_{i\alpha}T_{\alpha i}x_if_{\alpha}\\
    T(x)_{\alpha}&=\sum_{i}T_{\alpha i}x_i.
    \end{align}
\end{subequations}


Lorsque nous avons une base orthonormée nous avons aussi les propositions \ref{PROPooZKWXooWmEzoA} et \ref{PROPooZKWXooWmEzoA} qui donnent des formules avec produit scalaire :
\begin{enumerate}
    \item
    $T_{\alpha i}=e_{\alpha}\cdot T(e_i)$
    \item
    $x\cdot Ay=\sum_{kl}A_{kl}x_ky_l$.
\end{enumerate}
où le point est le produit scalaire usuel de \( \eR^n\).

%---------------------------------------------------------------------------------------------------------------------------
\subsection{Le changement de base}
%---------------------------------------------------------------------------------------------------------------------------

Soit un espace vectoriel \( V\) muni de deux bases \( (e_i)_{i=1,\ldots, n}\) et \( (f_{\alpha})_{\alpha=1,\ldots, n}\). Le lemme \ref{LEMooIHZGooOZoYZd} donne le lien entre les vecteurs de base :
\begin{enumerate}
    \item
        $f_{\alpha}=\sum_iQ_{i\alpha}e_i$
    \item
        $e_i=\sum_{\alpha}Q^{-1}_{\alpha i}f_{\alpha}$
\end{enumerate}
La proposition \ref{PROPooNYYOooHqHryX} donne un certain nombre de formules pour les coordonnées des vecteur :
\begin{enumerate}
    \item   
        \( y_{\alpha}=\sum_iQ^{-1}_{\alpha i}x_i\)
    \item  
        $x_i=\sum_{\alpha}Q_{i\alpha}y_{\alpha}$.
    \item
    $x_i=(Qy)_i$ 
    \item
    $x=Qy$
\end{enumerate}

La transformation de la matrice d'une application linéaire lors d'un changement de base est la proposition \ref{PROPooNZBEooWyCXTw}. Soit une application linéaire \( T\colon V\to V\) de matrices \( A\) et \( B\) dans les bases \( \{ e_i \}\) et \( \{ f_{\alpha} \}\). Si les bases sont liées par $f_{\alpha}=\sum_iQ_{i\alpha}e_i$, alors les matrices \( A\) et \( B\) sont liées par
    \begin{equation}
        B=Q^{-1}AQ.
    \end{equation}

%---------------------------------------------------------------------------------------------------------------------------
\subsection{Changement de base : matrice d'une forme bilinéaire}
%---------------------------------------------------------------------------------------------------------------------------

La proposition \ref{PROPooLBIOooUpzxXA} fait le changement de matrice d'une forme bilinéaire lors d'un changement de base. Si la matrice de \( q\) dans la base \( \{ e_i \}\) est \( A\) et celle dans la base \( \{ f_{\alpha} \}\) est \( B\), alors
\begin{equation}
    B=Q^tAQ.
\end{equation}
Pour comparaison avec la loi de transformation des matrices des applications liénaires, voir la remarque \ref{REMooNEJLooSqgeih}.



\newpage
\addcontentsline{toc}{chapter}{Table des matières}
\tableofcontents
\printindex

\addcontentsline{toc}{chapter}{Liste des notations}
\printnomenclature

% SCRIPT MARK -- ENGLISH INTRODUCTION

%\setcounter{chapter}{-2}
    \selectlanguage{english}
\chapter*{Introduction in English}
\input{AvertissementMazhe}

% SCRIPT MARK -- FRIDO
\selectlanguage{french}

\ifbool{isGiulietta}
{
    \part{Le Frido}
}{}

    \setcounter{chapter}{-1}
    \chapter{Introduction}
        %+++++++++++++++++++++++++++++++++++++++++++++++++++++++++++++++++++++++++++++++++++++++++++++++++++++++++++++++++++++++++++
\section{Auteurs, contributeurs, sources et remerciements}
%+++++++++++++++++++++++++++++++++++++++++++++++++++++++++++++++++++++++++++++++++++++++++++++++++++++++++++++++++++++++++++

Les remerciements, dans chaque catégorie, sont mis dans l'ordre chronologique approximatif. Les noms en couvertures sont ceux qui ont fourni du code \LaTeX\ (typiquement : un patch via github), par ordre chronologique approximatif d'entrée dans le projet.

%---------------------------------------------------------------------------------------------------------------------------
\subsection{Ceux qui ont travaillé sur le Frido}
%---------------------------------------------------------------------------------------------------------------------------

\begin{enumerate}
    \item
        Carlotta Donadello pour l'ensemble du cours de CTU de géométrie analytique 2010-2011. Une grosse partie de «analyse réelle» vient de là.
    \item
        Les étudiants de géométrie analytique en CTU 2010-2011 ont détecté d'innombrables coquilles. Les étudiants du cours présentiel de géométrie analytique 2011-2012 ont signalé un certain nombre d'incorrections dans les exercices et les corrigés. Les agrégatifs de Besançon 2011-2012 pour leurs plans et leurs développements.
    \item
        Lilian Besson pour m'avoir signalé un paquet de fautes, et quelques points pas clairs en statistiques.
    \item
        Plouf qui m'a signalé une coquille dans le fil \href{http://passeurdesciences.blog.lemonde.fr/2014/01/24/la-selection-scientifique-de-la-semaine-numero-106}{la-selection-scientifique-de-la-semaine-numero-106}.
    \item
        Benjamin de Block pour des coquilles et une mise au point sur les conventions à propos de \( \eR^+\) et \( (\eR^+)^*\).
    \item
        Olivier Garet pour avoir répondu à plein de questions de probabilités.
    \item
        François Gannaz pour de la relecture et une version plus claire de la preuve (et de l'énoncé) de la proposition~\ref{PROPooIOQEooGMcCJm}.
    \item
        Danarmk pour des réponses à des questions dans les commentaires (allongement pour éviter un Overfull hbox) \url{http://linuxfr.org/nodes/110155/comments/1675589}. Et aussi pour \href{ https://github.com/LaurentClaessens/mazhe/issues/87 }{ une discussion } à propos de la topologie sur \( \swD(\Omega)\).
    \item
        Cédric Boutilier pour des réponses à des questions de probabilité statistique. \url{https://github.com/LaurentClaessens/mazhe/issues/16}
    \item
        Remsirems pour des réponses à des questions d'analyse\footnote{\url{http://linuxfr.org/nodes/110155/comments/1675813}}
    \item
        Bertrand Desmons pour plusieurs patchs rendant plus clairs de nombreux passages sur les suites de Cauchy dans \( \eQ\).
    \item
        Anthony Ollivier pour m'avoir fait remarquer qu'il n'est pas vrai que \( A[X]\) est euclidien lorsque \( A\) est intègre (contre-exemple : \( A=\eZ\)). Ça fait une faute de moins dans le Frido.
    \item
        ybailly pour avoir détecté un bon nombre de coquilles dans la partie sur les ensembles de nombres.
    \item
        Éric Guirbal pour le remplacement de \info{frenchb} par \info{french}.
    \item
        cdrcprds pour une réponse à une question d'algèbre, démonstration à l'appui à propos de \href{https://github.com/LaurentClaessens/mazhe/issues/52#issuecomment-333251728}{pgcd}. Également pour sa relecture sans pitié de la partie sur la cardinalité (en particulier \( A\approx A\setminus B\)) et pour avoir pointé l'utilisé du théorème de comparabilité cardinale.
    \item
        Antoine Bensalah pour avoir répondu à une question sur Lax-Milgram tout en même temps que pointé une erreur dans la démonstration et fourni l'exemple~\ref{EXooTTBDooUNhBOc} sur l'optimalité de l'inégalité.
    \item
        Guillaume Deschamps pour ses remarques à propos du fait que le chapitre «constructions des ensembles» est très ardu.
    \item
        Guillaume Barriere pour sa relecture attentive jusqu'aux corps.
    \item
        Samy Clementz pour avoir découvert une faute dans la définition de mesure positive sur un espace mesurable.
    \item
        Sylvain Rousseau pour avoir clarifié une construction dans le théorème de Bower version \(  C^{\infty}\).
    \item
        Maxmax pour des typos dans l'index thématique.
    \item
        Laurent Choulette pour une typo dans les propriétés du neutre d'un groupe.
    \item
        Pierre Lairez pour la démonstration du théorème d'inversion de limite et de dérivée \ref{THOooXZQCooSRteSr} sans passer par les intégrales (et les lemmes correspondants à propos du module de continuité).
    \item
        Gregory Berhuy pour des réponses d'algèbres dans les catégories facile, moyen et difficile.
    \item
        Benoît Tran pour avoir signalé un paquet de typos dans la démonstration de l'ellipsoïde de John-Loewner et ses dépendances.
    \item
        Provatiscus pour avoir signalé un paquet de choses pas claires, et surtout pour avoir trouvé une faute dans la démonstration du fait qu'une fonction continue sur \( \eQ\) se prolonge en une fonction continue sur \( \eR\). Et pour cause : cet énoncé est faux. \url{https://github.com/LaurentClaessens/mazhe/issues/124}
    \item 
        William pour l'environnement \info{example} qui gère correctement le triangle.
    \item
        Colin Pitrat pour de nombreuses remarques, typos et relecture de théorèmes.
    \item
        Bruno Turgeon pour une très belle moisson de fautes de frappe (euphémisme pour dire «mon ignorance crasse de l'orthographe»).
\end{enumerate}

%---------------------------------------------------------------------------------------------------------------------------
\subsection{Aide directe, mais pas volontairement sur le Frido}
%---------------------------------------------------------------------------------------------------------------------------

\begin{enumerate}
    \item
        Plein de monde pour diverses contributions à des énoncés d'exercices. Pierre Bieliavsky pour les énoncés d'analyse numérique (MAT1151 à Louvain la Neuve 2009-2010). Jonathan Di Cosmo pour certaines corrections de MAT1151. François Lemeux, exercices sur les normes de matrices et correction de coquilles.  Martin Meyer et Mustapha Mokhtar-Kharroubi pour certains exercices du cours \emph{Outils mathématiques} (surtout ceux des DS et examens).
    \item
        Nicolas Richard et Ivik Swan pour les parties des exercices et rappels de calcul différentiel et intégral (Université libre de Bruxelles, 2003-2004) qui leurs reviennent.
    \item
	Carlotta Donadello pour la partie géométrie analytique : topologie dans \( \eR^n\), courbes, intégrales, limites. (Université de Franche-Comté 2010-2012)
    \item
        Le forum usenet de math, en particulier pour la construction des corps fini dans la fil « Vérifier qu'un polynôme est primitif » initié le 20 décembre 2011.
    \item
        Mihai Bostan nous a donné ses notes manuscrites de son cours présentiel de géométrie analytique 2009-2010. (Presque) Toute la structure du cours de géométrie analytique lui est due (qui est maintenant fondue un peu partout dans les chapitres d'analyse).
\end{enumerate}

%---------------------------------------------------------------------------------------------------------------------------
\subsection{Des gens qui ont fait un travail qui m'a bien servi}
%---------------------------------------------------------------------------------------------------------------------------

\begin{enumerate}
    \item
	Arnaud Girand pour avoir mis ses développements bien faits en ligne. Une bonne vingtaine de résultats un peu partout dans ces notes viennent de lui.
    \item
	Le site \url{http://www.les-mathematiques.net} m'a donné les preuves de nombreux résultats.
    \item
	Pierre Monmarché pour son document en ligne tout plein de développements, et des réponses à des questions.
    \item
        Tous les contributeurs du Wikipédia francophone (et aussi un peu l'anglophone) doivent être remerciés. J'en ai pompé des quantités astronomiques; des articles utilisés sont cités à divers endroits du texte, mais ce n'est absolument pas exhaustif. Il n'y a à peu près pas un résultat important de ces notes dont je n'aie pas lu la page Wikipédia, et souvent plusieurs pages connexes.
    \item
        Les intervenants du fil «\href{http://www.les-mathematiques.net/phorum/read.php?2,302266}{Antisymétrisation, alterné, déterminant et caractéristique}» sur \texttt{les-mathematiques.net} m'ont bien aidé pour la section sur les déterminants~\ref{SecGYzHWs} (bien qu'ils ne le savent pas).
    \item
        Xavier Mauquoy pour l'énoncé et la preuve du théorème~\ref{THOooYXJIooWcbnbm}.
    \item
        David Revoy pour les dessins de Pepper\&Carrot \href{https://www.peppercarrot.com/fr/article285/episode-8-pepper-s-birthday-party}{de la couverture}.
\end{enumerate}

J'ai souvent donné entre parenthèses à côté des mots « théorème », « lemme » ou « proposition » une ou plusieurs références vers les sources de la preuve que je donne. Ce sont parfois des liens vers la bibliographie; parfois aussi des liens hypertextes vers des sites, des blogs, etc. Tous ces gens ont fait du bon boulot. Sans toute cette « communauté », l'internet serait mort\footnote{Cette dernière phrase doit être comprise comme un appel à ne pas utiliser Moodle et autres iCampus pour diffuser vos cours de math, ou en tout cas pas comme moyen exclusif.}.

%+++++++++++++++++++++++++++++++++++++++++++++++++++++++++++++++++++++++++++++++++++++++++++++++++++++++++++++++++++++++++++
\section{Originalité}
%+++++++++++++++++++++++++++++++++++++++++++++++++++++++++++++++++++++++++++++++++++++++++++++++++++++++++++++++++++++++++++

Ces notes ne sont pas originales par leur contenu : ce sont toutes des choses qu'on trouve facilement sur internet; je crois que la bibliographie est éloquente à ce sujet. Ce cours se distingue des autres sur les points suivants.
\begin{description}
    \item[La longueur] J'ai décidé de ne pas me soucier de la taille du fichier. Il fera cinq mille pages s'il le faut, mais il restera en un bloc. Étant donné qu'il n'existe qu'une seule mathématique, il ne m'a pas semblé intéressant de produire une division artificielle entre l'analyse, la géométrie ou l'algèbre. Tous les résultats d'une branche peuvent (et sont) être utilisés dans toutes les autres branches.

        Dans cette optique, je me suis évertué à ne créer que des références «vers le haut». À moins d'oubli de ma part\footnote{Par exemple pour les théorèmes pour lesquels je n'ai pas lu ni a fortiori écrit de preuves.}, il n'y a aucun endroit du texte qui dépend d'un lemme démontré plus bas. Le fait qu'un théorème \( B\) soit plus bas qu'un théorème \( A\) signifie qu'on peut démontrer \( A\) sans savoir \( B\).

    \item[La licence] Ce document est publié sous une licence libre. Elle vous donne explicitement le droit de copier, modifier et redistribuer.

    \item[Les mises à jour] Ce document est régulièrement mis à jour. Des fautes d'orthographe sont corrigées (presque) chaque jour. Si vous me signalez une faute de mathématique, elle sera corrigée.
    \item[Transparence] Je ne fais pas semblant que ces notes soient parfaites. Les points sur lesquels je ne suis pas sûr, les preuves que j'ai inventées moi-même sont clairement indiqués pour inciter le lecteur à redoubler de prudence. Une liste de questions à résoudre est inclue en la section~\ref{SecooCKWWooBFgnea}. Voir \ref{SECooWVHBooCaYoXP} pour plus de détails.
\end{description}

%+++++++++++++++++++++++++++++++++++++++++++++++++++++++++++++++++++++++++++++++++++++++++++++++++++++++++++++++++++++++++++
\section{Les choses qui doivent vous faire tiquer}
%+++++++++++++++++++++++++++++++++++++++++++++++++++++++++++++++++++++++++++++++++++++++++++++++++++++++++++++++++++++++++++
\label{SECooWVHBooCaYoXP}

Un cours de math doit toujours être lu attentivement, surtout si vous avez l'intention de resservir à un jury le fruit de vos lectures. Dans ce livre, trois éléments doivent vous faire redoubler de prudence.

\begin{description}
    \item[La référence \cite{MonCerveau}] 
        D'abord les références à \cite{MonCerveau} indiquent qu'une bonne partie de ce qui suit est de l'invention personnelle de l'auteur. Cela ne veut évidemment pas dire que c'est moi qui ait découvert le résultat. Ça veut dire que je n'ai pas trouvé le résultat ou certaines parties de la preuve.
    \item[Les notes en bas de page]
        Certaines notes en bas de page sont écrites dans une fonte spéciale\quext{Les notes comme celle-ci signifient qu'il y a certaines chose dont je ne suis pas sûr.}. Elles indiquent des points sur lesquels je doute ou des étapes de calculs que je ne parviens pas à reproduire en suivant mes sources. Lorsque vous voyez une telle note, redoublez de prudence, allez voir la source, et écrivez-moi si vous pouvez résoudre le problème.
    \item[Les environnements dédiés]       
        Et enfin certains problèmes sont indiqués plus longuement dans un environnement dédié en petits caractères comme ceci :

        \begin{probleme}
            Les choses écrites comme ceci sont des questions ou des éléments sur lesquels j'ai un doute. Lisez-les attentivement. Ces notes mentionnent des points que personnellement je n'oserais pas affirmer plein d'aplomb à un jury d'agrégation.
        \end{probleme}
\end{description}

%+++++++++++++++++++++++++++++++++++++++++++++++++++++++++++++++++++++++++++++++++++++++++++++++++++++++++++++++++++++++++++
\section{Quelques choix qui peuvent provoquer des quiproquos}
%+++++++++++++++++++++++++++++++++++++++++++++++++++++++++++++++++++++++++++++++++++++++++++++++++++++++++++++++++++++++++++

Comme tout cours de mathématique, ce cours fait des choix qui sont parfois discutables. Voici quelques points sur lesquels les choix faits ici ne sont peut-être pas ceux fait par tout le monde. Ce sont donc des points sur lesquels vous devez faire attention pour éviter les quiproquos lors par exemple d'un oral dans un concours.

\begin{enumerate}
    \item
        Nous utilisons la définition usuelle de limite d'une fonction en un point. Elle diffère de celle donnée par le ministère de l'enseignement en France. Si votre but est de passer un concours d'enseignement en France, vous devriez lire~\ref{SUBSECooVHKCooYRFgrb}; dans tous les autres cas, la définition prise ici est celle qu'il vous faut.
    \item
        Un compact est une partie d'un espace topologique pour lequel tout recouvrement par des ouverts admet un sous-recouvrement fini. Le fait d'être séparable n'est pas inclus dans la définition de compact. De nombreux textes français incluent la séparabilité dans la compacité.
    \item
        Le logarithme sur \( \eC\) est une application \( \ln\colon \eC^*\to \eC\) définie partout sauf en zéro. Elle n'est donc pas continue sur la fameuse demi-droite. À ne pas confondre avec une \emph{détermination} du logarithme qui est par définition continue et donc non définie sur la demi-droite.

        Cela est un choix très discutable. La raison de donner à la notation «\( \ln\)» cette signification est simplement de suivre l'usage de Sage. Pour Sage, \( \ln(-1)\) existe et vaut \( i\pi\).

        Voir les remarques~\ref{REMooFBLLooDnkmjR}.
    \item
        Le mot «corps» n'implique pas la commutativité, et nous n'utilisons pas la terminologie «anneau à division». Voir la remarque~\ref{REMooYRNUooYgBBKF} et la discussion~\ref{NORMooGPWRooIKJqqw}.
\end{enumerate}

%+++++++++++++++++++++++++++++++++++++++++++++++++++++++++++++++++++++++++++++++++++++++++++++++++++++++++++++++++++++++++++ 
\section{Autres choix pas spécialement standards}
%+++++++++++++++++++++++++++++++++++++++++++++++++++++++++++++++++++++++++++++++++++++++++++++++++++++++++++++++++++++++++++

Nous listons ici quelques choix qui n'induisent pas de différences ou d'incompatibilité avec les autres cours, mais qui doivent être compris et justifiés.

\begin{enumerate}
    \item
        Nous n'utilisons pas les notations \( o(x)\) ou autres \( O(N^2)\). D'abord parce que je n'ai jamais très bien compris comment elles fonctionnent, et ensuite (surtout) parce que ces notations induisent en erreur. Ce sont des notations qui cachent sous des notations a peu près intuitive l'utilisation de théorèmes pas simples.

        Écrire
        \begin{equation}
            f(x)=P(x)+o(x^2),
        \end{equation}
        c'est un peu comme quand on écrit (horreur !)
        \begin{equation}
            F(x)=\int f(x)dx+C.
        \end{equation}
        Où est le \( x\) à droite ? Quel est le statut de \( C \) ? 

        Même chose pour la notation \( f(x)=P(x)+o(x^2)\). Le \( x\) de \( o(x^2)\) est-il le \( x\) qu'on a à gauche ? Si \( g(x)=Q(x)+o(x^2)\), est-ce le même \( o\) que celui de \( f\) ?
    \item
        Nous allons être plus calme avec la notation \( A[X]\) pour les polynômes sur l'anneau \( A\), et encore moins \( A[X_1,\ldots, X_n]\) pour les polynômes de \( n\) variables. Au lieu de cela nous utilisons \( \Poly(A)\) et \( \Poly_n(A)\).

        Est-ce que vous diriez que \( A[X]=A[Y]\) ? Quelle est exactement la nature de \( X\) dans \( P=X^2+1\) ou dans \( P(X)=X^2+1\) ? Si \( P\in A[X]\) vaut \( P(X)=X^2\)  et si \( Q\in A[Y]\) vaut \( Q(Y)=Y^2\), est-ce que vous oseriez écrire \( P=Q\) ?
\end{enumerate}

%+++++++++++++++++++++++++++++++++++++++++++++++++++++++++++++++++++++++++++++++++++++++++++++++++++++++++++++++++++++++++++
\section{Sage est là pour vous aider}
%+++++++++++++++++++++++++++++++++++++++++++++++++++++++++++++++++++++++++++++++++++++++++++++++++++++++++++++++++++++++++++

Il existe de nombreux logiciels de mathématique. Notre préféré est \href{http://www.sagemath.org}{Sage} pour une raison très précise : en tant que langage de programmation, Sage est python qui est un langage généraliste. La syntaxe et la structure de Sage ne sont pas \emph{ad hoc} pour faire de math, et ce qu'on apprend en Sage peut être recyclé pour faire n'importe quoi : navigateur web, script de manipulation de texte, traitement d'image, réseau neuronaux, \ldots

%Par ailleurs, le vingt et unième siècle est déjà largement entamé; si vous vous lancez dans une carrière scientifique, il vous faudra maitriser l'informatique un peu plus solidement qu'être virtuose es trouver le trajet le plus court en bus sur votre téléphone.

Sage est un logiciel disponible pour l'épreuve de modélisation de l'agrégation de mathématique; il y a donc de bonnes chances que vous en ayez l'usage.

%---------------------------------------------------------------------------------------------------------------------------
\subsection{Lancez-vous dans Sage}
%---------------------------------------------------------------------------------------------------------------------------


\begin{enumerate}
	\item
        Aller sur \url{http://www.sagemath.org},
	\item
		créer un compte,
	\item
		créer des feuilles de calcul et s'amuser !!
\end{enumerate}

Il y a beaucoup de \href{http://lmgtfy.com/?q=sage+documentation}{documentation} sur le \href{http://www.sagemath.org}{site officiel}\footnote{\href{http://www.sagemath.org}{http://www.sagemath.org}}, et nous vous conseillons particulièrement le livre \cite{ooBLMMooWTPsQy}.

Si vous comptez utiliser régulièrement ce logiciel, je vous recommande \emph{chaudement} de \href{http://mirror.switch.ch/mirror/sagemath/index.html}{l'installer} sur votre ordinateur.

%---------------------------------------------------------------------------------------------------------------------------
\subsection{Exemples de ce que Sage peut faire pour vous}
%---------------------------------------------------------------------------------------------------------------------------

Ce livre est émaillé de petits bouts de code en Sage montrant ses différentes fonctionnalités là où nous en avons besoin\footnote{Soit un vrai besoin comme tracer un graphique en 3D, soit de la paresse comme calculer une grosse dérivée.}. Voici une liste (non exhaustive) de ce que Sage peut faire pour vous.

\begin{enumerate}

	\item
        Calculer des limites de fonctions, exemples~\ref{ExBCRXooDVUdcf} et~\ref{ExCWDRooKxnjGL}.
	\item
        Tracer des graphes de fonctions, exemple~\ref{ExCWDRooKxnjGL}.
	\item
        Tracer des courbes en trois dimensions, voir exemple~\ref{ExempleTroisDxxyy}. Notez que pour cela vous devez installer aussi le logiciel Jmol. Pour Ubuntu, c'est dans le paquet \info{icedtea6-plugin}.
	\item
		Calculer des dérivées partielles de fonctions à plusieurs variables, voir exemple~\ref{exJMGTooZcZYNy}.
	\item
        Résoudre des systèmes d'équations linéaires. Voir les exemples~\ref{exKGDIooVefujD} et~\ref{ExBGCEooPIQgGW}. Lire aussi \href{http://www.sagemath.org/doc/constructions/linear_algebra.html#solving-systems-of-linear-equations}{la documentation}.
	\item
        Tout savoir d'une forme quadratique, voir exemple~\ref{exBNGVooIvKfTT}.
	\item
        Calculer la matrice hessienne de fonctions de deux variables, déterminer les points critiques, déterminer le genre de la matrice hessienne aux points critiques et écrire les extrémums de la fonctions (sous réserve d'être capable de résoudre certaines équations), voir les exemples~\ref{exZHGRooTQpVpq} et~\ref{exHWIHooOAvaDQ}.
	\item
        Indiquer une infinité de solutions à une équation en utilisant des paramètres. L'exemple \ref{exEEHPooKDxLTJ} montre ça avec une équation algébrique. Un exemple concernant des fonctions trigonométriques :
        \begin{verbatim}
sage: solve(sin(x)/cos(x)==1,x,to_poly_solve=True)
[x == 1/4*pi + pi*z1]
sage: solve(sin(x)**2==cos(x)**2,x,to_poly_solve=True)
[sin(x) == cos(x), x == -1/4*pi + 2*pi*z86, x == 3/4*pi + 2*pi*z84]
        \end{verbatim}

        Notez l'option \info{to\_poly\_solve=true} dans \info{solve}.

	\item
        Calculer des dérivées symboliquement, voir exemple~\ref{exRNZKooUIOfPU}.
	\item
        Calculer des approximations numériques comme dans l'exemple~\ref{exLFYFooNCXCJz}.
    \item
        Calculer dans un corps de polynômes modulo comme \( \eF_p[X]/P\) où \( P\) est un polynôme à coefficients dans \( \eF_p\). Voir l'exemple~\ref{ExemWUdrcs}.
\end{enumerate}

Sage peut en général faire tout ce que vous êtes capable de faire à l'entrée en master et probablement bien plus, à la notable exception des limites à deux variables.

\begin{remark}
    Sage peut toutefois vous induire en erreur si vous n'y prenez pas garde parce qu'il sait des choses en mathématique que vous ne savez pas. Par conséquent il peut parfois vous donner des réponses (mathématiquement exactes) auxquelles vous ne vous attendez pas. Voir par exemple~\ref{ooOPWYooDDSZWx} pour le logarithme de nombres négatifs. Et aussi ceci :

\lstinputlisting{tex/sage/sageSnip017.sage}

Sage fait une différence entre \info{Infinity} et \info{+Infinity} et donne
\begin{equation}
    \lim_{x\to 0} \frac{1}{ x }=\infty
\end{equation}
ainsi que
\begin{equation}
    \lim_{x\to 0} \frac{1}{ x^2 }=+\infty.
\end{equation}
\end{remark}

Voir aussi la compactification en un point d'Alexandroff \ref{PROPooHNOZooPSzKIN}.

%+++++++++++++++++++++++++++++++++++++++++++++++++++++++++++++++++++++++++++++++++++++++++++++++++++++++++++++++++++++++++++
\section{Comment contribuer et aider ?}
%+++++++++++++++++++++++++++++++++++++++++++++++++++++++++++++++++++++++++++++++++++++++++++++++++++++++++++++++++++++++++++
\label{SecooCKWWooBFgnea}

Ces notes ne sont pas relues de façon systématique. Aucune garantie. Merci de me signaler toute faute ou remarque : le relecteur c'est toi. Voici une petite liste de questions que je me pose ou de choses écrites dont je ne suis pas certain. Si vous avez un avis ou une réponse à un des points, merci de vous faire connaitre.

Les questions ouvertes sont divisées en trois niveaux de difficulté :
\begin{enumerate}
    \item
        Niveau facile : un étudiant de licence devrait pouvoir le faire.
    \item
        Niveau moyen : un candidat à l'agrégation de mathématique devrait pouvoir le faire.
    \item
        Demande probablement de connaissances avancées en mathématique; au moins être tout à fait à l'aise avec le niveau d'agrégation.
\end{enumerate}

Quel que soit votre niveau, vous pouvez faire ceci :

\begin{enumerate}
    \item
        M'écrire pour me signaler toutes les fautes que vous voyez, même si vous n'êtes pas sûr.
    \item
        Si vous n'êtes pas expert, me signaler tous les endroits qui vous semblent obscurs. Vu que ces notes sont destinées à \emph{apprendre}, les avis des non-experts sont très importants.
    \item
        Mettre une copie de (ou un lien vers) ces notes sur votre site.
\end{enumerate}

%--------------------------------------------------------------------------------------------------------------------------- 
\subsection{Des preuves qui manquent}
%---------------------------------------------------------------------------------------------------------------------------

Vous trouverez un peu partout des énoncés sans preuves. Certaines sont surement très faciles, et d'autres probablement assez compliquées. N'hésitez pas à rédiger une preuve et me l'envoyer.

Vous pouvez m'envoyer vos preuves sous forme de «c'est bien fait dans tel cours», avec une URL.

Ne me dites juste pas «c'est bien fait dans tel \emph{livre}». Je ne travaille pas à l'université, et je n'ai pas accès à une bibliothèque universitaire; je n'ai donc pas réellement accès à ces fameux «livres» dont tout le monde parle.

%--------------------------------------------------------------------------------------------------------------------------- 
\subsection{Du texte qui manque}
%---------------------------------------------------------------------------------------------------------------------------

Vous remarquerez que de nombreuses pages du Frido sont des enchainements de théorèmes et démonstrations sans articulations. Autrement dit, il manque ce qu'à l'agrégation on irait à l'oral quand on présente le plan. Si vous avez des idées de choses à ajouter ici où là, faites le moi savoir.

%---------------------------------------------------------------------------------------------------------------------------
\subsection{Mes questions de géométrie}
%---------------------------------------------------------------------------------------------------------------------------

%///////////////////////////////////////////////////////////////////////////////////////////////////////////////////////////
\subsubsection{Facile}
%///////////////////////////////////////////////////////////////////////////////////////////////////////////////////////////

\begin{enumerate}
    \item
        Donner une interprétation en termes de plans, droites ou je ne sais quoi de géométrique au \( 4\)-cycle parmi les permutations des sommets du tétraèdre. D'où sort géométriquement la matrice \eqref{EQooONDUooYlduup} ?
\end{enumerate}

%///////////////////////////////////////////////////////////////////////////////////////////////////////////////////////////
\subsubsection{Moyen}
%///////////////////////////////////////////////////////////////////////////////////////////////////////////////////////////

\begin{enumerate}
    \item
        Démontrer la proposition~\ref{PROPooCOZCooCghwaR} qui donne les relations de Chasles pour un espace affine.
    \item
        En géométrie projective, dans la sphère de Riemann \( \hat\eC=P_1(\eC)=\eC\cup\{ \infty \}\) est-ce qu'il existe une notion de cercle dont le centre est \( \infty\) ? Voir le point~\ref{NORMooCXVJooMTMqEU}.
    \item
        Géométrie projective. Tout le monde semble définir le birapport en identifiant \( P(\eK^2)\) à \( \hat\eK=\eK\cup\{ \infty \}\). Bien entendu, personne ne semble s'être attribué la mission d'expliciter la dépendance du birapport en le choix de l'identification. Je le fais à la définition~\ref{DEFooBFSKooDwzwmO}.

        Mais cette définition dépend du choix d'identification \( \varphi\colon P(\eK^2)\to \hat\eK\), comme le montre l'exemple~\ref{EXooYCOYooWFSfUv}. J'ai donc défini des classes d'identifications possibles \( A(\varphi)\) en~\ref{DEFooMLQUooGwvQMh}. Et je démontre la proposition~\ref{PROPooTFMQooIOQGvs} que si \( \varphi_a\in A(\varphi)\) alors les birapports construits à partir de \( \varphi\) et \( \varphi_a\) sont identiques.

        Question : pourquoi personne ne semble faire ce travail ? En quoi l'identification \( \varphi_0\) que tout le monde utilise est plus canonique qu'une autre ? Est-ce que l'on peut décrire simplement les classes \( A(\varphi)\) ? Le groupe qui conserve le birapport associé à \( \varphi\) est-il isomorphe au groupe qui conserve le birapport associé à \( \varphi'\) ? Quels que soient \( \varphi\) et \( \varphi'\) ?

        Suis-je la seule personne au monde à m'être demandé si le birapport était un objet canonique ?
    \item
        En géométrie projective, dans \( P_1(\eC)=\hat\eC=\eC\cup\{ \infty \}\), si \( \ell\) est une droite dans \( \eC\), est-ce que la droite correspondante dans \( \hat\eC\) contient le point \( \infty\) ? Moi j'ai envie de dire que \( \infty\) est sur toutes les droites. Voir le problème~\ref{PROBooZHHTooIFNwxR} et la remarque~\ref{REMooBMAEooHDvNID}.

    \item
        Géométrie affine, barycentre. Les mauvaises langues diraient que tout cela est du snobisme autour de la paresse d'écrire \( \vect{ xy }\) au lieu de \( y-x\). Est-ce qu'il y a des cas où toute l'étude des espaces affines et des barycentres en particulier apportent \emph{réellement} plus qu'une facilité d'écriture par rapport à travailler dans le cadre vectoriel pur ?
\end{enumerate}


%---------------------------------------------------------------------------------------------------------------------------
\subsection{Mes questions d'algèbre}
%---------------------------------------------------------------------------------------------------------------------------

%///////////////////////////////////////////////////////////////////////////////////////////////////////////////////////////
\subsubsection{Facile}
%///////////////////////////////////////////////////////////////////////////////////////////////////////////////////////////

%///////////////////////////////////////////////////////////////////////////////////////////////////////////////////////////
\subsubsection{Moyen}
%///////////////////////////////////////////////////////////////////////////////////////////////////////////////////////////

\begin{enumerate}
    \item La définition \ref{DEFooLNEXooYMQjRo} de \( \sum_{i\in I}f(i)\)  (\( I\) est fini et \( f\colon I\to (G,+)\) est une application) demande une preuve. Comment prouver que
    \begin{equation}
        \sum_{i=1}^nf\big( \sigma_1(i) \big)=\sum_{i=1}^nf\big( \sigma_2(i) \big)
    \end{equation}
    lorsque \( \sigma_1\) et \( \sigma_2\) sont des bijections ? 

    De façon peut-être plus simple, il faut prouver la proposition \ref{PROPooJBQVooNqWErk} qui exprime l'invariance d'une somme commutative finie sous le groupe des bijections des termes.

    Je crains que cela demande de décomposer \( \sigma_1\sigma_2^{-1}\) en cycles et de pas mal chipoter pour tout ramener à des permutations de deux termes.
\end{enumerate}

%///////////////////////////////////////////////////////////////////////////////////////////////////////////////////////////
\subsubsection{Difficile}
%///////////////////////////////////////////////////////////////////////////////////////////////////////////////////////////

\begin{enumerate}
    \item   \label{ITEMooYWPLooMHuGzQ}
        Que penser de~\ref{EXooJRSUooYhAZkR} qui dit que l'extension de corps \( \eK(\alpha)\) dépend en réalité du corps ambiant dans lequel on calcule l'extension ?

        Est-ce qu'il existe des exemples moins triviaux et plus utiles que celui que je donne ?
    \item       \label{ITEMooUBZIooDDcfWg}
        Pour quelle classe d'anneaux et de polynômes le quotient \( A[X]/(P)\) est-il un corps ?
\end{enumerate}

%///////////////////////////////////////////////////////////////////////////////////////////////////////////////////////////
\subsubsection{Non classées}
%///////////////////////////////////////////////////////////////////////////////////////////////////////////////////////////

\begin{enumerate}
    \item
        Est-ce qu'il existe une structure raisonnable d'espace vectoriel sur \( \eZ\) ? Est-ce qu'il existe des corps discrets infinis ?

        Dans cette question, j'ai derrière la tête que dans un espace vectoriel topologique nous avons une notion de suite de Cauchy, définition~\ref{DefZSnlbPc}. Donc dans ce cas la notion d'espace complet est une notion topologique. Or il y a l'exemple~\ref{EXooNMNVooXyJSDm} qui donne deux distances sur \( \eN\), qui donnent la même topologique, mais l'un étant complet, l'autre non.

        Si il y avait une structure vectorielle sur \( \eN\), cela créerait une contradiction. Au moins au sens où la définition~\ref{DefZSnlbPc} de suite de Cauchy «topologique» ne redonne pas la même que la notion «métrique» de la définition usuelle~\ref{DEFooXOYSooSPTRTn}.
    \item
        La «décomposition en facteurs premiers» dans \( \eZ[i\sqrt{2}]\) que je donne dans l'exemple~\ref{ExluqIkE} est-elle correcte ? En particulier le lemme~\ref{LemTScCIv} ?
    \item
        Est-ce que la fin de la démonstration~\ref{ThojCJpFW} avec cette histoire d'ensemble \( \{ \xi_k^q\tq q\in \eN \}\) fini est compréhensible ?
    \item
        Les représentations \emph{irréductibles} sont les modules \emph{indécomposables}. Quid des modules irréductibles ? C'est pas un peu dingue de ne pas utiliser le mot «irréductible» pour désigner les mêmes choses dans le cas des modules et celui des représentations ?
    \item
        Rendre rigoureuse la remarque \eqref{RemmQjZOA} qui dit que les matrices dont le polynôme minimal est égal au polynôme caractéristique sont denses dans les matrices.
    \item
        La partie initiation de récurrence (\( r=2\)) de la preuve de la proposition~\ref{PropSVvAQzi} à propos de convexe et de barycentre est-elle correcte ? Ce passage de l'espace affine à l'espace vectoriel sous-jacent me paraît un peu facile.
    \item
        Est-ce que l'énoncé et la démonstration de la proposition~\ref{PropyMTEbH} sont corrects ? Si \( a\) et \( b\) sont des racines de \( P\), alors \( \mu_a\mu_b\) divise \( P\) (si \( \mu_a\neq \mu_b\)). Cette proposition est utilisée dans la démonstration de l'irréductibilité des polynômes cyclotomiques (proposition~\ref{PropoIeOVh}).
    \item
        À quoi sert l'hypothèse «autre que \( \eF_2\)» dans le lemme~\ref{LemcDOTzM} ? Peut-être dans la notion de déterminant parce qu'en caractéristique \( 2\), l'antisymétrie d'une forme linéaire n'implique le fait qu'elle soit alternée.
    \item
        L'inversibilité de la somme de Gauss (proposition~\ref{PropciRUov}) est-elle bien démontrée ?
    \item
        Des commentaires sur l'exemple~\ref{ExfUqQXQ} qui montre que \( X^p-X+1\) est irréductible sur \( \eF_p\).
    \item
        Les idéaux de \( A/I\) sont en bijection avec les idéaux de \( A\) contenant \( I\). Justification de l'équation \eqref{EqKbrizu}.
    \item
        À propos d'extensions algébriques, est-ce que la proposition~\ref{PropURZooVtwNXE} est correcte ? Est-ce qu'implicitement, il n'y a pas un sur-corps de \( \eK\) dans lequel il faut travailler ?
    \item
        À propos de construction à la règle et au compas. Pour l'addition d'angles, l'exemple~\ref{ExOVDooXnWPDl} explique comment on construit la somme de deux angles. Le problème est que cette construction se fait par intersection de deux cercles. Une des deux intersections donne \( \alpha+\beta\) et l'autre donne \( \alpha-\beta\). Comment par construction peut-on choisir le bon point ?
    \item
        À propos de chiffrement RSA, quelle est la probabilité que le message \( M\) ne soit pas premier avec \( p\) ? Est-ce que Alice (qui est celle qui chiffre avec la clef de Bob) peut le vérifier ? Que penser des points que j'énumère à la page \pageref{PageAKTBooMDeQxY} au dessus du problème~\ref{ProbGAYFooZATuYy} ?
    \item
        Isomorphisme du corps \( \eR\). Que penser de la remarque~\ref{REMooGHEDooOYYUPk} ?
\end{enumerate}

%---------------------------------------------------------------------------------------------------------------------------
\subsection{Mes questions d'analyse}
%---------------------------------------------------------------------------------------------------------------------------

%///////////////////////////////////////////////////////////////////////////////////////////////////////////////////////////
\subsubsection{Facile}
%///////////////////////////////////////////////////////////////////////////////////////////////////////////////////////////


%///////////////////////////////////////////////////////////////////////////////////////////////////////////////////////////
\subsubsection{Moyen}
%///////////////////////////////////////////////////////////////////////////////////////////////////////////////////////////

\begin{enumerate}
    \item
        Définition de l'intégrale. Si \( f(x)=x^2\), alors \( \int_{\eR}f(x)dx\) utilise la définition de l'intégrale des fonctions partout positive en tant que supremum des fonctions étagées qui minorent \( f\). Pas de problèmes, \( \int_{\eR}x^2dx\) existe et vaut \( \infty\).

        Par contre \( g(x)=x^2-1\) possède une minuscule partie négative et doit donc prendre la définition des intégrales pour des fonctions de signe variable. Cette définition \ref{DefTCXooAstMYl} demande que les intégrales des parties positives et négatives soient séparément finies. Donc \( \int_{\eR} (x^2-1)dx\) n'existe pas.

        Ça me semble un peu restrictif. Il y a un problème à définir \( \int_{\eR}f\) lorsque la partie positive est infinie, mais la partie négative reste finie ?
    \item
        Que penser de la remarque~\ref{RemfdJcQF} qui dit qu'on doit avoir un théorème de complétion de partie orthonormale en une base orthonormale pour un espace de Hilbert ? C'est vrai ?
    \item
        Préciser l'énoncé et donner une démonstration de la proposition~\ref{PropMpBStL} qui traite de sommes dénombrables.
    \item
        Explosion en temps fini. C'est le corolaire~\ref{CorGDJQooNEIvpp}. Dans le cas où \( \lim_{t\to t_{max}} \| y(t) \|=\infty\) alors la dérivée de \( y\) n'est pas non plus bornée. Correct ?
    \item
        Est-ce qu'il y a moyen de définir la mesure produit (de deux espaces mesurés) sans passer par l'intégrale ? 

        Le théorème-définition \ref{ThoWWAjXzi} donne le produit de mesures par la formule
        \begin{equation}  
            (\mu_1\otimes \mu_2)(A)=\int_{\Omega_1}\mu_2\big( A_2(x) \big)d\mu_1(x)=\int_{\Omega_2}\mu_1\big( A_1(y) \big)d\mu_2(y).
        \end{equation}
        On peut faire plus abstrait ?

        Sans une telle définition, l'ordre imposé est :
        \begin{itemize}
            \item mesure
            \item intégrale
            \item mesure produit.
        \end{itemize}
        En particulier, quand on voit l'intégrale, la mesure de Lebesgue sur \( \eR^n\) n'est pas encore définissable.

        Il serait bien de pouvoir faire :
        \begin{itemize}
            \item mesure
            \item mesure de Lebesgue sur \( \eR\)
            \item mesure produit
            \item mesure de Lebesgue sur \( \eR^n\)
            \item intégrale.
        \end{itemize}

    \item
        La proposition \ref{PROPooMXCDooBffXbl} prouve que la fonction \( x\mapsto a^x\) (\( a>0\)) est dérivable. Pour ce faire, le concept de primitive est utilisé (pas jusqu'aux intégrales, cependant). Ça me semble incroyable qu'il faille ça pour prouver une dérivabilité.

        Prouver que la limite
        \begin{equation}
            \lim_{\epsilon\to 0}\frac{ a^{\epsilon}-1 }{ \epsilon }
        \end{equation}
        existe et n'est pas infinie sans recourir aux intégrales, aux primitives.
\end{enumerate}

%///////////////////////////////////////////////////////////////////////////////////////////////////////////////////////////
\subsubsection{Difficile}
%///////////////////////////////////////////////////////////////////////////////////////////////////////////////////////////

\begin{enumerate}
    \item
        Soit \( n\in \eN\), \( A\in \eR\) et \( x_0\in \eQ\). Nous considérons la suite\cite{BIBooMPXEooQLKhku}
        \begin{equation}
            x_{k+1}=\frac{1}{ n }\left( (n+1)x_k+\frac{ A }{ x_k^{n-1} } \right).
        \end{equation}
        Prouver que:
        \begin{itemize}
            \item C'est une suite de Cauchy
            \item \( x_k^n\to A\).
        \end{itemize}
        Il me faudrait une démonstration de cela sans passer par la méthode de Newton. Par exemple via le binôme de Newton.

        Mon but serait de définir \( a^{1/n}\) pour tout \( n\in \eN\) sans passer par de l'analyse (ou en tout cas pas en passant par le concept de fonction continue). Pour l'instant, c'est la définition \ref{DEFooJWQLooWkOBxQ} qui définit \( a^{1/n}\). Cela se base sur un argument de fonction continue strictement croissante pour obtenir une bijection.
    \item
        Est-ce que la proposition~\ref{PropooUEEOooLeIImr} qui donne le critère \( d(x_p,x_q)\leq \epsilon\) pour être une suite de Cauchy est valide dans un espace topologique métrique au lieu de normé ?  Dans quels cas a-t-on
        \begin{equation}
            d(a,b)=d(a+u)+d(b+u)
        \end{equation}
        lorsque \( d\) est une distance qui n'est pas spécialement induite d'une norme ?
    \item
        Soit \( V\), un espace vectoriel normé. Soit \( v\in V\). Est-ce qu'il existe un élément \( \varphi\in V'\) (application linéaire continue) telle que \( \varphi(v)=1\) et \( \| \varphi \|=1\) ?
    \item
        Que penser de~\ref{NORMooXTGBooKDnAhZ} qui tente d'expliquer pourquoi on ne définit pas l'intégrale d'une fonction non mesurable, malgré que le supremum qui la définirait existe forcément ?
    \item
        Changement de variables. À quel point la proposition \ref{PROPooILOEooBiumKD} est-elle équivalente au théorème usuel ?
\end{enumerate}

%///////////////////////////////////////////////////////////////////////////////////////////////////////////////////////////
\subsubsection{Non classées}
%///////////////////////////////////////////////////////////////////////////////////////////////////////////////////////////

\begin{enumerate}
    \item
        À propos de suites de Cauchy dans un espace vectoriel topologique et dans un espace métrique, est-ce que le théorème~\ref{THOooGQZSooAmQolf} est correct ?

        Soit \( V\) un espace vectoriel topologique métrisable\footnote{i.e. admet une base dénombrable de topologique, voir la proposition~\ref{PROPooPRLBooGtsRjr}}, alors il admet une métrique \( d\) compatible avec la topologie telle que une suite dans \( V\) est \( d\)-Cauchy si et seulement si elle est \( \tau\)-Cauchy.

        Dans cet ordre d'idée, il faut des exemples de :
        \begin{itemize}
            \item un espace vectoriel topologique métrisable et une métrique \( d\) compatible avec la topologie, mais dont les suites \( d\)-Cauchy ne sont pas celles \( \tau\)-Cauchy. Et en particulier dont la complétude est différente que celle de la «bonne» métrique donnée par la théorème~\ref{THOooAGBXooZnvQLK}.
            \item Et aussi un exemple pour la remarque~\ref{REMooUFQYooUVCCjs}.
        \end{itemize}
    \item
        L'exemple~\ref{ExfYXeQg} parle d'inverser une intégrale et une dérivée au sens des distributions pour prouver que la dérivée de \( \int_0^xg(t)dt\) par rapport à \( x\) est \( g\). Rendre cela rigoureux.
    \item
        À propos du théorème de récurrence de Poincaré~\ref{ThoYnLNEL}, l'application \( \phi\) doit être mesurable ? Répondre à la question posée sur la page de discussion de \wikipedia{fr}{Théorème_de_récurrence_de_Poincaré}{l'article sur wikipédia}.

        Toujours à propos du théorème de récurrence de Poincaré, il me semble qu'il y a un énoncé qui insiste sur la compacité de l'espace des phases et une démonstration utilisant la propriété de sous-recouvrement fini. Je serais content de retrouver cela. (ce serait sans doute mettable dans la leçon sur l'utilisation de la compacité)
        \item
            Dans \cite{OEVAuEz}, on parle de la proposition~\ref{PropZMKYMKI} à sa page \( 10\). Comment est-ce qu'on justifie le passage
            \begin{equation}
                \int_{\eR^d}T\big( y\mapsto \varphi(x)\psi(x-y) \big)dx=T\Big( y\mapsto\int_{\eR^d}\varphi(x)\psi(x-y)dx \Big).
            \end{equation}
            Sylvie Benzoni précise que «ceci demanderai quelques justifications». Où trouver lesdites justifications ? Il s'agit de permuter une distribution et une intégrale.

    \item
        Peut-on permuter une application linéaire et continue avec une somme pas spécialement dénombrable ? En supposant que \( \sum_{i\in I}f(v_i)\) existe, la proposition~\ref{PROPooWLEDooJogXpQ} semble dire que oui. Est-ce correct ?

        Peut-on avoir un exemple de partie sommable \( \{ v_i \}_{i\in I}\) et d'application linéaire continue \( f\) telle que la partie \( \{ f(v_i) \}\) ne soit pas sommable ?

        Peut-être ceci dans un espace de Hilbert.  \( v_i=\frac{1}{ i^2 }e_i\) puis \( f(e_i)=i^3e_i\).

    \item
        Soit une partie orthonormale \emph{pas spécialement dénombrable} \( \{ u_i \}_{i\in I}\) d'un espace de Hilbert (pas spécialement séparable). Si
        \begin{equation}
            x=\sum_{i\in I}x_iu_i,
        \end{equation}
        puis-je prendre le produit scalaire avec \( u_{k}\) et le permuter avec la somme pour déduire que \( x_k=\langle x, u_k\rangle \) ?

        C'est ce que je fais dans la proposition~\ref{PROPooWTOZooYZdlml}.
    \item
        Théorème de point fixe et équation différentielle. Que penser de l'exemple~\ref{EXooJXIGooQtotMc} qui itère la contraction de Cauchy-Lipschitz pour résoudre \( y'(t)=y(t)\), \( y(0)=1\) ? Est-ce que c'est générique comme comportement ? Est-ce que la convergence est efficace dans des cas moins triviaux ?
\end{enumerate}

%--------------------------------------------------------------------------------------------------------------------------- 
\subsection{Mes questions d'analyse numérique}
%---------------------------------------------------------------------------------------------------------------------------

\begin{enumerate}
    \item
        Différences finies. Il faut une analyse de consistance, stabilité et convergence du schéma à \( 9\) points pour le laplacien donné par \eqref{EQooKUMVooCVrzjt}. Je crois qu'il est d'ordre \( 6\), mais je n'en suis vraiment pas sûr.
\end{enumerate}

%---------------------------------------------------------------------------------------------------------------------------
\subsection{Mes questions de probabilité et statistiques.}
%---------------------------------------------------------------------------------------------------------------------------

%///////////////////////////////////////////////////////////////////////////////////////////////////////////////////////////
\subsubsection{Facile}
%///////////////////////////////////////////////////////////////////////////////////////////////////////////////////////////

%///////////////////////////////////////////////////////////////////////////////////////////////////////////////////////////
\subsubsection{Moyen}
%///////////////////////////////////////////////////////////////////////////////////////////////////////////////////////////

\begin{enumerate}
    \item
        Soit une variable aléatoire \( X\) à valeurs réelles. Est-ce que la tribu engendrée par \( X\) est d'une façon ou d'une autre engendrée par les «courbes de niveau» de \( X\) ? C'est-à-dire par les \( X^{-1}\big( \{ t \} \big)\) pour les \( t\in \eR\).

        C'est ce qui semble ressortir de l'exemple de~\ref{SUBSECooWOOGooVxflVZ}. Et intuitivement, je trouve que ça irait bien \ldots
\end{enumerate}


%///////////////////////////////////////////////////////////////////////////////////////////////////////////////////////////
\subsubsection{Difficile}
%///////////////////////////////////////////////////////////////////////////////////////////////////////////////////////////

%///////////////////////////////////////////////////////////////////////////////////////////////////////////////////////////
\subsubsection{Non classées}
%///////////////////////////////////////////////////////////////////////////////////////////////////////////////////////////

%---------------------------------------------------------------------------------------------------------------------------
\subsection{Mes questions de \LaTeX\ et programmation}
%---------------------------------------------------------------------------------------------------------------------------

%///////////////////////////////////////////////////////////////////////////////////////////////////////////////////////////
\subsubsection{Facile}
%///////////////////////////////////////////////////////////////////////////////////////////////////////////////////////////

\begin{enumerate}
    \item
        Comment faire en sorte que les mots commençant par «é» soient avec les «e» dans l'index, et non avant les «a» ? Il me faudrait un mécanisme plus automatique que faire \info{machin@truc}.
    \item 
        Il y a des problèmes dans la table des matières.  « Table des matières », « Index », et « Liste des notations » ne pointent pas vers la bonne page.
\end{enumerate}

%///////////////////////////////////////////////////////////////////////////////////////////////////////////////////////////
\subsubsection{Moyen}
%///////////////////////////////////////////////////////////////////////////////////////////////////////////////////////////

\begin{enumerate}
    \item
        Revoir le mécaniste de l'index thématique. Il faudrait pouvoir les trier avec des titres. Mais attention : il doit arriver avant la table des matières.
\end{enumerate}

%///////////////////////////////////////////////////////////////////////////////////////////////////////////////////////////
\subsubsection{Difficile}
%///////////////////////////////////////////////////////////////////////////////////////////////////////////////////////////

Je ne sais pas comment faire, et à mon avis il faudra innover.

\begin{enumerate}
    \item
        Si vous savez comment faire \info{pdf --> epub} pour créer un eBook, faites le moi savoir. Cahier des charges :
        \begin{itemize}
            \item libre, disponible sur Ubuntu
            \item en ligne de commande (en tout cas : exécutable depuis un script en python ou C++)
        \end{itemize}
        Attention : le Frido étant un truc assez compliqué, avant de répondre la première chose qui vous passe par la tête, assurez-vous que votre solution fait avancer les choses sur le Frido et non sur un petit document de test.

        Nous en avons déjà un peu discuté sur \url{https://github.com/LaurentClaessens/mazhe/issues/13}. Il faudra entre autres faire un script qui remplace tous les environnements tizk des fichiers \info{*.pstricks} (désolé pour la convention de nommage historique) par un simple \info{includegraphics} du fichier \info{pdf} correspondant que l'on trouvera dans le répertoire \info{auto/pictures\_tikz}.
    \item
        Écrire un script (en python ou autre) qui prend en argument deux numéros ou noms de chapitres et qui retourne l'ensemble des lignes du premier qui contient des \info{ref} ou \info{eqref} dont le label correspondant est dans le second.

        Attention : il faut tenir compte de \info{input} de façon récursive.

        Bonus : calculer le hash sha1 de chaque ligne du résultat et ne pas l'afficher si il se trouve dans la liste du fichier \info{commons.py}.
\end{enumerate}

%--------------------------------------------------------------------------------------------------------------------------- 
\subsection{Numérique}
%---------------------------------------------------------------------------------------------------------------------------

%///////////////////////////////////////////////////////////////////////////////////////////////////////////////////////////
\subsubsection{Moyen}
%///////////////////////////////////////////////////////////////////////////////////////////////////////////////////////////

\begin{enumerate}
    \item
        L'erreur de cancellation provoquée par la différence \( a-\tilde a\) lorsque \( a\) et \( \tilde a\) sont proches n'a pas de conséquences sur l'ordre de grandeur de la réponse. Seulement des conséquences sur la valeur des chiffres significatifs. Vrai ou faux ?

        Voir la remarque~\ref{REMooRQIJooNLdAZE}.
\end{enumerate}

%+++++++++++++++++++++++++++++++++++++++++++++++++++++++++++++++++++++++++++++++++++++++++++++++++++++++++++++++++++++++++++ 
\section{Taper du code pour le Frido}
%+++++++++++++++++++++++++++++++++++++++++++++++++++++++++++++++++++++++++++++++++++++++++++++++++++++++++++++++++++++++++++

Dans cette section nous donnons quelques indications sur la façon de taper du code \LaTeX\ pour le Frido.

Tout commence par télécharger les sources à l'adresse
\begin{center}
    \url{https://github.com/LaurentClaessens/mazhe}
\end{center}

%--------------------------------------------------------------------------------------------------------------------------- 
\subsection{Pour compiler le document vous même}
%---------------------------------------------------------------------------------------------------------------------------

Lisez le fichier \info{COMPILATION.md}.

%---------------------------------------------------------------------------------------------------------------------------
\subsection{Nommage des fichiers \info{tex}}
%---------------------------------------------------------------------------------------------------------------------------

J'ai pris l'habitude de préfixer les noms par un nombre. Par exemple \info{139\_EspacesVecto}. Le fait est qu'il est plus simple, pour ouvrir le ficher, de taper \info{139<TAB>} que de se souvenir si «EspacesVecto» est écrit avec une majuscule, en français, en anglais, \ldots

De plus un chapitre contenant plusieurs fichiers, nous nous retrouvons rapidement avec beaucoup de fichiers nommés \info{EspacesVecto1}, \info{EspacesVecto2}, etc.

Vous trouverez le prochain numéro disponible dans \info{réserve.tex}.

%---------------------------------------------------------------------------------------------------------------------------
\subsection{Inclure des exemples de code}
%---------------------------------------------------------------------------------------------------------------------------

Pour inclure du code Sage, nous utilisons la commande \info{\textbackslash lstinputlisting}. Ici encore, le fichier \info{réserve.tex} contient le prochain disponible.

%---------------------------------------------------------------------------------------------------------------------------
\subsection{Pour les exercices}
%---------------------------------------------------------------------------------------------------------------------------

ATTENTION : dans un futur proche, je vais supprimer tous les exercices et les mettre sous forme d'exemples. Une des raisons est de supprimer la dépendance en le paquet personnel \info{exocorr} qui rend compliqué la compilation du Frido par des tierces personnes.

\vspace{1cm}

Les exercices sont tapés dans les fichiers déjà pré-remplis \info{src\_exocorr/exo*.tex}. Les corrections sont dans le fichier \info{src\_exocorr/corr*.tex} correspondant. Ces fichiers ne sont pas inclus directement, mais via la macro \info{\textbackslash Exo}.

Le fichier \info{réserve.tex} contient le prochain disponible.

\paragraph{Exemple}

Vous voulez créer un exercice.
\begin{itemize}
    \item Allez voir dans \info{réserve.tex} la prochaine ligne \info{Exo} disponible.
    \item Mettons que ce soit   \info{\textbackslash Exo\{mazhe-0018\}}
    \item Supprimez cette ligne de \info{réserve.tex}, et mettez la où vous voulez voir paraitre votre exercice.
    \item Tapez votre exercice dans le fichier \info{src\_exocorr/exomazhe-0018.tex} et votre correction dans le fichier \info{src\_exocorr/corrmazhe-0018.tex}. Ces fichiers sont déjà créés et pré-remplis. Ne changez pas le code qui y est.
\end{itemize}

%+++++++++++++++++++++++++++++++++++++++++++++++++++++++++++++++++++++++++++++++++++++++++++++++++++++++++++++++++++++++++++
\section{Les politiques éditoriales}
%+++++++++++++++++++++++++++++++++++++++++++++++++++++++++++++++++++++++++++++++++++++++++++++++++++++++++++++++++++++++++++

Certaines parties de ce texte ne respectent pas les politiques éditoriales. Ce sont des erreurs de jeunesse, et j'en suis le premier triste.

%---------------------------------------------------------------------------------------------------------------------------
\subsection{Licence libre}
%---------------------------------------------------------------------------------------------------------------------------

Je crois que c'est clair.

%---------------------------------------------------------------------------------------------------------------------------
\subsection{pdflatex}
%---------------------------------------------------------------------------------------------------------------------------

Tout est compilable avec pdf\LaTeX. Pas de pstricks, de psfrag ou de ps<quoiquecesoit>.

%---------------------------------------------------------------------------------------------------------------------------
\subsection{utf8}
%---------------------------------------------------------------------------------------------------------------------------

Je crois que c'est clair.

%---------------------------------------------------------------------------------------------------------------------------
\subsection{Notations}
%---------------------------------------------------------------------------------------------------------------------------

On essaie d'être cohérent dans les notations et les conventions. Pour la transformée de Fourier par exemple, je crois que la définition du produit scalaire dans \( L^2\), des coefficients de Fourier, de la transformation et de la transformation inverse sont cohérents. Cela demande, lorsqu'on suit un livre qui ne suit pas les conventions utilisées ici, de convertir parfois massivement.

%---------------------------------------------------------------------------------------------------------------------------
\subsection{De la bibliographie}
%---------------------------------------------------------------------------------------------------------------------------

On évite d'écrire en haut de chapitre «les références pour ce chapitre sont \ldots». Il est mieux d'écrire au niveau des théorèmes, entre parenthèses, les références.

Lorsqu'on écrit l'énoncé d'un théorème sans retranscrire la démonstration, il faut mettre une référence vers un document \emph{en ligne} qui en contient la preuve. Il est vraiment fastidieux de chercher une preuve sur internet et de tomber sur des dizaines de documents qui donnent l'énoncé mais pas la preuve.

%---------------------------------------------------------------------------------------------------------------------------
\subsection{Faire des références à tout}
%---------------------------------------------------------------------------------------------------------------------------

Lorsqu'un utilise le théorème des accroissements finis, il ne faut pas écrire «d'après le théorème des accroissements finis, blablabla». Il faut écrire un \verb+\ref+ explicite vers le résultat. Cela alourdit un peu le texte, mais lorsqu'on joue avec un texte de plus de 2000 pages, il est parfois laborieux de trouver le résultat qu'on cherche (surtout s'il existe plusieurs versions d'un résultat et que l'on veut faire référence à une version particulière).

%---------------------------------------------------------------------------------------------------------------------------
\subsection{Des listes de liens internes}
%---------------------------------------------------------------------------------------------------------------------------

Le début du Frido contient une espèce d'index thématique. Il serait bon de l'étoffer.

%---------------------------------------------------------------------------------------------------------------------------
\subsection{Pas de références vers le futur}
%---------------------------------------------------------------------------------------------------------------------------

Dans le Frido, \emph{aucune} preuve ne peut faire une référence vers un résultat prouvé plus bas. On n'utilise pas le théorème 10 dans la démonstration du théorème 7. Cela est une contrainte forte sur le découpage en chapitres et sur l'ordre de présentation des matières.

Il est bien entendu accepté et même encouragé de mettre des notes du type «Nous verrons plus loin un théorème qui \ldots». Tant que ce théorème n'est pas \emph{utilisé}, ça va.

En faisant
\begin{quote}
    \begin{verbatim}
    pytex lst_frido.py --verif
    \end{verbatim}
\end{quote}
vous aurez une liste des références vers le bas. Cette liste doit être vide ! Ce programme cherche tous les \verb+\ref+ et \verb+\eqref+ ainsi que les \verb+\label+ correspondants et vous prévient lorsque le \verb+\label+ est après le \verb+\ref+.

Si vous pensez qu'une référence pointée doit être acceptée (par exemple parce c'est dans une des listes de liens internes), alors vous ajoutez son hash dans la liste du fichier \info{commons.py}. Si il s'agit vraiment d'une référence vers un résultat que vous utilisez, alors vous devez déplacer des choses. Soit votre résultat vers le bas, soit celui que vous utilisez vers le haut. Parfois cela demande de déplacer ou redécouper des chapitres entiers\ldots\ Si il n'y a vraiment pas moyen, bravo, vous venez de prouver que la mathématique est logiquement inconsistante.

%--------------------------------------------------------------------------------------------------------------------------- 
\subsection{Écriture inclusive}
%---------------------------------------------------------------------------------------------------------------------------

Je suis triste de devoir le préciser, mais le Frido est écrit en français. Nous n'utiliserons donc pas de féminisation abusives, et accepterons comme correcte des tournures comme, en parlant d'une fonction, «\emph{elle} est \emph{un} contre-exemple», ou en parlant d'un lemme que «\emph{il} est \emph{une} conséquence».

Parfois le genre d'un objet n'est pas bien défini. Par exemple \( 3/4\) est \emph{la} classe d'équivalence de \( (3,4)\) dans \( \eZ\times \eZ\setminus{0}\); mais en même temps c'est \emph{un} élément de \( \eQ\). Nous utiliserons alors, prudemment, un neutre en disant «\emph{il} est plus petit que \( 1\)».

%+++++++++++++++++++++++++++++++++++++++++++++++++++++++++++++++++++++++++++++++++++++++++++++++++++++++++++++++++++++++++++
\section{Vérifier si vous n'avez pas fait de bêtises}
%+++++++++++++++++++++++++++++++++++++++++++++++++++++++++++++++++++++++++++++++++++++++++++++++++++++++++++++++++++++++++++

Lorsqu'on fait de lourdes modifications (déplacement de parties, fusion de théorèmes, etc) il est toujours possible de faire des bêtises d'au moins deux types : créer des références vers le futur et supprimer des parties (genre couper-coller en oubliant le coller). Pour s'en prémunir, le script suivant lance quelques compilations et vérifications :

\begin{verbatim}
./testing.sh
\end{verbatim}
Aucune erreur ne devrait être signalée.

Attention : ce script fait quelques manipulations à base de \info{git stash} et crée une nouvelle branche (nom aléatoire assez long) pour tester votre dernière modification sans créer de commit.

%+++++++++++++++++++++++++++++++++++++++++++++++++++++++++++++++++++++++++++++++++++++++++++++++++++++++++++++++++++++++++++ 
\section{Acceptation des contributions}
%+++++++++++++++++++++++++++++++++++++++++++++++++++++++++++++++++++++++++++++++++++++++++++++++++++++++++++++++++++++++++++

TD;DR : pratiquement aucun patch n'est refusé.

Le premier critère d'acceptation d'une contribution est évidemment la correction mathématique.

%--------------------------------------------------------------------------------------------------------------------------- 
\subsection{Attention aux expressions rationnelles}
%---------------------------------------------------------------------------------------------------------------------------

Si vous trouvez une faute d'orthographe, rien ne vous empêche de faire une recherche de la même faute pour la corriger d'un seul coup dans \( 25\) fichiers. Faites toutefois attention à des remplacements automatiques sur base d'expressions rationnelles telles que
\begin{verbatim}
    des [a-z]*[^s]
\end{verbatim}
qui serait supposé détecter des erreurs de pluriel. Je vous laisse trouver au moins \( 5\) cas sans fautes qui satisfont cette expression.

Utilisez de telles expressions pour \emph{trouver} des fautes, pas pour les corriger.

%--------------------------------------------------------------------------------------------------------------------------- 
\subsection{Pas de modifications massives, automatiques pour des raisons cosmétiques}
%---------------------------------------------------------------------------------------------------------------------------

Il est un type de contributions que je ne vais plus accepter, ce sont les modifications massives et automatiques de \emph{tous} les fichiers pour des raisons de «propreté» du code. Exemples : 
\begin{itemize}
    \item Supprimer automatiquement tous les espaces en bout de lignes,
    \item Supprimer automatiquement toutes les lignes vides en fin de fichier
    \item Remplacer \info{\textbackslash ref} par \info{\textasciitilde\textbackslash ref}
    \item Remplacer \info{\textbackslash [} par \info{\textbackslash begin\{equation\}}
    \item Remplacer \info{\textbackslash og} par \info{«} (avec ou sans espaces devant ou derrière)
    \item \ldots
\end{itemize}

Ce type de substitutions automatiques créent des patch gigantesques qui prennent un temps astronomique à relire pour un bénéfice pas tellement évident.

Pire : ils ont des effets de bords pas toujours évident à détecter ou à prévoir.

N'oubliez pas que le Frido n'est pas que du \LaTeX. Il est aussi divers scripts de pré et post-compilation (y compris qui hackent des fichiers intermédiaires entre deux passes de \LaTeX). Ces scipts, écrits par votre très humble et très obéissant serviteur, ne sont pas parfaits et les parseurs reposent sur certaines hypothèses. Donc des choses qui ne devraient ne rien changer du point de vue de \LaTeX\ peuvent avoir des conséquences.

Bien entendu, si vous êtes en train de taper des math et que ce genre de «malpropreté» du code vous gêne, vous pouvez corriger dans les fichiers que vous modifiez.


\emptyInputPath
\addInputPath{tex/frido}

\chapter{Construction des ensembles de nombres}
% This is part of Mes notes de mathématique
% Copyright (c) 2011-2020
%   Laurent Claessens
% See the file fdl-1.3.txt for copying conditions.

%+++++++++++++++++++++++++++++++++++++++++++++++++++++++++++++++++++++++++++++++++++++++++++++++++++++++++++++++++++++++++++
\section{Quelques éléments sur les ensembles}
%+++++++++++++++++++++++++++++++++++++++++++++++++++++++++++++++++++++++++++++++++++++++++++++++++++++++++++++++++++++++++++

%---------------------------------------------------------------------------------------------------------------------------
\subsection{Petit mot d'introduction}
%---------------------------------------------------------------------------------------------------------------------------

\begin{normaltext}

Le Frido n'est pas supposé être lu dans l'ordre de la première à la dernière page; les matières y sont présentées dans l'ordre logique mathématique, et non dans l'ordre logique pédagogique, et encore moins par ordre de difficulté croissante.

En mathématique, si on lit une démonstration et que l'on veut vraiment tout justifier, et justifier toutes les étapes de tous les résultats utilisés, on tombe forcément un jour sur les axiomes.

Or l'axiomatique est un sujet particulièrement difficile. Nous n'allons donc pas «tout justifier» jusque là. Nous n'allons même pas préciser quel système d'axiome est utilisé. En particulier nous n'allons pas donner l'axiomatique des ensembles : nous allons supposer connus les ensembles et leurs principales propriétés.

Bref. Nous supposons avoir une théorie des ensembles qui tient la route. En particulier nous supposons connues les notions suivantes :
\begin{enumerate}
    \item
        ensemble vide,
    \item
        ensemble, appartenance, intersection, union,
    \item
        application entre deux ensembles, notation \( f(x)\) pour désigner l'image de \( x\) par \( f\),
    \item
        produit cartésien de plusieurs ensembles.
\end{enumerate}
Ce sont toutes des choses dont la construction à partir des axiomes n'est en aucun cas évidente. En particulier, des «définitions» comme «l'intersection de deux ensembles est l'ensemble contenant exactement les éléments communs aux deux ensembles» ne sont pas correctes parce qu'elles passent à côté de l'existence et de l'unicité d'un tel ensemble.
\end{normaltext}

Une petite définition cependant, que l'on utilisera :
\begin{definition}\label{DefEnsemblesDisjoints}
    Deux ensembles $A$ et $B$ sont \defe{disjoints}{ensembles!disjoints} si leur intersection est vide\footnote{Remarquez que les mots «intersection» et «vide» sont de ceux que nous avons décidé de ne pas définir.}; en d'autres termes, s'il n'existe aucun élément commun à $A$ et $B$.
\end{definition}

\begin{normaltext}
    Remarquez par exemple que la première phrase de l'article de Wikipédia sur la construction de \( \eN\) est «Partant de la théorie des ensembles, on identifie 0 à l'ensemble vide, puis on construit \ldots». Il est bien précisé que l'on part d'une théorie des ensembles.
\end{normaltext}

\begin{normaltext}
    La suite de ce chapitre sera essentiellement sans exemples parce qu'avant d'avoir construit les ensembles de nombres, je ne sais pas très bien quels exemples on peut donner de quoi que ce soit.
\end{normaltext}

%---------------------------------------------------------------------------------------------------------------------------
\subsection{Ensemble ordonné}
%---------------------------------------------------------------------------------------------------------------------------

\begin{definition}
    Soient deux ensembles \( E\) et \( F\). Une application \( f\colon E\to F\) est
    \begin{enumerate}
        \item
            \defe{surjective}{surjection} si pour tout \( y\in F\), il existe \( x\in E\) tel que \( y=f(x)\);
        \item
            \defe{injective}{injection} si pour tout \( y\in F\), il existe au plus un \(x\in E \) tel que \( y=f(x)\);
        \item
            \defe{bijective}{bijection} si elle est à la fois injective et surjective.
    \end{enumerate}
\end{definition}
La méthode la plus courante pour démontrer qu'une application \( f\colon E\to F\) est injective est de considérer \( x,y\in E\) tels que \( f(x)=f(y)\), et de prouver à partir de là que \( x=y\). Ou alors de supposer \( x\neq y\) et obtenir une contradiction.

La technique de la contradiction est évidemment la plus courante lorsque l'égalité \( f(x)=g(x)\) implique une équation faisant intervenir \( 1/(x-y)\).

\begin{normaltext}\label{NORooLMBYooYjUoju}
L'\defe{axiome du choix}{axiome!du choix} que nous acceptons peut s'énoncer comme ceci\cite{ooKLIXooHbpufL} : Étant donné un ensemble X d'ensembles non vides, il existe une fonction définie sur X, appelée fonction de choix, qui à chacun d'entre eux associe un de ses éléments.
\end{normaltext}

\begin{definition}      \label{DefooFLYOooRaGYRk}
    Une \defe{relation d'ordre}{ordre} sur un ensemble \( E\) est une relation binaire (notée \( \leq\)) sur \( E\) telle que pour tous \( x,y,z\in E\),
    \begin{description}
        \item[réflexivité] : \( x\leq x\)
         \item[antisymétrie] : \( x\leq y\) et \( y\leq x\) implique \( x=y\)
         \item[transitivité] : \( x\leq y\) et \( y\leq z\) implique \( x\leq z\).
    \end{description}
\end{definition}

\begin{definition}      \label{DEFooVGYQooUhUZGr}
    Un ensemble ordonné est \defe{totalement ordonné}{ordre!total} si deux éléments sont toujours comparables : si \( x,y\in E\) alors nous avons soit \( x\leq y\) soit \( y\leq x\). Si les éléments ne sont pas tous comparables, nous disons que l'ordre est \defe{partiel}{ordre!partiel}.
\end{definition}

\begin{definition}
    Soit un ensemble ordonné \( (E,\leq)\) et une partie \( A\) de \( E\). Nous disons que \( m\in A\) est un \defe{minimum}{minimum!ensemble ordonné} de \( A\) si pour tout \( x\in A\), l'élément \( m\) est comparable à \( x\) et \( m\leq x\).

    Un élément \( p\in E\) est un \defe{minorant}{minorant} de \( A\) si pour tout \( a\in A\), l'élément \( p\) et \( a\) sont comparables et \( p\leq a\).

    Les notions de \defe{maximum}{maximum} et de \defe{majorant}{majorant} sont définies de façon analogue.
\end{definition}

Lorsqu'une partie possède un minimum, ce dernier est nommé le «plus petit élément» de la partie. Attention : il n'en existe pas toujours. D'innombrables exemples pourront être vus lorsque nous aurons construits \( \eQ\) et \( \eR\). Typiquement les intervalles du type \( \mathopen] a , b \mathclose[\).

\begin{definition}   \label{DEFooLJEAooBLGsiS}
    Un ensemble ordonné est \defe{bien ordonné}{bon!ordre}\index{ordre!bon ordre} si toute partie non vide possède un plus petit élément : si \( A\) est une partie de \( E\), alors \( \exists x\in A,\forall y\in A, x\leq y\).
\end{definition}

\begin{normaltext}
    Quelques remarques.
    \begin{enumerate}
        \item
            L'inégalité stricte (définie par: \( x<y\) si et seulement si \( x\leq y\) et \( x\neq y\)) n'est pas une relation d'ordre parce qu'elle n'est pas réflexive.
        \item
            Nous verrons dans la remarque~\ref{REMooXOIOooHjwMvA} que l'intervalle \( \mathopen[ -1 , 1 \mathclose]\) dans \( \eR\) n'est pas bien ordonné.
        \item
            Un ensemble bien ordonné est forcément totalement ordonné parce que toutes les parties de la forme \( \{ x,y \}\) possèdent un minimum. Par conséquent \( x\) et \( y\) doivent être comparables : \( x\leq y\) ou \( y\leq x\).
    \end{enumerate}
\end{normaltext}

\begin{example}
    Si \( E\) est un ensemble, l'inclusion est un ordre sur l'ensemble des parties de \( E\), mais pas un ordre total parce que si \( X,Y\) sont des parties de \( E\), alors nous n'avons pas automatiquement soit \( X\subset Y\) ou \( Y\subset X\).
\end{example}

La notion d'ordre permet d'introduire la notion d'intervalle.

\begin{definition}  \label{DefEYAooMYYTz}
    Soit un ensemble totalement ordonné \( (E,\leq)\). Un \defe{intervalle}{intervalle} de \( E\) est une partie \( I\) telle que tout élément compris entre deux éléments de \( I \) soit dans \( I \). En formule, la partie \( I \) de \( E\) est un intervalle si
    \[
      \forall a,b\in I,(a\leq x\leq b)\Rightarrow x\in I.
    \]
\end{definition}

%---------------------------------------------------------------------------------------------------------------------------
\subsection{Lemme de Zorn}
%---------------------------------------------------------------------------------------------------------------------------

\begin{definition}[Ensemble inductif\cite{MathAgreg}]  \label{DefGHDfyyz}
    Un ensemble est \defe{inductif}{inductif} si tout sous-ensemble totalement ordonné admet un majorant.
\end{definition}


\begin{lemma}[Lemme de Zorn\cite{BIBooYDIJooWCVynX}]    \label{LemUEGjJBc}
    Tout ensemble ordonné inductif non vide admet au moins un élément maximal.
\end{lemma}
\index{lemme!de Zorn}

%---------------------------------------------------------------------------------------------------------------------------
\subsection{Complémentaire}
%---------------------------------------------------------------------------------------------------------------------------
\label{AppComplement}

\begin{definition}
    Soit $E$, un ensemble et $A$, une partie de $E$ (c'est-à-dire un sous-ensemble de $E$). Le \defe{complémentaire}{complémentaire} de l'ensemble $A$, dans $E$, noté $\complement A$\nomenclature[T]{$\complement A$}{Le complémentaire de l'ensemble $A$} est l'ensemble des éléments de $E$ qui ne font pas partie de $A$ :
    \begin{equation}
	    \complement A=E\setminus A=\{ x\in E\tq x\notin A \}.
    \end{equation}
\end{definition}

Nous allons aussi régulièrement noter le complémentaire de \( A\) par \( A^c\)\nomenclature[T]{\( A^c\)}{complémentaire de \( A\)}.

\begin{lemma}		\label{LemPropsComplement}
	Quelques propriétés à propos des complémentaires. Si $E$ est un ensemble et si $A$ et $B$ sont des sous-ensembles de $E$, nous avons
	\begin{enumerate}
		\item
			$\complement \complement A =A $, en d'autres termes, $E\setminus(E\setminus A)=A$,
		\item
			$\complement(A\cap B)=\complement A\cup\complement B$,
		\item
			$\complement(A\cup B)=\complement A\cap\complement B$,
		\item	\label{ItemLemPropComplementiii}
			$A\setminus B=A\cap\complement B$.
	\end{enumerate}
\end{lemma}

\begin{definition}[différence symétrique]    \label{DefBMLooVjlSG}
    Si \( A\) et \( B\) sont des ensembles, l'ensemble \( A\Delta B\)\nomenclature[T]{\( A\Delta B\)}{différence symétrique} est la \defe{différence symétrique}{ensemble!différence symétrique} d'ensembles :
    \begin{equation}
        A\Delta B=(A\cup B)\setminus(A\cap B).
    \end{equation}
\end{definition}
C'est l'ensemble des éléments étant soit dans \( A\) soit dans \( B\) mais pas dans les deux.

\begin{lemma}   \label{LemCUVoohKpWB}
    Si \( A\) et \( B\) sont des ensembles nous avons
    \begin{enumerate}
        \item\label{ItemVUCooHAztC}
            \( A^c\Delta B^c=A\Delta B\).
        \item\label{ItemVUCooHAztCii}
            \( (A\Delta B)\Delta B=A\).
    \end{enumerate}
\end{lemma}

\begin{proof}
    D'abord nous avons l'égalité \( X^c\setminus Y^c=Y\setminus X\). Cela se prouve de façon classique en séparant deux cas selon que \( B\) soit inclus dans \( A\) ou non.

    De là nous avons la première assertion :
    \begin{equation}
        A^c\Delta B^c=(A^c\cup B^c)\setminus(A^c\cap B^c)=(A\cap B)^c\setminus(A\cup B)^c=(A\cup B)\setminus (A\cap B)=A\Delta B.
    \end{equation}

    Pour la seconde assertion, il faut remarquer que \( (A\Delta B)\cup B=A\cup B\) et que \( (A\Delta B)\cap B=B\setminus A\), donc
    \begin{equation}
        (A\Delta B)\Delta B=(A\cup B)\setminus (B\setminus A)=A.
    \end{equation}
\end{proof}

%---------------------------------------------------------------------------------------------------------------------------
\subsection{Relations d'équivalence}
%---------------------------------------------------------------------------------------------------------------------------
\label{appEquivalence}

\begin{definition}  \label{DefHoJzMp}
Si $E$ est un ensemble, une \defe{relation d'équivalence}{relation d'équivalence} sur $E$ est une relation $\sim$ qui est à la fois
\begin{description}
	\item[réflexive] $x\sim x$ pour tout $x\in E$,
	\item[symétrique] $x\sim y$ si et seulement si $y\sim x$;
	\item[transitive] si $x\sim y$ et $y\sim z$, alors $x\sim z$.
\end{description}
\end{definition}

\begin{definition}      \label{DEFooRHPSooHKBZXl}
    Si \( E\) est un ensemble et si \( \sim\) est une relation d'équivalence sur \( E\), alors nous notons \( E/\sim\) l'\defe{ensemble quotient}{ensemble quotient}, c'est-à-dire l'ensemble des classes d'équivalence dans \( E\). Un élément de \( E/\sim\) est de la forme
    \begin{equation}
        [a]=\{ x\in E\tq x\sim a \}.
    \end{equation}
\end{definition}

\begin{lemma}
    Soit un ensemble \( E\) et une relation d'équivalence \( \sim\). Pour \( a,b\in E\), nous avons \( [a]=[b]\) si et seulement si \( a\sim b\).
\end{lemma}

\begin{proof}
    En deux parties.
    \begin{subproof}
        \item[\( \Rightarrow\)]
            Nous supposons que \( [a]=[b]\). Vu que \( a\sim a\), nous avons \( a\in [a]=[b]\). Mais \( a\in [b]\) signifie \( a\sim b\), ce qu'il fallait.
        \item[\( \Leftarrow\)]
            Nous supposons que \( a\sim b\), et nous démontrons que \( [a]\subset [b]\) (pour l'inclusion inverse, vous devriez vous en sortir par tout seul). Si \( x\in [a]\), alors \( x\sim a\). Mais \( a\sim b\). Donc \( x\sim a\sim b\), ce qui implique \( x\sim b\) par transitivité. Or dire \( x\sim b\) implique \( x\in [b]\).
    \end{subproof}
\end{proof}

\begin{example}
    Sur l'ensemble de tous les polygones du plan, la relation «a le même nombre de côtés» est une relation d'équivalence. Plus précisément, si $P$ et $Q$ sont deux polygones, nous disons que $P\sim Q$ si et seulement si $P$ et $Q$ ont le même nombre de côtés. Cela est une relation d'équivalence :
    \begin{itemize}
        \item
            un polygone $P$ a toujours le même nombre de côtés que lui-même : $P\sim P$;
        \item
            si $P$ a le même nombre de côtés que $Q$ ($P\sim Q$), alors $Q$ a le même nombre de côtés que $P$ ($Q\sim P$);
        \item
            si $P$ a le même nombre de côtés que $Q$ ($P\sim Q$) et que $Q$ a le même nombre de côtés que $R$ ($Q\sim R$), alors $P$ a le même nombre de côtés que $R$ ($P\sim R$).
    \end{itemize}
\end{example}

\begin{example}
Soit \( f\) une application entre deux ensembles \( E\) et \( F\). Nous définissons une relation d'équivalence sur \( E\) par
\begin{equation}
    x\sim y\Leftrightarrow f(x)=f(y).
\end{equation}
Nous notons par \( \pi\colon E\to E/\sim\) la projection canonique. L'application
\begin{equation}
    \begin{aligned}
        g\colon E/\sim&\to F \\
        [x]&\mapsto f(x)
    \end{aligned}
\end{equation}
est bien définie et injective. Elle n'est pas surjective tant que \( f\) ne l'est pas. La \defe{décomposition canonique}{canonique!décomposition}\index{décomposition!canonique} de \( f\) est
\begin{equation}
    f=g\circ\pi.
\end{equation}
\end{example}

%+++++++++++++++++++++++++++++++++++++++++++++++++++++++++++++++++++++++++++++++++++++++++++++++++++++++++++++++++++++++++++
\section{Les naturels}
%+++++++++++++++++++++++++++++++++++++++++++++++++++++++++++++++++++++++++++++++++++++++++++++++++++++++++++++++++++++++++++
\label{SECooPJSYooNYaIaq}
% Lorsque ce chapitre est fait, changer la phrase qui le référentie dans la partie sur la constante de Weiner.

Toutes les constructions sont faites dans \cite{RWWJooJdjxEK}. Les résultats énoncés ici sont utilisés plus bas et servent de guide à un éventuel contributeur qui voudrait écrire une partie sur la construction de \( \eN\) et \( \eZ\). Nous espérons que des preuves se trouvent dans \cite{RWWJooJdjxEK}. En tout cas, le lecteur est invité à ne pas les prendre pour évidents.

Nous supposons en particulier que \( \eN\) est construit avec sa relation d'ordre. Voici quelque affirmations que nous admettons.

\begin{lemma}       \label{LEMooYMRJooYIAhBb}
    Quelques affirmations sur l'ordre dans \( \eN\).
    \begin{enumerate}
        \item
            Il n'existe pas de \( n\in \eN\) tel que \( n<0\).
        \item
            Si \( a,b\in \eN\) vérifient \( a>b\), alors il n'existe pas de \( x\) dans \( \eN\) tel que \( a+x=b\).
        \item       \label{ITEMooNHRIooODBVNK}
            Toute partie non vide de \( \eN\) possède un unique élément minimal.
    \end{enumerate}
\end{lemma}

\begin{lemma}       \label{LEMooFHEOooSHPGgU}
    Toute partie non vide de \( \eN\) possède un unique minimum.
\end{lemma}


%+++++++++++++++++++++++++++++++++++++++++++++++++++++++++++++++++++++++++++++++++++++++++++++++++++++++++++++++++++++++++++ 
\section{Quelques résultats de cardinalité}
%+++++++++++++++++++++++++++++++++++++++++++++++++++++++++++++++++++++++++++++++++++++++++++++++++++++++++++++++++++++++++++

Je vous conseille fortement de ne pas considérer les résultats qui viennent comme évidents avant d'avoir lu quelques articles de Wikipédia sur la construction des naturels en théorie des ensembles.

Les notions d'équipotence, surpotence et de subpotence permettent de comparer les «tailles» des ensembles sans avoir besoin de la théorie des ordinaux. Tout ceci ne sera pas très souvent utile par la suite. Un exemple d'utilisation de ces notions est le théorème de Steinitz \ref{THOooEDQKooLEGlDv} qui démontre l'existence de clôture algébrique pour tout corps.

\begin{definition}[\cite{BIBooAKHUooProFGE,BIBooWNKRooETlebF}]      \label{DEFooXGXZooIgcBCg}
    Soient deux ensembles \( A\) et \( B\).
    \begin{enumerate}
        \item
            Les ensembles \( A\) et \( B\) sont \defe{équipotents}{équipotent} si il existe une bijection entre \( A\) et \( B\). Nous notons \( A\approx B\).
        \item
            L'ensemble \( A\) est \defe{surpotent}{surpotent} à \( B\) si il existe une surjection de \( A\) vers \( B\). Nous notons \( A\succeq B\).
        \item
            L'ensemble \( A\) est \defe{subpotent}{subpotent} à \( B\) si il existe une injection de \( A\) vers \( B\). Nous notons \( A\preceq B\).
    \end{enumerate}
    Nous disons également «strictement» surpotent quand il y a surpotence mais pas équipotence, et de même pour la subpotence. Les symboles \( \succ\) et \( \prec\) sont alors utilisés.
\end{definition}
Vu que l'ensemble des ensembles n'existe pas\footnote{Voir le corolaire \ref{CORooZMAOooPfJosM}.}, nous n'allons pas énoncer le fait que ces notions donnent une relation d'ordre sur les ensembles; il faudrait parler de classes et nous ne nous en sortirions pas. Nous allons toutefois énoncer quelque résultats qui vont dans ce sens. Pour en savoir plus, vous pouvez lire les différentes pages de Wikipédia sur les nombres cardinaux.

\begin{definition}      \label{DefEOZLooUMCzZR}
    Un ensemble est \defe{infini}{ensemble!infini} s'il peut être mis en bijection avec un de ses sous-ensembles propres (c'est-à-dire différent de lui-même).
\end{definition}

\begin{lemma}[\cite{MonCerveau}\quext{Écrivez-moi si vous connaissez une preuve de ceci.}]       \label{LEMooYHGCooAwsVQN}
    Soit un ensemble \( E\). Si \( I\) et \( J\) sont deux parties de \( E\) telles que \( I\cup J\) soit infini. Alors soit \( I\) soit \( J\) (soit le deux) est infini.
\end{lemma}

\begin{proposition}     \label{PROPooBYKCooGDkfWy}
    L'ensemble \( \eN\) est infini\footnote{Définition \ref{DefEOZLooUMCzZR}.}.
\end{proposition}

Cette proposition est à peu près prise comme définition d'un ensemble fini dans \cite{ooVAYLooJxVYex} qui donne également une preuve de l'équivalence avec notre définition. Rien de tout cela n'est évident\quext{Surtout que je n'ai pas défini la notation \( \{ 0,\ldots, N \}\).}
\begin{propositionDef}     \label{PROPooJLGKooDCcnWi}
    Si \( I\) est un ensemble fini, il existe un unique \( N\in \eN\) tel que \( I\) soit en bijection avec \( \{ 0,\ldots, N \}\).

    Dans ce cas, le nombre \( N+1\) est le \defe{cardinal}{cardinal} de \( I\). 
\end{propositionDef}
%TODOooNXZPooIMyiem définir la notation {0,..., N}
\nomenclature{$\eN_0$}{les naturels non nuls : $\eN_0=\eN\setminus\{ 0 \}$}

Nous ne définissons pas ce qu'est le cardinal d'un ensemble infini; c'est très compliqué et ça ne nous servira pas.

\begin{definition}\label{DefEnsembleDenombrable}
    Un ensemble est \defe{dénombrable}{dénombrable} s'il peut être mis en bijection avec \( \eN\). Il est \defe{non dénombrable}{non dénombrable} s'il est infini et ne peut pas être mis en bijection avec \( \eN\).
\end{definition}
Une chose vraiment amusante avec cette définition que l'on met en rapport avec la définition~\ref{DefEOZLooUMCzZR}, c'est qu'un ensemble fini n'est ni dénombrable ni non dénombrable\footnote{Beaucoup de sources disent qu'un ensemble est dénombrable lorsqu'il est en bijection avec une partie de \( \eN\). Cela laisse la porte ouverte aux ensembles finis. Par exemple Wikipédia\cite{ooLMVKooUiQUtb}.}.

\begin{lemma}       \label{LEMooGRGFooSWDeMA}
    Si \( A\) est un ensemble finie et si \( \sigma\colon A\to B\) est une application quelconque, alors \( \sigma(A)\) est fini dans \( B\).
\end{lemma}

\begin{lemma}       \label{LEMooTUIRooEXjfDY}
    Toute partie d'une ensemble fini est finie.
\end{lemma}

\begin{lemma}       \label{LEMooSRZWooASgEfy}
    Si \( A\) est un ensemble fini ou dénombrable, alors il existe une surjection \( \eN\to A\).
\end{lemma}

\begin{lemma}[\cite{MonCerveau}]        \label{LEMooXPSQooRaSrxv}
    Si \( A\) est un ensemble infini et si \( f\colon A\to B\) est une application injective, alors \( f(A)\) est infini.
\end{lemma}

\begin{proof}
    Vu que \( A\) est infini, il existe \( A'\) strictement inclus à \( A\) et une bijection \( \sigma\colon A'\to A\). Nous allons prouver que la partie \( f(A')\) est en bijection avec \( f(A)\) tout en étant un sous-ensemble strict de \( f(A)\).

    \begin{subproof}
        \item[Stricte inclusion]
            Si \( a\in A\setminus A'\), alors par injectivité de \( f\), nous avons aussi \( f(a)\in f(A)\setminus f(A')\). Autrement dit, \( f(a)\) est un élément de \( f(A)\) qui n'est pas dans \( f(A')\).
        \item[La candidate bijection]
            Vu que \( f\) est une injection nous pouvons considérer \( f^{-1}\) sur \( f(A)\) est poser
            \begin{equation}
                \begin{aligned}
                    \varphi\colon f(A')&\to f(A) \\
                    x&\mapsto  f\Big( \sigma\big( f^{-1}(x) \big) \Big).
                \end{aligned}
            \end{equation}
        \item[Surjection]
            Une élément de \( f(A)\) est de la forme \( y=f(a)\) avec \( a\in A\). L'application \( \sigma\) est une bijection, donc nous pouvons poser \( b=\sigma^{-1}(a)\). Il est alors facile de vérifier que \( x=f(b)\) satisfait \( \varphi(x)=f(a)\). En effet :
            \begin{equation}
                \varphi(x)=(\varphi f\sigma^{-1})(a)=(f\sigma f^{-1}f\sigma^{-1})(a)=f(a).
            \end{equation}
            Cela prouve que \( \varphi\) est surjective.
        \item[Injection]
            Soient \( a,b\in A'\) tels que \( \varphi\big( f(a) \big)=\varphi\big( f(b) \big)\). Nous devons prouver que \( f(a)=f(b)\). Nous avons
            \begin{equation}
                (\varphi f)(a)=(f\sigma f^{-1} f)(a)=(f\sigma)(a).
            \end{equation}
            Donc l'hypothèse dit que \( (f\sigma)(a)=(f\sigma)(b)\). Vu que \( \sigma\) est injective, cela implique que \( f(a)=f(b)\).
    \end{subproof}
\end{proof}

%--------------------------------------------------------------------------------------------------------------------------- 
\subsection{Dénombrabilité et ensemble des naturels}
%---------------------------------------------------------------------------------------------------------------------------

\begin{proposition}[\cite{MonCerveau}]      \label{PROPooOBKMooWEGCvM}
    Toute partie infinie de \( \eN\) est dénombrable.
\end{proposition}

\begin{proof}
    Soit un ensemble infini \( A\) dans \( \eN\). Le lemme \ref{LEMooFHEOooSHPGgU} nous indique que toute partie non vide de \( \eN\) possède un minimum. Nous en profitons pour définir \( \sigma\colon \eN\to A\) par récurrence :
    \begin{subequations}
        \begin{numcases}{}
            \sigma(0)=\min(A)\\
            \sigma(k+1)=\min\big( A\setminus \sigma\{ 0,\ldots, k \} \big)
        \end{numcases}
    \end{subequations}
    Nous montrons un certain nombre de choses à propos de cette application.
    \begin{subproof}
        \item[Elle est strictement croissante]
            Vu que \( A\setminus\alpha\{ 0,\ldots, k \}\subset A\setminus\{ 0,\ldots, k-1 \}\), le minimum est plus grand ou égal : \( \sigma(k+1)\geq \sigma(k)\). Mais \( \sigma(k+1)\) est sélectionné dans l'ensemble \( A\setminus\sigma\{ 0,\ldots, k \}\), qui ne contient justement pas \( \sigma(k)\). Donc \( \sigma(k+1)\neq \sigma(k)\).
        \item[Elle est définie sur \( \eN\)]
            Il faut montrer que pour tout \( k\), l'ensemble \( A\setminus\sigma\{ 0,\ldots, k \}\) est non vide. Si il l'était, cela signifierait que \( A\subset \sigma\{ 0,\ldots, k \}\). Par le lemme \ref{LEMooGRGFooSWDeMA}, la partie \( \sigma\{ 0,\ldots, k \}\) est finie dans \( \eN\). Le lemme \ref{LEMooTUIRooEXjfDY} dit alors qu'en tant que partie de \( \sigma\{ 0,\ldots, k \}\), l'ensemble \( A\) est fini. Mais comme les hypothèses disent que \( A\) est infini, nous avons une contradiction et nous concluons que \( \sigma\) est bien définie sur tout \( \eN\).
        \item[Elle est injective]
            Chaque \( \sigma(k)\) est sélectionné dans une partie de \( A\) qui exclu tous les \( \sigma(i)\) avec \( i<k\).
        \item[Elle est surjective]
            Soit \( a\in A\). Vu que \( \sigma\) est croissante et que \( \sigma(0)\geq 0\), nous avons \( \sigma(a)\geq a\). Si \( \sigma(a)=a\) nous avons terminé. Supposons \( \sigma(a)>a\). Alors
            \begin{equation}        \label{EQooNHTBooQexzwV}
                \min\big( A\setminus\sigma\{ 0,\ldots, a \} \big)>a.
            \end{equation}
            Si \( \sigma\big( \{ 0,\ldots, a \} \big)\) ne contenait pas \( a\), alors \( A\setminus \sigma(\{ 0,\ldots, a \})\) le contiendrait et nous n'aurions pas l'inégalité \eqref{EQooNHTBooQexzwV}. Donc \( a\in \sigma\big( \{ 0,\ldots, a \} \big)\) et \( a\) est bien dans l'image de \( \sigma\).
    \end{subproof}
\end{proof}

\begin{normaltext}
    La proposition \ref{PROPooOBKMooWEGCvM} pourrait être prouvée plus facilement en acceptant le théorème de Cantor-Schröder-Bernstein \ref{THOooRYZJooQcjlcl}. Il existe une injection \( A\to \eN\) parce que \( A\) est une partie de \( \eN\). Mais vu que \( A\) est infini, il possède une partie dénombrable. Cela donne une surjection \( A\to \eN\) et donc une injection \( \eN\to A\). Le théorème de Cantor-Schröder-Bernstein conclu.

    Cela dit, une telle preuve demanderait des outils plus complexes.
\end{normaltext}


\begin{normaltext}
    La proposition suivante donne une bijection explicite entre \( \eN\) et \( \eN\times \eN\). Elle n'a rien de transcendante, mais je ne résiste pas à la donner ici parce qu'elle est utilisée dans l'article \emph{Un peu de programmation transfinie} de David Madore\footnote{Et comme j'aime beaucoup cet article, il me fallait une excuse pour le placer ici.\\ \url{http://www.madore.org/~david/weblog/d.2017-08-18.2460.html}.}. Son utilité est de pouvoir créer un langage de programmation pouvant traiter des paires d'entiers rien qu'en traitant des entiers.
\end{normaltext}
\begin{proposition}[Une bijection \( \eN\times \eN\to \eN\)]        \label{PROPooLPKUooAlsYJg}
    La fonction
    \begin{equation}
        \begin{aligned}
            f\colon \eN\times \eN&\to \eN \\
            (x,y)&\mapsto \begin{cases}
                y^2+x    &   \text{si } x<y\\
                x^2+x+y    &    \text{si } y\leq x.
            \end{cases}
        \end{aligned}
    \end{equation}
    est une bijection.
\end{proposition}

\begin{proof}
    Il s'agit de prouver qu'elle est injective et surjective. Dans la suite, tous les nombres sont des entiers positifs.
    \begin{subproof}
        \item[\( f\) est injective]

            Pour \( k\in \eN\) donné, nous allons prouver que
            \begin{enumerate}
                \item
                    l'équation \( f(x,y)=k\) possède au maximum une solution avec \( x<y\),
                \item
                    l'équation \( f(x,y)=k\) possède au maximum une solution avec \( y\leq x\),
                \item
                    si \(   k=y^2+x \) avec \( x<y\) alors il est impossible que \( k=x'^2+x'+y'\) avec \( y'\leq x'\).
            \end{enumerate}
            On y va.
            \begin{enumerate}
                \item
                    Nous supposons \( y^2+x=t^2+z\) avec \( x<y\) et \( z<t\). Pour fixer les idées, nous supposons \( t>y\) et nous posons \( t=y+s\) (\( s\geq 1\)). En substituant, et en isolant \( z\),
                    \begin{subequations}
                        \begin{align}
                            z&=x-2sy-s^2\\
                            &<x-2sy\\
                            &<x-2sx\\
                            &=x(1-2s)\\
                            &<0.
                        \end{align}
                    \end{subequations}
                    Impossible parce que \( z\geq 0\).
                \item
                    De même nous supposons \( x^2+x+y=z^2+z+t\) avec \( y\leq x\) et \( t\leq z\). Nous posons \( z=x+s\), et nous déballons le même genre de calculs en isolant \( t\).
                \item
                    Enfin nous supposons \( y^2+x=z^2+z+t\) avec \( x<y\) et \( t\leq z\). Les plus courageux diviseront en trois cas : \( y<z\), \( y=z\) et \( y>z\) et feront les calculs. Par exemple, pour le cas \( y>z\) nous posons \( y=z+s\) et nous substituons :
                    \begin{equation}
                        (y+s)^2+x=z^2+z+t
                    \end{equation}
                    qui donne
                    \begin{equation}
                        x=z+t-2zs-s^2<2z-2zs-s^2=2z(1-s)-s^2\leq -s<0
                    \end{equation}
                    parce que \( s\geq 1\), donc \( 1-s\leq 0\).
            \end{enumerate}

        \item[\( f\) est surjective]

            Nous devons prouver que tous les éléments de \( \eN\) sont dans l'image de \( \eN\times \eN\) par \( f\). En premier lieu, \( 0=f(0,0)\). C'est un bon début. Soit \( a\in \eN\) non nul; nous montrons que tous les nombres de \( a^2\) à \( (a+1)^2\) sont des images de \( f\). D'abord \( a^2=f(0,a)\), ensuite les nombres
            \begin{equation}
                f(1,a),f(2,a),\ldots, f(a-1,a)
            \end{equation}
            prennent les valeurs \( a^2+1\), \ldots, \( a^2+a-1\). Enfin nous avons \( f(a,0)=a^2+a\) et les nombres \( f(a,1),\ldots, f(a,a)\) prennent les valeurs de \( a^2+a+1\) à \( a^2+2a=(a+1)^2-1\).
    \end{subproof}
\end{proof}
Sachez que cette fonction s'étend aux ordinaux (mais là ce n'est plus pour rigoler).

\begin{corollary}       \label{CORooNRPIooZPSmqa}
    Il existe des parties \( \{ \eN_i \}_{i\in \eN}\) telles que \( \bigcup_{i\in \eN}\eN_i=\eN\) et que chaque \( \eN_i\) soit en bijection avec \( \eN\)
\end{corollary}

\begin{proof}
    Nous considérons la bijection \( f\colon \eN\to \eN\times \eN\) donnée par (l'inverse de celle donnée) par la proposition \ref{PROPooLPKUooAlsYJg}, et nous posons
    \begin{equation}
        \eN_i=f^{-1}(i,\eN).
    \end{equation}
    L'application
    \begin{equation}
        f\colon \eN_i\to \{ (i,k) \}_{k\in \eN}
    \end{equation}
    est une bijection. Or l'ensemble \( \{ (i,k) \}_{k\in \eN}\) est évidemment en bijection avec \( \eN\). Par composition nous avons le résultat.
\end{proof}

\begin{lemma}[\cite{MonCerveau}]        \label{LEMooDLWFooNAJbbq}
    Si il existe une surjection \( \eN\to A\), alors \( A\) est fini ou dénombrable.
\end{lemma}

\begin{proof}
    Pour chaque \( a\in A\), l'ensemble \( f^{-1}(a)\) est une partie de \( \eN\). 
    \begin{subproof}
        \item[Une application]
            Le lemme \ref{LEMooYMRJooYIAhBb}\ref{ITEMooNHRIooODBVNK} nous permet de poser
            \begin{equation}
                \begin{aligned}
                    \sigma\colon A&\to \eN \\
                    a&\mapsto \min\big( f^{-1}(a) \big). 
                \end{aligned}
            \end{equation}
        \item[\( \sigma\) est injective]
            Supposons que \( \sigma(a)=\sigma(b)\). Nous appelons \( x\) ce nombre :
            \begin{equation}
                x=\min\big( f^{-1}(a) \big)=\min\big( f^{-1}(b) \big).
            \end{equation}
            Nous avons \( x\in f^{-1}(a)\cap f^{-1}(b)\), ce qui implique que \( f(x)=a\) et que \( f(x)=b\); donc \( a=b\).

            Donc \( \sigma\) est une injection.
        \item[\( A\) est infini]
            Si \( A\) est fini, le lemme est prouvé. Donc à partir de maintenant nous supposons que \( A\) est infini. Le but est de prouver qu'il est dénombrable, c'est à dire de construire une bijection \( A\to \eN\).
        \item[\( \sigma(A)\) est dénombrable]
            Vu que \( \sigma\colon A\to  \eN\) est injective et que \( A\) est infini, le lemme \ref{LEMooXPSQooRaSrxv} dit que \( \sigma(A)\) est infini dans \( \eN\). La proposition \ref{PROPooOBKMooWEGCvM} nous dit alors que \( \sigma(A)\) est dénombrable.

            Soit une bijection \( \varphi\colon \sigma(A)\to \eN\).
        \item[La candidate bijection]
            Nous posons
            \begin{equation}
                f=\varphi\circ \sigma\colon A\to \eN
            \end{equation}
            et nous allons prouver que c'est une bijection.
        \item[Injective]
            Vu que \( \varphi\) et \( \sigma\) sont injectives, l'égalité \( (\varphi\sigma)(a)=(\varphi\sigma)(b)\) implique immédiatement \( a=b\).
        \item[Surjective]
            Soit \( k\in \eN\). Vu que \( \varphi\) et \( \sigma\) sont des injections, nous pouvons poser \( a=(\sigma^{-1}\varphi^{-1})(k)\). Il est alors immédiat que \( f(a)=k\).
    \end{subproof}
\end{proof}

\begin{proposition}[\cite{MonCerveau,ooLMVKooUiQUtb}]     \label{PROPooENTPooSPpmhY}
    Une union dénombrable d'ensembles finis ou dénombrables est finie ou dénombrable.
\end{proposition}

\begin{proof}
    Soient \( A_i\) des ensembles finis ou dénombrables. Nous posons \( A=\bigcup_{i\in \eN}A_i\), et nous considérons les parties \( \eN_i\) du corolaire \ref{CORooNRPIooZPSmqa}. Vu que \( A_i\) est dénombrable ou fini et que \( \eN_i\) est dénombrable, il existe une surjection \( \varphi_i\colon \eN_i\to A_i\).

    Nous définissons \( s\colon \eN\to \eN\) par \( n\in \eN_{s(n)}\), et nous posons enfin
    \begin{equation}
        \begin{aligned}
            \varphi\colon \eN&\to A \\
            n&\mapsto \varphi_{s(n)}(n). 
        \end{aligned}
    \end{equation}
    Nous prouvons que \( \varphi\) est surjective.

    Soit \( a\in A_i\). Il existe \( n\in \eN_i\) tel que \( a=\varphi_i(n)\). Mais comme \( n\in \eN_i\), nous avons \( s(n)=i\). Donc
    \begin{equation}
        a=\varphi_i(n)=\varphi_{s(n)}(n)=\varphi(n).
    \end{equation}
    Donc \( \varphi\colon \eN\to A\) est surjective.

    Le lemme \ref{LEMooDLWFooNAJbbq} conclu que \( A\) est fini ou dénombrable.
\end{proof}

\begin{lemma}[\cite{MonCerveau}]        \label{LEMooRXSRooBUWOyb}
    Si \( N\) est un ensemble dénombrable, alors il existe une bijection \( g\colon \{ 1,2 \}\times N\to N\).
\end{lemma}

\begin{proof}
    D'abord nous définissons une bijection \( \varphi\colon \{ 0,1 \}\times \eN\to \eN\) par
    \begin{equation}
        \begin{aligned}
            \varphi\colon \{ 0,1 \}\times \eN&\to \eN \\
            (n,k)&\mapsto 2k+n. 
        \end{aligned}
    \end{equation}
    Ensuite si \( f\colon \eN\to N\) est une bijection, il suffit de poser \( g(n,k)=f\big( \varphi(n,k) \big)\).
\end{proof}

\begin{proposition}[\cite{ooLMVKooUiQUtb}]     \label{PROPooDMZHooXouDrQ}
    Si \( N\) est un ensemble dénombrable, alors pour tout \( n\in \eN\), l'ensemble \( N^n\) est dénombrable.
\end{proposition}

Les ensembles dénombrables sont les plus petits ensembles infinis possibles, comme en témoigne la proposition suivante.
\begin{proposition}      \label{PROPooUIPAooCUEFme}
    Tout ensemble infini contient une partie en bijection avec \( \eN\).
\end{proposition}

\begin{proof}
    Soient un ensemble infini \( E_0\) et une partie propre \( E_1\) en bijection avec \( E_0\). Nous notons \( \varphi\colon E_0\to E_1\) une bijection.

    Soit \( x_0\in E_0\setminus E_1\) (axiome du choix et tout ça). Nous définissons
    \begin{equation}
        \begin{aligned}
            \psi\colon \eN&\to \{\varphi^k(x_0)\} \\
            n&\mapsto \varphi^n(x_0)
        \end{aligned}
    \end{equation}
    et nous allons prouver que c'est une bijection. Que ce soit surjectif est immédiat. Pour l'injectivité, soit \( \varphi^k(x_0)=\varphi^l(x_0)\) avec \( k\neq l\). Supposons pour fixer les notations que \( k>l\). Alors, vu que \( \varphi\) est inversible nous pouvons écrire
    \begin{equation}
        x_0=\varphi^{k-l}(x_0)=\varphi\big( \varphi^{k-l-1}(x_0) \big)
    \end{equation}
    où il est entendu que \( \varphi^0(x_0)=x_0\). Cela signifie que \( x_0\) est dans l'image de \( \varphi\), c'est-à-dire dans $E_1$, ce que nous avons exclu par choix de \( x_0\) dans \( E_0\setminus E_1\). Donc en réalité \( \varphi^k(x_0)\neq \varphi^l(x_0)\) dès que \( k\neq l\).
\end{proof}

\begin{proposition} \label{PropQEPoozLqOQ}
    Toute partie d'un ensemble fini est finie, et toute partie d'un ensemble dénombrable est finie ou dénombrable.
\end{proposition}

\begin{lemma}   \label{LEMooGTOTooFbpvzU}
    Soit un ensemble \( E\) non dénombrable ainsi qu'une application \( f\colon E\to F\) où \( F\) est un ensemble quelconque. Si \( f(E)\) est dénombrable (ou fini), alors il existe \( y\in f(E)\) tel que \( f^{-1}(y)\) est indénombrable.
\end{lemma}

\begin{proof}
    Nous avons
    \begin{equation}
        E=\bigcup_{y\in F}f^{-1}(y).
    \end{equation}
    Si tous les \( f^{-1}(y)\) sont dénombrables, alors \( E\) est une union dénombrable (\( F\) est dénombrable) d'ensembles dénombrables. Il serait donc dénombrable (proposition \ref{PROPooENTPooSPpmhY}), ce qui est contraire à l'hypothèse.
\end{proof}

%--------------------------------------------------------------------------------------------------------------------------- 
\subsection{Théorème de Cantor-Schröder-Bernstein}
%---------------------------------------------------------------------------------------------------------------------------

\begin{lemma}[\cite{BIBooECJMooGPxBem}]     \label{LEMooTNMHooBpdzab}
    Soient un ensemble \( A\) et une partie \( B\) de \( A\). Si il existe une injection \( f\colon A\to B\) alors il existe une bijection \( \alpha\colon A\to B\).
\end{lemma}

\begin{proof}
    Nous posons \( Y=A\setminus B\) et nous décomposons la preuve en étapes.
    \begin{subproof}
        \item[Les \( f^k(Y)\) sont disjoints]
            Vu que \( f\) prend ses valeurs dans \( B\), nous avons \( f^k(Y)\subset B\) pour tout \( k\). Et vu que \( Y=A\setminus B\), nous avons
            \begin{equation}        \label{EQooDNHJooFJBrDq}
                f^k(Y)\cap Y=\emptyset
            \end{equation}
            pour tout \( k\). L'application \( f\) étant injective, elle vérifie \( f(C\cap D)=f(C)\cap f(D)\). Nous appliquons \( f^m\) des deux côtés de \eqref{EQooDNHJooFJBrDq} :
            \begin{equation}
                f^{k+m}(Y)\cap f^m(Y)=\emptyset
            \end{equation}
            pour tout \( k,m\in \eN\).
        \item[Une décomposition] 
            Vu que \( f(X)\subset B\) nous avons l'égalité
            \begin{equation}
                B=f(X)\cup\big(B\setminus f(X)\big)
            \end{equation}
        \item[\( A\setminus X=B\setminus f(X)\)]
            Supposons \( x\in A\setminus X\). Vu que \( A\setminus B\) est dans \( X\), l'élément \( x\) n'est pas dans \( A\setminus B\) et donc est dans \( B\) parce qu'il est dans \( A\). Mais \( x\) n'est pas dans \( X\) et en particulier pas dans \( f(X)\) parce que \( f(X)\subset X\). Donc \( x\) est dans \( B\setminus f(X)\).

            Dans l'autre sens, nous supposons que \( x\in B\setminus f(X)\). Vu que \( B\subset A\) nous avons \( x\in A\). Comme \( x\) est hors de \( f(X)\), il est hors des \( f^k(Y)\) pour \( k\geq 1\). Mais \( x\in B\), donc \( x\) est hors de \( A\setminus B=f^0(Y)\). Donc \( x\) est hors de \( k^k(Y)\) pour tout \( k\geq 0\). Donc \( x\) est hors de \( X\).

        \item[La bijection]
            Nous considérons l'application
            \begin{equation}
                \begin{aligned}
                    \alpha\colon A&\to B \\
                    x&\mapsto \begin{cases}
                        f(x)    &   \text{si } x\in X\\
                        x    &    \text{si } x\in A\setminus X.
                    \end{cases}
                \end{aligned}
            \end{equation}
            Nous démontrons dans les points suivants que \( \alpha\) est bijective.
        \item[Injective]
            Nous supposons \( \alpha(x)=\alpha(y)\). Il y a 4 possibilités suivant que \( x\) et \( y\) soient dans \( X\) ou \( A\setminus X\).

            Si \( x,y\in X\) alors \( f(x)=f(y)\) et donc \( x=y\) parce que \( f\) est injective.

            Si \( x\in X\) et \( y\in A\setminus X\), alors \( f(x)=y\). Mais \( f(x)\in f(X)\) et \( y\in A\setminus X=B\setminus f(X)\). Donc l'élément \( f(x)=y\) est dans \( f(X)\cap \big( B\setminus f(X) \big)=\emptyset\). Il n'est donc pas possible d'avoir \( \alpha(x)=\alpha(y)\) avec \( x\in X\) et \( y\in A\setminus X\).

            Si \( x\in A\setminus X\) et \( y\in X\), c'est la même chose.

            Si \( x,y\in A\setminus X\), alors \( x=\alpha(x)=\alpha(y)=y\).

        \item[Injective]
            Soit \( y\in B\). Il y a deux possibilités : \( y\in x\) et \( y\in A\setminus X\). La première se divise en deux : \( y\in Y\) et \( y\in \bigcup_{k=1}^{\infty}f^k(Y)\). On y va.

            \begin{subproof}
                \item[\( y\in Y\)]
                    Ce cas n'est pas possible parce que \( y\in B\) alors que \( Y=A\setminus B\).
                \item[\( y\in f^k(Y)\) avec \( k\geq 1\)]
                    Nous avons
                    \begin{equation}
                         y\in f\big( f^{k-1}(Y) \big)\subset f(X)\subset \alpha(A).
                    \end{equation}
                \item[\( y\in A\setminus X\)]
                    Alors \( y=\alpha(y)\).
            \end{subproof}
    \end{subproof}
\end{proof}

\begin{theorem}[Cantor-Schröder-Bernstein]      \label{THOooRYZJooQcjlcl}
    Soient deux ensembles \( A\) et \( B\) pour lesquels il existe des injections \( f\colon A\to B\) et \( g\colon B\to A\). Alors il existe une bijection entre \( A\) et \( B\).
\end{theorem}

\begin{proof}
    La composée \( g\circ f\colon A\to A\) est injective et prend ses valeurs dans \( g\big( f(A) \big)\subset g(B)\subset A\). Bref, l'application \( g\circ f\colon A \to g(B)\) est injective. Le lemme \ref{LEMooTNMHooBpdzab} donne alors une bijection \( \varphi\colon A\to g(B)\).

    Nous montrons que \( g^{-1}\circ\varphi\colon A\to B\) est une bijection.

    \begin{subproof}
        \item[Injective]
            Nous supposons \( x,y\in A\) tels que
            \begin{equation}
                g^{-1}\big( \varphi(x) \big)=g^{-1}\big( \varphi(y) \big).
            \end{equation}
            Nous appliquons \( g\) des deux côtés : \( \varphi(x)=\varphi(y)\). Vu que \( \varphi\) est une bijection, cela entraîne \( x=y\).
        \item[Surjective]
            Soit \( b\in B\). En posant \( a=\varphi^{-1}\big( g(b) \big)\) nous avons bien \( (g^{-1}\circ \varphi)(a)=b\).
    \end{subproof}
\end{proof}

%--------------------------------------------------------------------------------------------------------------------------- 
\subsection{Ensembles équipotents, surpotents, subpotents}
%---------------------------------------------------------------------------------------------------------------------------

\begin{proposition}[\cite{MonCerveau}]
    L'ensemble \( A\) est subpotent à \( B\) si et seulement si \( B\) est surpotent à \( A\).
\end{proposition}

\begin{proof}
    En deux parties.
    \begin{subproof}
        \item[Premier sens]
            Nous supposons que \( A\) est subpotent à \( B\). Il existe une injection \( \varphi\colon A\to B\). Nous définissons \( f\colon B\to A\) par
            \begin{equation}
                f(x)=\begin{cases}
                    \varphi^{-1}(x)    &   \text{si } x\in\varphi(A)\\
                    a    &    \text{sinon } 
                \end{cases}
            \end{equation}
            où \( a\) est un élément quelconque de \( A\). Cette application est bien définie parce que \( \varphi\) est injective, de telle sorte que \( \varphi^{-1}\) est bien définie. Vu que \( \varphi\) est définie sur tout \( a\), l'application \( f\) est une surjection.
        \item[L'autre sens]
            Nous supposons que \( A\) est surpotent à \( B\). Il existe une surjection \( \varphi\colon A\to B\). Pour chaque \( y\in B\), nous considérons un élément \( a_y\in \varphi^{-1}(y)\). Cela existe parce que \( \varphi\) est surjective\quext{Mais ça demande l'axiome du choix et j'avoue être un peu étonné que ce soit indispensable ici.}. Nous posons alors
            \begin{equation}
                \begin{aligned}
                    f\colon B&\to A \\
                    y&\mapsto a_y. 
                \end{aligned}
            \end{equation}
            Cela est injectif parce que \( f(y_1)=f(y_2)\) implique \( a_{y_1}=a_{y_2}\), c'est à dire que l'intersection \( \varphi^{-1}(y_1)\cap\varphi^{-1}(y_2)\) est non vide. Vu que \( \varphi\) est injective, cela implique que $y_1=y_2$.
    \end{subproof}
\end{proof}

Le théorème de comparabilité cardinale énonce que si \( A\) et \( B\) sont des ensemble, alors nous avons toujours \( A\succeq B\) ou \( A\preceq B\) (ou les deux; dans ce cas \( A\approx B\) par Cantor-Schröder-Bernstein).
\begin{theorem}[Théorème de comparabilité cardinale\cite{MonCerveau,BIBooTSOKooCWxMwj}]     \label{THOooCBSKooCmzfUf}
    Entre deux ensembles, il existe forcément une injection de l'un dans l'autre.
\end{theorem}

\begin{proof}
    Nous allons montrer que le graphe d'une injection de $A$ dans $B$ ou de $B$ dans $A$ est donné par un élément maximal (au sens de l'inclusion) de l'ensemble (inductif) des graphes d'injections d'une partie de $A$ dans une partie de $B$.

    Nous posons l'ensemble
    \begin{equation}
       \mA=\Big\{  (X,Y,\varphi)  \tq
        \begin{cases}
            X\subset A\\
            Y\subset B\\
            \varphi\colon X\to Y\text{ est injective.}
        \end{cases}
    \Big\}
    \end{equation}
    que nous ordonnons par, l'inclusion, c'est à dire par \( (X_1,Y_1,\varphi_1)<(X_2,Y_2,\varphi_2)\) lorsque \( X_1\subset X_2\), \( Y_{1}\subset Y_2\) et \( \varphi_2|_{X_1}=\varphi_1\).

    Nous passons rapidement sur le fait que cet ensemble est inductif, et nous considérons tout de suite un élément maximal \( (X,Y,\varphi)\).

    Il y a deux possibilités : soit \( X\) est équipotent à \( A\), soit non.
    \begin{subproof}
        \item[Si \( X\) est équipotent à \( A\)]
            Nous considérons une bijection \( \psi\colon A\to X\). Alors l'application
            \begin{equation}
                \begin{aligned}
                    \sigma\colon A&\to B \\
                    x&\mapsto (\varphi\circ\psi)(x) 
                \end{aligned}
            \end{equation}
            est une injection. Elle n'est pas surjective si \( Y\) est différent de \( A\), mais cela ne nous intéresse pas ici.
        \item[Si \( X\) n'est pas équipotent à \( A\)]
            Dans ce cas, en particulier \( X\) est strictement inclus à \( A\) et nous pouvons considérer \( a\in A\setminus X\).

            Il y a deux possibilités : soit \( Y=B\), soit non.

            \begin{subproof}
                \item[Si \( Y=B\)]
                    Alors \( \varphi\colon X\to B\) est une bijection, et donc \( \varphi^{-1}\colon B\to X\) est une bijection. En particulier \( \varphi^{-1}\) est une injection \( B\to A\).
                \item[Si \( Y\neq B\)]
                    Soit \( b\in B\setminus Y\). Nous considérons l'application
                    \begin{equation}
                        \begin{aligned}
                            \psi\colon X\cup\{ a \}&\to Y\cup\{ b \} \\
                            x&\mapsto \begin{cases}
                                \varphi(x)    &   \text{si } x\in X\\
                                b    &    \text{si }x=a.
                            \end{cases}
                        \end{aligned}
                    \end{equation}
                    Cela est une application injective. Donc le triple \( (X\cup \{ a \}, Y\cup\{ b \},\psi)\) majore \( (X,Y,\varphi)\). Nous avons une contradiction et le cas \( Y\neq B\) n'est pas possible.
            \end{subproof}
    \end{subproof}
\end{proof}

\begin{normaltext}
    Le théorème de comparabilité cardinale couplé au théorème de Cantor-Schröder-Bernstein nous indique que pour tout ensembles \( A\) et \( B\), nous avons soit \( A\preceq B\), soit \( B\preceq A\). Et si \( A\preceq B\preceq A\), alors \( A\approx B\).

    Nous ne sommes pas loin de dire que la relation \( \preceq\) donne un ordre total sur l'ensemble des ensembles. C'est très beau sauf que l'ensemble des ensembles n'existe pas\footnote{Corolaire \ref{CORooZMAOooPfJosM}.}. Il faudrait parler de \emph{classe} des ensembles, mais ça nous mènerait trop loin. Toujours est-il que ces deux théorèmes montrent qu'on n'est pas loin d'avoir un ordre sur les ensembles, et que cela est une des bases possibles pour développer les nombres cardinaux.
\end{normaltext}

%--------------------------------------------------------------------------------------------------------------------------- 
\subsection{Théorème de Cantor, ensemble des ensembles}
%---------------------------------------------------------------------------------------------------------------------------

\begin{theorem}[Cantor\cite{BIBooXHFNooSmqUar}]     \label{THOooJPNFooWSxUhd}
    Un ensemble est toujours strictement subpotent à son ensemble des parties.
\end{theorem}

\begin{proof}
    Soit un ensemble \( E\) et son ensemble des parties \( \mP(E)\). Nous commençons par prouver qu'il n'existe pas de surjection \( E\to \mP(E)\). Soit en effet une application \( f\colon E\to \mP(E)\). Nous posons
    \begin{equation}
        D=\{ x\in E\tq x\notin f(x) \}.
    \end{equation}
    Nous prouvons que \( D\) ne peut pas être dans l'image de \( f\). Supposons que \( y\in E\) soit tel que \( f(y)=D\).
    \begin{subproof}
        \item[Si \( y\in D\)]
            Alors par définition de \( D\), nous avons \( y\notin f(y)=D\). Contradiction.
        \item[Si \( y\notin D\)]
            Alors \( y\in f(y)=D\), contradiction.
    \end{subproof}
    Donc aucune surjection \( f\colon E\to \mP(E)\) n'existe. En particulier par de bijections.

    Par ailleurs, l'application \( g\colon \mP(E)\to E\) qui fait \( g(\{ a \})=a\) (et n'importe quoi d'autre sur les autres éléments de \( \mP(E)\)) est une surjection \( \mP(E)\to E\).

    Donc \( \mP(E)\) est toujours strictement surpotent à \( E\).
\end{proof}

\begin{normaltext}
    Le théorème de Cantor implique en particulier qu'il existe (au moins) une infinité dénombrable d'ensemble infinis de cardinalité différentes (plus évidemment une infinité dénombrable d'ensembles finis de cardinalité différentes).

    Pour tout ensemble \( A\), il est donc possible de dire «soit \( E\), un ensemble strictement surpotent à \( A\)».
\end{normaltext}

\begin{corollary}       \label{CORooZMAOooPfJosM}
    Il n'existe pas d'ensemble contenant tous les ensembles.
\end{corollary}

\begin{proof}
    Si \( E\) était un tel ensemble, nous aurions \( \mP(E)\subset E\) parce que les éléments de \( \mP(E)\) sont des ensembles. Or cela donnerait une surjection \( E\to \mP(E)\) alors que cela est impossible par le théorème de Cantor \ref{THOooJPNFooWSxUhd}.
\end{proof}

%--------------------------------------------------------------------------------------------------------------------------- 
\subsection{Ajouter ou soustraire des cardinalités}
%---------------------------------------------------------------------------------------------------------------------------

Nous allons prouver une série de résultats que nous pourrions résumer en  «ajouter ou retrancher des parties de cardinalité plus petite ne change pas la cardinalité».

\begin{lemma}[\cite{MonCerveau}]        \label{LEMooUFCAooSyZtZj}
    Si \( A\) est infini et \( B\) est fini, alors \( A\cup B\) est équipotent à \( A\).
\end{lemma}

\begin{proof}
    Nous supposons que \( A\) et \( B\) sont disjoints\footnote{Adaptez la démonstration au cas où l'intersection n'est pas vide.}. La proposition \ref{PROPooJLGKooDCcnWi} nous permet de considérer une bijection \( \psi\colon \{ 1,\ldots, n \}\to B\).
    
    Vu que \( A\) est infini, la proposition \ref{PROPooUIPAooCUEFme} nous permet de considérer \( N\subset A\) et une bijection \( \varphi\colon \eN\to N\).

    Maintenant, il s'agit seulement d'insérer \( B\) dans \( A\) en le mettant «au début» de \( N\) et en décalant les autres. La bijection est
    \begin{equation}
        \begin{aligned}
            f\colon A\cup B&\to A \\
            x&\mapsto \begin{cases}
                x    &   \text{si } x\in A\setminus N\\
                \varphi\big( \varphi^{-1}(x)+n \big)    &   \text{si } x\in N\\
                \varphi\big( \psi^{-1}(x) \big)    &    \text{si }x\in B.
            \end{cases}
        \end{aligned}
    \end{equation}
\end{proof}

\begin{lemma}       \label{LEMooXMVDooIWLWis}
    Si \( A\) est infini et si \( A\) est surpotent à \( B\), alors \( A\approx A\cup B\).
\end{lemma}

\begin{proof}
    Il existe évidemment une injection \( A\to A\cup B\). Donc le théorème de Cantor-Schröder-Bernstein \ref{THOooRYZJooQcjlcl} nous indique que trouver une injection \( A\cup B\to A\) suffira pour la peine.

    Nous allons utiliser le lemme de Zorn \ref{LemUEGjJBc} avec l'ensemble
    \begin{equation}
       \mA=\Big\{  (X,\varphi_X)  \tq
        \begin{cases}
            X\subset B\\
            \varphi_X\colon A\cup X\to A\text{ est injective.}
        \end{cases}
    \Big\}
    \end{equation}
    muni de l'ordre de l'inclusion : \( (X,\varphi_X)<(Y,\varphi_Y)\) si \( X\subset Y\) et \( \varphi_Y(x)=\varphi_X(x)\) pour tout \( x\in A\cup X\).
    
    \begin{subproof}
        \item[\( \mA\) est inductif]
            Soit une famille \( \mF=\{ (X_i,\varphi_i) \}_{i\in I}\) complètement ordonnée indexée par l'ensemble \( I\). En posant \( X=\bigcup_{i\in I}X_i\) et \( \varphi(x)=\varphi_i(x)\) dès que \( x\in A\cup X_i\), l'élément \( (X,\varphi)\) majore \( \mF\).
        \item[Un maximum]
            Le lemme de Zorn nous assure que \( \mA\) possède (au moins) un élément maximum. Soit un tel élément maximum \( (X,\varphi)\). 
        \item[\( X\approx B\)]
            Ah oui, vous auriez aimé avoir \( X=B\). Mais non; il n'y a pas de garanties. Nous allons montrer que \( X\approx B\), et ça suffira.

            Vu que \( X\subset B\), si \( X\) n'est pas équipotent à \( B\), il est strictement inclus à \( B\). Nous pouvons donc considérer
            \begin{subequations}
                \begin{align}
                    b&\in B\setminus X\\
                    a&\in A\setminus \varphi(X).
                \end{align}
            \end{subequations}
            Nous considérons alors l'élément \( (Y,\psi)\in \mA\) défini par
            \begin{subequations}
                \begin{align}
                    Y&=X\cup\{ b \}\\
                    \psi(x)&=\begin{cases}
                        a    &   \text{si } x=b\\
                        \varphi(x)    &    \text{sinon }.
                    \end{cases}
                \end{align}
            \end{subequations}
            Cet élément majore \( (X,\varphi)\).

            Donc \( X\approx B\).
        \item[Résumé de la situation]
            Nous avons \( A\approx A\cup X\) ainsi que une injection \( \varphi\colon A\cup X\to A\) et une bijection \( \psi\colon B\to X\).
        \item[Conclusion si \( A\) est disjoint de \( B\)]
            Si \( A\) et \( B\) sont disjoints, nous avons une bijection
            \begin{equation}
                \begin{aligned}
                    l\colon A\cup B&\to A \\
                    x&\mapsto \begin{cases}
                        \varphi(x)    &   \text{si }  x\in A\\
                        \varphi\big( \psi(x) \big)    &    \text{si } x\in B.
                    \end{cases}
                \end{aligned}
            \end{equation}
        \item[Conclusion si \( A\) n'est pas disjoint de \( B\)]
            Il suffit de poser \( C=B\setminus A\) et avons
            \begin{equation}
                A\cup B=[A\cup (A\cap B)]\cup C.
            \end{equation}
            Cette union est disjointe, \( A\cup(A\cap B)\) est surpotent à \( A\) et \( C\) est subpotent à $B$. La conclusion est donc encore valable.
    \end{subproof}
\end{proof}

La proposition suivante sera utilisée en théorie de la mesure, dans l'exemple~\ref{ExOIXoosScTC}. Ça utilise l'axiome du choix sous la forme du lemme de Zorn.
\begin{proposition}[\cite{ooFAOQooACUugI}] \label{PropVCSooMzmIX}
    Si \( S\) est un ensemble infini alors il existe une bijection \( \varphi\colon \{ 0,1 \}\times S\to S\).
\end{proposition}

\begin{proof}
    En plusieurs points.
    \begin{subproof}
        \item[Un gros ensemble]
            Nous considérons l'ensemble suivant :
                \begin{equation}
                   \mA=\Big\{  (X,\varphi)  \tq
                    \begin{cases}
                        X\subset S\text{ est infini}\\
                        \varphi\colon \{ 0,1 \}\times X\to X\,\text{est injective}
                    \end{cases}
                \Big\}
                \end{equation}
                Nous ordonnons par l'inclusion, c'est à dire que nous disons que \( (X_1,\varphi_1)<(X_2,\varphi_2)\) lorsque \( X_1\subset X_2\) et \( \varphi_1(k,x)=\varphi_2(k,x)\) pour tout \( (k,x)\in \{ 0,1 \}\times X_1\).
        \item[Il est inductif]
            Nous prouvons que \( \mA\) est inductif\footnote{Définition \ref{DefGHDfyyz}.}. Soit une partie totalement ordonnée \( \mF=\{ (X_i,\varphi_i) \}_{i\in I}\) indexé par un ensemble \( I\).

            Commençons par remarquer que si \( x\in X_i\cap X_j\) et si \( k\in \{ 0,1 \}\), alors \( \varphi_i(k,x)=\varphi_j(k,x)\). En effet, vu que \( \mF\) est totalement ordonné, nous pouvons supposer \( (X_i,\varphi_i)<(X_j,\varphi_j)\) (sinon c'est le contraire) et nous avons alors \( X_i\subset X_j\) et \( \varphi_j(k,x)=\varphi_i(k,x)\).

            Maintenant nous prouvons que \( \mF\) accepte un majorant dans \( \mA\). Il s'agit de construire une paire \( (X,\varphi)\). Nous posons d'abord
            \begin{equation}
                X=\bigcup_{i\in I}X_i.
            \end{equation}
            D'autre part, si \( x\in X\) et si \( k\in \{ 0,1 \}\), l'ensemble 
            \begin{equation}
             \{ \varphi_i(k, x)\tq i\in I, x\in X_i \}   
            \end{equation}
             contient un seul élément. Nous nommons \( \varphi(k,x)\) cet élément. Cela nous définit une application \( \varphi\colon \{ 0,1 \}\times X\to X \) qui prolonge tous les \( \varphi_i\).

            Donc nous avons \( (X,\varphi)>(X_i,\varphi_i)\) pour tout \( i\in I\).

        \item[Lemme de Zorn]

            Vu que \( \mA\) est un ensemble inductif, nous pouvons utiliser le lemme de Zorn \ref{LemUEGjJBc} et considérer un élément maximum, c'est à dire une paire \( (Y,\varphi)\) où \( Y\subset S\) et \( \varphi\colon \{ 0,1 \}\times Y\to Y\) est injective.

            Il y a deux cas : soit \( Y\) est équipotent à \( S\), soit il ne l'est pas.

        \item[Si \( Y\) est équipotent à \( S\)]

            Nous considérons une bijection \( \psi\colon S\to Y\) et nous montrons que l'application
            \begin{equation}
                \begin{aligned}
                    f\colon \{ 0,1 \}\times S&\to S \\
                    (n,a)&\mapsto \psi^{-1}\Big( \varphi\big( (n,\psi(a)\big) \Big) .
                \end{aligned}
            \end{equation}
            est une injection. Il s'agit seulement d'une vérification basée sur le fait que \( \psi^{-1}\), \( \varphi\) et \( \psi\) le sont.

            Une injection dans l'autre sens n'est pas compliquée à construire; par exemple \( s\mapsto (1,s)\). Ayant maintenant une injection dans les deux sens, le théorème de Cantor-Bernstein \ref{THOooRYZJooQcjlcl} dit qu'il existe une bijection.

        \item[Si \( Y\) n'est pas équipotent à \( S\)]

            Nous montrons que ce cas n'est pas possible. Notons que \( Y\) est certainement subpotent à \( S\) parce que \( Y\subset S\).
            \begin{subproof}
                \item[\( S\setminus Y\) est inifni]
                    Supposons le contraire. Si \( S\setminus Y\) est fini, alors \( Y\cup(S\setminus Y)\) est équipotent à \( Y\) par le lemme \ref{LEMooUFCAooSyZtZj}. Oui mais \( Y\cup(S\setminus Y)=S\), et nous avons dit que \( Y\) n'est pas équipotent à \( S\).

                    Bref, oui : \( S\setminus Y\) est infini.

                \item[La contradiction]
                    La proposition \ref{PROPooUIPAooCUEFme} nous donne une partie dénombrable\footnote{Et donc infinie par la proposition \ref{PROPooBYKCooGDkfWy} que nous n'allons plus citer à chaque fois.} \( N\subset S\setminus Y\). Nous allons construire une injection \( \psi\colon \{ 0,1 \}\times (Y\cup N)\to Y\cup N\) qui prolonge \( \varphi\), contredisant ainsi la maximalité de \( (Y,\varphi)\).

                    Nous considérons une bijection \( g\colon \{ 0,1 \}\times N\to N\) (lemme \ref{LEMooRXSRooBUWOyb}). Vu que \( N\) est dans \( S\setminus Y\), les parties \( Y\) et \( N\) sont disjointes et nous pouvons écrire
                    \begin{equation}
                        \{ 0,1 \}\times (Y\cup N)=\big( \{ 0,1 \}\times Y \big)\cup\big( \{ 0,1 \}\times N \big)
                    \end{equation}
                    où l'union à droite est disjointe.

                    Nous pouvons donc définir \( \psi\) correctement de la façon suivante :
                    \begin{equation}
                        \begin{aligned}
                            \psi\colon \{ 0,1 \}\times (Y\cup N)&\to Y\cup N \\
                            (n,x)&\mapsto \begin{cases}
                                \varphi(n,x)    &   \text{si } x\in Y\\
                                g(n,x)    &    \text{si }x\in N.
                            \end{cases}
                        \end{aligned}
                    \end{equation}
                    Voila qui contredit la maximalité de \( (Y,\varphi)\).
            \end{subproof}
    \end{subproof}
\end{proof}

\begin{corollary}       \label{CORooJCSIooOeOICJ}
    Si \( A\) est un ensemble infini, alors \( A\) possède deux sous-ensembles disjoints \( A_1\) et \( A_2\) qui sont tout deux en bijection avec \( A\).
\end{corollary}

\begin{proof}
    La proposition \ref{PropVCSooMzmIX} donne une bijection \( \varphi\colon \{ 1,2 \}\times A\to A\). Il suffit de poser \( A_1=\varphi(1,A)\) et \( A_2=\varphi(2,A)\).
\end{proof}

Maintenant que nous pouvons mettre dans \( A\) deux copies disjointes de \( A\), il n'est pas très étonnant que nous puissions en mettre une infinité dénombrable. C'est en substance ce que signifie la proposition suivante.
\begin{proposition} \label{PROPooFKBEooKXqujV}
    Si \( A\) est infini, alors \( A\times \eN\approx A\).
\end{proposition}

\begin{proof}
    La démonstration se base sur le fait qu'à l'intérieur de \( A\), nous pouvons construire autant de copies de \( A\) deux à deux disjointes que nous le voulons. La \( k\)\ieme\ «copie» sera naturellement l'image de \( k\times A\).

    Voyons tout cela en détail.
    \begin{subproof}
        \item[Ce que nous allons faire]
            Nous allons construire, pour tout \( i\in \eN\) des parties \( A_i,B_i\subset A\) telles que
            \begin{itemize}
                \item \( A_i,B_i\subset B_{i-1}\)
                \item \( A_i\approx B_i\approx A\).
                \item \( A_i\cap A_j=\emptyset\) si \( i\neq j\)
            \end{itemize}
        \item[La construction]
            Nous commençons à zéro en utilisant le corollaire \ref{CORooJCSIooOeOICJ} pour construire \( A_0\) et \( B_0\) dans \( A\) disjoints tels que \( A_0\approx B_0\approx A\) et tels que \( A_0\) et \( B_0\) sont disjoints.

            Ensuite, vu que \( B_0\approx A\), il existe \( A_1\) et \( B_1\) dans \( B_0\) tels que \(  A_1\cap B_1=\emptyset\) et \( A_1\approx B_1\approx B_0\approx A\). Cela est notre construction pour \( i=1\).

            Pour la récurrence, vu que \( A_i\approx B_i\approx A\), nous considérons \( A_{i+1}\) et \( B_{i+1}\) dans \( B_i\) tels que \( A_{i+1}\cap B_{i+1}=\emptyset\) et \( A_{i+1}\approx B_{i+1}\approx B_i\approx A\). C'est encore le corolaire \ref{CORooJCSIooOeOICJ} qui fait le travail.

        \item[Les propriétés]
            Nous avons \( A_i\cap A_{i+1}=\emptyset\) parce que \( A_i\cap A_{i+1}\subset A_i\cap B_i=\emptyset\).

            Nous devons encore montrer que \( A_i\cap A_j=\emptyset\) dès que \( i\neq j\). Supposons que \( j>i\). Nous avons les inclusions
            \begin{equation}
                A_j\subset B_{j-1}\subset B_{j-2}\subset \ldots \subset B_i.
            \end{equation}
            Donc \( A_j\cap A_i\subset B_i\cap A_i=\emptyset\).
        \item[Une injection]
            Nous pouvons à présent écrire une injection qui termine presque la preuve. Pour cela nous considérons pour tout \( i\), une bijection \( \psi_i\colon A\to A_i\). Ensuite nous posons
            \begin{equation}
                \begin{aligned}
                    \varphi\colon A\times \eN&\to A \\
                    (a,k)&\mapsto \psi_k(a). 
                \end{aligned}
            \end{equation}
                Si \( \varphi(a,k)=\varphi(b,l)\), alors \( \psi_k(a)=\psi_l(b)\). L'élément \( \psi_k(a)\) est donc dans \( A_k\cap A_l\); ce n'est possible que si \( k=l\). Donc \( \psi_l(a)=\psi_l(b)\). Cette dernière égalité n'est possible que si \( a=b\) parce que \( \psi_l\) est une bijection.
            
                Donc \( \varphi\) est une injection, et nous avons prouvé que \( A\times \eN\preceq A\).
            \item[La bijection]
                Nous venons de prouver que \( A\times \eN\preceq A\). La surpotence \( A\times \eN\succeq A\) étant évidente, le théorème de Cantor-Schröder-Bernstein \ref{THOooRYZJooQcjlcl} conclu que \( A\times \eN\approx A\).
    \end{subproof}
\end{proof}

\begin{lemma}[\cite{MonCerveau}]        \label{LEMooDHWSooFqhano}
    Sois un ensemble \( A\) muni de deux sous-ensembles \( B\) et \( B'\) équipotents et disjoints. Alors \( A\setminus B\) est équipotent à \( A\setminus B'\).
\end{lemma}

\begin{proof}
    Soit une bijection \( \psi\colon B'\to B\). L'application
    \begin{equation}
        \begin{aligned}
            \varphi\colon A\setminus B&\to A\setminus B' \\
            x&\mapsto \begin{cases}
                x    &   \text{si } x\notin B'\\
                \psi(x)    &    \text{si } x\in B'.
            \end{cases}
        \end{aligned}
    \end{equation}
    fait la bijection.
\end{proof}

\begin{lemma}[\cite{BIBooYBGLooUZuTrc}]       \label{LEMooIVCBooHWQiZB}
    Si \( A\) est un ensemble infini et si \( B\prec A\), alors \( A\approx A\setminus B\).
\end{lemma}

\begin{proof}
    Nous pouvons écrire
    \begin{equation}
        A=(A\setminus B)\cup B.
    \end{equation}
    Le théorème de comparabilité cardinale \ref{THOooCBSKooCmzfUf} nous indique que soit \( A\setminus B\preceq B\), soit \( A\setminus B\succeq B\). Nous allons voir les deux cas.
    \begin{subproof}
        \item[Si \( A\setminus B\succeq B\)] 
            Dans ce cas, \( (A\setminus B)\cup B\approx A\setminus B\) par le lemme \ref{LEMooXMVDooIWLWis}. Dans ce cas, notre résultat est prouvé parce que \( A=(A\setminus B)\cup B\approx A\setminus B\).
        \item[Si \( A\setminus B\preceq B\)] 
            Dans ce cas, le lemme \ref{LEMooXMVDooIWLWis} nous indique que \( A=(A\setminus B)\cup B\approx B\). Mais \( A\approx B\) est exclu par l'hypothèse. Ce cas est donc impossible.
    \end{subproof}
\end{proof}

\begin{lemma}[\cite{MonCerveau}]        \label{LEMooMRVQooUZSSyL}
    Si \( A\) est infini et si \( B\) est une partie strictement subpotente de \( A\), alors il existe \( U\subset A\) disjoint de \( B\) et équipotent à \( B\).
\end{lemma}

\begin{proof}
    Le lemme \ref{LEMooIVCBooHWQiZB} nous donne une bijection \( \varphi\colon A\to A\setminus B\). Il suffit alors de poser \( U=\varphi(B)\). Cette partie est disjointe de \( B\) parce que \( \varphi\) prend ses valeurs dans \( A\setminus B\).
\end{proof}

\begin{lemma}[\cite{BIBooDLDFooFwXSGV}]
    Soit un ensemble infini \( A\) ainsi qu'un sous-ensemble \( B\subset A\). Nous supposons l'existence d'une fonction surjective \( f\colon B\to B\times B\).

    Alors \( B\preceq B\times B\preceq B\preceq A\).
\end{lemma}

\begin{proof}
    La première est l'hypothèse sur \( f\). La seconde est l'existence (évidente) d'une surjection \( B\times B\to B\). La troisième est le fait que \( B\) soit inclus à \( A\).
\end{proof}

\begin{lemma}[\cite{BIBooDLDFooFwXSGV}]     \label{LEMooPOEFooXaifhT}
    Soit un ensemble infini \( A\) ainsi qu'un sous-ensemble strictement subpotent \( B\subset A\). Nous supposons l'existence d'une fonction surjective \( f\colon B\to B\times B\).

    Alors \( f\) peut être étendue en une injection \( f\colon D\to D\times D\) où \( D\subset A\) contient strictement \( B\).
\end{lemma}

\begin{proof}
    Par le lemme \ref{LEMooMRVQooUZSSyL}, nous considérons une partie \( U\subset A\) disjointe de \( B\) et équipotent à \( B\). Nous pouvons faire le développement
    \begin{equation}
        (B\cup U)\times (B\cup U)=(B\times B)\cup(B\times U)\cup (U\times B)\cup (U\times U).
    \end{equation}
    Nous savons que \( B\) est surpotent à \( U\) (il est même équipotent); donc le lemme \ref{LEMooXMVDooIWLWis} nous dit que \( B\cup U\approx B\). De plus il existe une bijection \( B\to U\), donc 
    \begin{equation}
        B\approx B\times B\approx B\times U\approx U\times B\approx U\times U.
    \end{equation}
    Et enfin, la réunion d'ensembles équipotents est équipotent. Bref, nous avons une bijection
    \begin{equation}
        \varphi\colon U\to (U\times B)\cup (B\times U)\cup (U\times U).
    \end{equation}
    Et enfin nous définissons
    \begin{equation}
        \begin{aligned}
            g\colon B\cup U&\to (B\times U)\times (B\cup U) \\
            x&\mapsto \begin{cases}
                f(x)    &   \text{si }  x\in B\\
                \varphi(x)    &    \text{si } x\in U.
            \end{cases}
        \end{aligned}
    \end{equation}
    Cette définition est bonne parce que \( U\) et \( B\) sont disjoints, et \( g\) est injective.
\end{proof}

Le théorème suivant est une généralisation de la proposition \ref{PropVCSooMzmIX}. Elle implique, entre autres choses, qu'il existe une bijection entre \( \eR\) et \( \eR\times \eR\). Pour le cas de \( \eN\times \eN\approx \eN\), il y a la proposition \ref{PROPooLPKUooAlsYJg} qui donne une bijection explicite et donc sans axiome du choix et sans lemme de Zorn.
\begin{theorem}     \label{THOooDGOVooRdURVi}
    Si \( A\) est infini, alors \( A\approx A\times A\).
\end{theorem}

\begin{proof}
    Une fois de plus, ce sera le lemme de Zorn qui va s'y coller. Soit l'ensemble
    \begin{equation}
       \mA=\Big\{  (X,\varphi)  \tq
        \begin{cases}
            X\subset A\\
            \varphi\colon X\to X\times X\text{ est surjective.}
        \end{cases}
    \Big\}
    \end{equation}
    Cet ensemble est non vide parce que \( A\) est infini; il contient donc une partie dénombrables \( N\), et nous connaissons la surjection \( \eN\to \eN\times \eN\) du lemme \ref{PROPooLPKUooAlsYJg}.

    Nous ordonnons \( \mA\) par l'inclusion : \( (X,\varphi)\leq (Y,\phi)\) lorsque \( X\subset Y\) et \( \phi|_X=\varphi\). La tambouille usuelle montre que \( \mA\) est un ensemble inductif et le lemme de Zorn \ref{LemUEGjJBc} donne l'existence d'un élément maximum que nous notons \( (B,\varphi)\).

    Vu que \( B\) est subpotent à \( A\) (parce qu'il est inclus), soit il est strictement subpotent soit il est équipotent. Nous commençons par montrer que \( B\) ne peut pas être strictement subpotent à \( A\).

    En effet, si nous avions une surjection \( B\to B\times B\), alors que \( B\) est strictement subpotent à \( A\). Le lemme \ref{LEMooPOEFooXaifhT} nous dit alors que \( \varphi\) peut être étendue, ce qui contredirait la maximalité de \( (B,\varphi)\).

    Donc la partie \( B\) est équipotente à \( A\) : il existe une bijection \( g\colon A\to B\). Mais nous avons une surjection \( B\to B\times B\) et donc aussi une injection \( B\times B\to B\). Vu que nous avons par ailleurs une injection \( B\to B\times B\), le théorème de Cantor-Schröder-Bernstein \ref{THOooRYZJooQcjlcl} nous donne une bijection \( \phi\colon B\times B\to B\). Avec ça, l'application
    \begin{equation}
        \begin{aligned}
            f\colon A\times A&\to A \\
            (a,b)&\mapsto \phi\big( g(a),g(b) \big) 
        \end{aligned}
    \end{equation}
    est une bijection. Donc les ensembles \( A\) et \( A\times A\) sont équipotents.
\end{proof}

\begin{lemma}       \label{LEMooNKKDooUvSYPO}
    Si \( A\) est infini, et si pour tout \( i\in \eN\) nous avons \( A_i\approx A\), alors
    \begin{equation}
        \bigcup_{i\in \eN}A_i\approx A.
    \end{equation}
\end{lemma}

\begin{proof}
    Pour chaque \( i\in \eN\) nous avons une bijection \( \varphi_i\colon A_i\to A\). Nous posons
    \begin{equation}        \label{EQooCHJAooRpHypV}
        \begin{aligned}
            \varphi\colon A\times \eN&\to \bigcup_{i=0}^{\infty}A_i \\
            (a,i)&\mapsto \varphi_i(a). 
        \end{aligned}
    \end{equation}
    Cette application est surjective mais peut-être pas injective parce que les \( A_i\) peuvent avoir des intersections non vides. Nous avons alors le calcul
    \begin{subequations}
        \begin{align}
            A&\approx A\times \eN        \label{SUBEQooICFEooTLuFHZ}\\
            &\succeq \bigcup_{i=0}^{\infty}A_i      \label{SUBEQooRVPRooJJevkv}\\
            &\succeq A      \label{SUBEQooFJRGooJnervy}
        \end{align}
    \end{subequations}
    Justifications :
    \begin{itemize}
        \item Pour \eqref{SUBEQooICFEooTLuFHZ}, c'est la proposition \ref{PROPooFKBEooKXqujV}.
        \item Pour \eqref{SUBEQooRVPRooJJevkv}, c'est la surjection \eqref{EQooCHJAooRpHypV}.
        \item Pour \eqref{SUBEQooFJRGooJnervy}, c'est le fait que seulement \( A_0\) possède déjà une surjection vers \( A\).
    \end{itemize}
    Donc \( \bigcup_iA_i\) est à la fois surpotent et subpotent à \( A\). Il est donc équipotent par le théorème \ref{THOooRYZJooQcjlcl}.
\end{proof}

%+++++++++++++++++++++++++++++++++++++++++++++++++++++++++++++++++++++++++++++++++++++++++++++++++++++++++++++++++++++++++++ 
\section{Groupes}
%+++++++++++++++++++++++++++++++++++++++++++++++++++++++++++++++++++++++++++++++++++++++++++++++++++++++++++++++++++++++++++

%---------------------------------------------------------------------------------------------------------------------------
\subsection{Définition, unicité du neutre}
%---------------------------------------------------------------------------------------------------------------------------

\begin{definition}[Groupe]      \label{DEFooBMUZooLAfbeM}
    Un \defe{groupe}{groupe} est un ensemble \( G\) muni d'une opération interne \( \cdot\colon G\times G\to G\) telle que
    \begin{enumerate}
        \item
            pour tous \( g\), \( h\), \( k\in G\), \( g\cdot(h\cdot k)=(g\cdot h)\cdot k\),
        \item
            il existe un élément \( e\in G\) tel que \( e\cdot g=g\cdot e=g\) pour tout \( g\in G\),
        \item
            pour tout \( g\in G\), il existe un élément \( h\in  G\) tel que \(g\cdot h=h\cdot g=e \).
    \end{enumerate}
\end{definition}

    Notons que nous avons écrit \( g\cdot h\) et non \( \cdot(g,h)\) comme une notation purement fonctionnelle nous l'aurait suggéré. Dans les exemples concrets, selon les cas, le loi de groupe appliquée à \( g\) et \( h\) sera notée tantôt \( g+h\), tantôt \( g\cdot h\) ou, le plus souvent pour un groupe générique, simplement \( gh\).

\begin{lemmaDef}[Unicités]  \label{LEMooECDMooCkWxXf}
    L'inverse et le neutre sont uniques, c'est-à-dire :
    \begin{enumerate}
        \item
            il existe un unique élément \( e\in G\) tel que \( e g=g e=g\) pour tout \( g\in G\),
        \item       \label{ITEMooOIWTooYqmMPP}
            pour tout \( g\in G\), il existe un unique élément \( h\in  G\) tel que \(g h=h g=e \).
    \end{enumerate}
    Le \( e\) ainsi défini est nommé \defe{neutre}{neutre!dans un groupe} de \( G\). Le \( h\) tel que \( g h=h g=e\) est nommé l'\defe{inverse}{inverse!dans un groupe} de \( g\) et est noté \( g^{-1}\).
\end{lemmaDef}

\begin{proof}
    Chaque point séparément.
    \begin{enumerate}
        \item
            Supposons que \( e_1\) et \( e_2\) vérifient la propriété. Nous avons pour tout \( g\in G\) : \( e_1g=ge_1=g\). En particulier pour \( g=e_2\) nous écrivons \( e_1e_2=e_2e_1=e_2\). Mais en partant dans l'autre sens : \( e_2g=ge_2=g\) avec \( g=e_1\) nous avons \( e_2e_1=e_1e_2=e_1\). En égalant ces deux valeurs de \( e_2e_1\) nous avons \( e_1=e_2\).

            Pour la suite de la preuve nous écrivons \( e\) l'unique neutre de \( G\).

        \item
            Supposons que \( k_1\) et \( k_2\) soient deux inverses de \( g\). On considère alors le produit \( k_1 g k_2 \). Puisque \(k_1 g = e \), on a \( k_1 g k_2 = e k_2 = k_2 \); mais, comme \(g k_2 = e \), on a aussi \( k_1 g k_2 = k_1 e = k_1 \). Le produit est donc à la fois égal à \( k_1 \) et à \( k_2 \), et donc \( k_1 = k_2 \).
    \end{enumerate}
\end{proof}

\begin{definition}
    Un groupe \( G\) est \defe{abélien}{abélien!groupe} ou \defe{commutatif}{commutatif!groupe} si pour tous \( g\) et \( h\) dans \( G\), \( gh=hg\).
\end{definition}

\begin{lemma}       \label{LEMooBIBFooBHxFYC}
    Si \( G\) est un groupe et si \( h\in G\), alors les applications
    \begin{equation}
        \begin{aligned}
            L_h\colon G&\to G \\
            g&\mapsto hg 
        \end{aligned}
    \end{equation}
    et
    \begin{equation}
        \begin{aligned}
            R_h\colon G&\to G \\
            g&\mapsto gh 
        \end{aligned}
    \end{equation}
    sont des bijections.
\end{lemma}

\begin{proof}
    D'abord si \( L_h(g_1)=L_h(g_2)\), alors \( hg_1=hg_2\) et en multipliant à gauche par \( h^{-1}\) nous avons \( g_1=g_2\); donc \( L_h\) est injective. Ensuite \( L_h\) est surjective parce que si \( g\in G\), alors \( g=L_h(h^{-1} g)\).

    Pour l'application \( R_h\), la preuve est une simple adaptation.
\end{proof}

%+++++++++++++++++++++++++++++++++++++++++++++++++++++++++++++++++++++++++++++++++++++++++++++++++++++++++++++++++++++++++++ 
\section{Anneaux}
%+++++++++++++++++++++++++++++++++++++++++++++++++++++++++++++++++++++++++++++++++++++++++++++++++++++++++++++++++++++++++++

\begin{definition}[Anneau\cite{Tauvel}]     \label{DefHXJUooKoovob}
    Un \defe{anneau}{anneau}\footnote{Nous faisons le choix qu'un anneau admet toujours un neutre pour la multiplication. Certains ouvrages parlent dans ce cas d'anneau unitaire.} est un triplet \( (A,+,\cdot)\) avec les conditions
    \begin{enumerate}
        \item
            \( (A,+)\) est un groupe abélien. Nous notons \( 0\) le neutre.
        \item
            La multiplication est associative et nous notons \( 1\) le neutre.
        \item
            La multiplication est distributive par rapport à l'addition.
    \end{enumerate}
\end{definition}

\begin{definition}[Morphisme d'anneaux\cite{ooZRUJooXyxPqQ}]      \label{DEFooSPHPooCwjzuz}
    Si \( (A,+,\cdot)\) et \( (B,+,\cdot)\) sont des anneaux, un \defe{morphisme d'anneaux}{morphisme!d'anneaux} est une application \( f\colon A\to B\) telle que
    \begin{enumerate}
        \item \( f(a+b)=f(a)+f(b)\)
        \item
            \( f(a\cdot b)=f(a)\cdot f(b)\)
        \item
            \( f(1)=1\).
    \end{enumerate}
    Étant bien entendu que les significations de \( 1\), $+$ et \( \cdot\) sont différentes à gauche et à droite.
\end{definition}

\begin{lemma}       \label{LEMooVUSMooWisQpD}
    Pour tout élément \( a\) d'un anneau nous avons \( a\cdot 0=0\).
\end{lemma}

\begin{proof}
    L'élément \( 0\) est le neutre de l'addition. Il peut être écrit \( 1-1\), et en utilisant la distributivité,
    \begin{equation}
        a\cdot 0=a\cdot (1-1)=a-a=0.
    \end{equation}
    Notons que la dernière égalité s'écrit en détail \( a-a=a+(-a)\) qui donne le neutre de l'addition.
\end{proof}

\begin{proposition}     \label{PROPooNCCGooXjVyVt}
    Dans un anneau non nul, le neutre pour l'addition est distinct du neutre pour la multiplication.
\end{proposition}
\begin{proof}
    Supposons par contraposée que dans un anneau $A$, \( 1 = 0 \). Alors, pour tout \( a \in A \), on a \( a = 1a = 0a = (1 - 1)a = a - a=0 \), d'où l'on déduit \( -a = 0  \) et par suite, \( a = 0. \)
\end{proof}

Soit \( X\) un ensemble et un anneau $(A, +, \times)$. Nous considérons \( \Fun(X,A)\)\nomenclature[A]{\( \Fun(X,Y)\)}{les applications de \( X\) vers \( Y\)} l'ensemble des applications \( X\to A\). Cet ensemble devient un anneau avec les définitions
\begin{subequations}
    \begin{align}
        (f+g)(x)=f(x)+g(x)\\
        (fg)(x)=f(x)g(x).
    \end{align}
\end{subequations}
Cela est la \defe{structure canonique}{structure d'anneau canonique} d'anneau sur \( \Fun(X,A)\).

\begin{definition}
    Le \defe{centralisateur}{centralisateur} de \( x\in A\) dans \( A\) est l'ensemble
    \begin{equation}
        \{ y\in A\tq xy=yx \},
    \end{equation}
    le \defe{centre}{centre!d'un anneau} de \( A\) est
    \begin{equation}
        \{ y\in A\tq xy=yx,\forall x\in A \}.
    \end{equation}
\end{definition}

\begin{definition}
    Nous disons que \( A\) est un \defe{anneau commutatif}{anneau commutatif} si pour tout \( a,b\in A\) nous avons \( a+b=b+a\).
\end{definition}

\begin{definition}
    Un \defe{isomorphisme d'anneaux}{isomorphisme!d'anneaux} est un morphisme bijectif.
\end{definition}

\begin{definition}[Idéal dans un anneau]  \label{DefooQULAooREUIU}
    Un sous-ensemble \( I\subset A\) est un \defe{idéal à gauche}{idéal!dans un anneau} à gauche si
    \begin{enumerate}
        \item
            \( I\) est un sous-groupe pour l'addition,
        \item
            pour tout \( a\in A\), \( aI\subset I\).
    \end{enumerate}

    Lorsqu'un ensemble est idéal à gauche et à droite, nous disons que c'est un \defe{idéal bilatère}{idéal!bilatère}. Lorsque nous parlons d'idéal sans précisions, nous parlons d'idéal bilatère.
\end{definition}

\begin{propositionDef}      \label{PROPooGXMRooTcUGbi}
    Soit \( A\), un anneau, \( I\) un idéal bilatère\footnote{Définition~\ref{DefooQULAooREUIU}.} de \( A\). Nous considérons la relation d'équivalence \( x\sim y\) si et seulement si \( x-y\in I\). Sur le quotient\footnote{Définition \ref{DEFooRHPSooHKBZXl}.}
    \begin{equation}
        A/\sim=A/I,
    \end{equation}
    nous mettons les opérations
    \begin{enumerate}
        \item
            \( [x]+[y]=[x+y]\)
        \item
            \( [x][y]=[xy]\).
    \end{enumerate}
    Nous avons alors les résultats suivants :
    \begin{enumerate}
        \item       \label{ITEMooEJPEooRKAqmS}
            Les opérations sont bien définies,
        \item       \label{ITEMooYBEGooTlHgNz}
            l'ensemble \( A/I\), muni de ces opérations, est un anneau
        \item       \label{ITEMooLNRLooMkoWXZ}
            la surjection canonique \( \pi\colon A\to A/I\) est un morphisme.
    \end{enumerate}
    Cet anneau est appelé \defe{anneau quotient}{anneau!quotient par un idéal}.
\end{propositionDef}

\begin{proof}
    En plusieurs parties.
    \begin{subproof}
        \item[Pour \ref{ITEMooEJPEooRKAqmS}]
            Nous savons que, par définition,
            \begin{equation}
                \bar x=\{ x+i\tq i\in I \}.
            \end{equation}
            Calculons le produit de représentants génériques de \( \bar x\) et de \( \bar y\) :
            \begin{equation}
                (x+i_1)(y+i_2)=xy+xi_2+yi_1+i_1i_2.
            \end{equation}
            Vu que \( I\) est un idéal nous avons \( xi_2+yi_1+i_1i_2\in I\) et donc bien
            \begin{equation}
                (x+i_1)(y+i_2)\in \overline{ xy }.
            \end{equation}
        \item[Pour \ref{ITEMooYBEGooTlHgNz}]
            Il s'agit de vérifier les conditions de la définition \ref{DefHXJUooKoovob}.

            Vu que \( (A,+)\) est un groupe abélien de neutre \( 0\), nous avons
            \begin{equation}
                [a]+[b]=[a+b]=[b+a]=[b+a],
            \end{equation}
            ainsi que
            \begin{equation}
                [a]+[0]=[a+0]=[a]
            \end{equation}
            et
            \begin{equation}
                [a]+([b]+[c])=[a]+[b+c]=[a+b+c]=[a+b]+[c].
            \end{equation}
            Donc \( (A/I,+)\) est un groupe abélien de neutre \( [0]\).

            L'associativité de \( A\) donne l'associativité dans \( A/I\) :
            \begin{equation}
                \big( [a][b] \big)[c]=[ab][c]=[abc]=[a][bc]=[a]\big( [b][c] \big).
            \end{equation}
            Et enfin pour la distributivité,
            \begin{equation}
                [a]\big( [b]+[c] \big)=[a][b+c]=[a(b+c)]=[ab+ac]=[ab]+[ac]=[a][b]+[a][c].
            \end{equation}
            Nous avons prouvé que \( A/I\) est un anneau de neutre \( [0]\) et d'unité \( [1]\).
        \item[Pour \ref{ITEMooLNRLooMkoWXZ}]
            Nous devons vérifier les trois conditions de la définition \ref{DEFooSPHPooCwjzuz}. Cela est immédiat parce que \( \pi(x)=[x]\).
    \end{subproof}
\end{proof}


\begin{definition}\label{DefrYwbct}
    Soient \( A\) un anneau commutatif et \( S\subset A\). Nous disons que \( \delta\in A\) est un \defe{PGCD}{pgcd!dans un anneau intègre} de \( S\) si
    \begin{enumerate}
        \item
            \( \delta\) divise tous les éléments de \( S\).
        \item
            si \( d\) divise également tous les éléments de \( S\), alors \( d\) divise \( \delta\).
    \end{enumerate}
    Nous disons que \( \mu\in A\) est un \defe{PPCM}{ppcm!dans un anneau intègre} de \( S\) si
    \begin{enumerate}
        \item
            \( S\divides \mu\),
        \item
            si \( S\divides m\), alors \( \mu\divides m\).
    \end{enumerate}
\end{definition}

\begin{remark}
    Au sens de la définition \ref{DefrYwbct}, le pgcd n'est pas unique. Dans \( \eZ\) par exemple les nombres \( 4\) et \( -4\) sont tout deux pgcd de \( \{4,16  \}\).

    Dans \( \eZ\) cependant, nous modifions implicitement la définition et nous n'acceptons que les positifs, de telle sorte à ce que l'unique pgcd soit effectivement le plus grand pour l'ordre usuel sur \( \eZ\).

    Pour l'unicité dans \( \eZ\), voir \ref{LEMooBJVJooFyuFeN}.
\end{remark}

%+++++++++++++++++++++++++++++++++++++++++++++++++++++++++++++++++++++++++++++++++++++++++++++++++++++++++++++++++++++++++++
\section{Les entiers}
%+++++++++++++++++++++++++++++++++++++++++++++++++++++++++++++++++++++++++++++++++++++++++++++++++++++++++++++++++++++++++++

\begin{lemma}       \label{LEMooMYEIooNFwNVI}
    Toute partie bornée de \( \eZ\) possède un plus grand élément.
\end{lemma}

\begin{proposition}     \label{PROPooYJBMooZrzkNX}
    Soit \( a,b\in \eZ\) tels que \( a\) divise \( b\). Alors \( a\leq b\).
\end{proposition}

%---------------------------------------------------------------------------------------------------------------------------
\subsection{Quelques autres résultats}
%---------------------------------------------------------------------------------------------------------------------------

Nous supposons ici connaitre, et avoir démontré les rudiments du calcul dans \( \eN\) et \( \eZ\), en particulier les produits remarquables.


%---------------------------------------------------------------------------------------------------------------------------
\subsection{Anneau intègre}
%---------------------------------------------------------------------------------------------------------------------------

\begin{definition}[Diviseurs dans un anneau]\label{DiviseursAnneau}
	Soient \( a, b \in A \). On dit que $a$ divise $b$, ou que $a$ est un \defe{diviseur (à gauche)}{diviseur!dans un anneau} de $b$ s'il existe \( c \in A \) tel que \( ac = b \). On dit que c'est un diviseur de $b$ à droite si \( ca = b \) pour un certain \( c \in A \).
\end{definition}
Un cas particulier est le cas des diviseurs de zéro. L'absence de tels diviseurs dans un anneau est une propriété intéressante: on dit dans ce cas que l'anneau est intègre. Nous étudions ces anneaux plus en détail en section~\ref{SECAnneauxIntegres}.

Un élément \( a\in A\) est \defe{régulier à droite}{régulier à droite} si \( ba=0\) implique \( b=0\). Il est régulier à gauche si \( ab=0\) implique \( b=0\).

\begin{definition}[Éléments nilpotents, unipotents et inversibles]
	On dit que \( a \in A \) est \defe{nilpotent}{nilpotent} s'il existe \( n \in \eN \) tel que \( a^n = 0 \). Il est dit \defe{unipotent}{unipotent} si \( a-1\) est nilpotent, c'est-à-dire si \( (a-1)^n =0\) pour un certain \( n \in \eN \).

	Un élément \( a \in A \) est dit \defe{inversible}{élément!inversible!dans un anneau} s'il existe \( b \in A \) tel que \( ab = 1 \).
\end{definition}

L'ensemble \( U(A)\)\nomenclature[A]{\( U(A)\)}{ensemble des inversibles} des éléments inversibles de \( A\) est un groupe pour la multiplication. Nous notons \( A^*=A\setminus\{ 0 \}\).

Conformément à la définition \ref{DiviseursAnneau} de diviseur, nous posons la définition suivante pour les diviseurs de zéro.
\begin{definition}[\cite{ooTNKJooSCSCZQ}]
    Un élément \( a\neq 0\) est un \defe{diviseur de zéro à gauche}{diviseur!de zéro} s'il existe \( x\neq 0\) tel que $xa=0$. L'élément \( a\) est un diviseur de zéro \defe{à droite}{diviseur!de zéro à droite} s'il existe \( b\) tel que \( ab=0\).

    Nous disons que \( a\) est un \defe{diviseur de zéro}{diviseur de zéro} si il est un diviseur de zéro à gauche ou à droite.
\end{definition}

\begin{propositionDef}[\cite{MonCerveau}]           \label{DEFooTAOPooWDPYmd}
    Soit $A$ un anneau non réduit à \( \{ 0 \}\). Les assertions suivantes sont équivalentes:
    \begin{enumerate}
        \item       \label{ITEMooMXMKooXMYpkN}
            \( A\) ne possède pas de diviseurs de zéro.
        \item       \label{ITEMooLAJCooFwxXrV}
            La règle du produit nul s'applique dans $A$: pour tous \( a, b \in A \), si \( ab=0\), alors \( a = 0\) ou \( b = 0\).
            \index{règle du produit nul}
        \item       \label{ITEMooQNTFooSRrVPK}
            On peut simplifier par un même élément non-nul, deux expressions produit dans $A$ qui sont égales: pour tous \( a, b, c \in A \) avec \( a \neq 0 \), si \( ab = ac \), alors \( b = c \).
    \end{enumerate}
    Un anneau non réduit à \( \{ 0 \}\) qui vérifie ces propriétés est dit \defe{intègre}{anneau intègre}.
\end{propositionDef}

\begin{proof}
    En trois implications.
    \begin{subproof}
        \item[\ref{ITEMooMXMKooXMYpkN} implique \ref{ITEMooLAJCooFwxXrV}]

            Si \( ab=0\) avec \( a,b\neq 0\) alors \( a\) est un diviseur de zéro. Vu que nous supposons que \( A\) n'a pas de diviseurs de zéros, soit \( a\) soi \( b\) est nul.
        \item[\ref{ITEMooLAJCooFwxXrV} implique \ref{ITEMooQNTFooSRrVPK}]

            Si \( ab=ac\), alors \( a(b-c)=0\) et l'hypothèse dit que soit \( a=0\), soit \( b-c=0\). Donc si \( a\neq 0\), alors \( b-c=0\).
        \item[\ref{ITEMooQNTFooSRrVPK} implique \ref{ITEMooMXMKooXMYpkN}]
            Si \( A=\{ 0 \}\), le point \ref{ITEMooQNTFooSRrVPK} n'est pas applicable.

            Si \( a\neq 0\) et \( xa=0\), alors nous avons aussi \( xa=0\times a\). Par propriété de simplification, \( x=0\). Donc \( a\) n'est pas un diviseur de zéro à gauche. Nous prouvons de la même façon qu'il n'y a pas de diviseurs de zéro à droite.
    \end{subproof}
\end{proof}

%+++++++++++++++++++++++++++++++++++++++++++++++++++++++++++++++++++++++++++++++++++++++++++++++++++++++++++++++++++++++++++ 
\section{Somme à valeurs dans un groupe abélien}
%+++++++++++++++++++++++++++++++++++++++++++++++++++++++++++++++++++++++++++++++++++++++++++++++++++++++++++++++++++++++++++

%--------------------------------------------------------------------------------------------------------------------------- 
\subsection{Somme à valeurs dans un anneau}
%---------------------------------------------------------------------------------------------------------------------------

\begin{definition}      \label{DEFooNEVNooJlmJOC}
    Si \( f\colon \{ 0,\ldots, N \}\to A\) est une application vers un anneau \( A\), alors nous définissons la notation \( \sum_{i=0}^Nf(i)\) par récurrence de la façon suivante :
    \begin{enumerate}
        \item
            \( \sum_{i=0}^0f(i)=f(0)\),
        \item
            \( \sum_{i=0}^{k}f(i)=\sum_{i=0}^{k-1}f(i)+f(k)\).
    \end{enumerate}
\end{definition}

%--------------------------------------------------------------------------------------------------------------------------- 
\subsection{Somme à valeurs dans un groupe abélien}
%---------------------------------------------------------------------------------------------------------------------------

Si \( S\) est un ensemble fini, nous savons de la proposition \ref{PROPooJLGKooDCcnWi} qu'il existe un unique \( N\in \eN\) pour lequel il existe une bijection \( \varphi\colon \{ 0,\ldots, N \}\to S\). Cette bijection n'est à priori pas unique.

\begin{lemmaDef}[\cite{MonCerveau}]       \label{DEFooLNEXooYMQjRo}
    Soient un groupe abélien \( (G,+)\) ainsi qu'un ensemble fini \( I\) contenant \( n\) éléments. Soit une application \( f\colon I\to G \). Si \( \sigma_1,\sigma_2\colon \{1,\ldots, n \}\to I\) sont deux bijections, alors\footnote{Pour rappel, le symbole \( \sum_{i=1}^n\) est défini par \ref{DEFooNEVNooJlmJOC}.}
    \begin{equation}
        \sum_{i=1}^nf\big( \sigma_1(i) \big)=\sum_{i=1}^nf\big( \sigma_2(i) \big).
    \end{equation}
    La valeur commune est notée
    \begin{equation}
        \sum_{i\in I}f(i)
    \end{equation}
\end{lemmaDef}

La définition \ref{DEFooLNEXooYMQjRo} donne lieu à un certain nombre de remarques.
\begin{enumerate}
    \item
        Elle donne la somme sur un ensemble fini. Un problème avec les ensembles infinis (outre la convergence) est l'ordre de sommation. Si vous voulez sommer sur \( \eZ\), dans quel ordre le faire ?
    \item
        Pour aller plus loin, et sommer sur des ensembles infinis, il faut regarder la définition \ref{DefHYgkkA}. 
\end{enumerate}

\begin{proposition}     \label{PROPooJBQVooNqWErk}
    Soient un groupe abélien \( (G,+)\), un ensemble fini \( I\) est un ensemble fini, une application \( f\colon I\to G\) et une bijection \( \sigma\colon I\to I\). Alors
    \begin{equation}
        \sum_i\in If(i)=\sum_{i\in I}f\big( \sigma(i) \big).
    \end{equation}
\end{proposition}

Si nous avons une application \( L\colon S\to S\), nous notons
\begin{equation}
    \sum_{s\in S}f\big( L(s) \big)=\sum_{s\in S}(f\circ L)(s).
\end{equation}
Cette façon d'écrire donne une interprétation pour la notation \( \sum_{g\in G}f(hg)\) qui arrive dans la proposition \ref{PROPooWJQQooFINSEc}. Il s'agit de considérer l'application \( L_h\) du lemme \ref{LEMooBIBFooBHxFYC}, de considérer\footnote{Le fait que \( L_h\) soit une bijection n'a pas d'importance ici.}
\begin{equation}        \label{EQooQQBEooFDOBVG}
    \sum_{g\in G}f(hg)=\sum_{g\in G}(f\circ L_h)(g)
\end{equation}
et de faire tourner la définition \ref{DEFooLNEXooYMQjRo}. La même chose tient pour définir \( \sum_{g\in G}(gh)\) à l'aide de \( R_h\).

\begin{proposition}[\cite{MonCerveau}]     \label{PROPooWJQQooFINSEc}
    Soient un groupe fini \( G\) et une fonction \( f\colon G\to A\) où \( A\) est un anneau commutatif. Alors
    \begin{equation}
        \sum_{g\in G}f(g)=\sum_{g\in G}f(gh)=\sum_{g\in G}f(hg)
    \end{equation}
    pour tout \( h\in G\).
\end{proposition}

\begin{proof}
    Nous avons une bijection \( \varphi\colon \{ 0,\ldots,  N \}\to G\) garantie par la proposition \ref{PROPooJLGKooDCcnWi}. La définition est que
    \begin{equation}
        \sum_{g\in G}f(g)=\sum_{i=0}^Nf\big( \varphi(i) \big).
    \end{equation}
    Par ailleurs, le lemme \ref{LEMooBIBFooBHxFYC} donne une bijection \( L_h\colon G\to G\) et permet de considérer la composée
    \begin{equation}
        \begin{aligned}
            \varphi'\colon \{ 0,\ldots,  N \}&\to G \\
            \varphi'=L_h\circ \varphi.
        \end{aligned}
    \end{equation}
    La proposition \ref{DEFooLNEXooYMQjRo} nous permet d'utiliser la bijection \( \varphi'\) au lieu de \( \varphi\) pour exprimer la somme \( \sum_{g\in G}\). Ensuite un jeu de notation utilisant \eqref{EQooQQBEooFDOBVG} donne
    \begin{equation}
        \begin{aligned}[]
            &\sum_{g\in G}f(g)=\sum_{i=0}^Nf\big( \varphi(i) \big)=\sum_{i=0}^Nf\big( \varphi'(i) \big)=\sum_{i=0}^N(f\circ L_h\circ \varphi)(i)\\
            &\quad=\sum_{i=0}^N(f\circ L_h)\big( \varphi(i) \big)=\sum_{g\in G}(f\circ L_h)(g)=\sum_{g\in G}f(hg).
        \end{aligned}
    \end{equation}
    En ce qui concerne \( \sum_{g\in G}f(gh)\), c'est la même chose, en utilisant \( R_h\) au lieu de \( L_h\).
\end{proof}

Tout cela nous permet de faire une somme sympathique et bien connue.
\begin{lemma}
    Soit \( n\in \eN\). Nous avons
    \begin{equation}
        \sum_{k=0}^nk=\frac{ n(n+1) }{ 2 }.
    \end{equation}
\end{lemma}

\begin{proof}
    La preuve est pratiquement immédiate par récurrence. Nous allons donner une preuve plus «constructive», qui formalise l'idée classique d'écrire la somme à l'endroit et à l'envers.


    Nous notons \( S\) la somme \( \sum_{k=0}^nk\). Le lemme \ref{DEFooLNEXooYMQjRo} dit que si \( \sigma_i\colon \{ 0,\ldots, n \}\to \{ 0,\ldots, n \}\) sont deux bijections, alors \( \sum_{k=0}^nf\big( \sigma_1(k) \big)=\sum_{k=0}^nf\big( \sigma_2(k) \big)\). Nous sommes intéressé au cas \( f(i)=i\).

    En prenant \( \sigma_1(k)=k\) et \( \sigma_2(k)=n-k\), nous avons
    \begin{equation}
        S=\sum_{k=0}^nk=\sum_{k=0}^n(n-k).
    \end{equation}
    Donc
    \begin{equation}
        2S=\sum_{k=0}^n\big( k+(n-k) \big)=\sum_{k=0}^nn=n\sum_{k=0}^n1=n(n+1).
    \end{equation}
    En divisant par deux, nous obtenons le résultat annoncé.
\end{proof}

Dans le même ordre d'idée, pour le produit.
\begin{proposition}     \label{PROPooQMUDooQQVRIe}
    Si \( E\) est un ensemble fini et si \( G\) est un groupe abélien, alors pour toute fonction \( f\colon E\to G\) et pour toute permutation \( \sigma\) de \( E\),
    \begin{equation}
        \prod_{i\in E}f(i)=\prod_{i\in E}f\big( \sigma(i) \big)
    \end{equation}
\end{proposition}

%--------------------------------------------------------------------------------------------------------------------------- 
\subsection{Fonction puissance}
%---------------------------------------------------------------------------------------------------------------------------

Voici une première définition de la fonction puissance. Il y en aura d'autres, de plus en plus générales. Voir le thème \ref{THEMEooBSBLooWcaQnR}.
\begin{definition}\label{DEFooGVSFooFVLtNo}
    Si \( A\) est un anneau, si \( a\in A\) et si \( n\in \eN\), nous définissons \( a^n\) par récurrence :
    \begin{enumerate}
        \item
            \( a^0=1\) (l'unité pour la multiplication dans \( A\)),
        \item       \label{ITEMooOUIPooGjAgQb}
            \( a^{k+1}=a\cdot a^{k}\).
    \end{enumerate}
\end{definition}

Le lemme suivant dit que le point \ref{ITEMooOUIPooGjAgQb} de la définition \ref{DEFooGVSFooFVLtNo} aurait pu être écrit \( a^k\cdot a\) au lieu de \( a\cdot a^k\).
\begin{lemma}[\cite{MonCerveau}]        \label{LEMooWPARooYLZlzr}
    Si \( A\) est un anneau, si \( a\in A\) et si \( n\in \eN\), alors
    \begin{equation}
        a^n=a\cdot a^{n-1}=a^{n-1}\cdot a.
    \end{equation}
\end{lemma}


%+++++++++++++++++++++++++++++++++++++++++++++++++++++++++++++++++++++++++++++++++++++++++++++++++++++++++++++++++++++++++++
\section{Corps}
%+++++++++++++++++++++++++++++++++++++++++++++++++++++++++++++++++++++++++++++++++++++++++++++++++++++++++++++++++++++++++++

%---------------------------------------------------------------------------------------------------------------------------
\subsection{Définitions, morphismes}
%---------------------------------------------------------------------------------------------------------------------------

\begin{definition}[\cite{ooLKFGooTUrnhx}]  \label{DefTMNooKXHUd}
    Un \defe{corps}{corps} est un anneau\footnote{Définition \ref{DefHXJUooKoovob}.} dans lequel tout élément non nul est inversible.
\end{definition}

\begin{remark}      \label{REMooYRNUooYgBBKF}
    Un anneau est ce qu'on appelle «\emph{ring}» en anglais. Un corps est en anglais «\emph{field}». De plus le mot «\emph{field}» comprend la commutativité. Donc certains utilisent le mot «corps» pour dire «corps commutatif» et parlent alors d'anneau \emph{à division} pour parler de corps non commutatifs.
\end{remark}

La proposition suivante donne une caractérisation d'un corps, en disant un tout petit peu plus que la définition~\ref{DefTMNooKXHUd}.
\begin{proposition}
    L'anneau $A$ est un corps si et seulement si \( U(A) = A^* \).
\end{proposition}

\begin{proof}
    En deux parties.
    \begin{subproof}
        \item[Sens direct]
            Nous supposons que \( A\) est un corps. D'une part tous les éléments non nuls sont inversibles, c'est-à-dire \( A^*\subset U(A)\).
            
            Pour l'inclusion inverse, nous montrons qu'une élément inversible ne peut pas être nul. Cela n'est autre que le lemme~\ref{LEMooVUSMooWisQpD} couplé à la proposition~\ref{PROPooNCCGooXjVyVt} : \( a\cdot 0=0\neq 1\) pour tout \( a\).
        \item[Sens inverse]
            Si \( U(A)=A^*\), nous avons immédiatement que tous les éléments non nuls sont inversibles et donc que \( A\) est un corps.
    \end{subproof}
\end{proof}

\begin{lemma}       \label{LemAnnCorpsnonInterdivzer}
    Un corps non nul est un anneau intègre\footnote{Définition \ref{DEFooTAOPooWDPYmd}.}.
\end{lemma}

\begin{proof}
    En effet si \( a\) est un diviseur de zéro, alors \( ax=0\) pour un certain \( x\neq 0\). Si \( a\) était inversible, nous aurions \( x=a^{-1}ax=0\), ce qui est impossible.
\end{proof}
Conséquence : dans un corps nous avons toujours la règle du produit nul, et l'élément nul n'est jamais inversible.

\begin{definition}[Morphisme de corps]
    Un corps étant un anneau sans plus de structure, un \defe{morphisme de corps}{morphisme!de corps}\index{isomorphisme!de corps} n'est qu'un morphisme des anneaux.
\end{definition}

Le lemme suivant montre que définir un morphisme de corps comme étant simplement un morphisme des anneaux est une bonne idée.
\begin{lemma}       \label{LEMooWBOPooZnsZgQ}
    Si \( \phi\colon \eK\to \eK'\) est un morphisme de corps, alors
    \begin{enumerate}
        \item
            pour tout \( a\in \eK\) nous avons \( \varphi(a^{-1})=\varphi(a)^{-1}\);
        \item
            le morphisme \( \varphi\) est injectif.
    \end{enumerate}
\end{lemma}

\begin{proof}
    Vu que \( \varphi(1)=1\), nous avons aussi
    \begin{equation}
        1=\varphi(aa^{-1})=\varphi(a)\varphi(a^{-1}).
    \end{equation}
    Donc, par unicité de l'inverse\footnote{Lemme~\ref{LEMooECDMooCkWxXf}\,\ref{ITEMooOIWTooYqmMPP}.}, \( \varphi(a^{-1})=\varphi(a)^{-1}\).

    Pour l'injectivité nous supposons \( \varphi(a)=\varphi(b)\). Étant donné que \( \eK'\) est un corps, nous pouvons multiplier par \( \varphi(b)^{-1}\) :
    \begin{equation}
        \varphi(a)\varphi(b)^{-1}=1.
    \end{equation}
    En utilisant le premier point nous avons \( 1=\varphi(a)\varphi(b^{-1})\), puis le morphisme d'anneaux : \( 1=\varphi(ab^{-1})\), et encore le morphisme d'anneaux nous permet de déduire \( ab^{-1}=1\) et donc \(a=b\).
\end{proof}

%---------------------------------------------------------------------------------------------------------------------------
\subsection{Corps des fractions}
%---------------------------------------------------------------------------------------------------------------------------

\begin{definition}[\cite{ooGSDHooLgtHCb}]       \label{DEFooGJYXooOiJQvP}
    Soit un anneau commutatif et intègre\footnote{Définition~\ref{DEFooTAOPooWDPYmd}.} \( A\) et \( E=A\times A\setminus\{ 0 \}\). Nous y définissons les deux opérations suivantes :
    \begin{enumerate}
        \item
            \( (a,b)+(c,d)=(ad+cb,bd)\);
        \item
            \( (a,b)(c,d)=(ac,bd)\).
    \end{enumerate}
    Et aussi la relation d'équivalence \( (a,b)\sim(c,d)\) si et seulement si \( ad=bc\).

    Le \defe{corps des fractions}{corps!des fractions} de \( A\) est le quotient
    \begin{equation}
        \Frac(A)=\big( A\times A\setminus\{ 0 \} \big)/\sim.
    \end{equation}
    Nous notons \( a/b\) la classe de \( (a,b)\).

    Les éléments de \( \Frac(A)\) sont des \defe{fractions rationnelles}{fractions!rationnelles} de \( A\).
\end{definition}
Le fait que \( A\) soit intègre est important pour être certain que \( bd\neq 0\) sous l'hypothèse que \( b,d\neq 0\).

La proposition suivante dit que \( \Frac(A)\) est le plus petit corps contenant \( A\).

\begin{proposition}[\cite{MonCerveau}]      \label{PROPooGSHDooJOnDsp}
    Si \( \eL\) est un corps contenant \( A\) en tant que sous-anneau\footnote{Bien entendu, nous pouvons trouver plein de corps contenant \( A\) en tant que sous-ensemble sans pour autant étendre \( A\) en tant qu'anneau; il suffit d'y mettre une loi de composition farfelue.}, alors il existe un morphisme de corps injectif \( \epsilon\colon \Frac(A)\to \eL\).
\end{proposition}

\begin{proof}
    Il suffit de vérifier que la formule
    \begin{equation}        \label{EQooCVOBooQEQJBM}
        \epsilon(a/b)=ab^{-1}
    \end{equation}
    vérifie toutes les conditions. Notons que dans le membre de droite de \eqref{EQooCVOBooQEQJBM}, l'inverse et le produit sont calculés dans \( \eL\).

    Le fait que \( \epsilon\) soit bien défini provient du fait que \( A\) soit commutatif :
    \begin{equation}
        \epsilon(ax/bx)=(ax)(bx)^{-1}=ab^{-1}xx^{-1}=ab^{-1}=\epsilon(a/b).
    \end{equation}

    Le fait que \( \epsilon\) soit un morphisme est une vérification de routine, par exemple ceci pour la somme :
    \begin{equation}
        \epsilon\big( a/b+c/d \big)=\epsilon\big( (ad+cb)/bd)\big)=(ad+cb)(bd^{-1})=ab^{-1}+cd^{-1},
    \end{equation}
    tandis que
    \begin{equation}
        \epsilon(a,b)+\epsilon(c,d)=ab^{-1}+cd^{-1},
    \end{equation}
    qui est égal (il faut aussi vérifier pour le produit).

    Enfin \( \epsilon\) est injective parce que si \( \epsilon(a/b)=\epsilon(c/d)\), alors \( ab^{-1}=cd^{-1}\), d'où il est facilement vu que \( ad=cb\), c'est-à-dire \( a/b=c/d\).
\end{proof}

Notons que si \( A\) est un anneau qui n'est pas un corps, le corps \( \Frac(A)\) existe, mais si \( R\in\Frac(A)\), il n'a pas de sens de vouloir calculer \( R(\alpha)\) pour \( \alpha\in A\).

\begin{definition}[Évaluation d'une fraction rationnelle]       \label{DEFooLBIWooCPCaSY}
    Soit un corps \( \eK\) contenant l'anneau \( A\). Si \( R=P/Q\in \Frac(A)\) et si \( \alpha\in \eK\) nous définissons
    \begin{equation}
        R(\alpha)=(P/Q)(\alpha)=P(\alpha)Q^{-1}(\alpha).
    \end{equation}
    Dans cette formule, les polynômes, l'inverse et le produit sont calculés dans \( \eK\) et non dans \( A\).
\end{definition}

\begin{theoremDef}     \label{ThogbhWgo}
    Soit \( \eA\) un anneau commutatif intègre.

    \begin{enumerate}
        \item
    Il existe un corps commutatif \( \eK\) et un morphisme d'anneaux injectif \( \epsilon\colon \eA\to \eK\) tels que pour tout \( \lambda\in\eK\), il existe \( (a,b)\in \eA\times \eA^*\) tels que
    \begin{equation}
        \lambda=\epsilon(a)\big( \epsilon(b) \big)^{-1}
    \end{equation}
\item
    Si \( (\eK',\epsilon')\) est un autre couple qui vérifie la propriété, les corps \( \eK\) et \( \eK'\) sont isomorphes.
    \end{enumerate}

    Le corps \( \eK\) associé à l'anneau \( \eA\) est le \defe{corps des fractions}{corps!des fractions}\index{fractions (corps)} de \( \eA\), et sera noté \( \Frac(\eA)\).\nomenclature[A]{\( \Frac(\eA)\)}{Le corps des fractions de l'anneau \( \eA\)}
\end{theoremDef}

\begin{lemma}[Simplification de fraction]
    L'application \( \eA\times \eA^*\to \eK\) donnée par \( (a,b)\mapsto \epsilon(a)\big( \epsilon(b) \big)^{-1}\) envoie \( (xa,xb)\) sur le même que \( (a,b)\).
\end{lemma}

La proposition suivante montre encore que le corps des fractions est le plus petit corps que l'on puisse imaginer à partir d'un anneau.
\begin{proposition}
    Soit un anneau \( A\). Tout corps contenant un sous-anneau isomorphe à \( A\) contient un sous-corps isomorphe à \( \Frac(A)\).
\end{proposition}

\begin{proof}
    Soit un corps \( \eK\) contenant un sous-anneau \( A'\) isomorphe à \( A\). Nous considérons la partie suivante de \( \eK\) :
    \begin{equation}
        S=\{ ab^{-1}\tq a,b\in A \}.
    \end{equation}
    Ensuite nous montrons que
    \begin{equation}
        \begin{aligned}
            \varphi\colon S&\to \Frac(A) \\
            ab^{-1}&\mapsto a/b
        \end{aligned}
    \end{equation}
    est un isomorphisme de corps.

    \begin{subproof}
        \item[Bien définie]

            Si \( ab^{-1}=xy^{-1}\) alors \( ay=xb\) et donc \( a/b=x/y\) par définition des classes de \( \Frac(A)\).

        \item[Surjectif]

            Tout élément de \( \Frac(A)\) est de la forme \( a/b\) avec \( a,b\in A\). Un tel élément est l'image par \( \varphi\) de \( ab^{-1}\in S\).

        \item[Injectif]

            Si \( \varphi(ab^{-1})=\varphi(xy^{-1})\) alors \( a/b=x/y\), et par définition des classes nous avons \( ay=bx\) qui donne immédiatement \( ab^{-1}=xy^{-1}\).
    \end{subproof}
\end{proof}

%---------------------------------------------------------------------------------------------------------------------------
\subsection{Suites de Cauchy dans un corps totalement ordonné}
%---------------------------------------------------------------------------------------------------------------------------

\begin{definition}      \label{DefKCGBooLRNdJf}
    Ordre et choses reliées dans un corps.
    \begin{enumerate}
        \item \label{ITEMooOOOVooJWwIQr}
            Un corps \( \eK\) est \defe{totalement ordonné}{ordre!dans un corps}\index{corps!ordonné} s'il existe une relation d'ordre total\footnote{Définition~\ref{DEFooVGYQooUhUZGr}.} tel que
            \begin{enumerate}
                \item
                    \( x\leq y\) implique \( x+z\leq y+z\) pour tout \( x,y,z\in \eK\)
                \item   \label{CONDooBYYDooElXgPO}
                    \( x\geq 0\) et \( y\geq 0\) implique \( xy\geq 0\).
            \end{enumerate}
        \item       \label{ItemooWUGSooRSRvYC}
            Si \( \eK\) est un corps totalement ordonné, nous y définissons la valeur absolue par
            \begin{equation}
                | x |=\begin{cases}
                    x    &   \text{si }x\geq 0\\
                    -x    &    \text{si } x\leq 0.
                \end{cases}
            \end{equation}
        \item       \label{ItemVXOZooTYpcYN}
    La suite \( (x_n)\) dans le corps totalement ordonné \( \eK\) est \defe{de Cauchy}{suite!de Cauchy!dans un corps} si pour tout \( \epsilon\in \eK^+\), il existe \( N\in \eN\) tel que si \( p,q\geq N\) alors \( | x_p-x_q |\leq \epsilon\).
\item       \label{ITEMooDERQooLmJwFR}
    La suite \( (x_n)\) dans le corps totalement ordonné \( \eK\) est \defe{convergente}{convergence!suite!dans un corps} s'il existe \( q\in \eK\) tel que pour tout \( \epsilon\in \eK^+\), il existe \( N\) tel que si \( k\geq N\) alors \( | x_k-q |\leq \epsilon\).
\item   \label{ItemooDZQKooPsqeRf}
            Un corps \( \eK\) est \defe{archimédien}{corps!archimédien}\index{archimédien} s'il est totalement ordonné et si pour tout \( x,y\in \eK\) avec \( x>0\), il existe \( n\in \eN\) tel que \( nx\geq y\).
        \item       \label{ITEMooKZZYooDaidGU}
            Un corps totalement ordonné est \defe{complet}{corps!complet}\index{complet!corps} si toute suite de Cauchy y est convergente.
        \item       \label{ITEMooMWASooEzhVyh}
            Si \( a,\epsilon\in \eK\) alors nous définissons la \defe{boule ouverte}{boule dans un corps} de centre \( a\) et de rayon \( \epsilon\) par
            \begin{equation}
                B(a,\epsilon)=\{ x\in \eK\tq | a-x |<\epsilon \},
            \end{equation}
            et la \defe{boule fermée}{boule dans un corps} par
            \begin{equation}
                \overline{ B(a,\epsilon) }=\{ x\in \eK\tq | a-x |\leq \epsilon \}.
            \end{equation}

    \end{enumerate}
\end{definition}

\begin{remark}
    Nous étudierons plus tard la notion de caractéristique d'un anneau\footnote{définition~\ref{LEMDEFooVEWZooUrPaDw}} ou d'un corps. Tout anneau totalement ordonné est nécessairement de caractéristique nulle, lemme~\ref{LEMooJQIKooQgukqn}.
\end{remark}

\begin{remark}
    En mettant côte à côte les points~\ref{ITEMooDERQooLmJwFR} et~\ref{ITEMooMWASooEzhVyh} nous pouvons dire que \( (x_k)\) converge vers \( q\) si et seulement si pour tout \( \epsilon>0\), nous avons \( x_k\in B(q,\epsilon)\) à partir d'un certain indice \( N\).

    Ces boules prendront une nouvelle force avec le super-théorème~\ref{ThoORdLYUu}.
\end{remark}

Parmi ces définitions, celles de suite convergente, de Cauchy et de corps complet seront utilisées dans le cas de \( \eQ\) (et de \( \eR\) pour la complétude). Elles seront prouvées être équivalentes aux définitions topologiques dans le cas particulier de \( \eR\) et \( \eQ\) lorsque la topologie métrique sera définie. Dans cet état d'esprit nous n'allons pas démontrer tout de suite que \( \eR\) est un corps complet. Nous allons directement démontrer que c'est un espace topologique complet.

\begin{lemma}[Propriétés de la valeur absolue]  \label{LemooANTJooYxQZDw}
    Soit \( \eK\) un corps totalement ordonné. Si \( x,y\in \eK\) alors
    \begin{enumerate}
        \item       \label{ItemooNVDIooSuiSoB}
            Si \( x\geq 0\) alors \( -x\leq 0\).
        \item
            \( | x |\geq 0\)
        \item
            \( | x |=0\) si et seulement si \( x=0\)
        \item\label{ItemooOMKNooRlanvk}
            \( | x+y |\leq | x |+| y |\).
    \end{enumerate}
\end{lemma}

\begin{proof}
    Point par point
    \begin{enumerate}
        \item
            Nous partons de \( x\geq 0\) et nous ajoutons \( -x\) des deux côtés en profitant de la définition d'un corps totalement ordonné : \( x-x\geq -x\) et donc \( 0\geq-x\), c'est-à-dire \( -x\leq 0\).
        \item
            Si \( x\geq 0\) alors c'est vrai. Sinon, \( x\leq 0\) et \( | x |=-x\geq 0\) par le point~\ref{ItemooNVDIooSuiSoB}.
        \item
            Si \( x=0\) alors \( x=-x\) et \( | x |=0\). Au contraire si \(x\neq 0\) alors \( -x\neq 0\) et que \( x\) soit positif ou négatif, nous aurons toujours \( \pm x\neq 0\).
        \item
            Nous supposons que \( x\leq y\) et nous distinguons divers cas suivant la positivité de \( x\) et \( y\).
            \begin{enumerate}
                \item
                    Si \( x,y\geq 0\). Dans ce cas, \( x+y\geq y\geq 0\), donc \( | x+y |=x+y=| x |+| y |\).
                \item
                    Si \( x,y\leq 0\). Dans ce cas, \( x+y\leq 0\) et nous avons \( | x+y |=-x-y=| x |+| y |\).
                \item
                    Si \( x\leq 0\) et \( y\geq 0\). Nous subdivisons encore en deux cas suivant que \( x+y\) est positif ou négatif. Si \( x+y\geq 0\), alors nous écrivons successivement
                    \begin{subequations}
                        \begin{align}
                            x&\leq 0\\
                            x+y&\leq y\leq y+| x |=| x |+| y |
                        \end{align}
                    \end{subequations}
                    et donc \( | x+y |=x+y\leq | x |+| y |\).

                    Nous supposons à présent que \( x\leq 0\), \( y\geq 0\) et \( x+y\leq 0\). Dans ce cas il suffit d'écrire \( | x+y |=| (-x)+(-y) |\) pour retomber dans le cas précédent à inversion près de \( x\) et \( y\).
            \end{enumerate}
    \end{enumerate}
\end{proof}

\begin{remark}      \label{RemooJCAUooKkuglX}
    La partie~\ref{ItemooOMKNooRlanvk} est très importante parce que c'est elle qui fera presque toutes les majorations dont nous aurons besoin en analyse. En effet elle donne l'inégalité triangulaire de la façon suivante : si \( x,y,z\in \eK\) nous avons
    \begin{equation}
        | x-y |= |  (x-z)+(z-y) |\leq | x-z |+| z-y |.
    \end{equation}
\end{remark}

\begin{lemma}[À propos de boules]
    Soient un corps totalement ordonné \( \eK\) et des éléments \( x,y,\epsilon\in \eK\).
    \begin{enumerate}
        \item       \label{ITEMooXJGVooSebiip}
            Nous avons \( y\in B(x,\epsilon)\) si et seulement si \( x-\epsilon<y<x+\epsilon\).
        \item       \label{ITEMooRUBBooRayiMs}
            Si \( y\in  \overline{ B(x,\epsilon) }  \) alors \( y\in B(x,\epsilon')\) pour tout \( \epsilon'>\epsilon\).
    \end{enumerate}
\end{lemma}

\begin{proof}
    Pour rappel,
    \begin{equation}
        | x-y |=\begin{cases}
               x-y    &     \text{si } x-y\geq 0 \\
                    y-x    &    \text{si } x-y\leq 0.
               \end{cases}
    \end{equation}
    Nous pouvons maintenant démontrer nos choses.
    \begin{subproof}
        \item[\ref{ITEMooXJGVooSebiip}]
            Des inégalités \( x-\epsilon<y\) et \( y<x+\epsilon\) nous tirons \( x-y<\epsilon\) et \( y-x<\epsilon\). Donc quel que soit le signe de \( x-y\) nous avons toujours \( | x-y |<\epsilon\).

            Dans l'autre sens, nous supposons que \( | x-y |<\epsilon\).

            Si \( x-y\geq 0\) alors l'hypothèse signifie \( x-y<\epsilon\), ce qui donne \( y>x-\epsilon\). Mais l'inégalité \( x-y\geq 0\) donne également \( x\geq y\) et donc \( x+\epsilon\geq y+\epsilon>y\). Notez le jeu de l'inégalité non stricte qui se change en inégalité stricte.

            Si \( x-y\leq 0\) nous pouvons faire le même raisonnement.

        \item[\ref{ITEMooRUBBooRayiMs}]

            C'est immédiat parce que
            \begin{equation}
                | x-y |\leq \epsilon<\epsilon'.
            \end{equation}
    \end{subproof}
\end{proof}

\begin{proposition}     \label{PROPooTFVOooFoSHPg}
    Toute suite convergente dans un corps totalement ordonné est de Cauchy.
\end{proposition}

\begin{proof}
    Soit un corps totalement ordonné \( \eK\) et une suite \( x_n\stackrel{\eK}{\longrightarrow}x\). Soit \( \epsilon>0\). Il est important de se rendre compte que \( \epsilon\in \eK\) et que l'inégalité est au sens de l'ordre dans \( \eK\); en particulier ce n'est pas \( \epsilon\in \eR\) ni \( \epsilon\in \eQ\). D'ailleurs nous n'avons encore pas défini ni \( \eR\) ni \( \eQ\).

    Vu que \( (x_n)\) converge vers \( x\), il existe \( N\in \eN\) tel que pour tout \( k>N\),
    \begin{equation}
        | x_k-x |<\epsilon.
    \end{equation}

    Soient \( p,q>N\). Alors en utilisant la majoration du lemme~\ref{LemooANTJooYxQZDw}\ref{ItemooOMKNooRlanvk},
    \begin{equation}
        | x_p-x_q |=\big| (x_p-x)+(x-x_q) \big|\leq | x_p-x |+| x-x_q |\leq 2\epsilon.
    \end{equation}
    Donc la suite \( (x_n)\) est de Cauchy.
\end{proof}

%+++++++++++++++++++++++++++++++++++++++++++++++++++++++++++++++++++++++++++++++++++++++++++++++++++++++++++++++++++++++++++
\section{Les rationnels}
%+++++++++++++++++++++++++++++++++++++++++++++++++++++++++++++++++++++++++++++++++++++++++++++++++++++++++++++++++++++++++++

Une construction très explicite est faite dans \cite{RWWJooJdjxEK}. Ici nous allons prendre plus court :
\begin{definition}
    Le corps des fractions de \( \eZ\) (définition~\ref{DEFooGJYXooOiJQvP}) est noté \( \eQ\) et ses éléments sont les \defe{rationnels}{rationnels}.
\end{definition}

Les résultats énoncés ici sont utilisés plus bas et servent de guide à un éventuel contributeur qui voudrait écrire une partie dédiée à \( \eQ\) et ses propriétés de base\quext{Par exemple, définir une relation d'ordre sur \( \eQ\) et expliciter l'inclusion de \( \eZ\) dans \( \eQ\).}. Nous espérons que des preuves se trouvent dans \cite{RWWJooJdjxEK}. En tout cas, le lecteur est invité à ne rien prendre comme évident.

\begin{lemma} \label{LEMooEBTIooGMoHsj}
    Tout rationnel est majoré par un naturel.
\end{lemma}

\begin{proposition}     \label{PROPooDHIAooZysvNs}
    L'ensemble des rationnels est dénombrable.
\end{proposition}

\begin{proposition}     \label{PROPooBTCCooVVvaeL}
    Si \( q<1\), alors \( qx<x\) pour tout \( x\in \eQ^+\).
\end{proposition}

\begin{proposition}     \label{PROPooMXGPooDUkOuv}
    Le corps \( \eQ\) est archimédien\footnote{Définition~\ref{DefKCGBooLRNdJf}\ref{ItemooDZQKooPsqeRf}.}.
\end{proposition}

%---------------------------------------------------------------------------------------------------------------------------
\subsection{Suites de Cauchy dans les rationnels}
%---------------------------------------------------------------------------------------------------------------------------

\begin{proposition}[\cite{RWWJooJdjxEK}]        \label{PropFFDJooAapQlP}
    Principales propriétés des suites de Cauchy dans \( \eQ\).
    \begin{enumerate}
        \item       \label{ItemRKCIooJguHdji}
            Toute suite convergente est de Cauchy\footnote{Et non la réciproque, qui sera justement la grande innovation des nombres réels.}.
        \item       \label{ItemRKCIooJguHdjii}
            Toute suite de Cauchy est bornée.
        \item       \label{ItemRKCIooJguHdjiii}
            Si \( x_n\to 0\) et si \( (y_n)\) est bornée, alors \( x_ny_n\to 0\)
        \item
            Si \( (x_n)\) et \( (y_n)\) sont de Cauchy alors \( (x_n+y_n)\), \( (x_n-y_n)\) et \( (x_ny_n)\) sont également de Cauchy.
        \item       \label{ITEMooIAFSooAIUpAN}
            Si il existe \( a,b\in \eQ\) tels que \( x_n\to a \) et \( y_n\to b \) alors \( x_n+y_n\to a+b\), \( x_n-y_n\to a-b\) et \(  x_ny_n\to ab  \).
        \item   \label{ItemRKCIooJguHdjvi}
            Soit \( (x_n)\) une suite de Cauchy qui ne converge pas vers zéro. Alors il existe \( n_0\) tel que la suite \( \left( \frac{1}{ x_n } \right)_{n\geq n_0}\) soit de Cauchy.
    \end{enumerate}
\end{proposition}

\begin{proof}
    Point par point.
    \begin{enumerate}
        \item

            C'est la proposition~\ref{PROPooTFVOooFoSHPg}.

        \item
            Soit \( (x_n)\) une suite de Cauchy dans \( \eQ\). Avec \( \epsilon=1\) dans la définition, si \( q>N_1\), nous avons
            \begin{equation}
                | x_q-x_{N_1} |\leq 1.
            \end{equation}
            Et donc pour tout \( q\) plus grand que \( N_1\), \( x_N-1\leq x_q\leq x_N+1\), ou encore, pour tout \( n\) :
            \begin{equation}
                | x_n |\leq\max\{ | x_1 |,| x_2 |,\ldots,| x_N |,| x_N+1 | \}.
            \end{equation}
            La suite est donc bornée.
        \item
            Soit \(\epsilon>0\). Les hypothèses disent qu'il existe un \( N\) tel que \( | x_n |\leq \epsilon\) dès que \( n\geq N\). Et il existe aussi \( M\geq 0\) tel que \( | y_n |\leq M\) pour tout \( n\). Du coup, lorsque \( n\geq N\) nous avons \( | x_ny_n |\leq M\epsilon\).
        \item
            En ce qui concerne la somme,
            \begin{equation}        \label{EqDCNBooAzrrBi}
                | x_p+y_p-x_q-y_q |\leq | x_p-x_q |+| y_p-y_q |.
            \end{equation}
            Soit \( N_1\) tel que si \( p,q\geq N_1\) alors \( | x_p-x_q |\leq \epsilon\) et \( N_2\) de même pour la suite \( (y_n)\). En prenant \( N=\max\{ N_1,N_2 \}\), la somme \eqref{EqDCNBooAzrrBi} est plus petite que \( 2\epsilon\) dès que \( p,q\geq N\).

            Passons à la démonstration du fait que le produit de deux suites de Cauchy est de Cauchy. Les suites \( (x_n)\) et \( (y_n)\) sont bornées et quitte à prendre le maximum, nous disons qu'elles sont toutes les deux bornées par le nombre \( M\) : pour tout \( n\) nous avons \( | x_n |\leq M\) et \( | y_n |\leq M\). Nous avons :
            \begin{equation}
                | x_py_p-x_qy_q |\leq | x_py_p-x_qy_p |+| x_qy_p-x_qy_q |\leq | y_p | |x_p-x_q |+| x_q | |y_p-y_q |.
            \end{equation}
            Vu que \( (x_n)\) et \( (y_n)\) sont de Cauchy, si \( p\) et \( q\) sont assez grands, les deux différences sont majorées par \( \epsilon\) et nous avons
            \begin{equation}
                | x_py_p-x_qy_q |\leq M\epsilon+M\epsilon=2M\epsilon,
            \end{equation}
            ce qui prouve que \( (x_ny_n)\) est de Cauchy.
        \item
            En ce qui concerne la somme, nous pouvons tout de suite calculer
            \begin{equation}
                | x_n+y_n-(a+b) |\leq | x_n-a |+| y_n-b |.
            \end{equation}
            Il existe une valeur de \( n\) à partir de laquelle le premier terme est plus petit que \( \epsilon\) et une à partir de laquelle le second terme est plus petit que \( \epsilon\). En prenant le maximum des deux, la somme est plus petite que \( 2\epsilon\).

            En ce qui concerne le produit,
            \begin{equation}
                | x_ny_n-ab |\leq | x_ny_n-ay_n |+| ay_n-ab |\leq | y_n || x_n-a |+| a || y_n-b |.
            \end{equation}
            Les suites \( | x_n-a |\) et \( | y_n-b |\) convergent vers zéro; la suite \( (y_n)\) est bornée parce que convergente (combinaison des points~\ref{ItemRKCIooJguHdji} et~\ref{ItemRKCIooJguHdjii})  et \( a\) (la suite constante) est également bornée. Donc par le point~\ref{ItemRKCIooJguHdjiii}, nous avons
            \begin{equation}
                y_n| x_n-a |+a| y_n-b |\to 0.
            \end{equation}
            Au passage nous avons également utilisé la propriété de la somme que nous venons de démontrer.
        \item Soit \( (x_n)\) une suite de Cauchy dans \( \eQ\) ne convergeant pas vers zéro : il existe \( \alpha>0\) tel que pour tout \( N\in \eN\), il existe \( n\geq N\) tel que \( | x_n |>\alpha\). Mais notre suite est de Cauchy, donc il existe \( n_0\in \eN\) tel que si \( p,q\geq n_0\) alors
            \begin{equation}
                | x_p-x_q |\leq \frac{ \alpha }{2}.
            \end{equation}
            En fixant \( N = n_0\), on obtient un naturel \( n\geq n_0\) tel que \( | x_n |\geq \alpha\). De plus, comme la suite est de Cauchy, si \( p>n\) nous avons aussi \( | x_n-x_p |\leq \frac{ \alpha }{2}\). Cela implique \( | x_p |\geq \frac{ \alpha }{2}\) et en particulier \( x_p\neq 0\).

            Nous venons de prouver que la suite ne s'annule plus à partir de l'indice \( n\), et même que \( | x_k |\geq\alpha/2\) pour tout \( k\geq n\). La suite \( (1/x_k)_{k\geq n}\) est donc bien définie.

            Soit \( \epsilon>0\). Soit \( n_0\) tel que \( | x_p-x_q |<\epsilon\) pour tout \( p,q>n_0\). Soit \( K\) plus grand que \( n_0\) et que \( n\). En prenant \( p,q\geq K\), nous avons \( |  x_p|>\frac{ \alpha }{2}\) et \( | x_q |>\frac{ \alpha }{2}\). Nous en déduisons que
            \begin{equation}
                \left| \frac{1}{ x_p }-\frac{1}{ x_q } \right| \leq \frac{ | x_q-x_p | }{ | x_px_q | }\leq \frac{ 4 }{ \alpha^2 }| x_q-x_p |\leq \frac{ 4 }{ \alpha^2 }\epsilon.
            \end{equation}
            Donc \( \left( \frac{1}{ x_n } \right)\) est de Cauchy.
    \end{enumerate}
\end{proof}

%---------------------------------------------------------------------------------------------------------------------------
\subsection{Insuffisance des rationnels}
%---------------------------------------------------------------------------------------------------------------------------

Nous allons voir qu'il n'existe pas de nombres rationnels \( x\) tels que \( x^2=2\), mais que pourtant il existe une infinité de suites de rationnels \( (x_n)\) tels que \(  x_n^2\to 2  \).

\begin{lemma}       \label{LemJPIUooWFHaFM}
    Un entier \( x\) est pair si et seulement si l'entier \( x^2\) est pair.
\end{lemma}

\begin{proof}
    Si \( x\) est un nombre pair, alors il existe un entier \( a\) tel que \( x=2a\) alors \( x^2=4a^2\) est pair.

    Inversement, si \( x\) est impair alors il existe un entier \( a\) tel que \( x=2a+1\) et alors \( x^2=4a^2+4a+1=2(2a^2+2a)+1\) est impair.
\end{proof}

Le théorème~\ref{THOooYXJIooWcbnbm} nous dira que tous les \( \sqrt{n}\) sont irrationnels dès que \( n\) n'est pas un carré parfait. Voici déjà le résultat pour \( n=2\). Le fait que \( \sqrt{ 2 }\) existe dans \( \eR\) sera la proposition \ref{PROPooUHKFooVKmpte}.
\begin{proposition}[Irrationalité de \( \sqrt{2}\)]     \label{PropooRJMSooPrdeJb}
    Il n'existe pas de fractions d'entiers dont le carré soit égal à \( 2\).
\end{proposition}
\index{irrationalité!\( \sqrt{2}\)}

\begin{proof}
    Nous supposons que la fraction d'entiers \( a/b\) est telle que \( a^2/b^2=2\), et nous allons construire une suite d'entiers strictement décroissante et strictement positive, ce qui est impossible.

    Grâce au lemme~\ref{LemJPIUooWFHaFM} nous avons successivement les affirmations suivantes :
    \begin{itemize}
        \item
        \(\frac{ a^2 }{ b^2 }=2 \)  avec \( a\neq 0\) et \( b\neq 0\).
    \item
        \( a^2=2b^2\), donc \( a^2\) est pair.
    \item
        \( a\) est alors pair et \( a^2\) est divisible par \( 4\). Soit \( a^2=4k\).
    \item
        \( 4k/b^2=2\), donc \( 4k=2b^2\), donc \( b^2=2k\) et \( b^2\) est pair.
    \item
        Nous déduisons que \( b\) est pair.
    \end{itemize}
    La fraction \( \frac{ a/2 }{ b/2 }\) est alors une nouvelle fraction d'entiers dont le carré vaut $2$. En procédant de la même façon, en remplaçant \( a\) par \( a/2\) et \( b\) par \( b/2\), on obtient que la fraction d'entiers \( \frac{ a/4 }{ b/4 }\) a la même propriété.

    En particulier, tous les nombres de la forme \( a/2^n\) sont des entiers.  Ils forment une suite strictement décroissante d'entiers strictement positifs. Impossible, me diriez-vous ? Et vous auriez bien raison : toute partie non vide de \( \eN\) admet un plus petit élément\footnote{Voir \cite{RWWJooJdjxEK}, et attention : ce n'est pas tout à fait évident.}. Il n'y a donc pas de fractions d'entiers dont le carré vaut \( 2\).
\end{proof}

\begin{lemma}[Série géométrique, voir aussi l'exemple~\ref{ExZMhWtJS}]      \label{LEMooOTVUooImvusn}
    Si \( q\in \eQ\) et \( p\in \eN\) nous avons
    \begin{equation}
        \sum_{k=0}^pq^k=\frac{ 1-q^{p+1} }{ 1-q }.
    \end{equation}
\end{lemma}

\begin{proof}
    En posant \( S_p=1+q+q^2+\cdots +q^{p}\), nous avons $S_p-qS_p=1-q^{p+1}$ et donc
    \begin{equation}
        S_p=\sum_{k=0}^pq^k=\frac{ 1-q^{p+1} }{ 1-q }.
    \end{equation}
\end{proof}

\begin{proposition}
    La suite donnée par
    \begin{equation}
        x_n=1+\frac{ 1 }{ 1! }+\cdots +\frac{1}{ n! }
    \end{equation}
    est de Cauchy et ne converge pas dans \( \eQ\).
\end{proposition}

\begin{proof}
    Si \( p>q>0\) nous avons
    \begin{subequations}
        \begin{align}
            x_p-x_q&=\sum_{k=q+1}^p\frac{1}{ k! }\\
            &\leq \sum_{k=q+1}^p\frac{1}{ (q+1)! }\frac{1}{ (q+1)^{k-q-1} }  \label{SUBEQooAXILooEAcpVB}\\
            &\leq \frac{1}{ (q+1)! }\lim_{p\to \infty} \sum_{k=0}^{p}\frac{1}{ (q+1)^k }  \label{SUBEQooNDPTooDSEYEJ}\\
            &=\frac{1}{ (q+1)! }\frac{1}{ 1-\frac{1}{ q+1 } } \label{SUBEQooEMHJooSnCUiK}  \\
            &=\frac{1}{ (q+1)! }\frac{q+1}{q}\\
            &=\frac{1}{ q!q }.
        \end{align}
    \end{subequations}
    Justifications :
    \begin{itemize}
        \item Pour \eqref{SUBEQooAXILooEAcpVB}, il s'agit de remplacer dans \( k!\) tous les facteurs plus grands que \( (q+1)\) par \( q+1\). Cela rend le dénominateur plus petit.
        \item Pour \eqref{SUBEQooNDPTooDSEYEJ}, il y a une inégalité parce que la suite \( p\mapsto \sum_{k=0}^p1/(q+1)^k\) est une suite strictement croissante.

        \item Pour \eqref{SUBEQooEMHJooSnCUiK}, le lemme~\ref{LEMooOTVUooImvusn} donne la valeur de la somme finie. En ce qui concerne la limite, nous avons demandé \( p>q>0\) et donc \( q+1>1\). Dans ce cas la limite fonctionne.
    \end{itemize}

    Cette inégalité une fois établie nous permet de prouver les assertions. La suite \( (x_n) \) est de Cauchy car, pour tout \( \epsilon\in\eQ\) s'écrivant \( \epsilon=\frac{ a }{ b }\) avec \( a,b\in \eN\), en prenant \( p,q>b\), nous avons
    \begin{equation}
        x_p-x_q\leq \frac{1}{ b!b }<\frac{1}{ b }<\frac{ a }{ b }=\epsilon.
    \end{equation}

    Montrons par l'absurde que cette suite ne converge pas dans \( \eQ\). Pour cela, nous supposons que \( \lim_{n\to \infty} x_n=\frac{ a }{b }\in \eQ\). Pour tout \( p>q\) nous avons établi
    \begin{equation}
        0<x_p-x_q<\frac{1}{ qq! }.
    \end{equation}
    Prenons la limite \( p\to \infty\); par stricte croissance de la suite, les inégalités restent strictes :
    \begin{equation}        \label{EqQLCTooOgQOdh}
        0<\frac{ a }{ b }-x_q<\frac{1}{ qq! }.
    \end{equation}
    Si \( n>b\) alors nous pouvons écrire
    \begin{equation}
        \frac{ a }{ b }-x_n=\frac{ \alpha }{ n! }
    \end{equation}
    avec \( \alpha\in \eZ\) parce que le dénominateur commun entre \( \frac{ a }{ b }\) et \( x_n\) est dans \( n!\). En prenant donc \( q>n\) dans \eqref{EqQLCTooOgQOdh} nous pouvons écrire
    \begin{equation}
        0<\frac{ \alpha }{ q! }<\frac{1}{ qq! },
    \end{equation}
    c'est-à-dire \( 0<\alpha<\frac{1}{ q }\), ce qui est impossible pour \( \alpha\in \eZ\).
\end{proof}

\begin{lemma}   \label{LEMooDTXYooKwmlZh}
    Soit \( A>0\) dans \( \eQ\). Il existe un rationnel \( q>0\) tel que \( q^2<A\).
\end{lemma}

\begin{proof}
    Vu que \( \eQ\) est archimédien (proposition \ref{PROPooMXGPooDUkOuv}), il existe \( n\in \eN\) tel que \( 1<nA\). Pour ce \( n\), nous avons
    \begin{equation}
        \left( \frac{1}{ n } \right)^2<\frac{1}{ n }<A.
    \end{equation}
\end{proof}

La proposition suivante donne une suite de rationnels qui convergerait dans \( \eR\) vers \( \sqrt{ A }\) (non encore défini à ce stade). Il est expliqué dans \cite{BIBooMPXEooQLKhku} que la suite est motivée par la méthode de Newton \ref{THOooDOVSooWsAFkx}.
\begin{proposition}[\cite{BIBooMPXEooQLKhku}]       \label{PROPooSTQXooHlIGVf}
    Soient \( A>0\) dans \( \eQ\) et \( x_0\in \eQ\). La suite \( (x_k)\) définie par
    \begin{equation}
        x_{k+1}=\frac{ 1 }{2}\left( x_k+\frac{ A }{ x_k } \right)
    \end{equation}
    a les propriétés suivantes :
    \begin{enumerate}
        \item
            La suite \( y_k=x_k^2 \) converge dans \( \eQ\) vers \( A\).
        \item
            La suite \( (x_k)\) est de Cauchy dans \( \eQ\).
        \item
            La suite \( (x_k)\) ne converge pas dans \( \eQ\) dans le cas de \( A=2\).
    \end{enumerate}
\end{proposition}

\begin{proof}
    En plusieurs points.
    \begin{subproof}
        \item[La suite \( s_k\)]
            En posant \( y_k=x_k^2\) nous calculons que
            \begin{equation}
                y_{k+1}-A=\frac{ (y_k-A)^2 }{ 4y_k }.
            \end{equation}
            Autrement dit, la suite \( s_k=y_k-A\) vérifie
            \begin{equation}
                s_{k+1}=\frac{ s_k^2 }{ 4(A+s_k) }.
            \end{equation}
            Quelle que soit la valeur de \( s_0=x_0^2-A\), nous avons
            \begin{equation}
                s_1=\frac{ s_0^2 }{ 4(A+s_1) }=\frac{ (x_0^2-A)^2 }{ 4(A+x_0^2-A) }=\frac{ (x_0^2-A)^2 }{ 4x_0^2 }>0.
            \end{equation}
            Donc à partir de \( s_1\), tous les éléments sont positifs. Vu que \( A>0\) nous avons aussi
            \begin{equation}
                s_{k+1}<\frac{ s_k^2 }{ 4s_k }=\frac{ s_k }{ 4 }
            \end{equation}
            et donc \( s_k<s_0/4^k\). Donc \( s_k\to 0\).
        \item[La suite \( (y_k)\)]
            Nous venons de prouver que si \( y_k=A+s_k\), alors \( s_k\to 0\). Autrement dit, la suite \( y_k\) converge vers \( A\) dans \( \eQ\). 

            La suite \( (y_k)\) est donc de Cauchy par la proposition \ref{PropFFDJooAapQlP}\ref{ItemRKCIooJguHdji}.
        \item[La suite \( (x_k)\) est de Cauchy]
            Soit \( \epsilon>0\) dans \( \eQ\). Vu que \( (y_k)\) est de Cauchy, il existe \( n_0\in \eN\) tel que 
            \begin{equation}
                | x^2_r-x_s^2 |<\epsilon
            \end{equation}
            pour tout \( r,s\geq n_0\).

            Soit par ailleurs \( q\in \eQ\) tel que \( q^2<A\), assuré par le lemme \ref{LEMooDTXYooKwmlZh}. Quitte à prendre \( n_0\) plus grand, nous supposons que \( x_r,x_s>q\), et en particulier que \( x_r+x_s\neq 0\). Cela permet d'écrire d'abord
            \begin{equation}
                x_r^2-x_s^2=(x_r+x_s)(x_r-x_s)
            \end{equation}
            et ensuite de prendre la valeur absolue et de diviser par \( | x_r+x_s |\) :
            \begin{equation}
                | x_r-x_s |=\frac{ | x_r^2-x_s^2 | }{ | x_r+x_s | }<\frac{ \epsilon }{ 2q }.
            \end{equation}
            Donc \( (x_k)\) est une suite de Cauchy.
        \item[Pas de convergence pour \( A=2\)]
            Supposons que \( x_k\to a\in \eQ\). Dans ce cas nous aurions \( x_k^2\to a^2=A=2\) (proposition~\ref{PropFFDJooAapQlP}\ref{ITEMooIAFSooAIUpAN}). Mais nous savons par la proposition~\ref{PropooRJMSooPrdeJb} que \( a^2=2\) est impossible dans \( \eQ\).
    \end{subproof}
\end{proof}

Notons que cette proposition ne présume en rien de l'existence ou de la non-existence dans \( \eQ\) d'un élément qui pourrait décemment être nommé \( \sqrt{ A }\). Il se fait que le théorème \ref{THOooYXJIooWcbnbm} dira que \( \sqrt{ n }\) est soit entier soit irrationnel.

\begin{normaltext}
    Un petit programme en python explorer la suite de la proposition \ref{PROPooSTQXooHlIGVf}.
    \lstinputlisting{tex/frido/codeSnip_4.py}
\end{normaltext}

%+++++++++++++++++++++++++++++++++++++++++++++++++++++++++++++++++++++++++++++++++++++++++++++++++++++++++++++++++++++++++++
\section{Les réels}
%+++++++++++++++++++++++++++++++++++++++++++++++++++++++++++++++++++++++++++++++++++++++++++++++++++++++++++++++++++++++++++

Une construction des réels via les coupures de Dedekind est donnée dans \cite{PaulinTopGmVegN}.

\begin{normaltext}      \label{NormooHRDZooRGGtCd}
    La construction des réels va nécessiter un petit «\wikipedia{fr}{bootstrap}{bootstrap}» au niveau de la topologie. En effet la notion de suite de Cauchy est une notion topologique (définition~\ref{DefZSnlbPc}) alors que la topologie métrique (celle entre autres de \( \eQ\)) ne sera définie que par le théorème~\ref{ThoORdLYUu}. Nous avons donc dû définir en la définition~\ref{DefKCGBooLRNdJf} \emph{ex nihilo} les notions de
\begin{itemize}
    \item
        suite de Cauchy
    \item
        suite convergente
    \item
        complétude
\end{itemize}
Nous allons ensuite construire \( \eR\) comme ensemble de classes d'équivalence de suites de Cauchy dans \( \eQ\). Ce ne sera que plus tard, après avoir défini la notion d'espace métrique que nous allons voir que sur \( \eR\), ces trois notions coïncident avec celles topologiques\footnote{Proposition~\ref{PropooUEEOooLeIImr}.}. Et par conséquent que \( \eR\) sera un espace métrique complet\footnote{Théorème~\ref{THOooUFVJooYJlieh} pour la complétude en tant que corps et théorème~\ref{PROPooTFVOooFoSHPg} pour la complétude en tant que espace métrique.}.
% position 11144-30436
% position 13984-18006

Dans cette optique, il est intéressant de lire ce que dit Wikipédia à propos des suites de Cauchy dans l'article consacré à la construction des nombres réels\cite{BIBooPIGUooHzurMI}.
\end{normaltext}

%---------------------------------------------------------------------------------------------------------------------------
\subsection{L'ensemble}
%---------------------------------------------------------------------------------------------------------------------------

Soit \( \modE\) l'ensemble des suites de Cauchy\footnote{Définition~\ref{DefKCGBooLRNdJf}\ref{ItemVXOZooTYpcYN}} dans \( \eQ\). Soit aussi l'ensemble \( \modE_0\) constituée des suites qui convergent vers zéro\footnote{Nous rappelons qu'à ce niveau nous n'avons pas encore prouvé que toutes les suites de Cauchy convergent.}.

En posant
\begin{equation}
    x+y=(x_n+y_n)
\end{equation}
et
\begin{equation}
    xy=(x_ny_n),
\end{equation}
l'ensemble \( \modE\) devient un anneau\footnote{Définition~\ref{DefHXJUooKoovob}.} commutatif dont le neutre de l'addition est la suite constante \( x_n=0\) et le neutre pour la multiplication est la suite constante \( x_n=1\).

\begin{proposition}     \label{PROPooNUQVooAAkicK}
    La partie \( \modE_0\) est un idéal\footnote{Définition~\ref{DefooQULAooREUIU}.} de l'anneau \( \modE\).
\end{proposition}

\begin{proof}
    Nous savons par la proposition~\ref{PropFFDJooAapQlP}\ref{ItemRKCIooJguHdji} que les suites convergentes sont de Cauchy; par conséquent \( \modE_0\subset\modE\).

    L'ensemble structuré \( (\modE_0,+)\) est un sous-groupe de \( \modE\) par les propriétés de la proposition~\ref{PropFFDJooAapQlP} (il s'agit du fait que la somme de deux suites convergent vers zéro est une suite convergente vers zéro).

    En ce qui concerne la propriété fondamentale des idéaux, si \( x\in\modE_0\) et \( y\in\modE\) nous devons prouver que \( xy\in \modE_0\). Vu que \( (\modE_0,\cdot)\) est commutatif, cela suffira pour être un idéal bilatère. Vu que \( y\) est une suite de Cauchy, elle est bornée; et étant donné que \( x\to 0\) nous avons alors \( xy\to 0\) (par la proposition~\ref{PropFFDJooAapQlP}\ref{ItemRKCIooJguHdjiii}).
\end{proof}

\begin{theoremDef}[L'anneau des réels\cite{RWWJooJdjxEK}]       \label{DefooFKYKooOngSCB}
    Sur l'ensemble quotient \( \modE/\modE_0\), les opérations
    \begin{enumerate}
        \item
            \( \bar u+\bar v=\overline{ u+v }\)
        \item
            \( \bar u\cdot \bar v=\overline{ uv }\)
    \end{enumerate}
    sont bien définies et donnent à \( \modE/\modE_0\) une structure de corps commutatif appelé \defe{corps des réels}{réel} et noté \( \eR\)\nomenclature[Y]{\( \eR\)}{l'ensemble des réels}
\end{theoremDef}
\index{construction!des réels}

\begin{proof}
    Nous divisons la preuve en plusieurs parties.
    \begin{subproof}
    \item[Les opérations sont bien définies]
        La partie \( \modE_0\) est un idéal par la proposition \ref{PROPooNUQVooAAkicK}. Le quotient est donc bien défini et est un anneau par la proposition \ref{PROPooGXMRooTcUGbi}\ref{ITEMooYBEGooTlHgNz}.
    \item[Caractérisation des classes]
        Soit \( q\in \eQ\) et une suite \( x\) convergente vers \( q\). Cette suite est de Cauchy comme toute suite convergente. Montrons que
        \begin{equation}
            \bar x=\{ \text{suites qui convergent vers } q \}.
        \end{equation}
        Si \( y\in\bar x\) alors \( y=x+h\) avec \( h\in \modE_0\), et comme \( h_n\to 0\), on a \( y_n\to q\). Réciproquement, si \( y_n\to q\) alors pour chaque \( n\) nous avons
        \begin{equation}
            y_n=x_n+(y_n-x_n),
        \end{equation}
        mais \( y_n-x_n\to 0\). Donc la suite \( y-x\in\modE_0\) ce qui signifie que \( y\in\bar x\).
    \item[Neutre et unité]
        Il est vite vérifié que \( \bar 0\), la classe de la suite constante égale à zéro est neutre pour l'addition. De même, \( \bar 1\), est un neutre pour la multiplication.
    \item[Corps]
        Nous devons prouver que tout élément non nul est inversible. C'est-à-dire que si \( x\in\modE\) ne converge pas vers zéro\footnote{\( x\in\modE\) peut soit ne pas converger du tout, soit converger vers autre chose que zéro.} alors il existe \( y\in \modE\) tel que \( xy\in\bar 1\).

        Nous savons par la proposition~\ref{PropFFDJooAapQlP}\ref{ItemRKCIooJguHdjvi} que \( x\) étant une suite de Cauchy dans \( \eQ\), il existe \( n_0\in \eN\) tel que \( \left( \frac{1}{ x_n } \right)_{n\geq n_0}\) est une suite de Cauchy. Nous posons alors
        \begin{equation}
            y_n=\begin{cases}
                0    &   \text{si } n\leq n_0\\
                \frac{1}{ x_n }    &    \text{si } n>n_0.
            \end{cases}
        \end{equation}
        Nous avons alors
        \begin{equation}
            (xy)_n=\begin{cases}
                0    &   \text{si } n\leq n_0\\
                1    &    \text{si } n>n_0
            \end{cases}
        \end{equation}
        et donc \( xy\in\bar 1\).
    \end{subproof}
\end{proof}

\begin{normaltext}[Quelques notations entre \( \eQ\) et \( \eR\)]      \label{NORMooWBYNooBQaPPk}
    Si \( k\mapsto x_k\) est une suite, nous notons \( (x_k)\) avec des parenthèses la suite elle-même. Le \( k\) dans \( (x_k)\) est un indice muet, et dans les cas où il peut y avoir une ambiguïté, nous pouvons noter \( (x_k)_{k\in \eN}\). Cette dernière notation est plus lourde, mais plus exacte.
    
    Le mieux est d'écrire simplement \( x\) la suite, mais alors il faut être prudent et ne pas noter \( x\) la limite. Nous éviterons donc d'écrire \( x_k\to x\).

    Si \( (x_k)\) est une suite de Cauchy dans \( \eQ\), nous notons \( \bar x\) l'élément de \( \eR\) qui lui correspond. En fait \( \bar x=(x_k)\) : \( \bar x\) est la suite-elle même, mais pour nous souvenir de l'origine nous allons adopter cette notation.

    D'autre part nous définissons
    \begin{equation}
        \begin{aligned}
            \varphi\colon \eQ&\to \eR \\
            q&\mapsto \overline{ [k\mapsto q]},
        \end{aligned}
    \end{equation}
    c'est-à-dire que \( \varphi(q)\) est la classe de la suite constante \( x_k=q\).
\end{normaltext}

\begin{proposition}     \label{PropooEPFCooMtDOfP}
    Soit l'application
    \begin{equation}
        \begin{aligned}
            \varphi\colon \eQ&\to \eR \\
            q&\mapsto \bar q .
        \end{aligned}
    \end{equation}
    où par \( \bar q\) nous entendons la classe de la suite constante égale à \( q\) (qui est de Cauchy).
    \begin{enumerate}
        \item
            C'est un homomorphisme de corps injectif.
        \item
            \( \Image(\varphi)\) est un sous-corps de \( \eR\)
        \item
            \( \varphi\colon \eQ\to \Image(\varphi)\) est un isomorphisme de corps.
    \end{enumerate}
\end{proposition}

\begin{proof}
    Le fait que ce soit un homomorphisme est simplement
    \begin{itemize}
        \item \( \varphi(q+q')=\overline{ q+q' }=\bar q+\overline{ q' }\)
        \item \( \varphi(qq')=\overline{ qq' }=\overline{ q }\overline{ q' }\).
    \end{itemize}
    En ce qui concerne l'injectivité, si \( q\) est tel que \( \varphi(q)=\bar 0=\modE_0\), c'est que
    \begin{equation}
        \varphi(q)=\{ \text{suites de Cauchy qui convergent vers zéro} \}
    \end{equation}
    Mais nous savons aussi que\footnote{Voir dans la démonstration du théorème~\ref{DefooFKYKooOngSCB}.}
    \begin{equation}
        \varphi(q)=\bar q=\{ \text{suites de Cauchy qui convergent vers } q \}
    \end{equation}
    Nous en déduisons que \( q=0\).
\end{proof}
Lorsque dans la suite nous parlerons d'un élément de \( \eQ\) comme étant un réel, nous aurons en tête l'image de cet élément par \( \varphi\).

%---------------------------------------------------------------------------------------------------------------------------
\subsection{Relation d'ordre}
%---------------------------------------------------------------------------------------------------------------------------

Nous définissons les parties \( \modE^+\) et \( \modE^-\) de \( \modE\) par
\begin{enumerate}
    \item
        \( x\in  \modE^+\) si et seulement si pour tout \( \epsilon>0\), il existe \( N_{\epsilon}\) tel que \( n>N_{\epsilon}\) implique \( x_n>-\epsilon\).
    \item
        \( x\in  \modE^-\) si et seulement si pour tout \( \epsilon>0\), il existe \( N_{\epsilon}\) tel que \( n>N_{\epsilon}\) implique \( x_n<\epsilon\).
\end{enumerate}
Nous notons aussi \( \modE^{++}=\modE^+\setminus\modE_0\).

\begin{lemma}
    Les parties \( \modE^+\) et \( \modE^-\) partitionnent \( \modE\) de la façon suivante :
    \begin{enumerate}
        \item
            \( \modE^+\cap\modE^-=\modE_0\)
        \item
            \( \modE^+\cup\modE^-=\modE\)
    \end{enumerate}
\end{lemma}

\begin{proof}
    On prouve d'abord que \( \modE^+\cap\modE^-\subset\modE_0\), l'inclusion inverse est évidente. Soit \( \epsilon>0\) et \( x\in \modE^+\cap\modE^-\). Il existe un \( N\in \eN\) tel que \( x_n>-\epsilon\) et \( x_n<\epsilon\) pour tout \( n\geq N\). Par conséquent, \( | x_n |\leq \epsilon\) pour tout \( n\geq N\) et la suite \( x\) converge vers zéro, c'est-à-dire \( x\in\modE_0\).

    Pour prouver le second point, soit \( x\in \modE\setminus\modE^-\), et prouvons que \( x\in\modE^+\). La condition \( x\notin \modE^-\) donne qu'il existe un \( \alpha>0\) (dans \( \eQ\)) tel que pour tout \( n\), il existe \( p>n\) avec \( x_p>\alpha\). Mais \( x\) est une suite de Cauchy, donc nous avons un \( n_0\) tel que si \( n,p\geq n_0\) alors \( | x_n-x_p |\leq \frac{ \alpha }{2}\). En particulier, si \( n\geq n_0\), et si \( p>n\) est tel que \( x_p>\alpha\), on obtient
    \begin{equation}
        x_n>\frac{ \alpha }{2}>0
    \end{equation}
    Par conséquent \( x\in\modE^+\) parce que \( x\in\modE\) et les \( x_n \) sont tous positifs à partir d'un certain rang.
\end{proof}

\begin{lemma}[\cite{RWWJooJdjxEK}]
    Quelques propriétés du partitionnement.
    \begin{enumerate}
        \item
            \( x\in\modE^-\) si et seulement si \( (-x)\in\modE^+\)
        \item
            \( x\in\modE^+\) et \( y\in\modE^+\) implique \( x+y\in\modE^+\)
        \item
            \( x\in\modE^+\) et \( y\in\modE^+\) implique \( xy\in\modE^+\)
        \item
            Si \( x,y\in\modE\) sont tels que \( x-y\in\modE_0\) alors soit \( x,y\in\modE^+\) soit \( x,y\in\modE^-\).
    \end{enumerate}
\end{lemma}

\begin{proof}
    Point par point.
    \begin{enumerate}
        \item
            Définition de \( \modE^+\) et \( \modE^-\).
        \item
            Pour \( n\geq N_{\epsilon/2}\) nous avons \( x_n>-\epsilon/2\) et \( y_n>-\epsilon/2\). Donc \( x_n+y_n>-\epsilon\).
        \item
            Si \( x\) ou \( y\) est dans \( \modE_0\) alors \( xy\in\modE_0\) et c'est bon. Si par contre \( x,y\in\modE^{++}\) alors pour \( n\) suffisamment grand, \( x_n>0\) et \( y_n>0\). Et dans ce cas, \( (xy)_n> 0\), c'est-à-dire \( xy\in\modE^+\).
        \item
            Supposons que \( x-y\in\modE_0\) avec \( x\in\modE^+\) et prouvons qu'alors \( y\in\modE^+\). Soit donc \( \epsilon>0\); il existe \( n_1\) tel que \( x_n>-\frac{ \epsilon }{2}\) dès que \( n\geq n_1\). Mais \( x-y\in\modE_0\), donc il existe \( n_2\) tel que \( | x_n-y_n |<\frac{ \epsilon }{2}\) dès que \( n\geq n_2\). En prenant \( n\) plus grand que \( n_1\) et \( n_2\), nous avons en même temps
            \begin{subequations}
                \begin{numcases}{}
                    x_n>-\frac{ \epsilon }{2}\\
                    | x_n-y_n |<\frac{ \epsilon }{2}.
                \end{numcases}
            \end{subequations}
            Cela implique que \( y_n>-\epsilon\) et donc que \( y\in\modE^+\).

            Nous pouvons de même prouver que si \( x\in\modE^-\) alors \( y\in\modE^-\).
    \end{enumerate}
\end{proof}

\begin{definition}[Positivité dans \( \eR\)]        \label{DefooLMQIooTgzZXd}
    Vocabulaire et notations.
    \begin{enumerate}
        \item
            Nous notons \( \eR=\modE/\modE_0\).
        \item
            Nous notons \( \eR^+=\modE^+\).\nomenclature[Y]{\( \eR^+\)}{les réels positifs ou nuls}
        \item
            Nous notons \( \eR^-=\modE^-\).
        \item
            Un élément de \( \eR\) est \defe{positif}{positif} s'il est la classe d'une suite de Cauchy appartenant à \( \modE^+\).
        \item
            Un élément de \( \eR\) est \defe{négatif}{négatif} s'il est la classe d'une suite de Cauchy appartenant à \( \modE^-\).
        \item
            Lorsque nous parlons de nombres réels, le symbole «\( 0\)» signifie \( \modE_0\) ou plus précisément la classe d'un élément de \( \modE_0\) modulo \( \modE_0\).
    \end{enumerate}
\end{definition}

\begin{normaltext}\label{REMooOCXLooKQrDoq}
    Avec les conventions de la définition~\ref{DefooLMQIooTgzZXd}, et en anticipant sur nos connaissances à propos des réels,
    \begin{enumerate}
        \item
            zéro est positif et négatif.
        \item
            L'intersection entre \( \eR^+\) et \( \eR^-\) est le singleton \( \{ 0 \}\).
        \item
            L'ensemble des nombres \emph{strictement} positifs est noté \( (\eR^+)^*\) ou \( \eR^+\setminus\{ 0 \}\).
        \item
            Le mot «positif» signifie «positif ou nul»; le mot «négatif» signifie «négatif ou nul».
    \end{enumerate}

    Cela vient des conventions de la remarque \ref{REMooOCXLooKQrDoq} qui sont également celles de Wikipédia\cite{ooSBSSooTlnuKi}.
\end{normaltext}


\begin{lemmaDef}[Ordre dans \( \eR\)]       \label{LemooRordonne}
    Si \( a,b\in \eR\) nous notons \( a\leq b\) si et seulement si \( b-a\) est positif. Nous notons aussi \( a>b\) si et seulement si \( b-a\in \eR^+\setminus\{ 0 \}\), etc.

    Avec ces définitions,
    \begin{enumerate}
        \item       \label{ITEMooNHIUooMQHZKZ}
            l'ensemble \( (\eR,\leq)\) est un corps totalement ordonné (définitions~\ref{DEFooVGYQooUhUZGr} et~\ref{DefKCGBooLRNdJf});
        \item
            l'application
            \begin{equation}
                \begin{aligned}
                    \varphi\colon \eQ&\to \eR \\
                    q&\mapsto \bar q
                \end{aligned}
            \end{equation}
            dont nous avons déjà parlé dans la proposition~\ref{PropooEPFCooMtDOfP} est strictement croissante.
    \end{enumerate}
\end{lemmaDef}

\begin{proof}
    Nous prouvons la stricte croissance de \( \varphi\). Si \( q< l\) alors \( \varphi(q)-\varphi(l)=\overline{ q-l }\) est la classe de la suite constante \( q-l\) qui est un élément strictement positif de \( \eQ\). Nous avons donc \( \overline{ q-l }\in \eR^+\), et donc \( \varphi(q)<\varphi(l)\).
\end{proof}

\begin{remark}
    Comme déjà mentionné plus haut, à chaque fois que nous parlerons d'un élément de \( \eQ\) comme étant un élément de \( \eR\), nous considérons la classe de la suite constante.
\end{remark}

\begin{lemma}       \label{LemooYNOVooOwoRwD}
    Si \( x,y,z\in \eR\) avec \( x>0\) sont tels que \( z>y/x\) alors \( zx>y\).
\end{lemma}

\begin{proof}
    Nous savons que
    \begin{equation}
        z-\frac{ y }{ x }\in \modE^+\setminus\{ 0 \}=\modE^{++}.
    \end{equation}
    Vu que \( x\in\modE^{++}\), multiplier par \( x\) fait rester dans \( \modE^{++}\) :
    \begin{equation}
        zx-x\frac{ y }{ x }\in \modE^{++}.
    \end{equation}
    Un représentant de \( x\frac{ y }{ x }\) est la suite \( n\mapsto x_n\frac{ y_n }{ x_n }=y_n\). Donc \( x\frac{ y }{ x }=y\). Cela signifie que \( zx-y\in\modE^{++}\) et donc que \( zx>y\).
\end{proof}

\begin{lemma}       \label{LemooMWOUooVWgaEi}
    Pour tout \( a\in \eR\), il existe \( p\in \eN\) tel que \( p>a\).
\end{lemma}

\begin{proof}
    Nous allons donner deux preuves différentes de ce lemme.
    \begin{subproof}
    \item[Première façon]

        L'élément \( a\) de \( \eR\) admet un représentant \( (a_n)\) qui est une suite de Cauchy dans \( \eQ\). C'est donc une suite bornée, c'est-à-dire qu'il existe \( m,q\in \eN\) tels que \( | a_n |\leq m/q\) pour tout \( n\) (proposition~\ref{PropFFDJooAapQlP}\ref{ItemRKCIooJguHdjii}). Soit \( M\) un naturel strictement plus grand que \( m/q\)\footnote{Lemme~\ref{LEMooEBTIooGMoHsj}.}.

    La suite de Cauchy \( (M-a_n)_{n\in \eN}\) est constituée de rationnels positifs et est donc dans \( \modE^+\). La classe de \( M-a\) est donc un réel positif\footnote{Et nous allons d'ailleurs arrêter de toujours préciser «la classe de» lorsque ce n'est pas nécessaire.}. Par définition de la relation d'ordre, \( M\geq a\).
\item[Seconde façon]

    La suite \( (a_n)\) est majorée par \( \frac{ m }{ q }\), donc on a dans \( \eQ\) et pour tout \( n\) :
    \begin{equation}
        a_n\leq \frac{ m }{ q }=M\leq qM.
    \end{equation}
    L'application \( \varphi\colon \eQ\to \eR\) est croissante, donc
    \begin{equation}
        \varphi\big( (a_n) \big)\leq \varphi(qM).
    \end{equation}
    \end{subproof}
\end{proof}

En corolaire, nous avons
\begin{lemma}      \label{LEMooMWOUooVWgbFi}
    Pour tout \( x\in \eR\), il existe \( q\in \eZ\) tel que \( q < x\).
\end{lemma}
\begin{proof}
    Utilisation du lemme précédent avec \( a = -x \): on prend \( q = -p \).
\end{proof}

\begin{theorem}[\cite{RWWJooJdjxEK}]        \label{ThoooKJTTooCaxEny}
    Le corps \( \eR\) est archimédien\footnote{Définition~\ref{DefKCGBooLRNdJf}\ref{ItemooDZQKooPsqeRf}.}.
\end{theorem}

\begin{proof}
    Le lemme~\ref{LemooRordonne} dit que \( \eR\) est totalement ordonné. Soient \( x,y\in \eR\) avec \( x>0\); posons \( a=\frac{ y }{ x }\). Le lemme~\ref{LemooMWOUooVWgaEi} nous donne un \( p \in \eN\) tel que \(p > a\).Nous concluons alors avec le lemme~\ref{LemooYNOVooOwoRwD} :
    \begin{equation}
        px>ax=\frac{ y }{ x }x=y.
    \end{equation}
\end{proof}

Le lemme suivant n'est pas loin de dire que \( \eQ\) est dense dans \( \eR\), à part que nous n'avons pas encore donné de topologie sur \( \eR\).
\begin{lemma}       \label{LemooHLHTooTyCZYL}
    Si \( x,y\in \eR\) sont tels que \( x<y\), alors il existe \( s\in \eQ\) tel que \( x<s<y\).
\end{lemma}

\begin{proof}
    Nous avons par hypothèse que \( y-x>0\) et donc le fait que \( \eR\) soit archimédien (théorème~\ref{ThoooKJTTooCaxEny}) nous donne \( q\in \eN\) tel que \( q(y-x)>1\). Soit
    \begin{equation}
        E=\{ n\in \eZ\tq \frac{ n }{ q }\leq x \}.
    \end{equation}
    Cet ensemble n'est pas vide à cause du lemme~\ref{LEMooMWOUooVWgbFi}; de plus, comme \( |x|q \leq n_0\) pour un certain \( n_0 \) (à cause du lemme~\ref{LemooMWOUooVWgaEi}), l'ensemble \( E\) est majoré par \( n_0\). Donc \( E\) possède un plus grand élément\footnote{Lemme~\ref{LEMooMYEIooNFwNVI}.} \( p\) qui vérifie
    \begin{equation}
        \frac{ p }{ q }\leq x<\frac{ p+1 }{ q }.
    \end{equation}
    De plus \( (p+1)/q<y\). En effet nous avons
    \begin{equation}
        \frac{ p+1 }{ q }=\frac{ p }{ q }+\frac{1}{ q }\leq x+\frac{1}{ q }<x+y-x=y
    \end{equation}
    où nous avons utilisé l'inégalité stricte \( y-x>\frac{1}{ q }\).

    Nous avons donc
    \begin{equation}
        x<\frac{ p+1 }{ q }<y,
    \end{equation}
    et le nombre \( (p+1)/q\) convient comme \( s\).
\end{proof}

\begin{remark}      \label{REMooXOIOooHjwMvA}
    Le lemme~\ref{LemooHLHTooTyCZYL} a également pour conséquence que des ensembles comme \( \mathopen[ -1 , 1 \mathclose]\) ne sont pas bien ordonnés (définition~\ref{DEFooVGYQooUhUZGr}). En effet la partie \( \mathopen] 0 , 1 \mathclose[\) ne possède pas de minimum parce que si \( x\in \mathopen] 0 , 1 \mathclose[\) alors \( 0<x\) et il existe \( s\in \eQ\) (a fortiori \( s\in \eR\)) tel que \( 0<s<x\), c'est-à-dire que \( x\) n'est pas un minimum de \( \mathopen] 0 , 1 \mathclose[\).
\end{remark}

Tant que nous y sommes dans les encadrements de réels\dots
\begin{normaltext}
  Soit \(q_0 \in \eQ \) tel que \( 0 \leq q_0 < 1 \). On définit alors \( d_1 \in \{0, 1\} \) comme valant \( 1 \) si \( 2 q_0 \geq 1 \) et \(0 \) sinon. Puis on pose \( q_1 = 2 q_0 - d_1 \).

  Poursuivant de la sorte, on crée une suite \( (d_n)_{n\geq 1} \): c'est le \defe{développement dyadique}{dyadique!développement} de \( q_0 \).
\end{normaltext}

\begin{lemma}[\cite{MonCerveau}]        \label{LEMooRSLIooVrZMxM}
  Soit \( q,\ q' \) deux rationnels tels que \( 0 \leq q < q' < 1 \). Il existe deux entiers naturels \( a \) et \( N \) tels que \( q < \frac a {2^N} < q' \).
\end{lemma}
\begin{proof}
  On crée les développements dyadiques de \( q \) et \( q' \), que l'on note respectivement \( (d_n)_{n\geq 1} \) et \( (d'_n)_{n\geq 1} \). Notons
  \begin{equation}
    E = \{ n \in \eN \tq d_n \neq d'_n \}.
  \end{equation}
Comme \( q \neq q' \), les développements dyadiques sont différents\quext{À vérifier tout de même\dots}, l'ensemble \(E\) est non-vide, et il admet un plus petit élément \(N \). Or, \( q < q' \), et donc nécessairement \( d_N < d'_N \). On construit alors \( a = \sum_{i=1}^{N} d_i 2^i \). 
\end{proof}

\begin{corollary}\label{CorDensiteDyadiques}
  Pour tous réels \(x,\ y\) tels que \( 0 \leq x < y \leq 1 \), il existe un nombre de la forme \( d = a / 2^n \), avec \( n \in \eN \) et \( a \in \eN,\ a \leq 2^n\), tel que \( x < d < y \).
\end{corollary}

%TODO: preuve complète.

\begin{lemma}[\cite{MonCerveau}]        \label{LEMooEGYLooCGrDrl}
    Soient des réels \( a,b,x,y\) tels que
    \begin{equation}
        a\leq x\leq b
    \end{equation}
    et
    \begin{equation}
        a\leq y\leq b,
    \end{equation}
    alors \( | x-y |\leq | b-a |\).
\end{lemma}

%---------------------------------------------------------------------------------------------------------------------------
\subsection{Complétude}
%---------------------------------------------------------------------------------------------------------------------------

Le théorème \ref{ThoKHTQJXZ} donne une complétion de tout espace métrique en un espace complet. Il serait tentant de l'utiliser ici pour définir \( \eR\) à partir de \( \eQ\). Cette méthode ne fonctionne cependant pas parce que la démonstration de \ref{ThoKHTQJXZ} utilise le fait que \( \eR\) est complet.

\begin{lemma}[\cite{RWWJooJdjxEK}]      \label{LemooRTGFooYVstwS}
    Toute suite de Cauchy dans \( \eQ\) converge dans \( \eR\) vers le réel qu'elle représente.

    Plus précisément, en suivant les notations de \ref{NORMooWBYNooBQaPPk}, si \( (x_k)\) est une suite de Cauchy dans \( \eQ\), alors la suite \( \varphi(x_k)\) dans \( \eR\)
    \begin{enumerate}
        \item
            est de Cauchy dans \( \eR\),
        \item
            converge dans \( \eR\) vers \( \bar x\).
    \end{enumerate}
\end{lemma}

\begin{proof}
    Soit \( (x_n)\) une suite de Cauchy de \( \eQ\), c'est-à-dire que \( x_k\in \eQ\) pour tout \( k\) et qu'elle est de Cauchy. Elle représente un réel \( \bar x\in \eR\), et nous voulons prouver que pour la topologie de \( \eR\) nous avons \( \lim_{n\to \infty} x_n=\bar x\). Dans cette dernière limite, chacun des \( x_n\) est vu dans \( \eR\).

    Si \( \epsilon\in \eQ\) est donné, il existe \( N_{\epsilon}\) tel que si \( p,q\geq N_{\epsilon}\) alors \( | x_p-x_q |< \epsilon\), c'est-à-dire
    \begin{equation}
        x_p-\epsilon<x_q<x_p+\epsilon.
    \end{equation}
    Soit \( p\geq N_{\epsilon}\) fixé. Pour tout \( q\geq N_{\epsilon}\) nous avons \(  x_p-\epsilon<x_q<x_p+\epsilon \). Par conséquent, la suite \( n\mapsto (x_p-\epsilon)-q_n\) est un élément de \( \modE^-\) et au niveau des classes nous pouvons écrire
    \begin{equation}
        \overline{ n\mapsto (x_p-\epsilon)-x_n }\leq 0.
    \end{equation}
    Vu que \( x_p-\epsilon\) représente la suite constante nous avons l'inégalité suivante dans \( \eR\) :
    \begin{equation}
        x_p-\epsilon-\bar x\leq 0
    \end{equation}
    ou encore : \( \bar x\geq x_p-\epsilon\). En faisant de même avec l'autre partie de l'inégalité, \( x_p-\epsilon\leq \bar x\leq x_p+\epsilon\), ce qui implique que
    \begin{equation}
        x_p\in B(\bar x,\epsilon)
    \end{equation}
    dès que \( p\geq N_{\epsilon}\). Cela signifie que \( x_p\to \bar x\) dans \( \eR\).
\end{proof}

\begin{proposition}     \label{PROPooZSQYooWRKNGY}
    Soit une suite convergente \( x_k\stackrel{\eQ}{\longrightarrow}q\). Alors
    \begin{equation}
        \varphi(x_k)\stackrel{\eR}{\longrightarrow}\varphi(q)
    \end{equation}
    où \( \varphi\) est la fonction qui à un rationnel fait correspondre la classe de la suite constante correspondante\footnote{Voir les notations en \ref{NORMooWBYNooBQaPPk}.}. 
\end{proposition}

\begin{proof}
    Le fait d'avoir une convergence \( x_k\to q\) dans \( \eQ\) implique que la suite \( (x_k)\) est de Cauchy, par la proposition \ref{PropFFDJooAapQlP}\ref{ItemRKCIooJguHdji}.
    
    Le lemme \ref{LemooRTGFooYVstwS} nous indique que \( \varphi(x_k)\) est une suite dans \( \eR\) qui converge vers \( \bar q\), la classe de la suite \( (x_k)\).

    À prouver : \( \varphi(x)=\bar q\). Autrement dit, nous devons prouver que la classe de la suite constante \( a_k=q\) et la classe de la suite \( x\) sont les mêmes.

    La suite \( (x_k-q)\) est de Cauchy dans \( \eQ\) et converge vers zéro par hypothèse. Donc les suites \(x\) et \( (q)\) sont dans la même classe.
\end{proof}

\begin{proposition}[\cite{MonCerveau}]     \label{PROPooFGBOooHiZqbs}
    Deux choses à propos de suites de rationnels convergeant vers un réel.
    \begin{enumerate}
        \item       \label{ITEMooMAVYooKFtqlx}
    Soit un réel \( x\). Il existe une suite de rationnels strictement croissante qui converge vers \( x\).
\item       \label{ITEMooVOVYooFUbccG}
    Si de plus \( x>0\), alors la suite (toujours strictement croissante) peut être choisie parmi les rationnels strictement positifs.
    \end{enumerate}
\end{proposition}

\begin{proof}
    Le lemme \ref{LemooHLHTooTyCZYL} nous sera d'une grande aide. Soit \( x\in \eR\). Il existe \( q_0\in \eQ\) tel que \( x-1<q_0<x\). Ensuite nous construisons la suite par récurrence : \( q_k\) est choisi tel que \( q_{k-1}<q_k<x\). Cela règle le point \ref{ITEMooMAVYooKFtqlx}.

    Pour \ref{ITEMooVOVYooFUbccG}. Il suffit de faire la même chose, en partant de \( 0<q_0<x\).
\end{proof}

\begin{theorem}[Complétude de \( \eR\), critère de Cauchy\cite{RWWJooJdjxEK}] \label{THOooUFVJooYJlieh}
    Nous avons :
    \begin{enumerate}
        \item
            Le corps \( \eR\) est un corps complet (définition~\ref{DefKCGBooLRNdJf}\ref{ITEMooKZZYooDaidGU})
        \item
            Une suite dans \( \eR\) est convergente (définition~\ref{DefKCGBooLRNdJf}\ref{ITEMooDERQooLmJwFR}) si et seulement si elle est de Cauchy (définition~\ref{DefKCGBooLRNdJf}\ref{ItemVXOZooTYpcYN}).
    \end{enumerate}
\end{theorem}
\index{complet!$\eR$!corps}
\index{critère!de Cauchy}
Notez la grande similitude entre ce théorème et le théorème~\ref{THOooNULFooYUqQYo}. Ils ne sont pas équivalents, ne parlent pas exactement du même objet «\( \eR\)», ni des mêmes notions de suites de Cauchy et de complétude.

\begin{proof}
    Soit \( (x_n)\) une suite de Cauchy dans \( \eR\). Pour chaque \( n\), il existe par le lemme~\ref{LemooHLHTooTyCZYL} un \( y_n\in \eQ\) tel que
    \begin{equation}
        x_n-\frac{1}{ n }<y_n<x_n+\frac{1}{ n }.
    \end{equation}
    \begin{subproof}
        \item[\( (y_n)\) est une suite de Cauchy dans \( \eQ\)]
            Nous prouvons que \( (y_n)\) est une suite de Cauchy dans \( \eQ\) (définition~\ref{DefKCGBooLRNdJf}\ref{ItemVXOZooTYpcYN}). Vu que \( (x_n)\) est de Cauchy pour le corps \( \eR\), si \( \epsilon>0\) dans \( \eR\) est donné, il existe \( n_{\epsilon}\) tel que si \( p,q\geq n_{\epsilon}\), alors \( | x_p-x_q |<\epsilon\).

        Nous avons :
        \begin{equation}
            | y_p-y_q |\leq | y_p-x_p |+| x_p-x_q |+| x_q-y_q |<\frac{1}{ p }+\epsilon+\frac{1}{ q }.
        \end{equation}
        En choisissant \( N_{\epsilon}>\max\{ n_{\epsilon},\frac{1}{ \epsilon } \}\) (ce qui est possible par le lemme~\ref{LemooMWOUooVWgaEi}), et en prenant \( p,q>N_{\epsilon}\), nous avons
        \begin{equation}
            | y_p-y_q |\leq 3\epsilon,
        \end{equation}
        ce qui prouve que \( (y_p)\) est une suite de Cauchy dans \( \eQ\), pour la notion de suite de Cauchy dans \( \eQ\).

    \item[Le réel représenté]

        Vu que \( (y_p)\) est de Cauchy dans \( \eQ\), elle représente un réel que nous notons \( \bar y\).

    \item[Convergence de \( (x_n)\)]

        Nous prouvons que \(     x_n\stackrel{\eR}{\longrightarrow}\bar y \).

        Nous savons qu'une suite de Cauchy de rationnels converge dans \( \eR\) vers le réel qu'elle représente, c'est-à-dire : \( y_n\stackrel{\eR}{\longrightarrow}\bar y\) où chaque \( y_n\in \eQ\) est vu comme la suite constante (cela est le lemme~\ref{LemooRTGFooYVstwS}). Autrement dit, pour \( \epsilon>0\), il existe un \( N_{\epsilon}\in \eN\) tel que si \( p>N_{\epsilon}\) alors \( | \bar y-y_p |<\epsilon\). Pour un tel \( p\) nous avons
        \begin{equation}
            | \bar y-x_p |\leq| \bar y-y_p |+| y_p-x_p |\leq \epsilon+\frac{1}{ p }.
        \end{equation}
        Donc dès que \( p\) est plus grand que \( \max\{ N_{\epsilon},\frac{1}{ \epsilon } \}\), nous avons \( | \bar y-x_p |<2\epsilon\), ce qui signifie que la suite \( (x_n) \) converge vers \( \bar y\) dans \( \eR\).

        Ceci achève de prouver que \( \eR\) est un corps complet.
        \end{subproof}

        En ce qui concerne l'équivalence entre les suites convergentes et de Cauchy, nous venons de prouver que toute suite de Cauchy dans \( \eR\) est convergente. La réciproque est la proposition~\ref{PROPooTFVOooFoSHPg}.

\end{proof}

Nous avons terminé avec la construction des réels. Les propriétés topologiques arrivent en la section~\ref{SECooGKHYooMwHQaD}. En particulier le théorème~\ref{THOooNULFooYUqQYo} pour la complétude de \( \eR\) en tant qu'espace métrique.

%--------------------------------------------------------------------------------------------------------------------------- 
\subsection{Intervalles}
%---------------------------------------------------------------------------------------------------------------------------

Nous avons déjà défini la notion d'intervalle pour un espace totalement ordonné en \ref{DefEYAooMYYTz}. Nous posons quelques notations dans \( \eR\).

\begin{definition}  \label{DEFooAQBUooKLChOW}
    Soient \( a\neq b\) dans \( \eR\). Nous définissons les parties suivantes de \( \eR\) :
    \begin{enumerate}
        \item
            \( \mathopen] a , b \mathclose[=\{ x\in \eR\tq a<x<b \}\)
        \item
            \( \mathopen[ a , b \mathclose[=\{ x\in \eR\tq a\leq x<b \}\)
            \item
            \( \mathopen] a , b \mathclose]=\{ x\in \eR\tq a<x\leq b \}\)
        \item
            \( \mathopen[ a , b \mathclose]=\{ x\in \eR\tq a\leq x\leq b \}\)
        \item
        \( \mathopen]-\infty , a \mathclose]=\{ x\in \eR\tq x\leq a \}\)
        \item
        \( \mathopen]-\infty , a \mathclose[=\{ x\in \eR\tq x< a \}\)
        \item
        \( \mathopen] a , \infty \mathclose[=\{ x\in \eR\tq x>a \}\)
        \item
            \( \mathopen[ a , \infty \mathclose[=\{ x\in \eR\tq x\geq a \}\).
            \item
            \( \mathopen] -\infty , \infty \mathclose[=\eR\).
    \end{enumerate}
    La proposition \ref{PROPooHPMWooQJXCAS} nous dira que tous les intervalles de \( \eR\) sont d'une de ces formes.
\end{definition}

%---------------------------------------------------------------------------------------------------------------------------
\subsection{Maximum, supremum et compagnie}
%---------------------------------------------------------------------------------------------------------------------------

Ce n'est un secret pour personne que $\eR$ est un ensemble totalement ordonné\footnote{Lemme \ref{LemooRordonne}\ref{ITEMooNHIUooMQHZKZ}.} : il y a des éléments plus grands que d'autres, et mieux : à chaque fois que je prends deux éléments différents dans $\eR$, il y en a un des deux qui est plus grand que l'autre. Il n'y a pas d'\emph{ex æquo} dans $\eR$.

\begin{definition}
    Soit \( A\), une partie de \( \eR\). 
    \begin{enumerate}
        \item
            Un nombre \( M\) est un \defe{majorant}{majorant} de \( A\) si \( M\) est plus grand que tous les éléments de \( A\) : pour tout \( x\in A\), \( M\geq x\).
        \item
            Un nombre \( m\) est un \defe{minorant}{minorant} de \( A\) si \( m\) est plus petit que tous les éléments de \( A\) : pour tout \( x\in A\), \( m\leq x\).
    \end{enumerate}
    Nous parlons de majorant ou de minorants \emph{stricts} lorsque les inégalités sont strictes.
\end{definition}

Nous insistons sur le fait que l'inégalité n'est pas stricte. Ainsi, $1$ est un majorant de $[0,1]$. Dès qu'un ensemble a un majorant, il en a plein. Si $s$ majore l'ensemble $A$, alors $s+1$, $s+4$, et \( s+\frac{ 3 }{ 7 }\) majorent également $A$.

\begin{example}
Une petite galerie d'exemples de majorants.
\begin{itemize}
\item L'intervalle fermé $[4,8]$ admet entre autres $8$ et $130$ comme majorants,
\item l'intervalle ouvert $]4,8[$ admet également $8$ et $130$ comme majorants,
\item $7$ n'est pas un majorant de $[1,5]\cup]8,32]$,
\item $10/10$ majore les notes qu'on peut obtenir à un devoir.
\item l'intervalle $[4,\infty[$ n'a pas de majorants.
\end{itemize}
\end{example}

\begin{propositionDef}[Least-upper-bound property\cite{BIBooRRUXooKWzcFo}]		\label{DefSupeA}
    Soit $A$ une partie majorée de $\eR$. Il existe un unique élément \( M\in \eR\) tel que
    \begin{enumerate}
        \item
            $M\geq x$ pour tout $x\in A$,
        \item
            pour tout $\varepsilon$, le nombre $M-\varepsilon$ n'est pas un majorant de $a$, c'est-à-dire qu'il existe un élément $x\in A$ tel que $x>M-\varepsilon$.
    \end{enumerate}

    Cet élément est nommé \defe{supremum}{supremum} de $A$ et est noté \( \sup(A)\). De la même façon, \defe{l'infimum}{infimum} de $A$, noté $\inf A$, est le plus grand de ses minorants.
\end{propositionDef}

Par convention, si la partie n'est pas bornée vers le haut, nous dirons que son supremum n'existe pas, ou bien qu'il est égal à $+\infty$, suivant les contextes. Pour votre culture générale, sachez toutefois que $\infty\notin\eR$.

\begin{proof}
    Nous faisons la preuve pour l'infimum.

    \begin{subproof}
    \item[Unicité]

    En ce qui concerne l'unicité, soient \( m_1\) et \( m_2\), deux infimums de \( A\). Supposons \( m_1>m_2\). Alors il existe \( \epsilon>0\) tel que \( m_2<m_2+\epsilon<m_1\) (c'est le lemme~\ref{LemooHLHTooTyCZYL}). Cela prouve que \( m_2+\epsilon\) est un minorant de \( A\) et donc que \( m_2\) n'est pas un infimum.

\item[Existence]

	Soit $A$, une partie de $\eR$. Nous allons trouver son infimum en suivant une méthode de dichotomie. Pour cela nous allons construire trois suites en même temps de la façon suivante. D'abord nous choisissons un point $x_0$ de $A$ et un point $x_1$ qui minore $A$ (qui existe par hypothèse) :
	\begin{equation}
		\begin{aligned}[]
			x_0&\text{ est un élément de }A,\\
			x_1&\text{ est un minorant de }A,\\
			a_0&=x_0\\
			b_0&=x_1\\
			b_1&=x_1.
		\end{aligned}
	\end{equation}
	Ensuite, nous faisons la récurrence suivante :
	\begin{equation}
		\begin{aligned}[]
			x_{n+1}&=\frac{ a_n+b_n }{2},\\
			a_{n+1}&=\begin{cases}
                a_{n}	&	\text{si }x_{n+1} \text{ minore } A \\
				x_{n+1}	&	 \text{sinon},
			\end{cases}\\
			b_{n+1}&=\begin{cases}
                x_{n+1}	&	\text{si }x_{n+1} \text{ minore } A\\
				b_n	&	 \text{sinon}.
			\end{cases}
		\end{aligned}
	\end{equation}
    Nous allons montrer que \( (a_n)\) et \( (b_n)\) sont des suites convergentes de même limite et que cette limite est l'infimum de \( A\).

	Soit $n\in\eN$; il y a deux possibilités. Soit $a_n=a_{n-1}$ et $b_n=x_n$, soit $a_n=x_n$ et $b_n=b_{n-1}$. Supposons que nous soyons dans le premier cas (le second se traite de façon similaire). Alors nous avons
	\begin{equation}
		\begin{aligned}[]
			| a_n-b_n |&=| a_{n-1}-x_n |\\
			&=\left| a_{n-1}-\frac{ a_{n-1}+b_{n-1} }{2} \right| \\
			&=\frac{ 1 }{2}| a_{n-1}-b_{n-1} |,
		\end{aligned}
	\end{equation}
	ce qui prouve que $| a_n-b_n |\to 0$. Nous montrons maintenant que la suite \( (a_n)\) est de Cauchy. En effet nous avons
    \begin{equation}
        | a_n-a_{n-1} |=\begin{cases}
          0\\
          \left| \frac{ a_n -b_n}{ 2} \right|
      \end{cases}\leq \frac{1}{ 2n }.
    \end{equation}
    Il en est de même pour la suite \( (b_n)\). Ce sont deux suites de Cauchy (donc convergentes par la proposition~\ref{PROPooTFVOooFoSHPg}) qui convergent vers la même limite. Soit \( \ell\) cette limite.

	Le nombre $\ell$ minore $A$. En effet si $a\in A$ est plus petit que $\ell$, les éléments $b_n$ tels que $| b_n-\ell |<| a-\ell |$ ne peuvent pas minorer $A$. D'autre part, pour tout $\epsilon$, le nombre $\ell+\epsilon$ ne peut pas minorer $A$. En effet, $\ell$ est la limite de la suite décroissante $(a_n)$, donc il existe $a_n$ entre $\ell$ et $\ell+\epsilon$. Mais $a_n$ ne minore pas $A$, donc $\ell+\epsilon$ ne minore pas non plus $A$.

	Nous avons prouvé que toute partie minorée de $\eR$ possède un infimum.
    \end{subproof}

    La preuve que toute partie majorée possède un supremum se fait de la même façon.
\end{proof}

%///////////////////////////////////////////////////////////////////////////////////////////////////////////////////////////
\subsubsection{Intervalles}
%///////////////////////////////////////////////////////////////////////////////////////////////////////////////////////////

\begin{proposition}     \label{PROPooHPMWooQJXCAS}
    Tous les intervalles de \( \eR\) sont d'une des formes listées dans la définition \ref{DEFooAQBUooKLChOW}.
\end{proposition}

%///////////////////////////////////////////////////////////////////////////
\subsubsection{Quelques exemples}
%///////////////////////////////////////////////////////////////////////////

En matière de notations, le maximum de l'ensemble $A$ est noté $\max A$, le supremum est noté $\sup A$. Le minimum et l'infimum sont notés $\min A$ et $\inf A$.

\begin{example}
Exemples de différence entre majorant, supremum et maximum.
\begin{itemize}
\item Le nombre $10$ est un supremum, majorant et maximum de l'intervalle fermé $[0,10]$,
\item Le nombre $10$ est un majorant et un supremum, mais pas un maximum de l'intervalle ouvert $]0,10[$,
\item Le nombre $136$ est un majorant, mais ni un maximum ni un supremum de l'intervalle $[0,10]$.
\end{itemize}
\end{example}

En utilisant les notations concises, ces différents cas s'écrivent ainsi :
\begin{equation}
    \begin{aligned}[]
10&=\max[0,10]=\sup[0,10]	& 10&=\sup[0,10[
    \end{aligned}
\end{equation}


\begin{example}
Si on dit que un pont s'effondre à partir d'une charge de $10$ tonnes, alors $10$ tonnes est un \emph{supremum} des charges que le pont peut supporter : si on met $9,999999$ tonnes dessus, il tient encore le coup, mais si on ajoute un gramme, alors il s'effondre (on sort de l'ensemble des charges acceptables).
\end{example}

\begin{example}
Si on dit qu'un pont résiste jusqu'à $10$ tonnes, alors $10$ tonnes est un \emph{maximum} de la charge acceptable. Sur ce pont-ci, on peut ajouter le dernier gramme. Mais à partir de là, le moindre truc qu'on ajoute, il s'effondre.
\end{example}

\begin{lemma}       \label{LEMooWCUXooFqTwDK}
    À propos de bornes d'un intervalle.
    \begin{enumerate}
        \item
	        La borne inférieure d'un intervalle est son infimum, 
        \item
            la borme supérieure est le supremum. 
        \item
            Si de plus l'intervalle est fermé, l'infimum est un minimum et le supremum est un maximum.
    \end{enumerate}
\end{lemma}


\begin{example}
    Quelques exemples dans les intervalles.
	\begin{enumerate}
		\item
			$A=\mathopen[ 1 , 2 \mathclose]$. Tous les nombres plus petits ou égaux à $1$ sont minorants, $1$ est infimum et minimum. Le nombre $2$ est un majorant, le maximum et le supremum.
		\item
			$B=\mathopen] 3 , \pi \mathclose[$. Le nombre $\pi$ est le supremum et est un majorant, mais n'est pas le maximum (parce que $\pi\notin B$). L'ensemble $B$ n'a pas de maximum. Bien entendu, $-1000$ est un minorant.
	\end{enumerate}
    Dans les deux cas, le nombre $53$ est un majorant.
\end{example}

Il existe évidemment de nombreux exemples plus vicieux.

\begin{example}
	Prenons $E=\{ \frac{1}{ n }\tq n\in\eN_0 \}$, dont les premiers points sont indiqués sur la figure~\ref{LabelFigSuiteUnSurn}. Cet ensemble est constitué des nombres $1$, $\frac{ 1 }{2}$, $\frac{1}{ 3 }$, \ldots Le plus grand d'entre eux est $1$ parce que tous les nombres de la forme $\frac{1}{ n }$ avec $n\geq 1$ sont plus petits ou égaux à $1$. Le nombre $1$ est donc maximum de $E$.

	L'ensemble $E$ n'a par contre pas de minimum parce que tout élément de $E$ s'écrit $\frac{1}{ n }$ pour un certain $n$ et est plus grand que $\frac{1}{ n+1 }$ qui est également dans $E$.

	Prouvons que zéro est l'infimum de $E$. D'abord, tous les éléments de $E$ sont strictement positifs, donc zéro est certainement un minorant de $E$. Ensuite, nous savons que pour tout $\varepsilon>0$, il existe un $n$ tel que $\frac{1}{ n }$ est plus petit que $\varepsilon$. L'ensemble $E$ possède donc un élément plus petit que $0+\varepsilon$, et zéro est bien l'infimum.
\end{example}

\newcommand{\CaptionFigSuiteUnSurn}{Les premiers points du type $x_n=1/n$.}
\input{auto/pictures_tex/Fig_SuiteUnSurn.pstricks}

L'exemple suivant est une source classique d'erreurs en ce qui concerne l'infimum. Il sera à relire après avoir vu la définition de limite (définition~\ref{PropLimiteSuiteNum}).

\begin{example}
	Les premiers points de l'ensemble $F=\{ \frac{ (-1)^n }{ n }\tq n\in\eN_0 \}$ sont représentés à la figure~\ref{LabelFigSuiteInverseAlterne}. Bien que (comme nous le verrons plus tard) la limite de la suite $x_n=(-1)^n/n$ soit zéro, il n'est pas correct de dire que zéro est l'infimum de l'ensemble $F$. Le dessin, au contraire, montre bien que $-1$ est le minium (aucun point est plus bas que $-1$), tandis que le maximum est $1/2$.

	Nous reviendrons avec cet exemple dans la suite. Pour l'instant, ayez bien en tête que zéro n'est rien de spécial pour l'ensemble $F$ en ce qui concerne les notions de maximum, minimum et compagnie.
\end{example}
\newcommand{\CaptionFigSuiteInverseAlterne}{Les quelques premiers points du type $(-1)^n/n$.}
\input{auto/pictures_tex/Fig_SuiteInverseAlterne.pstricks}

%--------------------------------------------------------------------------------------------------------------------------- 
\subsection{Racines}
%---------------------------------------------------------------------------------------------------------------------------
\label{SUBSECooMBCNooEqjjTY}

Dans cette section, nous définissons \( \sqrt{ x }\) pour \( x\in\eR^+\). Vous notez que c'est fait de façon assez algébrique\footnote{Discutable parce que des limites sont utilisées.}, ou en tout cas, en restant proche des définitions. Des définitions plus technologiques utilisant la continuité de \( x\mapsto x^n\) et qui prouvent que c'est bijectif sur un domaine choisi avec prudence existent, et c'est fait dans la définition \ref{DEFooJWQLooWkOBxQ}. Il est même expliqué dans \cite{BIBooMPXEooQLKhku} que la méthode décrite ici permet de définir \( \sqrt[n]{ x }\) pour tout \( n\) entier, et pas seulement pour \( n=2\).

\begin{proposition}     \label{PROPooUHKFooVKmpte}
    Soit \( q\in \eQ^+\). Il existe un unique \( r\in \eR\) tel que \( r^2=q\).
    
    Plus précisément, en termes des notations de \ref{NORMooWBYNooBQaPPk}, pour tout \( q\in \eQ^+\), il existe un unique \( r\in \eR^2\) tel que \( r^2=\varphi(q)\).
\end{proposition}

\begin{proof}
    En deux parties : d'abord l'existence et ensuite l'unicité.
    \begin{subproof}
        \item[Existence]
            Si \( q=0\), c'est \( r=0\). Nous supposons \( q>0\). La suite \( (x_k)\) de la proposition \ref{PROPooSTQXooHlIGVf} a la propriété d'être de Cauchy dans \( \eQ\). Donc il existe un réel \( r\) qui est la classe de cette suite. Nous posons donc
            \begin{equation}
                r=\bar x.
            \end{equation}
            
            En ce qui concerne \( r^2\), nous avons, par définition du produit dans \( \eR\),    
            \begin{equation}        \label{EQooPHLFooAZhebM}
                r^2=\bar x^2=\overline{ (x_k^2) },
            \end{equation}
            c'est la classe de la suite de Cauchy donnée par les \( x_k^2\). Posons \( y_k=x_k\); la relation \eqref{EQooPHLFooAZhebM} s'écrit
            \begin{equation}
                r^2=\bar y.
            \end{equation}
            
            La proposition \ref{PROPooSTQXooHlIGVf} nous dit également que \( y\) est une suite de Cauchy et que
            \begin{equation}
                y_k\stackrel{\eQ}{\longrightarrow}q
            \end{equation}
            La proposition \ref{PROPooZSQYooWRKNGY} donne alors \( \bar y=\bar q\), et finalement
            \begin{equation}
                r^2=\bar q=\varphi(q).
            \end{equation}
            Ici tout n'est pas encore terminé avec l'existence parce qu'il faut nous assurer que \( r\geq 0\). Ce n'est pas très compliqué : si \( r<0\), alors nous pouvons faire le choix \( -r\) qui convient tout aussi bien : \( (-r)^2=r^2\).
        \item[Unicité]
            Supposons \( r_1,r_2\in \eR\) tels que \( r_1^2=r_2^2\). Le lemme \ref{LemooRordonne} dit que \( \eR\) est totalement ordonné; disons pour fixer les idées que \( r1\leq r_2\). Cela signifie, par définition de l'ordre sur \( \eR\), que \( r_2-r_1\geq 0\). En posant \( s=r_2-r_1\) nous avons \( r_2=r_1+s\). Passons au carré; la distribution dans le calcul suivant provient du fait que \( \eR\) est un corps :
            \begin{equation}
                r_2^2=(r_1+s)^2=r_1^2+2r_1s+s^2.
            \end{equation}
            Vu que \( r_1^2=q=r_2^2\), nous avons \( 2r_1s+s^2=0\) ou encore
            \begin{equation}
                s(2r_1+s)=0.
            \end{equation}
            Vu que \( \eR\) est un corps, il est un anneau intègre\footnote{Lemme \ref{LemAnnCorpsnonInterdivzer}.} et la règle du produit nul s'applique : soit \( s=0\), soit \( 2r_2+s=0\). Vu que \( r_2>0\) et que \( s\geq 0\), nous avons \( 2r_2+s>0\) et donc \( s=0\). 

            Nous en déduisons que \( r_1=r_2\).
    \end{subproof}
\end{proof}

%+++++++++++++++++++++++++++++++++++++++++++++++++++++++++++++++++++++++++++++++++++++++++++++++++++++++++++++++++++++++++++
\section{Les complexes}
%+++++++++++++++++++++++++++++++++++++++++++++++++++++++++++++++++++++++++++++++++++++++++++++++++++++++++++++++++++++++++++

\begin{probleme}
    Encore une fois, cette section n'est pas du tout faite. Le but de cette section serait de
    \begin{itemize}
        \item Construire \( \eC\) en tant qu'ensemble
        \item Construire les opérations courantes.
        \item Démontrer les propriétés de base.
    \end{itemize}
    Attention : il est sans espoir de parler de forme trigonométrique ici parce que les exponentielles et fonctions trigonométriques ne sont définies qu'avec les séries.

    Nous donnons ici en vrac quelques propriétés et définitions que se doivent d'être présentes dans la version finale de cette section, si elle existe un jour.

    Comme partout dans ce chapitre sur la construction des ensembles de nombres, certaines propriétés qui ont l'air toute simples peuvent, en fonction des définitions prises, s'avérer pas du tout évidentes.
\end{probleme}


\subsection{Définitions}
Un nombre complexe s'écrit sous la forme $z = a + b i$, où $a$ et $b$
sont des nombres réels appelés (et notés) respectivement partie réelle
($a = \Re(z)$) et partie imaginaire ($b = \Im(z)$) de $z$. L'ensemble
des nombres de cette forme s'appelle l'ensemble des nombres complexes
; cet ensemble porte une structure de corps et est noté $\eC$. Le
nombre complexe $i = 0 + 1 i$ est un nombre imaginaire qui a la
particularité que $i^2 = -1$.

Deux nombres complexes $a + bi$ et $c + di$ sont égaux si et seulement
si $a = c$ et $b = d$, c'est-à-dire leurs parties réelles sont égales,
et leurs parties imaginaires sont égales.

Pour $z = a + bi$ un nombre complexe, on note $\bar z = a - bi$ le
\Defn{complexe conjugué} de $z$. Dans le plan de Gauss, il s'agit du
symétrique de $z$ par rapport à la droite réelle (généralement
dessinée horizontalement).

On définit le module du complexe $z$ par $\module z = \sqrt{z\bar z} =
\sqrt{a^2 + b^2}$. Dans le plan de Gauss, il s'agit de la distance
entre $0$ et $z$.

\begin{proposition} \label{PROPooXLARooYSDCsF}
 Si \( z_1\) et \( z_2\) sont des nombres complexes, alors 
 \begin{equation}
     | z_1z_2 |=| z_1 | |z_2 |.
 \end{equation}
 Nous avons aussi, pour tout \( n\in \eN\),
 \begin{equation}
     | z^n |=| z |^n.
 \end{equation}
\end{proposition}

\begin{proof}
    D'abord \( (a+bi)(c+di)=ac-db+(ad+bc)i\), de telle sorte que
    \begin{equation}
        | (a+bi)(c+di) |^2=(ac-bd)^2+(ad+bc)^2.
    \end{equation}
    Mais en calculant d'autre part \( | a+bi |^2| c+di |^2\), nous tombons sur la même valeur.

    Une simple récurrence permet de conclure que \( | z^n |=| z |^n\).
\end{proof}
Voila. Vous êtes déjà content d'apprendre que l'on peut démontrer \( | z^n |=| z |^n\) sans faire appel à la forme trigonométrique des nombres complexes.


\begin{proposition}     \label{PROPooUMVGooIrhZZg}
Pour tout nombres complexes $z = a+bi$ et $z^\prime$, nous avons
   \begin{enumerate}
   \item $z \bar z = a^2 + b^2$;
   \item $\bar{\bar{z}} = z$;
   \item $\module z = \module {\bar z}$;
   \item $\module{zz^\prime} = \module z \module{z^\prime}$;
   \item    \label{ITEMooDVMDooFDmOur}
       $\module{z+z^\prime} \leq \module z + \module{z^\prime}$.
   \end{enumerate}
\end{proposition}

\begin{lemma}   \label{LEMooONLNooXLNbtB}
    Pour tout \( z\in \eC\) nous avons \( z\bar z=\bar z z=| z |^2\).
\end{lemma}


\chapter{Théorie des groupes}
% This is part of Mes notes de mathématique
% Copyright (c) 2011-2019
%   Laurent Claessens
% See the file fdl-1.3.txt for copying conditions.

Pour rappel, la notion de groupe est définie en \ref{DEFooBMUZooLAfbeM}.

%+++++++++++++++++++++++++++++++++++++++++++++++++++++++++++++++++++++++++++++++++++++++++++++++++++++++++++++++++++++++++++
\section{Groupes}
%+++++++++++++++++++++++++++++++++++++++++++++++++++++++++++++++++++++++++++++++++++++++++++++++++++++++++++++++++++++++++++

\begin{definition}[\cite{Kropholler}]
    Soit \( G\) un groupe. Le \defe{centralisateur}{centralisateur} de \( H\subset G\) est
    \begin{equation}
        \mZ_G(H)=\{ g\in G\tq hg=gh\,\forall h\in H\};
    \end{equation}
    il contient donc tous les éléments de \( G\) qui commutent avec ceux de \( H\).

    Si \( H\) est un sous-groupe, son \defe{normalisateur}{normalisateur} est
    \begin{equation}
        N_G(H)=\{ g\in G\tq gH=Hg \}.
    \end{equation}
\end{definition}

\begin{definition}\label{defGroupeCentre}
Le \defe{centre}{centre!d'un groupe} d'un groupe \( G\) est l'ensemble des éléments de \( G\) qui commutent avec tous les autres:
\begin{equation}
    Z_G=\{ z\in G\tq gz=zg\;\forall g\in G \}.
\end{equation}
Si \( g\in G\) nous notons \( Z_G(g)\) le \defe{centralisateur}{centralisateur} de \( g\) dans \( G\) :
\begin{equation}
    Z_G(g)=\{h\in G\tq hg=gh\}.
\end{equation}
C'est l'ensemble des éléments de \( G\) qui commutent avec \( g\).


\end{definition}

\begin{definition}\label{DEFooNIIMooFkZgvX}
    Un sous-groupe \( N\) de \( G\) est \defe{normal}{normal!sous-groupe} ou \defe{distingué}{distingué!sous-groupe} si pour tout \( g\in G\) et pour tout \( n\in N\), \( gng^{-1}\in N\). Autrement dit lorsque \( gNg^{-1}\subset N\).

    Lorsque \( N\) est normal dans \( G\) il est parfois noté \( N\normal G\)\nomenclature[]{\(N \normal G\)}{Le sous-groupe \( N\) est normal dans \( G\)}.
\end{definition}

\begin{definition}      \label{DEFooUXXTooCCLmQe}
    Un sous-groupe \( H\) de \( G\) est un sous-groupe \defe{caractéristique}{sous-groupe!caractéristique}\index{caractéristique!sous-groupe} si \( \alpha(H)=H\) pour tout \( \alpha\in \Aut(G)\).
\end{definition}

\begin{definition}[Groupe simple]   \label{DefGroupeSimple}
    Un groupe est dit \defe{simple}{simple!groupe} si il est non trivial et si les seuls sous-groupes normaux qu'il admet sont lui-même et le sous-groupe réduit à l'identité.
\end{definition}

\begin{definition}[Sous-groupe engendré]        \label{DefooRDRXooEhVxxu}
    Soit \( A\) une partie du groupe \( G\). Le sous-groupe \defe{engendré}{sous-groupe!engendré}\index{engendré!sous-groupe} par \( A\) est l'intersection de tous les sous-groupes de \( G\) contenant \( A\). Nous notons ce groupe \( \gr(A)\)\nomenclature[R]{\( \gr\)}{groupe engendré}.

    Lorsque \( A \) est fini (disons \( A = \{a_1, \dots, a_n\} \)), on note aussi le sous-groupe engendré \( \langle a_1, \dots, a_n \rangle \).
\end{definition}

Un sous-groupe engendré n'est jamais vide parce qu'il contient toujours au moins le neutre (parce que c'est un sous-groupe). Si \( G\) est un groupe, le sous-groupe \( \gr(\emptyset)\) lui-même contient \( e\)\footnote{Demandez-vous si il est possible que \( \gr(\emptyset)\) contienne d'autres éléments que \( e\).}.

\begin{lemma}
    Si \( G\) est un groupe et \( A\) une partie de \( G\), alors \( \gr(A)\) est un sous-groupe de \( G\).
\end{lemma}

Le sous-groupe engendré par \( A \) est le plus petit (pour l'inclusion) groupe de \( G\) contenant \( A\). Plus formellement, nous avons le résultat suivant.
\begin{lemma}
    Tout sous-groupe de \( G\) contenant \( A\) contient \( \gr(A)\).
\end{lemma}

\begin{proof}
    Si \( H\) est un sous-groupe de \( G\) contenant \( A\), alors \( \gr(A)\) est l'intersection de \( H\) avec tous les autres sous-groupes de \( G\) contenant \( A\). Il contient donc \( \gr(A)\).
\end{proof}

\begin{lemma}[\cite{BIBooERNQooTXQPvD}]   \label{LemFUIZooBZTCiy}
    Si \( A\) est une partie du groupe \( G\), alors le sous-groupe \( \gr(A)\) engendré\footnote{Définition~\ref{DefooRDRXooEhVxxu}.} par \( A\) est l'ensemble de tous les produits finis d'éléments de \( A\) et de \( A^{-1}\) (l'identité est le produit à zéro éléments).

    C'est-à-dire que tout élément de \( \gr(A)\) peut être écrit sous la forme
    \begin{equation}
        \prod_{i=1}^ng_i^{a_i}
    \end{equation}
    où \( a_i\in \eZ\) et \( g\colon \eN\to A\) n'est pas spécialement injective : il peut arriver que \( g_i=g_j\).
\end{lemma}

\begin{proof}
    Nous nommons \( \gr(A)\) le groupe engendré par \( A\) et par \( H\) l'ensemble
    \begin{equation}
        H=\{ g_1\ldots g_n\tq g_i\in A\cup A^{-1} \}.
    \end{equation}
    Nous commençons par prouver que \( H\) est un groupe.
    \begin{itemize}
        \item Vu que \( A\) est non vide, nous considérons \( a\in A\). Dans ce cas, \( e=aa^{-1}\in H\). Donc \( e\in H\).
        \item L'inverse de \( g_1\ldots g_n\) est \( g_n^{-1}\ldots g_1^{-1}\) qui est également dans \( H\).
        \item Le produit de \( g_1\ldots g_n\) par \( h_1\ldots h_n\) est également dans \( H\)\footnote{Et c'est ici qu'on se rend compte que la décomposition n'est probablement que rarement unique.}.
    \end{itemize}
    Vu que \( H\) est un groupe contenant \( A\), nous avons \( \gr(A)\subset H\) parce que \( \gr(A)\) est une intersection dont un des éléments est \( H\).

    Par ailleurs tout groupe contenant \( A\) doit contenir les inverses et les produits finis, donc \( H\subset \gr(A)\).

    Au final, \( H=\gr(A)\), ce qu'il fallait.
\end{proof}

\begin{lemma}       \label{LEMooCFTVooKvmyKN}
    Soit un groupe \( G\) et un sous-groupe \( H=\gr(h_1,\ldots, h_n)\). Si \( \alpha\in G\), alors
    \begin{equation}
        \alpha H\alpha^{-1}=\gr(\alpha h_1\alpha^{-1},\ldots, \alpha h_n\alpha^{-1}).
    \end{equation}
\end{lemma}

\begin{proof}
    Il s'agit d'une conséquence du lemme \ref{LemFUIZooBZTCiy}. Un élément de \( \gr(\alpha h_1\alpha^{-1},\ldots, \alpha h_n\alpha^{-1})\) est un produit d'éléments de \( G\) de la forme \( \alpha h_i\alpha^{-1}\) ou \( (\alpha h_j\alpha^{-1})^{-1}=\alpha h_j^{-1}\alpha^{-1}\). Or nous avons
    \begin{equation}
        \alpha h_i\alpha^{-1}\alpha h_j\alpha^{-1}=\alpha h_ih_j\alpha^{-1}\in \alpha H\alpha^{-1}.
    \end{equation}
    Donc 
    \begin{equation}
        \gr(\alpha h_1\alpha^{-1},\ldots, \alpha h_n\alpha^{-1})\subset \alpha H\alpha^{-1}.
    \end{equation}
    L'inclusion dans l'autre sens est du même tonneau.
\end{proof}

\begin{definition}[Partie génératrice, groupe monogène]  \label{DEFooWMFVooLDqVxR}
    Soit \( G\), un groupe et \( A\) une partie de \( G\). Si \( \gr(A)=G\), alors nous disons que \( A\) est une \defe{partie génératrice}{partie génératrice} le groupe \( G\).

    Un groupe est \defe{monogène}{monogène} s'il a une partie génératrice réduite à un seul élément.
\end{definition}

\begin{definition}[Groupe cyclique]     \label{DefHFJWooFxkzCF}
    Un élément \( a\in G\) est un \defe{générateur}{générateur} de \( G\) si tous les éléments de \( G\) s'écrivent sous la forme \( a^n\) pour un certain \( n\in\eZ\). Un groupe fini et monogène est dit \defe{cyclique}{groupe cyclique}.
\end{definition}

\begin{definition}
    Soit le groupe \( \big( \eZ/10\eZ,+ \big)\). L'élément \( [2]_{10}\) n'est pas générateur parce que ses puissances\footnote{Attention aux notations; en général on écrit la loi de groupe de façon multiplicative et on parle des puissances d'un élément, mais ici on écrit la loi de groupe additivement, donc les «puissances» sont en réalité les multiples.} sont
    \begin{equation}
        \gr([2]_{10})=\{ [2]_{10},[4]_{10},[6]_{10},[8]_{10},[0]_{10} \}.
    \end{equation}
    Par contré l'élément \( [3]_{10}\) est générateur : ses puissances sont dans l'ordre
    \begin{equation}
        [3]_{10}, [6]_{10}, [9]_{10}, [2]_{10}, [5]_{10}, [8]_{10},[1]_{10},[4]_{10},[7]_{10},[0]_{10}.
    \end{equation}
\end{definition}

Un exemple presque identique, mais un peu masqué sera l'exemple \ref{EXooOXAAooZMdDfP}.

%+++++++++++++++++++++++++++++++++++++++++++++++++++++++++++++++++++++++++++++++++++++++++++++++++++++++++++++++++++++++++++
\section{Sous-groupe normal}
%+++++++++++++++++++++++++++++++++++++++++++++++++++++++++++++++++++++++++++++++++++++++++++++++++++++++++++++++++++++++++++

\begin{proposition}\label{propGroupeNormal}
    Soit \( N\) un sous-groupe de \( G\). Les propriétés suivantes sont équivalentes :
    \begin{enumerate}
        \item
            \( gNg^{-1}\subseteq N\) pour tout \( g\in G\),
        \item
            \( gNg^{-1}= N\) pour tout \( g\in G\),
        \item
            \( gN=Ng\) pour tout \( g\in G\),
        \item
            \( N\) est une union de classes de conjugaison de \( G\),
        \item
            \( N\) est normal\footnote{Définition \ref{DEFooNIIMooFkZgvX}.}.
    \end{enumerate}
\end{proposition}

\begin{definition}
    Soit \( g\in G\) et \( n\in \eZ\). Nous définissons \( g^n\) par
    \begin{enumerate}
        \item
            \( g^0=e\) et \( g^n=gg^{n-1}\) si \( n\) est positif.
        \item
            si \( n<0\), nous posons \( g^n=(g^{-1})^{-n}\).
    \end{enumerate}
\end{definition}

\begin{definition}[Ordre d'un groupe et d'un élément]       \label{DEFooKWBCooMlmpCP}
    Ce sont deux choses différentes.
    \begin{enumerate}
        \item

    Si \( G\) est un groupe, l'\defe{ordre}{ordre!d'un groupe} est la cardinalité de \( G\) et est noté \( | G |\).
\item

    L'\defe{ordre}{ordre!élément} d'un élément \( g\) de \( G\) est le naturel
    \begin{equation}
        \min\{ n\in\eN\tq g^n=e \},
    \end{equation}
    s'il existe; dans le cas contraire, nous disons que l'ordre de \( g\) est infini.
    \end{enumerate}
\end{definition}
Nous verrons que le corolaire~\ref{CorpZItFX} au théorème de Lagrange dira que l'ordre d'un élément divise l'ordre du groupe.

\begin{lemma}[\cite{PDFpersoWanadoo}]\label{LemHUkMxp}
    Si \( H\) et \( K\) sont normaux dans le groupe \( G\) et si \( H\cap K=\{ e \}\) alors \( HK\simeq H\times K\).
\end{lemma}

\begin{definition}  \label{DefvtSAyb}
    L'\defe{exposant}{exposant!d'un groupe} du groupe \( G\) est le plus petit entier non nul \( n\) tel que \( g^n=e\) pour tout \( g\in G\). S'il n'existe pas, nous disons que l'exposant du groupe est infini.
\end{definition}
Si l'ordre de tous les éléments acceptent un majorant commun, alors l'exposant du groupe est le plus petit commun multiple des ordres des éléments. En particulier pour un groupe fini, l'exposant est le $\ppcm$ des ordres des éléments du groupe.

Le théorème de Burnside~\ref{ThooJLTit} nous donnera un bon paquet d'exemples de groupes d'exposant fini dans \( \GL(n,\eC)\).

\begin{proposition} \label{PropSRMJooIDPBoW}
    Soit \( H\) un sous-groupe normal de \( G\) et \( \psi\colon G\to K\) un homomorphisme.
    \begin{enumerate}
        \item
            \( \psi(H)\) est normal dans \( \psi(G)\)
        \item
            Si \( G/H\) est abélien alors \( \psi(G)/\psi(H)\) est abélien.
    \end{enumerate}
\end{proposition}

\begin{proof}
    Soient \( h\in H\) et \( g\in G\). Alors \( \psi(g)\psi(h)\psi(g)^{-1}=\psi(ghg^{-1})\in\psi(H)\). Donc \( \psi(H)\) est normal dans \( \psi(G)\).

    Pour la seconde partie nous notons \( [\ldots]\) les classes par rapport à \( \psi(H)\) et \( \overline{ \vphantom{g}\ldots }\) celles par rapport à \( H\). Nous avons
    \begin{subequations}
        \begin{align}
            [\psi(g_1)][\psi(g_2)]&=\big[ \psi(g_1)\psi(g_2) \big]\\
            &=\big[ \psi(g_1g_2) \big]\\
            &=\{ \psi(g_1g_2)\psi(h)\tq h\in H \}\\
            &=\{ \psi(g_1g_2h)\tq h\in H \}\\
            &=\psi\Big(  \{ g_1g_2h\tq h\in H \}  \Big) \\
            &=\psi\big( \overline{ g_1g_2 } \big)\\
            &=\psi(\overline{ g_2g_1 })\\
            &=\text{refaire à l'envers}\\
            &=[\psi(g_2)][\psi(g_1)].
        \end{align}
    \end{subequations}
    Par conséquent \( \psi(G)/\psi(H)\) est abélien.
\end{proof}

%---------------------------------------------------------------------------------------------------------------------------
\subsection{Classes de conjugaison}
%---------------------------------------------------------------------------------------------------------------------------

\begin{definition}
    Soit un groupe \( G\) et un élément \( g\in G\). La \defe{classe de conjugaison}{classe!de conjugaison} de \( g\) est la partie
    \begin{equation}
        C_g=\{ kgk^{-1}\tq k\in G \}.
    \end{equation}
\end{definition}

\begin{lemma}       \label{LEMooQYBJooYwMwGM}
    Un groupe est abélien si et seulement si ses classes de conjugaison sont des singletons.
\end{lemma}

\begin{proof}
    Supposons que \( G\) soit abélien. Alors
    \begin{equation}
        C_g=\{ kgk^{-1}\tq k\in G \}=\{ g \}.
    \end{equation}
    Donc les classes de conjugaison sont des singletons.

    Dans l'autre sens, si les classes sont des singletons, on a \( kgk^{-1}=g\) pour tous \( k,g\in G\). Cela signifie immédiatement que \( G\) est abélien.
\end{proof}

%+++++++++++++++++++++++++++++++++++++++++++++++++++++++++++++++++++++++++++++++++++++++++++++++++++++++++++++++++++++++++++
\section{Groupe dérivé}
%+++++++++++++++++++++++++++++++++++++++++++++++++++++++++++++++++++++++++++++++++++++++++++++++++++++++++++++++++++++++++++

\begin{definition}
    Si \( G\) est un groupe et si \( g,h\in G\), nous notons \( [g,h]=ghg^{-1}h^{-1}\)\nomenclature[R]{\( [g,h]\)}{commutateur dans un groupe} le \defe{commutateur}{commutateur!dans un groupe} de \( g\) et \( h\). 
\end{definition}

L'élément neutre est toujours un commutateur : pour \( g=h \), \( [g,g]=ggg^{-1}g^{-1}=e \).

\begin{definition}      \label{DEFooBNLPooShKYXa}
    Le \defe{groupe dérivé}{dérivé!groupe}\index{groupe!dérivé} de \( G\) est le sous-groupe noté \( D(G)\)\nomenclature[R]{\( D(G)\)}{groupe dérivé} ou \( [G,G]\)\nomenclature[R]{\( [G,G]\)}{groupe dérivé} engendré\footnote{Définition \ref{DefooRDRXooEhVxxu}.} par les commutateurs.
\end{definition}
Autrement dit, \( D(G)\) est l'intersection de tous les sous-groupes de \( G\) contenant tous les commutateurs. Le groupe \( D(G)\) contient toujours au moins le neutre parce que c'est un groupe.

En vertu du lemme~\ref{LemFUIZooBZTCiy}, le groupe dérivé de \( G\) est l'ensemble des produits finis de commutateurs. C'est-à-dire que si \( S_m\) est l'ensemble des produits de \( m\) commutateurs, alors
\begin{equation}
    D(G)=\bigcup_{m=1}^{\infty}S_m.
\end{equation}

\begin{lemma}   \label{LemMMOCooDJJJhy}
    Le groupe dérivé est un sous-groupe caractéristique\footnote{Définition \ref{DEFooUXXTooCCLmQe}.}, et un sous-groupe normal\footnote{Définition \ref{DEFooNIIMooFkZgvX}.}.
\end{lemma}

\begin{proof}
    Il est évident que si \( \alpha\in\Aut(G)\) alors
    \begin{equation}
        \alpha\big( [g,h] \big)=\big[ \alpha(g),\alpha(h) \big],
    \end{equation}
    c'est-à-dire que \( D(G)\) est un sous-groupe caractéristique. En particulier si \( c\) est un commutateur, alors \( xcx^{-1}\) en est encore un, ce qui montre que \( D(G)\) est normal dans \( G\). Plus spécifiquement,
    \begin{subequations}
        \begin{align}
        x(ghg^{-1}h^{-1})x^{-1}&=(xgx^{-1})(xhx^{-1})(xg^{-1}x^{-1})(xh^{-1}x^{-1})\\
        &=(xgx^{-1})(xhx^{-1})(xgx^{-1})^{-1}(xhx^{-1})^{-1}.
        \end{align}
    \end{subequations}
\end{proof}

\begin{proposition}\label{PropAPRGooHBkELf}
    Le groupe quotient \( G/D(G)\) est abélien.
\end{proposition}

\begin{proof}
    En ce qui concerne le fait que \( G/D(G)\) soit abélien, nous savons que pour tout \( g,h\in G\) nous avons \( h^{-1}g^{-1}hg\in D(G)\) et donc
    \begin{equation}
        [g][h]=[gh]=[ghh^{-1}g^{-1}hg]=[hg]=[h][g].
    \end{equation}
\end{proof}

Le groupe quotient \( G/D(G)\) est appelé l'\defe{abélianisé}{abélianisé} de \( G\) et est parfois noté \( G^{ab}\)\nomenclature[R]{\( G^{ab}\)}{groupe abélianisé de \( G\)}.

Si \( f\colon G\to A\) est un homomorphisme entre le groupe \( G\) et un groupe abélien \( A\), alors \( f\big( D(G) \big)=\{ 0 \}\). Du coup \( f\) passe au quotient de \( G\) par \( D(G)\), et il existe une unique application \( \bar f\colon G/D(G)\to A\) telle que \( f=\bar f\circ \pi\) où \( \pi\colon G\to G/D(G)\) est la projection canonique.

%+++++++++++++++++++++++++++++++++++++++++++++++++++++++++++++++++++++++++++++++++++++++++++++++++++++++++++++++++++++++++++
\section{Théorèmes d'isomorphismes}
%+++++++++++++++++++++++++++++++++++++++++++++++++++++++++++++++++++++++++++++++++++++++++++++++++++++++++++++++++++++++++++

\begin{definition}      \label{DEFooWBIYooGNRYOp}
    Soient un groupe \( G\), un ensemble $X$ et une application \( f\colon X\to G\). Le \defe{noyau}{noyau!application vers un groupe} de \( f\) est la partie
    \begin{equation}
        \ker(f)=\{ x\in X\tq f(x)=e \}
    \end{equation}
    où \( e\) est l'unité de \( G\).
\end{definition}

Si \( G\) est un groupe et si \( N\) est un sous-groupe normal, alors l'ensemble \( G/N\) a une structure de groupe et la projection canonique \( \pi\colon G\to G/N\) est un homomorphisme surjectif de noyau~\( N\).

\begin{theorem}[Premier théorème d'isomorphisme]        \label{ThoPremierthoisomo}
    Soit \( \theta\colon G\to H\) un homomorphisme de groupe. Alors
    \begin{enumerate}
        \item
            \( \Kernel\theta\) est normal dans \( G\),
        \item
            \( \Image \theta\) est un sous-groupe de \( H\)
        \item   \label{ItemWLCLdk}
            nous avons un isomorphisme naturel
            \begin{equation}
                G/\Kernel\theta\simeq \Image\theta
            \end{equation}
    \end{enumerate}
\end{theorem}
\index{théorème!isomorphisme!premier!pour les groupes}

\begin{proof}
    Point par point.
    \begin{enumerate}
        \item
            Le fait que  \( \Kernel\theta\) soit un sous-groupe de \( G\) est clair; montrons qu'il est normal. Si \( g \in G \) et \( u \in \Kernel\theta\), alors \(\theta (g^{-1} u g) = \theta(g^{-1})\theta(u)\theta(g) = \bigl(\theta(g)\bigr)^{-1}\theta(g) = 1_H \), et donc \( g^{-1} u g \in \Kernel\theta\).
        \item
            Il suffit de remarquer que si \( h = \theta(g) \) et \( h' = \theta(g') \), alors \( h^{-1} h' = \theta(g^{-1} g') \).
        \item
            Si \( [g]\) représente la classe de \( g\) dans \( G/\Kernel\theta\), l'isomorphisme est donné par \( \varphi[g]=\theta(g)\).
    \end{enumerate}
\end{proof}

\ifbool{isGiulietta}{
Ce premier théorème d'isomorphismes permet entre autres de prouver que $SO(3)=\SU(2)/\eZ_2$, voir la proposition \ref{PROPooDKPTooBnLflt}.
}{}
\begin{theorem}[Deuxième théorème d'isomorphisme]
    Soient \( H\) et \( N\) deux sous-groupes de \( G\) et supposons que \( N\) soit normal. Alors
    \begin{enumerate}
        \item
            \( NH=HN\) est un sous-groupe.
        \item
            Le groupe \( N\) est normal dans \( NH\).
        \item
            Le groupe \( N\cap H\) est normal dans \( H\).

        \item\label{ItemjRPajc}
            Nous avons l'isomorphisme
            \begin{equation}
                \frac{ HN }{ N }\simeq\frac{ H }{ H\cap N }.
            \end{equation}
        \item   \label{ItembgDQEN}
            L'isomorphisme du point~\ref{ItemjRPajc} est encore valable si \( N\) n'est pas normal mais si seulement \( H\) normalise \( N\), c'est-à-dire si \( hNh^{-1}\in N\) pour tout \( h\in H\).
    \end{enumerate}
\end{theorem}
\index{théorème!isomorphisme!second}
%TODO : trouver une démonstration du dernier point.

\begin{proof}
    Point par point.
    \begin{enumerate}
        \item
            Il est clair que \( 1_G \in NH \). Soient $nh$ et $n'h'$ deux éléments de \( NH \); alors en tenant compte du fait que \( N\) est normal,
            \begin{equation}
                nhn'h'=n\underbrace{hn'h^{-1}}_{\in N}hh'\in NH.
            \end{equation}
            Cela prouve que \( NH\) est un groupe.

            De la même façon, nous prouvons que \( HN\) est un groupe par
            \begin{equation}
                hnh'n'=hh'\underbrace{h'^{-1}nh'}_{\in N}n'\in HN
            \end{equation}

            Nous devons encore prouver que \( HN=NH\). Pour cela, \( nh \in HN \), car \( nh = hh^{-1}nh \), les trois derniers facteurs formant un  élément de \( N \) par normalité; de même \( hn \in NH \), montrant que \( NH = HN \). Enfin, comme \( (nh)^{-1} = h^{-1} n^{-1}) \), les inverses de \( NH \) sont dans \( HN = NH \).
        \item
            \( N\) est normal dans \( G \), a fortiori dans l'un de ses sous-groupes.
        \item
            Il suffit de voir que, si \( h \in H \) et \( n \in N \cap H \), alors \( hnh^{-1} \in N \cap H \). Or, \( hnh^{-1} \in H \) puisque \( H\) est un sous-groupe; et \( hnh^{-1} \in N \) car \( N \) est un sous-groupe normal de \( G \).
        \item
            Il faut d'abord remarquer que \( H\) et \( N\) étant des groupes et le produit \( NH\) étant un groupe, nous avons \( NH=HN\). Soit le morphisme injectif
            \begin{equation}
                \begin{aligned}
                    j\colon H&\to HN \\
                    h&\mapsto h
                \end{aligned}
            \end{equation}
            et la surjection canonique
            \begin{equation}
                \sigma\colon HN\to HN/N
            \end{equation}
            Nous considérons ensuite l'application composée
            \begin{equation}
                \begin{aligned}
                    f\colon H&\to HN/N \\
                    h&\mapsto hN.
                \end{aligned}
            \end{equation}

            \begin{subproof}
                \item[\( f\) est surjective]

                    L'application \( f\) est surjective parce que l'élément \( hnN\in HN/N\) est l'image de \( h\), étant donné que \( hnN=hN\).

                \item[\( \Kernel(f)=H\cap N\)]

            Si \( a\in H\cap N\), nous avons \( f(a) =aN = N\), et donc \( H\cap N\subset \Kernel f\). D'autre part, si \( h\in H\) vérifie \( h\in\Kernel f\), alors \( f(h)=hN=N\), ce qui est uniquement possible si \( h\in N\).

            \end{subproof}
            Le premier théorème d'isomorphisme implique alors que \( H/\Kernel f\simeq \Image f\), c'est-à-dire
            \begin{equation}
                H/N\cap H\simeq HN/N.
            \end{equation}
    \end{enumerate}
\end{proof}

\begin{theorem}[Troisième théorème d'isomorphisme]  \label{ThoezgBep}
    Soient \( N\) et \( M\) deux sous-groupes normaux de \( G\) avec \( M\subset N\). Alors \( N/M\) est normal dans \( G/M\) et
    \begin{equation}
        (G/M)/(N/M)\simeq G/N.
    \end{equation}
\end{theorem}
\index{théorème!isomorphisme!troisième}

\begin{proof}
    Afin de montrer que \( N/M\) est normal dans \( G/M\), nous considérons \( g\in G\), \( nM\in N/M\) et nous calculons
    \begin{equation}
        gnMg^{-1}=gng^{-1}\underbrace{gMg^{-1}}_{=M}=\underbrace{gng^{-1}}_{\in N}M\in N/M.
    \end{equation}

    Pour prouver l'isomorphisme nous considérons le morphisme
    \begin{equation}
        \begin{aligned}
            \varphi\colon G/M&\to G/N \\
            gM&\mapsto gN.
        \end{aligned}
    \end{equation}
    C'est surjectif et le noyau est \( N/M\) parce que \( \varphi(gM)=N\) uniquement si \( g\in N\). Nous pouvons appliquer le premier théorème d'isomorphisme à \( \varphi\) en écrivant
    \begin{equation}
        (G/M)/\Kernel \varphi\simeq\Image \varphi,
    \end{equation}
    c'est-à-dire
    \begin{equation}
        (G/M)/(N/M)\simeq G/N.
    \end{equation}
\end{proof}

%+++++++++++++++++++++++++++++++++++++++++++++++++++++++++++++++++++++++++++++++++++++++++++++++++++++++++++++++++++++++++++
\section{Indice d'un sous-groupe et ordre des éléments}
%+++++++++++++++++++++++++++++++++++++++++++++++++++++++++++++++++++++++++++++++++++++++++++++++++++++++++++++++++++++++++++

\begin{lemma}       \label{LEMooFNVRooRCkjLc}
    Lorsque \( H\) est normal dans \( G\), alors la définition
    \begin{equation}        \label{EQooEUESooSeUWHK}
        [a]\cdot[b]=[ab]
    \end{equation}
    définit une loi de groupe sur l'ensemble \( G/H\).
\end{lemma}

\begin{proof}
    Le neutre est \( [e]\) et l'associativité ne pose pas plus de problèmes que l'existence d'un inverse. Le point à vérifier est que la formule \eqref{EQooEUESooSeUWHK} est une bonne définition : \( [ah]\cdot [bh']=[ab]\) pour tout \( h,h'\in H\). Nous avons :
    \begin{equation}
        [ah]\cdot [ah']=[ahah']=[ahb].
    \end{equation}
    Pour montrer que cela est \( [ab]\), l'astuce est d'introduire \( bb^{-1}\) à côté du \( a\) :
    \begin{equation}
        [ahb]=[abb^{-1}hb]=[ab]
    \end{equation}
    parce que \( b^{-1} hb\in H\) du fait que \( H\) soit normal dans \( G\).
\end{proof}

\begin{example}[\cite{ooTVPEooGOLEom}]      \label{EXooFNIKooHxePSs}
    Il ne faudrait pas croire que le groupe quotient \( G/H\) est forcément un sous-groupe de \( G\). Par exemple le quotient \( \eZ/2\eZ\) est l'ensemble \( \{ 0,1 \}\) muni de l'addition. En particulier \( 1+1=0\), ce qui est évidemment faux dans \( \eZ\). Le groupe \( (\eZ,+)\) ne possède aucun élément d'ordre \( 2\).

    Il n'en est pas moins vrai que l'application
    \begin{equation}
        \begin{aligned}
            f\colon G&\to G/H \\
            g&\mapsto [g]
        \end{aligned}
    \end{equation}
    est un morphisme de groupes.
\end{example}

\begin{definition}      \label{DEFooMPIAooIeZNaR}
    Si \( H\) est un sous-groupe d'un groupe fini l'\defe{indice}{indice} de \( H\) dans \( G\) est le nombre \( | G |/| H |\), souvent noté \( | G:H |\).
\end{definition}

Le théorème de Lagrange dira en particulier que l'indice est toujours un nombre entier. C'est à ne pas confondre avec le degré d'une extension de corps (définition~\ref{DefUYiyieu}).


\begin{theorem}[Théorème de Lagrange]    \label{ThoLagrange}
    Soit \( H\) un sous-groupe du groupe fini \( G\).  Alors
    \begin{enumerate}
        \item   \label{ITEMooDPKSooNpOusd}
    L'ordre de \( H\) divise l'ordre de \( G\).
\item
    Les trois nombres suivants sont égaux :
    \begin{itemize}
        \item
            le nombre de classes de \( H\) à gauche,
        \item
            le nombre de classes de \( H\) à droite,
        \item
            l'indice de \( H\) dans \( G\).
    \end{itemize}
    \end{enumerate}
    En particulier si \( H\) est distingué dans \( G\) nous avons
    \begin{equation}
        | G/H |=\frac{ | G | }{ | H | }.
    \end{equation}
\end{theorem}
\index{théorème!Lagrange}
\index{Lagrange!théorème}

\begin{proof}
    Nous commençons par montrer que les classes de \( H\) ont toutes le même nombre d'éléments que \( H\). En effet pour chaque \( g\in G\) nous avons la bijection
    \begin{equation}
        \begin{aligned}
            \varphi\colon H&\to gH \\
            h&\mapsto gh.
        \end{aligned}
    \end{equation}
    L'injectivité de \( \varphi\) est le fait que \( gh=gh'\) implique \( h=h'\). La surjectivité est par définition de la classe.

    Les classes à gauche formant une partition de \( G\), le cardinal de \( G\) est le produit de la taille des classes par le nombre de classes :
    \begin{equation}
        | G |=| H |\cdot\text{nombre de classes}.
    \end{equation}
    En particulier nous voyons que \( | H |\) divise \( | G |\).

    La dernière formule exprime simplement que \( G/H\) est par définition le nombre de classes de \( H\) à gauche (ou à droite) dans \( G\).
\end{proof}

\begin{corollary}       \label{CorpZItFX}
    L'ordre d'un élément d'un groupe fini divise l'ordre du groupe. En particulier dans un groupe d'ordre \( n\) tous les éléments vérifient \( q^n=e\).
\end{corollary}

\begin{proof}
    Soit \( G\) un groupe fini et considérons, à \( g \in G \) fixé, le sous-groupe
    \begin{equation}
        H=\{ g^k\tq k\in\eN \}.
    \end{equation}
    Par le théorème de Lagrange~\ref{ThoLagrange}, l'ordre de \( H\) divise \( | G |\), mais l'ordre de \( H\) est le plus petit \( k\) tel que \( g^k=e\), c'est-à-dire l'ordre de \( g\).
\end{proof}

D'autres résultats à propos d'ordres et d'indices de groupes finis dans la proposition \ref{PROPooVWVIooQzuAlA} et le lemme \ref{LemqAUBYn}. En particulier le théorème de Cauchy \ref{THOooSUWKooICbzqM} qui dit si \( p\) divise l'ordre du groupe \( G\), alors \( G\) contient au moins un élément d'ordre \( p\).

%+++++++++++++++++++++++++++++++++++++++++++++++++++++++++++++++++++++++++++++++++++++++++++++++++++++++++++++++++++++++++++
\section{Suite de composition}
%+++++++++++++++++++++++++++++++++++++++++++++++++++++++++++++++++++++++++++++++++++++++++++++++++++++++++++++++++++++++++++
\index{sous-groupe!normal}\index{groupe!quotient}\index{quotient!de groupe}

%TODO : citer la page de la wikiversité sur Jordan-Hölder.
%TODO : donner la définition d'un raffinement de suite de composition.

\begin{definition}  \label{DefJWZSooNcntfK}
Une \defe{suite de composition}{composition!suite de}\index{suite de composition} pour un groupe \( G\) est une suite finie de sous-groupes \( (G_i)_{i=0,\ldots, n}\) telle que
\begin{equation}
    \{ e \}=G_n\subseteq G_{n-1}\subseteq\ldots\subseteq G_1\subseteq G_0=G
\end{equation}
et telle que \( G_{i+1}\) est normal\footnote{Nous rappelons au cas où que «normal» signifie «distingué».} dans \( G_i\). Les groupes \( G_i/G_{i+1}\) sont les \defe{quotients}{quotient!dans une suite de composition} de la suite de composition.

    Une suite de \defe{Jordan-Hölder}{suite!de Jordan-Hölder}\index{Jordan-Hölder} est une suite de composition dont tous les quotients sont simples.
\end{definition}
L'objet de nos prochaines pérégrinations mathématiques est de montrer que tout groupe fini admet une suite de Jordan-Hölder (théorème~\ref{ThoLgxWIC}).

\begin{lemma}[du papillon\cite{NjCCfW}]\label{LemsKpXCG}
    Soit \( G\) un groupe et des sous-groupes \( A\) et \( B\). Soit \( A'\) normal dans \( A\) et \( B'\) normal dans \( B\). Alors
    \begin{enumerate}
        \item
            \( A'(A\cap B')\) est normal dans \( A'(A\cap B)\)
        \item
            \( (A'\cap B)B'\) est normal dans \( (A\cap B)B'\)
        \item
            Nous avons les isomorphismes de groupes
            \begin{equation}
                \frac{ A'(A\cap B) }{ A'(A\cap B') }\simeq\frac{ (A\cap B)B' }{ (A'\cap B)B' }\simeq\frac{ B'(A\cap B) }{ B'(A'\cap B) }.
            \end{equation}
    \end{enumerate}
\end{lemma}

\begin{proof}
    Nous n'allons pas démontrer chacun des points; pour plus de détails, nous dirons simplement que «la preuve est très similaire dans les autres cas».

    Commençons par montrer que \( A'(A\cap B')\) est un groupe. Si \( a,b\in A'\) et \( x,y\in A\cap B'\),
    \begin{equation}
        axby=xx^{-1}axbx^{-1}xy
    \end{equation}
    En utilisant la normalité, \( x^{-1}ax\in A'\), donc \( xx^{-1}axbx^{-1}\in A'\) et donc le tout est dans \( A'(A\cap B')\). L'ensemble \( A'(A\cap B')\) est également stable pour l'inverse parce que
    \begin{equation}
        x^{-1}a^{-1}=\underbrace{x^{-1}a^{-1}x}_{\in A'}x^{-1}.
    \end{equation}

    Nous montrons maintenant que \( A'(A\cap B')\) est normal dans \( A'(A\cap B)\). Soient \( a,b\in A'\), \( x\in A\cap B'\) et \( f\in A\cap B\). Alors
    \begin{subequations}
        \begin{align}
        (bf)^{-1}(ax)(bf)&=(bf)^{-1}(a\underbrace{xbx^{-1}}_{=c\in A'}xf)\\
        &=f^{-1}b^{-1}acxf\\
        &=f^{-1}b^{-1}acf\underbrace{f^{-1}xf}_{=y\in A\cap B'}\\
        &=\underbrace{f^{-1}b^{-1}acf}_{\in A'}y\\
        &\in A'(A\cap B').
        \end{align}
    \end{subequations}

    Pour prouver l'isomorphisme
    \begin{equation}
        \frac{ A'(A\cap B) }{ A'(A\cap B') }=\frac{ (A\cap B)B' }{ (A'\cap B)B' },
    \end{equation}
    nous allons utiliser le deuxième théorème d'isomorphisme (\ref{ThoezgBep}\ref{ItembgDQEN}). Que nous appliquons à \( H=A\cap B\) et \( N=A'(A\cap B')\). La vérification que \( H\) normalise \( N\) est usuelle. Nous commençons par écrire
    \begin{equation}    \label{EqkphNsE}
        \frac{ A'(A\cap B')(A\cap B) }{ A'(A\cap B') }\simeq\frac{ A\cap B }{ A\cap B\cap A'(A\cap B') }.
    \end{equation}
    Pour simplifier un peu cette expression nous prouvons d'abord que
    \begin{equation}    \label{EqkhsyNh}
        (A\cap B)\cap A'(A\cap B')=(A'\cap B)(A\cap B').
    \end{equation}
    L'inclusion \( \supset\) est facile. Pour l'autre sens, étant donné que \( A'(A\cap B')\subset A\) nous avons
    \begin{equation}
        A\cap B\cap A'(A\cap B)=B\cap A'(A\cap B).
    \end{equation}
    Un élément de \( B\cap A'(A\cap B)\) est un élément de \(   B\) qui s'écrit sous la forme \( s=ax\) avec \( a\in A'\) et \( x\in A\cap B'\). Nous avons alors \( a=sx^{-1}\) avec \( s\in B\) et \( x^{-1} \in B'\). Par conséquent \( a\in B\) et donc \( a\in A'\cap B\). Nous avons donc
    \begin{equation}
        (A\cap B)\cap A'(A\cap B')=B\cap A'(A\cap B)\subset (A'\cap B)(A\cap B'),
    \end{equation}
    et donc l'égalité \eqref{EqkhsyNh}. Toujours dans l'idée de simplifier \eqref{EqkphNsE} nous remarquons que \( A\cap B'\) est un sous-ensemble de \( A\cap B'\), donc \( A'(A\cap B')(A\cap B)=A'(A\cap B)\). Il reste donc
    \begin{equation}
        \frac{ A'(A\cap B) }{ A'(A\cap B') }=\frac{ A\cap B }{ (A'\cap B)(A\cap B') }.
    \end{equation}
    Étant donné que les hypothèses sur \( A\) et \( B\) sont symétriques, le membre de droite peut aussi s'écrire en inversant \( A\) et \( B\). Nous en sommes à
    \begin{equation}
        \frac{ B'(A\cap B) }{ B'(A'\cap B) }=\frac{ A'(A\cap B) }{ A'(A\cap B') }.
    \end{equation}
    Nous devons encore justifier \( B'(A\cap B)=(A\cap B)B'\) et \( B'(A'\cap B)=(A'\cap B)B'\). Faisons le premier et laissons le second \href{http://abstrusegoose.com/395}{au lecteur}.
    Si \( b\in B'\) et \( x\in A\cap B\), alors
    \begin{equation}
        bx=x\underbrace{x^{-1}bx}_{\in B'}\in (A\cap B)B'.
    \end{equation}
\end{proof}

\begin{proposition}
    Si \( G\) est un groupe fini et que \( (G_i)\) est une suite de composition pour \( G\), alors l'ordre de \( G\) est le produit des ordres de ses quotients.
\end{proposition}

\begin{proof}
    Étant donné que \( G_{i+1}\) est toujours normal dans \( G_i\), le théorème de Lagrange (\ref{ThoLagrange}) s'applique et nous avons à chaque pas de la suite de composition nous avons
    \begin{equation}
        | \frac{ G_i }{ G_{i+1} } |=\frac{ | G_i | }{ | G_{i+1} | }
    \end{equation}
    et il suffit d'écrire \( | G |\) de façon télescopique :
    \begin{equation}
        | G |=\prod_{0\leq i\leq n-1}\frac{ | G_i | }{ | G_{i+1} | }
    \end{equation}
\end{proof}

Nous disons que les deux suites de composition \( (G_i)_{0\leq i\leq r}\) et \( (G_j)_{0\leq j\leq s}\) sont \defe{équivalentes}{equivalence@équivalence!suite de composition} si \( r=s\) et s'il existe une permutation \( \sigma\in S_{r-1}\) telle que
\begin{equation}
    \frac{ G_i }{ G_{i+1} }\simeq\frac{ H_{\sigma(i)} }{ H_{\sigma(i)+1} }.
\end{equation}

\begin{proposition}[Schreider]\index{lemme!de Schreider}
    Deux suites de composition d'un même groupe admettent des raffinements équivalents.
\end{proposition}

\begin{proof}
    Soient les suites de composition
    \begin{subequations}
        \begin{align}
            \{ e \}=G_m\subseteq\ldots\subseteq G_1\subseteq G_0=G\\
            \{ e \}=H_m\subseteq\ldots\subseteq H_1\subseteq H_0=G
        \end{align}
    \end{subequations}
    Nous raffinons la suite \( (G_i)\) en remplaçant \( G_{i+1}\subseteq G_i\) par
    \begin{equation}
        G_{i+1}=G_{i+1}(G_i\cap H_n)\subset G_{i+1}(G_i\cap H_{n-1})\subseteq\ldots\subseteq G_{i+1}(G_i\cap H_0)=G_i,
    \end{equation}
    et de même pour \( (H_j)\). Le groupe \( G_{i+1}(G_i\cap H_k)\) est normal dans \( G_{i+1}(G_i\cap H_{k-1})\) parce que \( G_{i+1}\) étant normal dans \( G_i\) et \( H_k\) dans \( H_{k-1}\), le lemme~\ref{LemsKpXCG} s'applique. Nous avons donc bien défini un raffinement.

    Nous devons maintenant prouver que les deux raffinements ainsi construits sont des suites de composition équivalentes. D'abord elles ont la même longueur \( mn\) parce que chacun des \( m\) éléments de la suite \( (G_i)\) a été remplacé par \( n\) éléments et inversement, chacun de \( n\) éléments de la suite \( (H_j)\) a été remplacé par \( m\) éléments.

    Par ailleurs, les quotients du raffinement de \( (G_i)\) sont de la forme
    \begin{equation}    \label{EqPAYTCB}
        \frac{ G_{i+1}(G_i \cap H_k) }{ G_{i+1}(G_i\cap H_{k+1}) }\simeq \frac{ H_{k+1}(H_k\cap G_i) }{ H_{k+1}(H_k\cap G_{i+1}) }
    \end{equation}
    en vertu du lemme du papillon (\ref{LemsKpXCG}). Le membre de droite de \eqref{EqPAYTCB} est un des quotients du raffinement de \( (H_j)\).
\end{proof}

\begin{lemma}[Schreider strictement décroissant]    \label{LemBSicRJ}
    Soient \( \Sigma_1\) et \( \Sigma_2\), deux suites de composition strictement décroissantes du groupe \( G\). Alors elles admettent des raffinements équivalents strictement décroissants.
\end{lemma}

\begin{proof}
    Par hypothèse, \( \Sigma_1\) et \( \Sigma_2\) n'ont pas de répétitions. Soient \( \Sigma''_1\) et \( \Sigma''_2\), des raffinements équivalents donnés par le lemme de Schreider. Étant donné que ce sont des suites de composition équivalentes, elles ont le même nombre de quotients réduits à \( \{ e \}\), c'est-à-dire le même nombre de répétitions.

    Les suites \( \Sigma'_1\) et \( \Sigma'_2\) obtenues en retirant les répétitions de \( \Sigma''_1\) et \( \Sigma''_2\) sont des raffinements équivalents de \( \Sigma_1\) et \( \Sigma_2\) et strictement décroissants.
\end{proof}

\begin{theorem}[Jordan-Hölder]\label{ThoLgxWIC}
    Tout groupe fini admet une suite de Jordan-Hölder.

    Deux suites de Jordan-Hölder sont équivalentes.
\end{theorem}
% TODO : trouver une preuve du fait que tout groupe fini admet une suite de Jordan-Hölder.

\begin{proof}
    Nous ne prouvons que le second point.

    Par définition, une suite de Jordan-Hölder n'a pas de raffinement strictement décroissant (à part elle-même) parce que \( G_{i+1}\) est normal maximum dans \( G_i\). Si \( \Sigma_1\) et \( \Sigma_2\) sont des suites de Jordan-Hölder nous pouvons considérer les raffinements équivalents strictement décroissants \( \Sigma'_1\) et \( \Sigma'_2\) du lemme de Schreider~\ref{LemBSicRJ}. Nous avons \( \Sigma'_1\sim\Sigma'_2\), mais par ce que nous venons de dire à propos de la maximalité, \( \Sigma'_1=\Sigma_1\) et \( \Sigma'_2=\Sigma_2\). D'où le résultat.
\end{proof}

%+++++++++++++++++++++++++++++++++++++++++++++++++++++++++++++++++++++++++++++++++++++++++++++++++++++++++++++++++++++++++++
\section{Groupes résolubles}
%+++++++++++++++++++++++++++++++++++++++++++++++++++++++++++++++++++++++++++++++++++++++++++++++++++++++++++++++++++++++++++

\begin{definition}  \label{DefOSYNooTROIKs}
    Le groupe \( G\) est \defe{résoluble}{groupe!résoluble} s'il existe une suite finie de sous-groupes \( G_i\)
    \begin{equation}
        \{ e \}=G_n\subset G_{n-1}\subset\ldots\subset G_1\subset G_0=G
    \end{equation}
    avec \( G_i\) normal dans \( G_{i+1}\) et \( G_i/G_{i+1}\) abélien.
\end{definition}
Il s'agit d'un groupe qui admet une suite de composition\footnote{Voir définition~\ref{DefJWZSooNcntfK}.} dont les quotients sont abéliens.

\begin{lemma}[\cite{HQRooKGAfpu}]   \label{LemOARMooYhYmbH}
    Soit \( G\) un groupe et \( H\) un sous-groupe normal. Le groupe \( G/H\) est abélien si et seulement si \( D(G)\subset H\).
\end{lemma}

\begin{proof}
    Les propositions suivantes sont équivalentes :
    \begin{itemize}
        \item Le groupe \( G/H\) est abélien
        \item pour tout \( x,y\in G\), \( \bar x\bar y=\bar y\bar x\)
        \item $\bar x\bar y\bar x^{-1}\bar y^{-1}=\bar e$
        \item \( \overline{ xyx^{-1}y^{-1} }=\bar e\)
        \item \( [x,y]\in H\)
        \item \( D(G)\subset H\).
    \end{itemize}
\end{proof}

\begin{proposition}[\cite{HQRooKGAfpu}] \label{PropRWYZooTarnmm}
    Un groupe est résoluble si et seulement si sa suite dérivée termine sur \( \{ e \}\).
\end{proposition}

\begin{proof}
    Grâce au lemme~\ref{LemMMOCooDJJJhy} et à la proposition~\ref{PropAPRGooHBkELf}, si la suite dérivée termine sur \( \{ e \}\) alors la suite dérivée est une suite qui répond aux conditions de la définition~\ref{DefOSYNooTROIKs} de groupe résoluble.

    Il faut donc encore montrer le sens direct. Nous supposons que \( G\) est un groupe résoluble et nous étudions sa suite dérivée. Nous avons une suite
    \begin{equation}
        \{ e \}=G_n\subset G_{n-1}\subset\ldots\subset G_1\subset G_0=G
    \end{equation}
    avec \( G_i/G_{i+1}\) abélien et \( G_{i+1}\) normal dans \( G_i\). Nous allons prouver par récurrence que \( D^i(G)\subset G_i\).

    Pour \( i=0\) nous avons bien \( G\subset G_0\). Notre hypothèse de récurrence est :
    \begin{equation}    \label{EqEAQEooEaeIEo}
        D^i(G)\subset G_i
    \end{equation}
    Par le lemme~\ref{LemOARMooYhYmbH} nous avons aussi
    \begin{equation}    \label{EqEDJXooLOLQcr}
        D(G_i)\subset G_{i+1}.
    \end{equation}
    En dérivant \eqref{EqEAQEooEaeIEo} et en tenant compte de \eqref{EqEDJXooLOLQcr}, \( D^{i+1}(G)\subset D(G_i)\subset G_{i+1}\). Donc par récurrence nous avons bien \( D^k(G)\subset G_k\) pour tout \( k\). Mais \( G_r=\{ e \}\) pour un certain \( r\), donc pour ce \( r\) nous avons \( D^r(G)=\{ e \}\), ce qu'il fallait.
\end{proof}

\begin{proposition} \label{PropBNEZooJMDFIB}
    Soient des groupes \( G\) et \( H\). Nous supposons que \( G\) est résoluble et nous considérons un homomorphisme \( \psi\colon G\to H\). Alors \( \psi(G)\) est résoluble.
\end{proposition}

\begin{proof}
    Vu que \( G\) est résoluble, il existe une suite de sous-groupes \( G_i\) tels que
    \begin{equation}
        \{ e \}=G_n\subset G_{n-1}\subset\ldots\subset G_1\subset G_0=G
    \end{equation}
    avec \( G_i\) normal dans \( G_{i+1}\) et \( G_i/G_{i+1}\) abélien. Nous posons \( \psi(G)_i=\psi(G_i)\) et nous avons \( \psi(G)_n=\psi\big( \{ e \} \big)=\{ e \}\) ainsi que \( \psi(G)_0=\psi(G)\); donc
    \begin{equation}
        \{ e \}=\psi(G)_n\subset \psi(G)_{n-1}\subset\ldots\subset \psi(G)_1\subset \psi(G)_0=\psi(G).
    \end{equation}

    Les faits que \( \psi(G)_i\) soit normal dans \( \psi(G)_{i+1}\) et que \( \psi(G)_i/\psi(G)_{i+1}\) soit abélien est directement la proposition~\ref{PropSRMJooIDPBoW}.

\end{proof}

%+++++++++++++++++++++++++++++++++++++++++++++++++++++++++++++++++++++++++++++++++++++++++++++++++++++++++++++++++++++++++++
\section{Action de groupes}
%+++++++++++++++++++++++++++++++++++++++++++++++++++++++++++++++++++++++++++++++++++++++++++++++++++++++++++++++++++++++++++

\begin{definition}[Thème~\ref{THEMEooKZHBooRCULcr}]  \label{DefActionGroupe}
    Une \defe{action de groupe}{action}\index{action} \( G\) sur un ensemble \( E\) est la donnée, pour chaque élément \( g \in G\), d'une fonction \(\phi_g : E \to E \), de telle sorte que:
    \begin{gather*}
        \phi_{e}(x) = x, \hspace{2em} \forall x \in E;\\
        \phi_{gh}(x) = \phi_g (\phi_h (x)),  \hspace{2em} \forall g,h \in G, \forall x \in E.
     \end{gather*}
     On dit dans ce cas que \( G \) \defe{agit}{action} sur \( E \).
\end{definition}

\begin{lemma}
    Pour tout \( g\in G\),
    \begin{enumerate}
        \item
            L'application \( \phi_g\colon E\to E\) est injective,
        \item
            Pour l'inverse : \( (\phi_g)^{-1}=\phi_{g^{-1}}\).
    \end{enumerate}
\end{lemma}

\begin{proof}
    Si \( x,y\in E\) sont tels que \( \phi_g(x)=\phi_g(y)\) alors en appliquant \( \phi_{g^{-1}}\) aux deux membres nous trouvons
    \begin{equation}
        (\phi_{g^{-1}}\phi_g)(x)=(\phi_{g^{-1}}\phi_g)(y),
    \end{equation}
    ce qui donne \( x=y\) parce que \( \phi_{g^{-1}}\phi_g=\phi_{g^{-1}g}=\phi_e=\id\).

    Les dernières trois égalités écrites disent que \( \phi_{g^{-1}}\) est l'inverse\footnote{Si vous décidez de dire ça a un jury dans un concours, soyez prêts à préciser les domaines.} de \( \phi_g\).
\end{proof}

Pour alléger les notations, on convient d'écrire $g \cdot x$, voire plus simplement $gx$ au lieu de \( \phi_g(x) \). Le deuxième axiome d'action de groupe dit que la notation $ghx$ ne souffre d'aucune ambiguïté.

\begin{definition}[Quelques notions autour de l'action]

    Si \( G\) agit sur un ensemble \( E\), nous notons \( G\cdot x\) l'\defe{orbite}{orbite!d'un point sous une action} de \( x\in E\) sous l'action de $G$:
\begin{equation*}
    G\cdot x = \{ gx \tq g \in G\}.
\end{equation*}

Nous notons \( G_x\) ou \( \Fix(x)\) le \defe{stabilisateur}{stabilisateur} de \( x\) :
\begin{equation}
    \Fix(x)=G_x=\{ g\in G\tq g\cdot x=x \}.
\end{equation}

    Pour \( g\in G\), nous notons enfin \( \Fix(g)\) le \defe{fixateur}{fixateur} de \( g\) :
\begin{equation}
    \Fix(g)=\{ x\in E\tq g\cdot x=x \}.
\end{equation}

\end{definition}

\begin{definition}  \label{DefuyYJRh}
    L'action de \( G\) sur \( E\) est \defe{fidèle}{fidèle (action)}\index{action!fidèle} si l'identité est le seul élément de \( G\) à fixer tous les points de \( E\), c'est-à-dire si \( gx=x\,\forall x\in E\Rightarrow g=e\).
\end{definition}

Un exemple d'action fidèle tout à fait non trivial sera donné avec l'action du groupe modulaire sur le plan de Poincaré dans le théorème~\ref{ThoItqXCm}.

Le groupe \( G\) agit toujours sur lui même à gauche et à droite. L'action à gauche est \( g\cdot h=gh\); celle à droite est \( g\cdot h=hg^{-1}\).

\begin{definition}      \label{DEFooCORTooEeOLPT}
    L'action \defe{adjointe}{action!adjointe} définie par \( g\cdot h=ghg^{-1}\) est une manière pour un groupe d'agir sur lui-même par automorphismes. Cela est souvent noté \( \AD(g)h=ghg^{-1}\).
\end{definition}
En effet pour tout \( g\in G\), l'application \( \AD(g)\colon G\to G\) est un automorphisme de \( G\).

Si \( H\) est un sous-groupe de  \( G\), nous notons \( G/H\) le quotient de $G$ par la relation \( g\sim gh\) pour tout \( h\in H\). Lorsque la distinction est importante, nous noterons \( (G/H)_g\)\nomenclature[R]{$(G/H)_g$}{classes à gauche} pour les classes à gauche et \( (G/H)_d\) pour les classes à droite.

Nous avons une relation d'équivalence à gauche et une à droite. D'abord
\begin{equation}
    x\sim_g y\Leftrightarrow xh=y
\end{equation}
pour un certain \( h\in H\). Ensuite
\begin{equation}
    x\sim_d y\Leftrightarrow hx=y
\end{equation}
pour un certain \( h\in H\).

Le lemme suivant est une généralisation du théorème de Lagrange~\ref{ThoLagrange}.

\begin{lemma}
    L'ensemble \( (G/H)_g\) est fini si et seulement si l'ensemble \( (G/H)_d\) est fini. S'il en est ainsi, alors \( (G/H)_g\) et \( (G/H)_d\) ont même cardinal qui vaut l'indice de \( H\) dans \( G\).
\end{lemma}

\begin{proof}
    L'application
    \begin{equation}
        \begin{aligned}
            f\colon (G/H)_g&\to (G/H)_d \\
            [x]_g&\mapsto [x^{-1}]_d
        \end{aligned}
    \end{equation}
    est une bijection bien définie. En effet si \( x\sim_g y\), nous avons \( h\in H\) tel que \( y^{-1}h=x^{-1}\), c'est-à-dire que \( x^{-1}\sim_d y^{-1}\) et \( f\) est bien définie. Le fait que \( f\) soit surjective est évident. Pour l'injectivité, soient \( x, y \in G \) tels que
    \begin{equation}
        f([x]_g)=f([y]_g).
    \end{equation}
    Alors \( x^{-1}\sim_d y^{-1}\), ce qui implique l'existence de \( h\in H\) tel que \( hx^{-1}=y^{-1}\), ou encore que \( xh^{-1}=y\), ce qui signifie que \( x\sim_gy\).

    Pour l'énoncé à propos de l'indice, nous procédons en plusieurs étapes simples.
    \begin{enumerate}
        \item
            Les classes (les éléments de \( (G/H)_g\)) forment une partition de $G$.
        \item
            Toutes les classes ont le même nombre d'éléments par la bijection
            \begin{equation}
                \begin{aligned}
                    f\colon [x]_g&\to [y]_g \\
                    xh&\mapsto yh.
                \end{aligned}
            \end{equation}
        \item
            Le nombre d'éléments dans une classe est égal à \( | H |\) par la bijection
            \begin{equation}
                \begin{aligned}
                    g\colon [x]_g&\to H \\
                    xh&\mapsto h.
                \end{aligned}
            \end{equation}
    \end{enumerate}
    Par conséquent
    \begin{equation}
        | G |=| H |\cdot \text{nombre de classes}=| H |\cdot\text{cardinal de $(G/H)_g$},
    \end{equation}
    et nous avons bien
    \begin{equation}
        \text{cardinal de }(G/H)_g=\frac{ | G | }{ | H | }=| G:H |.
    \end{equation}
\end{proof}

\begin{proposition}[Orbite-stabilisateur\cite{Combes}]     \label{Propszymlr}
    Soit \( G\) un groupe agissant sur un ensemble \( E\) et \( x\in E\).
    \begin{enumerate}
        \item
            Les ensembles \( G\cdot x\) et \( G/G_x\) sont équipotents.
        \item       \label{ITEMooCWUGooCOFHYk}
            L'orbite de \(\Fix(x)\) est finie si et seulement si \( \Fix(x)\) est d'indice fini dans \( G\). Dans ce cas nous avons
            \begin{equation}        \label{EqnLCHCE}
                \Card(G\cdot x)=| G:\Fix(x) |.
            \end{equation}
            Une autre façon d'écrire la même formule :
            \begin{equation}        \label{EqCewSXT}
                | G |=| \Fix(x) | |\mO_x |.
            \end{equation}
    \end{enumerate}
\end{proposition}
\index{équation!orbite-stabilisateur}
C'est la formule \eqref{EqnLCHCE} qui est nommée \wikipedia{fr}{Action_de_groupe_(mathématiques)\#Formule_des_classes.2C_formule_de_Burnside}{formule des classes} sur wikipédia.

\begin{proof}
    \begin{enumerate}
        \item
    Soit l'application
    \begin{equation}
        \begin{aligned}
            \psi\colon G\cdot x&\to G/G_x \\
            a\cdot x&\mapsto [a].
        \end{aligned}
    \end{equation}
    Cette application est bien définie parce que si \( a\cdot x=b\cdot x\), alors il existe \( h\in G_x\) tel que \( b=ah\), et par conséquent \( [a]=[b]\). Cette application est une bijection et par conséquent \( G\cdot x\) est équipotent à \( G/G_x\).
    \item
        Soit \( y\in \mO_x\) et \( A_y=\{ g\in G\tq g\cdot x=y \}\). L'ensemble \( A_y\) est une classe à gauche de \( \Fix(x)\), par conséquent \( | A_y |=|\Fix(x)|\) pour tout \( y\in\mO_x\). Les \( A_y\) pour différents \( y\) sont disjoints et nous avons de plus
        \begin{equation}
            \bigcup_{y\in\mO_x}A_y=G.
        \end{equation}
        Les ensembles \( A_y\) divisent donc \( G\) en \( | \mO_x |\) paquets de \( | \Fix(x) |\) éléments. D'où la formule \eqref{EqCewSXT}.

    \end{enumerate}
\end{proof}

\begin{corollary}       \label{CORooRRVHooTyCjZZ}
    Soit \( C_g\) la classe de conjugaison d'un élément  \( g\) du groupe fini \( G\). Alors
    \begin{equation}
        \Card(C_g)=| G:Z_G(g) |
    \end{equation}
    où $Z_G(g)$ est le centralisateur de \( g\) dans \( G\)\footnote{Définition~\ref{defGroupeCentre}.} de \( G\).
\end{corollary}

\begin{proof}
    Cela est une application de la proposition~\ref{Propszymlr} (formule \eqref{EqnLCHCE}) dans le cas de l'action adjointe de \( G\) sur lui-même.

    En effet, si nous considérons l'action adjointe, l'orbite est la classe de conjugaison : \( C_g=G\cdot g\). Et le stabilisateur de \( g\) pour l'action adjointe n'est autre que le centralisateur de \( g\) :
    \begin{subequations}
        \begin{align}
        \Fix(g)&=\{ h\in G\tq h\cdot g=g \}\\
        &= \{ h\in G\tq hgh^{-1}=g \}\\
        &=\{ h\in G\tq gh=hg \}\\
        &=Z_G(g).
        \end{align}
    \end{subequations}

    Donc la formule \( \Card(G\cdot g)=| G:G_g |\) devient, dans le cas de l'action adjointe de \( G\) sur lui-même : \( \Card(C_g)=| G:Z_G(g) |\).
\end{proof}

\begin{lemma}
    Soit \( G\) un groupe agissant sur l'ensemble \( E\). On définit \( x\sim x'\) si et seulement s'il existe \( g\in G\) tel que \( g\cdot x=x'\). Alors
    \begin{enumerate}
        \item
            la relation \( \sim\) est une relation d'équivalence.
        \item
            la classe \( [x]\) est l'orbite \( \mO_x\) de \( x\) sous \( G\).
    \end{enumerate}
\end{lemma}

\begin{corollary}[Équation des orbites]\index{équation!des orbites} \label{CorARFVMP}
    Soit \( G\) un groupe agissant sur l'ensemble \( E\) et \( \mO_1,\ldots, \mO_k  \) la liste des orbites (distinctes). Alors
    \begin{enumerate}
        \item
            \( E=\bigcup_i\mO_i\), l'union est disjointe,
        \item
            \( \Card(E)=\sum_i\Card(\mO_i)\).
    \end{enumerate}
\end{corollary}

\begin{definition}  \label{DefcSuYxz}
    Soit \( G\) un groupe agissant sur l'ensemble \( E\). Un \defe{domaine fondamental}{domaine!fondamental d'une action}\index{fondamental!domaine d'une action}\index{action!domaine fondamental} ou une \defe{transversale}{transversale} est une partie de \( E\) contenant un et un seul élément de chaque orbite.
\end{definition}
Autrement dit, les images des éléments d'un domaine fondamental forment une partition de l'ensemble :
\begin{equation}
    E=\bigsqcup_{g\in G}g(F)
\end{equation}
où  \( g(F) = \phi_g(F) = \{ \phi_g(x) \tq x \in F\} \). L'union est disjointe, c'est-à-dire que si \( g\neq g'\), alors \( g(F)\cap g'(F)=\emptyset\).


%TODO:
% - définir ce qu'est un p-groupe, au bon endroit (à côté du théorème de Lagrange?).
% - pour l'instant, la définition de p-groupe est là où on l'utilise, c'est-à-dire un peu avant les théorèmes de Sylow.
% - Il vaut peut-être mieux créer un thème ``indice dans un groupe'' dans lequel on parlera entre autres du théorème de Lagrange et de p-groupe.


\begin{proposition}[Équation des classes\cite{FabricegPSFinis}]     \label{PropUyLPdp}
    Soit \( G\), un groupe fini opérant sur un ensemble \( E\). Si \( E''\) est un ensemble contenant exactement un élément de chaque orbite dans \( E\setminus\Fix_G(E)\), alors
    \begin{equation}        \label{EqobuzfK}
        | G |=| \Fix_G(E) |+\sum_{x\in E''}\frac{ | G | }{ | \Fix_G(x) | }.
    \end{equation}
    Si de plus \( G\) est un $p$-groupe, alors
    \begin{equation}    \label{EqbzLEVJ}
        | E |=| \Fix_G(E) |\mod p.
    \end{equation}
\end{proposition}

\begin{proof}
    Par le corolaire~\ref{CorARFVMP}, nous avons \( | G |=\sum_{x\in E'}| \mO_x |\) où \( E'\) est une transversale.  En séparant la somme entre les orbites à un élément et les autres,
    \begin{equation}    \label{EqeggkBs}
        | G |=\Card(\Fix_G(E))+\sum_{x\in E''}\frac{ | G | }{ | \Fix_G(x) | }
    \end{equation}  \label{EqDgYbhm}
    où nous avons utilisé le fait que \( | G |=| \Fix_G(x) | |\mO_x |\).

    Si \( G\) est un \( p\)-groupe alors si \( x\in E''\), \( \Fix_G(x)\) est un sous-groupe propre de \( G\) et donc \( | \Fix_G(x) |\) est un diviseur propre de \( | G |\). Du coup la fraction \( | G |/|\Fix_G(x)|\) est une puissance non nulle de \( p\) et l'équation \eqref{EqobuzfK} devient immédiatement \eqref{EqbzLEVJ}.
\end{proof}


\begin{corollary}[Équation des classes]\index{équation!des classes}
    Soit \( G\), un groupe et \( C_1\),\ldots, \( C_l\) la liste de ses classes de conjugaison contenant plus de un élément. Alors
    \begin{equation}        \label{EqkgGmoq}
        \Card(G)=\Card\big( Z(G) \big)+\sum_i| G:Z_{g_i} |=\Card\big( Z(G) \big)+\sum_i\frac{ \Card(G) }{ \Card\big( \Fix(g_i) \big) }
    \end{equation}
    si \( g_i\in C_i\).
\end{corollary}

\begin{proof}
    Étant donné que les classes de conjugaison sont disjointes, le cardinal de \( G\) est bien la somme des cardinaux de ses classes. Les classes ne contenant que un seul élément sont celles des éléments de \( Z(G)\). En ce qui concerne les autres orbites, \( \Card(C_{g_i})=| G:Z_{g_i} |\) par le théorème orbite-stabilisateur (proposition~\ref{Propszymlr}).
\end{proof}

\begin{theorem}[\wikipedia{fr}{Action_de_groupe_(mathématiques)}{Formule de Burnside}]      \label{THOooEFDMooDfosOw}
    Si \( G\) est un groupe fini agissant sur l'ensemble fini \( E\) et si \( \Omega\) est l'ensemble des orbites, alors
    \begin{equation}    \label{EqTUsblv}
        \Card(\Omega)=\frac{1}{ | G | }\sum_{g\in G}\Card\big( \Fix(g) \big).
    \end{equation}
\end{theorem}
\index{Burnisde!formule}
\index{formule!Burnside}

\begin{proof}
    Nous considérons l'ensemble
    \begin{equation}
        A=\{ (g,x)\in G\times E\tq gx=x \},
    \end{equation}
    et nous en calculons le cardinal de deux façons. D'abord
    \begin{subequations}
        \begin{align}
            \Card(A)&=\sum_{x\in E}\Card\{ g\in g\tq gx=x \}\\
            &=\sum_{x\in E}\Card(\Fix(x))\\
            &=\sum_{\omega\in \Omega}\sum_{x\in \omega}\Card(\Fix(x))\\
            &=\sum_{\omega\in \Omega}\frac{ | G | }{ \Card(\omega) }     \label{EqyVtkyf}\\
            &=| G |.
        \end{align}
    \end{subequations}
    Pour obtenir \eqref{EqyVtkyf} nous avons utilisé l'équation des classes \eqref{EqCewSXT}. L'autre façon de calculer \( \Card(A)\) est de regrouper ainsi :
    \begin{equation}
        \Card(A)=\sum_{g\in G}\Card\{ x\in E\tq gx=x \}=\sum_{g\in G}\Card(\Fix(g)).
    \end{equation}
    En égalisant les deux expressions de \( \Card(A)\) nous trouvons
    \begin{equation}
        | G |\Card(\Omega)=\sum_{g\in G}\Card(\Fix(g)).
    \end{equation}
\end{proof}

\begin{proposition}
    Soit \( G\) un groupe et \( H\), un sous-groupe du centre de \( G\).
    \begin{enumerate}
        \item
            \( H\) est normal dans \( G\).
        \item
            Si \( G/H\) est monogène, alors \( G\) est abélien.
        \item
            Si \( G\) est fini de centre \( Z\), alors \( | G:H |\) n'est pas premier.
    \end{enumerate}
\end{proposition}

% TODO: preuve

\begin{theorem}     \label{THOooRGSTooIWyhqt}
    Soit \( G\) un groupe cyclique\footnote{Définition \ref{DefHFJWooFxkzCF}.} d'ordre \( n\).
    \begin{enumerate}
        \item
            Tout sous-groupe de \( G\) est cyclique.
        \item
            Pour chaque diviseur \( d\) de \( n\), il existe un unique sous-groupe \( H_d\) de \( G\) d'ordre \( d\).
    \end{enumerate}
    Si \( a\) est un générateur de \( G\), alors \( H_d\) peut être décrit des façons suivantes :
    \begin{equation}
        H_d=\{ x\in G\tq x^d=e \}=\{ x\in G\tq\exists y\in G\tq y^{n/d}=x \}=\langle a^{n/d}\rangle.
    \end{equation}
\end{theorem}

% TODO: preuve

\begin{definition}      \label{DEFooQDHPooCfDEuL}
    Soit \( G\) un groupe agissant sur un ensemble \( E\). Nous disons que l'action est \defe{transitive}{transitive}\index{action!transitive} si elle possède une seule orbite. L'action est \defe{libre}{libre!action}\index{action!libre} si \( g\cdot x=g'\cdot x\) implique \( g=g'\).
\end{definition}

\input{47_groupes}

\chapter{Anneaux}
% This is part of Mes notes de mathématique
% Copyright (c) 2011-2020
%   Laurent Claessens
% See the file fdl-1.3.txt for copying conditions.

Attention aux conventions. Dans le Frido, un corps peut être réduit à \( \{ 0 \}\) et un idéal premier ne peut pas être \( \{ 0 \}\). Ces conventions ont une série de conséquences un peu partout, par exemple dans la proposition \ref{PROPooRUQKooIfbnQX} où nous parlons d'idéal maximum propre. Comparez par exemple avec \cite{ooWEUDooQybvIx}. Soyez attentifs.

En cas de doutes, nous suivons les conventions de Wikipédia.

%+++++++++++++++++++++++++++++++++++++++++++++++++++++++++++++++++++++++++++++++++++++++++++++++++++++++++++++++++++++++++++ 
\section{Inversible et niplolens}
%+++++++++++++++++++++++++++++++++++++++++++++++++++++++++++++++++++++++++++++++++++++++++++++++++++++++++++++++++++++++++++

Le concept d'anneau est la définition \ref{DefHXJUooKoovob}.

\begin{lemma}
    Si \( a\) et \( b\) commutent, nous avons, pour tout \( r \in \eN \) et \( r > 0\), la formule
    \begin{equation}        \label{Eqarpurmkbk}
        a^{r+1}-b^{r+1}=(a-b)\left(\sum_{k=0}^ra^{r-k}b^k\right).
    \end{equation}
\end{lemma}

\begin{proof}
  Démontrons cela par récurrence. Le cas \( r = 0 \) est évident. Pour
  un \( r \) donné, si \eqref{Eqarpurmkbk} est vraie, alors
  \begin{align*}
    a^{r+2}-b^{r+2}&= a^{r+1}a - a^{r+1}b +a^{r+1}b - b^{r+1}b\\
    &= a^{r+1}(a - b) + (a^{r+1} - b^{r+1})b\\
    &= a^{r+1}(a - b) + (a-b)\left(\sum_{k=0}^ra^{r-k}b^k\right)b\\
    &= (a - b) \left(a^{r+1} + \left(\sum_{k=0}^ra^{r-k}b^k\right)b\right)\\
    &= (a - b) \left(a^{r+1} + \sum_{k=0}^ra^{r-k}b^{k + 1}\right)\\
    &= (a - b) \left(a^{r+1} + \sum_{k'=1}^{r+1}a^{(r+1)-k'}b^{k'}\right)\\
    &= (a - b) \left(\sum_{k'=0}^{r+1}a^{(r+1)-k'}b^{k'}\right).
  \end{align*}
\end{proof}

\begin{proposition}
    Si \( a\) est un élément nilpotent de l'anneau \( A\), alors \( 1-a\) est inversible. Si \( a\) est nilpotent non nul, alors il est diviseur de zéro.
\end{proposition}

\begin{proof}
    Soit \( n\) le minimum tel que \( a^n=0\). En vertu de la formule \eqref{Eqarpurmkbk} nous avons
    \begin{equation}
        1=1-a^n=(1-a)(1+a+\cdots+a^{n-1})=(1+a+\cdots+a^{n-1})(1-a).
    \end{equation}
    La somme \( 1+a+\cdots+a^{n-1}\) est donc un inverse de \( (1-a)\).
\end{proof}

%+++++++++++++++++++++++++++++++++++++++++++++++++++++++++++++++++++++++++++++++++++++++++++++++++++++++++++++++++++++++++++ 
\section{PGCD, PPCM et éléments inversibles}
%+++++++++++++++++++++++++++++++++++++++++++++++++++++++++++++++++++++++++++++++++++++++++++++++++++++++++++++++++++++++++++

La définition de pgcd et ppcm dans un anneau commutatif est la définition \ref{DefrYwbct}. Dans la plus grande tradition, elle a été introduite sans motivations, et utilisée par-ci par-là. Nous revenons maintenant dessus.

Commençons par donner une autre vision de la divisibilité dans les anneaux intègres.
\begin{proposition}\label{PropDiviseurIdeaux}
    Dans un anneau intègre\footnote{Définition \ref{DEFooTAOPooWDPYmd}.} $A$, on a l'équivalence suivante concernant deux éléments \( a, b \in A \):
\begin{equation}
    a\divides b\Leftrightarrow (b)\subset (a).
\end{equation}
\end{proposition}

Donc la divisibilité devient en réalité une relation d'ordre dont nous pouvons chercher un maximum et un minimum. Si \( S\) est une partie de \( A\), nous notons \( a\divides S\) pour exprimer que \( a\divides x\) pour tout \( x\in S\); de la même façon, \( S\divides b\) signifie que \( x\divides b\) pour tout \( x\in S\).


Nous rappelons également la définition~\ref{DEFooSPHPooCwjzuz} de morphisme d'anneaux. Remarquons que si \( f\) est un morphisme, nous avons \( f(0)=0\) et \( f(x)^{-1}=f(x^{-1})\).

\begin{lemma}[\cite{ooLIOMooBuCPUS}]
    Les éléments inversibles d'un anneau sont diviseurs de tous les éléments.
\end{lemma}

\begin{proof}
    Soit \( k\) inversible d'inverse \( k'\) : \( kk'=1\); soit aussi \( a\in A\). Alors \( a=k(k'a)\), ce qui montre que \( k\) divise \( a\).
\end{proof}

\begin{lemma}[\cite{ooLIOMooBuCPUS}]
    Dans un anneau, \( 1\) est un pgcd de \( a\) et \( b\) si et seulement si les seuls diviseurs communs sont les inversibles.
\end{lemma}

\begin{proof}
    Supposons pour commencer que \( 1\) est un pgcd de \( a\) et \( b\). Un diviseur commun de \( a\) et \( b\) doit donc diviser \( 1\). Or un diviseur de \( 1\) est forcément inversible.

    Dans l'autre sens, les diviseurs communs de \( a\) et \( b\) sont tous inversibles et donc diviseurs de \( 1\). Donc \( 1\) est un pgcd de \( a\) et \( b\).
\end{proof}

%+++++++++++++++++++++++++++++++++++++++++++++++++++++++++++++++++++++++++++++++++++++++++++++++++++++++++++++++++++++++++++
\section{Le groupe et anneau des entiers}
%+++++++++++++++++++++++++++++++++++++++++++++++++++++++++++++++++++++++++++++++++++++++++++++++++++++++++++++++++++++++++++

Certes \( \eZ\) est un groupe pour l'addition, mais c'est également un anneau\footnote{Définition~\ref{DefHXJUooKoovob}.} parce que nous avons les deux opérations d'addition et de multiplication. Nous n'allons pas nous priver de cette belle structure juste parce que le titre du chapitre est «groupes».

%---------------------------------------------------------------------------------------------------------------------------
\subsection{Division euclidienne}
%---------------------------------------------------------------------------------------------------------------------------

\begin{theorem}[Division euclidienne\cite{ooRBKHooQJqglH}]     \label{ThoDivisEuclide}
    Soient \( a\in\eZ\) et \( b\in\eN^*\). Il existe un unique couple \( (q,r)\in\eZ\times\eN\), avec \( 0\leq r<b\), tel que
    \begin{equation}
        a=bq+r.
    \end{equation}
\end{theorem}

\begin{proof}
    Remarquons que \( r = a - bq \), et donc, une fois l'existence et l'unicité de $q$ établie, celle de $r$ suivra.

    \begin{subproof}
        \item[Unicité]
            Nous supposons avoir \( (q,r)\in \eZ\times \eN\) tel que \( 0\leq r<b\) et \( a=bq+r\). Alors forcément \( r=a-qb\) et $0\leq a-qb<b$, ou encore
            \begin{equation}
                qb\leq a<(1+q)b.
            \end{equation}
            Ces deux inéquations fixent \( q\in \eZ\). En effet nous démontrons maintenant que seul \( k=0\) permet à \( q+k\) de vérifier ces deux inéquations (parmi les \( k\in \eZ\)). Si \( q\geq 1\) alors
            \begin{equation}
                (q+k)b=(q+1)b+(k-1)b>a.
            \end{equation}
            Et si \( k\leq -1\) alors
            \begin{equation}
                (q+k+1)b\leq qb\leq a.
            \end{equation}
            D'où l'unicité de \( q\) et par conséquent celle de \( r\).

        \item[Existence]

            Nous considérons l'ensemble
    \begin{equation*}
        E = \{ q \in \eZ  | bq \leq a \}.
    \end{equation*}
    C'est un sous-ensemble d'entiers non-vide (il contient \( -|a| \) ) et admet \( |a| \) comme majorant; il admet donc un maximum $q$ par le lemme \ref{LEMooMYEIooNFwNVI}. Ce maximum vérifie
     \begin{equation}
         bq\leq a<b(q+1).
     \end{equation}
     Cela donne \( 0\leq a-bq<b\) et le résultat en posant \( r=a-qb\).
    \end{subproof}
\end{proof}

\begin{definition}
    L'opération \( (a,b)\mapsto(q,r)\) donnée par le théorème~\ref{ThoDivisEuclide} est la \defe{division euclidienne}{division!euclidienne}. Le nombre \( q\) est le \defe{quotient}{quotient} et \( r\) est le \defe{reste}{reste} de la division de \( a\) par \( b\).
\end{definition}

% TODO : À propos de restes, il n'est peut-être pas mal de parler d'algorithme de calcul de la date de pâques.
% L'algorithme de Gauss, Meeus utilise des arrondis.
% http://fr.wikipedia.org/wiki/Calcul_de_la_date_de_Pâques

%---------------------------------------------------------------------------------------------------------------------------
\subsection{Sous-groupes de \texorpdfstring{$(\eZ,+)$}{(Z,+)}}
%---------------------------------------------------------------------------------------------------------------------------

\begin{proposition} \label{PropSsgpZestnZ}
    Une partie \( H\) du groupe \( (\eZ,+)\) est un sous-groupe si et seulement s'il existe \( n\in\eN\) tel que \( H=n\eZ\).
\end{proposition}

\begin{proof}
    Soit \( H\neq\{ 0 \}\) un sous-groupe de \( \eZ\). L'ensemble \( H\cap\eN^*\) contient un élément minimum que nous notons \( n\). Nous avons certainement \( n\eZ\subset H\) parce que \( H\) est un groupe (donc \( n+n\) et \( -n\) sont dans \( H\) dès que \( n\) est dans \( H\)). Nous devons prouver que \( H\subset n\eZ\).

    Si \( x\in H\), par le théorème de division euclidienne~\ref{ThoDivisEuclide}, il existe \( q\in\eZ\) et \( r\in\eN \), uniques, tels que \( x=nq+r\) et \(0 \leq r < n \). Nous savons déjà que \( nq\in H\), donc \( r = x - nq \in H \). Le nombre \( r\) est donc un élément de \( H\) strictement plus petit que \( n\). Mais nous avions décidé que \( n\) serait le plus petit élément de \( H\cap\eN^*\). Par conséquent \( r=0\) et \( x=nq\in n\eZ\).
\end{proof}


Notons que si un sous-groupe \( H\) de \( \eZ\) est donné, alors le nombre \( n\) tel que \( H=n\eZ\) est unique. En effet si \( n\eZ=m\eZ\) nous avons que \( n\) divise \( m\) (parce que \( m\in m\eZ\subset n\eZ\)) et que \( m\) divise \( n\) parce que \( n\in m\eZ\). Par conséquent \( n=m\).


%---------------------------------------------------------------------------------------------------------------------------
\subsection{PGCD, PPCM et Bézout}
%---------------------------------------------------------------------------------------------------------------------------

Vu que \( \eZ\) est un anneau intègre, nous avons la définition \ref{DefrYwbct} de pgcd et de ppcm.
\begin{proposition}[PPCM et PGCD]       \label{PROPooAVRGooUfhjwF}
    Soient \( p,q\in\eZ^*\). 
    \begin{enumerate}
        \item
            Le pgcd de \( p\) et \( q\) est le plus grand diviseur commun de \( p\) et \( q\). 
        \item
            Le ppcm de \( p\) et \( q\) est leur plus petit multiple commun.
    \end{enumerate}
\end{proposition}

\begin{proof}
    Démontrons le premier point. Notons \( \delta\) le pgcd de \( p\) et \( q\). Si \( d\) est un diviseur commun de \( p\) et \( q\), alors \( d\) divise \( \delta\). Dans \( \eZ\), \( d\divides \delta\) implique \( d\leq\delta\) (proposition \ref{PROPooYJBMooZrzkNX}).
\end{proof}

\begin{lemma}
    Soient \( p,q\in\eZ^*\). Les entiers \( \ppcm(p,q)\) et \( \pgcd(p,q)\) fournissent les isomorphismes de groupes suivants :
\begin{subequations}
    \begin{align}
        p\eZ\cap q\eZ&=\ppcm(p,q)\eZ\\
        p\eZ + q\eZ&=\pgcd(p,q)\eZ.
    \end{align}
\end{subequations}
\end{lemma}

\begin{definition}  \label{DefZHRXooNeWIcB}
    Si \( \pgcd(p,q)=1\), nous disons que \( p\) et \( q\) sont \defe{premiers entre eux}{nombre!premier!deux nombres entre eux}. Si nous avons un ensemble d'entiers \( a_i\), nous disons qu'ils sont premiers \defe{dans leur ensemble}{nombre!premier!dans leur ensemble} si \( 1\) est le PGCD de tous les \( a_i\) ensemble.
\end{definition}

Les nombres \( 2\), \( 4\) et \( 7\) ne sont pas premiers deux à deux (à cause de \( 2\) et \( 4\)), mais ils sont premiers dans leur ensemble parce qu'il n'y a pas de diviseurs communs à tout le monde.

\begin{definition}
    Un \defe{nombre premier}{nombre!premier} est un naturel acceptant exactement deux diviseurs distincts.
\end{definition}
Avec cette définition, \( 0\) n'est pas premier, \( 1\) n'est pas premier et \( 2\) est premier.

\begin{theorem}[Théorème de Bézout\footnote{Il y a une super application ici : \url{https://perso.univ-rennes1.fr/matthieu.romagny/agreg/dvt/mauvais_prix.pdf}.}\cite{LSAmvR}, thème~\ref{THEMEooNRZHooYuuHyt}] \label{ThoBuNjam}
    Deux entiers non nuls \( a,b\in\eZ^*\) sont premiers entre eux si et seulement s'il existe \( u,v\in\eZ\) tels que
    \begin{equation}
        au+bv=1
    \end{equation}
\end{theorem}
\index{Bézout!nombres entiers}

\begin{proof}
    Soit \( d=\pgcd(a,b)\) et des nombres \( u,v\) tels que \( au+bv=1\). Le PGCD \( d\) divise à la fois \( a\) et \( b\), et donc divise \( au+bv\). Nous en déduisons que \( d\) divise \( 1\) et est par conséquent égal à \( 1\).

    Nous supposons maintenant que \( \pgcd(a,b)=1\) et nous considérons l'ensemble
    \begin{equation}
        E=\{ au+bv\tq u,v\in \eZ \}\cap \eN^*.
    \end{equation}
    C'est-à-dire l'ensemble des nombres strictement positifs pouvant s'écrire sous la forme \( au+bv\). Cet ensemble est non vide parce qu'il contient par exemple soit \( a\) soit \( -a\). Soit \( m\) le plus petit élément de \( E\) et écrivons
    \begin{equation}    \label{EqMBsfrP}
        m=au_1+bv_1.
    \end{equation}
    Par le théorème de division euclidienne\footnote{Théorème~\ref{ThoDivisEuclide}.} (avec \( a\) et \( m\)), il existe des entiers uniques $q$ et $r$ tels que
    \begin{equation}
        a=mq+r
    \end{equation}
    avec \( 0\leq r<m\). En remplaçant \( m\) par sa valeur \eqref{EqMBsfrP}, \( a=(au_1+bv_1)q+r\) et
    \begin{equation}
        r=a(1-u_1q)-bv_1q,
    \end{equation}
    c'est-à-dire que \( r\in \eZ a+\eZ b\) en même temps que \( 0\leq r<m\). Si \( r\) était strictement positif, il serait dans \( E\). Mais cela est impossible par minimalité de \( m\). Donc \( r=0\) et \( a\) est divisible par \( m\).

    De la même façon nous prouvons que \( b\) est divisible par \( m\). Vu que \( m\) divise à la fois \( a\) et \( b\) nous avons \( m=1\).
\end{proof}

\begin{corollary}       \label{CorgEMtLj}
    Soient \( p\) et \( q\) deux entiers premiers entre eux. Alors
    \begin{equation}
        p\eZ+q\eZ=\eZ;
    \end{equation}
    en particulier, pour tout \( x \in \eZ \), il existe \( u_x, v_x \) entiers tels que \(u_x p + v_x q = x \).
\end{corollary}

Notons que l'application \( p\eZ+q\eZ\) vers \( \eZ\) n'est évidemment pas injective: les $u_x$ et $v_x$ ne sont pas uniques à $x$ fixé.

\begin{proof}
    Soit \( x\in \eZ\). Le théorème de Bézout nous donne \( k\) et \( l\) tels que \( kp+lq=1\). Du coup, \( (xk)p+(xl)q=x\).
\end{proof}

La proposition suivante établit que si \( x\) est assez grand, alors il peut même être écrit comme une combinaison de \( p\) et \( q\) à coefficients positifs. Elle sera utilisée pour démontrer que les états apériodiques d'une chaine de Markov peuvent être atteints à tout moment (assez grand), voir la définition~\ref{DefCxvOaT} et ce qui suit.

\begin{proposition}     \label{PropLAbRSE}
    Soient \( a\) et \( b\) deux éléments de \( \eN\) premiers entre eux. Il existe \( N>0\) tel que tout \( x>N\) appartient à \( a\eN+b\eN\).
\end{proposition}

\begin{proof}
    Soient \( a\) et \( b\), premiers entre eux, et \( x\in \eN\). Disons tout de suite, pour éviter les cas triviaux et pénibles, que \( x\), \( a\) et \( b\) sont strictement positifs.

    \begin{subproof}
    \item[Une décomposition pour \( x\)]

    On applique le théorème~\ref{ThoDivisEuclide} de division euclidienne à $x$ et \( a + b \): il existe des entiers \( p_x, r_x \), uniques, tels que
    \begin{subequations}
        \begin{numcases}{}
       x = (p_x-1)(a+b) + r_x\\
       0 \leq r_x < a+b.
        \end{numcases}
    \end{subequations}
    En d'autres termes, \( p_x(a+b)\) est le premier multiple de \( a+b\) supérieur ou égal à $x$. De plus, $p_x$ est strictement positif car $x$ l'est. Il existe alors des entiers $u$ et $v$ tels que
    \begin{equation}    \label{EQooXYSZooJqxPui}
        ua + vb = p_x(a+b) - x
    \end{equation}
    par le corolaire~\ref{CorgEMtLj}. Ainsi, $x$ peut s'écrire
    \begin{equation}
        x = (p_x - u) a + (p_x - v) b.
    \end{equation}

\item[Des maximums]

    Il s'agit maintenant de savoir si nous pouvons être assuré d'avoir \( p_x > u\) et \( p_x > v\) dès que \( x\) est assez grand. Pour cela, grâce au corolaire~\ref{CorgEMtLj}, nous considérons les nombres \( u_i\) et \( v_i\) définis par
    \begin{equation}
        u_ia+v_ib=i
    \end{equation}
    pour \( i=1,\ldots, a+b\). Nous posons \( u^*=\max\{ u_i \}\), \( v^*=\max\{ v_i   \}\), et \( p^*=\max\{ u^*,v^* \}\).  Nous posons alors \( N = p^*(a+b)\), et considérons \( x>N \).

\item[Nouvelle décomposition pour \( x\)]

    Nous voulons écrire
    \begin{equation}        \label{EQooIKNWooBKItYz}
        x= (p_x - u_k) a + (p_x - v_k) b
    \end{equation}
    pour un certain \( k\). Cela demande \( u_ka+v_kb=ua+vb=p_x(a+b)-x\) par l'équation \eqref{EQooXYSZooJqxPui}. Vu que \( p_x(a+b)-x>0\), les nombres \( u_k\) et \( v_k\) existent : il suffit de prendre \( k=p_x(a+b)-x\).

\item[Conclusion]

    Avec tous ces choix, nous avons d'abord \( x>p^*(a+b)\) et donc
    \begin{equation}
        x=(p_x-1)(a+b)+r_x>p^*(a+b),
    \end{equation}
    ce qui donne
    \begin{equation}
        (p_x-1)(a+b)>p^*(a+b)-r_x>(p-1)(a+b).
    \end{equation}
    ou encore \( p_x>p^*\). Nous avons finalement
    \begin{equation}
       p_x \geq p^* \geq u^* \geq u_k
    \end{equation}
    et
    \begin{equation}
       p_x \geq p^* \geq v^* \geq v_k.
    \end{equation}
    De ce fait, la décomposition \eqref{EQooIKNWooBKItYz} est celle que nous voulions.
    \end{subproof}
\end{proof}


%\begin{proof}
    %Soit \( x\in \eN\) et \( k_1,l_1\in \eN\) les plus petits entiers tels que \( k_1p\geq x/2\) et \( l_1q\geq x/2\). Nous avons alors
    %\begin{equation}
        %x\leq k_1p+l_1q<x+(p+q).
    %\end{equation}
    %Nous posons \( \delta=k_1p+l_1q-x\).
   %
    %Soient des entiers \( a_i,b_i\) tels que \( a_ip+b_iq=i\). Nous notons
    %\begin{subequations}
        %\begin{align}
            %A=\max\{ a_i\tq i=1,\ldots, k+p \}\\
            %B=\max\{ b_i\tq i=1,\ldots, k+p \}
        %\end{align}
    %\end{subequations}
    %Notons que \( A\) et \( B\) sont donnés uniquement en termes de \( p\) et \( q\). Ils ne sont en aucun cas dépendants de \( x\).
   %
    %Nous avons
    %\begin{equation}
        %x=k_1p+lq-\delta=(k_1-a_{\delta})p+(l_1+b_{\delta})q
    %\end{equation}
    %avec \( k_1-a_{\delta}\geq k_1-A\) et \( l_1-b_{\delta}\geq l_1-B\). Si \( x\) est suffisamment grand pour avoir \( k_1>A\) et \( l_1>B\), alors la décomposition souhaitée est trouvée.
%
    %Une borne pour \( x\) est donnée par
    %\begin{equation}    \label{EqjQpURG}
        %x>\max\{ 2pA,2qB \}.
    %\end{equation}
%\end{proof}

\begin{normaltext}
    Une méthode pour obtenir les entiers naturels $u$ et $v$ qui permettent la décomposition \(x = au + bv \) est d'abord de choisir $u_0$ et $v_0$ tels que \( au_0 \) et \( bv_0 \) soient les plus proches possibles de $x/2$, puis de décomposer le nombre (relativement petit) \( x - au_0 - bv_0 \) en \( au_1 + bv_1 \). Deux nombres $u$ et $v$ qui fonctionnent sont alors $u = u_0 + u_1$ et $v = v_0 + v_1$.
\end{normaltext}

\begin{example}
    Écrivons \( 1000=u\cdot 7+v\cdot 5\) avec \( u,v\in \eN\). D'abord \( 72\cdot 7=504\) et \( 100\cdot 5=500\). Nous avons donc
    \begin{equation}
        1004=72\cdot 7+100\cdot 5.
    \end{equation}
    Ensuite \( 4=25-21=-3\cdot 7+5\cdot 5\). Au final,
    \begin{equation}
        1000=75\cdot 7+95\cdot 5.
    \end{equation}
\end{example}

%---------------------------------------------------------------------------------------------------------------------------
\subsection{Calcul effectif du PGCD et de Bézout}
%---------------------------------------------------------------------------------------------------------------------------
\label{subSecIpmnhO}

Soient \( a\) et \( b\), deux entiers que nous allons prendre positifs. Nous allons voir maintenant l'algorithme de \defe{Euclide étendu}{Euclide!algorithme étendu} qui est capable, pour \( a\) et \( b\) donnés, de calculer le PGCD de $a$ et $b$, et un couple de Bézout \( (u,v)\) tel que \( ua+vb=\pgcd(a,b)\). Ce calcul est indispensable si on veut implémenter RSA (\ref{SecEVaFYi}).

Cela se base sur le lemme suivant.

\begin{lemma}       \label{LemiVqita}
    Soient \( a,b\in \eN\) et des nombres \( q\) et \( r\) tels que \( a=qb+r\). Alors \( \pgcd(a,b)=\pgcd(r,b)\).
\end{lemma}

\begin{proof}
    Il suffit de voir que les diviseurs communs de \( a\) et \( b\) sont diviseurs de \( r\) et que les diviseurs communs de \( r\) et \( b\) divisent \( a\).

    Si \( s\) divise \( a\) et \( b\), alors dans l'équation
    \begin{equation*}
        \frac{ a }{ s }=\frac{ qb }{ s }+\frac{ r }{ s }
    \end{equation*}
    les termes \( a/s\) et \( qb/s\) sont entiers, donc \( r/s\) est aussi entier, et \( s\) divise \( r\).

    Inversement, si \( s\) divise \( r\) et \( b\), alors il divise \( qb+r\) et donc \( a\).
\end{proof}
\begin{remark}
    Ce lemme est surtout intéressant lorsque \( a=qb+r\) est la division euclidienne de \( a\) par \( b\): en effet, dans ce cas \( r < b \), et le calcul du PGCD de deux nombres ($a$ et $b$) est ramené à un calcul de PGCD de deux nombres plus petits ($b$ et $r$).

    L'algorithme pour calculer \( \pgcd(a,b)\) consiste à écrire des divisions euclidiennes successives de la manière suivante:
    \begin{subequations}
        \begin{align}
            a &= q_2 b   + r_2 && r_2<b\\
            b &= q_3 r_2 + r_3 && r_3<r_2\\
            &\vdots
        \end{align}
    \end{subequations}
    en remarquant que \( \pgcd(a,b)=\pgcd(b,r_2)=\pgcd(r_2,r_3) \). Étant donné que les inégalités \( r_2<b\) et \( r_3<r_2\) sont strictes, en continuant ainsi nous finissons sur zéro, c'est-à-dire qu'il existera un $n$ pour lequel \( r_{n+1} = 0 \); et donc
\begin{equation*}
    r_{n-1}=q_{n+1}r_n,
\end{equation*}
et à ce moment nous avons \( \pgcd(a,b)=\pgcd(r_{n-1},r_n)=r_n\).
\end{remark}

%///////////////////////////////////////////////////////////////////////////////////////////////////////////////////////////
\subsubsection{Algorithme d'Euclide pour le PGCD}
%///////////////////////////////////////////////////////////////////////////////////////////////////////////////////////////
\label{SUBSECooAEBLooFGJRkg}
\index{pgcd!calcul effectif}

Écrivons l'algorithme\cite{BezoutCos} en détail (parce que Bézout, ça va être la même chose en cinq fois plus compliqué). On pose
\begin{subequations}
    \begin{align}
        r_0=a\\
        r_1=b
    \end{align}
\end{subequations}
(ce qui explique que nous n'ayons pas utilisé $r_0$ et $r_1$ précédemment). Ensuite on écrit la division euclidienne \( a=q_2b+r_2\), c'est-à-dire \( r_0=q_2r_1+r_2\). Cela donne \( r_2\) et \( q_2\) en termes de \( r_0\) et \( r_1\) :
\begin{equation}
    r_2=r_0-q_2r_1.
\end{equation}
Ensuite, sachant \( r_2\) nous pouvons continuer :
\begin{equation}
    r_1=q_3r_2+r_3
\end{equation}
donne \( q_3\) et \( r_3=r_1-q_3r_2\). On continue avec \( r_2=q_4r_3+r_4\). Tout cela pour poser la suite
\begin{equation}
    \begin{aligned}[]
        r_0&=a\\
        r_1&=b\\
        r_k&=q_{k+2}r_{k+1}+r_{k+2}
    \end{aligned}
\end{equation}
où la troisième définit \( r_{k+2}\) et \( q_{k+2}\) en fonction de \( r_k\) et \( r_{k+1}\), à l'aide du théorème de la division euclidienne. La suite \( (r_k)\) ainsi construite est strictement décroissante et à chaque étape le lemme~\ref{LemiVqita} et le principe de l'algorithme d'Euclide nous donnent
\begin{subequations}
    \begin{numcases}{}
        \pgcd(r_k,r_{k+1})=\pgcd(r_{k+1},r_{k+2})=\pgcd(a,b)\\
        0\leq r_{k+1}<r_k.
    \end{numcases}
\end{subequations}
La suite étant strictement décroissante, nous prenons \( r_n\), le dernier non nul : \( r_{n+1}=0\). Dans ce cas la dernière équation sera
\begin{equation}
    r_{n-1}=q_nr_n
\end{equation}
avec \( \pgcd(a, b)=\pgcd(r_n,r_{n-1})=r_n\).

\begin{example}
    Calculons le PGCD de \( 18\) et \( 231\). Pour cela nous écrivons les divisions euclidiennes en chaine :
    \begin{subequations}
        \begin{align}
            231&=18\cdot 12+15\\
            18&=1\cdot 15 + 3\\
            15&=5\cdot 5+0.
        \end{align}
    \end{subequations}
    Donc le PGCD est \( 3\) (le dernier reste non nul).
\end{example}

%///////////////////////////////////////////////////////////////////////////////////////////////////////////////////////////
\subsubsection{Algorithme étendu: calcul effectif des coefficients de Bézout}
%///////////////////////////////////////////////////////////////////////////////////////////////////////////////////////////
\label{SUBSECooRHSQooEuBWbd}
\index{Bézout!calcul effectif}

La difficulté est de construire la suite qui donne des coefficients de Bézout. Elle va être construite à l'envers. Nous supposons déjà connaitre la liste complète des \( r_k\) jusqu'à \( r_n=\pgcd(a,b)\), ainsi que la liste complète des divisions euclidiennes
\begin{equation}
    r_k=q_{k+2}r_{k+1}+r_{k+2}.
\end{equation}

Nous voulons trouver les couples \( (u_k,v_k)\) de telle façon à avoir à chaque étape
\begin{equation}
    r_n=u_kr_k+v_kr_{k-1}.
\end{equation}
Notons que c'est à chaque fois \( r_n\) que nous construisons. La première équation de type Bézout à résoudre est
\begin{equation}
    r_n=u_nr_n+v_nr_{n-1},
\end{equation}
sachant que \( r_{n-1}=q_nr_n\). On pose \( v_n=0\) et \( u_n=1\) et c'est bon. Pour la récurrence, supposons les coefficients $u_k$ et $v_k$ connus, et déterminons les coefficients \( u_{k-1} \) et \( v_{k-1} \). Pour ce faire, nous égalons les deux expressions pour \( r_n\) :
\begin{equation}
    r_n=u_kr_k+v_kr_{k-1}=u_{k-1}r_{k-1}+v_{k-1}r_{k-2};
\end{equation}
dans laquelle nous substituons \( r_{k-2}=q_k-r_{k-1}+r_k\):
\begin{align}
    u_kr_k+v_kr_{k-1}&=u_{k-1}r_{k-1}+v_{k-1}(q_k r_{k-1}+r_k)\\
    &= (u_{k-1} + q_k v_{k-1}) r_{k-1} +v_{k-1} r_k
\end{align}
et nous égalons les coefficients de \( r_k\) et \( r_{k-1}\) pour obtenir
\begin{subequations}
    \begin{numcases}{}
        v_{k-1}=u_k\\
        u_{k-1}=v_k-v_{k-1}q_k.
    \end{numcases}
\end{subequations}
Dès que \( u_k\) et \( v_k\) ainsi que \( q_k\) sont connus, on peut calculer \( u_{k-1}\) et \( v_{k-1}\).

La dernière équation, celle avec \( k=1\), est
\begin{equation}
    r_n=u_1r_1+v_1r_0,
\end{equation}
c'est-à-dire
\begin{equation}        \label{EqNDMLooDvaiAc}
    \pgcd(a,b)=u_1b+v_1a.
\end{equation}
Nous avons ainsi trouvé des coefficients de Bézout pour $a$ et $b$.

%---------------------------------------------------------------------------------------------------------------------------
\subsection{Décomposition en facteurs premiers}
%---------------------------------------------------------------------------------------------------------------------------

\begin{lemma}[Lemme de Gauss]    \label{LemPRuUrsD}
    Soient \( a,b,c\in \eZ\) tels que \( a\) divise \( bc\). Si \( a\) est premier avec \( c\), alors \( a\) divise \( b\).
\end{lemma}
\index{lemme!de Gauss!pour des entiers}

\begin{proof}
    Vu que \( a\) est premier avec \( c\), nous avons \( \pgcd(a,c)=1\) et le théorème de Bézout~\ref{ThoBuNjam} nous donne donc \( s,t\in \eZ\) tels que \( sa+tc=1\). En multipliant par \( b\), nous avons $sab+tbc=b$. Mais les deux termes du membre de gauche sont multiples de \( a\) parce que \( a\) divise \( bc\). Par conséquent \( b\) est somme de deux multiples de \( a\) et donc est multiple de \( a\).
\end{proof}

\begin{lemma}[Lemme d'Euclide\cite{BTDWooZCyXfb}]       \label{LemAXINooOeuMJZ}
    Si un nombre premier $p$ divise le produit de deux nombres entiers $b$ et $c$, alors $p$ divise $b$ ou $c$.
\end{lemma}
\index{Euclide!lemme}

\begin{proof}
    Vu que \( p\) est premier, s'il ne divise pas \( a\) c'est que \( \pgcd(a,p)=1\). Dans ce cas le lemme de Gauss~\ref{LemPRuUrsD} implique que \( p\) divise \( b\).
\end{proof}
\index{lemme!d'Euclide}

Le théorème fondamental de l'arithmétique permet de décomposer des nombres en facteurs premiers.

\begin{theorem}[\cite{RATEooJuqgom}]        \label{ThoAJFJooAveRvY}
    Tout entier strictement positif peut être écrit comme un produit de nombres premiers d'une unique façon, à l'ordre près des facteurs.

    En d'autres termes, pour tout entier \( n>1\), il existe une suite finie unique $(p_1, k_1)$,\ldots $(p_r, k_r)$ telle que :
    \begin{enumerate}
        \item
    les \( p_i\) sont des nombres premiers tels que, si $i < j$, alors $p_i < p_j$ ;
    \item
    les \( k_i\) sont des entiers naturels non nuls ;
    \item
        \( n=\prod_{i=1}^rp_i^{k_i}\).
    \end{enumerate}
\end{theorem}

\begin{proof}
    Soit \( n\) un entier positif. Nous prouvons l'existence d'une décomposition en facteurs premiers par récurrence. Le nombre \( n=1\) est le produit d'une famille finie de nombres premiers : la famille vide.

    Supposons que tout entier strictement inférieur à un certain entier \( n>1\) est produit de nombres premiers. Deux possibilités apparaissent pour $n$ : il est premier ou non. Si $n$ est premier, et donc produit d'un unique entier premier, à savoir lui-même, le résultat est vrai. Si \( n\) n'est pas premier, il se décompose sous la forme $kl$ avec $k$ et $l$ strictement inférieurs à $n$. Dans ce cas, l'hypothèse de récurrence implique que les entiers $k$ et $l$ peuvent s'écrire comme produits de nombres premiers. Leur produit aussi, ce qui fournit une décomposition de $n$ en produit de nombres premiers.  Par application du principe de récurrence, tous les entiers naturels peuvent s'écrire comme produit de nombres premiers.

    Nous prouvons maintenant l'unicité. Prenons deux produits de nombres premiers qui sont égaux. Prenons n'importe quel nombre premier $p$ du premier produit. Il divise le premier produit, et, de là, aussi le second. Par le lemme d'Euclide~\ref{LemAXINooOeuMJZ}, $p$ doit alors diviser au moins un facteur dans le second produit. Mais les facteurs sont tous des nombres premiers eux-mêmes, donc $p$ doit être égal à un des facteurs du second produit. Nous pouvons donc simplifier par $p$ les deux produits. En continuant de cette manière, nous voyons que les facteurs premiers des deux produits coïncident précisément.
\end{proof}

\begin{lemma}[\cite{MonCerveau}]        \label{LEMooDTQQooYoJABt}
    Nous notons \( \mP\) l'ensemble des nombres premiers dans \( \eN\). Soient des suites finies \( (a_p)_{p\in \mP}\) et \( (b_p)_{p\in \mP}\). Nous posons
    \begin{equation}
        \begin{aligned}[]
            a&=\prod_{ p\mP}p^{a_p}&\text{ et }&&b=\prod_{ p\in \mP}p^{b_p}.
        \end{aligned}
    \end{equation}
    Alors \( a\divides b\) si et seulement si \( a_p\leq b_p\) pour tout \( p\).
\end{lemma}

\begin{proof}
    Dire que \( a\divides b\) signifie qu'il existe \( s\in \eN\) tel que \( as=b\); le théorème \ref{ThoAJFJooAveRvY} nous permet de décomposer \( s\) en \( s=\prod_{p\in\mP}p^{s_p}\). Vu que le produit dans \( \eN\) est commutatif et associatif,
    \begin{equation}
        b=as=\prod_{p\in\mP}p^{s_p+a_p}.
    \end{equation}
    Par unicité de la décomposition de \( b\) (toujours le théorème \ref{ThoAJFJooAveRvY}), nous en déduisons que \( b_p=s_p+a_p\geq a_p\).

    Dans l'autre sens, l'hypothèse \( a_p\leq b_p\) implique l'existence de \( s_p\geq 0\) tels que \( b+p=a_p+s_p\). En posant \( s=\prod_{p\in\mP}p^{s_p}\), nous avons
    \begin{equation}
        as=\prod_{p\in\mP}p^{s_p+a_p}=\prod_{p\in \mP}p^{b_p}=b.
    \end{equation}
    Donc \( a\divides b\).
\end{proof}

\begin{lemma}[\cite{MonCerveau}]       \label{LEMooBJVJooFyuFeN}
    Dans \( \eN\), le pgcd\footnote{Le pgcd et ppcm sont définis dans \ref{DefrYwbct}.} et le ppcm sont uniques.
\end{lemma}

\begin{proof}
    Supposons que \( \delta_1\) et \( \delta_2\) soient des pgcd de la partie \( S\). Vu que \( \delta_1\divides S\), nous avons \( \delta_1\divides \delta_2\) parce que \( \delta_2\) est un pgcd. Le même raisonnement, inversant \( \delta_1\) et \( \delta_2\) montre que \( \delta_2\divides \delta_1\). Si \( (a_p)\) sont les éléments de la décomposition de \( \delta_1\) et \( (b_p)\) ceux de \( \delta_2\), alors le lemme \ref{LEMooDTQQooYoJABt} nous indique que \( a_p\leq b_p\) et \( b_p\leq a_p\), ce qui implique que \( a_p=b_p\).

    Le ppcm se fait de même.
\end{proof}

\begin{proposition}     \label{PROPooNQBOooHWqTvs}
    Soient \( a,b\in \eZ\setminus\{ 0 \}\) décomposés en \( a=\prod_{p\in\mP}p^{a_p}\) et \( b=\prod_{p\in\mP}p^{b_p}\). En posant 
    \begin{subequations}
        \begin{align}
            m_p=\min\{ a_p,b_p \}\\
            M_p=\max\{ a_p,b_p \},
        \end{align}
    \end{subequations}
    nous avons
    \begin{subequations}
        \begin{align}
            \pgcd(a,b)&=\prod_{p\in\mP}p^{m_p}\\
            \ppcm(a,b)&=\prod_{p\in\mP}p^{M_p}.
        \end{align}
    \end{subequations}
\end{proposition}

\begin{proof}
    Nous notons $\delta=\prod_{p\in\mP}p^{m_p}$ et $\mu=\prod_{p\in\mP}p^{M_p}$. 
    
    Nous commençons par montrer que \( \delta\) est le pgcd de \( a\) et \( b\). Vu que \( \delta_p=\min\{ a_p,b_p \}\), nous avons \( \delta_p\leq a_p\) et \( \delta_p\leq b_p\). Le lemme \ref{LEMooDTQQooYoJABt} nous dit alors que \( \delta\divides a\) et \( \delta\divides b\). De même, si \(s\divides a\) et \( s\divides b\), nous avons \( s_p\leq a_p\) et \( s_p\leq b_p\), ce qui montre que \( s_p\leq m_p\) et donc que \( s\divides \delta\).

    Le fait que \( \mu\) soit le ppcm de \( a\) et \( b\) se montre de même.
\end{proof}

\begin{corollary}[\cite{MonCerveau}]  \label{CORooQIMHooUzLUJY}
    Un élément \( m\in \eZ^*\) vérifie \( m\leq p^n\) et \( \pgcd(m,p^n)\neq 1\) si et seulement si \( m=qp\) pour un certain \( q\leq p^{n-1}\).
\end{corollary}

\begin{probleme}
   Il faut vérifier si le corolaire~\ref{CORooQIMHooUzLUJY} est correct.
   Et puis rédiger des démonstrations de tout ce petit monde.
\end{probleme}

\begin{lemma}   \label{LemheKdsa}
    Un entier \( n\geq 1\) se décompose de façon unique en produit de la forme \( n=qm^2\) où \( q\) est un entier sans facteurs carrés et \( m\), un entier.
\end{lemma}

\begin{proof}
    Pour \( n=1\), c'est évident. Nous supposons \( n\geq 2\).

    En ce qui concerne l'existence, nous décomposons \( n\) en facteurs premiers\footnote{Théorème~\ref{ThoAJFJooAveRvY}.} et nous séparons les puissances paires des puissances impaires :
    \begin{subequations}
        \begin{align}
            n&=\prod_{i=1}^rp_p^{2\alpha_i}\prod_{j=1}^sq_{j}^{2\beta_j+1}\\
            &=\underbrace{\left( \prod_{i=1}^rp_i^{2\alpha_i}\prod_{j=1}^sq^{2\beta_j} \right)}_{m^2}\underbrace{\prod_{j=1}^sq_j}_{q}.
        \end{align}
    \end{subequations}

    Nous passons à l'unicité. Supposons que \( n=q_1m_1^2=q_2m_2^2\) avec \( q_1\) et \( q_2\) sans facteurs carrés (dans leur décomposition en facteurs premiers). Soit \( d=\pgcd(m_1,m_2)\) et \( k_1\), \( k_2\) définis par \( m_1=dk_1\), \( m_2=dk_2\). Par construction, \( \pgcd(k_1,k_2)=1\). Étant donné que
    \begin{equation}        \label{EqWPOtto}
        n=q_1d^2k_1^2=q_2d^2k_2^2,
    \end{equation}
    nous avons \( q_1k_1^2=q_2k_2^2\) et donc \( k_1^2\) divise \( q_2k_2^2\). Mais \( k_1\) et \( k_2\) n'ont pas de facteurs premiers en commun, donc \( k_1^2\) divise \( q_2\), ce qui n'est possible que si \( k_1=1\) (parce que \( k_1^2\) n'a que des facteurs premiers alors que \( q_2\) n'en a pas). Dans ce cas, \( d=m_1\) et \( m_1\) divise \( m_2\). Si \( m_2=lm_1\) alors l'équation \eqref{EqWPOtto} se réduit à  \( n=q_1m_1^2=q_2l^2m_1^2\) et donc
    \begin{equation}
        q_1=q_2l^2,
    \end{equation}
    ce qui signifie \( l=1\) et donc \( m_1=m_2\).
\end{proof}

Dans \( \eN\), il y a assez bien de nombres premiers. Nous allons voir maintenant que la somme des inverses des nombres premiers diverge. Pour comparaison, la somme des inverses des carrés converge. Il y a donc plus de nombres premiers que de carrés.

%--------------------------------------------------------------------------------------------------------------------------- 
\subsection{Ordre d'un élément dans un groupe fini}
%---------------------------------------------------------------------------------------------------------------------------

Il y a des informations en plus dans la partie sur les groupes monogènes, \ref{SECooXIHPooWVSjhT}.

\begin{theorem}[Théorème de Cauchy\cite{ooTZHGooEPFstf}]    \label{THOooSUWKooICbzqM}
    Soit un groupe fini d'ordre \( n\). Pour tout diviseur premier \( p\) de \( n\), le groupe \( G\) possède au moins un élément d'ordre \( p\).
\end{theorem}
\index{théorème!Cauchy!groupe}

Le lemme suivant indique que sous hypothèse de commutativité, l'ordre d'un élément est une notion multiplicative.
\begin{lemma}[\cite{rqrNyg}]    \label{LemyETtdy}
    Soit \( G\) un groupe et \( a,b\in G\) tels que \( ab=ba\) d'ordres respectivement \( r\) et \( s\), deux nombres premiers entre eux. Alors l'élément \( ab\) est d'ordre \( rs\).
\end{lemma}

\begin{proof}
    Étant donné que \( (ab)^{rs}=a^{rs}b^{rs}=1\), l'ordre de \( ab\) divise \( rs\). Et vu que \( r\) et \( s\) sont premiers entre eux, l'ordre de \( ab\) s'écrit sous la forme \( r_1s_1\) avec \( r_1\divides r\) et \( s_1\divides s\). Nous avons
    \begin{equation}
        a^{r_1s_1}b^{r_1s_1}=(ab)^{r_1s_1}=1,
    \end{equation}
    que nous élevons à la puissance \( r_2\) où \( r_2\) est définit en posant \(r=r_1r_2\) :
    \begin{equation}
        a^{rs_1}b^{rs_1}=1.
    \end{equation}
    Et comme \( a^{rs_1}=1\), nous concluons que \( b^{rs_1}=1\). Donc \( s\divides rs_1\). Par le lemme de Gauss \ref{LemPRuUrsD}, nous avons en fait \( s\divides s_1\). Vu qu'on a aussi \( s_1\divides s\), nous avons \( s=s_1\).

    Le même type d'argument donne \( r=r_1\), et finalement l'ordre de \( ab\) est \( r_1s_1=rs\).
\end{proof}

\begin{lemma}[\cite{Combes}]    \label{LemSkIOOG}
    Un sous-groupe d'indice \( 2\) est un sous-groupe normal.
\end{lemma}

\begin{proof}
    Si $H$ est un tel sous-groupe d'un groupe $G$, alors $G$ possède exactement deux classes à gauche par rapport à \( H\) (théorème de Lagrange~\ref{ThoLagrange}) et se partitionne donc en deux parties : \( G=H\cup xH\) avec \( x \notin H \). De même pour les classes à droite : \( G=H\cup Hx\). Puisque la classe à droite \( Hx \) n'est pas $H$, on a \( xH = Hx \), et $H$ est normal dans $G$ par la proposition~\ref{propGroupeNormal}.
\end{proof}

\begin{lemma}[\cite{NielsBMorph}]\label{PropubeiGX}
    Soit \( H\), un sous-groupe normal d'indice \( m\) de \( G\). Alors pour tout \( a\in G\) nous avons \( a^m\in H\).
\end{lemma}

\begin{proof}
    Par définition de l'indice, le groupe \( G/H\) est d'ordre \( m\). Donc si \( [a]\in G/H\), nous avons \( [a]^m=[e]\), ce qui signifie \( [a^m]=[e]\), ou encore \( a^m\in H\).
\end{proof}

\begin{proposition}[\cite{NielsBMorph}]     \label{PROPooVWVIooQzuAlA}
    Soit un groupe fini \( G\) et \( H\), un sous-groupe normal d'ordre \( n\) et d'indice \( m\) avec \( m\) et \( n\) premiers entre eux. Alors \( H\) est l'unique sous-groupe de \( G\) à être d'ordre \( n\).
\end{proposition}

\begin{proof}
    Soit \( H'\) un sous-groupe d'ordre \( n\). Si \( h\in H'\) alors \( h^n=1\) et \( h^m\in H\) par le lemme \ref{PropubeiGX}. Étant donné que \( m\) et \( n\) sont premiers entre eux, par le théorème de Bézout~\ref{ThoBuNjam}, il existe \( a,b\in \eZ\) tels que
    \begin{equation}
        am+bn=1.
    \end{equation}
    Du coup \( h=h^1=(h^m)^a(h^n)^b\). En tenant compte du fait que \( h^n=1\) et \( h^m\in H\), nous avons \( h\in H\). Ce que nous venons de prouver est que \( H'\subset H\) et donc que \( H=H'\) parce que \( | H' |=| H |=| G |/m\).
\end{proof}

\begin{normaltext}
    Notons que cette proposition ne dit pas qu'il existe un sous-groupe d'ordre \( n\) et d'indice \( m\). Il dit juste que s'il y en a un et s'il est normal, alors il n'y en a pas d'autres.
\end{normaltext}

\begin{lemma}       \label{LemqAUBYn}
    L'ensemble des ordres\footnote{Définition \ref{DEFooKWBCooMlmpCP}.} d'un groupe commutatif est stable par PPCM\footnote{Définition \ref{DefrYwbct}.}.

    Autrement dit, si \( x\in G\) est d'ordre \( r\) et si \( y\in G\) est d'ordre \( s\), alors il existe un élément d'ordre \( \ppcm(r,s)\).
\end{lemma}

\begin{proof}
    Soit \( m=\ppcm(r,s)\). Afin d'écrire \( m\) sous une forme pratique, nous considérons les décompositions en facteurs premiers de \( r\) et \( s\) :
    \begin{subequations}
        \begin{align}
            r&=\prod_{i=1}^kp_i^{\alpha_i}\\
            s&=\prod_{i=1}^kp_i^{\beta_i}
        \end{align}
    \end{subequations}
    où \( \{ p_i \}_{i=1\ldots k}\) est l'ensemble des nombres premiers arrivant dans les décompositions de \( r\) et de \( s\). Si nous posons
    \begin{subequations}
        \begin{align}
            r'&=\prod_{\substack{i=1\\\alpha_1>\beta_i}}^kp_i^{\alpha_i}\\
            s'&=\prod_{\substack{i=1\\\alpha_i\leq \beta_i}}^kp_i^{\beta_i},
        \end{align}
    \end{subequations}
    alors \( \ppcm(r,s)=r's'\) et \( r'\) et \( s'\) sont premiers entre eux. L'élément \( x^{r/r'}\) est d'ordre \( r'\) et l'élément \( y^{s/s'}\) est d'ordre \( s'\). Maintenant nous utilisons le fait que \( G\) soit commutatif et le lemme~\ref{LemyETtdy} pour conclure que l'ordre de \( x^{r/r'}y^{s/s'}\) est \( r's'=m\).
\end{proof}

%---------------------------------------------------------------------------------------------------------------------------
\subsection{Écriture des fractions}
%---------------------------------------------------------------------------------------------------------------------------

\begin{theorem}     \label{THOooWYQVooRBaAAM}
    Tout élément de \( \eQ^+\) s'écrit de façon unique comme quotient de deux entiers premiers entre eux.
\end{theorem}

\begin{proof}
    En deux parties\footnote{Définitions des pgcd et ppcm en \ref{DefrYwbct}.}
    \begin{subproof}
        \item[Unicité]
            Supposons avoir \( \frac{ a }{ b }=\frac{ c }{ d }\) avec \( \pgcd(a,b)=\pgcd(c,d)=1\). Nous avons
            \begin{equation}
                ad=bc
            \end{equation}
            donc
            \begin{enumerate}
                \item
                    \( a\) divise \( bc\) mais est premier avec \( b\) donc \( a\) divise \( c\) par le lemme de Gauss~\ref{LemPRuUrsD}.
                \item
                    \( c\) divise \( ad\) mais est premier avec \( d\) donc \( c\) divise \( a\) par le lemme de Gauss~\ref{LemPRuUrsD}.
            \end{enumerate}
            En conclusion \( a\) divise \( c\) et \( c\) divise \( a\), ergo \( a=c\). L'égalité \( b=d\) est alors immédiate.
        \item[Existence]
            Soit le quotient \( \frac{ a }{ b }\). Nous avons
            \begin{equation}
                \frac{ a }{ b }=\frac{ a/\pgcd(a,b) }{ b/\pgcd(a,b) },
            \end{equation}
            qui est encore un quotient d'entiers parce que \( \pgcd(a,b)\) divise aussi bien \( a\) que \( b\). Il faut montrer que les nombres \( a/\pgcd(a,b)\) et \( b/\pgcd(a,b)\) sont premiers entre eux. Pour cela nous supposons que \( k\) est un diviseur commun. En particulier, les nombres \( a/k\pgcd(a,b)\) et \( b/k\pgcd(a,b)\) sont des entiers, ce qui fait que \( k\pgcd(a,b)\) est un diviseur commun de \( a\) et \( b\). Étant donné que \( \pgcd(a,b)\) est le plus grand tel diviseur, nous devons avoir \( k\pgcd(a,b)=\pgcd(a,b)\) c'est-à-dire que \( k=1\). Donc les nombres \( a/\pgcd(a,b)\) et \( b/\pgcd(a,b)\) sont premiers entre eux.
    \end{subproof}
\end{proof}

\begin{proposition}     \label{PROPooRZDDooLJabov}
    Les entiers \( p\) et \( q\) sont premiers entre eux si et seulement si \( p^2\) et \( q^2\) sont premiers entre eux.
\end{proposition}

\begin{proof}
    Si \( p^2\) et \( q^2\) sont premiers entre eux, par le théorème de Bézout~\ref{ThoBuNjam} il existe \( a,b\in \eZ\) tels que
    \begin{equation}
        ap^2+bq^2=1
    \end{equation}
    Dans ce cas, \( (ap)p+(bq)q=1\), ce qui montre (par encore Bézout, mais l'autre sens) que \( p\) et \( q\) sont premiers entre eux.

    Réciproquement, supposons que \( p\) et \( q\) ne sont pas premiers entre eux. Alors \( \pgcd(p,q)=k\neq 1\). L'entier \( k\) divise \( p\) et donc \( p^2\); et l'entier \( k\) divise \( q\) et donc \( q^2\). Au final, \( k\) divise \( p^2\) et \( q^2\), ce qui montre que \( p^2\) et \( q^2\) ne sont pas premiers entre eux.
\end{proof}

Une des conséquences de ces résultats sera le fait que \( \sqrt{n}\) est irrationnelle dès que \( n\) n'est pas un carré parfait, théorème~\ref{THOooYXJIooWcbnbm}.

Nous avons déjà vu dans la proposition~\ref{PropooRJMSooPrdeJb} que \( \sqrt{2}\) était irrationnel. En fait le théorème suivant va nous montrer que le nombre \( \sqrt{ n }\) est soit entier, soit irrationnel.
\begin{theorem}     \label{THOooYXJIooWcbnbm}
    Soit \( n\in \eN\). Le nombre \( \sqrt{n}\) est rationnel si et seulement si \( n\) est un carré parfait.
\end{theorem}

\begin{proof}
    Supposons que \( \sqrt{n}\) soit rationnel. Le théorème~\ref{THOooWYQVooRBaAAM} nous donne \( p,q\in \eN\) premiers entre eux tels que \( \sqrt{n}=p/q\). La proposition~\ref{PROPooRZDDooLJabov} nous enseigne de plus que \( p^2\) et \( q^2\) sont premiers entre eux. Nous avons
    \begin{equation}
        p^2=nq^2.
    \end{equation}
    Le nombre $q$ est alors un diviseur commun de \( q^2\) et de \( p\). Donc \( q=1\) et \( n=p^2\) est un carré parfait.
\end{proof}

%---------------------------------------------------------------------------------------------------------------------------
\subsection{Équation diophantienne linéaire à deux inconnues}
%---------------------------------------------------------------------------------------------------------------------------
\label{subsecZVKNooXNjPSf}

\index{équation!diophantienne}


Soient \( a\), \( b\) et \( c\) des entiers naturels donnés. Nous considérons l'équation
\begin{equation}        \label{EqTOVSooJbxlIq}
    ax+by=c
\end{equation}
à résoudre\cite{PAYUooYVuNAB} pour \( (x,y)\in \eN^2\).

Si \( a\) ou \( b\) est nul, c'est facile; nous supposons donc que \( a\) et \( b\) sont tout deux non nuls. Nous commençons par simplifier l'équation en cherchant les diviseurs communs. Soit \( d=\pgcd(a,b)\) et notons \( a=da'\), \( b=db'\). Nous avons déjà l'équation
\begin{equation}
    d(a'x+b'y)=c,
\end{equation}
et donc si \( c\) n'est pas un multiple de \( d\), il n'y a pas de solutions\footnote{Exemple : \( 8x+2y=9\). Le membre de gauche est certainement un nombre pair et il n'y a donc pas de solutions.}. Si par contre \( c\) est un multiple de \( d\) alors nous écrivons \( c=c'd\) et l'équation devient
\begin{equation}
    a'x+b'y=c'
\end{equation}
C'est maintenant que nous utilisons le théorème de Bézout~\ref{ThoBuNjam} : vu que \( a'\) et \( b'\) sont premiers entre eux, nous avons la relation  \( a'u+b'v=1\) pour certains\footnote{Nous avons décrit un algorithme pour les trouver en démontrant l'équation~\ref{EqNDMLooDvaiAc}.} nombres entiers \( u\) et \( v\). Nous récrivons notre équation sous la forme \( a'x+b'y=c'(a'u+b'v)\) et rassemblons les termes :
\begin{equation}
    a'(x-c'u)=b'(c'v-y).
\end{equation}
C'est-à-dire que si \( (x,y)\) est une solution, alors \( a'\) divise \( b'(c'v-y)\). Mais comme \( a'\) et \( b'\) sont premiers entre eux, le nombre \( a'\) doit forcément\footnote{C'est Gauss~\ref{LemPRuUrsD} qui le dit, et vous savez que lorsque Gauss dit, c'est \emph{forcément} vrai.} diviser \( c'v-y\). Disons \( c'v-y=ka'\). Alors \( a'(x-c'u)=b'ka'\) et donc
\begin{equation}
    x=b'k+c'u.
\end{equation}
Nous trouvons alors une expression pour \( y\) en injectant cela dans  \( a'x+b'y=c'\) et en utilisant Bézout : \( a'c'u=(1-b'v)c'\). Au final nous avons prouvé que toutes les solutions sont de la forme
\begin{subequations}            \label{EqYCQVooZzHuRq}
    \begin{numcases}{}
        x=b'k+c'u\\
        y=vc'-a'k
    \end{numcases}
\end{subequations}
avec \( k\in\eZ\). Si nous voulons réellement seulement des solutions dans \( \eN\) et non dans \( \eZ\), il faut seulement un peu restreindre les valeurs de \( k\). Il en reste un nombre fini parce que l'équation pour \( x\) borne \( k\) vers le bas tandis que celle pour \( y\) borne \( k\) vers le haut.

Par ailleurs, il est très vite vérifié que tous les couples \( (x,y)\) de la forme \eqref{EqYCQVooZzHuRq} sont solutions.

\begin{example}
    Résoudre l'équation \( 2x+6y=52\).

    Nous pouvons factoriser \( 2\) dans le membre de gauche et simplifier alors toute l'équation par \( 2\) :
    \begin{equation}
        x+3y=26.
    \end{equation}
    Nous cherchons une relation de Bézout pour \( u+3v=1\). Ce n'est heureusement pas très compliqué : \( u=-5\), \( v=2\). Nous pouvons alors écrire
    \begin{equation}
        x+3y=26\times (-5+3\times 2),
    \end{equation}
    et donc \( x+5\times 26=3(y-26\times 6)\), et en posant \( k=y-26\times 6\) nous avons
    \begin{equation}
        x=3k-130.
    \end{equation}
    En injectant nous trouvons l'équation \( 3k-130+3y=26\) et donc
    \begin{equation}
        y=52-k.
    \end{equation}
\end{example}

%---------------------------------------------------------------------------------------------------------------------------
\subsection{Quotients}
%---------------------------------------------------------------------------------------------------------------------------

Nous écrivons \( a=b\mod p\) essentiellement s'il existe \( k\in \eZ\) tel que \( b+kp=a\). Plus généralement nous notons \( [a]_p=\{ a+kp|k\in \eZ \}\)\nomenclature[R]{\( [a]_p\)}{ensemble des \( a+kp\)} et l'écriture «\( a=n\mod p\)» peut tout autant signifier \( a=[b]_p\) que \( a\in [b]_p\). La différence entre les deux est subtile mais peut de temps en temps valoir son pesant d'or.

\begin{proposition}
    Soit \( n\in\eN\). Le groupe \( \eZ/n\eZ\) est monogène. Si \( n\neq 0\), le groupe \( \eZ/n\eZ\) est cyclique d'ordre \( n\).
\end{proposition}

\begin{proof}
    Nous considérons la surjection canonique \( \mu\colon \eZ\to \eZ/n\eZ\). Si \( a\in\eZ\), alors \( \mu(a)=a\mu(1)\). Par conséquent \( \eZ/n\eZ=\gr\bigl( \mu(1) \bigr)\) parce que tout groupe contenant \( \mu(1)\) contient tous les multiples de \( \mu(1)\), et par conséquent contient \( \mu(\eZ)=\eZ/n\eZ\).

    Soit \( x\in\eZ/n\eZ\) et considérons \( x_0\), le plus petit naturel représentant \( x\). Nous notons \( x=[x_0]\). Le théorème de la division euclidienne~\ref{ThoDivisEuclide} donne l'existence de \( q\) et \( r\) avec \( 0\leq r<n\) et \( q\geq 0\) tels que
    \begin{equation}
        x_0=nq+r.
    \end{equation}
    Nous avons \( [x_0]=[r]=\mu(r)\) parce que \( x_0-r\) est un multiple de \( n\). Nous avons donc \( [x_0]\in\mu(\eN_{n-1})\). Par conséquent
    \begin{equation}
        \eZ/n\eZ=\mu(\eZ)=\mu(\eN_{n-1}).
    \end{equation}
    La restriction \( \mu\colon \eN_{n-1}\to \eZ/n\eZ\) est donc surjective. Montrons qu'elle est également injective. Si \( \mu(x_0)=\mu(x_1)\), alors \( x_1=x_0+kn\). Si nous supposons que \( x_1>x_0\), alors \( k>0\) et si \( x_0\in\eN_{n-1}\), alors \( x_1>n-1\).

    L'ordre de \( \eZ/n\eZ\) est donc le même que le cardinal de \( \eN_{n-1}\), c'est-à-dire \( n\). Le groupe \( \eZ/n\eZ\) est donc fini, d'ordre \( n\) et monogène parce que \( \eZ/n\eZ=\gr(\mu(1))\). Il est donc cyclique.
\end{proof}

\begin{lemma}[\cite{KXjFWKA}]
    Soit \( q\in \eN\) avec \( q\geq 2\). Soient \( n,d\in \eN\) tels que \( q^d-1\divides q^n-1\). Alors \( d\divides n\).
\end{lemma}

\begin{proof}
    Par le théorème de division euclidienne~\ref{ThoDivisEuclide}, il existe \( a,b\in \eZ\) tels que \( n=ad+b\) avec \( 0\leq b<d\). En remarquant que \( q^d\in[1]_{q^d-1}\) nous avons
    \begin{equation}
        q^n=(q^d)^aq^b\in[1]_{q^d-1}q^b=[q^b]_{q^d-1}.
    \end{equation}
    Pour cela nous avons utilisé d'abord le fait que si \( a\in [z]_k\), alors \( a^n\in[z^n]_k\) et ensuite le fait que \( [1]_kx=[x]_k\). D'autre part l'hypothèse \( q^d-1\divides q^n-1\) implique
    \begin{equation}
        q^n\in[1]_{q^d-1}.
    \end{equation}
    Par conséquent le nombre \( q^n\) est à la fois dans \( [q^b]_{q^d-1}\) et dans \( [1]_{q^d-1}\). Cela implique que
    \begin{equation}
        [1]_{q^d-1}=[q^b]_{q^d-1},
    \end{equation}
    ou encore que \( q^b\in[1]_{q^d-1}\) ou encore que \( q^d-1\divides q^b-1\).

    Étant donné que \( b<d\) et que \( q\geq 2\), nous avons que \( q^b-1<q^d-1\); donc pour que \( q^d-1\) divise \( q^b-1\), il faut que \( q^b-1\) soit zéro, c'est-à-dire \( b=0\).

    Mais dire \( b=0\) revient à dire que \( d\divides n\), ce qu'il fallait démontrer.
\end{proof}

%+++++++++++++++++++++++++++++++++++++++++++++++++++++++++++++++++++++++++++++++++++++++++++++++++++++++++++++++++++++++++++
\section{Binôme de Newton et morphisme de Frobenius}
%+++++++++++++++++++++++++++++++++++++++++++++++++++++++++++++++++++++++++++++++++++++++++++++++++++++++++++++++++++++++++++

\begin{proposition}[\cite{ooPTQCooIWykWP}]     \label{PropBinomFExOiL}
Pour tout $x$, $y\in\eR$ et $n\in\eN$, nous avons
\begin{equation}        \label{EqNewtonB}
    (x+y)^n=\sum_{k=0}^n{n\choose k}x^{n-k}y^k
\end{equation}
où
\begin{equation}
    {n\choose k}=\frac{ n! }{ k!(n-k)! }
\end{equation}
sont les \defe{coefficients binomiaux}{coefficients binomiaux}.
\end{proposition}

\begin{proof}
    La preuve se fait par récurrence. La vérification pour $n=0$ et $n=1$ se fait aisément pour peu que l'on se rappelle que \( x^0=1\) et que \( 0!=1\), ce qui donne entre autres \( {0\choose 0}=1\).

    Supposons que la formule \eqref{EqNewtonB} soit vraie pour $n\geq1$, et prouvons la pour $n+1$. Nous avons
\begin{equation}        \label{EqBinTrav}
    \begin{aligned}[]
        (x+y)^{n+1} &=(x+y)\cdot  \sum_{k=0}^n{n\choose k}x^{n-k}y^k\\
                &= \sum_{k=0}^n{n\choose k}x^{n-k+1}y^k+\sum_{k=0}^n{n\choose k}x^{n-k}y^{k+1}\\
                &=x^{n+1}+ \sum_{k=1}^n{n\choose k}x^{n-k+1}y^k+\sum_{k=0}^{n-1}{n\choose k}x^{n-k}y^{k+1}+y^{n+1}.
    \end{aligned}
\end{equation}
La seconde grande somme peut être transformée en posant $i=k+1$ :
\begin{equation}
    \sum_{k=0}^{n-1}{n\choose k}x^{n-k}y^{k+1}  =\sum_{i=1}^n{n\choose i-1}x^{n-(i-1)}y^{i-1+1},
\end{equation}
dans lequel nous pouvons immédiatement renommer $i$ par $k$. En remplaçant dans la dernière expression de \eqref{EqBinTrav}, nous trouvons
\begin{equation}
    (x+y)^{n+1}=x^{n+1}+y^{n+1}+\sum_{k=1}^n\left[ {n\choose k}+{n\choose k-1} \right]x^{n-k+1}y^k.
\end{equation}
La thèse découle maintenant de la formule
\begin{equation}
    {n\choose k}+{n\choose k-1}={n+1\choose k}
\end{equation}
qui est vraie parce que
\begin{equation}
    \frac{ n! }{ k!(n-k)! }+\frac{ n! }{ (k-1)(n-k+1)! }=\frac{ n!(n-k+1)+n!k }{ k!(n-k+1)! }=\frac{ n!(n+1) }{  k!(n-k+1)!  },
\end{equation}
par simple mise au même dénominateur.
\end{proof}

%+++++++++++++++++++++++++++++++++++++++++++++++++++++++++++++++++++++++++++++++++++++++++++++++++++++++++++++++++++++++++++
\section{Idéal dans un anneau}
%+++++++++++++++++++++++++++++++++++++++++++++++++++++++++++++++++++++++++++++++++++++++++++++++++++++++++++++++++++++++++++

La définition d'un idéal dans un anneau est la définition~\ref{DefooQULAooREUIU}.


\begin{definition}[Idéal engendré par un élément]  \label{DefSKTooOTauAR}
    Si \( p\) est un élément d'un anneau \( A\) alors nous notons \( (p)\)\nomenclature[A]{\( (p)\)}{idéal engendré par \( p\)}\index{engendré!idéal dans un anneau} l'idéal dans \( A\) \defe{engendré}{engendré} par \( p\), c'est-à-dire \( pA\).
\end{definition}

\begin{definition}  \label{DefAJVTPxb}
    Un sous-ensemble \( B\subset A\) d'un anneau est un \defe{sous anneau}{sous anneau} si
    \begin{enumerate}
        \item
            \( 1\in B\)
        \item
            \( B\) est un sous-groupe pour l'addition
        \item
            \( B\) est stable pour la multiplication.
    \end{enumerate}
\end{definition}

\begin{remark}
    Un idéal n'est pas toujours un anneau parce que l'identité pourrait manquer. Un idéal qui contient l'identité est l'anneau complet.
\end{remark}

\begin{example}
    L'ensemble \( 2\eZ\) est un idéal de \( \eZ\). On peut aussi le noter \( (2) \).
\end{example}

\begin{proposition}[Premier théorème d'isomorphisme pour les anneaux]
    Soient \( A\) et \( B\) des anneaux et un homomorphisme \( f\colon A\to B\). Nous considérons l'injection canonique \( j\colon f(A)\to B\) et la surjection canonique \( \phi\colon A\to A/\ker f\). Alors il existe un unique isomorphisme
    \begin{equation}
        \tilde f \colon A/\ker f\to f(A)
    \end{equation}
    tel que \( f=j\circ\tilde f\circ\phi\).

    \begin{equation}
        \xymatrix{%
        A \ar[r]^{f}\ar[d]_{\phi}        &   B\ar[d]^{j}\\
           A/\ker f \ar[r]_{\tilde f}   &   f(A)\subset B
           }
    \end{equation}
\end{proposition}
\index{théorème!isomorphisme!premier!pour les anneaux}

\begin{proposition}     \label{PropIJJIdsousphi}
    Soient \( I\) un idéal de \( A\) et la projection canonique
    \begin{equation}
        \phi\colon A\to A/I.
    \end{equation}
    Elle est une bijection entre les idéaux de \( A\) contenant \( I\) et les idéaux de \( A/I\).

    Dit de façon imagée :
    \begin{equation}        \label{EqKbrizu}
        \{ \text{idéaux de } A\text{ contenant } I\}\simeq\{ \text{idéaux de } A/I \}.
    \end{equation}
\end{proposition}

\begin{proof}
    Si \( I\subset J\) et si \( J \) est un idéal de \( A\), alors \( \phi(J)\) est un idéal dans \( A/I\). En effet un élément de \( \phi(J)\) est de la forme \( \phi(j)\) et un élément de \( A/I\) est de la forme \( \phi(i)\). Leur produit vaut
    \begin{equation}
        \phi(i)\phi(j)=\phi(ij)\in\phi(J).
    \end{equation}

    Soit maintenant \( K\) un idéal dans \( A/I\) et soit \( J=\phi^{-1}(K)\). Étant donné qu'un idéal doit contenir \( 0\) (parce qu'un idéal est un groupe pour l'addition), \( [0]\in K\) et par conséquent \( I\subset\phi^{-1}(K)\).
\end{proof}
% TODO : il faudrait dire à peu près ici qu'une des utilités de Z_2 est le groupe modulaire PSL(2,Z)=SL(2,Z)/Z_2

\begin{proposition}[\cite{MonCerveau}]     \label{AnnCorpsIdeal}\label{PROPooUOCVooZGAVVk}
    Si \( A\) est un anneau, nous avons les équivalences
    \begin{enumerate}
        \item       \label{ITEMooLAAVooXhTcMe}
            \( A\) est un corps\footnote{Définition \ref{DefTMNooKXHUd}.}.
        \item       \label{ITEMooDGZIooRopYGx}
            \( A\) est non nul et ses seuls\footnote{Je vous laisse vous poser de grandes questions sur le fait que le vide est un idéal ou non.} seuls idéaux à gauche sont \( \{ 0 \}\) et \( A\).
        \item       \label{ITEMooLPWHooDJpTbR}
            \( A\) est non nul et ses seuls idéaux à droite sont \( \{ 0 \}\) et \( A\).
    \end{enumerate}
\end{proposition}

\begin{proof}
    Nous allons montrer que le point \ref{ITEMooLAAVooXhTcMe} est équivalent aux deux autres.
    \begin{subproof}
        \item[\ref{ITEMooLPWHooDJpTbR} implique \ref{ITEMooDGZIooRopYGx}]
            Si \( I\) est un idéal à gauche différent de \( \{ 0 \}\), alors il contient un certain \( a\neq 0\). Vu que \( A\) est un corps, il contient un inverse \( a^{-1}\), et comme \( I\) est un idéal, \( a^{-1} I\subset I\). En particulier \( a^{-1}a\in I\). Donc \( 1\in I\) et \( I=A\).
        \item[\ref{ITEMooDGZIooRopYGx} implique \ref{ITEMooLAAVooXhTcMe}]

            Supposons que les seuls idéaux de \( A\) soient \( \{ 0 \}\) et \( A\). Soit \( a\in A\). Si \( a\) est non nul, alors \( aA=A\), en particulier, \( 1\in aA\), c'est-à-dire qu'il existe \( b\in A\) tel que \( ab=1\). L'élément \( a\) est donc inversible.
    \end{subproof}
\end{proof}

\begin{definition}\label{DEFIdealMax}
Un idéal \( I\) dans un anneau \( A \) est dit \defe{idéal maximal}{idéal!maximal}\index{idéal!maximal} ou idéal maximum si tout idéal \( J \) vérifiant \( I \subset J \subset A \) est soit \( I \), soit \( A \).
\end{definition}

\begin{proposition}[Thème~\ref{THEMEooZYKFooQXhiPD}]     \label{PROPooSHHWooCyZPPw}
    Un idéal \( I\) dans un anneau \( A \) est maximum si et seulement si \( A/I\) est un corps.
\end{proposition}

\begin{proof}
    Soit un idéal maximum \( I\subset A\). Alors les idéaux contenant \( I\) sont \( A\) et \( I\). L'application \( \phi\) de la proposition~\ref{PropIJJIdsousphi} est une bijection, donc l'ensemble des idéaux de \( A/I\) ne contient que deux éléments. Les seuls idéaux de \( A/I\) sont donc \( \{ 0 \}\) et \( A/I\); donc \( A/I\) est un corps par la proposition~\ref{PROPooUOCVooZGAVVk}.

    Dans l'autre sens, c'est la même chose : si \( A/I\) est un corps, il possède exactement deux idéaux, donc \( A\) ne contient que deux idéaux contenant $I$. Donc \( I\) est un idéal maximum.
\end{proof}

%---------------------------------------------------------------------------------------------------------------------------
\subsection{Résultats supplémentaires sur l'anneau des entiers}
%---------------------------------------------------------------------------------------------------------------------------

\begin{corollary}       \label{CORooLINXooBlUKPG}
    Les quotients de \( \eZ\) sont \( \eZ/n\eZ\).
\end{corollary}

\begin{proof}
    Tous les idéaux de \( \eZ\) sont de la forme \( n\eZ\). En effet en vertu de la proposition~\ref{PropSsgpZestnZ}, les seuls sous-groupes de \( \eZ\) (en tant que groupe additif) sont les \( n\eZ\). Tous les idéaux sont donc de cette forme. De plus les \( n\eZ\) sont effectivement tous des idéaux : si \( a\in n\eZ\) et si \( k\in \eZ\) alors \( ak\in n\eZ\). Cela est donc un idéal.
\end{proof}

\begin{proposition}     \label{PropZpintssiprempUzn}
    Soient \( n\geq 2\) un entier et \( \phi\colon \eZ\to \eZ/n\eZ\) la surjection canonique. Nous noterons \( \tilde a=\phi(a)\). Alors l'ensemble des inversibles de \( \eZ/n\eZ\) est donné par
    \begin{equation}
        U(\eZ/n\eZ)=\phi(P_n)=\{ \tilde x\tq 0\leq x\leq n\tq\pgcd(x,n)=1 \}.
    \end{equation}
    où \( P_n\) est l'ensemble $P_n=\{ x\in\{ 0,\ldots,n-1 \}\tq\pgcd(x,n)=1 \}$.

    De plus,
    \begin{equation}
        \Card\big( U(\eZ/n\eZ) \big)=\varphi(n).
    \end{equation}
\end{proposition}

\begin{proof}
    Soit \( 0\leq x\leq n\) tel que \( \pgcd(x,n)=1\). Il existe donc\footnote{Théorème de Bézout~\ref{ThoBuNjam}} \( u,v\in\eZ\) tels que \( ux+vn=1\). En passant aux classes,
    \begin{equation}
        \tilde u\tilde x=\tilde 1,
    \end{equation}
    donc \( \tilde u\) est l'inverse de \( \tilde x\). Cela prouve que \( \phi(P_n)\subset U(\eZ/n\eZ)\).

    Nous prouvons maintenant l'inclusion inverse. Soient \( \tilde x\) et \( \tilde y\) inverses l'un de l'autre : $\tilde x\tilde y=\tilde 1$. Il existe donc \( q\in\eZ\) tel que \( xy-qn=1\), ce qui prouve\footnote{À nouveau avec le Théorème de Bézout.} que \( \pgcd(x,n)=1\).
\end{proof}

%+++++++++++++++++++++++++++++++++++++++++++++++++++++++++++++++++++++++++++++++++++++++++++++++++++++++++++++++++++++++++++
\section{Caractéristique}
%+++++++++++++++++++++++++++++++++++++++++++++++++++++++++++++++++++++++++++++++++++++++++++++++++++++++++++++++++++++++++++

\begin{lemmaDef}        \label{LEMDEFooVEWZooUrPaDw}
    Soit l'application
    \begin{equation}
        \begin{aligned}
            \mu\colon \eZ&\to A \\
            n&\mapsto n\cdot 1_A .
        \end{aligned}
    \end{equation}
    \begin{enumerate}
        \item
            C'est un morphisme d'anneaux.
        \item
            Le noyau est un sous-groupe de \( \eZ\)
        \item
            Il existe un unique \( p\in \eZ\) tel que \( \ker(\mu)=p\eZ\).
    \end{enumerate}
    Ce \( p\) est la \defe{caractéristique}{caractéristique!d'un anneau} de \( A\).
\end{lemmaDef}

Par exemple la caractéristique que \( \eQ\) est zéro parce qu'aucun multiple de l'unité n'est nul.

À propos de diagonalisation en caractéristique \( 2\), voir l'exemple~\ref{ExewINgYo}.

\begin{lemma}
    Si \( A\) est de caractéristique nulle, alors \( A\) est infini.
\end{lemma}

\begin{proof}
    En effet, \( \ker\mu=\{0\} \) implique que \( n1_A \neq  m1_A\) dès que \(n \neq m \) et par conséquent \( A\) contient \(\eZ 1_A \), et  est infini.
\end{proof}

\begin{lemma}       \label{LemHmDaYH}
    Si \( p\) est la caractéristique de l'anneau \( A\), alors nous avons l'isomorphisme d'anneaux
    \begin{equation}
         \eZ 1_A\simeq\eZ/p\eZ.
    \end{equation}
\end{lemma}

\begin{proof}
    L'isomorphisme est donné par l'application \( n1_A\mapsto \phi(n)\) si \( \phi\) est la projection canonique \( \eZ\to \eZ/p\eZ\).
\end{proof}

\begin{proposition}     \label{PropGExaUK}
    La caractéristique d'un anneau fini divise son cardinal.
\end{proposition}

\begin{proof}
    Si \( A\) est un anneau, le groupe \( \eZ\) agit sur \( A\) par
    \begin{equation}
        n\cdot a=a+n1_A.
    \end{equation}
    Chaque orbite de cette action est de la forme
    \begin{equation}
        \mO_a=\{ a+n1_A\tq n=0,\ldots, p-1 \}
    \end{equation}
    où \( p\) est la caractéristique de \( A\). Les orbites ont \( p\) éléments et forment une partition de \( A\), donc le cardinal de \( A\) est un multiple de \( p\).
\end{proof}

\begin{lemma}[\cite{ooIBWOooSjOvXd}]        \label{LEMooJQIKooQgukqn}
    Un anneau totalement ordonné est de caractéristique nulle.
\end{lemma}

\begin{proof}
    Le morphisme \( \mu\colon \eZ\to A\), \( n\mapsto n 1_A\) est strictement croissant, en particulier \( \mu(x)\neq \mu(y)\) dès que \( x\neq y\). Donc \( \ker(\mu)=\{ 0 \}\).
\end{proof}

L'ensemble typique de caractéristique \( p\) est \( \eF_p=\eZ/p\eZ\).



\begin{proposition}     \label{Propqrrdem}
    Soit \( A\) un anneau commutatif de caractéristique première \( p\). Alors \( \sigma(x)=x^p\) est un automorphisme de l'anneau \( A\). Nous avons la formule
    \begin{equation}
        (a+b)^p=a^p+b^p
    \end{equation}
    pour tout \( a,b\in A\).
\end{proposition}

\begin{proof}
    Nous utilisons la formule du binôme de la proposition~\ref{PropBinomFExOiL} et le fait que les coefficients binomiaux non extrêmes sont divisibles par \( p\) et donc nuls.
\end{proof}

\begin{proposition} \label{PropFrobHAMkTY}
    Soit \( A\) un anneau commutatif unitaire de caractéristique \( p\). L'application
    \begin{equation}
        \begin{aligned}
            \Frob_A\colon A&\to A \\
            x&\mapsto x^p
        \end{aligned}
    \end{equation}
    est un automorphisme d'anneau unitaire.
\end{proposition}
Nous le nommons le \defe{morphisme de Frobenius}{morphisme!Frobenius}\index{Frobenius!morphisme}. Nous utiliserons aussi les itérés du morphisme de Frobenius : \( \Frob^k\colon x\mapsto x^{p^k}\).

\begin{example}
    Soit à factoriser \( X^p-1\) dans \( \eF_p\). Grâce au morphisme de Frobenius, nous avons immédiatement
    \begin{equation}
        X^p-1=(X-1)^p.
    \end{equation}
\end{example}

%+++++++++++++++++++++++++++++++++++++++++++++++++++++++++++++++++++++++++++++++++++++++++++++++++++++++++++++++++++++++++++
\section{Module sur un anneau}
%+++++++++++++++++++++++++++++++++++++++++++++++++++++++++++++++++++++++++++++++++++++++++++++++++++++++++++++++++++++++++++

\begin{definition}[module sur un anneau\cite{ooJGVOooSjQBVh}]       \label{DEFooHXITooBFvzrR}
    Soit un anneau \( A\). Un \defe{module à gauche}{module!à gauche} sur \( A\) est la donnée d'un triple \( (M,+,\cdot)\) où
    \begin{enumerate}
        \item
            \( +\) est une loi de composition interne à \( M\), c'est-à-dire \( +\colon M\times M\to M\),
        \item
            \( \cdot\) est une loi de composition externe, c'est-à-dire \( \cdot\colon A\times M\to M\)
    \end{enumerate}
    telles que
    \begin{enumerate}
        \item
            \( (M,+)\) est un groupe\footnote{Nous verrons dans la proposition~\ref{PROPooGARGooDiMqtN} qu'il est forcément commutatif.}.
        \item
            \( a\cdot(x+y)=a\cdot x+a\cdot y\),
        \item
            \( (a+b)\cdot x=a\cdot x+b\cdot x\),
        \item
            \( (ab)\cdot x=a\cdot(b\cdot x)\)
        \item
            \( 1\cdot x=x\).
    \end{enumerate}
    pour tout \( a,b\in A\) et \( x,y\in M\).
\end{definition}

\begin{proposition}\label{PROPooGARGooDiMqtN}
    Si \( M\) est un module sur un anneau, alors \( (M,+)\) est un groupe commutatif.
\end{proposition}

\begin{proof}
    Il suffit de calculer \( (1+1)\cdot (x+y)\) de deux façons différentes :
    \begin{equation}
        (1+1)\cdot (x+y)=1\cdot (x+y)+1\cdot (x+y)=x+y+x+y
    \end{equation}
    d'une part et
    \begin{equation}
        (1+1)\cdot (x+y)=(1+1)\cdot x+(1+1)\cdot y=x+x+y+y,
    \end{equation}
    d'autre part. En égalant les deux expressions, il vient
    \begin{equation}
        x+y+x+y=x+x+y+y,
    \end{equation}
    qui se simplifie (nous sommes dans un groupe) en \( y+x=x+y\).
\end{proof}

\begin{definition}\label{DEFooKHWZooIfxdNc}
    Un \defe{espace vectoriel}{espace!vectoriel} est un module sur un corps commutatif\footnote{La condition de commutativité n'est pas indispensable, mais comme nous ne parlerons que de corps commutatifs\ldots}.
\end{definition}

\begin{definition}[\cite{BIBooSTWWooItiMUp}]        \label{DEFooRUKVooLnXxdS}
    Soient un \( A\)-module \( M\) et un ensemble \( I\). Une famille \( (m_i)_{i\in I}\) est \defe{libre}{partie libre!module} si ils sont \defe{linéairement indépendants}{linéairement indépendant!module}, c'est-à-dire si pour tout choix d'une partie finie \( J\) dans \( I\) et d'éléments \( (a_j)_{j\in J}\) dans \( A\), si nous avons
    \begin{equation}
        \sum_{j\in J}a_jm_j=0,
    \end{equation}
    alors \( a_j=0\) pour tout \( j\).
\end{definition}

\begin{definition}[\cite{BIBooNKWVooYfrwSd}]        \label{DEFooWBOBooJNyyBF}
    Soit \( S\), une partie du \( A\)-module \( M\). Le \defe{sous-module engendré}{sous-module engendré} par \( S\) est l'ensemble des éléments de \( M\) qui sont des combinaisons linéaires finies d'éléments de \( S\), c'est-à-dire de sommes de la forme
    \begin{equation}
        \sum_{t\in T}a_tt
    \end{equation}
    où \( T\) est fini dans \( S\) et \( a_t\in A\).
\end{definition}

%--------------------------------------------------------------------------------------------------------------------------- 
\subsection{Module produit}
%---------------------------------------------------------------------------------------------------------------------------

\begin{definition}[\cite{BIBooSTWWooItiMUp}]        \label{DEFooLCJEooBvVmkV}
    Soient un anneau \( A\) et un ensemble \( I\). Le \( A\)-\defe{module produit}{module produit} \( A^I\) est l'ensemble des applications \( I\to A\).

    En termes de notations, nous écrivons ceci :
    \begin{equation}
        A^I=\{ (a_i)_{i\in I},a_i\in A \}.
    \end{equation}
    L'ensemble \( A^I\) devient un module par les définition, pour \( x,y\in A^I\) et \( a\in A\) :
    \begin{subequations}
        \begin{align}
            ax&=(ax_i)_{i\in I}\\
            x+y&=(x_i+y_i)_{i\in I}     \label{EQooODBMooQKLUgd}.
        \end{align}
    \end{subequations}
    En d'autres termes, \( A^I=\Fun(I,A)\).
\end{definition}

\begin{lemma}
    Pour chaque \( i\in I\) nous considérons l'élément \( e_i\in A^I\) donné par
    \begin{equation}
        e_i=(\delta_{ij})_{j\in I}.
    \end{equation}
    La famille \( \{ e_i \}_{i\in I}\) est libre\footnote{Définition \ref{DEFooRUKVooLnXxdS}.} dans \( A^I\).
\end{lemma}

\begin{proof}
    Soient \( J\) fini dans \( I\) ainsi que des éléments \( a_j\in A\) (\( j\in J\)). Nous supposons que\footnote{Pour rappel, les sommes finies sont définies par \ref{DEFooLNEXooYMQjRo}.} \( \sum_{j\in J}a_je_j=0\). Calculons un peu :
    \begin{equation}
        \sum_{j\in J}a_je_j=\sum_{j\in J}(a_j\delta_{ji})_{i\in I}=\left( \sum_{j\in J}a_j\delta_{ji} \right)_{i\in I}.
    \end{equation}
    Pour que le tout soit nul dans \( A^I\), il faut que
    \begin{equation}
        \sum_{j\in J}a_j\delta_{ji}
    \end{equation}
    soit nul pour tout \( i\in I\). Si nous fixons \( i\in I\), la somme sur \( j\) possède un seul terme non annulé par \( \delta_{ji}\), et c'est le terme \( j=i\). Nous avons donc \( a_i=0\).
\end{proof}

\begin{normaltext}
    Nous notons \( A^{(I)}\) le sous-module de \( A^I\) engendré\footnote{Définition \ref{DEFooWBOBooJNyyBF}.} par les \( e_i\).
\end{normaltext}

\begin{theorem}[Propriété universelle de \( A^{(I)}\)\cite{BIBooSTWWooItiMUp}]      \label{THOooPDZCooJnHbOd}
    Soient un anneau \( A\) ainsi qu'un \( A\)-module \( P\). Pour \( \phi\in\Hom_A(A^{(I)}, P)\), nous considérons
    \begin{equation}
        \begin{aligned}
            \phi|_I\colon I&\to P \\
            i&\mapsto \phi(e_i). 
        \end{aligned}
    \end{equation}
    \begin{enumerate}
        \item
            
    L'application
    \begin{equation}
        \begin{aligned}
            f\colon \Hom_A(A^{(I)},P)&\to \Fun(I,P) \\
            \phi&\mapsto \phi|_I 
        \end{aligned}
    \end{equation}
    est une bijection.
\item
    L'application inverse est \( g\colon \Fun(I,P)\to \Hom_A(A^{(I)},P) \) donnée par
    \begin{equation}
        g(\psi)\big( \sum_{j\in J}a_je_j \big)=\sum_{j\in J}a_j\psi(j)
    \end{equation}
    pour tout \( J\) fini dans \( I\) et choix de \( a_j\in A\).
    \end{enumerate}
\end{theorem}

\begin{proof}
    Nous allons montrer que \( g\big( f(\phi) \big)=\phi\) et que \( f\big( g(\psi) \big)=\psi\) pour tout \( \phi\in\Hom_A(A^{(I)},P)\) et pour tout \( \psi\in \Fun(I,P)\).

    Dans un premier sens nous avons :
    \begin{subequations}
        \begin{align}
            g\big( f(\phi) \big)\big( \sum_ja_je_j \big)&=\sum_ja_jf(\phi)(j)\\
            &=\sum_ja_j\phi(e_j)\label{SUBALIGNooBWPLooHeIaQg}\\
            &=\phi(\sum_ja_je_j)        \label{SUBALIGNooUOQPooCwLgZo}.
        \end{align}
    \end{subequations}
    Justifications :
    \begin{itemize}
        \item 
            Pour \eqref{SUBALIGNooBWPLooHeIaQg}, nous avons utilisé le fait que \( f(\phi)(i)=\phi|_I(i)=\phi(e_i)\).
        \item
            Pour \eqref{SUBALIGNooUOQPooCwLgZo}, nous utilisons le fait que \( \phi\) est un morphisme de modules.
    \end{itemize}
    Et pour l'autre sens,
    \begin{equation}
        f\big( g(\psi) \big)(i)=g(\psi)(e_i)=\psi(i).
    \end{equation}
\end{proof}

%--------------------------------------------------------------------------------------------------------------------------- 
\subsection{Sous-module}
%---------------------------------------------------------------------------------------------------------------------------

Soient \( M\) un \( A\)-module et \( x=(x_i)_{i\in I}\) une famille d'éléments de \( M\) paramétrée par l'ensemble \( I\). Nous considérons l'application
\begin{equation}
    \begin{aligned}
        \mu_x\colon A^{(I)}&\to M \\
        (a_i)_{i\in I}&\mapsto \sum_{i\in I}a_ix_i.
    \end{aligned}
\end{equation}
Ici \( A^{(I)}\) désigne l'ensemble de toutes les applications \( I\to A\) de support fini.

\begin{definition}      \label{DefBasePouyKj}
    À l'instar des espaces vectoriels, les modules ont une notion de partie libre, génératrice et de bases :
    \begin{enumerate}
        \item
            Si \( \mu_x\) est surjective, nous disons que \( x\) est une partie \defe{génératrice}{génératrice!partie d'un module}.
        \item
            Si \( \mu_x\) est injective, nous disons que la partie \( x\) est \defe{libre}{libre!partie d'un module}.
        \item
            Si \( \mu_x\) est bijective, nous disons que la partie \( x\) est une \defe{base}{base!d'un module}.
    \end{enumerate}
\end{definition}

\begin{definition}
  Un sous-ensemble \( N\subset M\) est un \defe{sous-module}{sous-module} si \( (N,+)\) est un sous-groupe de \( (M,+)\) et si \( a\cdot x\in N\) pour tout \( x\in N\) et pour tout \( a\in A\).
\end{definition}

\begin{example}
    Un anneau \( A\) est lui-même un \( A\)-module et ses sous-modules sont les idéaux.
\end{example}

\begin{definition}
    Soit \( M\) un module sur un anneau commutatif \( A\). Un \defe{projecteur}{projecteur!dans un module} est une application linéaire \( p\colon M\to M\) telle que \( p^2=p\).

    Une famille \( (p_i)_{i\in I}\) sur \( M\) est \defe{orthogonale}{orthogonal!famille de projecteurs} si \( p_i\circ p_j=0\) pour tout \( i\neq j\). La famille est \defe{complète}{complète!famille de projecteurs} si \( \sum_{i\in I}p_i=\mtu\).
\end{definition}

\begin{theorem}     \label{ThoProjModpAlsUR}
    Soient des sous modules \( M_1,\ldots,M_n\) du module \( M \) tels que \( M=M_1\oplus\ldots\oplus M_n\). Les applications \( p_i\) définies par
    \begin{equation}
        p_i(x_1+\ldots+x_n)=x_i
    \end{equation}
    forment une famille orthogonale de projecteurs et \( p_1+\cdots +p_n=\id\).

    Inversement, si \( (p_1,\ldots, p_n)\) est une famille orthogonale de projecteurs dans un module \( \modE\) tel que \( \sum_{i=1}^np_i=\id\), alors
    \begin{equation}
        M=\bigoplus_{i=1}^np_i(M).
    \end{equation}
\end{theorem}

\begin{definition}
    Un module est \defe{simple}{simple!module}\index{module!simple} ou \defe{irréductible}{irréductible!module}\index{module!irréductible} s'il n'a pas d'autres sous-modules que \( \{ 0 \}\) et lui-même. Un module est \defe{indécomposable}{indécomposable!module}\index{module!indécomposable} s'il ne peut pas être écrit comme somme directe de sous-modules.
\end{definition}

Un module simple est a fortiori indécomposable. L'inverse n'est pas vrai comme le montre l'exemple suivant.

\begin{example}
    Soit \( \modE=\eC[X]/(X^2)\) vu comme \( \eC[X]\)-module. C'est le \( \eC[X]\)-module des polynômes de la forme \( aX+b\) avec \( a,b\in \eC\). L'ensemble des polynômes de la forme \( aX\) est un sous-module. Le module \( \modE\) n'est donc pas simple. Il est cependant indécomposable parce que \( \{ aX \}\) est le seul sous-module non trivial. En effet si \( \modF\) est un sous-module de \( \modE\) contenant \( aX+b\) avec \( b\neq 0\), alors \( \modF\) contient \( X(aX+b)=bX\) et donc contient tout \( \modE\).
\end{example}

\begin{definition}[Algèbre\cite{ZSyHmiy}]   \label{DefAEbnJqI}
    Si \( \eK\) est un corps commutatif\footnote{Définition~\ref{DefTMNooKXHUd}}, une \( \eK\)-\defe{algèbre}{algèbre} \( A\) est un espace vectoriel\footnote{Définition~\ref{DEFooKHWZooIfxdNc}.} muni d'une opération bilinéaire \( \times\colon A\times A\to A\), c'est-à-dire telle que pour tout \( x,y,z\in A\) et pour tout \( \alpha,\beta\in\eK\),
    \begin{enumerate}
        \item
            \( (x+y)\times z=x\times z+y\times z\)
        \item
            \( x\times (y+z)=x\times y+x\times z\)
        \item
            \( (\alpha x)\times (\beta y)=(\alpha\beta)(x\times y)\).
    \end{enumerate}
    Si \( A\) et \( B\) sont deux \( \eK\)-algèbres, une application \( f\colon A\to B\) est un \defe{morphisme d'algèbres}{morphisme!d'algèbres} entre \( A\) et \( B\) si pour tout \( x,y\in A\) et pour tout \( \alpha\in \eK\),
    \begin{enumerate}
        \item
            \( f(xy)=f(x)f(y)\)
        \item
            \( f(x+\alpha y)=f(x)+\alpha f(y)\)
    \end{enumerate}
    où nous avons noté \( xy\) pour \( x\times y\).
\end{definition}

\begin{lemma}[\cite{MonCerveau}]   \label{LEMooVKLKooSAHmpZ}
    Soient une algèbre \( A\) et une famille \( (X_i)_{i\in I}\) de sous-algèbres de \( A\) (ici \( I\) est un ensemble quelconque). Alors la partie \( X=\bigcap_{i\in I}X_i\) est une sous-algèbre de \( A\).
\end{lemma}

\begin{proof}
    Nous devons prouver que si \( x\) et \( y\) sont dans \( X\) et \( \lambda\in \eK\), alors \( xy\), \( x+y\) et \( \lambda x\) sont dans \( X\). Pour tout \( i\in I\) nous avons \( x,y\in X_i\) et donc \( xy\in X_i\), \( x+y\in X_i\) et \( \lambda x\in X_i\) (parce que \( X_i\) est une algèbre). Donc \( xy\),\( x+y\) et \( \lambda x\) sont dans \( X_i\) pour tout \( I\), et donc dans \( X\).
\end{proof}

\begin{definition}\label{DefkAXaWY}
    L'\defe{algèbre engendrée}{algèbre!engendrée} par \( X\) est l'intersection de toutes les sous-algèbres de \( A\) contenant \( X\) (qui est une algèbre par le lemme~\ref{LEMooVKLKooSAHmpZ}).
\end{definition}

%+++++++++++++++++++++++++++++++++++++++++++++++++++++++++++++++++++++++++++++++++++++++++++++++++++++++++++++++++++++++++++
\section{Polynômes}
%+++++++++++++++++++++++++++++++++++++++++++++++++++++++++++++++++++++++++++++++++++++++++++++++++++++++++++++++++++++++++++

%--------------------------------------------------------------------------------------------------------------------------- 
\subsection{Polynômes d'une variable}
%---------------------------------------------------------------------------------------------------------------------------

Et voila la définition que tout le monde attendait; la définition des anneaux de polynômes. Pour ne pas taper trop fort du premier coup, nous commençons par les polynômes d'une seule variable.

Nous allons définir et étudier ici l'anneau des polynômes sur un anneau \( A\), c'est-à-dire ce qui sera noté \( A[X]\). Pour \( \eK(X)\) lorsque \( \eK\) est un corps, voir~\ref{DEFooQPZIooQYiNVh}.

L'ensemble des polynômes sur \( A\) sera simplement \( A^{(\eN)}\). Vu que \( \eN\) est un ensemble bien particulier possédant plein de structure, nous allons pouvoir mettre sur \( A^{(\eN)}\) une structure non seulement de \( A\)-module (ça c'est déjà fait), mais en plus d'anneau ainsi qu'une évaluation.
\begin{definition}      \label{DEFooFYZRooMikwEL}
    L'ensemble des \defe{polynômes}{polynômes} en une indéterminée sur l'anneau \( A\) est \( A^{(\eN)}\) défini en \ref{DEFooLCJEooBvVmkV}.
\end{definition}

Notez que nous n'avons pas encore donné la notation \( A[X]\); nous verrons plus tard comment elle arrive. 

Vu que \( A^{(\eN)}\) est engendré par les \( e_i\), tout polynôme sur \( A\) s'écrit \( P=\sum_{i=1}^na_ie_i\).

\begin{definition}      \label{DEFooNXKUooLrGeuh}
    Nous ajoutons deux structures à \( A^{(\eN)}\).
    \begin{description}
        \item[L'évaluation] Si \( \alpha\in A\) et si \( P\in A^{(\eN)}\), nous définissons \( P(\alpha)\) par
            \begin{equation}        \label{EQooDJISooTEkMOw}
                P(\alpha)=(\sum_{i=0}^{n}a_ie_i)(\alpha)=\sum_{i=0}^na_i\alpha^i,
            \end{equation}
            étant entendu que \( \alpha^0=1\) dans \( A\).

            Cette définition s'étend immédiatement au cas où \( B\) est un anneau qui étend \( A\). Dans ce cas nous pouvons définir \( P(b)\) pour tout \( P\in \eA^{(\eN)}\) et \( b\in B\) avec la même formule \eqref{EQooDJISooTEkMOw}.
        \item[Le produit] C'est ici que la structure particulière de \( \eN\) est utilisée. Nous définissons le produit \( A^{\eN}\times A^{(\eN)}\to A^{(\eN)}\) de la façon suivante. Si \( (P_k)_{k\in \eN}\) est la suite (presque partout nulle) d'éléments de \( A\) qui définit \( P\) et si \( (Q_k)_{k\in \eN}\) est celle de \( Q\), nous notons
        \begin{equation}    \label{EQooTNCSooKklisb}
            (PQ)_n=\sum_{k=0}^nP_kQ_{n-k},
        \end{equation}
        et donc \( PQ=\sum_i(PQ)_ie_i\). Plus explicitement,
        \begin{equation}    \label{EQooCIBUooAgpxjL}
            (\sum_{i=0}^na_ie_i)(\sum_{j=0}^mb_je_j)=\sum_{k=0}^{\infty}\Big( \sum_{\substack{  (i,j)\in \eN^2 \\i+j=k}}a_ib_j \Big)e_k.
        \end{equation}
        Notons qu'à droite, la somme sur \( k\) est une somme finie.
    \end{description}
\end{definition}

\begin{proposition}     \label{PROPooGDQCooHziCPH}
    Soit un anneau \( A\). À propos de structure sur \( A^{(\eN)}\).
    \begin{enumerate}
        \item
            Avec le produit, l'ensemble \( A^{(\eN)}\) devient un anneau.
        \item
    L'application
    \begin{equation}
        \begin{aligned}
            g\colon A^{(\eN)}&\to A \\
            P&\mapsto P(\alpha)
        \end{aligned}
    \end{equation}
    est un morphisme d'anneaux\footnote{Définition \ref{DEFooSPHPooCwjzuz}.}. En particulier, \( (PQ)(\alpha)=P(\alpha)Q(\alpha)\).
    \end{enumerate}
\end{proposition}

\begin{proof}
    En plusieurs points
    \begin{subproof}
        \item[Anneau]
            L'identité pour le produit dans \( A^{(\eN)}\) est le polynôme donné par \( a_0=1\) et \( a_i=0\) pour \( i\neq 0\). Cela se vérifie en utilisant directement la définition \eqref{EQooCIBUooAgpxjL}. La distributivité aussi\quext{Je n'ai pas fait les calculs, écrivez-moi pour me dire si ça va facilement.}.
        \item[Le morphisme]
    Nous notons \( P_k\) les éléments de la suite définissant \( P\) et \( Q_k\) ceux de \( Q\). Alors nous avons
    \begin{equation}
        (P+Q)(\alpha)=\sum_k(P_k+Q_k)\alpha^k=\sum_kP_k\alpha^k+\sum_kQ_k\alpha^k=P(\alpha)+Q(\alpha).
    \end{equation}
    Vous aurez noté que la première égalité était la définition \eqref{EQooODBMooQKLUgd}. De même,
    \begin{subequations}
        \begin{align}
            P(\alpha)Q(\alpha)&=\big( \sum_nP_n\alpha^n \big)\big( \sum_kQ_k\alpha^k \big)=\sum_kQ_k\big( \sum_nP_n\alpha^n \big)\alpha^k=\sum_k\sum_nQ_kP_n\alpha^{n+k}\\
            &=\sum_m\big( \sum_{l=0}^mP_lQ_{m-l} \big)\alpha^m=\sum_m(PQ)_m\alpha^m=(PQ)(\alpha).
        \end{align}
    \end{subequations}
    \end{subproof}
\end{proof}

\begin{definition}  \label{DefDegrePoly}
    Soit \( P \in \polyP\), \( P \neq 0 \). On appelle \defe{degré}{degré!d'un polynôme} de $P$ le plus grand nombre naturel $n$ pour lequel le coefficient correspondant est non-nul. Ce naturel est noté \( \deg(P) \).
\end{definition}

%--------------------------------------------------------------------------------------------------------------------------- 
\subsection{La notation ${A[X]}$}
%---------------------------------------------------------------------------------------------------------------------------
\label{SUBSECooLEKVooFBPSJz}

Si \( A\) est un anneau, nous avons déjà défini les polynômes en une indéterminée sur \( A\) comme étant le module \( A^{(\eN)}\) qui est devenu un anneau par la proposition \ref{PROPooGDQCooHziCPH}.

Le polynôme donné par la suite \( (a_n)_{n\in \eN}\) est souvent notée
\begin{equation}
    \sum_ka_kX^k.
\end{equation}
Par exemple avec \( a=(4,2,8)\) nous avons \( a=8X^2+2X+4\). Nous utiliserons souvent cette notation, qui est très pratique parce qu'elle s'adapte bien aux règles de multiplication et d'addition, en particulier la distributivité.

Il y a (au moins) deux façons de comprendre ce que signifie réellement «\( X\)» dans cette notation.

%///////////////////////////////////////////////////////////////////////////////////////////////////////////////////////////
\subsubsection{Première façon (qui botte en touche)}
%///////////////////////////////////////////////////////////////////////////////////////////////////////////////////////////

La première est de dire qu'il n'a pas de significations, et que \( X^2\) est un simple abus de notations pour écrire \( (0,0,1,0,\cdots)\). Avec cette façon de voir, nous notons l'anneau des polynômes sur \( A\) par «\( A[X]\)» où le \( X\) n'a pas d'autres raisons d'être que d'avertir le lecteur que nous réservons la lettre «\( X\)» pour utiliser la notation pratique des polynômes.

%///////////////////////////////////////////////////////////////////////////////////////////////////////////////////////////
\subsubsection{Seconde façon (la bonne)}
%///////////////////////////////////////////////////////////////////////////////////////////////////////////////////////////
\label{SUBSUBSECooPNBYooWXEHrg}

La seconde façon de voir le «\( X\)» est de nous rappeler que \( A^{(\eN)}\) a une base en tant de que module : les \( e_k\) dont nous avons parlé plus haut. Nous posons \( X=e_1\), et nous prenons la convention \( X^0=1\). Alors nous avons \( e_k=X^k\) et nous notons \( A[X]\)\nomenclature[A]{\( A[X]\)}{tous les polynômes de degré fini à coefficients dans \( A\)} l'anneau \(A^{(\eN)}\) exprimé avec \( X\).

Dans les deux cas, il n'est pas vraiment légitime d'écrire des égalités comme « \( P(X)=X^2+2X-3\) », et encore moins de dire «Le polynôme \( P\), \emph{évalué} en \( X\) vaut \( X^2+2X-3\)»  : il est plus correct d'écrire « \( P=X^2+2X-3\) ».

Le lemme suivant montre que ces notations tombent vraiment à point. La véritable difficulté de l'énoncé est de comprendre qu'il n'est pas trivial.

Nous avons vu dans la définition \ref{DEFooNXKUooLrGeuh} que si \( B\) est un anneau qui étant \( A\), et si \(P\in A[X] \), alors nous avons une définition de \( P(b)\) pour tout \( b\in B\). Nous appliquons cela à \( B=A[X]\), qui est un anneau qui étend \( A\). Autrement dit, si \( P\) et \( Q\) sont des polynômes, ça a un sens d'écrire \( P(Q)\) et le résultat sera un élément de \( A[X]\). 

Dans le cas particulier \( Q=X\), nous avons une chouette formule.
\begin{lemma}       \label{LEMooGKWQooVOyeDX}
    Nous avons
    \begin{equation}
        P(X)=P
    \end{equation}
    pour tout \( P\in A[X]\).
\end{lemma}

\begin{proof}
    Si \( P=(a_k)_{k\in \eN}\) alors par définition \( P(\alpha)=\sum_ka_k\alpha^k\) dès que \( \alpha\) est dans un anneau \( B\) qui étend \( A\). Nous considérons le cas particulier \( B=\eA[X]\) et \( \alpha=X\), c'est-à-dire \( Q=(0,1,0,\ldots)\), l'élément \( P(X)\) de \( A[X]\) vaut
    \begin{equation}        \label{EQooABULooFCEasf}
        \sum_ka_kX^k,
    \end{equation}
    qui est exactement \( P\) lui-même.
\end{proof}

Mais il faut bien comprendre que si \( P\) est le polynôme \( (-3,2,1,0,\ldots)\), noté \( X^2+2X-3\), écrire \( P(X)=X^2+2X-3\) est une pirouette de notations que rien ne justifie par rapport à simplement écrire \( P=X^2+2X-3\).

%---------------------------------------------------------------------------------------------------------------------------
\subsection{Polynômes de plusieurs variables}
%---------------------------------------------------------------------------------------------------------------------------

\begin{definition}      \label{DEFooZNHOooCruuwI}
    L'ensemble des \defe{polynôme de \( n\) variables}{polynôme de plusieurs variables} sur l'anneau \( A\) est \( A^{(\eN^n)}\), c'est-à-dire l'ensemble des suites indexées par \( \eN^n\) et dont seulement une quantité finie de coefficients sont non nuls.

    Le produit sur \( A[X_1,\ldots, X_n]\) est défini par
    \begin{equation}
        (PQ)(k_1,\ldots, k_n)=\sum_{\substack{ (l_1,\ldots, l_n),(m_1,\ldots, m_n)\in \eN^n\times \eN^n   \\l_i+m_i=k_i}}P_{l_1,\ldots, l_n}Q_{m_1,\ldots, m_n}.
        \end{equation}
\end{definition}

\begin{normaltext}
    Dans \( A[X_1,\ldots, X_n]\), la multiplication n'est pas la multiplication de fonctions \( \eN^n\to \eK\), parce que le but est d'obtenir la multiplication usuelle au niveau des évaluations.
\end{normaltext}

\begin{definition}
    Si \( P\) est un polynôme de \( n\) variables sur \( A\), et si \( (x_1,\ldots, x_n)\in A^n\), \defe{l'évaluation}{évaluation!polynôme plusieurs variables} de \( P\) sur \( (x_1,\ldots, x_n)\) est
    \begin{equation}
        P(x_1,\ldots, x_n)=\sum_{(k_1,\ldots, k_n)\in \eN^n}P_{k_1,\ldots, k_n}x_1^{k_1}\ldots x_n^{k_n}.
    \end{equation}
    Notez que la somme, bien que sur \( \eN^n\), est une somme finie.
\end{definition}

\begin{normaltext}
    Comme dans le cas des polynômes d'une seule variable, les \( X_i\) dans la notation \( A[X_1,\ldots, X_n]\) sont à prendre à la légère. L'anneau des polynômes de \( n\) variables sur \( A\) aurait mieux fait d'être noté par exemple par \( \mP_n(A)\).

    Le fait est que nous avons les polynômes élémentaires définis par
    \begin{equation}
        X_1(k_1,\ldots, k_n)=\begin{cases}
            1    &   \text{si } (k_1,\ldots, k_n)=(1,0\ldots, 0)\\
            0    &    \text{sinon. }
        \end{cases}
    \end{equation}
    et que l'anneau des polynômes peut être vu comme \( A\) (les polynômes constants) étendus par les \( X_i\).

    Quoi qu'il en soit, les \( X_1\) dans la notation \( A[X_1,\ldots, X_n]\) sont des indices muets. L'anneau \( A[X_1,\ldots, X_n]\) est exactement le même que \( A[T_1,\ldots, T_n]\).
\end{normaltext}

%--------------------------------------------------------------------------------------------------------------------------- 
\subsection{Action du groupe symétrique}
%---------------------------------------------------------------------------------------------------------------------------

Par souci de notations, nous notons \( \Poly_n(A)\) l'anneau des polynômes de \( n\) variables sur \( A\). La propriété universelle de \( \Poly_n(A)=A^{(\eN^n)}\) du théorème \ref{THOooPDZCooJnHbOd} nous donne une application
\begin{equation}
    g\colon \Fun\big(\eN^n,\Poly_n(A)\big)\to \Hom_A\big( \Poly_n(A),\Poly_n(A) \big)
\end{equation}
Avec cela nous pouvons énoncer et démontrer le lemme qui donne l'action de \( S_n\)\footnote{Définition du groupe symétrique \( S_n\) en \ref{DEFooJNPIooMuzIXd}.} sur \( \Poly_n(A)\).

\begin{lemma}[\cite{BIBooFDZDooJQLjlB}]       \label{LEMooIRVQooHvoNBq}
    Pour \( \sigma\in S_n\) nous définissons 
    \begin{equation}
        \begin{aligned}
            \phi_{\sigma}\colon \eN^n&\to \Poly_n(A) \\
            m&\mapsto e_{\sigma(m)}. 
        \end{aligned}
    \end{equation}
    Alors l'application
    \begin{equation}
        \begin{aligned}
            \rho\colon S_n&\to \Hom_A\big( \Poly_n(A),\Poly_n(A) \big) \\
            \sigma&\mapsto g(\phi{\sigma}) 
        \end{aligned}
    \end{equation}
    est une action\footnote{Définition \ref{DefActionGroupe}.}.
\end{lemma}

\begin{proof}
    Nous commençons par donner une expression à notre \( \rho\). Un élément de \( \Poly_n(A)\) est de la forme \( \sum_{m\in \eN^n}a_me_m\), et nous avons\footnote{La somme est définie par \ref{DEFooLNEXooYMQjRo}, et ça va être important. Ah oui, en réalité partout, les sommes sont finies parce que les \( a_m\) (\( m\in \eN^n\)) sont presque tous nuls. Il faudrait écrire sur la somme sur \(\{ m\in \eN^2\tq a_m\neq 0 \}\), mais vous vous imaginez la complication dans la notation.}
    \begin{equation}
        \rho(\sigma)\big( \sum_{m\in \eN^n}a_me_m \big)=\sum_ma_m\phi_{\sigma}(m)=\sum_ma_me_{\sigma(m)}.
    \end{equation}
    
    Nous avons tout de suite \( \rho(\id)=\id\).

    En ce qui concerne la composition, nous avons d'une part
    \begin{equation}
        \rho(\sigma_1\sigma_2)\big( \sum_ma_me_m \big)=g(\phi_{\sigma_1\sigma_2})\big( \sum_ma_me_m \big)=\sum_ma_me_{\sigma_1\sigma_2(m)},
    \end{equation}
    et d'autre part,
    \begin{subequations}
        \begin{align}
            \rho(\sigma_1)\rho(\sigma_2)\big( \sum_ma_me_m \big)&=\rho(\sigma_1)\big( \sum_ma_me_{\sigma_2(m)} \big)\\
            &=\rho(\sigma_1)\big( \sum_ma_{\sigma_2^{-1}(m)}e_m \big)   \label{SUBEQooTSCYooCUWiRz}\\
            &=\sum_ma_{\sigma_2^{-1}(m)}e_{\sigma_1(m)}\\
            &=\sum_ma_me_{\sigma_1\sigma_2(m)}      \label{SUBEQooQPGPooVvqJdT}
        \end{align}
    \end{subequations}
    La proposition \ref{PROPooJBQVooNqWErk} est utilisée pour \eqref{SUBEQooTSCYooCUWiRz} et pour \eqref{SUBEQooQPGPooVvqJdT}.
\end{proof}

%+++++++++++++++++++++++++++++++++++++++++++++++++++++++++++++++++++++++++++++++++++++++++++++++++++++++++++++++++++++++++++
\section{Anneau intègre}
%+++++++++++++++++++++++++++++++++++++++++++++++++++++++++++++++++++++++++++++++++++++++++++++++++++++++++++++++++++++++++++
\label{SECAnneauxIntegres}

La définition d'un anneau intègre est la définition~\ref{DEFooTAOPooWDPYmd}.

\begin{example}     \label{EXooMXNTooZaRPPi}
    Un corps\footnote{Définition~\ref{DefTMNooKXHUd}.} est toujours un anneau intègre. En effet, soient un corps \( \eK\) et deux éléments \( x,y\in \eK\) tels que \( xy=0\). Si \( y\) est inversible, alors nous pouvons multiplier par \( y^{-1}\) pour trouver \( x=0\). Cela prouve que \( \eK\) est un anneau intègre.
\end{example}

\begin{example}     \label{EXooLDXRooSxUAXs}
    L'ensemble \( \eZ\) avec les opérations usuelles est un anneau intègre.
\end{example}

\begin{example}
    L'anneau \( \eZ/6\eZ\) n'est pas intègre parce que \( 3\cdot 2=0\) alors que ni \( 3\) ni \( 2\) ne sont nuls.
\end{example}

Nous verrons au théorème~\ref{ThoBUEDrJ} que l'anneau \( A\) est intègre si et seulement si \( A[X]\) est intègre.

\begin{corollary}   \label{CorZnInternprem}
    L'anneau \( \eZ/n\eZ\) est intègre si et seulement si \( n\) est premier.
\end{corollary}

\begin{proof}
    Supposons que \( n\) soit premier. La proposition \ref{PropZpintssiprempUzn} donne les inversibles de \( \eZ/n\eZ\) par
    \begin{equation}
        U(\eZ/n\eZ)=\{ \tilde x\tq 0\leq x\leq n\tq\pgcd(x,n)=1 \}.
    \end{equation}
    Mais comme \( n\) est premier, \( \pgcd(x,n)=1\) pour tout \( x\), et donc tous les éléments de \( \eZ/n\eZ\) sont inversibles. Donc \( \eZ/n\eZ\) est intègre.

    Si \( n\) n'est pas premier, alors \( n=pq\) avec \( 1<p\leq q<n\). Alors
    \begin{equation}
        [p]_n[q]_n=[pq]_n=[0]_n.
    \end{equation}
    Donc lorsque \( n\) n'est pas premier,  l'anneau \( \eZ/n\eZ\) possède des diviseurs de zéro et n'est alors pas intègre.
\end{proof}

%---------------------------------------------------------------------------------------------------------------------------
\subsection{Caractéristique d'un anneau intègre}
%---------------------------------------------------------------------------------------------------------------------------

\begin{lemma}       \label{LemCaractIntergernbrcartpre}
    La caractéristique\footnote{Définition~\ref{LEMDEFooVEWZooUrPaDw}.} d'un anneau intègre est zéro ou un nombre premier.
\end{lemma}

\begin{proof}
    Si \( A\) est intègre, alors \( \eZ 1_A\) est a fortiori intègre. Notons \( p \) la caractéristique de \( A \). Si \( p = 0 \), la preuve est finie; supposons donc que \( p \neq 0 \). Alors, l'anneau \( \eZ/p\eZ\) est isomorphe à \( \eZ 1_A\), et est donc intègre. Or, la proposition~\ref{CorZnInternprem} dit que \( \eZ/p\eZ\) est intègre si et seulement si \( p\) est premier, ce qui conclut la preuve.
\end{proof}

\begin{example}
    Il existe des corps dont la caractéristique n'est pas égale au cardinal (contrairement à ce que laisserait penser l'exemple des \( \eZ/p\eZ\)). En effet les matrices \( n\times n\) inversibles sur \( \eF_{3}\) forment un corps qui n'est pas de cardinal trois alors que la caractéristique est \( 3\) :
    \begin{equation}
        \begin{pmatrix}
            1    &       \\
                &   1
            \end{pmatrix}+\begin{pmatrix}
                1    &       \\
                    &   1
                \end{pmatrix}+\begin{pmatrix}
                    1    &       \\
                        &   1
                \end{pmatrix}=0.
    \end{equation}
\end{example}

\begin{example}
    Si \( \eK\) est un corps de caractéristique \( 2\), alors l'égalité \( x=-x\) n'implique pas \( x=0\), vu que \( 2x=0\) est vérifiée pour tout \( x\). Cela se répercute sur un certain nombre de résultats. Par exemple, en caractéristique deux, une forme antisymétrique n'est pas toujours alternée: voir le lemme~\ref{LemHiHNey}.
\end{example}

%---------------------------------------------------------------------------------------------------------------------------
\subsection{Divisibilité et classes d'association}
%---------------------------------------------------------------------------------------------------------------------------
\label{DivisibiliteAnneauxIntegres}

\begin{lemma}\label{LemRmVTRq}
    Si \( A\) est un anneau intègre et si \( a,b\in A\) sont tels que \( a\divides b\) et \( b\divides a\), alors il existe un inversible \( u\in A\) tel que \( a=ub\).
\end{lemma}

\begin{proof}
    Les hypothèses à propos de la divisibilité nous indiquent que \( a=xb\) et \( b=ya\) pour certains \( x,y\in A\). Du coup,
    \begin{equation}
        b(1-yx)=0.
    \end{equation}
    Étant donné que \( \eA\) est intègre, cela montre que \( b=0\) ou \( 1-yx=0\). Si \( b=0\) nous avons immédiatement \( a=0\) et le lemme est prouvé. Si au contraire \( yx=1\), c'est que \( y\) et \( x\) sont inversibles et inverses l'un de l'autre.
\end{proof}

\begin{definition}\label{DefrXUixs}
    On dit de deux éléments \( a,b\in A\) qu'ils sont \defe{associés}{associés!éléments d'un anneau} si ils vérifient les hypothèses du lemme~\ref{LemRmVTRq}; en d'autres termes, $a$ et $b$ sont associés s'il existe un inversible \( u\in A\) tel que \( a=ub\).

    La \defe{classe d'association}{classe d'association}\index{classe!d'association} d'un élément \( a \in A \) est l'ensemble des éléments qui lui sont associés; en d'autres termes, c'est \( a  U(A) \).
\end{definition}

\begin{example}
    Dans \( \eZ[i]\), les inversibles sont \( \pm 1\) et \( \pm i\). Donc les éléments associés à \( z\) sont \( z\), \( -z\), \( iz\) et \( -iz\).

    Notons au passage que la notion de divisibilité dans \( \eZ[i]\) n'est pas immédiatement intuitive. En effet bien que \( 7\) ne soit pas divisible par \( 2\) (ni dans \( \eZ\) ni dans \( \eZ[i]\)), le nombre \( 7+6i\) est divisible par \( 2+i\) dans \( \eZ[i]\). En effet :
    \begin{equation}
        (2+i)(4+i)=7+6i.
    \end{equation}
\end{example}

\begin{probleme}
    Est-ce que quelqu'un connaît un anneau contenant \( \eZ\) dans lequel \( 7\) est divisible en \( 2\) ?

    Peut-être \( \eZ\) étendu par tous les \( 1/2^n\) ?
\end{probleme}

%---------------------------------------------------------------------------------------------------------------------------
\subsection{PGCD et PPCM}
%---------------------------------------------------------------------------------------------------------------------------

<<<<<<< Updated upstream
Pour le théorème de Bézout et autres opérations avec des modulo, voir le thème~\ref{THEMEooNRZHooYuuHyt}. Le pgcd et le ppcm sont définis en \ref{DefrYwbct}.

\begin{lemma}
    Soient \( A\) un anneau intègre et \( S\subset A\). Si \( \delta\) est un PGCD de \( S\), alors l'ensemble des PGCD de \( S\) est la classe d'association de \( \delta\).

    De la même façon si \( \mu\) est un PPCM de \( S\), alors l'ensemble des PPCM de \( S\) est la classe d'association de \( \mu\).
\end{lemma}

\begin{proof}
    Soient \( \delta\) un PGCD de \( S\) et \( u\) un inversible dans \( A\). Si \( x\in S\) nous avons \( \delta\divides x\) et donc \( x=a\delta\). Par conséquent \( x=au^{-1}u\delta\) et donc \( u\delta\) divise \( x\). De la même manière, si \( d\) divise \( x\) pour tout \( x\in S\), alors \( d\) divise \( \delta\) et donc \( \delta=ad\) et \( u\delta=aud\), ce qui signifie que \( d\) divise \( u\delta\).

    Dans l'autre sens nous devons prouver que si \( \delta'\) est un autre PGCD de \( S\), alors il existe un inversible \( u\in \eA\) tel que \( \delta'=u\delta\). Vu que \( \delta'\) divise \( x\) pour tout \( x\in S\), nous avons \( \delta'\divides \delta\), et symétriquement nous trouvons \( \delta\divides\delta'\). Par conséquent (lemme~\ref{LemRmVTRq}), il existe un inversible \( u\) tel que \( \delta=u\delta'\).

    Le même type de raisonnement tient pour le PPCM.
\end{proof}

Si \( \delta\) est un PGCD de \( S\), nous dirons \emph{par abus de langage} que \( \delta\) est \emph{le} PGCD de \( S\), gardant en tête qu'en réalité toute sa classe d'association est PGCD. Nous noterons aussi, toujours par abus que \( \delta=\pgcd(S)\).

\begin{remark}
    La classe d'association d'un élément n'est pas toujours très grande. Les inversibles dans \( \eZ\) étant seulement \( \pm 1\), nous pouvons obtenir l'unicité du PGCD et du PPCM en imposant qu'ils soient positifs.

    Pour les polynômes, nous obtenons l'unicité en demandant que le PGCD soit unitaire.

    Dans les cas pratiques, il y a donc en réalité peu d'ambiguïté à parler du PGCD ou du PPCM d'un ensemble.
\end{remark}

%---------------------------------------------------------------------------------------------------------------------------
\subsection{Anneaux intègres et corps}
%---------------------------------------------------------------------------------------------------------------------------

Le fait d'être intègre pour un anneau n'assure pas le fait d'être un corps. Nous avons cependant ce résultat pour les anneaux finis.

\begin{proposition}     \label{PropanfinintimpCorp}
    Un anneau fini intègre est un corps.
\end{proposition}

\begin{proof}
    Soit \( A\) un tel anneau. Soit \( a\neq 0\). Les applications
    \begin{subequations}
        \begin{align}
            l_a\colon x\to ax\\
            r_a\colon x\to xa
        \end{align}
    \end{subequations}
    sont injectives. En tant que applications injectives entre ensembles finis, elles sont surjectives. Il existe donc \( b\) et \( c\) tels que \( 1=ba=ac\). Il se fait que \( b\) et \( c\) sont égaux parce que
    \footnote{Il faut être un peu souple avec les notations communément employées dans les ouvrages mathématiques, et que nous reprenons telles quelles. Dans l'équation qui suit, \( b(ac)\) est le produit de \( b\) par l'élément \( ac\), et non quelque chose comme le produit de \( b\) avec l'idéal \( (ac)\) par exemple.}
    \begin{equation}
        b=b(ac)=(ba)c=c.
    \end{equation}
    Par conséquent \( b\) est un inverse de \( a\).
\end{proof}



\begin{proposition}     \label{PropzhFgNJ}
    Soit \( n\in\eN^*\). Les conditions suivantes sont équivalentes :
    \begin{enumerate}
        \item
            \( n\) est premier.
        \item
            \( \eZ/n\eZ\) est un anneau intègre.
        \item
            \( \eZ/n\eZ\) est un corps.
    \end{enumerate}
\end{proposition}

\begin{proof}
    L'équivalence entre les deux premiers points est le contenu du corolaire~\ref{CorZnInternprem}. Le fait que \( \eZ/n\eZ\) soit un corps lorsque \( \eZ/n\eZ\) est intègre est la proposition~\ref{PropanfinintimpCorp}. Le fait que \( \eZ/n\eZ\) soit intègre lorsque \( \eZ/n\eZ\) est un corps est une propriété générale des corps : ce sont en particulier des anneaux intègres (lemme~\ref{LemAnnCorpsnonInterdivzer}).
\end{proof}

%--------------------------------------------------------------------------------------------------------------------------- 
\subsection{Élément irréductible}
%---------------------------------------------------------------------------------------------------------------------------

\begin{definition}[Élément irréductible\cite{ooWUNIooXKxRya}]  \label{DeirredBDhQfA}
    Un élément d'un anneau commutatif est \defe{irréductible}{irréductible!dans un anneau} si il n'est ni inversible, ni le produit de deux éléments non inversibles.
\end{definition}

\begin{normaltext}
    Nous allons voir dans la section \ref{SECooSWGKooEeOZTO} que le concept d'élément irréductible n'est vraiment utile que dans le cas des anneaux intègres.
\end{normaltext}

\begin{example}
    Un corps n'a pas d'éléments irréductibles parce qu'à part zéro tous les éléments sont inversibles. Mais \( 0\) n'est pas irréductible parce qu'il peut être écrit comme produit d'éléments non inversibles : \( 0=0\cdot 0\).
\end{example}

\begin{example}
    Les éléments irréductibles de l'anneau \( \eZ\) sont les nombres premiers. En effet les seuls inversibles de \( \eZ\) sont \( \pm 1\). Si \( p\) est premier et \( p=ab\) avec \( a,b\in \eZ\), alors nous avons soit \( a=\pm 1\) soit \( b=\pm 1\).
\end{example}

%+++++++++++++++++++++++++++++++++++++++++++++++++++++++++++++++++++++++++++++++++++++++++++++++++++++++++++++++++++++++++++
\section{Anneau factoriel}
%+++++++++++++++++++++++++++++++++++++++++++++++++++++++++++++++++++++++++++++++++++++++++++++++++++++++++++++++++++++++++++

\begin{definition}[Anneau factoriel]        \label{DEFooVCATooPJGWKq}
    Un anneau commutatif \( A\) est \defe{factoriel}{factoriel!anneau}\index{anneau!factoriel} s'il vérifie les propriétés suivantes.
    \begin{enumerate}
        \item
            L'anneau \( A\) est intègre (pas de diviseurs de zéro).
        \item
            Si \( a\in A\) est non nul et non inversible, alors il admet une décomposition en facteurs irréductibles: \( a=p_1\ldots p_k\) où les \( p_i\) sont irréductibles.
        \item
            Si \( a=q_1\ldots q_m\) est une autre décomposition de \( a\) en irréductibles, alors \( m=k\) et il existe une permutation\footnote{Définition~\ref{DEFooJNPIooMuzIXd}.} \( \sigma\in S_k\) telle que \( p_i\) et \( q_{\sigma(i)}\) soient associés\footnote{Définition~\ref{DefrXUixs}.}.
    \end{enumerate}
\end{definition}

Un anneau factoriel permet de caractériser le \( \pgcd\) et le \( \ppcm\) de nombres.

\begin{proposition}
Soit une famille \( \{ a_n \}\) d'éléments de \( A\) qui se décomposent en irréductibles comme
\begin{equation}
    a_i=\prod_k p_k^{\alpha_{k,i}}.
\end{equation}
Alors
\begin{equation}
    \pgcd\{ a_n \}=\prod_k p_k^{\min_i\{ \alpha_{k,i} \}}.
\end{equation}

De plus le PGCD est :
\begin{enumerate}
    \item
        Un multiple de tous les diviseurs communs des \( a_i\).
    \item
        Unique pour cette propriété à multiple près par un inversible\quext{Soyez prudent avec cette affirmation : je n'en n'ai pas de démonstrations sous la main et ne suis pas certain que ce soit vrai.}.
\end{enumerate}

\end{proposition}

De la même manière,
\begin{equation}
    \ppcm\{ a_n \}=\prod_kp_k^{\max_i\{ \alpha_{k,i} \}}.
\end{equation}
Un anneau factoriel a une relation de préordre partiel\index{ordre!sur un anneau factoriel} donnée par \( a<b\) si \( a\) divise \( b\). En termes d'idéaux, cela donne l'ordre inverse de celui de l'inclusion\footnote{Voir proposition~\ref{PropDiviseurIdeaux}.} : \( a<b\) si et seulement si \( (b)\subset (a)\).

\begin{example} \label{EXooCWJUooCDJqkr}
    L'anneau \( \eZ[i\sqrt{3}]\) n'est pas factoriel parce que \( 4\) a au moins deux décompositions distinctes en irréductibles :
    \begin{equation}
        4=2\cdot 2,
    \end{equation}
    et
    \begin{equation}
        4=(1+i\sqrt{3})(1-i\sqrt{3}).
    \end{equation}
\end{example}

Nous allons voir dans l'exemple~\ref{ExeDufyZI} que \( \eZ[i\sqrt{2}]\) est factoriel parce qu'il sera euclidien.

%+++++++++++++++++++++++++++++++++++++++++++++++++++++++++++++++++++++++++++++++++++++++++++++++++++++++++++++++++++++++++++
\section{Anneau principal et idéal premier}
%+++++++++++++++++++++++++++++++++++++++++++++++++++++++++++++++++++++++++++++++++++++++++++++++++++++++++++++++++++++++++++

\begin{definition}      \label{DEFooMZRKooBPLAWH}
    Un idéal \( I\) dans \( A\) est \defe{principal à gauche}{idéal!principal!à gauche} s'il existe \( a\in I\) tel que \( I= A a\). Il est \defe{principal à droite}{idéal!principal!à droite} s'il existe \( a\in I\) tel que \( I=a A\). Nous disons qu'il est \defe{principal}{principal!idéal} s'il est principal à gauche et à droite.
\end{definition}

\begin{definition}          \label{DEFooGWOZooXzUlhK}
    Un anneau est \defe{principal}{principal!anneau} si
    \begin{enumerate}
        \item
            il est commutatif et intègre
        \item
            tous ses idéaux sont principaux.
    \end{enumerate}
\end{definition}

Souvent pour prouver qu'un anneau est principal, nous prouvons qu'il est euclidien (définition~\ref{DefAXitWRL}) et nous utilisons la proposition~\ref{Propkllxnv} qui dit qu'un anneau euclidien est principal.

Une manière de prouver qu'un anneau n'est pas principal est de prouver qu'il n'est pas factoriel, théorème~\ref{THOooANCAooBChmwp}.

\begin{definition}      \label{DEFooAQSZooVhvQWv}
    Nous disons qu'un idéal \( I\) dans \( A\) est \defe{premier}{premier!idéal} si \( I\) est strictement inclus dans \( A\) et si pour tout \( a,b\in A\) tels que \( ab\in I\) nous avons \( a\in I\) ou \( b\in I\).
\end{definition}

\begin{lemma}       \label{LEMooYRPBooYxXXsi}
    L'idéal nul (réduit à \( \{ 0 \}\)) est premier si et seulement si \( A\) est intègre.
\end{lemma}

\begin{proof}
    En deux sens.
    \begin{subproof}
    \item[Si \( \{ 0 \}\) est premier]
        Alors \( A\neq \{ 0 \}\) parce que \( I=\{ 0 \}\) est propre (définition d'idéal premier).
        
        De plus, si \( ab=0\), alors \( ab\in I\) qui est un idéal premier. Donc soit \( a\) soit \( b\) est dans \( I\), c'est-à-dire que soit \( a\) soit \( b\) est nul. Donc \( A\) est intègre.

    \item[Si \( A\) est intègre]

        Alors \( A\neq \{ 0 \}\) et l'idéal \( I=\{ 0 \}\) est strictement inclus dans \( A\). Si \( ab\in I\), alors \( ab=0\) et comme \( A\) est intègre, soit \( a\) soit \( b\) est nul, c'est-à-dire appartient à \( I\).
    \end{subproof}
\end{proof}

\begin{proposition}[\cite{ooWEUDooQybvIx}]      \label{PROPooRUQKooIfbnQX}
    Soit un anneau commutatif\footnote{Tous les anneaux du Frido sont commutatifs} et un idéal \( I\) dans \( A\).
    \begin{enumerate}
        \item       \label{ITEMooUGBTooOGrnWl}
            \( I\) est un idéal premier si et seulement si \( A/I\) est un anneau intègre.
        \item   \label{ITEMooGLXSooUjINqR}
            \( I\) est un idéal maximal si et seulement si \( A/I\) est un corps.
        \item       \label{ITEMooTFFQooOUajFw}
            Tout idéal maximal propre est premier.
    \end{enumerate}
\end{proposition}


\begin{proof}
    En plein d'étapes.
    \begin{subproof}
        \item[\( I\) premier implique \( A/I\) intègre]
            Évacuons le cas trivial pour être sûr. Si \( I=\{ 0 \}\) alors \( A\) est intègre par le lemme \ref{LEMooYRPBooYxXXsi}. Donc \( A/I=A/\{ 0 \}=A\) est intègre également.

            Soient \( a,b\in A\) tels que \( [a][b]=[0]\). Donc \( [ab]=[0]\), c'est-à-dire \( ab\in I\). Vu que \( I\) est un idéal premier nous avons \( a\in I\) ou \( b\in I\), c'est-à-dire \( [a]=0\) ou \( [b]=0\); nous en déduisons que \( A/I\) est un anneau intègre.
        \item[\( A/I\) intègre implique \( I\) premier]
            Soit \( ab\in I\). Alors \( [ab]=0\), ce qui signifie que \( [a][b]=0\) donc que \( [a]=0\) ou que \( [b]=0\) parce que \( A/I\) est intègre. Mais la condition \( [a]=0\) signifie \( a\in I\), et \( [b]=0\) signifie \( b\in I\). Nous avons donc prouvé que soit \( a\) soit \( b\) est dans \( I\), c'est-à-dire que \( I\) est premier.
        \item[Si \( I\) est un idéal maximum]

            Nous devons montrer que tout élément non nul de \( A/I\) est inversible. Un élément non nul de \( A/I\) est \( [x]\) avec \( x\in A\setminus I\). 
            
            Nous considérons \( J=Ax+I\), qui est un idéal parce que pour tout \( a\in A\), \( aAx+aI\in Ax+I\). Mais comme \( I\) est maximal, \( J=I\) ou \( J=A\).

            Si \( J=I\), nous aurions que pour tout \( a\in A\) et pour tout \( i\in I\), \( ax+i\in I\). En particulier pour \( a=1\) et \( i=0\) nous aurions \( x\in I\), ce qui est contraire à l'hypothèse faite sur \( x\).

            Donc \( J=A\). En particulier, \( 1\in J\), c'est-à-dire qu'il existe \( a\in A\) et \( i\in I\) tels que \( ax+i=1\). En passant aux classes, \( [ax]=1\), c'est-à-dire \( [a][x]=1\) qui signifie que \( [a]\) est un inverse de \( [x]\) dans \( A/I\).

        \item[Si \( A/I\) est un corps]

            Si \( x\in A\setminus I\), il faut prouver que tout idéal contenant \( I\) et \( x\) est \( A\).

            Un idéal contenant \( I\) et \( x\) doit contenir l'idéal \( J=Ax+I\). Vu que \( x\notin I\), nous avons \( [x]\neq 0\) dans \( A/I\). Donc \( [x] \) est inversible et il existe \( a\in A\) tel que \( [ax]=[A]\). C'est-à-dire que $ax-1\in I$. Nous avons alors
            \begin{equation}
                1=ax+\underbrace{(1-ax)}_{\in I}.
            \end{equation}
            C'est-à-dire que \( 1\in Ax+I\) et donc \( Ax+I=A\).
    \end{subproof}
    Enfin nous prouvons que tout idéal maximal propre est premier. 

    Si \( I\) est maximal, \( A/I\) est un corps par le point \ref{ITEMooGLXSooUjINqR}, et vu que \( I\) est propre, le corps \( A/I\) n'est pas réduit à \( \{ 0 \}\). Donc le lemme \ref{LemAnnCorpsnonInterdivzer} dit que \( A/I\) est un anneau intègre. Le point \ref{ITEMooUGBTooOGrnWl} dit alors que \( I\) est un idéal premier.
\end{proof}

\begin{remark}
    Vu qu'un corps peut être réduit à \( \{0\}\), dans \ref{ITEMooGLXSooUjINqR}, l'idéal peut être \( A\). Mais pas dans \ref{ITEMooTFFQooOUajFw}, parce qu'un idéal premier est propre, ça fait partie de la définition \ref{DEFooAQSZooVhvQWv}.
\end{remark}

\begin{proposition}[\cite{ooOYKZooOJBDHS}]     \label{PROPooHABIooBZZQMj}
    Si \( A\) est un anneau commutatif intègre, alors un idéal \( I\) dans \( A\) est premier si et seulement si \( A/I\) est intègre.
\end{proposition}

\begin{proof}
    Supposons que \( I\) soit un idéal premier. Si \( \bar a,\bar b\in A/I\)  sont tels que \( \bar a\bar b=0\), alors \( \overline{ ab }=0\), ce qui signifie que \( ab\in I\). Mais alors, vu que \( I\) est premier, soit \( a\) soit \( b\) est dans \( I\). Cela signifie que soit \( \bar a\) soit \( \bar b\) est nul dans \( A/I\). Cela prouve que \( A/I\) est un anneau intègre.

    Dans l'autre sens, nous supposons que \( A/I\) est intègre. Cela implique immédiatement que \( I\neq A\) parce que \( A/A\) n'est pas un anneau intègre (tout le monde est évidemment diviseur de zéro).

    Soient donc \( a,b\in A\) tels que \( ab\in I\). Alors \( \bar a\bar b= \overline{ ab }=0\) dans \( A/I\), mais comme \( A/I\) est intègre, cela implique que soit \( \bar a\) soit \( \bar b\) est nul. Autrement dit, soit \( a\) soit \( b\) est dans \( I\).
\end{proof}

\begin{proposition}[Thème~\ref{THEMEooZYKFooQXhiPD}, \cite{MonCerveau}] \label{PropomqcGe}
    Soit \( A\) un anneau principal qui n'est pas un corps. Pour un idéal propre \( I\) de \( A\), les conditions suivantes sont équivalentes :
    \begin{enumerate}
        \item       \label{ITEMooNOVFooEHtcwE}
            \( I\) est un idéal maximal\footnote{Définition \ref{DEFIdealMax}.};
        \item       \label{ITEMooMQWVooNocVEU}
            \( I\) est un idéal premier non nul\footnote{Définition \ref{DEFooAQSZooVhvQWv}.};
        \item       \label{ITEMooJBXGooEISNuW}
            il existe \( p\) irréductible\footnote{Définition \ref{DeirredBDhQfA}.} dans \( A\) tel que \( I=(p)\).
    \end{enumerate}
\end{proposition}

\begin{proof}
    En plusieurs implications.
    \begin{subproof}
        \item[\ref{ITEMooNOVFooEHtcwE} implique~\ref{ITEMooMQWVooNocVEU}]

            Par hypothèse, \( I\) est un idéal propre, de plus il n'est pas égal à \( \{ 0 \}\), parce que lorsque \( A\) et \( \{ 0 \} \) sont les seuls idéaux, nous avons un corps (proposition~\ref{PROPooUOCVooZGAVVk}). Étant donné que \( I\) est un idéal maximal, le quotient \( A/I\) est un corps par la proposition~\ref{PROPooSHHWooCyZPPw}.

            Soient maintenant, pour entrer dans le vif du sujet, des éléments \( a,b\in A\) tels que \( ab\in I\). Dans le corps \( A/I\) nous avons \( \overline{ ab }=0\), et par définition du produit dans le quotient, \( \bar a\bar b=0\). Par intégrité de l'anneau \( A/I\) (un corps est un anneau intègre, exemple~\ref{EXooMXNTooZaRPPi}) nous avons soit \( \bar a=0\), soit \( \bar b=0\), soit les deux en même temps. Dans tous les cas, soit \( a\) soit \( b\) est dans \( I\).

        \item[\ref{ITEMooMQWVooNocVEU} implique~\ref{ITEMooJBXGooEISNuW}]

            Maintenant \( I\) est un idéal premier non réduit à \( \{ 0 \}\). Vu que \( A\) est un anneau principal, il existe \( x\in A\) tel que \( I=(x)\). Nous devons prouver que \( x\) peut être choisi irréductible; et nous allons faire plus : nous allons prouver que \( x\) ne peut être que irréductible\quext{ça me semble un peu trop facile. Lisez attentivement, et écrivez-moi pour dire si vous êtes d'accord ou pas.}.

            Supposons que \( x\) ne soit pas irréductible. Alors il existe \( a,b\in A\) non inversibles tels que \( x=ab\). Si \( a\in (x)\) alors il existe \( k\in A\) tel que \( a=xk\), et en particulier, \( a=abk\), c'est-à-dire \( 1=bk\) (parce que \( A\) est principal et donc intègre). Cela signifie que \( b\) est inversible alors que nous avions dit qu'il ne l'était pas. Nous en déduisons que \( a\) n'est pas dans \( (x)\). On montre de manière similaire que \( b\) n'est pas dans \( (x)\) non plus.

            Nous nous retrouvons donc avec \( a,b\in A\) tel que \( ab\in I\) sans que ni \( a\) ni \( b\) sont soient dans \( I\). Cela contredit le fait que \( I\) soit un idéal premier. En conclusion, \( x\) est irréductible.

        \item[\ref{ITEMooJBXGooEISNuW} implique~\ref{ITEMooNOVFooEHtcwE}]

            Nous avons \( I=(p)\) avec \( p\) irréductible dans \( A\). Supposons que \( J\) est un idéal différent de \( A\) contenant \( I\). Vu que \( A\) est principal, il existe \( y\in A\) tels que \( J=(y)\). En particulier \( p\in J\), donc \( p=ay\) pour un certain \( a\in A\). Mais \( p\) est irréductible, donc soit \( a\) est inversible, soit \( y\) est inversible. Si \( y\) est inversible, alors \( J=A\), ce qui est exclu. Si \( a\) est inversible, alors \( (y)=(p)\), et \( I=J\).
    \end{subproof}
\end{proof}

\begin{normaltext}
    Dans la proposition \ref{PropomqcGe}, l'hypothèse d'idéal propre est importante. En effet dans le cas \( I=A\), nous avons évidemment que \( I\) est un idéal maximum. Mais \( A\) n'est d'abord pas un idéal premier parce qu'un idéal premier doit être strictement inclus dans l'anneau. Et ensuite, \( A\) est en général loin d'être garanti d'être égal à \( (p)\) pour un de ses éléments \( p\).
\end{normaltext}

\begin{proposition}     \label{PropoTMMXCx}
    Soit \( A \) un anneau principal, et soit \( p \in A \) un élément irréductible. Alors
    \begin{enumerate}
        \item
            \( (p)\) est un idéal maximum.
        \item       \label{ITEMooKPJQooWuPZXS}
            \( A/(p)\) est un corps.
    \end{enumerate}
\end{proposition}

\begin{proof}
    Nous notons \( I=(p)\). Soit un idéal \( J\) contenant \( I\). Vu que \( A\) est principal, \( J\) aussi est monogène : \( J=(q)\). Mais comme \( p\) est dans \( I\) qui est dans \( J\), il existe \( a\in A\) tel que \( p=qa\).

    Vu que \( p\) est irréductible, soit \( q\) soit \( a\) est inversible. Si \( q\) est inversible, alors \( J=A\). Si \( a\) est inversible, alors nous avons \( p=qa\), donc \( q=pa^{-1}\), ce qui signifie que \( q\in(p)\) et donc que \( J=I\).

    Cela prouve que \( (p)\) est un idéal maximum.

    Le fait que \( A/(p)\) soit un corps est maintenant la proposition~\ref{PROPooSHHWooCyZPPw}.
\end{proof}

\begin{example}
    L'anneau \( \eZ\) est principal parce qu'il est intègre et que ses seuls idéaux sont les \( n\eZ\) qui sont principaux : \( n\eZ\) est engendré par \( n\).
\end{example}

\begin{example}[Les idéaux de $\eZ/n\eZ$]       \label{EXooCJRPooYkWdyr}

    Les idéaux de \( \eZ/n\eZ\) sont principaux, mais l'anneau \( \eZ/n\eZ\) n'est pas principal lorsque \( n\) n'est pas premier. Nous allons voir ça.

    \begin{subproof}
        \item[Les idéaux de \( \eZ/n\eZ\) sont principaux]

            Soit un idéal \( S\) dans \( \eZ/n\eZ\). Nous considérons la projection canonique \( \phi\colon \eZ\to \eZ/n\eZ\). La proposition~\ref{PropIJJIdsousphi} dit que  \( S=\phi(J)\) où \( J\) est un idéal de \( \eZ\) contenant \( n\eZ\). Mais le corolaire~\ref{CORooLINXooBlUKPG} nous dit qu'alors \( J=m\eZ\) pour un certain \( m\). Pour que \( m\eZ\) contienne \( n\eZ\), il faut que \( m\) divise \( n\).

            Bref, \( S=\phi(m\eZ)\) avec \( m\divides n\). Nous montrons maintenant que \( S\) est engendré par \( [m]_n\). D'abord, l'élément \( [m]_n\) est bien dans \( \phi(m\eZ)\). Ensuite un élément de \( \phi(m\eZ)\) est de la forme
            \begin{equation}
                [km]_n=k[m]_n\in ([m]_n).
            \end{equation}
            Donc \( S\subset ([m]_n)\). Et l'inclusion dans l'autre sens est tout aussi immédiate : un élément de \( ([m]_n)\) est de la forme
            \begin{equation}
                k[m]_n=[km]_n=\phi(km)\in \phi(m\eZ).
            \end{equation}

        \item[Si \( n\) n'est pas premier, \( \eZ/n\eZ\) n'est pas principal]

            Le fait est que lorsque \( n\) n'est pas premier, \( \eZ/n\eZ\) n'est pas intègre (corolaire~\ref{CorZnInternprem}).

        \item[Moralité]

            Un anneau comme \( \eZ/6\eZ\) est un anneau dont tous les idéaux sont principaux, mais qui n'est pas principal.

    \end{subproof}
\end{example}

\begin{example}
    Nous verrons dans la proposition~\ref{PROPooVWRPooGQMenV} que l'anneau des fonctions holomorphes sur un compact de \( \eC\) est principal.
\end{example}

\begin{definition}      \label{DEFooXSPFooPumQSy}
Nous disons que deux éléments d'un anneau principal sont \defe{premiers entre eux}{premier!deux éléments d'un anneau principal} si leur PGCD est \( 1\).
\end{definition}

\begin{theorem}\index{théorème!chinois!anneau principal}        \label{ThofPXwiM}
    Si \( A\) est un anneau principal et si \( p\) et \( q\) sont premiers entre eux dans \( A\), alors on a l'isomorphisme d'anneaux
    \begin{equation}
        A/pqA\simeq A/pA\times A/qA.
    \end{equation}
\end{theorem}
% TODO : trouver une preuve. Je parie que recopier la même que celle dans Z fonctionne très bien.

%---------------------------------------------------------------------------------------------------------------------------
\subsection{Bézout}
%---------------------------------------------------------------------------------------------------------------------------

\begin{theorem}[\cite{XPXxPl}]
    Toute partie \( S\) d'un anneau principal admet un PGCD et un PPCM. De plus
    \begin{equation}
        \begin{aligned}[]
            \delta=\pgcd(S)\Leftrightarrow (\delta)=\sum_{s\in S}(s)
            \mu=\ppcm(S)\Leftrightarrow (\mu)=\bigcap_{s\in S}(s)
        \end{aligned}
    \end{equation}
\end{theorem}

\begin{proof}
    Vu que l'anneau \( A\) est principal, tous ses idéaux sont principaux et donc engendrés par un seul élément. En particulier il existe \( \delta,\mu\in A\) tels que
    \begin{subequations}
        \begin{align}
            (\delta)&=\sum_{s\in S}(s)\\
            (\mu)&=\bigcap_{s\in S}(s)
        \end{align}
    \end{subequations}
    \begin{subproof}
    \item[PGCD]
        Montrons ce que \( \delta\) est un PGCD de \( S\). Pour tout \( x\in S\), nous avons \( (x)\subset (\delta)\), et donc \( \delta\divides x\). Par ailleurs si \( d\divides x\) pour tout \( x\in S\), nous avons \( (x)\subset (d)\) et donc
        \begin{equation}
            \sum_{x\in S}(x)\subset (d),
        \end{equation}
        puis \( (\delta)\subset (d)\) et finalement \( d\divides \delta\).
        \item[PPCM]
            Si \( x\in S\) nous avons \( (\mu)\subset (x)\) et donc \( x\divides \mu\). D'autre part si \( x\divides m\) pour tout \( x\in S\), alors \( (m)\subset (x)\) et donc \( (m)\subset(\mu)\), finalement \( \mu\divides m\).
    \end{subproof}
\end{proof}

\begin{corollary}[Théorème de Bézout\cite{XPXxPl}]\index{Bézout!anneau principal}\label{CorimHyXy}
    Soit un anneau principal \( A\). Deux éléments \( a,b\in A\) sont premiers entre eux si et seulement s'il existe un couple \( (u, v)\in A^2 \) tel que
    \begin{equation}
        ua+vb=1.
    \end{equation}
    À la place de \( 1\) on aurait pu écrire n'importe quel inversible.
\end{corollary}
\index{anneau!principal}

\begin{proof}
    Pour cette preuve, nous allons écrire \( \pgcd(a,b)\) l'ensemble de PGCD de \( a\) et \( b\), c'est-à-dire la classe d'association d'un PGCD.

    Si \( a\) et \( b\) sont premiers entre eux, alors
    \begin{equation}
        1\in\pgcd(a,b)=\sum_{x=a,b}(x)=(a)+(b).
    \end{equation}

    À l'inverse, si nous avons \( ua+vb=1\), alors \( 1\in (a)+(b)\), mais vu que \( (a)+(b)\) est un idéal principal, \( (1)=(a)+(b)\) et donc \( 1\in \pgcd(a,b)\).
\end{proof}

Le lemme de Gauss est une application immédiate de Bézout. Il y aura aussi un lemme de Gauss à propos de polynômes (lemme~\ref{LemEfdkZw}), et une généralisation directe au théorème de Gauss, théorème~\ref{ThoLLgIsig}.
\begin{lemma}[\href{http://ljk.imag.fr/membres/Bernard.Ycart/mel/ar/node6.html}{lemme de Gauss}]    \label{LemSdnZNX}
    Soit \( A\) un anneau principal et \( a,b,c\in A\) tels que \( a\) divise \( bc\). Si \( a\) est premier avec \( c\), alors \( a\) divise \( b\).
\end{lemma}
\index{lemme!Gauss!dans un anneau principal}

\begin{proof}
    Vu que \( a\) est premier avec \( c\), nous avons \( \pgcd(a,c)=1\) et Bézout (\ref{ThoBuNjam}) nous donne donc \( s,t\in \eA\) tels que \( sa+tc=1\). En multipliant par \( b\),
    \begin{equation}
        sab+tbc=b.
    \end{equation}
    Mais les deux termes du membre de gauche sont multiples de \( a\) parce que \( a\) divise \( bc\). Par conséquent \( b\) est somme de deux multiples de \( a\) et donc est multiple de \( a\).
\end{proof}
Un cas usuel d'utilisation est le cas de \( A=\eN^*\).

%--------------------------------------------------------------------------------------------------------------------------- 
\subsection{Élément premier}
%---------------------------------------------------------------------------------------------------------------------------

\begin{definition}[\cite{ooWBLYooLYwALS}]       \label{DEFooZCRQooWXRalw}
    Soit un anneau commutatif \( A\). Un élément \( p\in A\) est \defe{premier}{élément premier} si il est
    \begin{enumerate}
        \item
            non nul,
        \item
            non inversible,
        \item       \label{ITEMooPMTTooCVHPIm}
            si \( p\) divise un produit \( ab\), alors il divise soit \( a\) soit \( b\) (ou le deux).
    \end{enumerate}
\end{definition}

\begin{proposition}[\cite{ooTGPAooQTbamu}]     \label{PROPooWMNPooZdvOBt}
    Dans un anneau intègre\footnote{Si pas intègre, voir l'exemple \ref{EXooEIUEooCZCPMC}.} tout élément premier est irréductible\footnote{Toutes les définitions dans le thème \ref{THEMEooVIQIooOcFAQS}.}.
\end{proposition}
    
\begin{proof}
    Soit \( p\), un élément premier dans un anneau intègre \( A\).
    \begin{subproof}
        \item[\( p\) n'est pas inversible]
            Cela fait partie de la définition d'un élément premier.
        \item[\( p\) n'est pas un produit d'inversibles]
            Soient \( a,b\in A\) tels que \( p=ab\). Par le point \ref{ITEMooPMTTooCVHPIm} de la définition \ref{DEFooZCRQooWXRalw}, \( p\) divise soit \( a\) soit \( b\). Supposons que \( p\) divise \( a\). Alors il existe \( x\in A\) tel que \( a=px\). En remettant dans \( p=ab\) nous avons :
            \begin{equation}        \label{EQooPYBGooLFHMJZ}
                p=pxb.
            \end{equation}
            Mais l'anneau est intègre et permet donc des simplifications par tout élément non nul. La relation \ref{EQooPYBGooLFHMJZ} donne donc 
            \begin{equation}
                1=xb,
            \end{equation}
            ce qui signifie que \( b\) est inversible.

            Un travail similaire montre que \( a\) est inversible si \( p\) divise \( b\).
    \end{subproof}
\end{proof}

\begin{example}
    Si nous avons l'égalité \( 7=ab\) dans \( \eZ\), alors soit \( a\) soit \( b\) vaut \( 1\) et est donc inversible.
\end{example}

Sur un anneau non intègre, la notion d'élément premier n'est pas aussi intéressante que sur un anneau intègre. Par exemple la proposition \ref{PROPooWMNPooZdvOBt} devient fausse.

\begin{example}     \label{EXooEIUEooCZCPMC}
    Soit l'anneau \( \eZ^2\). L'élément \( (1,0)\) est premier mais pas irréductible.
    \begin{subproof}
        \item[\( (1,0)\) est premier]
            L'élément \( (1,0)\) est non nul, ok. Pour qu'il soit inversible, il faudrait \( (1,0)(x,y)=(1,1)\). Entre autres, \( 0\times y=1\), ce qui est impossible. Donc il n'est pas inversible.

            Supposons que \( (1,0)\) divise le produit \( (a,b)(c,d)=(ac,b)\). Alors il existe \( (x,y)\) tel que \( (1,0)(x,y)=(ac,bd)\). Cela signifie que \( x=ac\) et \( 0\times y=bd\). En particulier, soit \( b=0\) soit \( d=0\). Si \( b=0\), nous avons \( (a,b)=(a,0)\) et effectivement, \( (1,0)\) le divise.
        \item[\( (1,0)\) n'est pas irréductible]
            Nous avons \( (1,0)=(1,0)(1,0)\). Donc l'élément \( (1,0)\) est le produit de deux éléments non inversibles.
    \end{subproof}
\end{example}

\begin{example}
    Si \( \eK\) est un corps, l'élément \( XY\) de \( \eK[X,Y]\) n'est pas premier parce que \( XY\divides X^2Y^2\) alors que \( XY\) ne divise ni \( X^2\) ni \( Y^2\).
\end{example}


\begin{proposition}[\cite{ooJHFCooSbHtEC,MonCerveau}, thème \ref{THEMEooVIQIooOcFAQS}]     \label{PROPooZICGooNmblhl}
    Soit un anneau principal \( A\) et un élément \( p\neq 0\) dans \( A\). Nous avons équivalence de :
    \begin{enumerate}
        \item   \label{ITEMooBTEAooWlFUTX}
            \( (p)\) est un idéal premier,
        \item   \label{ITEMooKQRMooBNPDMX}
            \( p\) est un élément premier,
        \item   \label{ITEMooZYYJooCWiBhL}
            \( p\) est un élément irréductible,
        \item   \label{ITEMooHPAIooYoQzqD}
            \( (p)\) est un idéal maximum propre\quext{Ce «propre» n'est pas dans l'énoncé sur Wikipédia. Je ne comprends pas pourquoi, et j'ai posé la question sur la page de discussion.\\\url{https://fr.wikipedia.org/wiki/Discussion:Idéal_premier}}.
    \end{enumerate}
\end{proposition}

\begin{proof}
    En plusieurs implications.
    \begin{subproof}
        \item[\ref{ITEMooBTEAooWlFUTX} implique \ref{ITEMooKQRMooBNPDMX}]
            En plusieurs points.
            \begin{itemize}
                \item La condition \( p\neq 0\) est dans les hypothèses de la proposition.
                \item Si \( p\) était inversible, nous aurions \( (p)=A\) et donc pas que \( (p)\) est un idéal premier.
                \item Soient \( a,b\in A\) tels que \( p\divides ab\). En particulier, \( (ab)\in (p)\). Mais comme \( (p)\) est un idéal premier, cela implique soit \( a\in (p)\) soit \( b\in (p)\). Donc soit \( p\) divise \( a\) soit \( p\) divise \( b\).
            \end{itemize}
        \item[\ref{ITEMooKQRMooBNPDMX} implique \ref{ITEMooZYYJooCWiBhL}]
            Un anneau principal est intègre; c'est dans la définition \ref{DEFooGWOZooXzUlhK}. Dans un anneau intègre, tout élément premier est irréductible, c'est la proposition \ref{PROPooWMNPooZdvOBt}.
        \item[\ref{ITEMooZYYJooCWiBhL} implique \ref{ITEMooHPAIooYoQzqD}]
            Soit un idéal \( I\) contenant \( (p)\). Vu que \( A\) est principal, \( I\) est engendré par un seul élément; soit \( I=(a)\). Vu que \( p\in I\), l'élément \( a\) divise \( p\). Mais comme \( p\) est un élément premier, \( a\divides p\) implique \( a=p\) ou \( a=1\). Dans le premier cas, \( I=(a)=(p)\), et dans le second cas, \( I=(a)=(1)=A\). Donc \( (p)\) est bien un idéal maximum.

            De plus l'idéal \( (p)\) est propre. En effet avoir \( (p)=A\) dirait en particulier que \( 1\in (p)\), c'est-à-dire qu'il existe \( x\in A\) tel que \( xp=1\). Or \( p\) étant irréductible, il est non inversible.
        \item[\ref{ITEMooHPAIooYoQzqD} implique \ref{ITEMooBTEAooWlFUTX}]
            C'est la proposition \ref{PROPooRUQKooIfbnQX}\ref{ITEMooTFFQooOUajFw}.
    \end{subproof}
\end{proof}

Un exemple d'élément premier non irréductible est \( [4]_6\) dans l'anneau non principal \( \eZ/6\eZ\). Voir \ref{NORMooAXOKooAQMXoB} et le lemme \ref{LEMooZSELooGOFEIz}.

%---------------------------------------------------------------------------------------------------------------------------
\subsection{Anneau noethérien}
%---------------------------------------------------------------------------------------------------------------------------

\begin{definition}      \label{DEFooPWMHooCnrQuJ}
    Un anneau est dit \defe{noethérien}{anneau!noethérien} si toute suite croissante d'idéaux est stationnaire (à partir d'un certain rang).
\end{definition}

Montrer que tout anneau principal est noethérien est le premier pas pour montrer que tout anneau principal est factoriel.

\begin{lemma}       \label{LEMooHQPVooTfkhRV}
    Tout anneau principal\footnote{Définition \ref{DEFooGWOZooXzUlhK}.} est noethérien.
\end{lemma}

\begin{proof}
    Soit \( (J_n)\) une suite croissante d'idéaux et \( J\) la réunion. L'ensemble \( J\) est encore un idéal parce que les \( J_i\) sont emboités. Étant donné que l'idéal est principal nous pouvons prendre \( a\in J\) tel que \( J=(a)\). Il existe \( N\) tel que \( a\in J_N\). Alors pour tout \( n\geq N\) nous avons
    \begin{equation}
        J\subset J_N\subset J_n\subset J.
    \end{equation}
    La première inclusion est le fait que \( J=(a)\) et \( a\in J_N\). La seconde est la croissance des idéaux et la troisième est le fait que \( J\) est une union. Par conséquent pour tout \( n\geq N\) nous avons \( J_N=J_n=J\). La suite est par conséquent stationnaire.
\end{proof}

\begin{example}
    Il y a moyen d'avoir un anneau noetherien non principal. C'est le cas de \( \eZ/6\eZ\) dont nous parlerons dans \ref{LEMooZSELooGOFEIz}.
\end{example}

\begin{theorem}[\cite{FSwlnf}]      \label{THOooANCAooBChmwp}
    Tout anneau principal est factoriel.
\end{theorem}

\begin{example}[\( \eZ\lbrack i\sqrt{ 5 }\rbrack\) n'est ni factoriel ni principal]     \label{EXooYCTDooGXAjGC}
    Vu que \( (i\sqrt{ 5 })^2=-5\), les éléments de \( \eZ[i\sqrt{ 5 }]\) sont les éléments de \( \eC\) de la forme \( a+bi\sqrt{ 5 }\) avec \( a,b\in \eZ\). Nous définissons la \defe{norme}{norme!sur \( \eZ[i\sqrt{ 5 }]\)} sur \( \eZ[i\sqrt{ 5 }]\) par\footnote{C'est le carré de la norme usuelle, mais c'est l'usage dans le milieu.}
    \begin{equation}
        \begin{aligned}
            N\colon \eZ[i\sqrt{ 5 }]&\to \eN \\
            z&\mapsto z\bar z.
        \end{aligned}
    \end{equation}
    Le fait que ce soit à valeurs dans \( \eN\) est un simple calcul :
    \begin{equation}
        N(x+iy\sqrt{ 5 })=(x+iy\sqrt{ 5 })(x-iy\sqrt{ 5 })=x^2+5y^2.
    \end{equation}
    De plus \( N\) est multiplicative : \( N(z_1z_2)=N(z_1)N(z_2)\).

    Nous pouvons maintenant déterminer les inversibles de \( \eZ[i\sqrt{ 5 }]\). Si \( \alpha\) est inversible, alors il existe \( \beta\) tel que \( \alpha\beta=1\). Au niveau de la norme,
    \begin{equation}
        N(\alpha)N(\beta)=1,
    \end{equation}
    ce qui implique que \( N(\alpha)=1\). Or l'équation \( x^2+5y^2=1\) dans \( \eN\) donne \( y=0\), \( x=\pm 1\).

    Au final, les inversibles de \( \eZ[i\sqrt{ 5 }]\) sont \( \pm 1\).

    L'anneau \( \eZ[i\sqrt{ 5 }]\) n'est alors pas factoriel (définition~\ref{DEFooVCATooPJGWKq}) parce que
    \begin{equation}
        2\times 3=(1+i\sqrt{ 5 })(1-i\sqrt{ 5 }).
    \end{equation}
    Cela donne deux décompositions du nombre \( 6\) en produit d'éléments non associés\footnote{Définition~\ref{DefrXUixs}.} (\( 2\) n'est associé qu'à \( 2\) et \( -2\)) parce que les inversibles sont \( 1\) et \( -1\).

    Le fait que \( \eZ[i\sqrt{ 5 }]\) ne soit pas factoriel implique qu'il ne soit pas principal, théorème~\ref{THOooANCAooBChmwp}.
\end{example}

%+++++++++++++++++++++++++++++++++++++++++++++++++++++++++++++++++++++++++++++++++++++++++++++++++++++++++++++++++++++++++++ 
\section{Anneau \( \eZ/6\eZ\)}
%+++++++++++++++++++++++++++++++++++++++++++++++++++++++++++++++++++++++++++++++++++++++++++++++++++++++++++++++++++++++++++
\label{SECooSWGKooEeOZTO}

Nous allons donner quelques propriétés de cet anneau, et en particulier voir que dans cet anneau non intègre, la notion d'élément irréductible n'est pas très intéressante.

Voici pour commencer un calcul la table de multiplication de \( A=\eZ/6\eZ\). Pour les multiples de (par exemple) \( [4]_6\) nous écrivons
\begin{equation}
    1\times [4]_6=[4_6]
\end{equation}
et ensuite
\begin{equation}
    2\times [4]_6=[8]_6=[2]_6,
\end{equation}
puis
\begin{equation}
    3\times [4]_6=[2+4]_6=[6]_6=[0]_6,
\end{equation}
et caetera. Le résultat est :
\begin{equation}
\begin{array}{c|c|c|c|c|c|c}
    \times & [0]_6 & [1]_6  & [2]_6  & [3]_6 & [4]_6 & [5]_6  \\
\hline\hline
[0]_6 & 0 & 0 & 0 & 0 & 0 & 0 \\ 
\hline
[1]_6  & 0 & 1 & 2 & 3 & 4 & 5 \\ 
\hline
[2]_6 & 0 & 2 & 4 & 0 & 2 & 4 \\ 
\hline
[3]_6 & 0 & 3 & 0 & 3 & 0 & 3 \\ 
\hline
[4]_6 & 0 & 4 & 2 & 0 & 4 & 2 \\ 
\hline
[5]_6 & 0 & 5 & 4 & 3 & 2 & 1 \\ 
\hline
\end{array}
\end{equation}
Pour ne pas alourdir, nous n'avons pas écrit \( [x]_6\) partout au lieu de \( x\).

\begin{normaltext}[Inversibles]
    Les éléments inversibles de \( \eZ/6\eZ\) sont ceux qui ont un \( [1]_6\) dans leur table de multiplication. Ce sont donc
    \begin{equation}
        U(\eZ/6\eZ)=\big\{ [1]_6,[5]_6 \big\}.
    \end{equation}
\end{normaltext}

\begin{normaltext}[Diviseurs de zéro]
    Les diviseurs de zéro sont ceux qui ont un \( [0]_6\) dans leur table de multiplication, c'est-à-dire
    \begin{equation}
        \big\{ [2]_6,[3]_6,[4]_6 \big\}.
    \end{equation}
\end{normaltext}

\begin{normaltext}[Irréductibles]
    Les irréductibles sont ceux qui ne sont ni inversibles ni produit de deux éléments non inversibles. Les non inversibles sont :
    \begin{equation}
        \big\{ [0]_6,[2]_6,[3]_6,[4]_6 \}.
    \end{equation}
    Ils sont candidats à être irréductibles. Les produits de ces éléments (on oublie les crochets) sont :
    \begin{subequations}
        \begin{align}
            2\times 2&=4\\
            2\times 3&=0\\
            2\times 4&=2\\
            3\times 3&=3\\
            3\times 4&=0\\
            4\times 4&=4.
        \end{align}
    \end{subequations}
    Donc \( [0]_6\), \( [2]_6\), \( [3]_6\) et \( [4]_6\) ne sont plus candidats à être irréductible. Bref, il ne reste aucun candidats.

    L'anneau \( \eZ/6\eZ\) n'a aucun élément irréductible.
\end{normaltext}

\begin{normaltext}[Éléments premiers]       \label{NORMooAXOKooAQMXoB}
    Les éléments non nuls et non inversibles sont \( 2\), \( 3\) et \( 4\).
    \begin{subproof}
    \item[Pour \( 2\)]
        L'élément \( [2]_6\) divise \( 2\), \( 4\) et \( 0\).
        \begin{itemize}
            \item Les \( (a,b)\) tels que \( ab=2\) sont : $(1,2)$, \( (2,4)\) et \( (5,4)\). L'élément \( 2\) divise donc toujours \( a\) ou \( b\).
            \item Les \( (a,b)\) tels que \( ab=4\) sont : $(1,4)$, \( (2,5)\) et \( (4,4)\). L'élément \( 2\) divise donc toujours \( a\) ou \( b\).
            \item Les \( (a,b)\) tels que \( ab=0\) sont : \( (0,x)\),  $(3,2)$ et \( (4,3)\). L'élément \( 2\) divise donc toujours \( a\) ou \( b\). En particulier, \( [2]_6\) divise \( [0]_6\); c'est important.
        \end{itemize}
        Donc \( [2]_6\) est un élément premier.
    \item[Pour \( 3\)]
        L'élément \( [3]_6\) divise \( 3\) et \( 0\).
        \begin{itemize}
            \item Les \( (a,b)\) tels que \( ab=3\) sont : $(1,3)$ et \( (3,5)\). L'élément \( 3\) divise donc toujours \( a\) ou \( b\).
            \item Les \( (a,b)\) tels que \( ab=0\) sont : \( (0,x)\),  $(3,2)$ et \( (4,3)\). L'élément \( 3\) divise donc toujours \( a\) ou \( b\).
        \end{itemize}
        Donc \( [3]_6\) est un élément premier.
        L'élément \( [4]_6\) divise \( 4\), \( 2\) et \( 0\).
        \begin{itemize}
            \item Les \( (a,b)\) tels que \( ab=4\) sont : $(1,4)$, \( (2,5)\) et \( (4,4)\). L'élément \( 4\) divise donc toujours \( a\) ou \( b\).
            \item Les \( (a,b)\) tels que \( ab=2\) sont : $(1,2)$, \( (2,4)\) et \( (5,4)\). L'élément \( 4\) divise donc toujours \( a\) ou \( b\).
            \item Les \( (a,b)\) tels que \( ab=0\) sont : \( (0,x)\),  $(3,2)$ et \( (4,3)\). L'élément \( 4\) divise donc toujours \( a\) ou \( b\).
        \end{itemize}
        Donc \( [4]_6\) est un élément premier.
    \end{subproof}
    Au final, les éléments premiers dans \( \eZ/6\eZ\) sont 
    \begin{equation}
        \big\{ [2]_6, [3]_6, [4]_6  \big\}.
    \end{equation}
\end{normaltext}

Vous noterez que \( \eZ/6\eZ\) a des éléments premiers non irréductibles. Cela est un contre-exemple à la proposition \ref{PROPooZICGooNmblhl} dans le cas d'un anneau non-intègre.


\begin{lemma}[\cite{MonCerveau}]    \label{LEMooZSELooGOFEIz}
    L'anneau \( \eZ/6\eZ\) est noetherien, mais ni intègre ni principal\footnote{Toutes les définitions dans le thème \ref{THEMEooVIQIooOcFAQS}.}.
\end{lemma}

\begin{proof}
    Vu que c'est un anneau fini, toute suite croissante de quoi que ce soit devient stationnaire; donc \( \eZ/6\eZ\) est noetherien.

    Vu que \( \eZ/6\eZ\) a des diviseurs de zéro, il n'est pas intègre. Et vu qu'il n'est pas intègre, il n'est pas factoriel non plus.
\end{proof}

% This is part of Mes notes de mathématique
% Copyright (c) 2011-2019
%   Laurent Claessens
% See the file fdl-1.3.txt for copying conditions.

%+++++++++++++++++++++++++++++++++++++++++++++++++++++++++++++++++++++++++++++++++++++++++++++++++++++++++++++++++++++++++++
\section{Anneau euclidien}
%+++++++++++++++++++++++++++++++++++++++++++++++++++++++++++++++++++++++++++++++++++++++++++++++++++++++++++++++++++++++++++

\begin{definition}[\wikipedia{fr}{Anneau_euclidien}{Wikipédia}] \label{DefAXitWRL}
    Soit \( A\) un anneau intègre. Un \defe{stathme euclidien}{stathme euclidien} sur \( A\) est une application \( \alpha\colon A\setminus\{ 0 \}\to \eN\) tel que
    \begin{enumerate}
        \item       \label{ITEMooLVJAooLpjgEz}
            \( \forall a,b\in A\setminus\{ 0 \}\), il existe \( q,r\in A\) tel que
            \begin{equation}
                a=qb+r
            \end{equation}
            et \( \alpha(r)<\alpha(b)\).
        \item
            Pour tout \( a,b\in A\setminus\{ 0 \}\), \( \alpha(b)\leq \alpha(ab)\).
    \end{enumerate}
    Un anneau est \defe{euclidien}{euclidien!anneau} s'il accepte un stathme euclidien.
\end{definition}
Le stathme est la fonction qui donne le «degré» à utiliser dans la division euclidienne. La contrainte est que le degré du reste soit plus petit que le degré du dividende.

\begin{example} \label{ExwqlCwvV}
    Le stathme de \( \eN\) pour la division euclidienne usuelle est \( \alpha(n)=n\). Si \( a,b\in \eN\) nous écrivons
    \begin{equation}
        a=qb+r
    \end{equation}
    où \( q\) est l'entier le plus proche \emph{inférieur} à \( a/b\) (on veut que le reste soit positif) et \( r=a-qb\). Nous avons donc
    \begin{equation}
        r-b=a-b(q+1)<a-b\frac{ a }{ b }=0,
    \end{equation}
    ce qui montre que \( r<b\).
\end{example}

Cet exemple ne fonctionne pas avec \( \eZ\) au lieu de \( \eN\) parce que le stathme doit avoir des valeurs dans \( \eN\). Cela ne veut cependant pas dire qu'il n'existe pas de stathme sur \( \eZ\); cela veut seulement dire que \( \alpha(x)=x\) ne fonctionne pas.

\begin{proposition}[\cite{ooELVSooZIZCRn}]\label{Propkllxnv}
    Un anneau euclidien est principal.
\end{proposition}

\begin{proof}
    Soit \( A\) un anneau principal et \( \alpha\) un stathme sur \( A\). Nous considérons un idéal \( I\) non nul de \( A\). Nous devons montrer que \( I\) est généré par un élément. En l'occurrence nous allons montrer qu'un élément \( a\in I\setminus\{ 0 \}\) qui minimise \( \alpha(a)\) va générer\footnote{Un tel élément existe\dots}. Soit \( x\in I\). Par construction, il existe \( q,r\in A\) tels que \( x=aq+r\) avec \( r=0\) ou \( \alpha(r)<\alpha(a)\). Étant donné que \( x,a\in I\), \( r\in I\). Si \( r\neq 0\), alors \( r\) contredirait la minimalité de \( \alpha(a)\). Donc \( r=0\) et \( x=aq\), ce qui signifie que \( I\) est principal.
\end{proof}

\begin{proposition}     \label{PROPooPJGLooQSrJTU}
    L'anneau \( \eZ\) est principal et euclidien.
\end{proposition}

\begin{proof}
    Nous allons seulement montrer que \( \alpha(x)=| x |\) est un stathme euclidien. Ainsi \( \eZ\) sera euclidien et donc principal par la proposition~\ref{Propkllxnv}.

    D'abord \( \eZ\) est intègre, c'est l'exemple~\ref{EXooLDXRooSxUAXs}.

    La condition \( \alpha(b)\leq \alpha(ab)\) est immédiate : \( | a |\leq | ab |\) pour tout \( a,b\in \eZ\).

    Soient maintenant \( a,b\in \eZ\). Nous définissons \( q_0,r_0\in \eN\) tels que
    \begin{equation}
        | a |=q_0| b |+r_0
    \end{equation}
    avec \( r_0<| b |\). Cela existe parce que \( \alpha(x)=x\) est un stathme sur \( \eN\) par l'exemple~\ref{ExwqlCwvV}.

    \begin{subproof}
        \item[Si \( a>0\) et \( b>0\)]

            Alors \( a=q_0b+r_0\) et le couple \( (q_0,r_0)\) vérifie les conditions de la définition~\ref{DefAXitWRL}\ref{ITEMooLVJAooLpjgEz}.

        \item[Si \( a>0\) et \( b<0\)]

            Alors \( a=-q_0b+r_0\), et le couple \( (-q_0,r_0)\) vérifie les conditions de la définition~\ref{DefAXitWRL}\ref{ITEMooLVJAooLpjgEz}.


        \item[Si \( a<0\) et \( b>0\)]
            Alors \( a=-q_0b-r_0\), et le couple \( (-q_0,-r_0)\) vérifie les conditions de la définition~\ref{DefAXitWRL}\ref{ITEMooLVJAooLpjgEz} parce que
            \begin{equation}
                \alpha(-r_0)=r_0<| b |=\alpha(b).
            \end{equation}
        \item[Si \( a<0\) et \( b<0\)]
            Alors \( a=q_0b-r_0\), et le couple \( (q_0,-r_0)\) vérifie les conditions de la définition~\ref{DefAXitWRL}\ref{ITEMooLVJAooLpjgEz}.

    \end{subproof}
\end{proof}

Nous venons de voir que \( \eZ\) est principal; le lemme suivant nous dit que $\eZ[X]$ n'est pas principal, lui.
\begin{lemma}[\cite{ooRQHSooEBZpKe}]        \label{LEMooDJSUooJWyxCL}
    Si $A$ est un anneau intègre qui n'est pas un corps, alors \( A[X]\) n'est pas principal.
\end{lemma}

\begin{proof}
    Soit un élément non nul \( a\in A\).
    \begin{subproof}
        \newcommand{\foo}{A[X]}
    \item[Un idéal principal contenant \( a\) et \( X\) est {A[X]}]

            Soit \( (P)\) un idéal principal contenant \( a\) et \( X\). Vu que \( a\in(P)\), il existe \( Q\) tel que \( a=QP\). Donc \( P\) divise \( a\) dans \( \eZ[X]\). Les degrés font que \( P\) est un polynôme constant, c'est-à-dire en réalité un élément de \( A\). Soit \( P=k\in A\).

            Vu que \( P\) divise \( X\), nous avons aussi \( X=kQ\) pour un certain \( Q\in \eZ[X]\). Les degrés disent qu'il existe \( k'\in A\) tel que \( Q=k'X\) et donc \( Q=k'X=k'kQ\), ce qui implique que \( kk'=1\). L'idéal engendré par \( k\) contient donc en particulier \( kk'=1\) et donc contient \( A[X]\) en entier :
            \begin{equation}
                1=k'k\in k'(P)=(P).
            \end{equation}

        \item[Si \( (a,X)=\foo\) alors \( a\) est inversible]

            Si \( (a,X)=A[X]\), en particulier, \( 1\in (a,X)\), ce qui signifie qu'il existe des polynômes \( U,V\in A[X]\) tels que
            \begin{equation}
                1=UX+Va.
            \end{equation}
            Nous évaluons cette égalité en \( 0\) : vu que \( (UX)(0)=0\) nous avons \( 1=V(0)a\), ce qui signifie que \( V(0)\) est un inverse de \( A\). Donc \( a\) est inversible.


        \item[Si \( a\) n'est pas inversible alors \( (a,X)\) n'est pas principal]

            Si \( (a,X)\) était principal, alors nous aurions, par ce qui est dit plus haut, \( (a,X)=A[X]\). Mais cette dernière égalité impliquerait que \( a\) est inversible.
    \end{subproof}
    En conclusion, si \( A\) n'est pas un corps, il possède un élément ni nul ni inversible. Dans ce cas, l'idéal \( (a,X)\) n'est pas principal dans \( A[X]\) et nous en déduisons que \( A[X]\) n'est pas un anneau principal.
\end{proof}

Nous verrons dans le lemme~\ref{LEMooIDSKooQfkeKp} que si $\eK$ est un corps, alors \( \eK[X]\) est principal.

\begin{example} \label{ExeDufyZI}
    Prouvons que \( \eZ[i\sqrt{2}]\) est un anneau euclidien. Pour cela nous démontrons que
    \begin{equation}    \label{EqOZUIooZGmHWl}
        \begin{aligned}
            N\colon \eZ[i\sqrt{2}]&\to \eN \\
            a+bi\sqrt{2}&\mapsto a^2+2b^2
        \end{aligned}
    \end{equation}
    est un stathme euclidien.

    Soient \( z=a+bi\sqrt{2}\), \( t=a'+b'i\sqrt{2}\). Nous cherchons \( q\) et \( r\) tels que la division euclidienne s'écrive \( z=qt+r\). Soient \( \alpha,\beta\in \eQ\) tels que
    \begin{equation}
        \frac{ z }{ t }=\alpha+\beta i\sqrt{2}.
    \end{equation}
    Nous désignons par \( \alpha+\epsilon_1\) et \( \beta+\epsilon_2\) les entiers les plus proches de \( \alpha\) et \( b\). Nous avons \( | \epsilon_1 |,| \epsilon_2 |\leq \frac{ 1 }{2}\). Nous posons alors naturellement
    \begin{equation}
        q=(\alpha+\epsilon_1)+(\beta+\epsilon_2)i\sqrt{2}
    \end{equation}
    et nous calculons \( r=z-qt\) :
    \begin{equation}
        2b'\epsilon_2-a'\epsilon_1+i\sqrt{2}\big( \epsilon_1b'-a'\epsilon_2 \big).
    \end{equation}
    Nous trouvons
    \begin{equation}
        N(r)=a'^2\epsilon_1^2+4b'^2\epsilon_2^2+2a'^2\epsilon_1^2+2b'^2\epsilon_2^2\leq \frac{ 3 }{ 4 }a'^2+\frac{ 3 }{2}b'^2.
    \end{equation}
    D'autre part \( N(t)=a'^2+2b'^2\), et nous avons donc bien \( N(r)<N(t)\).

    En ce qui concerne la seconde propriété du stathme, un petit calcul montre que
    \begin{equation}
        N(zt)=(a^2+2b^2)(a'^2+2b'^2),
    \end{equation}
    et tant que \( t\neq 0\) nous avons bien \( N(zt)>N(z)\).
\end{example}

Notons en particulier que \( \eZ[i\sqrt{2}]\) est factoriel et principal.

\begin{example}[Décomposition en facteurs irréductibles dans ${\eZ[i\sqrt{2}]}$] \label{ExluqIkE}
    Les éléments inversibles de \( \eZ[i\sqrt{2}]\) sont \( \pm 1\), donc deux éléments \( a\) et \( b\) sont associés (définition~\ref{DefrXUixs}) si et seulement si \( a=\pm b\).

    De plus si \( p\) est irréductible\footnote{Définition \ref{DeirredBDhQfA}}, alors \( -p\) est irréductible. Les éléments irréductibles de \( \eZ[i\sqrt{2}]\) arrivent donc par pairs d'éléments associés. Soit \( \{ p_i \}\) une sélection de un élément irréductible parmi chaque paire. Tout élément \( x\) de \( \eZ[i\sqrt{2}]\) peut alors être écrit \( x=\pm p_1^{\alpha_1}\ldots p_n^{\alpha_n}\). Ce fait va être pratique pour comparer des décompositions en facteurs irréductibles d'éléments.
\end{example}

Le lemme suivant fait en pratique partie de l'exemple~\ref{ExmuQisZU}, mais nous l'isolons pour plus de clarté\footnote{Merci à \href{http://fr.wikipedia.org/wiki/Utilisateur:Marvoir}{Marvoir} pour m'avoir souligné le manque.}.
\begin{lemma}       \label{LemTScCIv}
    Si \( a\) et \( b\) sont deux éléments premiers entre eux de \( \eZ[i\sqrt{2}]\), et s'il existe \( y \in  \eZ[i\sqrt{2}]\) tel que \( ab=y^3\), alors \( a\) et \( b\) sont des cubes (dans \( \eZ[i\sqrt{2}]\)).
\end{lemma}

\begin{proof}
    D'après l'exemple~\ref{ExluqIkE} nous pouvons écrire
    \begin{subequations}
        \begin{align}
            y&=\pm p_1^{\sigma_1}\ldots p_n^{\sigma_n}\\
            a&=\pm p_1^{\alpha_1}\ldots p_n^{\alpha_n}\\
            b&=\pm p_1^{\beta_1}\ldots p_n^{\beta_n}
        \end{align}
    \end{subequations}
    où les \( p_i\) sont les irréductibles de \( \eZ[i\sqrt{2}]\) «modulo \( \pm 1\)» au sens où la liste des irréductibles est \( \{ p_i \}\cup\{ -p_i \}\) (union disjointe). Étant donné que \( a\) et \( b\) sont premiers entre eux, \( \alpha_i\) et \( \beta_i\) ne peuvent pas être non nuls en même temps alors que leur somme doit faire \( 3\sigma_i\). Nous avons donc pour chaque \( i\) soit \( \alpha_i=3\sigma_i\) soit \( \beta=3\sigma_i\) (et bien entendu si \( \sigma_i=0\) alors \( \alpha_i=\beta_i=0\)).

    Étant donné que \( \pm 1\) sont également deux cubes, \( a\) et \( b\) sont bien des cubes.

    Notons que nous avons utilisé de façon capitale le fait que \( \eZ[i\sqrt{2}]\) était factoriel.
\end{proof}

%---------------------------------------------------------------------------------------------------------------------------
\subsection{Équations diophantiennes}
%---------------------------------------------------------------------------------------------------------------------------
%TODO : il y a une équation diophantienne qui semple pas mal ici : http://fr.wikipedia.org/wiki/Entier_quadratique#x2_.2B_5.y2_.3D_p

\begin{example} \label{ExZPVFooPpdKJc}
    L'équation diophantienne
    \begin{equation}
        x^2=3y^2+8
    \end{equation}
    n'a pas de solutions. En effet si nous prenons l'équation modulo \( 3\) nous obtenons
    \begin{equation}
        [x^2]_3=[3y^2+8]_3=[8]_3=[2]_3.
    \end{equation}
    Or dans \( \eZ/3\eZ\), aucun carré n'est égal à deux : \( 0^2=0\neq 2\), \( 1^2=1\neq 2\) et \( 2^2=4=1\neq 2\).
\end{example}

\begin{example}     \label{ExmuQisZU}
    Résolvons l'équation diophantienne\index{équation!diophantienne}
    \begin{equation}
        x^2+2=y^3.
    \end{equation}
    Une première remarque est que \( x\) doit être impair. En effet si \( x=2k\), nous devons avoir \( y^3\) pair. Mais si un cube pair est divisible par \( 8\), donc \( y^3=8l\). L'équation devient \( 4k^2+2=8l^3\), c'est-à-dire \( 2k^2+1=4l^3\). Le membre de gauche est impair tandis que celui de droite est pair. Impossible.

    Nous pouvons écrire l'équation sous la forme \( x^2+2=(x+i\sqrt{2})(x-i\sqrt{2})\). Et nous considérons \( \eZ[i\sqrt{2}]\) muni de son stathme \( N\) donné par \eqref{EqOZUIooZGmHWl}.

    L'élément \( i\sqrt{2}\) est irréductible parce que \( N(i\sqrt{2})=2\), et si nous avions \( i\sqrt{2}=pq\), alors nous aurions \( N(p)N(q)=2\), ce qui n'est possible que si \( N(p)\) ou \( N(q)\) vaut \( 1\).

    Nous prouvons maintenant que les éléments \( x+i\sqrt{2}\) et \( x-i\sqrt{2}\) sont premiers entre eux. Supposons que \( d\) soit un diviseur commun; alors il divise aussi la somme et la différence. Donc \( d\) divise à la fois \( 2x\) et \( 2i\sqrt{2}\).

    Étant donné que \( i\sqrt{2}\) est irréductible et que \( 2i\sqrt{2}=(-i\sqrt{2})^3\), les diviseurs de \( 2i\sqrt{2}\) sont les puissances de \( (-i\sqrt{2})\). Du coup nous devrions avoir \( d=(i\sqrt{2})^{\beta}\) et donc
    \begin{equation}
        x=(i\sqrt{2})^{\beta}q
    \end{equation}
    pour un certain \( q\in\eZ[i\sqrt{2}]\). Dans ce cas nous avons \( N(x)=2^{\beta}N(q)\), mais nous avons déjà précisé que \( x\) ne pouvait pas être pair, donc \( \beta=0\) et nous avons \( d=1\).

    Vu que les nombres \( x\pm i\sqrt{2}\) sont premiers entre eux et que leur produit doit être un cube, ils doivent être séparément des cubes (lemme~\ref{LemTScCIv}). Nous devons donc résoudre séparément \( x\pm i\sqrt{2}=y^3\).

    Cherchons les \( x\) et \( y\) entiers tels que \( x+i\sqrt{2}=y^3\). Si nous posons \( z=a+bi\sqrt{2}\), il suffit de calculer \( z^3\) :
    \begin{verbatim}
----------------------------------------------------------------------
| Sage Version 4.8, Release Date: 2012-01-20                         |
| Type notebook() for the GUI, and license() for information.        |
----------------------------------------------------------------------
sage: var('a,b')
(a, b)
sage: z=a+I*sqrt(2)*b
sage: (z**3).expand()
3*I*sqrt(2)*a^2*b - 2*I*sqrt(2)*b^3 + a^3 - 6*a*b^2
    \end{verbatim}
    En identifiant cela à \( x+i\sqrt{2}\) nous trouvons le système
    \begin{subequations}
        \begin{numcases}{}
            x=a^3-6ab^2\\
            1=3a^2b-2b^3
        \end{numcases}
    \end{subequations}
    où, nous le rappelons, \( x\), \( a\) et \( b\) sont des entiers. La seconde équation montre que \( b\) doit être inversible : \( b(3a^2-2b^2)=1\). Il y a donc les possibilités \( b=\pm 1\). Pour \( b=1\) l'équation devient \( 3a^2-2=1\), c'est-à-dire \( a=\pm 1\). Pour \( b=-1\) l'équation devient \( 3a^2-2=-1\), impossible. En conclusion les possibilités sont
    \begin{subequations}
        \begin{align}
            (x,z)=(-5,1+i\sqrt{2})\\
            (x,z)=(5,-1+i\sqrt{2})\\
        \end{align}
    \end{subequations}
    Le travail avec \( x-i\sqrt{2}\) donne les mêmes résultats.

    Les deux solutions de l'équation \( x^2+2=y^3\) sont alors \( (5,3)\) et \( (-5,3)\).
\end{example}

%---------------------------------------------------------------------------------------------------------------------------
\subsection{Triplets pythagoriciens et équation de Fermat pour \texorpdfstring{$ n=4$}{n=4}}
%---------------------------------------------------------------------------------------------------------------------------

\begin{definition}
    Les solutions entières (positives) de l'équation \( x^2+y^2=z^2\) sont appelés \defe{triplets pythagoriciens}{triplet!pythagoricien}.
\end{definition}

Ils donnent toutes les possibilités de triangles rectangles dont les côtés ont des longueurs entières.

\begin{definition}
  On dit qu'un triplet pythagoricien est \defe{primitif}{primitif!triplet pythagoricien} si les trois nombres sont premiers dans leur ensemble\footnote{Définition \ref{DefZHRXooNeWIcB}.}.
\end{definition}

Remarquons que cela est équivalent à montrer que les trois nombres sont premiers deux à deux: en effet, si deux parmi \( x\), \( y\) et \( z\) sont divisibles par un nombre, alors tous les trois sont divisibles par ce nombre\footnote{Parce que \( k\) et \( k^2\) ont les mêmes facteurs premiers.}, donc les nombres \( x\), \( y\) et \( z\) sont premiers deux à deux.

\begin{lemma}    \label{LemTripletsPythagoriciensPrimitifs}
  Dans un triplet pythagoricien primitif \( (x, y, z) \), on a toujours $z$ impair et:
  \begin{itemize}
  \item
    soit $x$ impair et $y$ pair;
  \item
    soit $x$ pair et $y$ impair.
  \end{itemize}
\end{lemma}

\begin{proof}
  Remarquons que le fait d'imposer que le triplet soit primitif, interdit aux nombres $x$ et $y$ d'être pairs en même temps. En effet, si c'était le cas, alors \( x^2 \) et \( y^2 \) seraient aussi pairs, donc leur somme \( z^2 \) aussi, d'ou $z$ serait pair et les trois nombres ne seraient pas premiers entre eux.

  Nous montrons à présent que les nombres \( x\) et \( y\) ne sont pas tous les deux impairs. Par l'absurde, si \( x=2a+1\), nous avons \( x^2=4a^2+4a+1\in [1]_4\); de la même manière,  \( y^2 \in [1]_4\). On en déduit alors que \( z^2=x^2+y^2\in [2]_4\). Le nombre \(  z^2\) est donc pair, donc \( z\) est pair : disons \( z=2c\). Alors, \( z^2=4c^2\in [0]_4\). Comme les classes modulo 4 sont disjointes, nous aboutissons à une contradiction.
\end{proof}

\begin{proposition}[Triplets pythagoriciens\cite{fJhCTE,HARRooBvzbXo}]  \label{PropXHMLooRnJKRi}
    Un triplet \( (x,y,z)\in(\eN^*)^3\) est solution de \( x^2+y^2=z^2\) si et seulement s'il existe \( d\in \eN\) et \( u,v\in \eN^*\) premiers entre eux tels que
    \begin{subequations}        \label{subeqLVHFooVgWsFx}
        \begin{numcases}{}
            x=d(u^2-v^2)\\
            y=2duv\\
            z=d(u^2+v^2)
        \end{numcases}
    \end{subequations}
    ou
    \begin{subequations}
        \begin{numcases}{}
            x=2duv\\
            y=d(u^2-v^2)\\
            z=d(u^2+v^2)
        \end{numcases}
    \end{subequations}
    La différence entre les deux est seulement d'inverser les rôles de \( x\) et \( y\).
\end{proposition}

\begin{proof}
    Montrons d'abord que les formules proposées sont bien des solutions; nous vérifions \eqref{subeqLVHFooVgWsFx} :
    \begin{equation}
        x^2+y^2=d^2(u^2-v^2)+4d^2u^2v^2=d^2(u^2+v^2)^2,
    \end{equation}
    qui correspond bien au \( z^2\) proposé.

    Nous allons maintenant prouver la réciproque : toute solution est d'une des deux formes proposées. Déterminer les triplets primitifs suffira parce que si \( (x,y,z)\) n'est pas une solution primitive, alors en posant \( k=\pgcd(x,y,z)\), le triplet \( \big( \frac{ x }{ k },\frac{ y }{ k },\frac{ z }{ k } \big)\) est primitif. Connaissant les triplets primitifs, nous obtenons tous les autres par simple multiplication.

    Soit donc \( (x,y,z)\) un triplet pythagoricien primitif. Sans perte de généralité\footnote{En échangeant les rôles de $x$ et $y$ ici, nous obtenons à la fin la seconde forme des solutions.}, grâce au lemme \ref{LemTripletsPythagoriciensPrimitifs}, nous pouvons supposer  \( x\) est pair tandis que \( y\) et \( z\) sont impairs. Comme \( x^2=(z+y)(z-y)\), nous avons
    \begin{equation}
        \left( \frac{ x }{2} \right)^2=\left( \frac{ z+y }{2} \right)^2\left( \frac{ z-y }{ 2 } \right)^2.
    \end{equation}
    Vu que \( z\) et \( y\) sont premiers entre eux, les nombres \( z-y\) et \( z+y\) sont également premiers entre eux\footnote{Si \( z-y=kn\) et \( z+y=km\), faisant la somme et la différence on voit que \( y\) et \( z\) sont divisibles par \( k\).}. Donc les facteurs premiers (qui arrivent au moins au carré) de \( (x/2)^2\) sont chacun soit dans \( (z+y)/2\) soit dans \( (z-y)/2\). Nous en déduisons que ces derniers sont des carrés d'entiers. Nous posons
    \begin{equation}
        \begin{aligned}[]
            \frac{ z-y }{2}=u^2&&\frac{ z+y }{2}=v^2.
        \end{aligned}
    \end{equation}
    Bien entendu \( u\) et \( v\) sont non nuls parce que nous avons exclu la possibilité de triplets dont un élément serait nul. Avec tout cela nous avons \( (x/2)^2=u^2v^2\) et donc \( x=2uv\) puis par somme et différence :
    \begin{subequations}
        \begin{numcases}{}
            x=2uv\\
            y=v^2-u^2\\
            z=u^2+v^2,
        \end{numcases}
    \end{subequations}
    ce qu'il fallait.
\end{proof}

\begin{remark}
    Les solutions dans lesquelles \( x\), \( y\) ou \( z\) sont nuls sont faciles à classer. La solution \( (1,0,1)\) n'est pas dans les formes proposées. En effet elle ne peut pas être de la première forme : avoir \( y=0\) demanderait qu'un nombre parmi \( d\), \( u\) et \( v\) soit nul, ce qui est interdit. La solution \( (1,0,1) \) ne peut pas non plus être de la seconde forme parce que \( x\) y est pair.
\end{remark}

\begin{proposition}[\cite{fJhCTE}]      \label{propFKKKooFYQcxE}
    Les équations \( x^4+y^4=z^2\) et \( x^2+y^4=z^4\) n'ont pas de solutions dans \( (\eN^*)^3\).
\end{proposition}
\index{équation!diophantienne}

\begin{proof}
  Si la première équation n'a pas de solutions, alors la seconde n'en
  n'a pas non plus parce que \( z^4\) est un carré. Nous nous
  concentrons donc sur l'équation \( x^4+y^4=z^2\) et nous supposons
  qu'il existe au moins une solution dans \( (\eN^*)^3\). Nous en choisissons une \( (x,y,z)\) avec \( z\) minimum (les \( z\) dans différentes solutions étant dans \( \eN\), il en existe forcément un minimum\footnote{Voir quelque chose comme le lemme~\ref{PropQEPoozLqOQ}.}). Du coup, les trois nombres \( x\), \( y\) et \( z\) sont premiers dans leur ensembles parce que une
  division par leur \( \pgcd\) donnerait une nouvelle solution qui
  contredirait la minimalité de \( z\).

    Nous posons \( x^4=\bar x^2\) et \( y^4=\bar y^2\). Ils vérifient
    l'équation \( \bar x^2+\bar y^2=z^2\) et par la proposition
   ~\ref{PropXHMLooRnJKRi}, il existe \( u,v\in \eN^*\) premiers entre
    eux tels que, sans perte de généralité\footnote{En inversant les
      rôles de $x$ et $y$ au besoin.}, on ait
    \begin{subequations}
        \begin{numcases}{}
            \bar x=2uv\\
            \bar y=u^2-v^2\label{eqnFKKKooFYQcxE1}\\
            z=u^2+v^2.\label{eqnFKKKooFYQcxE2}
        \end{numcases}
    \end{subequations}
    Si \( u\) est pair, alors \( v\) est impair (et inversement) parce
    que \( \pgcd(u,v)=1\) Si \( u\) est pair, alors \( u=2a\) et \(
    v=2b+1\), ce qui donne \( \bar y=4a^2-4b^2-4b-1\in[-1]_4\). Or
    nous avons déjà vu qu'un carré est dans \( [0]_4\) ou dans \(
    [1]_4\). Il faut donc que \( u\) soit impair. Le lemme
   ~\ref{LemTripletsPythagoriciensPrimitifs} implique alors que \( v\)
    soit pair.

    De l'équation~\ref{eqnFKKKooFYQcxE1}, nous en déduisons que \(
    v^2+\bar y=u^2\); de plus \( u^2\), \( v^2\) et \( \bar y\) sont
    premiers dans leur ensemble: en effet, $u$ et $v$ sont premiers
    entre eux, et si l'un parmi \( u^2\) et \( v^2\) a un facteur
    commun avec \( \bar y\), alors l'autre doit l'avoir aussi (dans
    une égalité \( a+b=c\), si deux des nombres ont un diviseur
    commun, le troisième l'a aussi). Comme \( \bar y=y^2\), le triplet
    \( (v,y,u)\) est un triplet pythagoricien primitif. Nous
    réappliquons la proposition~\ref{PropXHMLooRnJKRi}, en se
    souvenant que $v$ est pair: il existe donc deux nombres \( r\) et
    \( s\) premiers entre eux tels que
    \begin{subequations} \label{eqnFKKKooFYQcxE3}
        \begin{numcases}{}
            v=2rs\\
            y=r^2-s^2\\
            u=r^2+s^2.
        \end{numcases}
    \end{subequations}
    Avec cela, \( \bar x=2uv=4rs(r^2+s^2)\). Remarquons que les trois
    nombres \( r\), \( s\) et \( r^2+s^2\) sont premiers entre
    eux dans leur ensemble; or, comme \( \bar x\) est un
    carré ces nombres doivent séparément être des carrés :
    \begin{subequations}
        \begin{numcases}{}
            r=\alpha^2\\
            s=\beta^2\\
            r^2+s^2=\gamma^2.
        \end{numcases}
    \end{subequations}
    En mettant les deux premiers dans la troisième, nous avons \( \alpha^4+\beta^4=\gamma^2\). Donc \( (\alpha^2,\beta^2,\gamma)\) est une solution. Nous allons prouver que \( \gamma<z\), ce qui terminera la preuve, puisque $z$ était supposé minimal. Nous avons :
    \begin{align*}
        z&=u^2+v^2&&\text{par~\ref{eqnFKKKooFYQcxE2}}\\
          &=r^2+s^2+4r^2s^2&&\text{par~\ref{eqnFKKKooFYQcxE3}}\\
          &=\gamma^2+4r^2s^2\\
          &> \gamma^2,
    \end{align*}
    et a fortiori \( \gamma<z\).
\end{proof}

%+++++++++++++++++++++++++++++++++++++++++++++++++++++++++++++++++++++++++++++++++++++++++++++++++++++++++++++++++++++++++++
\section{Polynômes à coefficients dans un anneau commutatif}
%+++++++++++++++++++++++++++++++++++++++++++++++++++++++++++++++++++++++++++++++++++++++++++++++++++++++++++++++++++++++++++
\label{SECooVMABooVdhbPo}

\begin{lemma}       \label{LEMooXFMAooMBgIrN}
    Nous considérons un polynôme \( P\in A[X]\), et le quotient \( A[X]/(P)\). Pour tout polynôme \( Q\in A[X]\) nous avons les égalités
    \begin{equation}
        Q(\bar X)=\overline{ Q(X) }=\bar Q.
    \end{equation}
\end{lemma}

\begin{proof}
    Si \( Q=\sum_ka_kX^k\), alors par la linéarité de la prise de classes,
    \begin{equation}        \label{EQooXQRMooIPGFVM}
        \bar Q=\sum_ka_k\overline{ X^k }.
    \end{equation}
    Nous insistons sur le fait que cette égalité n'est rien d'autre que l'itération de la définition de la somme dans l'espace quotient : \( \bar x+\bar y=\overline{ x+y }\) ainsi que du produit \( k\bar x=\overline{ kx }\) (définition~\ref{PROPooGXMRooTcUGbi}). Toujours par définition du produit appliqué à l'élément \( \bar X\) nous avons \( (\bar X)^2=\overline{ X^2 }\); par récurrence \( \overline{ X^k }=\bar X^k\), et
    \begin{equation}
        \bar Q=\sum_ka_k\bar X^k=Q(\bar X).
    \end{equation}

    Le fait que \( \bar Q=\overline{ Q(X) }\) n'est rien d'autre que le fait que dans \( A[X]\) nous avons \( Q=Q(X)\), comme expliqué dans le lemme~\ref{LEMooGKWQooVOyeDX}.
\end{proof}

%---------------------------------------------------------------------------------------------------------------------------
\subsection{Monômes}
%---------------------------------------------------------------------------------------------------------------------------

\begin{normaltext}
Les éléments de la forme \( \lambda X^k\) avec \( \lambda\in A\) et \( k\in\eN\) sont des \defe{monômes}{monôme}.

Nous allons aussi considérer\nomenclature[A]{\( A_n[X]\)}{les polynômes à coefficients dans \( A\) et de degré inférieur à \( n\)}
\begin{equation}
    A_n[X]=\{ P\in A[X]\tq \deg(P)\leq n \}.
\end{equation}
Cela est un sous-module libre.
\end{normaltext}

%---------------------------------------------------------------------------------------------------------------------------
\subsection{Évaluation}
%---------------------------------------------------------------------------------------------------------------------------

Soit \( P\in A[X]\). À priori, \( P\) n'est qu'une suite dans \( A\) indexée par \( \eN\). 


Nous avons déjà défini son évaluation sur un élément \( \alpha\in A\) dans la définition \ref{DEFooNXKUooLrGeuh} :
\begin{equation}
    P(\alpha)=\sum_ka_k\alpha^k.
\end{equation}
Cette somme est toujours finie.

\begin{normaltext}      \label{NORMooQFTJooLBcPxl}
    L'ensemble \( A[X]\) est une algèbre et donc un espace vectoriel. Il possède un unique élément nul qui est celui dont tous les coefficients sont nuls; cela est immédiat par la construction en tant que suites presque nulles.
\end{normaltext}

Il n'y a à priori pas équivalence entre le fait d'être un polynôme nul et le fait de s'évaluer à zéro sur tous les éléments de \( A\). Cela sera discuté dans le théorème~\ref{ThoLXTooNaUAKR} et l'exemple~\ref{exVQBooBMPLkD}.

\begin{definition}      \label{DEFooRFBFooKCXQsv}
    Soient un anneau \( A\) et un anneau \( B\) qui contient \( A\) (comme sous-anneau). Pour \( \alpha\in B\) nous définissons \( A[\alpha]_B\) comme étant l'intersection de tous les sous-anneaux de \( B\) contenant \( A\).
\end{definition}
Comme dit plus haut, nous nous permettons d'écrire \( A[\alpha]\) sans préciser \( B\) lorsque ce dernier sera clair dans le contexte.

\begin{proposition}     \label{PROPooPMNSooOkHOxJ}
    Soient un anneau \( A\) et un anneau \( B\) qui contient \( A\) (comme sous-anneau). Pour tout \( \alpha\in B\) nous avons
    \begin{equation}
        A[\alpha]=\{ P(\alpha)\tq P\in A[X] \}
    \end{equation}
    où encore une fois, \( P(\alpha)\) est calculé dans \( B\); le contexte est clair là-dessus.
\end{proposition}

\begin{proof}
    Si \( A'\) est un sous-anneau de \( B\) contenant \( A\) et \( \alpha\), alors \( A'\) contient tous les \( P(\alpha)\) avec \( P\in A[X]\). Nous avons donc
    \begin{equation}
        \{ P(\alpha)\tq P\in A[X] \}\subset A[\alpha].
    \end{equation}
    Par ailleurs, \( \{ P(\alpha)\tq P\in A[X] \}\) est un sous-anneau de \( B\) contenant \( A\) et \( \alpha\). Donc \( A[\alpha]\) y est inclus.
\end{proof}

%---------------------------------------------------------------------------------------------------------------------------
\subsection{Polynômes sur un anneau intègre}
%---------------------------------------------------------------------------------------------------------------------------

\begin{theorem}     \label{ThoBUEDrJ}
    L'anneau \( A\) est intègre si et seulement si \( A[X]\) est intègre.
\end{theorem}

\begin{proof}
    Soient \( P\) et \( Q\) des éléments non nuls de \( A[X]\). Vu que l'anneau \( A\) est intègre, nous avons
    \begin{equation}
        \deg(PQ)=\deg(P)+\deg(Q)
    \end{equation}
    et le produit ne peut pas être nul. L'anneau \( A[X]\) est donc intègre.

    Si \( A[X]\) est intègre, \( A\) est intègre parce qu'il peut être vu comme sous anneau.
\end{proof}

\begin{normaltext}
    Si \( A\) n'est pas intègre, soient \( \alpha,\beta\in A\) non nuls tels que \( \alpha\beta=0\). Le produit des polynômes \( X\mapsto \alpha X\) et \( X\mapsto \beta\) est \( (\alpha X)(\beta)=0\); le degré du produit n'est pas la somme des degrés.

    Les personnes qui ont tout compris jusqu'ici remarqueront que la notation «\( X\mapsto P(X)\)» n'est pas correcte parce que du point de vue que nous adoptons ici, un polynôme n'est pas une application.
\end{normaltext}

\begin{corollary}
    Si \( A\) est intègre, les inversibles de \( A[X]\) sont les éléments de \( U(A)\).
\end{corollary}

\begin{proof}
    Pour que \( Q\) soit inversible, il faut un \( P\) tel que \( PQ=1\). Mais l'anneau \( A\) étant intègre, les degrés s'additionnent. Par conséquent ils doivent être de degrés zéro et il faut que \( P,Q\in A\). Enfin pour qu'ils soient inversibles, ils doivent être dans \( U(A)\).
\end{proof}

La \defe{valuation}{valuation!d'un polynôme} du polynôme \( P=\sum_n a_nX^n\), notée \( \val(P)\), est
\begin{equation}
    \val(P)=\min\{ n\tq a_n\neq 0 \}.
\end{equation}
Nous avons \( \val(P)\leq \deg(P)\) et \( \val(P)=\deg(P)\) si et seulement si \( P\) est un monôme. Si \( P=0\), nous convenons que \( \val(0)=\infty\) et \( \deg(0)=-\infty\).

%---------------------------------------------------------------------------------------------------------------------------
\subsection{Division euclidienne}
%---------------------------------------------------------------------------------------------------------------------------

Le théorème suivant établit la \defe{division euclidienne}{division!euclidienne} dans \( A[X]\) du polynôme \( P\) par un polynôme \( D\).
\begin{theorem}     \label{ThodivEuclPsFexf}
    Soit \( D\neq 0\) dans \( A[X]\) de coefficient dominant inversible dans \( A\). Pour tout \( P\in A[X]\), il existe \( Q,R\in A[X]\) tels que
    \begin{equation}
        P=QD+R
    \end{equation}
    avec \( \deg(R)<\deg(D)\).

    Les polynômes \( Q\) et \( R\) sont déterminés de façon univoque par cette condition.
\end{theorem}

\begin{definition}\label{DefMPZooMmMymG}
    Le polynôme \( Q\) est le \defe{quotient}{quotient} et \( R\) est le \defe{reste}{reste} de la division euclidienne de \( P\) par \( D\). Si le reste de la division de \( P\) par $D$ est nul on dit que \( D\) \defe{divise}{diviseur!polynôme} \( P\) et on note \( D\divides P\)\nomenclature[A]{\( D\divides P\)}{\( D\) divise \( P\)}. Autrement dit \( D\) divise \( P\) s'il existe \( Q\) tel que \( P=QD\).\footnote{Ceci se rapproche tout naturellement des notions de divisibilité dans un anneau intègre général, vues en sous-section~\ref{DivisibiliteAnneauxIntegres}.}
\end{definition}

\begin{normaltext}
    Le théorème~\ref{ThodivEuclPsFexf} nous incite à utiliser le degré comme stathme euclidien sur \( A[X]\) dès que \( A\) est un anneau intègre. Or cela ne fonctionne en général pas parce que très peu de polynômes ont à priori un coefficient dominant inversible.
\end{normaltext}

\begin{lemma}[Thème~\ref{THEMEooZYKFooQXhiPD}]       \label{LEMooIDSKooQfkeKp}
    Si \( \eK\) est un corps\footnote{Définition~\ref{DefTMNooKXHUd}.}, alors l'anneau \( \eK[X]\) est euclidien et principal.
\end{lemma}

\begin{proof}
    Vu que \( \eK\) est un corps, tous les éléments sont inversibles et le degré donne un stathme par le théorème~\ref{ThodivEuclPsFexf}. L'anneau \( \eK[X]\) est donc euclidien et par conséquent principal (proposition~\ref{Propkllxnv}).
\end{proof}

Dans le théorème~\ref{ThoCCHkoU} nous donnerons une preuve directe du fait que \( \eK[X]\) est principal en montrant que tous ses idéaux sont principaux. Nous y démontrerons donc un peu moins pour un peu plus cher, mais avec le plaisir de ne pas devoir passer par un stathme.

\begin{definition}[\cite{ooSXFEooEehobn}]  \label{DefDSFooZVbNAX}
    Soit un anneau \( A\). Deux polynômes \( P\) et \( Q\) dans \( A[X]\) sont dits \defe{étrangers}{etranger@étrangers!polynômes} entre eux si \( 1\) est un pgcd\footnote{Définition~\ref{DefrYwbct}.} de \( P\) et \( Q\). Un ensemble de polynômes \( (P_i)_{i\in I}\) est étranger \defe{dans leur ensemble}{étranger!dans leur ensemble} si \( 1\) est un \( \pgcd\) des \( P_i\).

Les polynômes \( P\) et \( Q\) sont \defe{premiers entre eux}{premier!deux polynômes entre eux} si les seuls diviseurs communs de \( P\) et \( Q\) sont les inversibles.
\end{definition}

Les notions de polynômes étrangers entre eux ou de polynômes premiers entre eux ne sont pas identiques, comme le montre l'exemple suivant.

\begin{example}[\cite{MonCerveau}]
    Soient dans \( \eZ[X]\) les polynômes \( P(X)=2X+2\) et \( Q(X)=2X^2+2\). Le nombre \( 2\) est diviseur commun et n'est pas un diviseur de \( 1\). Donc \( 1\) n'est pas un pgcd de \( P\) et \( Q\). Ils ne sont pas étrangers.

    Mais ils sont premiers entre eux parce qu'ils n'ont pas d'autres diviseurs communs que les inversibles (\( 1\) et \( -1\)).
\end{example}

%---------------------------------------------------------------------------------------------------------------------------
\subsection{Polynôme primitif}
%---------------------------------------------------------------------------------------------------------------------------

\begin{definition}\label{DefContenuPolynome}
    Le \defe{contenu}{contenu}\index{polynôme!contenu} du polynôme \( P=\sum_ia_iX^i\in\eK[X]\) est le pgcd de ses coefficients : $c(P)=\pgcd(a_i)$.
\end{definition}

\begin{definition}[Ordre d'un polynôme]
    Soit \( P\) un polynôme irréductible de degré \( n\) sur \( \eF_p[X]\). L'\defe{ordre}{ordre!d'un polynôme} de \( P\) est
    \begin{equation}
        \min\{ k\tq P\divides X^k-1 \}.
    \end{equation}
\end{definition}

\begin{definition}[Polynôme primitif]           \label{DEFooDVOOooKaPZQC}
    Soit \( p\), un nombre premier et \( P\) un polynôme de degré $n$ dans \( \eF_p[X]\). Nous disons que \( P\) est \defe{primitif}{primitif!polynôme} si
    \begin{enumerate}
        \item
            \( P\) est unitaire et irréductible,
        \item
            les racines de \( P\) sont d'ordre \( p^n-1\) dans \( \eF_p[X]/P\).
    \end{enumerate}
\end{definition}

\begin{definition}[Polynôme primitif au sens du pgcd]       \label{DEFooAIYGooRAEfHU}
    Soit un anneau \( A\). Un polynôme \( P\in A[X]\) est \defe{primitif au sens du pgcd}{primitif!polynôme!au sens du pgcd} si ses coefficients sont premiers entre eux.
\end{definition}

\begin{normaltext}
    Pour rappel, il y a plusieurs façons de périphraser le fait que les coefficients soient premiers entre eux. Nous pouvons dire \ldots
    \begin{enumerate}
        \item
            Le pgcd de ses coefficients est \( 1\) parce que c'est la définition~\ref{DEFooXSPFooPumQSy} d'avoir des nombres premiers entre eux.
        \item
            Le contenu de ses coefficients est \( 1\). Parce que le contenu est précisément le pgcd, définition~\ref{DefContenuPolynome}.
    \end{enumerate}
\end{normaltext}

La notion de polynôme primitif au sens du pgcd est particulière aux polynômes à coefficients dans un anneau comme le montre le lemme suivant.

\begin{lemma}
    Si \( \eK\) est un corps, tout polynôme unitaire dans \( \eK[X]\) non nul est primitif au sens du pgcd.
\end{lemma}

\begin{proof}
    Un polynôme unitaire a un \( 1\) parmi ses coefficients, donc le pgcd est forcément \( 1\).
\end{proof}

Lorsque nous utiliserons la notion de polynôme primitif au sens du \( \pgcd\), nous le mentionnerons explicitement. C'est pas exemple le cas pour le corolaire~\ref{CORooZCSOooHQVAOV}.

%---------------------------------------------------------------------------------------------------------------------------
\subsection{Racines des polynômes}
%---------------------------------------------------------------------------------------------------------------------------

\begin{definition}
  Soient \( A \) un anneau et \( P \in A[X] \). On appelle
  \defe{racine}{racine!d'un polynôme} un élément \( \alpha \in A \)
  tel que \( P(\alpha) = 0 \); c'est-à-dire que, en remplaçant toutes
  les occurrences de $X$ par $\alpha$ dans l'expression de $P$, on
  obtient $0$.
\end{definition}

\begin{proposition} \label{PropHSQooASRbeA}
    Soient \( A\) un anneau et \( P\) un polynôme non nul dans \( A[X]\). Si \( \alpha\in A\) est une racine de \( P\) alors \( X-\alpha\) divise \( P\), et réciproquement.
\end{proposition}

\begin{proof}
  Nous notons le polynôme \( \mu=X-\alpha\) par analogie avec le polynôme minimal dont il sera question dans la très semblable proposition~\ref{PropXULooPCusvE}. Le sens réciproque est clair: si $\mu$ divise $P$, alors $\alpha$ est racine de $P$.

  Pour le sens direct, remarquons que si $\alpha$ est racine de $P$, alors $P$ est de degré au moins égal à \( 1\), et nous pouvons donc effectuer la division euclidienne\footnote{Théorème~\ref{ThodivEuclPsFexf}.} de \( P\) par \( \mu\) : il existe des polynômes \( Q\) et \( R\) tels que
    \begin{equation} \label{PropHSQooASRbeA1}
        P=Q\mu+R
    \end{equation}
    avec \( \deg(R)<\deg(\mu)\). Donc \( R\) est une constante,
    élément de $A$: appelons-le $a$. En évaluant
    \eqref{PropHSQooASRbeA1} en \( \alpha\), il vient
    \begin{equation}
        0 = P(\alpha)=Q(\alpha)\mu(\alpha)+a,
    \end{equation}
    et nous en déduisons que \( a=0\), ce qui montre que \( P=Q\mu\) et que \( \mu\) divise \( P\).
\end{proof}

\begin{definition}[Racine simple et multiple d'un polynôme]
  Soit \( A\) un anneau ainsi qu'un polynôme \( P\in A[X]\) et \( \alpha\in A\) racine de $P$. La \defe{multiplicité}{multiplicité!racine d'un polynôme} de \( \alpha\) par rapport à \( P\) est l'entier \( h\) tel que \( P\) est divisible par \( (X-\alpha)^h\) mais pas divisible par \( (X-\alpha)^{h+1}\).  Nous noterons \( \theta_{\alpha}(P)\)\nomenclature[A]{\( \theta_{\alpha}(P)\)}{la multiplicité de \( \alpha\) par rapport à \( P\)} la multiplicité de \( \alpha\) par rapport à \( P\).
\end{definition}

Pour une définition générale d'une racine simple de l'équation \( f(x)=0\), voir la définition~\ref{DEFooXSOQooAnWqKM}. La proposition~\ref{PropHSQooASRbeA} nous indique que toute racine est de multiplicité au moins \( 1\).


\begin{proposition}     \label{PROPooQCZSooVokxXQ}
    Soient un anneau \( A\), un polynôme \( P\in \eA[X]\) de degré \( n\), ainsi qu'une racine \( \alpha\in A\) de \( P\). Alors il existe un polynôme \( Q\in \eA[X]\) de degré \( n-1\) tel que \( P=(X-\alpha)Q\).
\end{proposition}


\begin{proposition} \label{PropahQQpA}
  L'élément \( \alpha\in A\) est une racine de multiplicité \( h\) du
  polynôme \( P\) si et seulement s'il existe \( Q\in A[X]\) tel que
  \( P=(X-\alpha)^hQ\) avec \( Q(\alpha)\neq 0\).
\end{proposition}

\begin{lemma}       \label{LemIeLhpc}
    Soient \( P\) et \( Q\) des polynômes non nuls de \( A[X]\) et \( \alpha\in A\). Alors
    \begin{enumerate}
        \item
            \( \theta_{\alpha}(P+Q)\leq\min\{
            \theta_{\alpha}(P),\theta_{\alpha}(Q) \}\), et l'égalité a
            lieu si \( \theta_{\alpha}(P)\neq \theta_{\alpha}(Q)\);
        \item     \label{ItemIeLhpciv}
            \( \theta_{\alpha}(PQ)\geq
            \theta_{\alpha(P)}+\theta_{\alpha}(Q)\), et l'égalité a
            lieu si \( A \) est intègre.
    \end{enumerate}
\end{lemma}

Dans le théorème suivant, la partie importante en pratique est la seconde partie parce qu'elle dit que, lorsque nous cherchons les racines d'un polynôme, nous pouvons nous arrêter lorsque nous en avons trouvé autant que le degré, multiplicité comprise.
\begin{theorem} \label{ThoSVZooMpNANi}
    Soit \( A\) un anneau intègre
  et \( P\in A[X]\setminus\{ 0 \}\), un polynôme de degré \( n\). 

  \begin{enumerate}
      \item
  Si \( \alpha_1,\ldots, \alpha_p\in A\) sont des racines deux à deux
  distinctes de multiplicités \( k_1,\ldots, k_p\), alors il existe \(
  Q\in A[X]\), de degré \( n-p\), tel que \(
  P=Q\prod_{i=1}^p(X-\alpha_i)^{k_i}\) et \( Q(\alpha_i)\neq 0\) pour
  tout $i$.
  \item     \label{ITEMooWGOBooGApPOo}
    La somme des multiplicités des racines de \( P\) est au plus \( \deg(P)\).
  \end{enumerate}
\end{theorem}
\index{factorisation!de polynôme}

\begin{proof}
    Si \( p=1\), soit \( \alpha\) une racine de multiplicité \( k\) de \( P\). La définition de la multiplicité d'une racine nous dit que \( P\) est divisible par \( (X-\alpha)^k\) mais pas par \( (X-\alpha)^{k+1}\). Donc il existe \( Q\in \eA[X]\) tel que \( P=Q(X-\alpha)^k\). Il reste à voir que \( Q(\alpha)\neq 0\). Cela est une conséquence de la proposition~\ref{PropHSQooASRbeA} : si \( Q(\alpha)\) était nul, on pourrait lui factoriser \( (X-\alpha)\) et donc avoir \( (X-\alpha)^{k+1}\) qui se factorise dans \( P\), ce qui n'est pas possible.

    Nous supposons que \( p\geq 2\) et nous effectuons une récurrence sur \( p\). Nous considérons donc les \( p-1\) premières racines \( \alpha_1,\ldots, \alpha_{p-1}\) et un polynôme \( R\in\eA[X]\) tel que \( R(\alpha_i)\neq 0\) pour \( i=1,\ldots, p-1\) et
    \begin{equation}
        P=\underbrace{(X-\alpha_1)^{k_1}\ldots (X-\alpha_{p-1})^{k_{p-1}}}_SR.
    \end{equation}
    Par hypothèse \( P(\alpha_p)=S(\alpha_p)R(\alpha_p)=0\). L'anneau \( \eA\) étant intègre, \( S(\alpha_p)\neq 0\) parce que \( \alpha_i\neq \alpha_p\) pour \( i\neq p\). Par conséquent, \( R(\alpha_p)=0\).

    Nous devons encore vérifier que la multiplicité \( \alpha_p\) est \( k_p\) par rapport à \( R\). Pour cela nous utilisons le point~\ref{ItemIeLhpciv} du lemme~\ref{LemIeLhpc} afin de dire que le degré de \( \alpha_p\) pour \( P=SR\) est \( k_p\). Par conséquent
    \begin{equation}
        R=(X-\alpha_p)^{k_p}T
    \end{equation}
    avec \( T(\alpha_p)\neq 0\) et enfin
    \begin{equation}
        P=\prod_{i=1}^p(X-\alpha_i)T.
    \end{equation}
    De plus \( T(\alpha_i)\neq 0\), sinon \( R(\alpha_i)\) serait nul.
\end{proof}

\begin{corollary}       \label{CORooUGJGooBofWLr}
    Un polynôme de degré \( n\) possède au maximum \( n\) racines distinctes.
\end{corollary}

\begin{proof}
    Le théorème \ref{ThoSVZooMpNANi}\ref{ITEMooWGOBooGApPOo} dit que la somme des multiplicités des racines de \( P\) est au maximum \( n\). Mais la proposition \ref{PropHSQooASRbeA} dit que toutes les racines ont une multiplicité au moins un. Donc il ne peut pas y en avoir plus de \( n\).
\end{proof}

\begin{corollary}[Conséquence du lemme de Gauss\cite{ooCDLEooEQGSvn}]       \label{CORooZCSOooHQVAOV}
    Soient \( A\) un anneau factoriel et \( \Frac(A)\) son corps des fractions. Un polynôme non constant \( P\in A[X]\) est irréductible (sur \( A\)) si et seulement s'il est irréductible et primitif au sens du pgcd\footnote{Définition~\ref{DEFooAIYGooRAEfHU}.} sur \( \Frac(A)[X]\).
\end{corollary}

\begin{example}
    Il ne faudrait pas croire qu'être irréductible dans un anneau \( A\) implique d'être irréductible dans le corps des fractions. En effet soit \( A=\eZ[\sqrt{ 5 }]\) et \( P=X^2-X-1\). Nous savons que sa factorisation est 
    \begin{equation}
        P=\left( X-\frac{ 1+\sqrt{ 5 } }{ 2 } \right)\left( X-\frac{ 1-\sqrt{ 5 } }{ 2 } \right).
    \end{equation}
    Si vous ne le saviez pas, faites juste le calcul pour vous en assurer.

    Ce polynôme est irréductible sur \( \eZ[\sqrt{ 5 }]\) mais pas irréductible sur \( \Frac\big( \eZ[\sqrt{ 5 }] \big)\).
\end{example}

%---------------------------------------------------------------------------------------------------------------------------
\subsection{Quelques identités}
%---------------------------------------------------------------------------------------------------------------------------

\begin{lemma}   \label{LemISPooHIKJBU}
    Quelques identités de polynômes.
    \begin{enumerate}
        \item   \label{ItemLTBooAcyMtN}
            Si \( n\) est impair, alors \( 1+X\) divise \( 1+X^n\).
        \item\label{ItemLTBooAcyMtNii}
            Pour tout \( n\) nous avons \( X^n-1=(X-1)(1+X+\cdots +X^{n-1})\).
        \item
            \( X^n-a^n=(X-a)\sum_{i=0}^{n-1}a^iX^{n-1-i}\).
    \end{enumerate}
\end{lemma}

\begin{proof}
  Nous démontrons uniquement le point~\ref{ItemLTBooAcyMtNii}, puisque
  les autres ont été vus en début de chapitre\footnote{Voir l'égalité
    \eqref{Eqarpurmkbk}.}. Le cas \( n=1\) est évident. Procédons
  alors par récurrence en considérant un nombre entier impair \( n\) :
    \begin{subequations}
        \begin{align}
            1+X^{n+2}&=1+X^n+X^{n+2}-X^n\\
                    &=(1+X)P+X^n(X^2-1)\\
                    &=(1+X)P+X^n(X+1)(X-1)\\
                    &=(1+X)\big( P+X^n(X-1) \big).
        \end{align}
    \end{subequations}
\end{proof}


\chapter{Espaces vectoriels (début)}
% This is part of Mes notes de mathématique
% Copyright (c) 2008-2020
%   Laurent Claessens
% See the file fdl-1.3.txt for copying conditions.

%+++++++++++++++++++++++++++++++++++++++++++++++++++++++++++++++++++++++++++++++++++++++++++++++++++++++++++++++++++++++++++
\section{Parties libres, génératrices, bases et dimension}
%+++++++++++++++++++++++++++++++++++++++++++++++++++++++++++++++++++++++++++++++++++++++++++++++++++++++++++++++++++++++++++

Nous avons déjà défini (dans~\ref{DEFooKHWZooIfxdNc}) un espace vectoriel comme étant un module sur un corps commutatif. En explicitant un peu, cela donne ceci\cite{ooQLVLooEUrNLS}.

Un espace vectoriel sur le corps \( \eK\) est un ensemble \( V\) muni de deux opérations :
\begin{itemize}
    \item une loi de composition interne \( +\colon V\times V\to V\),
    \item une loi de composition externe \( \cdot\colon \eK\times V\to V\)
\end{itemize}
telles que
\begin{enumerate}
    \item
        \( (V,+)\) soit un groupe abélien,
    \item
        pour tout \( u,v\in V\) et pour tout \( k,k'\in \eK\),
        \begin{subequations}
           \begin{align}
                k(u+v)=(ku)+(kv)\\
                (kk')u=k(k'u)\\
                (k+k')u=(ku)+(k'u)\\
                1u=u
            \end{align}
        \end{subequations}
        où \( 1\) est le neutre de \( \eK\) et où nous avons directement adopté la notation \( ku\) pour \( k\cdot u\).
\end{enumerate}
Si \( u\in V\), nous notons \( -u\) l'inverse de \( u\) dans le groupe \( (V,+)\).

\begin{definition}[Partie libre]
    Si \( E\) est un espace vectoriel, une partie \( A\) de \( E\) est \defe{libre}{libre!partie} si pour tout choix d'un nombre fini d'éléments \( \{ u_i \}_{i=1,\ldots, n}\), l'égalité
    \begin{equation}
        a_1 u_1+\cdots +a_nu_n=0
    \end{equation}
    implique \( a_i=0\) pour tout \( i\) (ici les \( a_i\) sont dans le corps de base).

    Une partie infinie est libre si toutes ses parties finies le sont.
\end{definition}

\begin{remark}
    Notons que le vecteur nul n'est dans aucune partie libre, ne fût-ce que parce que \( a0=0\) n'implique pas \( a=0\).
\end{remark}

Si \( A\) est une partie de l'espace vectoriel \( E\) nous notons \( \Span(A)\)\nomenclature[A]{$\Span(A)$}{l'ensemble des combinaisons linéaires finies d'éléments de \( A\)} l'ensemble des combinaisons linéaires finies d'éléments de \( A\). Les coefficients de ces combinaisons linéaires sont dans le corps de base \( \eK\).

\begin{definition}[Partie génératrice]
    Une partie $B$ d'un espace vectoriel \( E\) est \defe{génératrice}{partie!génératrice} si \( \Span(B)=E\).
\end{definition}

\begin{remark}
    Ces définitions demandent des commentaires en dimension infinie\footnote{Nous n'avons pas encore défini le concept de dimension, mais nous nous adressons au lecteur trop pressé.}.

    \begin{enumerate}
        \item
    Tout élément peut être écrit comme combinaison linéaire finie d'une partie génératrice. Cela ne signifie pas que nous pouvons extraire une partie finie qui convient pour tous les éléments à la fois. Lorsque l'espace est de dimension infinie, ceci est particulièrement important.
\item
    La définition séparée de liberté dans le cas des parties infinies a son importance lorsqu'on parle d'espaces vectoriels de dimension infinies (en dimension finie, aucune partie infinie n'est libre) parce que cela fera une différence entre une base algébrique et une base hilbertienne par exemple.
    \end{enumerate}
\end{remark}

\begin{definition}[Base]        \label{DEFooNGDSooEDAwTh}
    Une \defe{base}{base} de l'espace vectoriel \( E\) est une partie à la fois génératrice et libre.
\end{definition}

\begin{proposition}[\cite{MonCerveau}]      \label{PROPooEIQIooXfWDDV}
    Tout élément non nul d'un espace vectoriel possédant une base\footnote{Nous n'avons pas démontré que tout espace vectoriel possède une base. Donc à notre niveau, il est possible que ce théorème soit sans objet pour beaucoup d'espaces.} se décompose de façon unique en combinaison linéaire finie d'éléments d'une base.
\end{proposition}

\begin{proof}
    Soit un espace vectoriel \( E\) et une base \( \{ e_i \}_{i\in I}\) où \( I\) est un ensemble à priori quelconque. Soit \( v\in E\). Vu que \( E=\Span\{ e_i \}_{i\in I}\), il existe une partie finie \( J\) de \( I\) et des coefficients \( \{ v_j \}_{j\in J}\) dans \( \eK\) tels que
    \begin{equation}
        v=\sum_{j\in J}v_je_j.
    \end{equation}
    Cela donne l'existence.

    En ce qui concerne l'unicité, soient \( J \) et \( K\) des parties finies de \( I\) et des coefficients \( \{ v_j \}_{j\in J}\) et \( \{ w_{k} \}_{k\in K}\) tels que
    \begin{equation}
        v=\sum_{j\in J}v_je_j=\sum_{k\in K}w_{k}e_{k}.
    \end{equation}
    Nous posons \( L=J\cup K\) et, pour \( l\in L\),
    \begin{equation}
        \alpha_l=\begin{cases}
            v_l    &   \text{si } l\in J\setminus K\\
            w_l    &    \text{si } l\in K\setminus J\\
            v_l-w_l    &    \text{si } l\in K\cap J.
        \end{cases}
    \end{equation}
    Nous avons alors
    \begin{equation}
        \sum_{l\in L}\alpha_le_l=0,
    \end{equation}
    ce qui implique que \( \alpha_l=0\) pour tout \( l\in L\) parce que la partie \( \{ e_i \}_{i\in I}\) est libre et que \( L\) est finie.

    L'unicité de la décomposition de \( v\) signifie que
    \begin{equation}
        \{ j\in J \tq v_j\neq 0 \}=\{ k\in K\tq w_k\neq 0 \}
    \end{equation}
    et que pour \( l\) dans cet ensemble, \( v_l=w_l\).

    Soit \( j\in J\); il y a deux possibilités : \( j\in J\setminus K\) et \( j\in J\cap K\). Dans le premier cas nous avons déjà vu que \( \alpha_j=v_j=0\). Dans le second cas, \( \alpha_j=v_j-w_j=0\), c'est-à-dire \( v_j=w_j\).

    Donc \( j\in J\) vérifiant \( v_j\neq 0\) implique \( j\in J\cap K\) et l'égalité des coefficients. Idem avec \( k\in K\) tel que \( w_k\neq 0\) implique \( k\in J\cap K\).
\end{proof}

\begin{lemma}[\cite{MonCerveau}]        \label{LEMooDJSIooYcsvhO}
    Soit un espace vectoriel admettant des bases. Un endomorphisme est une bijection si et seulement si il change toute base en une base.
\end{lemma}

\begin{proof}
    En deux parties. Soit un espace vectoriel \( E\) possédant des bases et un endomorphisme \( f\colon E\to E\).
    \begin{subproof}
        \item[Si \( f\) est bijective]
            Soit une base \( \{ v_i \}_{i\in I}\); nous devons voir que \( \{ f(v_i) \}_{i\in I}\) est une base.
            \begin{subproof}
                \item[Libre]
                    Si \( J\) est une partie finie de \( I\) et si les \( \lambda\j\) sont des scalaires tels que \( \sum_{j\in J}\lambda_jf(v_j)=0\), alors
                    \begin{equation}
                        0=\sum_{j\in J}\lambda_jf(v_j)=f\big( \sum_{j\in J}\lambda_jv_j \big).
                    \end{equation}
                    Mais comme \( f\) est bijective, cela implique que \( \sum_{j\in J}\lambda_jv_j=0\). En retour, parce que \( \{ v_i \}\) est une base, cela implique que \( \lambda_j=0\) pour tout \( j\).
                \item[Générateur]
                    Soit \( x\in E\). Vu que \( f\) est bijective, il existe un unique \( y\in E\) tel que \( x=f(y)\). Comme \( \{ v_i \}_{i\in I}\) est une base, il existe une partie finie \( J\subset I\) et des scalaires \( \{ \lambda_j \}_{j\in J}\) tels que
                    \begin{equation}
                        y=\sum_{j\in J}\lambda_jv_j.
                    \end{equation}
                    Nous avons alors
                    \begin{equation}
                        x=f(y)=\sum_{j\in J}\lambda_jf(v_j),
                    \end{equation}
                    qui montre que \( \{ f(v_i) \}_{i\in I}\) est bien génératrice de \( E\)
            \end{subproof}
        \item[Si \( f\) change les bases en bases]
            Soit un endomorphisme changeant toute base en une base. Nous devons prouver qu'il est bijectif.
            \begin{subproof}
                \item[Injective]
                    Nous considérons une base \( \{ v_i \}_{i\in I}\). La partie \( \{ f(v_i) \}_{i\in I}\) est par hypothèse également une base.

                    Soient \( x,y\in E\) tels que \( f(x)=f(y)\). Il existe \( J\) et \( K\) finis dans \( I\) qui permettent de décomposer \( x\) et \( y\) respectivement dans la base \( \{ f(v_i) \}_{i\in I}\). Quitte à poser \( J'=J\cup K\), nous supposons que \( J\) suffit\footnote{Nous utilisons le fait que l'union de deux parties finies d'une ensemble est finie (lemme \ref{LEMooYHGCooAwsVQN}).}. Il existe donc des scalaires \( \{ \lambda_j \}_{j\in J}\) et \( \{ \mu_j \}_{j\in J}\) tels que \( x=\sum_{j\in J}\lambda_jf(v_j)\) et \( y=\sum_{j\in J}\mu_jf(v_j)\).

                    La relation \( f(x)=f(y)\) donne immédiatement, par la linéarité de \( f\),
                    \begin{equation}
                        \sum_{j\in J}(\lambda_j-\mu_j)f(v_j)=0.
                    \end{equation}
                    Du fait que \( \{ f(v_i) \}_{i\in I}\) soit une base, nous déduisons que \( \lambda_j-\mu_j=0\) pour tout \( j\). Donc \( x=y\), et \( f\) est injective.
                \item[Surjective]
                    Soit \( x\in E\). Vu que \( \{ f(v_i) \}_{i\in I}\) est une base, il existe des scalaires \( \lambda_j\) tels que 
                    \begin{equation}
                        x=\sum_{j\in J}\lambda_jf(v_j)=f\big( \sum_{j\in J}\lambda_jv_j \big).
                    \end{equation}
                    Donc \( f\) est surjective.
            \end{subproof}
    \end{subproof}
\end{proof}

\begin{definition}
    Un espace vectoriel est \defe{de type fini}{type!fini!espace vectoriel} s'il contient une partie génératrice finie.
\end{definition}
Nous verrons dans les résultats qui suivent que cette définition est en réalité inutile parce qu'une espace vectoriel sera de type fini si et seulement s'il est de dimension finie.

\begin{lemma}       \label{LemytHnlD}
    Si \( E\) a une famille génératrice de cardinal \( n\), alors toute famille de \( n+1\) éléments est liée.
\end{lemma}

\begin{proof}
    Nous procédons par récurrence sur \( n\). Pour \( n=1\), nous avons \( E=\Span(e)\) et donc si \( v_1,v_2\in E\) nous avons \( v_1=\lambda_1 e\), \( v_2=\lambda_2e\) pour certains éléments non nuls \( \lambda_1,\lambda_2\) du corps de base. Nous avons donc \( \lambda_2v_1-\lambda_1v_1=0\). Cela prouve que \( \{ v_1,v_2 \}\) est liée.

    Supposons maintenant que le résultat soit vrai pour \( k<n\), c'est-à-dire que pour tout espace vectoriel contenant une partie génératrice de cardinal \( k<n\), les parties de \( k+1\) éléments sont liées. Soit maintenant un espace vectoriel muni d'une partie génératrice \( G=\{ e_1,\ldots, e_n \}\) de \( n\) éléments, et montrons que toute partie \( V=\{ v_1,\ldots, v_{n+1} \}\) contenant \( n+1\) éléments est liée. Dans nos notations nous supposons que les \( e_i\) sont des vecteurs distincts et les \( v_i\) également. Nous les supposons également tous non nuls. Étant donné que \( \{ e_i \}\) est génératrice nous pouvons définir les nombres \( \lambda_{ij}\) par
    \begin{equation}
        v_i=\sum_{k=1}^n\lambda_{ij}e_j
    \end{equation}
    Vu que
    \begin{equation}
        v_{n+1}=\sum_{k=1}^n\lambda_{n+1,k}e_k\neq 0,
    \end{equation}
    quitte à changer la numérotation des \( e_i\) nous pouvons supposer que \( \lambda_{n+1,n}\neq 0\). Nous considérons les vecteurs
    \begin{equation}
        w_i=\lambda_{n+1,n}v_i-\lambda_{i,n}v_{n+1}.
    \end{equation}
    En calculant un peu,
    \begin{subequations}
        \begin{align}
            w_i&=\lambda_{n+1,n}\sum_k\lambda_{i,k}e_k-\lambda_{i,n}\sum_k\lambda_{n+1,k}e_k\\
            &=\sum_{k=1}^{n-1}\big( \lambda_{n+1,n}\lambda_{i,k}-\lambda_{i,n}\lambda_{n+1,} \big)e_k
        \end{align}
    \end{subequations}
    parce que les termes en \( e_n\) se sont simplifiés. Donc la famille \( \{ w_1,\ldots, w_n \}\) est une famille de \( n\) vecteurs dans l'espace vectoriel \( \Span\{ e_1,\ldots, e_{n-1} \}\); elle est donc liée par l'hypothèse de récurrence. Il existe donc des nombres \( \alpha_1,\ldots, \alpha_n\in \eK\) non tous nuls tels que
    \begin{equation}        \label{EqOQGGoU}
        0=\sum_{i=1}^n\alpha_iw_i=\sum_{i=1}^n\alpha_i\lambda_{n+1,n}v_i-\left( \sum_{i=1}^n\alpha_i\lambda_{i,n} \right)v_{n+1}.
    \end{equation}
    Vu que \( \lambda_{n+1,n}\neq 0\) et que parmi les \( \alpha_i\) au moins un est non nul, nous avons au moins un des produits \( \alpha_i\lambda_{n+1,n}\) qui est non nul. Par conséquent \eqref{EqOQGGoU} est une combinaison linéaire nulle non triviale des vecteurs de \( \{ v_1,\ldots, v_{n+1} \}\). Cette partie est donc liée.
\end{proof}

\begin{lemma}   \label{LemkUfzHl}
    Soit \( L\) une partie libre et \( G\) une partie génératrice. Si l'ensemble des parties libres \( L'\) telles que \( L\subset L'\subset G\) possède un élément maximum\footnote{Encore une fois, à part quelques cas triviaux, il n'est pas clair à ce point que ce maximum existe.}, alors cet élément est une base.
\end{lemma}
Qu'entend-on par «maximale» ? La partie \( B\) doit être libre, contenir \( L\), être contenue dans \( G\) et de plus avoir la propriété que \( \forall x\in G\setminus B\), la partie \( B\cup\{ x \}\) est liée.

\begin{proof}
    D'abord si \( G\) est une base, alors toutes les parties de \( G\) sont libres et le maximum est \( B=G\). Dans ce cas le résultat est évident. Nous supposons donc que \( G\) est liée.

    La partie \( B=\{ b_1,\ldots, b_l \}\) est libre parce qu'on l'a prise parmi les libres. Montrons que \( B\) est génératrice. Soit \( x\in G\setminus B\); par hypothèse de maximalité, \( B\cup\{ x \}\) est liée, c'est-à-dire qu'il existe des nombres \( \lambda_i\), \( \lambda_x\) non tous nuls tels que
    \begin{equation}    \label{EqxfkevM}
        \sum_{i=1}^l\lambda_ib_i+\lambda_xx=0.
    \end{equation}
    Si \( \lambda_x=0\) alors un de \( \lambda_i\) doit être non nul et l'équation \eqref{EqxfkevM} devient une combinaison linéaire nulle non triviale des \( b_i\), ce qui est impossible parce que \( B\) est libre. Donc \( \lambda_x\neq 0\) et
    \begin{equation}
        x=\frac{1}{ \lambda_x }\sum_{i=1}^l\lambda_ib_i.
    \end{equation}
    Donc tous les éléments de \( G\setminus B\) sont des combinaisons linéaires des éléments de \( B\), et par conséquent, \( G\) étant génératrice, tous les éléments de \( E\) sont combinaisons linéaires d'éléments de \( B\).
\end{proof}

\begin{theorem}[Théorème de la base incomplète] \label{ThonmnWKs}
    Soit \( E\) un espace vectoriel de type fini sur le corps \( \eK\).
    \begin{enumerate}
        \item     \label{ItemBazxTZ}
            Si \( L\) est une partie libre et si \( G\) est une partie génératrice contenant \( L\), alors il existe une base \( B\) telle que \( L\subset B\subset G\).
        \item     \label{ITEMooFVJXooGzzpOu}
            Toute partie libre peut être étendue en une base.
        \item     \label{ITEMooFBUAooSSZxgx}
            Toutes les bases sont finies et ont même cardinal.
        \item       \label{ITEMooJIJSooGuJMdt}
            Si \( V\) est un sous-espace vectoriel de \( E\), et si \( L\) est une base de \( V\), alors il existe une base \( E\) qui contient \( L\).
    \end{enumerate}
\end{theorem}
\index{théorème!base incomplète}

\begin{proof}
    Point par point.
    \begin{enumerate}
        \item
    Vu que \( E\) est de type fini, il admet une partie génératrice \( G\) de cardinal fini \( n\). Donc une partie libre est de cardinal au plus \( n\) par le lemme~\ref{LemytHnlD}. Soit \( L\), une partie libre contenue dans \( G\) (ça existe : par exemple \( L=\emptyset\)). La partie \( B\) maximalement libre contenue dans \( G\) et contenant \( L\) est une base par le lemme~\ref{LemkUfzHl}.
\item
Notons que puisque \( E\) lui-même est générateur, le point~\ref{ItemBazxTZ} implique que toute partie libre peut être étendue en une base.
\item
    Soient \( B\) et \( B'\), deux bases. En particulier \( B\) est génératrice et \( B'\) est libre, donc le lemme~\ref{LemytHnlD} indique que \( \Card(B')\leq \Card(B)\). Par symétrie on a l'inégalité inverse. Donc \( \Card(B)=\Card(B')\).
\item
    La partie \( L\) étant une base de \( V\), elle est en particulier libre dans \( E\). Par le point \ref{ITEMooFVJXooGzzpOu}, \( L\) peut être étendue en une base.
    \end{enumerate}
\end{proof}

\begin{remark}      \label{REMooYGJEooEcZQKa}
    Le théorème de la base incomplète~\ref{ThonmnWKs}\ref{ITEMooFVJXooGzzpOu} est ce qui permet de construire une base d'une espace vectoriel en « commençant par» une base d'un sous-espace. En effet si \( H\) est un sous-espace de \( E\) alors une base de \( H\) est une partie libre de \( E\) et donc peut être étendue en une base de \( E\).
\end{remark}

\begin{definition}      \label{DEFooWRLKooArTpgh}
    La \defe{dimension}{dimension} d'un espace vectoriel de type fini est le cardinal\footnote{Définition \ref{PROPooJLGKooDCcnWi}.} d'une\footnote{Le théorème de la base incomplète~\ref{ThonmnWKs}\ref{ITEMooFBUAooSSZxgx} montre que cette définition ne souffre d'aucune ambiguïté.} de ses bases.
\end{definition}
\index{dimension!définition}

Il existe une infinité de bases de $\eR^m$. On peut démontrer que le cardinal de toute base de $\eR^m$ est $m$, c'est-à-dire que toute base de $\eR^m$ possède exactement $m$ éléments.

\begin{example}
    La base de \defe{canonique}{canonique!base}\index{base canonique de $\eR^m$} de \( \eR^m\) est la partie $\{e_1,\ldots, e_m\}$, où le vecteur $e_j$ est
    \begin{equation}\nonumber
      e_j=
    \begin{array}{cc}
      \begin{pmatrix}
        0\\\vdots\\0\\1\\ 0\\\vdots\\0
      \end{pmatrix} &
      \begin{matrix}
        \quad\\\quad\\\leftarrow\textrm{j-ème} \quad\\\quad\\\quad\\
      \end{matrix}
    \end{array}.
    \end{equation}
    La composante numéro $j$ de $e_i$ est $1$ si $i=j$ et $0$ si $i\neq j$. Cela s'écrit $(e_i)_j=\delta_{ij}$ où $\delta$ est le \defe{symbole de Kronecker}{Kronecker} défini par
    \begin{equation}
        \delta_{ij}=\begin{cases}
            1	&	\text{si }i=j\\
            0	&	 \text{si }i\neq j
        \end{cases}
    \end{equation}
    Les éléments de la base canonique de $\eR^m$ peuvent donc être écrits $e_i=\sum_{k=1}^m\delta_{ik}e_k$.
\end{example}

Le théorème suivant est essentiellement une reformulation du théorème~\ref{ThonmnWKs}.
\begin{theorem} \label{ThoMGQZooIgrXjy}
    Soit \( E\) un espace vectoriel de dimension finie et \( \{ e_i \}_{i\in I}\) une partie génératrice de \( E\).

    \begin{enumerate}
        \item       \label{ITEMooTZUDooFEgymQ}
            Il existe \( J\subset I\) tel que \( \{ e_i \}_{i\in J}\) est une base. Autrement dit : de toute partie génératrice nous pouvons extraire une base.
        \item       \label{ITEMooCJQGooXwjsfm}
            Soit \( \{ f_1,\ldots, f_l \}\) une partie libre. Alors nous pouvons la compléter en utilisant des éléments \( e_i\). C'est-à-dire qu'il existe \( J\subset I\) tel que \( \{ f_k \}\cup\{ e_i \}_{i\in J}\) soit une base.
    \end{enumerate}
\end{theorem}

\begin{proposition}     \label{PROPooVEVCooHkrldw}
    Si \( E\) est un espace vectoriel de dimension finie \( n\), alors 
    \begin{enumerate}
        \item       \label{ITEMooZNLDooBISkJyBS}
            toute partie contenant \( n+1\) éléments est liée.
        \item       \label{ITEMooSGGCooOUsuBs}
            toute partie libre contenant \( n\) éléments est une base,
        \item
            toute partie génératrice contenant \( n\) éléments est une base.
    \end{enumerate}
\end{proposition}

\begin{proof}
    Soit une partie \( M\) contenant \( n+1\) éléments. L'espace \( E\) possède une partie génératrice contenant \( n\) éléments (n'importe quelle base). Donc \( M\) est liée par le lemme \ref{LemytHnlD}.

    Une partie libre contenant \( n\) éléments peut être étendue en une base; si ladite extension est non triviale (c'est-à-dire qu'on ajoute vraiment au moins un élément) une telle base contiendra une partie de \( n+1\) éléments qui serait liée par le lemme~\ref{LemytHnlD}.

    Pour l'autre assertion, soit une partie génératrice \( \{ v_i \}_{i\in I}\) où \( I\) contient \( n\) éléments. Par le théorème \ref{ThoMGQZooIgrXjy}\ref{ITEMooCJQGooXwjsfm} il existe \( J\subset I\) tel que \( \{ v_j \}_{j\in J}\) soit une base. Si l'inclusion \( J\subset I\) est stricte, alors la base \( \{ v_j \}_{j\in J}\) contiendrait moins de\( n\) éléments, ce qui serait en contradiction avec le théorème \ref{ThonmnWKs}\ref{ITEMooFBUAooSSZxgx}.
\end{proof}

\begin{definition}\label{DefCodimension}
Soit \( F\) un sous-espace vectoriel de l'espace vectoriel \( E\). La \defe{codimension}{codimension} de \( F\) dans \( E\) est
\begin{equation}
    \codim_E(F)=\dim(E/F).
\end{equation}
\end{definition}

\begin{probleme}
Voir que $E/F$ a une structure vectorielle, expliciter sa dimension en fonction de celles de $E$ et $F$.
\end{probleme}

%---------------------------------------------------------------------------------------------------------------------------
\subsection{Et en dimension infinie}
%---------------------------------------------------------------------------------------------------------------------------

Dans ZFC, en dimension infinie, il existe aussi une base pour tout espace vectoriel ainsi qu'un théorème de la base incomplète. Nous ne parlerons pas de ce qu'il se passe lorsque nous ne considérons que ZF\footnote{Si vous ne savez pas ce que signifient les sigles «ZF» et «ZFC» vous ne devriez pas être en train de lire ceci, et encore moins penser à le resservir à un jury d'agrégation.}.

\begin{lemma}[\cite{ooXEFKooHikcdE}]        \label{LEMooSSRXooIyfgNz}
    Soient un \( \eK\)-espace vectoriel \( E\) et un sous-espace vectoriel \( V\) de \( E\). Soient encore deux sous-espaces vectoriels \( W_1\) et \( W_2\) tels que
    \begin{enumerate}
        \item
            \( V\cap W_1=\{ 0 \}\);
        \item
            \( V+W_2=E\).
    \end{enumerate}
    Alors il existe un supplémentaire \( W\) de \( V\) tel que \( W_1\subset W\subset W_2\).
\end{lemma}

Juste une remarque : dans le Frido le symbole «\( \subset\)» ne signifie pas une inclusion stricte.

\begin{proof}
    Nous divisons en petits morceaux.
    \begin{subproof}
        \item[Un gros ensemble]
            Soit \( \mA\) l'ensemble des sous-espaces vectoriels \( S\) de \( E\) tels que \( W_1\subset S\subset W_2\) et \( S\cap V=\{ 0 \}\). Vu que \( W_1\subset \mA\), cet ensemble n'est pas vide. De plus \( \mA\) est partiellement ordonné pour l'inclusion.
        \item[\( \mA\) est inductif]
            Nous prouvons maintenant que \( \mA\) est inductif\footnote{Définition~\ref{DefGHDfyyz}.}. Pour cela, soit une partie \( \mA'\) totalement ordonnée et \( U=\bigcup_{A\in \mA'}A\).

            Alors, la partie \( U\) est un sous-espace vectoriel de \( E\). En effet si \( x,y\in U\), alors il existe \( A_1,A_2\in\mA'\) tels que \( x\in A_1\) et \( y\in A_2\). Vu que \( \mA'\) est totalement ordonné, l'un des ensembles parmi \( A_1\) et \( A_2\) est inclus dans l'autre. Sans perdre de généralité, disons \( A_1\subset A_2\). Alors les opérations s'effectuent dans \( A_2 \) : nous avons \( x,y\in A_2\), et donc \( \lambda x\in A_2\subset U\) ainsi que \( x+y\in A_2\subset U\).

            De plus, \( U \) contient \( W_1 \), et est contenu dans \( W_2\). Ainsi, \( U\in \mA\) et majore \( \mA'\) pour l'inclusion. En bref, \( \mA\) est bien inductif.
        \item[Utilisation de Zorn]

            Le lemme de Zorn~\ref{LemUEGjJBc} nous donne alors un maximum \( W\) de \( \mA\). Ce maximum vérifie
            \begin{enumerate}
                \item
                    \( W\cap V=\{ 0 \}\),
                \item
                    \( W_1\subset W\subset W_2\),
                \item
                    pour tout \( W'\in\mA\), nous avons \( W'\subset W\) parce que \( W\) est maximum.
            \end{enumerate}
        \item[Supplémentaire]
            Montrons que ce \( W\) est un supplémentaire de \( V\). Soit \( x\in E\). Le but est de trouver une décomposition de \( x\) en somme d'un élément de \( W\) et un de \( V\). Vu que \( V+W_2=E\) nous avons \( v\in V\) et \( w_2\in W_2\) tels que 
            \begin{equation}
                x=v+w_2. 
            \end{equation}
            Si \( w_2\in W\) alors c'est fait. Sinon \ldots

            Soit \( X=\Span\{ W,w_2 \}\). Vu que \( X\) contient strictement \( W\) et que \( W\) est maximum dans \( \mA\), la partie \( X\) n'est pas un élément de \( \mA\). Vu que \( X\) est un sous-espace vectoriel de \( E\) tel que \( W_1\subset X\subset W_2\), la seule possibilité pour que \( X\) ne soit pas dans \( \mA\) est que \( X\cap V\neq \{ 0 \}\). Soit donc \( y\neq 0\) dans \( X\cap V\). Par définition de \( X\),
            \begin{equation}\label{EqDecompo55:296}
                y=w'+\lambda w_2
            \end{equation}
            pour \( w'\in W\), \( w_2\in W_2\) et \( \lambda\in \eK\). Nous avons \( \lambda\neq 0\), sinon nous aurions \( y\in W\cap V \) et donc \(y = 0 \) puisque \( W \) est dans \( \mA \). La décomposition \eqref{EqDecompo55:296} permet alors d'écrire \( w_2=(y-w')/\lambda\) et finalement
            \begin{equation}
                x=v+\frac{1}{ \lambda }(y-w')=\underbrace{v+\frac{1}{ \lambda }y}_{\in V}-\underbrace{\frac{1}{ \lambda }w'}_{\in W}.
            \end{equation}
            La somme d'espaces vectoriels \( E=V+W\) est donc établie.
    \end{subproof}
\end{proof}

\begin{corollary}
    Tout sous-espace vectoriel d'un espace vectoriel possède un supplémentaire.
\end{corollary}

\begin{proof}
    Soit un espace vectoriel \( E\) ainsi qu'un sous-espace vectoriel \( V\). Si \( V=E\) nous sommes ok. Sinon nous considérons \( v\in E\setminus V\) et nous posons \( W_1=\eK v\) et \( W_2=E\).

    Vu que \( V\) et \( W_1\) sont des espaces vectoriels, nous avons \( V\cap W_1=\{ 0 \}\), et vu que \( W_2=E\) nous avons \( V+W_2=E\). Le lemme~\ref{LEMooSSRXooIyfgNz} nous donne alors un supplémentaire de \( V\).
\end{proof}

\begin{proposition}[Base incomplète]        \label{PROPooHDCEooMhDjPi}
    Tout espace vectoriel (non réduit à \( \{ 0 \}\)) possède une base.
\end{proposition}

\begin{proof}
    Soit \( \mA\) l'ensemble des familles libres de \( E\). Il n'est pas vide parce que \( \{ v \}\) en est une dès que \( v\) est non nul dans \( E\). Rapidement :
    \begin{itemize}
        \item l'ensemble \( \mA\) est ordonné pour l'inclusion,
        \item si \( \mA'\) est une partie totalement ordonnée, l'union est un majorant,
        \item donc \( \mA\) est inductif,
        \item soit un maximum \( F\) de \( \mA\).
    \end{itemize}
    La partie \( F\) est libre parce qu'elle est dans \( \mA\). Elle est génératrice parce que si \( v\) n'est pas dans \( \Span(F)\) alors la partie \( F\cup\{ v \}\) est encore libre, et majore strictement $F$ pour l'inclusion, ce qui n'est pas possible.

    Donc \( F\) est une base de \( E\).
\end{proof}

\begin{theorem}[Base incomplète, dimension quelconque]      \label{THOooOQLQooHqEeDK}
    Soit une partie \( \{ e_i \}_{i\in I}\) génératrice de l'espace vectoriel \( E\) (ici, \( I\) est un ensemble quelconque\footnote{Un cas d'utilisation intéressant est de poser \( I=E\) et \( e_i=i\). Pensez-y.}). Soit \( I_0\in I\) tel que \( \{ e_i \}_{i\in I_0}\) soit libre.

    Alors il existe \( I_1\) tel que \( I_0\subset I_1\subset I\) tel que \( \{ e_i \}_{i\in I_1}\) soit une base de \( E\).
\end{theorem}

Note : une telle partie \( I_0\) existe en prenant un singleton. Mais l'existence n'est pas le sujet ici.

\begin{proof}
    Soit \( \mA\) l'ensemble des parties \( J\) de \( I\) telles que \( I_0\subset J\subset I\) et telles que \( \{ e_i \}_{i\in J}\) soit libre.

    Encore une fois, \( \mA\) est inductif pour l'ordre partiel donné par l'inclusion. Soit \( J\) un maximum. Vu que \( J\in\mA\), la partie \( \{ e_i \}_{i\in J}\) est libre. Mais elle est également génératrice parce que si \( e_k\) n'est pas dedans, \( J\) ne serait pas maximum, étant majorée par \( J\cup\{ k \}\).

    Donc \( \{ e_i \}_{i\in J}\) engendre tous les \( e_i\) avec \( i\in I\) et donc tous les éléments de \( E\).
\end{proof}

%--------------------------------------------------------------------------------------------------------------------------- 
\subsection{Espace librement engendré}
%---------------------------------------------------------------------------------------------------------------------------

\begin{definition}[\cite{ooGNYOooGZKGba}]       \label{DEFooCPNIooNxsYMY}
    Soient un ensemble \( S\) et un corps \(\eK \). L'espace vectoriel \defe{librement engendré}{librement engendré} sur \( S\), noté \( F_{\eK}(S)\) est l'ensemble des applications \( S\to \eK\) qui sont non-nulles en un nombre fini de points de \( S\).

    Autrement dit, \( \sigma\colon S\to \eK\) est dans \( F_{\eK}(S) \) si \( \{ x\in S\tq \delta(x)\neq 0 \}\) est fini\footnote{Parce que nous l'aimons bien, nous ne résistons pas à faire un renvoi vers la définition \ref{DefEOZLooUMCzZR}.}.
\end{definition}

Le lemme suivant donne tout son sens à l'expression «librement» engendré. Il dit que \( F(S)\) possède une base indexée par \( S\) lui-même.
\begin{lemma}       \label{LEMooLOPAooUNQVku}
    L'ensemble des applications \( \delta_s\) données par
    \begin{equation}
        \begin{aligned}
            \delta_s\colon S&\to \eK \\
            t&\mapsto \begin{cases}
                1    &   \text{si } t=s\\
                0    &    \text{sinon }
            \end{cases}
        \end{aligned}
    \end{equation}
    avec \( s\in S\) forment une base\footnote{Définition \ref{DEFooNGDSooEDAwTh}.} de \( F(S)\).
\end{lemma}

\begin{proof}
    Pour prouver que les \( \delta_s\) sont générateurs, nous considérons \( g\colon S\to \eK\) non nul sur la partie finie \( \{ s_i \}_{i\in I}\) de \( S\). Alors nous avons
    \begin{equation}
        g=\sum_{i\in I}g(s_i)\delta_{s_i}.
    \end{equation}
    
    Pour prouver que les \( \delta_s\) forment une partie libre, nous supposons avoir \( \lambda_i\in \eK\) tels que
    \begin{equation}
        g=\sum_{i\in I}\lambda_i\delta_{s_i}=0
    \end{equation}
    Soit \( j\in I\). Nous avons
    \begin{equation}
        0=f(s_j)=\sum_{i\in I}\lambda_i \underbrace{\delta_{s_i}(s_j)}_{=\delta_{ij}}=\lambda_j.
    \end{equation}
    Donc les coefficients \( \lambda_i\) sont tous nuls, et nous avons prouvé que la partie est libre.
\end{proof}

Il est parfois pratique d'écrire les éléments de \( F(S)\) comme sommes «formelles» d'éléments de \( S\). Cela va encore lorsque \( S\) est un ensemble n'ayant aucune somme bien définie. 

Mais attention : si \( S=\eR\), l'élément \( 4+7\) de \( F(\eR)\) n'est pas \( 11\). L'élément \( 11\) de \( F(\eR)\) est un élément complètement différent. Bref, il n'est pas judicieux d'écrire les éléments de \( F(S)\) comme des combinaisons linéaires d'éléments de \( S\). Pour \( x\in S\) il vaut mieux écrire explicitement \( \delta_x\) que \( x\). La somme \( \delta_x+\delta_y\) est parfaitement bien définie dans l'esemble des applications de \( S\) vers \( \eK\).

%+++++++++++++++++++++++++++++++++++++++++++++++++++++++++++++++++++++++++++++++++++++++++++++++++++++++++++++++++++++++++++++
\section{Applications linéaires}
%+++++++++++++++++++++++++++++++++++++++++++++++++++++++++++++++++++++++++++++++++++++++++++++++++++++++++++++++++++++++++++++

%---------------------------------------------------------------------------------------------------------------------------
\subsection{Définition}
%---------------------------------------------------------------------------------------------------------------------------

\begin{definition}      \label{DEFooULVAooXJuRmr}
    Soient des espaces vectoriels \( E \) et \( F\) sur le corps \( \eK\). Une application \( T\colon E\to F\) est dite \defe{linéaire}{linéaire!application} si
    \begin{itemize}
        \item $T(x+y)=T(x)+T(y)$ pour tout $x$ et $y$ dans \( E\),
        \item $T(\lambda x)=\lambda T(x)$ pour tout $\lambda$ dans $\eK$ et \( x\) dans \( E\).
    \end{itemize}
\end{definition}
Si vous avez bien suivi, les égalités dans la définition~\ref{DEFooULVAooXJuRmr} sont des égalités dans \( F\).

\begin{lemmaDef} \label{DefDQRooVGbzSm}
    L'ensemble de toutes les applications linéaires de \( E\) vers \( F\) est noté \( \aL(E,F)\)\nomenclature{$\aL(E,F)$}{Ensemble des applications linéaires de $E$ dans $F$} et devient un espace vectoriel sur \( \eK\) avec les définitions suivantes :
    \begin{enumerate}
        \item
            \( (T_1+T_2)(x)=T_1(x)+T_2(x)\),
        \item
            \( (\lambda T)(x)=\lambda T(x)\).
    \end{enumerate}
\end{lemmaDef}

\begin{example}
Pour tout $b$ dans $\eR$ la fonction $T_b(x)= bx$ est une application linéaire de $\eR$ dans $\eR$. En effet,
\begin{itemize}
\item  $T_b(x+y)= b(x+y)= bx + by = T_b(x)+T_b(y)$,
\item $T_b(ax)=b(ax)= abx = a T_b(x)$.
\end{itemize}
De la même façon on peut montrer que la fonction $T_{\lambda}$ définie par $T_{\lambda}(x)=bx$ est un application linéaire de $\eR^m$ dans $\eR^m$ pour tout $\lambda$ dans $\eR$ et $m$ dans $\eN$.
\end{example}

\begin{example}\label{ex_affine}
	Soit $m=n$. On fixe $\lambda$ dans $\eR$ et $v$ dans $\eR^m$. L'application $U_{\lambda}$ de $\eR^m$ dans $\eR^m$ définie par $U_{\lambda}(x)=\lambda x+v$ n'est pas une application linéaire lorsque \( v \neq 0 \), parce que si \( a \) est un réel différent de \(0 \) et \( 1 \), alors \( av \neq v \), d'où
\[
U_{\lambda}(ax)=\lambda(ax)+v\neq a(\lambda x+v) =a U_{\lambda}(x).
\]
\end{example}

\begin{example}\label{exampleT_A}
	Soit $A$ une matrice fixée de $\mathcal{M}_{n\times m}$\nomenclature{$\mathcal{M}_{n\times m}$}{l'ensemble des matrices $n\times m$}. La fonction $T_A\colon \eR^m\to \eR^n$ définie par $T_A(x)=Ax$ est une application linéaire. En effet,
    \begin{itemize}
        \item  $T_A(x+y)= A(x+y)= Ax + Ay = T_A(x)+T_A(y)$,
        \item $T_A(ax)=A(ax)= a(Ax) = a T_A(x)$.
    \end{itemize}
\end{example}

On peut observer que, si on identifie $\mathcal{M}_{1\times 1}$ et $\eR$, on obtient le premier exemple comme cas particulier.

\begin{definition}[Quelques ensembles d'applications linéaires]      \label{DEFooOAOGooKuJSup}
    Soient \( E\) et \( F\) des espaces vectoriels.
    \begin{itemize}
        \item
            L'ensemble des applications linéaires de \( E\) vers \( F\) est noté $\aL(E,F)$, comme déjà dit en \ref{DefDQRooVGbzSm}.
        \item Une application linéaire \( E\to E\) est un \defe{endomorphisme}{endomorphisme} de \( E\). L'ensemble des endomorphismes de \( E\) est noté \( \End(E)\)\nomenclature[B]{$\End(E)$}{les endomorphismes de \( E\)}.
        \item Un endomorphisme bijectif est un \defe{automorphisme}{automorphisme!d'espace vectoriel}. L'ensemble des automorphismes de \( E\) est noté \( \Aut(E)\)\nomenclature[B]{$\Aut(E)$}{automorphisme de l'espace vectoriel \( E\)}.
        \item
            Une application linéaire bijective \( E\to F\) est un \defe{isomorphisme}{isomorphisme!espaces vectoriels} d'espace vectoriel. L'ensemble des isomorphismes \( E\to F\) est noté\footnote{Le fait d'utiliser une notation similaire à celle des matrices inversibles n'est pas anodine: le lecteur en est sans doute conscient.} \( \GL(E,F)\).
    \end{itemize}
\end{definition}

\begin{remark}
    Les ensembles définis en~\ref{DEFooOAOGooKuJSup} concernent la structure d'espace vectoriel seulement. Lorsque nous verrons la notion d'espace vectoriel normé, nous demanderons de plus la continuité, laquelle n'est pas automatique en dimension infinie. Voir aussi les définitions~\ref{DEFooTLQUooJvknvi}.
\end{remark}

\begin{definition}
    Si \( E\) est un espace vectoriel, si \( X\) est un espace vectoriel, et si \( f\colon X\to E\) est une application, le \defe{noyau}{noyau!vers un espace vectoriel} de \( f\) est le noyau de \( f\) lorsque \( E\) est vu comme un groupe pour l'addition\footnote{Définition \ref{DEFooWBIYooGNRYOp}.}, c'est-à-dire la partie
    \begin{equation}
        \ker(f)=\{ x\in X\tq f(x)=0 \}.
    \end{equation}
\end{definition}

\begin{proposition}     \label{PROPooRLLPooKYzsJp}
    Le noyau d'une application linéaire est un sous-espace vectoriel.
\end{proposition}

\begin{proof}
    Soit une application linéaire \( f\colon E\to F\). Si \( x,y\in \ker(f)\) et si \( \lambda\in \eK\) alors
    \begin{equation}
        f(x+y)=f(x)+f(y)=0+0=0,
    \end{equation}
    donc \( x+y\in \ker(f)\) et
    \begin{equation}
        f(\lambda x)=\lambda f(x)=0,
    \end{equation}
    donc \( \lambda x\in \ker(f)\).
\end{proof}

\begin{proposition}
    Si \( E\) et \( F\) sont des espaces vectoriels de dimension \( n\) et si \( \{ e_i \}_{i=1,\ldots, n}\) et \( \{ f_i \}_{i=1,\ldots, n}\) sont des bases respectivement de \( E\) et \( F\), alors il existe une unique application linéaire \( T\colon E\to F\) telle que \( T(e_i)=f_i\) pour tout \( i\).
\end{proposition}

\begin{proof}
    En deux parties.\begin{subproof}
        \item[Existence]
            Soit \( v\in E\). Vu que \( \{ e_i \}\) est une base, il se décompose de façon unique en \( v=\sum_iv_ie_i\). Alors définir
            \begin{equation}
                T(v)=\sum_iv_if_i
            \end{equation}
            est une bonne définition et satisfait aux exigences.
        \item[Unicité]
            Soient \( T\) et \( U\) satisfaisant aux exigences. Alors pour tout \( i\) nous avons \( T(e_i)=U(e_i)\). Si \( v\in E\) s'écrit de la forme \( v=\sum_iv_ie_i\) alors la linéarité impose \( T(v)=\sum_iv_iT(e_i)=\sum_iv_iU(e_i)=U(v)\). Donc \( T = U\).
    \end{subproof}
\end{proof}

\begin{lemma}[\cite{MonCerveau}]       \label{LEMooJXFIooKDzRWR}
    Soient des espaces vectoriels \( V\) et \( W\) de dimension finie. Soient des bases \( \{e_i\}\) de \( V\) et \( \{f_{\alpha}\}\) de \( W\). Nous posons
    \begin{equation}
        \begin{aligned}
            \varphi_{i\alpha}\colon V&\to W \\
            v&\mapsto v_if_{\alpha} 
        \end{aligned}
    \end{equation}
    où \( v_i\) est défini par la décomposition (unique) \( v=\sum_iv_ie_i\). 

    Alors :
    \begin{enumerate}
        \item
            La partie \( \{\varphi_{i\alpha}\} \) est une base de \( \aL(V,W)\).
        \item       \label{ITEMooPMLWooNbTyJI}
            Au niveau des dimensions : \( \dim\big( \aL(V,W) \big)=\dim(V)\dim(W)\).
    \end{enumerate}
\end{lemma}

\begin{proof}
    Il faut prouver que \( \{\varphi_{i\alpha}\}\) est libre et générateur.

    \begin{subproof}
        \item[Générateur]
            Soit une application linéaire \( b\colon V\to W\). En décomposant \( b(v)\) dans la base \( \{f_{\alpha}\}\), nous définissons \( b_{\alpha}\colon V\to \eK\) par
            \begin{equation}
                b(v)=\sum_{\alpha}b_{\alpha}(v)f_{\alpha}.
            \end{equation}
            Nous posons \( b_{\alpha i}=b_{\alpha}(e_i)\). Ainsi,
            \begin{equation}
                b(v)=\sum_{\alpha}v_ib_{\alpha i}f_{\alpha}=\sum_{\alpha i}b_{\alpha i}\varphi_{i\alpha}(v).
            \end{equation}
            Donc \( b\) peut être écrit comme combinaison linéaire des \( \varphi_{i\alpha}\).

        \item[Libre]
            Supposons que \( \sum_{i\alpha}a_{i\alpha}\varphi_{i\alpha}=0\) pour certains coefficients \( a_{i\alpha}\in \eK\). Nous avons, pour tout \( v\in V\) :
            \begin{equation}
                0=\sum_{i\alpha}a_{i\alpha}\varphi_{i\alpha}(v)=\sum_{i\alpha}a_{i\alpha}v_if_{\alpha},
            \end{equation}
            mais comme les \( f_{\alpha}\) forment une base, chaque terme de la somme sur \( \alpha\) est nul :
            \begin{equation}
                \sum_ia_{i\alpha}v_i=0.
            \end{equation}
            Et comme cela est valable pour tout \( v\) et donc pour tout choix de \( v_i\), nous avons \( a_{i\alpha}=0\) pour tout \( i\) et pour tout \( \alpha\).
    \end{subproof}
    La formule de dimension est simplement la cardinalité de la base trouvée; c'est la définition \ref{DEFooWRLKooArTpgh}.
\end{proof}

%--------------------------------------------------------------------------------------------------------------------------- 
\subsection{Linéarité et bases}
%---------------------------------------------------------------------------------------------------------------------------

\begin{proposition}[\cite{ooZLSSooMYdbEz}]
    Soient deux espaces vectoriels \( E\) et \( F\). Une application linéaire\footnote{Définition \ref{DEFooULVAooXJuRmr}.} \( f\colon E\to F\) est injective si et seulement si \( \ker\{ f \}=\{ 0 \}\).
\end{proposition}

\begin{proof}
    Nous supposons que \( f\) est injective. Si \( x\in\ker(f)\), alors \( f(x)=0\). Or \( f\) est linéaire, donc \( f(0)=0\). Nous avons donc \( f(x)=f(0)\) et donc \( x=0\) parce que \( f\) est injective.

    Dans l'autre sens, soient \( x,y\) tels que \( f(x)=f(y)\). Par linéarité de \( f\) nous avons \( f(x-y)=0\), et donc \( x-y=0\) parce que \( \ker(f)=\{0\}\). Donc \( x=y\) et \( f\) est injective.
\end{proof}

\begin{proposition}[\cite{ooZLSSooMYdbEz}]      \label{PROPooZFKZooBGLSex}
    Soit \( f\in \aL(E,F)\) où \( E\) et \( F\) sont deux espaces vectoriels.
    \begin{enumerate}
        \item   \label{ITEMooPPMEooIaZqtm}
            Si \( f\) est injective et si \( \{v_i\}_{i\in I}\) est libre, alors \( \{f(v_i)\}_{i\in I}\) est libre.
        \item   \label{ITEMooOZSPooQBrDGi}
            Si \( f\) est surjective et si \( \{v_i\}_{i\in I}\) est génératrice, alors \( \{f(v_i)\}_{i\in I}\) est génératrice.
        \item   \label{ITEMooOIEYooIfdFnv}
            Si \( f\) est une bijection, alors l'image d'une base par \( f\) est une base.
    \end{enumerate}
\end{proposition}

\begin{proof}
    En trois parties.
    \begin{subproof}
        \item[\ref{ITEMooPPMEooIaZqtm}]
            Nous devons montrer que \( \{f(v_j)\}_{j\in J}\) est libre pour tout \( J\) fini dans \( I\). Soit donc une partie finie \( J\in I\) et des scalaires\footnote{Des éléments du corps de base \( \eK\).} tels que \( \sum_{j\in J}\lambda_jf(v_j)=0\). La linéarité de \( f\) donne\footnote{Voir les propriétés de la définition \ref{DEFooULVAooXJuRmr}.}
            \begin{equation}
                f\big( \sum_{i\in J}\lambda_jv_j \big)=0.
            \end{equation}
            Par injectivité de \( f\) nous avons alors \( \sum_j\lambda_jv_j=0\). Vu que les \( v_j\) eux-même forment une partie libre, nous avons \( \lambda_j=0\) pour tout \( j\in J\).
        \item[\ref{ITEMooOZSPooQBrDGi}]
            Soit \( y\in F\). Vu que \( f\) est surjective, il existe \( x\in E\) tel que \( f(x)=y\). Étant donné que \( \{v\i\}_{i\in I}\) est générateur, il existe une partie finie \( J\subset I\) et des scalaires \( \lambda_j\in \eK\) tels que
            \begin{equation}
                x=\sum_{j\in J}\lambda_jv_j.
            \end{equation}
            En appliquant \( f\) aux deux côtés, et en tenant compte de la linéarité de \( f\),
            \begin{equation}
                y=f(x)=\sum_{j\in J}\lambda_jf(v_j),
            \end{equation}
            ce qui prouve que \( y\) est une combinaison linéaire des \( f(v_j)\).
        \item[\ref{ITEMooOIEYooIfdFnv}]
            Une base est à la fois libre et génératrice et une bijection est à la fois injective et surjective. Les deux premiers points permettent de conclure.
    \end{subproof}
\end{proof}

\begin{corollary}[\cite{MonCerveau}]        \label{CORooXIPKooWThOsr}
    Si \( E\) et \( F\) sont des espaces vectoriels isomorphes de dimensions finies. Alors leurs dimensions sont égales.
\end{corollary}

\begin{proof}
    Vu que \( E\) et \( F\) sont isomorphes, il existe une bijection \( f\colon E\to F\). Par la proposition \ref{PROPooZFKZooBGLSex}\ref{ITEMooOIEYooIfdFnv}, l'image d'une base de \( E\) est une base de \( F\). Donc les espaces \( E\) et \( F\) ont des bases contenant le même nombre d'éléments.
\end{proof}

%---------------------------------------------------------------------------------------------------------------------------
\subsection{Rang}
%---------------------------------------------------------------------------------------------------------------------------

La proposition~\ref{DefALUAooSPcmyK} et le théorème~\ref{ThoGkkffA} sont valables également en dimension infinie; ce sera une des rares incursions en dimension infinie de ce chapitre.
\begin{propositionDef}\label{DefALUAooSPcmyK}
    L'image d'une application linéaire est un espace vectoriel. La dimension de cet espace est le \defe{rang}{rang} de ladite application linéaire.
\end{propositionDef}

\begin{proof}
    Soit une application linéaire \( f\colon E\to F\). Nous considérons \( v,w\) dans l'image de \( f\) ainsi que \( \lambda\) dans le corps de base commun à \( E\) et \( F\).

    Soient \( v_0\in E\) et \( w_0\in E\) tels que \( v=f(v_0)\) et \( w=f(w_0)\). Alors \( v+w=f(v_0+w_0)\) et \( \lambda v=f(\lambda v_0)\). Donc l'image est bien un espace vectoriel.
\end{proof}

\begin{theorem}[Théorème du rang]       \label{ThoGkkffA}
    Soient \( E\) et \( F\) deux espaces vectoriels (de dimensions finies ou non) et soit \( f\colon E\to F\) une application linéaire. 
    
   Si \( (u_s)_{s\in S}\) est une base de \( \ker(f)\) et si \( \big( f(v_t) \big)_{t\in T}\) est une base de \( \Image(f)\) alors 
   \begin{equation}
   (u_s)_{s\in s}\cup (v_t)_{t\in T}
   \end{equation}
   est une base de \( E\).
    
   En dimension finie, nous avons en plus la formule suivante :
   \begin{equation}     \label{EQooUEOQooLySRiE}
       \rang(f)+\dim\ker f=\dim E,
   \end{equation}
   c'est-à-dire que le rang\footnote{Définition~\ref{DefALUAooSPcmyK}.} de \( f\) est égal à la codimension\footnote{Définition~\ref{DefCodimension}.} du noyau.
\end{theorem}
\index{théorème!du rang}

\begin{proof}
    Nous devons montrer que
    \begin{equation}
          (u_s)_{s\in S}\cup (v_t)_{t\in T}
    \end{equation}
    est libre et générateur.

    Soit \( x\in E\). Nous définissons les nombres \( x_t\) par la décomposition de \( f(x)\) dans la base \( \big( f(v_t) \big)\) :
    \begin{equation}
        f(x)=\sum_{t\in T}x_tf(v_t).
    \end{equation}
    Ensuite le vecteur \( x=\sum_tx_tv_t\) est dans le noyau de \( f\), par conséquent nous le décomposons dans la base \( (u_s)\) :
    \begin{equation}
        x-\sum_tx_tv_t=\sum_{s\in S} x_su_s.
    \end{equation}
    Par conséquent
    \begin{equation}
        x=\sum_sx_su_s+\sum_tx_tv_t.
    \end{equation}

    En ce qui concerne la liberté nous écrivons
    \begin{equation}
        \sum_tx_tv_t+\sum_sx_su_s=0.
    \end{equation}
    En appliquant \( f\) nous trouvons que
    \begin{equation}
        \sum_tx_tf(v_t)=0
    \end{equation}
    et donc que les \( x_t\) doivent être nuls. Nous restons avec \( \sum_sx_su_s=0\) qui à son tour implique que \( x_s=0\).
\end{proof}
Un exemple d'utilisation de ce théorème en dimension infinie sera donné dans le cadre du théorème de Fréchet-Riesz, théorème~\ref{ThoQgTovL}.
\ifbool{isGiulietta}{Il existe une généralisation du théorème du rang pour les variétés différentiables en le théorème \ref{THOooSWKVooTJQsXc}.}{}

\begin{proposition}[\cite{ooDSTAooKgSyCN}]      \label{PROPooQCIXooHIyPPq}
    Soit \( E\), un espace vectoriel de dimension finie sur le corps $\eK$. Soient \( V\) et \( W\) des sous-espaces vectoriels de \( E\). Alors
    \begin{equation}
        \dim(V+W)=\dim(V)+\dim(W)-\dim(V\cap W).
    \end{equation}
\end{proposition}

\begin{proof}
    Nous considérons l'application 
    \begin{equation}
        \begin{aligned}
            \varphi\colon V\times W&\to E \\
            (x,y)&\mapsto x+y. 
        \end{aligned}
    \end{equation}
    C'est une application linéaire dont l'image est \( V+W\). Nous avons donc, pour commencer
    \begin{equation}
        \dim(V+W)=\dim\big( \Image(\varphi) \big).
    \end{equation}
    Nous appliquons à présent le théorème du rang \ref{ThoGkkffA} à l'application \( \varphi\) :
    \begin{subequations}
        \begin{align}
            \dim(V+W)&=\dim\big( \Image(\varphi) \big)\\
            &=\dim(V\times W)- \dim\big( \ker(\varphi) \big)\\
            &=\dim(V)+\dim(W)-\dim\big( \ker(\varphi) \big).
        \end{align}
    \end{subequations}
    Nous devons maintenant étudier \( \ker(\varphi)\). D'abord, \( (v,w)\in V\times W\) appartient à \( \ker(\varphi)\) si et seulement si \( v+w=0\). Nous avons donc
    \begin{equation}
        \ker(\varphi)=\{ (x,-x)\tq x\in V\cap W \}.
    \end{equation}
    Nous montrons à partir de cela que \( \dim\big( \ker(\varphi) \big)=\dim(V\cap W)\) en montrant que l'application
    \begin{equation}
        \begin{aligned}
            \psi\colon V\cap W&\to \ker(\varphi) \\
            x&\mapsto (x,-x) 
        \end{aligned}
    \end{equation}
    est un isomorphisme d'espaces vectoriels. D'abord \( \psi\) est injective parce que si \( \psi(x)=\psi(y)\), alors \( (x,-x)=(y,-x)\) et donc \( x=y\). Ensuite, \( \psi\) est surjective parce qu'un élément générique de \( \ker(\varphi)\) est \( (x,-x)=\psi(x)\) avec \( x\in V\cap W\). L'application \( \psi\) étant un isomorphisme d'espaces vectoriels, nous avons bien \( \dim\big( \ker(\varphi) \big)=\dim(V\cap W)\).
\end{proof}

\begin{corollary}       \label{CORooCCXHooALmxKk}
    Soient deux espaces vectoriels \( E\) et \( F\) de même dimensions finies\footnote{Les deux mots sont importants : les dimensions doivent être égales et finies.}. Pour une application linéaire \( f\colon E\to F\), les trois conditions suivantes sont équivalentes :
    \begin{enumerate}
        \item
            \( f\) est injective;
        \item
            \( f\) est surjective;
        \item
            \( f\) est bijective.
    \end{enumerate}
\end{corollary}

\begin{proof}
    Si un endomorphisme \( f\colon E\to E\) est surjectif, alors \( \rang(f)=\dim(E)\), ce qui donne, par le théorème du rang~\ref{ThoGkkffA}, \( \dim\big( \ker(f) \big)=0\), c'est-à-dire que \( f\) est injectif.

    De la même façon, si \( f\) est injective, alors \( \dim\big( \ker(f) \big)=0\), ce qui donne \( \rang(f)=\dim(E)\) ou encore que \( f\) est surjective.
\end{proof}

\begin{example}
    Le corolaire \ref{CORooCCXHooALmxKk} n'est pas correct en dimension infinie. Par exemple en prenant \( f(e_1)=f(e_2)=e_1\) et ensuite \( f(e_k)=e_{k-1}\) pour tout \( k\geq 2\). Cette application est surjective mais pas injective.
\end{example}

Une conséquence du théorème du rang est que les endomorphismes ont un inverse à gauche et à droite égaux (lorsqu'ils existent). En résumé, ce que le corolaire \ref{CORooNFJLooJtzFwN} dit est que si \( AB=\mtu\), alors \( BA=\mtu\).
\begin{corollary}           \label{CORooNFJLooJtzFwN}
    Soit un endomorphisme \( f\) d'un espace vectoriel de dimension finie. Si \( f\) admet un inverse à gauche, alors
    \begin{enumerate}
        \item
            \( f\) est bijective,
        \item
            \( f\) admet également un inverse à droite,
        \item
            les inverses à gauche et à droite sont égaux.
    \end{enumerate}
    Tout cela tient également en remplaçant «gauche» par «droite».
\end{corollary}

\begin{proof}
    Soit \( g\), un inverse à gauche de \( f\) : \( gf=\id\). Cela implique que \( f\) est injective et que \( g\) est surjective, et donc qu'elles sont toutes deux bijectives par le corolaire~\ref{CORooCCXHooALmxKk}. Vu que \( f\) est bijective, elle admet également un inverse à droite, soit \( h\). Nous avons : \( gf=\id\) et \( fh=\id\).

    Alors \( gfh=h\) parce que \( gf=\id\), mais également \( gfh=g\) parce que \( fh=\id\). Donc \( g=h\).\footnote{C'est le même argument que celui employé pour la preuve du lemme~\ref{LEMooECDMooCkWxXf}~\ref{ITEMooOIWTooYqmMPP}, à ceci près que nous devions montrer l'existence de l'inverse à droite.}
\end{proof}
C'est ce corolaire qui nous permet d'écrire \( f^{-1}\) sans plus de précisions dès que \( f\) est une bijection.

\begin{example}[Pas en dimension infinie]
    Tout cela ne fonctionne pas en dimension infinie. Par exemple avec une base \( \{ e_k \}_{k\in \eN}\) nous pouvons considérer l'opérateur
    \begin{equation}
        f(e_k)=e_{k+1}.
    \end{equation}
    Il est injectif, mais pas surjectif. Si on pose
    \begin{equation}
        g(e_k)=\begin{cases}
            e_{k-1}    &   \text{si } k\geq 1\\
            0    &    \text{si } k=0
        \end{cases}
    \end{equation}
    alors nous avons \( gf=\id\), mais pas \( fg=\id\) parce que ce \( (fg)(e_0)=0\).
\end{example}

\begin{lemma}       \label{LEMooRZDTooEuLTrO}
    Si \( E\) et \( F\) sont des espaces vectoriels et si \( f\colon E\to F\) est une application linéaire inversible, alors son inverse est également linéaire.
\end{lemma}

\begin{proof}
    Nous avons \( f^{-1}(x+y)=f^{-1}(x)+f^{-1}(y)\). En effet,
    \begin{equation}
        f\big( f^{-1}(x)+f^{-1}(y) \big)=f\big( f^{-1}(x) \big)+f\big( f^{-1}(y) \big)=x+y.
    \end{equation}
    De la même façon,
    \begin{equation}
        f\big( \lambda f^{-1}(x) \big)=\lambda x, 
    \end{equation}
    donc \( f^{-1}(\lambda x)=\lambda f^{-1}(x)\).
\end{proof}

\begin{proposition}     \label{PROPooHLUYooNsDgbn}
    Soient un espace vectoriel \( E\) de dimension finie, un endomorphisme \( f\colon E\to E\) et une partie \( \{v_i\}_{i\in I}\) tel que \( \{f(v_i)\}_{i\in I}\) soit une base.

    Alors \( \{v_i\}_{i\in I}\) est une base.
\end{proposition}

\begin{proof}
    Soit \( x\in E\). Il existe une partie finie \( J\subset I\) et des scalaires \( \lambda_j\) tels que 
    \begin{equation}
        x=\sum_j\lambda_jf(v_j)=f\big( \sum_j\lambda_jv_j \big),
    \end{equation}
    ce qui prouve que \( f\) est surjective. Le corolaire \ref{CORooCCXHooALmxKk} nous dit alors que \( f\) est une bijection. L'application inverse est également linéaire par le lemme \ref{LEMooRZDTooEuLTrO}.

    Une application linéaire bijective (comme \( f^{-1}\)) transforme une base en une base par la proposition \ref{PROPooZFKZooBGLSex}. Donc 
    \begin{equation}
        f^{-1}\big( \{f(v_i)\} \big)
    \end{equation}
    est une base.
\end{proof}

\begin{proposition}     \label{PROPooADESooATJSrH}
    Soit un espace vectoriel \( E\) de dimension finie et deux applications linéaires \( f,g\colon E\to E\) telles que \( g\circ f=\id\). Alors \( f\) et \( g\) sont bijectives.
\end{proposition}

\begin{proof}
    En plusieurs étapes
    \begin{subproof}
        \item[\( f\) est injective]
            Si \( f(x)=f(y)\), alors en appliquant \( g\) nous avons 
            \begin{equation}
                g\big( f(x) \big)=g\big( f(y) \big),
            \end{equation}
            ce qui donne \( x=y\).
        \item[\( f\) est surjective]
            C'est maintenant le corolaire \ref{CORooCCXHooALmxKk}.
        \item[\( g\) est surjective]
            Pour tout \( x\in E\) nous avons \( g\big( f(x) \big)=x\). Donc l'image de \( f(E)\) par \( g\) est $E$. 
        \item[\( g\) est injective]
            C'est maintenant le corolaire \ref{CORooCCXHooALmxKk}.
    \end{subproof}
\end{proof}

%---------------------------------------------------------------------------------------------------------------------------
\subsection{Injection, surjection}
%---------------------------------------------------------------------------------------------------------------------------

\begin{definition}      \label{DEFooVTXWooVXfUnc}
    Soient deux espaces vectoriels \( E\) et \( V\). Une application \( f\colon E\to V\) est \defe{affine}{application affine} si il existe une application linéaire \( u\colon E \to V\) et un élément \( v\in V\) tel que
    \begin{equation}
        f(x)=u(x)+v
    \end{equation}
    pour tout \( x\in E\).
\end{definition}


\begin{definition}
    Soit un espace vectoriel \( V\). Une \defe{droite vectorielle}{droite vectorielle} dans \( V\) est un sous-espace vectoriel de dimension \( 1\). Un \defe{plan vectoriel}{plan vectoriel} est un sous-espace vectoriel de dimension \( 2\).

    Une partie \( D\) de \( V\) est une \defe{droite}{droite} si il existe \( v\in V\) tel que \( D-v\) soit une droite vectorielle.
\end{definition}

\begin{lemma}       \label{LEMooQQFFooEZYeck}
    Si \( D\) est une droite et si \( a,b\in D\), alors \( D-a=D-b\) et \( D-a\) est une droite vectorielle.
\end{lemma}

\begin{proof}
    Vu que \( D\) est une droite, il existe \( v\in V\) tel que \( D-v\) soit une droite vectorielle que nous notons \( L\). Nous allons montrer que \( D-a=D-v\). Vu que \( a\) est arbitraire, cela suffit.

    \begin{subproof}
        \item[\( D-a\subset D-v\)]
            Un élément de \( D-a\) est de la forme \( x-a\) avec \( x\in D\). Nous écrivons \( x-a\) sous la forme \( y-v\) et nous espérons que \( y\in D\). Allons-y : d'abord nous isolons \( y\) dans \( x-a=y-v\) :
            \begin{subequations}
                \begin{align}
                    y=x-a+v=(x-v)-(a-v)+v.
                \end{align}
            \end{subequations}
            Vu que \( x-v\) et \( a-v\) sont des éléments de \( L\), la somme est dans \( L\) et donc \( y=l+v\) pour un certain élément de \( l\in L\). Nous avons donc prouvé que \( y\in D\) et donc que \( x-a=y-v\in D-v\).
        \item[\( D-v\subset D-a\)]
            Nous notons \( x-v\) un élément générique que \( D-v\) (\( x\in D\)). En posant \( y-a=x-v\), nous trouvons
            \begin{equation}
                y=x-v+a=\underbrace{x-v}_{\in L}+\underbrace{(a-v)}_{\in L}+v
            \end{equation}
            Donc \( y\in D\) et \( x-v=y-a\in D-a\).
    \end{subproof}
\end{proof}

\begin{lemma}
    À propos de droites.
    \begin{enumerate}
        \item       \label{ITEMooYQCIooOrhRwj}
            Si \( L\) est une droite vectorielle, alors pour tout \( a\neq 0\) dans \( L\), nous avons \( L=\Image(f)\) où \( f\) est l'application linéaire donnée par
            \begin{equation}
                \begin{aligned}
                    f\colon \eK&\to V \\
                    \lambda&\mapsto \lambda a. 
                \end{aligned}
            \end{equation}
        \item       \label{ITEMooZIGMooGruFMP}
            Si \( D\) est une droite affine, alors pour tout \( a\neq b\) sur \( D\) nous avons \( D=\Image(f)\) où \( f\) est l'application affine donnée par
            \begin{equation}
                \begin{aligned}
                    g\colon \eK&\to V \\
                    \lambda&\mapsto a+\lambda(b-a). 
                \end{aligned}
            \end{equation}
    \end{enumerate}
\end{lemma}

\begin{proof}
    En deux parties.
    \begin{subproof}
        \item[Pour \ref{ITEMooYQCIooOrhRwj}]
            Vu que \( L\) est un sous-espace de dimension \( 1\), il possède une baser, disons \( \{ b \}\). En particulier \( a=\mu b\) pour un certain \( \mu\in \eK\). Si \( x\in L\) nous avons \( x=\lambda_x b\) pour un certain \( \lambda_x\), et donc
            \begin{equation}
                x=\frac{ \lambda_x }{ \mu }a.
            \end{equation}
            Donc \( x=f(\lambda_x/\mu)\). Cela prouve que \( L\subset\Image(f)\).

            L'inclusion inverse est simplement le fait que \( \lambda a\in L\) dès que \( a\in L\) parce que \( L\) est vectoriel.
        \item[Pour \ref{ITEMooZIGMooGruFMP}]
            Le lemme \ref{LEMooQQFFooEZYeck} nous indique qu'il existe une droite vectorielle \( L\) telle que \( D-x=L\) pour tout \( x\in D\).
            \begin{subproof}
                \item[\( D\subset\Image(g)\)]
                    Nous nommons \( f\colon \eK\to V\) l'application linéaire qui donne \( L\). Vu que \( b-a\in L\) nous avons
                    \begin{equation}
                        f(\lambda)=\lambda(b-a),
                    \end{equation}
                    et tout élément de \( L\) est de la forme \( f(\lambda)\). Nous avons aussi \( D=L+a\); donc un élément de \( D\) est de la forme \( f(\lambda)+a\) et donc de la forme \( \lambda(b-a)+a=g(\lambda)\).
                \item[\( \Image(g)\subset D\)]
                    Un élément de \( \Image(g)\) est de la forme \( a+\lambda(b-a)\) avec \( \lambda\in \eK\). Mais \( b-a\in L\), donc \( \lambda(b-a)\in L\) et 
                    \begin{equation}
                        g(\lambda)=a+\lambda(b-a)\in a+L=D.
                    \end{equation}
            \end{subproof}
    \end{subproof}
\end{proof}

\begin{example}
	Les exemples les plus courants d'applications affines sont les droites et les plans ne passant pas par l'origine.
	\begin{description}
		\item[Les droites] Une droite dans $\eR^2$ (ou $\eR^3$) qui ne passe pas par l'origine est l'image d'une fonction de la forme $s(t) =u t +v$, avec $t \in \eR$, et $u$ et $v$ dans $\eR^2$ ou $\eR^3$ selon le cas. 

		En choisissant des coordonnées adéquates, les droites peuvent être aussi vues comme graphes de fonctions affines. Dans le cas de $\eR^2$, on retrouve la fonction de l'exemple~\ref{ex_affine}, pour \( n = m = 1 \).

		\item[Les plans]
			De la même façon nous savons que tout plan qui ne passe pas par l'origine dans $\eR^3$ est le graphe d'une application affine, $P(x,y)= (a,b)^T\cdot(x,y)^T+(c,d)^T$, lorsque les coordonnées sont bien choisies.
	\end{description}
\end{example}

\begin{lemma}[\cite{ooEPEFooQiPESf}]        \label{LEMooDAACooElDsYb}
    Soit une application linéaire \( f\colon E\to F\).
    \begin{enumerate}
        \item       \label{ITEMooEZEWooZGoqsZ}
            L'application \( f\) est injective si et seulement s'il existe \( g\colon F\to E\) telle que \( g\circ f=\id|_E\).
        \item
            L'application \( f\) est surjective si et seulement s'il existe \( g\colon F\to E\) telle que \( f\circ g=\id|_F\).
    \end{enumerate}
\end{lemma}

\begin{proof}
    Nous démontrons séparément les deux affirmations.
    \begin{enumerate}
        \item
            Si \( f\) est injective, alors \( f\colon E\to \Image(f)\) est un isomorphisme. Si $V$ est un supplémentaire de \( \Image(f)\) dans \( F\) (c'est-à-dire \( F=\Image(f)\oplus V\)) alors nous pouvons poser \( g(x+v)=f^{-1}(x)\) où \( x+v\) est la décomposition (unique) d'un élément de \( F\) en \( x\in\Image(f)\) et \( v\in V\). Avec cela nous avons bien \( g\circ f=\id\).

            Inversement, s'il existe \( g\colon F\to E\) telle que \( g\circ f=\id\) alors \( f\colon E\to E\) doit être injective. Parce que si \( f(x)=0\) avec \( x\neq 0\) alors \( (g\circ f)(x)=0\neq x\).
        \item
            Si \( f\) est surjective nous pouvons choisir des éléments \( x_1,\ldots, x_p\) dans \( E\) tels que \( \{ f(x_i) \}\) soit une base de \( F\). Ensuite nous définissons
            \begin{equation}
                \begin{aligned}
                    g\colon F&\to E \\
                    \sum_k a_k f(x_k)&\mapsto \sum_k a_k x_k.
                \end{aligned}
            \end{equation}
            Cela donne \(  f\circ g=\id|_F\) parce que si \( v\in F\) alors \( v=\sum_kv_kf(x_k)\) avec \( v_k\in \eK\), et nous avons
            \begin{equation}
                (f\circ g)(v)=\sum_k v_k (f\circ g) \left(f(x_k)\right)
                             =f\left( \sum_k v_k x_k \right)
                             =\sum_k v_k f(x_k) = v.
            \end{equation}

            Inversement, s'il existe \( g\colon F\to E\) tel que \( f\circ g=\id\) alors \( f\) doit être surjective parce que
            \begin{equation}
                F=\Image(f\circ g)=f\big( \Image(g) \big)\subset \Image(f).
            \end{equation}
    \end{enumerate}
\end{proof}



%+++++++++++++++++++++++++++++++++++++++++++++++++++++++++++++++++++++++++++++++++++++++++++++++++++++++++++++++++++++++++++
\section{Matrices}
%+++++++++++++++++++++++++++++++++++++++++++++++++++++++++++++++++++++++++++++++++++++++++++++++++++++++++++++++++++++++++++

Les matrices et les applications linéaires sont deux choses différentes. Une application linéaire\footnote{Définition \ref{DEFooULVAooXJuRmr}.} est une application d'un espace vectoriel vers un autre, et une matrice est un simple tableau de nombres sur lesquels nous définissons des opérations, de telle sorte à fournir une structure d'espace vectoriel. Le lien entre ces opérations et les opérations correspondantes sur les applications linéaires sera fait plus tard.

%TODO: fixer les notations d'ensembles M( n x m, K) et matrice identité dans l'index des notations.

%--------------------------------------------------------------------------------------------------------------------------- 
\subsection{Définitions}
%---------------------------------------------------------------------------------------------------------------------------

\begin{definition}
    Soit un anneau \( \eA\) ainsi que des entiers \( m\), \( n\) strictement positifs. L'ensemble \( \eM(n\times n,\eA)\) est l'ensemble des applications
    \begin{equation}
        \{ 1,\ldots, n \}\times \{ 1,\ldots, m \}\to \eA,
    \end{equation}
    et est appelé ensemble des \defe{matrices}{matrice} \(n\times m\) sur \( \eA \).
\end{definition}
Si \( A\) est une matrice, nous notons \( A_{ij}\) au lieu de \( A(i,j)\) l'image de \( (i,j)\) par l'application \( A\).


\begin{definition}
Quelques ensembles de matrices particuliers.
  \begin{enumerate}
  \item Si \( n=m\), alors:
  \begin{itemize}
    \item nous disons que la matrice est \defe{carrée}{carrée!matrice},
    \item nous notons \( \eM(n,\eA)\) pour \( \eM(n\times n,\eA)\),
    \item \( n \) est appelée \defe{ordre}{ordre!d'une matrice carrée} de la matrice.
  \end{itemize}
  \item Si \( n = 1 \), alors la matrice est appelée \defe{matrice-ligne}{matrice-ligne}.
    \item Si \( m = 1 \), alors la matrice est appelée \defe{matrice-colonne}{matrice-colonne}.
  \end{enumerate}
\end{definition}

\begin{normaltext}
    On note les isomorphismes naturels \( \eM(1\times m,\eA) \simeq \eA^m\) et \( \eM(n\times 1,\eA) \simeq \eA^n\).
\end{normaltext}

\begin{lemmaDef}
    L'ensemble \( \eM(n\times m, \eA)\) muni des opérations
    \begin{description}
        \item[Somme] \( (A+B)_{ij}=A_{ij}+B_{ij}\),
        \item[Produit par un scalaire] \( (\lambda A)_{ij}=\lambda A_{ij}\),
    \end{description}
    pour tout \( A,B\in \eM(n\times m,\eA ) \) et \( \lambda\in \eA \) est un \( \eA\)-module (définition \ref{DEFooHXITooBFvzrR}).
\end{lemmaDef}

\begin{lemmaDef}
    Avec la multiplication
    \begin{equation}
        \begin{aligned}
             \eM(n\times p,\eA)\times \eM(p\times m,\eA)&\to \eM(n\times m,\eA) \\
             (A,B)&\mapsto (AB)_{ij}=\sum_{k=1}^pA_{ik}B_{kj},
        \end{aligned}
    \end{equation}
    l'espace \( \eM(n,\eK)\) est une \( \eK\)-algèbre\footnote{Définition \ref{DefAEbnJqI}.}.
\end{lemmaDef}

\begin{definition}
    Pour un élément \( A\in \eM(n\times m, \eA)\) nous définissons encore
    \begin{description}
        \item[La transposée] \( A^t_{ij}=A_{ji}\),
        \item[La trace] \( \tr(A)=\sum_iA_{ii}\).
    \end{description}
\end{definition}


\begin{remark}
    Quelques remarques directes sur les définitions.
    \begin{enumerate}
        \item
            La motivation de cette définition pour le produit apparaîtra plus loin, mais le Frido n'étant pas un livre d'introduction, j'imagine que le lecteur a déjà une idée.
        \item
            Nous verrons plus loin en \ref{SUBSECooGPXVooEYwIiJ} que la définition de transposée d'une application linéaire n'est pas tout à fait évidente; elle sera la définition \ref{DefooZLPAooKTITdd}.

            Ici nous avons bien défini la transposée d'une matrice, pas d'une application linéaire.
    \end{enumerate}
\end{remark}

\begin{remark}
    Quelques remarques à propos de structures supplémentaires.
\begin{enumerate}
    \item Nous utiliserons (presque) tout le temps des matrices à coefficients dans un corps. Il est clair que, si \( \eK \) est un corps (commutatif), alors \( \eM(n\times m,\eK) \) a une structure d'espace vectoriel sur \( \eK \).
    \item Par ailleurs, sur les matrices carrées d'ordre \( n \) fixé, le produit de deux matrices est bien défini. Ainsi, \( \eM(n,\eA)\) se voit conférer une structure d'anneau, dont le neutre pour la multiplication est la matrice carrée \( \mtu_n\) (notée aussi \( \mtu\) lorsqu'il n'y a pas d'ambiguïté sur la taille), donnée par
\begin{equation}
    \mtu_{ij}=\begin{cases}
        1    &   \text{si } i=j\\
        0    &    \text{sinon.}
    \end{cases}
\end{equation}
Il est vite vu que si \( A\) est une matrice carrée d'ordre \( n \), alors \( A\mtu=\mtu A=A\).
\end{enumerate}
\end{remark}

\begin{lemma}[\cite{MonCerveau}]        \label{LEMooUXDRooWZbMVN}
    Si \( A\), \( B\) et \( C\) sont des matrices nous avons
    \begin{enumerate}
        \item
            \( (AB)^t=B^tA^t\),
        \item       \label{ITEMooXDYQooAlnArd}
            \( \tr(ABC)=\tr(CAB)\).
    \end{enumerate}
\end{lemma}

\begin{proof}
    La première est un simple calcul :
    \begin{equation}
        (AB)^t_{ij}=(AB)_{ji}=\sum_kA_{jk}B_{ki}=\sum_kA^t_{kj}B^t_{ik}=(B^tA^t)_{ij}.
    \end{equation}
    Pour la seconde :
    \begin{equation}
        \tr(ABC)=\sum_{ikl}A_{ik}B_{kl}C_{li}=\sum_{ikl}C_{li}A_{ik}B_{kl}=\sum_l(CAB)_{ll}=\tr(CAB).
    \end{equation}
\end{proof}

\begin{normaltext}
    La seconde égalité est importante et est nommée \defe{invariance cyclique}{invariance cyclique!trace} de la trace. Elle sert entre autres nombreuses choses à prouver que la trace d'une matrice d'une application linéaire ne dépend pas de la base choisie. Ce sera la proposition \ref{PROPooRMYQooWkEpJJ}.
\end{normaltext}

\begin{lemma}       \label{LEMooLXAHooPRyHaF}
    Soient des matrices \( A,B\in \eM(n,\eK)\). Si pour tout \( x,y\in \eK^n\) nous avons
    \begin{equation}
        \sum_{ij}A_{ij}x_iy_j=\sum_{ij}B_{ij}x_iy_j
    \end{equation}
    alors \( A=B\).
\end{lemma}

\begin{proof}
    Il suffit de choisir \( x_i=\delta_{ik}\) et \( y_j=\delta_{jl}\), et d'effectuer les sommes; par exemple
    \begin{equation}
        \sum_{ij}A_{ij}\delta_{ik}y_j=\sum_jA_{kj}y_j.
    \end{equation}
    Après avoir effectué toutes les sommes nous nous retrouvons avec \( A_{kl}=B_{kl}\), ce qui signifie \( A=B\).
\end{proof}


%--------------------------------------------------------------------------------------------------------------------------- 
\subsection{Application linéaire associée}
%---------------------------------------------------------------------------------------------------------------------------

Soient deux espaces vectoriels de dimension finie \( E,F\) sur le corps \( \eK\). Nous considérons les bases\footnote{C'est le théorème~\ref{ThonmnWKs} qui nous permet de considérer des bases. Et ce théorème ne fonctionne que parce que nous avons supposé une dimension finie.} \( \{ e_i \}\) pour \( E\) et \( \{ f_{\alpha} \}\) pour \( F\). 

\begin{definition}      \label{DEFooJVOAooUgGKme}
    Nous considérons l'application
    \begin{equation}        \label{EQooVZQWooMyFFeO}
        \begin{aligned}
            \psi\colon \eM(n\times m, \eK)&\to \aL(E,F) \\
            A&\mapsto f_A 
        \end{aligned}
    \end{equation}
    où \( f_A\) est définie par
    \begin{equation}        \label{EQooBVGHooJhFbMs}
        f_A(x)=\sum_{i\alpha}A_{\alpha i}x_if_{\alpha}
    \end{equation}
    si \( x_i\) sont les coordonnées de \( x\in E\) dans la base \( \{ e_i \}\).
\end{definition}



\begin{normaltext}
    Nous allons prouver un certain nombre de résultats montrant que cette application a toutes les propriétés imaginables permettant d'identifier les matrices aux applications linéaires : elle est un isomorphisme pour toutes les structure que vous pouvez raisonnablement imaginer.

    À cette application \( \psi\) il manque cependant une propriété importante : elle n'est pas canonique. Elle dépend des bases choisies. Autrement dit : nous avons à priori autant d'applications \( \psi\) différentes qu'il y a de choix de bases sur \( E\) et \( F\)\quext{Bonne question. Est-ce qu'il y a moyen de construire deux choix de bases donnant la même application \( \psi\) ? Écrivez-moi si vous savez la réponse.}. %TODOooFKSRooBtOZYc

    Nous allons prouver maintenant quelques résultats montrant que les matrices et les applications linéaires, dans le cas des espaces vectoriels \( \eK^n\) sont deux présentations de la même chose.
\end{normaltext}

\begin{normaltext}
    Lorsque \( A\in \eM(n,\eK)\) est une matrice et \( x\in \eK^n\) un vecteur, nous notons \( Ax\) l'élément de \( \eK^n\) donné par
    \begin{equation}        \label{EQooQFVTooMFfzol}
        (Ax)_i=\sum_jA_{ij}x_j.
    \end{equation}
    Autrement dit, \( Ax=f_A(x)\).

    Cette convention et de nombreuses autres à propos de matrice sera rappelée dans \ref{SECooBTTTooZZABWA}.
\end{normaltext}

\begin{propositionDef}      \label{PROPooGXDBooHfKRrv}
    Soient deux espaces vectoriels de dimension finie \( E,F\) sur le corps \( \eK\). Nous considérons les bases \( \{ e_i \}\) pour \( E\) et \( \{ f_{\alpha} \}\) pour \( F\). 

    Nous considérons l'application
    \begin{equation}
        \begin{aligned}
            \psi\colon \eM(n\times m, \eK)&\to \aL(E,F) \\
            A&\mapsto f_A 
        \end{aligned}
    \end{equation}
    où \( f_A\) est définie par
    \begin{equation}        \label{EQooZKEKooNYjvhP}
        f_A(x)=\sum_{i\alpha}A_{\alpha i}x_if_{\alpha}
    \end{equation}
    si \( x_i\) sont les coordonnées de \( x\in E\) dans la base \( \{ e_i \}\).

    Alors
    \begin{enumerate}
        \item       \label{ITEMooKZYYooZPTkpq}
            Nous avons
            \begin{equation}
                f_A(e_i)_{\alpha}=A_{\alpha i}.
            \end{equation}
        \item       \label{ITEMooANXFooGIuxUR}
            Nous avons
            \begin{equation}                \label{EQooOKOJooYgteNP}
                f_A(e_i)=\sum_{\alpha}A_{\alpha i}f_{\alpha}.
            \end{equation}
        \item       \label{ITEMooXLLLooKfigfB}
            Nous avons
            \begin{equation}        \label{EQooAXRJooUwHbjB}
                \big( f_A(x) \big)_{\alpha}=\sum_{i}A_{\alpha i}x_i.
            \end{equation}
        \item       \label{ITEMooHSMLooRJZref}
            L'application \( \psi\) est une bijection.
    \end{enumerate}
    Si \( f\) est une application linéaire, alors la matrice \( \psi^{-1}(f)\) est la \defe{matrice associée}{matrice d'une application linéaire} à \( f\) dans les bases choisies.
\end{propositionDef}

Remarque : les bases ne sont supposées être canoniques en aucun sens du terme. Les dimensions de \( E\) et \( F\) ne sont pas non plus supposées identiques.

\begin{proof}
    En nous rappelant que \( (e_j)_i=\delta_{ij}\) nous avons
    \begin{equation}        \label{EQooWGZHooIBoygB}
        f_A(e_j)=\sum_{i\alpha}A_{\alpha i}(e_j)_if_{\alpha}=\sum_{\alpha}A_{\alpha j}f_{\alpha},
    \end{equation}
    donc \( f_A(e_i)_{\alpha}=A_{\alpha i}\). Cela prouve la formule du point \ref{ITEMooKZYYooZPTkpq}.

    Le point \ref{ITEMooANXFooGIuxUR} est une simple somme sur \( \alpha\) de \ref{ITEMooKZYYooZPTkpq}.

    La formule \eqref{ITEMooXLLLooKfigfB} est simplement la composante \( f_{\alpha}\) de la définition \ref{EQooZKEKooNYjvhP}.

    Prouvons que \( \psi\) est injective. Si \( f_A=f_B\), nous avons en particulier \( f_A(e_i)_{\alpha}=f_B(e_i)_{\alpha}\) et donc \( A_{\alpha i}=B_{\alpha i}\).

    Prouvons que \( \psi\) est surjective. Pour cela nous considérons \( f\in \aL(E,F)\) et nous posons \( A_{\alpha i}=f(e_i)_{\alpha}\). Nous avons alors \( f=f_A\) parce que
    \begin{equation}
        f_A(x)=\sum_{i\alpha}A_{\alpha i}x_if_{\alpha}=\sum_{i\alpha}f(e_i)_{\alpha}x_if_{\alpha}=\sum_{\alpha}f(\sum_ix_ie_i)_{\alpha}f_{\alpha}=\sum_{\alpha}f(x)_{\alpha}f_{\alpha}=f(x).
    \end{equation}
\end{proof}

La proposition suivante montre que le produit matriciel correspond à la composition d'applications linéaires, pourvu que l'on travaille avec les bases canoniques sur \( \eK^n\).
\begin{proposition}[\cite{MonCerveau}]      \label{PROPooIYVQooOiuRhX}
    Soit un corps commutatif \( \eK\). Nous considérons des espaces vectoriels \( E\) et \( F\) munis de bases \( \{ e_i \}_{i=1,\ldots, n}\) et \( \{ f_{\alpha}\}_{\alpha\in 1,\ldots, m} \).

    L'application déjà définie\footnote{Notez la position du \( n\) et du \( m\). Sachez noter les bornes des sommes écrites dans la démonstration.}
    \begin{equation}
        \psi\colon \eM(m\times n,\eK)\to \aL(E,F)
    \end{equation}
    est un isomorphisme d'espaces vectoriels.
\end{proposition}

\begin{proof}
    Le fait que \( \psi\) soit une bijection est la proposition \ref{PROPooGXDBooHfKRrv}. Nous devons montrer que c'est linéaire. 

    Pour \( \lambda\in \eK\) nous avons le calcul
    \begin{equation}
        \psi(\lambda A)(e_k)=f_{\lambda A}(e_k)=\sum_{\alpha i}(\lambda A)_{\alpha i}\underbrace{(e_k)_i}_{=\delta_{ki}}f_{\alpha}=\lambda\sum_{\alpha}A_{\alpha k}f_{\alpha}=\lambda f_A(e_j).
    \end{equation}
    Donc \( \psi(\lambda A)=\lambda\psi(A)\).

    Si \( A,B\in \eM(n,\eK)\) nous avons de la même façon \( f_{A+B}=f_A+f_B\).
\end{proof}

\begin{proposition}     \label{PROPooCSJNooEqcmFm}
    Soient des espaces vectoriels \( E\), \( F\) et \( G\) de dimensions \( n\), \( m\) et \( p\) munis de bases\footnote{Avec trois, nous renonçons à utiliser des alphabets différents pour numéroter les éléments des bases.} \( \{ e_i \}\), \( \{ f_i \}\) et \( \{ g_i \}\). Nous considérons les deux applications
    \begin{equation}
        \psi\colon \eM(m\times n,\eK)\to \aL(E,F)
    \end{equation}
    et
    \begin{equation}
        \psi\colon \eM(p\times m,\eK)\to \aL(F,G).
    \end{equation}
    Nous avons 
    \begin{equation}
        f_A\circ f_B=f_{AB}
    \end{equation}
    pour toutes matrices \( A\in \eM(p\times m,\eK)\) et \( B\in \eM(m\times n,\eK)\).
\end{proposition}

\begin{proof}
    Nous considérons les applications linéaires associées à \( A\) et \( B\) : \( f_A\colon F\to G\) et \( f_B\colon E\to F\) et la composée \( f_A\circ f_B\colon E\to G\). Et puis c'est le calcul :
    \begin{subequations}
        \begin{align}
            (f_A\circ f_B)(e_k)&=f_A\big( \sum_{ij}B_{ij}(e_k)_jf_i \big)\\
            &=\sum_i B_{ik}f_A(f_i)\\
            &=\sum_iB_{ik}\sum_{rs}A_{rs}(f_i)_sg_r\\
            &=\sum_{ir}B_{ik}A_{ri}g_r\\
            &=\sum_r(AB)_{rk}g_r\\
            &=f_{AB}(e_k).
        \end{align}
    \end{subequations}
    Donc \( f_A\circ f_B=f_{AB}\) comme il se doit.
\end{proof}
    
Nous pouvons particulariser au cas où \( E=F=G\).
\begin{proposition}     \label{PROPooFMBFooEVCLKA}
    Si \( E\) est un espace vectoriel muni d'une base \( \{ e_i \}\), alors l'application
    \begin{equation}
        \psi\colon \eM(n,\eK)\to \End(E)
    \end{equation}
    est un isomorphisme d'algèbre\footnote{Définition \ref{DefAEbnJqI}.} et d'anneaux\footnote{Définition \ref{DEFooSPHPooCwjzuz}}.
\end{proposition}

\begin{proof}
    Le fait que \( \psi\) soit un isomorphisme d'algèbre est juste la combinaison entre les propositions \ref{PROPooIYVQooOiuRhX} et \ref{PROPooCSJNooEqcmFm}.

    En ce qui concerne l'isomorphisme d'anneaux, il faut en plus identifier les neutres. Le neutre pour la composition d'applications linéaires est l'application identité et le neutre pour la multiplication de matrices est la matrice identité. Nous devons donc montrer que \( \psi(\delta)=\id\). Juste un calcul :
    \begin{equation}
        f_{\delta}(x)=\sum_{ij}\delta_{ij}x_je_i=\sum_ix_ie_i=x.
    \end{equation}
    Donc oui, \( f_{\delta}\) est l'identité.
\end{proof}

Voila. Soyez bien conscient que l'application \( \psi\) dont nous avons beaucoup parlé est surtout intéressante dans le cas des espaces de la forme \( \eK^n\). Dans ce cas, nous avons une identification canonique entre \( \eM(n,\eK)\) et \( \End(\eK^n)\) qui est un isomorphisme d'anneaux et d'algèbres.

Nous verrons que ce \( \psi\) respecte encore les inverses\footnote{Proposition \ref{PROPooNPMCooPmaCwu}.} et les déterminants\footnote{Proposition \ref{PROPooFKDXooKMSolt}.}.

\begin{normaltext}
    Il convient de ne pas confondre matrice et application linéaire (bien que nous le ferons sans vergogne). Une matrice est un bête tableau de nombres, tandis qu'une application linéaire est une application entre deux espaces vectoriels vérifiant certaines propriétés.

    Cependant si les espaces vectoriels \( E\) et \( F\) sont munis de bases, alors il y a une application
    \begin{equation}
        \psi\colon \eM(m\times n,\eK)\to \aL(E,F)
    \end{equation}
    qui a toutes les propriétés imaginables\footnote{Et elle en aura encore plus lorsque nous aurons vus les déterminants.}.

    Cette application dépend des bases choisies. Il n'y a donc pas de trucs comme «la matrice de telle application linéaire» ou comme «voici une matrice, nous considérons l'application linéaire associée». 

    Cependant, sur des espaces comme \( \eR^n\) ou plus généralement sur \( \eK^n\), nous avons une base canonique et toute personne raisonnable utilise toujours la base canonique (sauf mention du contraire). Dans ces cas il est sans danger de dire «la matrice associée à telle application linéaire» sans préciser les bases.

    Mais si un jour vous utilisez une base autre que la base canonique sur \( \eR^n\), précisez-le et plutôt deux fois qu'une\footnote{Au passage, non, les coordonnées polaires ne sont pas une base de \( \eR^2\). C'est un système de coordonnées, et ce n'est pas la même chose.}.
\end{normaltext}

\begin{normaltext}
    Tant que nous sommes à parler de matrice et d'applications linéaires, les plus acharnés anti-abus de language\footnote{Dont l'auteur de ces lignes fait partie.} remarqueront qu'il n'est pas vrai que «étant donné une base, une application linéaire a une matrice».

    En effet, une base est une partie libre et génératrice (définition \ref{DEFooNGDSooEDAwTh}). Or une partie d'un ensemble n'est pas muni d'un ordre. Toutes les permutations de colonnes de la matrice sont encore possible d'après l'ordre que l'on met sur les vecteurs de la base.

    Encore une fois, la base canonique n'a pas de problèmes parce que les \( \{ e_i \}\) de \( \eR^n\) viennent avec un ordre indiscutable. Plus généralement, très souvent, lorsqu'on construit une base, la construction suggère un ordre.
\end{normaltext}

%---------------------------------------------------------------------------------------------------------------------------
\subsection{Déterminant}
%---------------------------------------------------------------------------------------------------------------------------

\begin{definition}      \label{DEFooYCKRooTrajdP}
    Si \( A\in\eM(n,\eK)\) nous définissons le \defe{déterminant}{déterminant!matrice} de \( A\) par la formule
    \begin{equation}
        \det(A)=\sum_{\sigma\in S_n}(-1)^{\sigma}\prod_{i=1}^nA_{i\sigma(i)}
    \end{equation}
    où la somme est effectuée sur tous les éléments du groupe symétrique\footnote{Pour le groupe symétrique, c'est la définition \ref{DEFooJNPIooMuzIXd}, le fait que ce soit un groupe fini est le lemme \ref{LEMooSGWKooKFIDyT}, et pour la somme sur un groupe fini c'est la définition \ref{DEFooLNEXooYMQjRo}..} \( S_n\) et où \( (-1)^{\sigma}\) représente la parité de la permutation \( \sigma\).
\end{definition}
En se souvenant que \( | S_n |=n!\), nous sommes frappés de stupeur devant le fait que le nombre de termes dans la somme croît de façon factorielle (c'est plus qu'exponentiel, pour info) en la taille de la matrice. Cette formule est donc sans espoir pour une matrice plus grande que \( 3\times 3\) ou à la rigueur \( 4\times 4\) à la main. À l'ordinateur, il est possible de monter plus haut, mais pas tellement.

%---------------------------------------------------------------------------------------------------------------------------
\subsection{Déterminant en petite dimension}
%---------------------------------------------------------------------------------------------------------------------------

En dimension deux, le déterminant de la matrice
    $\begin{pmatrix}
        a    &   b    \\
        c    &   d
    \end{pmatrix}$
est le nombre
\begin{equation}        \label{EQooQRGVooChwRMd}
     \det\begin{pmatrix}
         a   &   b    \\
         c   &   d
     \end{pmatrix}=\begin{vmatrix}
          a  &   b    \\
        c    &   d
    \end{vmatrix}=ad-cb.
\end{equation}
Ce nombre détermine entre autres le nombre de solutions que va avoir le système d'équations linéaires associé à la matrice.

Pour une matrice $3\times 3$, nous avons le même concept, mais un peu plus compliqué; nous avons la formule
\begin{equation}
    \det
    \begin{pmatrix}
        a_{11}    &   a_{12}    &   a_{13}    \\
        a_{21}    &   a_{22}    &   a_{23}    \\
        a_{31}    &   a_{32}    &   a_{33}
    \end{pmatrix}
    =
    \begin{vmatrix}
        a_{11}    &   a_{12}    &   a_{13}    \\
        a_{21}    &   a_{22}    &   a_{23}    \\
        a_{31}    &   a_{32}    &   a_{33}
    \end{vmatrix}=
    a_{11}\begin{vmatrix}
        a_{22}  &   a_{23}    \\
        a_{32}    &   a_{33}
    \end{vmatrix}-
    a_{12}\begin{vmatrix}
        a_{21}  &   a_{23}    \\
        a_{31}    &   a_{33}
    \end{vmatrix}+
    a_{13}\begin{vmatrix}
        a_{21}  &   a_{22}    \\
        a_{31}    &   a_{32}
    \end{vmatrix}.
\end{equation}

%--------------------------------------------------------------------------------------------------------------------------- 
\subsection{Manipulations de lignes et de colonnes}
%---------------------------------------------------------------------------------------------------------------------------

Nous voudrions savoir ce qu'il se passe avec le déterminant d'une matrice lorsque nous substituons à une ligne ou une colonne une combinaison des autres lignes et colonnes. Lorsque une matrice est donnée, nous notons \( C_j\) sa \( j\)\ieme colonne.

\begin{lemma}[\cite{MonCerveau}]        \label{LEMooRSJTooQEoOtN}
    Si \( A\) et \( B\) sont des matrices, alors
    \begin{equation}
        (AB)^t=B^tA^t.
    \end{equation}
\end{lemma}

\begin{proof}
    Il suffit de calculer les éléments de matrice :
    \begin{equation}
        (AB)^t_{ij}=(AB)_{ji}=\sum_k A_{jk}B_{ki}=\sum_kB^t_{ik}A^t_{kj}=(B^tA^t)_{ij}.
    \end{equation}
\end{proof}

\begin{lemma}[\cite{MonCerveau,ooKYTYooJlzZMp}]        \label{LEMooCEQYooYAbctZ}
    Si \( A\) est une matrice, alors \( \det(A)=\det(A^t)\).
\end{lemma}

\begin{proof}
    Nous commençons par écrire la définition du déterminant :
    \begin{equation}
        \det(A^t)=\sum_{\sigma\in S_n}\epsilon(\sigma)\prod_{i=1}^n(A^t)_{i,\sigma(i)}=\sum_{\sigma}\epsilon(\epsilon)\prod_iA_{\sigma(i),i}.
    \end{equation}
    Pour chaque \( \sigma\) séparément, nous utilisant la proposition \ref{PROPooQMUDooQQVRIe} pour ré-indexer le produit :
    \begin{equation}
        \prod_i A_{\sigma(i),i}=\prod_iA_{i,\sigma^{-1}(i)}.
    \end{equation}
    Nous profitons du fait que l'application \( \varphi\colon S_n\to S_n\) donnée par \( \varphi(\sigma)=\sigma^{-1}\) soit une permutation de \( S_n\) pour appliquer la définition \ref{DEFooLNEXooYMQjRo} et faire la somme sur \( \sigma^{-1}\) :
    \begin{equation}
        \det(A^t)=\sum_{\sigma}\epsilon(\sigma)\prod_iA_{i,\sigma^{-1}(i)}=\sum_{\sigma}\epsilon(\sigma^{-1})\prod_iA_{i,\sigma(i)}=\det(A)
    \end{equation}
    où nous avons utilisé le fait que \(\epsilon(\sigma^{-1})=\epsilon(\sigma)\) (corolaire \ref{CORooZLUKooBOhUPG}).
\end{proof}

Le fait que \( \det(A)=\det(A^t)\) permet, dans toutes les propositions du type «ce qui arrive au déterminant si on change telle ligne ou colonnes» de ne donner qu'une preuve pour la partie «ligne» et déduire automatiquement le cas «colonne». Le lemme suivant donne un exemple d'utilisation.

\begin{lemma}[\cite{MonCerveau}]        \label{LEMooWMQWooGWFlmC}
    Soit une matrice \( A\). Nous considérons la matrice \( B\) obtenue à partir de \( A\) par la permutation de lignes \( L_k\leftrightarrow L_l\) ainsi que la matrice \( C\) obtenue à partir de \( A^t\) par la permutation de colonnes \( C_k\leftrightarrow C_l\).
    
    Alors \( C^t=B\).
\end{lemma}

\begin{proof}
    Calculons les éléments de matrice de \( C\) :
    \begin{equation}
        C_{ij}=\begin{cases}
            (A^t)_{ij}    &   \text{si }  j\neq k, j\neq l\\
            (A^t)_{ik}    &   \text{si } j=l\\
            (A^t)_{il}    &    \text{si }j=k
        \end{cases}=
        \begin{cases}
            A_{ji}    &   \text{si }  j\neq k, j\neq l\\
            A_{ki}    &   \text{si } j=l\\
            A_{li}    &    \text{si }j=k.
        \end{cases}
    \end{equation}
    Ensuite nous prouvons que \( C^t=B\) en écrivant les éléments de \( C^t\) :
    \begin{equation}
        (C^t)_{ij}=C_{ji}=\begin{cases}
            A_{ij}    &   \text{si } i\neq k, i\neq l\\
            A_{kj}    &   \text{si } i=l\\
            A_{lj}    &    \text{si }i=k.
        \end{cases}
    \end{equation}
    Cette dernière expression est la matrice \( A\) après permutation des lignes \( L_k\leftrightarrow L_l\), c'est-à-dire a matrice \( B\).
\end{proof}

Pour la suite nous écrivons \( \delta\) la matrice «identité», c'est-à-dire celle dont les entrées sont précisément les \( \delta_{ik}\).  Nous écrivons également \( E_{ij}\) la matrice contenant de zéros partout sauf en \( (i,j)\) où elle a un \( 1\), c'est-à-dire
\begin{equation}
    (E_{ij})_{kl}=\delta_{ik}\delta_{jl}.
\end{equation}

\begin{proposition}[Permuter des lignes ou des colonnes \( L_k\leftrightarrow L_l\)\cite{ooKBOMooSkKHvu,MonCerveau}]    \label{PROPooFQRDooRPfuxk}
    Soient une matrice \( A\in \eM(n,\eK)\), deux entiers \( k\neq l\) inférieurs ou égaux à \( n\). 
    \begin{enumerate}
        \item   \label{ITEMooAIHWooHXzeys}
            Si \( B\) est la matrice obtenue à partir de \( A\) en permutant deux lignes ou deux colonnes, alors
            \begin{equation}
                \det(A)=-\det(B).
            \end{equation}
        \item  \label{ITEMooDNHWooOMgmxa}
            Si \( B\) est la matrice obtenue à partir de \( A\) par la permutation de lignes \( L_k\leftrightarrow L_l\). Alors
            \begin{equation}
                B=SA
            \end{equation}
            avec \( S=\delta+E_{kl}+E_{lk}-E_{kk}-E_{ll}\).

            Autrement dit : la matrice \( S\) est une matrice de permutations de lignes.
        \item \label{ITEMooSHRQooQrqVdO}
            La matrice \( S\) vérifie \( \det(S)=-1\)
        \item       \label{ITEMooQXSEooMWiKbL}
            Nous avons 
            \begin{equation}
                \det(SA)=\det(S)\det(A).
            \end{equation}
    \end{enumerate}
\end{proposition}

\begin{proof}
    Point par point
    \begin{subproof}
    \item[\ref{ITEMooAIHWooHXzeys} pour les colonnes]

    Soient \( k\) et \( l\) fixés, et considérons la permutation des colonnes \( C_k\) et \( C_l\). Nous notons \( \alpha\) la permutation \( (kl)\) dans \( S_n\) (groupe symétrique, définition \ref{DEFooJNPIooMuzIXd}). Nous avons
    \begin{equation}
        B_{ij}=A_{i \alpha(j)},
    \end{equation}
    ou encore : \( A_{ij}=B_{i\alpha(j)}\). Par définition,
    \begin{equation}
        \det(A)=\sum_{\sigma\in S_{n}}\epsilon(\sigma)\prod_{i=1}^nA_{i\sigma(i)}
    \end{equation}
    C'est le moment d'utiliser la proposition \ref{PROPooWJQQooFINSEc} à propos de somme sur des groupes avec \( G=S_n\), \( h=\alpha\) et 
    \begin{equation}
        f(\sigma)=\epsilon(\sigma)\prod_iA_{i,\sigma(i)}.
    \end{equation}
    Nous savons que \( \epsilon(\alpha)=-1\) et que \( \epsilon\) est un homomorphisme par la proposition \ref{ProphIuJrC}\ref{ITEMooBQKUooFTkvSu}, donc
    \begin{equation}
        f(\alpha \sigma)=\epsilon(\alpha\sigma)\prod_iA_{i,(\alpha\sigma)(i)}=-\epsilon(\sigma)\prod_iB_{i,\sigma(i)}.
    \end{equation}
    Avec ça, nous concluons :
    \begin{equation}
        \det(A)=\sum_{\sigma\in S_n}f(\sigma)=\sum_{\sigma}f(\alpha \sigma)=-\sum_{\sigma\in S_n}\epsilon(\sigma)\prod_{i=1}^nB_{i\sigma(i)}=-\det(B).
    \end{equation}
    \item[\ref{ITEMooAIHWooHXzeys} pour les lignes]

    Que se passe-t-il si nous permutons les lignes \( L_k\) et \( L_{l}\) ?Si nous notons \( B'\) la matrice obtenue à partir de \( A\) par la permutation de lignes \( L_k\leftrightarrow L_l\), et \( C\) celle obtenue de \( A^t\) après permutation de colonnes \( C_k\leftrightarrow C_l\) alors nous avons \( C^t=B'\). Le lemme \ref{LEMooWMQWooGWFlmC} nous dit que \( C^t=B'\). En utilisant le lemme \ref{LEMooCEQYooYAbctZ} sur le déterminant de la transposée,
    \begin{equation}
        \det(B')=\det(C^t)=\det(C)=-\det(A^t)=-\det(A).
    \end{equation}
    Voila qui prouve le résultat pour les permutation de lignes.
        
\item[\ref{ITEMooDNHWooOMgmxa}]
    Si \( k=l\), il n'y a pas de permutations, et il est vite vu que la matrice \( S\) est l'identité parce qu'il y a quatre fois le terme \( E_{kk}\). Nous supposons donc que \( k\neq l\); en particulier \( \delta_{kl}=0\).

    Il s'agit surtout d'un beau calcul :
    \begin{subequations}
        \begin{align}
            (SA)_{ij}=\sum_{m}S_{im}A_{mj}&=A_{ij}+\sum_m(\delta_{ki}\delta_{lm}+\delta_{li}\delta_{lm}-\delta_{ki}\delta_{km}-\delta_{li}\delta_{lm})A_{mj}\\
            &=A_{ij}+\delta_{ki}A_{lj}+\delta_{li}A_{kj}-\delta_{ki}A_{kj}-\delta_{li}A_{lj}.
        \end{align}
    \end{subequations}
    Si \( i\neq j\) et \( i\neq l\), alors \( (SA)_{ij}=A_{ij}\). Si \( i=k\), alors 
    \begin{equation}
        (SA)_{kj}=A_{kj}+A_{lj}-A_{kj}=A_{lj},
    \end{equation}
    c'est-à-dire que la \( k\)\ieme ligne de \( SA\) est la \( l\)\ieme ligne de \( A\).

    Avec \( i=l\) nous obtenons la \( k\)\ieme ligne de \( A\).

    Tout cela montre que \( SA\) est la matrice \( A\) dans laquelle les lignes \( k\) et \( l\) ont été inversées, c'est-à-dire \( SA=B\).

\item[\ref{ITEMooSHRQooQrqVdO}]
            En utilisant la définition du déterminant,
            \begin{subequations}
                \begin{align}
                    \det(S)&=\sum_{\sigma\in S_n}\epsilon(\sigma)\prod_{i=1}^nS_{i\sigma(i)}\\
                    &=\sum_{\sigma}\epsilon(\sigma)\prod_i\big( \delta_{i\sigma(i)}+\delta_{ki}\delta_{l\sigma(i)}+\delta_{li}\delta_{k\sigma(i)}-\delta_{ki}\delta_{k\sigma(i)}-\delta_{li}\delta_{l\sigma(i)} \big).
                \end{align}
            \end{subequations}
            Nous utilisons l'associativité et la commutativité du produit pour séparer les facteurs \( i=k\) et \( i=l\) des autres :
            \begin{equation}
                \det(S)=\sum_{\sigma}\epsilon(\sigma)\prod_{\substack{i\neq k\\i\neq l}}\delta_{i\sigma(i)}(\delta_{k\sigma(k)}+\delta_{l\sigma(k)}-\delta_{k\sigma(k)})(\delta_{l\sigma(l)}+\delta_{k\sigma(l)}-\delta_{l\sigma(l)}).
            \end{equation}
            À cause des facteurs \( i\neq k\) et \( i\neq l\), les \( \sigma\) pour lesquels le tout n'est pas nuls doivent vérifier \( \delta_{i\sigma(i)}=1\) pour tout \( i\) différent de \( k\) et \( l\). Les deux seuls sont donc \( \sigma=\id\) et la permutation \( \sigma=(k,l)\). Pour \( \sigma=\id\), nous avons
            \begin{equation}
                \prod_{\substack{i\neq k\\i\neq l}}\delta_{ii}(\delta_{kk}+\delta_{lk}-\delta_{kk})(\delta_{ll}+\delta_{kl}-\delta_{ll})=0.
            \end{equation}
            Dernier espoir : \( \sigma=(k,l)\). Pour ce terme nous avons \( \epsilon(\sigma)=-1\) et
            \begin{equation}
                \prod_{\substack{i\neq k\\i\neq l}}\delta_{ii}(\delta_{kl}+\delta_{ll}-\delta_{kl})(\delta_{lk}+\delta_{kk}-\delta_{lk})=1.
            \end{equation}
            Au final dans \( \det(S)\) il n'y a de non nul que le terms \( \sigma=(k,l)\) et il vaut \( -1\). Donc
            \begin{equation}
                \det(S)=-1.
            \end{equation}
        \item[\ref{ITEMooQXSEooMWiKbL}]
            Il s'agit de mettre bout à bout les points déjà prouvés :
            \begin{equation}
                \det(SA)=-\det(A)=\det(S)\det(A).
            \end{equation}
    \end{subproof}
\end{proof}

\begin{corollary}[\cite{ooKBOMooSkKHvu}]        \label{CORooAZFCooSYINvBl}
    Soit une matrice \( A\in \eM(n,\eK)\). Si deux lignes ou deux colonnes de \( A\) sont égales, alors \( \det(A)=0\).
\end{corollary}

\begin{proof}
    Si deux colonnes sont égales, la matrice ne change pas lorsqu'on les permute, alors que le déterminant change de signes. La seule possibilité est que \( \det(A)=-\det(A)\), ce qui signifie que \( \det(A)=0\).
\end{proof}
Notons que si pour \( k\neq l\) nous avons \( C_k=\lambda C_l\), alors nous avons aussi \( \det(A)=0\).

La réciproque n'est pas vraie : il existe des matrices dont le déterminant est nul et dont aucune entrée n'est nulle. Par exempe
\begin{equation}
    \begin{pmatrix}
        1    &   2    \\ 
        1    &   2    
    \end{pmatrix}.
\end{equation}


\begin{proposition}[\cite{ooKBOMooSkKHvu}]      \label{PROPooNGZJooHjtMyn}
    Soient \( A\in \eM(n,\eK)\), et \( v\in \eK^n\). Si \( B\) est la matrice \( A\) avec la substitution \( L_j\to L_j+v\) et \( C\) est la matrice \( A\) avec la substitution \( L_j\to v\), alors
    \begin{equation}
        \det(B)=\det(A)+\det(C).
    \end{equation}
\end{proposition}

\begin{proof}
    En utilisant l'associativité de la multiplication,
    \begin{subequations}
        \begin{align}
            \det(B)&=\sum_{\sigma\in S_n}\epsilon(\sigma)\prod_{i=1}^nB_{i\sigma(i)}\\
            &=\sum_{\sigma}\epsilon(\sigma)\big( \prod_{i\neq j}B_{i\sigma(i)} \big)B_{j\sigma(j)}\\
            &=\sum_{\sigma}\epsilon(\sigma)\big( \prod_{i\neq j}A_{i\sigma(i)} \big)(A_{j\sigma(j)}+v_{\sigma(j)})\\
            &=\sum_{\sigma}\epsilon(\sigma)\prod_iA_{i\sigma(i)}+\sum_{\sigma}\epsilon(\sigma)\prod_{i\neq j}C_{i\sigma(i)}v_{\sigma(j)}         \label{SUBEQooKATCooVIbEpv}\\
            &=\det(A)+\sum_{\sigma}\epsilon(\sigma)\prod_{i\neq j}C_{i\sigma(i)}C_{j\sigma(j)}  \label{SUBEQooCOTDooPPrEYJ}\\    
            &=\det(A)+\det(C).
        \end{align}
    \end{subequations}
    Justifications :
    \begin{itemize}
        \item \ref{SUBEQooKATCooVIbEpv} parce que pour \( i\neq j\) nous avons \( A_{i\sigma(i)}=C_{i\sigma(i)}\)
        \item \ref{SUBEQooCOTDooPPrEYJ} parce que \( v_{\sigma(j)}=C_{j\sigma(j)}\).
    \end{itemize}
\end{proof}

\begin{proposition}[Combinaison de lignes ou colonnes \( L_k\to L_k+\lambda L_l\)\cite{ooKBOMooSkKHvu}]     \label{PROPooPYNHooLbeVhj}
    Soient une matrice \( A\in \eM(n,\eK)\), deux entiers \( k\neq l\) inférieurs ou égaux à \( n\).
    \begin{enumerate}
        \item       \label{ITEMooJSRDooTggEyO}
            Si \( B\) est la matrice obtenue à partir de \( A\) par la substitution \( L_k\to L_k+\lambda L_l\) ou \( C_k\to C_k+\lambda C_l\), alors
            \begin{equation}
                \det(A)=\det(B).
            \end{equation}
        \item   \label{ITEMooHKZWooVZDgnf}
            Si \( B\) est la matrice \( A\) dans laquelle nous avons fait la substitution \( L_k\to L_k+\lambda L_l\), alors
            \begin{equation}
                B=UA
            \end{equation}
            avec \( U=\delta+\lambda E_{kl}\), c'est-à-dire que \( U\) est une matrice de combinaison de lignes.
        \item           \label{ITEMooPGYJooWTTghT}
            La matrice \( U\) vérifie \( \det(U)=1\).
        \item       \label{ITEMooBBEAooZJVGNV}
            Nous avons 
            \begin{equation}
                \det(UA)=\det(U)\det(A).
            \end{equation}
    \end{enumerate}
\end{proposition}

\begin{proof}
    Point par point.
    \begin{subproof}
    \item[\ref{ITEMooJSRDooTggEyO}]
        Soit la matrice \( C\) obtenue à partir de \( A\) par \( L_k\to \lambda L_l\). En considérant le vecteur \( v=\lambda L_l\), nous sommes dans la situation de la proposition \ref{PROPooNGZJooHjtMyn}. Donc
        \begin{equation}
            \det(B)=\det(A)+\det(C).
        \end{equation}
        Mais dans la matrice \( C\), nous avons \( L_k=\lambda L_l\), ce qui implique \( \det(C)=0\) par le corolaire \ref{CORooAZFCooSYINvBl}. Donc \( \det(A)=\det(B)\) comme il se devait.
    \item[\ref{ITEMooHKZWooVZDgnf}]
        Encore un calcul :
        \begin{equation}
            (UA)_{ij}=\sum_m\big( \delta_{im}+\lambda(E_{kl})_{im} \big)A_{mj}=A_{ij}+\lambda\sum_m\delta_{ki}\delta_{lm}A_{mj}=A_{ij}+\lambda \delta_{li}A_{kj}.
        \end{equation}
        Cela donne, pour \( i=k\) la ligne
        \begin{equation}
            (UA)_{kj}=A_{kj}+\lambda A_{lj},
        \end{equation}
        ce qui correspond bien à \( L_k\to L_k+\lambda L_l\).

    \item[\ref{ITEMooPGYJooWTTghT}]
            Nous calculons le déterminant de \( U=\delta+\lambda E_{kl}\) avec \( k\neq l\). Nous avons dans un premier temps :
            \begin{equation}
                \det(U)=\sum_{\sigma\in S_n}\epsilon(\sigma)\prod_{i=1}^n(\delta_{i\sigma(i)}+\lambda \delta_{ki}\delta_{l\sigma(i)}).
            \end{equation}
            Vu que nous avons toujours \( \delta_{ki}\delta_{li}=0\), le terme \( \sigma=\id\) donne \( 1\).

            Pour les \( \sigma\neq \id\), le facteur \( \lambda\delta_{ki}\delta_{l\sigma(i)}\) ne s'annulle pas uniquement si \( i=k\) et \( \sigma(i)=k\). Donc le seul terme non nul autre que \( \sigma=\id\) peut provenir de \( \sigma=(k,l)\). Pour ce terme, nous isolons les termes \( i=l\) et \( i=k\) :
            \begin{equation}
                (\delta_{k\sigma(k)}+\lambda\delta_{kk}\delta_{k\sigma(k)})(\delta_{l\sigma(l)}+\lambda\delta_{kl}\delta_{k\sigma(l)}).
            \end{equation}
            Le dernier facteur est nul.
        \item[\ref{ITEMooBBEAooZJVGNV}]
            En mettant bout à bout les résultats prouvés,
            \begin{equation}
                \det(UA)=\det(A)=\det(U)\det(A).
            \end{equation}
    \end{subproof}
\end{proof}

\begin{proposition}[Multiplication par un scalaire d'une ligne ou colonne \( L_k\to \lambda L_k\)\cite{ooKBOMooSkKHvu}] \label{PROPooXUFKooOaPnna}
    Soient une matrice \( A\in \eM(n,\eK)\), un entier \( k\neq l\) inférieurs ou égal à \( n\). Soit la matrice \( B\) obtenue à partir de \( A\) en multipliant la ligne \( L_k\) par \( \lambda\in \eK\).
    \begin{enumerate}
        \item       \label{ITEMooBKIGooCDQEDt}
            \( \det(B)=\lambda\det(A)\)
        \item       \label{ITEMooWRRCooFXkRNW}
            En considérant la matrice \( T=\delta+(\lambda-1)E_{kk}\), nous avons
            \begin{equation}
                B=TA,
            \end{equation}
            c'est-à-dire que a matrice \( T\) est une matrice de multiplication de ligne par un scalaire.
        \item       \label{ITEMooOGGDooPVVRzk}
            Nous avons \( \det(T)=\lambda\).
        \item       \label{ITEMooIFRVooWQYgkK}
            Et aussi : \( \det(TA)=\det(T)\det(A)\)
    \end{enumerate}
\end{proposition}

\begin{proof}
    Point par point.
    \begin{subproof}
        \item[\ref{ITEMooBKIGooCDQEDt}]
            La matrice \( B\) est donnée par les éléments
            \begin{equation}
                B_{ij}=\begin{cases}
                    A_{ij}    &   \text{si } j\neq k\\
                    \alpha A_{ij}    &    \text{si } j=kn
                \end{cases}
            \end{equation}
            c'est-à-dire \( B_{ij}=\big( 1+(\alpha-1)\delta_{jk} \big)A_{ij}\). Nous mettons cela dans la définition du déterminant de \( B\) :
            \begin{equation}        \label{EQooGVMTooPntKew}
                \det(B)=\sum_{\sigma\in S_n}\epsilon(\sigma)\prod_{i=1}^nB_{i\sigma(i)}=\sum_{\sigma}\prod_i\big( 1+(\alpha-1)\delta_{\sigma(i)k}A_{i\sigma(i)} \big).
            \end{equation}
            L'associativité du produit dans \( \eK\) nous permet de séparer le produit de la façon suivante :
            \begin{equation}
                \prod_{i=1}^n\big( 1+(\alpha-1)\delta_{\sigma(i)k} \big)A_{i\sigma(i)}=\prod_i\big( 1+(\lambda-1)\delta_{\sigma(i)k} \big)\prod_iA_{i\sigma(i)}=\lambda\prod_iA_{i\sigma(i)}.
            \end{equation}
            En remettant dans \eqref{EQooGVMTooPntKew}, nous trouvons \( \det(B)=\det(A)\).
        \item[\ref{ITEMooWRRCooFXkRNW}]
            C'est un cas particulier de la proposition \ref{PROPooPYNHooLbeVhj}\ref{ITEMooHKZWooVZDgnf} en prenant \( k=l\) et en adaptant le \( \lambda\).
        \item[\ref{ITEMooOGGDooPVVRzk}]
            Nous calculons le déterminant de la matrice \( T=\delta+(\lambda-1)E_{kk}\). La formule du déterminant donne
            \begin{equation}
                \det(T)=\sum_{\sigma}\epsilon(\sigma)\prod_{i=1}^n\big( \delta_{i\sigma(i)}+(\lambda-1)\delta_{ki}\delta_{k\sigma(i)} \big).
            \end{equation}
            Si \( i\neq \sigma(i)\), alors non seulement \( \delta_{i\sigma(i)}=0\), mais en plus \( \delta_{ki}\delta_{k\sigma(i)}=0\). Donc suel \( \sigma=\id\) reste dans la somme sur \( \sigma\in S_n\). Il reste donc
            \begin{equation}
                \det(T)=\prod_{i=1}^n\big( 1+(\lambda-1)\delta_{ki} \big)=\left( \prod_{i\neq k}1 \right)(1+(\lambda-1))=\lambda
            \end{equation}
            où nous avons utilisé encore l'associativité pour isoler le facteur \( i=k\).
        \item[\ref{ITEMooIFRVooWQYgkK}]
            Il faut mettre bout à bout les résultats déjà faits :
            \begin{equation}
                \det(TA)=\lambda\det(A)=\det(T)\det(A).
            \end{equation}
    \end{subproof}
\end{proof}

%--------------------------------------------------------------------------------------------------------------------------- 
\subsection{Réduction de Gauss}
%---------------------------------------------------------------------------------------------------------------------------

Nous avons vu les matrices d'opérations élémentaire sur les lignes et colonnes :
\begin{itemize}
    \item Permutation de lignes \( L_k\leftrightarrow L_l\)  : \( S(n;k,l)=\delta+E_{kl}+E_{lk}-E_{kk}-E_{ll}\), proposition \ref{PROPooFQRDooRPfuxk}.
    \item Combinaisons de lignes \( L_k\to L_k+\lambda L_l\) : \( U(n;k,l,\lambda)=\delta+\lambda E_{kl}\), proposition \ref{PROPooPYNHooLbeVhj}.
    \item Multiplication d'une ligne par un scalaire \( L_k\to \lambda L_k\) : \( T=\delta+(\lambda-1)E_{kk}\), proposition \ref{PROPooXUFKooOaPnna}.
\end{itemize}

Ces matrices seront dans la suite notées \( G\). Et elles vérifient la grosse propriété
\begin{equation}        \label{EQooLQTVooBYjVYl}
    \det(GA)=\det(G)\det(A)
\end{equation}
pour toute matrice \( A\).

\begin{proposition}[Réduction de Gauss\cite{MonCerveau}]        \label{PROPooJBTZooNLobpf}
    Soit une matrice \( A\in \eM(n,\eK)\) de déterminant non nul : \( \det(A)\neq 0\). Alors il existe des matrices \( G_1,\ldots, G_N\) toutes de type \( S\), \( U\) ou \( T\) telles que
    \begin{equation}
        G_1\ldots G_NA=\delta.
    \end{equation}
\end{proposition}

\begin{proof}
    Nous faisons une récurrence sur \( n\). D'abord pour \( n=1\), la matrice \( A\) contient un seul élément \( A_{11}\) qui est non nul par hypothèse. Nous pouvons multiplier sa ligne par \( 1/A_{11}\) pour obtenir le résultat. Plus précisément, nous avons l'égalité
    \begin{equation}
        T(1;1,\frac{1}{ A_{11} })A=\delta
    \end{equation}
    dans \( \eM(1, \eK)\). Notons que \( \eK\) est un corps (donc \( A_{11}\) est inversible) commutatif, ce qui permet d'écrire \( 1/A_{11}\) sans ambiguïtés.

    Supposons le résultat prouvé pour \( n\), et voyons ce qu'il se passe pour \( n+1\). Vu que \( \det(A)\neq 0\), aucune de ses colonnes n'est nulle (corolaire \ref{CORooAZFCooSYINvBl}). Il existe donc un \( k\) tel que \( A_{k1}\neq 0\).

    Par la proposition \ref{PROPooFQRDooRPfuxk}, la matrice
    \begin{equation}
        B^{(1)}=S(n+1;k,1)A
    \end{equation}
    est une matrice telle que \( B^{(1)}_{11}=A_{k1}\neq 0\). Ensuite, par la proposition \ref{PROPooXUFKooOaPnna} la matrice
    \begin{equation}
        B^{(2)}=T(n+1;1,\frac{1}{ A_{k1} })B^{(1)}
    \end{equation}
    vérifie \( B^{(2)}_{11}=1\). 

    Vu que la multiplication par la matrice \( U(n+1;k;l;\lambda)\) fait par la proposition \ref{PROPooPYNHooLbeVhj} la substitution \( L_k\to L_{k}+\lambda L_l\), la matrice
    \begin{equation}
        B^{(3)}=\prod_{k=2}^{n+1}U(n+1;k,1,-B^{(1)}_{k1})B^{(1)}
    \end{equation}
    a toute sa première colonne nulle à l'exception de \( B^{(3)}_{11}=1\).

    Nous n'avons pas donné de nom ni démontré de théorèmes à propos de la substitution \( C_k\to C_k+\lambda C_l\). En passant éventuellement par les transposées et en utilisant les lemmes \ref{LEMooRSJTooQEoOtN} et \ref{LEMooCEQYooYAbctZ} nous obtenons une matrice \( U'(n+1;k,l,\lambda)\) ayant la propriété que la matrice
    \begin{equation}
        B^{(4)}=\prod_{k=2}^{n+1}U'(n+1;k,1,-B^{(3)}_{1k})B^{(3)}
    \end{equation}
    vérifie \( B^{(4)}_{1j}=B^{(4)}_{j1}=0\) pour tout \( j\) sauf \( j=1\). En d'autres termes, la matrice \( B^{(4)}\) est de la forme
    \begin{equation}
        B^{(4)}=\begin{pmatrix}
            1    &   \begin{matrix} 
                0    &   \ldots    &   0    
            \end{matrix}\\ 
            \begin{matrix}
                0    \\ 
                \vdots    \\ 
                0    
            \end{matrix}&   \begin{pmatrix}
                    &       &       \\
                    &   A'    &       \\
                    &       &   
            \end{pmatrix}
        \end{pmatrix}
    \end{equation}
    où \( A'\) est une matrice de taille \( n\).
    
    Voyons quelques propriétés de \( A'\). Nous savons que
    \begin{equation}
        B^{(4)}=\prod_i G_iA
    \end{equation}
    où les \( G_i\) sont de type \( S\), \( T\) ou \( U\). Vu que \( \det(SA)=\det(S)\det(A)\) (et idem pour \( T\) et \( U\)), nous avons
    \begin{equation}
        \det(B^{(4)})=\prod_i\det(G_i)\det(A),
    \end{equation}
    et comme aucun des \( \det(G_i)\) n'est nul, nous avons encore \( \det(B^{(4)})\neq 0\), ce qui implique \( \det(A')\neq 0\).

    La récurrence peut avoir lieu. Il existe des matrices \( G'_i\) telles que
    \begin{equation}
        G'_1\ldots G'_MA'=\delta
    \end{equation}
    où les \( G'_i\) sont de taille \( n\), ainsi que le \( \delta\). En remarquant que
    \begin{equation}
       S(n+1;k,l) =\begin{pmatrix}
            1    &   \begin{matrix} 
                0    &   \ldots    &   0    
            \end{matrix}\\ 
            \begin{matrix}
                0    \\ 
                \vdots    \\ 
                0    
            \end{matrix}& S(n;k-1,l-1)
        \end{pmatrix},
    \end{equation}
    et pareillement pour les matrices \( T\) et \( U\), nous voyons qu'en prenant
    \begin{equation}
       G_i =\begin{pmatrix}
            1    &   \begin{matrix} 
                0    &   \ldots    &   0    
            \end{matrix}\\ 
            \begin{matrix}
                0    \\ 
                \vdots    \\ 
                0    
            \end{matrix}& G'_i
        \end{pmatrix},
    \end{equation}
    nous avons
    \begin{equation}
        \prod_{i=1}^MG_iB^{(3)}=
      \begin{pmatrix}
            1    &   \begin{matrix} 
                0    &   \ldots    &   0    
            \end{matrix}\\ 
            \begin{matrix}
                0    \\ 
                \vdots    \\ 
                0    
            \end{matrix}& \prod_{i=1}^MG'_iA'
        \end{pmatrix}=\delta_{n+1}
    \end{equation}
    où nous avons mis un indice sur le dernier \( \delta\) pour être plus explicite.
\end{proof}

\begin{proposition} \label{PROPooPMYCooAAtHsB}
    Si \( A\in \eM(n,\eK)\) est telle que \( \det(A)=0\), alors il existe des matrices de manipulation de lignes et de colonnes \( G_1,\ldots, G_N\) telles que \( G_1\ldots G_NA\) ait une colonne de zéros.
\end{proposition}

\begin{proof}
    Si la matrice \( A\) elle-même n'a pas de colonnes de zéros, alors nous pouvons faire un pas de réduction de Gauss et obtenir des matrices \( G_1,\ldots,  G_{N_1}\) telles que
    \begin{equation}
        G_1\ldots G_{N_1}A=
      \begin{pmatrix}
            1    &   \begin{matrix} 
                0    &   \ldots    &   0    
            \end{matrix}\\ 
            \begin{matrix}
                0    \\ 
                \vdots    \\ 
                0    
            \end{matrix}& A^{(1)}
        \end{pmatrix}.
    \end{equation}
    Si \( A^{(1)}\) ne possède pas de colonnes de zéros, nous pouvons continuer. 

    Si nous parvenons à faire \( n\) pas de la sorte, alors nous aurions
    \begin{equation}
        G_1\ldots G_NA=\delta,
    \end{equation}
    et donc \( \det(G_1\ldots G_N)\det(A)=1\), ce qui est impossible lorsque \( \det(A)=0\). Nous en concluons que le processus doit s'arrêter et qu'une des matrices \( A^{(k)}\) doit avoir une colonne de zéros\footnote{En réalité, le processus tel que nous l'avons décrit ne s'arrête que lorsque la première colonne est remplie de zéros.}.
\end{proof}

%--------------------------------------------------------------------------------------------------------------------------- 
\subsection{Matrices inversibles}
%---------------------------------------------------------------------------------------------------------------------------

\begin{propositionDef}      \label{PROPooMLWRooRWfZXE}
    Soit une matrice \( A\in \eM(n,\eK)\). Si les matrices \( B_1\) et \( B_2\) de \( \eM(n,\eK)\) vérifient
    \begin{equation}
        AB_1=B_1A=\delta
    \end{equation}
    et
    \begin{equation}
        AB_2=B_2A=\delta,
    \end{equation}
    alors \( B_1=B_2\). Dans ce cas, nous disons que \( A\) est inversible et nous notons \( A^{-1}\) l'unique matrice telle que \( AA^{-1}=A^{-1}A=\delta\).
\end{propositionDef}

\begin{proof}
    La preuve est réalisée dans le cas général par le lemme \ref{LEMooECDMooCkWxXf}. Mais si vous en voulez une preuve avec les notations d'ici, en voici une.

    Nous avons $AB_1=AB_2$. En multipliant à gauche par \( B_1\), nous trouvons \( B_1AB_1=B_1AB_2\). En remplaçant \( B_1A\) par \( \delta\) des deux côtés, il reste \( B_1=B_2\).
\end{proof}

\begin{lemma}[\cite{ooKBOMooSkKHvu}]        \label{LEMooGZCTooQigDvC}
    Si \( A\in \eM(n,\eK)\), alors il existe au plus une matrice \( B\in \eM(n,\eK)\) telle que \( AB=\delta\).
\end{lemma}

\begin{proof}
    Soient des matrices \( B,C\in \eM(n,\eK)\) telles que \( AB=AC=\delta\). Nous allons montrer que \( B=C\).

    Pour cela nous considérons les applications linéaires \( f_A, f_B, f_C\in \End(\eK^)\) associées par la proposition \ref{PROPooGXDBooHfKRrv}. Vu que \( AB=\delta\), par la proposition \ref{PROPooCSJNooEqcmFm}, nous avons \( f_A\circ f_B = f_{AB}=\id\). La proposition \ref{PROPooADESooATJSrH} nous dit alors que \( f_A\) et \( f_B\) sont bijectives. 

    En particulier, vu que \( \{e_i\}\) est une base, son image par \( f_B\) est une base par la proposition \ref{PROPooZFKZooBGLSex}. La proposition \ref{PROPooHLUYooNsDgbn} dit alors que \( \{f_B(e_i)\}\) est une base. Nous décomposons \( f_B(e_k)-f_C(e_k)\) dans cette base :
    \begin{equation}
        f_B(e_k)-f_C(e_k)=\sum_j\alpha_jf_B(e_j)
    \end{equation}
    où les \( \alpha_j\) dépendent à priori de \( k\). Vu que \( f_A\circ(f_B-f_C)=0\), nous avons 
    \begin{equation}
        0=f_A\big( f_B(e_k)-f_C(e_k) \big)=\sum_j(f_A\circ f_B)(e_j)=\sum_j\alpha_je_j.
    \end{equation}
    Donc les \( \alpha_j\) sont tous nuls.

    Nous en déduisons que \( f_B(e_k)=f_C(e_k)\), et donc \( f_B=f_C\). Cela implique que \( B=C\) par la proposition \ref{PROPooGXDBooHfKRrv}\ref{ITEMooHSMLooRJZref}.
\end{proof}

\begin{proposition}[\cite{ooKBOMooSkKHvu}]      \label{PROPooECIIooVMCIwz}
    Si \( A,B\in \eM(n,\eK)\) vérifient \( AB=\delta\), alors \( BA=\delta\).
\end{proposition}

\begin{proof}
    L'astuce est de poser \( C=BA-\delta+B\) et de montrer que \( C=B\). Pour cela, un rapide calcul commence par montrer que
    \begin{equation}
        AC=ABA-A+AB=AB=\delta.
    \end{equation}
    Donc \( C\) est également un inverse à droite de \( A\). Le lemme \ref{LEMooGZCTooQigDvC} donne alors \( C=B\).
\end{proof}

\begin{corollary}       \label{CORooBQLXooTeVfgb}
    Soit \( A\in \eM(n,\eK)\). Si il existe \( B\in \eM(n,\eK)\) tel que \( AB=\delta\), alors \( A\) est inversible et son inverse est \( B\).
\end{corollary}

\begin{proof}
    Il s'agit d'une paraphrase de la proposition \ref{PROPooECIIooVMCIwz} et de la définition \ref{PROPooMLWRooRWfZXE}.
\end{proof}

\begin{lemma}       \label{LEMooZDNVooArIXzC}
    Si une matrice \( A\) n'est pas inversible, alors le produit \( AB\) n'est inversible pour aucune matrice \( B\).
\end{lemma}

\begin{proof}
    Supposons que \( AB\) soit inversible. Alors
    \begin{equation}
        AB(AB^{-1})=\delta,
    \end{equation}
    ce qui dirait que \( B(AB^{-1})\) serait un inverse de \( A\).
\end{proof}

\begin{proposition}     \label{PROPooNPMCooPmaCwu}
    Une matrice est inversible si et seulement si son application linéaire associée est inversible. Dans ce cas, nous avons
    \begin{equation}
        f_A^{-1}=f_{A^{-1}}.
    \end{equation}
\end{proposition}

\begin{proof}
    Dans le sens direct, si \( A\) est inversible nous avons \( AA^{-1}=\delta\). Donc
    \begin{equation}        \label{EQooQQOSooBKVqXh}
        f_A\circ f_{A^{-1}}=f_{AA^{-1}}=f_{\delta}=\id
    \end{equation}
    où nous avons utilisé la proposition \ref{PROPooCSJNooEqcmFm} pour la composition et la proposition \ref{PROPooFMBFooEVCLKA} pour l'identité. L'égalité \eqref{EQooQQOSooBKVqXh} indique que \( f_A\) est inversible et que son inverse est \( f_{A^{-1}}\).

    Dans l'autre sens, l'application \( f_A^{-1}\) existe. Soit \( B\in \eM(n,\eK)\) sa matrice. Alors nous avons
    \begin{equation}
        f_{\delta}=\id=f_A\circ f_B=f_{AB}.
    \end{equation}
    Le fait que l'application \(\psi\colon A\to f_A\) soit une bijection\footnote{Proposition \ref{PROPooGXDBooHfKRrv}\ref{ITEMooHSMLooRJZref}.} implique que \( AB=\delta\), c'est-à-dire que \( A\) est inversible et que \( B=A^{-1}\).
\end{proof}

%--------------------------------------------------------------------------------------------------------------------------- 
\subsection{Inversibilité et déterminant}
%---------------------------------------------------------------------------------------------------------------------------

\begin{proposition}     \label{PROPooAVIXooMtVCet}
    Une matrice au déterminant non nul est inversible.
\end{proposition}

\begin{proof}
    Si \( A\) est une matrice telle que \( \det(A)\neq 0\), alors la proposition \ref{PROPooJBTZooNLobpf} nous donne des matrices \( G_1,\ldots, G_N\) telles que
    \begin{equation}
        G_1\ldots G_NA=\delta.
    \end{equation}
    Donc la matrice \( G_1\ldots G_N\) est un inverse de \( A\) par le corolaire \ref{CORooBQLXooTeVfgb}.
\end{proof}

\begin{proposition}     \label{PROPooEOKBooKUROFg}
    Si une matrice \( A\) a une ligne ou une colonne de zéros, alors
    \begin{enumerate}
        \item
            \( \det(A)=0\),
        \item
            \( A\) n'est pas inversible.
    \end{enumerate}
\end{proposition}

\begin{proof}
    Par définition nous avons
    \begin{equation}
        \det(A)=\sum_{\sigma\in S_n}\epsilon(\sigma)\prod_{i=1}^nA_{i\sigma(i)}.
    \end{equation}
    Si la \( k\)\ieme ligne est nulle, alors \( A_{k\sigma(k)}=0\) pour tout \( \sigma\). Donc tous les produits contiennent un facteur nul. Donc \( \det(A)=0\).

    Pour toute matrice \( B\) nous avons
    \begin{equation}
        (AB)_{kk}=\sum_lA_{kl}B_{lk}.
    \end{equation}
    Si la \( k\)\ieme ligne de \( A\) est nulle nous avons \( (AB)_{kk}=0\) et donc pas \( AB=\delta\). Donc \( A\) n'est pas inversible.
\end{proof}

\begin{proposition}     \label{PROPooVUDJooLWjmSI}
    Une matrice dont le déterminant est nul n'est pas inversible.
\end{proposition}

\begin{proof}
    Par la proposition \ref{PROPooPMYCooAAtHsB}, il existe des matrices de manipulation de lignes et de colonnes \( G_1,\ldots, G_N\) telles que la matrice \( G_1\ldots G_NA\) ait une colonne de zéros. De là, la proposition \ref{PROPooEOKBooKUROFg} implique que la matrice
    \begin{equation}        \label{EQooQGXBooXxFOtb}
        G_1\ldots G_NA
    \end{equation}
    n'est pas inversible. Vu les déterminants des matrices \( G_i\),  la proposition \ref{PROPooAVIXooMtVCet} implique que \( G_1\ldots G_N\) est inversible. Si \( A\) était inversible, nous aurions
    \begin{equation}
        G_1\dots G_NAA^{-1}(G_1\ldots G_N)^{-1}=\delta,
    \end{equation}
    c'est-à-dire que \( A^{-1}(G_1\ldots G_N)^{-1}\) serait un inverse de la matrice \eqref{EQooQGXBooXxFOtb}. Cette dernière n'ayant pas d'inverse, nous concluons que \( A\) n'en a pas non plus.
\end{proof}

\begin{theorem}     \label{THOooSNXWooSRjleb}
    Une matrice sur un corps commutatif est inversible si et seulement si son déterminant est non nul.
\end{theorem}

\begin{proof}
    Dans un sens c'est la proposition \ref{PROPooAVIXooMtVCet} et dans l'autre sens c'est la proposition \ref{PROPooVUDJooLWjmSI}.
\end{proof}

%---------------------------------------------------------------------------------------------------------------------------
\subsection{Quelques ensembles de matrices particuliers}
%---------------------------------------------------------------------------------------------------------------------------
Certains ensembles de matrices ont une importance particulière, que nous développerons plus tard.

\begin{definition}[Groupe linéaire de matrices]
On note \( \GL(n,\eA) \) l'ensemble des matrices carrées d'ordre \( n \) à coefficients dans \( \eA \), qui sont inversibles. En d'autres termes, \( \GL(n,\eA) = U (\eM (n,\eA) ) \).
\end{definition}

\begin{definition}[Groupe orthogonal de matrices]\label{DefMatriceOrthogonale}
    On dit qu'une matrice \( A \) est \defe{orthogonale}{matrice!orthogonale} si son inverse est sa transposée, c'est-à-dire si \( A^{-1} = A^t \). On note \( \gO(n,\eA) \) l'ensemble des matrices carrées d'ordre \( n \) à coefficients dans \( \eA \), qui sont orthogonales.
\end{definition}

%--------------------------------------------------------------------------------------------------------------------------- 
\subsection{Déterminant et combinaisons de lignes et colonnes}
%---------------------------------------------------------------------------------------------------------------------------
\label{SUBSECooKMSVooBBHwkH}

\begin{proposition}     \label{PROPooUCZVooPkloQp}
    Soient des matrices \( A,B\in \eM(n,\eK)\) telles que \( \det(A)\neq 0\neq \det(B)\). Alors
    \begin{equation}
        \det(AB)=\det(A)\det(B).
    \end{equation}
\end{proposition}

\begin{proof}
    La proposition \ref{PROPooJBTZooNLobpf} nous donne des matrices de permutations de lignes et de colonnes \( G_1,\ldots, G_N\) et \( G'_1,\ldots, G'_N\) telles que\footnote{Les plus acharnés préciseront que pour avoir le même \( N\) des deux côtés, il a fallu compléter avec des matrices \( \delta\) là où il y en avait le moins.}
    \begin{subequations}        \label{EQooDNZUooHBhcZj}
        \begin{align}
            G_1\ldots G_NA&=\delta\\
            G'_1\ldots G'_NB&=\delta.
        \end{align}
    \end{subequations}
    Nous avons
    \begin{equation}
        (G'_1\ldots G'_N)\underbrace{(G_1\ldots G_N)A}_{=\delta}B=\delta.
    \end{equation}
    En prenant le déterminant des deux côtés et en tenant compte de \eqref{EQooLQTVooBYjVYl},
    \begin{equation}
        1=\det(\delta)=\det\big(  G'_1\ldots G'_NG_1\ldots G_NAB\big)=\det(G_1'\ldots G_N')\det(G_1\ldots G_N)\det(AB).
    \end{equation}
    Mais en même temps, les équations \ref{EQooDNZUooHBhcZj} donnent
    \begin{subequations}
        \begin{align}
            \det(G_1\ldots G_N)=\det(A)^{-1}\\
            \det(G_1'\ldots G'_N)=\det(B)^{-1}.
        \end{align}
    \end{subequations}
    Cela pour dire que
    \begin{equation}
        1=\det(A)^{-1}\det(B)^{-1}\det(AB),
    \end{equation}
    et donc ce qu'il nous fallait.
\end{proof}

\begin{proposition}     \label{PROPooWVJFooTmqoec}
    Soient des matrices \( A,B\in \eM(n,\eK)\) telles que \( \det(A)=0\) et \( \det(B)\neq0\). Alors
    \begin{equation}
        \det(AB)=\det(BA)=\det(A)\det(B).
    \end{equation}
\end{proposition}

\begin{proof}
    Il existe des matrices de manipulations de lignes et de colonnes \( G_1,\ldots, G_N\) telles que \( G_1\ldots G_NB=\delta\). Donc
    \begin{equation}
        0=\det(A)=\det(G_1\ldots G_NBA)=\det(G_1\ldots G_N)\det(BA).
    \end{equation}
    Donc \( \det(BA)=0\).
\end{proof}

\begin{proposition}     \label{PROPooHQNPooIfPEDH}
    Soient des matrices \( A\) et \( B\) sur un corps commutatif. Alors
    \begin{equation}
        \det(AB)=\det(A)\det(B).
    \end{equation}
\end{proposition}

\begin{proof}
    Les propositions \ref{PROPooUCZVooPkloQp} et \ref{PROPooWVJFooTmqoec} ont déjà fait une grosse partie du travail. Il ne reste que le cas où \( \det(A)=\det(B)=0\).

    Dans ce cas, les matrices \( A\) et \( B\) ne sont pas inversibles (proposition \ref{THOooSNXWooSRjleb}). Le produit \( AB\) n'est alors pas inversible non plus\footnote{Citez le lemme \ref{LEMooZDNVooArIXzC} si vous voulez justifier ça.}. La proposition \ref{THOooSNXWooSRjleb}, utilisée dans le sens inverse, nous dit alors que \( \det(AB)=0\).

    Au final dans le cas \( \det(A)=\det(B)=0\) nous avons \( 0=\det(AB)=\det(A)\det(B)=0\).
\end{proof}

Faisons maintenant le cas général des manipulations de lignes et colonnes.

\begin{proposition}     \label{PROPooSLLGooSZjQrv}
    Soit une matrice carré \( A\in \eM(n,\eK)\). La matrice \( B\) obtenue par la substitution simultanée
    \begin{equation}
        C_j\to \sum_ka_{kj}C_k
    \end{equation}
    a pour déterminant
    \begin{equation}
        \det(B)=\det(a)\det(A).
    \end{equation}
\end{proposition}

\begin{proof}
    L'élément \( B_{ij}\) de la matrice \( B\) est une combinaison linéaire de tous les éléments de sa ligne :
    \begin{equation}
        B_{ij}=\sum_ka_{kj}A_{ik}=(Aa)_{ij}.
    \end{equation}
    Donc \( B=Aa\). La proposition \ref{PROPooHQNPooIfPEDH} nous dit alors que \( \det(B)=\det(a)\det(A)\).
\end{proof}

%---------------------------------------------------------------------------------------------------------------------------
\subsection{Transvections}
%---------------------------------------------------------------------------------------------------------------------------

Nous nommons \( E_{ij}\) la matrice remplie de zéros sauf à la case \( ij\) qui vaut \( 1\). Autrement dit
\begin{equation}
    (E_{ij})_{kl}=\delta_{ik}\delta_{jl}.
\end{equation}
\begin{definition}
    Une \defe{matrice de transvection}{transvection (matrice)}\index{matrice!de transvection} est une matrice de la forme
    \begin{equation}
        T_{ij}(\lambda)=\id+\lambda E_{ij}
    \end{equation}
    avec \( i\neq j\).

    Une \defe{matrice de dilatation}{matrice!de dilatation}\index{dilatation (matrice)} est une matrice de la forme
    \begin{equation}
        D_i(\lambda)=\id+(\lambda-1)E_{ii}.
    \end{equation}
    Ici le \( (\lambda-1)\) sert à avoir \( \lambda\) et non \( 1+\lambda\). C'est donc une matrice qui dilate d'un facteur \( \lambda\) la direction \( i\) tout en laissant le reste inchangé.

    Si \( \sigma\) est une permutation (un élément du groupe symétrique \( S_n\)) alors la \defe{matrice de permutation}{matrice!de permutation}\index{permutation!matrice} associée est la matrice d'entrées
    \begin{equation}
        (P_{\sigma})_{ij}=\delta_{i\sigma(j)}.
    \end{equation}
\end{definition}

\begin{lemma}   \label{LemyrAXQs}
    La matrice \( T_{ij}(\lambda)A=(\mtu+\lambda E_{ij})A\) est la matrice \( A\) à qui on a effectué la substitution
    \begin{equation}
        L_i\to L_i+\lambda L_j.
    \end{equation}
    La matrice \( AT_{ij}(\lambda)\) est la substitution
    \begin{equation}
        C_j\to C_j+\lambda C_i.
    \end{equation}

    La matrice \( AP_{\sigma}\) est la matrice \( A\) dans laquelle nous avons permuté les colonnes avec \( \sigma\).

    La matrice \( P_{\sigma}A\) est la matrice \( A\) dans laquelle nous avons permuté les lignes avec \( \sigma^{-1}\).
\end{lemma}

\begin{proof}
    Calculons la composante \( kl\) de la matrice \( E_{ij}A\) :
    \begin{subequations}
        \begin{align}
            (E_{ij}A)_{kl}&=\sum_m(E_{ij})_{km}A_{ml}\\
            &=\sum_m\delta_{ik}\delta_{jm}A_{ml}\\
            &=\delta_{ik}A_{jl}.
        \end{align}
    \end{subequations}
    C'est donc la matrice pleine de zéros, sauf la ligne \( i\) qui est donnée par la ligne \( j\) de \( A\). Donc effectivement la matrice
    \begin{equation}
        A+\lambda E_{ij}A
    \end{equation}
    est la matrice \( A\) à laquelle on a substitué la ligne \( i\) par la ligne \( i\) plus \( \lambda\) fois la ligne \( j\).

    En ce qui concerne l'autre assertion sur les transvections, le calcul est le même et nous obtenons
    \begin{equation}
        (AE_{ij})=A_{ki}\delta_{jl}.
    \end{equation}

    Pour les matrices de permutation, nous avons
    \begin{equation}
        (AP_{\sigma})_{kl}=A_{k\sigma(l)}
    \end{equation}
    et
    \begin{equation}
        (P_{\sigma}A)_{kl}=\sum_m\delta_{k\sigma(m)}A_{ml}=\sum_m\delta_{\sigma^{-1}(k)m}A_{ml}=A_{\sigma^{-1}(k)l}.
    \end{equation}
\end{proof}

%---------------------------------------------------------------------------------------------------------------------------
\subsection{Mineur, rang}
%---------------------------------------------------------------------------------------------------------------------------

Pour la définition du rang d'une matrice, nous en donnons une qui est clairement inspirée de l'application linéaire associée.
\begin{definition}[\cite{ooKBOMooSkKHvu}]       \label{DEFooCSGXooFRzLRj}
    Le \defe{rang}{rang d'une matrice} d'une matrice de \( \eM(n,\eK)\) est la dimension de la partie de \( \eK^n\) engendrée par ses colonnes.
\end{definition}

Il est possible d'exprimer le rang d'une matrice de façon plus «intrinsèque» via le concept de mineur.
\begin{definition}[\cite{ooLFZTooJWJUed}]
    Les mineurs d'une matrice sont les déterminants de ses sous-matrices carrées.
\end{definition}
Dans la suite nous désignerons souvent par le mot «mineur» la sous-matrice carrée elle-même au lieu de son déterminant.

\begin{proposition}      \label{DEFooVVBYooJbliTi}
    Le rang d'une matrice est la taille de son plus grand mineur non nul.
\end{proposition}

Lorsque \( A\) est une matrice, nous notons \( f_A\) l'application linéaire associée à la matrice \( A\) par l'application \eqref{EQooVZQWooMyFFeO}.
\begin{lemma} \label{LEMVecsaRgFixe}
    Soit \( \eK \) un corps commutatif\footnote{Comme toujours.}. Si \( A \) est une matrice carrée d'ordre \( n \) et de rang \( r \) à coefficients dans \( \eK \), alors il existe des vecteurs \( (x_i)_{i=1,\dots,n} \) formant une base de \( \eK^n \) tels que 
    \begin{equation}
        f_A(x_i)\neq 0
    \end{equation}
    pour \( x\leq r\) et
    \begin{equation}
        f_A(x_i) = 0
    \end{equation}
    pour \( i > r \).
\end{lemma}

\begin{proof}
    Soit \( V\) le sous-espace de \( \eK^n\) engendré par les colonnes de \( A\). Nous considérons la base canonique \( \{ e_i \}\) de \( \eK^n\), ainsi que \( v_i\) le vecteur créé par la \( i\)\ieme colonne de \( A\). Nous avons
    \begin{equation}
        v_i=f_A(e_i).
    \end{equation}
    Les vecteurs \( v_i\) engendrent \( V\), donc nous pouvons en extraire une base par le théorème \ref{ThoMGQZooIgrXjy}\ref{ITEMooTZUDooFEgymQ}. Soit donc \( \{ v_j \}_{i\in J}\) une base de \( V\) avec \( J\subset\{ 1,\ldots, n \}\).

    La base de \( \eK^n\) que nous cherchons commence par les vecteurs \( \{ e_j \}_{j\in J}\). Ces vecteurs vérifient \( f_A(e_j)=v_j\neq 0\) parce que des vecteurs d'une base ne sont jamais nuls.

    % Note : toute la ligne suivante fait des références qui peuvent être vers le futur parce que ce sont des choses qui ne sont 
    %        pas utilisées dans la démonstration.
    Pour la suite de la base, nous pourrions penser au théorème de la base incomplète\footnote{Théorème \ref{ThonmnWKs}\ref{ITEMooFVJXooGzzpOu}.}, mais les vecteurs ainsi complétant la base ne sont pas garantis de s'annuler par \( f_A\). Voir l'exemple \ref{EXooRKVQooZOGDEf}.

    L'idée est d'utiliser le noyau de \( f_A\) qui est un sous-espace vectoriel par la proposition \ref{PROPooRLLPooKYzsJp}. Soit une base\footnote{Cette base contient \( n-r\) éléments, mais ce n'est pas très important pour la suite.} \(  \{ z_k \}  \) de \( \ker(f)\). Les vecteurs \( \{ e_j \}_{j\in J}\) forment une base de \( \Image(f_A)\). Vu que les \( z_i\) forment une base de \( \ker(f_A)\), le théorème du rang \ref{ThoGkkffA} dit alors que \( \{ e_j \}_{j\in J}\cup \{ z_k \}\) est une base de \( \eK^n\).

    Il y a \( r\) éléments dans \( J\) parce que l'espace engendré par les colonnes de \( A\) est de dimension \( r\) par hypothèse. Donc il y a \( n-r\) éléments dans les \( z_k\) pour que le tout ait le bon nombre d'éléments.
\end{proof}

\begin{example}     \label{EXooRKVQooZOGDEf}
    Soit la matrice 
    \begin{equation}
        A=\begin{pmatrix}
            1    &   1    \\ 
            2    &   2    
        \end{pmatrix}.
    \end{equation}
    Elle est de rang \( 1\). En suivant l'idée de la démonstration, nous commençons la base de \( \eR^2\) par le vecteur \( e_1\) qui vérifie
    \begin{equation}
        f_A(e_1)=\begin{pmatrix}
            1    \\ 
            2    
        \end{pmatrix}.
    \end{equation}
    L'utilisation du théorème de la base incomplète ne permet pas de trouver un second vecteur de base \( v\) tel que \( f_A(v)=0\). En effet ce théorème donne juste l'existence d'une completion de la base, mais pas de propriétés particulières de la base obtenue. Elle pourrait donner \( v=e_2\) comme second vecteur de base. Mais alors
    \begin{equation}
        f_A(v)=f_A(e_2)=\begin{pmatrix}
            1    \\ 
            2    
        \end{pmatrix}\neq 0.
    \end{equation}

    Au contraire, le noyau de \( f_A\) est donné par le sous-espace engendré par \( \begin{pmatrix}
        1    \\ 
        -1    
    \end{pmatrix}\). Une base convenable est donc \( \{ e_1, e_1-e_2 \}\).
\end{example}

\begin{proposition}     \label{PROPooEGNBooIffJXc}
    Le rang d'une application linéaire est égal au rang de sa matrice dans n'importe quelle base.
\end{proposition}

%---------------------------------------------------------------------------------------------------------------------------
\subsection{Matrices équivalentes et semblables}
%---------------------------------------------------------------------------------------------------------------------------

\begin{definition}  \label{DefBLELooTvlHoB}
    Deux relations d'équivalence entre les matrices.
    \begin{enumerate}
        \item   \label{ItemPFXCooOUbSCt}
    Deux matrices \( A\) et \( B\) sont \defe{équivalentes}{matrice!équivalence} dans \( \eM(n,\eK)\) s'il existe \( P,Q\in\GL(n,\eK)\) telles que \( A=PBQ^{-1}\).
\item
    Deux matrices sont \defe{semblables}{matrices!similitude} s'il existe une matrice \( P\in \GL(n,\eK)\) telle que \( A=PBP^{-1}\).
    \end{enumerate}
\end{definition}

\begin{lemma}   \label{LemZMxxnfM}
    Une matrice de rang\footnote{Définition~\ref{DEFooVVBYooJbliTi}.} \( r\) dans \( \eM(n,\eK)\) est équivalente à la matrice par blocs
    \begin{equation}
        J_r=\begin{pmatrix}
            \mtu_r    &   0    \\
            0    &   0
        \end{pmatrix}.
    \end{equation}
\end{lemma}
\index{rang!classe d'équivalence}

\begin{proof}
    Nous devons prouver que pour toute matrice \( A\in\eM(n,\eK)\) de rang \( r\), il existe \( P,Q\in\GL(n,\eK)\) telles que \(QAP=J_r\). Soit \( \{ e_i \}\) la base canonique de \( \eK^n\), puis \( \{ f_i \}\) une base telle que \( Af_i=0\) dès que \( i>r\), qui existe par le lemme~\ref{LEMVecsaRgFixe}.

    Nous considérons la matrice inversible \( P\) telle que \( Pe_i=f_i\); ses colonnes sont donc précisément les \( f_i \), si bien que
    \begin{equation}
        APe_i=Af_i=\begin{cases}
            0    &   \text{si } i>r\\
            \neq 0    &    \text{sinon}.
        \end{cases}
    \end{equation}
    La matrice \( AP\) se présente donc sous la forme
    \begin{equation}
        AP=\begin{pmatrix}
            M    &   0    \\
            *    &   0
        \end{pmatrix}
    \end{equation}
    où \( M\) est une matrice \( r\times r\). Nous considérons maintenant une base \( \{ g_i \}_{i=1,\ldots, n}\) dont les \( r\) premiers éléments sont les \( r\) premières colonnes de \( AP\) et une matrice inversible \( Q\) telle que \( Qg_i=e_i\). Alors
    \begin{equation}
        QAPe_i=\begin{cases}
            e_i    &   \text{si } i<r\\
            0    &    \text{sinon}.
        \end{cases}.
    \end{equation}
    Cela signifie que \( QAP\) est la matrice \( J_r\).
\end{proof}

\begin{corollary}[Équivalence et rang]      \label{CorGOUYooErfOIe}
    Deux matrices sont équivalentes\footnote{Définition~\ref{DefBLELooTvlHoB}\ref{ItemPFXCooOUbSCt}.} si et seulement si elles sont de même rang.
\end{corollary}

\begin{proof}
    D'abord il y a des implicites dans l'énoncé. Vu que nous voulons soit par hypothèse soit par conclusion que les matrices \( A\) et \( B\) soient équivalentes, nous supposons qu'elles ont même dimension. Soient donc \( A\) et \( B\) deux matrices carrées d'ordre \( n \).

    Par le lemme~\ref{LemZMxxnfM}, deux matrices de même rang \( r\) sont équivalentes à \( J_r\). Elles sont donc équivalentes entre elles.

    Inversement, supposons que \( A\) et \( B\) soient deux matrices équivalentes : \( A=PBQ^{-1}\) avec \( P\) et \( Q\) inversibles. Alors
    \begin{subequations}
        \begin{align}
            \Image(PBQ^{-1})&=\{ PBQ^{-1}v\tq v\in \eK^n \}\\
            &=PB\underbrace{\{ Q^{-1}v\tq v\in \eK^n \}}_{=\eK^n}\\
            &=P\big( B(\eK^n) \big).
        \end{align}
    \end{subequations}
    L'ensemble \( B(\eK^n)\) est un sous-espace vectoriel de \( \eK^n\). Vu que le rang de \( P\) est maximum, la dimension de \( P\big( B(\eK^n) \big)\) est la même que celle de \( B(\eK^n)\). Par conséquent
    \begin{equation}
        \dim\Big( \Image(PBQ^{-1}) \Big)=\dim\big( B(\eK^n) \big)=\rang(B).
    \end{equation}
    Le membre de gauche de cela n'est autre que \( \rang(A)=\dim\big( \Image(PBQ^{-1}) \big)\).
\end{proof}

%---------------------------------------------------------------------------------------------------------------------------
\subsection{Algorithme des facteurs invariants}
%---------------------------------------------------------------------------------------------------------------------------

\begin{proposition}[Algorithme des facteurs invariants\cite{KXjFWKA}]   \label{PropPDfCqee}
    Soit \( (\eA,\delta)\) un anneau euclidien muni de son stathme  et \( U\in \eM(n \times m,\eA)\). Alors il existe \( d_1,\ldots, d_s\in \eA^*\) et des matrices \( P\in\GL(m,\eA)\), \( Q\in \GL(n,\eA)\) tels que nous ayons
    \begin{equation}
        U=P \begin{pmatrix}
            \begin{matrix}
                d_1    &       &       \\
                    &   \ddots    &       \\
                    &       &   d_s
            \end{matrix}&   0    \\
            0    &   0
        \end{pmatrix}Q
    \end{equation}
    avec \( d_i\divides d_{i+1}\) pour tout \( i\).
\end{proposition}
\index{anneau!euclidien!facteurs invariants}
\index{algorithme!facteurs invariants}

\begin{proof}
    Nous allons donner la preuve plus ou moins sous forme d'algorithme.

    D'abord si \( U=0\) c'est bon, on a la réponse. Sinon, nous prenons l'élément \( (i_0,j_0)\) dont le stathme est le plus petit et nous l'amenons en \( (1,1)\) par les permutations
    \begin{equation}
        \begin{aligned}[]
            C_1&\leftrightarrow C_{j_0}\\
            L_1&\leftrightarrow L_{i_0}
        \end{aligned}
    \end{equation}
    Ensuite nous traitons la première colonne jusqu'à amener des zéros partout en dessous de \( u_{11}\) de la façon suivante : pour chaque ligne successivement nous calculons la division euclidienne
    \begin{equation}
        u_{i1}=qu_{11}+r_i,
    \end{equation}
    et nous faisons
    \begin{equation}
        L_i\to L_i-qL_1,
    \end{equation}
    c'est-à-dire que nous enlevons le maximum possible et il reste seulement \( r_i\) en \( u_{i1}\). Vu que le but est de ne laisser que des zéros dans la première colonne, si le reste n'est pas zéro nous ne sommes pas content\footnote{S'il est zéro, nous passons à la ligne suivante}. Dans ce cas nous permutons \( L_1\leftrightarrow L_i\), ce qui aura pour effet de strictement diminuer le stathme de \( u_{11}\) parce qu'on va mettre en \( u_{11}\) le nombre \( r_i\) dont le stathme est strictement plus petit que celui de \( u_{11}\).

    En faisant ce jeu de division euclidienne puis échange, on diminue toujours le stathme de \( u_{11}\), donc ça finit par s'arrêter, c'est-à-dire qu'à un certain moment la division euclidienne de \( u_{i1}\) par \( u_{11}\) va donner un reste zéro et nous serons content.

    Une fois la première colonne ramenée à la forme
    \begin{equation}
        C_1=\begin{pmatrix}
            u_{11}    \\
            0    \\
            \vdots    \\
            0
        \end{pmatrix},
    \end{equation}
    nous faisons tout le même jeu avec la première ligne en faisant maintenant des sommes divisions et permutations de colonnes. Notons que ce faisant nous ne changeons plus la première colonne.

    En fin de compte nous trouvons une matrice\footnote{Nous nommons toujours par la même lettre \( U\) la matrice originale et la modifiée, comme il est d'usage en informatique.}
    \begin{equation}
        U=\begin{pmatrix}
            u_{11}   &   0    &   \ldots    &   0    \\
             0   &       &       &       \\
             \vdots   &       &   A    &       \\
             0   &       &       &
         \end{pmatrix}
    \end{equation}
    Si l'élément \( u_{11}\) ne divise pas un des éléments de \( A\), disons \( a_{ij}\), alors nous faisons
    \begin{equation}
        C_1\to C_1-C_j.
    \end{equation}
    Cela nous détruit un peu la première colonne, mais ne change pas \( u_{11}\). Nous avons maintnant
    \begin{equation}
        U=\begin{pmatrix}
            u_{11}   &   0    &   \ldots    &   0    \\
             0   &       &       &       \\
             *   &       &       &       \\
             u_{ij}   &       &   A    &       \\
             *   &       &       &       \\
             0   &       &       &
         \end{pmatrix}
    \end{equation}
    Et nous refaisons tout le jeu depuis le début. Cependant lorsque nous allons nous attaquer à la ligne \( i\), \( u_{11}\) ne divisera pas \( u_{ij}\), ce qui donnera lieu à une division euclidienne et un échange \( L_1\leftrightarrow L_i\). L'échange consistant à mettre \( r_i\) à la place de \( u_{11}\) et inversement  diminuera encore strictement le stathme. Encore une fois nous allons travailler jusqu'à avoir la matrice sous la forme
    \begin{equation}    \label{EqADcNVgI}
        U=\begin{pmatrix}
            u_{11}   &   0    &   \ldots    &   0    \\
             0   &       &       &       \\
             \vdots   &       &   A    &       \\
             0   &       &       &
         \end{pmatrix},
    \end{equation}
    sauf que cette fois le stathme de \( u_{11}\) est strictement plus petit que la fois précédente. Si \( u_{11}\) ne divise toujours pas tous les éléments de \( A\), nous recommençons encore et encore. En fin de compte nous finissons par avoir une matrice de la forme \eqref{EqADcNVgI} avec \( u_{11}\) qui divise tous les éléments de \( A\).

    Une fois que cela est fait, il faut continuer en recommençant tout sur la matrice \( A\). Nous avons maintenant
    \begin{equation}
        U=\begin{pmatrix}
            \begin{matrix}
                u_{11}  &       \\
                &   u_{22}
            \end{matrix}&   0    \\
            0    &   B
        \end{pmatrix}.
    \end{equation}
    Sous cette forme nous avons \( u_{11}\divides u_{22}\) et \( u_{11}\) divise tous les éléments de \( B\). En effet \( u_{11}\) divisant tous les éléments de \( A\), il divise toutes les combinaisons de ces éléments. Or tout l'algorithme ne consiste qu'à prendre des combinaisons d'éléments.

    Nous finissons donc bien sûr une matrice comme annoncée. De plus n'ayant effectué que des combinaisons de lignes, nous avons seulement multiplié par des matrices inversibles (lemme~\ref{LemyrAXQs}).
\end{proof}

%+++++++++++++++++++++++++++++++++++++++++++++++++++++++++++++++++++++++++++++++++++++++++++++++++++++++++++++++++++++++++++ 
\section{Changement de base}
%+++++++++++++++++++++++++++++++++++++++++++++++++++++++++++++++++++++++++++++++++++++++++++++++++++++++++++++++++++++++++++

Soit un espace vectoriel \( V\) muni de deux bases \( (e_i)_{i=1,\ldots, n}\) et \( (f_{\alpha})_{\alpha=1,\ldots, n}\). Les deux bases sont liées entre elles par
\begin{equation}        \label{EQooFRQRooSMsQQB}
    f_{\alpha}=\sum_iQ_{i\alpha}e_i.
\end{equation}
Ici \( Q\) n'est pas une application linéaire \( V\to V\) : \( Q\) est seulement un tableau de nombres, donnant les coordonnées des vecteurs \( f_{\alpha}\) dans la base de \( e_i\). Éventuellement \( Q\) peut être vu comme une application linéaire \( \eK^n\to \eK^n\).

Dans la suite nous nommerons \( Q^{-1}\) la matrice inverse de \( Q\). Inverse au sens des bêtes tableaux de nombres, sans interprétations en tant qu'application linéaire. De même pour \( Q^t\) qui est la transposée de \( Q\).

%---------------------------------------------------------------------------------------------------------------------------
\subsection{Changement de base : vecteurs de base}
%---------------------------------------------------------------------------------------------------------------------------

\begin{lemma}       \label{LEMooIHZGooOZoYZd}
    Soit un espace vectoriel \( V\) sur \( \eK\) ainsi que deux bases \( (e_i)_{i=1,\ldots, n}\), \( (f_{\alpha})_{\alpha=1,\ldots, n}\) de \( V\) liées par \( f_{\alpha}=\sum_iQ_{i\alpha}e_i\). Alors
    \begin{equation}    \label{EQooZQPAooAbKAdg}
        e_i=\sum_{\alpha}Q^{-1}_{\alpha i}f_{\alpha}.
    \end{equation}
\end{lemma}

\begin{proof}
    Nous multiplions l'égalité \( f_{\alpha}=\sum_iQ_{i\alpha}e_i\) par le nombre\footnote{Attention à la bonne interprétation de ce nombre : on fait bien référence à l'élément situé en \( (\alpha, j) \) de la matrice \( Q^{-1} \), et pas autre chose.} \( Q^{-1}_{\alpha j}\in \eK\)  et nous sommons sur \( \alpha\) :
    \begin{equation}
        \sum_{\alpha}Q^{-1}_{\alpha j}f_{\alpha}=\sum_{i\alpha}(A_{i\alpha}Q^{-1}_{\alpha j})e_i=e_j.
    \end{equation}
\end{proof}

%---------------------------------------------------------------------------------------------------------------------------
\subsection{Changement de base : coordonnées}
%---------------------------------------------------------------------------------------------------------------------------

\begin{proposition}     \label{PROPooNYYOooHqHryX}
    Soit un espace vectoriel \( V\) sur \( \eK\). Soient deux bases \( (e_i)_{i=1,\ldots, n}\) et \( (f_{\alpha})_{\alpha=1,\ldots, n}\) liées par \( f_{\alpha}=\sum_iQ_{i\alpha}e_i\). Nous considérons un même vecteur dans les deux bases : \( \sum_ix_ie_i=\sum_{\alpha}y_{\alpha}f_{\alpha}\). Alors
    \begin{enumerate}
        \item       \label{ITEMooIBAEooNaUnPD}
            \( y_{\alpha}=\sum_iQ^{-1}_{\alpha i}x_i\)
        \item       \label{ITEMooKPWTooMwdbPu}
            $x_i=\sum_{\alpha}Q_{i\alpha}y_{\alpha}$.
    \end{enumerate}
    Attention à l'ordre des indices dans la dernière égalité : la matrice \( Q\) vient avec les indices dans l'ordre \( i\alpha\), tandis que la matrice \( Q^{-1}\) vient avec les indices dans l'ordre opposé : \( \alpha i\). C'est pour cela qu'il est intéressant de noter avec des lettres latines les indices se rapportant à la première base et avec des lettres grecques ceux se rapportant à la seconde base.
\end{proposition}

\begin{proof}
    Soit un vecteur \( x\in V\). Il peut être écrit dans les deux bases :
    \begin{equation}        \label{EQooGSJMooGQstMx}
        x=\sum_ix_ie_i=\sum_{\alpha}y_{\alpha}f_{\alpha}.
    \end{equation}
    En remplaçant \( e_i\) par sa valeur \eqref{EQooZQPAooAbKAdg} nous avons l'égalité
    \begin{equation}
        \sum_{i\alpha}x_iQ^{-1}_{\alpha i}f_{\alpha}=\sum_{\alpha}y_{\alpha}f_{\alpha}.
    \end{equation}
    Vu que les \( f_{\alpha}\) sont linéairement indépendants, l'égalité des sommes donne l'égalité de chacun de termes :
    \begin{equation}
        y_{\alpha}=\sum_ix_iQ^{-1}_{\alpha i}.
    \end{equation}
    En identifiant \( x\in V\) au vecteur dans \( \eK^n\) de ses coordonnées dans la base \( \{ e_i \}\) nous pouvons écrire
    \begin{equation}
        y_{\alpha}=(Q^{-1}x)_{\alpha},
    \end{equation}
    Le point \ref{ITEMooIBAEooNaUnPD} est prouvé.

    En ce qui concerne le point \ref{ITEMooKPWTooMwdbPu}, nous repartons encore de \eqref{EQooGSJMooGQstMx}, mais nous y substituons la définition des \( f_{\alpha}\) :
    \begin{equation}
        \sum_{i}x_ie_i=\sum_{\alpha i}y_{\alpha}Q_{i\alpha}e_i.
    \end{equation}
    Vous voulez des détails ? Allez, une étape de plus que le strict nécessaire : nous écrivons
    \begin{equation}
        \sum_i\big( x_i-\sum_{\alpha}y_{\alpha}Q_{i\alpha} \big)e_i=0.
    \end{equation}
    Par linéaire indépendance des \( e_i\), nous avons annulation de tous les coefficients, c'est à dire
    \begin{equation}
        x_i=\sum_{\alpha}Q_{i\alpha}y_{\alpha},
    \end{equation}
    comme annoncé.
\end{proof}


\begin{normaltext}      \label{NORMooNWKZooPMwYTO}
    Les formules de changement de coordonnées de la proposition \ref{PROPooNYYOooHqHryX} s'écrivent souvent de la façon suivante :
    \begin{enumerate}
            \item       \label{ITEMooLHQCooBRvSlp}
                \( y_{\alpha}=(Q^{-1}x)_{\alpha}\)
            \item       \label{ITEMooNXUGooJIeoBf}
                \( y=Q^{-1}x\).
            \item       \label{ITEMooEFILooNENamW}
            $x_i=(Qy)_i $
            \item       \label{ITEMooMOKHooFEJvIW}
            $x=Qy$
    \end{enumerate}
    Ces égalités reposent sur un petit paquet d'abus de notations qu'il convient de bien comprendre. Ici, \( x\) et \( y\) sont les éléments de \( \eK^n\) donnés par les composantes de \( x\) dans les bases \( \{ e_i \}\) et \( \{ f_{\alpha} \}\), et \( Q\) est vu comme une matrice, un opérateur linéaire sur \( \eK^n\). Autrement dit, le choix des bases permet d'identifier \( V\) avec \( \eK^n\) et la matrice \( Q\) avec l'application linéaire \( f_Q\) de la proposition \ref{PROPooGXDBooHfKRrv}.
\end{normaltext}

%---------------------------------------------------------------------------------------------------------------------------
\subsection{Matrice d'une application linéaire}
%---------------------------------------------------------------------------------------------------------------------------

\begin{proposition}     \label{PROPooNZBEooWyCXTw}
    Soit une application linéaire \( t\colon V\to V\) de matrices \( A\) et \( B\) dans les bases \( \{ e_i \}\) et \( \{ f_{\alpha} \}\). Si les bases sont liées par
    \begin{equation}
        f_{\alpha}=\sum_iQ_{i\alpha}e_i,
    \end{equation}
    alors les matrices \( A\) et \( B\) sont liées par
    \begin{equation}
        B=Q^{-1}AQ.
    \end{equation}
\end{proposition}

\begin{proof}
    L'hypothèse sur le fait que \( A\) et \( B\) sont les matrices de \( t\) signifie que pour tout \( x\in V\),
    \begin{equation}
        t(x)=\sum_{ij}A_{ji}x_ie_j=\sum_{\alpha\beta}B_{\alpha\beta}y_{\beta}f_{\alpha}.
    \end{equation}
    En remplaçant \( e_j\) par son expression \eqref{EQooZQPAooAbKAdg} en termes des \( f_{\alpha}\) et \( x_i\) par son expression \eqref{SUBEQooPVGBooDafCcBk}, nous avons
    \begin{subequations}
        \begin{align}
            (By)_{\alpha}&=\sum_{ij\alpha}A_{ji}(Qy)_iQ^{-1}_{\alpha j}f_{\alpha}\\
            &=\sum_{i \alpha}(Q^{-1}A)_{\alpha i}(Qy)_if_{\alpha}\\
            &=\sum_{\alpha}(Q^{-1} AQy)_{\alpha}f_{\alpha}.
        \end{align}
    \end{subequations}
    Vu que les \( f_{\alpha}\) forment une base nous en déduisons \( Q^{-1}AQy=By\). Et vu que \( y\) est un élément quelconque de \( \eK^n\), nous en déduisons l'égalité de matrices
    \begin{equation}        \label{ooWKTYooOJfclT}
        B=Q^{-1}AQ.
    \end{equation}
\end{proof}
Il s'agit bien d'une égalité de matrices, ou à la limite d'applications linéaires sur \( \eK^n\), et non d'une égalité d'application linéaire sur \( V\).

%++++++++++++++++++++++++++++++++++++++++++++++++++++++++++++++++++++++++++++++++++++++++++++++++++++++++++++++++++++++++++++++++++++++++
\section{Espaces de polynômes}
%++++++++++++++++++++++++++++++++++++++++++++++++++++++++++++++++++++++++++++++++++++++++++++++++++++++++++++++++++++++++++++++++++++++++
\label{SecEspacePolynomes}

Attention : les polynômes en soi sont définis par la définition~\ref{DEFooFYZRooMikwEL}.

Pour chaque $k>0$ donné nous définissons
\begin{equation}
\mathcal{P}_\eR^k=\{p:\eR\to \eR\,|\, p : x\mapsto a_0+a_1 x +a_2 x^2 + \cdots+a_k x^k, \, a_i\in\eR,\,\forall i=0,\ldots,k\}.
\end{equation}
Il est facile de se convaincre que la somme de deux polynômes de degré inférieur ou égal à $k$ est encore un polynôme de degré inférieur ou égal à $k$. En outre il est clair que la multiplication par un scalaire ne peut pas augmenter le degré d'un polynôme. L'ensemble $\mathcal{P}_\eR^k$ est donc un espace vectoriel muni des opérations héritées de $\mathcal{P}_{\eR}$.

La base canonique de l'espace $\mathcal{P}_\eR^k$ est donné par les monômes $\mathcal{B}=\{x\mapsto x^j \,|\, j=0, \ldots, k\}$. Le fait que cela soit une base est vraiment facile à démontrer et est un exercice très utile si vous ne l'avez pas encore vu dans un cours précédent.

Nous allons maintenant étudier trois applications linéaires de $\mathcal{P}_\eR^k$ vers des autres espaces vectoriels
\begin{description}
  \item[L'isomorphisme canonique  $\phi:\mathcal{P}_\eR^k \to\eR^{k+1}$] Nous définissons $\phi$ par les relations suivantes
\[
\phi(x^j)=e_{j+1}, \qquad \forall j\in\{0,\dots, k\}.
\]
Cela veut dire que pour tout $p$ dans $\mathcal{P}_\eR^k$, avec $p(x)=a_0+a_1 x +a_2 x^2 + \cdots+a_k x^k$, l'image de $p$ par $\phi$ est
\[
\phi(p)=\phi\left(\sum_{j=0}^k a_j x^j\right)=\sum_{j=0}^k a_j e_{j+1}.
\]
\begin{example} Soit $k=5$ on a
  \begin{equation}
    \phi(-8-7x-4x^2+4x^3+2x^5)=
  \begin{pmatrix}
    -8\\
    -7\\
    -4\\
    4\\
    0\\
    2
  \end{pmatrix}.
  \end{equation}
\end{example}

Cette application est clairement bijective et respecte les opérations d'espace vectoriel, donc elle est un isomorphisme d'espaces vectoriels. L'existence d'un isomorphisme entre $\mathcal{P}_\eR^k$  et $\eR^{k+1}$ est un cas particulier du théorème qui dit que  pour chaque $m$ dans $\eN_0$ fixée, tous les espaces vectoriels sur $\eR$ de dimension $m$ sont isomorphes à $\eR^m$. Vous connaissez peut être déjà ce théorème depuis votre cours d'algèbre linéaire.
    \item[La dérivation $d: \mathcal{P}_\eR^k \to \mathcal{P}_\eR^{k-1}$] L'application de dérivation $d$ fait exactement ce qu'on s'attend d'elle
\[
d(x^0)=d(1)=0, \qquad d(x^j)=j x^{j-1}, \quad \forall j\in\{1,\dots, k\}.
\]
Cette application n'est pas injective, parce que l'image de $p$ ne dépend pas de la valeur de $a_0$, donc si deux polynômes sont les mêmes à une constante près ils auront la même image par $d$.

\begin{example} Soit $k=3$ on a
  \begin{equation}
    d(-8-12x+4x^3)= -12 (1) + 4 (3x^2) = -12+12 x^2.
    \end{equation}

    Noter que $d(-30-12x+4x^3)=d(-8-12x+4x^3)$. Cela confirme, comme mentionné plus haut, que la dérivée n'est pas injective.
\end{example}
      \item[L'intégration $I: \mathcal{P}_\eR^k \to \mathcal{P}_\eR^{k+1}$] Nous pouvons définir une application que est <<à une constante prés>> l'application inverse de la dérivation. Cette application est définie sur les éléments de base par
          \begin{equation}
                I(x^j)= \frac{x^{j+1}}{j+1}.
          \end{equation}
          Bien entendu la raison d'être et la motivation de cette définition apparaîtra lorsque nous développerons une théorie générale de l'intégration.

\begin{example}
   Soit $k=4$ on a
  \begin{equation}
    I(6+2x+x^2+x^4)= 6x+x^2+\frac{x^3}{3}+\frac{x^5}{5}.
    \end{equation}
\end{example}

Remarquez que, étant donné que dans la définition de $I$ nous avons décidé d'intégrer entre zéro et $x$, tous les polynômes dans $\mathcal{P}_\eR^{k+1}$ qui sont l'image par $I$ d'un polynôme de $\mathcal{P}_\eR^{k}$ ont $a_0=0$. Cela veut dire que nous pouvons générer toute l'image de $I$ en utilisant un sous-ensemble de la base canonique de $\mathcal{P}_\eR^{k+1}$,  en particulier $\mathcal{B}_1=\{x\mapsto x^j \,|\, j=1, \ldots, k\}\subset \mathcal{B}$ nous suffira. Cela n'est guère surprenant, parce que l'image par une application linéaire d'un espace vectoriel de dimension finie ne peut pas être un espace de dimension supérieure.
\end{description}

Les applications de dérivation et intégration correspondent évidemment à des applications linéaires de $\mathcal{P}_\eR$ dans lui-même.

L'espace de tous les polynômes étant de dimension infinie, il peut servir de contre-exemple assez simple. Dans la sous-section~\ref{SubSecPOlynomesCE}, nous verrons que toutes les normes ne sont pas équivalentes sur l'espace des polynômes.

%+++++++++++++++++++++++++++++++++++++++++++++++++++++++++++++++++++++++++++++++++++++++++++++++++++++++++++++++++++++++++++
\section{Dualité}
%+++++++++++++++++++++++++++++++++++++++++++++++++++++++++++++++++++++++++++++++++++++++++++++++++++++++++++++++++++++++++++

\begin{proposition} \label{PropEJBZooTNFPRj}
    Si \( A\) est la matrice d'une application linéaire, alors le rang de cette application linéaire est égal au rang de \( A \), c'est-à-dire à la taille de la plus grande matrice carrée de déterminant non nul contenue dans \( A\).
\end{proposition}

\begin{definition}  \label{DefJPGSHpn}
    Soit \( E\) un espace vectoriel sur \( \eK\).

    Une \defe{forme linéaire}{forme linéaire} sur \( E \) est une application linéaire de \( E \) sur son corps de base \( \eK\).

    Le \defe{dual algébrique}{dual algébrique} de \( E\), noté \( E^*\), l'ensemble des formes linéaires sur \( E\). Ainsi, \( E^* = \GL(E,\eK)\).
\end{definition}

Nous verrons plus tard qu'en dimension infinie, les applications linéaires ne sont pas toujours continues. Nous définirons donc aussi un concept de dual topologique. Voir la proposition~\ref{PROPooQZYVooYJVlBd}, la remarque~\ref{RemOAXNooSMTDuN} et la définition~\ref{DEFooKSDFooGIBtrG}.

\begin{definition}      \label{DEFooTMSEooZFtsqa}
    Si \( E\) est un espace vectoriel et si \( \{ e_i \}\) est une base de \( E\), alors nous définissons la \defe{base duale}{base!duale} de \( E^*\) par
    \begin{equation}
        e_i^*(e_j)=\delta_{ij}
    \end{equation}
    est sa prolongation par linéarité.
\end{definition}
Notons que si \( v\in E\) est un vecteur, ça n'a aucun sens à priori de parler de \( v^*\). Il s'agit bien de définir \emph{toute} la base \( \{ e_i^* \}\) à partir de toute la base \( \{ e_i \}\).

%---------------------------------------------------------------------------------------------------------------------------
\subsection{Orthogonal}
%---------------------------------------------------------------------------------------------------------------------------

\begin{definition}      \label{DEFooEQSMooHVzbfz}
    Soit \( E\), un espace vectoriel, et \( F\) une sous-espace de \( E\). L'\defe{orthogonal}{orthogonal!sous-espace} de \( F\) est la partie \( F^{\perp}\subset E^*\) donnée par
    \begin{equation}    \label{Eqiiyple}
        F^{\perp}=\{ \alpha\in E^*\tq \forall x\in F,\alpha(x)=0 \}.
    \end{equation}
\end{definition}

Cette définition d'orthogonal via le dual n'est pas du pur snobisme. En effet, la définition «usuelle» qui ne parle pas de dual,
\begin{equation}
    F^{\perp}=\{ y\in E\tq \forall x\in F,y\cdot x=0 \},
\end{equation}
demande la donnée d'un produit scalaire. Évidemment dans le cas de \( \eR^n\) munie du produit scalaire usuel et de l'identification usuelle entre \( \eR^n\) et \( (\eR^n)^*\) via une base, les deux notions d'orthogonal coïncident.

La définition~\ref{DEFooEQSMooHVzbfz}, au contraire, est intrinsèque : elle ne dépend que de la structure d'espace vectoriel.

Si \( B\subset E^*\), on note \( B^o\)\nomenclature[G]{\( B^o\)}{orthogonal dans le dual} son orthogonal :
\begin{equation}
    B^o=\{ x\in E\tq \omega(x)=0\,\forall \omega\in B \}.
\end{equation}
Notons qu'on le note \( B^o\) et non \( B^{\perp}\) parce qu'on veut un peu s'abstraire du fait que \( (E^*)^*=E\). Du coup on impose que \( B\) soit dans un dual et on prend une notation précise pour dire qu'on remonte au pré-dual et non qu'on va au dual du dual.

\begin{proposition} \label{PropXrTDIi}
    Soient un espace vectoriel \( E\) et un sous-espace vectoriel \( F\). Nous avons
    \begin{equation}
        \dim F+\dim F^{\perp}=\dim E.
    \end{equation}
\end{proposition}

\begin{proof}
    Soit \( \{ e_1,\ldots, e_p \}\) une base de \( F\) que nous complétons en une base \( \{ e_1,\ldots, e_n \}\) de \( E\) par le théorème~\ref{ThonmnWKs}. Soit \( \{ e_1^*,\ldots, e^*_n \}\) la base duale. Alors nous prouvons que \( \{ e^*_{p+1},\ldots, e_n^* \}\) est une base de \( F^{\perp}\).

    Déjà c'est une partie libre en tant que partie d'une base.

    Ensuite ce sont des éléments de \( F^{\perp}\) parce que si \( i\leq p\) et si \( k\geq 1\), nous avons \( e^*_{p+k}(e_i)=0\); donc oui, \( e^*_{p+k}\in F^{\perp}\).

    Enfin \( F^{\perp}\subset\Span\{ e_{k}^*, k \in \{p+1, \dots, n\}\}\) parce que si \( \omega=\sum_{k=1}^n\omega_ke_k^*\), alors \( \omega(e_i)=\omega_i\), mais nous savons que si \( \omega\in F^{\perp}\), alors \( \omega(e_i)=0\) pour \( i\leq p\). Donc \( \omega=\sum_{k=p+1}^n\omega_ke^*_k\).
\end{proof}

La proposition \ref{PROPooNITTooCYcrrT} donnera une version plus terre à terre de la proposition \ref{PropXrTDIi} en disant que si nous avons un produit scalaire, alors \( V=F\oplus F^{\perp}\) où \( F^{\perp}\) est cette fois défini comme l'orthogonal pour le produit scalaire.

%---------------------------------------------------------------------------------------------------------------------------
\subsection{Transposée : pas d'approche naïve}
%---------------------------------------------------------------------------------------------------------------------------
\label{SUBSECooGPXVooEYwIiJ}

Il est légitime, si \( t\colon E\to E\) est une application linéaire, de dire que sa transposée soit l'application linéaire \( t^t\colon E\to E\) dont la matrice est la matrice transposée de celle de \( t\). Lorsque nous travaillons sur \( \eR^n\) muni de la base canonique, cela ne pose pas de problèmes et nous pouvons écrire des égalités du type \( \langle x, Ay\rangle =\langle A^tx, y\rangle \).

\begin{proposition}[Matrice transposée et produit scalaire]     \label{PROPooNARVooEuhweD}
    Soit une matrice réelle \( A\). En utilisant l'application linéaire associée\footnote{Définition \ref{DEFooJVOAooUgGKme}. Ici nous considérons la base canonique sur \( \eR^n\)} \( f_A\colon \eR^n\to \eR^n\), nous avons
    \begin{equation}
        x\cdot f_A(y)=f_{A^t}(x)\cdot y.
    \end{equation}
    Cette formule est souvent écrire \( x\cdot Ay=A^tx\cdot y\) ou \( \langle x, Ay\rangle =\langle A^tx, y\rangle \).
\end{proposition}

\begin{proof}
    Il s'agit d'une calcul utilisant la formule \eqref{EQooBVGHooJhFbMs} et le produit scalaire \eqref{EQooFITHooEXDCGd} :
    \begin{equation}
        x\cdot f_A(y)=\sum_ix_i\big( f_A(y) \big)_i=\sum_ix_i\sum_jA_{ij}y_j=\sum_{ij}A^t_{ji}x_iy_j=\sum_jf_{A^t}(x)_jy_j=f_{A^t}(x)\cdot y.
    \end{equation}
\end{proof}

Hélas nous allons voir que cette façon de définir une transposée est mauvaise.

Soit une application linéaire \( t\colon E\to E\) de matrice \( A\) dans la base \( \{ e_i \}_{i=1,\ldots, n}\) et de matrice \( B\) dans la base \( \{ f_{\alpha} \}_{\alpha=1,\ldots, n}\).

Nous nommons \( t_1\) l'application linéaire associée à \( A^t\) dans la base \( \{ e_i \}\) et \( t_2\) l'application linéaire associée à la matrice \( B^t\) dans la base \( \{ f_{\alpha} \}\). Définir la transposée d'une application linéaire comme étant l'application linéaire associée à la transposée de sa matrice ne sera une bonne définition que si \( t_1=t_2\).

La première chose facile à voir est
\begin{equation}        \label{EQooAMHPooUQEkJo}
    t_1(e_i)_j=\sum_k(A^t)_{jk}(e_i)_k=A^t_{ji}=A_{ij}.
\end{equation}
Pour calculer \( t_2(e_i)_j\), c'est un peu plus laborieux :
\begin{subequations}
    \begin{align}
        t_2(e_i)&=\sum_{\alpha}Q_{\alpha i}^{-1} t_2(f_\alpha)=\sum_{\beta\gamma\alpha}Q_{\alpha i}^{-1}B^t_{\gamma\beta}\underbrace{(t_{\alpha})_{\beta}}_{\delta_{\alpha\beta}}f_{\gamma}=\sum_{\beta\gamma}Q_{\beta i}^{-1}B^t_{\gamma\beta}f_{\gamma}\\
        &=(B^tQ^{-1})_{\gamma i}Q_{j\gamma}e_j\\
        &=\sum_j(QB^tQ^{-1})_{ji}e_j.
    \end{align}
\end{subequations}
Donc \( t_2(e_i)_j=(QB^tQ^{-1})_{ji}\). En tenant compte du fait que \( B=Q^{-1}AQ\) nous avons
\begin{equation}
    t_2(e_i)_j=(QQ^tA^t(Q^{-1})^tQ^{-1})_{ji}.
\end{equation}
Cela est égal à l'expression \eqref{EQooAMHPooUQEkJo} lorsque \( Q^t=Q^{-1}\). Nous voyons que confondre transposée d'une application linéaire avec transposée de la matrice associée n'est valable que si nous sommes certain de ne considérer que des changements de base par des matrices orthogonales.

Cela est la situation typique dans laquelle nous nous trouvons lorsque nous considérons des applications linéaires sur \( \eR^n\) muni de la base canonique et que nous n'avons aucune intention de changer de base, et encore moins de chercher une base non orthonormale. Cette situation est clairement la situation la plus courante.

\begin{example}[\cite{ooLIOMooBuCPUS}]
    Soit la base canonique \( \{ e_1,e_2 \}\) de \( \eR^2\). Nous considérons l'application linéaire \( t\colon \eR^2\to \eR^2\) définie par
    \begin{subequations}
        \begin{align}
            t(e_1)&=e_1\\
            t(e_2)&=0.
        \end{align}
    \end{subequations}
    La matrice de \( t\) dans cette base est
    \begin{equation}
        A=\begin{pmatrix}
            1    &   0    \\
            0    &   0
        \end{pmatrix}.
    \end{equation}
    Elle est symétrique : elle vérifie \( A^t=A\). Si nous comptions sur la transposée de matrice pour définir la transposée de \( t\), nous aurions \( t^t=t\).

    Soit maintenant la base \( f_1=e_1\), \( f_2=e_1+e_2\). Nous avons \( t(f_1)=f_1\) et
    \begin{equation}
        t(f_2)=t(e_1)+t(e_2)=e_1=f_1.
    \end{equation}
    Donc la matrice de \( t\) dans cette base est
    \begin{equation}
        B=\begin{pmatrix}
            1    &   1    \\
            0    &   0
        \end{pmatrix}.
    \end{equation}
    Et là, nous avons \( B^t\neq B\). Donc en comptant sur cette base pour définir la transposée de \( t\) nous aurions \( t^t\neq t\).
\end{example}

\begin{normaltext}      \label{NooMZVRooExWVKJ}
    Autrement dit, la façon «usuelle» de voir la transposée d'une application linéaire ne fonctionne dans les livres pour enfant uniquement parce qu'on n'y considère toujours \( \eR^n\) muni de la base canonique ou de bases orthonormées.

    Notons que nous avons tout de même les notions d'opérateur adjoint et autoadjoint pour parler d'application orthogonale sans passer par la transposée, voir~\ref{DEFooYKCSooURQDoS}.
\end{normaltext}

%---------------------------------------------------------------------------------------------------------------------------
\subsection{Transposée : la bonne approche}
%---------------------------------------------------------------------------------------------------------------------------

\begin{definition}      \label{DefooZLPAooKTITdd}
    Si \( f\colon E\to F\) est une application linéaire entre deux espaces vectoriels, la \defe{transposée}{transposée} est l'application \( f^t\colon F^*\to E^*\) donnée par
    \begin{equation}
        f^t(\omega)(x)=\omega\big( f(x) \big).
    \end{equation}
    pour tout \( \omega\in F^*\) et \( x\in E\).
\end{definition}

\begin{lemma}
    Soit \( E\) muni de la base \( \{ e_i \}\) et \( F\) muni de la base \( \{ g_i \}\) et une application \( f\colon E\to F\). Si \( A\) est la matrice de \( f\) dans ces bases, alors \( A^t\) est la matrice de \( f^t\) dans les bases \( \{ e^*_i \}\) et \( \{ g^*_i \}\) de \( E^*\) et \( F^*\).
\end{lemma}

\begin{proof}
    Nous allons montrer que les formes \( f^t(g^*_i)\) et \( \sum_k(A^t)_{ik}g^*_k\) sont égales en les appliquant à un vecteur.

    Par définition de la matrice d'une application linéaire dans une base,
    \begin{equation}
        f^t(g_i^*)=\sum_j(f^t)_{ij}e^*_j,
    \end{equation}
    et
    \begin{equation}
        f(e_k)=\sum_lA_{kl}g_l.
    \end{equation}
    Du coup, si \( x=\sum_kx_ke_k\), nous avons
    \begin{equation}    \label{EqCzwftH}
        f^t(g_i^*)x=\sum_{kl}x_kg_i^*A_{kl}g_l=\sum_{kl}x_kA_{kl}\delta_{il}=\sum_k x_kA_{ki}=\sum_k(A^t)_{ik}x_k.
    \end{equation}
    D'autre part,
    \begin{equation}    \label{EqWlQlrR}
        \sum_k(A^t)_{ik}g_k^*x=\sum_{kl}(A^t)_{ik}g^*_kx_le_l=\sum_k(A^t)_{ik}x_k.
    \end{equation}
    Le fait que \eqref{EqCzwftH} et \eqref{EqWlQlrR} donnent le même résultat prouve le lemme.
\end{proof}
En corolaire, les rangs de \( f\) et de \( f^t\) sont égaux parce que le rang est donné par la plus grande matrice carrée de déterminant non nul. Nous prouvons cependant ce résultat de façon plus intrinsèque.

\begin{lemma}[\href{http://gilles.dubois10.free.fr/algebre_lineaire/dualite.html}{Gilles Dubois}]   \label{LemSEpTcW}
    Si \( f\colon E\to F\) est une application linéaire, alors
    \begin{equation}
        \rang(f)=\rang(f^t).
    \end{equation}
\end{lemma}

\begin{proof}
    Nous posons \( \dim\ker(f)=p\) et donc \( \rang(f)=n-p\). Soit \( \{ e_1,\ldots, e_p \}\) une base de \( \ker(f)\) que l'on complète en une base \( \{ e_1,\ldots, e_n \}\) de \( E\). Nous considérons maintenant les vecteurs
    \begin{equation}
        g_i=f(e_{p+i})
    \end{equation}
    pour \( i=1,\ldots, n-p\). C'est-à-dire que les \( g_i\) sont les images des vecteurs qui ne sont pas dans le noyau de \( f\). Prouvons qu'ils forment une famille libre. Si
    \begin{equation}
        \sum_{k=1}^{n-p}a_kf(e_{p+k})=0,
    \end{equation}
    alors \( f\big( \sum_ka_ke_{p+k} \big)=0\), ce qui signifierait que \( \sum_ka_ke_{p+k}\) se trouve dans le noyau de \( f\), ce qui est impossible par construction de la base \( \{ e_i \}_{i=1,\ldots, n}\). Étant donné que les vecteurs \( g_1,\ldots, g_{n-p}\) sont libres, nous les complétons en une base
    \begin{equation}
        \{ \underbrace{g_1,\ldots, g_{n-p}}_{\text{images}},\underbrace{g_{n-p+1},\ldots, g_r}_{\text{complétion}} \}
    \end{equation}
    de \( F\).

    Nous prouvons maintenant que \( \rang(f^t)\geq n-p\) en montrant que les formes \( \{ g_i^* \}_{i=1,\ldots, n-p}\) forment une partie libre (et donc l'espace image de \( f^t\) est au moins de dimension \( n-p\)). Pour cela nous prouvons que \( f^t(g_i^*)=e^*_{i+p}\). En effet
    \begin{equation}
        f^t(g^*_i)e_k=g_i^*(fe_k),
    \end{equation}
    Si \( k=1,\ldots, p\), alors \( fe_k=0\) et donc \( g_i^*(fe_k)=0\); si \( k=p+l\) alors
    \begin{equation}
        f^t(g_i^*)e_k=g_i^*(fe_{k+l})=g^*_i(g_l)=\delta_{i,l}=\delta_{i,k-p}=\delta_{k,i+p}.
    \end{equation}
    Donc \( f^t(g_i^*)=e^*_{i+p}\). Cela prouve que les formes \( f^t(g_i^*)\) sont libres et donc que
    \begin{equation}
        \rang(f^t)\geq n-p=\rang(f).
    \end{equation}
    En appliquant le même raisonnement à \( f^t\) au lieu de \( f\), nous trouvons
    \begin{equation}
        \rang\big( (f^t)^t \big)\geq \rang(f^t)
    \end{equation}
    et donc, vu que \( (f^t)^t=f\), nous obtenons \( \rang(f)=\rang(f^t)\).

\end{proof}

\begin{proposition}[\cite{DualMarcSAge}]        \label{PropWOPIooBHFDdP}
    Si \( f\) est une application linéaire entre les espaces vectoriels \( E\) et \( F\), alors nous avons
    \begin{equation}
        \Image(f^t)=\ker(f)^{\perp}.
    \end{equation}
\end{proposition}

\begin{proof}
    Soient donc l'application \( f\colon E\to F\) et sa transposée \( f^t\colon F^*\to E^*\). Nous commençons par prouver que \( \Image(f^{t})\subset(\ker f)^{\perp}\). Pour cela nous prenons \( \omega\in \Image(f^t)\), c'est-à-dire \( \omega=\alpha\circ f\) pour un certain élément \( \alpha\in F^*\). Si \( z\in\ker(f)\), alors \( \omega(z)=(\alpha\circ f)(z)=0\), c'est-à-dire que \( \omega\in (\ker f)^{\perp}\).

    Pour prouver qu'il y a égalité, nous n'allons pas démontrer l'inclusion inverse, mais plutôt prouver que les dimensions sont égales. Après, on sait que si \( A\subset B\) et si \( \dim A=\dim B\), alors \( A=B\). Nous avons
    \begin{subequations}
        \begin{align}
            \dim\big( \Image(f^t) \big)&=\rang(f^t)\\
            &=\rang(f)  &\text{lemme~\ref{LemSEpTcW}}\\
            &=\dim(E)-\dim\ker(f)   &\text{théorème~\ref{ThoGkkffA}}\\
            &=\dim\big( (\ker f)^{\perp} \big)  &\text{proposition~\ref{PropXrTDIi}}.
        \end{align}
    \end{subequations}
\end{proof}

\begin{lemma}[\cite{ooEPEFooQiPESf}]
    Soit \( \eK\) un corps, \( E\) et \( F\) deux \( \eK\)-espaces vectoriels de dimension finie et une application linéaire \( f\colon E\to F\). L'application \( f\) est injective si et seulement si sa transposée\footnote{Définition~\ref{DefooZLPAooKTITdd}.} \( f^t\) est surjective.
\end{lemma}

\begin{proof}
    Supposons que \( f\) soit injective. Alors par le lemme~\ref{LEMooDAACooElDsYb}, il existe \( g\colon F\to E\) tel que \( g\circ f=\id|_E\). Nous avons alors aussi \( (g\circ f)^t=\id|_{E^*}\), mais \( (g\circ f)^t=f^t\circ g^t\), donc \( f^t\) est surjective.

    Inversement, nous supposons que \( f^t\colon F^*\to E^*\) est surjective. Alors en nous souvenant que \( E\) et \( F\) sont de dimension finie et en faisant jouer les identifications \( (f^t)^t=f\) et \( (E^*)^*=E\) nous savons qu'il existe \( s\colon E^*\to F^*\) tel que \( f^t\circ s=\id|_{E^*}\). En passant à la transposée,
    \begin{equation}
        s^t\circ f=\id|_{E},
    \end{equation}
    qui implique que \( f\) est injective.
\end{proof}

%---------------------------------------------------------------------------------------------------------------------------
\subsection{Polynômes de Lagrange}
%---------------------------------------------------------------------------------------------------------------------------

Soit \( E=\eR_n[X]\) l'ensemble des polynômes à coefficients réels de degré au plus \( n\). Soient les \( n+1\) réels distincts \( a_0,\ldots, a_n\). Nous considérons les formes linéaires associées \( f_i\in E^*\),
\begin{equation}
    f_i(P)=P(a_i).
\end{equation}
\begin{lemma}
    Ces formes forment une base de \( E^*\).
\end{lemma}

\begin{proof}
    Nous prouvons que l'orthogonal est réduit au nul :
    \begin{equation}
        \Span\{ f_0,\ldots, f_n \}^{\perp}=\{ 0 \}
    \end{equation}
    pour que la proposition~\ref{PropXrTDIi} conclue. Si \( P\in\Span\{ f_i \}^{\perp}\), alors \( f_i(P)=0\) pour tout \( i\), ce qui fait que \( P(a_i)=0\) pour tout \( i=0,\ldots, n\). Un polynôme de degré au plus \( n\) qui s'annule en \( n+1\) points est automatiquement le polynôme nul.
\end{proof}

Les \defe{polynômes de Lagrange}{Lagrange!polynôme}\index{polynôme!Lagrange} sont les polynômes de la base (pré)duale de la base \( \{ f_i \}\).

\begin{proposition}
    Les polynômes de Lagrange sont donnés par
    \begin{equation}
        P_i=\prod_{k\neq i}\frac{ X-a_k }{ a_i-a_k }.
    \end{equation}
\end{proposition}

\begin{proof}
    Il suffit de vérifier que \( f_j(P_i)=\delta_{ij}\). Nous avons
    \begin{equation}
        f_j(P_i)=P_i(a_j)=\prod_{k\neq i}\frac{ a_j-a_k }{ a_i-a_k }.
    \end{equation}
    Si \( j\neq i\) alors un des termes est nul. Si au contraire \( i=j\), tous les termes valent \( 1\).
\end{proof}

%---------------------------------------------------------------------------------------------------------------------------
\subsection{Dual de \texorpdfstring{$ \eM(n,\eK)$}{M(n,K)}}
%---------------------------------------------------------------------------------------------------------------------------

\begin{proposition}[\cite{KXjFWKA}]     \label{PropHOjJpCa}
    Soit \( \eK\), un corps. Les formes linéaires sur \( \eM(n,\eK)\) sont les applications de la forme
    \begin{equation}
        \begin{aligned}
            f_A\colon \eM(n,\eK)&\to \eK \\
            M&\mapsto \tr(AM).
        \end{aligned}
    \end{equation}
\end{proposition}
\index{trace!dual de \( \eM(n,\eK)\)}
\index{dual!de \( \eM(n,\eK)\)}


\begin{proof}
    Nous considérons l'application
    \begin{equation}
        \begin{aligned}
            f\colon \eM(n,\eK)&\to \eM(n,\eK)^* \\
            A&\mapsto f_A
        \end{aligned}
    \end{equation}
    et nous voulons prouver que c'est une bijection. Étant donné que nous sommes en dimension finie, nous avons égalité des dimensions de \( \eM(n,\eK)\) et \( \left(\eM(n,\eK)\right)^*\), et il suffit de prouver que \( f\) est injective. Soit donc \( A\) telle que \( f_A=0\). Nous l'appliquons à la matrice \( (E_{ij})_{kl}=\delta_{ik}\delta_{jl}\) :
    \begin{equation}
            0=f_A(E_{ij})
            =\sum_{k}(AE_{ij})_{kk}
            =\sum_{kl}A_{kl}(E_{ij})_{lk}
            =\sum_{kl}A_{kl}\delta_{il}\delta_{jk}
            =A_{ij}.
    \end{equation}
    Donc \( A=0\).
\end{proof}

\begin{corollary}[\cite{KXjFWKA}]
    Soit \( \eK\) un corps et \( \phi\in\eM(n,\eK)^*\) telle que pour tout \( M,N\in \eM(n,\eK)\) on ait
    \begin{equation}
        \phi(MN)=\phi(NM).
    \end{equation}
    Alors il existe \( \lambda\in \eK\) tel que \( \phi=\lambda\Tr\).
\end{corollary}
\index{trace!unicité pour la propriété de trace}

\begin{proof}
    La proposition~\ref{PropHOjJpCa} nous donne une matrice \( A\in \eM(n,\eK)\) telle que \( \phi=f_A\). L'hypothèse nous dit que \( f_A(MN)=f_A(NM)\), c'est-à-dire
    \begin{equation}
        \Tr(AMN)=\Tr(ANM)
    \end{equation}
    pour toutes matrices \( M, N\in \eM(n,\eK)\). L'invariance cyclique de la trace\footnote{Lemme~\ref{LEMooUXDRooWZbMVN}.} appliqué au membre de droite nous donne \( \Tr(AMN)=\Tr(MAN)\), ce qui signifie que
    \begin{equation}
        \Tr\big( (AM-MA)N \big)=0
    \end{equation}
    ou encore que \( f_{AM-MA}=0\), et ce, pour toute matrice \( M\). La fonction \( f\) étant injective nous en déduisons que la matrice \( A\) doit satisfaire
    \begin{equation}
        AM=MA
    \end{equation}
    pour tout \( M\in\eM(n,\eK)\). En particulier, en prenant pour \( M \) les fameuses matrices \( E_{ij}\) et en calculant un peu,
    \begin{equation}
        A_{li}\delta_{jm}=\delta_{il}A_{jm}
    \end{equation}
    pour tout \( i,j,l,m\). Cela implique que \( A_{ll}=A_{mm}\) pour tout \( l\) et \( m\) et que \( A_{jm}=0\) dès que \( j\neq m\). Il existe donc \( \lambda\in \eK\) tel que \( A=\lambda\mtu\). En fin de compte,
    \begin{equation}
        \phi(X)=f_{\lambda\mtu}(X)=\lambda\Tr(X).
    \end{equation}
\end{proof}

\begin{corollary}[\cite{KXjFWKA}]       \label{CorICUOooPsZQrg}
    Soit \( \eK\) un corps. Tout hyperplan de \( \eM(n,\eK)\) coupe \( \GL(n,\eK)\).
\end{corollary}
\index{groupe!linéaire!hyperplan}

\begin{proof}
    Soit \( \mH\) un hyperplan de \( \eM\). Il existe une forme linéaire \( \phi\) sur \( \eM(n,\eK)\) telle que \( \mH=\ker(\phi)\). Encore une fois la proposition~\ref{PropHOjJpCa} nous donne \( A\in \eM\) telle que \( \phi=f_A\); nous notons \( r\) le rang de \( A\). Par le lemme~\ref{LemZMxxnfM} nous avons \( A=PJ_rQ\) avec \( P,Q\in \GL(n,\eK)\) et
    \begin{equation}
        J_r=\begin{pmatrix}
            \mtu_r    &   0    \\
            0    &   0
        \end{pmatrix}.
    \end{equation}
    Pour tout \( M\in \eM\) nous avons
    \begin{equation}
        \phi(M)=\Tr(AM)=\Tr(PJ_rQM)=\Tr(J_rQMP),
    \end{equation}
la dernière égalité découlant de l'invariance cyclique de la trace\footnote{Lemme~\ref{LEMooUXDRooWZbMVN}.}. Ce que nous cherchons est \( M\in \GL(n,\eK)\) telle que \( \phi(M)=0\). Nous commençons par trouver \( N\in\GL(n,\eK)\) telle que \( \Tr(J_rN)=0\). Celle-là est facile : c'est
    \begin{equation}
        N=\begin{pmatrix}
            0    &   1    \\
            \mtu_{n-1}    &   0
        \end{pmatrix}.
    \end{equation}
    Les éléments diagonaux de \( J_rN\) sont tous nuls. Par conséquent en posant \( M=Q^{-1}NP^{-1}\) nous avons notre matrice inversible dans le noyau de \( \phi\).
\end{proof}
\index{hyperplan!de \( \eM(n,\eK)\)}

%+++++++++++++++++++++++++++++++++++++++++++++++++++++++++++++++++++++++++++++++++++++++++++++++++++++++++++++++++++++++++++ 
\section{Représentation de groupe}
%+++++++++++++++++++++++++++++++++++++++++++++++++++++++++++++++++++++++++++++++++++++++++++++++++++++++++++++++++++++++++++

\begin{definition}[Représentation]      \label{DEFooXVMSooXDIfZV}
    Soit un groupe \( G\) et un espace vectoriel \( V\). Nous disons qu'une application \( \rho\colon G\to \GL(V)\) est une \defe{représentation}{représentation} de \( G\) sur \( V\) si pour tout \( g,h\in G\) nous avons
    \begin{equation}
        \rho(g)\circ\rho(g)=\rho(gh).
    \end{equation}
    Très souvent, nous disons que la représentation est le couple \( (V,\rho)\).
\end{definition}

\begin{definition}
    Une représentation\footnote{Définition \ref{DEFooXVMSooXDIfZV}.} est \defe{fidèle}{représentation!fidèle} si elle est injective en tant que application \( G\to \GL(V)\). Ce ne sont pas chacun des \( \rho(g)\) qui doivent être injectifs. La dimension de \( V\) est le \defe{degré}{degré!d'une représentation} de la représentation \( (V,\rho)\).
\end{definition}

\begin{proposition}     \label{PROPooHNQOooSzeEFG}
    Soit un corps \( \eK\). Si \( G\) est un groupe dans \( \eM(n,\eK)\) (c'est-à-dire un groupe de matrices à coefficients dans \( \eK\)), alors l'application
    \begin{equation}
        \begin{aligned}
                \rho\colon G&\to \GL(\eK^n) \\
            A&\mapsto f_A 
        \end{aligned}
    \end{equation}
    où \( f_A\) est l'application linéaire associée à \( A\) est une représentation de \( G\).
\end{proposition}




\chapter{Classification de certains groupes}
% This is part of Mes notes de mathématique
% Copyright (c) 2011-2020
%   Laurent Claessens
% See the file fdl-1.3.txt for copying conditions.

%+++++++++++++++++++++++++++++++++++++++++++++++++++++++++++++++++++++++++++++++++++++++++++++++++++++++++++++++++++++++++++
\section{Théorèmes de Sylow}
%+++++++++++++++++++++++++++++++++++++++++++++++++++++++++++++++++++++++++++++++++++++++++++++++++++++++++++++++++++++++++++

\begin{lemma}
    Soient \( H\) et \( K\) des sous-groupes finis de \( G\). Alors
    \begin{equation}
        \Card(HK)=\frac{ | H |\cdot | K | }{ | H\cap K | }.
    \end{equation}
\end{lemma}
Attention : dans ce lemme, l'ensemble \( HK\) n'est pas spécialement un groupe. Ce serait le cas si \( H\) normaliserait \( K\), c'est-à-dire si nous avions \( hkh^{-1}\in K,\,\forall h,k\in H\times K\).

\begin{theorem}[Théorème de Cauchy\cite{ooBZOQooUnlnoI}]\label{ThoCauchyGpFini}
    Soit \( G\) un groupe fini et \( p\) un nombre premier divisant \( | G |\). Alors
    \begin{enumerate}
        \item
            \( G\) contient un élément d'ordre \( p\).
        \item
            Si \( G\) est un \( p\)-groupe, il existe un élément central d'ordre \( p\) dans \( G\).
    \end{enumerate}
\end{theorem}
\index{Cauchy!théorème}

\begin{lemma}[Théorème de Cayley]    \label{ThoIfdlEB}   \index{Cayley!théorème}
    Si \( G\) est un groupe d'ordre \( n\) alors il est isomorphe à un sous-groupe du groupe symétrique \( S_n\).
\end{lemma}

\begin{proof}
    L'action à gauche de \( G\) sur lui-même
    \begin{equation}
        \begin{aligned}
            \varphi\colon G&\to S_n \\
            \varphi(x)g&\mapsto xg
        \end{aligned}
    \end{equation}
    est une permutation des éléments de \( G\). Cela donne un morphisme injectif parce que si \( \varphi(x)=\varphi(y)\) nous avons \( xg=yg\) pour tout \( g\) et en particulier pour \( g=e\) nous trouvons \( x=y\).
\end{proof}

\begin{lemma}       \label{LemaQxjcm}
    Soit \( p\) un diviseur premier de \( n\). Alors le groupe symétrique \( S_n\) se plonge dans \( \GL_n(\eF_p)\).
\end{lemma}

\begin{proof}
    Soit \( \{ e_i \}\) la base canonique de \( \eF_p\). Nous avons le morphisme injectif $\varphi\colon S_n\to \GL(n,\eF)$ donné par \( \varphi(\sigma)e_i=e_{\sigma(i)}\).
\end{proof}

\begin{remark}  \label{RemFzxxst}
    En mettant bout à bout les lemmes~\ref{ThoIfdlEB} et~\ref{LemaQxjcm}, nous trouvons que si \( p\) est un diviseur premier de \( | G |\), alors \( G\) peut être vu comme un sous-groupe de \( \GL(n,\eF_p)\).
\end{remark}

\begin{definition}      \label{DEFooPRCHooVZdwST}
    Soit \( p\) un nombre premier. Un \defe{$p$-groupe}{$p$-groupe}\index{groupe!$p$-groupe} est un groupe dont tous les éléments sont d'ordre \( p^m\) pour un certain \( m\) (dépendant de l'élément).

    Soit \( G\) un groupe fini et \( p\), un diviseur premier de $| G |$. Un \defe{\(p\)-Sylow}{$p$-Sylow}\index{Sylow!$ p$-Sylow} dans \( G\) est un \( p\)-sous-groupe d'ordre \( p^n\) où \( p^n\) est la plus grande puissance de \( p\) divisant \( | G |\).
\end{definition}
Notons que si \( p\) est un nombre premier, alors tout groupe d'ordre \( p^m\) est un \( p\)-groupe.

\begin{lemma}
    Soit \( G\) un groupe fini et \( P\), \( Q\) des \( p\)-sous-groupes. Nous supposons que \( Q\) normalise \( P\). Alors \( PQ\) est un \( p\)-sous-groupe de \( G\).
\end{lemma}

Si \( S\) est un \( p\)-Sylow, alors \( p\) ne divise pas le nombre \( | G:S |=| G |/| S |\).

\begin{proposition}     \label{Propvocmon}
    Soit le corps fini \( \eF_p=\eZ/p\eZ\) (\( p\) premier). Soit \( T\) le sous-ensemble de \( \GL_n(\eF_p)\) formé des matrices triangulaires supérieures de rang\footnote{Définition~\ref{DefALUAooSPcmyK}.} \( n\) et dont les éléments diagonaux sont \( 1\). Alors \( T\) est un \( p\)-Sylow de \( \GL_n(\eF_p)\).
\end{proposition}

\begin{proof}
    Nous commençons par étudier le cardinal de \( \GL_n(\eF_p)\). Pour la première colonne, la seule contrainte à vérifier est qu'elle ne soit pas nulle. Il y a donc \( p^n-1\) possibilités. Pour la seconde, il faut ne pas être multiple de la première. Il y a donc \( p^n-p\) possibilités (parce qu'il y a \( p\) multiples possibles de la premières colonne). Pour la \( k\)-ième colonne, il faut éviter toutes les combinaisons linéaires des \( (k-1)\) premières colonnes. Il y a \( p^{k-1}\) telles combinaisons et donc \( p^n-p^{k-1}\) possibilités pour la \( k\)-ième colonne. Nous avons donc
    \begin{subequations}
        \begin{align}
            \Card\big( \GL(n,\eF_{p}) \big)&=(p^n-1)(p^n-p)\ldots(p^n-p^{n-1})\\
            &=p\cdot p^2\cdots p^{n-1}(p^n-1)(p^{n-1}-1)\ldots (p-1)\\
            &=p^{\frac{ n(n-1) }{2}}m
        \end{align}
    \end{subequations}
    où \( m\) est un entier qui ne divise pas \( p\).

    En ce qui concerne le cardinal de \( T\), le calcul est plus simple : pour la première ligne nous avons \( p^{n-1}\) choix (parce qu'il y a un \( 1\) qui est imposé sur la diagonale), pour la seconde \( p^{n-2}\), etc. En tout nous avons alors
    \begin{equation}
        | T |=p^{\frac{ n(n-1) }{2}},
    \end{equation}
    et \( T\) est un \( p\)-Sylow de \( \GL_n(\eF_p)\).
\end{proof}


\begin{proposition}
    Soit \( p\) un nombre premier. Un groupe fini \( G\) est un $p$-groupe si et seulement l'ordre de \( G\) est \( p^n\) pour un certain \( n\).
\end{proposition}

\begin{proof}
    Supposons que \( G\) est un $p$-groupe. Soit \( q\) un nombre premier divisant \( | G |\). Par le théorème de Cauchy (\ref{ThoCauchyGpFini}), le groupe \( G\) contient un élément d'ordre \( q\), soit \( g\) un tel élément. Étant donné que \( G\) est un $p$-groupe, \( g^{p^n}=g^q=e\) pour un certain \( n\). Donc $q=p^n$ et \( q=p\) parce que \( q\) est premier. Nous venons de prouver que \( p\) est le seul nombre premier qui divise \( | G |\). L'ordre de \( G\) est par conséquent une puissance de \( p\).

    Nous nous intéressons maintenant à l'implication inverse. Nous supposons que \( | G |=p^n\) pour un certain entier \( n\geq 0\). Soit \( g\in G\); nous notons \( r\) l'ordre de \( G\). Le sous-groupe \( \gr(g)\) est d'ordre \( r\), donc \( r\) divise \( | G |\) (par le théorème~\ref{ThoLagrange} de Lagrange). Le nombre \( r\) est alors une puissance de \( p\).
\end{proof}

\begin{lemma}       \label{LemwDYQMg}
    Soit \( G\), un groupe fini de cardinal \( | G |=n\) et \( p\), un diviseur premier de \( n\). Nous notons \( n=p^m\cdot r\) où \( p\) ne divise pas \( r\). Soit \( H\) un sous-groupe de \( G\) et \( S\), un \( p\)-Sylow de \( G\). Alors il existe \( g\in G\) tel que
    \begin{equation}
        gSg^{-1}\cap H
    \end{equation}
    soit un \( p\)-Sylow de \( H\).
\end{lemma}

\begin{proof}
    Nous considérons l'ensemble \( G/S\) sur lequel \( H\) agit. Si \( a\in G\), le stabilisateur de \( [a]\) dans \( G/S\) est
    \begin{subequations}
        \begin{align}
            \Fix\big( [a] \big)&=\{ h\in H\tq [ha]=[a] \}\\
            &=\{ h\in H\tq a^{-1}ha\in S\}\\
            &=aSa^{-1}\cap H.
        \end{align}
    \end{subequations}
    Nous cherchons \( a\in G\) tel que l'entier
    \begin{equation}        \label{EqZpUbWx}
        \frac{ \Card(H) }{ \Card\big( aSa^{-1}\cap H \big) }
    \end{equation}
    soit premier avec \( p\). En effet, dans ce cas le groupe \( \Fix([a])\) est un $p$-Sylow de \( H\) parce que \( | H:aSa^{-1}\cap H |\) ne divise pas \( p\). La formule des orbites (équation \eqref{EqCewSXT}) nous dit que
    \begin{equation}
        \frac{ | H | }{ | aSa^{-1}\cap H | }=\Card\big( \mO_{[a]} \big).
    \end{equation}
    Supposons que toutes les orbites aient un cardinal divisible par \( p\). Étant donné que \( G/S\) est une réunion disjointe de ses orbites, nous aurions
    \begin{equation}
        p\divides \Card(G/S)=\frac{ | G | }{ | S | }
    \end{equation}
    alors que \( S\) étant un $p$-Sylow, \( p\) ne peut pas diviser \( | G |/| S |\). Toutes les orbites n'ont donc pas un cardinal divisible par \( p\), et il existe un \( a\in G\) tel que \eqref{EqZpUbWx} soit vérifiée.
\end{proof}


\begin{theorem}[Théorème de Sylow]  \label{ThoUkPDXf}
    Soit \( G\) un groupe fini et \( p\), un diviseur premier de \( | G |\). Alors
    \begin{enumerate}
        \item       \label{ITEMooETYHooXlUMQZ}
            \( G\) possède au moins un \( p\)-Sylow\footnote{Définition~\ref{DEFooPRCHooVZdwST}.}.
        \item
            Tout \( p\)-sous-groupe de \( G\) est contenu dans un \( p\)-Sylow.
        \item   \label{ItemMzNRVf}
            Les \( p\)-Sylow de \( G\) sont conjugués.
        \item   \label{ItemkYbdzZ}
            Si \( n_p\) est le nombre de $p$-Sylow de \( G\), alors \( n_p\) divise \( | G |\) et \( n_p\in[1]_p\).
    \end{enumerate}
\end{theorem}
\index{groupe!fini}

\begin{proof}
    En plusieurs points.
    \begin{enumerate}
        \item

            Nous savons de la remarque~\ref{RemFzxxst} que \( G\) est un sous-groupe de \( \GL_n(\eF_p)\) et que ce dernier a un $p$-Sylow par la proposition~\ref{Propvocmon}. Par conséquent \( G\) possède un $p$-Sylow par le lemme~\ref{LemwDYQMg}.

        \item

            Soit \( H\) un \( p\)-sous-groupe de \( G\) et \( S\), un $p$-Sylow de \( G\) (qui existe par le point précédent). Par le lemme~\ref{LemwDYQMg} il existe \( a\in G\) tel que \( aSa^{-1}\cap H\) soit un $p$-Sylow de \( H\). Mais \( H\) est un \(p\)-groupe et un $p$-Sylow dans un \( p\)-groupe est automatiquement le groupe entier. Par conséquent,
            \begin{equation}
                H=aSa^{-1}\cap H
            \end{equation}
            et \( H\subset aSa^{-1}\), ce qui signifie que \( H\) est inclus dans un $p$-Sylow.

        \item

            Soit \( H\) un $p$-Sylow. Nous venons de voir que si \( S\) est un $p$-Sylow quelconque, alors \( H\) est inclus au $p$-Sylow \( aSa^{-1}\) pour un certain \( a\in G\). Donc \( H\) est un $p$-Sylow inclus dans le $p$-Sylow \( aSa^{-1}\), donc \( H=aSa^{-1}\).

        \item

            Le fait que \( n_p\) divise \( n\) est parce que tous les $p$-Sylow ont le même nombre d'éléments (ils sont conjugués) et sont deux à deux disjoints. Donc ils forment une partition de \( G\) et \( | G |=n_p| S |\) si \( S\) est un $p$-Sylow quelconque.

            Montrons maintenant que \( n_p\) est congru à un modulo \( p\). Soit \( E\) l'ensemble des $p$-Sylow de \( G\). Le groupe \( G\) agit sur \( E\) par conjugaison. Soit \( S\) un $p$-Sylow et considérons l'ensemble
            \begin{equation}
                E_S=\{ T\in E\tq s\cdot T=T\forall s\in S \}.
            \end{equation}
            où l'action est celle par conjugaison. C'est l'ensemble des points fixes de \( E\) sous l'action de \( S\). L'ensemble \( E\) est la réunion des orbites sous \( S\) et chacune de ces orbites a un cardinal qui divise \( | S |=p^m\). Par conséquent \( | \mO_T |\) vaut \( 1\) lorsque \( T\in E_S\) et est un multiple de \( p\) sinon. Nous avons donc
            \begin{equation}
                | E |\equiv | E_S |\mod p.
            \end{equation}
            Nous voulons obtenir \( | E_S |=1\). Évidemment \( S\in E_S\) parce que si \( s\in S\) alors \( sSs^{-1}=S\). Nous voudrions montrer que \( S\) est le seul élément de \( E_S\). Soit \( T\in E_S\), c'est-à-dire que \( T\) est un $p$-Sylow de \( G\) tel que
            \begin{equation}
                sTs^{-1}=T
            \end{equation}
            pour tout \( s\in S\). Soit \( N\) le groupe engendré par \( S\) et \( T\). Montrons que \( T\) est normal dans \( N\). Un élément \( g\) dans \( N\) s'écrit
            \begin{equation}
                g=s_1t_1\cdots s_rt_r
            \end{equation}
            avec \( s_i\in S\) et \( t_i\in T\). Si \( t\in T\), en utilisant le fait que \( T\) est un groupe et le fait que \( S\) le normalise, nous avons
            \begin{equation}
                gtg^{-1}=s_1t_1\ldots s_rt_rtt_r^{-1}s_r^{-1}\ldots t_1^{-1}s_1^{-1}\in T.
            \end{equation}
            Donc \( T\) est un sous-groupe normal de \( N\). Mais \( S\) et \( T\) sont conjugués dans \( N\) (parce que ils sont des $p$-Sylow de \( N\)), donc il existe un élément \( a\in N\) tel que \( aTa^{-1}=S\). Mais étant donné que \( T\) est normal,
            \begin{equation}
                S=aTa^{-1}=T.
            \end{equation}
            Ceci achève la démonstration des théorèmes de Sylow.

    \end{enumerate}
\end{proof}

\begin{proposition}
    Si \( S\) est un \( p\)-Sylow dans le groupe \( G\) alors pour tout \( g\in G\), l'ensemble \( gSg^{-1}\) est encore un \( p\)-groupe.
\end{proposition}

\begin{proof}
    Si les éléments de \( S\) sont d'ordre \( p^n\), alors nous avons
    \begin{equation}
        (gsg^{-1})^q=gs^qg^{-1}=e.
    \end{equation}
    Pour avoir \( gs^qg^{-1}=e\), il faut et suffit que \( gs^q=g\), alors \( s^q=e\), c'est-à-dire \( q=p^n\). Donc \( gSg^{-1}\) est encore un \( p\)-Sylow.
\end{proof}

\begin{lemma}[\cite{ooGQNTooEiWtsy}]\label{Lemcmbzum}
    Soit \( G\), un groupe fini et \( p\), un nombre premier. Si \( H\) et \( K\) sont des groupes distincts d'ordre \( p\), alors \( H\cap K=\{ e \}\).
\end{lemma}

\begin{proof}
    L'ensemble \( H\cap K\) est un sous-groupe de \( H\). Par conséquent son ordre divise celui de \( H\) qui est un nombre premier. Par conséquent soit \( | H\cap K |=1\), soit \( | H\cap K |=| H |\). Dans le second cas nous aurions \( H=K\), alors que nous avons supposé que \( H\) et \( K\) étaient distincts.
\end{proof}

\begin{proposition}[\cite{ooGQNTooEiWtsy}] \label{PropyfhTmf}
    Soit \( G\) un groupe fini et \( n\) le nombre de sous-groupes d'ordre \( p\) dans \( G\). Alors le nombre d'éléments d'ordre \( p\) dans \( G\) vaut \( n(p-1)\).
\end{proposition}

\begin{proof}
    Si \( g\) est un élément d'ordre \( p\) dans \( G\), le groupe \( H\) engendré par \( g\) est d'ordre \( p\). Réciproquement si \( H\) est un groupe d'ordre \( p\), tous les éléments de \( H\setminus\{ e \}\) sont d'ordre \( p\) (parce que l'ordre d'un élément divise l'ordre du groupe). Donc l'ensemble des éléments d'ordre \( p\) dans \( G\) est la réunion des ensembles \( H\setminus\{ e \}\) où \( H\) parcourt les sous-groupes d'ordre \( p\) dans \( G\). Chacun de ces ensembles possède \( p-1\) éléments et le lemme~\ref{Lemcmbzum} nous assure qu'ils sont disjoints. Par conséquent nous avons \( n(p-1)\) éléments d'ordre \( p\) dans \( G\).
\end{proof}

\begin{corollary}
    Un groupe d'ordre premier est cyclique.
\end{corollary}

\begin{proof}
    Soit \( p\) l'ordre de \( G\). Le nombre de sous-groupes d'ordre \( p\) est \( n=1\) (et c'est \( G\) lui-même). La proposition~\ref{PropyfhTmf} nous dit alors que le nombre d'éléments d'ordre \( p\) dans \( G\) est \( p-1\). Donc tout élément est générateur.
\end{proof}


%+++++++++++++++++++++++++++++++++++++++++++++++++++++++++++++++++++++++++++++++++++++++++++++++++++++++++++++++++++++++++++
\section{Groupe monogène}
%+++++++++++++++++++++++++++++++++++++++++++++++++++++++++++++++++++++++++++++++++++++++++++++++++++++++++++++++++++++++++++
\label{SECooXIHPooWVSjhT}

Le théorème suivant donne quelques informations à propos de groupes monogènes. Il impliquera dans le corolaire~\ref{CORooMBLSooMHKmAq} qu'un groupe monogène d'ordre \( n\) possède \( \varphi(n)\) générateur où \( \varphi\) est la fonction indicatrice d'Euler définie en~\ref{DEFooWYIGooRVBTil}.

\begin{theorem}     \label{THOooDOMZooOEYHAe}
    Un groupe monogène est abélien. Plus précisément,
    \begin{enumerate}
        \item
            un groupe monogène infini est isomorphe à \( \eZ\),
        \item
            un groupe monogène fini est isomorphe à \( \eZ/n\eZ\) pour un certain \( n\).
    \end{enumerate}
\end{theorem}

\begin{proof}
    Le groupe est abélien parce que $g=a^n$, \( g'=a^{n'}\) implique \( gg'=q^{n+n'}=g'g\). Nous considérons un générateur \( a\) de \( G\) (qui existe parce que $G$ est monogène) et le morphisme surjectif
    \begin{equation}
        \begin{aligned}
            f\colon \eZ&\to G \\
            p&\mapsto a^p.
        \end{aligned}
    \end{equation}
    Si \( G\) est infini, alors \( f\) est injective parce que si \( a^n=a^{n'}\), alors \( a^{n-n'}=e\), ce qui rendrait \( G\) cyclique et par conséquent non infini. Nous concluons que si \( G\) est infini, alors \( f\) est une bijection et donc un isomorphisme \( \eZ\simeq G\).

    Si \( G\) est fini, alors \( f\) n'est pas injective et a un noyau \( \ker f\). Étant donné que \( \ker f\) est un sous-groupe de \( G\), il existe un (unique) \( n\) tel que \( \ker f=n\eZ\) et le premier théorème d'isomorphisme (théorème~\ref{ThoPremierthoisomo}) nous indique que
    \begin{equation}
        \eZ/\ker f=\eZ/n\eZ=\Image f=G.
    \end{equation}

\end{proof}

Le lemme suivant donne une démonstration alternative, avec une construction plus explicite de l'isomorphisme.

\begin{lemma}[\cite{MonCerveau}]   \label{LemZhxMit}

    À propos de groupes monogènes\footnote{Définition~\ref{DEFooWMFVooLDqVxR}.}

    \begin{enumerate}
        \item


    Soit un groupe monogène \( G\) d'ordre fini \( n\) dont \( g\) est un générateur. Alors il existe un isomorphisme
    \begin{equation}
        \phi\colon G\to (\eZ/n\eZ,+)
    \end{equation}
    tel que \( \phi(g)=1\).

\item

    Si \( G\) est un groupe monogène d'ordre infini et si \( g\) est un générateur, alors il existe un isomorphisme
    \begin{equation}
        \phi\colon G\to (\eZ,+)
    \end{equation}
    tel que \( \phi(g)=1\).

   \item

    Soient \( G\) et \( H\) deux groupes monogènes de même ordre. Soient \( g\) un générateur de \( G\) et \( h\), un générateur de \( H\). Il existe un isomorphisme de \( G\) sur \( H\) qui envoie \( g\) sur \( h\).
    \end{enumerate}
\end{lemma}

\begin{proof}

    Commençons par enfoncer une porte ouverte : vu que le groupe est monogène, l'ordre du groupe est égal à l'ordre de son générateur. Nous séparons les cas selon quel l'ordre soit fini ou non.

    \begin{subproof}
        \item[L'ordre de \( G\) est fini et vaut \( n\)]
            Si \( k\in\eZ\), nous notons \( [k]_n\) la classe de \( k\) modulo \( n\), c'est-à-dire l'ensemble \( \{ k+pn\tq p\in \eZ \}\).

            Nous construisons l'isomorphisme \( \phi\colon G\to \eZ/n\eZ\) de la façon suivante:
            \begin{equation}
                \phi(g^m)=[m]_n.
            \end{equation}
            Cela est une bonne définition parce que une égalité du type \( g^m=g^{m'}\) implique que \( m\) et \( m'\) soient dans la même classe modulo \( n\). Nous vérifions que cela est une isomorphisme entre \( G\) et \( \eZ/n\eZ\).

            \begin{subproof}
            \item[Morphisme]
                Pour l'identité, si \( x=e\) alors \( m=0\) et \( \phi(e)=[0]_n\). Et si \( x=g^k\), \( y=g^l\) alors \( \phi(xy)=\phi(g^{k+l})=[k+l]_n=[k]_n+[l]_n=\phi(x)+\phi(y) \).
            \item[Injectif]
                Supposons \( \phi(g^k)=\phi(g^l)\) avec \( k\geq l\). Nous avons \( h^k=h^l\), dont \( h^{k-l}=e\), ce qui donne \( k-l\in [0]_n\) ou encore \( [k]_n=[l]_n\). En particulier \( g^k=g^l\).
            \item[Surjectif]
                La classe \( [k]_n\) est l'image de \( g^k\).
            \end{subproof}

        \item[L'ordre de \( G\) est infini]

                Si l'ordre de \( G\) est infini alors un élément \( x\in G\) s'écrit de façon unique sous la forme \( x=g^m\) avec \( m\in \eZ\). Dans ce cas nous définissons directement \( \phi(g^m)=m\).

                Le reste de la preuve est alors identique au cas d'ordre fini, mais sans les complications liées au modulo.

    \end{subproof}

    La dernière assertion s'obtient des précédentes par composition d'isomorphismes.

\end{proof}

%+++++++++++++++++++++++++++++++++++++++++++++++++++++++++++++++++++++++++++++++++++++++++++++++++++++++++++++++++++++++++++
\section{Automorphismes du groupe \texorpdfstring{$ \eZ/n\eZ$}{Z/nZ}}
%+++++++++++++++++++++++++++++++++++++++++++++++++++++++++++++++++++++++++++++++++++++++++++++++++++++++++++++++++++++++++++

Notons que \( \eZ/n\eZ=\eF_n\) est un groupe pour l'addition tandis que \( (\eZ/n\eZ)^*\) est un groupe pour la multiplication. Il ne peut donc pas y avoir d'équivoque.

\begin{theorem}[\cite{ooUDKTooTWYpzN}]   \label{ThoozyeSn}
    Pour chaque \( x\in (\eZ/n\eZ)^*\) nous considérons l'application
    \begin{equation}
        \begin{aligned}
            \sigma_x\colon \eZ/n\eZ&\to \eZ/n\eZ \\
            y&\mapsto xy.
        \end{aligned}
    \end{equation}
    L'application
    \begin{equation}
        \sigma\colon \big( (\eZ/n\eZ)^*,\cdot\big)\to \Aut\big( \eZ/n\eZ,+ \big)
    \end{equation}
    ainsi définie est un isomorphisme de groupes.
\end{theorem}
L'énoncé de ce théorème s'écrit souvent rapidement par
\begin{equation}
    \Aut(\eZ/n\eZ)=(\eZ/n\eZ)^*,
\end{equation}
mais il faut bien garder à l'esprit qu'à gauche on considère le groupe additif et à droite celui multiplicatif.

\begin{proof}
    Nous notons \( [x]\) la classe de \( x\) dans \( \eZ/n\eZ\). Nous avons \( \eZ/n\eZ=[1]\). Soit \( f\) un automorphisme de \( (\eZ/n\eZ,+)\); pour tout \( r\in \eZ\) nous avons
    \begin{equation}
        f([r])=f(r[1])=rf([1])=[r]f([1]).
    \end{equation}
En particulier, vu que \( f\) est surjective, il existe un \( r\) tel que \( f([r])=[1]\). Pour un tel \( r\) nous avons \( [1]=[r]f([1])\), c'est-à-dire que nous avons montré que \( f([1])\) est inversible dans \(  \big( (\eZ/n\eZ)^*,\cdot\big)\). Nous montrons à présent que\footnote{Le \( \sigma\) donné ici est l'inverse de celui donné dans l'énoncé. Cela ne change évidemment rien à la validité de l'énoncé et de la preuve.}
    \begin{equation}
        \begin{aligned}
            \sigma\colon \Aut( (\eZ/n\eZ,+))&\to \big( (\eZ/n\eZ)^*,\cdot \big)\\
            f&\mapsto f([1])
        \end{aligned}
    \end{equation}
    est un isomorphisme.

    Nous commençons par la surjectivité. Soit \( [a]\in (\eZ/n\eZ)^*\). Les élément \( [a]\) et \( [1]\) étant tous deux des générateurs de \( (\eZ/n\eZ,+)\), il existe un automorphisme de \( \eZ/n\eZ\) qui envoie \( [1]\) sur \( [a]\) par le lemme~\ref{LemZhxMit}. Cela prouve la surjectivité de \( \sigma\).

    En ce qui concerne l'injectivité, considérons des automorphismes \( f_1\) et \( f_2\) de \( (\eZ/n\eZ,+)\) tels que \( f_1([1])=f_2([1])\). Les automorphismes \( f_1\) et \( f_2\) prennent la même valeur sur un générateur et donc sur tout le groupe. Donc \( f_1=f_2\).

    Enfin nous prouvons que \( \sigma\) est un morphisme, c'est-à-dire que \( \sigma(f\circ g)=\sigma(f)\sigma(g)\). Nous avons
    \begin{subequations}
        \begin{align}
            f\big( g([1]) \big)&=f\big( g([1])[1] \big)=g([1])f([1])=\sigma(f)\sigma(g).
        \end{align}
    \end{subequations}
\end{proof}

Ce dernier résultat s'étend aux groupes cycliques.
\begin{proposition}     \label{PROPooBZOMooVOHoYf}
    Si \( G\) est un groupe cyclique\footnote{Définition \ref{DefHFJWooFxkzCF}.} d'ordre \( n\), alors
    \begin{equation}
        \Aut(G)=(\eZ/n\eZ)^*.
    \end{equation}
\end{proposition}

\begin{corollary}       \label{CorwgmoTK}
    Si \( p\) divise \( q-1\) alors \( \Aut(\eF_q)\) possède un unique sous-groupe d'ordre \( p\).
\end{corollary}

\begin{proof}
    Si \( a\) est un générateur de \( \eF_q^*\) alors le groupe
    \begin{equation}    \label{EqAdGiil}
        \gr\left( a^{\frac{ q-1 }{ p }} \right)
    \end{equation}
    est un sous-groupe d'ordre \( p\). En ce qui concerne l'unicité, soit \( S\) un sous-groupe d'ordre \( p\). Il est donc d'indice \( (q-1)/p\) dans \( \eF_q^*\) et le lemme~\ref{PropubeiGX} nous enseigne que le groupe donné en \eqref{EqAdGiil} est contenu dans \( S\). Il est donc égal à \( S\) parce qu'il a l'ordre de \( S\). Le fait que \( S\) soit normal est dû au fait que \( \eF_q^*\) est abélien.
\end{proof}




%+++++++++++++++++++++++++++++++++++++++++++++++++++++++++++++++++++++++++++++++++++++++++++++++++++++++++++++++++++++++++++
\section{Groupes abéliens finis}
%+++++++++++++++++++++++++++++++++++++++++++++++++++++++++++++++++++++++++++++++++++++++++++++++++++++++++++++++++++++++++++

Source : \cite{FabricegPSFinis}.

Nous rappelons que l'exposant\index{exposant} d'un groupe fini est le \( \ppcm\) des ordres de ses éléments. Dans le cas des groupes abéliens finis, l'exposant joue un rôle important du fait qu'il existe un élément dont l'ordre est l'exposant. Cela est le théorème suivant.

\begin{theorem}[Exposant dans un groupe abélien fini]
    Un groupe abélien fini contient un élément dont l'ordre est l'exposant du groupe.
\end{theorem}

\begin{proof}
    Soit \( G\) un groupe abélien fini et \( x\in G\), un élément d'ordre maximum \( m\). Nous montrons par l'absurde que l'ordre de tous les éléments de \( G\) divise \( m\). Soit donc \( y\in G\), un élément dont l'ordre ne divise pas \( m\); nous notons $q$ son ordre. Vu que \( q\) ne divise pas \( m\), le nombre \( q\) possède au moins un facteur premier plus de fois que \( m\) : soit \( p\) premier tel que la décomposition de \( q\) contienne \( p^{\beta}\) et celle de \( m\) contienne \( p^{\alpha}\) avec \( \beta>\alpha\). Autrement dit,
    \begin{subequations}
        \begin{align}
            m=p^{\alpha}m'\\
            q=p^{\beta}q'
        \end{align}
    \end{subequations}
    où \( m'\) et \( q'\) ne contiennent plus le facteur \( p\). L'élément \( x\) étant d'ordre \( m\), l'élément \( x^{p^{\alpha}}\) est d'ordre \( m'\). De la même manière, l'élément \( y^{q'}\) est d'ordre \( p^{\beta}\). Étant donné que \( p^{\beta}\) et \( m'\) sont premiers entre eux, l'élément  \( x^{p^{\alpha}}y^{q'}\) est d'ordre \( p^{\alpha}m'>m\). D'où une contradiction avec le fait que \( x\) était d'ordre maximal.

    Par conséquent l'ordre de tous les éléments de $G$ divise celui de \( x\) qui est alors le \( \ppcm\) des ordres de tous les éléments de \( G\), c'est-à-dire l'exposant de \( G\).
\end{proof}

\begin{proposition} \label{PropfPRVxi}
    Soit \( G\) un groupe abélien fini et \( x\in G\), un élément d'ordre maximum. Alors
    \begin{enumerate}
        \item
            Il existe un morphisme \( \varphi\colon G\to \gr(x)\) tel que \( \varphi(x)=x\).
        \item   \label{ItemKRYwjU}
            Il existe un sous-groupe \( K\) de \( G\) tel que \( G=\gr(x)\oplus K\).
    \end{enumerate}
\end{proposition}

\begin{proof}
    Nous notons \( a\) l'ordre de \( x\) qui est également l'exposant du groupe \( G\).

    Nous allons prouver la première partie par récurrence sur l'ordre du groupe. Si \( G=\gr(x)\), alors c'est évident. Soit \( H\) un sous-groupe propre de \( G\) contenant \( x\) et tel que le problème soit déjà résolu pour \( H\) : il existe un morphisme \( \varphi\colon H\to \gr(x)\) tel que \( \varphi(x)=x\). Soit \( y\in G\setminus H\), d'ordre \( b\). Nous allons trouver un morphisme $\hat\varphi\colon \gr(H,y)\to \gr(x) $ telle que \( \hat\varphi(x)=x\).

    Pour cela nous commençons par construire les applications suivantes :
    \begin{equation}
        \begin{aligned}
            \tilde \varphi\colon \eZ/b\eZ\times H&\to \gr(x) \\
            (\bar k,h)&\mapsto x^{kl}\varphi(h)
        \end{aligned}
    \end{equation}
    où \( l\) est encore à déterminer, et
    \begin{equation}
        \begin{aligned}
            p\colon \eZ/b\eZ\times H&\to \gr(y,H) \\
            (\bar k,h)&\mapsto y^kh.
        \end{aligned}
    \end{equation}
    Pour que \( \tilde \varphi\) soit bien définie, il faut que \( a\) divise \( bl\). L'application \( p\) est bien définie parce que \( \bar k\) est pris dans \( \eZ/b\eZ\) et que \( b\) est l'ordre de \( y\).

    Nous allons construire le morphisme \( \hat \varphi\) en considérant le diagramme
    \begin{equation}
    \xymatrix{%
    \ker(p) \ar@{^{(}->}[r]        &   \eZ/b\eZ\times H\ar[d]_{\tilde \varphi}\ar[r]^p&\gr(y,H)\ar[ld]^{\hat \varphi}\\
          &   \gr(x)
       }
    \end{equation}
    que l'on voudra être commutatif. Vu que \( p\) est surjective, les théorèmes d'isomorphismes nous disent que
    \begin{equation}
        \gr(y,H)\simeq\frac{ \eZ/b\eZ\times H }{ \ker p }.
    \end{equation}
    Si \( [\bar k,h]\) est la classe de \( (\bar k,h)\) modulo \( \ker(p)\) alors nous voudrions définir \( \hat \varphi\) par
    \begin{equation}        \label{EqeesVxc}
        \hat\varphi\big( [\bar k,h] \big)=\tilde \varphi(\bar k,h).
    \end{equation}
    Pour que cela soit bien définit, il faut que si \( (\bar r,z)\in \ker p\),
    \begin{equation}
        \hat\varphi\big( [\bar k\bar r,hz] \big)=\hat\varphi\big( [\bar k,h] \big),
    \end{equation}
    c'est-à-dire que \( \tilde \varphi(\bar r,z)=e\). Du coup la définition \eqref{EqeesVxc} n'est bonne que si et seulement si
    \begin{equation}
        \ker(p)\subset\ker(\tilde\varphi ).
    \end{equation}
    Nous pouvons obtenir cela en choisissant bien \( l\).

    Déterminons d'abord le noyau de \( p\). Pour cela nous considérons un nombre \( \beta\) divisant \( b\) tel que \( \gr(y)\cap H=\gr(y^{\beta})\). Nous aurons \( p(\bar k,h)=e\) si et seulement si \( y^h=e\). En particulier \( h=y^{-k}\in\gr(y)\cap H=\gr(y^{\beta})\). Si \( h=(y^{\beta})^m=y^{m\beta}\), alors \( k=-m\beta\) et nous avons
    \begin{equation}
        \ker(p)=\{ (-m\beta,y^{m\beta})\tq m\in \eZ \}.
    \end{equation}
    En plus court : \( \ker(p)=\gr(\beta,y^{-\beta})\). Nous devons donc fixer \( l\) de telle sorte que \( \tilde \varphi(\beta,y^{-\beta})=e\). Étant donné que \( \varphi\) prend ses valeurs dans \( \gr(x)\), il existe un entier \( \alpha\) tel que \( \varphi(y^{-\beta})=x^{\alpha}\); en utilisant cet \( \alpha\), nous écrivons
    \begin{equation}
        \tilde \varphi(\beta,y^{-\beta})=x^{\beta l}\varphi(y^{-\beta})=x^{\beta l+\alpha}.
    \end{equation}
    Par conséquent nous choisissons \( l=-\alpha/\beta\). Nous devons maintenant vérifier que ce choix est légitime, c'est-à-dire que \( a\) divise \( bl\) et que \( \alpha/\beta\) est un entier.

    Étant donné que \( y\) est d'ordre \( b\),
    \begin{equation}
        e=\varphi(y^b)=\varphi(y^{-\beta b/\beta})=\varphi(y^{-\beta})^{b/\beta}=x^{b\beta/\alpha}.
    \end{equation}
    Par conséquent \( a\) divise \( \frac{ b\alpha }{ \beta }=-bl\).

    Pour voir que \( l\) est entier, nous nous rappelons que \( a\) est l'exposant de \( G\) (parce que \( x\) est d'ordre maximum) et que par conséquent \( b\) divise \( a\). Mais \( a\) divise \( \alpha\frac{ b }{ \beta }\). Donc \( \alpha/\beta\) est entier.

    Nous passons maintenant à la seconde partie de la preuve. Nous considérons un morphisme \( \varphi\colon G\to \gr(x)\) tel que \( \varphi(x)=x\). La première partie nous en assure l'existence. Nous montrons que
    \begin{equation}
        \begin{aligned}
            \psi\colon G&\to \gr(x)\oplus \ker(\varphi) \\
            g&\mapsto \big( \varphi(g),g\varphi(g)^{-1} \big)
        \end{aligned}
    \end{equation}
    est un isomorphisme. D'abord \( g\varphi(g)^{-1}\) est dans le noyau de \( \varphi\) parce que \( \varphi(g)^{-1}\) étant dans \( \gr(x)\), et \( \varphi\) étant un morphisme,
    \begin{equation}
        \varphi\big( g\varphi(g)^{-1} \big)=\varphi(g)\varphi(g)^{-1}=e.
    \end{equation}
    L'application \( \psi\) est un morphisme parce que, en utilisant le fait que \( G\) est abélien,
    \begin{subequations}
        \begin{align}
            \psi(g_1g_2)&=\big( \varphi(g_1g_2),g_1g_2\varphi(g_1g_2)^{-1} \big)\\
            &=\big( \varphi(g_1)\varphi(g_2),g_1\varphi(g_1)^{-1}g_2\varphi(g_2)^{-1} \big)\\
            &=\psi(g_1)\psi(g_2).
        \end{align}
    \end{subequations}
    L'application \( \psi\) est injective parce que si \( \psi(g)=(e,e)\) alors \( \varphi(g)=e\) et \( g\varphi(g)^{-1}=e\), ce qui implique \( g=e\).

    Enfin \( \psi\) est surjective parce qu'elle est injective et que les ensembles de départ et d'arrivée ont même cardinal. En effet par le premier théorème d'isomorphisme (théorème~\ref{ThoPremierthoisomo}) appliqué à \( \varphi\) nous avons
    \begin{equation}
        | G |=| \gr(x) |\cdot | \ker(\varphi) |.
    \end{equation}
\end{proof}

\begin{theorem} \label{ThoRJWVJd}
    Tout groupe abélien fini (non trivial) se décompose en
    \begin{equation}
        G\simeq \eZ/d_1\eZ\oplus\ldots\oplus \eZ/d_r\eZ
    \end{equation}
    avec \( d_1\geq 1\) et \( d_i\) divise \( d_{i+1}\) pour tout \( i=1,\ldots, r-1\).

    De plus la liste \( (d_1,\ldots, d_r)\) vérifiant ces propriétés est unique.
\end{theorem}

\begin{proof}
    Soit \( x_1\) un élément d'ordre maximal dans \( G\). Soit \( n_1\) son ordre et
    \begin{equation}
        H_1=\gr(x_1)=\eF_{n_1}.
    \end{equation}
    D'après la proposition~\ref{PropfPRVxi}\ref{ItemKRYwjU}, il existe un supplémentaire \( K_1\) tel que \( G=\eF_{n_1}\oplus K_1\). Si \( K_1=\{ 1 \}\) on s'arrête et on garde \( G=\eF_{n_1}\). Sinon on continue de la sorte en prenant \( x_2\) d'ordre maximal dans \( K_1\) etc.

    Nous devons maintenant prouver l'unicité de cette décomposition. Soit
    \begin{equation}
        G=\eF_{d_1}\oplus\ldots\oplus \eF_{d_r}=\eF_{s_1}\oplus\ldots\oplus \eF_{s_q}.
    \end{equation}
    L'exposant de \( G\) est \( d_r\) et \( s_q\). Donc \( d_r=s_q\). Les complémentaires étant égaux nous avons
    \begin{equation}
        \eF_{d_1}\oplus\ldots\oplus \eF_{d_{r-1}}=\eF_{s_1}\oplus\ldots\oplus \eF_{s_{q-1}}.
    \end{equation}
    En continuant nous trouvons \( r=q\) et \( d_i=s_i\).
\end{proof}

%+++++++++++++++++++++++++++++++++++++++++++++++++++++++++++++++++++++++++++++++++++++++++++++++++++++++++++++++++++++++++++
\section{Groupes d'ordre \texorpdfstring{$ pq$}{pq}}
%+++++++++++++++++++++++++++++++++++++++++++++++++++++++++++++++++++++++++++++++++++++++++++++++++++++++++++++++++++++++++++
\index{quotient!de groupes}\index{sous-groupe!normal}

\begin{lemma}
    Soit \( G\) un groupe d'ordre \( pq\) où \( p\) et \( q\) sont des nombres premiers distincts. Nous supposons que \( p<q\).
    \begin{enumerate}
        \item
            Le groupe \( G\) possède un unique \( q\)-Sylow.
        \item
            Cet unique \( q\)-Sylow est normal dans \( G\).
        \item
            Il n'est ni \( \{ e \}\) ni \( G\).
        \item
            Le groupe \( G\) n'est pas un groupe simple\footnote{Pas de sous-groupes normaux non triviaux,~\ref{DefGroupeSimple}.}.
    \end{enumerate}
\end{lemma}

\begin{proof}
    Soit \( n_q\) le nombre de \( q\)-Sylow; par le théorème de Sylow~\ref{ThoUkPDXf}\ref{ITEMooETYHooXlUMQZ} le groupe \( G\) possède des \( q\)-Sylow et par~\ref{ThoUkPDXf}\ref{ItemkYbdzZ},
    \begin{equation}
        n_q\in[1]_q.
    \end{equation}
    De plus le nombre \( n_q\) divise \( | G |=pq\). Donc \( n_q\) vaut \( p\), \( q\) ou \( 1\). Avoir \( n_q=p\) n'est pas possible parce que \( n_q\in[1]_q\) et \( p<q\). Avoir \( n_q=q\) n'est pas possible non plus, pour la même raison. Donc \( n_q=1\). Notons \( H\) l'unique \( q\)-Sylow de \( G\).

    Le fait que \( H\) soit normal est une conséquence de~\ref{ThoUkPDXf}\ref{ItemMzNRVf} parce que le conjugué de \( H\) est encore un \( q\)-Sylow alors que \( H\) est l'unique \( q\)-Sylow.

    Vu que
    \begin{equation}
        1<p=| H |<pq=| G |,
    \end{equation}
    le sous-groupe \( H \) n'est ni réduit à l'identité ni le groupe entier.

    Par conséquent \( G\) n'est pas simple parce qu'il contient un sous-groupe normal non trivial.
\end{proof}

Avant le lire le théorème suivant, n'oubliez pas de lire la définition d'un produit semi-direct~\ref{DEFooKWEHooISNQzi}.
\begin{theorem}[\cite{ooUWQNooKHTzdO}] \label{ThoLnTMBy}
    Soient deux nombres premiers distincts\footnote{Le cas \( p=q\) sera traité par la proposition~\ref{PropssttFK}.} \( p\) et \( q\) avec \( q>p\).
    \begin{enumerate}
        \item
    Si \( p\) ne divise pas \( q-1\) alors tout groupe d'ordre \( pq\) est cyclique et plus précisément le seul groupe (à isomorphisme près) d'ordre \( pq\) est \( \eZ/pq\eZ\).
\item       \label{ITEMooFQXIooFLAiUD}
        Si \( p\divides q-1\), alors il n'existe que deux groupes d'ordre \( pq\) :
        \begin{itemize}
            \item Le groupe abélien et cyclique \( \eZ/pq\eZ\).
            \item Le produit semi-direct non abélien
                \begin{equation}    \label{EqNuuTRE}
                    G=\eZ/q\eZ\times_{\varphi}\eZ/p\eZ
                \end{equation}
                où \( \varphi(\bar 1)\) est d'ordre \( p\) dans \( \Aut(\eZ/q\eZ)\).
        \end{itemize}

    \item

        Si \( p\) et \( q\) sont premiers entre eux, le produit est direct\quext{Cette affirmation me semble très bizarre. Comment deux nombres premiers distincts pourraient ne pas être premiers entre eux ???}.
    \end{enumerate}
\end{theorem}
\index{sous-groupe!distingué}
\index{groupe!fini}
\index{anneau!\( \eZ/n\eZ\)}
\index{nombre!premier}

\begin{proof}
    Division de la preuve en plusieurs parties.
    \begin{subproof}
    \item[Préliminaires avec Sylow]

        Soit un groupe \( G\) d'ordre \( pq\). Soient \( H\), un \( q\)-Sylow et \( K\), un \( p\)-Sylow de \( G\). Ils existent parce que \( p\) et \( q\) sont des diviseurs premiers de \( | G |\) (théorème de Sylow~\ref{ThoUkPDXf}). Si \( n_q\) est le nombre de \( q\)-Sylow dans \( G\) alors \( n_q\) divise \( | G |\) et \( n_q=1\mod q\). Donc d'abord \( n_q\) vaut \( 1\), \( p\) ou \( q\). Ensuite \( n_q=q\) est exclu par la condition \( n_q=1\mod q\); la possibilité \( n_q=p\) est également impossible parce que \( p=1\mod q\) est impossible avec \( p<q\). Donc \( n_q=1\) et \( H\) est normal dans \( G\).

        L'ensemble \( H\cap K\) est un sous-groupe à la fois de \( H\) et de \( K\), ce qui entraine que (théorème de Lagrange~\ref{ThoLagrange}) \( | H\cap K |\) divise à la fois \( p\) et \( q\). Nous en déduisons que \( | H\cap K |=1\) et donc que \( H\cap K=\{ e \}\).

        Étant donné que \( H\) est normal, l'ensemble \( HK\) est un sous-groupe de \( G\). De plus l'application
        \begin{equation}
            \begin{aligned}
                \psi\colon H\times K&\to HK \\
                (h,k)&\mapsto hk
            \end{aligned}
        \end{equation}
        est un bijection. Nous ne devons vérifier seulement l'injectivité. Supposons que \( hk=h'k'\). Alors \( e=h^{-1}h'k'k^{-1}\), et donc
        \begin{equation}
            h^{-1} h'=(k'k^{-1})^{-1}\in H\cap K=\{ e \}.
        \end{equation}
        Par conséquent \( | pq |=| H\times K |=| HK |\), et \( HK=G\). Le corolaire~\ref{CoroGohOZ} nous indique que
        \begin{equation}    \label{EqGjQjFN}
            G=H\times_{\varphi}K
        \end{equation}
        où \( \varphi\) est l'action adjointe. Nous devons maintenant identifier cette action. En d'autres termes, nous savons que \( H=\eZ/q\eZ\) et \( K=\eZ/p\eZ\) et que \( \varphi\colon \eZ/p\eZ\to \Aut(\eZ/q\eZ)\) est un morphisme. Nous devons déterminer les possibilités pour \( \varphi\).

        Soit \( n_p\) le nombre de \( p\)-Sylow de \( G\). Comme précédemment, \( n_p\) vaut \( 1\), \( p\) ou \( q\) et la possibilité \( n_p=p\) est exclue. Donc \( n_p\) est \( 1\) ou \( q\).

    \item[Si \( p\) ne divise pas \( q-1\)]

        Si \( p\) ne divise pas \( q-1\) alors il n'est pas possible d'avoir \( n_p=q\) parce que \( n_p\in [1]_p\). Or dire \( n_p=q\) demanderait \( q\in [1]_p\), c'est-à-dire \( q=kp+1\), qui impliquerait que \( p\) divise \( q-1\).

        La seule possibilité est que \( n_p=1\). Dans ce cas, \( K\) est également normal dans \( G\). Du coup le produit semi-direct \eqref{EqGjQjFN} est en réalité un produit direct (\( \varphi\) est triviale) et nous avons
        \begin{equation}
            G=\eZ/q\eZ\times \eZ/p\eZ=\eZ/pq\eZ.
        \end{equation}

    \item[Si \( p\) divise \( q-1\)]

        Cette fois \( n_p=1\) et \( n_p=q\) sont tous deux possibles. Ce que nous savons est que \( \varphi(\eZ/p\eZ)\) est un sous-groupe de \( \Aut(\eZ/q\eZ)\). Par le premier théorème d'isomorphisme~\ref{ThoPremierthoisomo}, nous avons
        \begin{equation}
            | \varphi(\eZ/p\eZ) |=\frac{ | \eZ/p\eZ | }{ | \ker\varphi | },
        \end{equation}
        ce qui signifie que \( | \varphi(\eZ/p\eZ) |\) divise \( | \eZ/p\eZ |=p\). Par conséquent, \( | \varphi(\eZ/p\eZ) |\) est égal à \( 1\) ou \( p\). Si c'est \( 1\), alors l'action est triviale et le produit est direct.

        Nous supposons que \( | \varphi(\eZ/p\eZ) |=p\). Le corolaire~\ref{CorwgmoTK} nous indique que \( \Aut(\eZ/q\eZ)\) possède un unique sous-groupe d'ordre \( p\) que nous notons \( \Gamma\); c'est-à-dire que \( \Gamma=\Image(\varphi)\). Vu que \( \varphi\colon \eZ/p\eZ\to \Aut(\eZ/q\eZ)\) est un morphisme, \( \Gamma\) est généré par \( \varphi(\bar 1)\) qui est alors un élément d'ordre \( p\), comme annoncé.

    \item[Unicité]
        Nous nous attaquons maintenant à l'unicité. Soient \( \varphi\) et \( \varphi'\) deux morphismes non triviaux \( \eZ/p\eZ\to \Aut(\eZ/q\eZ)\). Étant donné que \( \Aut(\eZ/q\eZ)\) ne possède qu'un seul sous-groupe d'ordre \( p\), nous savons que \( \Image(\varphi)=\Image(\varphi')=\Gamma\). Nous pouvons donc parler de \( \varphi'^{-1}\) en tant qu'application de \( \eZ/p\eZ\) dans \( \Gamma\). Nous montrons que
        \begin{equation}
            \begin{aligned}
                f\colon \eZ/q\eZ\times_{\varphi}\eZ/p\eZ&\to \eZ/q\eZ\times_{\varphi'}\eZ/p\eZ \\
                (h,k)&\mapsto (h,\alpha(k))
            \end{aligned}
        \end{equation}
        où \( \alpha=\varphi'^{-1}\circ\varphi\) est un isomorphisme de groupes. Le calcul est immédiat :
        \begin{subequations}
            \begin{align}
                f(h_1,k_1)f(h_2mk_2)&=\big( h_1,\alpha(k_1) \big)(h_2,\alpha(k_2))\\
                &=\big( h_1\varphi'(\alpha(k_1))h_2m\alpha(k_1k_2) \big)\\
                &=f\big( h_1\varphi(k_1)h_2,k_1k_2 \big)\\
                &=f\big( (h_1,k_1),(h_2,k_2) \big).
            \end{align}
        \end{subequations}
        Par conséquent \( \eZ/q\eZ\times_{\varphi}\eZ/p\eZ\simeq \eZ/q\eZ\times_{\varphi'} \eZ/p\eZ\).
    \end{subproof}
\end{proof}

Note : il existe des nombres premiers \( p\) et \( q\) tels que \( q=1\mod p\). Par exemple \( 7=1\mod 3\).

\begin{proposition}[\cite{PDFpersoWanadoo}]
    Soit \( G\) un groupe fini d'ordre \( pq\) où \( p\) et \( q\) sont deux nombres premiers distincts vérifiant
    \begin{subequations}
        \begin{numcases}{}
            p\neq 1\mod q\\
            q\neq 1\mod p.
        \end{numcases}
    \end{subequations}
    Alors \( G\) est cyclique, abélien et
    \begin{equation}
        G\simeq \eZ/p\eZ\times \eZ/q\eZ.
    \end{equation}
\end{proposition}

\begin{proof}
    Soient \( n_p\) et \( n_q\) les nombres de \( p\)-Sylow et \( q\)-Sylow. Par le théorème de Sylow~\ref{ThoUkPDXf}, \( n_p\) divise \( pq\) et \( n_p=1\mod p\). Le second point empêche \( n_p\) de diviser \( p\). Par conséquent \( n_p\) divise \( q\) et donc \( n_p\) vaut \( 1\) ou \( q\). La possibilité \( n_p=q\) est exclue par l'hypothèse \( q\neq 1\mod p\). Donc \( n_p=1\), et de la même façon nous obtenons \( n_q=1\).

    Soient \( S\) l'unique \( p\)-Sylow et \( T\), l'unique \( q\)-Sylow. Pour les mêmes raisons que celles exposée plus haut, ce sont deux sous-groupes normaux dans \( G\). Étant donné que \( S\) est d'ordre \( p^n\) pour un certain \( n\) et que l'ordre de \( S\) doit diviser celui de \( G\), nous avons \( |S|=p\). De la même façon, \( | T |=q\). Par conséquent \( S\) est un groupe cyclique d'ordre \( p\) et nous considérons \( x\), un de ses générateurs. De la même façon soit \( y\), un générateur de \( T\).

    Nous montrons maintenant que \( x\) et \( y\) commutent, puis que \( xy\) engendre \( G\). Nous savons que \( S\cap T\) est un sous-groupe à la fois de \( S\) et de \( T\), de telle façon que \( | S\cap T |\) divise à la fois \( | S |=p\) et \( | T |=q\). Nous avons donc \( | S\cap T |=1\) et donc \( S\cap T\) se réduit au neutre. Par ailleurs, \( S\) et \( T\) sont normaux, donc
    \begin{subequations}
        \begin{align}
            (xyx^{-1})y^{-1}\in T\\
            x(yx^{-1})y^{-1})\in S,
        \end{align}
    \end{subequations}
    donc \( xyx^{-1}y^{-1}=e\), ce qui montre que \( xy=yx\).

    Montrons que \( xy\) engendre \( G\). Soit \( m>0\) tel que \( (xy)^m=e\). Pour ce \( m\) nous avons \( x^m=y^{-m}\) et \( y^{-m}=x^m\), ce qui signifie que \( x^m\) et \( y^m\) appartiennent à \( S\cat T\) et donc \( x^m=y^m=e\). Les nombres \( p\) et \( q\) divisent donc tous deux \( m\); par conséquent \( \ppcm(p,q)=pq\) divise \( m\). Nous en concluons que \( xy\) est d'ordre \( pq\) (il ne peut pas être plus) et qu'il est alors générateur.

    Pour la suite nous allons d'abord prouver que \( G=ST\) puis que \( G\simeq S\times T\). Nous savons déjà que \( | S\cap T |=1\), ce qui nous amène à dire que \( | ST |=| S | |T |\). En effet si \( s,s'\in S\) et \( t,t'\in t\) et si \( st=s't'\), alors \( t=s^{-1}s't'\), ce qui voudrait dire que \( s^{-1}s'\in T\) et donc que \( s^{-1}s'=e\). Au final nous avons
    \begin{equation}
        | ST |=| S | |T |=pq=| G |.
    \end{equation}
    Par conséquent \( G=ST\). En nous rappelant du fait que \( S\cap T=\{ e \}\) et que \( S\) et \( T\) sont normaux, le lemme~\ref{LemHUkMxp} nous dit que \( G\simeq S\times T\). Le groupe \( S\) étant cyclique d'ordre \( p\) nous avons \( S=\eZ/p\eZ\) et pour \( T\), nous avons la même chose : \( T=\eZ/q\eZ\). Nous concluons que
    \begin{equation}
        G\simeq \eZ/p\eZ\times \eZ/q\eZ.
    \end{equation}
\end{proof}



\begin{theorem}[Théorème de Burnside\cite{FabricegPSFinis}] \label{ThoImkljy}
    Le centre d'un \( p\)-groupe non trivial est non trivial.
\end{theorem}

\begin{proof}
    Soit \( G\) un $p$-groupe non trivial. Nous considérons l'action adjointe \( G\) sur lui-même. Les points fixes de cette action sont les éléments du centre :
    \begin{equation}
        \mZ_G=\{ z\in G\tq \sigma_x(z)=z\forall x\in G \}=\Fix_G(G).
    \end{equation}
    Nous utilisons l'équation aux classes \eqref{PropUyLPdp} pour dire que \( | G |=| \mZ_G |\mod p\). Mais \( | \mZ_G |\) n'est pas vide parce qu'il contient l'identité. Donc \( | \mZ_G |\) est au moins d'ordre \( p\).
\end{proof}

\begin{proposition} \label{PropssttFK}
    Si \( p\) est un nombre premier, tout groupe d'ordre \( p\) ou \( p^2\) est abélien.
\end{proposition}
Rappel : un groupe d'ordre \( p\) ou \( p^2\) est automatiquement un $p$-groupe.

\begin{proof}
    Si \( | G |=p\), alors le théorème de Cauchy~\ref{ThoCauchyGpFini} nous donne l'existence d'un élément d'ordre \( p\). Cet élément est alors automatiquement générateur, \( G\) est cyclique et donc abélien.

    Si par contre \( G\) est d'ordre \( p^2\), alors les choses se compliquent (un peu). D'après le théorème de Burnside~\ref{ThoImkljy}, le centre \( \mZ\) n'est pas trivial; il est alors d'ordre \( p\) ou \( p^2\). Supposons qu'il soit d'ordre \( p\) et prenons \( x\in G\setminus\mZ\). Alors le stabilisateur de \( x\) pour l'action adjointe contient au moins \( \mZ\) et \( x\), c'est-à-dire que \( |\Fix_G(x)|\geq p+1\). Étant donné que \( \Fix_G(x)\) est un sous-groupe, son ordre est automatiquement \( 1\), \( p\) ou \( p^2\). En l'occurrence, il doit être \( p^2\) (parce que plus grand que \( p\)), et donc \( x\) doit être central, ce qui est une contradiction.
\end{proof}

%+++++++++++++++++++++++++++++++++++++++++++++++++++++++++++++++++++++++++++++++++++++++++++++++++++++++++++++++++++++++++++
\section{Groupe symétrique, groupe alterné}
%+++++++++++++++++++++++++++++++++++++++++++++++++++++++++++++++++++++++++++++++++++++++++++++++++++++++++++++++++++++++++++
\label{SECooZFYQooFfopMa}

La définition des permutations et du groupe symétrique sont~\ref{DEFooJNPIooMuzIXd}. Voir aussi le thème~\ref{THEMEooQEEWooXDhvhv}.

%---------------------------------------------------------------------------------------------------------------------------
\subsection{Le groupe alterné}
%---------------------------------------------------------------------------------------------------------------------------

\begin{definition}      \label{DEFooEIVIooFvVkHH}
    Le groupe \( A_n\)\nomenclature[R]{\( A_n\)}{groupe alterné} des permutations paires\footnote{Définition \ref{PROPooKRHEooAxtmRv}.} dans \( S_n\) est la \defe{groupe alterné}{alterné!groupe}\index{groupe!alterné}.
\end{definition}

\begin{proposition} \label{PROPooCPXOooVxPAij}
    À propos du groupe alterné dans le groupe symétrique.
    \begin{enumerate}
        \item
            Le groupe alterné \( A_n\) est un sous-groupe caractéristique\footnote{Définition~\ref{DEFooUXXTooCCLmQe}.} de \( S_n\)
        \item   \label{ITEMooWXXUooOWvFgE}
            Le sous-groupe \( A_n\) est d'indice \( 2\) dans \( S_n\).
        \item       \label{ITEMooGGAHooRYgNqq}
            Le sous-groupe \( A_n\) est l'unique sous-groupe d'indice\footnote{Définition \ref{DEFooMPIAooIeZNaR}.} \( 2\) de \( S_n\).
    \end{enumerate}
\end{proposition}

\begin{proof}
    Soit \( \alpha\in \Aut(S_n)\). Étant donné que \( \epsilon\circ\alpha\) est un homomorphisme surjectif sur \( \{ -1,1 \}\), par unicité de cet homomorphisme, nous avons \( \epsilon\circ\alpha=\epsilon\), et donc \( \alpha(A_n)=A_n\). Par le premier théorème d'isomorphisme~\ref{ThoPremierthoisomo}, il existe un isomorphisme
    \begin{equation}
        f\colon S_n/\ker(\epsilon)\to \Image(\epsilon).
    \end{equation}
    En égalant le nombre d'éléments nous avons \( | S_n:\ker\epsilon |=| S_n:A_n |=2\).

    Nous prouvons maintenant l'unicité. Soit \( H\) un sous-groupe d'indice \( 2\) dans \( S_n\). Par le lemme \ref{LemSkIOOG}, \( H\) est distingué et nous pouvons considérer le groupe \( S_n/H\). Ce dernier ayant \( 2\) éléments, il est isomorphe à \( \{ -1,1 \}\). Soit \( \theta\) l'isomorphisme. On note \( \varphi\) le morphisme canonique \( \varphi\colon S_n\to S_n/H\) :
    \begin{equation}    \label{EqSZBPTH}
        \xymatrix{%
        S_n \ar[r]^{\varphi}        &   S_n/H\ar[r]^{\theta}&\{ -1,1 \}.
           }
    \end{equation}
    La composition \( \varphi\circ \theta\) est alors un homomorphisme surjectif de \( S_n\) sur \( \{ -1,1 \}\) et nous avons \( \varphi\circ\theta=\epsilon\) par la proposition~\ref{ProphIuJrC}. L'enchainement \eqref{EqSZBPTH} nous montre que \( H=\ker(\theta\circ\varphi)=\ker(\epsilon)=A_n\).
\end{proof}

\begin{proposition}[\cite{ooFCRKooQdAaqw}]      \label{PROPooPSZVooSmAgPA}
    Le groupe symétrique \( S_n\) peut être écrit comme un produit semi-direct\footnote{Définition~\ref{DEFooKWEHooISNQzi}.} du groupe alterné :
    \begin{equation}
        S_n=A_n\times_{\varphi}\eZ/2\eZ
    \end{equation}
    où l'action de \( \eZ/2\eZ\) sur \( A_n\) est la conjugaison par \( \sigma=(12)\), c'est-à-dire \( \rho(-1)\tau=\sigma\tau\sigma^{-1}\).
\end{proposition}

\begin{proof}
    Nous avons la suite exacte
    \begin{equation}
        1\stackrel{i}{\longrightarrow}A_n\stackrel{i}{\longrightarrow}S_n\stackrel{\epsilon}{\longrightarrow}\{ \pm 1 \}\longrightarrow 1
    \end{equation}
    où les \( i\) représentent des inclusions et \( \epsilon\) est la signature définie en~\ref{DEFooWPYSooPWuwWO}. Grâce à cette suite et au fait que la signature soit un isomorphisme à partir de la partie \( \{ \id,\sigma \}\) (pour \( \sigma\) d'ordre \( 2\), par exemple \( \sigma=(12)\)), le théorème~\ref{THOooZNYTooPhnIdE} nous dit que
    \begin{equation}
        S_n\simeq A_n\times_{\varphi}\{ \id,\sigma \}
    \end{equation}
    où \( \varphi\) est l'action adjointe de \( \{ \id,\sigma \}\) sur \( A_n\).
\end{proof}

\begin{proposition}     \label{PROPooZOWBooIMxxlj}
    Si \( \beta\in S_n\) est une transposition, nous avons les égalités suivante d'ensembles :
    \begin{equation}
        S_n=A_n\cup A_n\beta=A_n\cup \beta A_n.
    \end{equation}
\end{proposition}

\begin{proof}
    Les parties \( A_n\) et \( \beta A_n\) ont le même nombre d'éléments. En effet, l'application
    \begin{equation}
        \begin{aligned}
            \varphi\colon A_n&\to A_n\beta \\
            \sigma&\mapsto \sigma\beta 
        \end{aligned}
    \end{equation}
    est une bijection.

    De plus ces deux ensembles sont disjoints à cause de la proposition \ref{ProphIuJrC}. En effet si \( \sigma\in A_n\), alors \( \epsilon(\sigma)=1\). Mais un élément de \( A_n\beta\) est de la forme \( \sigma\beta\) avec \( \sigma\in A_n\). Or \( \epsilon\) est une homomorphisme, donc \( \epsilon(\sigma\beta)=\epsilon(\sigma)\epsilon(\beta)=-1\).

    Enfin, la proposition \ref{PROPooCPXOooVxPAij}\ref{ITEMooWXXUooOWvFgE} dit que \( A_n\) est d'indice deux dans \( S_n\). Donc la partie
    \begin{equation}
        A_n\cup A_n\beta
    \end{equation}
    contient \( | S_n |/2+| S_n |/2=| S_n |\) éléments. C'est donc \( S_n\).
\end{proof}

\begin{lemma}   \label{LemiApyfp}   \index{groupe!dérivé!du groupe symétrique}
    Le groupe dérivé du groupe symétrique est le groupe alterné : \( D(S_n)=A_n\).
\end{lemma}

\begin{proof}
    Tout élément de \( D(S_n)\) s'écrit sous la forme \( ghg^{-1}h^{-1}\). Quel que soit le nombre de transpositions dans \( g\) et \( h\), le nombre de transpositions dans \( [g,h]\) est pair.
\end{proof}

\begin{proposition}[\cite{LoFdlw}]     \label{PropsHlmvv}
    Soit \( n\geq 3\). Les \( 3\)-cycles \( c_i=(1,2,i)\) avec \( i=3,\ldots, n\) engendrent le groupe alterné \( A_n\).
\end{proposition}

\begin{proof}
    Soit \( H\), le groupe engendré par les \( c_i\). D'abord nous avons
    \begin{equation}
        c_i=(1,2,i)=(1,2)(2,i),
    \end{equation}
    de telle sorte que \( \epsilon(c_i)=1\). Par conséquent nous avons \( H\subset A_n\). Nos montrons par récurrence que \( A_n\subset H\).

    Pour \( n=3\) il suffit de vérifier que \( A_3=\{ \id,c_3,c_3^2 \}\). Supposons avoir obtenu le résultat pour \(A_{n-1}\), et prouvons le pour \( A_n\). Soit \( s\in A_n\).

    Si \( s(n)=n\), alors \( s\) se décompose de la même manière que sa restriction \( s'\) à \( \{ 1,\ldots, n-1 \}\). Par l'hypothèse de récurrence, cette restriction, appartenant à \( A_{n-1}\),  se décompose en produit des \( c_3,\ldots, c_{n-1}\) et de leurs inverses.

    Si \( s(n)=k\) alors nous considérons l'élément \( c^2_nc_ks\). Cet élément envoie \( n\) sur \( n\) et peut donc être décomposé avec les \( c_i\) (\( i=1,\ldots, n-1\)) en vertu du point précédent.
\end{proof}

\begin{proposition} \label{PropiodtBG}
    Lorsque \( n\geq 5\), tous les \( 3\)-cycles de \( A_n\) sont conjugués. Autrement dit, la classe de conjugaison d'un \( 3\)-cycle est l'ensemble des \( 3\)-cycles.
\end{proposition}

\begin{proof}
    Soient les \( 3\)-cycles \( \sigma=(i_1,i_2,i_3)\) et \( \varphi=(j_1,j_2,j_3)\). Nous considérons une bijection \( \alpha\) de \( \{ 1,\ldots, n \}\) telle que \( \alpha(i_s)=j_s\). Nous avons immédiatement que \( \alpha\in S_n\) et que \( \alpha\sigma\alpha^{-1}=\varphi\). Donc les \( 3\)-cycles sont conjugués dans \( S_n\). Il reste à prouver qu'ils le sont dans \( A_n\).

    Si \( \alpha\) est une permutation paire, la preuve est terminée. Si \( \alpha\) est impaire, alors nous devons un peu la modifier. Vu que \( n\geq 5\), nous pouvons prendre \( s\) et \( t\), des éléments distincts dans \( \{ 1,\ldots, n \}\setminus\{ j_1,j_2,j_3 \}\) et poser \( \tau=(st)\). Vu que la signature est un homomorphisme et que \( \tau\) et \( \alpha\) sont impairs, l'élément \( \tau\alpha\) est pair (lemme et proposition~\ref{LemhxnkMf} et~\ref{PropPWIJbu}) et est donc dans \( A_n\). Les supports de \( \tau\) et \( \varphi\) étant disjoints, ces derniers commutent et nous avons
    \begin{equation}
        (\tau\alpha)\sigma(\tau\alpha)^{-1}=\tau(\alpha\sigma\alpha^{-1})\tau^{-1}=\tau\varphi\tau^{-1} = \varphi.
    \end{equation}
    Donc \( \sigma\) et \( \varphi\) sont conjugués par \( \tau\alpha\) qui est dans \( A_n\).
\end{proof}

\begin{theorem}[\cite{PDFpersoWanadoo}] \label{ThoURfSUXP}
    Le groupe alterné \( A_n\) est simple\footnote{Pas de sous-groupes normaux non triviaux, définition~\ref{DefGroupeSimple}.} pour \( n\geq 5\).
\end{theorem}
\index{sous-groupe!distingué!dans le groupe alterné}
\index{groupe!fini!alterné}
\index{groupe!partie génératrice}


\begin{proof}
    Soit \( N\), un sous-groupe normal de \( A_n\) non réduit à l'identité. Étant donné que les \( 3\)-cycles engendrent \( A_n\) (proposition~\ref{PropsHlmvv}) et que tous les \( 3\)-cycles sont conjugués dans \( A_n\) (proposition~\ref{PropiodtBG}), il suffit de montrer que \( N\) contient un \( 3\)-cycle. En effet si \( N\) contient un \( 3\)-cycle, le fait qu'il soit normal implique (par conjugaison) qu'il les contienne tous et donc qu'il contient une partie génératrice de \( A_n\).

    Soit donc \( \sigma\in N\) différent de l'identité. Nous prenons \( i\) dans le support de \( \sigma\) et \( j=\sigma(i)\). Nous choisissons ensuite \( k\in\{ 1,\ldots, n \}\setminus\{ i,j,\sigma^{-1}(i) \}\) et \( m=\sigma(k)\). Nous considérons la permutation \( \alpha=(ijk)\). Étant donné que \( N\) est normal l'élément
    \begin{equation}
        \theta=(\alpha^{-1}\sigma\alpha)\sigma^{-1}
    \end{equation}
    est dans \( N\). De plus en utilisant le lemme~\ref{LemmvZFWP} et le fait que \( \alpha^{-1}=(ikj)\) nous avons
    \begin{equation}
        \theta=(ikj)(j\sigma(j)m).
    \end{equation}
    Cela n'est pas spécialement un \( 3\)-cycle, mais nous allons en construire un. Nous allons déterminer que \( \theta\) est soit un \( 5\)-cycle, soit un \( 3\)-cycle , soit un \( 2\times 2\)-cycle suivant les valeurs de \( \sigma(j)\) et \( m\).

    Souvenons-nous que nous avons :
    \begin{itemize}
        \item
            \( i \neq j = \sigma(i) \), puisque $i$ est dans le support de \( \sigma \);
        \item
            \( k \neq i \) et \( k \neq j \), par définition de $k$ (rappelons aussi que \( k \neq \sigma^{-1}(i) \));
        \item
            \( m \neq i \), \( m \neq j \) et \( m \neq  \sigma(j) \) puisque \( m = \sigma(k) \).
    \end{itemize}
    Il ne nous reste alors seulement les deux possibilités suivantes :
    \begin{enumerate}
        \item
            soit \( m=k\), soit \( m \neq k \), d'une part;
        \item
            soit \( \sigma(j) = i \), soit \( \sigma(j) = k \), soit \( \sigma(j) \) n'est ni $i$, ni $k$, ni $m$, d'autre part.
    \end{enumerate}

    Supposons dans un premier temps que \( m=k\); alors
    \begin{equation}
        \theta=(ik)(j\sigma(j)).
    \end{equation}
    C'est à priori un \( 2\times 2\)-cycle. Mais si de plus \( \sigma(j) = i \), alors
    \begin{equation}
        \theta=(ijk)
    \end{equation}
    qui est un \( 3\)-cycle; et si \( \sigma(j) = k \), alors
    \begin{equation}
        \theta=(ikj)
    \end{equation}
    qui est un autre \( 3\)-cycle.

    Supposons à présent que \( m \neq k \). Si \( \sigma(j) \) n'est ni $i$, ni $k$, ni $m$, alors \( i\), \( j\), \( k\), \( \sigma(j)\) et \( m\) sont cinq nombres différents, et
    \begin{equation}
        \theta=(i,j,\sigma(j),m,k)
    \end{equation}
    est un \( 5\)-cycle. Si \( \sigma(j) = i\), alors
    \begin{equation}
        \theta=(ikj)(jim) = (imk)
    \end{equation}
    qui est un \( 3\)-cycle. Si \( \sigma(j)=k\), alors
    \begin{equation}
        \theta=(ikj)(jkm) = (ikm)
    \end{equation}
    qui est encore un \( 3\)-cycle.

    Bref nous avons montré que \( \theta\) est soit un \( 3\)-cycle, soit un \( 5\)-cycle, soit un \( 2\times 2\)-cycle. Si \( \theta\) est un \( 3\)-cycle, la preuve est terminée.

    Si \( \theta=(ab)(cd)\), alors on considère \( e\in \{ 1,\ldots, n \}\setminus\{ a,b,c,d \}\) et nous avons
    \begin{equation}
        \underbrace{(abe)^{-1}\theta(abe)}_{\in N}\theta^{-1}=(aeb)(ab)(cd)(abe)(an)(cd)=(abe)\in N.
    \end{equation}

    Si \( \theta\) est le \( 5\)-cycle \( (abcde)\), alors l'élément suivant est dans \( N\) :
    \begin{equation}
        (abc)^{-1}\theta(abc)\theta^{-1}=(acb)(abcde)(abc)(aedcb)=(acd).
    \end{equation}

    Dans tous les cas nous avons trouvé un \( 3\)-cycle dans \( N\) et nous avons par conséquent \( N=A_n\), ce qui fait que \( A_n\) ne contient pas de sous-groupes normaux non triviaux. Le groupe alterné \( A_n\) est donc simple.
\end{proof}

Nous en déduisons immédiatement que si \( n\geq 5\), le groupe dérivé de \( A_n\) est \( A_n\) parce que \( A_n\) ne contient pas d'autres sous-groupes non triviaux.\index{groupe!dérivé!du groupe alterné}

\begin{lemma}       \label{LEMooICEHooGSSpkq}
    Le groupe alterné\footnote{Définition~\ref{DEFooEIVIooFvVkHH}.} \( A_6\) n'accepte pas de sous-groupes normaux d'ordre \( 60\).
\end{lemma}

\begin{proof}
    Soit \( G\) normal dans \( A_6\), et \( a\), un élément d'ordre \( 5\) dans \( G\) (qui existe parce que \( 5\) divise \( 60\)). Soit aussi un élément \( b\) d'ordre \( 5\) dans \( A_6\). Les groupes \( \gr(a)\) et \( \gr(b)\) sont deux \( 5\)-Sylow dans \( A_6\). En effet, \( 5\) un nombre premier et est la plus grande puissance de \( 5\) dans la décomposition de \( 60\); donc \( \gr(a)\) est un \( 5\)-Sylow dans \( G\). D'autre part, l'ordre de \( A_6\) (qui est \( \frac{ 1 }{2}6!\)) ne possède également que \( 5\) à la puissance \( 1\) dans sa décomposition.

    En vertu du théorème de Sylow~\ref{ThoUkPDXf}\ref{ItemMzNRVf}, les \( 5\)-Sylow \( \gr(a)\) et \( \gr(b)\) sont conjugués et il existe \( \tau\in A_6\) tel que \( b=\tau a\tau^{-1}\). Mais \( G\) étant normal dans \( A_6\), l'élément \( \tau a\tau^{-1}\) est encore dans \( G\), de telle sorte que \( b\in G\). Du coup \( G\) doit contenir tous les éléments d'ordre \( 5\) de \( A_6\).

    Les éléments d'ordre $5$ de \( A_6\) doivent fixer un des points de \( \{ 1,2,3,4,5,6 \}\) puis permuter les autres de façon à n'avoir qu'un seul cycle. Un cycle correspond à écrire les nombres \( 1,2,3,4,5\) dans un certain ordre. Ce faisant, le premier n'a pas d'importance parce qu'on considère la permutation cyclique, par exemple \( (3,5,2,1,4)\) est la même chose que \( (5,2,1,4,3)\). Le nombre de cycles sur \( \{ 1,2,3,4,5 \}\) est donc de \( 4!\), et par conséquent le nombre d'éléments d'ordre \( 5\) dans \( A_6\) est \( 6\cdot 4!=144\).

    Le groupe \( G\) doit contenir au moins \( 144\) éléments alors que par hypothèse il en contient \( 60\); contradiction.
\end{proof}

Le théorème suivant montre que tout groupe peut être vu, en agissant sur lui-même, comme une partie du groupe symétrique.
\begin{theorem}
    Un groupe \( G\) est isomorphe à un sous-groupe de son groupe symétrique \( S(G)\).
\end{theorem}

\begin{proof}
    Nous considérons \( \varphi\), la translation à gauche :
    \begin{equation}
        \begin{aligned}
            \varphi\colon G&\to S(G) \\
            g&\mapsto t_g
        \end{aligned}
    \end{equation}
    où \( f_g(h)=gh\). Étant donné que
    \begin{equation}
        \varphi(gh)= ghx=g(t_hx)=t_g\circ t_h(x),
    \end{equation}
    l'application \( \varphi\) est un morphisme de groupes. Il est injectif parce que si \( gx=hx\) pour tout \( x\), en particulier pour \( x=e\) nous trouvons \( g=h\).

    De la même manière, \( \varphi(g)x=\varphi(g)y\) implique \( x=y\). Cela montre que l'image est bien dans le groupe symétrique.

    L'ensemble \( \Image(\varphi)\) est donc un sous-groupe de \( S(G)\), et \( \varphi\) est un isomorphisme vers ce groupe.
\end{proof}

\begin{lemma}       \label{LEMooMVUGooRiDaDz}
    Si \( n\geq 3\), alors
    \begin{enumerate}
        \item
            Le centre de \( S_n\) est trivial.
        \item
            Le groupe \( S_n\) est non abélien.
    \end{enumerate}
\end{lemma}

\begin{proof}
    Soit \( s\in Z(S_n)\) et trois éléments distincts \( a\),  \( b\) et \( c\) de \( \{ 1,\ldots, n \}\). Nous posons \( \tau=(ab)\) et nous avons \( s\tau=\tau s\). En notant \( a'=s(a)\) et \( b'=s(b)\) nous avons
    \begin{subequations}
        \begin{align}
            a'=s(a)=(\tau s\tau^{-1})(a)=(\tau s)(b)=\tau(b')\\
            b'=s(b)=(\tau s\tau^{-1})(b)=(\tau s)(a)=\tau(a').
        \end{align}
    \end{subequations}
    Donc \( \tau\) permute \( a'\) et \( b'\). Mais comme \( \tau\) ne permute que \( a\) et \( b\), en tant qu'ensembles, \( \{ a,b \}=\{ s(a), s(b) \}\). Le même raisonnement sur \( \{ b,c \}\) donne $\{ b,c \}=\{ s(b),s(c) \}$. Et vu que \( a\), \( b\) et \( c\) sont distincts,
    \begin{equation}
        \{ b \}=\{ b,c \}\cap\{ a,b \}=\{ s(b) \}.
    \end{equation}
    Cela montre que \( s(b)=b\), et donc que le centre de \( S_n\) est réduit à la permutation identité.

    En ce qui concerne le fait que \( S_n\) est non abélien, si nous avions \( st=ts\) pour tout \( s,t\in S_n\) alors \( s=tst^{-1}\) pour tout \( t\). Alors \( s\) serait dans le centre de \( S_n\). En bref, si \( S_n\) était abélien, son centre serait \( S_n\) et non \( \{ \id \}\).

\end{proof}

\begin{proposition}[\cite{Exo7Sylow,ooGQNTooEiWtsy}]        \label{PROPooUBIWooTrfCat}
    Tout groupe simple\footnote{Définition~\ref{DefGroupeSimple}.} d'ordre \( 60\) est isomorphe au groupe alterné \( A_5\).
\end{proposition}

\begin{proof}
    Nous avons la décomposition en nombres premiers \( 60=2^2\cdot 3\cdot 5\). Déterminons pour commencer le nombre \( n_5\) de \( 5\)-Sylow dans \( G\). Le théorème de Sylow~\ref{ThoUkPDXf}\ref{ItemkYbdzZ} nous renseigne que \( n_5\) doit diviser \( 60\) et doit être égal à \( 1\mod 5\). Les deux seules possibilités sont \( n_5=1\) et \( n_5=6\). Étant donné que tous les \( p\)-Sylow sont conjugués, si \( n_5=1\) alors le \( 5\)-Sylow serait un sous-groupe invariant à l'intérieur de $G$, ce qui est impossible vu que \( G\) est simple. Donc \( n_5=6\).

    Par le point~\ref{ItemMzNRVf} du théorème de Sylow, le groupe \( G\) agit transitivement sur l'ensemble des \( 5\)-Sylow par l'action adjointe :
    \begin{equation}
        g\cdot S=gSg^{-1}.
    \end{equation}
    Cela donne donc un morphisme \( \theta\colon G\to S_6\). Le noyau de \( \theta\) est un sous-groupe normal. En effet si \( k\in \ker\theta\) et si \( g\in G\) nous avons
    \begin{subequations}
        \begin{align}
            (gkg^{-1})\cdot S&=gkg^{-1} Ggk^{-1}g^{-1}\\
            &=gkTk^{-1}g^{-1}\\
            &=gTg^{-1}\\
            &=S
        \end{align}
    \end{subequations}
    où \( T\) est le Sylow \( T=g^{-1}Sg\). Étant donné que \( k\in \ker\theta\) nous avons utilisé \( kTk^{-1}=aT\). Au final \( gkg^{-1}\cdot S=S\), ce qui prouve que \( gkg^{-1} \in\ker\theta\).

    Étant donné que \( \ker\theta\) est normal dans \( G\), soit est soit réduit à \( \{ e \}\) soit il vaut \( G\). La seconde possibilité est exclue parce qu'elle reviendrait à dire que \( G\) agit trivialement, ce qui n'est pas correct étant donné qu'il agit transitivement. Nous en déduisons que \( \ker\theta=\{ e \}\), que \( \theta\) est injective et que \( G\) est isomorphe à un sous-groupe de \( S_6\).

    Par ailleurs le groupe dérivé de \( G\) est un sous-groupe normal (et non réduit à l'identité parce que \( G\) est non commutatif). Donc \( D(G)=G\). Étant donné que \( G\subset S_6\), nous avons
    \begin{equation}
        G=D(G)\subset D(S_6)=A_6
    \end{equation}
    parce que le groupe dérivé du groupe symétrique est le groupe alterné (lemme~\ref{LemiApyfp}).

    L'ensemble \( \theta^{-1}(A_6)\) est distingué dans \( G\). En effet si \( \sigma\in A_6\) et si \( g\in G\) nous avons
    \begin{equation}
        \theta\big( g\theta^{-1}(\sigma)g^{-1} \big)=\theta(g)\sigma \theta(g)^{-1}\in A_6.
    \end{equation}
    Nous en déduisons que \( \theta^{-1}(A_6)\) est soit \( G\) entier soit réduit à \( \{ e \}\). Si \( \theta^{-1}(A_6)=\{ e \}\), alors pour tout \( g\in G\) nous aurions \( g^2=e\) parce que \( \theta(g^2)\in A_6\). L'ordre de \( G\) étant \( 60\), il n'est pas possible que tous ses éléments soient d'ordre \( 2\). Nous en déduisons que \( \theta(G)\subset A_6\).

    Nous nommons \( H=\theta(G)\) et nous considérons l'ensemble \( X=A_6/H\) où les classes sont prises à gauche, c'est-à-dire
    \begin{equation}
        [\sigma]=\{ h\sigma\tq h\in H \}.
    \end{equation}
    Évidemment \( A_6\) agit sur \( X\) de façon naturelle. Au niveau de la cardinalité,
    \begin{equation}
        \Card(X)=\frac{ | A_6 | }{ | G | }=\frac{ 360 }{ 60 }=6.
    \end{equation}
    Le groupe \( A_6\) agit sur \( X\) qui a \( 6\) éléments. Nous avons donc une application \( \varphi\colon A_6\to A_6\). Encore une fois, la simplicité de \( A_6\) montre que \( \varphi(A_6)=A_6\).

    Nous étudions maintenant \( \varphi(H)\) agissant sur \( X\). Un élément \( x\in A_6\) fixe la classe de l'unité \( [e]\) si et seulement si \( x\in H\) et par conséquent \( \varphi(H)\) est la fixateur de \( [e]\) dans \( X\). À la renumérotation près, nous pouvons identifier \( \varphi(H)\) au sous-groupe de \( A_6\) agissant sur \( \{ 1,\ldots, 6 \}\) et fixant \( 6\). Nous avons alors \( \varphi(H)=S_5\cap A_6=A_5\). Nous venons de prouver que \( \varphi\) fournit un isomorphisme entre \( A_5\) et \( H\). Étant donné que \( H\) était isomorphe à \( G\), nous concluons que \( G\) est isomorphe à \( A_6\).
\end{proof}

%---------------------------------------------------------------------------------------------------------------------------
\subsection{Sous-groupes normaux}
%---------------------------------------------------------------------------------------------------------------------------

\begin{normaltext}[\cite{ooDZHIooQIYqwZ}]     \label{NORMooQAZTooBQLqDn}
    Soit le groupe \( V_4\) engendré par les bitranspositions de \( S_4\). Nous savons de l'exemple~\ref{ExVYZPzub}\ref{ITEMooGCMYooKZgFHX} que ce groupe contient exactement \( 3\) éléments non triviaux et l'identité. De plus, comme c'est une classe de conjugaison, \( V_4\) est normal dans \( S_4\).
\end{normaltext}

\begin{lemma}
    Les sous-groupes \( \Fix_{S_n}(a)\) (avec \( a\in\{ 1,\ldots, n \}\)) sont conjugués entre eux.
\end{lemma}

\begin{proof}
    Soit \( \sigma\in \Fix(a)\) et \( s\in S_n\) nous devons prouver que \( s \sigma s^{-1}\) est le fixateur d'un élément de \( \{ 1,\ldots, n \}\). Nous notons \( s(a)=b\). Alors
    \begin{equation}
        (s\sigma s^{-1})(b)=(s\sigma)(a)=s(a)=b.
    \end{equation}
    Donc \( s\Fix(a)s^{-1}\subset \Fix(b)\).

    Dans l'autre sens, si \( \sigma\in \Fix(b)\) alors \( s^{-1} \sigma s\in\Fix(a)\). Mais \( \sigma=s(s^{-1}\sigma s)s^{-1}\), donc \( \sigma\in s\Fix(a)s^{-1}\).
\end{proof}

\begin{proposition}[Sous-groupes normaux de \( S_n\) \cite{ooDZHIooQIYqwZ}]     \label{PROPooOTJAooUbzGZm}
    Les sous-groupes normaux de \( S_n\) ne sont pas légions.
    \begin{enumerate}
        \item
            Pour \( n=4\), le sous-groupes normaux de \( S_4\) sont \(  \{ \id \}  \), \( V_4\), \( A_4\) et \( S_4\).
        \item
            Pour \( n\neq 4\), les sous-groupes normaux de \( S_n\) sont \( \{ \id \}\), \( A_n\) et \( S_n\).
    \end{enumerate}
\end{proposition}

\begin{proof}
    Les cas \( n\leq 2\) sont un peu triviaux, donc nous faisons \( n\geq 3\). Soit \( H\) normal dans \( S_n\) et \( s\neq \id\) dans \( H\); par le lemme~\ref{LEMooMVUGooRiDaDz}, \( s\) n'est pas dans le centre de \( S_n\) et il existe \( u\in S_n\) tel que \( us\neq su\). Vu que \( u\) est un produit de transpositions (proposition~\ref{PropPWIJbu}), il existe une transposition \( t\) telle que \( st\neq ts\). Le sous-groupe \( H\) est normal et que \( s\in H\) nous avons aussi \( ts^{-1}t^{-1}\in H\). Mais en même temps, la combinaison \( sts^{-1}\) est le conjugué d'une transposition et est donc également une transposition (classe de conjugaison de \( S_4\) dans~\ref{ExVYZPzub}). Nous en concluons que \( sts^{-1}t^{-1}\) est un produit de deux transpositions appartenant \( H\).

    Nous venons de prouver que \( H\) contient au moins un produit de deux transpositions. Et ce produit est différent de \( \id\) parce que \( sts^{-1}t^{-1}=\id\) impliquerait \( st=ts\).

    Soient donc deux transpositions \( t_1,t_2\in H\) telles que \( t_1t_2\neq \id\). Les supports de \( t_1\) et \( t_2\) ont soit \( 1\) soit aucun élément communs.

    \begin{subproof}
        \item[Premier cas]

            Supposons \( t_1=(a,b)\), \( t_2=(b,c)\) avec \( a,b,c\) distincts dans \( \{ 1,\ldots, n \}\). Dans ce cas \( t_1t_2=(a,b,c)\) et \( H\) contient un cycle de longueur \( 3\). Vu que \( H\) est normal et que les cycles de longueur trois sont une classe de conjugaison (exemple~\ref{ExVYZPzub}) et que \( A_n\) est engendré par ceux-ci (proposition~\ref{PropsHlmvv}), \( A_n\subset H\). Mais \( A_n\) est d'indice deux dans \( S_n\) (proposition~\ref{ITEMooWXXUooOWvFgE}\ref{ITEMooWXXUooOWvFgE}). Quel nombre plus grand que \( n!/2\) divise \( n!\) ? Seulement \( n\) lui-même. Donc \( H\) est soit \( A_n\) soit \( S_n\).

        \item[Second cas]

            Le groupe \( H\) contient un élément de la forme \( (ab)(cd)\) avec \( a,b,c,d\) distincts dans \( \{ 1,\ldots, n \}\).

            \begin{subproof}

                \item[Si \( n=3\)]

                    Impossible parce que avec \( n=3\) nous n'avons pas quatre éléments distincts.

                \item[Si \( n=4\)]

                    Le sous-groupe \( H\) de \( S_4\) contient un élément de \( V_4\) qui n'est pas l'identité. Par normalité et classes de conjugaison, \( H\) contient \( V_4\). Nous devons maintenant prouver que si \( H\) n'est pas \( V_4\) alors \( H\) est \( A_4\) ou \( S_4\). Nous avons les inclusions \( V_4\subset H\subset S_4\) et donc les inégalités
                    \begin{equation}
                        4\leq | H |\leq 24.
                    \end{equation}
                    Donc le nombre \( | H |\) est un multiple de \( 4\) qui divise \( 24\). Les possibilités sont \( | H |=4,8,12,24\). La possibilité \( | H |=4\) donne \( H=V_4\); si \( |H |=24\) alors \( H=S_4\); si \( | H |=12\) alors \( H\) est d'indice \( 2\) dans \( S_4\) et \( H=A_n\) (proposition~\ref{PROPooCPXOooVxPAij}\ref{ITEMooGGAHooRYgNqq}). Quid de \( | H |=8\) ?

                    D'après le corolaire~\ref{CorpZItFX} au théorème de Lagrange, l'ordre d'un élément divise l'ordre du groupe. Soit \( x\) dans \( H\) mais pas dans \( V_4\). L'ordre de \( x\) peut être \( 1\), \( 2\), \( 4\) ou \( 8\). Ordre \( 1\) serait \( x=\id\). Ordre \( 8\), pas possible parce que \( S_4\) n'a pas d'éléments d'ordre \( 8\).
                    \begin{subproof}
                        \item[\( x\) d'ordre \( 2\)]

                            Prenons la décomposition de \( x\) en cycles disjoints. Vu qu'on est dans \( S_4\), ces cycles ne peuvent être que des transpositions. Soit il y en a un (alors \( H\) contient une transposition et donc \( H=S_4\)), soit il y en a deux et alors \( x\) est dans \( V_4\).

                        \item[\( x\) d'ordre \( 4\)]

                            L'élément \( x\) serait alors un cycle de longueur \( 4\). Et alors \( H\) contient tous les cycles de longueur \( 4\). Par exemple il contient le produit \( (abcd)(bacd)=(adc)\). Le sous-groupe \( H\) contient alors \( A_4\) (parce qu'il contient tous les \( 3\)-cycles).
                    \end{subproof}

                \item[Si \( n\geq 5\)]

                    Soit un élément \( e\) distinct de \( a,b,c\) et \( d\). Par notre liste préférée des classes de conjugaisons (exemple~\ref{ExVYZPzub}\ref{ITEMooGCMYooKZgFHX}), le \( 2\)-cycle \( (c,e)(a,b)\) est conjugué à \( (a,b)(c,d)\) et appartient donc à \( H\). Mais alors le produit suivant est également dans \( H\) :
                    \begin{equation}
                        (ce)(ab)(ab)(cd)=(ce)(cd)=(ecd).
                    \end{equation}
                    Donc \( H\) contient un \( 3\)-cycle, et par conséquent tous les \( 3\)-cycles. Encore une fois, cela prouve que \( H\) est soit \( A_n\) soit \( S_n\).

        \item[Pourquoi \( n=4\) est spécial ?]

            Dans le premier cas, nous montrons tout de suite que \( H=V_4\) n'est pas possible. Dans le deuxième cas, nous montrons que, grâce à un élément différent de \( a,b,c\) et \( d\), la possibilité \( H=V_4\) est exclue. La possibilité \( H=V_4\) n'existe que pour \( n=4\).

    \end{subproof}
    \end{subproof}

\end{proof}

%---------------------------------------------------------------------------------------------------------------------------
\subsection{Indice}
%---------------------------------------------------------------------------------------------------------------------------

\begin{theorem}
    Tout sous-groupe d'indice \( n\) dans \( S_n\) est isomorphe à \( S_n\).
\end{theorem}

\begin{proof}
    Pour \( n=1\), il n'y a rien. Pour \( n=2\), un sous-groupe d'indice \( 2\) ne peut contenir que \( 1\) élément qui est donc l'identité. Ok pour que \( \{ \id \}\) soit égal à \( S_1\) ?

    Pour les autres, il y a un peu plus de travail.

    \begin{subproof}
        \item[Pour \( n=3\)]

            Nous avons \( | S_3 |=6\). Donc un sous-groupe d'indice $3$ dans \( S_3\) contient exactement \( 2\) éléments. Il contient \( \id\) et un autre élément \( \sigma\in S_3\) qui doit vérifier \( \sigma^2=\id\) ou \( \sigma^2=\sigma\). Aucun élément de \( S_3\) ne vérifie \( \sigma^2=\sigma\) (à part l'identité). Donc \( \sigma^2=\id\), ce qui fait que \( \sigma\) est une transposition. Donc
            \begin{equation}
                H=\{ \id,(12) \}
            \end{equation}
            ou l'identité avec \( (23)\) ou avec \( (13)\). Dans tous les cas c'est isomorphe à \( S_2\).

        \item[Pour \( n=4\)]

            Nous avons \( | S_4:H |=4\), donc \( | H |=6\). Mais \( 6=2\times 3\) et \( 2\divides 3-1\), donc le théorème~\ref{ThoLnTMBy} nous dit que \( H\) est soit cyclique\footnote{Définition~\ref{DefHFJWooFxkzCF}.} (et donc abélien), soit un produit semi-direct. Vu que \( S_4\) n'a pas d'éléments d'ordre $6$, aucun sous-groupe d'ordre \( 6\) ne peut être cyclique. Nous sommes donc dans le cas du produit semi-direct
            \begin{equation}        \label{EQooSHWGooFMNRvf}
                H=\eZ_3\times_{\varphi}\eZ_2
            \end{equation}
            où \( \varphi\colon \eZ_2\to \Aut(\eZ_3)\) et \( \varphi(1)\) est d'ordre \( 2\) dans \( \Aut(\eZ_3)\). Il convient de nous attarder un peu pour être sûr d'avoir bien compris tout ce qui se trouve dans l'identification \eqref{EQooSHWGooFMNRvf}. D'abord un point de notations : ici nous considérons les groupes \( \eZ_p=\eZ/p\eZ\) munis de l'addition. Donc \( 1\) n'est pas le neutre. Ensuite nous savons du théorème~\ref{ThoozyeSn} que \( \Aut(\eZ/3\eZ)=(\eZ/3\eZ)^*\), et que via cette identification, \( \varphi(1)=2\in(\eZ/3\eZ)^*\) au sens où \( \varphi(1)x=2x\). Nous avons alors \( \varphi(1)^2x=4x=x\) dans \( \eZ/3\eZ\). Cela montre bien que \( \varphi(1)\) est d'ordre \( 2\).

            Par rapport à la proposition~\ref{PROPooPSZVooSmAgPA}, ici nous écrivons \( \eZ_2=\big( \{ 0,1 \},+ \big)\) alors que là nous écrivons \( \eZ_2=\big( \{ -1,1 \},\cdot \big)\). Ce sont les mêmes groupes, mais il convient de remarquer que le \( 1\) ici est le \( -1\) là.

            Nous savons par la proposition~\ref{PROPooPSZVooSmAgPA} que \( S_n=A_n\times_{\varphi}\eZ_2\); en comparant avec \eqref{EQooSHWGooFMNRvf} nous voyons qu'il suffit de prouver que \( A_3=\eZ/3\eZ\) pour avoir \( H=S_3\).

            Le groupe \( A_3\) possède \( | S_3 |/2=3\) éléments. Il est vite vu que \( A_3=\{ \id,(12)(31), (12)(32) \}\) : ce sont trois éléments de signature paire dans \( S_3\); donc c'est \( S_3\). La correspondance \( \id\mapsto 0\), \( (12)(13)\mapsto 1\), \( (13)(12)\mapsto 2\) donne un isomorphisme avec \( (\eZ_3,+)\).

        \item[Pour \( n\geq 5\)]

            Soit un sous-groupe \( H\) d'indice \( n\) dans \( S_n\) et l'action à gauche de \( S_n\) sur \( E=S_n/H\) (qui n'est à priori pas un groupe) donnée par \( g\cdot [s]=[gs]\).

            \begin{subproof}
                \item[Morphisme \( \varphi\colon S_n\to S_E\)]

                    Le \( \varphi\) définit par l'action est un morphisme parce que
                    \begin{equation}
                        \varphi(g_1g_2)[s]=[g_1g_2s]=\varphi(g_1)[g_2s]=\varphi(g_1)\varphi(g_2)[s].
                    \end{equation}
                    Mais il faut également vérifier que pour chaque \( g\in G\), l'application \( \varphi(g)\colon E\to E\) est bien une permutation. Pour l'injectivité, si \( \varphi(g)[s_1]=\varphi(g)[s_2]\) alors \( [gs_1]=[gs_2]\), donc il existe \( h\in H\) tel que \( gs_1=gs_2h\), ce qui prouve que \( s_1=s_2h\) et donc que \( [s_1]=[s_2]\). Pour la surjectivité, soit \( [t]\in S_n/H\) et résolvons \( \varphi(g)[s]=[t]\) par rapport à \( s\). L'élément \( s=g^{-1} t\) fonctionne.

                \item[\( \ker(\varphi)\) est normal]

                    Soit \( z\in\ker(\varphi)\), c'est-à-dire que \( \varphi(z)=\id_E\). Alors pour \( \sigma\in S_n\) nous avons \( \varphi(\sigma z\sigma^{-1})=\varphi(\sigma)\varphi(z)\varphi(\sigma^{-1})=\id_E\).

                \item[\( \ker(\varphi)=\bigcap_{g\in S_n}gHg^{-1}\)]

                    Supposons que \( z\in gHg^{-1}\) pour tout \( g\), et calculons \( \varphi(z)[s]\). D'abord par hypothèse il existe \( h\in H\) tel que \( z=shs^{-1}\), donc
                    \begin{equation}
                        \varphi(z)[s]=[zs]=[zhs^{-1}s]=[s],
                    \end{equation}
                    ce qui prouve que \( \varphi(z)=\id\).

                    Dans l'autre sens, soit \( z\in\ker(\varphi)\). Donc \( \varphi(z)[s]=[s]\). Il existe donc \( h\in H\) tel que \( zs=sh\), c'est-à-dire tel que \( z=shs^{-1}\). La formule demandée est donc prouvée.

                \item[Questions d'ordre]

                    Nous savons que \( | H |=(n-1)!\) alors que \( | A_n |=\frac{ n! }{2}\). Donc \( | H |<| A_n |\) avec une inégalité stricte. En même temps nous avons \( | \ker(\varphi) |\leq | H |\) parce que \( \ker(\varphi)\) est une intersection dont un des termes est \( H\) lui-même. Nous avons alors les inégalités
                    \begin{equation}
                        | \ker(\varphi) |\leq | H |=(n-1)!<| A_n |.
                    \end{equation}
                    Mais le seul sous-groupes normaux de \( S_n\) sont \( A_n\) et \( S_n\) et \( \{ \id \}\) (proposition~\ref{PROPooOTJAooUbzGZm}). Donc \( \ker(\varphi)=\id\) et \( \varphi\) est une injection entre deux ensembles finis de même cardinalité. Cela fait de \( \varphi\) une bijection et donc un isomorphisme de groupes
                    \begin{equation}
                        \varphi\colon S_n\to S_E.
                    \end{equation}
                    Soit une fonction de numérotation \( \psi\colon E\to \{ 1,\ldots, n \}\). Avec cela nous définissons un isomorphisme de groupes
                    \begin{equation}
                        \begin{aligned}
                            \tilde \psi\colon S_E&\to S_n \\
                            \sigma&\mapsto \psi\sigma\psi^{-1}.
                        \end{aligned}
                    \end{equation}

                \item[Fixateur]

                    Nous montrons à présent que \( (\tilde \psi\circ\varphi)(H)=\Fix\big( \psi[\id] \big)\) où le stabilisateur est pris dans \( S_n\). Pour la première inclusion, soit \( h\in H\). Nous avons \( (\tilde \psi\circ\varphi)(h)=\psi\circ\varphi(h)\psi^{-1}\), qui nous appliquons à \( \psi[\id]\) :
                    \begin{equation}
                        (\tilde \psi\circ\varphi)(h)\psi[\id]=\psi\circ\varphi(h)[\id]=\psi[h]=\psi[\id].
                    \end{equation}
                    Donc \( (\tilde \psi\circ\varphi)(H)\subset\Fix\big( \psi[\id] \big)\).

                    Pour l'autre inclusion, soit \( \sigma\in S_n\) tel que \( \sigma\psi[\id]=\psi[\id]\). Vu que \( \sigma\in S_n\) nous avons \( s\in S_E\) tel que \( \sigma=\tilde \psi(s)\). Pour ce \( s\) nous avons donc
                    \begin{equation}
                        \big( \tilde \psi(s)\circ\psi \big)[\id]=\psi[\id],
                    \end{equation}
                    d'où nous déduisons \( s[\id]=[\id]\). Cela prouve que \( s\) stabilise \( [\id]\) dans \( S_E\). Donc \( s=\varphi(h)\) pour un certain \( h\in H\), et au final \( \sigma=\tilde \psi\big( \varphi(h) \big)\).

                \item[Conclusion]

                    L'application \( \tilde \psi\circ\varphi\colon H\to S_n\) est une application dont l'image est le fixateur d'un point. Plus précisément,
                    \begin{equation}
                        \tilde \psi\circ\varphi\colon H\to \Fix\big( \psi[\id] \big)
                    \end{equation}
                    est un isomorphisme de groupe. Mais le stabilisateur d'un point dans \( S_n\) est \( S_{n-1}\).
            \end{subproof}
    \end{subproof}
\end{proof}


%+++++++++++++++++++++++++++++++++++++++++++++++++++++++++++++++++++++++++++++++++++++++++++++++++++++++++++++++++++++++++++
\section{Isométriques du cube}
%+++++++++++++++++++++++++++++++++++++++++++++++++++++++++++++++++++++++++++++++++++++++++++++++++++++++++++++++++++++++++++
\label{SecPVCmkxM}
\index{isométrie!espace euclidien!isométries du cube}
\index{groupe!et géométrie!isométries du cube}
Les isométries du cube proviennent de \cite{KXjFWKA}.

\begin{wrapfigure}{r}{6.0cm}
   \vspace{-0.5cm}        % à adapter.
   \centering
   \input{auto/pictures_tex/Fig_MCKyvdk.pstricks}
\end{wrapfigure}
Nous considérons le cube centré en l'origine de \( \eR^3\) et \( G\), le groupe des isométries de \( \eR^3\) préservant ce cube. Nous notons aussi \( G^+\) le sous-groupes de \( G\) constitué des éléments de déterminant positif. Nous notons
\begin{equation}
    \mD=\{ D_1,\ldots, D_4 \}
\end{equation}
l'ensemble des grandes diagonales, c'est-à-dire les segments \( [AG]\), \( [EC]\), \( [FD]\), et \( [BH]\). Nous savons que \( G\) préserve les longueurs et que ces segments sont les plus longs possibles à l'intérieur du cube. Donc \( G\) agit sur \( \mD\) parce qu'il ne peut transformer une grande diagonale qu'en une autre grande diagonale. Nous avons donc un morphisme de groupes
\begin{equation}
    \rho\colon G\to S_4.
\end{equation}
Nous montrons ce que morphisme est surjectif en montrant qu'il contient les transpositions. Le groupe \( G\) contient la symétrie axiale passant par le milieu \( M\) de \( [AE]\) et le milieu \( N\) de \( CG\). Il est facile de voir que cette symétrie permute \( [AG]\) avec \( [EC]\). De plus elle laisse \( [FD]\) inchangée. En effet, aussi incroyable que cela paraisse en regardant le dessin, nous avons \( FD\perp MN\), parce qu'en termes de vecteurs directeurs,
\begin{equation}
    \begin{aligned}[]
        \vect{ ON }&=\begin{pmatrix}
            1    \\
            -1    \\
            0
        \end{pmatrix}&\vect{ OF }&=\begin{pmatrix}
            1    \\
            1    \\
            -1
        \end{pmatrix}.
    \end{aligned}
\end{equation}

Étudions à présent le noyau \( \ker(\rho)\). Si \( f\in\ker(\rho)\) n'est pas l'identité, alors \( f(D_i)=D_i\) pour tout \( i\), mais au moins pour une des diagonales les sommets sont inversés. Quitte à renommer les sommets du cube nous supposons que la diagonale \( [AG]\) est retournée : \( f(A)=G\) et \( f(G)=A\). Regardons où peut partir \( B\) sous l'effet de \( f\). Étant donné que \( f\) préserve les diagonales, \( f(B)\in\{ B,C \}\), mais étant donné que \( f\) est une isométrie, \( d\big( f(B),f(G) \big)=d(B,G)\), et nous concluons que \( f(B)=H\). Donc la diagonale \( [BH]\) est retournée sous l'effet de \( f\). En raisonnant de même, nous voyons que \( f\) retourne toutes les diagonales. Donc les éléments non triviaux de \( \ker(\rho)\) retournent toutes les diagonales; il n'y en a donc qu'un seul et c'est la symétrie centrale :
\begin{equation}
    \ker(\rho)=\{ \id,s_0 \}.
\end{equation}
Le premier théorème d'isomorphisme~\ref{ThoPremierthoisomo} nous permet d'écrire le quotient de groupes :
\begin{equation}
    \frac{ G }{ \{ \id,s_0 \} }\simeq S_4.
\end{equation}
Une classe d'équivalence modulo \( \ker(\rho)\) dans \( G\) est donc toujours de la forme \( \{ f,f\circ s_0 \}\). Et vu que \( s_0\) est de déterminant \( -1\), une classe contient toujours exactement un élément de déterminant \( 1\) et un de déterminant \( -1\).

D'autre part \( \ker(\rho)\) est normal dans \( G\) parce que en tant que matrice, \( s_0=-\mtu\), donc les problèmes de commutativité ne se posent pas. L'application
\begin{equation}
    \begin{aligned}
        \varphi\colon \frac{ G }{ \{ \id,s_0 \} }&\to G^+ \\
        [g]&\mapsto \begin{cases}
            g    &   \text{si } \det(g)>0\\
            g\circ s_0    &    \text{sinon}
        \end{cases}
    \end{aligned}
\end{equation}
est un isomorphisme de groupes. Et enfin nous pouvons écrire
\begin{equation}
    G^+\simeq S_4.
\end{equation}

Nous allons maintenant utiliser le corolaire~\ref{CoroGohOZ} pour montrer que \( G=G^+\times_{\sigma}\ker(\rho)\). Les conditions sont remplies :
\begin{itemize}
    \item \( \ker(\rho)\) normalise \( G^+\) parce que \( \ker(\rho)\) ne contient que \( \pm\mtu\).
    \item \( \ker(\rho)\cap G^+=\{ \id \}\).
    \item \( \ker(\rho)G^+=G\) parce que les classes d'équivalence de \( G\) modulo \( \ker(\rho)\) sont composées de \( \{ f,f\circ s_0 \}\).
\end{itemize}
Vu que \( G^+\simeq S_4\) et \( \ker(\rho)\simeq \eZ/2\eZ\) nous pouvons écrire de façon plus brillante que
\begin{equation}
    G\simeq S_4\times_{\sigma}\eZ/2\eZ.
\end{equation}
Mais étant donné que la conjugaison par \( s_0\) est triviale, le produit semi-direct est un produit direct :
\begin{equation}
    G\simeq S_4\times\eZ/2\eZ.
\end{equation}
Il est maintenant du meilleur gout de pouvoir identifier géométriquement ces éléments. Les éléments de \( \eZ/2\eZ=\{ \id,s_0 \}\) ne font pas de mystères. Dans \( S_4\) nous avons les classes de conjugaison des éléments \( \id\), \( (12)\), \( (123)\), \( (1234)\) et \( (12)(34)\) déterminées durant l'exemple~\ref{ExVYZPzub}.
\begin{enumerate}
    \item
        L'élément \( (12)\) consiste à permuter deux diagonales et laisser les autres en place. Nous avons déjà vu que c'était une symétrie axiale passant par les milieux de deux côtés opposés. Cela fait \( 6\) axes d'ordre \( 2\).
    \item
        L'élément \( (123)\) fixe une des diagonales. C'est donc la symétrie axiale le long de la diagonale fixée. Par exemple la symétrie d'axe \( (AG)\) fait bouger le point \( B\) de la façon suivante :
        \begin{equation}
            B\to D\to E\to B.
        \end{equation}
        C'est une rotation est d'angle \( \frac{ 2\pi }{ 3 }\). Cela sont \( 8\) rotations d'ordre \( 3\).

        Notons à ce propos que la différence entre \( (234)\) et \( (243)\) est que la première fait une rotation d'angle \( 2\pi/3\) tandis que la seconde fait une rotation d'angle \( -2\pi/3\).

    \item
        L'élément \( (1234)\) ne maintient aucune des diagonales et est d'ordre \( 4\). C'est donc la rotation d'angle \( \pi/2\) ou \( -\pi/2\) autour de l'axe passant par les milieux de deux faces opposées. Il y en a \( 6\) comme ça (\( 3\) paires de faces puis pour chaque il y a \( \pi/2\) et \( -\pi/2\)), et ça tombe bien \( 6\) est justement la taille de la classe de conjugaison de \( (1234)\) dans \( S_4\).

    \item
        L'élément \( (12)(34)\) est le carré de la précédente\footnote{En fait c'est \( (13)(24)\), le carré de la précédente, mais c'est la même classe de conjugaison.}, c'est-à-dire les rotations d'angle \( \pi\) autour des mêmes axes. Cela fait \( 3\) éléments d'ordre \( 2\).

\end{enumerate}



\chapter{Corps}
% This is part of Mes notes de mathématique
% Copyright (c) 2011-2019
%   Laurent Claessens
% See the file fdl-1.3.txt for copying conditions.

%+++++++++++++++++++++++++++++++++++++++++++++++++++++++++++++++++++++++++++++++++++++++++++++++++++++++++++++++++++++++++++
\section{Généralités}
%+++++++++++++++++++++++++++++++++++++++++++++++++++++++++++++++++++++++++++++++++++++++++++++++++++++++++++++++++++++++++++

\begin{normaltext}      \label{NORMooGPWRooIKJqqw}
    Nous trouvons parfois le terme \defe{anneau à division}{anneau!à division}. Cela provient du fait que dans beaucoup de cas on considère uniquement des corps commutatifs; donc on voudrait une façon de parler d'un anneau dont tous les éléments non nuls sont inversibles. Dans ce cadre on dit :
    \begin{itemize}
        \item Un anneau à division est un anneau dont tous les éléments non nuls sont inversibles,
        \item Un corps est un anneau à division commutatif.
    \end{itemize}
    Pour prendre un exemple de cette différence, le théorème de Wedderburn~\ref{ThoMncIWA} est énoncé ici sous les termes «Tout corps fini est commutatif». Sous-entendu : la commutativité ne fait pas partie de la définition d'un corps. Par contre dans \cite{KXjFWKA} il est énoncé sous les termes «Tout anneau à division fini est un corps». Chez lui, un corps est toujours commutatif et un anneau à division est ce que nous appelons ici un corps.
\end{normaltext}

%---------------------------------------------------------------------------------------------------------------------------
\subsection{Corps ordonnés}
%---------------------------------------------------------------------------------------------------------------------------

Nous avons vu la définition de corps totalement ordonné en~\ref{DefKCGBooLRNdJf}.

\begin{definition}[\cite{ooTKEHooQuaFuD}]
    Un corps est \defe{formellement réel}{corps!formellement réel} si \( -1\) n'est pas une somme de carrés.
\end{definition}

\begin{proposition}
    Un corps totalement ordonné est formellement réel.
\end{proposition}

\begin{proof}
    Soit un corps totalement ordonné \( (\eK,\leq)\) et \( a\in \eK\) alors \( a^2\geq 0\). En effet si \( a\geq 0\) alors \( a^2=a\times a\geq 0\) directement par la définition~\ref{DefKCGBooLRNdJf}\ref{CONDooBYYDooElXgPO}. Si \( a\leq 0\) alors \( -a\geq 0\) et
    \begin{equation}
        a^2=(-a)^2\geq 0.
    \end{equation}
    Vu que \( -1<0\), il ne peut donc pas être écrit comme un carré. A fortiori comme somme de carrés.
\end{proof}

%---------------------------------------------------------------------------------------------------------------------------
\subsection{Automorphismes de \texorpdfstring{$ \eR$}{R} et \texorpdfstring{$ \eC$}{C}}
%---------------------------------------------------------------------------------------------------------------------------

\begin{proposition}[\cite{ooEKUSooDDDWuT,MonCerveau}]     \label{PROPooLLPMooIVanaO}
    L'identité est l'unique automorphisme du corps \( \eR\).
\end{proposition}

\begin{proof}
    Soit un automorphisme \( \sigma\colon \eR\to \eR\). Comme pour tout automorphisme,
    \begin{equation}
        \sigma(a)=\sigma(1a)=\sigma(1)\sigma(a).
    \end{equation}
    Donc \( \sigma(1)=1\).

    \begin{subproof}
    \item[Identité sur les rationnels]
    De plus
    \begin{equation}
        \sigma(n)=\sigma(1+\ldots +1)=\sigma(1)+\ldots +\sigma(1)=n,
    \end{equation}
    et
    \begin{equation}
        \sigma\left( \frac{1}{ n } \right)+\ldots +\sigma\left( \frac{1}{ n } \right)=\sigma\left( \frac{1}{ n }+\ldots +\frac{1}{ n } \right)=\sigma(1)=1.
    \end{equation}
    Donc \( \sigma(1/n)=1/n\).

    Nous en déduisons que pour tout \( q\in \eQ\), \( \sigma(q)=q\). Cela ne suffit pas pour déduire \( \sigma(x)=x\) pour tout \( x\in \eR\) parce que rien n'indique que \( \sigma\) soit continue.
        \item[Positive sur les positifs]

            Si \( x>0\) alors \( \sigma(x)=\sigma(\sqrt{ x })^2>0\).

        \item[Croissance]

            Si \( x>y\) alors \( x-y>0\) et \( \sigma(x-y)>0\). Cela donne \( \sigma(x)>\sigma(y)\).

        \item[Identité sur les réels]

            Soit un irrationnel \( x\in \eR\) et une suite \( (q_i)\) dans \( \eQ\) qui converge de façon croissante vers \( x\). Soit \( \epsilon>0\) dans \( \eQ\). Il existe \( N\) tel que si \( i>N\) alors \( q_i+\epsilon>x\); en appliquant \( \sigma\) à cette inégalité et en se souvenant que \( \sigma\) est l'identité sur \( \eQ\),
            \begin{equation}
                q_i+\epsilon>\sigma(x).
            \end{equation}
            Mais de plus, \( q_i<x\) donne \( \sigma(q_i)<\sigma(x)\), c'est-à-dire \( q_i<\sigma(x)\). En regroupant ces deux inégalités,
            \begin{equation}        \label{EQooLZOUooPhUNTI}
                q_i<\sigma(x)<q_i+\epsilon
            \end{equation}
            pour tout \( \epsilon>0\) dans \( \eQ\) et \( i>N\). Ce \( \epsilon\) étant fixé nous pouvons prendre la limite des inégalités \eqref{EQooLZOUooPhUNTI} :
            \begin{equation}
                x\leq \sigma(x)\leq x+\epsilon.
            \end{equation}
            Cela étant valable pour tout \( \epsilon>0\) dans \( \eQ\), nous avons bien \( x=\sigma(x)\).
    \end{subproof}
\end{proof}

\begin{remark}      \label{REMooGHEDooOYYUPk}
    Certains\cite{ooEKUSooDDDWuT} pensent que l'énoncé de cette proposition, ne parlant que de \emph{corps} \( \eR\) n'autorise pas l'utilisation d'autre structure réelle que celle de corps. Du coup il faut reconstruire la notion d'ordre à partir seulement du langage des corps. Par exemple en disant que \( a>b\) si et seulement si il existe \( k\) tel que \( a=b+k^2\).

    On peut s'en sortir en donnant l'énoncé suivant : «Si \( \eK\) est un corps isomorphe (en tant que corps) à \( \eR\) alors son unique automorphisme est l'identité». Cela se démontre immédiatement en disant que si \( f\) est un automorphisme de \( \eK\) et si \( \phi\) est un isomorphisme \( \eK\to \eR\) alors \( \phi\circ f\circ \phi^{-1}\) est un automorphisme de \( \eR\). Donc il est l'identité et \( f\) l'est également.

    Attention cependant à prouver que \( \phi^{-1}\) est un morphisme. En effet en posant \( \phi^{-1}(x)=a\) et \( \phi^{-1}(y)=b\) nous avons
    \begin{equation}
        \phi\big( \phi^{-1}(x)+\phi^{-1}(y) \big)=x+y
    \end{equation}
    parce que \( \phi\) est un morphisme. D'autre part,
    \begin{equation}
        \phi\big( \phi^{-1}(x)+\phi^{-1}(y) \big)=\phi(a+b).
    \end{equation}
    Donc
    \begin{equation}
        \phi^{-1}(x+y)=\phi^{-1}\big( \phi(a)+\phi(b) \big)=\phi^{-1}\big( \phi(a+b) \big)=a+b=\phi^{-1}(x)+\phi^{-1}(y).
    \end{equation}
\end{remark}

\begin{proposition}     \label{PROPooEATMooIPPrRV}
    Un automorphisme du corps \( \eC\) qui fixe \( \eR\) est soit l'identité soit la conjugaison complexe\footnote{Par «fixer \( \eR\)» nous entendons que \( \sigma(\eR)=\eR\), pas spécialement que \( \sigma(x)=x\) pour tout \( x\in \eR\)}.
\end{proposition}

\begin{proof}
    Soit un automorphisme \( \sigma\) vérifiant la condition de fixer \( \eR\). Alors la restriction de \( \sigma\) à \( \eR\) est un automorphisme de \( \eR\) et y est donc l'identité par la proposition~\ref{PROPooLLPMooIVanaO}.

    En ce qui concerne les imaginaires purs,
    \begin{equation}
        -1=\sigma(-1)=\sigma(ii)=\sigma(i)^2.
    \end{equation}
    Donc \( \sigma(i)\) est un élément de \( \eC\) vérifiant \( \sigma(i)^2=-1\). C'est-à-dire \( \sigma(i)=\pm i\).

    Si \( \sigma(i)=i\) alors \( \sigma=\id\). Si \( \sigma(i)=-i\) alors \( \sigma\) est la conjugaison complexe.
\end{proof}

%---------------------------------------------------------------------------------------------------------------------------
\subsection{Corps premier}
%---------------------------------------------------------------------------------------------------------------------------
\label{subseccorpspremhBlYIv}

\begin{definition}
    Un corps est \defe{premier}{corps!premier}\index{premier!corps} s'il est son seul sous corps. Le \defe{sous corps premier}{premier!sous corps} d'un corps est l'intersection de tous ses sous corps.
\end{definition}

\begin{lemma}
    Un corps premier est commutatif.
\end{lemma}

\begin{proof}
    Le centre d'un corps est certainement un sous corps. Par conséquent un corps premier doit être contenu dans son propre centre, c'est-à-dire être commutatif.
\end{proof}

\begin{definition}  \label{DefXIHLooBAcqYH}
Soit \( p\) un nombre premier. Nous notons \( \eF_p=\eZ_p=\eZ/p\eZ\)\nomenclature[A]{\( \eF_p\)}{lorsque \( p\) est premier}.
\end{definition}

Nous verrons plus loin (section~\ref{SecCorpsFinizkAcbS}) comment nous pouvons définir \( \eF_{p^n}\) lorsque \( p\) est premier, ainsi que l'unicité d'un tel corps.

Nous avons par exemple
\begin{equation}
    \eF_2=\eZ/2\eZ=\{ 0,1 \}
\end{equation}
avec la loi \( 2=0\).

Notons que \( \eF_p\) est un corps possédant \( p\) éléments. L'ensemble \( \eF_p^*\) est un groupe d'ordre \( p-1\).

\begin{lemma}
    Les corps \( \eQ\) et \( \eF_p\) (avec \( p\) premier) sont premiers.
\end{lemma}

\begin{proof}
    Tout sous corps de \( \eQ\) doit contenir \( 1\), et par conséquent \( \eZ\). Devant également contenir tous les inverses, il contient \( \eQ\).

    Tout sous corps de \(\eF_p \) doit contenir \( 1\) et donc \( \eF_p\) en entier. Par ailleurs nous savons de la proposition~\ref{PropzhFgNJ} que \( \eF_p\) est un corps lorsque \( p\) est premier.
\end{proof}

\begin{proposition}
    Soit \( \eK\) un corps de caractéristique \( p\) et \( \eP\) son sous corps premier. Si \( p=0\) alors \( \eP=\eQ\). Si \( p>0\), alors \( \eP=\eF_p\).
\end{proposition}

\begin{proof}
    Notons d'abord que la caractéristique d'un corps est toujours soit 0 soit un nombre premier, parce qu'un corps est en particulier un anneau intègre (proposition~\ref{LemCaractIntergernbrcartpre}).

    Étant donné que \( 1\) est dans tout sous corps, nous devons avoir \( \eZ 1\subseteq \eP\). Si \( p=0\), alors \( \eZ 1\simeq \eZ\), et nous avons
    \begin{equation}
        \eZ 1_{\eA}\subset \eP\subset \eK.
    \end{equation}
    Pour chaque \( (n,m)\in \eZ 1_{\eA}\times (\eZ 1_{\eA})^*\) l'élément \( nm^{-1}\in \eK\) est dans \( \eP\) parce que \( \eP\) est un corps. Nous en déduisons que le corps des fractions de \( \eZ\) est contenu dans \( \eP\) par conséquent \( \eP=\eQ\) (théorème~\ref{ThogbhWgo}).

    Si par contre la caractéristique de \( \eK\) est \( p\neq 0\), nous avons \( \eZ 1_{\eA}\simeq\eZ/p\eZ=\eF_p\) par le lemme~\ref{LemHmDaYH}. L'ensemble \( \eF_p\) étant un corps, c'est le corps premier de \( \eK\).
\end{proof}

\begin{proposition}     \label{PropqPPrgJ}
    Soit \( \eK\) un corps et \( \eP\) son sous-corps premier. Si \( \sigma\in\Aut(\eK)\) alors \( \sigma|_{\eP}=\id\), c'est-à-dire que $\sigma(x)=x$ pour tout \( x\in \eP\).
\end{proposition}

%---------------------------------------------------------------------------------------------------------------------------
\subsection{Petit théorème de Fermat}
%---------------------------------------------------------------------------------------------------------------------------

\begin{theorem}[Petit théorème de Fermat]       \label{ThoOPQOiO}   \index{théorème!petit de Fermat}\index{petit théorème de Fermat}
    Soit \( p\) un nombre premier. Si \( x\in \eF_p\) alors \( x^p=x\). Si \( x\in(\eF_p)^*\), alors \( x^{p-1}=1\).

    En particulier si \( x\in \eF_p^*\) alors \( x^{-1}=x^{p-2}\).
\end{theorem}

\begin{proof}
    Étant donné que \( \eF_p\) est un corps commutatif et que \( p\) est premier, la proposition~\ref{Propqrrdem} nous indique que \( \sigma(x)=x^p\) est un automorphisme. La proposition~\ref{PropqPPrgJ} nous indique alors que
    \begin{equation}
        a^p=a.
    \end{equation}
    Si \( a\) est inversible alors \( a^{p-1}=a^pa^{-1}=1\).
\end{proof}

\begin{remark}      \label{RemCoSnxh}
    Une autre façon d'énoncer le petit théorème de Fermat~\ref{ThoOPQOiO} est que si \( p\) est premier et si \( a\) est premier avec \( p\), alors \( a^{p-1}\in[1]_p\). Le nombre \( a\) n'est pas premier avec \( p\) uniquement lorsque \( a\) est multiple de \( p\). Dans ce cas c'est \( a=0\) dans \( \eF_p\) et donc \( a^{p-1}=0\).
\end{remark}

\begin{example}
    Soit \( \eK=\eF_{29}\). Le nombre \( 29\) étant premier, \( \eK\) est un corps premier. C'est le corps des entiers modulo \( 29\). Nous avons donc
    \begin{equation}
            -142=-113=-84=-55=-26=3=32=61=90=119.
    \end{equation}
    Le petit théorème de Fermat nous permet aussi de calculer des exposants et des inverses. En effet, puisque \( 1=x^{28}\) pour tout \( x \in \eF_{29}^* \), nous avons \( x^{-1}=x^{27}\), et par suite, pour tout entier \( a \),
    \begin{equation}
        x^{-a}=(x^a)^{27}=x^{27a}.
    \end{equation}
    Le nombre \( 27 a\) peut être grand par rapport à \( 29\). Mais en réutilisant le fait que \( 1=x^{28}\), on obtient
    \begin{equation}
        x^{-a}=x^{[27a]_{28}}.
    \end{equation}
    Cette expression doit être comprise comme disant que pour tout \( k\in [27a]_{28}\) nous avons \( x^{-a}=x^{k}\).

    Chose à retenir : dans les exposants nous calculons modulo \( 28\).
\end{example}

%+++++++++++++++++++++++++++++++++++++++++++++++++++++++++++++++++++++++++++++++++++++++++++++++++++++++++++++++++++++++++++
\section{Théorème des deux carrés}
%+++++++++++++++++++++++++++++++++++++++++++++++++++++++++++++++++++++++++++++++++++++++++++++++++++++++++++++++++++++++++++

\begin{proposition} \label{PropleGdaT}
    Soit \( p\) un nombre premier et \( P\) un élément de \( \eF_p[X]\). L'anneau \( \eF_p[X]/(P)\) est intègre si et seulement si \( P\) est irréductible dans \( \eF_p[X]\).
\end{proposition}

\begin{proof}
    Supposons que \( P\) soit réductible dans \( \eF_p[X]\), c'est-à-dire qu'il existe \( Q,R\in \eF_p[X]\) tels que \( P=QR\). Dans ce cas, \( \bar Q\) est diviseur de zéro dans \( \eF_p[X]/(P)\) parce que \( \bar Q\bar R=0\).

    Nous supposons maintenant que \( \eF_p[X]/(P)\) ne soit pas intègre : il existe des polynômes \( R,Q\in \eF_p[X]\) tels que \( \bar Q\bar R=0\). Dans ce cas le polynôme \( QR\) est le produit de \( P\) par un polynôme : \( QR=PA\). Tous les facteurs irréductibles de \( A \) étant soit dans \( Q\) soit dans \( R\), il est possible de modifier un peu \( Q\) et \( R\) pour obtenir \( QR=P\), ce qui signifie que \( P\) n'est pas irréductible.
\end{proof}

%---------------------------------------------------------------------------------------------------------------------------
\subsection{Un peu de structure dans \texorpdfstring{$ \eZ[i]$}{Zi}}
%---------------------------------------------------------------------------------------------------------------------------

\begin{lemma}   \label{LemSCAlICY}
     L'application
     \begin{equation}
         \begin{aligned}
             N\colon \eZ[i]&\to \eN \\
             a+bi&\mapsto a^2+b^2
         \end{aligned}
     \end{equation}
     est un stathme euclidien pour \( \eZ[i]\).
\end{lemma}
\index{stathme!sur \( \eZ[i]\)}

\begin{proof}
    Soient \( t,z\in \eZ[i]\setminus\{ 0 \}\) dont le quotient s'écrit
    \begin{equation}
        \frac{ z }{ t }=x+iy
    \end{equation}
    dans \( \eC\). Nous considérons \( q=a+bi\) où \( a\) et \( b\) sont les entiers les plus proches de \( x\) et \( y\). S'il y a \emph{ex aequo}, on prend au hasard\footnote{Dans l'exemple~\ref{ExwqlCwvV}, nous prenions toujours l'inférieur parce que le stathme tenait compte de la positivité.}. Alors nous avons
    \begin{equation}
        | \frac{ z }{ t }-q |\leq \frac{ | 1+i | }{ 2 }=\frac{ \sqrt{2} }{2}<1.
    \end{equation}
    On pose \( r=z-qt\) qui est bien un élément de \( \eZ[i]\). De plus
    \begin{equation}
        | r |=| z-qt |=| t | |\frac{ z }{ t }-q |<| t |,
    \end{equation}
    c'est-à-dire que \( | r |^2<| t |^2\) et donc \( N(r)<N(t)\).
\end{proof}
Étant donné que \( \eZ[i]\) est euclidien, il est principal (proposition~\ref{Propkllxnv}).

\begin{lemma}   \label{LemBMEIiiV}
    Les éléments inversibles de \( \eZ[i]\) sont \( \{ \pm 1,\pm i \}\).
\end{lemma}

\begin{proof}

    Déterminons les éléments inversibles de \( \eZ[i]\). Si \( z\in \eZ[i]^*\), alors il existe \( z'\in \eZ[i]^*\) tel que \( zz'=1\). Dans ce cas nous aurions
    \begin{equation}
        1=N(zz')=N(z)N(z'),
    \end{equation}
    ce qui est uniquement possible avec \( N(z)=N(z')=1\), c'est-à-dire \( z=\pm 1\) ou \( z=\pm i\). Nous avons donc
    \begin{equation}
        \eZ[i]^*=\{ \pm 1,\pm i \}.
    \end{equation}
\end{proof}

\begin{definition}[\cite{ooHZAVooDDUQce}]      \label{DEFooUCSHooJqGuVB}
    Un \defe{monoïde}{monoïde} est un triple \( (E,*,e)\) où \( E\) est un ensemble, \( e\) est un élément de \( E\) et \( *\colon E\times E\to E\) est une loi de composition telle que pour tout \( x,y\in E\),
    \begin{enumerate}
        \item
            \( x*(y*z)=(x*y)*z\) (associativité)
        \item
            \( e*x=x*e=x\) (\( e\) est un neutre).
    \end{enumerate}
\end{definition}

Nous notons \( \Sigma=\{ a^2+b^2\tq a,b\in \eN \}\).
\begin{lemma}   \label{LemIBDPzMB}
    L'ensemble \( \Sigma=  \{ a^2+b^2\tq a,b\in \eN \}  \) est un sous-monoïde\footnote{Définition \ref{DEFooUCSHooJqGuVB}.} de \( \eN\).
\end{lemma}

\begin{proof}
    Il suffit de prouver que si \( m,n\in \Sigma\), alors le produit \( mn\) est également dans \( \Sigma\). Si \( N\) est le stathme euclidien sur \( \eZ[i]\), alors  \( n\in \Sigma\) si et seulement s'il existe \( z\in \eZ[i]\) tel que \( N(z)=n\). Si \( z,z'\in \eZ[i]\), alors \( zz'\in \eZ[i]\) et de plus
    \begin{equation}
        N(zz')=N(z)N(z')=nm.
    \end{equation}
    Donc \( nm\) est l'image de \( zz'\) par \( N\), ce qui prouve que \( nm\in \Sigma\).
\end{proof}

\begin{theorem}[Théorème des deux carrés, version faible]   \label{ThospaAEI}
    Un nombre premier est somme de deux carrés si et seulement si \( p=2\) ou \( p\in[1]_4\).
\end{theorem}
\index{anneau!principal}
\index{nombre!premier}
\index{théorème!des deux carrés!version faible}

\begin{remark}
    Il n'est pas dit que les nombres dans \( [1]_4\) sont premiers (\( 9=8+1\) ne l'est pas par exemple). Le théorème signifie que (à part \( 2\)), si un nombre premier est dans \( [1]_4\) alors il est somme de deux carrés, et inversement, si un nombre premier est somme de deux carrés, il est dans \( [1]_4\).
\end{remark}

\begin{proof}
    Soit \( p\) un nombre premier dans \( \Sigma\). Si \( a=2k\), alors \( a^2=4k^2\) et \( a^2=0\mod 4\). Si au contraire \( a\) est impair, \( a=2k+1\) et \( a^2=4k^2+1+4k=1\mod 4\). La même chose est valable pour \( b\). Par conséquent, \( a^2+b^2\) est automatiquement \( [0]_4\), \( [1]_4\) ou \( [2]_4\). Évidemment les nombres de la forme \( 0\mod 4\) ne sont pas premiers; parmi les \( 2\mod 4\), seul \( p=2\) est premier (et vaut \( 1^2+1^2\)).

    Nous avons démontré que les seuls premiers de la forme \( a^2+b^2\) sont \( p=2\) et les \( p=1\mod 4\). Il reste à faire le contraire : démontrer que si un nombre premier \( p\) vaut \( 1\mod 4\), alors il est premier. Nous considérons l'anneau
    \begin{equation}
        \eZ[i]=\{ a+bi\tq a,b\in \eZ \}.
    \end{equation}
    puis l'application
    \begin{equation}
        \begin{aligned}
            N\colon \eZ[i]&\to \eN \\
            a+bi&\mapsto a^2+b^2.
        \end{aligned}
    \end{equation}
    Un peu de calcul dans \( \eC\) montre que pour tout \( z,z'\in \eZ[i]\), \( N(zz')=N(z)N(z')\).


    Nous savons que les éléments inversibles de \( \eZ[i]\) sont \( \pm 1\) et \( \pm i\) (lemme~\ref{LemBMEIiiV}).

    Le lemme~\ref{LemSCAlICY} montre que \( \eZ[i]\) est un anneau euclidien parce que \( N\) est un stathme. L'anneau \( \eZ[i]\) étant euclidien, il est principal (proposition~\ref{Propkllxnv}).



    Pour la suite, nous allons d'abord montrer que \( p\in\Sigma\) si et seulement si \( p\) n'est pas irréductible dans \( \eZ[i]\), puis nous allons voir quels sont les irréductibles de \( \eZ[i]\).

    Soit \( p\), un nombre premier dans \( \Sigma\). Si \( p=a^2+b^2\), alors nous avons \( p=(a+ib)(a-bi)\), mais étant donné que \( p\) est premier, nous avons \( a\neq 0\) et \( b\neq 0\). Du coup \( p\) n'est pas inversible dans \( \eZ[i]\), mais il peut être écrit comme le produit de deux non inversibles. Le nombre \( p\) est donc non irréductible dans \( \eZ[i]\).

    Dans l'autre sens, nous supposons que \( p\) est un nombre premier non irréductible dans \( \eZ[i]\). Nous avons alors \( p=zz'\) avec ni \( z\) ni \( z'\) dans \( \{ \pm 1,\pm i \}\). En appliquant \( N\) nous avons
    \begin{equation}
        p^2=N(p)=N(z)N(z').
    \end{equation}
    Vu que \( p\) est premier, cela est uniquement possible avec \( N(z)=N(z')=p\) (avoir \( N(z)=1\) est impossible parce que cela dirait que \( z\) est inversible). Si \( z=a+ib\), alors \( p=N(z)=a^2+b^2\), et donc \( p\in \Sigma\).

    Nous savons déjà que \( \eZ[i]\) est un anneau principal et n'est pas un corps; la proposition~\ref{PropomqcGe} s'applique donc et \( p\) sera non irréductible si et seulement si l'idéal \( (p)\) sera non premier. Le fait que \( (p)\) soit un idéal non premier implique que le quotient \( \eZ[i]/(p)\) est non intègre (c'est la définition d'un idéal premier). Nous cherchons donc les nombres premiers pour lesquels le quotient \( \eZ[i]/(p)\) n'est pas intègre.

    Nous commençons par écrire le quotient \( \eZ[i]/(p)\) sous d'autres formes. D'abord en remarquant que si \( I\) et \( J\) sont deux idéaux, on a \( (\eA/I)/J\simeq (\eA/J)/I\), du coup, en tenant compte du fait que \( \eZ[i]=\eZ[X]/(X^2+1)\), nous avons
    \begin{equation}
        \eZ[i]/(p)=(\eZ[X]/(p))/(X^2+1)=\eF_p[X]/(X^2+1).
    \end{equation}
    Nous avons donc équivalence des propositions suivantes :
    \begin{subequations}
        \begin{align}
            p\in\Sigma\\
            \eF_p[X]/(X^2+1)\text{ n'est pas intègre}\\
            X^2+1\text{ n'est pas irréductible dans } \eF_p \label{EqZkdrvh}\\
             X^2+1\text{ admet une racine dans } \eF_p\\
            -1\in (\eF_p^*)^2\\
            \exists y\in \eF_p^*\tq y^2=-1.
        \end{align}
    \end{subequations}
    Le point \eqref{EqZkdrvh} vient de la proposition~\ref{PropleGdaT}. Maintenant nous utilisons le fait que \( p\) soit un premier impair (parce que le cas de \( p=2\) est déjà complètement traité), donc \( (p-1)/2\in \eN\) et nous avons, pour le \( y\) de la dernière ligne,
    \begin{equation}
        (-1)^{(p-1)/2}=(y^2)^{(p-1)/2}=y^{p-1}=1
    \end{equation}
    parce que dans \( \eF_p\) nous avons \( y^{(p-1)}=1\) par le petit théorème de Fermat (théorème~\ref{ThoOPQOiO}). Du coup \( p\) doit vérifier
    \begin{equation}
        1=(-1)^{(p-1)/2},
    \end{equation}
    c'est-à-dire \( \frac{ p-1 }{2}=0\mod 2\) ou encore \( p=1\mod 4\).
\end{proof}

\begin{theorem}[Théorème des deux carrés\cite{KXjFWKA}]
    Soit \( n\geq 2\) un nombre dont nous notons
    \begin{equation}    \label{EqBMHTzCT}
        n=\prod_{p\in\pP}p^{v_p(n)}
    \end{equation}
    où \( \pP\) est l'ensemble des nombres premiers. Alors \( n\in \Sigma\) si et seulement si pour tout \( p\in\pP\cap[3]_4\), nous avons \( v_p(n)\in [0]_2\) (c'est-à-dire \( v_p(n)\) est pair).
\end{theorem}
\index{théorème!des deux carrés}
\index{nombre!premier!théorème des deux carrés}
\index{anneau!principal!utilisation}
%TODO : il y a un lien entre le théorème des deux carrés et les triplets pytagoritiens http://fr.wikipedia.org/wiki/Triplet_pythagoricien

\begin{proof}
    \begin{subproof}
    \item[Condition suffisante.]

        Le produit \eqref{EqBMHTzCT} est évidemment un produit fini que nous pouvons alors regrouper en quatre parties : \( \pP\cap[0]_4\), \( \pP\cap[1]_4\), \( \pP\cap[2]_4\) et \( \pP\cap[3]_4\).

        \begin{itemize}
            \item Il n'y a pas de nombres premiers dans \( [0]_4\).
            \item Les nombres premiers de \( [1]_4\) sont dans \( \Sigma\). Le produit d'éléments de \( \Sigma\) étant dans \( \Sigma\), nous avons
                \begin{equation}
                    \prod_{p\in\pP\cap[1]_4}p^{v_p(n)}\in \Sigma.
                \end{equation}
            \item
                Le seul nombre premier dans \( [2]_4\) est \( 2\). C'est un élément de \( \Sigma\).
            \item
                Le produit
                \begin{equation}
                    \prod_{p\in\pP\cap[3]_4}p^{v_p(n)}
                \end{equation}
                est par hypothèse un produit de carrés (\( v_p(n)\) est pair), et est donc un carré.
        \end{itemize}
        Au final le produit \( \prod_{p\in\pP}p^{v_p(n)}\) est un produit d'un carré par un élément de \( \Sigma\), ce qui est encore un élément de \( \Sigma\).

        Pour cette partie, nous avons utilisé et réutilisé le lemme~\ref{LemIBDPzMB}.

    \item[Condition nécessaire.]

        Soit \( p\), un nombre premier. Nous voulons montrer que
        \begin{equation}
            \{ v_p(n)\tq n\in \Sigma \}\subset [2]_2.
        \end{equation}
        Pour montrer cela nous allons procéder par récurrence sur les ensembles
        \begin{equation}
            E_k=\{ v_p(n)\tq n\in \Sigma \}\cap\{ 0,\ldots, k \}.
        \end{equation}
        Il est évident que les éléments de \( E_0\) sont pairs, vu qu'il n'y a que zéro, qui est pair.

        Supposons que \( E_k\subset[0]_2\), et montrons que \( E_{k+1}\subset[0]_2\). Soit un élément de \( E_{k+1}\), c'est-à-dire \( v_p(n)\leq k+1\) avec \( n=a^2+b^2\). Si \( v_p(n)=0\) alors l'affaire est réglée; sinon c'est que \( p\) divise \( n\). Mais dans \( \eZ[i]\) nous avons
        \begin{equation}
            n=a^2+b^2=(a+bi)(a-bi)
        \end{equation}
        Vu que \( \eZ[i]\) est principal, le lemme de Gauss~\ref{LemSdnZNX} nous dit que si \( p\) divise \( n\), alors il doit diviser soit \( a+bi\), soit \( a-bi\) (et du coup en fait les deux). Nous avons alors \( p\divides a\) et \( p\divides b\) en même temps. Du coup
        \begin{equation}
            p^2\divides a^2+b^2=n.
        \end{equation}
        Posons \( a=pa'\) et \( b=pb'\) avec \( a',b'\in \eN\). Nous avons
        \begin{equation}
            \frac{ n }{ p^2 }=\frac{ p^2a'^2+p^2b'^2 }{ p^2 }=a'^2+b'^2\in \Sigma.
        \end{equation}
        Mais par construction,
        \begin{equation}
            v_p\left( \frac{ n }{ p^2 } \right)=v_p(n)-2<k.
        \end{equation}
        Donc \( v_p(\frac{ n }{ p^2 })\) est pair et du coup \( v_p(n)\) doit également être pair.

    \end{subproof}
\end{proof}

%---------------------------------------------------------------------------------------------------------------------------
\subsection{Résultats chinois}
%---------------------------------------------------------------------------------------------------------------------------

Nous allons maintenant parler du système d'équations
\begin{subequations}
    \begin{numcases}{}
        x=a_1\mod p\\
        x=a_2\mod q
    \end{numcases}
\end{subequations}
avec \( a_1\), \( a_2\) donnés dans \( \eZ\) et \( p,q\) des entiers premiers entre eux. Le lemme chinois nous donne la liste des solutions ainsi qu'une manière de les construire. Le théorème chinois en sera une espèce de corolaire qui établira l'isomorphisme d'anneaux \( \eZ/pq\eZ\simeq \eZ/p\eZ\times \eZ/q\eZ\). Voir \href{http://www.les-mathematiques.net/b/a/d/node10.php}{les-mathematiques.net}.

\begin{lemma}[Lemme chinois \cite{CongrDuchSyl}]        \label{LemCtUeGA}
    Soient \( n_1,n_2\) deux entiers premiers entre eux. Soient \( a_1,a_2\in \eZ\). Les solutions du système
    \begin{subequations}        \label{SysVwvLKv}
        \begin{numcases}{}
            x=a_1\mod n_1\\
            x=a_2\mod n_2
        \end{numcases}
    \end{subequations}
    pour \( x\in \eZ/n_1n_2\eZ\) sont données de la façon suivante. Soient \( u_1,u_2\) deux entiers qui satisfont la relation de Bézout\footnote{voir le théorème~\ref{ThoBuNjam}}
    \begin{equation}        \label{EqWcucUG}
        u_1n_1+u_2n_2=1,
    \end{equation}
    et
    \begin{equation}        \label{EqHGchlQ}
        a=\big( a_1u_2n_2+a_2 u_1n_1 \big)\mod(n_1).
    \end{equation}
    Alors \( x=a\mod(n_1n_2)\).
\end{lemma}

\begin{proof}
    Vérifions que le \( x\) donné par \(x=a\mod(n_1n_2)\) est bien une solution. D'abord
    \begin{subequations}
        \begin{align}
            a\mod n_2&=a_1u_2n_2\mod n_1\\
            &=a_1(1-u_1n_1)\mod n_1\\
            &=a_1\mod n_1
        \end{align}
    \end{subequations}
    où nous avons utilisé l'identité de Bézout \eqref{EqWcucUG}. La vérification de \( a\mod n_2=a_2\mod n_2\) est la même.

    Soit maintenant \( x\in \eZ/n_1n_2\eZ\) une solution du système \eqref{SysVwvLKv} et \( a\) donné par la formule \eqref{EqHGchlQ}. Alors
    \begin{subequations}
        \begin{align}
            (x-a)\mod n_1&=\Big( a_1-(a_1n_2u_2+a_2u_1n_1) \Big)\mod n_1\\
            &=a_1-a_1u_2n_2\mod n_1\\
            &=0,
        \end{align}
    \end{subequations}
    donc \( (x-a)\mod n_1=0\), ce qui signifie que \( x-a\) est divisible par \( n_1\). De la même façon, \( (x-a)\mod n_2=0\) et \( x-a\) est divisible par \( n_2\). Nous savons maintenant que \( x-a\) est divisible par \( n_1\) et \( n_2\). Étant donné que \( n_1\) et \( n_2\) sont premiers entre eux, nous en déduisons que \( x-a\) est divisible par \( n_1n_2\), ou encore que \( x=a\mod n_1n_2\).
\end{proof}

\begin{theorem}[Théorème chinois]
    Soient \( p,q\) deux naturels premiers entre eux. Si \( p,q\geq 2\) alors l'application
    \begin{equation}
        \begin{aligned}
            \phi\colon \eZ/pq\eZ&\to \eZ/p\eZ\times \eZ/q\eZ \\
            [x]_{pq}&\mapsto \big( [x]_p,[x]_q \big)
        \end{aligned}
    \end{equation}
    est un isomorphisme d'anneaux.
\end{theorem}

\begin{proof}
    Nous devons prouver que l'application \( \phi\) respecte la somme, le produit et qu'elle est bijective. En ce qui concerne la somme,
    \begin{subequations}
        \begin{align}
            \phi([q]_{pq}+[y]_{pq})&=            \phi\big( (x+y)\mod pq \big)\\
            &=\big( [x+y]_{p},[x+y]_q \big)\\
            &=\big( [x]_p+[y]_p,[x]_q+[y]_q \big)\\
            &=\big( [x]_p,[x]_q \big)+\big( [y]_p,[y]_q \big)\\
            &=\phi(x)+\phi(y).
        \end{align}
    \end{subequations}
    En ce qui concerne le produit, c'est le même jeu : nous obtenons
    \begin{equation}
        \phi\big( [xy]_{pq} \big)=\phi([x]_{pq}])\phi([y]_{pq})
    \end{equation}
    en utilisant le fait que \( [xy]_{p}=[x]_p[y]_p\).

    Montrons maintenant que \( \phi\) est surjective. Soient \( y_1,y_2\in \eZ\) et \( x\in \eZ\). Demander
    \begin{equation}
        \phi([x]_{pq})=\big( [y_1]_p,[y_2]_q \big)
    \end{equation}
    revient à demander que \( [x]_p=[y_1]_p\) et \( [x]_q=[y_2]_q\), c'est-à-dire que \( x\) résolve le système
    \begin{subequations}
        \begin{numcases}{}
            x=y_1\mod p\\
            x=y_2\mod q.
        \end{numcases}
    \end{subequations}
    Le lemme chinois~\ref{LemCtUeGA} nous assure qu'une solution existe.

    En ce qui concerne l'injectivité, nous supposons que \( x\) et \( y\) soient deux entiers tels que
    \begin{equation}
        \phi([x]_{pq})=\phi([y]_{pq}).
    \end{equation}
    Nous en déduisons le système
    \begin{subequations}
        \begin{numcases}{}
            x\mod p=y\mod p\\
            x\mod q=y\mod q
        \end{numcases}
    \end{subequations}
    c'est-à-dire qu'il existe des entiers \( k\) et \( l\) tels que \( x=y+kp\) et \( x=y+lq\) ou encore tels que
    \begin{equation}
        kp+lq=0.
    \end{equation}
    Étant donné que \( p\) et \( q\) sont premiers entre eux, la seule possibilité est \( k=l=0\), c'est-à-dire \( x=y\).
\end{proof}

\begin{theorem}[Théorème chinois]\index{théorème!chinois}   \label{THOooVIGQooUhwBLS}
    Soit \( A\) un anneau commutatif, \( n\geq 2\), des éléments \( x_1,\ldots,x_n\) dans \( A\) et des idéaux \( I_1,\ldots,I_n\) tels que \( I_i+I_j=A\) pour tout \( i\neq j\).

    Alors il existe un \( x\in A\) tel que \( x-x_i\in I_i\) pour tout \( 1\leq i\leq n\).
\end{theorem}

\begin{proof}
    Pour \( i\in\{ 1,\ldots,n \}\) nous notons \( J_i\) le produit \( J_i=\prod_{k\neq i}I_k\). Étant donné que chaque \( I_i\) est un idéal, nous avons \( I_k\subset J_i\) lorsque \( i\neq k\).

    Soit \( i\) fixé. Pour tout \( j\neq i\), puisque \( I_i+I_j=A\), nous pouvons trouver \( a_j\in I_i\) et \( b_j\in I_j\) tels que \( a_j+b_j=1\). Nous avons alors
    \begin{equation}
        1=\prod_{j\neq i}(a_j+b_j).
    \end{equation}
    Par ailleurs, \( I_i+J_i=A\) parce que \( J_i\) contient \( I_k\) avec \( k\neq i\) et \( I_i+I_k=A\). Nous pouvons donc prendre \( \alpha_i\in I_i\) et \( \beta_i\in J_i\) tels que
    \begin{equation}
        \alpha_i+\beta_i = 1 = \prod_{j\neq i}(a_j+b_j).
    \end{equation}
    Nous considérons alors l'élément \( x=\beta_1x_1+\cdots+\beta_nx_n\). Il vient alors
    \begin{subequations}
        \begin{align}
            x-x_1&=(\beta_1-1)x_1+\beta_2x_2+\cdots+\beta_nx_n\\
            &=-\alpha_1x_1+\beta_2x_2+\cdots+\beta_nx_n.
        \end{align}
    \end{subequations}
    Mais \( \alpha_1\in I_1\) et tous les autres termes sont dans les \( J_i\) avec \( i\neq 1\), donc aussi dans \( I_1 \) par définition des \( J_i \). (Par exemple,  \( \beta_2\in J_2\subset I_1\).  Nous en déduisons \( x - x_1 \in I_1\).

    L'argument que nous venons de donner pour justifier que \(x - x_1 \in I_1 \) peut être généralisé à tous les indices. En effet, soit \( k \) un indice quelconque; nous avons
     \begin{subequations}
        \begin{align}
            x-x_k&=(\beta_k-1)x_k+\sum_{i \neq k} \beta_ix_i\\
            &=-\alpha_kx_k+\sum_{i \neq k} \beta_ix_i;
        \end{align}
    \end{subequations}
    et \( \alpha_k\in I_k\) et pour tout \( i \neq k \), \( \beta_i\in J_i\subset I_k\)); donc \( x - x_k \in I_k\).
\end{proof}

\begin{remark}
    Ce théorème chinois est bien une généralisation du lemme chinois~\ref{LemCtUeGA}. En effet, l'élément \( x\) dont il est question est solution du problème \( x=x_i\mod I_i\). L'hypothèse \( I_i+I_j=A\) n'est pas nouvelle non plus étant donné que si \( p\) et \( q\) sont des entiers premiers entre eux nous avons \( p\eZ+q\eZ=\eZ\) par le corolaire~\ref{CorgEMtLj}.
\end{remark}

%+++++++++++++++++++++++++++++++++++++++++++++++++++++++++++++++++++++++++++++++++++++++++++++++++++++++++++++++++++++++++++
\section{Polynômes à coefficients dans un corps}
%+++++++++++++++++++++++++++++++++++++++++++++++++++++++++++++++++++++++++++++++++++++++++++++++++++++++++++++++++++++++++++
\label{SECooFYOGooQHitgE}

Nous supposons que \( \eK\) est un corps commutatif, et nous étudions l'anneau \( \eK[X]\), défini en~\ref{DEFooFYZRooMikwEL}.

\begin{proposition}     \label{PropqGZXvr}
    L'anneau \( \eK[X]\) des polynômes sur un corps commutatif \( \eK\) est factoriel.
\end{proposition}
%TODO : une preuve.

Le théorème suivant est un cas particulier pour \( \eK[X]\) du théorème chinois~\ref{ThofPXwiM}.
\begin{theorem}[Théorème chinois]\index{théorème!chinois!anneau des polynômes}
    Si \( P\) et \( Q\) sont deux polynômes premiers entre eux, alors nous avons l'isomorphisme
    \begin{equation}
        \eK[X]/(P,Q)\simeq\eK[X]/(P)\times \eK[X]/(Q).
    \end{equation}
\end{theorem}
% TODO : s'assurer que c'est bien un cas particulier du théorème chinois de plus haut.


%---------------------------------------------------------------------------------------------------------------------------
\subsection{Irréductibilité}
%---------------------------------------------------------------------------------------------------------------------------

\begin{definition}[\cite{ooJJSGooBXOPGF}]      \label{DefIrredfIqydS}
    Un polynôme à coefficients dans un anneau commutatif est irréductible si il
\begin{enumerate}
        \item
            n'est pas inversible,
        \item
            n'est pas le produit de deux non inversibles.
    \end{enumerate}
\end{definition}

    Un polynôme est irréductible dans \( \eK[X]\) au sens de la définition~\ref{DeirredBDhQfA} si et seulement s'il est irréductible au sens de la définition~\ref{DefIrredfIqydS} parce que seules les constantes (non nulles) sont inversibles dans \( \eK[X]\).

\begin{example}
    Si un polynôme \( P\in \eZ[X]\) n'a que des racines complexes, ça ne l'empêche pas d'être réductible sur \( \eZ\). La réductibilité ne signifie pas qu'on peut mettre des racines en évidence. Par exemple le polynôme \( P=X^4+3X^2+2\) est réductible sur \( \eZ\) en
    \begin{equation}
        P=(X^2+1)(X^2+2),
    \end{equation}
    mais n'a pas de racines dans \( \eZ\). Par contre, il est vrai que si on veut réduire plus, il faut sortir de \( \eZ\).

\end{example}

\begin{definition}  \label{DefCPLSooQaHJKQ}
    Nous disons que \( P\in\eK[X]\) est \defe{scindé}{polynôme scindé} sur \(\eK\) s'il est produit dans \(\eK[X]\) de polynômes de degré~\( 1\).
\end{definition}
Note : les constantes ne sont donc pas des polynômes scindés.

\begin{proposition}[\wikipedia{fr}{Critère_d'Eisenstein}{Critère d'Eisenstein}]
    Soit le polynôme \( P=\sum_{k=0}^n a_nX^n\) dans \( \eZ[X]\). Nous supposons avoir un nombre premier \( p\) tel que
    \begin{enumerate}
        \item
            \( p\) divise tous les \( a_0,\ldots, a_{n-1}\),
        \item
            \( p\) ne divise pas \( a_n\),
        \item
            \( p^2\) ne divise pas \( a_0\).
    \end{enumerate}
    Alors \( P\) est irréductible dans \( \eQ[X]\).

    Si de plus \( P\) est primitif au sens du \( \pgcd\) (définition~\ref{DEFooAIYGooRAEfHU}) alors \( P\) est irréductible dans \( \eZ[X]\).
\end{proposition}

\begin{proof}
    Nous considérons \( \bar P\) le polynôme réduit modulo \( p\), c'est-à-dire \( \bar P\in \eF_p[X]\). Étant donné que par hypothèse tous les coefficients sont multiples de \( p\) sauf \( a_n\), nous avons \( \bar P=cX^n\). Supposons par l'absurde que \( P=QR\) avec \( Q,R\in \eQ[X]\). Alors le lemme de Gauss (\ref{LemSdnZNX}) impose \( P,Q\in \eZ[X]\).

    Nous avons aussi, au niveau des réductions modulo \( p\) que $\bar Q\bar R=\bar P$. Or \( \bar P\) est un monôme, donc \( \bar Q\) et \( \bar R\) doivent également l'être. Donc \( \bar Q=dX^k\) et \( \bar R=eX^{n-k}\) et en particulier \( \bar Q(0)=\bar R(0)=0\), c'est-à-dire que \( Q(0)\) et \( R(0)\) sont divisibles par \( p\). Cela impliquerait que \( a_0=Q(0)R(0)\) soit divisible par \( p^2\), ce qui est exclu par les hypothèses. Donc \( P\) est irréductible.

    Supposons de surcroît que \( P\) est primitif au sens du \( \pgcd\). Il est donc irréductible et primitif sur \( \eQ[X]\) et le corolaire~\ref{CORooZCSOooHQVAOV} nous dit alors que \( P\) est irréductible sur \( \eZ[X]\).
\end{proof}

\begin{example}
    Soit le polynôme \( P=3X^4+15 X^2+10\). Pour faire fonctionner le critère d'Eisenstein il nous faut un nombre premier \( p\) divisant \( 15\) et \( 10\), mais pas \( 3\) et dont le carré ne divise pas \( 10\). C'est vite vu que \( p=5\) fait l'affaire. Le polynôme \( P\) est donc irréductible sur \( \eQ[X]\).
\end{example}

%---------------------------------------------------------------------------------------------------------------------------
\subsection{Idéaux}
%---------------------------------------------------------------------------------------------------------------------------

Soit \( P\in \eK[X]\) un polynôme. Nous notons \( (P)\) l'idéal engendré par \( P\) :
\begin{equation}        \label{EqDefxMkDtW}
    (P)=\{ PR\tq R\in\eK[X] \}.
\end{equation}

\begin{lemma}
    Nous avons
    \begin{enumerate}
        \item
            \( (P)\subset (Q)\) si et seulement si \( Q\) divise \( P\),
        \item
            \( (P)=(Q)\) si et seulement si \( P\) et \( Q\) sont multiples (non nuls) l'un de l'autre.
    \end{enumerate}
\end{lemma}

\begin{proof}
    Si \( (P)\subset (Q)\), en particulier \( P\in(Q)\) et il existe \( R\in\eK[X]\) tel que \( P=QR\), ce qui signifie que \( Q\) divise \( P\).

    Si les idéaux de \( P\) et de \( Q\) sont identiques, l'un divise l'autre et l'autre divise l'un. Ils sont donc multiples l'un de l'autre.
\end{proof}

\begin{theorem}     \label{ThoCCHkoU}
    Soit \( \eK\) un corps commutatif.
    \begin{enumerate}
        \item       \label{ITEMooLZWMooDRsRwW}
            L'anneau \( \eK[X]\) est euclidien et principal.
        \item
            Si \( I\) est un idéal dans \( \eK[X]\) et si \( P \in I\) est de degré minimal, alors \( (P)=I\).
        \item   \label{ITEMooASHKooZqkiCH}
            De plus si \( I\neq \{  0\}\), il existe un unique polynôme unitaire \( \mu\) tel que \( I=(\mu)\).
    \end{enumerate}
\end{theorem}

\begin{proof}
    Le point~\ref{ITEMooLZWMooDRsRwW} a déjà été démontré dans le lemme~\ref{LEMooIDSKooQfkeKp} via le fait que \( \eK[X]\) est euclidien. Nous allons cependant donner ici une preuve directe que tous les idéaux de \( \eK[X]\) sont principaux. Si \( I=\{ 0 \}\), le résultat est évident. Nous supposons donc \( I\) non nul. Soit \( P\) de degré minimum parmi les éléments de \( I\). Évidemment \( (P)\subset I\). Nous allons démontrer qu'en réalité \( (P)=I\).

    Soit \( P'\in I\). Par le théorème~\ref{ThodivEuclPsFexf} de la division euclidienne\footnote{Ici \( \eK\) est un corps et donc l'hypothèse d'inversibilité est automatiquement vérifiée.}, il existe \( Q\) et \( R\) dans \( \eK[X]\) tels que \( P'=PQ+R\) avec \( \deg(R)<\deg(P)\). Étant donné que \( R=P'-PQ\) nous avons \( R\in I\) et par conséquent \( R=0\) parce que \( P\) a été choisi de degré minimum dans \( I\). Nous avons donc \( P'=PQ\) et \( I\subset (P)\).

    L'existence d'un polynôme unitaire qui génère \( I\) est obtenu en choisissant \( \mu =P/a_n\) où \( a_n\) est le coefficient du terme de plus haut degré. L'unicité d'un tel polynôme est obtenu par le fait que si \( \mu \) et \( \mu' \) génèrent le même idéal, alors ils sont multiples l'un de l'autre, or puisqu'ils sont unitaires, ils sont égaux.
\end{proof}
Nous voyons que n'importe quel polynôme de degré minimum dans un idéal génère l'idéal. Une importante conséquence du théorème~\ref{ThoCCHkoU} que nous verrons plus bas est que tout polynôme annulateur d'un endomorphisme est divisé par le polynôme minimal (proposition~\ref{PropAnnncEcCxj}).

\begin{corollary}       \label{CorsLGiEN}
    Si \( \eK\) est un corps et si \( P\) est un polynôme irréductible, alors l'ensemble \( \eL=\eK[X]/(P)\) est un corps. De plus \( \eL\) est un espace vectoriel de dimension \( \deg(P)\).
\end{corollary}

\begin{proof}
    En effet \( \eK[X]\) est un anneau principal par le théorème~\ref{ThoCCHkoU}, par conséquent la proposition~\ref{PropoTMMXCx}\ref{ITEMooKPJQooWuPZXS} déduit que \( \eK[X]/(P)\) est un corps.

    Une base de \( \eL\) est donnée par les projections de \( 1,X,X^2,\ldots, X^{n-1}\). En effet ces éléments forment une famille libre parce que si \( \sum_{k=0}^{n-1}a_k\bar X^n=0\) alors un représentant de cette classe doit être de la forme \( SP\) dans \( \eK[X]\), c'est-à-dire
    \begin{equation}
        \sum_{k=0}^{n-1}a_kX^k=SP,
    \end{equation}
    ce qui n'est possible que si \( S=0\) et \( a_k=0\). D'autre part c'est un système générateur. En effet si \( P=X^n+Q\) avec \( \deg(Q)=n-1\) alors
    \begin{equation}
        \bar X^{n+l}=\bar X^n\bar X^l=(\bar P-\bar Q)\bar X^l=\bar Q\bar X^l.
    \end{equation}
    Nous avons donc exprimé \( \bar X^{n+l}\) comme une somme de termes de degré \( n+l-1\). Par récurrence nous pouvons exprimer tout \( \bar X^{n+l}\) comme combinaison d'éléments de degré plus petit que \( n\).
\end{proof}

\begin{normaltext}
    Ce corolaire prendra une nouvelle jeunesse lorsque nous parlerons de polynômes d'endomorphismes, en particulier la proposition~\ref{PropooCFZDooROVlaA} va donner des précisions.
\end{normaltext}

\begin{lemma}[\cite{ooUHHUooONXDDl}]        \label{LEMooGRIMooPxCXAZ}
    Soit un isomorphisme de corps \( \tau\colon \eK\to \eK'\). Alors
    \begin{enumerate}
        \item
            L'application étendue
            \begin{equation}
                \begin{aligned}
                    \tau\colon \eK[X]&\to \eK'[X] \\
                    \sum_ia_iX^i&\mapsto \sum_i\tau(a_i)X^i
                \end{aligned}
            \end{equation}
            est un isomorphisme d'anneaux;
        \item
            pour tout \( P\in \eK[X]\), le passage au quotient
            \begin{equation}
                \begin{aligned}
                    \phi_{\tau}\colon \eK[X]/(P)&\to \eK'[X]/\big( \tau(P) \big) \\
                    \bar Q&\mapsto \overline{ \tau(Q) }
                \end{aligned}
            \end{equation}
            est un isomorphisme d'anneaux (et d'abord est bien définie).
    \end{enumerate}
\end{lemma}

\begin{proof}
    Nous n'allons pas faire explicitement toutes les vérifications, mais tout de même les principales. Montrons que \( \tau\) respecte le produit entre \( \eK[X]\) et \( \eK'[X]\). Nous rappelons que ce produit est défini par a formule \eqref{EQooTNCSooKklisb}. En notant \( P_i\) les coefficients de \( P\) et \( Q_i\) ceux de \( Q\) et en remarquant que la définition de \( \tau\) est essentiellement que \( \tau(P)_i=\tau(P_i)\), nous avons :
    \begin{subequations}
        \begin{align}
            \tau(PQ)&=\tau\Bigl( \sum_k\bigl(\sum_{l=0}^kP_lQ_{k-l}\bigr)X^k \Bigr)\\
            &=\sum_k X^k \sum_{l=0}^k\tau(P_lQ_{k-l})\\
            &=\sum_k X^k \sum_{l=0}^k\tau(P_l)\tau(Q_{k-l})\\
            &=\sum_k X^k \sum_{l=0}^k\tau(P)_l\tau(Q)_{k-l}\\
            &=\sum_{i} \bigl(\tau(P)_iX^i\bigr)\sum_j\bigl(\tau(Q)_j X^j\bigr)\\
            &=\tau(P)\tau(Q).
        \end{align}
    \end{subequations}

    Passons à l'isomorphisme d'anneaux donné par \( \phi_{\tau}\).
    \begin{subproof}
        \item[Bien définie]

            Si \( \bar Q_1=\bar Q_2\) alors \( Q_2=Q_1+RP\) pour un certain \( R\in \eK[X]\). Dans ce cas,
            \begin{subequations}
                \begin{align}
                    \phi_{\tau}(Q_2)=\overline{ \tau(Q_2) }&=\overline{ \tau(Q_1)+\tau(RP) }\\
                    &=\overline{ \tau(Q_1)+\tau(R)\tau(P) }\\
                    &=\overline{ \tau(Q_1) }.
                \end{align}
            \end{subequations}
            Ok  pour bien définie.

        \item[Injection]

            Si \( \phi_{\tau}(\bar Q_1)=\phi_{\tau}(\bar Q_2)\) alors \( \overline{ \tau(Q_1) }=\overline{ \tau(Q_2) }\), ce qui signifie que
            \begin{equation}
                \tau(Q_1)=\tau(Q_2)+R\tau(P)
            \end{equation}
            pour un certain \( R\in \eK'[X]\). Vu que \( \tau\colon \eK[X]\to \eK'[X]\) est un isomorphisme, nous pouvons y appliquer \( \tau^{-1}\) pour trouver :
            \begin{equation}
                Q_1=Q_2+\tau^{-1}(R)P,
            \end{equation}
            ce qui signifie que \( \bar Q_1=\bar Q_2\).

        \item[Surjection]

            Un élément de \( \eK'[X]/\big( \tau(P) \big)\) est de la forme \( \bar Q\) avec \( Q\in \eK'[X]\). Cela est l'image par \( \phi_{\tau}\) de l'élément \( \overline{ \tau^{-1}(Q) }\in \eK[X]/(P)\).

        \item[Morphisme]

            Nous vous laissons vérifier que l'application \( \phi_{\tau}\) est un morphisme d'anneaux.

    \end{subproof}
\end{proof}

%---------------------------------------------------------------------------------------------------------------------------
\subsection{Bézout}
%---------------------------------------------------------------------------------------------------------------------------

\begin{theorem}[Bézout] \label{ThoBezoutOuGmLB}
    Les polynômes \( P_1,\ldots,P_n\) dans \( \eK[X]\) sont étrangers entre eux si et seulement s'il existe des polynômes \( Q_1,\ldots,Q_n\in\eK[X]\) tels que
    \begin{equation}
        P_1Q_1+\cdots+P_nQ_n=1.
    \end{equation}
\end{theorem}
\index{Bézout!polynômes}
\index{théorème!Bézout!polynômes}

Deux polynômes \( P\) et \( Q\) ne sont donc pas premiers entre eux s'il existe des polynômes \( x\) et \( y\) tels que l'identité de Bézout soit vérifiée :
\begin{equation}    \label{EqkbbzAi}
    xP+yQ=0;
\end{equation}
cette dernière pourra être écrite en termes de la matrice de Sylvester, voir sous-section~\ref{subsecSQBJfr}.

\begin{lemma}       \label{LemuALZHn}
    Soient \( (P_i)_{i=1,\ldots,n}\in \eK[X]\) des polynômes étrangers deux à deux. Alors les polynômes \begin{equation} Q_i=\prod_{j\neq i}P_j \end{equation}
    sont étrangers entre eux\footnote{Et non seulement deux à deux.}.
\end{lemma}

\begin{lemma}[\cite{SQxrsoL}]   \label{LemzwkYdn}
    Soit \( \eK\) un corps commutatif et \( \eA\subset \eK\) un sous
    anneau de \( \eK\).  Alors \( \eA[X] \), vu comme idéal de \( \eK[X]
    \), est un idéal premier.

    En d'autres termes, si \( \phi\in \eK[X]\), et s'il existe \( Q\in \eK[X]\) unitaire tel que \( \phi Q\in \eA[X]\), alors \( \phi\in \eA[X]\).
\end{lemma}

%---------------------------------------------------------------------------------------------------------------------------
\subsection{Lemme et théorème de Gauss}
%---------------------------------------------------------------------------------------------------------------------------

\begin{theorem}[Théorème de Gauss]  \label{ThoLLgIsig}
    Soient \( P,Q,R\in \eK[X]\) tels que \( P\) soit premier avec \( Q\) et divise \( QR\). Alors \( P\) divise \( R\).
\end{theorem}
\index{théorème!Gauss!polynômes}

\begin{proof}
    Étant donné que \( P\) est premier avec \( Q\), le théorème de Bézout\footnote{théorème~\ref{ThoBezoutOuGmLB}.} nous donne \( U,V\in \eK[X]\) tels que \( PU+QV=1\). De plus il existe un polynôme \( S\) tel que \( PS=QR\). En multipliant l'identité de Bézout par \( R\), nous obtenons
    \begin{equation}
        R=PUR+QVR=PUR+VPS=P(UR+VS),
    \end{equation}
    ce qui signifie que \( P\) divise \( R\).
\end{proof}

Le lemme suivant est une généralisation du lemme de Gauss dans \( \eZ\) (lemme~\ref{LemSdnZNX}).
\begin{lemma}[Lemme de Gauss\cite{fJhCTE}]       \label{LemEfdkZw}   \index{lemme!Gauss!polynômes}\index{Gauss!lemme!polynômes}
    Soient les polynômes unitaires \( P,Q\in \eQ[X]\). Si \( PQ\in\eZ[X]\), alors \( P\) et \( Q\) sont tous deux dans \( \eZ[X]\).
\end{lemma}

\begin{proof}
    Soit \( a>0\) le plus petit entier tel que \( aP\in\eZ[X]\) (c'est le PPCM des dénominateurs) et de la même façon \( b>0\) le plus petit entier tel que \( bQ\in \eZ[X]\). On pose \( P_1=aP\) et \( Q_1=bQ\).

    Si \( ab=1\), alors \( a=b=1\) et nous avons tout de suite \( P,Q\in \eZ[X]\). Nous supposons donc \( ab>1\) et nous considérons \( p\), un diviseur premier de \( ab\). Ensuite nous considérons la projection
    \begin{equation}
        \pi_p\colon \eZ[X]\to (\eZ/p\eZ)[X].
    \end{equation}
    Par définition \( abPQ=P_1Q_1\in \eZ[X]\); en prenant la projection,
    \begin{equation}
        \pi_p(P_1)\pi_p(Q_1)=\pi_p(P_1Q_1)=\pi_P(ab)\pi_p(PQ)=0
    \end{equation}
    parce que \( \pi_p(ab)=0\). Étant donné que \( (\eZ/p\eZ)[X]\) est intègre (théorème~\ref{ThoBUEDrJ}), nous avons soit \( \pi_p(P_1)=0\) soit \( \pi_p(Q_1)=0\). Supposons pour fixer les idées que \( \pi_p(P_1)=0\). Alors \( P_1=pP_2\) pour un certain \( P_2\in \eZ[X]\). Par ailleurs \( P\) est unitaire et \( P_1=aP\), donc le coefficient de plus haut degré de \( P_1\) est \( a\), et nous concluons que \( p\) divise \( a\).

    Mettons \( a=pa'\). Dans ce cas, \( pa'P=P_1=pP_2\), et donc \( a'P=P_2\in \eZ[X]\). Cela contredit la minimalité de \( a\).
\end{proof}

%---------------------------------------------------------------------------------------------------------------------------
\subsection{Polynômes sur un corps et pgcd}
%---------------------------------------------------------------------------------------------------------------------------

Nous savons qu'un corps est un anneau intègre (lemme~\ref{LemAnnCorpsnonInterdivzer}). De plus l'ensemble des polynômes sur un anneau intègre est lui-même un anneau intègre (théorème~\ref{ThoBUEDrJ}). Donc la notion de pgcd à utiliser dans le cas de \( \eK[X]\) est celle de la définition~\ref{DefrYwbct}.

\begin{lemma}[Unicité du pgcd à inversibles près]      \label{LEMooXISOooNAMeVX}
    Soit un corps commutatif \( \eK\) et \( S\subset \eK[X]\). Si \( \delta_1\) et \( \delta_2\) sont des pgcd\footnote{Définition \ref{DefrYwbct}.} de \( S\), alors \( \delta_1=k\delta_2\) avec \( k\in \eK\).
\end{lemma}

\begin{proof}
    Nous savons que \( \delta_1\) est un pgcd de \( S\), mais que \( \delta_2\) divise \( S\). Donc \( \delta_2\divides \delta_1\). De la même manière, \( \delta_1\divides \delta_2\). Il existe donc \( A,B\in \eK[X]\) tels que \( \delta_1=A\delta_2\) et \( \delta_2=B\delta_1\). En substituant,
    \begin{equation}
        \delta_1=AB\delta_1.
    \end{equation}
    Mais \( \eK[X]\) possède la propriété de simplification par la proposition~\ref{DEFooTAOPooWDPYmd}\ref{ITEMooQNTFooSRrVPK}. Donc \( AB=1\). Cela signifie entre autres que \( A\) et \( B\) sont des inversibles de \( \eK[X]\).

    Or les seuls inversibles dans \( \eK[X]\) sont les éléments de \( \eK\); si vous en doutez, pensez que le degré de \( AB\) est supérieur ou égal à celui de \( A\).
\end{proof}

\begin{normaltext}
    En général, lorsque nous dirons «le» pgcd d'un ensemble, nous parlerons du pgcd unitaire, qui existe et est bien défini par le lemme~\ref{LEMooXISOooNAMeVX}.
\end{normaltext}


\begin{lemma}[\cite{ooSGHGooAJzIjz}]        \label{LEMooIAGMooHUQtUs}
    Soit un corps commutatif \( \eK\), deux polynômes quelconques \( A,B\in \eK[X]\) et un polynôme unitaire \( G\).

    Nous avons \( G=\pgcd(A,B)\) si et seulement si les deux conditions suivantes sont satisfaites :
    \begin{enumerate}
        \item
            Il existe \( U,V\in \eK[X]\) tels que \( AU+BV=G\),
        \item
            \( G\) divise \( A\) et \( B\).
    \end{enumerate}
\end{lemma}

\begin{proof}
    Une implication dans chaque sens.

    \begin{subproof}
        \item[\( \Rightarrow\)]

        Si $G$ est le pgcd de $A$ et $B$, il est clair que $G|A$ et $G|B$.  Il reste donc à montrer l'existence des polynômes $U$ et $V$ vérifiant $AU+BV=G$. Vu que \( G\) divise \( A\) et \( B\), il existe des polynômes $A_1,B_1$ tels que $A=GA_1$ et $B=GB_1$.

        Nous montrons que les polynômes $A_1$ et $B_1$ sont premiers entre eux. S'ils ont un diviseur commun $D$, alors $GD$ est un diviseur commun à $A$ et $B$.  Or, $G$ est le pgcd de $A$ et $B$ donc $GD|G$ ; $D$ ne peut être qu'un polynôme constant (c'est-à-dire un élément de \( \eK\)). Mais comme \( G\) est unitaire, le coefficient du terme de plus haut degré de \( GD\) doit être \( 1\). Donc \( D=1\).  L'élément \( 1\) est l'unique diviseur commun de \( A_1\) et \( B_1\); donc $A_1$ et $B_1$ sont donc bien premiers entre eux.

        D'après le théorème de Bézout~\ref{ThoBezoutOuGmLB}, il existe donc $U$ et $V$ tels que $A_1U+B_1V=1$. En multipliant par $G$, nous obtenons l'égalité voulue : $AU+BV=G$.

        \item[\( \Leftarrow\)]

        Si $G$ vérifie les deux conditions, montrons que $G$ est le pgcd de $A$ et $B$. Nous savons déjà (par hypothèse) que $G$ divise $A$ et $B$, il reste à montrer que tous les diviseurs commun à $A$ et $B$ divisent aussi $G$. Soit donc $D$ un diviseur commun à $A$ et $B$ : il existe $A_1$ et $B_1$ tels que $A=DA_1$ et $B=DB_1$. Nous savons que $G=AU+BV$ donc $G=D(A_1U+B_1V)$, et $D|G$.

        Par définition, $G$ est bien le pgcd de $A$ et $B$.
        \end{subproof}
\end{proof}
Notons qu'en supprimant la condition d'unitarité de \( G\), le résultat tient presque : il suffit de remplacer partout «le pgcd» par «un pgcd».

\begin{lemma}[\cite{ooSGHGooAJzIjz}]       \label{LEMooGNAMooXRpgBn}
Soient deux polynômes $A,B$ premiers entre eux. Si le polynôme \( P\) est divisible par $A$ et par $B$ alors $P$ est divisible par $AB$.
\end{lemma}

\begin{proof}
    Vu que \( A\divides P\), il existe \( Q_1\in \eK[X]\) tel que \( P=AQ_1\). Mais \( B\) divise \( P=AQ_1\) alors que \( B\) est premier avec \( A\); donc d'après le théorème de Gauss~\ref{ThoLLgIsig} : $B|Q_1$.

    Il existe donc $Q_2\in \eK[X]$ tel que $Q_1=BQ_2$. On a donc $P=ABQ_2$ : $P$ est bien divisible par $AB$.
\end{proof}

% TODO : voir qui référentie ce lemme et mettre la référence vers le point correct.
\begin{lemma}[\cite{ooSGHGooAJzIjz}]   \label{LemUELTuwK}
    Quelques propriétés du PGCD\footnote{Définition \ref{DefrYwbct}.} dans les polynômes. Soient des polynômes \( P,Q,R\in \eK[X]\).
    \begin{enumerate}
        \item       \label{ITEMooBPOZooYeFGjl}
            Nous avons l'égalité\footnote{Notez l'analogie avec le lemme~\ref{LemiVqita}.}
            \begin{equation}
                \pgcd(P,PQ+R)=\pgcd(P,R).
            \end{equation}
        \item       \label{ITEMooUVGRooNSGDZn}
            Si \( Q \) et \( R\) sont premiers entre eux,
            \begin{equation}
                \pgcd(P,QR)=\pgcd(P,Q)\pgcd(P,R)
            \end{equation}
        \item       \label{ITEMooYXAHooXibkgV}
            Si \( P\) et \( Q\) sont premiers entre eux,
            \begin{equation}
                \pgcd(P,QR)=\pgcd(P,R)
            \end{equation}
    \end{enumerate}
\end{lemma}
\index{pgcd!polynômes}

\begin{proof}
    Dans la suite si \( A\) et \( B\) sont des polynômes, nous dirons «les diviseurs de \( \{ A,B \}\)» pour parler des diviseurs communs de \( A\) et \( B\).

    \begin{enumerate}
        \item[\ref{ITEMooBPOZooYeFGjl}]

    Nous montrons que \( \{ P,PQ+R \}\) a les mêmes diviseurs que \( \{ P,R \}\).

    D'une part, si \( A\divides\{ P,PQ+R \}\), alors il existe des polynômes \( B_1\) et \( B_2\) tels que \( P=AB_1\) et \( PQ+R=AB_2\). Donc
            \begin{equation}
                R=AB_2-PQ=AB_2-AB_1Q=A(B_2-B_1Q),
            \end{equation}
            et nous concluons que \( A\) divise \( R\).

            D'autre part, si \( A\divides\{ P,R \}\) alors il existe des polynômes \( B_1\) et \( B_2\) tels que \( P=AB_1\) et \( R=AB_2\). Donc
            \begin{equation}
                PQ+R=AB_1Q+AB_2=A(B_1Q+B_2),
            \end{equation}
            et \( A\) divise \( PQ+R\).

            Conclusion : les paires \( \{ P,PQ+R \}\) et \( \{ P,R \}\) ont même ensemble de diviseurs, et donc même \( \pgcd\).

        \item[\ref{ITEMooUVGRooNSGDZn}]

            Nous avons trois polynômes $P,Q,R$ et nous savons que $Q$ et $R$ sont premiers entre eux. Nous notons : $G_1=\pgcd(P,Q)$ et $G_2=\pgcd(P,R)$.  Il faut montrer que $G_1G_2$ est le pgcd de $P$ et $QR$; pour cela nous allons utiliser le lemme~\ref{LEMooIAGMooHUQtUs}.

            \begin{subproof}
                \item[\( \exists U,V\) tels que $G_1G_2=PU+QRV$ ]

                    Vu que $G_1=\pgcd(P,Q)$, il existe $U_1$ et $V_1$ tels que $G_1=PU_1+QV_1$ (lemme~\ref{LEMooIAGMooHUQtUs}).
On a de même : $G_2=PU_2+RV_2$. En prenant le produit :
                    \begin{equation}
                        G_1G_2=(PU_1+QV_1)(PU_2+RV_2)=P(PU_1U_2+RU_1V_2+QV_1V_2)+QR(V_1V_2).
                    \end{equation}
                    Donc c'est bon pour ce point.

                \item[\( G_1\) et \( G_2\) sont premiers entre eux]

                    Si $D$ est un diviseur commun à $G_1$ et $G_2$, alors $D$ divise $Q$ et $R$ qui sont premiers entre eux ; $D$ ne peut être qu'un polynôme constant. Tous les diviseurs communs de \( G_1\) et \( G_2\) sont dans \( \eK\). Mais le \( \pgcd\) est par définition un diviseur commun unitaire, donc \( \pgcd(G_1,G_2)=1\). Cela signifie que \( G_1\) et \( G_2\) sont premiers entre eux (définition~\ref{DefZHRXooNeWIcB}).

                \item[\( G_1G_2\divides QR\)]
                    En effet : $G_1|Q$ et $G_2|R$ donc $G_1G_2|QR$.
                \item[\( G_1G_2\divides P\)]
                    Le polynôme $P$ est divisible par $G_1$ et par $G_2$, et de plus $G_1$ et $G_2$ sont premiers entre eux. Donc le lemme~\ref{LEMooGNAMooXRpgBn} conclu que \( P\) est divisible par \( G_1G_2\).

            \end{subproof}

\item[\ref{ITEMooYXAHooXibkgV}]

Supposons d'abord que \( A\in \eK[X]\) divise \( P\) et \( QR\). Le théorème de Bézout~\ref{ThoBezoutOuGmLB} assure l'existence de polynômes $U$ et $V$ tels que $PU+QV=1$. Ensuite l'hypothèse de division nous donne des polynômes \( B_1\) et \( B_2\) tels que $P=AB_1$ et $QR=AB_2$.  Nous avons :
            \begin{equation}
                    1=PU+QV=AB_1U+QV.
            \end{equation}
            Cela prouve que \( A\) est premier avec $Q$ grâce encore à Bézout, mais dans l'autre sens. Donc \( A\) est premier avec \( Q\) et \( A\divides QR\). Donc \( A|R\) par le théorème de Gauss~\ref{ThoLLgIsig}.

    Dans l'autre sens, si $A|R$ alors on a évidemment : $A|QR$.

    Les diviseurs de $\{P,QR\}$ sont exactement les diviseurs de $\{P,R\}$. En conséquence, nous concluons que les paires $\{P,QR\}$ et $\{P,R\}$ ont le même $\pgcd$.

    \end{enumerate}

\end{proof}


% This is part of Mes notes de mathématique
% Copyright (c) 2011-2020
%   Laurent Claessens
% See the file fdl-1.3.txt for copying conditions.

%+++++++++++++++++++++++++++++++++++++++++++++++++++++++++++++++++++++++++++++++++++++++++++++++++++++++++++++++++++++++++++
\section{Extension de corps}
%+++++++++++++++++++++++++++++++++++++++++++++++++++++++++++++++++++++++++++++++++++++++++++++++++++++++++++++++++++++++++++
\label{SECooLQVJooTGeqiR}

\begin{lemma}       \label{LemobATFP}
    Soit \( \eL\) un corps\footnote{Définition \ref{DefTMNooKXHUd}.} fini et \( \eK\) un sous corps de \( \eL\). Alors il existe \( s\in \eN\) tel que
    \begin{equation}        \label{EqUgqlJQ}
        \Card(\eL)=\Card(\eK)^s.
    \end{equation}
\end{lemma}

\begin{proof}
    Le corps \( \eL\) est un \( \eK\)-espace vectoriel de dimension finie. Si \( s\) est la dimension alors nous avons la formule \eqref{EqUgqlJQ} parce que chaque élément de \( \eL\) est un \( s\)-uple d'éléments de \( \eK\).
\end{proof}

\begin{definition}[\cite{ooOXISooAFtXsZ}]     \label{DEFooFLJJooGJYDOe}
    Soit \( \eK\) un corps commutatif. Une \defe{extension}{extension!de corps} de \( \eK\) est un couple \( (\eL,j)\) où \( \eL\) est un corps et \( j\colon \eK\to \eL\) est un morphisme de corps.
\end{definition}

    Nous identifions le plus souvent \( \eK\) avec \( i(\eK)\subset \eL\), mais il faut savoir que le corps \( \eL\) étendant \( \eK\) n'est pas toujours un sur-corps de \( \eK\).

\begin{lemma}       \label{LemooOLIIooXzdppM}
    Si \( (\eL,i)\) est une extension de \( \eK\), alors \( \eL\) est un espace vectoriel sur \( \eK\).
\end{lemma}

\begin{proof}
    Il faut définir le produit d'un élément de \( \eL\) par un élément de \( \eK\); si \( \lambda\in \eK\) et \( x\in \eL\) nous la définissons par
    \begin{equation}
        \lambda\cdot x=i(\lambda)x
    \end{equation}
    où la multiplication du membre de droite est celle du corps \( \eL\).
\end{proof}

\begin{definition}      \label{DefUYiyieu}
    Le \defe{degré}{degré!extension de corps} de \( \eL\) est la dimension de cet espace vectoriel. Il est noté \( [\eL:\eK]\)\nomenclature[A]{$ [\eL:\eK]$}{degré d'une extension de corps}; notons qu'il peut être infini.
\end{definition}

\begin{example}
    L'ensemble \( \eC\) est une extension de \( \eR\) et son degré est \( [\eC:\eR]=2\).
\end{example}

\begin{proposition}[Composition des degrés\cite{ooGIIFooVMVloY}]        \label{PROPooEGSJooBSocTf} \label{PropGWazMpY}
    Si \( \eL_2\) est une extension de \( \eL_1\) qui est elle-même une extension de \( \eK\), alors \( \eL_2\) est une extension de \( \eK\) et on a :
    \begin{equation}        \label{EQooOLLQooFdYtnh}
        [\eL_2:\eK]=[\eL_2:\eL_1][\eL_1:\eK].
    \end{equation}
    Dans ce cas, si \( \{ v_i \}_{i\in I}\) est une \( \eK\)-base de \( \eL_1\) et si \( \{ w_{\alpha} \}_{\alpha\in A}\) est une \( \eL_1\)-base de \( \eL_2\) alors \( \{ v_iw_{\alpha} \}_{\substack{i\in I\\\alpha\in A}}\) est une \( \eK\)-base de \( \eL_2\).
\end{proposition}

Notons que la formule \eqref{EQooOLLQooFdYtnh} n'est pas très instructive dans le cas des extensions non finies. La seconde partie, sur les bases, est en réalité nettement plus intéressante.

\begin{proof}
    Soit \( a\in \eL_2\). Vu que les \( w_{\alpha}\) forment une \( \eL_2\)-base nous avons une décomposition
    \begin{equation}
        a=\sum_{\alpha}a_{\alpha}w_{\alpha}
    \end{equation}
    pour des éléments \( a_{\alpha}\in \eL_1\). Mais les \( v_i\) forment une \( \eK\)-base de \( \eL_1\), donc chacun des \( a_{\alpha}\) peut être décomposé comme \( a_{\alpha}=\sum_ia_{\alpha i}v_iw_{\alpha}\). Donc :
    \begin{equation}
        a=\sum_{\alpha i}a_{\alpha i}v_iw_{\alpha},
    \end{equation}
    qui donne une décomposition de \( a\) en éléments de \( \{ v_iw_{\alpha} \}\) à coefficients dans \( \eK\). La partie proposée est donc génératrice.

    Pour prouver qu'elle est également libre, nous supposons avoir des éléments \( a_{\alpha i}\in \eK\) tels que
    \begin{equation}
        \sum_{\alpha i}a_{\alpha i}v_iw_{\alpha}=0.
    \end{equation}
    En récrivant sous la forme
    \begin{equation}
        \sum_{\alpha}\Big( \sum_ia_{\alpha i}v_i \Big)w_{\alpha}=0,
    \end{equation}
    nous reconnaissons une combinaison linéaire nulle des \( w_{\alpha}\) à coefficients dans \( \eL_1\). Les coefficients sont donc nuls : \( \sum_i a_{\alpha i}v_i=0\). Cela est une combinaison linéaire nulle des \( v_i\) à coefficients dans \( \eK\). Vu que les \( v_i\) forment une base, les coefficients sont nuls : \( a_{\alpha i}=0\).
\end{proof}

%---------------------------------------------------------------------------------------------------------------------------
\subsection{Extension et polynôme minimal}
%---------------------------------------------------------------------------------------------------------------------------

\begin{lemmaDef}[Polynôme minimal]    \label{DefCVMooFGSAgL}
    Soit \( \eL\) une extension de \( \eK\) et \( a\in \eL\). Nous considérons la partie
    \begin{equation}
        I_a=\{ P\in \eK[X]\tq P(a)=0 \}
    \end{equation}
    que nous supposons non réduite à \( \{ 0 \}\)\footnote{La non trivialité de \( I_a\) est une vraie hypothèse. En effet si nous prenons \( \eK=\eQ\) et l'extension \( \eL=\eR\), alors il suffit de prendre un réel \( a\) non algébrique sur \( \eQ\) pour que \( I_a\) soit réduit au seul polynôme identiquement nul.}

    \begin{enumerate}
        \item       \label{ITEMooUNLCooIfYZry}
            La partie \( I_a\) est un idéal dans \( \eK[X]\),
        \item       \label{ITEMooDCDRooPDnnbu}
            la partie \( I_a\) est un idéal principal dans \( \eK[X]\),
        \item       \label{ITEMooXFYQooTuMzIu}
            l'idéal \( I_a\) possède un unique générateur unitaire.
    \end{enumerate}

    Cet unique générateur unitaire est le \defe{polynôme minimal}{polynôme!minimal!d'un élément d'une extension} de \( a\) sur \( \eK\).
\end{lemmaDef}

\begin{proof}
    En plusieurs parties.
    \begin{subproof}
    \item[Pour \ref{ITEMooUNLCooIfYZry}]
        Soit \( P\in I_a\) : \( P(a)=0\). Si \( Q\in \eK[X]\) alors la proposition \ref{PROPooGDQCooHziCPH} nous indique que
        \begin{equation}
            (PQ)(a)=P(a)Q(a)=0.
        \end{equation}
        Donc \( PQ\in I_a\). Comme de plus \( I_a\) est clairement vectoriel, \( I_a\) est un idéal.

        Notez que nous avons utilisé la règle du produit nul justifiée par le fait que \( \eK\) soit un corps\quext{Si vous connaissez un contre-exemple à cette proposition dans le cas où \( \eK\) serait remplacé par un anneau, écrivez-moi.} et donc soumis au point \ref{ITEMooQNTFooSRrVPK} de la proposition \ref{DEFooTAOPooWDPYmd}.
    \item[Pour \ref{ITEMooDCDRooPDnnbu}]
        Nous savons par le théorème \ref{ThoCCHkoU} que \( \eK[X]\) est un anneau principal. En particulier, tous ses idéaux sont principaux, c'est dans la définition \ref{DEFooGWOZooXzUlhK} d'un anneau principal.
    \item[Pour \ref{ITEMooXFYQooTuMzIu}]
        Le théorème~\ref{ThoCCHkoU}\ref{ITEMooASHKooZqkiCH} nous informe alors que \( I_a\) possède un unique générateur unitaire.
    \end{subproof}
\end{proof}

Si nous avons un corps et un élément dans une extension du corps, il n'est pas autorisé de dire «soit le polynôme minimal de cet élément dans le premier corps» parce qu'il n'existe peut-être pas de polynôme annulateur.

\begin{example}
    Le polynôme minimal dépend du corps sur lequel on le considère. Par exemple le nombre imaginaire pur \( i\) accepte \( X-i\) comme polynôme minimal sur \( \eC\) et \( X^2+1\) sur \( \eQ[X]\).
\end{example}

\begin{proposition}[\cite{MonCerveau}]  \label{PropRARooKavaIT}
    Soit \( \eL\) une extension de \( \eK\) et \( a\in \eL\) dont le polynôme minimal sur \( \eK\) est \( \mu_a\in\eK[X]\). Alors
    \begin{enumerate}
        \item   \label{ItemDOQooYpLvXri}
            le polynôme \( \mu_a\) est irréductible\footnote{Définition~\ref{DefIrredfIqydS}.} sur \( \eK\);
        \item
            Le polynôme \( \mu_a\) est premier\footnote{Définition~\ref{DefDSFooZVbNAX}.} avec tout polynôme de \( \eK[X]\) non annulateur de \( a\).
    \end{enumerate}
\end{proposition}

\begin{proof}
    Une chose à la fois.
    \begin{enumerate}
        \item
            D'abord le polynôme \( \mu_a\) n'est pas inversible parce que seuls les éléments de \( \eK\) (ceux de degré zéro) peuvent être inversibles\footnote{Et d'ailleurs, le sont, mais ce n'est pas important ici.}. Mais ces polynômes sont constants et ne peuvent donc pas être des polynômes annulateurs de quoi que ce soit.

            Ensuite, supposons la décomposition \( \mu_a=PQ\) avec \( P,Q\in \eK[X]\). En évaluant cette égalité en \( a\) nous avons
            \begin{equation}
                0=P(a)Q(a).
            \end{equation}
            Vu que nous sommes sur un corps, nous avons la règle du produit nul\footnote{Parce que un corps est un anneau intègre par le lemme \ref{LemAnnCorpsnonInterdivzer} et qu'un anneau intègre est justement un anneau sur lequel nous avons la règle du produit nul, voir la définition \ref{DEFooTAOPooWDPYmd}.} et nous déduisons que soit \( P(a)\) soit \( Q(a)\) est nul, ou les deux. Pour fixer les idées, nous supposons \( P(a)=0\).

            Dans ce cas, \( P\) fait partie de l'idéal annulateur de \( a\), lequel idéal est engendré par \( \mu_a\). Donc il existe \( S\in \eK[X]\) tel que \( P=S\mu_a\). En récrivant \( \mu_a=PQ\) avec cela nous avons :
            \begin{equation}
                \mu_a=S\mu_aQ
            \end{equation}
            ou encore : \( SQ=1\), ce qui signifie que \( S\) et \( Q\) sont dans \( \eK\) et inversibles.

            Nous concluons que \( \mu_a\) ne peut pas être écrit sous forme de produit de deux non inversibles.
        \item
            Soit \( Q\) un polynôme non annulateur de \( a\). Soit aussi un diviseur commun \( P\) de \( Q\) et \( \mu_a\) dans \( \eK[X]\). Nous devons prouver que \( P\) est un inversible, c'est-à-dire un élément de \( \eK\) (le fait que \( P\) ne soit pas le polynôme nul est évident).
            Nous avons \( \mu_a=PR_1\) et \( Q=PR_2\) pour certains polynômes \( R_1,R_2\in \eK[X]\). Vu que \( \mu_a\) est irréductible par~\ref{ItemDOQooYpLvXri}, il n'est pas produit de deux non inversibles. En d'autres termes, soit \( P\) soit \( R_1\) est inversible. Si \( P \) n'est pas inversible, alors \( R_1\) est inversible; disons \( R_1=k\in \eK\). Alors
            \begin{equation}
                0=\mu_a(a)=P(a)k,
            \end{equation}
            donc \( P(a)=0\). Mais alors
            \begin{equation}
                Q(a)=P(a)R_2(a)=0,
            \end{equation}
            ce qui est contraire à l'hypothèse selon laquelle \( Q\) n'était pas annulateur de \( a\).

            Nous retenons donc que \( P\) est inversible, ce qu'il fallait montrer.
    \end{enumerate}
\end{proof}

\begin{definition}
    Deux éléments \( \alpha\) et \( \beta\) dans \( \eL\) sont dit \defe{conjugués}{conjugués!éléments d'une extension} s'ils ont même polynôme minimal. Par exemple \( i\) et \( -i\) sont conjugués dans \( \eC\) vu comme extension de \( \eQ\).
\end{definition}

%---------------------------------------------------------------------------------------------------------------------------
\subsection{Extensions algébriques et éléments transcendants}
%---------------------------------------------------------------------------------------------------------------------------

%///////////////////////////////////////////////////////////////////////////////////////////////////////////////////////////
\subsubsection{Éléments algébriques et transcendants}
%///////////////////////////////////////////////////////////////////////////////////////////////////////////////////////////


%TODOooAAOWooWNqqVO. Dans LEMooLVPLooEkWYDN, LEMooTZSSooZmwYji, et DEFooREUHooLVwRuw il faut modifier les énoncés pour être explicite que
% une extension L de K est bien une application i:K->L; nous n'avons pas spécialement que K est un sous-ensemble de L.
% Ça ajoute des niveaux d'indirections, mais on est névrosé des abus de notations et on assume.

\begin{definition}      \label{DEFooBBYGooWoOloR}
    L'ensemble \( A[X]\) devient un \( \eK\)-espace vectoriel avec la définition
    \begin{equation}
        (\lambda P)_k=\lambda P_k.
    \end{equation}
\end{definition}

Voici une définition d'un élément algébrique sur un corps. Une caractérisation plus «pratique» sera donnée dans le lemme \ref{LEMooTZSSooZmwYji}.
\begin{lemmaDef}[Élément algébrique et transcendant\cite{ooTGTKooFenWAc}] \label{LEMooLVPLooEkWYDN}
    Soit une extension \( \eL\) de \( \eK\) et \( \alpha\in \eL\). Nous considérons l'application
    \begin{equation}
        \begin{aligned}
            \varphi\colon \eK[X]&\to \eL \\
            P&\mapsto P(\alpha).
        \end{aligned}
    \end{equation}
    Alors
    \begin{enumerate}
        \item
            L'application \( \varphi\) est un morphisme d'anneaux\footnote{Définition \ref{DEFooSPHPooCwjzuz}.}.
        \item
            L'application \( \varphi\) est un morphisme de \( \eK\)-espace vectoriel.
    \end{enumerate}
    Si \( \varphi\) est injective, nous disons que \( \alpha\) est \defe{transcendant}{transcendant}. Sinon, nous disons qu'il est \defe{algébrique}{algébrique}.
\end{lemmaDef}

\begin{proof}
    Le fait que \( \varphi\) soit un morphisme d'anneaux est le lemme \ref{PROPooGDQCooHziCPH} déjà prouvé.

    Pour le morphisme de \( \eK\)-espace vectoriel, il faut seulement ajouter le calcul
    \begin{equation}
        \varphi(\lambda P)=(\lambda P)(\alpha)=\lambda P(\alpha)=\lambda \varphi(P).
    \end{equation}
    Notons la justification suivante qui n'est pas tout à fait triviale :
    \begin{equation}
        (\lambda P)(\alpha)=\sum_k(\lambda P)_k\alpha^k=\sum_k\lambda P_k\alpha^k=\lambda P(\alpha)
    \end{equation}
    qui utilise la définition \ref{DEFooBBYGooWoOloR}.
\end{proof}

\begin{example}
    L'injectivité de \( \varphi\) n'est pas automatique. Prenons par exemple \( \eL=\eQ[\sqrt{ 2 }]\) dans \( \eR\). Les polynômes dans \( \eQ[X]\) ont des degrés arbitrairement élevés en \( X\), tandis que les éléments de \( \eL\) n'ont pas de degré très élevés en \( \sqrt{ 2 }\) parce que \( \sqrt{ 2 }\sqrt{ 2 }=2\). L'ensemble \( \eQ[\sqrt{ 2 }]\) ne contient donc que des éléments de la forme \( a+b\sqrt{ 2 }\) avec \( a,b\in \eQ\).

    Si par contre \( x_0\in \eR\) n'est racine d'aucun polynôme (cela existe parce que \( \eR\) n'est pas dénombrable), alors \( \eQ[x_0]\) contient tous les \( \sum_{k=0}^Na_kx_0^k\) avec \( N\) arbitrairement grand. Et tous ces nombres sont différents.
\end{example}

Le lemme suivant donne une caractérisation d'élément algébrique moins abstraire que la définition \ref{LEMooLVPLooEkWYDN}.
\begin{lemma}       \label{LEMooTZSSooZmwYji}
    Soit \( \eK\), un corps et \( \eL\), une extension de \( \eK\). Un élément \( \alpha\in \eL\) est algébrique sur \( \eK\) si et seulement si existe un polynôme non nul \( P\in \eK[X]\) tel que \( P(a)=0\).
\end{lemma}

\begin{proof}
    Nous considérons l'application \( \varphi\) de la définition \ref{LEMooLVPLooEkWYDN}. Si \( \varphi\) n'est pas injective, c'est qu'il existe un polynôme \( P\) dans \( \eK[X]\) tel que \( \varphi(P)=0\). Dans ce cas, \( P(\alpha)=0\).

    À l'inverse si il existe \( P\) non nul dans \( \eK[X]\) tel que \( P(\alpha)=0\), alors \( \varphi(P)=0\) et \( \varphi\) n'est pas injective.
\end{proof}

\begin{definition}[Extension algébrique, clôture algébrique]      \label{DEFooREUHooLVwRuw}
    Soient un corps \( \eK\) et une extension \( \eL\).
    \begin{enumerate}
        \item
            L'extension \( \eL\) est une extension \defe{algébrique}{extension!de corps!algébrique}\index{algébrique!extension} de \( \eK\) si tous ses éléments sont algébriques\footnote{Définition \ref{LEMooLVPLooEkWYDN}.} sur \( \eK\), c'est à dire sont racines de polynômes sur \( \eK\), voir le lemme \ref{LEMooTZSSooZmwYji}.
        \item       \label{ITEMooEIWVooVjJRoR}
            L'extension \( \eL\) est \defe{algébriquement close}{algébriquement clos} si tout polynôme non-constant à coefficients dans \( \eK\) admet des racines dans \( \eL \).
        \item
            L'extension \( \eL\) est une \defe{clôture algébrique}{clôture algébrique} du corps \( \eK\) si elle est une extension algébrique qui est algébriquement close. 
    \end{enumerate}
\end{definition}

\begin{normaltext}
    Donc une extension est algébrique si elle contient seulement des racines de polynômes; elle est close si elle contient au moins une racine de chaque polynôme. Elle est une clôture algébrique si elle est les deux en même temps.
\end{normaltext}

\begin{example}
    Le corps \( \eR\) n'est pas une extension algébrique de \( \eQ\). En effet il existe seulement une infinité \emph{dénombrable} de polynômes dans \( \eQ[X]\) et donc une infinité dénombrable de racines de tels polynômes. Toute extension algébrique de \( \eQ\) est donc dénombrable. Voir aussi la proposition \ref{PROPooVPQFooScWvkS}.
\end{example}

%--------------------------------------------------------------------------------------------------------------------------- 
\subsection{Extension algébrique et polynôme minimal}
%---------------------------------------------------------------------------------------------------------------------------

\begin{proposition}[\cite{ooLIOMooBuCPUS}]      \label{PROPooALFJooDjmIcb}
    Soit une extension algébrique\footnote{Définition \ref{DEFooREUHooLVwRuw}.} \( \eL\) du corps $\eK$.
    \begin{enumerate}
        \item
            Pour tout \( a\in \eL\), il existe un polynôme \( P\in \eK[X]\) tel que \( P(a)=0\).
        \item       \label{ITEMooEFNFooKYqXDk}
            Le polynôme minimal de \( a\) dans \( \eK[X]\) est l'unique polynôme unitaire irréductible annulant \( a\).
    \end{enumerate}
\end{proposition}
\index{polynôme!minimal}

\begin{proof}
    Le premier point est seulement la définition~\ref{DEFooREUHooLVwRuw} d'une extension algébrique.

    L'idéal annulateur \( I_a=\{ P\in \eK[X]\tq P(a)=0 \}\) n'est pas réduit à \( \{ 0 \}\) parce que \( \eL\) est une extension algébrique. L'existence du polynôme minimal est le lemme~\ref{DefCVMooFGSAgL} et le fait qu'il soit irréductible est la proposition~\ref{PropRARooKavaIT}\ref{ItemDOQooYpLvXri}.

    Ce qui nous intéresse ici est l'unicité. Soit \( \mu_1\in \eK[X]\), un polynôme annulateur de \( a\) irréductible et unitaire. Vu que \( \mu_1\in I_a\) et que par définition, \( I_a=(\mu)\), il existe \( P\in \eK[X]\) tel que \( \mu_1=P\mu\). Vu que \( \mu\) n'est pas inversible et que \( \mu_1\) est irréductible, \( P\) doit être inversible : \( \mu_1=k\mu\) pour un certain \( k\in \eK\).

    Vu que \( \mu\) et \( \mu_1\) sont unitaires, \( k=1\). Donc \( \mu_1=\mu\).
\end{proof}

\begin{lemma}       \label{LEMooHKTMooKEoOuK}
    Soient un corps \( \eK\), une extension \( \eL\) de \( \eK\) et \( \alpha\in \eL\), un élément algébrique\footnote{Définition \ref{LEMooLVPLooEkWYDN}.} sur \( \eK\). Si \( \mu\) est le polynôme minimal de \( \alpha\) sur \( \eK\) alors
    \begin{equation}
        \begin{aligned}
            \varphi\colon \eK[\alpha]&\to \eK[X]/(\mu) \\
            Q(\alpha)&\mapsto \bar Q
        \end{aligned}
    \end{equation}
    avec \( Q\in\eK[X]\) est un isomorphisme de corps et de \( \eK\)-espaces vectoriels.
    % position 118885898.
\end{lemma}

\begin{proof}
    D'abord, \( \alpha\) est algébrique, donc l'idéal annulateur \( I_{\alpha}\) n'est pas réduit à \( \{ 0 \}\), et l'existence d'un polynôme minimal est assurée par le lemme~\ref{DefCVMooFGSAgL}.

    Ensuite, le fait que \( \eK[X]/(\mu)\) soit un corps est le corolaire~\ref{CorsLGiEN}. Nous montrons à présent que \( \varphi\) est un isomorphisme (d'anneaux); cela suffit pour en déduire que \( \eK[\alpha]\) est également un corps.

    Ces préliminaires étant dits, nous commençons.
    \begin{subproof}
        \item[Bien définie]
            Nous devons prouver que \( \varphi\) est bien définie, c'est-à-dire que tout élément de \( \eK[\alpha]\) peut être écrit sous la forme \( Q(\alpha)\) pour un \( Q\in \eK[X]\), et que si \( Q_1(\alpha)=Q_2(\alpha)\) alors \( \bar Q_1=\bar Q_2\).

            Le fait que tous les éléments de \( \eK[\alpha]\) peuvent être écrits sous la forme \( Q(\alpha)\) est le proposition~\ref{PROPooPMNSooOkHOxJ}. Supposons que \( Q_1(\alpha)=Q_2(\alpha)\). Alors nous définissons \( R\in \eK[X]\) par \( Q_1=Q_2+R\), et en évaluant cette égalité en \( \alpha\) nous avons
            \begin{equation}
                Q_1(\alpha)=Q_2(\alpha)+R(\alpha),
            \end{equation}
            autrement dit \( R(\alpha)=0\). Donc \( R\) est dans l'idéal annulateur de \( \alpha\) et est donc dans \( (\mu)\), c'est-à-dire que dans le quotient \( \eK[X]/(\mu)\) nous avons \( \bar R=0\) et donc \( \bar Q_1=\bar Q_2\).

        \item[Surjective]

            Tout élément de \( \eK[X]/(\mu)\) est de la forme \( \bar Q\) pour un \( Q\in \eK[X]\). Or ces éléments sont ceux de l'ensemble d'arrivée de \( \varphi\).

        \item[Injective]

            Si \( \bar Q_1=\bar Q_2\), alors \( Q_1=Q_2+R\) avec \( R\) dans l'idéal engendré par \( \mu\), c'est-à-dire entre autres \( R(\alpha)=0\). Donc \( Q_1(\alpha)=Q_2(\alpha)\).

    \end{subproof}

    Nous devons encore montrer que nous avons là un morphisme de \( \eK\)-espaces vectoriels.
    \begin{enumerate}
        \item
            Si \( k\in \eK\) alors \( \varphi\big( kQ(\alpha) \big)=\overline{ kQ }\). Mais par définition de la structure d'espace vectoriel sur \( \eK[X]/(\mu)\), \( \overline{ kQ }=k\bar Q\) (vérifier que cette définition de la multiplication par un scalaire sur \( \eK[X]/(\mu)\) est correcte).
        \item
            Nous avons aussi \( \varphi\big( Q_1(\alpha)+Q_2(\alpha) \big)=\varphi\big( (Q_1+Q_2)(\alpha) \big)=\overline{ Q_1+Q_2 }=\bar Q_1+\bar Q_2\).
    \end{enumerate}
\end{proof}


%---------------------------------------------------------------------------------------------------------------------------
\subsection{Extensions et polynômes}
%---------------------------------------------------------------------------------------------------------------------------

Nous savons déjà depuis la définition~\ref{DEFooFYZRooMikwEL} ce qu'est \( A[X]\) pour tout anneau \( A\) et donc a fortiori pour un corps.

\begin{definition}  \label{DEFooQPZIooQYiNVh}
    Soit un corps commutatif\footnote{Sauf mention du contraire, tous les corps du Frido sont commutatifs.}. Nous notons \( \eK(X)\) le corps des fractions\footnote{Définition~\ref{DEFooGJYXooOiJQvP}.} de \( \eK[X]\).
\end{definition}

\begin{lemmaDef}        \label{DEFooZHBZooKlNfGZ}
    Si \( R\in \eK(X)\), avec \( R=P/Q\) et si \( \eL\) est une extension\footnote{Définition \ref{DEFooFLJJooGJYDOe}.} de \( \eK\) contenant l'élément \( \alpha\), alors nous définissons
    \begin{equation}
        R(\alpha)=P(\alpha)Q(\alpha)^{-1}.
    \end{equation}
    Cela est une bonne définition au sens où elle ne dépend pas du choix du représentant \( (P,Q)\) pris dans la classe \( P/Q\).
\end{lemmaDef}

\begin{proof}
    Supposons \( R=P_1/Q_1=P_2/Q_2\). Par définition des classes (définition~\ref{DEFooGJYXooOiJQvP}) nous avons
    \begin{equation}        \label{EQooKHVNooABuHaO}
        P_1Q_2=Q_1P_2.
    \end{equation}
    Vu que l'évaluation est un morphisme \( \eK[X]\to\eK\) \footnote{Lemme~\ref{PROPooGDQCooHziCPH}.
    % laisser ce saut de ligne
    Certes ce lemme ne parle que d'anneaux, mais à y bien penser, dans le passage de \eqref{EQooKHVNooABuHaO} à \eqref{EQooJAIGooRADgiD}, nous ne considérons que les structures d'anneaux sur \( \eK[X]\) et \( \eK\).} nous pouvons évaluer l'équation \eqref{EQooKHVNooABuHaO} en \( \alpha\) :
    \begin{equation}        \label{EQooJAIGooRADgiD}
        P_1(\alpha)Q_2(\alpha)=Q_1(\alpha)P_2(\alpha).
    \end{equation}
    Cette dernière est une égalité dans le corps \( \eK\). Nous pouvons donc la multiplier par \( Q_2(\alpha)^{-1}P_2(\alpha)^{-1}\) (et utiliser toutes les hypothèses de commutativité des anneaux et corps) pour obtenir
    \begin{equation}
        P_1(\alpha)Q_1(\alpha)^{-1}=P_2(\alpha)Q_2(\alpha)^{-1},
    \end{equation}
    c'est-à-dire
    \begin{equation}
        (P_1/Q_1)(\alpha)=(P_2/Q_2)(\alpha).
    \end{equation}
\end{proof}

\begin{propositionDef}[\cite{MonCerveau}]  \label{DEFooVSKGooMyeGel}
    Soient un corps \( \eK\), une extension \( (\eL,j_{\eL})\) de \( \eK\) et un élément \( \alpha\in\eL\). Nous définissons \( \eK(\alpha)_{\eL} \) comme étant l'intersection de tous les sous-corps de \( \eL\) contenant \( j_{\eL}(\eK)\) et \( \alpha\).

    Alors
    \begin{enumerate}
        \item
            \( \eK(\alpha)_{\eL}\) est un sous-corps de \( \eL\),
        \item
            \( \eK(\alpha)_{\eL}\) est une extension\footnote{Définition \ref{DEFooFLJJooGJYDOe}.} de \( \eK\).
    \end{enumerate}
\end{propositionDef}

\begin{proof}
    Nous commençons par prouver que \( \eK(\alpha)_{\eL}\) est bien un corps. Si \( a,b\in \eK(\alpha)_{\eL}\) alors il suffit de calculer \( ab\), \( a+b\) et \( a^{-1}\) dans n'importe quel sous-corps de \( \eL\) contenant \( \eK\) et \( \alpha\); nous avons une garantie que \( a\), \( b\), \( ab  \), \( a+b\) et \( a^{-1}\) sont dans tous les tels sous-corps.

    Pour prouver que \( \eK(\alpha)_\eL\) est bien une extension, nous devons trouver une homomorphisme de corps \( j\colon \eK\to \eK(\alpha)_{\eL}\). Il se fait que prendre \( j=j_{\eL}\) fonctionne parce que par définition, \( \eK(\alpha)_{\eL}\) est une partie de \( \eL\) contenant l'image de \( j_{\eL}\).
\end{proof}

\begin{lemma}       \label{LEMooHZLCooPLHkLS}
    Soit \( n\) tel que \( \sqrt{ n }\) ne soit pas un rationnel. Si \( \alpha\in \{ a+b\sqrt{ n } \}_{a,b\in \eQ}\), alors il existe un unique choix \( (x,y)\in \eQ^2\) tel que
    \begin{equation}
        \alpha=x+y\sqrt{ n }.
    \end{equation}
\end{lemma}

\begin{example}
    Nous avons 
    \begin{equation}
        \eQ(\sqrt{ 2 })_{\eR}=\{ a+b\sqrt{ 2 } \}_{a,b\in \eQ}
    \end{equation}
    où à droite nous calculons les sommes et et les produits dans \( \eR\). Le tout est un sous-ensemble de \( \eR\) qui se révèle être un corps contenant \( \eQ\) et \( \sqrt{ 2 }\).

    En particulier, dans \( \eQ(\sqrt{ 2 })_{\eR}\) nous avons \( \sqrt{ 2 }\sqrt{ 2 }=2\).
\end{example}

\begin{lemma}   \label{LEMooKVPZooPqPrce}
    Les corps \( \eQ(\sqrt{ 2 })_{\eR}\) et \( \eQ(\sqrt{ 3 })_{\eR}\) ne sont pas isomorphes.
\end{lemma}

\begin{proof}
    Supposons l'existence d'un morphisme de corps\footnote{Définition \ref{DEFooSPHPooCwjzuz}. Oui, c'est un bête morphisme d'anneaux. Il n'y a pas plus de structure dans un corps que dans un anneau.}
    \begin{equation}
        \psi\colon \eQ(\sqrt{ 2 })_{\eR}\to \eQ(\sqrt{ 3 })_{\eR}.
    \end{equation}
    Nous notons «\( 1\)» à la fois le neutre de la multiplication dans \( \eQ(\sqrt{ 2 })_{\eR}\) et \( \eQ(\sqrt{ 3 })_{\eR}\) (qui s'évèrent être les mêmes en tant qu'élément de \( \eR\), mais ça n'a pas d'importance ici).

    Soit \( \alpha\in \eQ(\sqrt{ 2 })_{\eR}\) tel que \( \alpha^2-1=0\). Alors nous avons aussi
    \begin{equation}
        \psi(\alpha)^2-1=\psi(\alpha^2)-\psi(1)=\psi(\alpha^2-1)=\psi(0)=0.
    \end{equation}
    Donc \( \psi(\alpha)\) est un élément de \( \eQ(\sqrt{ 3 })_{\eR}\) qui est une racine de \( X^2-1\).

    Or un tel élément n'existe pas dans \( \eQ(\sqrt{ 3 })_{\eR}\) parce que nous savons que dans \( \eR\) entier, il n'y a que deux racines : \( \pm\sqrt{ 2 }\), et aucune des deux n'est dans \( \eQ(\sqrt{ 3 })_{\eR}\).
\end{proof}

\begin{example}      \label{EXooJRSUooYhAZkR}
    Est-ce que \( \eK(\alpha)_{\eL}\) dépend réellement de \( \eL\) ? Si \( \eL_2\) est une extension de \( \eL\) alors nous avons évidemment\footnote{Vérifiez-le tout de même.} \( \eK(\alpha)_{\eL_2}=\eK(\alpha)_{\eL}\).

    Nous commençons par construire un corps \( \eK\) un peu idiot qui, comme ensemble, est comme \( \eQ(\sqrt{ 2 })_{\eR}\), c'est-à-dire la partie
    \begin{equation}
        \{ a+b\sqrt{ 2 } \}_{a,b\in \eQ},
    \end{equation}
    de \( \eR\).

    Mais cette fois nous définissons la multiplication suivante :
    \begin{equation}
        (a+b\sqrt{ 2 })(c+d\sqrt{ 2 })=ac+3bd+(ad+bc)\sqrt{ 2 }.
    \end{equation}
    
    Cela est un corps parce que tout élément non nul est inversible. En effet, l'équation
    \begin{equation}        \label{EQooIZLEooLPOBcC}
        (a+b\sqrt{ 2 })(x+y\sqrt{ 2 })=1
    \end{equation}
    donne 
    \begin{equation}
        \begin{pmatrix}
            a    &   3b    \\ 
            b    &   a    
        \end{pmatrix}\begin{pmatrix}
            x    \\ 
            y    
        \end{pmatrix}=\begin{pmatrix}
            1    \\ 
            0    
        \end{pmatrix}.
    \end{equation}
    Ce système a une unique solution si et seulement si \( \det\begin{pmatrix}
        a    &   3b    \\ 
        b    &   a    
    \end{pmatrix}=0\). Cela survient si et seulement si
    \begin{equation}
        a^2-3b^2=0.
    \end{equation}
    Les solutions de cela dans \( \eR\) sont \( a=\pm\sqrt{ 3 }| b |\). Dès que \( a\) ou \( b\) est non nul, cela ne peut pas satisfaire \( a,b\in \eQ\). Donc le déterminant est toujours non nul et il existe \( x,y\in \eQ\) tels que \eqref{EQooIZLEooLPOBcC} soit satisfaite.

    Tout cela nous a donné une corps \( \eK\) dont \( \eQ\) est un sous-corps et qui contient l'élément \( \sqrt{ 2 }\) de \( \eR\). Il n'est cependant pas un sous-corps de \( \eR\).

    Ce corps est isomorphe à \( \eQ(\sqrt{ 3 })_{\eR}\). En effet, nous montrons que
    \begin{equation}
        \begin{aligned}
            \psi\colon \eK&\to \eQ(\sqrt{ 3 })_{\eR} \\
            a+b\sqrt{ 2 }&\mapsto q+b\sqrt{ 3 } 
        \end{aligned}
    \end{equation}
    est un isomorphisme de corps. Pour le produit, nous avons
    \begin{subequations}
        \begin{align}
            \psi\big( (a+b\sqrt{ 2 })(c+d\sqrt{ 2 }) \big)&=\psi\big( qc+3bd+(ad+bc)\sqrt{ 2 } \big) \label{SUBEQooQSZBooHZDTKo}\\
            &=ac+3bd+(ad+bc)\sqrt{ 3 }\label{SUBEQooPEKHooNPcIjE}\\
            &=(a+b\sqrt{ 3 })(c+d\sqrt{ 3 })\label{SUBEQooIGBZooMwrmFe}\\
            &=\psi(a+b\sqrt{ 2 })\psi(c+d\sqrt{ 2 }).
        \end{align}
    \end{subequations}
    Remarques :
    \begin{itemize}
        \item L'application \( \psi\) est bien définie grâce au lemme \ref{LEMooHZLCooPLHkLS} couplé au théorème \ref{THOooYXJIooWcbnbm} appliqué à \( n=2\) et \( n=3\).
        \item Dans le membre de gauche de \eqref{SUBEQooQSZBooHZDTKo}, \( b\sqrt{ 2 }\) est un produit dans \( \eR\) (d'où l'importance du lemme \ref{LEMooHZLCooPLHkLS} qui permet de re-séparer les éléments de \( \eR\) partie rationnelle et partie multiple de \( \sqrt{ 2 }\)), et le produit entre \( (a+b\sqrt{ 2 })\) et \( (c+d\sqrt{ 2 })\) est un produit dans \( \eK\).
        \item
            Dans \eqref{SUBEQooPEKHooNPcIjE} et \eqref{SUBEQooIGBZooMwrmFe}, tous les produits sont dans \( \eR\).
    \end{itemize}

    En comparant avec le lemme \ref{LEMooKVPZooPqPrce}, nous avons alors
    \begin{equation}
        \eQ(\sqrt{ 2 })_\eK=\eQ(\sqrt{ 3 })_\eR\neq \eQ(\sqrt{ 2 })_{\eR}
    \end{equation}
\end{example}

\begin{normaltext}
    Nous allons encore enfoncer le clou sur le fait que \( \eK(\alpha)_{\eL}\) dépend de \( \eL\).

    Le fait est que si on y pense, l'objet \( \sqrt{ 2 }\) n'a aucun rapport avec \( \eQ\). En effet les objets de \( \eQ\) sont des classes d'équivalence de couples d'éléments de \( \eZ\), alors que l'élément \( \sqrt{ 2 }\) est une classe d'équivalence de suites de Cauchy dans \( \eQ\).

    Lorsque nous écrivons \( \eQ(\sqrt{ 2 })\), nous associons des objets de nature complètement différentes, et il n'y a aucune raison à priori de définir la multiplication entre eux d'une façon plutôt qu'une autre.

    Plus généralement, dans ZF (que nous faisons du semblant de suivre tout en sachant que nous ne savons pas ce que c'est réellement\footnote{En lisant quelques pages de Wikipédia, vous pourrez briller en société, mais ne tentez pas le coup à l'agrégation.}), tout est ensemble. Peut-on dire ce que serait \( \eQ(I)\) si \( I\) est un ensemble quelconque ? Attention : en écrivant \( \eQ(I)\), nous entendons un corps dont \( I\) est un élément, pas un corps qui contiendrait comme éléments tous les éléments de \( I\).

    Si \( I\) est juste un ensemble, quelle définition donner de \( I^2\) ? Il y a plein de choix et rien ne se dégage clairement comme étant pertinent. Si par contre, en guise de \( I\) nous considérons l'ensemble \( \sqrt{ 2 }\) (oui, c'est un ensemble : un ensemble de suites de Cauchy dans \( \eQ\)), alors tout de suite nous nous disons que la bonne façon de faire est \( \sqrt{ 2 }^2=2\). Ce réflexe est juste conditionné par le fait que nous connaissons déjà par ailleurs le corps \( \eR\). Rien de plus.

    Donc oui, \( \eK(\alpha)_{\eL}\) dépend de \( \eL\), mais dans les cas particuliers où \( \eK\) est un sous-corps de \( \eC\), il y a un implicite comme quoi \( \eL=\eC\). Cela étant dit, il n'y a plus d'ambiguïtés en écrivant \( \eQ(\sqrt{ 2 })\).
\end{normaltext}

Dans l'énoncé suivant, la notation \( R(\alpha)_{\eL}\) signifie que l'évaluation de \( R\) sur \( \alpha\) se fait en calculant dans le sur-corps \( \eL\) de \( \eK\).  Cette proposition semble indiquer que \( \eK(\alpha)\) est donné en termes de \( \eK(X)\), lequel est défini de façon très intrinsèque sans faire appel à un corps ambiant de \( \eK\).

\begin{proposition}[\cite{MonCerveau}]     \label{PROPooYSFNooFGbbCi}
    Soit une extension \( \eL\) du corps \( \eK\) et \( \alpha\in \eL\). Alors nous avons les isomorphismes de corps suivants :
    \begin{enumerate}
        \item
            \( \eK(\alpha)_{\eL}=\Frac\big( \eK[\alpha]_{\eL} \big)\),
        \item       \label{ITEMooATPTooVXKdlK}
            \( \eK(\alpha)_{\eL}=\{ R(\alpha)_{\eL}\tq R\in \eK(X) \}\).
    \end{enumerate}
\end{proposition}

\begin{proof}
    Le corps \( \eK(\alpha)\) est un sous-corps de \( \eL\) contenant \( \eK[\alpha]\) comme sous-anneau. La proposition~\ref{PROPooGSHDooJOnDsp} nous dit alors que l'application suivante est un morphisme injectif de corps :
    \begin{equation}
        \begin{aligned}
            \epsilon\colon \Frac\big( \eK[\alpha] \big)&\to \eK(\alpha) \\
            P/Q&\mapsto PQ^{-1}.
        \end{aligned}
    \end{equation}
    Pour rappel, la notation \( P/Q\) est bien une notation pour la classe d'équivalence du couple \( (P,Q)\) pour la relation définie en~\ref{DEFooGJYXooOiJQvP}.

    Par ailleurs, la partie \( \epsilon\Big( \Frac\big( \eK[\alpha] \big) \Big) \) est inclue à \( \eL\) et est un corps contenant \( \eK\) et \( \alpha\). Donc le corps \( \eL\) fait partie des corps sur lesquels on prend l'intersection pour définir \( \eK(\alpha)\). Cela prouve que
    \begin{equation}
        \eK(\alpha)\subset  \epsilon\Big( \Frac\big( \eK[\alpha] \big) \Big).
    \end{equation}
    L'application \( \epsilon\) est donc surjective sur \( \eK(\alpha)\). Vu qu'elle était déjà injective, elle est bijective.

    Pour la seconde partie, veuillez lire la définition~\ref{DEFooLBIWooCPCaSY} de l'évaluation d'une fraction rationnelle sur un élément de l'anneau. Si \( R=P/Q\in \eK(X)\) et si \( \alpha\in \eL\), nous avons
    \begin{equation}
        R(\alpha)=P(\alpha)Q(\alpha)^{-1}.
    \end{equation}
    Tout sous-corps de \( \eL\) contenant \( \eK\) et \( \alpha\) doit contenir en particulier \( \{ P(\alpha)\tq P\in \eK[X] \} \), les inverses \( \{ P(\alpha)^{-1}\tq P\in \eK[X],\,P(\alpha)\neq 0 \}\) et les produits d'iceux. Donc tout sous-corps de \( \eL\) contenant \( \eK\) et \( \alpha\) contient \( \{ R(\alpha)\tq R\in \eK(X) \}\).

    Nous avons donc
    \begin{equation}
        \{ R(\alpha)\tq R\in \eK(X) \}\subset \eK(\alpha).
    \end{equation}
    Mais vu que \( \eK(\alpha)\) est lui-même un sous-corps de \( \eL\) contenant \( \eK\) et \( \alpha\), il est contenu dans \( \{ R(\alpha)\tq R\in \eK(X) \}\). D'où l'égalité.
\end{proof}

Pourquoi cela ne contredit pas l'exemple~\ref{EXooJRSUooYhAZkR} ? Lorsque nous écrivons
\begin{equation}
    \eK(\alpha)=\{ R(\alpha)\tq R\in \eK(X) \},
\end{equation}
certes \( \eK(X)\) est défini sans faire appel à un corps contenant \( \eK\). Mais l'évaluation \( R(\alpha)\), oui. Pour calculer \( R(\alpha)\), il faut écrire \( R=P/Q\) et calculer \( P(\alpha)Q(\alpha)^{-1}\). Tous les calculs de cette dernière expression doivent se faire dans un sur-corps de \( \eK\). Il suffit que le sur-corps en question soit un monceau de mauvaise foi comme celui de l'exemple~\ref{EXooJRSUooYhAZkR}, et en réalité \( \eK(\alpha)\) peut ne pas être ce que l'on croit.

Le corolaire suivant montre que les choses s'arrangent.

\begin{corollary}
    Soient un corps \( \eK\), une extension \( \eL_1\) de \( \eK\), un élément \( \alpha\in \eL_1\) et une extension \( \eL_2\) de \( \eL_1\). Alors
    \begin{equation}
        \eK(\alpha)_{\eL_1}=\eK(\alpha)_{\eL_2}.
    \end{equation}
\end{corollary}

\begin{proof}
    La proposition~\ref{PROPooYSFNooFGbbCi} nous dit que
    \begin{subequations}
        \begin{align}
            \eK(\alpha)_{\eL_1}=\{ R(\alpha)_{\eL_1}\tq R\in\eK(X) \}\\
            \eK(\alpha)_{\eL_2}=\{ R(\alpha)_{\eL_2}\tq R\in\eK(X) \}.
        \end{align}
    \end{subequations}
    Mais lorsque \( R\in \eK(X)\), le calcul de \( R(\alpha)\) est exactement le même dans \( \eL_1\) et dans \( \eL_2\) parce que \( \eL_2\) est un sur-corps de \( \eL_1\) et que les calculs effectifs de \( R(\alpha)=P(\alpha)Q(\alpha)^{-1}\) ne font intervenir que des quantités de \( \eK\) et des puissances de \( \alpha\).
\end{proof}

Ce que ce corolaire nous dit est que si le contexte fixe une extension de \( \eK\), nous pouvons faire tous les calculs dans cette extension, même si il y a des piles d'extensions à côté.

Typiquement, à chaque fois que nous considérons des sous-corps de \( \eC\), les extensions se feront dans \( \eC\) : pour tout \( \alpha\in \eC\), les corps \( \eQ(\alpha)\), \( \eR(\alpha)\) se calculent dans \( \eC\).


\begin{proposition}     \label{PROPooSYQWooFbfQtm}
    Soit un corps \( \eK\), une extension \( \eL\) et un élément \( \alpha\in \eL\). Nous considérons l'application
    \begin{equation}
        \begin{aligned}
            \varphi\colon \eK[X]&\to \eL \\
            P&\mapsto P(\alpha).
        \end{aligned}
    \end{equation}
    \begin{enumerate}
        \item       \label{ITEMooUZDQooOasiRQ}
            Si \( \alpha\) est transcendant, alors \( \eK[\alpha]=\eK[X]\) (isomorphisme d'anneaux).
        \item
            Si \( \alpha\) est transcendant, alors \( \eK(\alpha)_{\eL}=\eK(X)\) (isomorphisme de corps),
        \item
            Si \( \alpha\) est algébrique, alors \( \ker(\varphi)\) est un idéal possédant un unique générateur unitaire, lequel est le polynôme minimal\footnote{Définition~\ref{DefCVMooFGSAgL}.} de \( \alpha\) sur \( \eK\).
    \end{enumerate}
\end{proposition}

\begin{proof}
    Point par point.
    \begin{enumerate}
        \item
            Nous savons que \( \eK[\alpha]=\{ Q(\alpha)\tq Q\in \eK[X] \}\) (c'est la proposition~\ref{PROPooPMNSooOkHOxJ}). Donc \( \varphi\) est surjective sur \( \eK[\alpha]\), et est donc bijective. Elle est un isomorphisme\footnote{Les amateurs d'écriture inclusive ne seront, je l'espère, pas choqué par «\emph{elle} est \emph{un} isomorphisme»; c'est une tournure que je propose ici sur le modèle de l'immonde «\emph{elle} est \emph{un} ministre» ou, à peine moins grave, «\emph{il} est \emph{une} sommité».} parce que le lemme~\ref{LEMooLVPLooEkWYDN} dit déjà que c'est un morphisme.
        \item
            Nous supposons encore que \( \alpha\) est transcendant et nous considérons l'application
            \begin{equation}
                \begin{aligned}
                    \psi\colon \eK(X)&\to \eK(\alpha) \\
                    P&\mapsto R(\alpha).
                \end{aligned}
            \end{equation}
            Note : cette application n'est pas \( \varphi\). En effet \( \varphi\) n'est définie que sur \( \eK[X]\); le corps des fractions \( \eK(X)\) est nettement plus grand (classes d'équivalence de couples).

            Le fait que cette application soit surjective est la proposition~\ref{PROPooYSFNooFGbbCi}\ref{ITEMooATPTooVXKdlK}. Pour l'injectivité nous supposons que \( \psi(R)=0\), c'est-à-dire que \( R(\alpha)=0\). Nous considérons un représentant \( (P,Q)\) de \( R\); c'est-à-dire \( R=P/Q\). L'égalité \( R(\alpha)=0\) signifie \( P(\alpha)Q(\alpha)^{-1}=0\) (égalité dans \( \eL\)). Vu que \( \eL\) est un corps, c'est un anneau intègre et nous avons la règle du produit nul; soit \( P(\alpha)=0\), soit \( Q(\alpha)^{-1}=0\). La seconde possibilité est impossible parce que zéro n'est pas inversible. Donc \( P(\alpha)=0\). Donc \( \varphi(P)=0\) et \( \varphi\) étant injective, \( P=0\).

            Lorsque \( P=0\), la classe \( P/Q\) est nulle dans \( \eK(X)= \Frac\big(\eK[X]\big)\).

        \item

            C'est le lemme-définition~\ref{DefCVMooFGSAgL}.
    \end{enumerate}
\end{proof}

\begin{proposition}\label{PropXULooPCusvE}
    Soit un corps \( \eK\) et une extension \( \eL\). Soit \( P\in \eK[X]\) et  \( a\in \eL\), une racine de \( P\). Alors le polynôme minimal d'une racine divise\footnote{Définition~\ref{DefMPZooMmMymG}.} tout polynôme annulateur.

    Autrement dit, l'idéal engendré par le polynôme minimal est l'idéal des polynômes annulateurs.
\end{proposition}

\begin{proof}
    Nous considérons l'idéal
    \begin{equation}
        I=\{ Q\in \eK[X]\tq Q(a)=0 \}.
    \end{equation}
    Le fait que cela soit un idéal est simplement dû à la définition du produit : \( (PQ)(a)=P(a)Q(a)\). Par le théorème~\ref{ThoCCHkoU}, le polynôme minimal \( \mu_a\) de \( a\) est dans \( I\) et qui plus est le génère : \( I=(\mu_a)\). Par conséquent tout polynôme annulateur de \( a\) est divisé par \( \mu_a\).
\end{proof}

%///////////////////////////////////////////////////////////////////////////////////////////////////////////////////////////
\subsubsection{Extension algébrique, degré}
%///////////////////////////////////////////////////////////////////////////////////////////////////////////////////////////

\begin{proposition}
    Toute extension finie est algébrique.
\end{proposition}

\begin{proof}
    Soient un corps \( \eK\), une extension \( \eL\) de degré\footnote{Définition \ref{DefUYiyieu}.} \( n\) de \( \eK\) et \( a\in \eL\). Nous devons montrer qu'il existe un polynôme annulateur de \( a\) à coefficients dans \( \eK\).

    Soit la partie \( S=\{1,a,a^2,\ldots, a^n\}\) de \( \eL\). Si cette partie contient des éléments non distincts, alors c'est plié. En effet, si \( a^k=a^l\), alors le polynôme \( X^{k-l}\) est un polynôme annulateur de \( a\).

    Nous supposons donc que \( S\) contienne exactement \( n+1\) éléments distincts. Le lemme~\ref{LemytHnlD} nous assure que \( S\) est une partie liée : il existe des éléments \( k_i\in \eK\) tels que \( \sum_{i=0}^nk_ia^i=0\).

    Donc le polynôme \( \sum_ia_iX^i\) est un polynôme annulateur de \( a\).
\end{proof}

\begin{proposition}[Propriétés d'extensions algébriques\cite{MonCerveau}]   \label{PropURZooVtwNXE}
    Soit \( \eK\) un corps commutatif\footnote{Juste en passant nous rappelons que tous les corps considérés ici sont commutatifs} et \( a\) un élément algébrique\footnote{Définition \ref{LEMooLVPLooEkWYDN}.} sur \( \eK\), de polynôme minimal \( \mu_a\) de degré \( n\). Alors
    \begin{enumerate}
        \item\label{ItemJCMooDgEHajmi}
            En considérant l'application d'évaluation
            \begin{equation}
                \begin{aligned}
                    \varphi_a\colon \eK[X]&\to \eL \\
                    Q&\mapsto Q(a),
                \end{aligned}
            \end{equation}
            nous avons \( \eK[a]=\Image(\varphi_a)\).
        \item\label{ItemJCMooDgEHajiv}
            Une base de \( \eK[a]\) comme espace vectoriel sur \( \eK\) est donnée par \( \{ 1,a,a^2,\ldots, a^{n-1} \}\).
        \item\label{ItemJCMooDgEHajiii}
            Le degré de l'extension \( \eK[a]\) est égal au degré du polynôme minimal :
            \begin{equation}
                \big[ \eK[a]:\eK \big]=n.
            \end{equation}
         \item
            L'anneau \( \eK[a]\) est l'ensemble des polynômes en \( a\) de degré \( n-1\) à coefficient dans \( \eK\).
        \item\label{ItemJCMooDgEHaji}
            \( \eK(a)=\eK[a]\).
        \item   \label{ItemJCMooDgEHajii}
            Il existe un isomorphisme d'anneaux \( \varphi\colon \eK[a]\to \eK[X]/(\mu_a)\) tel que \( \varphi(k)=\bar k\) pour tout \( k\in \eK\).
            \( \eK[a]\simeq\eK[X]/(\mu_a)\) (isomorphisme d'anneau).
    \end{enumerate}
\end{proposition}
\index{extension!de corps!algébrique}
L'intérêt de~\ref{ItemJCMooDgEHajii} est qu'il permet de caractériser \( \eK[a]\) sans avoir recours à un sur-corps de \( \eK\). Le point~\ref{ItemJCMooDgEHajiii} indique que le degré d'une extension algébrique est égal au degré du polynôme minimal.


\begin{proof}
    \begin{enumerate}
        \item
            Nous avons \( \eK[a]\subset \Image(\varphi_a)\) parce que \( \Image(\varphi_a)\) est lui-même un sous-anneau de \( \eL\) contenant \( \eK\) et \( a\). Pour rappel, \( \eK[a]\) est l'intersection de tous les tels sous-anneaux.

            L'inclusion inverse est le fait que si \( Q\in \eK[X]\) alors \( Q(a)\in \eK[a]\) parce que \( \eK[a]\) est un anneau et contient donc tous les \( a^n\).
        \item
            La partie \( \{ 1,a,a^2,\ldots, a^{n-1} \}\) est libre parce qu'une combinaison linéaire de ces éléments est un polynôme de degré \( n-1\) en \( a\). Un tel polynôme ne peut pas être nul parce que nous avons mis comme hypothèse que le polynôme minimal de \( a\) est \( n\).

            Rappelons qu'en vertu de la définition~\ref{DefCVMooFGSAgL}, le polynôme minimal \( \mu_a\) est unitaire; donc le polynôme \( \mu_a(X)-X^n\) est un polynôme de degré \( n-1\). Par conséquent en posant \( S(X)=X^n-\mu_a(X)\), le polynôme \( S\) est de degré \( n-1\) et vérifie \( a^n=S(a)\).

            En vertu du point~\ref{ItemJCMooDgEHajmi}, un élément de \( \eK[a]\) s'écrit \( Q(a)\) pour un certain \( Q\in\eK[X]\). Supposons que \( Q\) soit de degré \( p>n-1\); alors nous le décomposons en une partie contenant les termes de degré jusqu'à \( n-1\) et une partie contenant les autres :
            \begin{equation}
                Q(X)=Q_1(X)+X^nQ_2(X)
            \end{equation}
            où \( Q_1\) est de degré \( n-1\) et \( Q_2\) de degré \( p-n\). Nous évaluons cette égalité en \( a\) :
            \begin{equation}
                Q(a)=Q_1(a)+S(a)Q_2(a).
            \end{equation}
            Donc \( Q(a)\) est l'image de \( a\) par le polynôme \( Q_1+SQ_2\) qui est de degré \( p-1\). Par récurrence, \( Q(a)\) est l'image de \( a\) par un polynôme de degré \( n-1\).

            Notons que l'idée est très simple : il s'agit de remplacer récursivement tous les \( a^n\) par \( S(a)\).
    \item
        Conséquence immédiate de~\ref{ItemJCMooDgEHajiv}.
    \item
        Conséquence immédiate de~\ref{ItemJCMooDgEHajiv}.
    \item
        Un élément général non nul de \( \eK[a]\) est de la forme \( Q(a)\) avec \( Q\in\eK[X]\); il s'agit de lui trouver un inverse. Pour cela nous remarquons que les polynômes \( \mu_a(X)\) et \( Q(x)\) sont premiers entre eux, sinon \( \mu_a\) ne serait pas un polynôme minimal (voir la proposition~\ref{PropRARooKavaIT}). Donc le théorème de Bézout~\ref{ThoBezoutOuGmLB} affirme l'existence d'éléments \( U,V\in \eK[X]\) tels que
        \begin{equation}
            U\mu_a+VQ=1
        \end{equation}
        dans \( \eK[X]\). Nous évaluons cette égalité en \( a\) en tenant compte de \( \mu_a(a)=0\) dans \( \eK[a]\) :
        \begin{equation}
            U(a)\mu_a(a)+V(a)Q(a)=1
        \end{equation}
        dans \( \eK[a]\). Par conséquent \( V(a)Q(a)=1\), ce qui signifie que \( V(a)\) est l'inverse de \( Q(a)\).
        \item
            Nous considérons l'application
            \begin{equation}
                \begin{aligned}
                    \psi\colon \eK[X]/(\mu_a)&\to \eK[a] \\
                    \bar R&\mapsto R(a)
                \end{aligned}
            \end{equation}
            et nous montrons qu'elle convient. Pour cela, nous nous souvenons que la proposition~\ref{PropXULooPCusvE} nous enseigne que \( (\mu_a)\), l'idéal engendré par \( \mu_a\), est égal à l'idéal des polynômes annulateurs de \( a\) dans \( \eK[X]\). Le polynôme \( \mu_a\) divise tous les éléments de cet idéal; voir aussi la définition~\ref{DefSKTooOTauAR} de l'idéal \( (\mu_a)\). Cela étant mis au point, nous passons à la preuve.
            \begin{subproof}
            \item[\( \psi\) est bien définie]

                Si \( \bar R=\bar S\) alors \( R=S+Q\) avec \( Q\in(\mu_a)\), et par conséquent \( R(a)=S(a)+Q(a)\) avec \( Q(a)=0\).

            \item[Surjective]

                Nous savons que \( \eK[a]=\Image(\varphi_a)\). Si \( x\in \eK[a]\) alors il existe \( Q\in \eK[X]\) tel que \( x=Q(a)\). Dans ce cas nous avons aussi \( x=\psi(\bar Q)\).

            \item[Injective]

                Si \( \psi(\bar R)=0\) alors \( R(a)=0\), mais comme mentionné plus haut, \( \mu_a\) engendre l'idéal est polynômes annulateurs de \( a\). Donc \( R\in (\mu_a)\) et nous avons \( \bar R=0\) dans \( \eK[X]/(\mu_a)\).

            \end{subproof}

    \end{enumerate}
\end{proof}

\begin{example}
    Un fait connu est que \( \frac{1}{ \sqrt{2} }=\frac{ \sqrt{2} }{ 2 }\). Donc l'inverse de \( \sqrt{2}\) s'exprime bien comme un polynôme en \( \sqrt{2}\) à coefficients dans \( \eQ\), ce qui confirme le point~\ref{ItemJCMooDgEHaji} de la proposition~\ref{PropURZooVtwNXE}. Du point de vue de Bézout, \( \mu_{\sqrt{2}}(X)=X^2-2\), et nous cherchons des polynômes \( U\) et \( V\) tels que
    \begin{equation}
        U(X^2-2)+VX=1.
    \end{equation}
    cette égalité est réalisée par \( U=-\frac{ 1 }{2}\) et \( V=\frac{ 1 }{2}X\). Et effectivement \( V(\sqrt{2})\) est bien l'inverse de \( \sqrt{2}\) :
    \begin{equation}
        V(\sqrt{(2)})=\frac{ 1 }{2}\sqrt{2}.
    \end{equation}
\end{example}

\begin{lemma}[\cite{UQerHHk}]
    Un nombre complexe algébrique dont tous les conjugués sont de module \( 1\) est une racine de l'unité.
\end{lemma}

\begin{proposition}[\cite{ooTGTKooFenWAc}]      \label{PROPooNGJWooYSpwVn}
    Soient un corps \( \eK\), un extension \( \eL\) de \( \eK\) et un élément \( \alpha\) de \( \eL\). Il y a équivalence entre les trois points suivants :
    \begin{enumerate}
        \item   \label{ITEMooYTEBooUuEfBz}
            \( \alpha\) est algébrique sur \( \eK\),
        \item   \label{ITEMooWMQTooLnepQl}
            \( \eK[\alpha]=\eK(\alpha)\),
        \item   \label{ITEMooAQIUooMVZojp}
            \( \eK[\alpha]\) est un \( \eK\)-espace vectoriel de dimension finie.
    \end{enumerate}
    Si ces affirmation sont vraies, alors \( [\eK(\alpha):\eK]\) est le degré du polynôme minimal de \( \alpha\) sur \( \eK\).
\end{proposition}

\begin{proof}
    Démonstration décomposée en plusieurs implications.
    \begin{subproof}
        \item[\ref{ITEMooYTEBooUuEfBz} implique~\ref{ITEMooWMQTooLnepQl}]

            Soit \( \alpha\) algébrique sur \( \eK\). Nous considérons le polynôme minimal de \( \alpha\) sur \( \eK\) (définition~\ref{DefCVMooFGSAgL}). Nous savons par le lemme~\ref{LEMooHKTMooKEoOuK} (qui fonctionne parce que \( \alpha\) est algébrique) que \( \eK[\alpha]=\eK[X]/(\mu)\) en tant qu'anneaux.

            Mais \( \eK[X]\) est un anneau principal et \( \mu\) en est un élément irréductible. Donc la proposition~\ref{PropomqcGe} dit que \( (\mu)\) est un idéal maximum; la proposition~\ref{PropoTMMXCx} avance encore un peu en disant que \( \eK[X]/(\mu)\) est un corps.

            Donc \( \eK[X]/(\mu)\) est un corps isomorphe à \( \eK[\alpha]\) en tant qu'anneaux. En conséquence de quoi \( \eK[\alpha]\) est un corps.

            Le corps \( \eK[\alpha]\) est un sous-corps de \( \eL\) contenant \( \eK\) et \( \alpha\); par définition nous avons donc \( \eK(\alpha)\subset \eK[\alpha]\). Mais d'autre part, \( \eK[\alpha]\) est contenu dans tout sous-corps de \( \eL\) contenant \( \eK\) et \( \alpha\), donc il est inclus dans l'intersection de tout ces corps, donc \( \eK[\alpha]\subset \eK(\alpha)\).

            Les deux inclusions sont prouvées.

        \item[\ref{ITEMooWMQTooLnepQl} implique~\ref{ITEMooYTEBooUuEfBz}]

            Nous montrons que non-\ref{ITEMooYTEBooUuEfBz} implique non-\ref{ITEMooWMQTooLnepQl}. Nous disons donc que \( \alpha\) est transcendant sur \( \eK\); cela implique par la proposition~\ref{PROPooSYQWooFbfQtm}\ref{ITEMooUZDQooOasiRQ} que \( \eK[\alpha]=\eK[X]\) en tant qu'anneaux. Donc \( \eK[\alpha]\) n'est pas un corps parce que \( \eK[X]\) ne l'est pas.

            N'étant pas un corps, \( \eK[\alpha]\) ne peut pas être égal à \( \eK(\alpha)\) qui, lui, est un corps.

        \item[\ref{ITEMooYTEBooUuEfBz} implique~\ref{ITEMooAQIUooMVZojp}]

            L'élément \( \alpha\) est maintenant algébrique et nous considérons son polynôme minimal \( \mu\). Nous savons par le lemme~\ref{LEMooHKTMooKEoOuK} que \( \eK[\alpha]=\eK[X]/(\mu)\) en tant qu'espaces vectoriels. Or \( \eK[X]/(\mu)\) est de dimension finie \( \deg(\mu)\). Donc \( \eK[\alpha]\) est également de dimension finie.

        \item[\ref{ITEMooAQIUooMVZojp} implique~\ref{ITEMooYTEBooUuEfBz}]

            Nous démontrons la contraposée. En supposant que \( \alpha\) est transcendant nous avons \( \eK[\alpha]=\eK[X]\) par la proposition~\ref{PROPooSYQWooFbfQtm}. Or \( \eK[X]\) n'est pas de dimension finie sur \( \eK\), donc \( \eK[\alpha]\) non plus.

    \end{subproof}
\end{proof}

\begin{lemma}[\cite{SNDooCQYseS}]
    Soit \( \eL\) un corps commutatif et \( (\eK_i)_{i\in I}\) une famille de sous-corps de \( \eL\). Alors \( \bigcup_{i\in I}\eK_i\) est un sous-corps de \( \eL\).
\end{lemma}

\begin{definition}  \label{DefZCYIbve}
    Soit une extension\footnote{Définition \ref{DEFooFLJJooGJYDOe}.} de corps \( j\colon \eK\to \eL\). Soit \( A\subset \eL\).
    \begin{enumerate}
        \item
            Nous notons \( \eK(A)_{\eL}\)\nomenclature[A]{$\eK(A)$}{corps contenant $\eK$ et $A$} le plus petit sous corps de \( \eL\) contenant \( j(\eK)\) et \( A\). C'est l'intersection de tous les sous-corps de \( \eL\) contenant \( A\) et \( j(\eK)\).
\item
    Nous notons \( \eK[A]_{\eL}\)\nomenclature[A]{$\eK[A]$}{anneau contenant $ \eK$ et $ A$} le plus petit sous anneau de \( \eL\) contenant \( j(\eK)\) et \( A\). C'est l'intersection de tous les sous-anneaux de \( \eL\) contenant \( A\) et \( j(\eK)\).
    \end{enumerate}
    Le plus souvent, l'indice \( \eL\) dans \( \eK(A)_{\eL}\) et \( \eK[A]_{\eL}\) est omis parce que le contexte est clair\quext{Et je me demande si il est possible de trouver un cas tordu où \( \eK(A)_{\eL}\neq \eK(A)_\eM\). Par exemple lorsque \( A\) est dans \( \eL\) et \( \eM\), mais que \( \eL\) n'est pas inclus à \( \eM\), ni \( \eM\) dans \( \eL\).}, et nous avons même très souvent \( \eK\subset \eL\) en tant que ensembles. Dans ce cas, l'application \( j\) est l'identité et elle sera omise.


    Nous disons que l'extension \( \eL\) de \( \eK\) est \defe{monogène}{monogène!extension de corps} ou \defe{\wikipedia{fr}{Extension_simple}{simple}}{extension!de corps!simple}\index{simple!extension de corps} s'il existe \( \theta\in\eL\) tel que \( \eL=\eK(\theta)\). Un tel élément \( \theta\) est dit \defe{élément primitif}{primitif!élément d'une extension de corps} de \( \eL\). Il n'est pas nécessairement unique.
\end{definition}

\begin{remark}
    Les ensembles \( \eK(A)\) et \( \eK[A]\) sont aussi appelés respectivement corps \defe{engendré}{engendré!corps, extension} et anneau engendré par \( A\). Cependant il faut bien remarquer que ce sont les parties de \( \eL\) engendrées par \( A\). Il n'est pas question à priori de parler de corps engendré par \( A\) sans dire dans quel corps plus grand nous nous plaçons.
\end{remark}

\begin{example}
    Nous savons que \( \eR\) est une extension de \( \eQ\). Si \( a\in \eR\) alors \( \eQ(a)\) est le plus petit corps contenant \( \eQ\) et \( a\).
\end{example}

\begin{example}
    Nous avons déjà vu à l'occasion de la définition~\ref{DEFooFYZRooMikwEL} que \( A[X]\) est l'anneau de tous les polynômes de degré fini en \( X\). Cela rentre dans le cadre de la définition~\ref{DefZCYIbve} parce un anneau contenant \( X\) doit contenir tous les \( X^n\).

    Notons que même si \( \eK\) est un corps, \( \eK[X]\) reste un anneau parce qu'un éventuel inverse de \( X\) n'est pas dedans\footnote{Lorsqu'on multiplie, les degrés montent toujours.}. Par contre, \( \eK(X)\) est un corps parce qu'il contient également les fractions rationnelles.
\end{example}

\begin{example} \label{ExLQhLhJ}
    Si nous prenons \( \eF_5\) et que nous l'étendons par \( i\), nous obtenons le corps \( \eK=\eF_5(i)\). Nous savons que tous les éléments \( a\in \eF_5\) sont racines de \( X^5-X\). Mais étant donné que \( i^5=i\), nous avons aussi \( x^5=x\) pour tout \( x\in \eF_5(i)\). Pour le prouver, utiliser le morphisme de Frobenius. Le polynôme \( X^5-X\) est donc le polynôme nul dans \( \eK\).

    Ceci est un cas très particulier parce que nous avons étendu \( \eF_p\) par un élément \( \alpha\) tel que \( \alpha^p=\alpha\). En général sur \( \eF_p(\alpha)\), le polynôme \( X^p-X\) n'est pas identiquement nul, et possède donc au maximum \( p\) racines. Pour \( x\in \eF_p(\alpha)\), nous avons \( x^p=x\) si et seulement si \( x\in \eF_p\).
\end{example}

\begin{lemma}
    Soit \( P\in\eK[X]\) un polynôme unitaire irréductible de degré \( n\). Il existe une extension \( \eL\) de \( \eK\) et \( a\in \eL\) telle que \( \eL=\eK(a)\) et \( P\) est le polynôme minimal de \( a\) dans \( \eL\).
\end{lemma}

\begin{proof}
    Nous prenons \( \eL=\eK[X]/(P)\) où \( (P)\) est l'idéal dans \( \eK[X]\) généré par \( P\). Cela est un corps par le corolaire~\ref{CorsLGiEN}. Nous identifions \( \eK\) avec \( \phi(\eK)\) où
    \begin{equation}
        \phi\colon \eK[X]\to \eL
    \end{equation}
    est la projection canonique. Nous considérons également \( a=\phi(X)\).

    Nous avons alors \( P(a)=0\) dans \( \eL\). En effet \( P(a)=P\big( \phi(X) \big)\) est à voir comme l'application du polynôme \( P\) au polynôme \( X\), le résultat étant encore un élément de \( \eL\). En l'occurrence le résultat est \( P\) qui vaut \( 0\) dans \( \eL\).

    Le polynôme \( P\) étant unitaire et irréductible, il est minimum dans \( \eL\).

    Nous devons encore montrer que \( \eL=\eK(a)\). Le fait que \( \eK(a)\subset \eL\) est une tautologie parce qu'on calcule \( \eK(a)\) dans \( \eL\). Pour l'inclusion inverse soit \( Q(X)=\sum_iQ_iX^i\) dans \( \eK[X]\). Dans \( \eL\) nous avons évidemment \( Q=\sum_iQ_ia^i\).
\end{proof}

\begin{proposition}[\cite{ooLIOMooBuCPUS}] \label{PropyMTEbH}
    Soit \( \eK\), un corps et \( P\in \eK[X]\) un polynôme. Soient \( a\) et \( b\), deux racines de \( P\) dans (éventuellement) une extension \( \eL\) de \( \eK\). Si \( \mu_a\) et \( \mu_b\) sont les polynômes minimaux de \( a\) et \( b\) (dans \( \eK[X]\)) et si \( \mu_a\neq \mu_b\), alors \( \mu_a\mu_b\) divise \( P\) dans \( \eK[X]\).
\end{proposition}

\begin{proof}
    Nous considérons les idéaux
    \begin{subequations}
        \begin{align}
            I_a=\{ Q\in \eK[X]\tq Q(a)=0 \};\\
            I_b=\{ Q\in \eK[X]\tq Q(b)=0 \}.
        \end{align}
    \end{subequations}
    Même si \( Q(a)\) est calculé dans \( \eL\), ce sont des idéaux de \( \eK[X]\). Le polynôme \( \mu_a\) est par définition le générateur unitaire de \( I_a\), et vu que \( a\) est une racine de \( P\), nous avons \( P\in I_a\) et il existe un polynôme \( Q\in \eK[X]\) tel que
    \begin{equation}    \label{EqvTPoSq}
        P=\mu_aQ.
    \end{equation}

    Montrons que \( \mu_a(b)\neq 0\). Pour cela, nous  supposons que \( \mu_a(b)=0\), c'est-à-dire que \( \mu_a\in I_b\). Il existe alors \( R\in \eK[X]\) tel que \( \mu_a=\mu_bR\). Mais par la proposition~\ref{PropRARooKavaIT}, le polynôme \( \mu_a\) est irréductible, donc soit \( \mu_b\) soit \( R\) est inversible. Vu que les inversibles sont les éléments de \( \eK\) (polynômes de degré zéro), \( \mu_b\) n'est pas inversible (sinon il serait constant et ne pourait pas être annulateur de \( b\)). Donc \( R\) est inversible. Disons \( R=k\).

    Donc \( \mu_a=k\mu_b\). Mais vu que \( \mu_a\) et \( \mu_b\) sont unitaires, nous avons obligatoirement \( k=1\). Cela donnerait \( \mu_a=\mu_b\), ce qui est contraire aux hypothèses. Nous en déduisons que \( \mu_a(b)\neq 0\).

    Étant donné que \( \mu_a(b)\neq 0\), l'évaluation de \eqref{EqvTPoSq} en \( b\) montre que \( Q(b)=0\), de telle sorte que \( Q\in I_b\) et il existe un polynôme \( S\) tel que \( Q=\mu_bS\), c'est-à-dire tel que \( P=\mu_a\mu_bS\), ce qui signifie que \( \mu_a\mu_b\) divise \( P\).
\end{proof}

\begin{example}
    Soit \( P=(X^2+1)(X^2+2)\) dans \( \eR[X]\). Dans \( \eC\) nous avons les racines \( a=i\) et \( b=\sqrt{2}i\) dont les polynômes minimaux sont \( \mu_a=X^2+1\) et \( \mu_2=X^2+2\). Nous avons effectivement \( \mu_a\mu_b\) divise \( P\) dans \( \eR[X]\).

    Si par contre nous considérions les racines \( a=i\) et \( b=-i\), nous aurions \( \mu_a=\mu_b=X^2+1\), tandis que le polynôme \( \mu_a^2\) ne divise pas \( P\).
\end{example}

%---------------------------------------------------------------------------------------------------------------------------
\subsection{Racines de polynômes}
%---------------------------------------------------------------------------------------------------------------------------

\begin{corollary}[Factorisation d'une racine]   \label{CorDIYooEtmztc}
    Soit \( P\in \eK[X]\), un polynôme de degré \( n\) et \( \alpha\in \eK\) tel que \( P(\alpha)=0\). Alors il existe un polynôme \( Q\) de degré \( n-1\) tel que \( P(x)=(X-\alpha)Q\).
\end{corollary}
\index{factorisation!de polynôme}

\begin{proof}
    Il s'agit d'un cas particulier de la proposition~\ref{PropXULooPCusvE} : si \( \alpha\in \eK\) alors son polynôme minimal dans \( \eK\) est \( X-\alpha\); donc \( X-\alpha\) divise \( P\). Il existe un polynôme \( Q\) tel que \( P=(X-\alpha)Q\). Le degré est alors immédiat.
\end{proof}

Avant de lire l'énoncé suivant, allez relire la définition \ref{NORMooQFTJooLBcPxl} pour savoir ce qu'est un polynôme nul.
\begin{theorem}[Polynôme qui a tellement de racines qu'il s'annule]\label{ThoLXTooNaUAKR}
    Soit \( \eK\) un corps et \( P\in \eK[X]\) un polynôme de degré \( n\) possédant \( n+1\) racines distinctes \( \alpha_1\),\ldots, \( \alpha_{n+1}\), alors \( P=0\).
\end{theorem}
\index{racine!de polynôme}

\begin{proof}
    Si \( P\) est de degré \( 1\), il s'écrit \( P=aX+b\); s'il a comme racines \( \alpha\) et \( \beta\), nous avons le système
    \begin{subequations}
        \begin{numcases}{}
            a\alpha+b=0\\
            a\beta+b=0.
        \end{numcases}
    \end{subequations}
    La différence entre les deux donne \( a(\alpha-\beta)=0\). Vu que \( \alpha\neq \beta\), la règle du produit nul (lemme~\ref{LemAnnCorpsnonInterdivzer}) nous donne \( a=0\). Maintenant que \( a=0\), l'annulation de \( b\) est alors immédiate.

    Nous faisons maintenant la récurrence en supposant le théorème vrai pour le degré \( n\) et en considérant un polynôme \( P\) de degré \( n+1\) possédant \( n+2\) racines distinctes. Vu que \( P(\alpha_1)=0\), le corolaire~\ref{CorDIYooEtmztc} nous donne un polynôme \( Q\) de degré \( n\) tel que
    \begin{equation}    \label{EqQGSooNdTWfz}
        P=(X-\alpha_1)Q.
    \end{equation}
    Étant donne que pour tout \( i\neq 1\) nous avons \( \alpha_i\neq \alpha_1\),
    \begin{equation}
        0=P(\alpha_i)=\underbrace{(\alpha_i-\alpha_1)}_{\neq 0}Q(\alpha_i),
    \end{equation}
    et la règle du produit nul donne \( Q(\alpha_i)=0\). Par conséquent le polynôme \( Q\) est de degré \( n\) et possède \( n+1\) racines distinctes; tous ses coefficients sont alors nuls par hypothèse de récurrence. Tous les coefficients du produit \eqref{EqQGSooNdTWfz} sont alors également nuls.
\end{proof}

\begin{example}\label{ExGRHooBNpjSP}
    Un polynôme à plusieurs variables peut s'annuler en une infinité de points sans être nul. Par exemple le polynôme \( X^2+Y^2-1\in\eR[X,Y]\) s'annule sur tout un cercle de \( \eR^2\) mais n'est pas nul, loin s'en faut.

    Nous verrons dans la proposition~\ref{PropTETooGuBYQf} une condition pour qu'un polynôme à plusieurs variables s'annule du fait qu'il ait «trop» de racines.
\end{example}

\begin{remark}
    L'intérêt du théorème~\ref{ThoLXTooNaUAKR} est que si l'on prouve qu'un polynôme s'annule sur un corps infini, alors il s'annulera sur n'importe quel autre corps. Nous aurons un exemple d'utilisation de cela dans le théorème de Cayley-Hamilton~\ref{ThoHZTooWDjTYI}.
\end{remark}

%---------------------------------------------------------------------------------------------------------------------------
\subsection{Corps de rupture}
%---------------------------------------------------------------------------------------------------------------------------

\begin{definition}      \label{DEFooVALTooDJJmJv}
    Soit \( P\in\eK[X]\) un polynôme irréductible. Une extension \( \eL\) de \( \eK\) est un \defe{corps de rupture}{corps!de rupture}\index{rupture!corps} pour \( P\) s'il existe \( a\in \eL\) tel que \( P(a)=0\) et \( \eL=\eK(a)\).
\end{definition}

\begin{normaltext}      \label{NORMALooTPOIooVZAfUo}
    Nous insistons sur le fait que nous ne définissons le concept de corps de rupture pour un polynôme irréductible à coefficients dans un corps. Les deux points sont importants : irréductible et à coefficient dans un corps.

    Nous discuterons brièvement le pourquoi de cela dans la section~\ref{SUBSECooEDMJooTXBfOu} et surtout dans la question~\ref{ITEMooUBZIooDDcfWg} des questions difficiles d'algèbre.
\end{normaltext}

\begin{example}     \label{ExemGVxJUC}
    Soit \( \eK=\eQ\) et \( P=X^2-2\). On pose \( a=\sqrt{2}\) et \( \eL=\eQ(\sqrt{2})\subset\eR\). De cette façon \( P\) est scindé dans \( \eL \):
    \begin{equation}
        P=(X-\sqrt{2})(X+\sqrt{2}).
    \end{equation}
    Le corps \( \eQ(\sqrt{2})\) est donc un corps de rupture pour \( P\).
\end{example}

\begin{example}
    Dans l'exemple~\ref{ExemGVxJUC}, nous avions un corps de rupture dans lequel le polynôme \( P\) était scindé. Il n'en est pas toujours ainsi. Prenons
    \begin{equation}
        P=X^3-2
    \end{equation}
    et \( a=\sqrt[3]{2}\). Nous avons, certes, \( P(a)=0\) dans \( \eQ(\sqrt[3]{2})\), mais \( P\) n'est pas scindé parce qu'il y a deux racines complexes.
\end{example}

\begin{example}
    Nous considérons le corps \( \eZ/p\eZ\) où \( p\) est un nombre premier. Si \( s\in \eZ/p\eZ\) n'est pas un carré, alors le polynôme \(P= X^2+s\) est irréductible et un corps de rupture de \( P\) sur \( \eZ/p\eZ\) est donné par \( (\eZ/p\eZ)[X]/(X^2+s)\), c'est-à-dire l'ensemble des polynômes de degré \( 1\) en \( \sqrt{s}\). Le cardinal en est \( p^2\).
\end{example}

Vu que nous allons abondamment parler du quotient \( \eK[X]/(P)\), nous nous permettons un petit lemme.
\begin{lemma}       \label{LEMooWYYFooXYacdF}
    Soit un corps \( \eK\) et \( P\in \eK[X]\) non constant. Alors \( \eK[X]/(P)\) est un corps si et seulement si \( P\) est irréductible.
\end{lemma}

\begin{proof}
    Nous utilisons le trio d'enfer dont il est question dans le thème~\ref{THEMEooZYKFooQXhiPD}. D'abord \( \eK[X]\) est un anneau principal par le lemme~\ref{LEMooIDSKooQfkeKp}. Donc \( \eK[X]/(P)\) sera un corps si et seulement si \( (P)\) est un idéal maximum (proposition~\ref{PROPooSHHWooCyZPPw}), et cela sera le cas si et seulement si \( (P)\) est engendré par un polynôme irréductible (proposition~\ref{PropomqcGe}).

    Il ne nous reste qu'à montrer que \( (P)\) est engendré par un irréductible si et seulement si \( P\) est irréductible. Il y a un sens dans lequel c'est évident.

    Soit un irréductible \( \mu\) tel que \( (P)=(\mu)\). En particulier \( \mu\in (P)\), c'est-à-dire qu'il existe \( Q\) tel que \( \mu=PQ\). Vu que \( \mu\) est irréductible, soit \( P\) soit \( Q\) est inversible. Si \( P\) est inversible, c'est-à-dire constant, ce que nous avons exclu par hypothèse. Si par contre \( Q\) est inversible, alors \( P=k\mu\) pour un certain \( k\in \eK\), ce qui montre que \( P\) est irréductible autant que \( \mu\).
\end{proof}

\begin{proposition}[Existence d'un corps de rupture]        \label{PROPooUBIIooGZQyeE}
    Soit un corps \( \eK\) et un polynôme irréductible non constant \( P\). Alors
    \begin{enumerate}
        \item
            Le corps \( \eL=\eK[X]/(P)\) est un corps de rupture pour \( P\).
        \item
            L'élément \( \bar X\) de \( \eL\) est une racine de \( P\).
        \item
            \( \eL=\eK(\bar X)_{\eL}\)
    \end{enumerate}
\end{proposition}

\begin{proof}
    Commençons par nous convaincre que \( \eK[X]/(P)\) est une extension de \( \eK\) (définition~\ref{DEFooFLJJooGJYDOe}). Le fait que ce soit un corps est le lemme~\ref{LEMooWYYFooXYacdF}. Le morphisme \( j\colon \eK\to \eK[X]/(P)\) est simplement \( k\mapsto \bar k\) où à droite, \( \bar k\) voit \( k\) dans \( \eK[X]\) comme étant le polynôme constant. Notez qu'il est automatiquement injectif (lemme~\ref{LEMooWBOPooZnsZgQ}).

    Il faut maintenant voir que \( \eK[X]/(P)=\eK(\alpha)\) pour un certain \( \alpha\in \eK[X]/(P)\). Grâce à notre compréhension des notations acquise dans~\ref{SUBSUBSECooPNBYooWXEHrg}, nous savons que \( X\in\eK[X]\) et qu'il est donc parfaitement légitime de poser \( \alpha=\bar X\) dans \( \eK[X]/(P)\). Il s'agit simplement de l'ensemble \( \bar X=\{ X+QP\tq Q\in \eK[X] \}\) où \( X\) est une notation pour la suite \( (0,1,0,0,\ldots)\).

    Bref, nous notons \( \alpha=\bar X\) et nous démontrons que \( P(\alpha)=0\) et que \( \eK[X]/(P)=\eK(\alpha)\) (isomorphisme de corps).
    \begin{subproof}
        \item[\( P(\bar X)=0\)]

            C'est le moment de nous souvenir comment la notation des \( X\) fonctionne, et en particulier la pirouette autour de \eqref{EQooABULooFCEasf}. D'abord la définition du produit sur \( \eK[X]/(P)\) est \( \bar P\bar Q=\overline{ PQ }\); en particulier si \( P=\sum_ka_kX^k\), alors \( P(\bar X)=\sum_ka_k\bar X^k=\sum_ka_k\overline{ X^k }\), et
            \begin{equation}
                P(\bar X)=\overline{ P(X) }=\bar P=0.
            \end{equation}
        \item[L'égalité]

            Nous montrons à présent que \( \eK(\bar X)_{\eL}=\eL\). C'est-à-dire que \( \eL\) est bien engendrée par \( \eK\) et un seul élément. D'abord, \( \eL=\eK[X]/(P)\) contient bien évidemment \( \eK\) et \( \bar X\). Ensuite nous devons prouver que tout sous-corps de \( \eL\) contenant \( \eK\) et \( \bar X\) est en réalité \( \eL\) entier.

            Soit \( Q\in \eK[X]\), et montrons que \( \bar Q\) est dans tout sous-corps de \( \eL\) contenant \( \eK\) et \( \bar X\).

            Par le lemme~\ref{LEMooXFMAooMBgIrN} nous avons \( \bar Q=Q(\bar X)\). Et si un corps contient \( \eK\) et \( \bar X\), il doit contenir tous les polynômes en \( \bar X\) à coefficients dans \( \eK\). Donc un tel corps doit contenir \( Q(\bar X)\) et donc \( \bar Q\).

    \end{subproof}
\end{proof}

\begin{example}
    Soit le polynôme \( P=X^2+1\in \eZ[X]\). Dans le quotient \( \eZ[X]/(P)\) nous avons \( \bar X^2+1=0\) et donc \( \bar X^2=-1\). C'est-à-dire que \( \eZ[X]/(P)\) contient un élément dont le carré est \( -1\). Avouez que c'est bien ce à quoi nous nous attendions.

    Notons que \( -\bar X\) est également une racine de \( P\) dans \( \eZ[X]/(P)\).

    En calculant dans les polynômes à coefficients dans \( \eZ(\bar X)\) nous avons :
    \begin{equation}
        (X+\bar X)(X-\bar X)=X^2-\bar X^2=X^2+1,
    \end{equation}
    c'est-à-dire que \( P\) est bien factorisé, et que nous avons retrouvé la multiplication \( x^2+1=(x+i)(x-i)\).
\end{example}

\begin{normaltext}
    Il n'y a évidemment pas unicité d'un corps de rupture pour un polynôme donné. Une raison est qu'un polynôme peut accepter plusieurs racines complètement indépendantes. Le corps étendu par l'une ou l'autre racine donne deux corps de rupture différents. Par exemple dans \( \eQ[X]\), le polynôme
    \begin{equation}
        P=X^4-X^2-2
    \end{equation}
    a pour racines (dans \( \eC\)) les nombres \( \sqrt{ 2 }\) et \( i\). Donc on a deux corps de rupture complètement différents : \( \eQ(\sqrt{ 2 })\) et \( \eQ(i)\).
\end{normaltext}

\begin{normaltext}
    La proposition suivante donne une unicité du corps de rupture dans le cas d'un polynôme irréductible. Et nous comprenons pourquoi : un polynôme irréductible n'a fondamentalement qu'une seule racine «indépendante». Par exemple \( X^2-2\) a pour racines \( \pm\sqrt{ 2 }\). Autre exemple, le polynôme \( X^2+6X+13\) a pour racines, dans \( \eC\), les nombres complexes conjugués \( z=-3+2i\) et \( \bar z=-3-2i\).
\end{normaltext}

\begin{proposition}[\cite{ooUHHUooONXDDl}]          \label{PROPooVJACooNDmlfb}
    Soient un corps \( \eK\) et un polynôme irréductible \( P\in \eK[X]\). Alors toute extension \( \eL\) contenant une racine \( \alpha\) de \( P\) admet un unique morphisme de corps
            \begin{equation}
                \psi\colon \eK[X]/(P)\to \eL
            \end{equation}
            tel que \( \psi(\bar X)=\alpha\).

    Dans un tel cas,
    \begin{enumerate}
        \item
            l'image de \( \psi\) est $\eK(\alpha)_{\eL}$ ,
        \item       \label{ITEMooHRFHooWLIdWU}
            si \( \eL=\eK(\alpha)_{\eL}\) alors \( \psi\) est un isomorphisme.
    \end{enumerate}

\end{proposition}

\begin{proof}
    L'idéal annulateur de \( \alpha\) parmi les polynôme de \( \eK[X]\) n'est pas réduit à \( \{ 0 \}\) parce qu'il contient \( P\). Le lemme~\ref{DefCVMooFGSAgL} s'applique donc et nous avons le polynôme minimal \( \mu\) de \( \alpha\) dans \( \eK[X]\). Il divise \( P\) qui est irréductible, donc
    \begin{equation}
        P=\lambda \mu
    \end{equation}
    pour un certain \( \lambda\in \eK\).

    Nous posons
    \begin{equation}
        \begin{aligned}
            \psi\colon \eK[X]/(P)&\to \eL \\
            \bar Q&\mapsto Q(\alpha).
        \end{aligned}
    \end{equation}
    \begin{subproof}
        \item[Bien définie]
            Si \( \bar Q_1=\bar Q_2\) alors il existe un \( R\in \eK[X]\) tel que \( Q_1=Q_2+RP\). Mais alors \( \psi(\bar Q_1)=Q_1(\alpha)=Q_2(\alpha)+R(\alpha)P(\alpha)=Q_2(\alpha)\).
        \item[Injective]

            Si \( \psi(\bar Q_1)=\psi(\bar Q_2)\) alors \( Q_1-Q_2=R\) pour un certain \( R\in \eK[X]\) vérifiant \( R(\alpha)=0\). Nous avons alors un polynôme \( S\) tel que \( R=S\mu=\lambda^{-1}SP\). Donc \( \bar R=0\) et donc \( \bar Q_1=\bar Q_2\).

        \item[Morphisme]

            Laissé comme exercice; la paresse de l'auteur de ces lignes attend vos contributions.

        \item[La condition]

            Le morphisme \( \psi\) respecte de plus la condition
            \begin{equation}
                \psi(\bar X)=X(\alpha)=\alpha.
            \end{equation}

    \end{subproof}

    En ce qui concerne l'unicité, fixer \( \psi(\bar X)\) est suffisant pour fixer un morphisme. En effet si \( \psi(\bar X)=\alpha\), alors
    \begin{equation}
        \psi(\bar Q)=\psi\Big( \sum_ka_k\bar X^k \Big)=\sum_ka_k\psi(\bar X)^k=\sum_ka_k\alpha^k.
    \end{equation}

    Pour le second point de l'énoncé, il faut remarquer que \( \alpha\) est algébrique et non transcendant. Donc en utilisant les propositions~\ref{PROPooPMNSooOkHOxJ} et~\ref{PropURZooVtwNXE}\ref{ItemJCMooDgEHaji} nous trouvons
    \begin{equation}
        \Image(\psi)=\{ Q(\alpha)\tq Q\in \eK[X] \}=\eK[\alpha]=\eK(\alpha).
    \end{equation}

    Et finalement pour le dernier point, un morphisme de corps est toujours injectif. Si il est également surjectif, il sera bijectif.
\end{proof}

%---------------------------------------------------------------------------------------------------------------------------
\subsection{Pile d'extensions}
%---------------------------------------------------------------------------------------------------------------------------

\begin{lemma}[\cite{MonCerveau}]        \label{LEMooTURZooXnjmjT}
    Soient un corps \( \eK\), des extensions \( \eL_1\),\ldots, \( \eL_n\) et des éléments \( \alpha_i\in \eL_i\) tels que
    \begin{equation}    \label{EQooOCQSooFMkzTc}
        \eL_1=\eK(\alpha_1)_{\eL_1}
    \end{equation}
    et
    \begin{equation}
        \eL_k=\eL_{k-1}(\alpha_k)_{\eL_k}.
    \end{equation}
    Alors
    \begin{equation}
        \eL_n=\eK(\alpha_1,\ldots, \alpha_n)_{\eL_n}.
    \end{equation}
\end{lemma}

\begin{proof}
    Nous démontrons par récurrence sur \( n\). Le cas \( n=1\) est simplement l'hypothèse \eqref{EQooOCQSooFMkzTc}. 
    
    Supposons donc que le lemme soit correct pour \( n\), et étudions le cas \( n+1\). Nous avons, par définition et par hypothèse de récurrence :
    \begin{equation}
        \eL_{n+1}=\eL_n(\alpha_{n+1})_{\eL_{n+1}}=\Big( \eK(\alpha_1,\ldots, \alpha_n)_{\eL_n} \Big)(\alpha_{n+1})_{\eL_{n+1}}.
    \end{equation}
    Notre tâche sera donc de montrer que
    \begin{equation}\label{EQooIHMGooTlPcsd}
        \Big( \eK(\alpha_1,\ldots, \alpha_n)_{\eL_n} \Big)(\alpha_{n+1})=\eK(\alpha_1,\ldots,\alpha_{n+1})
    \end{equation}
    où nous n'écrivons plus les indices \( \eL_{n+1}\) partout.

    Le membre de gauche est un sous-corps de \( \eL_{n+1}\) contenant à la fois \( \eK \) et tous les \(\alpha_i \), si bien que
    \begin{equation}\label{EQooLLRHooHOjLfk}
        \eK(\alpha_1,\ldots,\alpha_{n+1})\subset \big( \eK(\alpha_1,\ldots, \alpha_n)_{\eL_n} \big)(\alpha_{n+1})_{\eL_{n+1}}.
    \end{equation}

    Il faut donc prouver l'inclusion inverse; c'est-à-dire montrer que tout élément \( x \) du corps \( \big( \eK(\alpha_1,\ldots, \alpha_n)_{\eL_n} \big)(\alpha_{n+1})\) est forcément dans tout sous-corps de \( \eL_{n+1}\) contenant \( \eK\) et les \( \alpha_i\). Un tel élément \( x \) est, par la proposition~\ref{PROPooYSFNooFGbbCi}\ref{ITEMooATPTooVXKdlK}, de la forme \( r(\alpha_{n+1})\) avec \( r\in \eK(\alpha_1,\ldots, \alpha_{n})(X)\), c'est-à-dire
            \begin{equation}
                P(\alpha_{n+1})Q(\alpha_{n+1})^{-1}
            \end{equation}
            avec \( P,Q\in \eK(\alpha_1,\ldots, \alpha_n)[X]\).

            Prouvons d'abord que si \( P\in \eK(\alpha_1,\ldots, \alpha_n)[X]\), alors \( P(\alpha_{n+1})\) est dans tout sous-corps de \( \eL_{n+1}\) contenant \( \eK\) et les \( \alpha_i\). Nous pouvons écrire \( P=\sum_ia_iX^i\) avec \( a_i\in \eK(\alpha_1,\ldots, \alpha_n)\), et donc
            \begin{equation}
                P(\alpha_{n+1})=\sum_ia_i\alpha_{n+1}^i.
            \end{equation}
            Tout corps contenant \( \eK\) et les \( \alpha_1\),\ldots, \( \alpha_n\) contient les \( a_i\). Par produit, tout corps contenant \( \eK\), \( \alpha_1\),\ldots,  \( \alpha_{n+1}\) contient les termes \( a_i\alpha_{n+1}^i\), et donc \( P(\alpha_{n+1})\) par somme.

        De la même façon, si un corps contient \( \eK\) et les \( \alpha_i\)  (\( i=1,\ldots, n+1\)), alors il contient \( Q(\alpha_{n+1})\). Comme c'est un corps, il contient aussi son inverse \( Q(\alpha_{n+1})^{-1}\), et il contient aussi le produit
            \begin{equation}
                r(\alpha_{n+1})=P(\alpha_{n+1})Q(\alpha_{n+1})^{-1}.
            \end{equation}

On vient ainsi de montrer que tout élément \( x \in  \big( \eK(\alpha_1,\ldots, \alpha_n)_{\eL_n} \big)(\alpha_{n+1})\) était dans tout sous-corps de \( \eL_{n+1} \) qui contient \( \eK \) et les \( \alpha_i\)  (\( i=1,\ldots, n+1\)); en d'autres termes:
    \begin{equation}\label{EQooUFSMooJozpqL}
       \big( \eK(\alpha_1,\ldots, \alpha_n)_{\eL_n} \big)(\alpha_{n+1})_{\eL_{n+1}} \subset \eK(\alpha_1,\ldots,\alpha_{n+1}).
    \end{equation}
    Les inclusions \eqref{EQooLLRHooHOjLfk} et \eqref{EQooUFSMooJozpqL} prouvent l'égalité d'ensembles \eqref{EQooIHMGooTlPcsd} que nous voulions montrer.
\end{proof}

%--------------------------------------------------------------------------------------------------------------------------- 
\subsection{Clôture algébrique}
%---------------------------------------------------------------------------------------------------------------------------

Le concept de clôture algébrique a été défini dans \ref{DEFooREUHooLVwRuw}. Voici un lemme qui dit qu'une clôture algébrique est en quelque sorte une extension algébrique maximale.
\begin{lemma}[\cite{MonCerveau}]        \label{LEMooQSCGooMyCktA}
    Soient un corps \( \eK\) et une extension algébrique \( \eF\) de \( \eK\). Nous supposons que pour toute extension algébrique de \( \eL\) nous avons \( \eL=\eF\)

    Alors \( \eF\) est algébriquement clos\footnote{Définition \ref{DEFooREUHooLVwRuw}\ref{ITEMooEIWVooVjJRoR}.}.
\end{lemma}

\begin{proof}
    Soit un polynôme \( P\in \eK[X]\). Nous voudrions prouver que \( P\) a des racines dans \( \eF\). Pour cela, nous voyons \( P\) comme un polynôme sur \( \eF[X]\) et, grace à la proposition \ref{PROPooUBIIooGZQyeE} nous considérons un corps de rupture \( \eL\) pour \( P\). Vu que \( \eL\) est une extension de \( \eF\), nous avons \( \eL=\eF\). Donc \( \eF\) contient des racines de \( P\).
\end{proof}


\begin{normaltext}
    Nous avons défini le concept d'extension algébrique en \ref{DEFooREUHooLVwRuw}. Nous allons en construire un petit exemple très piéton.

    D'abord la proposition \ref{PROPooUHKFooVKmpte} nous donne l'existence et l'unicité d'un réel \( \sqrt{ 2 }\) strictement positif dont le carré est \( 2\). Ce réel est irrationnel par la proposition \ref{PropooRJMSooPrdeJb}. Cela étant posé, nous y allons.
\end{normaltext}

\begin{proposition}[\cite{BIBooAOINooYRFZFb}]
    Soit \( \eL=\{ a+b\sqrt{ 2 } \}_{a,b\in \eQ}\).
    \begin{enumerate}
        \item
            C'est un sous-corps de \( \eR\).
        \item   \label{ITEMooUSOAooZoBhla}
            Tout sous-corps de \( \eR\) contenant \( \eQ\) et \( \sqrt{ 2 }\) contient \( \eL\).
    \end{enumerate}
\end{proposition}

\begin{proof}
    Nous devons d'abord prouver que \( \eL\) est un corps en vérifiant d'une part que c'est un anneau (définition \ref{DefHXJUooKoovob}) et d'autre part le fait que tous les éléments non nuls sont inversibles.
    \begin{itemize}
        \item La partie \( \eL\) de \( \eR\) est stable pour l'addition : dès que \( a,b,a',b'\in \eQ\),
            \begin{equation}
                (a+b\sqrt{ 2 })+(a'b'\sqrt{ 2 })=(a+a')+(b+b')\sqrt{ 2 }\in \eL.
            \end{equation}
        \item
            Les neutres \( 0\) et \( 1\) sont dans \( \eL\).
        \item
            Si \( \alpha\in \eL\), alors \( -\alpha\in \eL\) :
            \begin{equation}
                -(a+b\sqrt{ 2 })=-a-b\sqrt{ 2 }.
            \end{equation}
        \item
            La partie \( \eL\) est stable pour le produit parce que
            \begin{equation}
                (a+b\sqrt{ 2 })(a'+b'\sqrt{ 2 })=(aa'+2bb')+(ab'+ba')\sqrt{ 2 }.
            \end{equation}
        \item
            L'inverse d'un élément de \( \eL\) est dans \( \eL\). C'est le seul point pas tout à fait évident. D'abord, l'ensemble \( \eR\) est un corps par le théorème \ref{DefooFKYKooOngSCB}. Donc pour tout \( a,b\in \eR\), le nombre
            \begin{equation}
                \frac{1}{ a+b\sqrt{ 2 } }
            \end{equation}
            existe dans \( \eR\).
            
            D'abord \( a-b\sqrt{ 2 }\) n'est pas nul, parce que si il l'était, nous aurions \( \sqrt{ 2 }=-a/b\in \eQ\) alors que \( \sqrt{ 2 }\) n'est pas rationnel par la proposition \ref{PropooRJMSooPrdeJb}. Nous pouvons donc faire le coup de multiplier le numérateur et le dénominateur par le binôme conjugué :
            \begin{equation}
                \frac{1}{ a+b\sqrt{ 2 } }=\frac{ a-b\sqrt{ 2 } }{ (a+b\sqrt{ 2 })(a-b\sqrt{ 2 }) }=\frac{ a }{ a^2-2b^2 }-\frac{ b }{ a^2-2b^2 }\sqrt{ 2 }.
            \end{equation}
            Cela est un rationnel. Donc le éléments non nuls de \( \eL\) ont un inverse qui appartient également à \( \eL\).
    \end{itemize}
    Nous passons à la preuve du point \ref{ITEMooUSOAooZoBhla}. Si \( \eL'\) est un corps qui contient \( \eQ\) et \( \sqrt{ 2 }\), il doit contenir \( b\sqrt{ 2 }\) pour tout \( b\in \eQ\) et donc aussi tous les \( a+b\sqrt{ 2 }\) avec \( a,b\in \eQ\). En conséquence de quoi \( \eL'\) doit contenir au moins tout \( \eL\).
\end{proof}

\begin{proposition}
    Soit \( \eL=\{ a+b\sqrt{ 2 } \}_{a,b\in \eQ}\).
    \begin{enumerate}
        \item   \label{ITEMooOMDMooLNhlyh}
            C'est un espace vectoriel de dimension \( 2\) sur \( \eQ\).
        \item       \label{ITEMooWGGDooSbsesf}
            Si \( \alpha\in \eL\), alors il existe un polynôme \( P\in \eL[X]\) de degré \( 2\) ou moins tel que \( P(\alpha)=0\).
        \item   \label{ITEMooPNNYooPtKYwQ}
            Le corps \( \eL\) est une extension algébrique de \( \eQ\).
    \end{enumerate}
\end{proposition}

\begin{proof}
    En plusieurs parties.
    \begin{subproof}
        \item[\ref{ITEMooOMDMooLNhlyh}]
            Pour la dimension, notez que \( \{ 1,\sqrt{ 2 } \}\) est une partie libre et génératrice de \( \eL\).

        \item[\ref{ITEMooWGGDooSbsesf}]

            Soit \( \alpha\in \eL\). La partie \( \{ 1,\alpha,\alpha^2 \}\) est de cardinal \( 1\), \( 2\) ou \( 3\). Si c'est \( 1\) ou \( 2\), c'est que \( 1=\alpha\) ou \( 1=\alpha^2\) ou \( \alpha=\alpha^2\). Si par exemple \( 1=\alpha\) alors avec \( P=X-1\) nous avons \( P(\alpha)=0\).

            Si au contraire \( \{ 1,\alpha,\alpha^2 \}\) est de cardinal \( 3\), alors c'est une partie liée par la proposition \ref{PROPooEIQIooXfWDDV}. Il existe donc des rationnels \( a,b,c\) tels que \( a+b\alpha+c\alpha^2=0\), c'est-à-dire \( P(\alpha)=0\) avec \( P=cX^2+bX+a\).
        \item[\ref{ITEMooPNNYooPtKYwQ}]
            Nous venons de voir que tous les éléments de \( \eL\) sont des racines de polynômes de \( \eQ[X]\).
    \end{subproof}
\end{proof}

\begin{lemma}       \label{LEMooHWPHooZeWqns}
    Si \( \eK\) est un corps infini, alors \( \eK[X]\) est équipotent\footnote{Définition \ref{DEFooXGXZooIgcBCg}.} à \( \eK\).
\end{lemma}

\begin{proof}
    Notons provisoirement \( \eK_n[X]\) l'ensemble des polynômes de degré \( n\). Nous avons une surjection
    \begin{equation}    \label{EQooFGZVooKIMKRA}
        \begin{aligned}
            \varphi\colon \eK^{n+1}&\to \eK_n[X] \\
            (k_0,\ldots, k_n)&\mapsto \sum_{i=0}^nk_iX^i. 
        \end{aligned}
    \end{equation}
    Par récurrence sur le théorème\quext{J'ai quand même du mal à croire qu'il faut vraiment le lemme de Zorn pour prouver que \( \eK[X]\) est équipotent à \( \eK\). Si vous connaissez un moyen plus direct, écrivez-moi.} \ref{THOooDGOVooRdURVi}, nous avons \( \eK^{n+1}\approx \eK\). La surjection \eqref{EQooFGZVooKIMKRA} dit alors que
    \begin{equation}
        \eK_n[X]\preceq \eK^{n+1}\approx \eK.
    \end{equation}
    Mais vu qu'il y a une surjection \( \eK\to \eK_n[X]\), nous avons aussi \( \eK_n[X]\succeq \eK\). Le théorème \ref{THOooRYZJooQcjlcl} dit alors que \( \eK_n[X]\approx \eK\).

    Le lemme \ref{LEMooNKKDooUvSYPO} nous permet alors de conclure que
    \begin{equation}
        \eK[X]=\bigcup_{n=0}^{\infty}\eK_n[X]\approx \eK.
    \end{equation}
\end{proof}

\begin{proposition}[\cite{MonCerveau}]      \label{PROPooVPQFooScWvkS}
    Soit un corps \( \eK\). Une extension algébrique de \( \eK\) est
    \begin{enumerate}
        \item
            au plus dénombrable si \( \eK\) est fini,
        \item
            équipotente à \( \eK\) si \( \eK\) est infini.
    \end{enumerate}
\end{proposition}

\begin{proof}
    En deux parties.
    \begin{subproof}
        \item[Si \( \eK\) est fini]
            Un polynôme non nul possède toujours au maximum un nombre fini de racines (éventuellement zéro) par la proposition \ref{ThoLXTooNaUAKR}. Par ailleurs, chaque degré de polynôme ayant seulement un nombre fini de possibilités, l'ensemble \( \eK[X]\) est au plus dénombrable (proposition \ref{PROPooENTPooSPpmhY}).

            Pour \( P\in \eK[X]\) nous avons une surjection de \( \eN\) vers l'ensemble des racines de \( P\). Nous la notons \( \varphi_P\colon \eN\to \eL\), en posant par exemple \( \varphi_P(n)=1\) si \( P\) n'a pas de racines. Enfin nous posons
            \begin{equation}
                \begin{aligned}
                    \varphi\colon \eK[X]\times \eN&\to \eL \\
                    (P,n)&\mapsto \varphi_P(n). 
                \end{aligned}
            \end{equation}
            C'est la fonction qui à un polynôme \( P\) et un nombre \( n\) fait correspondre la \( n\)\ieme racine de \( P\).

            Vu que \( \eL\) est une extension algébrique, \( \varphi\) est surjective.

            En termes de cardinalité, que \( \eK[X]\) soit fini ou dénombrable, dans les deux cas, \( \eK[X]\times \eN\) est dénombrable (proposition \ref{PROPooLPKUooAlsYJg}). Il existe donc une surjection d'un ensemble dénombrable vers \( \eL\). Le lemme \ref{LEMooDLWFooNAJbbq} conclu que \( \eL\) est fini ou dénombrable.
        \item[Si \( \eK\) est infini]
            Nous procédons de la même manière, mais nous devons faire appel à des résultats plus technologiques pour maitriser la cardinalité. Nous considérons à nouveau l'application
            \begin{equation}
                \begin{aligned}
                    \varphi\colon \eK[X]\times \eN&\to \eL \\
                    (P,n)&\mapsto \varphi_P(n).
                \end{aligned}
            \end{equation}
            Cette application est encore surjective : \( \eL\preceq \eK[X]\times \eN\). Le lemme \ref{LEMooHWPHooZeWqns} nous assure que \( \eK[X]\approx \eK\) parce que \( \eK\) est infini. Ensuite la proposition \ref{PROPooFKBEooKXqujV} nous dit que \( \eK[X]\times \eN\approx \eK[X]\). Donc
            \begin{equation}
                \eK\approx \eK[X]\approx \eK[X]\times \eN\succeq \eF.
            \end{equation}
            Mais \( \eF\) est une extension de \( \eK\). Donc il existe une injection \( \eK\to \eF\), c'est à dire \( \eK\preceq \eF\).

            Vu que \( \eK\preceq \eF\preceq \eK\), le théorème \ref{THOooRYZJooQcjlcl} implique que \( \eK\approx \eF\).
    \end{subproof}
\end{proof}

\begin{lemma}[\cite{MonCerveau}]
    Soient des corps \( \eK\) et \( \eL\) ainsi qu'un homomorphisme de corps \( \rho \colon \eK\to \eL\). Si \( P\in \eK[X]\) a une racine dans \( \eK\), alors le polynôme \( \rho(P)\) a une racine dans \( \eL\).
\end{lemma}

\begin{proof}
    Nous notons \( P=\sum_kP_kX^k\). Si \( a\in \eK\) est une racine de \( P\), alors \( \sum_kP_ka^k=0\). Nous appliquons \( \rho\) à cette égalité : \( \sum_k\rho(P_k)\rho(a)^k=0\), c'est à dire \( \rho(P)\big( \rho(a) \big)=0\). Donc \( \rho(a)\in \eL\) est une racine de \( \rho(P)\).
\end{proof}

\begin{lemma}[\cite{BIBooFQQWooIuoZyf}] % ooLLYUooBfIjxj
    Soit un corps algébriquement clos \( \eF\). Soient des corps \( \eK\) et \( \eL\) tels que
    \begin{enumerate}
        \item
            \( \eK\) est un sous-corps de \( \eF\)
        \item
            \( \eL\) est une extension algébrique de \( \eK\).
    \end{enumerate}
    Alors il existe un morphisme \( \varphi\colon \eL\to \eF\) tel que \( \varphi|_{\eK}=\id\).
\end{lemma}

\begin{proof}
    Nous allons utiliser le lemme de Zorn \ref{LemUEGjJBc} sur l'ensemble
    \begin{equation}
        \mA=\Big\{  (\eM,f)  \tq
        \begin{cases}
            \eM\text{ est un sous-corps de } \eL\\
            \eK\subset \eM\\
            f\colon \eM\to \eF \text{ vérifie } f|_{\eK}=\id.
        \end{cases}
    \Big\}
    \end{equation}
    Nous ordonnons \( \mA\) en disant que \( (\eM_1,f_1)<(\eM_2,f_2)\) lorsque \( \eM_1\subset \eM_2\) et \( f_2|_{\eM}=f_1\).

    Si \( \mF=\{ (\eM_i,f_i) \}_{i\in I}\) est une partie totalement ordonnée de \( \mA\), nous construisons un majorant en posant \( \eM=\bigcup_{i\in I}\eM_i\) et \( f\colon \eM\to \eF\) définie par \( f|_{\eM_i}=f_i\).
\end{proof}

\begin{lemma}[\cite{BIBooYYZPooMgFzkp}]     \label{LEMooIIKYooHMNqYn}
    Nous considérons un triple \( (\eK,\eL,\eF)\) où
    \begin{enumerate}
        \item
            \( \eK\), \( \eL\) et \( \eF\) sont des corps
        \item
            il existe \( a\in \eL\) algébrique sur \( \eK\) tel que \( \eL=\eK(a)\) et un morphisme de corps \( \alpha\colon \eK\to \eL\).
        \item
            \( \eF\) est une extension algébriquement close de \( \eK\) : il existe un morphisme \( \beta\colon \eK\to \eF\).
    \end{enumerate}
    Alors il existe un morphisme de corps \( \sigma\colon \eL\to \eF\) tel que \( \sigma|_{\alpha(\eL)}=\beta\).
\end{lemma}

Note : en pratique, les corps \( \eL\) et \( \eF\) sont le plus souvent des sur-corps de \( \eK\), de telle sorte que les applications \( \alpha\) et \( \beta\) sont l'identité. En particulier, la conclusion de ce lemme s'écrit le plus souvent \( \sigma|_{\eK}=\id\). Il faut juste savoir que le Frido est un névrosé des notations précises.

\begin{proof}
    Vu que \( \eL\) est monogène, si \( \mu_a\in \eK[X]\) est le polynôme minimal de \( a\in \eL\), alors les points \ref{ItemJCMooDgEHaji} et \ref{ItemJCMooDgEHajii} de la proposition \ref{PropURZooVtwNXE} disent que \( \eL\simeq \eK[a]\simeq \eK[X]/(\mu_a)\).

    Les coefficients de \( \mu_a\) sont dans \( \eK\), donc nous pouvons voir \( \mu_a\in \eF[X]\). Plus précisément, si \( \mu_a=\sum_ka_kX^k\), nous définissons
    \begin{equation}
        \mu'_a=\sum_k\beta(a_k)X^k\in \eF[X].
    \end{equation}
    Comme \( \eF\) est algébriquement clos, le polynôme \( \mu'_a\) possède une racine (au moins) \( b\in \eF\) : \( \mu'_a(b)=0\).

    Nous posons
    \begin{equation}
        \begin{aligned}
            \sigma'\colon \eK[X]/(\mu_a)&\to \eF \\
            \overline{ \sum_k s_kX^k }&\mapsto \sum_k\beta(s_k)b^k. 
        \end{aligned}
    \end{equation}
    Montrons que cela est bien défini. Si \( P=\sum_ks_kX^k\) et \( \mu_a=\sum_ka_kX^k\) (\( a_k, s_k\in \eK\)), alors
    \begin{equation}
        \sigma'(\overline{ P+\mu_a })=\sigma'\big( \overline{ \sum_k (a_k+s_k)X^k } \big)=\sum_k\beta(a_k)b^k+\sum_ks_kb^k=\mu'_a(b)+\sigma'(\overline{ P })=\sigma'(\overline{ P }).
    \end{equation}
    L'application \( \sigma'\) est un morphisme de corps.

    Si \( \varphi\colon \eL\to \eK[X]/(\mu_a)\) est l'isomorphisme dont nous parlions plus haut, il vérifie \( \varphi\big( \alpha(k) \big)=\bar k\) (la classe du polynôme de degré zéro). En posant \( \sigma=\sigma'\circ \varphi\) nous avons, pour tout \( k\in \eK\), 
    % TODO: il faut un peu justifier ce \varphi(alpha(k))=bar k, à partir de la forme explicite du varphi. C'est mon todo 368.
    \begin{equation}
        (\sigma'\varphi\alpha)(k)=\sigma'(\bar k)=\beta(k),
    \end{equation}
    c'est ce qu'il fallait.
\end{proof}

\begin{lemma}[\cite{BIBooYYZPooMgFzkp}] \label{LEMooUULTooYcytat}
    Soit un corps \( \eK\) muni de deux extensions \( \alpha\colon \eK\to \eL\) et \( \beta\colon \eK\to \eF\) où \( \eL\) est algébrique sur \( \eK\) et \( \eF\) est algébriquement clos.

    Alors il existe un morphisme de corps \( \sigma\colon \eL\to \eF\) tel que \( \sigma\circ \alpha=\beta\).
\end{lemma}

\begin{proof}
    Nous allons utiliser le lemme de Zorn \ref{LemUEGjJBc} sur l'ensemble
    \begin{equation}
        \mA=\Big\{  (\eM,\varphi)  \tq
        \begin{cases}
            \eM\text{ est un sous-corps de } \eL\\
            \alpha(\eK)\subset \eM\\
            \varphi\colon \eM\to \eF\text{ est une extension de corps}\\
            \varphi\circ \alpha=\beta
        \end{cases}
    \Big\}.
    \end{equation}
    Nous ordonnons (partiellement) cet ensemble en disant que \( (\eM_1,\varphi_1)<(\eM_2,\varphi_2)\) si \( \eM_1\subset \eM_2\) et \( \varphi_2|_{\eM_1}=\varphi_1\). Il se fait que \( \mA\) est un ensemble inductif et que nous pouvons donc lui appliquer le lemme de Zorn.

    Soit un élément maximum \( (\eM,\varphi)\). Nous allons montrer que \( \eM=\eL\).
    
    Soit \( l\in \eL\). Vu que \( \eL\) est une extension algébrique de \( \eK\), il existe un polynôme \( P\) à coefficients dans \( \alpha(\eK)\) tel que \( P(l)=0\). Mais vu que \( \alpha(\eK)\subset \eM\), ce polynôme est également à coefficients dans \( \eM\). Donc \( l\) est un élément algébrique sur \( \eM\).

    Nous pouvons donc considérer le triple \( \big( \eM,\eM(l),\eF \big)\) qui vérifie les hypothèses du lemme \ref{LEMooIIKYooHMNqYn}. Il existe donc un morphisme de corps \( \sigma\colon \eM(l)\to \eF\) tel que \( \sigma|_{\eM}=\varphi\).

    Nous avons
    \begin{equation}
        \sigma\circ\alpha=\sigma|_{\sigma(\eK)}\circ \alpha=\sigma|_{\eM}\circ \alpha=\varphi\circ \alpha=\beta.
    \end{equation}
    Donc l'élément \( \big( \eM(l),\sigma \big)\) majore \( (\eM,\varphi)\) dans \( \mA\).

    Par maximalité, nous déduisons que \( \eM=\eL\). Donc le morphisme \( \varphi\colon \eL\to \eF\) vérifie \( \varphi\circ \alpha=\beta\), ce qu'il nous fallait.
\end{proof}

\begin{theorem}[Steinitz\cite{BIBooRTTUooSPwAKJ,BIBooFQQWooIuoZyf}]       \label{THOooEDQKooLEGlDv}
    À propos de clôture algébrique.
    \begin{enumerate}
        \item
            Tout corps possède une clôture algébrique.
        \item
            Si \( \alpha_1\colon \eK\to \eF_1\) et \( \alpha_2\colon \eK\to \eF_2\) sont deux clôtures algébriques du même corps \( \eK\), alors il existe un isomorphisme de corps \( \varphi\colon \eF_1\to \eF_2\) tel que \( \varphi\circ\alpha_1=\alpha_2\).
    \end{enumerate}
\end{theorem}

\begin{proof}
    Nous commençons par l'existence, en plusieurs points.
    \begin{subproof}
    \item[Un ensemble]
    Nous considérons un ensemble \( \Omega\) qui contient \( \eK\), qui est strictement surpotent\footnote{Définition \ref{DEFooXGXZooIgcBCg}.} à \( \eK\) et qui est infini non dénombrable si \( \eK\) est fini. Par exemple \( \mP(\eK)\cup \eK\) si \( \eK\) est infini et \( \eR\cup \eK\) si \( \eK\) est fini (voir le théorème de Cantor \ref{THOooJPNFooWSxUhd}).

    \item[L'ensemble pour Zorn]
    Nous considérons l'ensemble des extensions algébriques de \( \eK\) contenues dans \( \Omega\), c'est à dire
    \begin{equation}
        \mA=\Big\{  (\eL, +_{\eL}, \times_{\eL})  \tq
        \begin{cases}
            \eK\subset \eL\subset \Omega\\
            (\eL, +_{\eL}, \times_{\eL}) \text{ est une extension algébrique de } (\eK, +, \times).
        \end{cases}
    \Big\}
    \end{equation}
    Nous ordonnons \( \mA\) par l'inclusion : nous disons que
    \begin{equation}
        (\eL_1, +_{1}, \times_1)< (\eL_2,+_2,\times_2)
    \end{equation}
    lorsque $(\eL_2, +_2, \times_2)$ est un sur-coprs de $(\eL_1,+_1,\times_1)$ (en particulier \( \eL_1\subset \eL_2\)).

\item[\( \mA\) est inductif]
    Soit une partie \( \mF=\{ (\eL_i, +_i,\times_i) \}_{i\in I}\) de \( \mA\) que nous supposons être totalement ordonnées. Nous allons lui trouver un majorant dans \( \mA\). Nous posons \( \eL=\bigcup_{i\in I}\eL_i\), et si \( a\in \eL_1\), \( b\in \eL_j\), alors nous définissons
    \begin{subequations}
        \begin{numcases}{}
            a+_{\eL}b=a+_kb\\
            a\times_{\eL}b=a\times_kb
        \end{numcases}
    \end{subequations}
    où \( k\in I\) est sélectionné de telle façon à avoir \( (\eL_i,+_i,\times_i )<(\eL_k,+_k,\times_k)\) et \( (\eL_j,+_j,\times_j )<(\eL_k,+_k,\times_k)\). Vu que tous les corps \( L_i\) sont des sous-corps les uns des autres, cela est une bonne définition.
\item[Lemme de Zorn]
    Nous utilisons le lemme de Zorn \ref{LemUEGjJBc}. Nous notons \( (\eF,+,\times)\) un élément maximal de \( \mA\). Vu que \( \eK\) en est un sous-corps, il n'y a pas d'ambiguïté de noter \( +\) et \( \times\) ses opérations.
\item[Stratégie pour la suite]
    Nous allons montrer que si \( \eE\) est une extension algébrique de \( \eF\), alors \( \eE=\eF\) (le but est d'utiliser le lemme \ref{LEMooQSCGooMyCktA}).

\item[Un peu de cardinalité]
    D'abord, vu que \( \eF\) est algébrique sur \( \eK\), l'ensemble \( \eF\) est équipotent à \( \eK\) si \( \eK\) est infini, et au plus dénombrable sur \( \eK\) est fini; c'est la proposition \ref{PROPooVPQFooScWvkS}. En bref :
    \begin{itemize}
        \item Si \( \eK\) est infini, \( \eK\approx \eF\approx \eE\prec\Omega\).
        \item Si \( \eK\) est fini, \(  \eK\preceq \eF\preceq \eE\prec \Omega \) où \( \eE\) est au maximum dénombrable et \( \Omega\) est indénombrable.
    \end{itemize}
    Dans tous les cas, \( \Omega\) est strictement surpotent à \( \eF\), et le lemme \ref{LEMooIVCBooHWQiZB} permet de dire
    \begin{equation}
        \eE\setminus \eF\preceq \eE\prec \Omega\approx\Omega\setminus \eF.
    \end{equation}
\item[Quelques injections]
    Il existe donc une injection \( \varphi\colon \eE\setminus \eF\to \Omega\setminus \eF\). Nous posons 
    \begin{equation}
        \begin{aligned}
            f\colon \eE&\to \Omega \\
            x&\mapsto \begin{cases}
                x    &   \text{si }x\in \eF\\
                \varphi(x)    &    \text{si } x\notin \eF.
            \end{cases}
        \end{aligned}
    \end{equation}
    Nous montrons que \( f\) est injective. Soient \( x,y\in \eE\) tels que \( f(x)=f(y)\). Si \( x,y\in \eF\), alors \( x=f(x)=f(y)=y\). Si \( \in \eF\) et \( y\notin \eF\), alors \( x=\varphi(y)\) alors que \( x\in \eF\) et \( \varphi(y)\in \Omega\setminus F\); ce cas est impossible. Enfin si \( x\) et \( y\) sont hors de \( \eF\), alors \( f(x)=\varphi(x)\) et \( f(y)=\varphi(y)\); donc \( \varphi(x)=\varphi(y)\) et \( x=y\) par injectivité de \( \varphi\).

    Nous avons donc bien une injection \( f\colon \eE\to \Omega\).

\item[La maximalité]

    Nous pouvons mettre sur \( f(\eE)\subset \Omega\) la structure de corps venant de \( \eE\). Vu que \( f(\eF)=\eF\), le corps \( f(\eE)\) est une extension algébrique de \( \eF\). Par maximalité, \( f(\eE)=\eF\).

    Mais si \( x\in\eE\setminus \eF\), alors \( f(x)\in \Omega\setminus \eF\). Donc en réalité nous avons aussi \( \eE\subset \eF\).
\item[Conclusion]
    En conclusion \( \eE=\eF\) et le lemme \ref{LEMooQSCGooMyCktA} termine en disant que \( \eF\) est une clôture algébrique de \( \eK\).
    \end{subproof}
    Nous passons à la partie «unicité» de la clôture algébrique. Étant donné que \( \eF_1\) est une extension algébrique de \( \eK\) et que \( \eF_2\) est algébriquement clos, le lemme \ref{LEMooUULTooYcytat}
\end{proof}



\begin{normaltext}
    Bien que \( \eC\) soit une extension algébriquement close de \( \eQ\), l'ensemble \( \eC\) n'est pas une clôture algébrique de \( \eQ\). C'est ce que nous montrons maintenant.
\end{normaltext}

\begin{lemma}       \label{LEMooRDIZooRjWNMa}
    Le corps \( \eC\) n'est pas une clôture algébrique\footnote{Clôture algébrique, définition \ref{DEFooREUHooLVwRuw}.} de \( \eQ\).    
\end{lemma}

\begin{proof}
    Nous montrons qu'il existe des éléments de \( \eC\) qui ne sont pas des racines de polynômes à coefficients rationnels. L'ensemble \( \eQ\) est dénombrable par la proposition \ref{PROPooDHIAooZysvNs}. L'ensemble des polynômes de degré \( n\) à coefficients dans \( \eQ\) est en bijection avec les \( n\)-uples de rationnels, c'est-à-dire avec \( \eQ^n\) qui est également dénombrable par la proposition \ref{PROPooDMZHooXouDrQ}. Enfin l'ensemble des polynômes à coefficients sur \( \eQ\) est l'union des polynômes de degré fixés, donc dénombrable par la proposition \ref{PROPooENTPooSPpmhY}.

    Jusqu'ici nous avons prouvé que l'ensemble des polynômes à coefficients rationnels était dénombrable. Or chaque polynôme possède une quantité finie de racines par le corolaire \ref{CORooUGJGooBofWLr}. Donc la partie de \( \eC\) constituée des nombres qui sont des racines de polynômes à coefficients dans \( \eQ\) est dénombrable. Mais \( \eC\) n'est pas dénombrable, donc possède des éléments qui ne sont pas des racines de polynômes.
\end{proof}

%---------------------------------------------------------------------------------------------------------------------------
\subsection{Polynômes à plusieurs variables}
%---------------------------------------------------------------------------------------------------------------------------

Nous avons déjà vu \( A[X,Y]\) lorsque \( A\) est un anneau en la définition~\ref{DEFooZNHOooCruuwI}.

\begin{definition}      \label{DEFooRHRKooPqLNOp}
    Soit un corps \( \eK\). Le corps \( \eK(X_1,\ldots, X_n)\) est le corps des fractions de l'anneau \( \eK[X_1,\ldots, X_n]\).
\end{definition}

\begin{definition}  \label{DEFooOCPHooXneutp}
    Soient un corps \( \eK\) et une extension \( \eL\) de \( \eK\) contenant les éléments \( \alpha_1\),\ldots, \( \alpha_n\) de \( \eK\). Nous définissons \( \eK(\alpha_1,\ldots, \alpha_n)\) comme étant l'intersection de tous les sous-corps de \( \eL\) contenant \( \eK\) et les \( \alpha_i\).
\end{definition}

La proposition suivante est analogue à~\ref{PROPooYSFNooFGbbCi}\ref{ITEMooATPTooVXKdlK}.

\begin{lemma}[\cite{MonCerveau}]        \label{LEMooQEJHooAmSNxU}
    Soient un corps \( \eK\), une extension \( \eL\) et des éléments \( \alpha_1,\ldots, \alpha_n\) dans \( \eL\). Alors
    \begin{equation}
        \eK(\alpha_1,\ldots, \alpha_n)=\{ r(\alpha_1,\ldots, \alpha_n)\tq r\in \eK(X_1,\ldots, X_n) \}.
    \end{equation}
\end{lemma}

\begin{proof}
    Ce que nous avons à droite est un corps : par exemple pour l'inverse, si \( r=P/Q\) alors \( r(\alpha_1,\ldots,\alpha_n)=P(\alpha_1,\ldots, \alpha_n)Q(\alpha_1,\ldots, \alpha_n)^{-1}\). Cet élément a un inverse en la personne de \( (Q/P)(\alpha_1,\ldots, \alpha_n)\).

    Donc à droite nous avons un sous-corps de \( \eL\) contenant \( \eK\) ainsi que les \( \alpha_i\). Donc
    \begin{equation}
        \eK(\alpha_1,\ldots, \alpha_n)\subset \big\{ r(\alpha_1,\ldots, \alpha_n)\tq r\in \eK(X_1,\ldots, X_n) \big\}.
    \end{equation}

    D'autre part, tout corps contenant \( \eK\) et les \( \alpha_i\) doit contenir tous les \( P(\alpha_1,\ldots, \alpha_n)\) (\( P\in \eK[X_1,\ldots, X_n]\)), leurs inverses ainsi que leurs produits; bref doit contenir tous les \( r(\alpha_1,\ldots, \alpha_n)\) avec \( r\in\eK[X_1,\ldots, X_n]\).
\end{proof}




% This is part of Mes notes de mathématique
% Copyright (c) 2011-2020
%   Laurent Claessens
% See the file fdl-1.3.txt for copying conditions.

%---------------------------------------------------------------------------------------------------------------------------
\subsection{Corps de décomposition}
%---------------------------------------------------------------------------------------------------------------------------

\begin{definition}      \label{DEFooEKGZooSkvbum}
    Soit \( \eK\) un corps commutatif et \( F=(P_i)_{i\in I}\) une famille d'éléments non constants de \( \eK[X]\). Un \defe{corps de décomposition}{corps!de décomposition}\index{décomposition!corps} de \( F\) est une extension \( \eL\) de \( \eK\) telle que
    \begin{enumerate}
        \item
            les \( P_i\) sont scindés sur \( \eL\),
        \item
            \( \eL=\eK(R)\) où \( R=\bigcup_{i\in I}\{ x\in\eL\tq P_i(x)=0 \}\).
    \end{enumerate}
    C'est-à-dire que \( \eL\) étend \( \eK\) par toutes les racines de tous les polynômes de \( F\).
\end{definition}

\begin{proposition}[\cite{ooJTWFooASSMjI}]      \label{PROPooDPOYooFHcqkU}
    Tout polynôme admet un corps de décomposition. Plus précisément, soit un corps \( \eK\) et un polynôme \( P\in \eK[X]\) de degré \( n\). Il existe un corps de décomposition \( \eD\) de la forme \( \eD=\eK(\alpha_1,\ldots,\alpha_n)\) où les \( \alpha_i\) sont des racines de \( P\) dans \( \eD\).
\end{proposition}

Notons que rien dans l'énoncé ne prétend que les \( \alpha_i\) soient tous distincts, ni même que certains (voire tous) ne seraient pas dans \( \eK\).

\begin{proof}
    Soient un corps \( \eK\) et un polynôme \( P\in \eK[X]\). Si le degré de \( P\) est \( 0\) ou \( 1\), alors \( \eK\) est un corps de décomposition pour \( P\). Pour le reste nous faisons une récurrence sur le degré de \( P\).

    Il y a deux possibilités, soit il existe \( \alpha\in \eK\) tel que \( P(\alpha)=0\), soit non.

    \begin{subproof}
        \item[Si racine dans \( \eK\)]
            Alors le corolaire~\ref{CorDIYooEtmztc} nous permet de factoriser \( X-\alpha\) :
            \begin{equation}
                P=(X-\alpha)Q
            \end{equation}
            avec \( \deg(Q)=\deg(P)-1\). Dans ce cas, \( \eK\) est un corps de rupture de \( P\).

        \item[Si pas de racines dans \( \eK\)]

            Nous prenons alors un corps de rupture \( \eL=\eK(\alpha)\) avec \( P(\alpha)=0\) (c'est la proposition~\ref{PROPooUBIIooGZQyeE} qui donne l'existence d'un corps de rupture). Dans \( \eL_1\) nous avons
            \begin{equation}
                P=(X-\alpha)Q
            \end{equation}
            avec \( Q\in \eL_1[X]\) et \( \deg(Q)=\deg(P)-1\).

        \item[Dans les deux cas]

            Dans les deux cas, par hypothèse de récurrence nous avons un corps de décomposition pour \( Q\) qui se présente sous la forme
            \begin{equation}
                \eL=\eK(\alpha_1,\ldots, \alpha_{n-1}).
            \end{equation}
            De plus, \( \eL\) est une extension de \( \eL_1\) parce que c'est une extension du corps sur lequel est \( Q\).

    \end{subproof}
    Pour terminer la preuve nous prouvons que
    \begin{equation}
        \eD=\eK(\alpha_1,\ldots, \alpha_{n-1},\alpha)
    \end{equation}
    est un corps de décomposition de \( P\). Vu que \( \eD\) contient \( \eK(\alpha)\) (comme cas particulier du lemme \ref{LEMooQEJHooAmSNxU}), dans \( \eD\) nous avons l'égalité \( P=(X-\alpha)Q\). Et vu que \( \eD\) contient également \( \eK(\alpha_1,\ldots, \alpha_{n-1})\), toujours dans \( \eD\) nous avons aussi
    \begin{equation}
         Q=(X-\alpha_1)\ldots(X-\alpha_{n-1}).
    \end{equation}
    Donc nous avons dans \( \eD\) l'égalité
    \begin{equation}
        P=(X-\alpha)(X-\alpha_1)\ldots (X-\alpha_{n-1}).
    \end{equation}
\end{proof}

\begin{lemma}[\cite{MonCerveau}]        \label{LEMooJNGWooTXdGre}
    Soit un polynôme \( P\in \eK[X]\), et un corps \( \eL\) dans lequel \( P\) est scindé. Si \( P=P_1\ldots P_r\) est la décomposition de \( P\) en irréductibles dans \( \eK\), alors chacun des \( P_i\) est scindé dans \( \eL\).
\end{lemma}

\begin{proof}
    Juste pour le mentionner, le fait que \( P\) ait une décomposition en irréductibles est le fait que \( \eK[X]\) soit factoriel, c'est-à-dire la proposition~\ref{PropqGZXvr}.

    Le polynôme \( P\) est scindé dans \( \eL\), c'est-à-dire que, en notant \( n\) le degré de \( P\),
    \begin{equation}        \label{EQooNFXSooUkWeQu}
        P=\prod_{i=1}^n(X-\lambda_i)
    \end{equation}
    avec \( \lambda_i\in \eL\).

    Soit \( \eL_1\), une extension de \( \eL\) dans laquelle \( P_1\) est scindé. Ensuite, \( \eL_2\) une extension de \( \eL_1\) dans laquelle \( P_2\) est scindé et ainsi de suite. Nous avons construit \( \eL_r\), une extension de \( \eL\) dans laquelle tous les \( P_i\) sont scindés ainsi que \( P\) lui-même. Dans ce corps nous avons l'égalité
    \begin{equation}        \label{EQooBEFUooBnqSUS}
        P=\prod_{k=1}^n(X-\mu_k)
    \end{equation}
    où les \( \mu_k\) sont des éléments des diverses extensions \( \eL_i\), et sont les racines des \( P_i\). En tout cas, tous sont dans \( \eL_r\).

    Les deux décompositions \eqref{EQooNFXSooUkWeQu} et \eqref{EQooBEFUooBnqSUS} sont des décompositions dans \( \eL_r[X]\) du polynôme \( P\). Vu que ce dernier est factoriel, en réalité les deux décompositions sont identiques (se souvenir de la définition~\ref{DEFooVCATooPJGWKq}), et nous avons \( \mu_k\in \eL\) pour tout \( k\). Toutes les extensions \( \eL_i\) sont en réalité triviales, et nous avons \( \eL_r=\eL\).

    Bref, tout cela pour dire que les \( P_i\) ont toutes leurs racines dans \( \eL\).
\end{proof}

\begin{theorem}[\cite{ooUHHUooONXDDl}]      \label{THOooQVKWooZAAYxK}
    Soient :
    \begin{itemize}
        \item un isomorphisme de corps \( \tau\colon \eK\to \eK'\);
        \item un polynôme non constant \( P\in \eK[X]\) de degré $n$;
        \item un corps de décomposition \( \eL\) de \( P\) sur \( \eK\);
        \item un corps de décomposition \( \eL'\) de \( P\) sur \( \eK'\);
    \end{itemize}
    Alors \( \tau\) se prolonge en un isomorphisme \( \sigma\colon \eL\to \eL'\).
\end{theorem}

\begin{proof}
    Soit \( m\) le nombre de racines de \( P\) dans \( \eL\setminus \eK\). Nous faisons une récurrence sur \( m\).

    Si \( m=0\) alors \( \eK\) est un corps de rupture de \( P\); nous avons
    \begin{equation}
        P=(X-\lambda_1)\ldots (X-\lambda_n)
    \end{equation}
    avec \( \lambda_i\in \eK\). Dans ce cas nous avons aussi
    \begin{equation}
        \tau(P)=\big( X-\tau(\lambda_1) \big)\ldots \big( X-\tau(\lambda_n) \big)
    \end{equation}
    avec \( \tau(\lambda_i)\in \eK'\). Nous avons donc \( \eL'=\eK'\) et prendre \( \sigma=\tau\) fonctionne.

    Nous supposons à présent que \( m>0\). Plus précisément nous considérons un polynôme possédant exactement \( m\) racines dans \( \eL\setminus \eK\). Soit
    \begin{equation}
        P=P_1\ldots P_r
    \end{equation}
    sa décomposition en irréductibles dans \( \eK[X]\) (notons que \( r\leq n\) parce que chacun des facteurs est de degré au moins \( 1\)). Au moins un des \( P_i\) est de degré plus grand ou égal à \( 2\). Nous savons de la proposition~\ref{PropqGZXvr} que \( \eK[X]\) est un anneau factoriel. Le lemme~\ref{LEMooJNGWooTXdGre} nous assure que les polynômes \( P_i\) sont également scindés dans \( \eL\). Et l'unicité de la décomposition fait en sorte que les racines des \( P_i\) sont celles de \( P\).

    Soit \( \alpha\in \eL\), une racine de \( P_1\). Vu que \( P_1\) est irréductible sur \( \eK\), l'application suivante est un isomorphisme de corps par le lemme~\ref{LEMooHKTMooKEoOuK} :
    \begin{equation}
        \begin{aligned}
            \psi\colon \eK[X]/(P_1)&\to \eK[\alpha] \\
            \bar Q&\mapsto Q(\alpha).
        \end{aligned}
    \end{equation}
    Notons que le lemme parle du quotient par le polynôme minimal, mais ici nous avons un irréductible. Un polynôme annulateur irréductible est multiple du polynôme minimal, et l'idéal engendré est le même.

    Nous avons aussi la décomposition
    \begin{equation}
        \tau(P)=\tau(P_1)\ldots \tau(P_r),
    \end{equation}
    et chacun des \( \tau(P_i)\) a ses racines dans \( \eL'\). Soit \( \beta\), une racine de \( \tau(P_1)\) dans \( \eL'\). Alors nous avons l'isomorphisme
    \begin{equation}
        \psi'\colon \eK'[X]/\big( \tau(P_1) \big)\to \eK'[\beta].
    \end{equation}
    De plus, par le lemme~\ref{LEMooGRIMooPxCXAZ}, nous savons que \( \tau\) passe aux classes :
    \begin{equation}
        \phi_{\tau}\colon \eK[X]/(P_1)\to \eK'[X]/\big( \tau(P_1) \big)
    \end{equation}
    est un isomorphisme d'anneaux. Et enfin, dernier résultat externe à invoquer, la proposition~\ref{PropURZooVtwNXE} nous assure que \( \eK[\alpha]=\eK(\alpha)\) et \( \eK'[\beta]=\eK'(\beta)\). Posons pour l'occasion \( \eK_1=\eK(\alpha)\) et \( \eK'_1=\eK'(\beta)\).

    Nous avons l'enchainement suivant d'isomorphismes de corps\footnote{En réalité il est plus exact de dire «isomorphisme d'anneaux», parce que la structure de corps n'est en réalité aucune nouvelle structure par rapport à l'anneau.} :
    \begin{equation}
        \tau_1=\psi'\circ\phi_{\tau}\circ\psi^{-1} \colon \eK_1\to \eK[X]/(P_1)\to\eK'[X]/\big( \tau(P_1) \big)\to \eK'_1.
    \end{equation}
    Cet isomorphisme \( \tau_1\colon \eK_1\to \eK_2\) prolonge \( \tau\). Si vous aimez les diagrammes, en voici un sur lequel les \( i\) représentent des inclusions et où \( \tau\) et \( \tau_1\) sont des isomorphismes
    \begin{equation}
        \xymatrix{%
            \eK \ar[r]^-{i}\ar[d]_-{\tau}       &   \eK_1   \ar[d]^{\tau_1} \ar[r]^i    &   \eL\\
            \eK' \ar[r]_{i}                     &   \eK'_1  \ar[r]^{i}                  &   \eL'
        }
    \end{equation}
    Le corps \( \eL\) est un corps de décomposition de \( P\) sur \( \eK_1\), et le nombre de racines de \( P\) dans \( \eL\setminus\eK_1\) est strictement plus petit que \( m\) parce qu'il y en a exactement \( m\) dans \( \eL\setminus \eK\) et que \( \eK_1\) en a au moins une de plus que \( \eK\). Même raisonnement pour \( \eK'\), \( \eK'_1\) et \( \eL'\).

    Résumons la situation :
    \begin{itemize}
        \item \( \tau_1\colon \eK_1\to \eK'_1\) est un isomorphisme de corps;
        \item \( P\in \eK_1[X]\) est un polynôme non constant;
        \item \( \eL\) est un corps de décomposition de \( P\) sur \( \eK_1\);
        \item \( \eL'\) est un corps de décomposition de \( P\) sur \( \eK'_1\);
        \item le nombre de racines de \( P\) dans \( \eL\setminus \eK_1\) est strictement inférieur à \( m\).
    \end{itemize}
    Donc, par hypothèse de récurrence sur \( m\), il existe un isomorphisme \( \sigma\colon \eL\to \eL'\) qui prolonge \( \tau_1\). Vu que \( \tau_1\) prolonge \( \tau\), nous avons également \( \sigma\) qui prolonge \( \tau\).
\end{proof}

L'énoncé le plus compact pour l'unicité du corps de décomposition (à isomorphisme près) est le suivant :
\begin{proposition}     \label{PropTMkfyM}
    Soit \( \eK\) un corps et \( P\in\eK[X]\). Soient \( \eL\) et \( \eF\) deux corps de décomposition de \( P\). Alors il existe un isomorphisme \( f\colon \eL\to \eF\) tel que \( f|_{\eK}=\id\).
\end{proposition}
\begin{proof}
    C'est un cas particulier du théorème~\ref{THOooQVKWooZAAYxK}, où nous considérons \( \eK=\eK'\) muni de l'isomorphisme identité.
\end{proof}

Cependant le passage par l'énoncé plus compliqué~\ref{THOooQVKWooZAAYxK} est nécessaire pour les besoins de la récurrence.

\begin{normaltext}
    À propos de terminologie. Lorsque nous disons «\emph{un} corps de décomposition» nous référons à la définition~\ref{DEFooEKGZooSkvbum} et il n'y a pas vraiment unicité. Si nous disons «\emph{le} corps de décomposition» nous référons en général à celui construit dans la proposition~\ref{PROPooDPOYooFHcqkU} qui n'est en réalité même pas très explicite.

    Quoi qu'il en soit, nous nous permettons de dire «le» corps de décomposition lorsque nous parlons de propriétés invariantes par isomorphisme.
\end{normaltext}

\begin{normaltext}
    La construction du corps de décomposition d'un polynôme fonctionne en prenant successivement le corps de rupture des facteurs irréductibles. Nous insistons sur le fait que cette opération se fait sur chaque facteur irréductible séparément.

    L'exemple suivant montre dans quel ordre se passent les choses.

    \begin{example}
        Soit le polynôme \( P=X^4-5X^2+6\). Sa factorisation en irréductibles est :
        \begin{equation}
            P=(X^2-2)(X^2-3).
        \end{equation}
        Ce polynôme n'est pas irréductible sur \( \eQ\) et il ne s'agit donc pas de prendre brutalement un corps de rupture pour \( P\). Il s'agit de poser \( P=P_1P_2\) avec
        \begin{subequations}
            \begin{align}
                P_1&=X^2-2\\
                P_2&=X^2-3,
            \end{align}
        \end{subequations}
        de remarquer que \( P_1\) et \( P_2\) sont irréductibles sur \( \eQ\) et de chercher des corps de rupture pour eux. On commence par \( P_1\). Nous avons le corps de rupture \( \eL_1=\eQ(\sqrt{ 2 })\) et la factorisation
        \begin{equation}
            P_1=(X+\sqrt{ 2 })(X-\sqrt{ 2 }).
        \end{equation}
        Ensuite nous considérons \( P_2\) dans \( \eL_1[X]\). Ce \( P_2\) est encore irréductible. Nous lui cherchons un corps de rupture, et c'est \( \eL_2=\eL_1(\sqrt{ 3 })\) dans lequel nous avons
        \begin{equation}
            P_2=(X-\sqrt{ 3 })(X+\sqrt{ 3 }).
        \end{equation}
        Nous savons (par le lemme~\ref{LEMooTURZooXnjmjT}) que
        \begin{equation}
            \eL_2=\eL_1(\sqrt{ 3 })=\big( \eQ(\sqrt{ 2 }) \big)(\sqrt{ 3 })=\eQ(\sqrt{ 2 },\sqrt{ 3 }).
        \end{equation}
        Nous pouvons donc écrire en toute confiance, dans \( \eQ(\sqrt{ 2 },\sqrt{ 3 })\) la factorisation
        \begin{equation}
            P=(X+\sqrt{ 2 })(X-\sqrt{ 2 })(X+\sqrt{ 3 })(X-\sqrt{ 3 }).
        \end{equation}

        Et nous notons que si nous avions commencé par \( P_2\) au lieu de \( P_1\), nous aurions eu le même résultat final.
    \end{example}
\end{normaltext}

\begin{corollary}[\cite{MonCerveau}]    \label{CORooELAUooPQGLkR}
    Le corps de décomposition d'un polynôme est une extension finie.
\end{corollary}

\begin{proof}
    Soient un corps \( \eK\), un polynôme \( P\in \eK[X]\) et un corps de décomposition \( \eD\) de \( P\) de la forme \( \eD=\eK(\alpha_1,\ldots, \alpha_n)\) où les \( \alpha_i\) sont les racines de \( P\) dans \( \eD\). Cela existe par la proposition~\ref{PROPooDPOYooFHcqkU}.

    Vu que le lemme~\ref{LEMooTURZooXnjmjT} donne
    \begin{equation}
        \eK(\alpha_1,\ldots, \alpha_n)=\big( \eK(\alpha_1,\ldots, \alpha_{n-1}) \big)(\alpha_n),
    \end{equation}
    le corps \( \eD\) se construit comme une pile d'extensions finies. Les degrés se composant par le lemme~\ref{PROPooEGSJooBSocTf}, au final ce corps de décomposition est une extension finie.

    Soit maintenant un corps de décomposition quelconque \( \eL\). La proposition~\ref{PropTMkfyM} donne un isomorphisme de corps\footnote{Un isomorphisme de corps est juste un isomorphisme d'anneaux.} \( f\colon \eL\to \eD\) tel que \( f\) soit l'identité sur \( \eK\).

    Si \( \{ v_i \}_{i\in I}\) est une base de \( \eD\) comme espace vectoriel sur \( \eK\), êtes-vous prêts à parier que \( \{ f(v_i) \}_{i\in I}\) est une base de \( \eL\) comme espace vectoriel sur \( \eK\)\quext{Personnellement, je n'ai pas vérifié. Vérifiez vous-même et écrivez-moi pour dire si c'est bon ou non.} ?
\end{proof}

%---------------------------------------------------------------------------------------------------------------------------
\subsection{Non irréductible ou pas corps ?}
%---------------------------------------------------------------------------------------------------------------------------
\label{SUBSECooEDMJooTXBfOu}

Nous avons déjà mentionné que nous ne définissons le corps de rupture d'un polynôme que dans le cas de polynôme irréductible à coefficients dans un corps.

D'abord si \( P\) n'est pas irréductible, la question de chercher un corps de rupture n'a pas beaucoup de sens.

Si \( A\) est un anneau intègre et si \( P\) est un polynôme irréductible sur \( A\), nous pouvons considérer le corps des fractions de \( A\), dire \( P\in\Frac(A)[X]\) et continuer. Étendre la définition de corps de rupture de cette façon aux polynômes à coefficients dans un anneau intègre n'est pas une grande révolution.

Au lieu de cela, nous pouvons nous demander dans quel cas nous aurions que \( A[X]/(P)\) est directement un corps.

\begin{example}
    Soit le polynôme constant \( P=2\) dans \( \eZ[X]\). Il y est irréductible parce qu'il ne peut pas être écrit comme produit de deux non inversibles. Ce polynôme a deux propriétés ennuyeuses :
    \begin{itemize}
        \item Il n'est plus irréductible sur \( \eQ\),
        \item Il n'existe aucun corps contenant une racine de \( P\) tout en contenant \( \eZ\) comme sous-anneau.
    \end{itemize}
\end{example}

%---------------------------------------------------------------------------------------------------------------------------
\subsection{Clôture algébrique}
%---------------------------------------------------------------------------------------------------------------------------

\begin{theorem}     \label{THOooQFWWooMWXEhT}
    Tout corps \( \eK\) possède une clôture algébrique\footnote{Définition \ref{DEFooREUHooLVwRuw}.} \( \Omega\). De plus si \( \eL\) est une extension de \( \eK\), alors \( \eL\) est \( \eK\)-isomorphe à un sous corps de \( \Omega\).
\end{theorem}
Les deux parties de ce théorème utilisent l'axiome du choix.

Notons en particulier que si \( \Omega'\) est une autre clôture algébrique de \( \eK\), alors \( \Omega\) et \( \Omega'\) sont des sous corps l'un de l'autre et sont donc \( \eK\)-isomorphes.

\begin{lemma}
    Les polynômes \( P,Q\in \eK[X]\) ne sont pas premiers entre eux si et seulement s'ils ont une racine commune dans la clôture algébrique \( \Omega\) de \( \eK\).
\end{lemma}

\begin{proof}
    Soit \( A\) un polynôme non inversible divisant \( P\) et $Q$. Par définition de \( \Omega\), ce polynôme \( A\) a une racine dans \( \Omega\) qui est alors une racine commune à \( P\) et \( Q\) dans \( \Omega\).

    Pour le sens inverse, si \( \alpha\) est une racine commune de \( P\) et \( Q\), alors le polynôme \( X-\alpha\) divise \( P\) et \( Q\) et donc \( P\) et \( Q \) ne sont pas premiers entre eux.
\end{proof}

\begin{example}     \label{ExfUqQXQ}
    Soit \( p\) un nombre premier. Montrons que le polynôme
    \begin{equation}
        Q(X)=X^p-X+1
    \end{equation}
    est irréductible dans \( \eF_p\).

    Nous supposons qu'il n'est pas irréductible, c'est-à-dire que
    \begin{equation}
        Q(X)=R(X)S(X)
    \end{equation}
    avec \( R\) et \( S\), des polynômes de degrés \( \geq 1\) dans \( \eF_p[X]\)

    Soit \( \bar\eF_p\) une clôture algébrique\footnote{Définition \ref{DEFooREUHooLVwRuw}. Pour l'existence c'est le théorème \ref{THOooQFWWooMWXEhT}.} de \( \eF_p\) et \( \alpha\in \bar \eF_p\) tel que \( R(\alpha)=0\). Pour tout \( a\in \eF_p\), nous avons
    \begin{subequations}
        \begin{align}
            Q(\alpha+a)&=(\alpha+a)^p-(\alpha+a)+1\\
            &=\alpha^p+a^p-\alpha-a+1\\
            &=\alpha^p-\alpha+1\\
            &=Q(\alpha)\\
            &=0
        \end{align}
    \end{subequations}
    où nous avons utilisé le fait que \( a^p=a\) et que \( \alpha\) était une racine de \( Q\). Ce que nous venons de prouver est que l'ensemble des racines de \( Q\) dans \( \bar\eF_p\) est donné par \( \{ \alpha+a\tq a\in \eF_p \}\).

    Les polynômes \( R\) et \( S\) sont donc formés de produits de termes \( X-(\alpha+a)\) avec \( a\in \eF_p\). L'un des deux --disons \( R\) pour fixer les idées-- doit bien en avoir plus que \( 1\). Nous avons alors
    \begin{equation}
        R(X)=\prod_{i=1}^{k}\big( X-(\alpha+a_i) \big)
    \end{equation}
    où les \( a_i\) sont les éléments de \( \eF_p\). En développant un peu,
    \begin{equation}
        R(X)=X^k-\sum_{i=1}^k(\alpha+a_i^{k-1})+\text{termes de degré plus bas en } X.
    \end{equation}
    Le coefficient devant \( X^{k-1}\) n'est autre que \( k\alpha+\sum_ia_i\). Étant donné que \( k\neq 0\) et que \( R\in \eF_p[X]\), nous devons avoir \( \alpha\in \eF_p\). Par conséquent nous avons \( \alpha^p=\alpha\) et une contradiction :
    \begin{equation}
        Q(\alpha)=\alpha^p-\alpha+1=1\neq 0.
    \end{equation}

    Le polynôme \( X^p-X+1\) est donc irréductible sur \( \eF_p\).
\end{example}

%---------------------------------------------------------------------------------------------------------------------------
\subsection{Extensions séparables}
%---------------------------------------------------------------------------------------------------------------------------

Notons que dans ce qui va suivre nous allons parler de \( \eK[X]\), l'ensemble des polynômes sur un corps. Cela ne s'applique donc pas à \( \eZ[X]\) par exemple.

Une des choses intéressantes avec les extensions séparables c'est qu'elles vérifient le théorème de l'élément primitif~\ref{ThoORxgBC}.

\begin{definition}      \label{DEFooLXSBooCHIUFU}
    Soit \( \eK\) un corps. Un polynôme \emph{irréductible} \( P\in \eK[X]\) est \defe{séparable}{séparable!polynôme irréductible}\index{polynôme!irréductible!séparable} sur $\eK$ si dans un corps de décomposition, ses racines sont distinctes, c'est-à-dire que si \( P\) est de degré \( n\), alors il possède \( n\) racines distinctes dans un corps de décomposition.

    Si \( P\) est un polynôme non constant dont la décomposition en irréductibles est \( P=P_1\ldots P_r\), nous disons qu'il est \defe{séparable}{séparable!polynôme non constant}\index{polynôme!séparable} si tous les \( P_i\) le sont.
\end{definition}

La proposition suivante donne un sens à la définition de polynôme irréductible séparable.
\begin{proposition}
    Soit \( P\) irréductible dans \( \eK[X]\) ayant des racines distinctes dans le corps de décomposition \( \eL\). Si \( \eL'\) est un autre corps de décomposition pour \( P\), alors \( P\) a aussi ses racines distinctes dans \( \eL'\).
\end{proposition}

\begin{proof}
    L'ingrédient est la proposition~\ref{PropTMkfyM} qui donne l'unicité du corps de décomposition à \( \eK\)-isomorphisme près. Soit donc \( \psi\colon \eL\to \eL'\) un isomorphisme laissant invariant les éléments de \( \eK\). D'une part, étant donné que \( P\) est à coefficients dans \( \eK\), nous avons \( \psi(P)=P\). D'autre part dans \( \eL\) le polynôme \( P\) s'écrit
    \begin{equation}
        P=a(X-\alpha_1)\ldots (X-\alpha_n)
    \end{equation}
    avec \( a\in \eK\) et \( \alpha_i\in \eL\). Nous avons donc
    \begin{equation}
        P=\psi(P)=a(X-\psi(\alpha_1))\ldots (X-\psi(\alpha_n)).
    \end{equation}
    Donc les racines de \( P\) dans \( \eL'\) sont les éléments \( \psi(\alpha_i)\) qui sont distincts.
\end{proof}

\begin{example}
    Un polynôme peut être séparable sur un corps, mais non séparable sur un autre. Soit \( \eL=\eF_p(T)\) et \( \eK=\eF_p(T^p)\). Nous considérons le polynôme
    \begin{equation}
        P=X^p-T^p
    \end{equation}
    dans \( \eK[X]\). Par le morphisme de Frobenius nous avons
    \begin{equation}
        P=(X-T)^p
    \end{equation}
    dans \( \eL[X]\). Le polynôme \( P\) est irréductible sur \( \eK[X]\) parce que ses diviseurs sont de la forme \( (X-T)^k\) qui contiennent \( T^k\) qui n'est pas dans \( \eK\) (sauf si \( k=n\) ou \( k=0\)).

    Ce polynôme n'est pas séparable sur \( \eK\) parce que dans le corps de décomposition \( \eL\), la racine \( T\) est multiple. Notons bien le raisonnement : \( P\) étant irréductible, pour savoir s'il est séparable, on le regarde dans un corps de décomposition.

    Par contre si nous regardons \( P\) dans \( \eL[X]\) alors \( P\) n'est plus irréductible parce que ses facteurs irréductibles sont \( (X-T)\). N'étant pas irréductible, nous regardons les racines de \emph{ses facteurs irréductibles}. Or chacun des facteurs irréductibles étant \( X-T\), les racines sont simples.
\end{example}

\begin{example}
    Le polynôme \( (X-1)^3\) est séparable sur \( \eQ\) parce que ses facteurs irréductibles dans \( \eQ[X]\) sont \( X-1\) et \(X^2 + X + 1\), et ces deux polynômes ont des racines simples (dans \( \eQ(i)\)).
\end{example}

\begin{example}
    Le polynôme \( (X^2+1)^2\) est séparable dans \( \eQ[X]\). En effet, il a pour facteurs irréductibles le polynôme \( X^2+1\) (en deux exemplaires), et ce polynôme a pour racines \( \pm i\) dans l'extension \( \eQ(i)\), racines qui sont simples pour ce polynôme.
\end{example}

\begin{proposition}[\cite{vgQYwF}]  \label{PropolyeZff}
    Soit \( P\in \eK[X]\) un polynôme non constant. Les propriétés suivantes sont équivalentes.
    \begin{enumerate}
        \item\label{ItemdqPFUi}
            \( P\) a une racine multiple dans une extension de \( \eK\). C'est-à-dire qu'il existe une extension de \( \eK\) dans laquelle \( P\) a une racine multiple.
        \item\label{ItemdqPFUib}
            \( P\) a une racine multiple dans tout corps de décomposition .
        \item\label{ItemdqPFUii}
            \( P\) et \( P'\) ont une racine commune dans une extension de \( \eK\).
        \item\label{ItemdqPFUiii}
            le degré de \( \pgcd(P,P')\) est \( \geq 1\).
    \end{enumerate}
\end{proposition}
\index{corps!extension}

\begin{proof}
    \begin{subproof}
    \item[\ref{ItemdqPFUi}\( \Rightarrow\)\ref{ItemdqPFUib}] Soit \( a\), une racine multiple de \( P\) dans une extension \( \eL\) de \( \eK\), et \( \eE\), un corps de décomposition de \( P\). Alors nous voulons prouver que \( P\) ait une racine multiple dans \( \eE\).

        Nous pouvons voir \( P\in \eL[X]\), et construire une corps de décomposition \( \eE'\) qui est une extension de \( \eL\). Vu que \( \eE\) et \( \eE'\) sont deux corps de décomposition de \( P\)
        % iDIUoR
        nous avons un isomorphisme \( \psi\colon \eE'\to \eE\). Si \( a\in \eE\) est une racine multiple de \( P\), alors \( \psi(a)\) est une racine multiple de \( P\) dans \( \eE'\) parce que
        \begin{equation}
            P\big( \psi(a) \big)=\psi\big( P(a) \big).
        \end{equation}
    \item[\ref{ItemdqPFUib}\( \Rightarrow\)\ref{ItemdqPFUii}] Soit \( \eL\) un corps de décomposition de \( P\) sur \( \eK\) et \( a\in \eL\), une racine multiple de \( P\). On a alors \( P=(X-a)^2Q\) avec \( Q\in \eL[X]\). En dérivant,
        \begin{equation}
            P'=2(X-a)Q+(X-a)^2Q',
        \end{equation}
        et donc \( a\) est également une racine de \( P'\).
    \item[\ref{ItemdqPFUii}\( \Rightarrow\)\ref{ItemdqPFUiii}] Soit \( D\) un \( \pgcd\) de \( P\) et \( P'\). D'après le théorème de Bézout il existe \( A,B\in \eK[X]\) tels que
        \begin{equation}
            AP+BP'=D.
        \end{equation}
        Si \( a\) est une racine commune de \( P\) et \( P'\) dans une extension \( \eL\), alors c'est aussi une racine de \( D\) et donc \( \deg(D)\geq 1\).
    \item[\ref{ItemdqPFUiii}\(\Rightarrow\)\ref{ItemdqPFUi}] Si le degré de \( D\) est plus grand ou égal à \( 1\), alors nous considérons une racine \( a\) de \( D\) dans \( \eL\) (une extension de \( \eK\)). Étant donné que \( D\) divise \( P\) et \( P'\), l'élément \( a\) est une racine commune de \( P\) et \( P'\). Nous montrons maintenant que \( a\) est alors une racine multiple de \( P\). Vu que \( P(a)=0\) nous avons
        \begin{equation}
            P=(X-a)Q,
        \end{equation}
        et \( P'=Q+(X-a)Q'\). Mais alors \( P'(a)=Q(a)\) et donc \( Q(a)=0\) et donc \( a\) est une racine double de \( P\). Par conséquent \( a\) est une racine multiple de \( P\) dans \( \eK\).
    \end{subproof}
\end{proof}
Notons que si \( P\) est irréductible, cette proposition donne des conditions pour que \( P\) ne soit pas séparable.

\begin{proposition}
    Soit \( P\in \eK[X]\) irréductible. Le polynôme \( P\) est séparable si et seulement si \( P'\neq 0\).
\end{proposition}

\begin{proof}
    Soit \( D=\pgcd(P,P')\) et nous voudrions prouver que \( \deg(D)\geq 1\) si et seulement si \( P'=0\). Si \( P'=0\), alors \( \pgcd(P,P')=P\) est donc \( \deg(D)\geq 1\).

    Dans l'autre sens, si \( P\) est irréductible, il est associé à \( D\) parce qu'il n'a pas d'autres diviseurs que lui-même et le polynôme constant \( 1\). Ainsi, \( D \in \eK \), ou bien \( P=\lambda D\) avec \( \lambda\in \eK\). et donc \( \deg(P)\geq 1\). Dans les deux cas, \( P' \) est nécessairement non-nul.
\end{proof}

\begin{corollary}   \label{CorUjfJSE}
    Si \( \eK\) est de caractéristique nulle, alors tout polynôme de \( \eK[X]\) est séparable.
\end{corollary}

\begin{proof}
    Il suffit de montrer que les irréductibles sont séparables. Soit \( P\) un polynôme irréductible et unitaire de degré \( d\). Le terme de plus haut degré de \( P'\) est alors \( dX^{d-1}\) qui est non nul parce que \( d\neq 0\) en caractéristique nulle. Donc \( P'\neq 0\) et donc \( P\) est séparable par la proposition~\ref{PropolyeZff}.
\end{proof}

\begin{definition}      \label{DEFooKTVHooTydOTM}
    Soit \( \eL\) une extension algébrique de \( \eK\).
    \begin{enumerate}
        \item       \label{ITEMooOFYPooLYkIPr}
            On dit que l'élément \( a\in \eL\) est \defe{séparable}{séparable!élément d'une extension} sur \( \eK\) si son polynôme minimal dans \( \eK[X]\) est séparable sur \( \eK\) (définition~\ref{DEFooLXSBooCHIUFU}).
        \item
            L'extension \( \eL\) est \defe{séparable}{séparable!extension de corps} si tous ses éléments sont séparables.
    \end{enumerate}
\end{definition}
\index{extension!séparable}

\begin{proposition} \label{PropUmxJVw}
    Soit \( \eK\) un corps. Les conditions suivantes sont équivalentes :
    \begin{enumerate}
        \item       \label{ITEMooUSKRooDmsGmw}
            toutes les extensions algébriques de \( \eK\) sont séparables;
        \item       \label{ITEMooJGWLooKInxSG}
            tout polynôme irréductible de \( \eK[X]\) est séparable.
    \end{enumerate}
    En particulier les extensions algébriques des corps de caractéristique nulle sont toutes séparables.
\end{proposition}

\begin{proof}

    En plusieurs parties.

    \begin{subproof}
        \item[\ref{ITEMooUSKRooDmsGmw} implique~\ref{ITEMooJGWLooKInxSG}]

            Soit un polynôme irréductible \( P\) de \( \eK[X]\), et un corps de décomposition \( \eL\) de \( P\). Cela est une extension algébrique par le corolaire~\ref{CORooELAUooPQGLkR}. Elle est donc séparable par hypothèse.

            Voilà une première chose de dite.

            Maintenant, nous voudrions montrer que \( P\) est un polynôme séparable. Dans \( \eL\) nous avons
            \begin{equation}
                P=\prod_{i=1}^n(X-a_i),
            \end{equation}
            et tout le défi est de prouver que les \( a_i\) sont tous distincts.

            Soient deux racines \( a,b\in \eL\) de \( P\). Nous considérons les polynômes minimaux \( \mu_a\) et \( \mu_b\) dans \( \eK[X]\). Ces deux polynômes divisent \( P\) parce que \( P\) est à la fois dans l'idéal annulateur de \( a\) et de \( b\). Mais comme \( P\) est irréductible, il existe \( k_a,k_n\in \eK\) tels que \( P=k_a\mu_a\) et \( P=k_b\mu_b\). Donc les polynômes \( \mu_a,\mu_b\) et \( P\) sont multiples les uns des autres. Vu que \( \mu_a\) et \( \mu_n\) sont unitaires, \( \mu_a=\mu_b\).

            Nous avons :
            \begin{equation}
                P=k\mu=\prod_{i=1}^n(X-a_i).
            \end{equation}
            Or le polynôme \( \mu\) est irréductible par la proposition~\ref{PropRARooKavaIT}\ref{ItemDOQooYpLvXri}, et l'extension \( \eL\) est séparable, donc \( \mu\) n'a que des racines simples, Donc tous les \( a_i\) sont distincts.

        \item[\ref{ITEMooJGWLooKInxSG} implique~\ref{ITEMooUSKRooDmsGmw}]

            Soit une extension algébrique \( \eL\) de \( \eK\). Soit \( a\in \eL\). Nous devons prouver que le polynôme minimal de \( a\) dans \( \eK\) est séparable, c'est-à-dire qu'il n'a que des racines simples.

            Le polynôme minimal \( \mu_a\in \eK[X]\) de \( a\) est irréductible et donc séparable. Donc \( \eL\) est séparable.

    \end{subproof}

    La dernière phrase est une conséquence du corolaire~\ref{CorUjfJSE}.
\end{proof}

\begin{corollary}  \label{CORooNZZMooIoBYXY}
    Toute les extensions algébriques de \( \eQ\) sont séparables.
\end{corollary}

\begin{proof}
    Le corps \( \eQ\) est de caractéristique nulle (définition~\ref{LEMDEFooVEWZooUrPaDw}). Le corolaire~\ref{CorUjfJSE} dit alors que tout polynôme sur \( \eQ\) est séparable. La proposition~\ref{PropUmxJVw} conclut en disant que toutes les extensions algébriques de \( \eQ\) sont séparables.
\end{proof}

\begin{theorem}[\cite{rqrNyg}]      \label{ThobkwCMm}
    Soit \( \eK\) un corps (pas spécialement fini). Tout sous-groupe fini de \( \eK^*\) est cyclique.
\end{theorem}

\begin{proof}
    Soit \( G\) un sous-groupe fini de \( \eK^*\) et \( \omega\) son exposant (qui est le PPCM des ordres des éléments de \( G\)). Étant donné que \( | G |\) est divisé par tous les ordres, il est divisé par le PPCM des ordres. Bref, nous avons
    \begin{equation}
        x^{\omega}=1
    \end{equation}
    pour tout \( x\in G\). Mais ce polynôme possède au plus \( \omega\) racines dans \( \eK\). Du coup \( | G |\leq \omega\). Et comme on avait déjà vu que \( \omega\divides | G |\), on a \( \omega=| G |\). Il suffit plus que trouver un élément d'ordre effectivement \( \omega\). Cela est fait par le lemme~\ref{LemqAUBYn}.
\end{proof}

\begin{theorem}[Théorème de l'élément primitif\cite{rqrNyg}]   \label{ThoORxgBC}
    Toute extension de corps séparable finie admet un élément primitif\footnote{Définition~\ref{DefZCYIbve}.}.

    Autrement dit, soient des éléments algébriques \( \alpha_1,\ldots, \alpha_n\) séparables\footnote{Définition \ref{DEFooKTVHooTydOTM}\ref{ITEMooOFYPooLYkIPr}.} sur \( \eK\), et soit l'extension engendrée \( \eL=\eK(\alpha_1,\ldots, \alpha_n)\). Alors \( \eL \) admet un élément primitif, c'est-à-dire un élément \( \theta \) tel que \( \eL = \eK(\theta)\).
\end{theorem}
\index{théorème!élément primitif}

\begin{proof}
    Si le corps \( \eK\) est fini, alors \( \eL\) est également fini. Donc \( \eL^*\) est cyclique par le théorème~\ref{ThobkwCMm}. Si \( \theta\) est un générateur de \( \eL^*\), alors \( \eL=\eK(\theta)\).

    Passons au cas où \( \eK\) est infini. Il suffit d'examiner le cas \( n=2\); en effet pour \( n=1\) c'est trivial et si \( n>2\), alors
    \begin{equation}
        \eK(\alpha_1,\ldots, \alpha_n)=\eK(\alpha_1,\ldots, \alpha_{n-1})(\alpha_n),
    \end{equation}
    et donc si \( \eK(\alpha_1,\ldots, \alpha_{n-1})=\eK(\theta)\), nous avons
    \begin{equation}
        \eK(\alpha_1,\ldots, \alpha_n)=\eK(\theta,\alpha_n)
    \end{equation}
    et nous sommes réduit au cas \( n=2\) par récurrence.

    Soit donc \( \eL=\eK(\alpha,\beta)\); soit \( P\) le polynôme minimal de \( \alpha\) sur \( \eK\) et \( Q\) celui de \( \beta\). Nous nommons \( \eE\), un corps de décomposition de \( PQ\). Nous avons \( \eL\subset \eE\). Vu que \( P\) et \( Q\) sont polynômes minimaux d'éléments qui sont par hypothèse séparables, les polynômes \( P\) et \( Q\) sont séparables. Donc dans \( \eE\) les racines de \( P\) sont distinctes parce que \( P\) est irréductible (et idem pour \( Q\)). Soient les racines
    \begin{equation}
        \alpha_1=\alpha,\alpha_2,\ldots, \alpha_r
    \end{equation}
    de \( P\) dans \( \eE\) et les racines
    \begin{equation}
        \beta_1=\beta,\beta_2,\ldots, \beta_s
    \end{equation}
    de \( Q\) dans \( \eE\). Ici \( r\) et \( s\) sont les degrés de \( P\) et \( Q\).

    Si \( s=1\) alors \( Q=X-\beta\) et donc \( \beta\in \eK\) (parce que \( Q\in \eK[X]\)). Du coup nous avons \( \eL=\eK(\alpha)\) et le théorème est démontré. Nous supposons donc maintenant que \( s\geq 2\).

    Pour chaque \( (i,j)\in \llbracket 1,r\rrbracket \times \llbracket 2,s\rrbracket \), l'équation \( \alpha_i+x\beta_k=\alpha_1+x\beta_1\) pour \( x\in \eK\) a au plus\footnote{La solution \eqref{EqWzUFHe} peut être dans $ \eL$ et non dans $\eK$. L'équation peut donc très bien ne pas avoir de solutions $x\in \eK$.} une solution donnée le cas échéant par
    \begin{equation}    \label{EqWzUFHe}
        x=(\alpha_i-\alpha_1)(\beta_1-\beta_k)^{-1}
    \end{equation}
    Notons que cela est de toutes façons dans \( \eL\) et qu'étant donné que \( \beta_1\neq \beta_k\), cette solution a un sens (ici on utilise l'hypothèse de séparabilité). Étant donné que \( \eK\) est infini nous pouvons donc trouver un \( c\in \eK\) qui ne résout aucune des équations \eqref{EqWzUFHe} :
    \begin{equation}\label{EQooIIMVooSmvrjP}
        \alpha_i+c\beta_k\neq \alpha_1+c\beta_1.
    \end{equation}
    Nous posons \( \theta=\alpha_1+c\beta_1\) et nous prétendons que \( \eL=\eK(\theta)\).

    Pour cela, commençons par montrer que \( \beta_1 \in \eK(\theta)\). On considère, dans \( \eK(\theta)[T]\), les polynômes \( Q(T)\) et \( S(T)=P(\theta-cT)\), et on nomme \( R\) le PGCD de ces deux polynômes. Alors, une racine de \( R\) doit être une racine de \( Q\), et est donc un des \( \beta_i\). Or, d'une part, le choix de \( \theta\) fait que \( \beta_1\) est une racine de \( R\) parce que
    \begin{equation}
        S(\beta_1)=P(\theta-c\beta_1)=P(\alpha_1+c\beta_1-c\beta_1)=P(\alpha_1)=0.
    \end{equation}
    D'autre part, si \( k\geq 2\), alors
    \begin{equation}
        S(\beta_k)=P(\alpha_1 + c \beta_1 - c \beta_k) = P\big(\alpha_1 +c(\beta_1-\beta_k)\big)\neq 0
    \end{equation}
    parce que \( \alpha_1 +c(\beta_1 - \beta_k)\) ne vaut ni \( \alpha_1 \) (le second terme est non-nul), ni un autre \( \alpha_i\) (à cause de \eqref{EQooIIMVooSmvrjP}).

    Il s'ensuit que \( Q \) et \(S \) n'ont qu'une racine commune \( \beta_1 = \beta \), qui est donc l'unique racine de \( R\). Ainsi,
    \begin{equation}
        R=X-\beta\in \eK(\theta)[T],
    \end{equation}
    et donc \( \beta\in \eK(\theta)\).

    Dès lors \( \alpha=\alpha_1=\theta-c\beta\) est alors immédiatement dans \( \eK(\theta)\); puisque les deux éléments \( \alpha\) et \( \beta\) sont dans \( \eK(\theta)\), nous avons obtenu \( \eL=\eK(\alpha,\beta)=\eK(\theta)\).

\end{proof}

\begin{example}
    Le théorème de l'élément primitif~\ref{ThoORxgBC} ne tient pas pour les corps non commutatifs. Considérons par exemple le corps \( \eK\) des quaternions\index{quaternion} et le groupe à \( 8\) éléments \( G=\{ \pm 1,\pm i,\pm j,\pm k \}\). Ce dernier groupe n'est pas cyclique alors qu'il est un groupe fini dans \( \eK^*\).
\end{example}

\begin{example}
    Il est aussi possible pour un groupe fini d'avoir \( \omega(G)=| G |\) sans pour autant que \( G\) soit cyclique. Par exemple pour \( G=S_3\), nous avons \( | S_3 |=6\) alors que les éléments de \( S_3\) sont soit d'ordre \( 2\) soit d'ordre \( 3\) et \( \omega(G)=\ppcm(2,3)=6\). Pourtant \( S_3\) n'est pas cyclique.
\end{example}

%+++++++++++++++++++++++++++++++++++++++++++++++++++++++++++++++++++++++++++++++++++++++++++++++++++++++++++++++++++++++++++
\section{Idéal maximum}
%+++++++++++++++++++++++++++++++++++++++++++++++++++++++++++++++++++++++++++++++++++++++++++++++++++++++++++++++++++++++++++

%---------------------------------------------------------------------------------------------------------------------------
\subsection{Idéal maximum}
%---------------------------------------------------------------------------------------------------------------------------

\begin{definition}  \label{DefWHDdTrC}
    Une \( \eK\)-algèbre est de \defe{type fini}{type!fini!en algèbre} si elle est le quotient de \( \eK[X_1,\ldots, X_n]\) par un idéal (pour un certain \( n\)).
\end{definition}

\begin{theorem}[\cite{OorXst}]      \label{ThonoZyKa}
    Soit \( \eK\) un corps et \( B\), une \( \eK\)-algèbre de type fini. Si \( B\) est un corps, alors c'est une extension algébrique finie de \( \eK\).
\end{theorem}
%TODO : faire la démonstration

\begin{theorem}[\cite{OorXst}]  \label{ThowgZYqx}
    Si \( \eK\) est un corps algébriquement clos, les idéaux maximaux de \( \eK[X_1,\ldots, X_n]\) sont de la forme
    \begin{equation}
        (X_1-a_1,\ldots, X_n-a_n)
    \end{equation}
    où les \( a_i\) sont des éléments de \( \eK\).
\end{theorem}

\begin{proof}
    Nous commençons par montrer que
    \begin{equation}
        J=(X_1-a_1,\ldots, X_n-a_n)
    \end{equation}
    est un idéal maximum. Pour cela nous considérons le morphisme surjectif d'anneaux
    \begin{equation}
        \begin{aligned}
            \phi\colon \eK[X_1,\ldots, X_n]&\to \eK \\
            P&\mapsto P(a_1,\ldots, a_n).
        \end{aligned}
    \end{equation}
    Soit \( P\in\ker(\phi)\); nous écrivons la division euclidienne de \( P\) par \( X-a_1\) puis celle du reste par \( X-a_2\) et ainsi de suite :
    \begin{equation}    \label{EqDAkijH}
        P=(X-a_1)Q_1+\cdots +(X_n-a_n)Q_n+R
    \end{equation}
    où \( R\) doit être une constante parce que le premier reste est de degré zéro en \( X_1\), le second est de degré zéro en \( X_1\) et \( X_2\), etc. Afin d'identifier cette constante, nous appliquons l'égalité \eqref{EqDAkijH} à \( (a_1,\ldots, a_n)\) et en nous rappelant que \( P\in \ker(\phi)\) nous obtenons
    \begin{equation}
        0=P(a_1,\ldots, a_n)=R,
    \end{equation}
    donc \( R=0\) et \( P=(X_1-a_1)Q_1+\cdots +(X_n-a_n)Q_n\), c'est-à-dire \( P\in J\). Nous avons donc \( \ker(\phi)\subset J\). Par ailleurs \( J\subset \ker(\phi)\) est évident, donc \( J=\ker(\phi)\).

    Vu que \( J\) est le noyau de l'application \( \eK[X_1,\ldots, X_n]\to \eK\), nous avons
    \begin{equation}
        \frac{ \eK[X_1,\ldots, X_n] }{ J }=\eK.
    \end{equation}
    Donc \( J\) est un idéal maximal parce que tout polynôme n'étant pas dans \( J\) doit avoir un terme indépendant non nul et donc être dans \( \eK\) vis à vis du quotient \( \eK[X_1,\ldots, X_n]/J\).

    Nous montrons maintenant l'implication inverse. Nous supposons que \( I\) est un idéal maximum et nous montrons qu'il doit être égal à \( J\) (pour un certain choix de \( a_1,\ldots, a_n\)).

    Le quotient
    \begin{equation}
        \frac{ \eK[X_1,\ldots, X_n] }{ I }
    \end{equation}
    est une \( \eK\)-algèbre de type fini (définition~\ref{DefWHDdTrC}). De plus c'est un corps par la proposition~\ref{PROPooSHHWooCyZPPw}. C'est donc une extension algébrique finie de \( \eK\) par le théorème~\ref{ThonoZyKa}. Mais \( \eK\) étant algébriquement clos, il est sa propre et unique extension algébrique; nous en déduisons que
    \begin{equation}
        \frac{ \eK[X_1,\ldots, X_n] }{ I }=\eK.
    \end{equation}
    Donc pour tout \( 1\leq i\leq n\), il existe \( a_i\in \eK\) tel que \( X_i-a_i\in I\), sinon le monôme \( X_i\) ne se projetterait pas sur un élément dans \( \eK\) dans le quotient. Cela prouve que \( J\) est contenu dans \( I\); par maximalité nous avons donc \( I=J\).
\end{proof}

\begin{corollary}
    Soit \( \eK\) un corps algébriquement clos et \( I\), un idéal de \( \eK[X_1,\ldots, X_n]\). Si nous notons
    \begin{equation}
        V(I)=\{ x\in \eK^n\tq P(x_1,\ldots, x_n)=0 \}
    \end{equation}
    l'ensemble des racines communes à tous les éléments de \( I\), on a \( V(I)=\emptyset\) si et seulement si \( I=\eK[X_1,\ldots, X_n]\).
\end{corollary}

\begin{proof}
    Si \( I=\eK[X_1,\ldots, X_n]\) en particulier \( 1\in I\) et nous avons évidemment \( V(I)=\emptyset\). Le sens difficile est l'autre sens.

    Supposons que \( I\neq \eK[X_1,\ldots, X_n]\) et que \( K\) est un idéal maximum contenu dans \( I\). Nous savons déjà par le théorème~\ref{ThowgZYqx} que \( K\) est de la forme \( K=(X_1-a_1,\ldots, X_n-a_n)\). Un élément de \( I\) est dans \( K\), donc si \( P\in I\) nous avons
    \begin{equation}
        P(a_1,\ldots, a_n)=0,
    \end{equation}
    c'est-à-dire que \( (a_1,\ldots, a_n)\in V(I)\) et donc que \( V(I)\neq \emptyset \).
\end{proof}

%+++++++++++++++++++++++++++++++++++++++++++++++++++++++++++++++++++++++++++++++++++++++++++++++++++++++++++++++++++++++++++
\section{Polynômes symétriques et alternés}
%+++++++++++++++++++++++++++++++++++++++++++++++++++++++++++++++++++++++++++++++++++++++++++++++++++++++++++++++++++++++++++

%---------------------------------------------------------------------------------------------------------------------------
\subsection{Polynômes symétriques, alternés ou semi-symétriques}
%---------------------------------------------------------------------------------------------------------------------------


Nous rappelons que le groupe symétrique \( S_n\) agit sur l'anneau des polynômes de \( n\) variables sur l'anneau \( A\) par le lemme \ref{LEMooIRVQooHvoNBq}.

\begin{definition}
    Un polynôme \( Q\) en \( n\) indéterminées est
    \begin{enumerate}
        \item
            \defe{symétrique}{polynôme!symétrique}\index{symétrique!polynôme} si \( Q=\sigma\cdot Q\) pour tout \( \sigma\in S_n\);
        \item
            \defe{alterné}{polynôme!alterné}\index{alterné!polynôme} si \( \sigma\cdot Q=\epsilon(\sigma)Q\) pour tout \( \sigma\in S_n\);
        \item
            \defe{semi-symétrique}{semi-symétrique!polynôme}\index{polynôme!semi-symétrique} si \( \sigma\cdot Q=Q\) pour tout \( \sigma\in A_n\)
    \end{enumerate}
\end{definition}
Le polynôme \( T_1+T_2\) est symétrique; le polynôme \( T_1+T_2^2\) ne l'est pas.

%---------------------------------------------------------------------------------------------------------------------------
\subsection{Polynôme symétrique élémentaire}
%---------------------------------------------------------------------------------------------------------------------------

\begin{definition}  \label{DEFooTREUooZKoXeg}
    Le \( k\)-ième \defe{polynôme symétrique élémentaire}{élémentaire!polynôme symétrique}\index{polynôme!symétrique!élémentaire} à \( n\) inconnues est le polynôme
    \begin{equation}
        \sigma_k(T_1,\ldots, T_n)=\sum_{s\in F_k}\prod_{i=1}^kT_{s(i)}
    \end{equation}
    où \( F_k\) est l'ensemble des fonctions strictement croissantes \( \{ 1,2,\ldots, k \}\to\{ 1,2,\ldots, n \}\).
\end{definition}

Une autre façon de décrire ces polynômes élémentaires est
\begin{equation}
    \sigma_k=\sum_{1\leq i_1<\ldots<i_k\leq n}X_{i_1}\ldots X_{i_k}.
\end{equation}
Par exemple
\begin{subequations}
    \begin{align}
        \sigma_1(T_1,\ldots, T_n)&=T_1+T_2+\cdots +T_n\\
        \sigma_2(T_1,\ldots, T_n)&=T_1T_2+\cdots +T_1T_n+T_2T_3+\cdots +T_2T_n+\cdots +T_{n-1}T_n\\
        \sigma_n(T_1,\ldots, T_n)&=T_1\ldots T_n.
    \end{align}
\end{subequations}
En particulier, \( \sigma_2(x,y,z)=xy+yz+xz\).

\begin{theorem}[\cite{PoloPolSym}]  \label{TholReBiw}
    Si \( Q\) est un polynôme symétrique en \( T_1,\ldots, T_n\), alors il existe un et un seul polynôme \( P\) en \( n\) indéterminées tel que
    \begin{equation}
        Q(T_1,\ldots, T_n)=P\big( \sigma_1(T_1,\ldots, T_n),\ldots, \sigma_n(T_1,\ldots, T_n) \big).
    \end{equation}
\end{theorem}
%TODO : la preuve de ce théorème

\begin{example}
    Nous voulons décomposer \( P(x,y,z)=x^3+y^3+z^3\) en polynômes symétriques élémentaires, c'est-à-dire en
    \begin{subequations}
        \begin{numcases}{}
            \sigma_1=x+y+z\\
            \sigma_2=xy+xz+yz\\
            \sigma_3=xyz.
        \end{numcases}
    \end{subequations}
    Étant donné que \( P\) est de degré \( 3\), les seules combinaisons des \( \sigma_i\) qui peuvent intervenir sont \( \sigma_1^3\), \( \sigma_1\sigma_2\) et \( \sigma_3\). Étant donné que dans \( P\) le coefficient de \( x^3\) est un, il est obligatoire d'avoir un coefficient \( 1\) devant \( \sigma_1^3\). Nous le calculons :
    \begin{verbatim}
----------------------------------------------------------------------
| Sage Version 4.8, Release Date: 2012-01-20                         |
| Type notebook() for the GUI, and license() for information.        |
----------------------------------------------------------------------
sage: var('x,y,z')
(x, y, z)
sage: P=x**3+y**3+z**3
sage: S1=x+y+z
sage: S2=x*y+x*z+y*z
sage: S3=x*y*z
sage: (S1**3).expand()
x^3 + 3*x^2*y + 3*x^2*z + 3*x*y^2 + 6*x*y*z + 3*x*z^2 + y^3 
                + 3*y^2*z + 3*y*z^2 + z^3
sage: (S1**3-P).expand()
3*x^2*y + 3*x^2*z + 3*x*y^2 + 6*x*y*z + 3*x*z^2 + 3*y^2*z + 3*y*z^2
x^3 + 3*x^2*y + 3*x^2*z + 3*x*y^2 + 6*x*y*z + 3*x*z^2 
            + y^3 + 3*y^2*z + 3*y*z^2 + z^3
    \end{verbatim}
    Dans la différence \( \sigma_1^3-P\) nous voyons que le terme en \( xyz\) est \( 6xyz\); par conséquent nous savons que le coefficient de \( \sigma_3\) sera \( -6\). Il nous reste :
    \begin{verbatim}
sage: (S1**3+6*S3-P).expand()
3*x^2*y + 3*x^2*z + 3*x*y^2 + 12*x*y*z + 3*x*z^2 + 3*y^2*z + 3*y*z^2
    \end{verbatim}
    que nous identifions facilement avec \( 3\sigma_1\sigma_2\). Nous avons donc
    \begin{equation}
        P=\sigma_1^3-3\sigma_1\sigma_2+3\sigma_3.
    \end{equation}
\end{example}


\begin{lemma}[\cite{fJhCTE}]    \label{LemSoXCQH}
    Soit \( \eK\) une extension de degré \( \delta\) de \( \eQ\) et \( P\in \eK[T_1,\ldots, T_m]\). Alors il existe \( \bar P\in \eQ[T_1,\ldots, T_m]\) tel que
    \begin{enumerate}
        \item
            $\deg\bar P=\delta\deg(P)$
        \item
            pour tout \( (z_1,\ldots, z_m)\in \eC^m\) tel que \( P(z_1,\ldots, z_m)=0\), on a \( \bar P(z_1,\ldots, z_m)=0\).
    \end{enumerate}
\end{lemma}
\index{polynôme!symétrique}
\index{polynôme!racines}
\index{extension!de corps}
\index{corps!extension}

\begin{proof}
    En vertu de la proposition~\ref{PropUmxJVw} et du corolaire~\ref{CORooNZZMooIoBYXY}, \( \eK\) est une extension séparable de \( \eQ\), et donc vérifie le théorème de l'élément primitif (\ref{ThoORxgBC}). Il existe \( \theta\in \eK\) tel que \( \eK=\eQ(\theta)\). Soit \( P_{\theta}\in\eQ[X]\) le polynôme minimal de \( \theta\). L'extension \( \eK\) étant de degré \( \delta\), et \( \theta\) étant un générateur, une base de \( \eK\) comme espace vectoriel sur \( \eQ\) est
    \begin{equation}
        \{ 1,\theta,\ldots, \theta^{\delta-1} \}.
    \end{equation}
    Mais par ailleurs la proposition~\ref{PropURZooVtwNXE}\ref{ItemJCMooDgEHajiv} nous indique qu'une base de \( \eQ(\theta)\) sur \( \eQ\) est donnée par
    \begin{equation}
        \{ 1,\theta,\ldots, \theta^{n-1} \}
    \end{equation}
    où \( n\) est le degré de \( P_{\theta}\). Donc \( P_{\theta}\) est de degré \( \delta\). Nous nommons \( \theta_1,\ldots, \theta_{\delta}\) les racines de \( P_{\theta}\) dans un corps de décomposition. Ici nous notons \( \theta=\theta_1\) et nous ne prétendons pas que \( \theta_k\in \eK\). Notons que ces \( \theta_i\) sont toutes des racines simples de \( P_{\theta}\), sinon nous aurions un facteur irréductible \( (X-\theta_k)^2\), et \( P_{\theta}\) ne serait pas irréductible sur \( \eQ\).

    Soit \( \sigma_k\) le morphisme canonique
    \begin{equation}
        \begin{aligned}
            \sigma_k\colon \eQ(\theta)&\to \eQ(\theta_k) \\
            \sum_i q_i\theta^i&\mapsto \sum_iq_i\theta_k^i
        \end{aligned}
    \end{equation}
    Nous avons \( \sigma_1\colon \eK\to \eK\) qui est l'identité.

    Notons \( N\) le degré du polynôme \( P\in \eK[T_1,\ldots, T_m]\) dont il est question dans l'énoncé. Nous le décomposons alors en
    \begin{equation}
        P=\sum_{l=0}^N\sum_{i=1}^mc_{il}T_i^l
    \end{equation}
    avec \( c_{il}\in \eK\). Nous voyons \( c_{i,.}\) comme un élément de \( \eK^m\) et donc nous écrivons\footnote{Il me semble qu'il manque la somme sur \( i\) dans \cite{fJhCTE}.}
    \begin{equation}
        P=\sum_{l=0}^N\sum_{i=1}^m c_l(\theta)_iT_i^l
    \end{equation}
    où \( c_l\in \eQ[X]^m\). Nous pouvons choisir \( \deg(c_l)<\delta\) parce que les puissances plus grandes de \( \theta\) ne génèrent rien de nouveau.

    Nous posons aussi
    \begin{equation}
        P^{\sigma_k}=\sum_{l,i} c_l(\theta_k)_iT_i^l\in \eQ(\theta_k)[T_1,\ldots, T_m],
    \end{equation}
    et \( \bar P=PP^{\sigma_2}\ldots P^{\sigma_k}\). Le coefficient de \( T_i^l\) dans \( \bar P\) est
    \begin{equation}
        \bar c_l(\theta_1,\ldots, \theta_{\delta})_i=\sum_{l_1+\cdots +l_{\delta}=l}c_{l_1}(\theta_1)_i\ldots c_{l_{\delta}}(\theta_{\delta})_i.
    \end{equation}
    Ce dernier est un polynôme en les \( \theta_k\) à coefficients dans \( \eQ\). Qui plus est, c'est un polynôme symétrique. En effet un terme contenant \( \theta_k^a\theta_l^b\) provenant de \( c_{l_i}(\theta_k)c_{l_j}(\theta_l)\) a un terme correspondant \( \theta_k^b\theta_l^a\) provenant de \( c_{l_j}(\theta_k)c_{l_i}(\theta_l)\).

    C'est donc le moment d'utiliser le théorème~\ref{TholReBiw} à propos des polynômes symétriques élémentaires qui nous dit que les coefficients de \( \bar P\) sont en réalité des polynômes en ceux de \( P_{\theta}\) qui sont dans \( \eQ\). Donc \( \bar P\in \eQ[T_1,\ldots, T_m]\). Par ailleurs nous avons que
    \begin{equation}
        \deg(\bar P)=\delta \deg(P)
    \end{equation}
    parce que \( \bar P\) est le produit de \( \delta\) «copies»  de \( P\). De plus \( P=P^{\sigma_1}\) divise \( \bar P \) donc on a bien que si \( P(z)=0\) alors \( \bar P(z)=0\). Le polynôme \( \bar P\) est celui que nous cherchions.
\end{proof}

%---------------------------------------------------------------------------------------------------------------------------
\subsection{Relations coefficients racines}
%---------------------------------------------------------------------------------------------------------------------------

\begin{theorem}[Relations coeffitients-racines] \label{ThoOQRgjpl}
    Soit le polynôme \( P=a_nX^n+\cdots +a_1X+a_0\) et \( r_i\) ses \( n\) racines. Alors nous avons pour chaque \( 1\leq k\leq n\) la relation
    \begin{equation}
        \sigma_k(r_1,\ldots, r_n)=(-1)^k\frac{ a_{n-k} }{ a_n }
    \end{equation}
    où \( \sigma_k\) est le \( k\)\ieme polynôme symétrique défini en~\ref{DEFooTREUooZKoXeg}.
\end{theorem}
\index{relations!coefficient-racines}
\index{polynôme!symétrique!élémentaire}

%TODO : citer Wikipédia pour l'exemple suivant.
%TODO : ici aussi il faudra faire référence au théorème sur le fait qu'un polynôme ait toutes ses racines dans \eC.

\begin{example} \label{ExHIfHhBr}
    Soit le polynôme
    \begin{equation}
        P(x)=x^3+2x^2+3x+4
    \end{equation}
    et ses racines que nous nommons \( a,b,c\). Nous voudrions calculer \( a^2+b^2+c^2\). D'abord nous décomposons \( Q(a,b,c)=a^2+b^2+c^2\) en polynômes symétriques élémentaires : \( Q(a,b,c)=\sigma_1(a,b,c)^2-2\sigma_2(a,b,c)\).

    Mais les relations coefficients-racines\footnote{Théorème \ref{ThoOQRgjpl}} nous donnent \( \sigma_1(a,b,c)=-2\) et \( \sigma_2(a,b,c)=3\), donc
    \begin{equation}
        a^2+b^2+c^2=(-2)^2-2\cdot 3=-2.
    \end{equation}

    Cela nous assure déjà qu'au moins une des solutions n'est pas réelle.

    Nous pouvons en avoir une vérification directe en calculant explicitement les racines (ce qui est possible pour le degré \( 3\)) :
    \lstinputlisting{tex/frido/VAYVmNRpolynomeSym.py}

    Notez qu'il faut un peu chipoter pour isoler les solutions depuis la réponse de la fonction \info{solve}.
\end{example}

En suivant le même cheminement que dans l'exemple, si \( P\) est un polynôme de degré \( n\) et si \( r_i\) sont ses racines, il est facile de calculer \( Q(r_1,\ldots, r_n)\) pour n'importe quel polynôme symétrique \( Q\)

\begin{proposition}[Annulation de fonctions polynomiales\cite{WARooZoFOBn}] \label{PropTETooGuBYQf}
    Soit \( \eK\) un corps et \( P\) un polynôme à \( n\) indéterminées. Nous supposons que \(P\) s'annule sur un ensemble de la forme \( A_1\times\cdots\times A_n\) avec \( \Card(A_j)>\deg_{X_j}(P)\) pour tout \( j\). Alors \( P=0\).

    De plus si \( P=0\) alors tous ses coefficients sont nuls\footnote{L'intérêt de cela est qu'un polynôme de \( \eZ[X_1,\ldots, X_n]\) peut s'évaluer sur un élément de n'importe quel corps; il restera le polynôme nul.}.
\end{proposition}

\begin{proof}
    Nous prouvons le résultat par récurrence sur le nombre \( n\) d'indéterminées. Si \( n=1\), cela est le théorème~\ref{ThoLXTooNaUAKR}. Nous classons les monômes du polynôme \( P\) par ordre de puissance de \( X_n\) et nous le factorisons :
    \begin{equation}
        P=\sum_{i=1}^mP_iX_n^i
    \end{equation}
    avec \( P_i\in \eK[X_1,\ldots, X_{n-1}]\). Soit \( (a_1,\ldots, a_{n-1})\in A_1\times \ldots \times A_{n-1}\) et posons
    \begin{equation}
        Q(T)=P(a_1,\ldots, a_{n-1},T)= \sum_{i=1}^mP_i(a_1,\ldots, a_{n-1})T^i.
    \end{equation}
    Le polynôme \( Q\) s'annule sur \( A_n\) avec \( \deg(Q)=\deg_{X_n}(P)<\Card(A_n)\) et le théorème~\ref{ThoLXTooNaUAKR} nous donne \( Q=0\). Or les coefficients des différentes puissances de \( T\) dans \( Q(T) \) sont les \( P_i(a_1,\ldots, a_{n-1})\); ils sont donc nuls.

    Nous avons montré que le polynôme \( P_i\) s'annule pour tout élément de \( A_1\times \ldots \times A_{n-1}\), mais nous avons
    \begin{equation}
        \deg_{X_j}(P_i)\leq \deg_{X_j}P<\Card(A_j),
    \end{equation}
    donc l'hypothèse de récurrence donne \( P_i=0\). Par suite, \( P=0\) également.
\end{proof}

%+++++++++++++++++++++++++++++++++++++++++++++++++++++++++++++++++++++++++++++++++++++++++++++++++++++++++++++++++++++++++++
\section{Minuscule morceau sur la théorie de Galois}
%+++++++++++++++++++++++++++++++++++++++++++++++++++++++++++++++++++++++++++++++++++++++++++++++++++++++++++++++++++++++++++

Vous trouverez des détails et des preuves à propos de la théorie de Galois dans \cite{GalIEl,rqrNyg}.

\begin{definition}
    Soit $\eK$, un corps.

    Le \defe{groupe de Galois}{groupe!de Galois} d'une extension \( \eL\) de \( \eK\) est le groupe des automorphismes de \( \eL\) laissant \( \eK\) invariant.

    Le groupe de Galois d'un polynôme sur \( \eK\) est le groupe de Galois de son corps de décomposition sur \( \eK\).
\end{definition}

\begin{definition}
    Des éléments \( b_1,\ldots, b_n\) d'une extension de \( \eK\) sont \defe{algébriquement indépendants}{algébriquement!indépendant}\index{indépendance!algébrique} si ils ne satisfont à aucune relation du type
    \begin{equation}
        \sum \alpha_{i_1\ldots i_n}b_1^{i_1}\ldots b_n^{i_n}=0
    \end{equation}
    avec \( \alpha_{i_1\ldots i_n}\in \eK\).
\end{definition}

Nous disons que l'équation
\begin{equation}
    x^n+a_{n-1}x^{n-1}+\cdots+a_1x+a_0=0
\end{equation}
est l'\defe{équation générale}{equation@équation!générale de degré $n$} de degré \( n\) si les coefficients \( a_i\) sont algébriquement indépendants sur \( \eK\).

\begin{theorem}
    Le groupe de Galois d'un polynôme de degré \( n\) est isomorphe au groupe symétrique \( S_n\).
\end{theorem}

\begin{corollary}[\cite{FWZHooBLvuCJ}]
    L'équation générale de degré \( n\) est résoluble par radicaux si et seulement si \( n\le 5\).
\end{corollary}


\chapter{Topologie générale}
% This is part of Mes notes de mathématique
% Copyright (c) 2008-2019
%   Laurent Claessens, Carlotta Donadello
% See the file fdl-1.3.txt for copying conditions.

%+++++++++++++++++++++++++++++++++++++++++++++++++++++++++++++++++++++++++++++++++++++++++++++++++++++++++++++++++++++++++++
\section{Éléments généraux de topologie}
%+++++++++++++++++++++++++++++++++++++++++++++++++++++++++++++++++++++++++++++++++++++++++++++++++++++++++++++++++++++++++++

%---------------------------------------------------------------------------------------------------------------------------
\subsection{Définitions et propriétés de base}
%---------------------------------------------------------------------------------------------------------------------------

\begin{definition}[\cite{BIBooAYWDooPwDIOH}]		\label{DefTopologieGene}
Soit \( X \), un ensemble et \( \mT \), une partie de l'ensemble de ses parties qui vérifie les propriétés suivantes.
\begin{enumerate}
\item
  Les ensembles \( \emptyset \) et \( X \) sont dans \( \mT \),
\item
  Une union quelconque\footnote{Par «quelconque» nous entendons vraiment quelconque : c'est-à-dire indicée par un ensemble qui peut autant être \( \eN\) que \( \eR\) qu'un ensemble encore considérablement plus grand.} d'éléments de \( \mT\) est dans \( \mT\).
\item
  Une intersection \emph{finie} d'éléments de \( \mT\) est dans \( \mT\).
\end{enumerate}
Un tel choix \( \mT \) de sous-ensembles de \( X \) est une  \defe{topologie}{topologie} sur \( X \), et les éléments de \( \mT \) sont appelés des \defe{ouverts}{ouvert}. On dit aussi que \( (X,\mT) \) (voire simplement \( X \) lorsqu'il n'y a pas d'ambiguïté) est un \defe{espace topologique}{espace topologique}.
\end{definition}

\begin{definition}		\label{DefFermeVoisinage}
Si \(X \) est un espace topologique, un sous-ensemble \( F \) de \( X \) est dit \defe{fermé}{fermé} si son complémentaire, \( F^c \), est ouvert.

Si \(a \in X\), on dit que \(V \subset X\) est un \defe{voisinage}{voisinage} de \(a\) s'il existe un ouvert \(\mO \in \mT\) tel que \(a \in \mO\) et \(\mO \subset V\).
\end{definition}

\begin{lemma}   \label{LemQYUJwPC}
    Union et intersection de fermés.
    \begin{enumerate}
        \item
            Une intersection quelconque de fermés est fermée.
        \item       \label{ItemKJYVooMBmMbG}
            Une union finie de fermés est fermée.
    \end{enumerate}
\end{lemma}

\begin{proof}
    Soit \( \{ F_i \}_{i\in I} \) un ensemble de fermés; nous avons
    \begin{equation}
        \left( \bigcap_{i\in I}F_i \right)^c=\bigcup_{i\in I}F_i^c.
    \end{equation}
    Le membre de droite est une union d'ouverts, c'est donc un ouvert; donc l'intersection qui apparaît dans le membre de gauche est le complémentaire d'un ouvert: c'est donc un fermé.

    De la même manière, le complémentaire d'une union finie de fermés est une intersection finie de complémentaires de fermés, et est donc ouvert\footnote{Un bon exercice est d'écrire ces unions et intersections, pour se convaincre que ça fonctionne.}.
\end{proof}

Dans un espace topologique, nous avons une caractérisation très importante des ouverts.
\begin{theorem}		\label{ThoPartieOUvpartouv}
    Une partie d'un espace topologique est ouverte si et seulement si elle contient un voisinage\footnote{Définition~\ref{DefFermeVoisinage}.} ouvert de chacun de ses éléments.
\end{theorem}

\begin{proof}
    Soit \( X\) un espace topologique et \( A\subset X\). Le sens direct est évident : $A$ lui-même est un ouvert autour de $x\in A$, qui est inclus dans $A$.

Pour le sens inverse, nous supposons que \( A\) contienne un ouvert autour de chacun de ses points. Pour chaque $x\in A$, choisissons $\mO_x\subset A$ un ouvert autour de $x$. Alors,
\begin{equation}	\label{EqAUniondesOx}
	A=\bigcup_{x\in A}\mO_x
\end{equation}
en effet, d'une part, $A\subset\bigcup_{x\in A}\mO_x$ parce que chaque élément $x$ de $A$ est dans le $\mO_x$ corrrespondant, par construction; et d'autre part, $\bigcup_{x\in A}\mO_x\subset A$ parce que chacun des $\mO_x$ est inclus dans $A$.

Ainsi, $A$ est égal à une union d'ouverts, cela prouve que $A$ est un ouvert.
\end{proof}
Le lemme \ref{LemMESSExh} est une version particulière de celui-ci, pour l'espace topologique \( \eR \). Une autre application typique est la proposition~\ref{PropMMKBjgY} et le théorème~\ref{ThoESCaraB}.

%---------------------------------------------------------------------------------------------------------------------------
\subsection{Quelques exemples}
%---------------------------------------------------------------------------------------------------------------------------

\subsubsection{Une première vague}
%///////////////////////////

\begin{example}\label{DefTopologieGrossiere}
  Pour un ensemble \( X \) quelconque, on considère l'ensemble \( \mT = \{ \emptyset; X\} \). Avec cet ensemble, on confère à \(X \) une structure d'espace topologique - même si elle nous apprend peu de choses\dots{} La topologie ainsi posée sur \(X \) est appelée \defe{topologie grossière}{topologie!grossière}.
\end{example}

\begin{example}\label{DefTopologieDiscrete}
  Pour un ensemble \( X \) quelconque, on considère l'ensemble \( \mT \) constitué de toutes les parties de \( X \). Avec cet ensemble, on confère à nouveau une structure d'espace topologique à \(X \); toutes les parties sont des ouverts, et aussi des fermés. La topologie ainsi posée sur \(X \) est appelée \defe{topologie discrète}{topologie!discrète}.
\end{example}

\begin{example} [Toutes les topologies d'un ensemble à 3 éléments]
    On pose \( X = \{1, 2, 3\} \). Alors on peut munir \( X \) de 29 topologies différentes\cite{BIBooSLBZooRYtdIi}; saurez-vous les retrouver toutes?
\end{example}

\subsubsection{Topologie engendrée}
%//////////////////////////

\begin{propositionDef}[Topologie engendrée, prébase\cite{BIBooTAMKooWwOwAL}]\label{DefTopologieEngendree}
    Soient un ensemble \( X \) et \( \mT_0 \) un ensemble de parties de \( X \). Nous définissons \( \tau(\mT_0)\) comme étant l'union quelconque d'intersections finies d'éléments de \( \mT_0 \). 

    Plus précisément, nous faisons les constructions suivantes :
    \begin{enumerate}
        \item
            Nous notons \( \{\mO_i \}_{i\in I}\) les éléments de \( T_0\) indexés par l'ensemble \( I\).
        \item 
            Soit  \( B(\mT_0)\) l'ensemble des intersections finies d'éléments de \( \mT_0\) :
            \begin{equation}
                B(\mT_0)=\big\{ \bigcap_{j\in J}\mO_j \big\}_{J\text{ fini dans }I}
            \end{equation}
            où nous convenons que \( \bigcap_{j\in\emptyset \mO_j}=X\)\footnote{Bref, nous mettons \( X\) dans \( B(\mT_0)\).}.
        \item
            Soit \( A\) un ensemble qui indexe \(   B(\mT_0) \) :
            \begin{equation}
                B(\mT_0)=\{ B_{\alpha} \}_{\alpha\in A}.
            \end{equation}
        \item 
            Nous posons
            \begin{equation}
                \tau(\mT_0)=\big\{    \bigcup_{\alpha\in S}B_{\alpha}   \big\}_{S\subset A}.
            \end{equation}
    \end{enumerate}
    Alors \( \tau(\mT_0) \) est une topologie sur \(X\), qu'on appelle \defe{topologie engendrée}{topologie!engendrée par une famille} par \( \mT_0 \). La partie \( \mT_0\) est appelée \defe{prébase}{prébase} de la topologie \(  \tau(\mT_0)  \).
\end{propositionDef}

\begin{proof}
    Nous devons montrer les différents points de la définition \ref{DefTopologieGene} d'une topologie.
    \begin{enumerate}
        \item
            L'ensemble vide est dans \( \tau(\mT_0)\) parce que \( \emptyset=\bigcup_{\alpha\in \emptyset}B_{\alpha}\). L'ensemble \( X\) est également dans \( \tau(\mT_0)\) parce que \( X\in B(\mT_0)\).

        \item
            Soient \( \{ D_l \}_{l\in L}\) des éléments de \( \tau(\mT_0)\) indexés par un ensemble \( L\). Pour chaque \( l\) nous avons un ensemble \( S\subset A\) tel que \( D_l=\bigcup_{\alpha\in S_l}B_{\alpha}\). En posant \( S=\bigcup_{l\in L}S_l\) nous avons
            \begin{equation}
                \bigcup_{l\in L} D_l=\bigcup_{\alpha\in S}B_{\alpha}\in \tau(\mT_0).
            \end{equation}
            Donc \( \tau(\mT_0)\) est stable par union quelconque.
        \item
            Soient \( D_1\) et \( D_2\) des éléments de \( \tau(\mT_0)\). Nous posons \( D_i=\bigcup_{\alpha\in S_i}B_{\alpha}\). Alors nous avons
            \begin{equation}        \label{EQooUCJOooCbKVpw}
                \bigcup_{\alpha\in S_1}B_{\alpha}\cap\bigcup_{\beta\i S_2}B_{\beta}=\bigcup_{\alpha,\beta\in S_1\times S_2}(B_{\alpha}\cap B_{\beta}).
            \end{equation}
            Mais \( B_{\alpha}\) et \( B_{\beta}\) sont dans \( B(\mT_0)\). Donc \( B_{\alpha}\cap B_{\beta}\in B(\mT_0)\). Donc \eqref{EQooUCJOooCbKVpw} est un union d'éléments de \( B(\mT_0)\).
    \end{enumerate}
    Au final nous avons prouvé que \( \tau(\mT_0)\) est une topologie sur \( X\).
\end{proof}

\begin{lemma}
    Soient un ensemble \( X\) et un ensemble \( \mT_0\) de parties de \( X\). Toute topologie sur \( X\) contenant \( \mT_0\) contient \( \tau(\mT_0)\).
\end{lemma}

\begin{proof}
    Soit une topologie \( \mu\) sur \( X\) contenant \( \tau(\mT_0)\). Vu que \( \mu\) est une topologie, les intersections finies d'éléments de \( \mu\) sont dans \( \mu\), donc, en suivant les notations de \ref{DefTopologieEngendree}, \( B(\mT_0)\subset \mu\).

    Vu que toutes les unions d'éléments de \( \mu\) sont dans \( \mu\), l'inclusion de \( B(\mT_0)\) dans \( \mu\) implique celle de \( \tau(\mT_0)\).
\end{proof}

La proposition suivante montre que vérifier la convergence d'une suite sur une prébase suffit pour vérifier la convergence.
\begin{proposition}     \label{PROPooJTJBooNtczsO}
    Soit \( \mT_0\) un ensemble de parties de l'ensemble \( X\). Soient une suite \( (x_n)\) dans \( X\) ainsi que \( x\in X\). Nous supposons que la suite \( (x_n)\) satisfasse la propriété suivante : pour tout \( A\in \mT_0\) tel que \( x\in A\), il existe \( K\in \eN\) tel que \( k\geq K\) implique \( x_k\in A\). 

    Alors nous avons la convergence de suite\footnote{Définition \ref{DefXSnbhZX}.}
    \begin{equation}
        x_n\stackrel{\big( X,\tau(\mT_0) \big)}{\longrightarrow}x.
    \end{equation}
\end{proposition}

\begin{proof}
    Nous considérons la topologie \( \tau(\mT_0)\) sur \( X\). Soit un ouvert \( \mO\) contenant \( x\). Nous le décomposons en suivant (à l'envers) la construction de la définition \ref{DefTopologieEngendree} :
    \begin{equation}
        \mO=\bigcup_{\alpha\in S}B_{\alpha}
    \end{equation}
    avec \( B_{\alpha}\in B(\mT_0)\). Donc pour chaque \( \alpha\), il existe un ensemble fini \( J_{\alpha}\) tel que
    \begin{equation}
        B_{\alpha}=\bigcap_{j\in J_{\alpha}}A_j
    \end{equation}
    avec \( A_j\in \mT_0\). Vu que \( x\in \mO\), nous avons un \( \alpha_0\) tel que \( x\in B_{\alpha_0}\). Donc \( x\in A_j\) pour tous les \( j\in J_{\alpha_0}\). 
    
    Pour chaque \( j\in J_{\alpha_0}\), il existe \( K_j\in \eN\) tel que \( k\geq K_j\) implique \( x_k\in A_j\). Vu que \( J_{\alpha_0}\) est un ensemble fini, nous pouvons poser \( K=\max_{j\in J_{\alpha_0}}K_j\).

    Maintenant, si \( k\geq K\), nous avons \( x_k\in A_j\) pour tout \( j\), et donc \( x_k\in B_{\alpha_0}\). Par conséquent aussi \( x_k\in \mO\).
\end{proof}


\subsubsection{Topologie produit}

\begin{definition}[Produit d'espaces topologiques, thème~\ref{THEMEooYRIWooDXZnhX}]      \label{DefIINHooAAjTdY}
    Soient \( X_1\),\ldots, \( X_n\) des espaces topologiques. Leur \defe{produit}{produit!espaces topologiques}\index{topologie!produit} est l'ensemble
    \begin{equation}
        X=\prod_{i=1}^nX_i
    \end{equation}
    muni de la topologie engendrée\footnote{Définition \ref{DefTopologieEngendree}.} par les produits \(A_1\times \cdots\times A_n\), avec \( A_i\in X_i \) ouverts de chacun des ensembles.
\end{definition}

\begin{proposition}[Convergence composante par composante]
    Soient des espaces topologiques \( X_i\) (\( i=1,\ldots, n\)) et une suite \( (a^{(1)}_k,\ldots, a^{(n)}_k)\) dans \( X_1\times\ldots \times X_n\). Nous avons la convergence
    \begin{equation}
        (a^{(1)}_k,\ldots, a^{(n)}_k)\stackrel{X_1\times\ldots \times X_n}{\longrightarrow}(a^{(1)},\ldots, a^{(n)})
    \end{equation}
    si et seulement si \( a^{(i)}_k\to a^{(i)}\) pour chaque \( i\).
\end{proposition}

\begin{proof}
    En deux parties.
    \begin{subproof}
        \item[Sens direct]
            Soient des ouverts \( \mO_i\) autour de \( a^{(i)}\) dans \( X_i\). Vu que \( \mO_1\times \ldots\times \mO_n\) est un ouvert autour de \( (a^{(1)},\ldots, a^{(n)})\), il existe \( K\in \eN\) tel que si \( k\geq K\) nous avons \( (a^{(1)}_k,\ldots, a^{(n)}_k)\in \mO_1\times \ldots \times \mO_n\). Pour ce \( K\) nous avons séparément \( a^{(i)}_k\in \mO_i\) pour chaque \( i\).

        \item[Sens inverse]
            Une prébase de la topologie sur \( X_1\times \ldots\times X_n\) est donné par les \( \mO_1\times \ldots \times \mO_n\) où \( \mO_i\) est un ouvert de \( X_i\). Voir la définition \ref{DefIINHooAAjTdY} de la topologie produit et la définition \ref{DefTopologieEngendree} de ce qu'est une prébase.

            La proposition \ref{PROPooJTJBooNtczsO} nous permet de ne vérifier la convergence de \( (a^{(1)}_k,\ldots, a^{(n)}_k)\) que sur la prébase. Soit donc \(\mO= \mO_1\times \ldots \mO_n\) avec \( (a^{(1)},\ldots, a^{(n)})\in \mO\). Vu que \( (a^{(i)}_k)_{k\in \eN}\to a^{(i)}\), pour chaque \( i\), il existe \( K_i\in \eN\) tel que si \( k\geq K_i\) alors \( a^{(i)}_k\in \mO_i\).

            En posant \( K=\max_i(K_i)\), nous avons \( (a^{(1)}_k,\ldots, a^{(n)}_k)\in \mO_1\times \ldots \mO_n\) pour tout \( k\geq K\).

            La proposition \ref{PROPooJTJBooNtczsO} permet de conclure.
    \end{subproof}
\end{proof}

\subsubsection{Topologie induite}
%//////////////////////////

\begin{definition}[Topologie induite] \label{DefVLrgWDB}
  Soit un espace topologique \( (X, \mT) \), et soit \( Y \subset X \). Alors on peut munir \( Y \) de la topologie constituée des \( Y \cap \mO \), pour \( \mO \in \mT \): c'est ce qu'on appelle la \defe{topologie induite}{topologie!induite}.
\end{definition}

\begin{lemma}[\cite{MonCerveau}]        \label{LemBWSUooCCGvax}
    Soit \( (X,\tau_X)\) un espace topologique et \( S\subset X\), un fermé de \( X\) sur lequel nous considérons la topologie induite \( \tau_S\). Si \( F\) est un fermé de \( (S,\tau_S)\) alors \( F\) est fermé de \( (X,\tau_X)\).
\end{lemma}

\begin{proof}
    Nous savons que le complémentaire de \( F\) dans \( S\) est un ouvert de \( (S,\tau_S)\) : il existe un ouvert \( \Omega\in \tau_X\) tel que \( S\setminus F=S\cap \Omega\). Si maintenant nous considérons le complémentaire de \( S\) dans \( X\) nous avons
    \begin{equation}
        F^c=(S\setminus F)\cup (X\setminus S)=(S\cap \Omega)\cup S^c=(S\cap \Omega)\cup(S^c\cap \Omega)\cup S^c=\Omega\cup S^c.
    \end{equation}
    Vu que \( \Omega\) et \( S^c\) sont des ouverts de \( X\), l'union est un ouvert. Donc \( F^c\in \tau_X\) et \( F\) est un fermé de \( X\).
\end{proof}

\begin{lemma}       \label{LemkUYkQt}
    Si \( B\subset A\) alors la fermeture de \( B\) pour la topologie de \( A\) (induite de \( X\)) que nous noterons \( \tilde B\) est
    \begin{equation}
        \tilde B=\bar B\cap A
    \end{equation}
    où \( \bar B\) est la fermeture de \( B\) pour la topologie de \( X\).
\end{lemma}

\begin{proof}
    Si \( a\in \bar B\cap A\), un ouvert de \( A\) autour de \( a\) est un ensemble de la forme \( \mO\cap A\) où \( \mO\) est un ouvert de \( X\). Vu que \( a\in\bar B\), l'ensemble \( \mO\) intersecte \( B\) et donc \( (\mO\cap A)\cap B\neq \emptyset\). Donc \( a\) est bien dans l'adhérence de \( B\) au sens de la topologie de \( A\).

    Pour l'inclusion inverse, soit \( a\in \tilde  B\), et montrons que \( a\in \bar B\cap A\). Par définition \( a\in A\), parce que \( \tilde B\) est une fermeture dans l'espace topologique \( A\). Il faut donc seulement montrer que \( a\in\bar B\). Soit donc \( \mO\) un ouvert de \( X\) contenant \( a\); par hypothèse \( \mO\cap A\) intersecte \( B\) (parce que tout ouvert de \( A\) contenant \( a\) intersecte \( B\)). Donc \( \mO\) intersecte \( B\). Cela signifie que tout ouvert (de \( X\)) contenant \( a\) intersecte \( B\), ou encore que \( a\in \bar B\).
\end{proof}

Si \( A\) est un ouvert de \( X\), on pourrait croire que la topologie induite n'a rien de spécial. Il est vrai que \( B\) sera ouvert dans \( A\) si et seulement s'il est ouvert dans \( X\), mais certaines choses surprenantes se produisent tout de même.

\begin{example} \label{ExloeyoR}
Prenons \( X=\eR\) et \( A=\mathopen] 0 , 1 \mathclose[\). Si \( B=\mathopen] \frac{ 1 }{2} , 1 \mathclose[ \), alors la fermeture de \( B\) dans \( A\) sera \( \tilde B=\mathopen[ \frac{ 1 }{2} , 1 [\) et non \( \mathopen[ \frac{ 1 }{2} , 1 \mathclose]\) comme on l'aurait dans \( \eR\).
\end{example}

Prendre la topologie induite de \( \eR\) vers un fermé de \( \eR\) donne des boules un peu spéciales comme le montre l'exemple suivant.

\begin{example}  \label{ExKYZwYxn}
    Quid de la boule ouverte \( B(1,\epsilon)\) dans le compact \( \mathopen[ 0 , 1 \mathclose]\) ? Par définition c'est
    \begin{equation}
        B(1,\epsilon)=\{ x\in\mathopen[ 0 , 1 \mathclose]\tq | x-1 |<\epsilon \}=\mathopen] 1-\epsilon , 1 \mathclose].
    \end{equation}
    Oui, cela est \emph{ouvert} dans \( \mathopen[ 0 , 1 \mathclose]\). C'est d'ailleurs un des ouverts de la topologie induite de \( \eR\) sur \( \mathopen[ 0 , 1 \mathclose]\).

    Donc pour la topologie de \( \mathopen[ 0 , 1 \mathclose]\), toutes les boules ouvertes \( B(x,\delta)\) avec \( x\in\mathopen[ 0 , 1 \mathclose]\) sont incluses à \( \mathopen[ 0 , 1 \mathclose]\).
\end{example}


%---------------------------------------------------------------------------------------------------------------------------
\subsection{Adhérence, fermeture, intérieur, point d'accumulation et isolé}
%---------------------------------------------------------------------------------------------------------------------------

\subsubsection{Intérieur}
%///////////////////////

\begin{definition}      \label{DEFooSVWMooLpAVZRInt}
    Soient un espace topologique \( X\) et une partie \( A\) de \( X\).
    \begin{enumerate}
        \item
            Un point \( x\in X\) est \defe{intérieur}{point intérieur} à \( A\) s'il est contenu dans un ouvert inclus dans \( A\). L'ensemble des points intérieurs de \( A\) est noté $\Int(A)$.\nomenclature[T]{$\Int(A)$}{intérieur de \( A\)}
        \item
            L'\defe{intérieur}{intérieur} de \( A\), notée \( \mathring A\), est l'union de tous les ouverts de \( X\) contenus dans \( A\).
    \end{enumerate}
\end{definition}
\begin{remark}\label{RemIntOuvert}
Quelques remarques en vrac.
\begin{enumerate}
\item Pour tout \( A \subset X\), l'ensemble \( \mathring A\) est un ouvert, comme union quelconque d'ouverts.

\item Par ailleurs, on a \( \mathring A = \Int A \): en effet, \( x \in \Int A \) si et seulement s'il existe un ouvert contenant \( x \) et inclus dans \( A \), si et seulement si \( x \) est dans l'union de tous les ouverts contenus dans \( A \), si et seulement si \( x \in \mathring A \).

\item On a  \( \mathring A \subset A \), et \( \mathring A = A \) si et seulement si $A$ est un ouvert: en sens direct, c'est clair par égalité d'ensembles; en sens inverse, c'est aussi clair puisque $A$ est alors un ouvert contenu à $A$, donc  \( A \subset \mathring A \).
\end{enumerate}
\end{remark}

\subsubsection{Adhérence et fermeture}
%///////////////////////

Disons-le tout de suite : «adhérence» et «fermeture» sont synonymes. Dans le Frido, nous allons nous évertuer à utiliser le mot «adhérance» et la notation \( \Adh(A)\) au lieu de \( \bar A\) que l'on rencontre assez souvent. Le fait que est \( \bar z\) est le conjugué complexe de \( z\). Dans certains cas, ça peut mener à des confusions.
\begin{definition}      \label{DEFooSVWMooLpAVZR}
    Soient un espace topologique \( X\) et une partie \( A\) de \( X\). Un point \( x\in X\) est \defe{adhérent}{point adhérent} à \( A\) si tout ouvert de \( X\) contenant \( x\) a une intersection non vide avec \( A\). L'ensemble des points d'adhérence de \( A\) est noté $\Adh(A)$.\nomenclature[T]{$\Adh(A)$}{adhérence de \( A\)}
\end{definition}

\begin{lemma}       \label{LEMooILNCooOFZaTe}
    L'adhérence de \( A\) est l'intersection de tous les fermés de \( X\) contenant \( A\).

    Par ailleurs, nous avons le lien
    \begin{equation}
      (\Int(A))^c = \Adh(A^c).
    \end{equation}
\end{lemma}

\begin{proof}
    Commençons par prouver la dernière égalité d'ensembles. On a les équivalences entre les éléments suivants, pour tout $x \in X$:
    \begin{itemize}
    \item $x$ n'est pas dans $\mathring A$;
    \item il n'y a aucun ouvert contenant $x$ et inclus dans $A$;
    \item tout ouvert contenant $x$ a une intersection non-vide avec $A^c$;
    \item $x$ est dans $\widebar{A^c}$.
    \end{itemize}
    Nous allons à présent montrer l'égalité d'ensembles \( \Adh(A)=\bar A \) en prouvant la double inclusion par contraposée.
    \begin{subproof}
        \item[Si \( x\in \bar A\) alors \( x\in\Adh(A)\)]
            Si \( x\) n'est pas dans \( \bar A\) alors nous avons un fermé \( F\) contenant \( A\) et pas \( x\). Le complémentaire \( F^c\) est un ouvert qui contient \( x\) et dont l'intersection avec \( A\) est vide. Donc \( x\) n'est pas dans \( \Adh(A)\).

        \item[Si \( x\in\bar A\) alors \( x\in \Adh(A)\)]

            Si \( x\) n'est pas dans \( \Adh(A)\) alors il existe un ouvert \( \mO\) contenant \( x\) et n'intersectant pas \( A\). Le complémentaire \( \mO^c\) est un fermé qui contient \( A\) et qui ne contient pas \( x\).

            Vu que \( \bar A\) est l'intersection de tous les fermés contenant \( A\), nous avons \( \bar A\subset\mO^c\) et donc \( x\) n'est pas dans \( \bar A\).
    \end{subproof}
\end{proof}

\begin{remark}\label{RemAdhFerme}
  Comme corolaire du lemme précédent, et grâce aux remarques faites pour les intérieurs, on obtient que pour \( A \subset X \) :
  \begin{enumerate}
  \item l'ensemble \( \bar A \) est fermé: c'est en effet le complémentaire d'un ouvert, précisément l'intérieur de \( A^c \);
  \item \( A \) est fermé si et seulement si \( \bar A = A \): en effet, \( A \) est fermé si et seulement si \( A^c \) est ouvert, si et seulement si l'intérieur de \( A^c \) est \( A^c \) lui-même; or, l'intérieur de \( A^c \) est le complémentaire de \( \bar A \) par le lemme \ref{LEMooILNCooOFZaTe}, si bien que \( A \) est fermé si et seulement si \( (\bar A)^c  = A^c \), ou encore\dots{} ce qu'on affirmait au début.
  \end{enumerate}
\end{remark}

\begin{definition}\label{DefEnsembleDense}
  Soit \( X \) un espace topologique. Un sous-ensemble \( A \) de \( X \) est \defe{dense}{dense} dans \( X \) si \( \bar A = X\). 
\end{definition}

\subsubsection{Frontière}
%/////////////////////////

\begin{definition}
  Soit \( X \) un espace topologique, et \( A \subset X \). La \defe{frontière}{frontière} de \( A \), notée \( \partial A \), est l'ensemble des points adhérents de \( A \) qui ne sont pas intérieurs à \( A \). Ainsi,
  \begin{equation}
    \partial A = \bar A \setminus \mathring A.
  \end{equation}
\end{definition}

\subsubsection{Points d'accumulation et isolés}
%/////////////////////////

\begin{definition}      \label{DEFooGHUUooZKTJRi}
    Soient un espace topologique \( X\) et une partie \( A\) de \( X\). Un point \( s\in X \) est un \defe{point d'accumulation}{point d'accumulation} de \( A\) si tout ouvert contenant \( s\) contient au moins un élément de \( A\setminus\{ s \}\).
\end{definition}

Quelle est la différence entre un point d'accumulation et un point d'adhérence ? La différence est que tous les points de \( A\) sont des points d'adhérence de \( A\), parce que tout voisinage de \( a\in A\) contient au moins \( a\) lui-même, alors que certains points de \( A\) peuvent ne pas être des points d'accumulation de \( A\). Voir l'exemple \ref{EXooWOYQooJolaTV}.

Notons qu'un point d'accumulation de \( A\) dans \( X\) n'est pas spécialement dans \( A\).

\begin{definition}      \label{DEFooXIOWooWUKJhN}
    Soient un espace topologique \( X\) et une partie \( A\) de \( X\). Un point \( s\in A \) est un \defe{point isolé}{point isolé} de \( A\) si il existe un voisinage ouvert \( \mO\) de \( s\) dans \( X\) tel que \( A\cap\mO=\{ s \}\).
\end{definition}

%+++++++++++++++++++++++++++++++++++++++++++++++++++++++++++++++++++++++++++++++++++++++++++++++++++++++++++++++++++++++++++ 
\section{Suites et convergence}
%+++++++++++++++++++++++++++++++++++++++++++++++++++++++++++++++++++++++++++++++++++++++++++++++++++++++++++++++++++++++++++

\begin{normaltext}
    À propos de notations. La pire notation possible pour une suite est \( (a_n)_n\). Mais que vient faire le second indice \( n\) ? Il peut être raisonnable d'écrire \( (a_n)_{n\in I}\) lorsqu'on veut dire dans quel ensemble se déplace \( n\). Si nous parlons de \emph{suite}, il faut une sérieuse raison de prendre autre chose que \( \eN\) comme ensemble d'indices.

    Une suite étant une fonction, de la même façon qu'on ne devrait pas dire «la fonction \( f(x)\)», mais «la fonction \( f\)» ou «la fonction \( x\mapsto f(x)\)», nous devrions simplement écrire \( a\) pour désigner la suite dont les éléments sont \( a_n\). 

    Par conséquent, il est parfaitement légal, et même conseillé, d'écrire «\( a+b\)» pour la somme des suites \( a\) et \( b\). Et il est tout aussi légal d'écrire «\( \lim a\)» au lieu de \( \lim_{n\to \infty} a_n\).

    Le hic est que nous écrivons souvent \( x\) la limite de la suite \( n\mapsto x_n\). Dans ce cas, nous sommes évidemment obligé d'écrire l'indice \( n\) pour parler de la suite.

    Tout cela pour dire qu'il faut être souple avec les notations.
\end{normaltext}

Dès que nous avons une topologie nous avons une notion de convergence de suite.
\begin{definition}[Convergence de suite] \label{DefXSnbhZX}
    Une suite $(x_n)$ d'éléments de $E$ \defe{converge}{convergence!de suite} vers un élément $y$ de $E$ si pour tout ouvert $\mO$ contenant $y$, il existe un $K\in \eN$ tel que $k>K$ implique $x_k\in\mO$.
\end{definition}
\index{limite!de suite!espace topologique}

%--------------------------------------------------------------------------------------------------------------------------- 
\subsection{Convergence dans un fermé}
%---------------------------------------------------------------------------------------------------------------------------

\begin{proposition}[\cite{MonCerveau}]      \label{PROPooBBNSooCjrtRb}
    Une suite contenue dans un fermé ne peut converger que vers un élément de ce fermé.
\end{proposition}

\begin{proof}
    Soient un espace topologique \( X\) et un fermé \( F\) dans \( X\). Nous supposons que la suite \( (x_k)\) soit contenue dans \( F\). Nous allons prouver qu'aucun élément de \( F^c\) ne peut être limite.

    Soit \( a \in F^c\). Vu que le complémentaire de \( F\) est un ouvert, et vu le théorème \ref{ThoPartieOUvpartouv}, il existe un ouvert \( \mO_a\) contenant \( a\), et contenu dans \( F^c\). Le voisinage \( \mO_a\) de \( a\) ne contient donc aucun élément de la suite \( (x_k)\), qui ne peut donc pas converger vers \( a\).
\end{proof}

\begin{corollary}\label{CorLimAbarA}
  Soit \( A \) un sous-ensemble d'un espace topologique \(X \). Toute suite d'éléments de \(A \) qui converge, admet pour limite un élément de \( \bar A \).
\end{corollary}
\begin{proof}
  Une fois la suite \( (x_n) \) fixée, il suffit de remarquer que tous les \( x_n \) sont dans \( \bar A \), et puis d'appliquer la proposition~\ref{PROPooBBNSooCjrtRb}. 
\end{proof}


\begin{lemma}   \label{LemPESaiVw}
    Soit \( A\subset X\) muni de la topologie induite de \( X\) et \( (x_n)\) une suite dans \( A\). Si \( (x_n) \) converge vers un élément \( x \) dans \(A \), alors elle converge aussi vers \(x \) dans \( X \).
\end{lemma}

\begin{proof}
    Soit \( \mO\) un ouvert autour de \( x\) dans \( X\). Alors \( A\cap\mO\) est un ouvert autour de \( x\) dans \( A\) et il existe \( N\in \eN\) tel que si \( n\geq N\), alors \( x_n\in A\cap\mO\subset\mO\).
\end{proof}

%---------------------------------------------------------------------------------------------------------------------------
\subsection{Pour des limites uniques : séparabilité}
%---------------------------------------------------------------------------------------------------------------------------

Notons que l'on a parlé d'\emph{une} limite de suite jusqu'à présent: en effet, s'il existe deux éléments distincts $x$ et $y$ tels que tout ouvert contenant $x$ contient $y$, alors la définition \ref{DefXSnbhZX} dit que toute suite convergeant vers $y$ converge aussi vers $x$\dots


\begin{example} \label{EXooSHKAooZQEVLB}
    Oui, il y a moyen de converger vers plusieurs points distincts si l'espace n'est pas super cool. Nous pouvons par exemple \cite{EJVQuas} considérer la droite réelle munie de sa topologie usuelle et y ajouter un point $0'$ (qui clone le réel $0$) dont les voisinages sont les voisinages de $0$ dans lesquels nous remplaçons $0$ par $0'$. Dans cet espace, la suite $(1/n)$ converge à la fois vers $0$ et $0'$.

    En fait, on «voit» le problème: on ne peut pas distinguer d'un point de vue topologique le $0$ et le $0'$.
\end{example}

Nous posons la définition suivante, qui nous permettra de donner une assez grande classe d'espaces topologiques dans lesquels nous avons unicité de la limite\footnote{Voir la proposition \ref{PropFObayrf}.}.
\begin{definition}[Espace topologique séparé]  \label{DefYFmfjjm}\label{DefWEOTrVl}
    Si deux points distincts admettent toujours deux voisinages disjoints\footnote{Définition~\ref{DefEnsemblesDisjoints}.}, nous disons que l'espace est \defe{séparé}{espace!séparé} ou \defe{Hausdorff}{Hausdorff}.
\end{definition}

Attention, cette notion est à ne pas confondre avec :
\begin{definition}[Espace topologique séparable]  \label{DefUADooqilFK}
    Un espace topologique est \defe{séparable}{séparable!espace topologique} s'il possède une partie dénombrable\footnote{Définition~\ref{DefEnsembleDenombrable}.} dense\footnote{Définition~\ref{DefEnsembleDense}.}.
\end{definition}

\begin{proposition}\label{PropUniciteLimitePourSuites}
  Dans un espace séparé, si une suite converge, alors sa limite est unique.
\end{proposition}
\begin{proof}
  Supposons que la suite \( (x_k)\) converge vers deux éléments distincts \( x \) et \( y \). L'espace étant séparé, il existe deux ouverts \( \mO_x \) et \( \mO_y \), disjoints, contenant respectivement \( x \) et \( y \). La suite convergeant à la fois vers \( x \) et \( y \), il existe \( k_x \) et \( k_y \), tels que, si \( k \geq \max\{k_x, k_y\} \), l'élément  \( x_k \) est (à la fois) dans  \( \mO_x \) et \( \mO_y \). Cela est en contradiction avec le fait que ces deux ensembles sont disjoints.
\end{proof}

\begin{normaltext}
  Donc, on pourra parler, avec des espaces séparés, de «la limite d'une suite». On notera \( x_n\to a\), ou \(\lim_{n\to \infty} x_n = a \), pour signifier que la suite \( (x_n) \) converge vers \( a \). 
\end{normaltext}

\begin{proposition}[\cite{MonCerveau}]      \label{PROPooNRRIooCPesgO}
    La convergence de suite pour la topologie de l'espace produit\footnote{Définition \ref{DefIINHooAAjTdY}.} est équivalente à la convergence des suites «composante par composante».
\end{proposition}

\begin{proof}
    En deux parties
    \begin{subproof}
        \item[Sens direct]
            Pour simplifier les notations, nous allons considérer le produit de deux espaces. Soit donc \( (x_k,y_k)\stackrel{X\times Y}{\longrightarrow}(x,y)\) et des ouverts \( \mO_1\) dans \( X\) autour de \( x\) et \( \mO_2\) autour de \( y\) dans \( Y\). 

            La partie \( \mO_1\times \mO_2\) est ouverte dans \( X\times Y\). Donc il existe \( K\) tel que \( k>K\) implique \( (x_k,y_k)\in \mO_1\times \mO_2\).

            Nous avons prouvé que pour tout ouvert \( \mO_1\) autour de \( x\) il existe \( K\) tel que \( k>K\) implique \( x_k\in \mO_1\). Donc \( x_k\stackrel{X}{\longrightarrow}x\). Idem pour \( y\).

        \item[Dans l'autre sens]
            klm
    \end{subproof}
\end{proof}

\begin{lemma}[\cite{MonCerveau}]        \label{LEMooSJKMooKSiEGq}
    Soit un espace topologique \( X\). Soient dans \( X\) une suite \( (x_n)\) et un élément \( x\) tels que toute sous-suite de \( (x_n)\) contient une sous-suite convergente vers \( x\). Alors \( x_n\to x\).
\end{lemma}

\begin{proof}
    Supposons que \( (x_n)\) ne converge pas vers \( x\). Il existe alors un ouvert \( \mO\) autour de \( x\) tel que pour tout \( N>0\), il existe \( n\geq N\) tel que \( x_n\) n'est pas dans \( \mO\).

    Cela nous permet de construire une sous-suite de \( (x_n)\) composée d'éléments hors de \( \mO\). Aucune sous-suite de cette sous-suite ne peut converger vers \( x\).
\end{proof}

%--------------------------------------------------------------------------------------------------------------------------- 
\subsection{Fonctions équivalentes}
%---------------------------------------------------------------------------------------------------------------------------

\begin{propositionDef}[\cite{ooZGTXooHrIgMQ}]       \label{DEFooWDSAooKXZsZY}
    Soit un espace topologique \( X\) et \( D\subset X\). Soient encore des fonctions \( f,g\colon D\to \eC\) et un point \( a\in\Adh(D)\)\footnote{Adhérence ou fermeture, c'est la même chose. Voir la définition \ref{DEFooSVWMooLpAVZR} et le lemme \ref{LEMooILNCooOFZaTe}.}.

    Nous définissons sur \( \Fun(D,\eC)\) la relation \( f\sim_ag\) lorsque qu'il existe un voisinage \( V\) de \( a\) dans \( X\) et une fonction \( \alpha\colon V\to \eR\) telles que
    \begin{enumerate}
        \item
            \( \lim_{x\to a} \alpha(x)=0\),
        \item
            pour tout \( x\in (V\cap D)\setminus\{ a \}\), 
            \begin{equation}        \label{EQooQXKYooSDPpNq}
                f(x)=\big( 1+\alpha(x) \big)g(x).
            \end{equation}
    \end{enumerate}
    Cette relation est une relation d'équivalence.

    Lorsque \( f\sim_a g\), nous disons que \( f\) et \( g\) sont \defe{équivalentes}{fonctions équivalentes} en \( a\).
\end{propositionDef}

Notons que la notion d'équivalence de fonctions, de même que la notion de limite, ne dépend pas des valeurs exactes atteintes par les fonctions au point.

\begin{lemma}
    Si \( f\) et \( g\) sont équivalentes en \( a\), et si \( g\) ne s'annule pas sur un voisinage de \( a\), alors pour tout \( \epsilon>0\), il existe \( r\) tel que
    \begin{equation}
        \frac{ f(x) }{ g(x) }\in B(1,\epsilon)
    \end{equation}
    pour tout \( x\in B(a,r)\).
\end{lemma}

\begin{proof}
    Nous considérons un voisinage \( V\) de \( a\) sur lequel en même temps :
    \begin{itemize}
        \item 
            la fonction \( \alpha\) de la définition d'équivalence est définie,
        \item
            \( | \alpha(x) |<\epsilon\) pour tout \( x\in V\),
        \item
            \( g(x)\neq 0\), pour tout \( x\in V\).
    \end{itemize}
    Ensuite nous considérons \( r>0\) tel que \( B(a,r)\subset V\). En divisant la condition \eqref{EQooQXKYooSDPpNq} par \( g(x)\) nous trouvons
    \begin{equation}
        \frac{ f(x) }{ g(x) }=1+\alpha(x).
    \end{equation}
    Donc
    \begin{equation}
        | \frac{ f(x) }{ g(x) }-1 |=| \alpha(x) |\leq \epsilon,
    \end{equation}
    ce qu'il fallait prouver.
\end{proof}


%+++++++++++++++++++++++++++++++++++++++++++++++++++++++++++++++++++++++++++++++++++++++++++++++++++++++++++++++++++++++++++
\section{Connexité}
%+++++++++++++++++++++++++++++++++++++++++++++++++++++++++++++++++++++++++++++++++++++++++++++++++++++++++++++++++++++++++++

L'idée de la connexité, c'est de s'assurer qu'un ensemble est «d'un seul tenant».

\begin{definition}  \label{DefIRKNooJJlmiD}
     Lorsque $E$ est un espace topologique, nous disons qu'un sous-ensemble $A$ est \defe{non connexe}{connexité!définition} quand on peut trouver des ouverts $O_1$ et $O_2$ disjoints tels que
    \begin{equation}    \label{EqDefnnCon}
        A=(A\cap O_1)\cup (A\cap O_2),
    \end{equation}
    et tels que $A\cap O_1\neq\emptyset$, et $A\cap O_2\neq\emptyset$. Si un sous-ensemble n'est pas non-connexe, alors on dit qu'il est \defe{connexe}{ensemble connexe}.
\end{definition}
Une autre façon d'exprimer la condition \eqref{EqDefnnCon} est de dire que $A$ n'est pas connexe quand il est contenu dans la réunion de deux ouverts disjoints qui intersectent tous les deux $A$.

\begin{proposition} \label{PropHSjJcIr}
    Soit \( X\) un espace topologique. Les conditions suivantes sont équivalentes.
    \begin{enumerate}
        \item
            L'espace \( X\) est connexe.
        \item
            Si \( X=A\sqcup B\) avec \( A\) et \( B\) fermés disjoints dans \( X\), alors \( A=\emptyset\) ou \( B=\emptyset\).
        \item       \label{ITEMooNIPZooIDPmEf}
            Si \( A\subset X\) avec \( A\) ouvert et fermé en même temps, alors \( A=\emptyset\) ou \( A=X\).
    \end{enumerate}
\end{proposition}

Nous verrons plus tard (proposition~\ref{PropConnexiteViaFonction}) une autre caractérisation de la connexité.

\begin{proposition}
    Si \( A\subset X\) est connexe et si \( A\subset B\subset \bar A\), alors \( B\) est connexe.
\end{proposition}
%TODO : une preuve.

\begin{proposition} \label{PropIWIDzzH}
    Stabilité de la connexité par union.
    \begin{enumerate}
        \item
    Une union quelconque de connexes ayant une intersection non vide est connexe.
\item
    Pour tout \( n \in \eN, n > 0 \), si \( A_1,\ldots, A_n\) sont des connexes de \( X\) avec \( A_i\cap A_{i+1}\neq \emptyset\), alors l'union \( \bigcup_{i=1}^nA_i\) est connexe.
    \end{enumerate}
\end{proposition}

\begin{proof}
    Point par point.
    \begin{enumerate}
        \item
    Soient \( \{ C_i \}_{i\in I}\) un ensemble de connexes et un point \( p\) dans l'intersection : \( p\in\bigcap_{i\in I}C_i\). Supposons que l'union ne soit pas connexe. Alors nous considérons \( A\) et \( B\), deux ouverts disjoints recouvrant tous les \( C_i\) et ayant chacun une intersection non vide avec l'union.

    Supposons pour fixer les idées que \( p\in A\) et prenons \( x\in B\cap\bigcup_{i\in I}C_i\). Il existe un \( j\in I\) tel que \( x\in C_j\). Avec tout cela nous avons
    \begin{enumerate}
        \item
            \( C_j\subset A\cup B\) parce que \(A \cup B\) recouvre tous les \( C_i \),
        \item
            \( C_j\cap A\neq \emptyset\) parce que \( p\) est dans l'intersection,
        \item
            \( C_j\cap B\neq\emptyset\) parce que \( x\) est dans cette intersection.
    \end{enumerate}
    Cela contredit le fait que \( C_j\) soit connexe.

\item

    Pour la seconde partie nous procédons de proche en proche\footnote{Parce qu'on a la flemme de faire une preuve par récurrence!}. D'abord \( A_1\cup A_2\) est connexe par la première partie, ensuite \( (A_1\cup A_2)\cup A_3\) est connexe parce que les connexes \( A_1\cup A_2\) et \( A_3\) ont un point d'intersection par hypothèse, et ainsi de suite.
    \end{enumerate}
\end{proof}


%+++++++++++++++++++++++++++++++++++++++++++++++++++++++++++++++++++++++++++++++++++++++++++++++++++++++++++++++++++++++++++
\section{Compacité}
%+++++++++++++++++++++++++++++++++++++++++++++++++++++++++++++++++++++++++++++++++++++++++++++++++++++++++++++++++++++++++++

La compacité est le thème~\ref{THEMEooQQBHooLcqoKB}.

%---------------------------------------------------------------------------------------------------------------------------
\subsection{Définition et notions connexes}
%---------------------------------------------------------------------------------------------------------------------------

Soit $E$, un sous-ensemble de $\eR$. Nous pouvons considérer les ouverts suivants :
\begin{equation}
    \mO_x=B(x,1)
\end{equation}
pour chaque $x\in E$. Évidemment,
\begin{equation}
    E\subseteq \bigcup_{x\in E}\mO_x.
\end{equation}
Cette union contient en général de nombreuses redondances. Si par exemple $E=[-10,10]$, l'élément $3\in E$ est contenu dans $\mO_{3.5}$, $\mO_{2.7}$ et bien d'autres. Pire : même si on enlève par exemple $\mO_2$ de la liste des ouverts, l'union de ce qui reste continue à être tout $E$. La question est : \emph{est-ce qu'on peut en enlever suffisamment pour qu'il n'en reste qu'un nombre fini ?}

\begin{definition} 
Soit $E$, un sous-ensemble de $\eR$. Une collection d'ouverts $\mO_i$ est un \defe{recouvrement}{recouvrement} de $E$ si $E\subseteq \bigcup_{i}\mO_i$.
\end{definition}

\begin{definition} \label{DefJJVsEqs}
    Une partie $A$ d'un espace topologique est \defe{compacte}{compact} s'il vérifie la propriété de Borel-Lebesgue : pour tout recouvrement de $A$ par des ouverts (c'est-à-dire une collection d'ouverts dont la réunion contient $A$) on peut extraire un recouvrement fini.
\end{definition}

\begin{remark}
    Certaines sources (dont \wikipedia{fr}{Compacité_(mathématiques)}{wikipédia}) disent que pour être compact il faut aussi être séparé\footnote{Définition~\ref{DefWEOTrVl}.}. Pour ces sources, un espace qui ne vérifie que la propriété de Borel-Lebesgue est alors dit \defe{quasi-compact}{quasi-compact}\index{compact!quasi}.
\end{remark}

\begin{normaltext}
    La définition \ref{DefJJVsEqs} en cache deux. En effet, si la partie \( A\) est l'espace topologique lui-même, cela définit un espace topologique compact. Un espace topologique est compact \emph{en soi} lorsque de tout recouvrement par des ouverts, nous pouvons extraire un sous-recouvrement fini. Dans ce cas, si \( X\) est l'espace et si \( \{ A_i \}_{i\in I}\) est le recouvrement, nous avons \( X=\bigcup_{i\in I}A_i\) et non une simple inclusion \( X\subset \bigcup_{i\in I}A_i\).
\end{normaltext}

\begin{lemma}       \label{LEMooVYTRooKTIYdn}
    Si \( K\) est une partie compacte de l'espace topologie \( X\), alors \( K\) est un espace topologique compact pour la topologie induite\footnote{Définition \ref{DefVLrgWDB}.} de \( X\).
\end{lemma}

\begin{proof}
    Nous notons \( \tau\) la topologie de \( X\) et \( \tau_K\) la topologie induite de \( X\) vers \( K\), c'est-à-dire
    \begin{equation}
        \tau_K=\{ \mO\cap K\tq \mO\in\tau \}.
    \end{equation}
    Soient des ouverts \( A_i\in \tau_K\) (\( i\in I\) où \( I\) est un ensemble quelconque) tels que \( \bigcup_iA_i=K\). Pour chaque \( i\in I\), il existe un \( \mO_i\in \tau\) tel que \( A_i=K\cap\mO_i\). Nous avons
    \begin{equation}
        K=\bigcup_{i\in I}(K\cap\mO_i)\subset\bigcup_{i\in I}\mO_i.
    \end{equation}
    Donc les \( \mO_i\) forment un recouvrement de \( K\) par des ouverts de \( X\). Vu que \( K\) est une partie compacte de \( X\), il existe un sous-ensemble fini \( J\) de \( I\) tel que
    \begin{equation}
        K\subset\bigcup_{j\in J}\mO_i.
    \end{equation}
    Nous avons donc aussi
    \begin{equation}
        K\subset\bigcup_{j\in J}K\cap\mO_i=\bigcup_{j\in J}A_j.
    \end{equation}
    Nous avons prouvé que \( \{ A_j \}_{j\in J}\) est un recouvrement fini de \( K\) par des ouverts de \( K\). Donc \( K\) est un espace topologique compact.
\end{proof}

\begin{definition}
    Une partie d'un espace topologique est \defe{relativement compact}{compact!relativement}\index{relativement!compact} si son adhérence est compacte.
\end{definition}

\begin{definition}  \label{DefEIBYooAWoESf}
    Un espace topologique est \defe{localement compact}{compact!localement} si tout élément possède un voisinage compact.
\end{definition}

\begin{definition}[Séquentiellement compact]
    Nous disons qu'un espace topologique est \defe{séquentiellement compact}{compact!séquentiellement} si toute suite admet une sous-suite convergente.
\end{definition}

\begin{definition}      \label{DefFCGBooLpnSAK}
    Un espace topologique est \defe{dénombrable à l'infini}{dénombrable!à l'infini} s'il est réunion dénombrable de compacts.
\end{definition}

\begin{definition}
    Une famille \( \mA\) de parties de \( X\) a la \defe{propriété d'intersection finie non vide}{propriété d'intersection non vide} si tout sous-ensemble fini de \( \mA\) a une intersection non vide.
\end{definition}

\begin{proposition}\label{PropXKUMiCj}
    Soient \( X\) un espace topologique et \( K\subset X\). Les propriétés suivantes sont équivalentes :
    \begin{enumerate}
        \item\label{ItemXYmGHFai}
            \( K\) est compact.
        \item\label{ItemXYmGHFaii}
            Si \( \{ F_i \}\) est une famille de fermés telle que \( \bigcap_{i\in I}F_i \cap K =\emptyset\), alors il existe une partie finie non vide \( A\) de \( I\) tel que \( \bigcap_{i\in A}F_i \cap K =\emptyset\).
        \item\label{ItemXYmGHFaiii}
            Si \( \{ F_i \}_{i\in I}\) est une famille de fermés telle que pour tout choix de \( A\) fini dans \( I\), \( \bigcap_{i\in A}F_i \bigcap K \neq\emptyset\), alors l'intersection complète est non vide : \( \bigcap_{i\in I}F_i \bigcap K\neq\emptyset\).
        \item\label{ItemXYmGHFaiv}
            Toute famille de fermés de \( X \), à laquelle \( K \) est joint, et qui a la propriété d'intersection finie non vide, a une intersection non vide.
    \end{enumerate}
\end{proposition}

\begin{proof}
    Les propriétés~\ref{ItemXYmGHFaiii} et~\ref{ItemXYmGHFaii} sont équivalentes par contraposition. De plus le point~\ref{ItemXYmGHFaiv} est une simple\footnote{Enfin, simple\dots{} il faut remarquer que dans la formulation de~\ref{ItemXYmGHFaiv}, les intersections peuvent ne pas faire intervenir \( K \), mais, au final, on s'en moque.} reformulation en français de la propriété~\ref{ItemXYmGHFaiii}.

    Prouvons~\ref{ItemXYmGHFai} \( \Rightarrow\)~\ref{ItemXYmGHFaii}. Soit \( \{ F_i \}_{i\in I}\) une famille de fermés tels que \( K\bigcap_{i\in I}F_i=\emptyset\). Les complémentaires \( \mO_i\) de \( F_i\) dans \( X\) recouvrent \( K\) et donc on peut en extraire un sous-recouvrement fini :
    \begin{equation}
        K\subset\bigcup_{i\in A}\mO_i
    \end{equation}
    pour un certain sous-ensemble fini \( A\) de \( I\). Pour ce même choix \( A\), nous avons alors aussi
    \begin{equation}
        \bigcap_{i\in A}F_i \cap K =\emptyset.
    \end{equation}

    L'implication~\ref{ItemXYmGHFaii} \( \Rightarrow\)~\ref{ItemXYmGHFai} est la même histoire de passage aux complémentaires.
\end{proof}

Le théorème \ref{ThoCQAcZxX} est en général celui qu'on nomme «théorème des fermés emboîtés», mais le corolaire suivant en mériterait également le nom.
\begin{corollary}[\cite{MonCerveau}]       \label{CORooQABLooMPSUBf}
    Soient un espace topologique compact \( X\) et une suite \( (F_i)_{i\in \eN}\) de fermés emboîtés\footnote{C'est-à-dire que \( F_{i+1}\subset F_i\).} dans \( X\) telle que
    \begin{equation}
        \bigcap_{i\in \eN}F_i=\emptyset.
    \end{equation}
    Alors il existe \( j_0\in \eN\) tel que \( F_i=0\) pour tout \( i\geq j_0\).
\end{corollary}

\begin{proof}
    La proposition \ref{PropXKUMiCj} nous dit qu'il existe une partie finie non vide \( J\) de \( \eN\) telle que \( \bigcup_{j\in J}F_j=\emptyset\). Si \( j_0=\min(J)\), alors \( F_j\subset F_{j_0}\) pour tout \( j\in J\) et nous avons
    \begin{equation}
        \emptyset=\bigcap_{j\in J}F_j=F_{j_0}.
    \end{equation}
    Dès que \( F_{j_0}=\emptyset\), tous les suivants sont également vides.
\end{proof}

%--------------------------------------------------------------------------------------------------------------------------- 
\subsection{Base de topologie}
%---------------------------------------------------------------------------------------------------------------------------

\begin{definition}[Base de topologie\cite{ooKBUGooWCSiXh}]   \label{DefQELfbBEyiB}
    Une famille \( \mB\) d'ouverts de \( X\) est une \defe{base de la topologie}{base!de topologie} de \( X\) si pour tout \( x\in X\) et pour tout voisinage \( V\) de \( x\), il existe \( A\in \mB\) tel que \( x\in A\subset V\).
\end{definition}

\begin{proposition} \label{PropMMKBjgY}
    Si \( \mB\) est une base de la topologie de \( X\) alors tout ouvert de \( X\) est une union d'éléments de \( \mB\).
\end{proposition}

\begin{proof}
    Soit \( \mO\) un ouvert de \( X\); pour chaque \( x\in\mO\) nous considérons un ouvert \( U(x)\) tel que \( x\in U(x)\subset \mO\) (possible par le théorème~\ref{ThoPartieOUvpartouv}). Nous prenons alors \( B(x)\in\mB\) tel que
    \begin{equation}
        x\in B(x)\subset U(x)\subset \mO.
    \end{equation}
    Alors nous avons \( \mO=\bigcup_{x\in \mO}B(x)\).
\end{proof}
Notons toutefois que nous sommes loin d'avoir une union dénombrable en général.

%---------------------------------------------------------------------------------------------------------------------------
\subsection{Quelques propriétés}
%---------------------------------------------------------------------------------------------------------------------------

\begin{lemma}   \label{LemOWVooZKndbI}
    Une partie \( K\) d'un espace topologique est compacte si et seulement si de tout recouvrement par des ouverts d'une base de topologie nous pouvons extraire un sous-recouvrement fini.
\end{lemma}
Remarquons que la partie qui est réellement à prouver est que, si \og ça marche \fg{} pour des ouverts d'une base de topologie, alors \og ça marche\fg{} pour tous types d'ouverts.
\begin{proof}
    Soit \( K\) une partie d'un espace topologique et \( \{ \mO_i \}_{i\in I}\) un recouvrement de \( K\) par des ouverts. Chacun des \( \mO_i\) est une union d'éléments de la base de topologie par la proposition~\ref{PropMMKBjgY}: disons \( \mO_i = \bigcup_{j \in J_i} A_{(i,j)} \). Soit \( J = \{ j = (i, j_i) | i \in I, j_i \in J_i \} \); alors nous obtenons  \( \bigcup_{j\in J}A_j=\bigcup_{i\in I}\mO_i\).

    Par hypothèse nous pouvons extraire un ensemble fini \( J_0\subset J\) tel que \( K\subset\bigcup_{j\in J_0}A_j\). Par construction chacun des \( A_j\) est inclus dans (au moins) un des \( \mO_i\). Le choix d'un élément de \( I\) pour chacun des éléments de \( J_0\) donne une partie finie \( I_0\) de \( I\) telle que \( K\subset\bigcup_{j\in J_0}A_j\subset\bigcup_{i\in I_0}\mO_i\).
\end{proof}


\begin{example}[Un compact non fermé]
    En général, un compact n'est pas toujours fermé. Si nous prenons par exemple un ensemble \( X\) de plus de deux points muni de la topologie grossière \( \{ \emptyset,X \}\). Toutes les parties de cet espace sont compactes, mais les seuls fermés sont \( \{ \emptyset,X \}\). Toutes les autres parties sont alors compactes et non fermées.
\end{example}

\begin{proposition}[\cite{OCBrmKo}]\label{PropUCUknHx}
    Tout compact d'un espace topologique séparé est fermé.
\end{proposition}
\index{compact!implique fermé}

\begin{proof}
    Soient \( X\) un espace séparé et \( K\) compact dans \( X\). Nous considérons \( y \in\complement K\) et, par hypothèse de séparation, pour chaque \( x\in K\) nous considérons un voisinage ouvert \( V_x\) de \( x\) et un voisinage ouvert~\footnote{Oui, la notation du voisinage peut surprendre, mais elle est quand même pratique pour ce qu'on veut en faire.} \( W_x\) de \( y\) tels que \( V_x\cap W_x=\emptyset\). Bien entendu les \( V_x\) forment un recouvrement ouvert de \( K\) dont nous pouvons extraire un sous-recouvrement fini : soit \( S\) fini dans \( K\) tel que
    \begin{equation}
        K\subset\bigcup_{x\in S}V_x.
    \end{equation}
    L'ensemble \( W=\bigcap_{x\in S}W_x\) est une intersection finie d'ouverts autour de \( y\) et est donc un ouvert autour de \( y\). 
    
    Montrons que \( W\cap K=\emptyset\). Soit \( a\in K\); par définition de \( S\), il existe \( s\in S\) tel que \( a\in V_s\). Par conséquent, \( a\) n'est pas dans \( W_s\) et donc pas non plus dans \( W\).
    

    L'ouvert \( W\) prouve que \( y\) est dans l'intérieur du complémentaire de \( K\), et comme \( y \) est arbitraire, nous concluons que le complémentaire de \( K\) est ouvert (théorème~\ref{ThoPartieOUvpartouv}), en d'autres termes, que \( K\) est fermé.
\end{proof}

\begin{lemma}[\cite{SNSposN}]   \label{LemnAeACf}
    Une partie fermée d'une compact est elle-même compacte.
\end{lemma}
\index{fermé!dans un compact}

\begin{proof}
%    Nous allons utiliser la caractérisation de la compacité en termes de suites donnée par le théorème de Bolzano-Weierstrass~\ref{ThoBWFTXAZNH}. Soit \( K\) un compact et \( F\) un fermé dans \( K\). Nous considérons une suite \( (x_n)\) dans \( F\); par la compacité de \( K\) nous pouvons considérer une sous-suite \( (y_n)\) qui converge dans \( K\) (proposition~\ref{ThoBWFTXAZNH}). Étant donné que \( (y_n)\) est une suite convergente contenue dans \( F\) et étant donné que \( F\) est fermé, la limite est dans \( F\), ce qui prouve que \( (x_n)\) possède une sous-suite convergente dans $F$ et par conséquent que \( F\) est compact.

    Soient \( F\) fermé dans un compact \( K\) et \( \{ \mO_i \}_{i\in I}\) un recouvrement de \( F\) par des ouverts. Vu que \( F\) est fermé, \( F^c\) est ouvert et \( \{ \mO_i \}_{i\in I}\cup\{ K\setminus F \}\) est un recouvrement de \( K\) par des ouverts. Si nous en extrayons un sous-recouvrement fini, c'est un recouvrement de \( F\), et en supprimant éventuellement l'ouvert \( K\setminus F\), ça reste un sous-recouvrement fini de \( F\) tout en étant extrait de \( \{ \mO_i \}_{i\in I}\).
\end{proof}

\begin{proposition}     \label{PropGBZUooRKaOxy}
    Si \( V\) est une partie de l'espace topologique \( X\) muni de la topologie induite\footnote{Définition \ref{DefVLrgWDB}.} \( \tau_V\) de celle de \( X\), et si \( K\) est un compact de \( (V,\tau_V)\) alors \( K\) est un compact de \( (X,\tau_X)\).
\end{proposition}

\begin{proof}
    Soient \(   (\mO_\alpha)_{\alpha\in A}  \) des ouverts de \( X\) recouvrant \( K\). Alors les ensembles \( V\cap \mO_{\alpha}\) recouvrent également \( K\), mais sont des ouverts de \( V\). Donc il en existe un sous-recouvrement fini. Soient donc \( (V\cap\mO_i)_{i\in I}\) recouvrant \( K\) avec \( I\) un sous-ensemble fini de \( A\). Les ensembles \( (\mO_i)_{i\in I}\) recouvrent encore \( K\) et sont des ouverts de \( X\).
\end{proof}

%--------------------------------------------------------------------------------------------------------------------------- 
\subsection{Compactifié d'Alexandrov}
%---------------------------------------------------------------------------------------------------------------------------

\begin{propositionDef}[\cite{ooEDBNooKshWkw}]       \label{PROPooHNOZooPSzKIN}
    Soit un espace topologique localement compact\footnote{Définition \ref{DefEIBYooAWoESf}.} \( X\). Nous considérons un élément \( \omega\notin X\) et l'ensemble \( \hat X =\ X\cup\{ \omega \}\). Nous nommons «ouverts de \( \hat X\)» les parties suivantes :
    \begin{itemize}
        \item les ouverts de \( X\),
        \item les parties de la forme \( K^c\cup\{ \omega \}\) où \( K\) est compact de \( X\). Ici, le complémentaire de \( K\) est pris dans \( X\), pas dans \( \hat X\).
    \end{itemize}
   Alors \( \hat X\) est un espace topologique compact (cela justifie le nom «ouvert» donné aux parties sus-définies).
\end{propositionDef}

Oh bien entendu les plus férus de questions embarrassantes demanderont, si \( X\) est l'espace considéré, où prendre ce \( \omega\) ? Quel «objet» exist en-dehors de \( X\) ? Qui m'assure que \( X\) n'est pas tellement grand que tout est dedans ? Je vous laisse dormir sur ces questions; sachez que \( X\) lui-même n'est certainement pas un élément de \( X\).

En ce qui concerne \( \eR\) auquel nous pouvons attacher deux infinis (\( +\infty\) et \( -\infty\)), ce sera la définition \ref{DEFooRUyiBSUooALDDOa}.

Pour \( \eC\), nous donnerons une caractérisation de la limite en \( \infty\) dans le lemme \ref{LEMooERABooQjLBzW}.

%+++++++++++++++++++++++++++++++++++++++++++++++++++++++++++++++++++++++++++++++++++++++++++++++++++++++++++++++++++++++++++
\section{Limites et continuité de fonctions}
%+++++++++++++++++++++++++++++++++++++++++++++++++++++++++++++++++++++++++++++++++++++++++++++++++++++++++++++++++++++++++++

%---------------------------------------------------------------------------------------------------------------------------
\subsection{Limites}
%---------------------------------------------------------------------------------------------------------------------------

\begin{definition}[Limite d'une fonction, thème~\ref{THEMEooGVCCooHBrNNd}]\label{DefYNVoWBx}
    Soient \( X\) et \( Y\) des espaces topologiques, et un point d'accumulation \( a\) de \( X\). Soit encore une fonction \( f\colon X\to Y\). L'élément \( y\in Y\) est une \defe{limite}{limite!d'une fonction} de \( f\) en \( a\) si pour tout voisinage \( W\) de \( y\) (pour la topologie de \( Y\)), il existe un voisinage \( V\) de \( a\) dans \( X\) tel que
    \begin{equation}        \label{EQooXLJJooZDcOtU}
        f\big( V\setminus\{a\} \big)\subset W.
    \end{equation}

    Si un tel élément est unique\footnote{Rappelons que ce n'est pas toujours le cas, mais que ça l'est si l'espace topologique est séparé -- définition~\ref{DefYFmfjjm}.}, alors nous disons que cet élément est la \defe{limite}{limite!d'une fonction} de \( f\) et nous notons
    \begin{equation}
        \lim_{x\to a} f(x)=y.
    \end{equation}
\end{definition}

\begin{normaltext}
    Souvent, nous considérons une fonction \( f\colon D\to \eR\) avec \( D\subset \eR\). Dans ce cas, le \( X\) de la définition de la limite est \( D\) muni de la topologie induite de \( \eR\) vers \( D\). Dans ce cas, la condition \eqref{EQooXLJJooZDcOtU} s'écrit sous la forme
    \begin{equation}
        f\big( V\cap D\setminus\{a\} \big)\subset W.
    \end{equation}
    où \( V\) est un voisinage de \( a\) dans \( \eR\) et non un voisinage de \( D\) dans \( D\).
\end{normaltext}


\begin{remark}
    Nous ne saurions trop insister sur le fait que la valeur de \( f\) en \( a\) n'intervient pas dans la définition de la limite de \( f\) en \( a\). Il n'est même pas pas nécessaire que \( f\) soit définie en \( a\) pour que l'on puisse parler de limite de \( f\) en \( a\). Par exemple nous avons
    \begin{equation}
        \lim_{x\to 1} \frac{ x^2-1 }{ x-1 }=2,
    \end{equation}
    alors que la fonction n'est pas définie en \( x=1\).

    Plus généralement, un peu par principe, toutes les fois que la notion de limite apporte une information, le point où l'on prend la limite est spécial. Sinon on ne calculerait pas la limite, mais on regarderait directement la valeur de la fonction. Cela est typiquement le cas lorsque nous verrons les dérivées. En effet, regardons (en faisans du semblant d'anticiper) la définition  \eqref{DEFooOYFZooFWmcAB}. Dans la formule
    \begin{equation}
        f'(a)=\lim_{x\to a} \frac{ f(x)-f(a) }{ x-a },
    \end{equation}
    la fonction sur laquelle nous prenons la limite n'est \emph{jamais} définie en \( x=a\).

    Cela est intimement lié à ce qu'on raconte dans~\ref{SUBSECooVHKCooYRFgrb}.
\end{remark}

\begin{proposition}[Unicité de la limite pour un espace séparé]\label{PropFObayrf}
    Soient \( X\) un espace topologique, \( A\) une partie de \( X\) et \( Y\) un espace topologique séparé\footnote{Définition~\ref{DefYFmfjjm}.}. Nous considérons une fonction \( f\colon A\to Y\). Si \( a\in\bar A\), alors \( f\) admet au plus une limite en \( a\).
\end{proposition}
\index{limite!unicité}

\begin{proof}
    Soient \( y\) et \( y'\) des limites de \( f\) en \( a\), ainsi que des voisinages \( V\) et \( V'\) de \( y\) et \( y'\). Nous prenons également les voisinages \( W\) et \( W'\) correspondants :
    \begin{subequations}
        \begin{numcases}{}
            f(W\cap A)\subset V\\
            f(W'\cap A)\subset V'.
        \end{numcases}
    \end{subequations}
    Quitte à prendre des sous-ensembles nous pouvons supposer que \( W\) et \( W'\) sont ouverts. Il s'ensuit alors que:
    \begin{itemize}
      \item l'ensemble \( W\cap W'\) est un ouvert contenant \( a\) et intersecte donc \( A\);
      \item l'ensemble \( (W\cap W')\cap A\) est donc non vide;
      \item et donc, \( f(W\cap W'\cap A) \) est aussi non vide.
    \end{itemize}
    Mais
    \begin{equation}
            f(W\cap W'\cap A)\subset f(W\cap A)\subset V,
    \end{equation}
    et
    \begin{equation}
            f(W\cap W'\cap A)\subset f(W'\cap A)\subset V',
    \end{equation}
    d'où \( V \) et \( V'\) ont une intersection. Puisque ces ensembles sont arbitraires, nous avons prouvé que tout voisinage de \( y\) et tout voisinage de \( y'\) ont une intersection non vide; étant donné que \( Y\) est séparé, nous devons avoir \( y=y'\).
\end{proof}

%---------------------------------------------------------------------------------------------------------------------------
\subsection{Continuité}
%---------------------------------------------------------------------------------------------------------------------------

\subsubsection{Définitions et propriétés}
%////////////////////////////

La définition suivante est \emph{la} définition de la continuité dans tous les cas.
\begin{definition}[Fonction continue\cite{ooBFBXooLJWsFq}]\label{DefOLNtrxB}
    Deux définitions :
    \begin{enumerate}
        \item   \label{ITEMooXARPooNzsWLr}
            Soient une fonction \( f\colon X\to Y\) entre les espaces topologiques \( X\) et \( Y\) et un point \( a\in X\). Nous disons que \( f\) est \defe{continue}{fonction continue en un point} en \( a\) si pour tout ouvert \( W\) contenant \( f(a)\), il existe un voisinage \( V\) de \( a\) dans \( X\) tel que \( f(V)\subset W\).
\item
    Une fonction \( f\colon X\to Y\) est \defe{continue}{continue!fonction entre espaces topologiques} sur \( X\) si pour tout ouvert \( \mO\) de \( Y\), l'ensemble
    \begin{equation}      \label{defFminus1ofaset}
      f^{-1}(\mO) = \{ x \in X \tq f(x) \in \mO\}
    \end{equation}
est ouvert dans \( X\).
    \end{enumerate}
\end{definition}

\begin{normaltext}
    Lorsque nous écrivons \( f\colon X\to Y\), nous entendons que \( f\) est définie sur tout \( X\), mais pas qu'elle soit surjective sur \( Y\). En particulier, pour que \( f\) soit continue en \( a\), il faut que \( a\) soit dans le domaine de \( f\). 

    Dans le cas de fonctions \( \eR\to \eR\), l'espace \( X\) sera la partie de \( \eR\) sur laquelle \( f\) sera définie, et la topologie sera la topologie induite de \( \eR\).
\end{normaltext} 

\begin{example}[\cite{MonCerveau}]
    Un truc bien avec la définition \ref{DefOLNtrxB}\ref{ITEMooXARPooNzsWLr} est que la continuité de \( f\) en un point est définie pour tout point du domaine; pas seulement les points d'accumulation. Soit par exemple une fonction simple
    \begin{equation}
        \begin{aligned}
            f\colon \{a\}&\to \eR \\
            a&\mapsto 4. 
        \end{aligned}
    \end{equation}
    Si \( W\) est un ouvert de \( \eR\) contenant \( 4\), nous avons l'ouvert \( V=\{a\}\) tel que \( f(V)\subset W\). Donc \( f\) est continue au point \( 4\).

    Mais \( f\) est également continue sur \( \{4\}\) en tant qu'espace topologique. En effet, si \( W\) est un ouvert de \( \eR\), l'ensemble \( f^{-1}(W)\) est soit \( \emptyset\) soit \( \{a\}\). Dans les deux cas c'est un ouvert.
\end{example}

La proposition~\ref{PropQZRNpMn} donnera des détails sur ce qu'il se passe lorsque l'espace est métrique.

\begin{theorem} \label{ThoESCaraB}
    Une fonction \( f\colon X\to Y\) est une fonction continue si et seulement si elle est continue en chacun des points de \( X\).
\end{theorem}

\begin{proof}
    En deux parties.
    \begin{subproof}
    \item[Sens direct]
        Nous supposons que \( f\) est une fonction continue. Soient \( a\in X\) et \( W\) un voisinage de \( f(a)\). Nous considérons \( \mO\), un voisinage ouvert de \( f(a)\) contenu dans \( W\); l'ensemble \( f^{-1}(\mO)\) est alors un ouvert contenant \( a\), et l'image de \( f^{-1}(\mO)\) par \( f\) est bien entendu contenue dans \( W\).

    \item[Sens inverse]

        Soit \( \mO\) un ouvert de \( Y\). Pour prouver que \( f^{-1}(\mO)\) est un ouvert de \( X\), nous allons considérer un élément \( a\in f^{-1}(\mO)\) et montrer qu'il existe un voisinage ouvert de \( a\) contenu dans \( f^{-1}(\mO)\); le théorème~\ref{ThoPartieOUvpartouv} nous assurera alors que \( f^{-1}(\mO)\) est ouvert.

        L'ensemble \( \mO\) est un voisinage ouvert de \( f(a)\) parce que \( a\) a été choisi dans \( f^{-1}(\mO)\). Donc la continuité de \( f\) en \( a\) nous assure qu'il existe un voisinage \( W\) de \( a\) tel que \( f(W)\subset\mO\). En prenant un ouvert contenant \( a\) à l'intérieur de \( W\) nous avons un voisinage ouvert de \( a\) contenu dans \( f^{-1}(\mO)\).
    \end{subproof}
\end{proof}

\begin{remark}
    À cause de l'éventuelle non unicité de la limite, deux fonctions continues et égales sur un sous-ensemble dense ne sont pas spécialement égales. Ce sera vrai sur les espaces métriques et plus généralement pour les espaces séparés. Voir l'exemple~\ref{EXooSHKAooZQEVLB} et la proposition~\ref{PropFObayrf}.
\end{remark}

\begin{lemma}[\cite{MonCerveau}]
Soient une fonction \( f\colon X\to Y\), et un point d'accumulation \( a\in X\)\footnote{Un point d'accumulation de \( X\) n'est pas spécialement dans \( X\), si \( X\) est un sous-espace d'un autre. Par exemple \( 0\) est un point d'accumulation de \( \mathopen] 0 , 1 \mathclose[\) dans \( \eR\). Ici nous supposons que \( a\in X\), sinon il n'y a de toutes façons pas de continuité en \( a\).}. La fonction \( f\) est continue en \( a\) si et seulement si \( f(a)\) est une limite de \( f\) en \( a\).
\end{lemma}

\begin{proof}
    En deux parties.
    \begin{subproof}
        \item[Sens direct]
            Nous supposons que \( f\) est continue en \( a\in X\). Soit un voisinage \( W\) de \( f(a)\) dans \( Y\). Par continuité de \( f\) en \( a\), il existe un voisinage \( V\) de \( A\) tel que \( f(V)\subset W\). A forciori, \( f\big( V\setminus{{a}} \big)\subset W\) comme le demande la définition de la limite.
        \item[Sens inverse]
            Nous supposons que \( f(a)\) est une limite de \( f(x)\) lorsque \( x\) tend vers \( a\). Si \( W\) est un ouvert de \( Y\) contenant \( f(a)\), il existe un voisinage \( V\) de \( a\) dans \( X\) tel que \( f\big( V\setminus{{a}} \big)\subset W\). Mais vu que \( f(a)\in W\), nous avons \( f(V)\subset W\).
    \end{subproof}
\end{proof}

\subsubsection{Continuité séquentielle}
%///////////////////////

\begin{definition}  \label{DefENioICV}
    Si \( X\) et \( Y \) sont deux espaces topologiques, une fonction \( f\colon X\to \eR\) est \defe{séquentiellement continue}{continuité!séquentielle} en un point \( a\) si pour toute suite convergente \( x_n\to a\) dans \( X\) nous avons \( f(x_n)\to f(x)\) dans \( Y\).
\end{definition}

\begin{proposition}[Caractérisation séquentielle de la limite\cite{MonCerveau}] \label{fContEstSeqCont}
  Soient deux espaces topologiques \( X\) et \( Y\) ainsi qu'une fonction \( f\colon X\to Y\). Soit \( a\in X\) et \( \ell\in Y\). Si
  \begin{equation}
    \lim_{x\to a} f(x)=\ell,
  \end{equation}
  alors, pour toute suite \( (x_k) \) telle que \( x_k \to a \), on a
  \begin{equation}
    \lim f(x_k)=\ell.
  \end{equation}
\end{proposition}

\begin{proof}
  Nous considérons une suite \( (x_k)\) qui converge vers \( a\) dans \( X\). Soient \( V\) un voisinage de \( \ell \) et \( W\) un voisinage de \( a\) tels que \( f(W)\subset V\) (définition~\ref{DefYNVoWBx} de la continuité en un point). Par la convergence \( a_k\to a\),  il existe \( N\) tel que pour tout \( k>N\), \( a_k\in W\), et donc tel que \( f(a_k)\in V\), ce qui donne la continuité séquentielle de \( f\).
\end{proof}

\begin{corollary}[Caractérisation séquentielle de la continuité en un point\cite{MonCerveau}]		\label{PropFnContParSuite}
    Un application entre deux espaces topologiques est continue en un point y est séquentiellement continue.
\end{corollary}

\begin{proof}
    Soit une application \( f\colon X\to Y\) entre les espaces topologies \( X\) et \( Y\). Nous supposons que \( f\) est continue en \( a\in X\). Soit une suite convergente \( x_k\stackrel{X}{\longrightarrow}a\). Nous devons prouver que \( f(x_k)\to f(a)\).

    Soit un voisinage \( V\) de \( f(a)\) dans \( Y\). Le fait que \( f\) soit continue en \( a\) signifie\footnote{C'est la définition \ref{DefOLNtrxB} de la continuité en un point.} que \( f(a)\) est une limite de \( f\) en \( a\), c'est-à-dire\footnote{Définition \ref{DefYNVoWBx} d'être une limite.} qu'il existe un voisinage \( W\) de \( a\) tel que \( f(W\setminus\{ a \})\subset V\).

    Vu que \( x_k\to a\), il existe \( N\) tel que \( x_k\in W\) pour tout \( k\geq N\). Pour ces valeurs de \( k\), nous avons \( f(x_k)\in V\).

    Nous avons prouvé que pour tout voisinage \( V\) de \( f(a)\) dans \( Y\), il existe \( N\) tel que \( f(x_k)\in V\) dès que \( k\geq N\). Cela signifie exactement que \( f(x_k)\to f(a)\).
\end{proof}


\subsubsection{Application réciproque}
%//////////////////////

\begin{definition}[injection, surjection, bijection]        \label{DEFooBFCQooPyKvRK}
    Soient des ensembles \( A\) et \( B\) ainsi qu'une application \( f\colon A\to B\).
    \begin{enumerate}
        \item
            La fonction \( f\) est \defe{injective}{injection} si \( f(x_1)=f(x_2)\), implique \( x_1=x_2\).
        \item
            La fonction \( f\) est \defe{surjective}{surjection} si tous les éléments de \( B\) sont atteints, c'est-à-dire si pour tout \( y\in B\) il existe \( x\in A\) tel que \( f(x)=y\).
        \item
            La fonction \( f\) est une \defe{bijection}{bijection} entre \( A\) et \( B\) si elle est injective et surjective, c'est-à-dire si pour tout \( y\in B\) il existe un unique \( x\in A\) tel que \( f(x)=y\).
    \end{enumerate}
\end{definition}
La surjection et l'injection sont des propriétés bien différentes qu'il convient de prouver séparément. De plus une même «formule» peut définir une application injective, surjective, bijective ou non selon le domaine sur laquelle nous la considérons.

\begin{definition}      \label{DEFooTRGYooRxORpY}
    Soit \( f\colon A\to B\) une bijection. L'\defe{application réciproque}{application réciproque} de \( f\) est la fonction
    \begin{equation}
        \begin{aligned}
            f^{-1}\colon B&\to A \\
            y&\mapsto \text{le } x\in A\text{ tel que } f(x)=y.
        \end{aligned}
    \end{equation}
\end{definition}

Plus généralement si \( f\colon X\to Y\) est une application quelconque et si \( S\subset Y\) nous notons
\begin{equation}
    f^{-1}(S)=\{ x\in X\tq f(x)\in S \},
\end{equation}
et dans le cas où \( S\) est réduit à un unique élément \( y\), nous notons \( f^{-1}(y)\) au lieu de \( f^{-1}\big( \{ y \} \big)\). Si de plus \( f^{-1}(S)\) est un singleton \( x\), nous noterons \( f^{-1}(S)=x\) et non \( f^{-1}(S)=\{ x \}\).

Les plus acharnés parmi les lecteurs se rendront compte de la différence ontologique fondamentale entre \( x\) et \( \{ x \}\).

\begin{proposition}	\label{PropoInvCompCont}
Soit $f\colon A\subset\eR^n\to B\subset\eR^m$ une bijection continue. Si $A$ est compact, alors $f^{-1}\colon B\to A$ est continue.
\end{proposition}
\index{réciproque!continuité}

\begin{proposition}		\label{PropIntContMOnIvCont}
Soient $I$ un intervalle dans $\eR$ et $f\colon I\to \eR$ une fonction continue strictement monotone. Alors la fonction réciproque $f^{-1}\colon f(I)\to \eR$ est continue sur l'intervalle $f(I)$.
\end{proposition}
\index{réciproque!continuité}

\subsubsection{Homéomorphisme}
%/////////////////

\begin{definition}
    Un \defe{homéomorphisme}{homéomorphisme} est une application bijective continue entre deux espaces topologiques dont la réciproque est continue. Deux espaces topologiques $X$ et $Y$ pour lesquels il existe un homéomorphisme entre $X$ et $Y$, sont dits \defe{isomorphes}{isomorphisme!d'espaces topologiques}.
\end{definition}

%---------------------------------------------------------------------------------------------------------------------------
\subsection{Continuité et topologie induite}
%---------------------------------------------------------------------------------------------------------------------------
\begin{proposition}[\cite{MonCerveau}]     \label{PROPooNPLBooPfmmym}
    Soit une fonction \( f\colon X\to Y\), continue sur l'ouvert \( A\) de \( X\) au sens où elle est continue en chaque point de \( A\). Alors la fonction restriction \( \tilde f\colon A\to Y\) est également continue pour la topologie sur \( A\), induite\footnote{Exemple~\ref{DefVLrgWDB}.} de \( X\).
\end{proposition}

\begin{proof}
    Soit \( a\in A\), et montrons que \( \tilde f\) est continue en \( a\), c'est-à-dire que \( \tilde f(a)=f(a)\) soit une limite de \( \tilde f\) en \( a\). Soit un voisinage \( V\) de \( \tilde f(a)\) dans \( Y\). Par la continuité de \( f\), nous avons un ouvert \( W\) de \( X\) tel que 
    \begin{equation}
        f\big( W\setminus\{ a \} \big)\subset V.
    \end{equation}
    La partie \( W\cap A\) est un voisinage de \( a\) pour la topologie de \( A\), et vérifie
    \begin{equation}
        f\big( W\cap A\setminus\{ a \} \big)\subset V.
    \end{equation}
    donc \( f(a)\) est une limite de \( \tilde f\) pour \( x\to a\). La fonction \( \tilde f\colon A\to Y\) est continue en chaque point de \( A\).
\end{proof}

Au niveau de la notion de continuité, il n'y a pas trop de changements en passant de \( \eR\) à \( \eQ\) muni de la topologie induite.

\begin{example}     \label{EXooHWIIooYYbfGE}
    Que signifie d'être continue pour une fonction \( f\colon \eQ\to \eR\) ? D'après le théorème~\ref{ThoESCaraB}, il s'agit d'être continue en chaque point de \( \eQ\). Il s'agit donc, par la définition~\ref{DefOLNtrxB} que pour tout \( q\in \eQ\), le nombre \( f(q)\) soit une limite de \( f\) pour \( x\to q\).

    L'espace d'arrivée étant \( \eR\), un voisinage de \( f(q)\) est pris comme une boule de taille \( \epsilon\). La continuité de \( f\) exige qu'il y ait un voisinage \( W\) de \( q\) dans \( \eQ\) tel que pour tout \( q'\in W\) (différent que \( q\)), \( | f(q)-f(q') |<\epsilon\).

    Qu'est-ce qu'un ouvert dans \( \eQ\) ? D'après la définition~\ref{DefVLrgWDB} de la topologie induite, ce sont les ensembles \( \eQ\cap\mO\) avec \( \mO\) ouvert dans \( \eR\). Tout cela pour dire que pour tout \( \epsilon>0\), il doit exister \( \delta>0\) tel que pour tout \( q'\in \eQ\) tel que \( 0<| q-q' |<\delta\), nous ayons \( | f(q)-f(q') |\).

    Bref, c'est exactement le mécanisme usuel de la continuité sur \( \eR\), sauf qu'il faut seulement considérer les rationnels.
\end{example}

\begin{lemma}[Application partielle\cite{MonCerveau}]       \label{LEMooHAODooYSPmvH}
    Soient trois espaces topologiques \( X_1\), \( X_2\) et \( Y\). Nous considérons une fonction continue \( f\colon X_1\times X_2\to Y\) ainsi que \( x_1\in X_1\). Alors l'application
    \begin{equation}
        \begin{aligned}
            g\colon X_2&\to Y \\
            x_2&\mapsto f(x_1,x_2)
        \end{aligned}
    \end{equation}
    est continue.
\end{lemma}

\begin{proof}
    Soit un ouvert \( \mO\) de \( Y\); par hypothèse sur \( f\), la partie \( f^{-1}(\mO)\) est ouverte dans \( X_1\times X_2\). Notre but est de prouver que \( g^{-1}(\mO)\) est un ouvert de \( X_2\). Nous avons :
    \begin{equation}
        g^{-1}(\mO)=\{ x_2\in X_2\tq (x_1,x_2)\in f^{-1}(\mO) \}.
    \end{equation}
    Nous considérons \( x_2\in g^{-1}(\mO)\) et nous prouvons qu'il existe dans \( X_2\) un voisinage de \( x_2\) entièrement contenu dans \( g^{-1}(\mO)\).

    Étant donné que \( (x_1,x_2)\) est dans \( f^{-1}(\mO)\) qui est ouvert, la définition~\ref{DefIINHooAAjTdY} de la topologie sur \( X_1\times X_2\) nous donne des ouverts \( A_1\) dans \( X_1\) et \( A_2\) dans \( X_2\) tels que
    \begin{equation}
        (x_1,x_2)\in A_1\times A_2\subset f^{-1}(\mO).
    \end{equation}

    Nous montrons à présent que \( A_2\subset g^{-1}(\mO)\). Soit \( y_2\in A_2\). Par construction \( (x_1,y_2)\in A_1\times A_2\subset f^{-1}(\mO)\), donc
    \begin{equation}
        g(y_2)=f(x_1,x_2)\in \mO.
    \end{equation}
    Cela termine la démonstration.
\end{proof}

%---------------------------------------------------------------------------------------------------------------------------
\subsection{Continuité et connexité}
%---------------------------------------------------------------------------------------------------------------------------


\begin{proposition} \label{PropConnexiteViaFonction}
  Un espace topologique \( X \) est connexe si et seulement si toute application continue \( X\to \eZ\) est constante.
\end{proposition}

\begin{proposition}\label{PropGWMVzqb}
    L'image d'un ensemble connexe par une fonction continue est connexe.
\end{proposition}

\begin{proof}
    Soit \( f\colon X\to Y\) une application continue entre deux espaces topologiques, et \( E\) une partie connexe de \( X\). Nous devons montrer que \( f(E)\) est connexe dans \( Y\).

    Par l'absurde nous considérons \( A\) et \( B\), deux ouverts de \( Y\) disjoints recouvrant \( f(E)\). Étant donné que \( f\) est continue, les ensembles \( f^{-1}(A)\) et \( f^{-1}(B)\) sont ouverts dans \( X\). De plus ces deux ensembles recouvrent \( E\).

    Si \( x\) est un élément de \( f^{-1}(A)\cap f^{-1}(B)\), alors \( f(x)\in A\cap B\), ce qui est impossible parce que nous avons supposé que \( A\) et \( B\) étaient disjoints. Par conséquent \( f^{-1}(A)\) et \( f^{-1}(B)\) sont deux ouverts disjoints recouvrant \( E\). Contradiction avec la connexité de \( E\). Nous concluons que \( f(E)\) est connexe.
\end{proof}
Une application de ce théorème sera le théorème de valeurs intermédiaires~\ref{ThoValInter}.

\begin{example}
    Les espaces topologiques \( \eR\) et \( \eR^2\) ne sont pas homéomorphes.
\end{example}

\begin{proof}
    Supposons par l'absurde que \( f\colon \eR\to \eR^2\) soit un  homéomorphisme. Nous posons \( E=f\big( \eR\setminus\{ 0 \} \big)\) et \( z_0=f(0)\). Vu que \( f\) est bijective nous avons
    \begin{equation}
        E=\eR^2\setminus\{ z_0 \},
    \end{equation}
    qui est connexe.

    Vu que \( E\) est connexe et que \( f^{-1}\) est continue, la proposition~\ref{PropGWMVzqb} nous dit que \( f^{-1}(E)\) est connexe. Mais par définition, \( f^{-1}(E)=\eR\setminus\{ 0 \}\) qui n'est pas connexe.
\end{proof}

%---------------------------------------------------------------------------------------------------------------------------
\subsection{Continuité et compacité}
%---------------------------------------------------------------------------------------------------------------------------

\begin{theorem}     \label{ThoImCompCotComp}
L'image d'un compact\footnote{Définition~\ref{DefJJVsEqs}.} par une fonction continue est un compact.
\end{theorem}
Dans le cadre des espaces vectoriels normés, ce théorème est démontré en la proposition~\ref{PropContinueCompactBorne}.

\begin{proof}
    Soit $K\subset X$, un ensemble compact, et regardons $f(K)$; en particulier, nous considérons $\Omega$, un recouvrement de $f(K)$ par des ouverts. Nous avons que
    \begin{equation}
        f(K)\subseteq\bigcup_{\mO\in\Omega}\mO.
    \end{equation}
    Par construction, nous avons aussi
    \begin{equation}
        K\subseteq\bigcup_{\mO\in\Omega}f^{-1}(\mO),
    \end{equation}
    en effet, si $x\in K$, alors $f(x)$ est dans un des ouverts de $\Omega$, disons $f(x)\in \mO$, et évidemment, $x\in f^{-1}(\mO)$.  Les $f^{-1}(\mO)$ recouvrent le compact $K$, et donc on peut en choisir un sous-recouvrement fini, c'est-à-dire un choix de $\{ f^{-1}(\mO_1),\ldots,f^{-1}(\mO_n) \}$ tels que
    \begin{equation}
        K\subseteq \bigcup_{i=1}^nf^{-1}(\mO_i).
    \end{equation}
    Dans ce cas, nous avons que
    \begin{equation}
        f(K)\subseteq\bigcup_{i=1}^n\mO_i,
    \end{equation}
    ce qui prouve la compacité de $f(K)$.
\end{proof}

%+++++++++++++++++++++++++++++++++++++++++++++++++++++++++++++++++++++++++++++++++++++++++++++++++++++++++++++++++++++++++++
\section{Topologie, distances et normes}
%+++++++++++++++++++++++++++++++++++++++++++++++++++++++++++++++++++++++++++++++++++++++++++++++++++++++++++++++++++++++++++
Certains ensembles ont plus de structures qu'une topologie. Nous fixons quelques bases maintenant, et nous détaillerons certains résultats plus tard.

%---------------------------------------------------------------------------------------------------------------------------
\subsection{Distance et topologie métrique}
%---------------------------------------------------------------------------------------------------------------------------

\begin{definition}  \label{DefMVNVFsX}
    Si $E$ est un ensemble, une \defe{distance}{distance} sur $E$ est une application $d\colon E\times E\to \eR$ telle que pour tout $x,y\in E$,
    \begin{enumerate}

    \item
    $d(x,y)\geq 0$

    \item
    $d(x,y)=0$ si et seulement si $x=y$,

    \item
    $d(x,y)=d(y,x)$

    \item
    $d(x,y)\leq d(x,z)+d(z,y)$.

    \end{enumerate}
    La dernière condition est l'\defe{inégalité triangulaire}{inégalité!triangulaire}.

    Un couple $(E,d)$ formé d'un ensemble et d'une distance est un \defe{espace métrique}{espace!métrique}.
\end{definition}

La définition-théorème suivante donne une topologie sur les espaces métriques en partant des boules.

\begin{theoremDef}     \label{ThoORdLYUu}
    Soit \( (E,d)\) un espace métrique. Nous définissons les \defe{boules ouvertes}{boule!ouverte} par
    \begin{equation}        \label{EQooYCWSooIhibvd}
        B(x,r)=\{ y\in E\tq d(x,y)<r \}.
    \end{equation}
    pout tout \( x\in E\) et \( r>0\).
    Alors en posant
    \begin{equation}        \label{EqGDVVooDZfwSf}
        \mT=\big\{  \mO\subset E  \tq\forall x\in \mO,\exists r>0\tq B(x,r)\subset \mO \big\}
    \end{equation}
    nous définissons une topologie sur \( E\).

    Cette topologie sur \( E\) est la \defe{topologie métrique}{topologie!métrique} de \( (E,d)\). En présence d'une distance, sauf mention explicite du contraire, c'est toujours cette topologie-là que nous utiliserons.
\end{theoremDef}

\begin{proof}
    D'abord \( \emptyset\in\mT\) parce que tout élément de l'ensemble vide \ldots heu \ldots enfin parce que d'accord hein\footnote{Pour qui ne seraient pas d'accord, allez ajouter \( \emptyset\) dans la définition des ouverts et puis c'est tout.}. Ensuite si \( (A_i)_{i\in I}\) sont des éléments de \( \mT\) et si \( x\in\bigcup_{i\in I}A_i\) alors il existe \( k\in I\) tel que \( x\in A_k\). Par hypothèse il existe une boule \( B(x,r)\subset A_k\subset\bigcup_{i\in I}A_i\).

    Enfin si \( (A_i)_{i\in\{ 1,\ldots, n \}}\) sont des éléments de \( \mT\) alors pour tout \( i\) il existe \( r_i>0\) tel que \( B(x,r_i)\subset A_i\). En prenant \( r=\min\{ r_i \}_{i=1,\ldots, n}\) nous avons $B(x,r)\subset\bigcap_{i=1}^nA_i.$
\end{proof}

\begin{remark}  \label{RemQDRooKnwKk}
    Quatre remarques à propos de cette définition.
    \begin{enumerate}
    \item
      Cette définition est faite exprès pour respecter le théorème~\ref{ThoPartieOUvpartouv}. Même si, à priori, on aurait dû utiliser la topologie engendrée faite à l'exemple \ref{DefTopologieEngendree}\dots\ mais on peut montrer que les deux topologies sont les mêmes.
    \item      \label{ITEMooUIHJooXAFaIz}
      Par construction, les boules ouvertes sont une base de la topologie (définition~\ref{DefQELfbBEyiB}) des espaces métriques.
    \item       \label{ITEMooUIHJooXAFaJa}
      Si \( V\) est un voisinage de \( x\), alors il existe \( r\) tel que \( B(x,r)\subset V\).
    \item
      Tout espace métrique est séparé. En effet, si deux éléments \( x \) et \( y \) sont distincts, alors en posant \( r = d(x , y) / 3 > 0 \), les boules \( B(x,r) \) et \( B(y,r)\) sont disjointes. Très pratique pour les limites : elles sont uniques, grâce aux propositions~\ref{PropUniciteLimitePourSuites} et \ref{PropFObayrf}!
    \end{enumerate}
\end{remark}

\begin{normaltext}
    Si vous avez un peu de temps, vous pouvez vérifier que si \( \eK\) est un corps totalement ordonné, alors avec toutes les définitions de~\ref{DefKCGBooLRNdJf}, en posant \( d(x,y)=| x-y |\) nous avons une distance sur \( \eK\).

    De plus, les boules définies en~\ref{DefKCGBooLRNdJf} sont alors les mêmes que celles définies en \eqref{EQooYCWSooIhibvd}, ce qui donne à tout corps totalement ordonné une structure d'espace topologique.
\end{normaltext}

\subsubsection{Les boules, une base de topologie}
%////////////////////////////

\begin{proposition} \label{PropNBSooraAFr}
    Un espace métrique séparable\footnote{Qui possède une partie dense dénombrable, définition~\ref{DefUADooqilFK}.} accepte une base de topologie dénombrable.

     Soit \( A\) dense et dénombrable dans l'espace métrique séparable \( (E,d)\). Si \( \{ a_i \}_{i\in \eN}\) est une énumération de \( A\) et \( \{ r_i \}_{i\in \eN}\) une énumération de \( \eQ\), alors
    \begin{equation}
        \mB=\{ B(a_i,r_j) \}_{i,j\in \eN}
    \end{equation}
    est une base de la topologie\footnote{Définition \ref{}.} de \( E\).
\end{proposition}
\index{base!de topologie!espace métrique}
\index{espace!métrique!base de topologie}
\index{base!de topologie!dénombrable}

\begin{proof}
    Soient \( x\in E\) et \( V\) un voisinage de \( x\). Ce dernier contient une boule \( B(x,r)\) et quitte à prendre \( r\) un peu plus petit nous supposons que \( r\in \eQ\) (existence d'un tel rationnel par le lemme \ref{LemooHLHTooTyCZYL}).

    Soit \( a\in A\) avec \( \| a-x \|<\frac{ r }{ 3 }\) (existe par densité de \( A\) dans \( E\)); nous avons \( B(a,\frac{ 2r }{ 3 })\subset B(x,r)\) parce que si \( y\in B( a,\frac{ 2r }{ 3 } )\) alors
    \begin{equation}
        \| y-x \|\leq \| y-a \|+\| a-x \|<\frac{ 2 }{ 3 }r+\frac{ 1 }{ 3 }r=r.
    \end{equation}
    La seconde inégalité est stricte parce que les boules sont ouvertes. Le tout montre que \( y\in B(x,r)\). Par ailleurs \( x\in B(a,\frac{ 2 }{ 3 }r)\) et nous avons trouvé un élément de \( \mB\) contenant \( x\) tout en étant inclus dans \( V\). Cela prouve que \( \mB\) est bien une base de la topologie de \( E\).
\end{proof}


\begin{remark}      \label{RemIPVLooHUXyeW}
    Il est vite vu que les cubes ouverts forment aussi une base de la topologie de \( \eR^n\). Cela est à mettre en rapport avec le fait que toutes les normes sont équivalentes sur \( \eR^n\) (proposition~\ref{ThoNormesEquiv}).

    % position 13268

    Voir aussi le corolaire~\ref{CorTHDQooWMSbJe} qui donnera tout ouvert comme union de pavés presque disjoints.
\end{remark}

\begin{definition}\label{DefEnsembleBorne}
  Soit \( (X, d) \) un espace métrique. Un sous-ensemble $A \subset X$ est \defe{borné}{borné} s'il existe une boule de $X$ contenant $A$.
\end{definition}

\begin{proposition}
  Toute réunion finie d'ensembles bornés est un ensemble borné. Toute partie d'un ensemble borné est un ensemble borné.
\end{proposition}

\subsubsection{Continuité et compacité}
%/////////////////

Un résultat important dans la théorie des fonctions sur les espaces vectoriels normés est qu'une fonction continue sur un compact est bornée et atteint ses bornes. Ce résultat sera (dans d'autres cours) énormément utilisé pour trouver des maximums et minimums de fonctions. Le théorème exact est le suivant.

\begin{lemma}[de Lebesgue\cite{AntoniniAndAl-EspacesMetriquesCompacts}]    \label{LemQFXOWyx}
    Soit \( (X,d)\) un espace métrique tel que toute suite ait une sous-suite convergente à l'intérieur de l'espace. Si \( \{ V_i \}\) est un recouvrement par des ouverts de \( X\), alors il existe \( \epsilon\) tel que pour tout \( x\in X\), nous ayons \( B(x,\epsilon)\subset V_i\) pour un certain \( i\).
\end{lemma}

\begin{proof}
    Par l'absurde, nous supposons que pour tout \( n\), il existe un \( x_n\in X\) tel que la boule \( B(x_n,\frac{1}{ n })\) n'est contenue dans aucun des \( V_i\). Ce des \( x_n\) nous extrayons une sous-suite convergente (que nous nommons encore \( (x_n)\)) et nous posons \( x_n\to x\). Pour \( n\) assez grand (\( \frac{1}{ n }<\epsilon\)) nous avons \( x_n\in B(x,\epsilon)\), donc tous les \( x_n\) suivants sont dans le \( V_i\) qui contient \( x\).
\end{proof}

\begin{lemma}[\cite{AntoniniAndAl-EspacesMetriquesCompacts}]   \label{LemMGQqgDG}
    Soit \( (X,d)\) un espace métrique tel que toute suite possède une sous-suite convergente. Pour tout \( \epsilon>0\), il existe un ensemble fini \( \{ x_i \}_{i\in I}\) tel que les boules \( B(x_i,\epsilon)\) recouvrent \( X\).
\end{lemma}

\begin{proof}
    Soit par l'absurde un \( \epsilon>0\) contredisant le lemme. Il n'existe pas d'ensemble finis autour des points duquel les boules de taille \( \epsilon\) recouvrent \( X\).

    Nous construisons par récurrence une suite ne possédant pas de sous-suites convergente. Le premier terme, \( x_0\) est pris arbitrairement dans \( X\). Ensuite si nous en avons \( N\) termes, nous savons que les boules de rayon \( \epsilon\) et centrées en les points \( \{ x_i \}_{i=1,\ldots, N}\) ne recouvrent pas \( X\). Donc nous prenons \( x_{N+1}\) hors de l'union de ces boules.

    Ainsi nous avons une suite \( (x_n)\) dont tous les termes sont à distance plus grande que \( \epsilon\) les uns des autres. Une telle suite ne peut pas contenir de sous-suite convergente. Contradiction.
\end{proof}

\begin{theorem}[Bolzano-Weierstrass\cite{AntoniniAndAl-EspacesMetriquesCompacts}, thème \ref{THEMEooQQBHooLcqoKB}]\label{ThoBWFTXAZNH}
    Un espace métrique est compact si et seulement si toute suite admet une sous-suite qui converge à l'intérieur de l'espace.
\end{theorem}
\index{théorème!Bolzano-Weierstrass}
\index{Bolzano-Weierstrass!espaces métriques}
\index{compacité}

\begin{proof}
   Soient \( X\) un espace métrique compact et \( (x_n)\) une suite dans \( X\). Nous considérons la suite de fermés emboîtés
   \begin{equation}
       X_n=\overline{ \{ x_k\tq k>n \} }.
   \end{equation}
   Ce sont des fermés ayant la propriété d'intersection finie non vide, et donc la proposition~\ref{PropXKUMiCj} nous dit qu'ils ont une intersection non vide. Un élément de cette intersection est automatiquement un point d'accumulation de la suite\footnote{Définition \ref{DEFooGHUUooZKTJRi}.}.

   Nous passons à l'autre sens. Nous supposons que toute suite dans \( X\) contient une sous-suite convergente, et nous considérons \( \{ V_i \}_{i\in I}\), un recouvrement de \( X\) par des ouverts. Par le lemme~\ref{LemQFXOWyx}, nous considérons un \( \epsilon\) tel que pour tout \( x\), il existe un \( i\in I\) avec \( B(x,\epsilon)\subset V_i\). Par le lemme~\ref{LemMGQqgDG}, nous considérons un ensemble fini \( \{ y_i \}_{i\in A}\) tel que le boules \( B(y_i,\epsilon)\) recouvrent \( X\).

   Par construction, chacune de ces boules \( B(y_i,\epsilon)\) est contenue dans un des ouverts \( V_i\). Nous sélectionnons donc parmi les \( V_i\) le nombre fini qu'il faut pour recouvrir les \( B(y_i,\epsilon)\) et donc pour recouvrir \( X\).
\end{proof}

\begin{example}[Non compacité de la boule unité en dimension infinie]\label{ExEFYooTILPDk}
    Le théorème de Bolzano-Weierstrass permet de voir tout de suite que la boule unité n'est pas compacte dans un espace vectoriel de dimension infinie : la suite des vecteurs de base ne possède pas de sous-suites convergentes.
\end{example}


Le théorème de Bolzano–Weierstrass~\ref{ThoBWFTXAZNH} a l'importante conséquence suivante.
\begin{theorem}[Weierstrass]		\label{ThoWeirstrassRn}
	Une fonction continue à valeurs réelles définie sur un compact est bornée et atteint ses bornes.
\end{theorem}
\index{théorème!Weierstrass}
\index{compact!et fonction continue}

\begin{proof}
	Soient \( K\) un compact et $f\colon K\to \eR$ une fonction continue. Nous désignons par $A$ l'ensemble des valeurs prises par $f$ sur $K$ :
	\begin{equation}
		A=f(K)=\{ f(x)\tq x\in K \}.
	\end{equation}
	Nous considérons le supremum $M=\sup A=\sup_{x\in K}f(x)$ avec la convention comme quoi si $A$ n'est pas borné supérieurement, nous posons $M=\infty$ (voir définition~\ref{DefSupeA}).

	Nous allons maintenant construire une suite $(x_n)$ de deux façons différentes suivant que $M=\infty$ ou non.
	\begin{enumerate}
		\item
			Si $M=\infty$, nous choisissons, pour chaque $n\in\eN$, un $x_n\in K$ tel que $f(x_n)>n$. Cela est certainement possible parce que si $A$ n'est pas borné, nous pouvons y trouver des nombres aussi grands que nous voulons.
		\item
			Si $M<\infty$, nous savons que pour tout $\varepsilon$, il existe un $y\in A$ tel que $y>M-\varepsilon$. Pour chaque $n$, nous choisissons donc $x_n\in K$ tel que $f(x_n)>M-\frac{1}{ n }$.
	\end{enumerate}
    Quel que soit le cas dans lequel nous sommes, la suite $(x_n)$ est une suite dans $K$ qui est compact, et donc nous pouvons en extraire une sous-suite convergente à l'intérieur de \( K\) par le théorème de Bolzano-Weierstrass~\ref{ThoBWFTXAZNH}. Afin d'alléger la notation, nous allons noter $(x_n)$ la sous-suite convergente. Nous avons donc
	\begin{equation}
		x_n\to x\in K.
	\end{equation}
	Par la proposition~\ref{PropFnContParSuite}, nous avons que $f$ prend en \( x\) la valeur
	\begin{equation}
		f(x)=\lim_{n\to \infty} f(x_n).
	\end{equation}
	Donc $f(x)<\infty$. Évidemment, si nous avions été dans le cas où $M=\infty$, la suite $x_n$ aurait été choisie pour avoir $f(x_n)>n$ et donc il n'aurait pas été possible d'avoir $\lim_{n\to \infty} f(x_n)<\infty$. Nous en concluons que $M<\infty$, et donc que $f$ est bornée sur $K$.

	Afin de prouver que $f$ atteint sa borne, c'est-à-dire que $M\in A$, nous considérons les inégalités
	\begin{equation}
		M-\frac{1}{ n }<f(x_n)\leq M.
	\end{equation}
	En passant à la limite $n\to \infty$, ces inégalités deviennent
	\begin{equation}
		M\leq f(x)\leq M,
	\end{equation}
	et donc $f(x)=M$, ce qui prouve que $f$ atteint sa borne $M$ au point $x\in K$.
\end{proof}

\begin{lemma}[\cite{MonCerveau}]       \label{LEMooQLVAooICaPvR}
    Soient des compacts \( A,B\) et une fonction continue \( f\colon A\times B\to \eR\). Alors
    \begin{equation}
        \sup_{(x,y)\in A\times B}| f(x,y) |=\sup_{x\in A}\big( \sup_{y\in B}| f(x,y) | \big).
    \end{equation}
\end{lemma}

\begin{proof}
    Pour chaque \( x\in A \), la fonction \( f_x\colon B\to \eR\) donnée par \( f_x(y)=| f(x,y) |\) est continue et atteint donc sa borne\footnote{Théorème \ref{ThoWeirstrassRn}.} en \( y_M(x)\). Notons que cela ne définit pas univoquement \( y_M(x)\) parce que \( f_x\) peut atteindre son maximum en plusieurs points. L'important est que pour tout \( x\), le nombre \( | f\big( x,y_M(x) \big) |\) ne dépend pas du choix de \( y_M(x)\) parmi les \( y\) qui réalisent le maximum.


    Notons \( (x_0,y_0)\) le point de \( A\times B\) sur lequel \( | f |\) réalise son maximum\footnote{Encore une fois, ce point n'est pas déterminé de façon unique par cette propriété.} :
    \begin{equation}        \label{EQooDDXDooVsnlKG}
        \sup_{(x,y)\in A\times B}| f(x,y) |=| f(x_0,y_0) |.
    \end{equation}
    

    Nous avons d'une part
    \begin{equation}
        \sup_{x\in A}\big( \sup_{y\in B}| f(x,y) | \big)=\sup_{x\in A}| f\big( x,y_M(x) \big) |\leq | f(x_0,y_0) |
    \end{equation}

    Et d'autre part,
    \begin{subequations}        \label{SUBEQooPYJPooBJyEgN}
        \begin{align}
            \sup_{x\in A}\big( \sup_{y\in B}| f(x,y) | \big)&\leq \sup_{x\in A}\sup_{y\in B}| f(x_0,y_0) |\\
            &=| f(x_0,y_0) |\\
            &\leq | f\big(x_0,y_M(x_0)\big) |\\
            &\leq \sup_{x\in A}| f(x),y_M(x) |\\
            &\leq \sup_{x\in A}\big( \sup_{y\in B}| f(x,y) | \big).
        \end{align}
    \end{subequations}
    Vu que les premiers et derniers termes des inégalités \eqref{SUBEQooPYJPooBJyEgN} sont égaux, toutes les inégalités sont en réalité des égalités. En particulier, en en reprenant \eqref{EQooDDXDooVsnlKG},
    \begin{equation}
        \sup_{(x,y)\in A\times B}| f(x,y) |=| f(x_0,y_0) |=\sup_{x\in A}\big( \sup_{y\in B}| f(x,y) | \big).
    \end{equation}
\end{proof}

%--------------------------------------------------------------------------------------------------------------------------- 
\subsection{Distance à un ensemble}
%---------------------------------------------------------------------------------------------------------------------------

\begin{definition}      \label{DEFooGNNUooFUZINs}
    Si \( A\) est une partie de l'espace métrique \( (X,d)\), et si \( b\in X\), nous définissons
    \begin{equation}
        d(b,A)=\inf_{y\in A}d(b,y).
    \end{equation}
\end{definition}

\begin{lemma}[\cite{MonCerveau}]        \label{LEMooAIARooQADaxM}
    Si \( A\) est fermé dans \( (X,d)\), et si \( b\in X\) vérifie \( d(b,A)=0\), alors \( b\in A\).
\end{lemma}

\begin{proof}
    Vu que \( A\) est fermé, le complémentaire \( A^c\) est ouvert (c'est la définition \ref{DefFermeVoisinage}). Supposons que \( b\in A^c\). Alors il existe \( r>0\) tel que \( B(b,r)\subset A^c\). Si \( a\in A\) nous avons alors \( d(b,A)\geq r\) et donc \( d(b,A)\geq r>0\). Cela contredit l'hypothèse \( d(b,A)=0\).

    Nous en déduisons que \( b\) n'est pas dans \( A^c\) et qu'il est donc dans \( A\).
\end{proof}


\begin{example}[Pas avec un ouvert]
    En prenant l'ouvert \( A=\mathopen] 0 , 1 \mathclose[\) dans \( \eR\) nous avons \( d(0,A)=0\), alors que \( 0\) n'est pas dans \( A\).
\end{example}

\begin{lemma}[\cite{MonCerveau}]    \label{LEMooJNRTooZyKiFC}
    Soient un espace métrique \( (X,d)\) ainsi qu'une partie \( A\subset X\). Soit \( r>0\). La partie
    \begin{equation}
        \mO=\{ x\in X\tq d(x,A) \}
    \end{equation}
    est ouverte.
\end{lemma}

\begin{proof}
   Soit \( y\in \mO\); nous avons \( d(y,A)<r\). Autrement dit,
   \begin{equation}
       \inf_{a\in A}d(y,a)<r
   \end{equation}
   et donc il existe \( a\in A\) tel que \( d(y,a)<r\). Soit \( \delta=d(y,a)<r\). Nous montrons à présent que \( B(y,r-\delta)\) est dans \( \mO\). En effet si \( z\in B(y,r-\delta)\), alors
   \begin{equation}
       d(z,a)\leq d(z,y)+d(y,a)<r-\delta+\delta=r.
   \end{equation}
\end{proof}

\begin{lemma}[\cite{MonCerveau}]        \label{LEMooEQIZooLpsbOe}
    Si \( F\) est un fermé dans \( (X,d)\) et si \( x\) n'est pas dans \( F\), alors \( d(x,F)>0\).
\end{lemma}

\begin{lemma}[\cite{MonCerveau}]        \label{LEMooCFGTooIfdcfk}
    Si \( A\) est une partie de \( (X,d)\), alors la fonction
    \begin{equation}
        \begin{aligned}
            f\colon \Omega&\to \mathopen[ 0 , \infty \mathclose[ \\
            x&\mapsto d(x,A) 
        \end{aligned}
    \end{equation}
    est continue.
\end{lemma}

%------------------------------------------------------------------------------------------------------
\subsection{Norme}
%------------------------------------------------------------------------------------------------------

\begin{definition}[\cite{BrunelleMatricielle}, thème~\ref{THEMEooUJVXooZdlmHj}]  \label{DefNorme}
    Soit \( E\) un espace vectoriel (pas spécialement de dimension finie) sur le corps \( \eK\) (\( =\eR\) ou \( \eC\)). Une  \defe{norme}{norme} sur $E$ est une application $N\colon E\to \eR^+$ telle que
	\begin{enumerate}
		\item
            \( N(x)=0\) si et seulement si \( x=0\);
		\item\label{ItemDefNormeii}
			$N(\lambda x)=| \lambda |N(x)$ pour tout $\lambda\in\eR$ et $x\in E$;
		\item\label{ItemDefNormeiii}
			$N(x+y)\leq N(x)+N(y)$
	\end{enumerate}
    pour tout $x,y\in E$ et pour tout $\lambda\in\eK$.

    La propriété~\ref{ItemDefNormeiii} est appelée \defe{inégalité triangulaire}{inégalité!triangulaire}.

    Un espace vectoriel muni d'une norme est un \defe{espace vectoriel normé}{espace vectoriel normé}.
\end{definition}
En prenant $\lambda=-1$ dans la propriété~\ref{ItemDefNormeii}, nous trouvons immédiatement que $N(-x)=N(x)$.

\begin{proposition}		\label{PropNmNNm}
	Toute norme $N$ sur l'espace vectoriel $E$ vérifie l'inégalité
	\begin{equation}
		\big| N(x)-N(y) \big|\leq N(x-y)
	\end{equation}
	pour tout $x,y\in E$.
\end{proposition}

\begin{proof}
	Nous avons, en utilisant le point~\ref{ItemDefNormeiii} de la définition~\ref{DefNorme},
	\begin{subequations}
		\begin{align}
			N(x)&=N(x-y+y)\leq N(x-y)+N(y),	\label{subEqNNNxxyyya}\\
			N(y)&=N(y-x+x)\leq N(y-x)+N(x).	\label{subEqNNNxxyyyb}
		\end{align}
	\end{subequations}
	Supposons d'abord que $N(x)\geq N(y)$. Dans ce cas, en utilisant \eqref{subEqNNNxxyyya},
	\begin{equation}
		\big| N(x)-N(y) \big|=N(x)-N(y)\leq N(x-y)+N(y)-N(y)=N(x-y).
	\end{equation}
	Si par contre $N(x)\leq N(y)$, alors nous utilisons \eqref{subEqNNNxxyyyb} et nous trouvons
	\begin{equation}
		\big| N(x)-N(y) \big|=N(y)-N(x)\leq N(y-x)+N(x)-N(x)=N(y-x).
	\end{equation}
	Dans les deux cas, nous avons retrouvé l'inégalité annoncée.
\end{proof}
Cette proposition signifie aussi que
\begin{equation}	\label{EqNleqNNleqNvqlqbs}
	-N(x-y)\leq N(x)-N(y)\leq N(x-y).
\end{equation}

\begin{normaltext}	
Afin de suivre une notation proche de celle de la valeur absolue, à partir de maintenant, la norme d'un vecteur $v$ sera notée $\| v\|$ au lieu de $N(v)$. La proposition~\ref{PropNmNNm} s'énoncera donc
\begin{equation}
\big| \| x \|-\| y \| \big|\leq \| x-y \|.
\end{equation}
Un espace vectoriel $E$ muni d'une norme est, on l'a déjà dit, un \defe{espace vectoriel normé}{normé!espace vectoriel}; on le notera $(E,\| . \|)$ pour distinguer la norme fixée.
\end{normaltext}

Une autre inégalité utile de temps en temps.
\begin{corollary}       \label{CORooDFBGooAqVRfS}
    Si \( a\) et \( b\) sont dans un espace vectoriel normé, alors
    \begin{equation}
        \big| \| a-b \|-\| b \| \big|\leq \| a \|.
    \end{equation}
\end{corollary}

\begin{proof}
    Il s'agit seulement de la proposition \ref{PropNmNNm} avec \( x=a-b\) et \( y=-b\).
\end{proof}

\begin{lemmaDef}[Distance induite par une norme]        \label{LEMooWGBJooYTDYIK}
    Soit un espace vectoriel normé \( (E,\| . \|)\). Nous posons
    \begin{equation}        \label{EQooZYJRooAHnsIG}
        d(x,y)=\| x-y \| .
    \end{equation}
    Alors
    \begin{enumerate}
        \item       \label{ITEMooLITDooPeReOk}
            \( d\) est invariante par translations : $d(a,b)=d(a+u,b+u)$
        \item
            \( d\) est une distance\footnote{Définition~\ref{DefMVNVFsX}.} sur \( E\).
    \end{enumerate}
    C'est la \defe{distance induite}{distance!associée à une norme} par la norme.
\end{lemmaDef}

\begin{proof}
    Le fait que la formule \eqref{EQooZYJRooAHnsIG} soit invariante par translations est immédiat. En ce qui concerne le fait que ce soit une distance, le seul point délicat à vérifier est l'inégalité triangulaire. Mais, pour tous \( x, y, z \in E\), on a
    \begin{equation}
            d(x,y)=\| x-y \| = \| x-z+z-y \|  \leq\| x - z \|+\| z - y\| =d(x,z)+d(z,y).
    \end{equation}
\end{proof}


\begin{corollary}
Un espace vectoriel normé est un espace vectoriel topologique : en d'autres mots, l'addition et la multiplication par un élément du corps sont continues.
\end{corollary}

Nous étudierons plus en détail les espaces vectoriels topologiques à partir de la définition~\ref{DefEVTopologique}.

% This is part of Mes notes de mathématique
% Copyright (c) 2008-2020
%   Laurent Claessens, Carlotta Donadello
% See the file fdl-1.3.txt for copying conditions.

%+++++++++++++++++++++++++++++++++++++++++++++++++++++++++++++++++++++++++++++++++++++++++++++++++++++++++++++++++++++++++++
\section{Topologie sur l'ensemble des réels}
%+++++++++++++++++++++++++++++++++++++++++++++++++++++++++++++++++++++++++++++++++++++++++++++++++++++++++++++++++++++++++++
\label{SECooGKHYooMwHQaD}

Nous allons à présent donner la topologie sur \( \eR\) et ainsi résoudre les questions laissées en suspens lors de la construction des réels, voir~\ref{NormooHRDZooRGGtCd}.


Afin de pouvoir étudier la topologie des espaces métriques, il faut savoir quelques propriétés des réels parce que nous allons étudier la fonction distance qui est une fonction continue à valeurs dans les réels.

La valeur absolue de la définition~\ref{DefKCGBooLRNdJf}\ref{ItemooWUGSooRSRvYC} permet de définir une norme sur \( \eR\).
\begin{lemma}
    L'application
    \begin{equation}
         x\mapsto | x |
    \end{equation}
     est une norme sur $\eR$.
\end{lemma}

\begin{proof}
  Grâce au lemme \ref{LemooANTJooYxQZDw} et à la remarque \ref{RemooJCAUooKkuglX}, on a, pour tous \(x,\ y,\ \lambda \in \eR \):
\begin{enumerate}
\item $| x |=0$ implique $x=0$,
\item $| \lambda x |=| \lambda | |x |$,
\item $| x+y |\leq | x |+| y |$,
\end{enumerate}
et donc, les conditions de la définition \ref{DefNorme} sont immédiatement vérifiées.
\end{proof}

\begin{normaltext}      \label{ooLCMFooQjMaxV}
    Nous verrons plus tard que cette norme donne lieu à une structure d'espace topologique. Tant sur \( \eQ\) que sur \( \eR\), nous considérons la topologie métrique correspondant à cette norme (hors cas rarissimes qui seront signalés). De plus, nous utiliserons toujours les caractérisations de la proposition~\ref{PropooUEEOooLeIImr} pour parler de suites convergentes et de suites de Cauchy.
\end{normaltext}

\begin{proposition}     \label{PropooUHNZooOUYIkn}
    Les rationnels sont denses dans les réels.
\end{proposition}
\index{densité!de \( \eQ\) dans \( \eR\)}

\begin{proof}
    Soient \( r\in \eR\) et \( \epsilon\in \eR^+\). Nous devons prouver l'existence d'un rationnel dans \( B(x,\epsilon)\). Le lemme~\ref{LemooHLHTooTyCZYL} dit qu'il existe un rationnel dans \( \mathopen] x-\epsilon/2 , x+\epsilon/2 \mathclose[\) et donc dans \( B(x,\epsilon)\).
\end{proof}

\begin{proposition}[\cite{MonCerveau}] \label{PropSLCUooUFgiSR}
    Quel que soit le réel \( r\), il existe une suite croissante de rationnels convergente vers \( r\).
\end{proposition}

\begin{proof}
    Soient \( x\in \eR\) et \( \delta\in \eR\); vu que \( x-\delta\) et \( x\) sont des réels, le lemme~\ref{LemooHLHTooTyCZYL} donne un élément \( x_{\delta}\in \eQ \) tel que
    \begin{equation}
        x-\delta<x_{\delta}<x.
    \end{equation}
    Il suffit alors de pêcher parmi ces \( x_{\delta}\) pour trouver une suite croissante, et on montrera que cette suite converge vers \( x \).

    Soit \( x_0\) un rationnel plus petit que \( x\). Nous posons \( \delta_0=x-x_0\) et ensuite :
    \begin{subequations}
        \begin{numcases}{}
            \delta_i=x-x_i\\
            x_{i+1}=x_{\delta_i/2} \in \eQ.
        \end{numcases}
    \end{subequations}
    Ainsi nous avons pour tout \( i\) les inégalités
    \begin{equation}
        x_i=x-\delta_i<x-\frac{ \delta_i }{ 2 }<x_{i+1}<x.
    \end{equation}
    La suite \( (x_i) \) est donc une suite de rationnels, croissante et toujours plus petite que \( x\). Mais nous avons à chaque étape \( \delta_{i+1}<\frac{ \delta_i }{ 2 }\), ce qui implique que la suite des  \( \delta_i \) converge vers \( 0 \). Soit \( \epsilon>0\). Il existe \( k_0\) tel que pour tout \( k > k_0 \), \( \delta_k<\epsilon\). Pour un tel \( k \), nous avons alors
    \begin{equation}
        x_{k+1}\in B(x,\frac{ \delta_k }{ 2 })\subset B(x,\epsilon).
    \end{equation}
Tous les \( x_k \), pour \( k > k_0 + 1 \), sont tels que \( |x - x_k| < \epsilon \): la suite des \( x_k \) converge donc vers \( x \).
\end{proof}

%--------------------------------------------------------------------------------------------------------------------------- 
\subsection{Compacité pour les réels}
%---------------------------------------------------------------------------------------------------------------------------

Pour la définition générale d'un compact, c'est \ref{DefJJVsEqs}.

\begin{proposition}     \label{PROPooBFSAooKSugMj}
    Les parties compactes de \( \eR\) sont fermées et bornées.
\end{proposition}

\begin{proof}
Prouvons d'abord qu'un ensemble compact est borné. Pour cela, supposons que $K$ est un compact non borné vers le haut\footnote{Nous laissons à titre d'exercice le cas où $K$ est borné par le haut et pas par le bas.}. Donc il existe une suite infinie de nombres strictement croissante $x_1<x_2<\ldots$ tels que $x_i\in K$. Prenons n'importe quel recouvrement ouvert de la partie de $K$ plus petite ou égale à $x_1$, et complétons ce recouvrement par les ouverts $\mO_i=]x_{i-1},x_i[$. Le tout forme bien un recouvrement de $K$ par des ouverts.

Il n'y a cependant pas moyen d'en tirer un sous recouvrement fini parce que si on ne prend qu'un nombre fini parmi les $\mO_i$, on en aura fatalement un maximum, disons $\mO_k$. Dans ce cas, les points $x_{k+1}$, $x_{k+1}$,\ldots ne seront pas dans le choix fini d'ouverts.

Cela prouve que $K$ doit être borné.

Pour prouver que $K$ est fermé, nous allons prouver que le complémentaire est ouvert. Et pour cela, nous allons prouver que si le complémentaire n'est pas ouvert, alors nous pouvons construire un recouvrement de $K$ dont on ne peut pas extraire de sous recouvrement fini.

Si $\eR\setminus K$ n'est pas ouvert, il possède un point, disons $x$, tel que tout voisinage de $x$ intersecte $K$. Soit $B(x,\epsilon_1)$, un de ces voisinages, et prenons $k_1\in K\cap B(x,\epsilon_1)$. Ensuite, nous prenons $\epsilon_2$ tel que $k_1$ n'est pas dans $B(x,\epsilon_1)$, et nous choisissons $k_2\in K\cap B(x,\epsilon_2)$. De cette manière, nous construisons une suite de $k_i\in K$ tous différents et de plus en plus proches de $x$. Prenons un recouvrement quelconque par des ouverts de la partie de $K$ qui n'est pas dans $B(x,\epsilon_1)$. Les nombres $k_i$ ne sont pas dans ce recouvrement.

Nous ajoutons à ce recouvrement les ensembles $\mO=]k_i,k_{i+1}[$. Le tout forme un recouvrement (infini) par des ouverts dont il n'y a pas moyen de tirer un sous recouvrement fini, pour exactement la même raison que la première fois.
\end{proof}

\begin{theorem}[Borel-Lebesgue]   \label{ThoBOrelLebesgue}
    Un intervalle de \( \eR\) est compact si et seulement si il est de la forme \( \mathopen[ a , b \mathclose]\).
\end{theorem}

\begin{proof}
    Tous les intervalles de \( \eR\) sont listés dans la proposition \ref{PROPooHPMWooQJXCAS}. Un compact est fermé et borné (proposition \ref{PROPooBFSAooKSugMj}). Donc les intervalles dont une borne est \( \pm\infty\) ne sont pas compacts. Parmi les intervalles \( \mathopen] a , b \mathclose[\), \( \mathopen] a , b \mathclose]\), \( \mathopen[ a , b \mathclose[\) et \( \mathopen[ a , b \mathclose]\), seul le dernier est fermé. Nous avons prouvé que si un intervalle est compact, alors il est de la forme \( \mathopen[ a , b \mathclose]\). 

    Nous prouvons à présent l'implication inverse : tous les intervalles de la forme \( \mathopen[ a , b \mathclose]\) sont compacts.

    Soit $\Omega$, un recouvrement du segment $[a,b]$ par des ouverts, c'est-à-dire que
    \begin{equation}
        [a,b]\subseteq\bigcup_{\mO\in\Omega}\mO.
    \end{equation}
    Nous notons par $M$ le sous-ensemble de $[a,b]$ des points $m$ tels que l'intervalle $[a,m]$ peut être recouvert par un sous-ensemble fini de $\Omega$. C'est-à-dire que $M$ est le sous-ensemble de $[a,b]$ sur lequel le théorème est vrai. Le but est maintenant de prouver que $M=[a,b]$.
    \begin{description}
        \item[$M$ est non vide] En effet, $a\in M$ parce que il existe un ouvert $\mO\in\Omega$ tel que $a\in\mO$. Donc $\mO$ tout seul recouvre l'intervalle $[a,a]$.
        \item[$M$ est un intervalle] Soient $m_1$, $m_2\in M$. Le but est de montrer que si $m'\in[m_1,m_2]$, alors $m'\in M$. Il y a un sous recouvrement fini de l'intervalle $[a,m_2]$ (par définition de $m_2\in M$). Ce sous recouvrement fini recouvre évidemment aussi $[a,m']$ parce que $[a,m']\subseteq [a,m_2]$, donc $m'\in M$.
        \item[$M$ est une ensemble ouvert] Soit $m\in M$. Le but est de prouver qu'il y a un ouvert autour de $m$ qui est contenu dans $M$. Mettons que $\Omega'$ soit un sous recouvrement fini qui contienne l'intervalle $[a,m]$. Dans ce cas, on a un ouvert $\mO\in\Omega'$ tel que $m\in\mO$. Tous les points de $\mO$ sont dans $M$, vu qu'ils sont tous recouverts par $\Omega'$. Donc $\mO$ est un voisinage de $m$ contenu dans $M$.
        \item[$M$ est un ensemble fermé] $M$ est un intervalle qui commence en $a$, en contenant $a$, et qui finit on ne sait pas encore où. Il est donc soit de la forme $[a,m]$, soit de la forme $[a,m[$. Nous allons montrer que $M$ est de la première forme en démontrant que $M$ contient son supremum $s$. Ce supremum est un élément de $[a,b]$, et donc il est contenu dans un des ouverts de $\Omega$. Disons $s\in\mO_s$. Soit $c$, un élément de $\mO_s$ strictement plus petit que $c$; étant donné que $s$ est supremum de $M$, cet élément $c$ est dans $M$, et donc on a un sous recouvrement fini $\Omega'$ qui recouvre $[a,c]$. Maintenant, le sous recouvrement constitué de $\Omega'$ et de $\mO_s$ est fini et recouvre $[a,s]$.
    \end{description}
    Nous pouvons maintenant conclure : le seul intervalle non vide de $[a,b]$ qui soit à la fois ouvert et fermé est $[a,b]$ lui-même (proposition \ref{PropHSjJcIr}), ce qui prouve que $M=[a,b]$, 
    et donc que $[a,b]$ est compact\footnote{Si vous n'aimez pas le coup du fermé et ouvert, le lemme \ref{LemOACGWxV} donne une autre preuve.}.
\end{proof}


\begin{lemma}[\cite{JUwQXOF}]\label{LemOACGWxV}
    Si \( a<b\in \eR\) alors le segment \( \mathopen[ a , b \mathclose]\) est compact\footnote{Définition~\ref{DefJJVsEqs}}.
\end{lemma}
\index{compact!intervalle \( \mathopen[ a , b \mathclose]\)}

\begin{proof}
    Soit \( \{ \mO_i \}_{i\in I}\) un recouvrement de \( \mathopen[ a , b \mathclose]\) par des ouverts. Nous posons
    \begin{equation}
        M=\{ x\in\mathopen[ a , b \mathclose]\tq \mathopen[ a , x \mathclose] \text{ admet un sous-recouvrement fini extrait de } \{ \mO_i \}_{i\in I} \}.
    \end{equation}
    Notre but est de prouver que \( b\in M\).
    \begin{subproof}

    \item[\( a\) est dans \( M\)]

        Le point \( a\) est naturellement dans un des \( \mO_i\). L'intervalle \( \mathopen[ a , a \mathclose]\) est donc recouvert par un seul des \( \mO_i\).

    \item[\( M\) est un intervalle]

        Soient \( m\in M\) et \( m'\in\mathopen[ a , m [\). Le sous-recouvrement fini qui recouvre \( \mathopen[ a , m \mathclose]\) recouvre a fortiori \( \mathopen[ a , m' \mathclose]\).

    \item[Les trois possibilités restantes]
        À ce niveau de la preuve, il reste trois possibilités pour \( M\) soit il est de la forme \( \mathopen[ a , c \mathclose]\) ou \( \mathopen[ a , c [\) avec \( c<b\), soit il est de la forme \( \mathopen[ a , b \mathclose]\). Nous allons maintenant éliminer les deux premiers cas.

    \item[Ce que \( M\) n'est pas]

        D'abord \( M\) n'est pas de la forme \( \mathopen[ a , c [\) avec \( c<b\). Par l'absurde, commençons par considérer \( \mO_{i_0}\) un ouvert du recouvrement qui contient \( c\); choisissons  \(m \in \mO_{i_0}\) tel que \( m<c\). Alors \( m \in M \), et, si nous joignons \( \mO_{i_0}\) à un recouvrement fini de \( \mathopen[ a , m \mathclose]\) alors nous avons un recouvrement fini de \( \mathopen[ a , c \mathclose]\). On en déduit \( c\in M\).

        Ensuite \( M\) n'est pas de la forme \( \mathopen[ a , c \mathclose]\) avec \( c<b\). En effet si on a un recouvrement fini de \( \mathopen[ a , c \mathclose]\) par des ouverts, alors un de ces ouverts contient \( c\) et donc contient des éléments de \( \mathopen[ a , b \mathclose]\) plus grands que \( c\).
    \end{subproof}
    Nous déduisons que \( M=\mathopen[ a , b \mathclose]\) et qu'il est possible d'extraire un sous-recouvrement fini recouvrant \( \mathopen[ a , b \mathclose]\).
\end{proof}

\begin{lemma}[\cite{MonCerveau}]\label{LemCKBooXkwkte}
    Si \( K_1\) et \( K_2\) sont des compacts dans \( \eR\) alors \( K_1\times K_2\) est compact dans \( \eR^2\).
\end{lemma}

\begin{proof}
    Soit \( \{ \mO_i \}_{i\in I}\) un recouvrement de \( K_1\times K_2\) par des ouverts; grâce au lemme~\ref{LemOWVooZKndbI} nous pouvons supposer que ce sont des carrés. Pour chaque \( x\in K_1\), l'ensemble \( \{ x \}\times K_2\) est compact et donc recouvert par un nombre fini des \( \mO_i\). Soit \( R_x\) un ensemble fini des \( \mO_i\) recouvrant \( \{ x \}\times K_2\).

    Vu que \( R_x\) est une collection finie de carrés nous pouvons considérer \( m_x\), le minimum des rayons. L'ensemble \( K_1\) est recouvert par les boules \( B(x,m_x)\) et il existe donc une collection finie de \( \{ x_i \}_{i\in A}\) tels que \( B(x_i,m_{x_i})\) recouvre \( K_1\).

    Alors \( \{ R_{x_i} \}_{i\in A}\) recouvre \( K_1\times K_2\) parce que \( R_{x_i}\) recouvre l'ensemble \( B(x_i,m_{x_i})\times \{ K_2 \}\).
\end{proof}

%--------------------------------------------------------------------------------------------------------------------------- 
\subsection{Conséquence: les fermés bornés sont compacts}
%---------------------------------------------------------------------------------------------------------------------------

\begin{theorem}[Théorème de Borel-Lebesgue] \label{ThoXTEooxFmdI}
    Une partie d'un espace vectoriel normé réel de dimension finie est compacte si et seulement si elle est fermée et bornée.
\end{theorem}
\index{théorème!Borel-Lebesgue}
\index{compact!fermé et borné}

\begin{proof}
    Sens direct.
    \begin{subproof}
    \item[Compact implique borné]
        En effet si \( K\) est non borné dans \( E\) alors \( K\) contient une suite \( (x_n)\) avec \( \| x_n \|>n\). Les boules \( B_i(x_i,\frac{ 1 }{3})\) sont disjointes. On pose \( \mO_0=\complement\bigcup_i\overline{ B(x_i,\frac{1}{ 5 }) }\), qui est ouvert comme complément d'un fermé. Pour \( i\geq 1\) nous posons \( \mO_i=B(x_i,\frac{1}{ 4 })\). Nous avons
        \begin{equation}
            K\subset\bigcup_{i\in \eN}\mO_i
        \end{equation}
        mais vu que \( x_i\) est uniquement dans \( \mO_i\), nous ne pouvons pas extraire de sous-recouvrement fini.
    \item[Compact implique fermé]
        Cela est la proposition~\ref{PropUCUknHx}.
    \end{subproof}
    Sens réciproque.
    \begin{subproof}
    \item[Un intervalle fermé et borné est compact dans \( \eR\)]
        C'est le lemme~\ref{LemOACGWxV}.
    \item[Un produit de segments est compact]
        Le produit de deux compacts de \( \eR\) est un compact dans \( \eR^2\) par le lemme~\ref{LemCKBooXkwkte}.
    \item[Un fermé et borné est compact]
        Soit \( K\) fermé et borné. Vu que \( K\) est borné, il est contenu dans un produit de segments. L'ensemble \( K\) est donc compact parce que fermé dans un compact, lemme~\ref{LemnAeACf}.
    \end{subproof}
\end{proof}

\begin{example}[Compacité de la boule unité]
    La boule unité fermée \( \overline{ B(0,1) }\) d'un espace vectoriel normé de dimension finie est compacte parce que fermée et bornée. En dimension infinie, cela n'est plus le cas. Certes la boule unité est encore fermée et bornée, mais elle n'est plus compacte. En effet nous allons donner un recouvrement par des ouverts duquel il ne sera pas possible d'extraire un sous-recouvrement fini.

    Autour de chacune des extrémités des vecteurs de base, nous considérons la boule \( A_i=B(e_i,\frac{1}{ 3 })\). Ensuite aussi l'ouvert
    \begin{equation}
        B(0,1)\setminus\bigcup_i\overline{ B(e_i,\frac{1}{ 4 })}.
    \end{equation}
    Le tout recouvre \( B(0,1)\) mais toutes les premières boules sont nécessaires.
\end{example}
\index{compact!boule unité}

Le théorème de Bolzano-Weierstrass \ref{THOooRDYOooJHLfGq} nous permettra de prouver plus simplement la non compacité en dimension infinie. Voir l'exemple~\ref{ExEFYooTILPDk}.


%---------------------------------------------------------------------------------------------------------------------------
\subsection{Suites et limites dans les réels}
%---------------------------------------------------------------------------------------------------------------------------

\subsubsection{Limites, convergence}
%////////////////////////////////

Dans le cas de suites réelles, nous avons la caractérisation suivante qui est souvent donnée comme une définition lorsque seule la topologie sur \( \eR\) est considérée.
\begin{proposition}[Limite d'une suite numérique]	\label{PropLimiteSuiteNum}
	La suite $(x_n)$ est convergente si et seulement s'il existe un réel $\ell$ tel que
	\begin{equation}		\label{EqDefLimSuite}
		\forall \epsilon>0,\,\exists N\in\eN\tq\forall n\geq N,\,| x_n-\ell |<\epsilon.
	\end{equation}
	Dans ce cas, le nombre $\ell$ est la limite de la suite $(x_n)$.
\end{proposition}
\index{convergence!suite numérique}
\index{limite!suite numérique}

\begin{propositionDef}		\label{PROPooOSXCooJWXkWH}
    Une suite $(x_n)$ dans un espace vectoriel normé $V$ est convergente\footnote{Définition \ref{DefXSnbhZX}.} si et seulement si il existe un élément $\ell\in V$ tel que
	\begin{equation}
		\forall \varepsilon>0,\,\exists N\in\eN\tq n\geq N\Rightarrow \| x_n-l \|<\varepsilon.
	\end{equation}
	Dans ce cas, $\ell$ est la limite de la suite $(x_n)$.
\end{propositionDef}
    \index{convergence!dans un espace vectoriel normé}

\begin{proof}
    En deux parties.
    \begin{subproof}
        \item[Sens direct]
            Si \( x_n\to \ell\) et si \( \epsilon>0\) il existe \( N_{\epsilon}\) tel que pour tout \( n\geq N\) nous avons \( x_n\in B(\ell,\epsilon)\) (parce que cette boule est un ouvert contenant \( \ell\)). Vue la définition d'une boule, cette condition s'écrit bien \( \| x_n-\ell \|<\epsilon\).

        \item[Sens inverse]

            Dans l'autre sens, soit \( \mO\) un ouvert contenant \( \ell\). Par définition de la topologie, il existe \( \epsilon>0\) tel que \( B(\ell,\epsilon)\subset \mO\). La condition \eqref{EqDefLimSuite} nous assure qu'il existe \( N_{\epsilon} \) tel que pour tout \( n\geq N_{\epsilon}\) nous ayons
            \begin{equation}
             x_n\in B(\ell,\epsilon)\subset\mO,
             \end{equation}
            ce qui assure que la suite \( (x_n)\) converge vers \( \ell\) pour la topologie métrique de \( V\).
    \end{subproof}
\end{proof}

Une façon équivalente d'exprimer le critère \eqref{EqDefLimSuite} est de dire que pour tout $\epsilon$ positif, il existe un rang $N\in\eR$ tel que l'intervalle $\mathopen[ \ell-\epsilon , \ell+\epsilon \mathclose]$ contient tous les termes $x_n$ au-delà de $N$.

Il est à noter que le rang $N$ dont il est question dans la définition de suite convergente dépend de~$\epsilon$.

\begin{definition}      \label{DEFooHNCTooMlQUvx}
    Nous disons qu'une suite réelle $(x_n)$ converge\footnote{Voir la définition~\ref{PropLimiteSuiteNum} pour plus de détail.} vers $\ell$ lorsque pour tout $\varepsilon$, il existe un $N$ tel que
    \begin{equation}
        n>N\Rightarrow | x_n-\ell |\leq\varepsilon.
    \end{equation}
\end{definition}

Le concept fondamental de cette définition est la notion de valeur absolue qui permet de donner la «distance» entre deux réels. Dans un espace vectoriel normé quelconque, cette notion est généralisée par la distance associée à la norme (définition~\ref{DefNorme}). Nous pouvons donc facilement définir le concept de convergence d'une suite dans un espace vectoriel normé.

%--------------------------------------------------------------------------------------------------------------------------- 
\subsection{Opérations sur les limites}
%---------------------------------------------------------------------------------------------------------------------------

\begin{proposition}[\cite{MonCerveau}]     \label{PROPooIQOAooJPMoDD}
    Soient des suites à valeurs réelles \( (a_i)\) et \( (b_j)\) si elles sont convergentes, alors la suite \( ab\) est convergente et
    \begin{equation}
        \big( \lim_ia_i \big)\big( \lim_jb_j \big)=\lim_i(a_ib_i).
    \end{equation}
\end{proposition}

\begin{proof}
    Nous nommons \( a\) et \( b\) les limites des suites \( (a_i)\) et \( (b_j)\). Soit \( \epsilon>0\) ainsi que \( i\in \eN\). Nous avons la majoration
    \begin{subequations}
        \begin{align}
            | a_ib_i-ab |&\leq | a_ib_i-a_ib |+| a_ib-ab |\\
            &\leq | a_i | |b_i-b |+b| a_i-a |.
        \end{align}
    \end{subequations}
    Vu que la suite \( (a_i)\) est convergente, elle est bornée. Nous pouvons donc majorer \( | a_i |\) par \( R>0\) qui ne dépend pas de \( i\). Soit \( \eta>0\) tel que \( (R+b)\eta<\epsilon\). Alors en prenant \( i\) assez grand pour que \( | b_i-b |<\eta\) et \( | a_i-a |<\eta\), nous avons bien
    \begin{equation}
        | a_ib_i-ab |\leq (R+b)\eta<\epsilon.
    \end{equation}
\end{proof}

\begin{proposition}     \label{PROPooICZMooGfLdPc}
    Soient des suites \( (x_n)\) et \( (y_n)\) dans un espace vectoriel normé \( V\). Si \( x_n\stackrel{V}{\longrightarrow}x\) et \( y_n\stackrel{V}{\longrightarrow}y\), alors
    \begin{equation}
        x_n+y_n\stackrel{V}{\longrightarrow}x+y.
    \end{equation}
\end{proposition}

\begin{proof}
    Soit \( \epsilon>0\). Nous considérons \( N\) tel que si \( n\geq N\), alors \( \| x_n-x \|\leq \epsilon\) et \( \| y_n-y \|\leq \epsilon\). En utilisant l'inégalité \ref{DefNorme}\ref{ItemDefNormeiii},
    \begin{equation}
        \| x_y+y_n-(x+y) \|\leq \| x_n-x \|+\| y_n-y \|\leq 2\epsilon.
    \end{equation}
    Donc la suite \( (x_n+y_n)\) converge vers \( x+y\).
\end{proof}

 
\subsection{Exemples}
%//////////////////////////

\begin{example}
	Quelques suites usuelles.
	\begin{enumerate}
		\item
			La suite $x_n=\frac{1}{ n }$ converge vers $0$.
		\item
			La suite $x_n=(-1)^n$ ne converge pas.
	\end{enumerate}
\end{example}

Deux limites pour voir comment ça fonctionne.
\begin{lemma}
    Si \( r>1\) nous avons :
    \begin{enumerate}
        \item
            \( \lim_{n\to \infty} r^n=\infty\).
        \item
            \( \lim_{n\to \infty} \frac{ r^n }{ n }=\infty\).
    \end{enumerate}
\end{lemma}

\begin{proof}
    Vu que \( r>1\) nous pouvons écrire \( r=1+\delta\) avec \( \delta>0\). La formule du binôme de Newton \eqref{EqNewtonB} nous donne
    \begin{equation}
        (1+\delta)^n=\sum_{k=0}^n{k\choose n}\delta^k>{1\choose n}\delta=n\delta.
    \end{equation}
    La proposition \ref{ThoooKJTTooCaxEny} (\( \eR\) est archimédien) nous indique que \( n\delta\) est arbitrairement grand lorsque \( n\) est grand, quelle que soit \( \delta>0\). Cela finit la preuve de la première limite.

    Pour la seconde, nous posons \( a_n=\frac{ r^n }{ n }\). Nous avons
    \begin{equation}
        \frac{ a_{n+1} }{ a_n }=\frac{ n }{ n+1 }r.
    \end{equation}
    Vu que \( \frac{ n }{ n+1 }\to 1\), la suite \( \frac{ n }{ n+1 }r\) tend vers \( r>0\), et en particulier pour tout \( \delta>0\) tel que \( r>1+\delta\), il existe \( N\in \eN\) tel que, pour tout \( n > N \),
    \begin{equation}
        \frac{ n }{ n+1 }r>1+\delta.
    \end{equation}
    Soit maintenant \( k\in \eN\). En utilisant un produit télescopique,
    \begin{equation}
        a_{N+k}=a_N\frac{ a_{N+1} }{ a_N }\frac{ a_{N+2} }{ a_{N+1} }\cdots\frac{ a_{N+k} }{ a_{N+k-1} }>a_N(1+\delta)^{k-1}.
    \end{equation}
    Or \( (1+\delta)^{k-1}\) tend vers \( \infty\) lorsque \( k\to \infty\) par le premier point. Donc nous avons \( \lim_{n\to \infty} r^n/n=\infty\).
\end{proof}


\begin{definition}      \label{DEFooEWRTooKgShmT}
    Nous disons que deux suites \( (u_n)\) et \( (v_n)\) sont \defe{équivalentes}{equivalence@équivalence!de suites} s'il existe une fonction \( \alpha\colon \eN\to \eR\) telle que
    \begin{enumerate}
        \item
            pour tout \( n\) à partir d'un certain rang, \( u_n=v_n\alpha(n)\)
        \item
            \( \alpha(n)\to 1\).
    \end{enumerate}
\end{definition}

%--------------------------------------------------------------------------------------------------------------------------- 
\subsection{Suites croissantes et bornées}
%---------------------------------------------------------------------------------------------------------------------------

Une suite est dite \defe{contenue}{} dans un ensemble $A$ si $x_n\in A$ pour tout $n$. Une suite est \defe{bornée supérieurement}{bornée!suite} s'il existe un $M$ tel que $x_n\leq M$ pour tout $n$. De la même manière, la suite est bornée inférieurement s'il existe un $m$ tel que $x_n\geq m$ pour tout $n$.

Le lemme suivant est souvent utilisé pour prouver qu'une suite est convergente.
\begin{lemma}		\label{LemSuiteCrBorncv}
	Une suite croissante et bornée supérieurement converge. Une suite décroissante bornée inférieurement est convergente.
\end{lemma}

Une erreur courante est de croire que la borne est la limite : le lemme n'affirme pas ça. Par contre il est vrai que la borne donne \ldots hum \ldots une borne inférieure (ou supérieure) pour la limite.

\begin{theorem}[Bolzano-Weierstrass, thème \ref{THEMEooQQBHooLcqoKB}]     \label{THOooRDYOooJHLfGq}
    Toute suite contenue dans un compact admet une sous-suite convergente.
\end{theorem}

\begin{proof}
    Nous faisons la preuve par l'absurde en supposant que \( (x_k)\) n'admette pas de sous-suite convergente. Soit \( a\in K\); aucune sous-suite de \( (x_k)\) ne converge vers \( a\). En particulier, il existe un voisinage ouvert \( \mO_a\) de \( a\) et une partie finie \( I_a\) de \( \eN\) tel que \( x_k\in \mO_a\) seulement pour \( k\in I_a\).

    Les ouverts \( \mO_a\) recouvrent \( K\); nous pouvons en extraire un sous-recouvrement fini (c'est la définition \ref{DefJJVsEqs} de la compactié). Nous avons donc des points \( a_1,\ldots, a_n\) tels que 
    \begin{equation}
        K\subset \bigcup_{i=1}^n\mO_{a_i}
    \end{equation}
    et tels que pour chaque \( \mO_{a_i}\), nous avons \( x_k\in \mO_{a_i}\) seulement pour \( k\in I_{a_i}\). Bien entendu, toute la suite est dans \( K\) et donc dans l'union.

    En conclusion, nous avons \( \eN=\bigcup_{i=1}^n I_{a_i}\), ce qui prouve que \( \eN\) est un ensemble fini. Contradiction avec la proposition \ref{PROPooBYKCooGDkfWy} qui dit que \( \eN\) est infini.
\end{proof}

% Inutile de replacer cette proposition plus loin : on en a besoin pour démontrer Weierstrass. Quitte à maintenir, il faut réénoncer pour un espace vectoriel normé et prouver.
\begin{proposition}		\label{PropCvRpComposante}
	Une suite $(x_n)$ dans $\eR^m$ est convergente dans $\eR^m$ si et seulement si les suites de chaque composante sont convergentes dans $\eR$. Dans ce cas nous avons
	 \begin{equation}
		 \lim x_n=\Big( \lim(x_n)_1,\lim (x_n)_2,\ldots,\lim (x_n)_m \Big)
	 \end{equation}
	 où $(x_n)_k$ dénote la $k$-ième composante de $(x_n)$.
\end{proposition}

\begin{example}
	La suite $x_n=\big( \frac{1}{ n },1-\frac{1}{ n } \big)$ converge vers $(0,1)$ dans $\eR^2$. En effet, en utilisant la proposition~\ref{PropCvRpComposante}, nous devons calculer séparément les limites
	\begin{equation}
		\begin{aligned}[]
			\lim\frac{1}{ n }&=0\\
			\lim\big( 1-\frac{1}{ n } \big)&=1.
		\end{aligned}
	\end{equation}
\end{example}

\begin{example}
	Étant donné que la suite $(-1)^n$ n'est pas convergente, la suite $x_n=\big( (-1)^n,\frac{1}{ n } \big)$ n'est pas convergente dans $\eR^2$.
\end{example}

%--------------------------------------------------------------------------------------------------------------------------- 
\subsection{Suites adjacentes}
%---------------------------------------------------------------------------------------------------------------------------

\begin{definition}[\cite{ooZZNWooSIipwW}]       \label{DEFooDMZLooDtNPmu}
    Les suites \( (a_n)\) et \( (b_n)\) sont \defe{adjacentes}{suites adjacentes} si l'une est croissante, l'autre décroissante et si \( a_n-b_n\to 0\).
\end{definition}

\begin{theorem}[Théorème des suites adjaentes]      \label{THOooZJWLooAtGMxD}
    Nous considérons des suites adjacentes \( (a_n)\) et \( (b_n)\) avec \( (a_n)\) croissante et \( (b_n)\) décroissante. Alors
    \begin{enumerate}
        \item
            \( b_n\geq a_n\) pour tout \( n\),
        \item
            \( a_n\leq b_q\) pour tout \( n\) et \( q\). C'est-à-dire que toute la suite \( a\) est plus petite que toute la suite \( b\).
        \item
            les suites \( a\) et \( b\) sont convergentes,
        \item
            les suites \( a\) et \( b\) convergent vers la même limite, notée \( \ell\),
        \item
            nous avons \( a_n\leq \ell\leq b_n\) pour tout \( n\).
    \end{enumerate}
\end{theorem}

\begin{proof}
    La suite \( n\mapsto b_n-a_n\) est décroissante parce que \( b_n-a_n\geq b_{n+1}-a_{n+1}\). Comme en plus \( b_a-a_n\to 0\) nous avons
    \begin{equation}
        b_n-a_n\geq 0
    \end{equation}
    pour tout \( n\in \eN\). De plus \( a_n\leq b_0\) pour tout \( n\) parce que si \( a_N>b_0\) alors, \( b\) étant décroissante, \( a_N>b_0\geq b_N\) qui est contraire à ce que nous venons de prouver. La suite \( a\) étant croissante et majorée, elle est convergente\footnote{Proposition \ref{LemSuiteCrBorncv}.}; notons \( \ell\) sa limite.

    La suite \( b\) peut maintenant être écrite par
    \begin{equation}
        b_n=(b_n-a_n)+a_n
    \end{equation}
    qui est une somme de deux suites convergentes. Elle est donc convergente et sa limite est la somme des limites\footnote{Proposition \ref{PROPooICZMooGfLdPc}.}, donc
    \begin{equation}
        \lim_{n\to \infty} b_n=\lim_{n\to \infty} (b_n-a_n)+a_n=0+\ell=\ell.
    \end{equation}
    Voila. Donc les suites \( a\) et \( b\) convergent et ont la même limite.

    Pour tout \( n,q\in \eN\) nous avons l'inégalité \( a_n\leq b_q\). En prenant la limite \( n\to \infty\) nous trouvons
    \begin{equation}
        \ell\leq b_q
    \end{equation}
    pour tout \( q\). Et de la même façon, \( b_n\geq a_q\) donne \( \ell\geq a_q\). L'un avec l'autre donne
    \begin{equation}
        a_q\leq \ell\leq b_q
    \end{equation}
    pour tout \( q\in \eN\).
\end{proof}

\begin{proposition}[\cite{ooXFPIooCLUvzV}]      \label{PROPooXOOCooGMqJNe}
    Soit une suite \( (a_n)\) dans \( \eR\).  Nous supposons que les suites extraites \( (a_{2n})\) et \( (a_{2n+1})\) convergent vers la même limite notée \( \ell\).

    Alors \( a_n\to \ell\).
\end{proposition}

\begin{proof}
    Soit \( \epsilon>0\). Il existe \( N_1\) tel que \( | a_{2n}-\ell |\leq \epsilon\) dès que \( n\geq N_1\). Il existe également \( N_2\) dès que \( | a_{2n+1}-\ell |\leq \epsilon\) dès que \( n\geq N_2\).

    Nous posons \( N=\max\{ 2N_1,2N_2+2 \}\) et nous avons, pour tout \( n\geq N\) :
    \begin{equation}
        | a_n-\ell |\leq \epsilon,
    \end{equation}
    c'est-à-dire que \( a\to \ell\).
\end{proof}

%---------------------------------------------------------------------------------------------------------------------------
\subsection{Limite supérieure et inférieure}
%---------------------------------------------------------------------------------------------------------------------------

\begin{lemmaDef}      \label{ooMVZAooVVCOnP}
    Soit \( (a_n)\) une suite dans \( \bar \eR\). Les limites suivantes existent dans \( \bar \eR\)
    \begin{equation}
        \limsup_{n\to\infty}a_n=\lim_{n\to \infty}\big( \sup_{k\geq n}a_k \big)
    \end{equation}
    et
    \begin{equation}
        \liminf_{n\to \infty}a_n=\lim_{n\to\infty}\big( \inf_{k\geq n}a_k \big).
    \end{equation}
    Elles sont nommées \defe{limite supérieure}{limite!supérieure} et la \defe{limite inférieure}{limite!inférieure} de la suite \( (a_k)\).
\end{lemmaDef}
\nomenclature[Y]{\( \limsup a_n\)}{limite supérieure}
\nomenclature[Y]{\( \liminf a_n\)}{limite inférieure}

\begin{proof}
    Pour la limite supérieure, l'ensemble des \( k\geq n\) est de plus en plus petit lorsque \( n\) grandit. Donc les ensembles \( A_n=\{ a_k\tq k\geq n \}\) sont emboîtés et la suite \( n\to \sup A_n\) est une suite décroissante. Elle a donc une limite dans \( \bar \eR\).
\end{proof}

\begin{normaltext}      \label{ooEEQJooRMFzVR}
    En ce qui concerne les suites d'ensembles, utiles en théorie des probabilités, nous définissons de même. Si les \( A_n\) sont des parties de \( \Omega\), nous définissons la \defe{limite supérieure}{limite!supérieure} et la \defe{limite inférieure}{limite!inférieure} de la suite \( A_n\) par
\begin{equation}
    \limsup_{n\to\infty}A_n=\bigcap_{n\geq 1}\bigcup_{k\geq n}A_k
\end{equation}
et
\begin{equation}
    \liminf_{n\to\infty}A_n=\bigcup_{n\geq 1}\bigcap_{k\geq n}A_k
\end{equation}

Nous avons
\begin{equation}
    \limsup A_n=\{ \omega\in\Omega\tq \omega\in A_n\text{pour une infinité de } n \}.
\end{equation}
\end{normaltext}

\begin{lemma}     \label{ooAQTEooYDBovS}
    Nous avons les formules pratiques suivantes :
    \begin{subequations}
        \begin{align}
            \limsup a_n&=\inf_{n\geq 1}\big( \sup_{k\geq n}a_k \big)\\
            \liminf a_n&=\sup_{n\geq 1}\big( \inf_{k\geq n}a_k \big).
        \end{align}
    \end{subequations}
\end{lemma}

\begin{proof}
    La suite \( n\mapsto \sup_{k\geq n}a_k\) est une suite décroissante, donc la limite est l'infimum. Même argument pour l'autre.
\end{proof}

\begin{lemma}       \label{ooIQIKooXWwAmM}
    La suite \( (a_n)\) dans \( \eR\) converge si et seulement si
    \begin{equation}
        \limsup a_n=\liminf a_n.
    \end{equation}
    Dans ce cas, \( \lim a_n=\limsup a_n=\liminf a_n\).
\end{lemma}

\begin{proof}
    Nous commençons par supposer que \( \limsup a_n=\liminf a_n=l\), et nous prouvons que \( \lim a_n\) existe et vaut \( l\). Soit \( \epsilon>0\). Il existe \( N\) tel que si \( n\geq N\) nous avons
    \begin{equation}
        \big| \sup_{k\geq n}a_k-l \big|<\epsilon
    \end{equation}
    et
    \begin{equation}
        \big| \inf_{k\geq n}a_k-l \big|<\epsilon.
    \end{equation}
    Pour tout \( k\geq N\) nous avons alors \( a_k\leq l+\epsilon\) et \( a_k\geq l-\epsilon\). Cela donne \( a_n\in B(l,\epsilon)\), c'est-à-dire \( a_k\to l\) par la proposition~\ref{PropLimiteSuiteNum}.

    Dans l'autre sens, nous supposons que \( \lim_n a_n=l\) et nous prouvons que les limites supérieures et inférieures sont toutes deux égales à \( l\). Soit \( \epsilon>0\) et \( N_{\epsilon}\) tel que \( | a_n-l |<\epsilon\) pour tout \( n\geq N_{\epsilon}\). Si \( n\geq N_{\epsilon}\) nous avons
    \begin{equation}
        \big| \sup_{k\geq n}a_k-l \big|\leq \epsilon
    \end{equation}
    et donc la limite de \( \sup_{k\geq n}a_k\) lorsque \( n\to \infty\) est bien \(l\).
\end{proof}

%---------------------------------------------------------------------------------------------------------------------------
\subsection{Ouverts, voisinage, topologie}
%---------------------------------------------------------------------------------------------------------------------------

Lorsque $x\in E$, nous rappelons qu'un voisinage\footnote{Définition~\ref{DefFermeVoisinage}.} de $x$ est n'importe quel sous-ensemble de $E$ qui contient une boule ouverte centrée en $x$. La proposition \ref{ThoPartieOUvpartouv} nous dit qu'un ensemble est ouvert s'il contient un voisinage de chacun de ses points. Au passage, rappelons que l'ensemble vide est ouvert.

Pour rappel, la remarque \ref{RemQDRooKnwKk}\ref{ITEMooUIHJooXAFaIz} dit que l'ensemble des boules ouvertes d'un espace métrique génère la topologie de l'espace.

Nous rappelons qu'une partie $A$ d'un espace métrique est dite bornée\footnote{Définition~\ref{DefEnsembleBorne}.} s'il existe une boule\footnote{À titre d'exercice, convainquez-vous que l'on peut dire boule \emph{ouverte} ou \emph{fermée} au choix sans changer la définition.} qui contient $A$.

Mais revenons à \( \eR \)\dots
\begin{lemma}  \label{LemSupOuvPas}
    Une partie ouverte de \( \eR\) ne contient pas son supremum.
\end{lemma}

\begin{proof}
Soit $\mO$, un ensemble ouvert et $s$, son supremum. Si $s$ était dans $\mO$, on aurait un voisinage $B=B(s,r)$ de $s$ contenu dans $\mO$. Le point $s+r/2$ est alors à la fois dans $\mO$ et plus grand que $s$, ce qui contredit le fait que $s$ soit un supremum de $\mO$.
\end{proof}

Par le même genre de raisonnements, on montre que l'union et l'intersection de deux ouverts sont encore des ouverts.

\begin{remark}
L'intersection d'une \emph{infinité} d'ouverts n'est pas spécialement un ouvert comme le montre l'exemple suivant :
\[
  \mO_i=]1,2+\frac{ 1 }{ i }[.
\]
Tous les ensembles $\mO_i$ contiennent le point $2$ qui est donc dans l'intersection. Mais quel que soit le $\epsilon>0$ que l'on choisisse, le point $2+\epsilon$ n'est pas dans $\mO_{(1/\epsilon)+1}$. Donc aucun point au-delà de $2$ n'est dans l'intersection, ce qui prouve que $2$ ne possède pas de voisinages contenus dans $\bigcap_{i=1}^{\infty}\mO_i$.
\end{remark}

\begin{proposition}     \label{PROPooANIOooIJHelX}
Quels que soient les ensembles $A$ et $B$ dans $\eR$, nous avons
\[
  \sup(A\cap B)\leq\sup A\leq\sup(A\cup B).
\]
\end{proposition}
Nous laissons le lecteur le prouver, même si ce n'est pas dans notre habitude.

%---------------------------------------------------------------------------------------------------------------------------
\subsection{Intervalles et connexité}
%---------------------------------------------------------------------------------------------------------------------------

Nous allons déterminer tous les sous-ensembles connexes\footnote{Définition~\ref{DefIRKNooJJlmiD}.} de $\eR$. Pour cela nous relisons d'abord la notion d'intervalle donnée en~\ref{DefEYAooMYYTz} ainsi que la proposition \ref{PROPooHPMWooQJXCAS} qui liste tous les intervalles de \( \eR\). La partie \( I\subset \eR\) est un intervalle si pour tout \( a,b\in I\), tout nombre entre \( a\) et \( b\) est également dans \( I\). Cette définition englobe tous les exemples connus d'intervalles ouverts, fermés avec ou sans infini : $[a,b]$, $[a,b[$, $]-\infty,a]$, \ldots L'ensemble \( \eR\) lui-même est un intervalle.

Si \( I\) est un intervalle, les nombres \( \inf(I)\) et \( \sup(I)\)\footnote{Qui existent par la proposition~\ref{DefSupeA}, quitte à poser \( \pm\infty\) comme infimum et supremum lorsque \( I\) n'est pas borné.} sont les \defe{extrémités}{extrémité!d'un intervalle} de \( I\).

\begin{definition}      \label{DefLISOooDHLQrl}
	Étant donnés deux points $a$ et $b$ dans $\eR^p$ on appelle \defe{segment}{segment!dans $\eR^p$} d'extrémités $a$ et $b$, et on note $[a,b]$, l'image de $[0,1]$ par l'application $s: [0,1]\to \eR^p$, $s(t)= (1-t)a+tb$.  On pose $]a,b[=s\left(]0,1[\right)$, et  $]a,b]=s\left(]0,1]\right)$.
\end{definition}
Il faut observer que le segment $[a,b]$ est une courbe orientée : certes en tant que ensembles, $[a,b]=[b,a]$, mais si nous regardons la fonction de $t$ correspondante à $[b,a]$, nous voyons qu'elle va dans le sens inverse de celle qui correspond à $[a,b]$. Nous approfondirons ces questions lorsque nous parlerons d'arcs paramétrés autour de la section~\ref{SecArcGeometrique}.

Le segment $[b,a]$ est l'image de l'application $r\colon [0,1]\to \eR^p$ donnée par $r(t)=(1-t)b+ta$.

\begin{proposition} \label{PropInterssiConn}
    Une partie de $\eR$ est connexe si et seulement si c'est un intervalle.
\end{proposition}
\index{connexité!et intervalles}

\begin{proof}
    La preuve est en deux parties. D'abord nous démontrons que si un sous-ensemble de $\eR$ est connexe, alors c'est un intervalle; et ensuite nous démontrons que tout intervalle est connexe.

    Afin de prouver qu'un ensemble connexe est toujours un intervalle, nous allons prouver que si un ensemble n'est pas un intervalle, alors il n'est pas connexe. Prenons $A$, une partie de $\eR$ qui n'est pas un intervalle. Il existe donc $a$, $b\in A$ et un $x_0$ entre $a$ et $b$ qui n'est pas dans $A$. Comme le but est de prouver que $A$ n'est pas connexe, il faut couper $A$ en deux ouverts disjoints. L'élément $x_0$ qui n'est pas dans $A$ est le bon candidat pour effectuer cette coupure. Prenons $M$, un majorant de $A$ et $m$, un minorant de $A$, et définissons
    \begin{align*}
        \mO_1&=]m,x_0[\\
        \mO_2&=]x_0,M[.
    \end{align*}
    Si $A$ n'a pas de minorant, nous remplaçons la définition de $\mO_1$ par $]-\infty,x_0[$, et si $A$ n'a pas de majorant, nous remplaçons la définition de $\mO_2$ par $]x_0,\infty[$. Dans tous les cas, ce sont deux ensembles ouverts dont l'union recouvre tout $A$. En effet, $\mO_1\cup \mO_2$ contient tous les nombres entre un minorant de $A$ et un majorant sauf $x_0$, mais on sait que $x_0$ n'est pas dans $A$. Cela prouve que $A$ n'est pas connexe.

    Jusqu'à présent nous avons prouvé que si un ensemble n'est pas un intervalle, alors il ne peut pas être connexe. Pour remettre les choses à l'endroit, prenons un ensemble connexe, et demandons-nous s'il peut être autre chose qu'un intervalle ? La réponse est \emph{non} parce que s'il était autre chose, il ne serait pas connexe.

    Prouvons à présent que tout intervalle est connexe. Pour cela, nous refaisons le coup de \href{http://fr.wikipedia.org/wiki/Contraposée}{la contraposée}. Nous allons donc prendre une partie $A$ de $\eR$, supposer qu'elle n'est pas connexe et puis prouver qu'elle n'est alors pas un intervalle. Nous avons deux ouverts disjoints $\mO_1$ et $\mO_2$ tels que $A\subset \mO_1\cup \mO_2$. Notons \( A_1 = A \cap \mO_1 \) et  \( A_2 = A \cap \mO_2 \); et prenons $a\in A_1$ et $b\in A_2$. Pour fixer les idées, on suppose que $a<b$. Maintenant, le jeu est de montrer qu'il existe une point $x_0$ entre $a$ et $b$ qui ne soit pas dans $A$ (cela montrerait que $A$ n'est pas un intervalle). Nous allons prouver que c'est le cas du point
    \[
      x_0=\sup\{ x\in\mO_1\tq x<b \}.
    \]
    Étant donné que l'ensemble $\mA=\{ x\in\mO_1\tq x<b \}$ est ouvert\footnote{C'est l'intersection entre l'ouvert $\mO_1$ et l'ouvert $\{x\tq x<b \}$.}, le point $x_0$ n'est pas dans l'ensemble par le lemme~\ref{LemSupOuvPas}. Nous avons donc
    \begin{itemize}
        \item soit $x_0$ n'est pas dans $\mO_1$,
        \item soit $x_0\leq b$,
        \item soit les deux en même temps.
    \end{itemize}
    Nous allons montrer qu'un tel $x_0$ ne peut pas être dans $A$. D'abord, remarquons que $\sup\mA\leq\sup\mO$ parce que $\mA$ est une intersection de $\mO$ avec quelque chose. Ensuite, il n'est pas possible que $x_0$ soit dans $\mO_2$ parce que tout élément de $\mO_2$ possède un voisinage contenu dans $\mO_2$. Un point de $\mO_2$ est donc toujours strictement plus grand que le supremum de $\mO_1$.

    Maintenant, remarque que si $x_0\leq b$, alors $x_0=b$, sinon $b$ serait un majorant de $\mA$ plus petit que $x_0$, ce qui n'est pas possible vu que $x_0$ est le supremum de $\mA$ et donc le plus petit majorant. Oui mais si $x_0=b$, c'est que $x_0\in\mO_2$, ce qu'on vient de montrer être impossible. Nous voilà déjà débarrassé des deuxièmes et troisièmes possibilités.

    Si la première possibilité est vraie, alors $x_0$ n'est pas dans $A$ parce qu'on a aussi prouvé que $x_0\notin\mO_2$. Or n'être ni dans $\mO_1$ ni dans $\mO_2$ implique de ne pas être dans $A$. Ce point $x_0=\sup\mA$ est donc hors de $A$.

    Oui, mais comme $a\in\mA$, on a obligatoirement que $x_0\geq a$. Mais par construction, on a aussi que $x_0\leq b$ (ici, l'inégalité est même stricte, mais ce n'est pas important). Donc
    \[
      a\leq x_0\leq b
    \]
    avec $a$, $b\in A$, et $x_0\notin A$. Cela finit de prouver que $A$ n'est pas un intervalle.
\end{proof}

\begin{theorem}[Théorème des bornes atteintes]\label{ThoMKKooAbHaro}
    Une fonction à valeurs réelles continue sur un compact est bornée et atteint ses bornes.

	C'est-à-dire qu'il existe $x_0\in K$ tel que $f(x_0)=\inf\{ f(x)\tq x\in K \}$ ainsi que $x_1$ tel que $f(x_1)=\sup\{ f(x)\tq x\in K \}$.
\end{theorem}
\index{compact!et fonction continue}

\begin{proof}
    Soient un espace topologique compact \( K\) et une fonction continue \( f\colon K\to \eR\). Alors le théorème~\ref{ThoImCompCotComp} indique que \( f(K)\) est compact. Par conséquent \( f(K)\) est un fermé borné de \( \eR\) par le théorème de Borel-Lebesgue~\ref{ThoXTEooxFmdI}. Vu que \( f(K)\) est borné, la fonction \( f\) est bornée.

    De plus \( f(K)\) étant fermé, son infimum est un minimum et son supremum est un maximum : il existe \( x\in K\) tel que \( f(x)=\sup f(K)\) et il existe \( y\in K\) tel que \( f(y)=\inf f(K)\).
\end{proof}

Le théorème suivant est essentiellement inutile pour les raisons suivantes :
\begin{itemize}
    \item 
        Il est un cas particulier du théorème~\ref{ThoBWFTXAZNH} qui donne pour tout espace métrique, l'équivalence entre la compacité et la compacité séquentielle.
    \item
        Il est un cas particulier du théorème \ref{THOooRDYOooJHLfGq} qui le donne pour tous les espaces compacts.
    \item
        Il utilise le cas particulier de \( \eR\), qui n'est pas démontré directement dans le Frido.
\end{itemize}
Bref, nous ne le laissons que pour le lecteur qui n'aurait pas en tête d'autres définitions de «compact» à part «fermé borné».

% Pour les raisons invoquées, il ne fait pas faire de références vers ce théorème. Le label ici ne sert qu'à le mettre dans l'index thématique.
\begin{theorem}[Théorème de Bolzano-Weierstrass]		\label{ThoBolzanoWeierstrassRn}
	Toute suite contenue dans un compact de \( \eR^m\) admet une sous-suite convergente.
\end{theorem}

\begin{proof}
    Nous rappelons qu'une partie compacte de \( \eR^n\) est fermée et bornée par le théorème de Borel-Lebesgue~\ref{ThoXTEooxFmdI}.

    Soit $(x_n)$ une suite contenue dans une partie bornée de $\eR^m$. Considérons $(a_n)$, la suite réelle des premières composantes des éléments de $(x_n)$ : pour chaque $n\in\eN$, le nombre $a_n$ est la première composante de $x_n$. Étant donné que la suite $(x_n)$ est bornée, il existe un $M$ tel que $\| x_n \|<M$. La croissance de la fonction racine carrée donne
	\begin{equation}
        | a_n |\leq\| x_n \|\leq M.
	\end{equation}
    La suite $(a_n)$ est donc une suite réelle bornée et donc contient une sous-suite convergente par le théorème correspondant dans \( \eR\) :  \ref{ThoBWFTXAZNH}. Soit $a_{I_1}$ une sous-suite convergente de $(a_n)$. Nous considérons maintenant $x_{I_1}$, c'est-à-dire la suite de départ dont on a enlevé tous les éléments qu'il faut pour qu'elle converge en ce qui concerne la première composante.

	Si nous considérons la suite $b_{I_1}$ des \emph{secondes} composantes de $x_{I_1}$, nous en extrayons, de la même façon que précédemment, une sous-suite convergente, c'est-à-dire que nous avons un $I_2\subset I_1$ tel que $b_{I_2}$ est convergent. Notons que $a_{I_2}$ est une sous-suite de la (sous) suite convergente $x_{I_1}$, et donc $a_{I_2}$ est encore convergente.

	En continuant ainsi, nous construisons une sous-sous-sous-suite $x_{I_3}$ telle que la suite des \emph{troisièmes} composantes est convergente. Lorsque nous avons effectué cette procédure $m$ fois, la suite $x_{I_m}$ est une suite dont toutes les composantes convergent, et donc est une suite convergente par la proposition~\ref{PropCvRpComposante}.

	Le tableau suivant donne un petit schéma de la façon dont nous procédons. Les $\bullet$ sont les éléments de la suite que nous gardons, et les $\times$ sont ceux que nous «jetons».
	\begin{equation}
		\begin{array}{lccccccccccc}
			x_{\eN}	&	\bullet&\bullet&\bullet&\bullet&\bullet&\bullet&\bullet&\bullet&\bullet&\bullet&\ldots\\
			x_{I_1}	&	\times&\bullet&\bullet&\times&\bullet&\times&\times&\bullet&\bullet&\bullet&\ldots\\
			x_{I_2}	&	\times&\bullet&\times&\times&\bullet&\times&\times&\bullet&\bullet&\times&\ldots\\
			\vdots\\
			x_{I_m}	&	\times&\times&\times&\times&\bullet&\times&\times&\times&\bullet&\times&\ldots
		\end{array}
	\end{equation}
	La première ligne, $x_{\eN}$, est la suite de départ.
\end{proof}

\begin{corollary}   \label{CorFHbMqGGyi}
    Si un suite est croissante et bornée alors elle est convergente.
\end{corollary}

\begin{proof}
    Nous nommons \( (x_n)\) la suite et nous prenons un majorant \( M\). Toute la suite est alors contenue dans le compact \( \mathopen[ x_0 , M \mathclose]\), ce qui donne une sous-suite \( (x_{\alpha(n)})\) convergente par le théorème de Bolzano-Weierstrass~\ref{THOooRDYOooJHLfGq}. Si \( \ell\) est la limite de cette sous-suite alors nous avons \( \ell\geq x_n\) pour tout \( n\).

    Pour tout \( \epsilon>0\) il existe \( K\) tel que si \( n>K\) alors \( | \ell-x_{\alpha(n)} |<\epsilon\). Vu que \( \ell\) majore la suite nous avons même
    \begin{equation}
        x_{\alpha(n)}+\epsilon>\ell.
    \end{equation}
    Vu que la suite est croissante pour tout \( m>\alpha(K)\) nous avons \( x_m+\epsilon>l\), ce qui signifie \( | x_m-\ell |<\epsilon\).
\end{proof}
Nous aurons une version pour les fonctions croissantes et bornées en la proposition~\ref{PropMTmBYeU}.

La proposition suivante dit que la notion d'ensemble non dénombrable ne prend pas réellement de force entre \( \eR\) et \( \eR^n\) : il n'y a pas moyen de caser \( \eR\) dans \( \eR^n\) de façon à ce qu'il y tienne à son aise.

\begin{proposition}
    Une partie non dénombrable de \( \eR^n\) possède un point d'accumulation\footnote{Définition \ref{DEFooGHUUooZKTJRi}.}.
\end{proposition}

\begin{proof}
    Soit une partie \( A\subset \eR^n\) sans point d'accumulation. Nous allons prouver que \( A\) est dénombrable.

    Soient les compacts \( K_n=\overline{ B(0,n) }\). La partie \( A\cap K_n\) est finie; sinon elle aurait une partie en bijection avec \( \eN\) (proposition~\ref{PROPooUIPAooCUEFme}) et donc une suite. Or une suite dans un compact possède un point d'accumulation par le théorème~\ref{THOooRDYOooJHLfGq}.

    Donc tous les \( A\cap K_n\) sont finis. Vu que \( A=\bigcup_nA\cap K_n\), l'ensemble \( A\) est une réunion dénombrable d'ensembles finis. Il est donc dénombrable.
\end{proof}

%--------------------------------------------------------------------------------------------------------------------------- 
\subsection{Recouvrement d'un compact par des intervalles ouverts}
%---------------------------------------------------------------------------------------------------------------------------

Soit un ensemble \( E\) et un ensemble \( \mA\) de parties de \( E\). Soit \( A\in \mA\). Nous aimerions savoir quelles sont les éléments de \( \mA\) qui sont atteignables en partant de \( A\) et en ne «sautant» que d'intersection en intersection.

Nous notons \( \mA=\{ B_i \}_{i\in I}\) où \( I\) est un ensemble d'indices (un ensemble quelconque).
\begin{subequations}
    \begin{align}
        s_1(A)&=\{  i\in I\tq B_i\cap A\neq \emptyset   \}\\
        \sigma_1(A)&=\bigcup_{B\in s_1(A)}B.
    \end{align}
\end{subequations}
Et ensuite :
\begin{subequations}
    \begin{align}
        s_{k+1}(A)&=\{ i\in I\tq B_i\cap \sigma_k(A)\neq \emptyset \}\\
        \sigma_{k+1}(A)&=\bigcup_{B\in s_{k+1}(A)}B
    \end{align}
\end{subequations}

\begin{lemma}
    Soient un intervalle \( A\) de \( \eR\) et \( \mA=\{ I_i \}_{i=1,\ldots, N}\) un recouvrement de \( A\) par des intervalles ouverts. Si \( I_1\cap A\neq \emptyset\) alors
    \begin{enumerate}
        \item
            \( \sigma_{N}=\sigma_{N+1}\)
        \item
            \( A\subset \sigma_N(I_1)\).
    \end{enumerate}
\end{lemma}

\begin{proof}
    Si \( \sigma_{k+1}=\sigma_k\), alors tous les \( \sigma_{k+l}\) sont identiques. De plus si \( \sigma_{k+1}\neq \sigma_k\), alors \( \sigma_{k+1}\) contient au moins un élément de plus que \( \sigma_k\). Donc \( \Card(\sigma_k)\geq k\) et en particulier \( N\leq \Card(\sigma_N)\leq N\). Cela prouve le premier point.

    L'ensemble \( \sigma_N(I_1)\) est une union d'ouverts et est donc un ouvert. Quitte à renuméroter nous écrivons
    \begin{equation}
        \sigma_N(I_1)=I_1\cup \ldots \cup I_n.
    \end{equation}
    L'ensemble 
    \begin{equation}
        \tau=\bigcup_{k=n+1}^NI_k
    \end{equation}
    est ouvert et est disjoint de \( \sigma_N(I_1)\) parce que si \( I_l\) ($I_l\geq n+1$) intersectait \( \sigma_N(I_1)\), nous aurions \( l\in s_{N+1}\) ou encore \( I_l\subset \sigma_{N+1}\setminus\sigma_N\).

    Donc \( \tau\) et \( \sigma_N\) sont deux ouverts disjoints qui recouvrent \( A\). Vu que \( A\) est un intervalle, il est connexe\footnote{Définition \ref{DefIRKNooJJlmiD} et proposition \ref{PropInterssiConn}.}. Donc soit \( A\subset \tau\) soit \( A\subset \sigma_N\). Comme \( I_1\cap A\neq \emptyset\) nous sommes dans le cas \( A\subset \sigma_N\).
\end{proof}


%---------------------------------------------------------------------------------------------------------------------------
\subsection{Connexité par arcs}
%---------------------------------------------------------------------------------------------------------------------------

\begin{definition}
  Le sous-ensemble $A \subset \eR^n$ est \defe{connexe par arcs}{connexe!par arc} si pour tout $x, y \in A$, il existe un chemin\footnote{Attention : ici quand on dit \emph{chemin}, on demande que l'application soit continue. Dans de nombreux cours de géométrie différentielle, on demande $ C^{\infty}$. Il faut s'adapter au contexte.} contenu dans $A$ les reliant, c'est-à-dire une application continue
  \begin{equation*}
    \gamma : [0,1] \to \eR^n \tq \gamma(0) = x~\text{et}~\gamma(1) = y
  \end{equation*}
  avec $\gamma(t) \in A$ pour tout $t\in [0,1]$.
\end{definition}

\begin{normaltext}      \label{NORMooQXKVooXOmMlX}
    La connexité d'un ensemble n'implique pas sa connexité par arc. Il suffit pour cela de prendre un ensemble constitué de deux connexes reliés par un chemin de longueur infinie (le graphe d'une fonction de type \( \sin(1/x)\) par exemple).
\end{normaltext}

%TODO : placer ici l'exemple de ce graphe à qui on ajoute le segment vertical de (0,0) à (0,1).

%---------------------------------------------------------------------------------------------------------------------------
\subsection{Topologie de la droite réelle complétée}
%---------------------------------------------------------------------------------------------------------------------------
\label{SUBSECooKRRUooSlZSmM}

Nous introduisons l'ensemble \( \bar\eR=\eR\cup\{ \pm\infty \}\). À présent les symboles \( +\infty\) et \( -\infty\) n'ont aucune signification particulière; il s'agit seulement de deux éléments que nous ajoutons à \( \eR\) pour former un ensemble que nous notons \( \bar \eR\).

Pas plus tard qu'immédiatement nous leur donnons une signification en définissant une topologie sur \( \bar\eR\). Les ouverts sur \( \bar \eR\) sont
\begin{enumerate}
    \item
        tous les ouverts de \( \eR\),
    \item
        les intervalles de la forme \( \mathopen] -\infty , a \mathclose[\) pour tous les \( a\in \eR\),
    \item
        les intervalles de la forme \( \mathopen] a , +\infty \mathclose[\) pour tous les \( a\in \eR\),
    \item
        la topologie engendrée par toutes ces parties de \( \bar \eR\).
\end{enumerate}

Par construction, les boules de \( \eR\) et les intervalles \( \mathopen] -\infty , a \mathclose[\) et \( \mathopen] a , +\infty \mathclose[\) forment une base de topologie pour \( \bar \eR\).

Si \( f\) est une fonction \( f\colon \eR\to \eR\), que signifie \( \lim_{x\to \infty} f(x)\) ? Il s'agit de considérer la fonction élargie
\begin{equation}
    \begin{aligned}
        \tilde f\colon \bar \eR&\to \bar\eR \\
        x&\mapsto \begin{cases}
            f(x)    &   \text{si } x\in \eR\\
            0    &    \text{si } x=\pm\infty.
        \end{cases}
    \end{aligned}
\end{equation}
Ensuite, c'est la définition topologie usuelle de la limite. Notons que les limites en \( a\) ne dépendent pas de la valeur effective de \( f\) en \( a\), donc le prolongement par \( 0\) est sans conséquences. Nous pouvions tout aussi bien prolonger par \( 4\).

Le même raisonnement tient pour donner un sens à \( \lim_{x\to a} f(x)=\pm \infty\).

%---------------------------------------------------------------------------------------------------------------------------
\subsection{Quelques mots à propos de la droite réelle complétée} 
%---------------------------------------------------------------------------------------------------------------------------

\begin{definition}
    La \defe{droite réelle complétée}{droite réelle complétée} est l'ensemble \( \eR\cup\{ \pm \infty \}\) où \( \pm\infty\) sont deux nouveaux éléments. Nous la notons \( \overline{ \eR }\) pour des raisons que nous verrons à peine plus bas.
\end{definition}

Cette définition ne servirait à rien si nous n'y mettions pas une topologie pour positionner les éléments \( \pm\infty\) par rapport à ceux qui existaient déjà dans \( \eR\).

\begin{definition}[Topologie sur \( \bar\eR\)]
La topologie sur \(\bar \eR\) est celle sur \( \eR\) à laquelle nous ajoutons les voisinages de \( \pm\infty\) de la façon suivante. Une partie \( V\) de \( \bar \eR\) est un voisinage de \( +\infty\) s'il existe \( m>0\) tel que \( \mathopen] m , +\infty \mathclose]\subset V\).
\end{definition}

Le lemme suivant justifie la notation \( \overline{ \eR }\) pour la droite réelle complétée\footnote{Mais ne justifie pas le qualificatif «complété» parce que l'espace métrique \( \eR\) était déjà complet.}.
\begin{lemma}       \label{LEMooPZXHooEEXsTC}
    L'adhérence de \( \eR\) dans \( \overline{ \eR }\) est \( \overline{ \eR }\).
\end{lemma}

Pour la suite nous utilisons la notation (pratique en probabilité)
\begin{equation}
    \{ f<a \}=\{ x\in S\tq f(x)<a \}.
\end{equation}

%---------------------------------------------------------------------------------------------------------------------------
\subsection{Limite pointée ou épointée ?}
%---------------------------------------------------------------------------------------------------------------------------
\label{SUBSECooVHKCooYRFgrb}

Si vous êtes dans l'enseignement en France\footnote{En particulier si vous voulez passer l'agrégation.}, vous devriez lire ceci à propos de limite pointée. Dans tous les autres cas, la limite pointée est une notion qui ne vous intéresse à priori pas.

\begin{definition}[\cite{ooCNVFooHdbArS}]
    Soient $X$ et $Y$ deux espaces topologiques, $A$ une partie de $X$, $f$ une application de $A$ dans $Y$, $a$ un point de $X$ adhérent à $A$ et \(\ell \) un point de $Y$. On dit que \( \ell\) est une \defe{limite pointée}{limite pointée} de $f$ au point $a$ si pour tout voisinage $V$ de \( \ell\), il existe un voisinage $W$ de a tel que pour tout point $x$ de $W\cap A$, l'image $f(x)$ appartient à $V$.
\end{definition}

La notion de limite pointée ne diffère de la limite que du fait que pour calculer la limite pointée en \( a\), nous tenons compte des valeurs de \( f\) sur \emph{tout} le voisinage de \( a\), y compris le point \( a\) lui-même.

Le choix entre la limite pointée ou épointée a été discuté en de nombreuses occasions \cite{BIBooKNWHooBRoxme,BIBooNUKAooVMqppa,BIBooDILKooUcmUVD,BIBooJDPPooVONaQV}.

\begin{enumerate}
    \item       \label{ITEMooXSRLooMVwIHU}
        Dans la majorité des cas, la limite pointée donne le même résultat que la limite parce que, fondamentalement, si nous voulons calculer une limite de \( f\) au point \( a\), c'est que \( f\) n'est pas définie en \( a\). C'est en particulier toujours le cas pour les limites en l'infini ou les limites définissant les dérivées.
    \item
        La limite pointée est un peu plus simple au départ.
    \item
        La limite épointée est un peu plus riche. Par exemple si on dit « la limite de \( f\) en \( a\) existe » , ça donne une régularité pour \( f\) autour de \( 0\) que la limite pointée ne parvient pas à exprimer.
    \item
        La limite pointée n'est connue qu'en France.
\end{enumerate}

Le point \ref{ITEMooXSRLooMVwIHU} est le plus important parce qu'il explique pourquoi il y a moyen de finir l'agrégation, et même de faire de la recherche en ne tombant jamais sur un cas où la différence est importante.

\begin{example}
    La fonction \( f\) donnée par
    \begin{equation}
        f(x)=\begin{cases}
            0    &   \text{si } x\neq 0\\
            4    &    \text{si }x=0
        \end{cases}
    \end{equation}
    a une limite épointée pour \( x\to 0\) qui vaut \( 0\). Elle n'a par contre pas de limite pointée en \( 0\).

    Cette fonction est l'exemple-type de différence entre limite usuelle et limite pointée.
\end{example}

\begin{example}
    Si vous voulez un cas dans lequel la différence se voit de façon macroscopique, aller lire le lemme \ref{LEMooYLIHooFBQyzC}, sa démonstration et l'exemple \ref{EXooHSYNooBZhDbE}.
\end{example}

Dans le Frido, nous choisissons de prendre la limite épointée comme définition de limite. Nous donnons ici quelques raisons pour ce choix.

\begin{enumerate}
    \item
       C'est la définition unanimement acceptée dans la communauté mathématique hors France.
   \item
       La limite pointée ne donne à peu près rien de nouveau par rapport à la continuité.
       
       Ce que le concept de limite apporte est la possibilité d'étudier le comportement de \( f\) pour les points «proches» de \( a\), sans regarder la valeur en \( a\) lui-même. Si l'idée est de regarder le comportement «proche» de \( a\) y compris au point \( a\) lui-même, c'est la notion de continuité qui fait le travail.

       Donc les contextes dans lesquels le concept de limite est intéressant sont justement les contextes dans lesquels la fonction étudiée n'existe pas au point étudié. Dans ce cas, les limites pointées et épointées coïncident.
\end{enumerate}

Que devez-vous faire ?
\begin{description}
    \item[Enseignement en France] La notion de limite pointée est celle nommée «limite» dans les programmes, et ce que nous nommons ici «limite» est nommé «limite épointée». Peut-être pour induire en erreur tout le reste de la planète ?
    \item[Recherche] Si vous faites de la recherche où que ce soit y compris en France, la seule définition de limite est la limite dite «épointée», celle qui sera toujours utilisée dans le Frido.
    \item[Doctorat] Vous commencez un doctorat en math, et vous avez vu la limite pointée comme seule définition de limite durant vos études ? Oubliez-la. Ou alors attendez-vous à vous à de sérieux quiproquos lorsque vous discuterez de mathématique avec des étrangers. 

        Disons clairement que si vous utilisez la limite pointée devant des non Français, ils se diront juste que vous devriez relire vos cours de base. Et si vous leur expliquez, il y a de bonnes chance qu'ils ne vous croient pas.
\end{description}

%///////////////////////////////////////////////////////////////////////////////////////////////////////////////////////////
\subsubsection{Le théorème de composition}
%///////////////////////////////////////////////////////////////////////////////////////////////////////////////////////////

Une des raisons fréquentes pour utiliser la limite pointée est que le théorème de composition est plus simple\cite{BIBooDILKooUcmUVD}.

Le voici avec des limites pointées. Pour faire la différence, j'adopte la notation \( {LP}\) pour la limite pointée et \( {LE}\) pour la limite épointée.
\begin{proposition}     \label{PROPooCQZZooMiZfQE}
    Soient \( f\) et \( g\) des fonctions \( \eR\to \eR\). Si \( {LP}_{x\to a}g(x)=\ell\) et si \( {LP}_{y\to \ell}f(y)=b\), alors
    \begin{equation}
        {LP}_{x\to a} (f\circ g)(x)={LP}_{y\to \ell} f(y)=b.
    \end{equation}
\end{proposition}

\begin{proof}
    Soit \( \epsilon>0\). L'hypothèse de limite pour \( f\) donne \( \eta>0\) tel que 
    \begin{equation}        \label{EQooLWGIooLqKThy}
        | y-\ell |<\eta \Rightarrow | f(y)-b |<\epsilon.
    \end{equation}

    Soit \( \delta>0\) tel que \( | x-a |<\delta\) implique \( | g(x)-\ell |<\eta\).

    Avec tout ça, si \( | x-a |<\delta\) nous avons \( | g(x)-\ell |<\eta\) et en appliquant l'implication \eqref{EQooLWGIooLqKThy} à \( y=g(x)\) nous trouvons \( | f\big( g(x) \big)-b |<\epsilon\).
\end{proof}

Avant d'énoncer et de démontrer le résultat correspondant pour les limites épointées, nous avons besoin d'un lemme, pour comprendre la différence d'hypothèse.

\begin{lemma}
    Si \( {LP}_{x\to a}f(x)=b\) alors il y a deux possibilités : 
    \begin{enumerate}
        \item
            Soit \( f\) est définie en \( a\) et alors \( f\) y est continue,
        \item 
            soit \( f\) n'est pas définie en \( a\) et alors poser \( f(a)=b\) donne un prolongement continu.
    \end{enumerate}
\end{lemma}

\begin{proposition}     \label{PROPooNWCMooCaDMex}
    Soient \( f\) et \( g\) des fonctions \( \eR\to \eR\). Nous supposons \( {LE}_{x\to a}g(x)=\ell\) et que \( f\) admette le prolongement continu \(b\) en \( \ell\)\footnote{Cette hypothèse est équivalente à dire que \( f\) a une limite pointée \( b\) en \( \ell\), c'est-à-dire la même hypothèse que dans la proposition \ref{PROPooNWCMooCaDMex}.} Alors
    \begin{equation}
        {LE}_{x\to a}(f\circ g)(x)=b.
    \end{equation}
\end{proposition}
Bon. Woaw. La différence est énorme.

\begin{proof}
    Soit \( \epsilon>0\). Par hypothèse de prolongement continu, il existe \( \eta>0\) tel que
    \begin{equation}
        | t-\ell |<\eta\Rightarrow | f(y)-b |<\epsilon.
    \end{equation}
    Soit \( \delta>0\) tel que
    \begin{equation}
        0<| x-a |<\delta\Rightarrow | g(x)-\ell |\leq \eta.
    \end{equation}
    Avec ce \( \delta\) nous avons que \( | 0<| x-a |<\delta |\) implique \( | f\big( g(x) \big)-b |<\epsilon\).
\end{proof}

Cela est surement une raison de présenter la limite pointée chez les petits. Mais ce n'est pas une raison pour les grands, que du contraire. L'énoncé de la proposition \ref{PROPooCQZZooMiZfQE} est à peine plus compliquée que celui de \ref{PROPooCQZZooMiZfQE}, mais elle dit un peu plus alors que la démonstration est la même.

Donc non, la proposition \ref{PROPooNWCMooCaDMex} n'ajoute pas d'hypothèse par rapport à \ref{PROPooCQZZooMiZfQE}. Au contraire, elle en enlève : la proposition \ref{PROPooNWCMooCaDMex} demande pour \( g\) une limite épointée au lieu d'une limite pointée. En ce qui concerne \( f\), les hypothèses sont les mêmes. La proposition \ref{PROPooNWCMooCaDMex} concerne une classe de fonctions un peu plus grande.

Au final, la proposition «épointée» \ref{PROPooNWCMooCaDMex} est un poil plus compliquée, mais elle a une hypothèse un peu plus faible et une conclusion un peu plus faible (existence d'une limite pointée). Bref, il s'agit d'un résultat différent. Mais comme maintenant nous sommes grands, nous sommes prêts à avoir des énoncés plus compliquée pour avoir des résultats plus complets.

%///////////////////////////////////////////////////////////////////////////////////////////////////////////////////////////
\subsubsection{Et les filtres ?}
%///////////////////////////////////////////////////////////////////////////////////////////////////////////////////////////

Si vous ne savez pas ce qu'est un filtre, vous pouvez sauter ces paragraphes. Sinon, vous pouvez vous dire que le débat «limite pointée» contre «limite épointée» n'a aucun sens parce que de toutes façons, la bonne façon de définir une limite passe par des filtres.

Alors le mieux est de se demander comment on construit, à partir de la notion de filtre, le nombre \( \lim_{x\to a} f(a)\) ?

Pas de bol, ça dépend du filtre choisi. Le premier filtre auquel on pense pour trouver une définition raisonnable de la limite de \( f(x)\) quand \( x\to a\) est le filtre des voisinages de \( a\). La notion de limite associée est la limite pointée. En ce sens la limite pointée est plus naturelle que la limite épointée. Cependant «naturel» signifie souvent «le premier qui nous tombe sous la main», ce qui ne signifie pas spécialement «le plus intéressant à utiliser».

La notion de limite épointée est la limite associée au filtre des voisinages épointés. Ce n'est, certes, pas le premier filtre qui nous tombe sous la main, mais il est, au moins dans le cadre de l'étude des fonctions sur \( \eR^n\), le plus efficace; celui qui donne le plus de nouvelles informations par rapport à la continuité.

%+++++++++++++++++++++++++++++++++++++++++++++++++++++++++++++++++++++++++++++++++++++++++++++++++++++++++++++++++++++++++++
\section{Formes bilinéaires et quadratiques}
%+++++++++++++++++++++++++++++++++++++++++++++++++++++++++++++++++++++++++++++++++++++++++++++++++++++++++++++++++++++++++++

Plus à propos de formes bilinéaires dans le thème \ref{THEMEooOAJKooEvcCVn}.

\begin{definition}[\cite{ooUQBZooCAKfrE}]      \label{DEFooEEQGooNiPjHz}
    Soient trois espaces vectoriels \( E,F\) et \( V\) sur le même corps commutatif \( \eK\). Une application \( b\colon E\times F\to V\) est \defe{bilinéaire}{application bilinéaire} si elle est séparément linéaire en ses deux variables, c'est-à-dire si
    \begin{enumerate}
        \item 
            \( b(u_1+u_2,v)=b(u_1,v)+b(u_2,v)\),
        \item
            \( b(u,v_1+v_2)=b(u,v_1)+b(u,v_2)\)
        \item
            \( b(\lambda u,v)=b(u,\lambda v)=\lambda b(u,v)\)
    \end{enumerate}
    pour tout \( u,u_1,u_2\in E\), \( v,v_1,v_2\in F\) et pour tout \( \lambda\in \eK\).

    Dans le cas \( E=F\) et \( V=\eK\), nous parlons de \defe{forme bilinéaire}{forme!bilinéaire} sur \( E\).

    Nous parlons de forme bilinéaire \defe{symétrique}{forme bilinéaire symétrique} si de plus \( b(u,v)=b(v,u)\).
\end{definition}

\begin{normaltext}
    Une application bilinéaire \( E\times E\to \eK\) n'est pas une application linéaire; la distinction est importante. La linéarité est
    \begin{equation}
        b(\lambda u,\lambda v)= b\big( \lambda(u,v) \big)=\lambda b(u,v)
    \end{equation}
    et la bilinéarité est
    \begin{equation}
        b(\lambda u,v)=b(u,\lambda v)=\lambda b(u,v).
    \end{equation}
    En réalité la seule forme qui soit à la fois linéaire et bilinéaire est la forme identiquement nulle : la condition
    \begin{equation}
        b(\lambda u,\lambda v)=\lambda^2b(u,v)=\lambda b(u,v)
    \end{equation}
    pour tout \( \lambda\in \eK\) implique \( b(u,v)=0\).
\end{normaltext}

\begin{example}[\cite{BIBooJMSXooYUADgm}]
    L'application
    \begin{equation}
        \begin{aligned}
            b\colon \eM(n,\eK)\times \eM(n,\eK)&\to \eK \\
            (A,B)&\mapsto \trace(AB) 
        \end{aligned}
    \end{equation}
    est une forme bilinéaire symétrique.

    La vérification est un calcul :
    \begin{equation}
        \trace(BA)=\sum_{i}(BA)_{ii}=\sum_{ik}B_{ik}A_{ki}=\sum_{ik}A_{ki}A_{ik}=\sum_k(AB)_{kk}=\trace(AB).
    \end{equation}
\end{example}

%+++++++++++++++++++++++++++++++++++++++++++++++++++++++++++++++++++++++++++++++++++++++++++++++++++++++++++++++++++++++++++
\section{Produit scalaire, produit hermitien}
%+++++++++++++++++++++++++++++++++++++++++++++++++++++++++++++++++++++++++++++++++++++++++++++++++++++++++++++++++++++++++++

\begin{definition}[Définie positive, thème~\ref{THEMEooYEVLooWotqMY}]      \label{DEFooJIAQooZkBtTy}
    Si $g$ est une application bilinéaire\footnote{Définition~\ref{DEFooEEQGooNiPjHz}.} sur un espace vectoriel \( E\) nous disons qu'elle est
    \begin{enumerate}
        \item
            \defe{définie positive}{application!définie positive} si $g(x,x)\geq 0$ pour tout $x\in E$ et $g(x,x)=0$ si et seulement si $x=0$.
        \item
            \defe{semi-définie positive}{application!semi-définie positive} si $g(x,x)\geq 0$ pour tout $x\in E$. Nous dirons aussi parfois qu'elle est simplement «positive».
        \end{enumerate}
\end{definition}
Cela est évidemment à lier à la définition~\ref{DefAWAooCMPuVM} et la proposition~\ref{PROPooUAAFooEGVDRC} : une application bilinéaires est définie positive si et seulement si sa matrice symétrique associée l'est.

\begin{definition}\label{DefVJIeTFj}
    Un \defe{produit scalaire}{produit!scalaire!en général} sur un espace vectoriel réel est une forme bilinéaire\footnote{Définition~\ref{DEFooEEQGooNiPjHz}.} symétrique strictement définie positive\footnote{Définition~\ref{DEFooJIAQooZkBtTy}.}.
\end{definition}

La définition suivante est utile pour celles qui veulent faire de la relativité\footnote{Voir le théorème \ref{THOooYHDWooWxVovH} qui établit les transformations de Lorentz.}.
\begin{definition}      \label{DEFooLPBGooXLxubc}
    Un \defe{produit pseudo-scalaire}{produit pseudo-scalaire} sur un espace vectoriel réel est une forme bilinéaire et symétrique.
\end{definition}

Vu que nous allons voir un pâté d'espaces avec des produits scalaires, nous leur donnons un nom.
\begin{definition}\label{DefLZMcvfj}
    Un espace vectoriel \defe{euclidien}{euclidien!espace} est un espace vectoriel de dimension finie muni d'un produit scalaire (définition~\ref{DefVJIeTFj}).
\end{definition}
Avouez que c'est drôle qu'un espace vectoriel est euclidien lorsqu'il possède une \emph{multiplication} alors qu'un anneau est euclidien lorsqu'il possède une \emph{division} (voir la définition~\ref{DefAXitWRL}). C'est pas très profond, mais si ça peut vous servir de moyen mnémotechnique\ldots

\begin{definition}[\cite{ooJUXBooVrwvfP}]  \label{DefMZQxmQ}
    Soit \( E\) est un espace vectoriel sur \( \eC\). Une application \( \langle ., .\rangle \colon E\times E\to \eC\) est \defe{sesquilinéaire à droite}{sesquilinéaire} si pour tout \( x,y\in E\) et pour tout \( \lambda\in \eC\),
    \begin{enumerate}
        \item
            \( \langle \lambda x, y\rangle =\lambda\langle x,y, \rangle =\langle x, \bar\lambda y\rangle \),
        \item
            \( \langle x+y, z\rangle =\langle x, y\rangle+\langle y, z\rangle  \),
        \item
            \( \langle x, y+z\rangle =\langle x, y\rangle +\langle x, z\rangle \).
    \end{enumerate}
    Cette forme est \defe{hermitienne}{hermitienne} si de plus
    \begin{equation}
        \langle x, y\rangle =\overline{ \langle y, x\rangle  }.
    \end{equation}
    Un \defe{produit hermitien}{produit hermitien} est une forme hermitienne strictement définie positive, c'est-à-dire telle que \( \langle x, x\rangle \geq 0\) pour tout \( x\in E\) et \( \langle x, x\rangle =0\) si et seulement si \( x=0\).
\end{definition}

\begin{example}
    L'ensemble \( E=\eC^n\) vu comme espace vectoriel de dimension \( n\) sur \( \eC\)  est muni d'une forme sesquilinéaire
    \begin{equation}    \label{EqFormSesqQrjyPH}
        \langle x, y\rangle =\sum_{k=1}^nx_k\bar y_k
    \end{equation}
    pour tout \( x,y\in\eC^n\). Cela est un espace vectoriel hermitien.
\end{example}

%---------------------------------------------------------------------------------------------------------------------------
\subsection{Norme, produit scalaire et Cauchy-Schwarz (cas réel)}
%---------------------------------------------------------------------------------------------------------------------------

Dans la suite, le produit scalaire de \( x\) et \( y\) pourra être noté indifféremment par \( x\cdot y\), \( \langle x, y\rangle \) ou \( b(x,y)\) lorsque une forme bilinéaire est donnée.

Nous rappelons au passage que les espaces vectoriels réels sont susceptibles de recevoir un produit scalaire, alors que les espaces vectoriels complexes sont susceptibles de recevoir un produit hermitien. Bien que de nombreux résultats soient identiques ou très similaires, ces deux notions sont à ne pas confondre.

Nous commençons par prouver qu'un produit scalaire étant donné, nous pouvons définir une norme par la formule \( \| x \|^2=\langle x, x\rangle \). Pour cela nous aurons besoin de l'inégalité de Cauchy-Schwarz.

\begin{theorem}[Inégalité de Cauchy-Schwarz, cas réel]      \label{ThoAYfEHG}
    Soit un espace vectoriel muni d'un produit scalaire \( (x,y)\mapsto x\cdot y\). En posant\footnote{Attention à la notation : pour l'instant nous ne savons pas que c'est une norme. Ce sera justifié dans la proposition~\ref{PropEQRooQXazLz}.}
    \begin{equation}
        \| x \|=\sqrt{ x\cdot x },
    \end{equation}
    nous avons
    \begin{equation}        \label{EQooZDSHooWPcryG}
		| x\cdot y |\leq \| x \|\| y \|.
	\end{equation}
    Nous avons une égalité si et seulement si \( x\) et \( y\) sont multiples l'un de l'autre.
\end{theorem}
\index{Cauchy-Schwarz}
\index{inégalité!Cauchy-Schwarz}

\begin{proof}
	Étant donné que les deux membres de l'inéquation sont positifs, nous allons travailler en passant au carré afin d'éviter les racines carrés dans le second membre.

	Nous considérons le polynôme
	\begin{equation}
		P(t)=\| x+ty \|^2=(x+ty)\cdot(x+ty)=x\cdot x+x\cdot ty+ty\cdot x+t^2y\cdot y.
	\end{equation}
    En utilisant la bilinéarité (pour sortir les \( t\)) et la symétrique du produit scalaire, puis en ordonnant les termes selon les puissances de $t$,
	\begin{equation}
		P(t)=\| y \|^2t^2+2(x\cdot y)t+\| x \|^2.
	\end{equation}
    %TODOooRUEZooGCVyQZ : faire la résolution.
	Cela est un polynôme du second degré en $t$ dont le signe est toujours positif (ou nul). Par conséquent le discriminant\footnote{Le fameux $b^2-4ac$.} doit être négatif ou nul. Nous avons donc
	\begin{equation}
		\Delta=4(x\cdot y)^2-4\| x \|^2\| y \|^2\leq 0,
	\end{equation}
	ce qui donne immédiatement
	\begin{equation}
		(x\cdot y)^2\leq\| x \|^2\| y \|^2.
	\end{equation}

    En ce qui concerne le cas d'égalité, si nous avons \( x\cdot y=\| x \|\| y \|\), alors le discriminant \( \Delta\) ci-dessus est nul et le polynôme \( P\) admet une racine double \( t_0\). Pour cette valeur nous avons
    \begin{equation}
        P(t_0)=| x+t_0y |=0,
    \end{equation}
    ce qui implique \( x+t_0y=0\) et donc que \( x\) et \( y\) sont liés.
\end{proof}

La proposition suivante montre que toute norme dérivant d'un produit scalaire vérifie l'identité du parallélogramme. Ce résultat sert souvent à prouver que des normes ne dérivent pas d'un produit scalaire. C'est le cas de la norme \( N(x,y)=| x |+| y |\) du lemme \ref{LEMooRWJYooOIJkZc} ainsi que du théorème de Weinersmith \ref{THOooCCMBooGulxkQ}.
\begin{proposition}[Norme dérivant d'un produit scalaire] \label{PropEQRooQXazLz}
    Si \( x,y\mapsto x\cdot y\) est un produit scalaire sur un espace vectoriel réel \( E\). Nous posons \( \| x \|=\sqrt{x\cdot x}\). Alors
    \begin{enumerate}
        \item
            L'opération \( \| . \|\) est une norme\footnote{Définition \ref{DefNorme}.}.
        \item
            Cette norme vérifie l'identité du parallélogramme :
            \begin{equation}        \label{EqYCLtWfJ}
                \| x-y \|^2+\| x+y \|^2=2\| x \|^2+2\| y \|^2.
            \end{equation}
    \end{enumerate}
\end{proposition}

\begin{proof}
    En deux parties.
    \begin{subproof}
        \item[C'est une norme]
            Nous allons nous contenter de prouver l'inégalité triangulaire. Si \( x,y\in E\) nous avons
            \begin{equation}
                \| x+y \|=\sqrt{\| x \|^2+\| y \|^2+2x\cdot y}.
            \end{equation}
            Par l'inégalité de Cauchy-Schwarz, théorème~\ref{ThoAYfEHG} nous avons aussi
            \begin{equation}
                2x\cdot y\leq 2\| x \|\| y \|.
            \end{equation}
            Nous pouvons donc majorer ce qui est dans la racine carrée :
            \begin{equation}
                \| x \|^2+\| y \|^2+2x\cdot y\leq \| x \|^2+\| y \|^2+2\| x \|\| y \|=\big( \| x \|+\| y \| \big)^2.
            \end{equation}
            En remettant les bouts ensemble,
            \begin{equation}
                \| x+y \|  =\sqrt{\| x \|^2+\| y \|^2+2x\cdot y}  \leq \sqrt{\big( \| x \|+\| y \| \big)^2}=\| x \|+\| y \|.
            \end{equation}

        \item[Inégalité du parallélogramme]
            Cette assertion est seulement un calcul :
            \begin{equation}
                \begin{aligned}[]
                    \| x-y \|^2+\| x+y \|^2&=(x-y)\cdot (x-y)+(x+y)\cdot(x+y)\\
                    &=x\cdot x-x\cdot y-y\cdot x+y\cdot y\\
                    &\quad +x\cdot x+x\cdot y+y\cdot x+y\cdot y\\
                    &=2x\cdot x+2y\cdot y\\
                    &=2\| x \|^2+2\| y \|^2.
                \end{aligned}
            \end{equation}
    \end{subproof}
\end{proof}

\begin{normaltext}
    Un produit scalaire fourni donc toujours une norme et donc une topologie. Il ne faudrait cependant pas croire que toute norme dérive d'un produit scalaire, même pas en dimension finie. Et ce, malgré l'équivalence de toutes les normes du théorème~\ref{ThoNormesEquiv} dont vous avez déjà peut-être entendu parler.
\end{normaltext}


L'intérêt du lemme suivant sera apparent en \ref{NORMooNKBCooKziIjx}.
\begin{lemma}   \label{LEMooRWJYooOIJkZc}
    Sur \( \eR^2\), l'application \( N(x,y)=| x |+| y |\) est une norme\footnote{Définition \ref{DefNorme}.} qui ne dérive pas d'un produit scalaire\footnote{La norme d'un produit scalaire est la proposition  \ref{PropEQRooQXazLz}.}.
\end{lemma}

\begin{proof}
    Nous commençons par montrer que \( N\) est une norme. Il faut vérifier les trois conditions de la définition \ref{DefNorme}.
    \begin{enumerate}
        \item
            Il faut utiliser le lemme \ref{LemooANTJooYxQZDw}\ref{ItemooNVDIooSuiSoB} dans les deux sens. Si \( (x,y)=(0,0)\), alors évidemment \( N(x,y)=0\). Dans l'autre sens, si \( N(x,y)=0\) nous avons
            \begin{equation}
                0=| x |+| y |\geq | x |.
            \end{equation}
            Donc \( | x |\leq 0\), mais comme \( | x |\geq 0\), nous avons \( | x |=0\) et donc \( x=0\). Le même raisonnement tient pour \( y\).
        \item
            En tenant compte du fait que \( | \lambda x |=| \lambda | |x |\), nous avons
            \begin{equation}
                N\big( \lambda(x,y) \big)=N(\lambda x,\lambda y)=| \lambda | |x |+| \lambda | |y |=| \lambda |(| x |+| y |)=| \lambda |N(x,y).
            \end{equation}
        \item
            Nous avons le calcul
            \begin{subequations}
                \begin{align}
                    N\big( (x,y)+(a,b) \big)&=N(x+a,y+b)\\
                    &=| x+a |+| y+b |\\
                    &\leq | x |+| a |+| y |+| b |       \label{SUBEQooIXKWooTNQFnu}\\
                    &=N(x,y)+N(a,b)
                \end{align}
            \end{subequations}
            Justification : pour \eqref{SUBEQooIXKWooTNQFnu} nous avons utilité \( | a+b |\leq | a |+| b |\), du lemme \ref{LemooANTJooYxQZDw}.
    \end{enumerate}
    Pour voir qu'elle ne dérive pas d'un produit scalaire, nous montrons qu'elle ne vérifie pas l'identité du parallélogramme de la proposition \ref{PropEQRooQXazLz}.

    Voici un petit bout de code qui nous permet de ne pas faire de recherches à la main :
    \lstinputlisting{tex/sage/sageSnip018.sage}

    Il est vite vu qu'avec \( v=(-1,1)\) et \( w=(1,1)\), l'identité du parallélogramme n'est pas vérifiée.
\end{proof}

\begin{lemma}[\cite{KXjFWKA}]   \label{LemLPOHUme}
    Soit \( V\) un espace vectoriel muni d'un produit scalaire et de la norme associée. Si \( x,y\in V\) satisfont à \( \| x+y \|=\| x \|+\| y \|\), alors il existe \( \lambda\geq 0\) tel que \( x=\lambda y\).
\end{lemma}

\begin{proof}
    Quitte à raisonner avec \( x/\| x \|\) et \( y/\| y \|\), nous supposons que \( \| x \|=\| y \|=1\). Dans ce cas l'hypothèse signifie que \( \| x+y \|^2=4\). D'autre part en écrivant la norme en termes de produit scalaire,
    \begin{equation}
        \| x+y \|^2=\| x \|^2+\| y \|^2+2\langle x, y\rangle ,
    \end{equation}
    ce qui nous mène à affirmer que \( \langle x, y\rangle =1=\| x \|\| y \|\). Nous sommes donc dans le cas d'égalité de l'inégalité de Cauchy-Schwarz\footnote{Théorème~\ref{ThoAYfEHG}.}, ce qui nous donne un \( \lambda\) tel que \( x=\lambda y\). Étant donné que \( \| x \|=\| y \|=1\) nous avons obligatoirement \( \lambda=\pm 1\), mais si \( \lambda=-1\) alors \( \langle x, y\rangle =-1\), ce qui est le contraire de ce qu'on a prétendu plus haut. Par souci de cohérence, nous allons donc croire que \( \lambda=1\).
\end{proof}

\begin{proposition}			\label{PropVectsOrthLibres}
	si $v_1,\cdots,v_k$ sont des vecteurs non nuls, orthogonaux deux à deux, alors ces vecteurs forment une famille libre.
\end{proposition}

\begin{lemma}       \label{LEMooYXJZooWKRFRu}
    Une isométrie d'un espace euclidien fixe l'origine.
\end{lemma}

\begin{proof}
    Soit une isométrie \( f\) d'un espace euclidien : \( f(x)\cdot f(y)=x\cdot y\) pour tout \( x,y\in E\). En particulier pour \( x=0\) nous avons
    \begin{equation}
        f(0)\cdot f(y)=0
    \end{equation}
    pour tout \( y\). Vu que \( f\) est une bijection, nous avons \( f(0)\cdot x=0\) pour tout \( x\). Comme le produit scalaire est non dégénéré cela implique que \( f(0)=0\).
\end{proof}

%--------------------------------------------------------------------------------------------------------------------------- 
\subsection{Cauchy-Schwarz etc. cas complexe}
%---------------------------------------------------------------------------------------------------------------------------

\begin{theorem}[Inégalité de Cauchy-Schwarz, cas complexe\cite{HilbertLi}]      \label{THOooSUCBooFnpkaF}
     Soit un espace vectoriel complexe muni d'un produit hermitien \( \langle ., .\rangle \). Alors pour tout vecteurs \( x,y\) nous avons
     \begin{equation}
         | \langle x, y\rangle  |\leq \| x \|\| y \|
     \end{equation}
     où nous avons posé \( \| x \|=\sqrt{ \langle x, x\rangle  }\).
\end{theorem}

\begin{proof}
    Si \( \langle x, y\rangle =0\), le résultat est évident; nous supposons que non. Nous posons
    \begin{equation}
        \theta=\frac{ \langle x, y\rangle  }{ | \langle x, y\rangle  | }.
    \end{equation}
    C'est un élément de \( \eC\) de norme \( 1\). Nous avons
    \begin{equation}
        \langle \frac{1}{ \theta }x, y\rangle =\frac{ | \langle x, y\rangle  | }{ \langle x, y\rangle  }\langle x, y\rangle =| \langle x, y\rangle  |\geq 0
    \end{equation}
    où le symbole «\( \geq\)» signifie «est réel et positif». Nous posons \( x'=\frac{1}{ \theta }x\) et nous considérons \( t\in \eR\). Remarquons que \( \| x' \|^2=\| x \|^2\) :
    \begin{equation}
        \| x' \|^2=\langle x', x'\rangle =\frac{1}{ \theta\bar\theta }\langle x, x\rangle =\| x \|^2
    \end{equation}
    parce que \( | \theta |=1\).

    En utilisant le fait que \( \langle a, b\rangle +\langle b, a\rangle =\real(\langle a, b\rangle )\) nous avons :
    \begin{subequations}
        \begin{align}
            0\leq \| x'+ty \|^2&=\| x' \|^2+t\langle x', y\rangle +t\langle y, x'\rangle +t^2\| y \|^2\\
            &=\| y \|^2t^2+2\real(\langle x', y\rangle )t+\| x' \|^2.
        \end{align}
    \end{subequations}
    Cela est un polynôme de degré \( 2\) en \( t\) qui n'est jamais strictement négatif. Autrement dit, il a au maximum une seule racine, ce qui signifie que son discriminant est négatif ou nul :
    \begin{equation}
        \real(\langle x', y\rangle )^2-\| y \|^2\| x' \|^2\leq 0.
    \end{equation}
    Mais nous avons choisi \( x'\) de telle sorte que \( \langle x', y\rangle =| \langle x, y\rangle  |\in \eR\) et \( \| x' \|^2=\| x \|^2\); nous avons donc
    \begin{equation}
        | \langle x, y\rangle  |^2\leq \| x \|^2\| y \|^2,
    \end{equation}
    comme il se devait.
\end{proof}

\begin{proposition}[Identité du parallélogramme\cite{BIBooXLLGooAFwpyU}]       \label{PROPooSSYJooHAXAnC}
    Soit une espace vectoriel complexe \( E\) muni d'un produit hermitien \( \langle ., .\rangle \). Nous posons \( \| x \|=\sqrt{ \langle x, x\rangle  }\). Nous avons
    \begin{enumerate}
        \item
            \( \| . \|\) est une norme.
        \item
            Elle vérifie l'identité du parallélogramme :
            \begin{equation}
                \| x+y \|^2+\| x-y \|^2=2\| x \|^2+2\| b \|^2
            \end{equation}
            pour tout \( x,y\in E\).
    \end{enumerate}
\end{proposition}

\begin{proof}
    En ce qui concerne le fait que \( \| . \|\) soit une norme, tout est essentiellement dans la définition \ref{DefMZQxmQ} d'un produit hermitien. Voyons tout de même l'inégalité triangulaire. Nous avons :
    \begin{subequations}
        \begin{align}
            \| x+y \|^2&=\langle x+y, x+y\rangle\\
            &=\| x \|^2+\| y \|^2+\langle x, y\rangle +\langle y, x\rangle\\
            &=\| x \|^2+\| y \|^2+2\Re\big( \langle x, y\rangle  \big)\\
            &\leq\| x \|^2+\| y \|^2+2|\Re\big( \langle x, y\rangle  \big)|\\
            &\leq\| x \|^2+\| y \|^2+2| \langle x, y\rangle  |\\
            &\leq \| x \|^2+\| y \|^2+2\| x \|\| y \|\label{SUBEQooQGQBooMRJcUc}\\
            &=\big( \| x \|+\| y \| \big)^2.
        \end{align}
    \end{subequations}
    Pour \eqref{SUBEQooQGQBooMRJcUc} nous avons utilisé Cauchy-Schwarz \ref{THOooSUCBooFnpkaF}.
\end{proof}

%---------------------------------------------------------------------------------------------------------------------------
\subsection{Projection et orthogonalité}
%---------------------------------------------------------------------------------------------------------------------------

\begin{proposition}[Propriétés du produit scalaire]
	Si $X$ et $Y$ sont des vecteurs de $\eR^3$, alors
	\begin{description}
		\item[Symétrie] $X\cdot Y=Y\cdot X$;
		\item[Linéarité] $(\lambda X+\mu X')\cdot Y=\lambda(X\cdot Y)+\mu(X'\cdot Y)$ pour tout $\lambda$ et $\mu$ dans $\eR$;
		\item[Défini positif] $X\cdot X\geq 0$ et $X\cdot X=0$ si et seulement si $X=0$.
	\end{description}
\end{proposition}
Note : lorsque nous écrivons $X=0$, nous voulons voulons dire $X=\begin{pmatrix}
	0	\\
	0	\\
	0
\end{pmatrix}$.


\begin{definition}
	La \defe{norme}{norme!vecteur} du vecteur $X$, notée $\| X \|$, est définie par
	\begin{equation}
		\| X \|=\sqrt{X\cdot X}=\sqrt{x^2+y^2+z^2}
	\end{equation}
	si $X=(x,y,z)$. Cette norme sera parfois nommée «norme euclidienne».
\end{definition}
Cette définition est motivée par le théorème de Pythagore. Le nombre $X\cdot X$ est bien la longueur de la «flèche» $X$. Plus intrigante est la définition suivante :
\begin{definition}
	Deux vecteurs $X$ et $Y$ sont \defe{orthogonaux}{orthogonal!vecteur} si $X\cdot Y=0$.
\end{definition}
Cette définition de l'orthogonalité est motivée par la proposition suivante.

\begin{proposition}		\label{PropProjScal}
	Si nous écrivons $\pr_Y$  l'opération de projection sur la droite qui sous-tend $Y$, alors nous avons
	\begin{equation}
		\| \pr_YX \|=\frac{ X\cdot Y }{ \| Y \| }.
	\end{equation}
\end{proposition}

\begin{proof}
	Les vecteurs $X$ et $Y$ sont des flèches dans l'espace. Nous pouvons choisir un système d'axe orthogonal tel que les coordonnées de $X$ et $Y$ soient
	\begin{equation}
		\begin{aligned}[]
			X&=\begin{pmatrix}
				x	\\
				y	\\
				0
			\end{pmatrix},
			&Y&=\begin{pmatrix}
				l	\\
				0	\\
				0
			\end{pmatrix}
		\end{aligned}
	\end{equation}
	où $l$ est la longueur du vecteur $Y$. Pour ce faire, il suffit de mettre le premier axe le long de $Y$, le second dans le plan qui contient $X$ et $Y$, et enfin le troisième axe dans le plan perpendiculaire aux deux premiers.

	Un simple calcul montre que $X\cdot Y=xl+y\cdot 0+0\cdot 0=xl$. Par ailleurs, nous avons $\| \pr_YX \|=x$. Par conséquent,
	\begin{equation}
		\| \pr_YX \|=\frac{ X\cdot Y }{ l }=\frac{ X\cdot Y }{ \| Y \| }.
	\end{equation}
\end{proof}

\begin{corollary}
	Si la norme de $Y$ est $1$, alors le nombre $X\cdot Y$ est la longueur de la projection de $X$ sur $Y$.
\end{corollary}

\begin{proof}
	Poser $\| Y \|=1$ dans la proposition~\ref{PropProjScal}.
\end{proof}

\begin{remark}
    Outre l'orthogonalité, le produit scalaire permet de savoir l'angle entre deux vecteurs à travers la définition~\ref{DEFooSVDZooPWHwFQ}. D'autres interprétations géométriques du déterminant sont listées dans le thème~\ref{THMooUXJMooOroxbI}.
\end{remark}

Nous sommes maintenant en mesure de déterminer, pour deux vecteurs quelconques $u$ et $v$, la projection orthogonale de $u$ sur $v$. Ce sera le vecteur $\bar u$ parallèle à $v$ tel que $u-\bar u$ est orthogonal à $v$. Nous avons donc
\begin{equation}
    \bar u=\lambda v
\end{equation}
et
\begin{equation}
    (u-\lambda v)\cdot v=0.
\end{equation}
La seconde équation donne $u\cdot v-\lambda v\cdot v=0$, ce qui fournit $\lambda$ en fonction de $u$ et $v$ :
\begin{equation}
    \lambda=\frac{ u\cdot v }{ \| v \|^2 }.
\end{equation}
Nous avons par conséquent
\begin{equation}
    \bar u=\frac{ u\cdot v }{ \| v \|^2 }v.
\end{equation}
Armés de cette interprétation graphique du produit scalaire, nous comprenons pourquoi nous disons que deux vecteurs sont orthogonaux lorsque leur produit scalaire est nul.

Nous pouvons maintenant savoir quel est le coefficient directeur d'une droite orthogonale à une droite donnée. En effet, supposons que la première droite soit parallèle au vecteur $X$ et la seconde au vecteur $Y$. Les droites seront perpendiculaires si $X\cdot Y=0$, c'est-à-dire si
\begin{equation}
	\begin{pmatrix}
		x_1	\\
		y_1
	\end{pmatrix}\cdot\begin{pmatrix}
		y_1	\\
		y_2
	\end{pmatrix}=0.
\end{equation}
Cette équation se développe en
\begin{equation}		\label{Eqxuyukljsca}
	x_1y_1=-x_2y_2.
\end{equation}
Le coefficient directeur de la première droite est $\frac{ x_2 }{ x_1 }$. Isolons cette quantité dans l'équation \eqref{Eqxuyukljsca} :
\begin{equation}
	\frac{ x_2 }{ x_1 }=-\frac{ y_1 }{ y_2 }.
\end{equation}
Donc le coefficient directeur de la première est l'inverse et l'opposé du coefficient directeur de la seconde.

\begin{example}
	Soit la droite $d\equiv y=2x+3$. Le coefficient directeur de cette droite est $2$. Donc le coefficient directeur d'une droite perpendiculaires doit être $-\frac{ 1 }{ 2 }$.
\end{example}

\begin{proof}[Preuve alternative]
	La preuve peut également être donnée en ne faisant pas référence au produit scalaire. Il suffit d'écrire toutes les quantités en termes des coordonnées de $X$ et $Y$. Si nous posons
	\begin{equation}
		\begin{aligned}[]
			X&=\begin{pmatrix}
				x_1	\\
				x_2	\\
				x_2
			\end{pmatrix},
			&Y&=\begin{pmatrix}
				y_1	\\
				y_2	\\
				y_3
			\end{pmatrix},
		\end{aligned}
	\end{equation}
	l'inégalité à prouver devient
	\begin{equation}
		(x_1y_1+x_2y_2+x_3y_3)^2\leq (x_1^2+x_2^2+x_3^2)(y_1^2+y_2^2+y_3^2).
	\end{equation}
	Nous considérons la fonction
	\begin{equation}
		\varphi(t)=(x_1+ty_1)^2+(x_2+ty_2)^2+(x_3+ty_3)^2
	\end{equation}
	En tant que norme, cette fonction est évidemment positive pour tout $t$. En regroupant les termes de chaque puissance de $t$, nous avons
	\begin{equation}
		\varphi(t)=(y_1^2+y_2^2+y_3^2)t^2+2(x_1y_1+x_2y_2+x_3y_3)t+(x_1^2+x_2^2+x_3^2).
	\end{equation}
	Cela est un polynôme du second degré en $t$. Par conséquent le discriminant doit être négatif. Nous avons donc
	\begin{equation}
		4(x_1y_1+x_2y_2+x_3y_3)^2-(x_1^2+x_2^2+x_3^2)(y_1^2+y_2^2+y_3^2)\leq 0.
	\end{equation}
	La thèse en découle aussitôt.
\end{proof}

\begin{proposition}     \label{PROPooVSVMooZrqxdc}
	La norme euclidienne a les propriétés suivantes :
	\begin{enumerate}
		\item
			Pour tout vecteur $X$ et réel $\lambda$,  $\| \lambda X \|=| \lambda |\| X \|$. Attention à ne pas oublier la valeur absolue !
		\item
			Pour tout vecteurs $X$ et $Y$, $\| X+Y \|\leq \| X \|+\| Y \|$.
	\end{enumerate}
\end{proposition}

\begin{proof}
    Pour le second point, nous avons les inégalités suivantes :
	\begin{subequations}
		\begin{align}
			\| X+Y \|^2&=\| X \|^2+\| Y \|^2+2X\cdot Y\\
			&\leq\| X \|^2+\| Y \|^2+2|X\cdot Y|\\
			&\leq\| X \|^2+\| Y \|^2+2\| X \|\| Y \|\\
			&=\big( \| X \|+\| Y \| \big)^2
		\end{align}
	\end{subequations}
    Nous avons utilisé d'abord la majoration $| x |\geq x$ qui est évidente pour tout nombre $x$; et ensuite l'inégalité de Cauchy-Schwarz~\ref{ThoAYfEHG}.
\end{proof}

%--------------------------------------------------------------------------------------------------------------------------- 
\subsection{Théorème de Pythagore}
%---------------------------------------------------------------------------------------------------------------------------

Nous allons donner une preuve du théorème de Pythagore.

\begin{theorem}[Pythagone\cite{MonCerveau}]     \label{THOooHXHWooCpcDan}
    Soient \( A,B,S\in \eR^2\) un triangle rectangle en \( A\), c'est-à-dire tel que
    \begin{equation}        \label{EQooRAWAooBxlBcZ}
        (B-A)\cdot (A-S)=0.
    \end{equation}
    Alors
    \begin{equation}
        \| S-B \|^2=\| S-A \|^2+\| B-A \|^2.
    \end{equation}
\end{theorem}

\begin{proof}
    En développant l'hypothèse \eqref{EQooRAWAooBxlBcZ} nous avons :
    \begin{equation}    \label{EQooYTDGooXzYQwi}
        B\cdot A-B\cdot S-\| A \|^2+A\cdot S=0.
    \end{equation}
    Et de même,
    \begin{equation}
        \| S-B \|^2=(S-B)\cdot(S-B)=\| S \|^2-2B\cdot S+\| B \|^2.
    \end{equation}
    En substituant dans cette dernière \( B\cdot S\) par \( B\cdot S=B\cdot A -\| A \|^2+A\cdot S \) tirée de \eqref{EQooYTDGooXzYQwi}, nous trouvons
    \begin{equation}
        \| S-B \|^2=\| S \|^2-2B\cdot A+2\| A \|^2-2A\cdot S+\| B \|^2=\| S-A \|^2+\| B-A \|^2.
    \end{equation}
\end{proof}

Je profite de l'occasion pour montrer mon scepticisme quant aux preuves de Pythagore basées sur différents pliages et découpages des carrés construits sur les côtés du triangle. Pour autant que je le sache, la géométrie dans «le plan» (c'est-à-dire pas dans \( \eR^2\) muni de son produit scalaire) ne définit pas «longueur» et «aire». Donc bon \ldots Il y a peut-être moyen de s'en sortir, mais je ne le connais pas.

%---------------------------------------------------------------------------------------------------------------------------
\subsection{Produit vectoriel}
%---------------------------------------------------------------------------------------------------------------------------

\begin{definition}      \label{DEFooTNTNooRjhuJZ}
	Soient $u$ et $v$, deux vecteurs de $\eR^3$. Le \defe{produit vectoriel}{produit!vectoriel} de $u$ et $v$ est le vecteur $u\times v$ défini par
    \begin{equation}        \label{EQooCUJRooFuFPaZ}
		u\times v=\det\begin{pmatrix}
			e_1	&	e_2	&	e_3	\\
			u_1	&	u_2	&	u_3	\\
			v_1	&	v_2	&	v_3
		\end{pmatrix}
    \end{equation}
	où les vecteurs $e_1$, $e_2$ et $e_3$ sont les vecteurs de la base canonique de $\eR^3$.
\end{definition}

\begin{lemma}
    Le produit vectoriel \( u\times v\) est également exprimé par
    \begin{subequations}        \label{EQSooOWGZooNYruoy}
        \begin{align}
            u\times v&=(u_2v_3-u_3v_2)e_1+(u_3v_1-u_1v_3)e_2+(u_1v_2-u_2v_1)e_3     \label{SEBEQooVROKooRpUOIr}\\
                &=\sum_{i,j,k}\epsilon_{ijk}v_iw_je_k
        \end{align}
    \end{subequations}
    où $\epsilon_{ijk}$ est défini par $\epsilon_{xyz}=1$ et ensuite $\epsilon_{ijk}$ est $1$ ou $-1$ suivant que la permutation des $x$, $y$ et $z$ est paire ou impaire. C'est-à-dire que \( \epsilon_{ijk}\) est la signature de la permutation qui amène \( (1,2,3)\) sur \( (i,j,k)\).
\end{lemma}

\begin{proof}
    Il s'agit seulement de développer explicitement le déterminant \eqref{EQooCUJRooFuFPaZ}.
\end{proof}

\begin{normaltext}
    Mettons que \( a\times b=v\). En calculant le même produit vectoriel dans la base \( f_i=-e_i\), les composantes de \( a\) et \( b\) changent de signe et la formule \eqref{EQSooOWGZooNYruoy} dit que le produit vectoriel ne change pas. On serait tenter d'écrire, dans la base \( \{ f_i \}\)
    \begin{equation}
        (-a)\times (-b)=v,
    \end{equation}
    tout en pleurant parce que dans la base des \( f_i\), le vecteur \( v\) devient \( -v\).

    Il a des personnes que cela tracasse tellement qu'on entend parler de «le produit vectoriel est une pseudo-vecteur sous \( \SO(2)\)».

    Il suffit d'être clair. Le produit vectoriel n'est défini que sur \( \eR^3\), et est définit par sa formule dans la base canonique, point barre. Si vous avez des vecteurs \( a\) et \( b\) dont vous connaissez les composantes dans une autre base, vous devez calculer les composantes dans la base canonique, utiliser la formule pour trouver les composantes de \( a\times b\) dans la base canonique. Ensuite, si ça vous chante, vous pouvez calculer à nouveau les composantes de \( a\times b\) dans une autre base.

    Tout cela pour dire que le produit vectoriel n'est pas une opération très généralisable. Il est possible, pour sembler plus intrinsèque, de tenter cette définition : le produit vectoriel \( a\times b\) est le vecteur perpendiculaire à \( a\) et \( b\), de longueur égale à l'aire du parallélogramme construit sur \( a\) et \( b\).

    Cette «définition» a plusieurs inconvénients.
    \begin{itemize}
        \item Elle demande quand même un produit scalaire et des aires; bref, elle demande une structure métrique,
        \item Elle ne donne pas le sens. En effet, dans \( \eR^3\), il y a deux vecteurs de longueur donnée perpendiculaires à \( a\) et \( b\). Il faut donc préciser le sens. Cela revient à donner une orientation et donc, fondamentalement, à choisir une base.
    \end{itemize}
    
    Bref, on retiendra que le produit vectoriel est une opération accrochée à \( \eR^3\) et a sa base canonique.
\end{normaltext}

Une des principales utilités du produit vectoriel est donnée dans la proposition suivante.
\begin{proposition}     \label{PROPooIQMTooFHNjfu}
    Si \( u\) et \( v\) sont des vecteurs de \( \eR^3\) alors le vecteur \( u\times v\) est perpendiculaire à \( u\) et à \( v\).
\end{proposition}
La chose importante à retenir est que le produit vectoriel permet de construire un vecteur simultanément perpendiculaire à deux vecteurs donnés. Le vecteur $u\times v$ est donc linéairement indépendant de $u$ et $v$. En pratique, si $u$ et $v$ sont déjà linéairement indépendants, alors le produit vectoriel permet de compléter une base de $\eR^3$.

\begin{lemmaDef}
    Nous avons l'égalité suivante pour tout \( u,v,w\in \eR^3\) :
    \begin{equation}        \label{EQooKJYUooSQgfXU}
        (u\times v)\cdot w=\det\begin{pmatrix}
                u_1	&	u_2	&	u_3	\\
                v_1	&	v_2	&	v_3	\\
                w_1	&	w_2	&	w_3
        \end{pmatrix}.
    \end{equation}
    Le résultat est nommé le \defe{produit mixte}{produit!mixte} de trois vecteurs de \( \eR^3\).
\end{lemmaDef}

\begin{normaltext}
    Nous avons donné un nom à la combinaison \( (u\times v)\cdot w\). J'imagine que vous voyez pourquoi nous ne considérons pas la combinaison $(u\cdot v)\times w$.
\end{normaltext}

Le lemme suivant donne un moyen compliqué et peu pratique de calculer la valeur absolue du produit mixte. La formule \eqref{EQooWZUQooYydphW} ne sera utilisée que pour faire le lien entre un jacobien et un élément de volume en dimension trois lorsque nous verrons les intégrales sur des variétés. Voir l'équation \eqref{EQooYIJSooHtkXfu}. 

% TODO lier à la vraie définition de l'intégrale sur une carte quand elle sera faite.
% TODOooWZMDooEJhpNS
\begin{lemma}[\cite{MonCerveau}]        \label{LEMooSMWNooCmEZeY}
    Le produit mixte peut également être exprimé par
    \begin{equation}        \label{EQooWZUQooYydphW}
           |(u\times v)\cdot w|^2=\det\begin{pmatrix}
            \| u \|^2    &   u\cdot v    &   u\cdot w    \\
            v\cdot u    &   \| v \|^2    &   v\cdot w    \\
            w\cdot u    &   w\cdot v    &   \| w \|^2
        \end{pmatrix}.
    \end{equation}
\end{lemma}

\begin{proof}
    Si nous notons 
    \begin{equation}
        a= \begin{pmatrix}
                u_1	&	u_2	&	u_3	\\
                v_1	&	v_2	&	v_3	\\
                w_1	&	w_2	&	w_3
        \end{pmatrix},
    \end{equation}
    il faut simplement remarquer que
    \begin{equation}
           \begin{pmatrix}
            \| u \|^2    &   u\cdot v    &   u\cdot w    \\
            v\cdot u    &   \| v \|^2    &   v\cdot w    \\
            w\cdot u    &   w\cdot v    &   \| w \|^2
        \end{pmatrix}=aa^t.
    \end{equation}
    Donc au niveau des déterminants, en utilisant les propositions \ref{PROPooHQNPooIfPEDH} et le lemme \ref{LEMooCEQYooYAbctZ} nous avons
    \begin{equation}
           \det\begin{pmatrix}
            \| u \|^2    &   u\cdot v    &   u\cdot w    \\
            v\cdot u    &   \| v \|^2    &   v\cdot w    \\
            w\cdot u    &   w\cdot v    &   \| w \|^2
        \end{pmatrix}=\det(aa^t)=\det(a)\det(a^t)=\det(a)^2.
    \end{equation}
    Et maintenant, par définition, \( \det(a)=(u\times w)\cdot w\). Donc le résultat annoncé.
\end{proof}

\begin{proposition}		 \label{PropScalMixtLin}
	Les applications produit scalaire, vectoriel et mixte sont multilinéaires. Spécifiquement, nous avons les propriétés suivantes.
	\begin{enumerate}
		\item
			Les applications produit scalaire et vectoriel sont bilinéaires. C'est-à-dire que pour tout vecteurs $a$, $b$, $c$ et pour tout nombre $\alpha$ et $\beta$ nous avons
    \begin{equation}
        \begin{aligned}[]
            a\times (\alpha b +\beta c)&=\alpha(a\times b)+\beta(a\times c)\\
            (\alpha a+\beta b)\times c&=\alpha(a\times c)+\beta(b\times c).
        \end{aligned}
    \end{equation}

        \item
            Le produit mixte est trilinéaire.
		\item
			Le produit vectoriel est antisymétrique, c'est-à-dire $u\times v=-v\times u$.
		\item
			Nous avons $u\times v=0$ si et seulement si $u$ et $v$ sont colinéaires, c'est-à-dire si et seulement si l'équation $\alpha u+\beta v=0$ a une solution différente de la solution triviale $(\alpha,\beta)=(0,0)$.
		\end{enumerate}
\end{proposition}

\begin{proposition}[Identité de Lagrange\cite{ooHFUZooGakvHi}]     \label{PROPooMXAIooJureOD}
    Si \( x,y\in \eR^n\), alors
    \begin{equation}
        \| x \|^2\| y \|^2-(x\cdot y)^2=\sum_j\sum_{i<j}(x_iy_j-x_jy_i)^2.
    \end{equation}
    Et si \( n=3\) alors
    \begin{equation}
        \| x\times y \|=\| y \|^2\| y \|^2-(x\cdot y)^2.
    \end{equation}
\end{proposition}

\begin{proof}
    C'est un calcul. D'abord nous avons
    \begin{equation}
        \| x \|^2\| y \|^2-(x\cdot y)^2=\sum_ix_i^2\sum_jy_j^2-\big( \sum_k x_ky_k  \big)^2=\sum_{ij}x_i^2y_j^2-\sum_{kl}x_ky_kx_ly_l.
    \end{equation}
    Ensuite nous coupons les sommes de la façon suivante
    \begin{equation}
        \sum_{ij}=\sum_j\sum_{i<j}+\sum_j(i=j)+\sum_j\sum_{i>j}
    \end{equation}
    pour obtenir
    \begin{equation}
        \begin{aligned}[]
            \| x \|^2\| y \|^2-(x\cdot y)^2&=\sum_j\sum_{i<j}x_i^2y_j^2+\sum_jx_j^2y_j^2+\sum_j\sum_{i>j}x_i^2y_j^2\\
                &\quad-\sum_l\sum_{k<l}x_ky_kx_ly_l-\sum_kx_k^2y_k^2-\sum_l\sum_{k>l}x_ky_kx_ly_l.
        \end{aligned}
    \end{equation}
    Il y a deux termes qui se simplifient. Notez que si \( A_{kl}\) est symétrique en \( kl\) nous avons
    \begin{equation}
        \sum_l\sum_{k<l}A_{kl}=\sum_k\sum_{l<k}A_{lk}=\sum_k\sum_{l<k}A_{kl}.
    \end{equation}
    La première égalité était seulement un renommage des indices. Le coup des indices symétriques est justement ce qu'il se passe dans les deux termes en\( x_ky_kx_ly_l\), donc nous les regroupons :
    \begin{subequations}
        \begin{align}
            \| x \|^2\| y \|^2-(x\cdot y)^2&=\sum_j\big( \sum_{i<j}x_i^2x_j^2+\sum_{i>j}x_i^2y_j^2-2\sum_{i>j}x_iy_ix_jy_j \big)\\
            &=\sum_j\sum_{i<j}(x_i^2y_j^2+x_j^2y_i^2-2x_iy_ix_jy_j)\\
            &=\sum_j\sum_{i<j}(x_iy_j-x_jy_i)^2.
        \end{align}
    \end{subequations}
    Voila qui prouve la première formule. Pour la seconde, il faut seulement poser \( n=3\) et écrire les sommes explicitement.

    \begin{itemize}
        \item 
    Pour \( j=1\), la somme sur \( i\) est \( \sum_{i<1}\), c'est-à-dire aucun termes.
\item
    Pour \( j=2\), il y a seulement \( i=1\), donc le terme \( (x_1y_2-x_2y_1)^2\).

\item
    Pour \( j=3\), il y a les termes \( i=1\) et \( i=2\), donc les termes \( (x_1y_3-x_3y_1)^2+(x_2y_3-x_3y_2)^2\).
    \end{itemize}
    Ces trois termes collectés sont justement les composants (au carré) de \( x\times y\) données dans la formule \eqref{SEBEQooVROKooRpUOIr}.
\end{proof}

Les trois vecteurs de base $e_x$, $e_y$ et $e_y$ ont des produits vectoriels faciles à retenir :
\begin{equation}
    \begin{aligned}[]
        e_x\times e_y&=e_z\\
        e_y\times e_z&=e_x\\
        e_z\times e_x&=e_y
    \end{aligned}
\end{equation}

Les deux formules suivantes, qui mêlent le produit scalaire et le produit vectoriel, sont souvent utiles en analyse vectorielle :
\begin{equation}
	\begin{aligned}[]
		(u\times v)\cdot w&=u\cdot(v\times w)\\
		(u\times v)\times w&=-(v\cdot w)u+(u\cdot w)v		\label{EqFormExpluxxx}
	\end{aligned}
\end{equation}
pour tout vecteurs $u$, $v$ et $w$ dans $\eR^3$. Nous les admettons sans démonstration. La seconde formule est parfois appelée \defe{formule d'expulsion}{formule!d'expulsion (produit vectoriel)}.

\begin{example}
    Calculons le produit vectoriel $v\times w$ avec
    \begin{equation}
        \begin{aligned}[]
            v&=\begin{pmatrix}
                3    \\
                -1    \\
                1
            \end{pmatrix}&w=\begin{pmatrix}
                1    \\
                2    \\
                -1
            \end{pmatrix}.
        \end{aligned}
    \end{equation}
    Les vecteurs s'écrivent sous la forme $v=3e_x-e_y+e_z$ et $w=e_x+2e_y-e_z$. Le produit vectoriel s'écrit
    \begin{equation}
        \begin{aligned}[]
            (3e_x-e_y+e_z)\times (e_x+2e_y-e_z)&=6e_x\times e_y-3e_x\times e_z\\
                                &\quad -e_y\times e_x + e_y\times e_z\\
                                &\quad + e_z\times e_x + 2e_z\times e_y\\
                                &=6e_z+3e_y+e_z+e_x+e_y-2e_x\\
                                &=-e_x+4e_y+7e_z.
        \end{aligned}
    \end{equation}
\end{example}

%---------------------------------------------------------------------------------------------------------------------------
\subsection{Produit mixte}
%---------------------------------------------------------------------------------------------------------------------------

Si $a$, $b$ et $c$ sont trois vecteurs, leur \defe{produit mixte}{produit!mixte} est le nombre $a\cdot(b\times c)$. En écrivant le produit vectoriel sous forme de somme de trois déterminants $2\times 2$, nous avons
\begin{equation}
    \begin{aligned}[]
        a\cdot& (b\times c)\\&=(a_1e_x+a_2e_y+a_3e_z)\cdot\left(
        \begin{vmatrix}
            b_2    &   b_3    \\
            c_2    &   c_3
        \end{vmatrix}e_x-\begin{vmatrix}
            b_1    &   b_3    \\
            c_1    &   c_3
        \end{vmatrix}e_y+\begin{vmatrix}
            b_1    &   b_2    \\
            c_1    &   c_2
        \end{vmatrix}\right)\\
        &=a_1\begin{vmatrix}
            b_2    &   b_3    \\
            c_2    &   c_3
        \end{vmatrix}-a_2\begin{vmatrix}
            b_1    &   b_3    \\
            c_1    &   c_3
        \end{vmatrix}+a_3\begin{vmatrix}
            b_1    &   b_2    \\
            c_1    &   c_2
        \end{vmatrix}\\
        &=\begin{vmatrix}
            a_1    &   a_2    &   a_3    \\
            b_1    &   b_2    &   b_3    \\
            c_1    &   c_2    &   c_3
        \end{vmatrix}.
    \end{aligned}
\end{equation}
Le produit mixte s'écrit donc sous forme d'un déterminant. Nous retenons cette formule:
\begin{equation}        \label{EqProduitMixteDet}
    a\cdot (b\times c)=\begin{vmatrix}
        a_1    &   a_2    &   a_3    \\
        b_1    &   b_2    &   b_3    \\
        c_1    &   c_2    &   c_3
    \end{vmatrix}.
\end{equation}

Un grand intérêt du produit vectoriel est qu'il fournit un vecteur qui est simultanément perpendiculaire aux deux vecteurs donnés.
\begin{proposition}     \label{PROPooTUVKooOQXKKl}
    Le produit vectoriel\footnote{Définition \ref{DEFooTNTNooRjhuJZ}.} $a\times b$ est un vecteur orthogonal à $a$ et $b$.
\end{proposition}

\begin{proof}
    Vérifions que $a\perp (a\times b)$. Pour cela, nous calculons $a\cdot (a\times b)$, c'est-à-dire le produit mixte
    \begin{equation}
        a\cdot(a\times b)=\begin{vmatrix}
            a_1    &   a_2    &   a_3    \\
            a_1    &   a_2    &   a_3    \\
            b_1    &   b_2    &   b_3
        \end{vmatrix}=0.
    \end{equation}
    L'annulation de ce déterminant est due au fait que deux de ses lignes sont égales.
\end{proof}

Ces résultats admettent une intéressante généralisation.
\begin{lemma}       \label{LEMooFRWKooVloCSM}
    Soit \( X\in \eR^n\) ainsi que \( v_1,\ldots, v_{n-1}\in \eR^n\). Alors
    \begin{enumerate}
        \item
            Nous avons
            \begin{equation}        \label{EQooMQNPooRHHBjz}
                \det(X,v_1,\ldots, v_{n-1})=X\cdot
                \det\begin{pmatrix}
                     e_1   &   \ldots    &   e_n    \\
                        &   v_1    &       \\
                        &   \vdots    &       \\
                        &   v_{n-1}    &
                 \end{pmatrix}
            \end{equation}
        \item
            Le vecteur
            \begin{equation}
                \det\begin{pmatrix}
                     e_1   &   \ldots    &   e_n    \\
                        &   v_1    &       \\
                        &   \vdots    &       \\
                        &   v_{n-1}    &
                 \end{pmatrix}
            \end{equation}
            est orthogonal à tous les \( v_i\).
    \end{enumerate}
\end{lemma}

\begin{proof}
    Vu que les deux côtés de \eqref{EQooMQNPooRHHBjz} vus comme fonctions de \( X\), sont des applications linéaires de \( \eR^n\) dans \( \eR\), il suffit de vérifier l'égalité sur une base.

    Nous posons \( \tau_i\colon \eR^n\to \eR^{n-1}\),
    \begin{equation}
        \tau_i(v)_k=\begin{cases}
            v_k    &   \text{si } k<i\\
            v_{k+1}    &    \text{si } k\geq i\text{.}
        \end{cases}
    \end{equation}
    et nous avons d'une part
    \begin{equation}
        e_k\cdot
                \det
                \begin{pmatrix}
                     e_1   &   \ldots    &   e_n    \\
                        &   v_1    &       \\
                        &   \vdots    &       \\
                        &   v_{n-1}    &
                 \end{pmatrix}
                 =\det\begin{pmatrix}
                     \tau_kv_1   \\
                     \vdots   \\
                     \tau_kv_{n-1}
                 \end{pmatrix}
            \end{equation}
     et d'autre part,
     \begin{equation}
         \det(e_k,v_1,\ldots, v_{n-1})=\det
         \begin{pmatrix}
             0&&&\\
             \vdots&&&\\
             1&v_1&\cdots&v_{n-1}\\
             \vdots&&&\\
             0&&&
         \end{pmatrix}=\det(\tau_k v_1,\ldots, \tau_k v_{n-1}).
     \end{equation}
     La première assertion est démontrée.

     En ce qui concerne la seconde, il suffit d'appliquer la première et se souvenir qu'un déterminant est nul lorsque deux lignes sont égales\footnote{Corolaire \ref{CORooAZFCooSYINvBl}.}. En effet :
     \begin{equation}
         v_k\cdot \det
                \begin{pmatrix}
                     e_1   &   \ldots    &   e_n    \\
                        &   v_1    &       \\
                        &   \vdots    &       \\
                        &   v_{n-1}    &
                 \end{pmatrix}
                 =
                 \det(v_k,v_1,\ldots, v_n)=0.
     \end{equation}
\end{proof}



%---------------------------------------------------------------------------------------------------------------------------
\subsection{Procédé de Gram-Schmidt}
%---------------------------------------------------------------------------------------------------------------------------

\begin{proposition}[Procédé de Gram-Schmidt]    \label{PropUMtEqkb}
    Un espace euclidien possède une base orthonormée.
\end{proposition}
\index{espace!euclidien}
\index{Gram-Schmidt}

\begin{proof}
    Soit \( E\) un espace euclidien et \( \{ v_1,\ldots, v_n \}\), une base quelconque de \( E\). Nous posons d'abord
    \begin{equation}
        \begin{aligned}[]
            f_1&=v_1,&e_1&=\frac{ f_1 }{ \| f_1 \| }.
        \end{aligned}
    \end{equation}
    Ensuite
    \begin{equation}
        \begin{aligned}[]
            f_2&=v_2-\langle v_2, e_1\rangle e_1,&e_2&=\frac{ f_2 }{ \| f_2 \| }.
        \end{aligned}
    \end{equation}
    Notons que \( \{ e_1,e_2 \}\) est une base de \( \Span\{ v_1,v_2 \}\). De plus elle est orthogonale :
    \begin{equation}
        \langle e_1, f_2\rangle =\langle e_1, v_2\rangle -\langle v_2, e_1\rangle \underbrace{\langle e_1, e_1\rangle}_{=1} =0.
    \end{equation}
    Le fait que \( \| e_1 \|=\| e_2 \|=1\) est par construction. Nous avons donc donné une base orthonormée de \( \Span\{ v_1,v_2 \}\).

    Nous continuons par récurrence en posant
    \begin{equation}
        \begin{aligned}[]
            f_k&=v_k-\sum_{i=1}^{k-1}\langle v_k, e_i\rangle e_i,&e_k&=\frac{ f_k }{ \| f_k \| }.
        \end{aligned}
    \end{equation}
    Pour tout \( j<k\) nous avons
    \begin{equation}
        \langle e_j, f_k\rangle =\langle e_j, v_k\rangle -\sum_{i=1}^{k-1}\langle v_k, e_i\rangle \underbrace{\langle e_i, e_j\rangle}_{=\delta_{ij}} =0
    \end{equation}
\end{proof}
Cet algorithme de Gram-Schmidt nous donne non seulement l'existence de bases orthonormée pour tout espace euclidien, mais aussi le moyen d'en construire à partir de n'importe quelle base.

%---------------------------------------------------------------------------------------------------------------------------
\subsection{Approximation}
%---------------------------------------------------------------------------------------------------------------------------

Le lemme suivant est surtout intéressant en dimension infinie.
\begin{lemma}
    Soit un espace vectoriel normé \( V\) et un sous-espace vectoriel dense \( A\). Soit \( v\in V\); il existe une suite \( (v_n)\) dans \( A\) telle que \( v_n\stackrel{V}{\longrightarrow}v\) et \( \| v_n \|\leq \| v \|\) pour tout \( n\).
\end{lemma}

\begin{proof}
    Vu que \( A\) est dense, il existe une suite \( a_n\) dans \( A\) telle que \( a_n\to v\). Ensuite il suffit de poser
    \begin{equation}
        v_n=\frac{ n }{ n+1 }\frac{ \| v \| }{ \| a_n \| }a_n.
    \end{equation}
    Par construction nous avons toujours
    \begin{equation}
        \| v_n \|=\frac{ n }{ n+1 }\| v \|\leq \| v \|.
    \end{equation}
    Et de plus, la norme étant continue\footnote{Où dans le calcul suivant nous utilisons la continuité de la norme ? Posez-vous la question.},
    \begin{equation}
        \lim_{n\to \infty} v_n=\lim_{n\to \infty} \frac{ n }{ n+1 }\lim_{n\to \infty} \frac{ \| v \| }{ \| v_n \| }\lim_{n\to \infty} v_n=v.
    \end{equation}

    Le fait que \( v_n\) soit dans \( A\) est dû au fait que \( A\) soit vectoriel.
\end{proof}

\begin{proposition}     \label{PROPooVEMGooYKhMFy}
    Soit un espace vectoriel normé \( V\) et un sous-espace vectoriel dense \( A\). Soit \( v\in V\); pour tout \( a\in \eR\) nous avons
    \begin{equation}
        \sup\{ | v\cdot a |\tq a\in A\text{ et }\| a \|\leq \lambda \}=\lambda\| v \|.
    \end{equation}
\end{proposition}

\begin{proof}
    D'abord pour tout \( a\in A\) vérifiant \( \| a \|\leq \lambda\) l'inégalité de Cauchy-Schwarz~\ref{ThoAYfEHG} donne
    \begin{equation}
        | v\cdot a |\leq \| v \|\| a \|\leq \lambda\| v \|.
    \end{equation}
    Donc le supremum dont on parle est majoré par \( \lambda\| v \|\).

    Il nous faut l'inégalité dans l'autre sens. Par densité nous pouvons choisir une suite \( v_n\in A\) tel que \( v_n\to v\). Ensuite nous posons
    \begin{equation}
        a_n=\frac{ \lambda }{ \| v_n \| }v_n.
    \end{equation}
    Nous avons \( \| a_n \|=\lambda\) pour tout \( n\) et
    \begin{equation}
        | v\cdot a_n |=\frac{ \lambda }{ \| v_n \| }| v\cdot v_n |,
    \end{equation}
    et en passant à la limite,
    \begin{equation}
        \lim_{n\to \infty} | v\cdot a_n |=\frac{ \lambda }{ \| v \| }\| v\cdot v \|=\lambda\| v \|.
    \end{equation}
    Donc l'ensemble sur lequel nous prenons le supremum contient une suite convergente vers \( \lambda\| v \|\). Le supremum est donc au moins aussi grand que cela.
\end{proof}

%---------------------------------------------------------------------------------------------------------------------------
\subsection{Quelques exemples de normes sur \texorpdfstring{$\eR^n$}{Rn}}
%---------------------------------------------------------------------------------------------------------------------------

Il est possible de définir de nombreuses normes sur $\eR^n$. Citons-en quelques-unes.

\begin{propositionDef}      \label{PROPooCLZRooIRxCnZ}
    Les formules suivantes définissent des normes sur \( \eR^n\).
    \begin{enumerate}
        \item
    Les normes $\| . \|_{L^p}$ ($p\in\eN$) sont définies de la façon suivante :
    \begin{equation}		\label{EqDeformeLp}
        \| x \|_{L^p}=\Big( \sum_{i=1}^n| x_i |^p\Big)^{1/p},
    \end{equation}
    pour tout $x=(x_1,\ldots,x_n)\in\eR^n$.
\item
    La norme $L^2$ est la \defe{norme euclidienne}{norme!euclidienne}.
\item
    Nous définissons également la \defe{norme supremum}{norme!supremum} par
    \begin{equation}
	    \| x \|_{\infty}=\max_i| x_i |.
    \end{equation}
    \end{enumerate}
\end{propositionDef}

\begin{proof}
    Point par point\quext{Preuve non terminée}.
    \begin{enumerate}
        \item
            Le cas \( p=1\) est déjà fait dans le lemme \ref{LEMooRWJYooOIJkZc}.
        \item
    Le fait que \( x\mapsto\| x \|_{L^2}\) soit une norme provient de la propriété suivante :
    \begin{equation}
        \sqrt{ (a+b)^2 }\leq \sqrt{ a^2 }+\sqrt{ b^2 },
    \end{equation}
    laquelle se démontre en passant au carré :
    \begin{equation}        \label{EQooRYNYooTzZpPz}
        (a+b)^2=a^2+b^2+2ab\leq a^2+b^2+2| ab |=\big( \sqrt{ a^2 }+\sqrt{ b^2 } \big)^2.
    \end{equation}
\item
    \end{enumerate}
\end{proof}

Parmi ces normes, celles qui seront le plus souvent utilisées dans ces notes sont
\begin{equation}
	\begin{aligned}[]
		\| x \|_{L^1}&=\sum_{i=1}^n| x_i |,\\
		\| x \|_{L^2}&=\Big( \sum_{i=1}^n| x_i |^2 \Big)^{1/2}.
	\end{aligned}
\end{equation}

\newcommand{\CaptionFigDistanceEuclide}{La \emph{norme} euclidienne induit la \emph{distance} euclidienne. D'où son nom. Le point $C$ est construit aux coordonnées $(A_x,B_y)$.}
\input{auto/pictures_tex/Fig_DistanceEuclide.pstricks}

Soient $A=(A_x,A_y)$ et $B=(B_x,B_y)$ deux éléments de $\eR^2$. La distance\footnote{Ne pas confondre «distance» et «norme».} euclidienne entre $A$ et $B$ est donnée par $\| A-B \|_2$. En effet, sur la figure~\ref{LabelFigDistanceEuclide}, la distance entre les points $A$ et $B$ est donnée par
\begin{equation}
	| AB |^2=| AC |^2+| CB |^2=| A_x-B_x |^2+| A_y-B_y |^2,
\end{equation}
par conséquent,
\begin{equation}
	| AB |=\sqrt{| A_x-B_x |^2+| A_y-B_y |^2}=\| A-B \|_2.
\end{equation}

\begin{remark}
	Si $A$, $B$ et $C$ sont trois points dans le plan $\eR^2$, alors l'inégalité triangulaire $| AB |\leq| AC |+| CB |$ est précisément la propriété~\ref{ItemDefNormeiii} de la norme (définition~\ref{DefNorme}). En effet l'inégalité triangulaire s'exprime de la façon suivante en termes de la norme $\| . \|_2$ :
	\begin{equation}	\label{EqNDeuxAmBNNdd}
		\| A-B \|_2\leq \| A-C \|_2+\| C-B \|_2.
	\end{equation}
	En notant $u=A-C$ et $v=C-B$, l'équation \eqref{EqNDeuxAmBNNdd} devient exactement la propriété de définition de la norme :
	\begin{equation}
		\| u+v \|_2\leq \| u \|_2+\| v \|_2.
	\end{equation}
	Ceci explique pourquoi cette propriété des normes est appelée «inégalité triangulaire».
\end{remark}


%+++++++++++++++++++++++++++++++++++++++++++++++++++++++++++++++++++++++++++++++++++++++++++++++++++++++++++++++++++++++++++
\section{Équivalence des normes}
%+++++++++++++++++++++++++++++++++++++++++++++++++++++++++++++++++++++++++++++++++++++++++++++++++++++++++++++++++++++++++++
\label{normes_equiv}

Au premier coup d'œil, les notions dont nous parlons dans ce chapitre ont l'air très générales. Nous prenons en effet n'importe quel espace vectoriel $V$ de dimension finie, et nous le munissons de n'importe quelle norme (rien que dans $\eR^m$ nous en avons défini une infinité par l'équation \eqref{EqDeformeLp}). À partir de ces données, nous définissons les boules, la topologie, l'adhérence, etc.

%---------------------------------------------------------------------------------------------------------------------------
\subsection{En dimension finie}
%---------------------------------------------------------------------------------------------------------------------------

Dans $\eR^n$, les normes $\| . \|_{L^1}$, $\| . \|_{L^2}$ et $\| . \|_{\infty}$ ne sont pas égales. Cependant elles ne sont pas complètement indépendantes au sens où l'on sent bien que si un vecteur sera grand pour une norme, il sera également grand pour les autres normes; les normes «vont dans le même sens». Cette notion est précisée par le concept de norme équivalente.

\begin{definition}		\label{DefEquivNorm}
    Deux normes $N_1$ et $N_2$ sur $\eR^m$ sont \defe{\wikipedia{fr}{Norme_équivalente}{équivalentes}}{equivalence@équivalence!norme}\index{norme!équivalence}\index{équivalence!de norme} s'il existe deux nombres réels strictement positifs $k_1$ et $k_2$ tels que
	\begin{equation}
		k_1N_1(x)\leq N_2(x)\leq k_2 N_1(x),
	\end{equation}
	pour tout $x$ dans $\eR^m$. Dans ce cas nous écrivons que $N_1\sim N_2$.
\end{definition}

\begin{lemma}       \label{LEMooHAITooWdtLAN}
    La définition de norme équivalentes donne une relation d'équivalence (définition~\ref{DefHoJzMp}) sur l'ensemble des normes existantes sur $\eR^m$.
\end{lemma}

\begin{proposition} \label{PropLJEJooMOWPNi}
    Pour \( \eR^N\), nous avons les équivalences de normes $\| . \|_{L^1}\sim\| . \|_{L^2}$, $\| . \|_{L^1}\sim\| . \|_{\infty}$ et $\| . \|_{L^2}\sim\| . \|_{\infty}$. Plus précisément nous avons les inégalités
    \begin{enumerate}
        \item\label{ItemABSGooQODmLNi}
           $ \| x \|_2\leq \| x \|_1\leq\sqrt{n}\| x \|_2$
        \item\label{ItemABSGooQODmLNii}
            $\| x \|_{\infty}\leq \| x \|_1\leq n \| x \|_{\infty}$
        \item\label{ItemABSGooQODmLNiii}
            $\| x \|_{\infty}\leq \| x \|_2\leq \sqrt{n}\| x \|_{\infty}$
    \end{enumerate}
\end{proposition}


\begin{proof}
    En mettant au carré la première inégalité nous voyons que nous devons vérifier l'inégalité
    \begin{equation}
        | x_1 |^2+\cdots+| x_n |^2\leq\big( | x_1 |+\cdots+| x_n | \big)^2
    \end{equation}
    qui est vraie parce que le membre de droite est égal au carré de chaque terme plus les double produits. La seconde inégalité provient de l'inégalité de Cauchy-Schwarz (théorème~\ref{ThoAYfEHG}) sur les vecteurs
    \begin{equation}
        \begin{aligned}[]
            v&=\begin{pmatrix}
                1/n    \\
                \vdots    \\
                1/n
            \end{pmatrix},
            &w&=\begin{pmatrix}
                | x_1 |    \\
                \vdots    \\
                | x_n |
            \end{pmatrix}.
        \end{aligned}
    \end{equation}
    Nous trouvons
    \begin{equation}
        \frac{1}{ n }\sum_i| x_i |\leq\sqrt{n\cdot\frac{1}{ n^2 }}\sqrt{\sum_i| x_i |^2},
    \end{equation}
    et par conséquent
    \begin{equation}
        \sum_i| x_i |\leq\sqrt{n}\| x \|_2.
    \end{equation}

    La première inégalité de~\ref{ItemABSGooQODmLNiii} se démontre en remarquant que si \( a\) et \( b\) sont positifs, \( a\leq\sqrt{a^2+b}\). En appliquant cela à \( a=\max_i| x_i |\), nous avons
    \begin{equation}
        \max_i| x_i |\leq\sqrt{ | x_1 |^2+\cdots+| x_n |^2  }
    \end{equation}
    parce que \( \max_i| x_i |\) est évidemment un des termes de la somme. Pour la seconde inégalité de~\ref{ItemABSGooQODmLNiii}, nous avons
    \begin{equation}
        \sqrt{\sum_k| x_k |^2}\leq\left( \sum_k\max_i| x_i |^2 \right)^{1/2}=\sqrt{n}\| x \|_{\infty}.
    \end{equation}
    Pour obtenir cette inégalité, nous avons remplacé tous les termes \( | x_k |\) par le maximum.
\end{proof}

Pour les autres normes \( \| . \|_p\), il y a des inégalités dans \ref{THOooPPDPooJxTYIy} et \ref{CORooMBQMooWBAIIH}; voir aussi le thème \ref{THEMEooUJVXooZdlmHj}.

En réalité, toutes les normes \( \| . \|_{L^p}\) et \( \| . \|_{\infty}\) sont équivalentes et, plus généralement, nous avons le résultat suivant, très étonnant à première vue, et en réalité assez difficile à prouver :
\begin{theorem}[\cite{TrenchRealAnalisys}]		\label{ThoNormesEquiv}
	Sur un espace vectoriel de dimension finie, toutes les normes sont équivalentes.
\end{theorem}
% TODO : la preuve est à la page 583 de Trench.

\begin{corollary}       \label{CORooBRDYooLmGJDE}
    Soit \( V\) un espace vectoriel de dimension finie et \( \| . \|_1\), \( \| . \|_2\) deux normes sur \( V\). Alors l'identité \( \id\colon V\to V\) est un isomorphisme d'espace topologique \( (V,\| . \|_1)\to (V,\| . \|_2)\).

    De plus les ouverts sont les mêmes : une partie de \( V\) est ouverte dans \( (V,\| . \|_1)\) si et seulement si elle est ouverte dans \( (V,\| . \|_2)\).
\end{corollary}

\begin{normaltext}      \label{NORMooNKBCooKziIjx}
    Le lemme \ref{LEMooRWJYooOIJkZc} donnera une norme sur \( \eR^2\) qui ne dérive pas d'un produit scalaire. Vu que toutes les normes sur \( \eR^2\) produisent la même topologie (c'est le corolaire~\ref{CORooBRDYooLmGJDE}), il y a parfaitement moyen pour deux espaces vectoriels topologiques d'être isomorphes alors que l'un a une norme dérivant d'un produit scalaire et l'autre non.
\end{normaltext}

\begin{normaltext}
    Le théorème d'équivalence de norme sera utilisé pour montrer que l'ensemble des formes quadratiques non dégénérées de signature \( (p,q)\) est ouvert dans l'ensemble des formes quadratiques, proposition~\ref{PropNPbnsMd}. Plus généralement il est utilisé à chaque fois que l'on fait de la topologie sur les espaces de matrices en identifiant \( \eM(n,\eR)\) à \( \eR^{n^2}\), pour se rassurer en se disant que ce qu'on fait ne dépend pas de la norme choisie.

    Voir aussi ce qu'on en fait en \ref{NORMooDAZZooDiGFoW} pour démontrer la différentiabilité à partir des dérivées partielles.
\end{normaltext}

\begin{proposition}[\cite{MonCerveau}] \label{PROPooNTCFooEcwZwt}
    Let \( V\) be a finite dimensional complex vector space. For a basis \( B=\{ e_1,\ldots, e_n \}\) of \( V\) we define
    Soit un espace vectoriel \( V\) de dimension finie sur \( \eC\). Pour une base \( B= \{ e_i \}\) de \( V\) nous définissons
    \begin{equation}        \label{EQooEGXVooLASQIC}
        \| \sum_kv_ke_k \|_B= \sqrt{ \sum_k| v_k |^2 }.
    \end{equation}
    \begin{enumerate}
        \item
            La formule \eqref{EQooEGXVooLASQIC} définit une norme sur \( V\).
        \item
            Si \( B\) et \( B'\) sont des bases de \( V\), alors les topologies induites par le norme \( \| . \|_B\) et \( \| . \|_{B'}\) sont égales.
    \end{enumerate}
\end{proposition}

\begin{proof}
    Nous commençons par fixer une base \( B=\{ e_i \}_{i=1,\ldots, n}\) de \( V\). Cette base nous permet de définir
    \begin{equation}
        \begin{aligned}
            \varphi\colon V&\to \eC^n \\
            \sum_kv_ke_k&\mapsto (v_1,\ldots, v_n). 
        \end{aligned}
    \end{equation}
    Cette application linéaire permet d'écrire
    \begin{equation}
        \| v \|_V=\| \varphi(v) \|_{\eC^n}.
    \end{equation}
    À partir de là, la vérification des propriétés de la définition \ref{DefNorme} est immédiate. Par exemple :
    \begin{equation}
        \| v+w \|=\| \varphi(v+w) \|=\| \varphi(v)+\varphi(w) \|\leq \| \varphi(v) \|+\| \varphi(w) \|=\| v \|+\| w \|.
    \end{equation}

    En ce qui concerne la seconde assertion, c'est le théorème \ref{ThoNormesEquiv}.
\end{proof}

%---------------------------------------------------------------------------------------------------------------------------
\subsection{Contre-exemple en dimension infinie}
%---------------------------------------------------------------------------------------------------------------------------
\label{SubSecPOlynomesCE}

Lorsque nous considérons des espaces vectoriels de dimension infinie, les choses ne sons plus aussi simples. Nous voyons ici sur l'exemple de l'espace des polynômes que le théorème~\ref{ThoNormesEquiv} n'est plus valable si on enlève l'hypothèse de dimension finie.

On considère l'ensemble des fonctions polynomiales à coefficients réels sur  l'intervalle $[0,1]$.
\begin{equation}
\mathcal{P}_\eR([0,1])=\{p:[0,1]\to \eR\,|\, p : x\mapsto a_0+a_1 x +a_2 x^2 + \ldots, \, a_i\in\eR,\,\forall i\in \eN\}.
\end{equation}
Cet ensemble, muni des opérations usuelles de somme entre polynômes et multiplications par les scalaires, est un espace vectoriel.

Sur $\mathcal{P}(\eR)$ on définit les normes suivantes
\begin{equation}
\begin{aligned}
&\|p\|_\infty=\sup_{x\in[0,1]}\{p(x)\},\\
&\|p\|_1 =\int_0^1|p(x)|\, dx,\\
&\|p\|_2 =\left(\int_0^1|p(x)|^2\, dx\right)^{1/2}.\\
\end{aligned}
\end{equation}
Les inégalités suivantes sont  immédiates
\begin{equation}
\begin{aligned}
&\|p\|_1 =\int_0^1|p(x)|\, dx\leq \|p\|_\infty,\\
&\|p\|_2 =\left(\int_0^1|p(x)|^2\, dx\right)^{1/2}\leq \|p\|_\infty,\\
\end{aligned}
\end{equation}
mais la norme $\|\cdot\|_\infty$ n'est  équivalente ni à $\|\cdot\|_1$, ni à $\|\cdot\|_2$. Soit $p_k(x)= x^k$. Alors
\begin{equation}
\begin{aligned}
&\|p_k\|_\infty=1,\\
&\|p_k\|_1 =\int_0^1x^k\, dx=  \frac{1}{k+1},\\
&\|p_k\|_2 =\left(\int_0^1x^{2k}\, dx\right)^{1/2}=\sqrt{\frac{1}{2k+1}}.
\end{aligned}
\end{equation}
Pour $k\to \infty$ les normes $\|p_k\|_1$, $\|p_k\|_2$ tendent vers zéro, alors que la norme $\|p_k\|_\infty$ est constante, donc les normes ne sont pas équivalentes parce que il n'existe pas un nombre positif $m$ tel que
\begin{equation}
\begin{aligned}
& m \|p_k\|_\infty\leq \|p_k\|_1 ,\\
& m \|p_k\|_\infty\leq \|p_k\|_2 ,\\
\end{aligned}
\end{equation}
uniformément pour tout $k$ dans $\eN$.

%+++++++++++++++++++++++++++++++++++++++++++++++++++++++++++++++++++++++++++++++++++++++++++++++++++++++++++++++++++++++++++
\section{Produit fini d'espaces vectoriels normés}
%+++++++++++++++++++++++++++++++++++++++++++++++++++++++++++++++++++++++++++++++++++++++++++++++++++++++++++++++++++++++++++
\label{sec_prod}

Dans cette sections nous parlons de produits finis d'espaces. Cela ne signifie pas que chacun des espaces soient séparément de dimension finie.

%---------------------------------------------------------------------------------------------------------------------------
\subsection{Distance et norme produit}
%---------------------------------------------------------------------------------------------------------------------------

\begin{propositionDef}[Distance produit]    \label{DefZTHxrHA}
    Si \( (E_1,d_1)\),\ldots, \( (E_n,d_n)\) sont des espaces métriques alors la formule
    \begin{equation} 
        d(x,y)=\max_{i=1,\ldots, n}d_i(x_i,y_i)
    \end{equation}
    définit une distance sur le produit cartésien \( E=E_1\times\ldots\times E_n\). Elle est la \defe{distance produit}{distance produit}.
\end{propositionDef}

La définition de la norme sur un produit d'espaces vectoriels normés découle immédiatement de la définition de la distance~\ref{DefZTHxrHA} :
\begin{lemmaDef}[\cite{ooALKGooMAzKpz}]  \label{DefFAJgTCE}
    Soient $V$ et $W$ deux espaces vectoriels normés. 
    \begin{enumerate}
        \item
            L'ensemble
            \begin{equation}
            V\times W=\{(v,w)\,|\, v\in V,\, w\in W\}
            \end{equation}
            est un espace vectoriel.
        \item 
            L'opération
            \begin{equation}	\label{EqNormeVxWmax}
                \|(v,w) \|_{V\times W}=\max\{\|v\|_{V},\|w\|_W\}.
            \end{equation}
            est une norme\footnote{Définition \ref{DefNorme}.} sur \( V\times V\).
        \item
            La topologie de cette norme est la même que la topologie produit sur \( V\times W\).
    \end{enumerate}
    L'espace vectoriel \( V\times W\) muni de cette norme est l'\defe{espace produit}{produit!d'espaces vectoriels normés} de $V$ et $W$. La topologie ainsi définie de deux manières est la topologie qui sera toujours considérée dans le cas de produit d'espaces vectoriels normés.
\end{lemmaDef}

\begin{proof}
    En plusieurs parties.
    \begin{subproof}
        \item[Espace vectoriel]
                Il est presque immédiat de vérifier que le produit cartésien $V\times W$ est un espace vectoriel pour les opération de somme et multiplication par les scalaires définies composante par composante. C'est-à-dire,  si $(v_1,w_1)$, $(v_2,w_2)$ sont dans $V\times W$ et $a$, $b$ sont des scalaires, alors
                \begin{equation}
                    a (v_1,w_1)+ b(v_2,w_2)=(av_1,aw_1)+ (bv_2,bw_2)=(av_1+bv_2,aw_1+bw_2).
                \end{equation}

            \item[Norme]
                On doit vérifier les trois conditions de la définition~\ref{DefNorme}.
                \begin{itemize}
                    \item Soit $(v,w)$ dans $V\times W$ tel que $\|(v,w)\|_{V\times W}=\max\{\|v\|_{V},\|w\|_W\}=0$. Alors $\|v\|_V=0$ et $\|w\|_W=0$, donc $v=0_V$ et $w=0_W$. Cela implique $(v,w)=(0_v,0_w)=0_{V\times W}$.
                    \item Pour tout $a$ dans $\eR$ et $(v,w)$ dans $V\times W$, la norme $\|a (v,w)\|_{V\times W}$ se calcule de la façon suivante :
                        \begin{equation}
                            \|a (v,w)\|_{V\times W}= \max\{ \| av \|_V,\| aw \|_W \} =|a|\max\{\|v\|_{V},\|w\|_W\}=|a|\|(v,w)\|_{V\times W}.
                        \end{equation}
                    \item Soient $(v_1,w_1)$ et $(v_2,w_2)$ dans $V\times W$.
                    \begin{equation}
                        \begin{aligned}
                            \|(v_1,w_1)+(v_2,w_2)\|_{V\times W}&=\max\{\|v_1+v_2\|_{V},\|w_1+w_2\|_W\}\\
                            &\leq \max\{\|v_1\|_V+\|v_2\|_{V},\|w_1\|_W+\|w_2\|_W\}\\
                            &\leq\max\{\|v_1\|_V,\|w_1\|_W\}+ \max\{\|v_2\|_{V},\|w_2\|_W\}\\
                            &=\|(v_1,w_1)\|_{V\times W}+\|(v_2,w_2)\|_{V\times W}.
                        \end{aligned}
                    \end{equation}
                \end{itemize}
            \item[Équivalence]

    Dans cette preuve, nous considérons la «topologie de \( V\times W\)» comme étant la topologie produit et «la topologie métrique de \( V\times W\)» la topologie de la norme produit.
    \begin{subproof}
        \item[Dans un sens]
            La proposition \ref{LEMooKJJNooMHNcSP} dit qu'une prébase de \( V\times W\) est donnée par 
            \begin{equation}
                \big\{   B(v,r)\times B(w,s)\tq v\in V;w\in W;r,s>0   \big\}.
            \end{equation}
            Nous prouvons maintenant que \( S= B(v_0,r)\times B(w_0,s)\) est un ouvert de \( \big( V\times W,\| . \|_{V\times W} \big)\). Pour cela nous prouvons que tout élément de \( S\) contient un voisinage métrique contenu dans \( S\).

            Soit \( (v_1,w_1)\in S\). Nous posons
            \begin{equation}
                d\big( (v_1,w_1), (v_0,w_0) \big)=\delta<\max\{ r,s \}.
            \end{equation}
            Nous considérons \( \epsilon>0\) et nous montrons que si \( \epsilon\) est assez petit, \( B\big( (v_1,w_1),\epsilon \big)\subset S\). Pour cela nous considérons \( (v,w)\in B\big( (v_1,w_1),\epsilon \big)\) et nous calculons un tout petit peu :
            \begin{subequations}
                \begin{align}
                    d\big( (v,w),(v_0,w_0) \big)&\leq d\big( (v,w),(v_1,w_1) \big)+d\big( (v_1,w_1),(v_0,w_0) \big)\\
                    &<\epsilon+\delta.
                \end{align}
            \end{subequations}
            Si \( \epsilon\) est assez petit, le tout reste plus petit que \( \max\{ r,s \}\).

            Donc \( S\) est bien un ouvert métrique par le théorème \ref{ThoPartieOUvpartouv}. Vu que la topologie métrique contient une prébase de la topologie produit, tout ouvert de la topologie produit est un ouvert de la topologie métrique.
        \item[Dans l'autre sens]
            Soient un ouvert métrique \( \mO\) ainsi que \( (v_0,w_0)\in \mO\); il existe \( r>0\) tel que
            \begin{equation}
                B\big( (v_0,w_0),r \big)\subset \mO.
            \end{equation}
            Nous affirmons que \( B(v_0,r)\times B(w_0,r)\) est contenu dans \( \mO\), de telle sorte que \( \mO\) soit un ouvert de la topologie produit. Pour \( (v_1,w_1)\in B(v_0,r)\times B(w_0,r)\) nous avons
            \begin{equation}
                    d\big( (v_1,w_1),(v_0,w_0) \big)=\max\{ \| v_1-v_0 \|,\| w_1-w_0 \| \}<r
            \end{equation}
            parce que \( v_1\in B(v_0,r)\) et \( w_1\in B(w_0,r)\).

            Donc tout élément de \( \mO\) admet un voisinage «produit» contenu dans \( \mO\); donc \( \mO\) est ouvert pour le topologie produit.
    \end{subproof}
    \end{subproof}
\end{proof}

\begin{normaltext}
    En particulier, pour la topologie de la norme maximum, la convergence d'une suite implique la convergence «composante par composante» par la proposition~\ref{PROPooNRRIooCPesgO}.
\end{normaltext}


%+++++++++++++++++++++++++++++++++++++++++++++++++++++++++++++++++++++++++++++++++++++++++++++++++++++++++++++++++++++++++++
\section{Topologie réelle en dimension $n$}
%+++++++++++++++++++++++++++++++++++++++++++++++++++++++++++++++++++++++++++++++++++++++++++++++++++++++++++++++++++++++++++

%---------------------------------------------------------------------------------------------------------------------------
\subsection{Ouverts et fermés}
%---------------------------------------------------------------------------------------------------------------------------

La définition suivante est là juste pour la facilité des notations; nous ne disons pas qu'elle est liée à la topologie sur \( \eR^n\).
\begin{definition}  \label{DefZVuBbqp}
	La \defe{boule ouverte}{boule!ouverte} de centre $x_0 \in \eR^n$ et de rayon $r \in
	\eR^+$ est définie par
	\begin{equation}
		B(x_0,r) = \{ x \in \eR^n \tq \norme{x - x_0} < r \},
	\end{equation}
	tandis que la \defe{boule fermée}{boule!fermée} de centre $x_0$ et de rayon $r$ est
	\begin{equation}
        \overline{  B(x_0,r)} = \{ x \in \eR^n \tq \norme{x - x_0} \leq r \};
	\end{equation}
	la différence est que l'inégalité dans la première est stricte.
\end{definition}

\begin{propositionDef}
    Sur \( \eR^n\), nous considérons les deux topologies suivantes :
    \begin{enumerate}
        \item       \label{ITEMooWACPooFBAWhx}
            la topologie produit \( \eR\times \ldots\times \eR\) des espaces topologiques \( (\eR,| . |)\),
        \item       \label{ITEMooJPJHooGTuLen}
            la topologie de la norme
            \begin{equation}
                \| (x_1,\ldots, x_n) \|_{\infty}=\max_i\{ | x_i | \},
            \end{equation}
        \item       \label{ITEMooEBYQooXiOOtb}
            la topologie de la norme
            \begin{equation}
                \| (x_1,\ldots, x_n) \|_2=\sqrt{ \sum_{i=1}^nx_i^2 }.
            \end{equation}
    \end{enumerate}
    Ces deux topologies sont égales et sont la topologie que nous allons toujours considérer sur \( \eR^n\) (saut mention très explicite du contraire).
\end{propositionDef}

\begin{proof}
    Les topologies \ref{ITEMooWACPooFBAWhx} et \ref{ITEMooJPJHooGTuLen} sont identiques par le lemme \ref{DefFAJgTCE}. Les topologies \ref{ITEMooJPJHooGTuLen} et \ref{ITEMooEBYQooXiOOtb} sont identiques par la proposition \ref{PropLJEJooMOWPNi} (ou plus généralement par le théorème \ref{ThoNormesEquiv} qui donne l'équivalence de toutes les normes).
\end{proof}

\begin{proposition}\label{PROPooEQYJooBbPiAj}
    Une partie \( A\) de \( \eR^n\) est ouverte si et seulement si pour tout \( a\in A\) il existe \( r>0\) tel que \( B(a,r)\subset A\).
\end{proposition}
Cette proposition est évidemment à mettre en rapport avec le théorème~\ref{ThoPartieOUvpartouv}.

Le lemme suivant justifie le vocabulaire des définitions~\ref{DefZVuBbqp}.
\begin{lemma}   \label{LemMESSExh}
    Pour tout $x \in \eR^n$ et tout $r >0$ la boule \( B(x,r)\) est ouverte.
\end{lemma}

\begin{proof}
    Afin de prouver que la boule est ouverte, nous prenons un point $p\in B(x,r)$, et nous allons montrer qu'il existe une boule autour de $p$ qui est contenue dans $B(x,r)$.

    Étant donné que $p\in B(x,r)$, nous avons $d(p,x)<r$. Prouvons que la boule $B\big(p,r-d(p,x)\big)$ est contenue dans $B(x,r)$. Pour cela, nous prenons $p'\in B\big(p,r-d(p,x)\big)$, et nous essayons de prouver que $p'\in B(x,r)$. En effet, en utilisant l'inégalité triangulaire,
    \begin{equation}
	    d(x,p')\leq d(x,p)+d(p,p')\leq d(x,p)+r-d(p,x)=r.
    \end{equation}
\end{proof}

%---------------------------------------------------------------------------------------------------------------------------
\subsection{Intérieur, adhérence et frontière}
%---------------------------------------------------------------------------------------------------------------------------

\begin{definition}
  Soient $A \subset \eR^n$ et $x \in \eR^n$. Le point $x$ est \defe{intérieur}{intérieur} à $A$ s'il existe une boule autour de $x$ complètement contenue dans $A$. L'ensemble des points intérieurs à $A$ est noté $\Int A$ ou $\mathring A$, de sorte qu'on a précisément
  \begin{equation*}
    x \in \Int A \iffdefn  \exists \epsilon > 0 \tq
    B(x,\epsilon) \subset A.
  \end{equation*}
\end{definition}

\begin{normaltext}

La notion d'adhérence a déjà été définie en~\ref{DEFooSVWMooLpAVZR}, et précisé par le lemme~\ref{LEMooILNCooOFZaTe}. Dans le cas de \( \eR^n\) dans lequel les boules forment une base de la topologie nous pouvons encore préciser de la façon suivante:
\begin{equation}
	x \in \Adh A \iffdefn \forall \epsilon > 0, B(x,\epsilon) \cap A \neq \emptyset
\end{equation}
\end{normaltext}

\begin{proposition}
Pour $A \subset \eR^n$, nous avons
\begin{equation*}
	\Int A \subseteq A  \subseteq \Adh A
\end{equation*}
\end{proposition}

\begin{definition}      \label{DEFooACVLooRwehTl}
  La \defe{frontière}{frontière} ou le \defe{bord}{bord} de $A$ est défini par $\partial A = \Adh A \setminus \Int A$. L'ensemble $A$ est un \defe{ouvert}{ouvert} si $A = \Int A$, et c'est un \defe{fermé}{fermé} si $A = \Adh A$.
\end{definition}

\begin{lemma}[Caractérisation équivalente de la frontière]      \label{LEMooEUYEooYcUfKr}
    Soient \( X\) un espace topologique et \( S\subset X\). Un point \( x\in X\) est dans \( \partial S\) si et seulement si tout voisinage de \( x\) contient un point de \( S\) et un point de \( S^c\).
\end{lemma}

\begin{proof}
    Supposons que tout voisinage de \( x\) contienne un point de \( S\) et un point de \( S^c\). Alors \( x\in Adh(S)\) (définition~\ref{DEFooSVWMooLpAVZR}), mais pas dans l'intérieur de \( S\) parce que \( x\) ne possède pas de voisinage contenu dans \( S\). Donc \( x\in \partial S\).

    À l'inverse, si \( x\in\partial S\) alors \( x\) est dans l'adhérence de \( S\) et tout voisinage de \( x\) contient un point de \( S\). Mais \( x\) n'est pas dans l'intérieur de \( S\) et tout voisinage de \( x\) contient un point qui n'est pas dans \( S\), aka un point de \( S^c\).
\end{proof}

\begin{corollary}
    Un ensemble et son complémentaire ont même frontière.
\end{corollary}

\begin{proof}
    Conséquence du lemme~\ref{LEMooEUYEooYcUfKr}. Les points de \( \partial(S^c)\) sont caractérisés par le fait que tout voisinage contient un point de \( S^c\) et un point de \( (S^c)^c=S\).
\end{proof}

\begin{example}
    Soit \( X=\mathopen[ 0 , 1 \mathclose]\) muni de la topologie de la distance \( | x-y |\) (définition~\ref{ThoORdLYUu}). Les points \( 0\) et \( 1\) \emph{ne sont pas} dans la frontière de $X$. En effet une boule ouverte autour de \( 1\) est un ensemble de la forme
    \begin{equation}
        B(1,r)=\{ x\in X\tq | x-1 |<r \}=\mathopen] 1-r , 1 \mathclose]
    \end{equation}
    où nous avons supposé \( r<1\).

    Les points \( 0\) et \( 1\) sont par contre sur la frontière de \( \mathopen[ 0 , 1 \mathclose]\) lorsque cet ensemble est vu comme partie de l'espace métrique \( \eR\).
\end{example}

\begin{lemma}[Passage de douane\cite{ooDKEWooFqlDyN,ooWBUCooAdPjMK}]        \label{LEMooLKWEooItGnkP}
    Dans un espace topologique, toute partie connexe qui rencontre à la fois une partie \( A\) et son complémentaire rencontre nécessairement la frontière de \( A\).
\end{lemma}

\begin{proof}
    Nommons \( \gamma\) la partie connexe qui intersecte \( A\) et \( A^c\). Les ouverts \( \Int(A)\) et \( X\setminus \bar A\) ne peuvent pas recouvrir \( \gamma\) parce que ce sont deux ouverts disjoints alors que \( \gamma\) est connexe (voir la définition~\ref{DefIRKNooJJlmiD} de la connexité). Donc \( \gamma\) doit contenir des points qui sont dans \( \bar A\) mais pas dans \( \Int(A)\). C'est-à-dire des points de \( \partial A\).
\end{proof}

On vérifiera que les notations et les dénominations sont cohérentes en prouvant la proposition suivante.
\begin{proposition}Pour $\epsilon > 0$,
  \begin{enumerate}
  \item l'adhérence de $B(x,\epsilon)$ est $\bar B(x,\epsilon)$,
  \item l'intérieur de $\bar B(x,\epsilon)$ est $B(x,\epsilon)$,
  \item la boule ouverte $B(x,\epsilon)$ est un ouvert,
  \item la boule fermée $\bar B(x,\epsilon)$ est un fermé.
  \end{enumerate}
\end{proposition}

Nous avons également les liens suivants entre intérieur, adhérence, ouvert, fermé et passage au complémentaire (noté ${}^c$)~:
\begin{proposition}
Si $A \subset \eR^n$ et $A^c = \eR^n\setminus A$, nous
  avons
  \begin{enumerate}
  \item $(\Int A)^c = \Adh (A^c)$ et $(\Adh A)^c = \Int
    (A^c)$,
  \item $A$ est ouvert si et seulement si $A^c$ est fermé,
  \item $\Int A$ est le plus grand ouvert contenu dans $A$,
  \item $\Adh A$ est le plus petit fermé contenant $A$,
  \end{enumerate}
\end{proposition}

\begin{example} \label{ExBFLooUNyvbw}
    Il n'est en général pas vrai que \( \overline{ A\cap B }=\bar A\cap \bar B\). Par exemple si \( A=\mathopen[ 0 , 1 [\) et \( B=\mathopen] 1 , 2 \mathclose]\). Dans ce cas, \( A\cap B=\emptyset\) alors que \( \bar A\cap\bar B=\{ 1 \}\).
\end{example}

%---------------------------------------------------------------------------------------------------------------------------
\subsection{Point d'accumulation, point isolé}
%---------------------------------------------------------------------------------------------------------------------------

Les définitions de point d'accumulation et de point isolé sont \ref{DEFooGHUUooZKTJRi} et \ref{DEFooXIOWooWUKJhN}. Nous voyons maintenant ce que ces définitions donnent dans le cas de l'espace topologique \( \eR\).

\begin{lemma}
    Soit $D\subset\eR$. Un point $a\in D$ est isolé dans $D$ si et seulement si il existe $\varepsilon>0$ tel que
    \begin{equation}
        \mathopen[ a-\varepsilon , a+\varepsilon \mathclose]\cap D=\{ a \}.
    \end{equation}
    Autrement dit, il existe un intervalle autour de $a$ dans lequel $a$ est le seul élément de $D$.
\end{lemma}

\begin{lemma}
    Un point $a\in \eR$ est un point d'accumulation de $D$ si pour tout $\varepsilon>0$,
    \begin{equation}
        \Big( \mathopen[ a-\varepsilon , a+\varepsilon \mathclose]\setminus\{ a \} \Big)\cap D\neq\emptyset.
    \end{equation}
    Autrement dit, quel que soit l'intervalle autour de  $a$ que l'on considère, le point $a$ n'est pas tout seul dans $D$.
\end{lemma}

\begin{example}
	Prenons $D=\mathopen[ 0 , 1 [\cup\mathopen] 2 , 3 \mathclose]$. Cet ensemble n'a pas de point isolé, et l'ensemble de ses points d'accumulation est $\mathopen[ 0 , 1 \mathclose]\cup\mathopen[ 2,3  \mathclose]$.

	Notez que les points $1$ et $2$ sont des points d'accumulation de $D$ qui ne font pas partie de $D$. Il est possible d'être un point d'accumulation de $D$ sans être dans $D$, mais pour être un point isolé dans $D$, il faut être dans $D$.
\end{example}

\begin{example}
	Soit $D=\{ \frac{1}{ n }\}_{n\in\eN}$. Tous les points de cet ensemble sont des points isolés (vérifier !).  Aucun point de $D$ n'est point d'accumulation. Cependant $0$ est un point d'accumulation.
\end{example}

\begin{example}     \label{EXooWOYQooJolaTV}
    Soit \( D=\mathopen] 1 , 2 \mathclose[\cup\{ 12 \}\). Le point \( 12\) est adhérence, mais pas d'accumulation parce que le voisinage \( \mathopen] 9 , 14 \mathclose[\) n'intersectionne pas \( D\setminus \{ 12 \}\).
\end{example}

%---------------------------------------------------------------------------------------------------------------------------
\subsection{Limite de suite}
%---------------------------------------------------------------------------------------------------------------------------

\begin{definition}[Limite d'une suite dans $\eR^m$]
	Une suite de points $(x_n)$ dans $\eR^m$ est dite \defe{convergente}{convergence!suite dans $\eR^m$} s'il existe un élément $\ell\in\eR^m$ tel que
	\begin{equation}	\label{EqCondLimSuite}
		\forall\varepsilon>0,\,\exists N\in \eN\tq\,\forall n\geq N,\,\| x_n-\ell \|<\varepsilon.
	\end{equation}
	Dans ce cas, nous disons que $\ell$ est la \defe{limite}{limite!suite dans $\eR^m$} de la suite $(x_n)$ et nous écrivons $\lim x_n=\ell$ ou plus simplement $x_n\to \ell$.
\end{definition}
Notez aussi la similarité avec la définition~\ref{PropLimiteSuiteNum}.

\begin{remark}
	Nous n'écrivons pas «$\lim_{n\to\infty}x_n$» parce que, lorsqu'on parle de suites, la limite est \emph{toujours} lorsque $n$ tend vers l'infini. Il n'y a aucun intérêt à chercher par exemple $\lim_{n\to 4}x_n$ parce que cela vaudrait $x_4$ et rien d'autre.

	Ceci est une différence importante avec les limites de fonctions.
\end{remark}

\begin{lemma}[Unicité de la limite]
	Il ne peut pas y avoir deux nombres différents qui satisfont à la condition \eqref{EqCondLimSuite}. En d'autres termes, si $\ell$ et $\ell'$ sont deux limites de la suite $(x_n)$, alors $\ell=\ell'$.
\end{lemma}

\begin{proof}
	Soit $\varepsilon>0$. Nous considérons $N$ tel que
	\begin{equation}
		\| x_n-\ell \|<\varepsilon
	\end{equation}
	pour tout $n\geq N$, et $N'>0$ tel que
	\begin{equation}
		\| x_n-\ell' \|<\epsilon
	\end{equation}
	pour tout $n>N'$. Maintenant, nous prenons $n$ plus grand que $N$ et $N'$ de telle façon que les deux équations pour $x_n$ soient vérifiées en même temps. Alors
	\begin{equation}
		\| \ell-\ell' \|=\| \ell-x_n+x_n-\ell' \|\leq\| \ell-x_n \|+\| x_n-\ell' \|<2\varepsilon.
	\end{equation}
	Cela prouve que $\| \ell-\ell' \|=0$.
\end{proof}
Le théorème de Bolzano-Weierstrass~\ref{ThoBWFTXAZNH} dit que dans le cas métrique, la compacité séquentielle est équivalente à la compacité.

%TODO : le théorème sur l'équivalence des normes sur les espaces vectoriels normés devrait être énoncé comme le fait que si N1 et N2 sont deux normes sur V, alors
%       nous avons un isomorphisme d'espace topologique (V,N1) ~ (V,N2). L'isomorphisme étant donné par l'identité.

% This is part of Mes notes de mathématique
% Copyright (c) 2008-2019
%   Laurent Claessens, Carlotta Donadello
% See the file fdl-1.3.txt for copying conditions.

%+++++++++++++++++++++++++++++++++++++++++++++++++++++++++++++++++++++++++++++++++++++++++++++++++++++++++++++++++++++++++++
\section{Topologie et distance}
%+++++++++++++++++++++++++++++++++++++++++++++++++++++++++++++++++++++++++++++++++++++++++++++++++++++++++++++++++++++++++++


\begin{lemma}   \label{LemDUJXooWsnmpL}
    Soient \( (X_1,d_1)\) et \( (X_2,d_2)\) des espaces métriques séparables. Alors \( X_1\times X_2\) admet une base dénombrable de topologie constituée de produits de boules de \( X_1\) par des boules de \( X_2\). Plus précisément si $A_i$ est dénombrable et dense dans \( X_i\) alors l'ensemble des produits
    \begin{equation}
        \big\{ B(y_1,r_1)\times B(y_2,r_2)\big\}_{\substack{y_i\in A_i\\r_i\in \eQ^+}}
    \end{equation}
    est une base de topologie pour \( X_1\times X_2\).
\end{lemma}

\begin{proof}
    Soit \( \mO\) un ouvert de \( X_1\times X_2\) et \( (x_1,x_2)\in \mO\). Par définition de la topologie produit\footnote{Définition~\ref{DefIINHooAAjTdY}.}, il existe \( r_1,r_2\in \eQ^+\) tels que \( B(x_1,r_1)\times B(x_2,r_2)\subset\mO\). Les parties \( A_i\) étant denses, il existe \( y_i\in B(x_i,r_i/2)\cap A_i\). Avec ces choix nous avons $x_i\in B(y_i,\frac{ r_i }{2})$. Nous avons donc
    \begin{equation}
        (x_1,x_2)\in B(y_1,\frac{ r_1 }{ 2 })\times B(y_2,\frac{ r_2 }{2}).
    \end{equation}
    Il est facile de voir que \( B(y_i,r_i/2)\subset B(x_i,r_i)\). En effet si \( z_i\in B(y_i,r_i/2)\) alors
    \begin{equation}
        d_i(z_i,x_i)\leq d(z_i,y_i)+d(y_i,x_i)\leq \frac{ r_i }{2}+\frac{ r_i }{2}=r_i.
    \end{equation}
    Au final,
    \begin{equation}
        (x_1,x_2)\in B(y_1,\frac{ r_1 }{ 2 })\times B(y_2,\frac{ r_2 }{2})\subset \mO.
    \end{equation}
\end{proof}


\begin{definition}
    Si \( (X,d_X)\) et \( (Y,d_Y)\) sont des espaces métriques, une \defe{isométrie}{isométrie d'espaces métriques} est une application bijective \( f\colon X\to Y\) telle que pour tout \( x,y\in X\) nous ayons
    \begin{equation}        \label{EQooVUOXooKJntMN}
        d_Y\big( f(x),f(y) \big)=d_X(x,y).
    \end{equation}
\end{definition}

\begin{remark}
    Une application vérifiant \eqref{EQooVUOXooKJntMN} est automatiquement injective. En pratique, il ne faut donc vérifier que la surjectivité.
\end{remark}

\begin{example}[Manque de surjectivité]
    Si \( X=\mathopen[ 0 , \infty \mathclose[\) et \( f(x)=x+1\) alors \( f\) vérifie \eqref{EQooVUOXooKJntMN} pour la distance \( d(x,y)=| x-y |\), mais n'est pas surjective.
\end{example}

\begin{propositionDef}[Groupe des isométries]
    Si \( (X,d)\) est un espace métrique,
    \begin{enumerate}
        \item
            l'ensemble des isométries de \( X\), noté \( \Isom(X)\)\nomenclature[Y]{$\Isom(X)$}{Le groupe des isométries de \( X\)} est un groupe pour la composition\index{isométrie!groupe}\index{groupe!des isométries!espace métrique}.
        \item
            Ce groupe agit fidèlement\footnote{Si vous ne savez pas ce que c'est, alors vous avez zappé la définition~\ref{DefuyYJRh}.} sur \( X\).
    \end{enumerate}
\end{propositionDef}
\begin{proposition}\label{PropLYMgVMJ}
    Une isométrie entre deux espaces métriques est continue.
\end{proposition}

\begin{proof}
    Soient \( f\colon X\to Y\) une application isométrique et \( \mO\) un ouvert de \( Y\). Soit \( a\in f^{-1}(\mO)\); si \( d(a,b)<r\), alors \( d\big( f(a),f(b) \big)<r\) et donc \( b\in f^{-1}\big( B(f(a),r) \big)\). Donc autour de chaque point de \( f^{-1}(\mO)\) nous pouvons trouver une boule ouverte contenue dans \( f^{-1}(\mO)\), ce qui prouve que \( f^{-1}(\mO)\) est ouvert.
\end{proof}

\begin{example}
    Si \( X\) est un ensemble, nous pouvons écrire la \defe{distance discrète}{distance discrète} :
    \begin{equation}
        d(x,y)=\begin{cases}
            0    &   \text{si } x=y\\
            1    &    \text{si } x\neq y\text{.}
        \end{cases}
    \end{equation}
    La topologie résultante est la topologie discrète, côtoyée dans l'exemple~\ref{DefTopologieDiscrete}\footnote{Vérifiez-le tout de même!}.

    Pour cette métrique, le groupe des isométries est le groupe symétrique de \( X\), c'est-à-dire le groupe de toutes les bijections de \( X\) sur lui-même.
\end{example}

\subsubsection{Distance point-ensemble}
%////////////////////////

\begin{definition}
	Si $A$ est une partie de l'espace métrique $(X,d)$ et si $x\in X$, nous disons que la \defe{distance}{distance!point et ensemble} entre $A$ et $x$ est le nombre
	\begin{equation}		\label{EqdefDistaA}
		d(x,A)=\inf_{a\in A}d(x,a).
	\end{equation}
\end{definition}
%The result is on the figure~\ref{LabelFigDistanceEnsemble}
\newcommand{\CaptionFigDistanceEnsemble}{La distance entre $x$ et $A$ est donnée par la distance entre $x$ et $p$. Les distances entre $x$ et les autres points de $A$ sont plus grandes que $d(x,p)$.}
\input{auto/pictures_tex/Fig_DistanceEnsemble.pstricks}

%---------------------------------------------------------------------------------------------------------------------------
\subsection{Suites et espaces métriques}
%---------------------------------------------------------------------------------------------------------------------------

%TODO : il y a un contre-exemple à faire à la page http://www.les-mathematiques.net/phorum/read.php?14,787368,787582

\begin{proposition}[Caractérisation séquentielle de la limite\cite{MonCerveau}]     \label{PROPooJYOOooZWocoq}
    Soient deux espaces métriques \( X\) et \( Y\) ainsi qu'une fonction \( f\colon X\to Y\). Soit \( a\in X\) et \( \ell\in Y\). On a
    \begin{equation}\label{EqLimooJYOOooZWocoqG}
        \lim_{x\to a} f(x)=\ell,
    \end{equation}
    si et seulement si, pour toute suite \( (x_k) \) telle que \( x_k \to a \), on a
    \begin{equation}\label{EqLimooJYOOooZWocoqS}
        \lim f(x_k)=\ell.
    \end{equation}
    Par ailleurs, l'une des deux limites existe si et seulement si l'autre existe.
\end{proposition}

\begin{proof}
  Le sens direct est la proposition~\ref{fContEstSeqCont}. Pour la réciproque, nous passons par la contraposée. C'est-à-dire que nous supposons que \( \ell\) n'est pas une limite de \( f\) pour \( x\to a\). Il existe un \( \epsilon\) tel que pour tout \( \delta\), il existe un \( x\) vérifiant \( d_X(x;a) <\delta\) et \( d_Y(f(x);\ell) >\epsilon\).

  Nous construisons à présent une suite de la manière suivante. Pour \( \delta=\frac{1}{ n }\) nous considérons \( x_n\) tel que \( d_X( x_n; a) <\delta\) et \( d_Y(f(x_n);\ell) > \epsilon \). Cette suite converge vers \( a\), mais la suite \( f(x_n)\) ne converge manifestement pas vers \( \ell\) : elle ne rentre jamais dans la boule \( B(\ell,\epsilon)\).
\end{proof}

Une fonction continue est séquentiellement continue. Dans les espaces métriques la proposition suivante montre que la réciproque est également vraie et la continuité est équivalente à la continuité séquentielle. Cela n'est cependant pas vrai pour n'importe quel espace topologique.

\begin{corollary}[Caractérisation séquentielle de la continuité en un point\cite{MonCerveau}]  \label{ItemWJHIooMdugfu}
    Si \( X\) et \( Y\) sont des espaces métriques, alors une fonction \( f\colon X\to Y\) est continue en un point si et seulement si elle est séquentiellement continue en ce point.
\end{corollary}

\begin{proof}
  Paraphrasons la preuve précédente. Nous supposons que \( X\) et \( Y\) sont métriques. Si \( f\) n'est pas continue en \( a\), il existe \( \epsilon>0\) tel que pour tout \( \delta>0\), il existe \( x\) tel que \( \| x-a \|\leq\delta\) et \( \| f(x)-f(a) \|>\epsilon\). Nous considérons un tel \( \epsilon\) et pour chaque \( n\geq1\in \eN\) nous considérons un \( x_n\) correspondant à \( \delta=\frac{1}{ n }\). Cela nous donne une suite \( x_n\to a\) dans \( X\) mais \( \| f(x_n) -f(a)\|\) reste plus grand que \( \epsilon\). Cela montre que \( f\) n'est pas non plus séquentiellement continue.
\end{proof}

Les espaces métriques ont une propriété importante que la \wikipedia{fr}{Espace_séquentiel}{fermeture séquentielle} est équivalente à la fermeture.
\begin{proposition}[Caractérisation séquentielle d'un fermé]    \label{PropLFBXIjt}
    Soient \( X\) un espace métrique et \( F\subset X\). L'ensemble \( F\) est fermé si et seulement si toute suite contenue dans \( F\) et convergeant dans \( X\) converge vers un élément de \( F\).
\end{proposition}
\index{fermeture séquentielle}
\index{séquentiellement fermé}

\begin{proof}
   Une suite contenue dans un fermé ne peut converger que vers un élément de ce fermé: c'était la proposition \ref{PROPooBBNSooCjrtRb}. Le point le plus important est donc l'autre sens: si toute suite d'éléments de \( F \) converge dans \( F \) alors \( F \) est fermé.
    
   Par contraposée, supposons que \( X\setminus F\) ne soit pas ouvert. Alors il existe \( x\in X\setminus F\) pour lequel tout voisinage intersecte \( F\). En prenant \( x_k\in B(x,\frac{1}{ k })\), nous construisons une suite contenue dans \( F\), convergeant vers \( x\) qui n'est pas dans \( F \).
\end{proof}


\begin{lemma}		\label{LemLimAbarA}
  Soit $X$ un espace métrique, et soit $(x_n)$ une suite convergente contenue dans un ensemble $A\subset X$. Alors la limite $x_n$ appartient à $\bar A$.
\end{lemma}
Ce lemme est précisément la version «espace métrique» du corolaire \ref{CorLimAbarA}; mais, donnons-en une preuve tout de même.
\begin{proof}
	Supposons que nous ayons une partie $A$ de $X$, et une suite $(x_n)$ dont la limite $\ell$ se trouve hors de $\bar A$. Dans ce cas, il existe un $r>0$ tel que\footnote{Une autre manière de dire la même chose : si $\ell\notin\bar A$, alors $d(\ell,A)>0$.} $B(\ell,r)\cap A=\emptyset$. Si tous les éléments $x_n$ de la suite sont dans $A$, il n'y en a donc aucun tel que $d(x_n,\ell)<r$. Cela contredit la notion de convergence $x_n\to \ell$.
\end{proof}

\begin{corollary}		\label{CorAdhEstLim}
  Soit $X$ un espace métrique, $A \subset X$ et $a \in \bar A$. Alors il existe une suite d'éléments dans $A$ qui converge vers $a$.
\end{corollary}

\begin{proof}
  Si $a\in A$, alors nous pouvons prendre la suite constante $x_n=a$. Si $a$ n'est pas dans $A$, alors $a$ est dans $\partial A$, et pour tout $n$, il existe un point de $A$ dans la boule $B(a,\frac{1}{ n })$. Si nous nommons $x_n$ ce point, la suite ainsi construite est une suite contenue dans $A$ et qui converge vers $a$ (ce dernier point est laissé à la sagacité du lecteur ou de la lectrice).
\end{proof}

En termes savants, ce corolaire signifie que la fermeture $\bar A$ est composé de $A$ plus de toutes les limites de toutes les suites contenues dans $A$.

\begin{proposition}[Caractérisation séquentielle de la continuité\cite{MonCerveau}]     \label{PropXIAQSXr}
    Soient \( X\) et \( Y\) deux espaces métriques. Une application \( f\colon X\to Y\) est continue sur \( X\) si et seulement si elle est séquentiellement.
\end{proposition}

\begin{proof}
    Le sens direct est déjà prouvé dans la proposition \ref{fContEstSeqCont}. Nous nous concentrons donc sur la réciproque.

    Soit \( \mO\) un ouvert de \( Y\); nous allons voir que le complémentaire de \( f^{-1}(\mO)\) est fermé dans \( E\). Pour cela nous considérons une suite convergente \( x_k\stackrel{E}{\longrightarrow} x\) avec \( x_k\in\complement f^{-1}(\mO)\) pour tout \( k\). Nous allons montrer que \( x\in \complement f^{-1}(\mO)\) et la caractérisation séquentielle\footnote{Proposition~\ref{PropLFBXIjt}.} de la fermeture conclura que \( \complement f^{-1}(\mO)\) est fermé.

    Pour tout \( k\), nous avons \( f(x_k)\in\complement \mO\), mais \( \mO\) est ouvert et \( f(x_k)\stackrel{Y}{\longrightarrow}f(x)\) parce que \( f\) est séquentiellement continue. Par conséquent \( f(x)\in\complement \mO\) et \( x\in\complement f^{-1}(\mO)\).
\end{proof}

%TODO : il y a ici trois théorèmes sur la continuité séquentielle. Il faut sans doute les fusionner.

\begin{proposition} \label{PropCJGIooZNpnGF}
    Si \( X\) et \( Y\) sont deux espaces métriques et \( f,g\colon X\to Y\) sont deux fonctions continues égales sur une partie dense de \( X\) alors \( f=g\).
\end{proposition}
\index{fonction!continue!égales}

\begin{proof}
    Les fonctions \( f\) et \( g\) sont séquentiellement continues (proposition~\ref{PropFnContParSuite}, ou proposition \ref{ItemWJHIooMdugfu}). Soient \( A\) un ensemble dense dans \( X\) sur lequel \( f\) et \( g\) sont égales, et \( x\notin A\). Vu que \( A\) est dense, il existe une suite \( a_n\) dans \( A\) telle que \( a_n\to x\). La séquentielle continuité de \( f\) et \( g\) donnent
    \begin{subequations}
        \begin{align}
            f(a_n)\to f(x)\\
            g(a_n)\to g(x),
        \end{align}
    \end{subequations}
    mais pour tout \( n\), \( f(a_n)=g(a_n)\). Par unicité de la limite\footnote{Proposition~\ref{PropFObayrf}.} dans \( Y\), \( f(x)=g(x)\).
\end{proof}

%--------------------------------------------------------------------------------------------------------------------------- 
\subsection{Espace métrisable}
%---------------------------------------------------------------------------------------------------------------------------

\begin{definition}[Espace vectoriel topologique métrisable\cite{ooOFEPooVFgTXm}]
    Un espace topologique est \defe{métrisable}{métrisable!espace vectoriel topologique} si il existe une distance compatible avec la topologie.
\end{definition}
\index{espace!vectoriel topologique!métrisable}

\begin{proposition}[\cite{ooCGEHooVTyTuY}]      \label{PROPooXWBTooCvGLOj}
    Soit un espace topologique métrisable \( X\).
    \begin{enumerate}
        \item   \label{ITEMooOXVRooBsKwuq}
            Tout fermé de \( X\) est une intersection dénombrable d'ouverts.
        \item
            Tout ouvert de \( X\) est une union dénombrable de fermés.
    \end{enumerate}
\end{proposition}

\begin{proof}
    Soit une métrique \( d\) compatible avec la topologie de \( X\) et un fermé \( A\). Nous posons
    \begin{equation}
        V_n=\{ x\in X\tq d(x,A)<\frac{1}{ n } \}.
    \end{equation}
    Et juste pour faire simple nous notons \( V_0=X\).
    \begin{subproof}
        \item[Les parties \( V_n\) sont ouvertes]
            Soit \( x\in V_n\). Trouvons un voisinage de \( x\) contenu dans \( V_n\) afin de pouvoir encore invoquer le théorème~\ref{ThoPartieOUvpartouv}. D'abord, vu que \( x\in V_n\), il existe \( a\in A\) tel que \( d(x,a)<\frac{ 1 }{ n }\) (ici les inégalités strictes sont importantes).

            Soient \( \epsilon>0\) que nous fixerons plus bas, et \( y\in B(x,\epsilon)\). L'inégalité triangulaire donne
            \begin{equation}
                d(y,a)\leq d(y,x)+d(x,a)<\epsilon+\frac{1}{ n }.
            \end{equation}
            Nous pouvons donc choisir \( \epsilon\) de telle sorte que \( d(y,a)<1/n\). Avec ce \( \epsilon\), nous avons, pour tout \( y\in B(x,\epsilon)\) :
            \begin{equation}
                d(y,A)\leq d(y,a)<\frac{1}{ n }
            \end{equation}
            et donc \( y\in V_n\).
        \item[\( A\) est l'intersection des \( V_n\)]
            Nous avons évidemment \( A\subset V_n\) pour tout \( n\). Et d'autre part, si \( a\in\bigcap_{n\in \eN} V_n\) alors \( d(a,A)<\frac{1}{ n }\) pour tout \( n\). Cela implique \( d(a,A)=0\), et donc \( a\in A\) par le lemme \ref{LEMooAIARooQADaxM}.
        \end{subproof}

        Ceci démontre le point \ref{ITEMooOXVRooBsKwuq}.

    En ce qui concerne la seconde partie, nous appliquons la première partie au complémentaire. Si \( \mO\) est ouvert, \( \mO^c\) est fermé et
    \begin{equation}
        \mO^c=\bigcap_{n\in \eN}V_n,
    \end{equation}
    ce qui donne immédiatement
    \begin{equation}
        \mO=\bigcup_{n\in \eN}V_n^c
    \end{equation}
    où les \( V_n^c\) sont fermés.
\end{proof}

\begin{corollary}       \label{CORooTWFYooCNMieM}
    Si \( X\) est un espace topologique métrisable, alors \( X\) accepte une base dénombrable de topologie autour de chaque point.
\end{corollary}

\begin{proof}
    Il s'agit seulement de remarquer que les singletons sont fermés et d'appliquer la proposition~\ref{PROPooXWBTooCvGLOj}.
\end{proof}

%+++++++++++++++++++++++++++++++++++++++++++++++++++++++++++++++++++++++++++++++++++++++++++++++++++++++++++++++++++++++++++
\section{Suites de Cauchy, métrique et espaces complets}
%+++++++++++++++++++++++++++++++++++++++++++++++++++++++++++++++++++++++++++++++++++++++++++++++++++++++++++++++++++++++++++

%---------------------------------------------------------------------------------------------------------------------------
\subsection{Généralités}
%---------------------------------------------------------------------------------------------------------------------------

\begin{definition}[Suite de \( \tau\)-Cauchy, espace vectoriel topologique\cite{TQSWRiz,ooMKWJooLSkGfh}]   \label{DefZSnlbPc}
    Soit \( E\) un espace vectoriel topologique. Une suite \( (x_k)\) dans \( E\) est une \defe{suite \( \tau\)-Cauchy}{suite!de Cauchy} si pour tout voisinage \( \mU\) de \( 0\) il existe \( N\in \eN\) tel que \( x_k-x_l\in\mU\) pour tout \( k,l\geq N\).
\end{definition}

\begin{definition}[Espace \( \tau\)-complet]      \label{DEFooVQDBooNxprFU}
    Nous disons qu'une partie \( A\) d'un espace vectoriel topologique est \defe{\( \tau\)-complet}{complet!espace topologique} si toute suite \(  \tau\)-Cauchy d'éléments de \( A\) converge\footnote{Définition~\ref{DefXSnbhZX}.} vers un élément de \( A\).
\end{definition}

\begin{definition}[Suite de Cauchy, espace métrique]      \label{DEFooXOYSooSPTRTn}
    Une suite \( (a_k)\) dans un espace métrique \( (V,d)\) est \defe{de Cauchy}{suite!de Cauchy} si pour tout \( \epsilon\in \eR\), il existe \( N\) tel que si \( n,m\geq N\) alors \( d(a_n,a_m)<\epsilon\).
\end{definition}

Notons qu'ici, même si l'espace \( V\) n'a rien à voir avec \( \eR\), nous prenons \( \epsilon\) dans \( \eR\) et la distance à valeurs dans \( \eR\). Cela semble une évidence, mais il faut se rendre compte que \( \eR\) commence à prendre une place centrale dans nos constructions. Ce n'était pas le cas du temps où nous parlions de suites de Cauchy et de complétude dans des corps totalement ordonnés (définitions~\ref{DefKCGBooLRNdJf}). Dans ce contexte, le \( \epsilon\) était pris dans le corps lui-même.

\begin{definition}[Métrique complète]       \label{DEFooHBAVooKmqerL}
    Soit \( (E,d)\) un espace métrique. Nous disons que la métrique \( d\) est \defe{complète}{complet!métrique} si toute suite de Cauchy dans \( (E,d)\) converge dans \( E\).
\end{definition}

\begin{normaltext}
    Ces définitions méritent quelques remarques.
    \begin{enumerate}
        \item
            Dans le cas des espaces vectoriels topologiques, nous définissons les notions de suite \( \tau\)-Cauchy et d'espace topologique \( \tau\)-complet. Nous ajoutons le préfixe \( \tau\) pour indiquer que ce sont des notions topologiques.
        \item
            Dans le cas des espaces métriques, nous définissons la notion de \emph{métrique} complète. C'est bien la métrique qui est complète, et non l'espace. En effet nous allons voir dans l'exemple \ref{EXooNMNVooXyJSDm} que le même espace topologique peut accepter plusieurs distances différentes (donnant la même topologie) donnant lieu à des suites de Cauchy différentes.
        \item
            Si un espace vectoriel a une topologie issue d'une distance, rien ne dit que ses suites \( \tau\)-Cauchy et ses suites de Cauchy sont les mêmes. Ce sont deux notions à priori séparées. Si \( V\) est un espace vectoriel topologique que l'on peut munir de deux distances \( d_1, d_2\) donnant toutes deux la topologie, dire que \( V\) est \( \tau\)-complet, dire que \( d_1\) est complète et dire que \( d_2\) est complète sont trois choses différentes. Même si les trois topologies sont identiques.
        \item
            Nous allons bien entendu voir que dans de larges gammes d'exemples, les notions de suite de Cauchy et \( \tau\)-Cauchy coincident.
    \end{enumerate}
\end{normaltext}

\begin{definition}  \label{DefVKuyYpQ}
    Un \defe{espace de Banach}{espace!Banach}\index{Banach!espace} est un espace vectoriel normé complet\footnote{Définition \ref{DEFooHBAVooKmqerL}.} pour la topologie de la norme.
\end{definition}

\begin{example}[La complétude n'est pas une propriété topologique\cite{ooSCDYooWutzzr}]     \label{EXooNMNVooXyJSDm}
    Le fait pour un espace d'être complet n'est pas une propriété topologique, mais une propriété métrique. Plus exactement, il existe des espaces topologiques isomorphes, mais dont l'un est complet et l'autre non.

    Nous considérons la distance suivante sur \( \eN\) :
    \begin{equation}
        d_1(x,y)=\Bigl| \frac{1}{ x }-\frac{1}{ y } \Bigr|.
    \end{equation}
    Pour vérifier que cette formule définit bien une distance (définition~\ref{DefMVNVFsX}), le seul point non immédiat est l'inégalité triangulaire :
    \begin{equation}
        d_1(x,y)=\Bigl| \frac{1}{ x }-\frac{1}{ y } \Bigr|\leq\Bigl| \frac{1}{ x }-\frac{1}{ z } \Bigr|+\Bigl| \frac{1}{ z }-\frac{1}{ y } \Bigr|=d_1(x,z)+d_1(z,y).
    \end{equation}

    Au niveau de la topologie induite par cette distance, c'est la topologie discrète. En effet, soit \( x\in \eN\) et \( \epsilon>0\); nous voulons déterminer la boule \( B(x,\epsilon)\) en résolvant l'équation
    \begin{equation}
        \Bigl| \frac{1}{ x }-\frac{1}{ y } \Bigr|<\epsilon
    \end{equation}
    pour \( y\in \eN\). Nous trouvons que $\frac 1 y > \frac 1 x  - \epsilon$ et $\frac 1 y < \frac 1 x + \epsilon$, soit
    \begin{subequations}
        \begin{numcases}{}
            y > \frac 1 {\frac 1 x  + \epsilon}\\
            y < \frac 1 {\frac 1 x - \epsilon}.
        \end{numcases}
    \end{subequations}
    Si \( \epsilon \) est assez petit, la seule solution entière est \( y=x\). Les ouverts sont donc toutes les parties parce que tous les singletons sont ouverts.

    L'espace topologique associé à \( (\eN,d_1)\) est donc la topologie discrète\footnote{Celle dont toutes les parties sont des ouverts.}.

    Si nous considérons par contre la distance usuelle sur \( \eN\), à savoir \( d(x,y)=| x-y |\), nous obtenons encore la topologie discrète. Nous avons donc un isomorphisme d'espaces topologiques
    \begin{equation}
        (\eN,d)\simeq (\eN,d_1).
    \end{equation}
    Nous pouvons même donner un isomorphisme explicite : \( f(n)=n\).

    La suite \( (x_n)=n\) est une suite de Cauchy dans \( (\eN,d_1)\) parce que si \( \epsilon>0\) est donné, il suffit de prendre \( N\) assez grand pour avoir \( \frac{1}{ N }<\epsilon\) (possible par le lemme~\ref{LemooHLHTooTyCZYL}) nous avons, pour \( n,m>N\) :
    \begin{equation}
        \Bigl| \frac{1}{ n }-\frac{1}{ m } \Bigr|<\frac{1}{ n }<\frac{1}{ N }<\epsilon.
    \end{equation}
    Or cette suite ne converge pas. Soit en effet un candidat limite \( k\). Calculons
    \begin{equation}
        d_1(x_n,k)= \Bigl| \frac{1}{ n }-\frac{1}{ k } \Bigr |\to \frac{1}{ k }\neq 0.
    \end{equation}
    L'espace \( (\eN,d_1)\) n'est pas complet.

    Notons que cette suite n'est pas de Cauchy dans \( (\eN,d)\).

    En résumé :
    \begin{enumerate}
        \item
            Les espaces topologiques \( (\eN,d)\) et \( (\eN,d_1)\) sont isomorphes.
        \item
            Ils ont les mêmes notions de suites convergentes : une suite convergente pour l'un est convergente pour l'autre.
        \item
            Ils n'ont pas les mêmes notions de suites de Cauchy.
        \item
            Dans \(  (\eN,d_1)  \), il existe des suites de Cauchy qui ne convergent pas (pas complet).
        \item
            L'espace \( (\eN,d)\) est complet, mais \( (\eN,d_1)\) n'est pas complet.
        \item
            Le fait pour un espace topologique métrique d'être complet n'est pas intrinsèque à sa topologie : la complétude est une propriété de la distance. La complétude est une propriété de la métrique, et non de la topologie qui s'en suit.
    \end{enumerate}
\end{example}

%---------------------------------------------------------------------------------------------------------------------------
\subsection{Espace topologique métrique}
%---------------------------------------------------------------------------------------------------------------------------

Dans les espaces vectoriels topologiques métriques, il n'y a pas d'ambiguïté.
\begin{proposition}[Caractérisations avec la distance \( d \)]     \label{PropooUEEOooLeIImr}
    Soit \( (E,d)\) un espace vectoriel topologique métrique.
    \begin{enumerate}
        \item   \label{ItemooROYMooAQCXnj}
            Une suite \( (x_n)\) dans \( E\) est convergente\footnote{Définition~\ref{DefXSnbhZX}.} vers \( x\) si et seulement si pour tout \( \epsilon\in \eR\) il existe \( N_{\epsilon}\) tel que pour tout \( n\geq N_{\epsilon}\) nous avons \( d(x_n,x)\leq \epsilon\).
        \item
            Une suite \( (x_n)\) dans \( E\) est de Cauchy\footnote{Définition~\ref{DefZSnlbPc}.} si pour tout \( \epsilon\in \eR\), il existe un \( N_{\epsilon}\) tel que si \( p,q\geq N_{\epsilon}\), nous avons \( d(x_p,x_q)\leq \epsilon\).
    \end{enumerate}
\end{proposition}

\begin{proof}
   En ce qui concerne la convergence :
    \begin{subproof}
        \item[Sens direct]

            Nous supposons que \( x_k\to x\) dans \( E\). Soit \( \epsilon>0\); vu que \( B(x,\epsilon)\) est un ouvert contenant \( x\), il existe un \( N_{\epsilon}>0 \) tel que \( k>N_{\epsilon}\) implique \( x_k\in B(x,\epsilon)\). Cela signifie \( d(x,x_k)\leq \epsilon\).

        \item[Réciproque]

            Nous supposons que pour tout \( \epsilon>0\), il existe \( N_{\epsilon}>0\) tel que si \( k>N_{\epsilon}\) alors \( x_k\in B(x,\epsilon)\). Soit un ouvert \( \mO\) autour de \( x\). Nous sommes dans un espace métrique; ergo la topologie est donné par le théorème~\ref{ThoORdLYUu} et en particulier la liste des ouverts est donnée par \eqref{EqGDVVooDZfwSf}. Il existe donc une boule \( B(x,\epsilon)\) incluse à \( \mO\). Pour tout \( k>N_{\epsilon}\) nous avons alors \( x_k\in B(x,\epsilon)\subset\mO\).
    \end{subproof}
    En ce qui concerne les suites de Cauchy :
    \begin{subproof}
    \item[Sens direct]
        Si \( (x_n)\) est une suite de Cauchy et si \( \epsilon>0\) est donné, alors \( B(0,\epsilon)\) est un voisinage de \( 0\) et il existe \( N_{\epsilon}\) tel que si \( p,q\geq N_{\epsilon}\) alors \( x_p-x_q\in B(0,\epsilon)\). Posons \( u=x_p-x_q\); en utilisant l'invariance par translation (lemme~\ref{LEMooWGBJooYTDYIK}\ref{ITEMooLITDooPeReOk}) nous avons
        \begin{equation}
            d(u,0)=d(x_p-x_q,0)=d(x_p,x_q).
        \end{equation}
        Par conséquent \( d(x_p,x_q)\leq \epsilon\).
    \item[Réciproque]
        Soit \( \mO\) un voisinage de \( 0\). Il existe \( \epsilon\) tel que \( B(0,\epsilon)\subset \mO\). Par hypothèse il existe \( N_{\epsilon}\) tel que \( d(x_p,x_q)\leq \epsilon\) dès que \( p,q\geq N_{\epsilon}\). En utilisant encore l'invariance par translation nous avons
        \begin{equation}
            d(x_p,x_q)=d(x_p-x_q,0),
        \end{equation}
        et comme cela est plus petit que \( \epsilon\), nous avons \( x_p-x_q\in B(0,\epsilon)\subset\mO\).
    \end{subproof}
\end{proof}

\begin{proposition}[\cite{IRWFPQB}]     \label{PROPooZZNWooHghltd}
    Toute suite convergente dans un espace métrique est de Cauchy.
\end{proposition}

\begin{proof}
    Nous utilisons les caractérisations de la proposition~\ref{PropooUEEOooLeIImr} des suites convergentes et de Cauchy.

    Soit un espace métrique \( (X,d)\) et \( x_n\to\ell\) une suite convergente. Si \( \epsilon>0\), la proposition~\ref{PropooUEEOooLeIImr}\ref{ItemooROYMooAQCXnj}, dit qu'il existe \( N\) tel que pour tout \( n>N\) nous ayons \( d(x_n,\ell)<\epsilon\). Par conséquent si \( n,m>N\) alors
    \begin{equation}
        d(x_n,x_m)\leq d(x_m,\ell)+d(l,x_m)\leq 2\epsilon.
    \end{equation}
    Cela prouve que \( (x_n)\) est une suite de Cauchy.
\end{proof}

%+++++++++++++++++++++++++++++++++++++++++++++++++++++++++++++++++++++++++++++++++++++++++++++++++++++++++++++++++++++++++++
\section{Topologie et espace vectoriel}
%+++++++++++++++++++++++++++++++++++++++++++++++++++++++++++++++++++++++++++++++++++++++++++++++++++++++++++++++++++++++++++

%---------------------------------------------------------------------------------------------------------------------------
\subsection{Espace vectoriel topologique}
%---------------------------------------------------------------------------------------------------------------------------

\begin{definition}\label{DefEVTopologique}
  Un espace vectoriel \( V\) sur le corps \( \eK\) muni d'une topologie est un \defe{espace vectoriel topologique}{espace vectoriel!topologique} si
    \begin{enumerate}
        \item
            la somme de deux vecteurs est une application continue\footnote{Naturellement, l'espace \(V \times V \) est muni de la topologie produit.} \( V\times V\to V \); et
        \item
            la multiplication par un scalaire est une application continue\footnote{Naturellement, l'espace \(\eK \times V \) est muni (lui aussi) de la topologie produit.} \( \eK\times V\to V\).
    \end{enumerate}
\end{definition}
On le redit quand même: le corps\footnote{Définition~\ref{DefTMNooKXHUd}} lui-même doit avoir sa topologie. Dans la grande majorité des cas, ce corps est \( \eR\) ou \( \eC\) muni de la topologie usuelle.

Mine de rien, le fait que les deux opérations usuelles soient continues a de belles conséquences sur la topologie de l'espace\dots

\begin{proposition}[\cite{ooMKWJooLSkGfh}]
  Pour \(x \in V \) et \(\lambda \in \eK, \ \lambda \neq 0 \) fixés, les fonctions \( T_x \) et \( M_\lambda \) définies par:
  \begin{align}
    T_x:&V \to V & &\text{et}&M_\lambda:&V \to V\\
    & y \mapsto x+y & & & &y \mapsto \lambda y
  \end{align}
sont des homéomorphismes de \(V \) dans \(V \).
\end{proposition}

\begin{proof}
  Ce sont des bijections continues, dont les inverses sont respectivement \( T_{-x} \) et \( M_{1/\lambda} \).
\end{proof}

\begin{corollary}[Invariance de la topologie~\cite{ooMKWJooLSkGfh}]\label{PropInvarianceTopologie}
  Toute base de voisinage de \( 0 \) se transporte en tout point de l'espace vectoriel topologique.
\end{corollary}

\begin{lemma}[\cite{ooMKWJooLSkGfh}]\label{PropSommeTopologique}
  Soit \( V \) un espace vectoriel topologique, et \( W \) un voisinage de \( 0 \). Il existe \( U \) un voisinage de \( 0 \), symétrique\footnote{C'est-à-dire que, pour tout \( x \in V \), on a \( x \in U \) si et seulement si \( -x \in U \).}, tel que \( U + U = W \).
\end{lemma}

\begin{proof}
  Par continuité de l'addition et par la définition de la topologie produit, il existe \(U_1 \) et \(U_2 \) tels que \( U_1 + U_2 \subset W \). En posant \( U = U_1 \cap U_2 \cap (-U_1) \cap (-U_2) \), on a un sous-ensemble symétrique de \( U_1\) et \(U_2\), si bien que \( U + U = W \).
\end{proof}

\begin{definition}      \label{DEFooGTOZooRcvGHg}
    Une distance \( d\) sur un espace vectoriel topologique \( V\) est dite \defe{compatible}{distance!compatible} avec la topologie si la topologie induite\footnote{Définition~\ref{ThoORdLYUu}.} de \( d\) est celle de \( V\).

    Une distance \( d\) sur un espace vectoriel \( V\) est dite \defe{invariante}{distance!invariante} si pour tout \( x,y,u\in V\) nous avons
    \begin{equation}
        d(x+u,y+u)=d(x,y).
    \end{equation}
\end{definition}
Notons que lorsque nous parlons d'une distance compatible avec un espace vectoriel topologique, nous parlons de compatibilité avec la topologie, pas avec la structure vectorielle.

\begin{theorem}[\cite{ooMKWJooLSkGfh}]      \label{THOooAGBXooZnvQLK}
    Si $V$ est un espace vectoriel topologique possédant en tout point une base de topologie dénombrable, alors il existe une distance \( d\) sur \( V\) telle que
    \begin{enumerate}
        \item
            \( d\) est compatible avec la topologie de \( V\),
        \item
            \( d\) est invariante par translation.
    \end{enumerate}
\end{theorem}

\begin{proof}
Grâce à la proposition~\ref{PropInvarianceTopologie}, on peut tout ramener en \(0 \) puis faire les transports en tous les points de l'espace. Mieux: grâce à la proposition~\ref{PropSommeTopologique} (appliquée deux fois de suite), on peut créer une base de voisinage \( (U_n) \) de \( 0 \) telle que pour tout \(n \in \eN\),
\begin{equation}\label{EqBaseTopoMetriquePf1}
  U_{n+1} + U_{n+1} + U_{n+1} + U_{n+1} \subset U_n.
\end{equation}
Pour tous entiers naturels \(n\) et \(k\), on obtient alors
\begin{equation}\label{EqBaseTopoMetriquePf2}
  U_{n+1} + U_{n+2} + \cdots  + U_{n+(k-1)} + U_{n+k} \subset  U_{n+1} + U_{n+1} \subset U_n.
\end{equation}

On construit à présent, pour tout \( n \in \eN \), l'ensemble
\begin{equation}
  D_n = \Bigl\{\sum_{i=1}^n \frac {c_i} {2^i} \tq \forall i = 1,\cdots, n, c_i \in \{0,1\}\Bigr\},
\end{equation}
et \( D = \cup_{n>0} D_n\). Ensuite, définissons \(\phi \) sur \(D \cup \mathopen[1,+ \infty\mathclose[ \) et à valeurs dans les parties de \( V \):
\begin{equation}
  \phi (r) =
  \begin{cases}
    V & \text{si } r \geq 1;\\
    c_1 U_1 + \cdots + c_n U_n & \text{si } r \in D_n.
  \end{cases}
\end{equation}
Quelques remarques sur cette fonction.
\begin{enumerate}
  \item \emph{\(\phi(r) + \phi(s) \subset \phi(r+s)\) :} Si déjà \(r + s \geq 1 \), c'est clair. Sinon, on se place dans \( D_n \) avec le \(n \) qui va bien -- de telle sorte que \(r,\ s\) et \(r+s\) soient dedans. Notons:
    \begin{gather}
      r = \sum_{i=1}^n \frac {r_i}{2^i};\\
      s = \sum_{i=1}^n \frac {s_i}{2^i};\\
      r+s = \sum_{i=1}^n \frac {t_i}{2^i}.
    \end{gather}
    Deux cas se produisent. Si pour tout \(i\), \( t_i = r_i + s_i\), alors
    \begin{equation}
      \phi(r+s) = \sum_i t_i U_i = \sum_i r_i U_i + \sum_i s_i U_i = \phi(r) + \phi(s);
    \end{equation}
    l'égalité a lieu car \(r_i\) et \(s_i\) ne peuvent jamais valoir \(1\) en même temps.

    Sinon, posons \(k \) le plus petit entier tel que \( t_k \neq r_k + s_k\). Alors, nécessairement, \( r_k = 0,\ s_k = 0\) et \( t_k = 1\). Il s'ensuit, grâce à \eqref{EqBaseTopoMetriquePf1} et \eqref{EqBaseTopoMetriquePf2}, que
    \begin{gather}
      \phi(r) = \sum_{i=1}^{k-1} r_i V_i + \sum_{i=k+1}^n r_i V_i \subset \sum_{i=1}^{k-1} r_i V_i + V_{k+1}+ V_{k+1};\\
      \phi(s) = \sum_{i=1}^{k-1} s_i V_i + \sum_{i=k+1}^n s_i V_i \subset \sum_{i=1}^{k-1} s_i V_i + V_{k+1}+ V_{k+1};\text{ d'où}\\
      \phi(r)+\phi(s) = \sum_{i=1}^{k-1} r_i V_i + \sum_{i=1}^{k-1} s_i V_i + V_{k+1}+ V_{k+1} + V_{k+1}+ V_{k+1} = \sum_{i=1}^{k-1} t_i V_i + V_k \subset \phi(r+s).
    \end{gather}
  \item \emph{\(0  \in \phi(r)\) pour tout \(r\):} en effet, \(\phi(r)\) n'est jamais vide, c'est toujours un voisinage de \(0\).
  \item \label{PhiEstTotalementOrdonne} \emph{si \(r < s \) alors \( \phi(r) \subset \phi(s) \):} il suffit d'écrire
    \begin{equation}
      \phi(r) \subset \phi(r) + \phi(s-r) \subset \phi(s).
    \end{equation}
\end{enumerate}
Enfin, on définit
\begin{equation}\label{EqDefDistanceCompatible}
d(x,y) = \inf\{r \in \mathopen[0;1\mathclose] \tq y - x \in \phi(r)\}.
\end{equation}
Il suffit alors de voir que \(d\) convient. De par sa définition, il est clair qu'elle est invariante par translation; reste à voir que c'est bien une distance, et qu'elle est compatible avec la topologie.
\begin{subproof}
\item [$d(x,x) = 0$] Oui, car \(0\) est dans \(\phi(r)\), pour tout \(r \), puisque les \( U_i \) sont des voisinages de \(0\).
\item[$d(x,y) = d(y,x)$] Oui, car tous les voisinages considérés sont symétriques: pour tout \(i\) et tout \(x \in V\), on a \(x \in U_i\) si et seulement si  \(-x \in U_i\).
\item[$d(x,z) \leq d(x,y) + d(y,z)$] Soit \(\epsilon > 0 \). Par définition des distances comme infimums, et grâce au corolaire~\ref{CorDensiteDyadiques}, il existe \(r\) et \(s\) dans \( D \) tels que:
  \begin{equation}
    d(x,y) < r < d(x,y) + \frac \epsilon 2 \quad\text{ et }\quad d(y,z) < s < d(y,z) + \frac \epsilon 2 .
  \end{equation}
  Comme \(d(x,y) + d(y,z) <  r + s\), et par la remarque~\ref{PhiEstTotalementOrdonne} sur \(\phi\), on a \(y - x \in \phi(r)\) et \(z - y \in \phi(s)\); donc
  \begin{equation}
    (y - x) + (z - y) = z - x \in \phi(r) + \phi(s) \subset \phi(r+s)
  \end{equation}
  Ainsi, pour tout \(\epsilon > 0 \), on a
  \begin{equation}
    d(x,z) \leq r+s < d(x,y) + d(y,z) + \epsilon.
  \end{equation}

\item[Compatibilité avec la topologie] Si \(d(0,y) < r \), alors \( y  \in \phi(r) \); en particulier pour \( r = 1/2^k \), on a \(y \in \phi(r) = V_k\). D'où, pour tout \( n \in \eN,\ B(0, 1/2^n) \subset V_n \).
\end{subproof}
\end{proof}

\begin{proposition}     \label{PROPooPRLBooGtsRjr}
    Un espace vectoriel topologique\footnote{Définition \ref{DefEVTopologique}.} est métrisable si et seulement si il possède en tout point une base dénombrable de topologie.
\end{proposition}

\begin{proof}
    Il s'agit seulement de mettre bout à bout les corolaires~\ref{CORooTWFYooCNMieM} et théorème~\ref{THOooAGBXooZnvQLK}.
\end{proof}

%---------------------------------------------------------------------------------------------------------------------------
\subsection{Équivalence entre Cauchy et \texorpdfstring{$\tau-$}{tau-}Cauchy}
%---------------------------------------------------------------------------------------------------------------------------

\begin{lemma}       \label{LEMooIAHSooFkXjvr}
    Soit un espace vectoriel topologique\footnote{Définition~\ref{DefEVTopologique}.} \( V\) et une distance \( d\colon V\times V\to \eR^+\) compatible\footnote{Définition~\ref{DEFooGTOZooRcvGHg}.} avec la topologie de \( V\). Si \( d\) est invariante\footnote{Définition~\ref{DEFooGTOZooRcvGHg}.}, alors les suites de Cauchy pour \( d \) et les suites \( \tau\)-Cauchy sont les mêmes.
\end{lemma}

\begin{proof}
    Nous avons deux implications à prouver. 
    \begin{subproof}
    \item[Cauchy pour \( d\) implique \( \tau\)-Cauchy]
        Soit \( (x_n)\), une suite de Cauchy dans \( V\) pour \( d\), et un voisinage \( U\) de \( 0\). Vu que \( d\) est compatible avec la topologie de \( V\), il existe une boule ouverte \( B(0,\epsilon)\) incluse à \( U\). Soit \( N>0\) tel que \( m,n>N\) implique \( d(x_n,x_m)<\epsilon\). Par invariance de la métrique, nous avons aussi
        \begin{equation}
            d(0,x_m-x_n)<\epsilon,
        \end{equation}
        c'est-à-dire \( x_m-x_n\in B(0,\epsilon)\subset U\). La suite \( (x_n)\) est donc \( \tau\)-Cauchy.
    \item[\( \tau\)-Cauchy implique Cauchy pour \( d\)]
        Soit $(x_n)$, une suite \( \tau\)-Cauchy dans \( V\) et \( \epsilon>0\). Vu que \( B(0,\epsilon)\) est un voisinage de \( 0\) dans \( V\), il existe \( N\) tel que \( m,n>N\) implique \( x_n-x_m\in B(0,\epsilon)\). Cela signifie que \( d(0,x_n-x_m)<\epsilon\) et toujours par invariance, que \( d(x_n,x_m)<\epsilon\).
    \end{subproof}
\end{proof}

Tout ceci nous mène à donner une large classe d'espaces vectoriels topologiques sur lesquelles les notions de suites de Cauchy pour une distance et \( \tau\)-Cauchy coïncident.

\begin{theoremDef}     \label{THOooGQZSooAmQolf}
    Soit \( V\) un espace vectoriel topologique métrisable\footnote{i.e. admet une base dénombrable de topologie, voir la proposition~\ref{PROPooPRLBooGtsRjr}}, alors il admet une métrique \( d\) compatible avec la topologie telle que une suite dans \( V\) est de Cauchy pour \( d\) si et seulement si elle est \( \tau\)-Cauchy.

    Une \defe{suite de Cauchy}{Cauchy!suite} dans un espace vectoriel métrique \( (E,d)\) est une suite \( \tau\)-Cauchy ou de Cauchy pour \( d \).
\end{theoremDef}

\begin{proof}
    Soit \( d\) une métrique sur \( V\) satisfaisant au théorème~\ref{THOooAGBXooZnvQLK}. Vu qu'elle est invariante par translation, les suites \( d\)-Cauchy sont exactement les suites \( \tau\)-Cauchy par le lemme~\ref{LEMooIAHSooFkXjvr}.
\end{proof}

\begin{remark}  \label{REMooUFQYooUVCCjs}
    Même si \( V\) est métrisable, si on choisit la métrique n'importe comment, on ne peut rien espérer.
\end{remark}

\begin{normaltext}
    Sur les espaces vectoriels topologiques métrisables, nous pouvons donc parler de suite de Cauchy sans préciser si nous parlons de \( \tau\)-Cauchy ou de \( d\)-Cauchy, parce que nous sous-entendons avoir choisi une métrique non seulement compatible avec la topologie, mais également invariante par translation.

    Il reste cependant à traiter le cas d'un espace vectoriel topologique non métrisable. Dans ce cas, il n'y a pas de métrique, et la question de l'équivalence des définitions ne se pose pas.
\end{normaltext}

Le théorème suivant donne la complétude de \( \eR\) et le critère de Cauchy pour les définitions métriques et topologiques usuelles. Lorsqu'on dit que \( \eR\) est complet, le plus souvent nous parlons de ce théorème, et non de~\ref{THOooUFVJooYJlieh} qui en est un lemme indispensable mais qui parle de notions différentes, bien que très liées.
\begin{theorem}[Complétude de \( \eR\), critère de Cauchy\cite{RWWJooJdjxEK}]       \label{THOooNULFooYUqQYo}
    Nous avons :
    \begin{enumerate}
        \item
            L'espace métrique \( (\eR,d)\) est complet (définition~\ref{DEFooHBAVooKmqerL}).
        \item       \label{ITEMooUUFCooIVtGgz}
            Une suite dans \( \eR\) est convergente (définition~\ref{DefXSnbhZX}) si et seulement si elle est de Cauchy (définition~\ref{THOooGQZSooAmQolf}).
    \end{enumerate}
\end{theorem}
\index{complet!$\eR$!espace métrique}
\index{critère!de Cauchy}

\begin{proof}
    Tout ce théorème se base sur le fait que la définition de suite de Cauchy dans \( (\eR,d)\) et de suite convergente dans \( (\eR,d)\) coïncident avec les définitions correspondantes dans \( \eR\) vu comme simple corps ordonné (définitions~\ref{DefKCGBooLRNdJf}).

    Donc si \( (x_n)\) est de Cauchy dans \( (\eR,d)\), elle est de Cauchy dans le corps ordonné \( (\eR,\leq)\). Donc le théorème~\ref{THOooUFVJooYJlieh} nous dit que \( (x_n)\) est convergente dans \( (\eR,\leq)\). Et donc convergente dans \( (\eR,d)\).

    Toutes les autres affirmations se prouvent de la même manière.
\end{proof}

Si vous n'êtes pas sûr ou si vous ne voulez pas étudier les notations de convergence et de suites de Cauchy dans les corps, vous pouvez simplement recopier la démonstration du théorème~\ref{THOooUFVJooYJlieh} en remplaçant partout \( \eQ\) par \( \eR\), et aussi en remplaçant les \( | x-y |\) par \( d(x,y)\).

\begin{normaltext}
    Nous pouvons également mettre une structure d'espace métrique sur \( \eC\) en posant
    \begin{equation}
        d(z,z')=| z-z' |.
    \end{equation}
\end{normaltext}

\begin{proposition}
    L'espace métrique \( (\eC,d)\) est complet.
\end{proposition}

\begin{proof}
    Commençons par nous rendre compte que pour tout \( z\in \eC\) nous avons \( | \real(z) |\leq | z |\). C'est bon ? Vous vous en êtes rendu compte ? Ok. Continuons.

    Soit une suite de Cauchy \( (z_k)\) dans \( \eC\) et \( \epsilon>0\). Si \( x_k=\real(z_j)\), nous avons
    \begin{equation}
        | x_k-x_l |=| \real(z_k-z_l) |\leq | z_k-z_l |.
    \end{equation}
    Vu que \( (z_k)\) est de Cauchy, il existe un \( N\) tel que si \( k,l\geq N\),
    \begin{equation}
        | x_k-x_l |\leq | z_k-z_l |\leq \epsilon.
    \end{equation}

    Donc la suite des parties réelles converge par la complétude de \( (\eR,d)\) du théorème~\ref{THOooNULFooYUqQYo}. Notez que le \( d\) ici n'est pas tout à fait le même, et que la démonstration fonctionne parce que la distance prise sur \( \eR\) est la restriction à \( \eR\) de la distance prise sur \( \eC\). Notons \( x\) la limite de \( (x_k)\).

    De la même manière la suite des parties imaginaires \( y_k=\imag(z_k)\) converge vers un réel que nous notons \( y\). Avec tout cela, la suite \( z_k\) converge dans \( \eC\) vers \( x+iy\). En effet pour \( \epsilon\) donné et pour un \( k\) suffisament grand,
    \begin{equation}
        | z_k-(x+iy) |=\big| \real(z_k)-x+i(\imag(z_k)-y) \big|\leq | x_k-x |+| y_k-y |\leq \epsilon.
    \end{equation}
\end{proof}

%+++++++++++++++++++++++++++++++++++++++++++++++++++++++++++++++++++++++++++++++++++++++++++++++++++++++++++++++++++++++++++ 
\section{Norme; espace vectoriel normé}
%+++++++++++++++++++++++++++++++++++++++++++++++++++++++++++++++++++++++++++++++++++++++++++++++++++++++++++++++++++++++++++
\label{SECooWKJNooKOqpsx}

La valeur absolue est essentielle pour introduire les notions de limite et de continuité pour les fonctions d'une variable. Par exemple nous verrons dans la proposition \ref{PROPooVNGEooPwbxXP} que la fonction \( f\colon \eR\to \eR\) est continue en \( a\) si et seulement si pour tout $\varepsilon > 0$, il existe un $\delta > 0$ tel que
  \begin{equation}
    | x-a |\leq\delta \Rightarrow | f(x)-f(a) |\leq \varepsilon.
  \end{equation}
La quantité $| x-a |$ donne la «distance» entre $x$ et $a$; la définition de la continuité signifie que pour tout $\varepsilon$, il existe un $\delta$ tel que si $a$ et $x$ sont au plus à la distance $\delta$ l'un de l'autre, alors $f(x)$ et $f(a)$ ne seront éloignés au plus d'une distance $\varepsilon$.

La valeur absolue, dans $\eR$, nous sert donc à mesurer des distances entre les nombres. Les principales propriétés de la valeur absolue sont :
\begin{enumerate}

	\item
		$| x |=0$ implique $x=0$,
	\item
		$| \lambda x |=| \lambda | |x |$,
	\item
		$| x+y |\leq | x |+| y |$

\end{enumerate}
pour tout $x,y\in\eR$ et $\lambda\in\eR$.

Afin de donner une notion de limite pour les fonctions de plusieurs variables, nous devons trouver un moyen de définir les notions de «taille» d'un vecteur et de distance entre deux points de $\eR^n$, avec $n>1$. La notion de «taille» doit satisfaire propriétés analogues à celles de la valeur absolue.

La première notion de «taille» pour un vecteur de $\eR^2$ que nous vient à l'esprit est la longueur du segment entre l'origine et l'extrémité libre du vecteur. Cela peut être calculée à l'aide du théorème de Pythagore :
\begin{equation}
  \textrm{taille de } (a,b) = \sqrt{a^2+b^2}.
\end{equation}
Nous pouvons introduire une notion de distance entre les éléments de $\eR^2$ de façon similaire :
\begin{equation}
	d\big((a_x,a_y),(b_x,b_y)\big)=\sqrt{  (a_x-b_x)^2+(a_y-b_y)^2  }.
\end{equation}
Cette définition a l'air raisonnable; est-elle mathématiquement correcte ? Peut-elle jouer le rôle de la valeur absolue dans $\eR^2$ ? Est-elle la seule définition possibles de «taille» et distance en $\eR^2$ ?

Nous voulons formaliser les notions de «taille» et de distance dans $\eR^n$, et plus généralement dans un espace vectoriel $V$ de dimension finie. Pour cela nous nous inspirons des propriétés de la valeur absolue.



\subsubsection{Critère de Cauchy}
%///////////////////////////////

\begin{lemma}
    Une suite de Cauchy\footnote{Définition \ref{DEFooXOYSooSPTRTn}.} dans un espace vectoriel normé admettant une sous-suite convergente est elle-même convergente vers la même limite.
\end{lemma}

\begin{proof}
    Soit \( (a_n)\) une suite de Cauchy dans un espace vectoriel normé \( E\) et \( \ell\) la limite d'une sous-suite de \( (a_n)\). Soit \( \epsilon>0\) et \( N\in \eN\) tel que \( \| a_m-a_p \|<\epsilon\) dès que \( m,p\geq N\). Nous allons montrer que si \( k>N\) alors \( \| a_k-\ell \|<2\epsilon\). Pour cela nous considérons un \( n>N\) tel que \( \| a_n-\ell \|\leq \epsilon\) et nous calculons
    \begin{equation}
        \| a_k-\ell \|\leq \| a_k-a_n \|+\| a_n-\ell \|\leq 2\epsilon.
    \end{equation}
\end{proof}

Dans le cas des espaces de dimension finie, le fait d'être complet est automatique, comme le montre la proposition suivante.
\begin{proposition}     \label{PROPooGJDTooXOoYfw}
    Soit \( \big( E,\| . \| \big)\) un espace vectoriel normé de dimension finie sur un corps \( \eK\) qui est complet\footnote{La définition est~\ref{DefKCGBooLRNdJf}, mais si vous n'avez pas envie de vous embarquer trop loin, dites juste «toutes les suites de Cauchy convergent». Typiquement c'est \( \eR\) ou \( \eC\).}. Alors \( E\) est complet\footnote{Définition~\ref{DEFooHBAVooKmqerL}.}.
\end{proposition}
Pour rappel, la complétude de l'espace métrique \( \eR\) est la proposition~\ref{PROPooTFVOooFoSHPg}.

\begin{proof}
    Nous considérons une suite de Cauchy \( (f_n)\) dans \( E\) et si \( \{ e_{\alpha} \} \) est une base orthonormée de \( E\) nous définissons les coefficients \( f_n=\sum_{\alpha}a_{n\alpha}e_{\alpha} \). La somme sur \( \alpha\) est finie par hypothèse sur la dimension de \( E\).

    Nous avons
    \begin{equation}
        \| f_n-f_m \|=\| \sum_{\alpha}(a_{n\alpha}-a_{m\alpha})e_{\alpha} \|=\sum_{\alpha}| a_{n\alpha}-a_{m\alpha} |^2.
    \end{equation}
    Pour tout \( \epsilon\), il existe \( N\) tel que si \( m,n>N\) alors \( | a_{n\alpha}-a_{m\alpha} |<\sqrt{ \epsilon }\). Autrement dit, pour chaque \( \alpha\), la suite \( (a_{n\alpha})_{\alpha\in \eN}\) est de Cauchy dans \( \eK\) et converge donc dans \( \eK\). Soit \( a_{\alpha}\) la limite et définissons \( f=\sum_{\alpha}a_{\alpha}e_{\alpha}\). Nous avons alors
    \begin{equation}
        \| f_n-f \|=\| \sum_{\alpha}(a_{n\alpha}-a_{\alpha})e_{\alpha} \|,
    \end{equation}
    dont la limite \( n\to \infty\) est bien zéro. Donc la suite \( (f_n)\) converge vers \( f\in E\). L'espace \( E\) est alors complet.
\end{proof}




\begin{proposition}		\label{PropContinueCompactBorne}
	Soient $V$ et $W$ deux espaces vectoriels normés. Soient $K$ une partie compacte de $V$ et $f\colon K\to W$ une fonction continue. Alors l'image $f(K)$ est compacte dans $W$.
\end{proposition}
Ce résultat est démontré dans un cadre plus général par le théorème~\ref{ThoImCompCotComp}.

\begin{proof}
	Nous allons prouver que $f(K)$ est fermée et bornée.
    \begin{subproof}
		\item[$f(K)$ est fermé] Nous allons prouver que si $(y_n)$ est une suite convergente contenue dans $f(K)$, alors la limite est également contenue dans $f(K)$. Dans ce cas, nous aurons que l'adhérence de $f(K)$ est contenue dans $f(K)$ et donc que $f(K)$ est fermé. Pour chaque $n\in\eN$, le vecteur $y_n$ appartient à $f(K)$ et donc il existe un $x_n\in K$ tel que $f(x_n)=y_n$. La suite $(x_n)$ ainsi construite est une suite dans le fermé $K$ et possède donc une sous-suite convergente (proposition~\ref{THOooRDYOooJHLfGq}). Notons $(x'_n)$ cette sous-suite convergente, et $a$ sa limite : $\lim(x'_n)=a\in K$. Le fait que la limite soit dans $K$ provient du fait que $K$ est fermé.

			Nous pouvons considérer la suite $f(x'_n)$ dans $W$. Cela est une sous-suite de la suite $(y_n)$, et nous avons $\lim f(x'_n)=a$ parce que $f$ est continue. Par conséquent nous avons
			\begin{equation}
				f(a)=\lim f(x'_n)=\lim y_n.
			\end{equation}
			Cela prouve que la limite de $(y_n)$ est dans $f(K)$ et par conséquent que $f(K)$ est fermé.

		\item[$f(K)$ est borné]
			Si $f(K)$ n'est pas borné, nous pouvons trouver une suite $(x_n)$ dans $K$ telle que
			\begin{equation}		\label{EqfxnWgeqn}
				\| f(x_n) \|_W>n
			\end{equation}
			Mais par ailleurs, l'ensemble $K$ étant compact (et donc fermé), nous avons une sous-suite $(x'_n)$ qui converge dans $K$. Disons $\lim(x'_n)=a\in K$.

			Par la continuité de $f$ nous avons alors $f(a)=\lim f(x'_n)$, et donc
			\begin{equation}
				| f(a) |=\lim | f(x'_n) |.
			\end{equation}
			La suite $f(x'_n)$ est alors une suite bornée, ce qui n'est pas possible au vu de la condition \eqref{EqfxnWgeqn} imposée à la suite de départ $(x_n)$.
    \end{subproof}
\end{proof}

\begin{corollary}	\label{CorFnContinueCompactBorne}
	Si $f\colon K\to \eR$ est une application continue où $K$ est une partie compacte d'un espace vectoriel normé, alors \( f(K)\) est borné.
\end{corollary}

\begin{proof}
	En effet, la proposition~\ref{PropContinueCompactBorne} montre que $f(K)$ est compact et donc borné.
\end{proof}

% TODO: regarder ceci à propos des compacts.
% En particulier, si on recouvre $A$ par l'ensemble des boules
% $B(x,1)$ où $x$ parcourt $A$ (de sorte que tout point de $A$ est
% dans « sa » boule, et donc la réunion des boules recouvre bien
% $A$), on doit pouvoir en tirer un recouvrement fini, c'est-à-dire
% des boules $B(x_1,1), B(x_2,1), \ldots, B(x_k,1)$ (avec $k$ un
% naturel) dont la réunion contient $A$.

% Il me semble que c'est le coup qu'il ne faut vérifier le sous-recouvrement que pour des recouvrements composés d'ouverts issus d'une base donnée de la topologie.

% This is part of Mes notes de mathématique
% Copyright (c) 2011-2018
%   Laurent Claessens, Carlotta Donadello
% See the file fdl-1.3.txt for copying conditions.

%+++++++++++++++++++++++++++++++++++++++++++++++++++++++++++++++++++++++++++++++++++++++++++++++++++++++++++++++++++++++++++
\section{Espaces métriques}
%+++++++++++++++++++++++++++++++++++++++++++++++++++++++++++++++++++++++++++++++++++++++++++++++++++++++++++++++++++++++++++

%---------------------------------------------------------------------------------------------------------------------------
\subsection{Espaces métrisables}
%---------------------------------------------------------------------------------------------------------------------------

\begin{definition}
    Un espace topologique est \defe{métrisable}{espace!topologique!métrisable} s'il est homéomorphe à un espace métrique.
\end{definition}


\begin{proposition} \label{PROPooKNVUooMbLZoy}
    Une fonction séquentiellement continue sur un espace métrisable et à valeurs dans un espace métrique est continue.
\end{proposition}

\begin{proof}
    Soient \( E\) un espace métrique et \( \phi\colon X\to (E,d)\) un homéomorphisme. Nous supposons que \( f\colon X\to Y\) est séquentiellement continue. Nous considérons l'application \( \tilde f=f\circ\phi^{-1}\), c'est-à-dire
    \begin{equation}
        \begin{aligned}
            \tilde f\colon E&\to Y \\
            a&\mapsto f\big( \phi^{-1}(a) \big).
        \end{aligned}
    \end{equation}
    L'application \( \phi^{-1}\) est continue et donc séquentiellement continue. De plus \( \tilde f\) est séquentiellement continue. En effet si \( a_k\stackrel{E}{\longrightarrow}a\), alors
    \begin{equation}
        \tilde f(a_k)=f\big( \phi^{-1}(a_k) \big),
    \end{equation}
    mais \( \phi^{-1}\) est séquentiellement continue, donc \( \phi^{-1}(a_k)\stackrel{X}{\longrightarrow}\phi^{-1}(a)\), ce qui signifie que \( \phi^{-1}(a_k)\) est une suite convergente dans \( X\) et donc
    \begin{equation}
        \lim_{k\to \infty} \tilde f(a_k)=\lim_{k\to \infty} f\big( \phi^{-1}(a_k) \big)=f\big( \phi^{-1}(a) \big)=\tilde f(a).
    \end{equation}
    L'application \( \tilde f\) est donc séquentiellement continue. Mais étant donné que \( \tilde f\) est définie sur un espace métrique (\( E\)) et à valeurs dans un métrique, elle est continue par la proposition~\ref{PropXIAQSXr}. L'application \( f=\tilde f\circ\phi\) est donc continue en tant que composée d'applications continues.
\end{proof}

%---------------------------------------------------------------------------------------------------------------------------
\subsection{Fonctions continues}
%---------------------------------------------------------------------------------------------------------------------------

La propriété suivante donne des caractérisations importantes de la continuité dans le cas des espaces métriques.
\begin{proposition}[Continuité, ouverts et voisinages et limite\cite{DHpsZoY}] \label{PropQZRNpMn}
    Soient \( f\colon E\to F\) une application entre espaces métriques et \( a\in E\). Alors nous avons équivalence entre les choses suivantes :
    \begin{enumerate}
        \item\label{ItemCBUoRWJi}
            \( f\) est continue en \( a\),
        \item\label{ItemCBUoRWJii}
            Pour tout voisinage ouvert \( W\) de \( f(a)\), il existe un voisinage ouvert \( V\) de \( a\) tel que \( f(V)\subset W\).
        \item\label{ItemCBUoRWJiii}
            Pour toute boule \( W'=B\big( f(a),\epsilon \big)\), il existe une boule \( V'=B(a,\delta)\) telle que \( f(V)\subset W\).
        \item\label{ItemCBUoRWJiv}
            $\forall \epsilon>0,\,\exists \delta>0\,\tq f\big( B(a,\delta) \big)\subset B\big( f(a),\epsilon \big)$.
        \item\label{ItemYNQpikrii}
            \( \lim_{x\to a}f(x)=f(a)\) où la limite est donnée par la définition~\ref{DefYNVoWBx},
        \item\label{ItemYNQpikriii}
            Pour tout \( \epsilon>0\), il existe \( \delta>0\) tel que \( \| x-a \|<\delta\) implique \( \| f(x)-f(a) \|<\epsilon\).
    \end{enumerate}
\end{proposition}
\index{continue!fonction entre espaces métriques}
La proposition~\ref{PropNGjQnqF} nous montrera que ces équivalences tiennent encore lorsque l'espace a une topologie de semi-normes.

\begin{proof}
    L'équivalence~\ref{ItemCBUoRWJi} \( \Leftrightarrow\)~\ref{ItemCBUoRWJii} est la définition~\ref{DefOLNtrxB}. L'équivalence~\ref{ItemCBUoRWJiii} \( \Leftrightarrow\)~\ref{ItemCBUoRWJiv} est une simple paraphrase.

    Montrons~\ref{ItemCBUoRWJii} \( \Rightarrow\)~\ref{ItemCBUoRWJiii}. Si \( W'=B\big( f(a),\delta \big)\), nous avons un voisinage \( V\) de \( a\) tel que \( f(V)\subset W\). L'ensemble \( V\) contenant une boule autour de chacun de ses points\footnote{Cela est le théorème-définition~\ref{ThoORdLYUu} des ouverts dans un espace métrique, à ne pas confondre avec le théorème~\ref{ThoPartieOUvpartouv}.}, il en contient un autour de \( a\) : \( V'=B(a,\delta)\subset V\). A fortiori nous avons \( f(V')\subset W\).

    Montrons~\ref{ItemCBUoRWJiii} \( \Rightarrow\)~\ref{ItemCBUoRWJii}. Si \( W\) est un ouvert autour de \( f(a)\), il contient une boule autour de \( f(a)\) : \( B\big( f(a),\epsilon \big)\subset W\). Il existe donc une boule \( V'=B(a,\delta)\) telle que \( f(V')\subset B\big( f(a),\epsilon \big)\subset W\).

    L'équivalence~\ref{ItemCBUoRWJi} \( \Leftrightarrow\)~\ref{ItemYNQpikrii} est la définition~\ref{DefOLNtrxB} de la continuité en un point couplée à l'unicité de la limite due à la proposition~\ref{PropFObayrf} parce qu'un espace métrique est séparé.

    Prouvons~\ref{ItemYNQpikrii} \( \Rightarrow\)~\ref{ItemYNQpikriii}. Soient \( \epsilon>0\) et \( V=B\big( f(a),\epsilon \big)\). Étant donné que \( f(a)\) est une limite de \( f\) pour \( x\to a\), il existe un voisinage \( W\) de \( a\) tel que \( f(W)\subset V\). Soit \( \delta>0\) tel que \( B(a,\delta)\subset W\); alors si \( \| x-a \|<\delta\) nous avons \( x\in B(x,\delta)\subset W\) et donc \( f(x)\in B\big( f(a),\epsilon \big)\), c'est-à-dire \( \| f(a)-f(x) \|<\epsilon\).

    Enfin l'implication~\ref{ItemCBUoRWJii} \( \Rightarrow\)~\ref{ItemYNQpikrii} est une réécriture de la définition de la limite en un point.
\end{proof}

Voici un théorème qui parle de fermés emboîtés dans un espace métrique. Le corolaire \ref{CORooQABLooMPSUBf} parle du cas \( \cap_iA_i=\emptyset\) dans un compact.
\begin{theorem}[Théorème \wikipedia{fr}{Théorème_des_fermés_emboités}{des fermés emboîtés}\cite{OIywOjl}]   \label{ThoCQAcZxX}
    Soit \( (E,d)\) un espace métrique. Il est complet si et seulement si toute suite décroissante de fermés non vides dont le diamètre tend vers zéro a une intersection qui se réduit à un seul point.
\end{theorem}

\begin{proof}
    En deux parties.
    \begin{subproof}
    \item[Condition suffisante]

        Soit \( \{ F_n \}_{n\in \eN}\) une telle suite de fermés emboités. Si nous choisissons des points \( x_n\in F_n\), nous obtenons une suite \( (x_n)\) de Cauchy et qui est par conséquent convergente vu que l'espace est par hypothèse complet. De plus, pour chaque \( N\geq n\), la queue de suite \( (x_n)_{n\geq N}\) est contenue dans \( F_N\) et donc converge vers un élément de \( F_N\) (parce que ce dernier est fermé). Donc la limite de \( (x_n)\) est dans \( \bigcap_{n\in \eN}F_n\).

        De plus cette intersection a diamètre nul parce que le diamètre de \( \bigcap_{n\in \eN}F_n\) est majoré par tous les diamètres des \( F_n\), lesquels sont arbitrairement petits par hypothèse. Donc l'intersection est réduite a un point.

    \item[Condition nécessaire]

        Soit \( (x_n)\) un suite de Cauchy. Nous considérons les ensembles
        \begin{equation}
            F_n=\overline{ \{ x_i\tq i\geq n \} }.
        \end{equation}
        Le fait que la suite soit de Cauchy implique que \( \diam(F_n)\to 0\). Par hypothèse, nous avons alors
        \begin{equation}
            \bigcap_{n\in \eN}F_n=\{ a \}.
        \end{equation}
        Pour s'assurer que \( a\) est bien la limite de \( (x_n)\), il suffit de remarquer que
        \begin{equation}
            d(x_n,a)\leq \diam F_n\to 0.
        \end{equation}
    \end{subproof}
\end{proof}

\begin{proposition}     \label{PropGULUooNzqZKj}
    Soient \( (X,d) \) un espace topologique métrique et \( F\) un fermé de \( X\). Nous avons \( d(x,F)=0\) si et seulement si \( x\in F\).
\end{proposition}

\begin{proof}
    Si \( x\in F\) alors \( d(x,F)=0\) parce que \( d(x,x)\) fait partie de l'ensemble sur lequel nous prenons l'infimum.

    Si réciproquement \( d(x,F)=0\), cela signifie que pour tout \( \epsilon\), il existe \( x_{\epsilon}\in F\) tel que \( d(x_{\epsilon},x)\leq \psi\). En prenant \(\epsilon=1/k\) nous construisons une suite \( (x_k)\) d'éléments dans \( F\) vérifiant \( d(x_k,x)=\frac{1}{ k }\). Cela signifie que \( \lim_{k\to \infty} x_k=x\) par la proposition~\ref{PropooUEEOooLeIImr}\ref{ItemooROYMooAQCXnj}.

    Par la caractérisation séquentielle des fermés (un fermé contient les limites de toutes ses suites, proposition~\ref{PropLFBXIjt}), la suite \( (x_k)\) étant dans \( F\), la limite est dans \( F\). Donc \( x\in F\).
\end{proof}


\begin{lemma}       \label{LemooynkH}
    Soit \( A_n\) une suite décroissante de fermés dans un espace métrique\footnote{L'hypothèse métrique provient de l'utilisation de Bolzano-Weierstrass, lequel est vrai pour les espaces séquentiellement compacts, dont les espaces métriques.} compact \( K\). Alors
    \begin{equation}
        C=\bigcap_{n\in \eN}A_n
    \end{equation}
    est non vide.
\end{lemma}

\begin{proof}
    Soit \( (x_n)\) une suite dans \( K\) telle que \( x_n\in A_n\). La suite étant contenue dans \( A_1\), et \( A_1\) étant compact (lemme~\ref{LemnAeACf}), elle possède une sous-suite \( (y_n=x_{\sigma_1(n)})\) convergente dont la limite est dans \( A_1\) par le théorème de Bolzano-Weierstrass~\ref{ThoBWFTXAZNH}. Une queue de la suite \( y_n\) est dans \( A_2\) et nous considérons donc une sous-suite convergente dans \( A_2\) donnée par
    \begin{equation}
        z_n=y_{\sigma_2(n)}=x_{\sigma_1\sigma_2(n)}.
    \end{equation}
    En continuant ainsi nous construisons une suite convergente dans \( A_k\). Nous considérons enfin la suite
    \begin{equation}
        y_n=x_{\sigma_1\ldots \sigma_n(n)}.
    \end{equation}
    Pour tout \( k\), une queue de cette suite est une sous-suite de \( x_{\sigma_1\ldots \sigma_k(n)}\) et par conséquent cette suite converge dans \( A_k\). La limite de cette suite est donc dans l'intersection demandée.
\end{proof}

\begin{remark}
    Cette propriété est fausse pour les ouverts. Par exemple
    \begin{equation}
        \bigcap_{n>1}\mathopen] 0 , \frac{1}{ n } \mathclose[=\emptyset.
    \end{equation}
\end{remark}

\begin{lemma}   \label{LemKIcAbic}
    Si \( K\) est un compact dans un espace métrique et \( F\) un fermé disjoint de \( K\), alors \( d(K,F)>0\).
\end{lemma}

\begin{proof}
    La fonction
    \begin{equation}
        \begin{aligned}
             K&\to \eR \\
            x&\mapsto d(x,F)
        \end{aligned}
    \end{equation}
    est une fonction continue sur \( K\), et donc atteint son minimum par le théorème de Weierstrass~\ref{ThoWeirstrassRn}. Soit \( x_0\in K\) un point de \( K\) qui réalise ce minimum. Si \( d(x_0,F)=0\), alors on aurait une suite \( (x_n)\) dans \( F\) qui convergerait vers \( x_0\), mais \( F\) étant fermé cela signifierait que \( x_0\) serait dans \( F\), ce qui contredirait l'hypothèse que \( F\) et \( K\) sont disjoints.
\end{proof}

\begin{proposition}[\cite{AntoniniAndAl-EspacesMetriquesCompacts}]
    Une isométrie d'un espace métrique compact sur lui-même est une bijection.
\end{proposition}

\begin{proof}
    Soient \( X\) un espace métrique compact et \( f\colon X\to X\) une isométrie. Le fait que \( f\) soit injective est obligatoire (sinon il y a des images dont la distance est nulle). Il faut montrer que \( f\) est surjective.

    Soit \( x\in X\) hors de \( f(X)\). Le lemme~\ref{LemKIcAbic} appliqué au fermé \( \{ x \}\) et au compact \( f(K)\) donne un \( r>0\) tel que
    \begin{equation}
        d\big( x,f(K)\big)>r.
    \end{equation}
    Soit la suite \( u_n=f^n(x)\); c'est une suite dans \( K\) et possède donc une sous-suite convergente (Bolzano-Weierstrass\ref{ThoBWFTXAZNH}) que l'on nomme \( (y_n)\). Vu que \( f\) est une isométrie,
    \begin{equation}
        d(y_{n},y_{n+1})=d(x,y_m)>r
    \end{equation}
    pour un certain \( m\leq n+1\). Cela signifie que pour tout \( n\), nous avons \( d(y_n,y_{n+1})>r\), ce qui contredit le fait que la suite \( (y_n)\) converge.
\end{proof}

\begin{proposition} \label{PropLHWACDU}
    Soient \( (X,d)\) un espace métrique compact et \( (u_n)\) une suite de \( X\) telle que
    \begin{equation}
        \lim_{n\to \infty} d(u_n,u_{n+1})=0.
    \end{equation}
    Alors l'ensemble des points d'accumulation\footnote{Définition \ref{DEFooGHUUooZKTJRi}.} de \( (u_n)\) est connexe.
\end{proposition}
\index{connexité!points d'accumulation}
\index{compacité}

\begin{proof}
    Nous notons \( \Gamma\) l'ensemble des points d'accumulation de la suite.
    \begin{subproof}
    \item[\( \Gamma\) est compact]
        Nous notons \( A_p=\{ u_n\tq n\geq p \}\) et nous avons
        \begin{equation}
            \Gamma=\bigcap_{p\in \eN}\overline{ A_p }
        \end{equation}
        parce que si \( x\in\Gamma\), alors pour tout \( n\), il existe \( m>n\) tel que \( x_m\in B(x,\epsilon)\), et donc tel que \( x\in B(x_m,\epsilon)\). Donc pour tout \( \epsilon\) et pour tout \( p\), l'intersection \( B(x,\epsilon)\cap A_p\) est non vide.

        En tant qu'intersection de fermés, \( \Gamma\) est fermé (lemme~\ref{LemQYUJwPC}). En tant que fermé dans un compact, \( \Gamma\) est compact (lemme~\ref{LemnAeACf}).

    \item[Recouvrement par deux compacts]

        Supposons que \( \Gamma\) ne soit\quext{est-ce qu'il faut vraiment un subjonctif ici ?} pas connexe. Nous pouvons alors considérer \( S\) et \( O\), deux ouverts disjoints recouvrant \( \Gamma\) et intersectant tout deux \( \Gamma\). Nous posons alors
        \begin{subequations}
            \begin{align}
                A&=S\cap\Gamma\\
                B&=O\cap\Gamma,
            \end{align}
        \end{subequations}
        et nous avons évidemment \( \Gamma=A\cup B\). Montrons que \( A\) est fermé (\( B\) le sera aussi par le même raisonnement). Soit une suite d'éléments de \( S\cap \Gamma\) convergent dans \( X\). Alors la limite est dans \( \bar\Gamma=\Gamma\) et donc elle est donc \( O\) ou \( S\), mais elle est certainement dans \( \bar S\). Cependant \( \bar S\) n'intersecte pas \( O\). En effet si \( x\in \bar S\cap O\), alors tout voisinage de \( x\) intersecterait \( S\), mais il y a des voisinages de \( x\) étant inclus dans \( O\) parce que \( O\) est ouvert; cela donnerait une intersection entre \( O\) et \( S\), ce qui est impossible. Donc la limite n'est pas dans \( O\) et donc elle est dans \( S\). Au final la limite est dans \( S\cap \Gamma\), ce qui prouve son caractère fermé.

        Comme d'habitude, \( \Gamma\cap S\) est compact parce que fermé dans un compact\footnote{Lemme \ref{LemnAeACf}.}.

    \item[Décomposition en trois morceaux]

        Vu que \( A\) et \( B\) sont des compacts disjoints, nous avons \( d(A,B)=\alpha>0\) pour un certain \( \alpha\) par le lemme~\ref{LemKIcAbic}. Nous notons
        \begin{subequations}
            \begin{align}
                A'&=\{x\in X\tq d(x,A)<\frac{ \alpha }{ 3 }\}\\
                B'&=\{x\in X\tq d(x,B)<\frac{ \alpha }{ 3 }\}
            \end{align}
        \end{subequations}
        Nous avons \( A'=\bigcup_{x\in A}B(x,\frac{ \alpha }{ 3 })\) et donc en tant qu'union d'ouverts, \( A'\) est ouvert (définition de la topologie). Même chose pour \( B'\).

        Enfin nous notons
        \begin{equation}
            K=X\setminus(A'\cup B')
        \end{equation}
        qui est fermé en tant que complémentaire d'ouvert, et donc compact. Étant donné que \( A\subset A'\) et \( B\subset B' \), nous avons \( K\cap \Gamma=\emptyset\).

        L'idée est maintenant de montrer que \( K\) contient un point d'accumulation de \( (u_n)\).

    \item[Sous-suites de \( (u_n)\)]

        L'hypothèse sur la suite \( (u_n)\) nous indique qu'il existe un \( N_0\) tel que \( \forall n\geq N_0\),
        \begin{equation}    \label{EqIHioHjW}
            d(u_{n},u_{n+1})<\frac{ \alpha }{ 3 }.
        \end{equation}
        Soient \( N>N_0 \) et \( x_0\in A\). Étant donné que \( x_0\) est point d'accumulation de la suite, il existe \( n_1>N\) tel que \( d(x_0,u_{n_1})<\frac{ \alpha }{ 3 }\). Même chose dans \( B\) : nous prenons \( y_0\in B\) et un naturel \( n_2>n_1\) tel que \( d(y_0,u_{n_2})<\frac{ \alpha }{ 3 }\). Nous avons \( u_{n_1}\in A'\) et \( u_{n_2}\in B'\).

        Soit \( n_0\) le plus petit naturel supérieur à \( n_1\) tel que \( u_{n_0}\notin A'\). Cela existe parce que \( u_{n_2}\in B'\) et \( B'\cap A'=\emptyset\), mais \( n_0\) n'est pas \( n_2\) lui-même parce que \( d(A',B')\geq \frac{ \alpha }{ 3 }\) alors que nous considérons \( n_0,n_1,n_2>N_0\) et donc pour tous les \( i\) entre \( n_1\) et \( n_2\) (compris), \( d(u_i,u_{i+1})<\frac{ \alpha }{ 3 }\). Notons qu'ici le strict dans la condition \eqref{EqIHioHjW} est important. Nous avons donc \(N_0<n_1<n_0<n_2\).

        Nous allons maintenant montrer que \( u_{n_0}\) est dans \( K\). C'est fait pour : il est loin en même temps de \( A'\) et de \( B'\). En utilisant l'inégalité triangulaire à l'envers, nous avons
        \begin{equation}
            \begin{aligned}[]
            d(u_{n_0},B)&\geq d(u_{n_0-1},B)-d(u_{n_0-1},u_{n0})\\
            &\geq d(A,B)-d(u_{n_0-1},A)-d(u_{n_0-1},u_{n_0})\\
            &\geq \alpha-\frac{ \alpha }{ 3 }-\frac{ \alpha }{ 3 }\\
            &=\frac{ \alpha }{ 3 }.
            \end{aligned}
        \end{equation}
        Pour la dernière inégalité nous avons utilisé le fait que \( u_{n_0-1}\) n'est pas dans \( A'\). Bref, nous avons montré que \( u_{n_0}\) n'est pas dans \( B'\) (dans la définition de ce dernier nous avons bien une inégalité stricte). Vu que par définition \( u_{n_0}\) n'est pas non plus dans \( A'\), nous avons \( u_{n_0}\in K\).

        Nous avons montré jusqu'à présent que pour tout \( N\geq N_0\), il existe un \( n_0\geq N\) tel que \( u_{n_0}\in K\). Cela nous construit donc une sous-suite \( (v_n)\) de \( (u_n)\) contenue dans \( K\). En tant que suite dans le compact \( K\), la suite \( (v_n)\) admet un point d'accumulation dans \( K\). Ce point est également point d'accumulation de la suite \( (u_n)\) complète, ce qui donne un point d'accumulation de \( (u_n)\) dans \( K\) et donc une contradiction.

    \end{subproof}
    Nous concluons que \( \Gamma\) est connexe.
\end{proof}

Encore une petite conséquence sans ambition du théorème de Bolzano-Weierstrass.
\begin{proposition}\label{PropHNylIAW}
    Si \( (x_n)\) est une suite dans un compact telle que toute sous-suite convergente ait le même point \( x\) comme limite. Alors la suite entière converge vers \( x\).
\end{proposition}

\begin{proof}
    Supposons que ce ne soit pas le cas. Alors il existe un \( \epsilon\) tel que pour tout \( N>0\), il existe \( n>N\) avec \( d(x_n,x)>\epsilon\). Cela nous donne une sous-suite de \( (x_n)\) composée d'éléments tous à une distance de \( x\) supérieure à \( \epsilon\). Nous la nommons \( (y_n)\); c'est une suite dans un compact qui admet donc une sous-suite convergente (et une telle sous-suite est une sous-suite de \( (x_n)\)) dont la limite devrait être \( x\), mais c'est impossible par construction.
\end{proof}

\begin{lemmaDef}       \label{LemGDeZlOo}
    Soi \( \Omega\) un ouvert dans un espace métrique \( E\). Il existe une suite \( (K_n)\) de compacts tels que
    \begin{enumerate}
        \item
            \( K_n\subset \Omega\)
        \item
            \( \bigcup_{n=0}^{\infty}K_n=\Omega\)
        \item
            \( K_n\subset\Int(K_{n+1})\).
    \end{enumerate}
    Une telle suite de compacts vérifie alors
    \begin{enumerate}
        \item
            Il existe \( \delta_n\) tel que pour tout \( z\in K_n\), \( B(z,\delta_n)\subset K_{n+1}\).
        \item
            Tout compact de \( \Omega\) est inclus dans \( \Int(K_n)\) pour un certain \( n\).
    \end{enumerate}
    Une telle suite de compacts est une \defe{suite exhaustive}{compact!suite exhaustive}\index{exhaustive (suite de compacts)!} de compacts pour \( \Omega\).
\end{lemmaDef}

\begin{proof}
    Nous considérons les ensembles
    \begin{equation}
        V_n=\{ z\in E\tq | z | \}\cup\bigcup_{a\notin\Omega}B(a,\frac{1}{ n }),
    \end{equation}
    et nous définissons \( K_n=\complement V_n\). Vérifions que ces ensembles vérifient tout ce qu'il faut.
    \begin{enumerate}
        \item
            Si \( a\notin\Omega\) alors \( a\) est dans tous les \( V_n\) et donc dans aucun des \( K_n\); nous avons donc bien \( K_n\subset\Omega\).
        \item
            Si \( z\in \Omega\) alors nous prenons \( n_1>| z |\) puis \( n_2\) tel que \( B(z,\frac{1}{ n_2 })\subset \Omega\). Alors \( z\in K_n\) avec \( n>\max(n_1,n_2)\).
        \item
            Une chose à comprendre est que si \( z\in K_n\), alors \( d(z,\complement \Omega)\geq \frac{1}{ n }\). Du coup si nous prenons \( \delta\) tel que
            \begin{equation}
                \frac{1}{ n+1 }<\delta<\frac{1}{ n }
            \end{equation}
            alors \( B(z,\delta)\subset K_{n+1}\).
        \item
            Enfin, les \( K_n\) sont tous compacts. En effet ils sont bornés parce que \( K_n\subset B(0,n)\) et ensuite \( K_n\) est fermé en tant que complémentaire d'un ouvert (\( V_n\) est ouvert en tant qu'union d'ouverts).
    \end{enumerate}

    Nous passons maintenant aux propriétés, qui sont indépendantes de la façon dont nous avons construit les \( K_n\) vérifiant les conditions.
    \begin{enumerate}
        \item

            Nous pouvons considérer la fonction \( K_n\to \eR\) donnée par \( z\mapsto d(z,\complement K_{n+1})\). Vu que \( K_n\subset\Int(K_{n+1})\), c'est une fonction (continue sur le compact \( K_n\)) prenant des valeurs strictement positives. Elle a donc un minimum strictement positif. Si \( \delta_n\) est plus petit que ce minimum nous avons \( B(z,\delta_n)\subset K_{n+1}\) pour tout \( z\in K_n\).

        \item

            D'abord nous avons \( \Omega=\bigcup_{n=0}^{\infty}\Int(K_n)\). En effet nous avons
            \begin{equation}
                \Omega=\bigcup_{n=0}^{\infty}K_n\subset\bigcup_{n=0}^{\infty}\Int(K_{n+1})\subset\bigcup_{n=0}^{\infty}\Int(K_n).
            \end{equation}
            L'inclusion dans l'autre sens est facile.

            Soit \( K\) compact dans \( \Omega\). Vu que \( \Omega\) est l'union des \( \Int(K_n)\), nous avons
            \begin{equation}
                K\subset\bigcup_{n=0}^{\infty}\Int(K_n).
            \end{equation}
            Cela donne à \( K\) un recouvrement par des ouverts dont nous pouvons extraire un sous-recouvrement fini par compacité. Les \( K_n\) étant croissants, du recouvrement fini, il suffit de prendre le plus grand (disons \( K_m\)) et nous avons \( K\subset\Int(K_m)\).

    \end{enumerate}
\end{proof}
Notons qu'avec la suite de \( K_n\) telle que construite, le dernier point est réglé en prenant
\begin{equation}
    \frac{1}{ n+1 }<\delta_n<\frac{1}{ n }.
\end{equation}


\begin{theorem}[Tykhonov]\index{théorème!Tykhonov}\label{ThoFWXsQOZ}
    Un produit quelconque d'espaces métriques non vides est compact si et seulement si chacun de ses facteurs est compact.
\end{theorem}
Nous n'allons donner la preuve que dans le cas d'un produit fini dans le théorème~\ref{THOIYmxXuu}.

%---------------------------------------------------------------------------------------------------------------------------
\subsection{Ensembles enchaînés}
%---------------------------------------------------------------------------------------------------------------------------

Soit \( (x,d)\) un espace métrique.
\begin{definition}
    Une \defe{\( \epsilon\)-chaine}{chaine} joignant les points \( a\) et \( b\) de \( X\) est une suite finie \( (u_0,\ldots, u_n)\) dans \( X\) telle que \( u_0=a\), \( u_n=b\) et pour tout \( 0\leq i\leq n-1\) nous avons \( d(u_n,u_{n+1})\leq \epsilon\).

    Une partie \( A\) de \( X\) est \defe{bien enchaînée}{bien!enchaîné} si pour tout \( \epsilon>0\) et pour tout \( a,b\in A\), il existe une \( \epsilon\)-chaine joignant \( a\) et \( b\) dans $A$.
\end{definition}
Les rationnels dans \( \eR\) sont bien enchaînés.

\begin{proposition}
    Un espace connexe est bien enchaîné.
\end{proposition}
%TODO: une preuve.

\begin{proposition}
    La fermeture d'un ensemble bien enchaîné dans un espace métrique compact \( (X,d)\) est connexe.
\end{proposition}
\index{connexité}
\index{compacité}

\begin{proof}
    Soit \( A\subset X\) un ensemble bien enchaîné, et soient \( a,b\in \bar A\). Nous construisons une suite \( (u_k)\) dans \( A\) de la façon suivante. Pour chaque \( n>0\) nous prenons \( a'\in B(a,\frac{1}{ n })\cap A\) et \( b'\in B(b,\frac{1}{ n })\cap A\). Ensuite nous considérons une \( \frac{1}{ n }\)-chaine \( \{ v_i^{(n)} \}_{i\in I_n}\) dans \( A\) entre \( a'\) et \( b'\). Ici l'ensemble \( I_n\) est fini. La suite \( (u_k)\) est simplement construite en mettant bout à bout les éléments \( v_i^{(n)}\).

    La suite ainsi construite est une suite dans \( A\) admettant \( a\) et \( b\) comme points d'accumulation (les autres points d'accumulation sont également dans \( \bar A\)) et telle que \( \lim_{k\to \infty} d(u_k,u_{k+1})=0\). Par conséquent la proposition~\ref{PropLHWACDU} nous dit que l'ensemble des points d'accumulation de \( (u_k)\) est connexe dans \( X\). Nous le notons \( C_{a,b}\).

    Si nous fixons \( a\in \bar A\), alors nous avons
    \begin{equation}
        \bigcup_{x\in \bar A}C_{a,x}=\bar A.
    \end{equation}
    Vu que le membre de gauche est une union de connexes, c'est un connexe par la proposition~\ref{PropIWIDzzH}.
\end{proof}

\begin{corollary}       \label{CORooSIKCooTncoQm}
    Un espace métrique compact est connexe si et seulement s'il est bien enchaîné.
\end{corollary}

%---------------------------------------------------------------------------------------------------------------------------
\subsection{Produit fini d'espaces métriques}
%---------------------------------------------------------------------------------------------------------------------------

Pour rappel, la distance sur un espace produit est donnée par la définition \ref{DefZTHxrHA}.
\begin{theorem}[\cite{MonCerveau}]\label{THOIYmxXuu}
    Un produit fini d'espaces métriques non vides est compact si et seulement si chacun de ses facteurs est compact.
\end{theorem}
\index{compact!produit fini}
\index{théorème!Tykhonov!fini}

\begin{proof}
    Soient \( K_1\),\ldots, \( K_n\) des compacts et \( K=K_1\times \ldots\times K_n\) le produit muni de sa métrique usuelle de la définition \eqref{DefZTHxrHA} (attention : chacun des \( K_i\) peut être de dimension infinie) :
    \begin{equation}
        d(\alpha,\beta)=\max\{ d_i(\alpha_i,\beta_i) \}
    \end{equation}
    où \( d_i\) est la distance sur \( K_i\). Si \( (\alpha_n)\) est une suite dans \( K\) alors la suite \( (\alpha_n)_1\) est une suite dans le compact \( K_1\) dont nous pouvons extraire une sous-suite convergente (Bolzano-Weierstrass~\ref{ThoBWFTXAZNH}). De la sous-suite de \( \alpha\) correspondante nous extrayons la sous-suite pour la seconde composante, etc.

    En fin de compte nous avons une sous-suite (que nous nommons \( \alpha\) également) donc chacune des composantes est convergente. Nous nommons \( \ell_k\) les limites correspondantes. Soit \( \epsilon>0\) pour chaque \( k=1,\ldots, n\), il existe \( N_k>0\) tel que si \( p>N_k\) alors
    \begin{equation}
        d\big( (\alpha_p)_k-\ell_p \big)\leq \epsilon.
    \end{equation}
    Ici \( \alpha_p\in K\) est le \( p\)\ieme élément de la suite \( \alpha\) et \( (\alpha_p)_i\in K_i\) est la \( i\)\ieme composante de \( \alpha_p\). En prenant \( N=\max_kN_k\) et \( n>N\) nous avons
    \begin{equation}
        d\big( \alpha_n,(\ell_1,\ldots, \ell_n) \big)\leq\epsilon.
    \end{equation}
    Par conséquent de la suite \( (\alpha)\) nous avons extrait une sous-suite convergente et la partie «réciproque» de Bolzano-Weierstrass nous assure alors que \( K\) est compact.

    À l'inverse si un des facteurs n'est pas compact (mettons \( K_1\)) alors nous prenons un recouvrement \( \{ \mO_i \}_{i\in I}\) de \( K_1\) par des ouverts duquel il est impossible d'extraire un sous-recouvrement fini. Ensuite nous posons
    \begin{equation}
        \mP_i=\mO_i\times K_2\times\ldots\times K_n,
    \end{equation}
    qui est un recouvrement de \( K\) par des ouverts (de \( K\)) d'où aucun sous-recouvrement fini ne peut être extrait.
\end{proof}

Pour la culture générale, il y a bien entendu moyen de faire des produits dénombrables et pire d'espaces métriques.
\begin{definition}[\cite{AntoniniAndAl-TheoremeTykhonov}]
    Soient \( (E_n,d_n)\) des espaces métriques. Sur l'ensemble produit \( E=\prod_{i=1}^{\infty}E_i\) nous définissons la métrique
    \begin{equation}
        d(x,y)=\sum_{k=1}^{\infty}\frac{1}{ 2^k }d'_k(x_i,y_i)
    \end{equation}
    où \( d'_i=\min(d_i,1)\).
\end{definition}
On peut montrer que ce \( d\) est bien une distance et que \( (E,d)\) devient un espace métrique.

\begin{theorem}[Tykhonov dénombrable\cite{AntoniniAndAl-TheoremeTykhonov}] \label{ThoKKBooNaZgoO}  % Ce résultat n'est pas censé être utilisé dans l'agrégation.
    Un produit dénombrable d'espaces métriques non vides est compact si et seulement si chacun de ses facteurs est compact.
\end{theorem}
\index{compact!produit dénombrable}
\index{théorème!Tykhonov!dénombrable}
Note : ce résultat est encore valable pour un produit quelconque, c'est le théorème de Tykhonov~\ref{ThoFWXsQOZ}.

%--------------------------------------------------------------------------------------------------------------------------- 
\subsection{Équicontinuité}
%---------------------------------------------------------------------------------------------------------------------------

\begin{definition}[\cite{ooYDVWooPWLUGW}]       \label{DEFooSGMVooASNbxo}
    Soit une famille de fonctions \( f_i\colon X\to E\) indexée par un ensemble \( I\) où \( X\) est un espace topologique et \( E\) un espace métrique. Cette famille est \defe{équicontinue}{famille équicontinue} en \( x\in X\) si pour tout \( \epsilon>0\), il existe un voisinage \( V\) de \( x\) tel que
    \begin{equation}
        \| f_i(x)-f_i(y) \|<\epsilon
    \end{equation}
    pour tout \( i\) dès que \( x,y\in V\).

    Nous disons qu'une famille est équicontinue sans préciser en quel point si elle est équicontinue en tout pout.
\end{definition}

La proposition suivante permet de montrer que certaines fonctions définies par une limite sont continues. Ce sera par exemple le cas de la fonction puissance, proposition \ref{PROPooUQNZooSSHLqr}.
\begin{proposition}[\cite{MonCerveau,ooYDVWooPWLUGW}]     \label{PROPooICNNooAMjcut}
    Soit une suite équicontinue \( (f_i)\) de fonctions qui converge simplement vers \( f\), alors \( f\) est continue.
\end{proposition}

\begin{proof}
    Soit une suite équicontinue \( f_i\colon X\to E\) convergeant simplement vers \( f\). Soit \( a\in X\). Nous prouvons que \( f\) est continue en \( a.\). Pour cela nous considérons \( \epsilon>0\) et, conformément à l'hypothèse équicontinuité un voisinage \( V\) de \( a\) tel que \( | f_i(a)-f_i(x) |<\epsilon\) pour tout \( x\in V\).

    Nous avons la majoration
    \begin{subequations}
        \begin{align}
            | f(x)-f(a) |\leq | f(x)-f_i(x) |+| f_i(x)-f_i(a) |+| f_i(a)-f(a) |.
        \end{align}
    \end{subequations}
    Plusieurs majorations.
    \begin{itemize}
        \item 
            Vu que \( f_i\to f\), il existe \( N_1\) tel que \( | f(x)-f_i(x) |<\epsilon\) pour tout \( i>N_1\).
        \item
            De plus, par définition de \( V\), nous avons aussi \( | f_i(x)-f_i(a) |\leq \epsilon\).
        \item
            Vu que \( f_i\to f\), il existe \( N_2\) tel que \( | f_i(a)-f(a) |<\epsilon\) pour tout \( i>N_2\).
    \end{itemize}
    Donc en prenant \( x\in V\) et \( i>\max\{ N_1,N_2 \}\) nous avons
    \begin{equation}
        | f(x)-f(a) |\leq 3\epsilon.
    \end{equation}
\end{proof}


%--------------------------------------------------------------------------------------------------------------------------- 
\subsection{Continuité uniforme}
%---------------------------------------------------------------------------------------------------------------------------

\begin{definition}[\cite{ooDMOSooBYWrkwgTc}]\label{DEFooYIPXooQTscbG}
    Soient deux espaces métriques \( (E,d)\) et \( (E',d')\). Une application \( f\colon E\to E'\) est \defe{uniformément continue}{uniformément continue} si pour tout \( \epsilon>0\), il existe \( \delta>0\) tel que \( d(x,y)\leq \delta\) implique \( d'\big( f(x),f(y) \big)\leq \epsilon\).
\end{definition}
Dans l'uniforme continuité, le \( \alpha\) qui fait fonctionner \( \epsilon\) doit le faire fonctionner pour tous les \( x,y\in E\). C'est la différence avec la continuité simple dans laquelle nous pouvons choisir, pour un même \( \epsilon\), un \( \delta\) différent en chaque point.

%+++++++++++++++++++++++++++++++++++++++++++++++++++++++++++++++++++++++++++++++++++++++++++++++++++++++++++++++++++++++++++
\section{Ensembles nulle part denses}
%+++++++++++++++++++++++++++++++++++++++++++++++++++++++++++++++++++++++++++++++++++++++++++++++++++++++++++++++++++++++++++

Nous allons nous limiter au cas de \( \eR\), mais je crois que ça se généralise sans trop de peine aux espaces métriques, voire plus. Voir aussi la section~\ref{SecBDlaUrz} sur les espaces de Baire.

\begin{definition}
    Un ensemble est dit \defe{nulle part dense}{nulle part dense}\index{dense!nulle part} s'il n'est dense dans aucun intervalle.

    Un ensemble dans \( \eR\) est de \defe{première catégorie}{catégorie!ensemble de première} ou \defe{maigre}{maigre (ensemble)} s'il est une union dénombrable d'ensembles nulle part dense (c'est-à-dire d'ensembles denses sur aucun intervalle).
\end{definition}

\begin{theorem}[Baire\cite{BaireZied}]      \label{ThoQGalIO}
    Une réunion dénombrable d'ensembles nulle part denses est d'intérieur vide.
\end{theorem}
\index{Baire!théorème}
\index{théorème!Baire}

\begin{proof}
    Soient \( a\in S\) et \( \epsilon>0\). Nous allons trouver un élément dans \( B(a,\epsilon)\) qui n'est pas dans \( S\). Nous commençons par choisir \( x_1\in B(a,\epsilon)\) et \( r_1<\frac{ \epsilon }{2}\) tel que
    \begin{equation}
        B(x_1,r_1)\cap A_1=\emptyset.
    \end{equation}
    Ensuite nous choisissons \( x_2\in B(x_1,r_1)\) et \( r_2<\epsilon/4\) tel que \( B(x_2,r_2)\subset B(x_1,r_1)\) et \( B(x_2,r_2)\cap A_2=\emptyset\). Notons que \( B(x_2,r_2)\cap A_1=\emptyset\) aussi, par construction.

    Par récurrence nous construisons une suite d'éléments \( x_n\) et de rayons \( r_n<\epsilon/2^n\) tels que
    \begin{enumerate}
        \item
            \( B(x_n,r_n)\cap A_j=\emptyset\) pour tout \( j\leq n\),
        \item
            \( \overline{ B(x_n,r_n) }\subset B(x_{n-1},r_{r-1})\).
    \end{enumerate}
    Cette suite étant de Cauchy (parce que contenue dans des intervalles emboîtés de rayon décroissant vers zéro), elle converge\footnote{Par la proposition~\ref{PROPooTFVOooFoSHPg}} donc vers un point qui en particulier appartient à \( B(a,\epsilon)\). Mais la limite n'est dans aucun des \( A_n\) et donc pas dans \( S\).
\end{proof}

%+++++++++++++++++++++++++++++++++++++++++++++++++++++++++++++++++++++++++++++++++++++++++++++++++++++++++++++++++++++++++++
\section{Topologie des semi-normes}
%+++++++++++++++++++++++++++++++++++++++++++++++++++++++++++++++++++++++++++++++++++++++++++++++++++++++++++++++++++++++++++

Les principaux espaces topologiques construit avec des semi-normes seront les espaces de fonctions de la définition~\ref{DefFGGCooTYgmYf}. Nous verrons également la topologie \( *\)-faible sur \( \swD'(\Omega)\) en la définition~\ref{DefASmjVaT}.

\begin{definition}  \label{DefPNXlwmi}
    Si \( E\) est un espace vectoriel, une \defe{semi-norme}{semi-norme} sur \( E\) est une application \( p\colon E\to \eR\) telle que
    \begin{enumerate}
        \item
            \( p(x)\geq 0\),
        \item   \label{ItemSHnimhDii}
            \( p(\lambda x)=| \lambda |p(x)\)
        \item   \label{ItemSHnimhDiii}
            \( p(x+y)\leq p(x)+p(y)\).
    \end{enumerate}
\end{definition}

La seule différence avec une norme est simplement qu'une semi-norme peut s'annuler en des éléments non-nuls de l'espace.

\begin{lemma}[\cite{DRcmzcB}]
    Si \( p\) est une semi-norme nous avons
    \begin{equation}
        | p(x)-p(y) |\leq p(x-y).
    \end{equation}
\end{lemma}

\begin{proof}
    Nous avons d'une part \( p(x+h)\leq p(x)+p(h)\) et d'autre part \( p(x)\leq p(x+h)+p(-h)=p(x+h)+p(h)\). En isolant \( p(x+h)-p(x)\) dans chacune ce des deux inégalités,
    \begin{equation}
        -p(h)\leq p(x+h)-p(x)\leq p(h)
    \end{equation}
    ou encore
    \begin{equation}
        |p(x+h)-p(x)|\leq p(h)
    \end{equation}
    qui donne le résultat demandé en posant \( h=y-x\).
\end{proof}

Soit \( (p_i)_{i\in I}\) une famille de semi-normes sur \( E\). Nous construisons alors une topologie sur \( E\) de la façon suivante.

\begin{definition}[Topologie et semi-normes\cite{SOdaAdx,MUbDonp}]
    Pour tout \( J\) fini dans \( I\) nous définissons les \defe{boules ouvertes}{boule!avec semi-normes}
    \begin{equation}
        B_J(x,r)=\{ y\in E\tq p_j(y-x)<r\,\forall j\in J \}.
    \end{equation}
    La \defe{topologie}{topologie!et semi-normes} sur \( E\) donnée par la famille de semi-norme est définie en disant que \( \mO\subset E\) est ouvert si et seulement si chaque point de \( \mO\) est dans une boule contenue dans \( \mO\).
\end{definition}

\begin{proposition} \label{PropQPzGKVk}
    Une suite \( (x_n)\) dans \( E\) converge vers \( x\) au sens de la topologie des semi-normes si et seulement si pour tout \( i\in I\),
    \begin{equation}
        p_i(x-x_n)\to 0.
    \end{equation}
\end{proposition}

\begin{proof}
    Si la suite \( (x_n)\) converge\footnote{Définition~\ref{DefXSnbhZX}.} vers \( x\), alors pour tout ouvert \( \mO\) autour de \( x\), il existe un \( N\) tel que si \( n\geq N\), alors \( x_n\in\mO\). En particulier pour tout \( j\) et pour tout \( \epsilon>0\), il doit exister un \( n\geq N_j\) tel que \( x_n\in B_j(x,\epsilon)\).

    Voyons l'implication inverse. Soit \( \epsilon>0\). Pour tout \( i\in I\), il existe un \( N_i\) tel que \( n\geq N_i\) implique \( p_i(x-x_n)\leq \epsilon\). Si \( \mO\) est un ouvert, il doit contenir une boule du type \( B_J(x,r)\) pour un certain ensemble fini \( J\subset I\).

    En prenant \( N=\max\{ N_j\tq j\in J \}\), nous avons \( p_j(x-x_n)\leq \epsilon\) pour tout \( j\) et donc \( x_n\in B_J(x,r)\).
\end{proof}

La proposition suivante est un vulgaire plagiat de la proposition~\ref{PropQZRNpMn}.
\begin{proposition} \label{PropNGjQnqF}
    Soit \( f\colon \eR\to (E,p_i)_{i\in I}\) une application. Nous avons équivalence entre
    \begin{enumerate}
        \item   \label{ItemHNxGMpCi}
            la fonction \( f\) est continue en \( t_0\in \eR\),
        \item\label{ItemHNxGMpCii}
            si \( W\) est un voisinage ouvert de \( f(t_0)\) il existe un voisinage ouvert \( V\) de \( t_0\) (dans \( \eR\)) tel que \( f(V)\subset W\),
        \item\label{ItemHNxGMpCiii}
            pour tout \( i\in I\) et \( \epsilon>0\) il existe \( \delta>0\) tel que
            \begin{equation}
                f\big( B(t_0,\delta) \big)\subset B_i\big( f(t_0),\epsilon \big).
            \end{equation}
    \end{enumerate}
\end{proposition}

\begin{proof}
    L'équivalence~\ref{ItemHNxGMpCi} \( \Leftrightarrow\)~\ref{ItemHNxGMpCii} est la définition~\ref{DefOLNtrxB}.

    Prouvons~\ref{ItemHNxGMpCii} \( \Rightarrow\)~\ref{ItemHNxGMpCiii}. Soient \( i\in I\) et \( \epsilon>0\). Considérons la boule \( B_i\big( f(t_0),\epsilon \big)\), qui est un ouvert de \( E\) contenant \( f(t_0)\). Il existe donc un ouvert \( V\) autour de \( t_0\) tel que \( f(V)\subset B_i\big( f(t_0),\epsilon \big)\). En particulier \( V\) contient une boule \( B(t_0,\delta)\) et nous avons
    \begin{equation}
        f\big( B(t_0,\delta) \big)\subset f(V)\subset B_i\big( f(t_0),\epsilon \big).
    \end{equation}

    Prouvons~\ref{ItemHNxGMpCiii} \( \Rightarrow\)~\ref{ItemHNxGMpCii}. Soit \( W\) un ouvert autour de \( f(t_0)\). Il existe un \( i\in I\) et \( \epsilon>0\) tel que \( B_i\big( f(t_0),\epsilon \big)\subset W\). Nous avons alors un \( \delta>0\) tel que
    \begin{equation}
        f\big( B(t_0,\delta) \big)\subset B_i\big( f(t_0),\epsilon \big)\subset W.
    \end{equation}
\end{proof}

Lorsqu'on a un espace \( E\) muni d'une quantité dénombrable de semi-normes \( \{ p_k \}_{k\in I}\) nous définissons l'écart\footnote{Dans le cas de \( E=\swD(K)\), la première semi-norme est numérotée à zéro, donc il faudra poser \( d(\varphi_1,\varphi_2)\) avec \( p_{k-1}\) au lieu de \( p_k\).}
\begin{equation}        \label{EqAAghiUR}
    d(x,y)=\sup_{k\geq 1}\min\big\{  \frac{1}{ k },p_k(x-y) \big\}.
\end{equation}
Notons que cet écart est invariant par translation au sens où pour tout \( x,y,h\) dans \( E\) nous avons
\begin{equation}
    d(x+h,y+h)=\sup_{k\geq 1}\min\big\{ \frac{1}{ k },p_k(x-y) \big\}=d(x,y).
\end{equation}


\begin{proposition}     \label{PROPooMJEQooHtIyeX}
    Si \( X\) est un espace topologique dont la topologie est donnée par une famille dénombrable de semi-normes, alors il est métrisable.
\end{proposition}
%TODO : une preuve

\begin{proposition}[\cite{DRcmzcB}] \label{PropLOwUvCO}
    La topologie donnée par les boules
    \begin{equation}    \label{EqGHfYIlQ}
        B_k(a,r)=\{ x\in E\tq \forall\,k\leq \frac{1}{ r },p_k(x-a)<r\}
    \end{equation}
    est la même que celle «usuelle» donnée par les semi-normes. En disant «la même» nous entendons le fait que les ouverts sont les mêmes : \( A\) est ouvert pour une des deux topologies si et seulement s'il est ouvert pour l'autre.
\end{proposition}

\begin{proof}
    Pour cette démonstration nous allons préfixer par \( d\) les notions topologiques issues des boules \eqref{EqGHfYIlQ} et par \( P\) celle des semi-normes : \( P\)-continue, \( d\)-ouvert, etc.

    D'abord nous avons
    \begin{equation}    \label{EqRIURpQo}
        B(a,r)=\bigcap_{k\leq \frac{1}{ k }}B_k(a,r).
    \end{equation}
    Si \( \mO\) est un \( d\)-ouvert, il contient une \( d\)-boule autour de chacun de ses points. Or d'après la formule \eqref{EqRIURpQo}, une \( d\)-boule est une intersection \emph{finie} de \( P\)-ouverts et donc est un \( P\)-ouvert par définition. Donc \( \mO\) contient un \( P\)-ouvert autour de tous ses points et est donc \( P\)-ouvert.

    Inversement nous supposons que \( \mO\) est un \( P\)-ouvert. Commençons par prouver que les semi-normes \( p_k\) sont \( d\)-continues. En effet soient \( k\in \eN\), \( \epsilon\leq \frac{1}{ k }\) et \( x,y\in E\) tels que \( d(x,y)\leq \epsilon\); nous avons
    \begin{subequations}
        \begin{align}
            | p_k(y)-p_k(x) |&\leq p_k(x-y)\\
            &=\min\{ \frac{1}{ k },p_k(x-y) \}\\
            &\leq d(x,y)\\
            &\leq \epsilon.
        \end{align}
    \end{subequations}
    Montrons à présent que \( \mO\) est \( d\)-ouverte. Si \( a\in\mO\), il existe \( k\) et \( r\) tels que \( B_k(a,r)\subset\mO\). Soit \( x\in B_k(a,r)\). Montrons que si \( \epsilon\) est suffisamment petit, la \( d\)-boule \( B(x,\epsilon)\) est inclue à \( B_k(a,r)\). Pour cela prenons \( y\in B(x,\epsilon)\); nous avons
    \begin{equation}
        \big| p_k(a-x)-p_k(a-y) \big|\leq d(x,y)\leq \epsilon.
    \end{equation}
    Par conséquent le nombre \( p_k(a-y)\) est dans l'intervalle
    \begin{equation}
        p_k(a-x)\pm\epsilon
    \end{equation}
    et il suffit de prendre \( \epsilon<\frac{ r-p_k(a-x) }{2}\).
\end{proof}

%---------------------------------------------------------------------------------------------------------------------------
\subsection{Espace dual}
%---------------------------------------------------------------------------------------------------------------------------

Nous parlerons plus en détail d'espace dual d'un espace normé en la section~\ref{SECooKOJNooQVawFY}.

\begin{definition}  \label{DefHUelCDD}
    Soient \( F\) un espace métrique et \( E\) un espace topologique vectoriel. Une topologie possible\footnote{C'est, dans l'idée, celle qui sera choisie pour les espaces de distributions, voir la définition~\ref{DefASmjVaT}.} sur l'espace des applications linéaires \( \aL(E,F)\) est la \defe{topologie \( *\)-faible}{topologie!$*$-faible} qui est la topologie des semi-normes
    \begin{equation}
        p_v(T)=\| T(v) \|_F.
    \end{equation}
\end{definition}
C'est une famille de semi-normes indicées par les éléments de \( E\). Si \( E\) est un espace métrique, c'est cette topologie qui sera considérée sur son dual topologique\index{topologie!sur dual topologique} \( E'\) des applications continues \( E\to \eR\).

La proposition suivante indique qu'elle est un peu la topologie de la convergence ponctuelle.
\begin{proposition}
    Soient \( E\) un espace muni de la topologie des semi-normes \( \{ p_i \}_{i\in I}\) et \( F\) un espace métrique. Soient une suite \( (T_n)\) dans \( \aL(E,F)\) et \( T\in \aL(E,F)\). Nous avons \( T_n\stackrel{*}{\longrightarrow}T\) si et seulement si \( T_n(v)\stackrel{F}{\longrightarrow}T(v)\) pour tout \( v\in E\).
\end{proposition}

\begin{proof}
    Nous avons équivalence entre les lignes suivantes :
    \begin{subequations}
        \begin{align}
            T_n\stackrel{*}{\longrightarrow}T\\
            p_v(T_n-T)\to 0\,\forall v\in E &&\text{proposition~\ref{PropQPzGKVk}}\\
            \| T_n(v)-T(v) \|_E\to 0\,\forall v\in E\\
            T_n(v)\stackrel{E}{\longrightarrow}T(v).
        \end{align}
    \end{subequations}
\end{proof}

%---------------------------------------------------------------------------------------------------------------------------
\subsection{Espace \texorpdfstring{$ C^k(\eR,E')$}{C(R,E')}}
%---------------------------------------------------------------------------------------------------------------------------

Nous revenons à nos histoires de limites de la définition~\ref{DefXSnbhZX}.
\begin{proposition}[Unicité de la limite dans un dual topologique] \label{PropRBCiHbz}
    Soient \( E\) un espace métrique et \( E'\) son dual topologique muni de sa topologie de la définition~\ref{DefHUelCDD}. Il y a unicité de l'élément de \( E'\) vers lequel une fonction \( u\colon \eR\to E' \) peut converger.
\end{proposition}

\begin{proof}
    Soit \( T\) un élément vers lequel \( u_t\) converge lorsque \( t\to t_0\). Soient \( \epsilon>0\) et \( x\in E\). La boule \( B_x(T,\epsilon)\) de \( E'\) subordonnée à la norme \( p_x\) et centrée en \( T\) est un ouvert de \( E'\). Étant donné que \( u\) converge vers \( T\) il existe \( \delta>0\) tel que \( u_t\in B_x(T,\epsilon)\) dès que \( | t-t_0 |\leq \delta\). Nous avons donc, pour tout \( x\in E\), la limite (dans \( \eR\)) :
    \begin{equation}
        \lim_{t\to t_0} u_t(x)=T(x).
    \end{equation}
    Cela prouve que la convergence de \( u\) vers \( T\) implique l'existence pour tout \( x\) de la limite de \( u_t(x)\) dans \( \eR\). Si \( T'\) est un autre élément vers lequel \( u_t\) converge, nous avons par le même raisonnement que
    \begin{equation}
        \lim_{t\to t_0} u_t(x)=T'(x).
    \end{equation}
    Par unicité de la limite dans \( \eR\)
    %TODO : prouver ça et mettre une référence.
    nous devons alors avoir \( T(x)=T'(x)\) pour tout \( x\), c'est-à-dire \( T=T'\).
\end{proof}

\begin{proposition} \label{PropVKSNflB}
    Soit \( u\colon \eR\to E'\) une fonction continue. Alors
    \begin{enumerate}
        \item   \label{ItemLSJjfZdi}
            pour tout \( x\in E\) la fonction \( t\mapsto u_t(x)\) est continue,
        \item\label{ItemLSJjfZdii}
            pour tout \( x\in E\) nous avons la limite dans \( \eR\)
            \begin{equation}    \label{EqWKdFPVO}
                \lim_{t\to t_0} u_t(x)=u_{t_0}(x),
            \end{equation}
        \item\label{ItemLSJjfZdiii}
            nous avons la limite dans \( E'\)
            \begin{equation}
                \lim_{t\to t_0} u_t=u_{t_0}.
            \end{equation}
    \end{enumerate}
\end{proposition}

\begin{proof}
    Soient \( x\in E\) et \( \epsilon> 0\). Par la proposition~\ref{PropNGjQnqF} la continuité de \( u\) donne un \( \delta>0\) tel que
    \begin{equation}
        u_{B(t_0,\delta)}\subset B_x(u_{t_0},\epsilon).
    \end{equation}
    C'est-à-dire que si \( | t-t_0 |\leq \delta\) nous avons
    \begin{equation}
        \big| u_{t_0}(x)-u_t(x) \big|<\epsilon,
    \end{equation}
    ce qui signifie bien que la fonction \( t\mapsto u_t(x)\) est continue en tant que fonction \( \eR\to \eR\). Cela est le point~\ref{ItemLSJjfZdi}. Le théorème de limite et continuité dans \( \eR\) nous donne immédiatement la limite \eqref{EqWKdFPVO}.

    Nous passons à la preuve du point~\ref{ItemLSJjfZdiii}. Soit \( \mO\) un ouvert de \( E'\) contenant \( u_{t_0}\). Il existe donc un \( i\in I\) et \( \epsilon>0\) tel que \( B_i(u_{t_0},\epsilon)\subset \mO\). Étant donné que \( u\) est continue, il existe \( \delta>0\) tel que
    \begin{equation}
        u_{B(t_0,\delta)}\subset B_i(u_{t_0},\epsilon)\subset \mO.
    \end{equation}
    Cela signifie bien que
    \begin{equation}
        | t-t_0 |\leq \delta\Rightarrow u_t\in\mO,
    \end{equation}
    c'est-à-dire que nous avons la limite \( \lim_{t\to t_0} u_t=u_{t_0}\) dans \( E'\). Pour dire cela nous avons utilisé la définition~\ref{DefYNVoWBx} de la limite et le résultat d'unicité~\ref{PropRBCiHbz}.
\end{proof}

\begin{definition}  \label{DefDZsypWu}
    Si nous avons une application \( u\colon \eR\to E'\) nous considérons sa \defe{dérivée}{dérivée!fonction à valeurs dans $E'$} donnée par la limite
    \begin{equation}
        u'_{t_0}=\lim_{t\to t_0} \frac{ u_t-u_{t_0} }{ t-t_0 }.
    \end{equation}
    Cela est un nouvel élément de \( E'\) (pour peu que la limite existe). La fonction \( u'\colon \eR\to E'\) ainsi définie peut être continue ou non. Cela nous permet de définir les espaces \( C^k(\eR,E')\) et \( C^{\infty}(\eR,E')\).
\end{definition}
Une des principales utilisations que nous ferons de ces espaces seront les espaces de fonctions à valeurs dans les distributions tempérées dont nous parlerons dans la section~\ref{SecTEgDVWO}.

%+++++++++++++++++++++++++++++++++++++++++++++++++++++++++++++++++++++++++++++++++++++++++++++++++++++++++++++++++++++++++++
\section{Espaces de Baire}
%+++++++++++++++++++++++++++++++++++++++++++++++++++++++++++++++++++++++++++++++++++++++++++++++++++++++++++++++++++++++++++
\label{SecBDlaUrz}

\begin{definition}
    Un \defe{espace de Baire}{espace!de Baire}\index{Baire!espace} est un espace topologique dans lequel toute intersection dénombrable d'ouverts denses est dense.
\end{definition}

\begin{theorem}[Théorème de Baire\cite{SIdTHwW}]    \label{ThoBBIljNM}
    Les espaces suivants sont de Baire :
    \begin{enumerate}
        \item
            les espaces topologiques localement compacts,
        \item
            les espaces métriques complets (donc ceux de Banach en particulier),
        \item
            tout ouvert d'un espace de Baire.
    \end{enumerate}
\end{theorem}
\index{théorème!de Baire}
\index{Baire!théorème}
%TODO : une preuve c'est sans doute bien, et ça a l'air d'être pas trop dur et donné sur Wikipédia.

\begin{proof}
    \begin{subproof}
    \item[Espaces topologiques localement compacts]
        \item[Espaces métriques complets]
            Soit \( (E,d)\) un espace métrique complet. Soient \( V\) un ouvert quelconque de \( E\) et \( U_n\) une suite d'ouverts denses. Le but est de prouver que l'ensemble \( \bigcap_{n\in \eN}U_n\) intersecte \( V\). Vu que \( V\) est ouvert dans un espace métrique, il contient une boule ouverte et donc une boule fermée \( B_0\) de rayon strictement positif. L'ensemble \( U_1\) est dense et intersecte donc un ouvert contenu dans \( B_0\). L'intersection est un ouvert qui contient alors une boule fermée \( B_1\) de rayon strictement positif. Continuant ainsi nous construisons une suite de fermés emboités \( B_n\) telle que
            \begin{equation}
                \bigcap_{n\in \eN}U_n\cap V
            \end{equation}
            contient l'intersection des \( B_n\). Par le théorème~\ref{ThoCQAcZxX} des fermés emboîtés (que nous utilisons parce que \( E\) est métrique et complet), cette intersection est non vide.
        \item[Ouvert d'un espace de Baire]
    \end{subproof}
\end{proof}

Une des applications du théorème de Baire est le théorème de Banach-Steinhaus~\ref{ThoPFBMHBN}.




\chapter{Espaces affines}
% This is part of Mes notes de mathématique
% Copyright (c) 2011-2014,2020
%   Laurent Claessens
% See the file fdl-1.3.txt for copying conditions.

\begin{definition}
    Soit \( E\), un espace vectoriel. Un \defe{espace affine modelé sur}{affine!espace} \( E\) est un ensemble \( \affE\) sur lequel le groupe \( (E,+)\) agit à droite transitivement et librement\footnote{Définition \ref{DEFooQDHPooCfDEuL}.}.
\end{definition}

Étant donné que \( E\) est un groupe commutatif, l'action peut être vue indifféremment à gauche ou à droite. Si \( M\in\affE\) et si \( x\in E\) nous notons \( M+x\) au lieu de \( x\cdot M\) le résultat de l'action de \( x\) sur \( M\).

\begin{normaltext}      \label{NORMooZANAooQdXqlh}
    Lorsque nous écrivons «\( M+x\)», le symbole plus n'est pas une loi de composition interne de \( \affE\), mais une action.

    Soient \( N,M\in\affE\). Par liberté et transitivité de l'action, il existe un unique \( x\in E\) tel que \( M+x=N\). Ce vecteur \( x\) sera noté \( \vect{ MN }\).
\end{normaltext}

\begin{proposition}     \label{PROPooCOZCooCghwaR}
    Si \( A,B,C\in\affE\) nous avons les égalités suivantes dans \( E\) :
    \begin{enumerate}
        \item   \label{ITEMooSDMIooUQiKeW}
            \( \vect{ AB }+\vect{ BC }=\vect{ AC }\) (relations de Chasles)\index{relations!de Chasles}\index{Chasles},
        \item
            \( \vect{ AA }=0\),
        \item
            \( \vect{ BA }=-\vect{ AB }\).
    \end{enumerate}
\end{proposition}

\begin{normaltext}      \label{NORMooXAJLooIupekj}
    Si \( E\) est un espace vectoriel, le groupe \( (E,+)\) agit sur \( E\) par l'action \( t_y(x)=y+x\). Utilisant cette action nous construisons l'\defe{espace affine canonique}{canonique!espace affine} de \( E\). En particulier nous notons \( \affE_n(\eK)\) l'espace affine canonique de \( \eK^n\) vu comme espace vectoriel sur \( \eK\).
    \begin{itemize}
        \item
            En tant qu'ensembles, \( \affE_n(\eK)=\eK^n\).
        \item
            Sur cet espace en particulier, si \( M,N\in\affE_n(\eK)\), nous avons \( \vect{ MN }=N-M\) où à droite, la différence est la différence vectorielle dans \(\eK^n\).
    \end{itemize}

    Ces deux points se généralisent immédiatement à un espace vectoriel \( E\) au lieu de \( \eK^n\).
\end{normaltext}

%+++++++++++++++++++++++++++++++++++++++++++++++++++++++++++++++++++++++++++++++++++++++++++++++++++++++++++++++++++++++++++
\section{Repères cartésiens affines}
%+++++++++++++++++++++++++++++++++++++++++++++++++++++++++++++++++++++++++++++++++++++++++++++++++++++++++++++++++++++++++++

Soit \( E\) un \( \eK\)-espace vectoriel de dimension \( n\) et \( \affE\) un espace affine construit sur \( E\).
\begin{definition}      \label{DEFooQELZooEXvxgw}
    Un multiplet \( (A,e_1,\ldots, e_n)\) où \( A\) est un point de \( \affE\) et \( \{ e_i \}\) est une base de \( E\) est un \defe{repère cartésien}{repère!cartésien!espace affine} de \( \affE\).

    Nous disons que \( \{ e_i \}\) est la \defe{base associée}{base associée à un repère cartésien} au repère.
\end{definition}

\begin{proposition}
    Si \( \affE\) est un espace affine modelé sur l'espace vectoriel \( E\) de dimension \( n\) sur le corps \( \eK\), et si \(  \big( A,\{ e_i \}_{i=1,\ldots, n} \big)\) est un repère cartésien, alors
\begin{equation}
    \begin{aligned}
        \phi\colon \eK^n&\to \affE \\
        (x_1,\ldots, x_n)&\mapsto A+\sum_ix_ie_i.
    \end{aligned}
\end{equation}
est une bijection.

Ces nombres \( x_i\) sont les \defe{coordonnées}{coordonnées!dans un espace affine} du point \( A+\sum_ix_ie_i\) dans le repère \( (A,e_i)\).
\end{proposition}

\begin{proof}
    L'application \( \varphi\) est surjective parce que l'action de \( E\) sur \( \affE\) est transitive et injective parce que l'action est libre.
\end{proof}

%+++++++++++++++++++++++++++++++++++++++++++++++++++++++++++++++++++++++++++++++++++++++++++++++++++++++++++++++++++++++++++
\section{Classification affine des conique}
%+++++++++++++++++++++++++++++++++++++++++++++++++++++++++++++++++++++++++++++++++++++++++++++++++++++++++++++++++++++++++++

Soit une conique \( f(x,y)=0\) avec
\begin{equation}
    f(x,y)=ax^2+2bxy+cy^2+2dx+2ey+f
\end{equation}
dans le repère \( R=(A,e_i)\). La signature de la quadratique
\begin{equation}
    q(x,y)=ax^2+2bx+cy^2
\end{equation}
ne dépend pas de la base choisie et un changement de variables
\begin{subequations}
    \begin{numcases}{}
        \tilde x=\alpha x+\beta y\\
        \tilde y=\gamma x+\delta y
    \end{numcases}
\end{subequations}
peut nous amener dans trois cas :
\begin{equation}
    q(x,y)=\begin{cases}
        \tilde x^2+\tilde y^2    &   \text{genre ellipse}\\
        \tilde x^2-\tilde y^2    &    \text{genre hyperbole}\\
        \tilde x^2               &  \text{genre parabole}.
    \end{cases}
\end{equation}
Dans le troisième cas, la matrice de \( q\) est de rang \( 1\).

Nous cherchons maintenant à savoir si un point \( I=(x_0,y_0)\) est un centre de symétrie de \( f(x,y)=0\). Pour cela nous choisissons le repère centré en \( I\), c'est-à-dire que nous posons
\begin{subequations}
    \begin{numcases}{}
        x=x_0+\tilde x\\
        y=y_0+\tilde y.
    \end{numcases}
\end{subequations}
Un peu de calcul montre qu'alors la conique s'écrit
\begin{equation}
    f(x_0,y_0)+q(\tilde x,\tilde y)+(2ax_0+2by_0+2d)\tilde x+(2bx_0+2cy_0+2e)\tilde y=0.
\end{equation}
Le point \( I\) sera un centre de symétrique si les termes linéaires en \( \tilde x\) et \( \tilde y\) s'annulent, c'est-à-dire si
\begin{subequations}        \label{SyskhiOvW}
    \begin{numcases}{}
        ax_0+by_0+d=0\\
        bx_0+cy_0+e=0.
    \end{numcases}
\end{subequations}
Nous supposons que \( (d,e)\neq (0,0)\), sinon la conique de départ serait déjà centrée. Le déterminant du système \eqref{SyskhiOvW} est
\begin{equation}
    \delta=ac-b^2.
\end{equation}
Si ce dernier est différent de zéro, le système possède une unique solution et la conique aura alors un unique centre de symétrie.

Si le déterminant du système est nul, il y a soit pas de centre de symétrie, soit une infinité. Dans le premier cas nous sommes en présence d'une parabole, et dans le second cas de deux droites parallèles.

\begin{example}
    Soit
    \begin{equation}    \label{EqOgsEcz}
        f(x,y)=x^2+2xy-y^2-6x+2y-1=0
    \end{equation}
    donnée dans le repère affine \( R=(A,\{ e_i \})\). Nous commençons par étudier la signature de \( q(x,y)=x^2+2xy-y^2\) dont la matrice symétrique est
    \begin{equation}
        Q=\begin{pmatrix}
            1    &   1    \\
            1    &   -1
        \end{pmatrix}.
    \end{equation}
    Son polynôme caractéristique est \( \lambda^2-2\) sont les racines sont \( \pm\sqrt{2}\). La signature est donc \( (1,-1)\) et nous sommes en présence d'une conique de genre hyperbole. Nous cherchons le centre en posant \( x=\tilde x+x_0\), \( y=\tilde y+y_0\). Le système à résoudre est
    \begin{subequations}
        \begin{numcases}{}
            x_0+y_0-3=0\\
            x_0-y_0+1=0,
        \end{numcases}
    \end{subequations}
    dont l'unique solution est \( (x_0,y_0)=(1,2)\). Nous considérons le repère centré en \( (x_0,y_0)\), c'est-à-dire le repère
    \begin{equation}
        R'=(I,\{ e_i \})
    \end{equation}
    avec \( I=A+x_0e_1+y_0e_2\) où \( A\) est l'origine du repère dans lequel l'équation \eqref{EqOgsEcz} était donnée.

    Par construction dans ce repère nous avons la conique
    \begin{equation}
        f(x_0,y_0)+q(\tilde x,\tilde y)=0,
    \end{equation}
    c'est-à-dire
    \begin{equation}
        \tilde x^2+2\tilde x\tilde y-\tilde y^2=0.
    \end{equation}
    Maintenant la nous avons une quadrique centrée nous voulons la mettre sous une forme plus canonique :
    \begin{equation}
        \left( \frac{1}{ \sqrt{2} }(\tilde x+\tilde y) \right)^2-\tilde y^2-1=0.
    \end{equation}
    Nous posons donc
    \begin{subequations}
        \begin{numcases}{}
            X=\frac{1}{ \sqrt{2} }(\tilde x+\tilde y)\\
            Y=\tilde y,
        \end{numcases}
    \end{subequations}
    et nous trouvons l'hyperbole
    \begin{equation}
        X^2-Y^2-1=0.
    \end{equation}
    Cela revient à faire le changement de base
    \begin{subequations}    \label{EqfiVwym}
        \begin{numcases}{}
            e'_1=\sqrt{2}e_1\\
            e'_2=-e_1+e_2.
        \end{numcases}
    \end{subequations}
    Pour rappel, les vecteurs de bases se transforment avec la matrice inverse des coefficients. Étant donné que
    \begin{equation}
        \begin{pmatrix}
            X    \\
            Y
        \end{pmatrix}=\begin{pmatrix}
            1/\sqrt{2}    &   1/\sqrt{2}    \\
            0    &   1
        \end{pmatrix}\begin{pmatrix}
            \tilde x    \\
            \tilde y
        \end{pmatrix},
    \end{equation}
    nous avons
    \begin{equation}
        \begin{pmatrix}
            e'_1    \\
            e'_2
        \end{pmatrix}=\begin{pmatrix}
            1/\sqrt{2}    &   1/\sqrt{2}    \\
            0    &   1
        \end{pmatrix}^{-1}\begin{pmatrix}
            e_1    \\
            e_2
        \end{pmatrix}.
    \end{equation}
    C'est de là que provient le changement \eqref{EqfiVwym}.
\end{example}

%+++++++++++++++++++++++++++++++++++++++++++++++++++++++++++++++++++++++++++++++++++++++++++++++++++++++++++++++++++++++++++
\section{Applications affines}
%+++++++++++++++++++++++++++++++++++++++++++++++++++++++++++++++++++++++++++++++++++++++++++++++++++++++++++++++++++++++++++

La définition \ref{DEFooVTXWooVXfUnc} donnait déjà la définition d'une application affine entre deux espaces vectoriels. Voici maintenant le pendant entre deux espaces affines.
\begin{definition}      \label{DEFooUAWZooXcMKve}
    Soient \( \affE\) et \( \affE'\) deux espaces affines sur les espaces vectoriels \( E\) et \( E'\) (sur le même corps \( \eK\)). Une application \( f\colon \affE\to \affE'\) est dite \defe{affine}{affine!application} si pour tout \( M\in \affE\), il existe une application linéaire \( u_M\colon E\to E'\) telle que
    \begin{equation}    \label{EqMqIoWX}
        f(M+x)=f(M)+u_M(x)
    \end{equation}
    pour tout \( x\in E\).
\end{definition}

La définition suivante permet de décomposer une application affine en une partie linéaire et une translation. À partir de là, la proposition \ref{PROPooBPKKooJRAMeT} nous donnera une structure de groupe sur \( \Aff(\eR^n)\).

\begin{lemmaDef}[\cite{MonCerveau}]       \label{LEMooYJCDooOGAHkF}
    Soient \( \affE\) et \( \affE'\) deux espaces affines sur les espaces vectoriels \( E\) et \( E'\) (sur le même corps \( \eK\)). Nous considérons une application affine \( f\colon \affE\to \affE'\).

    Il existe une unique application linéaire \( u\colon E\to E'\) telle que
    \begin{equation}
        f(M+x)=f(M)+u(x)
    \end{equation}
    pour tout \( x\in E\) et pour tout \( M\in \affE\).

    Cette application linéaire est appelée \defe{partie linéaire}{partie linéaire} de \( f\). Pour varier les notations, nous noterons souvent \( f=\alpha\circ\tau_v\) pour une application linéaire \( \alpha\) et la translation \( \tau_v\) de vecteur \( v\).
\end{lemmaDef}

\begin{proof}
    En plusieurs coups.
    \begin{subproof}
        \item[Unicité]
            Supposons que \( u_1\) et \( u_2\) vérifient la propriété, alors pour tout \( x\in E\) et tout \( M\in \affE\) nous avons \( f(M+x)=f(M)+u_1(x)\) et \( f(M+x)=f(M)+u_2(x)\). Cela suffit à nous convaincre que \( u_1=u_2\).
        \item[\( u_M=u_N\)]
            Avant de prouver l'existence, nous considérons \( M,N\in \affE\) et les applications linéaires \( u_M\) et \( u_N\) vérifiant l'équation \eqref{EqMqIoWX} pour \( M\) et \( N\) respectivement. Prouvons que \( u_M=u_N\).

            Posons
            \begin{subequations}
                \begin{align}
                    f(M+x)&=f(M)+u_M(x)\\
                    f(N+y)&=f(N)+u_N(y).
                \end{align}
            \end{subequations}
            Définissons \( a\in E\) par \( N=M+a\); nous avons d'une part que
            \begin{equation}
                f(N+y)=f(M+y+a)=f(M)+u_M(y+a),
            \end{equation}
            et d'autre part
            \begin{equation}
                f(N+y)=f(M+a)+u_N(y)=f(M)+u_M(a)+u_N(y).
            \end{equation}
            Par conséquent \( u_M(y+a)=u_M(a)+u_N(y)\). Par linéarité \( u_N=u_M\).
        \item[Existence]
            Soit \( M\in \affE\). Nous affirmons que \( u_M\) fait l'affaire. En effet, soient \( N\in \affE\) et \( x\in E\). Vu que \( u_M=u_N\) nous avons
            \begin{equation}
                f(N+x)=f(M)+u_N(x)=f(M)+u_M(x).
            \end{equation}
            Donc effectivement \( u_M\) peut être utilisé en tout point de \( \affE\).
  \end{subproof}
\end{proof}
Ce lemme est important car il permet de démonter qu'une application est affine en prouvant la linéarité des \( u_M\) séparément sans devoir prouver qu'elles sont égales.

%--------------------------------------------------------------------------------------------------------------------------- 
\subsection{Autres propriétés}
%---------------------------------------------------------------------------------------------------------------------------

\begin{lemma}[\cite{MonCerveau}]       \label{LEMooXXTPooKYFGGM}
    Soient \( M\in \affE\) et \( A,B\in \affE\) deux points donnés par \( A=M+x_a\), \( B=M+x_b\). Soit encore une application affine \( f\) sur \( \affE\). Alors
    \begin{equation}
        \vect{ AB }=u_f(x_b-x_a).
    \end{equation}
\end{lemma}

\begin{proof}
    En appliquant \( f\) à \( A=M+x_a\) et \( B=M+x_b\),
    \begin{subequations}
        \begin{align}
            f(A)&=f(M)+u_f(x_a)\\
            f(B)&=f(M)+u_f(x_b).
        \end{align}
    \end{subequations}
    Donc \( f(B)=f(A)-u_f(x_a)+u_f(x_b)\) ou encore
    \begin{equation}
        f(B)=f(A)+u_f(x_b-x_a).
    \end{equation}
\end{proof}

\begin{remark}
    La condition \eqref{EqMqIoWX} pour tout \( M\in\affE\) est équivalente à demander
    \begin{equation}
        f\circ t_x=t_{u(x)}\circ f
    \end{equation}
    pour tout \( x\in E\).
\end{remark}

\begin{proposition}     \label{PROPooALXYooHoMdqQ}
    Soit \( f\) une application affine.
    \begin{enumerate}
        \item       \label{ITEMooSKCYooHyRZYN}
            Il existe une unique application linéaire \( u_f\) telle que \( f(M+x)=f(M)+u_f(x)\) pour tout \( M\in \affE\) et tout \( x\in E\).
        \item
            L'application \( u_f\) est injective si et seulement si \( f\) est injective.
        \item
            L'application \( u_f\) est surjective si et seulement si \( f\) est surjective.
    \end{enumerate}
    Si de plus les espaces \( \affE\) et \( \affE'\) ont même dimension finie, alors \( f\) est injective si et seulement si \( f\) est surjective.
\end{proposition}

\begin{proof}
    La partie \ref{ITEMooSKCYooHyRZYN} est le lemme \ref{LEMooYJCDooOGAHkF}.
\end{proof}

\begin{example}     \label{EXooAGINooYmvPML}
    L'espace \( \eR^n\) est très particulier parce qu'il agit sur lui-même; il est donc un espace affine à lui tout seul : \( \affE=E=\eR^n\).

    Dans le cas de \( \eR^n\), en posant \( M=0\) dans la condition \eqref{EqMqIoWX}, si \( f\) est une application affine il existe une application linéaire \( \alpha\) et un vecteur \( v\) tel que \( f=\tau_v\circ \alpha\).

    Notons que ça n'a pas de sens de poser \( M=0\), et la décomposition \( f=\tau_v\circ \alpha\) n'a aucun sens en général. En particulier, nous ne pouvons pas appliquer une application linéaire à un élément d'un espace affine général.
\end{example}

\begin{proposition}
    Si \( f\colon \affE\to \affE'\) et \( g\colon \affE\to \affE''\) sont des applications affines, alors \( g\circ f\colon \affE\to \affE''\) est affine et \( u_{g\circ f}=u_g\circ u_f\).
\end{proposition}

\begin{proof}
    Si \( M\in\affE\) et \( x\in E\) nous avons
    \begin{equation}
        \begin{aligned}[]
            (g\circ f)(M+x)&=g\big( f(M)+u_f(x) \big)\\
            &=f\big( f(M) \big)+u_g\big( u_f(x) \big)\\
            &=(g\circ f)(M)+(u_g\circ u_f)(x).
        \end{aligned}
    \end{equation}
\end{proof}

\begin{theorem}
    Soient \( \affE\) et \( \affE'\) deux espaces affines de dimensions finies \( p\) et \( q\) sur \( \eK\). Soient les repères cartésiens \( R=(O,\{ e_i \})\) et \( R'=(O',\{ e_i' \})\). Une application \( f\colon \affE\to \affE'\) est affine si et seulement s'il existe une matrice \( a\in\eM_{p,q}(\eK)\) et \( b\in \eK^q\) tels que
    \begin{equation}    \label{EqCmNHjs}
        f(x)=b+ax.
    \end{equation}
\end{theorem}

\begin{remark}
    L'équation \eqref{EqCmNHjs} est écrite en utilisant un abus de notation entre le vecteur \( x\in \eK^p\) et le point de \( \affE\) qui est représenté par \( x\) dans le repère \( (A,\{ e_i \})\).
\end{remark}

%+++++++++++++++++++++++++++++++++++++++++++++++++++++++++++++++++++++++++++++++++++++++++++++++++++++++++++++++++++++++++++
\section{Isomorphismes}
%+++++++++++++++++++++++++++++++++++++++++++++++++++++++++++++++++++++++++++++++++++++++++++++++++++++++++++++++++++++++++++

\begin{definition}
    Un \defe{isomorphisme}{isomorphisme!espace affine} entre les espaces affines \( \affE\) \( \affE'\) est une application affine \( f\colon \affE\to \affE'\) inversible dont l'inverse est affine.
\end{definition}

\begin{proposition} \label{PropxtFeDE}
    Une application affine bijective est un isomorphisme. Si \( f\) est un isomorphisme d'espaces affines, alors \( u_{f^{-1}}=(u_f)^{-1}\).
\end{proposition}

\begin{proposition}
    Un espace affine de dimension finie \( n\) sur un corps \( \eK\) est isomorphe à l'espace affine canonique \( \affE_n(\eK)\).
\end{proposition}

\begin{proof}
    Si nous considérons le repère \( R=(A,\{ e_i \})\) de l'espace affine \( \affE\) alors l'application
    \begin{equation}
        \begin{aligned}
            \varphi\colon \eK^n&\to \affE \\
            (x_1,\ldots,x_n)&\mapsto A+\sum_ix_ie_i
        \end{aligned}
    \end{equation}
    est un isomorphisme.
\end{proof}

%+++++++++++++++++++++++++++++++++++++++++++++++++++++++++++++++++++++++++++++++++++++++++++++++++++++++++++++++++++++++++++
\section{Sous espaces affines}
%+++++++++++++++++++++++++++++++++++++++++++++++++++++++++++++++++++++++++++++++++++++++++++++++++++++++++++++++++++++++++++

\begin{definition}
    Soit \( \affE\) un espace affine sur l'espace vectoriel \( E\). Un \defe{sous-espace affine}{affine!sous-espace} de \( \affE\) est une orbite de l'action d'un sous-espace vectoriel de \( E\).
\end{definition}

Si \( \affF\) est un sous-ensemble de \( \affE\), il sera un sous-espace affine de \( \affE\) si et seulement si l'ensemble
\begin{equation}
    F=\{ AB\tq A,B\in\affF \}
\end{equation}
est un sous-espace vectoriel de \( E\). Dans ce cas nous disons que \( F\) est la \defe{direction}{direction!sous-espace affine} de \( \affF\). Si \( A\in\affF\), alors l'orbite de \( A\) sous \( F\) est \( \affF\). La \defe{dimension}{dimension!sous espace affine} de \( \affF\) est la dimension de sa direction.

Si \( \affF\) et \( \affG\) sont des sous-espaces affines de \( \affE\) de directions \( F\) et \( G\), nous disons que \( \affF\) est \defe{parallèle}{parallèle!sous-espaces affines} à \( \affG\) si \( F\subset G\).

\begin{proposition}
    Soit \( \affF\) un sous-espace affine de dimension \( k\) dans l'espace affine \( \affE\) de dimension \( n\). Alors il existe une application affine \( f\colon \affE\to \eK^{n-k}\) telle que \( \affF=f^{-1}(0)\).
\end{proposition}

\begin{proof}
    Soient \( F\) la direction de \( \affF\) et \( A\in\affF\). Nous considérons une base \( \{ e_i \}\) adaptée à \( F\) au sens \( \{ e_1,\ldots, e_k \}\) est une base de \( F\). Nous considérons maintenant le repère cartésien \( (A,\{ e_i \})\) avec \( A\in\affF\) et nous construisons l'application affine
    \begin{equation}
        \begin{aligned}
            f\colon \affE&\to \eK^{n-k} \\
            A+\sum_{i=1}^nx_ie_i&\mapsto \begin{pmatrix}
                x_{k+1}    \\
                \vdots    \\
                x_n
            \end{pmatrix}.
        \end{aligned}
    \end{equation}
    Par construction nous avons \( f(M)=0\) si et seulement si \( M\in\affF\).
\end{proof}

\begin{proposition}[\cite{Combes}]      \label{PropomhBwi}
    Soit \( \sigma\) une partie de l'espace affine \( \affE\).
    \begin{enumerate}
        \item
            L'intersection de tous les sous-espaces affines contenant \( \sigma\) est un sous-espace affine, noté \( \affF\).
        \item
            Si \( A\in \sigma\), alors la direction de \( \affF\) est le sous-espace vectoriel
            \begin{equation}        \label{EqnRAUfg}
                F=\Span\{ \overrightarrow{ AM }\tq M\in \sigma \}.
            \end{equation}
    \end{enumerate}
\end{proposition}
Le sous-espace affine donné par la proposition~\ref{PropomhBwi} est le sous-espace affine \defe{engendré}{sous-espace!affine engendré par une partie} par la partie~\( \sigma\), et il est noté \( \eae(\sigma)\). \index{engendré!sous-espace affine}

\begin{proposition}     \label{PROPooAKJBooMkmsiV}
    Soit \( \affE\) un espace affine de dimension \( n\) sur \( \eK\), soit \( f\colon \affE\to \eK^r\) une fonction affine. Pour tout \( a=(a_1,\ldots, a_r)\in \eK^r\), l'ensemble \( f^{-1}(a)\) est un sous-espace affine de dimension \( \dim\ker(u_f)\).
\end{proposition}

\begin{proof}
    Nous considérons le repère \( (A,\{ e_i \})\) de \( \affE\). Étant donné que \( f\) est affine nous avons
    \begin{equation}
        f\big( A+\sum_ix_ie_i \big)=f(A)+u_f\big( \sum_ix_ie_i \big).
    \end{equation}
    Nous avons donc \( f\big( A+\sum_ix_ie_i \big)=a\) lorsque
    \begin{equation}
        u_f(\sum_ix_ie_i)=a-f(A).
    \end{equation}
    Nous avons donc
    \begin{equation}
        f^{-1}(a)=A+(u_f)^{-1}\big( a-f(A) \big),
    \end{equation}
    dont la dimension est le rang de \( (u_f)^{-1}=u_{f^{-1}}\) (proposition~\ref{PropxtFeDE}). Le rang de \( (u_f)^{-1}\) est le dimension du noyau de \( u_f\).
\end{proof}

\begin{definition}[Partie convexe]        \label{DEFooQQEOooAFKbcQ}
    Une partie \( A\) d'un espace vectoriel est \defe{convexe}{partie convexe} si pour tout \( a,b\in A\) et pour tout \( t\in \mathopen[ 0 , 1 \mathclose]\), le point \( ta+(1-t)b\) est dans \( A\).

    Autrement dit, une partie est convexe lorsqu'elle contient tous les segments joignant ses points.
\end{definition}

\begin{proposition}     \label{PROPooUQLUooDQfYLT}
    Soit un espace vectoriel normé\footnote{Définition \ref{DefNorme}.} \( (V,\| . \|)\). Pour tout \( a\in V\) et \( r>0\), la boule \( B(a,r)\) est convexe. La boule fermée \( \overline{ B(a,r) }\) également.
\end{proposition}

\begin{proof}
    En deux parties.
    \begin{subproof}
        \item[La boule centrée en zéro]
            Soient \( x,y\in B(0,r)\) et \( \lambda\in\mathopen] 0 , 1 \mathclose[\). Alors
                \begin{equation}
                    \| \lambda x+(1-\lambda)y \|\leq | \lambda |\| x \|+| 1-\lambda |\| y \|< (| \lambda | +| 1-\lambda |)r\leq r
                \end{equation}
                où nous avons utilisé le fait que \( | \lambda |=\lambda\) et \( | 1-\lambda |=1-\lambda\).

                Cela prouve que \( \lambda x+(1-\lambda)y\in B(0,r)\). Notez l'inégalité stricte due au fait que \( \| x \|<r\) et \( \| y \|<r\). Dans le cas de la boule fermée, nous avons une inégalité large.

            \item[La boule centrée autre part]

                Soient \( x,y\in B(a,r)\). Alors \( x-a\) et \( y-a\) sont dans \( B(0,r)\), de telle sorte que
                \begin{equation}
                    \lambda(x-a)+(1-\lambda)(y-a)\in B(0,r)
                \end{equation}
                par la première partie. En développant et simplifiant,
                \begin{equation}
                    \lambda x+(1-\lambda)y-a\in B(0,r),
                \end{equation}
                ce qui signifie que \( \lambda x+(1-\lambda)y\in B(a,r)\).
    \end{subproof}
\end{proof}

\begin{proposition}     \label{PropPoNpPz}
    Soit \( A\) un ensemble convexe\footnote{Définition \ref{DEFooQQEOooAFKbcQ}.} dans un espace vectoriel et \( v_1,\ldots, v_n\) des éléments de \( A\). Alors toute combinaison
    \begin{equation}
        a_1v_1+\cdots +a_nv_n
    \end{equation}
    telle que \( a_1+\cdots +a_n=1\) et \( a_i\in\mathopen[ 0 , 1 \mathclose]\) appartient à \( A\).
\end{proposition}

\begin{proof}
    Nous prouvons la proposition pour \( n=3\). Nous devons trouver des nombres \( t_1,t_2\in \mathopen[ 0 , 1 \mathclose]\) tels que
    \begin{equation}
        t_2\big( t_1v_1+(1-t_1)v_2 \big)+(1-t_2)v_3=av_1+bv_2+cv_3.
    \end{equation}
    La réponse est immédiatement donnée par
    \begin{subequations}
        \begin{align}
            t_2a=1-c\\
            t_1=a/t_2.
        \end{align}
    \end{subequations}
    Étant donné que \( c\in \mathopen[ 0 , 1 \mathclose]\) nous avons \( t_2\in\mathopen[ 0 , 1 \mathclose]\). En ce qui concerne \( t_1\) nous avons
    \begin{equation}
        t_1=\frac{ a }{ t_2 }\leq \frac{ 1-c }{ 1-c }=1.
    \end{equation}
\end{proof}

%+++++++++++++++++++++++++++++++++++++++++++++++++++++++++++++++++++++++++++++++++++++++++++++++++++++++++++++++++++++++++++
\section{Barycentre}
%+++++++++++++++++++++++++++++++++++++++++++++++++++++++++++++++++++++++++++++++++++++++++++++++++++++++++++++++++++++++++++

Soit \( \affE\) un espace affine sur le \( \eK\)-espace vectoriel \( E\). Un couple \( (A,\lambda)\) avec \( A\in \affE\) et \( \lambda\in \eK\) est un \defe{point pondéré}{point!pondéré}.

\begin{lemmaDef}[\cite{sZiwBQ}]        \label{LemtEwnSH}
    Soit une famille de points pondérés \( \{ (A_i,\lambda_i) \}_{i=1\ldots r}\). Si \( \sum_i\lambda_i\neq 0\), alors il existe un unique \( G\in \affE\) tel que
    \begin{equation}
        \sum_{i=1}^r\lambda_i\overrightarrow{ GA_i }=0.
    \end{equation}
    Le point \( G\) donné par le lemme~\ref{LemtEwnSH} est le \defe{barycentre}{barycentre!cas affine} des points pondérés \( (A_i,\lambda_i)\).
\end{lemmaDef}


Notons que l'on peut toujours supposer que \( \sum_i\lambda_i=1\) parce que le barycentre ne change pas lorsque tous les \( \lambda_i\) sont multipliés par un même nombre.
\begin{definition}[Combinaison convexe]\label{DefIMZooLFdIUB}
    Des nombres \( \lambda_1\),\ldots, \( \lambda_n\) vérifiant \( \sum_i\lambda_i=1\) forment une \defe{combinaison convexe}{combinaison!convexe}.
\end{definition}

Le théorème suivant donné quelques caractérisations équivalentes du barycentre.
\begin{theorem}[\cite{sZiwBQ}]      \label{ThoIJVzxr}
    Soient \( \{ (A_i,\lambda_i) \}_{i=1,\ldots, r}\) une famille de points pondérés. Les conditions suivantes sur le point \( G\in \affE\) sont équivalentes.
    \begin{enumerate}
        \item
            Le point \( G\) est barycentre de la famille.
        \item
            Pour tout \( \alpha\in \eR^*\), \( \sum_i(\alpha\lambda_i)\overrightarrow{ GA_i }=0\).
        \item
            Il existe \( A\in\affE\) tel que \( \big( \sum_i\lambda_i \big)\overrightarrow{ AG }=\sum_i\lambda_i\overrightarrow{ AA_i }\).
        \item   \label{ItemEgOQBX}
            Pour tout \( B\in\affE\), nous avons \( \big( \sum_i\lambda_i \big)\overrightarrow{ BG }=\sum_i\lambda_i\overrightarrow{ BA_i }\).
    \end{enumerate}
\end{theorem}

\begin{definition}
    Si \( A,B\in \affE\), le \defe{segment}{segment!dans un espace affine} \( [AB]\) est l'ensemble des barycentres de \( A\) et \( B\) pondérés par des poids positifs (ouvert ou fermé suivant que l'on accepte que l'un ou l'autre des poids soit nul).
\end{definition}

Lorsque tous les \( \lambda_i\) sont égaux, nous parlons d'\defe{isobarycentre}{isobarycentre}. Autrement dit, l'isobarycentre des points \( A_i\) est le barycentre des points pondérés \( (A_i,1)\).

%---------------------------------------------------------------------------------------------------------------------------
\subsection{Sous-espaces affines}
%---------------------------------------------------------------------------------------------------------------------------

\begin{proposition}
    Une partie \( \affF\) des \( \affE\) est un sous-espace affine si et seulement si elle est stable par barycentrisation.
\end{proposition}

\begin{proof}
    Soit \( \affF\) une sous-espace affine de direction \( F\) et \( A_1,\ldots, A_n\) des points de \( \affF\). Nous devons voir que le barycentre des points \( A_i\) pondérés de n'importe quelles masses appartient à \( \affF\). Pour ce faire nous faisons appel à la caractérisation~\ref{ItemEgOQBX} du théorème~\ref{ThoIJVzxr} : pour tout \( B\in\affF\),
    \begin{equation}
        \overrightarrow{ BG }=\sum_i\lambda_i\overrightarrow{ BA_i }.
    \end{equation}
    Vu que \( B\) et \( A_i\) sont dans \( \affF\), nous avons \( \overrightarrow{ BA_i\in F }\) et donc \( \overrightarrow{ BG }\in F\). Mais comme \( B\in\affF\), le point \( G\) est à son tour dans \( \affF\).

    Réciproquement, nous supposons que \( \affF\) est stable par barycentrisme. Nous voudrions montrer que l'ensemble
    \begin{equation}        \label{EqCmyWGi}
        F=\{ \overrightarrow{ AB }\tq A,B\in \affF \}
    \end{equation}
    est un sous-espace vectoriel. Soit \( A\in\affF\). Nous commençons par prouver que les vecteurs de la forme \( \overrightarrow{ AX }\) (\( X\in \affF\)) forment un espace vectoriel. Considérons \( \overrightarrow{ AX }+\overrightarrow{ AY }\) qui est un élément de \( E\); il existe donc \( V\in \affE\) tel que
    \begin{equation}
        \overrightarrow{ AV }=\overrightarrow{ AX }+\overrightarrow{ AY }.
    \end{equation}
    Par les relations de Chasles,
    \begin{equation}
        \overrightarrow{ AV }=\overrightarrow{ AV }+\overrightarrow{ VX }+\overrightarrow{ AV }+\overrightarrow{ VY },
    \end{equation}
    donc
    \begin{equation}
        0=\overrightarrow{ VX }-\overrightarrow{ VA }+\overrightarrow{ VY },
    \end{equation}
    ce qui prouve que \( V\) est un barycentre de \( X,A,Y\), et donc que \( V\in\affF\). De la même manière si \( W\in \affE\) est défini par \( \overrightarrow{ AW }=\mu \overrightarrow{ AX }\), alors
    \begin{equation}
        \overrightarrow{ AW }=\mu\overrightarrow{ AX }=\mu(\overrightarrow{ AW }+\overrightarrow{ WX }),
    \end{equation}
    ce qui signifie que
    \begin{equation}
        (1-\mu)\overrightarrow{ AW }+\mu\overrightarrow{ XW }=0
    \end{equation}
    et que \( W\) est un barycentre.

    Afin de montrer que \eqref{EqCmyWGi} est bien un espace vectoriel, nous devons considérer \( A,B,X,Y\in\affF\) et prouver que \( \overrightarrow{ AX }+\overrightarrow{ BY }\in F\). Nous avons
    \begin{subequations}
        \begin{align}
            \overrightarrow{ AX }+\overrightarrow{ BY }&=\overrightarrow{ AX }+\overrightarrow{ BA }+\overrightarrow{ AY }\\
            &=\overrightarrow{ AV }+\overrightarrow{ BA }   & V\text{ est celui donné plus haut}\\
            &=\overrightarrow{ AV }-\overrightarrow{ AB }  \\
            &=\overrightarrow{ AV }+\overrightarrow{ AW }   & W\text{ est donné par } \mu=-1\text{.}\\
            &=\overrightarrow{ AV' }.
        \end{align}
    \end{subequations}
\end{proof}

\begin{proposition}[\cite{sZiwBQ}]      \label{PropBVbCOS}
    Soient \( A_0,\ldots, A_r\) des points de \( \affE\). L'ensemble des barycentres de ces points (avec des masses de somme \( 1\)) est le sous-espace affine engendré par les \( A_i\) que nous nommons \( \affF\).
\end{proposition}

\begin{proof}
    Soit \( G\) le barycentre associé aux poids \( \lambda_i\). Nous avons
    \begin{equation}
        G=A_0+\overrightarrow{ A_0G }=A_0+\sum_{i=1}^r\lambda_i\overrightarrow{ A_0A_i }.
    \end{equation}
    Notons que les vecteurs \( \overrightarrow{ A_0A_i }\) sont dans la direction du sous-espace affine engendré par les \( A_i\) par \eqref{EqnRAUfg}. Donc \( G\) est bien dans \( \affF\).

    Inversement si $X$ est dans \( \affF\), on a
    \begin{equation}
        X=A_0+\sum_i\lambda_i\overrightarrow{ A_0A_i }
    \end{equation}
    parce que \( \sum_i\lambda_i\overrightarrow{ A_0A_i }\) est un élément général de la direction de \( \affF\). Du coup
    \begin{equation}
        \overrightarrow{ A_0X }=\sum_i\lambda_i\overrightarrow{ A_0A_i },
    \end{equation}
    et en utilisant la relation de Chasles sur chacun des \( \overrightarrow{ A_0A_i }\),
    \begin{equation}
        \overrightarrow{ A_0X }=\sum_i\lambda_i\big( \overrightarrow{ A_0X }+\overrightarrow{ XA_i } \big).
    \end{equation}
    De là nous concluons que
    \begin{equation}
        \big( 1-\sum_i\lambda_i \big)\overrightarrow{ A_0X }+\sum_i\lambda_i\overrightarrow{ A_iX }=0,
    \end{equation}
    ce qui signifie précisément que \( X\) est un barycentre des \( A_i\).
\end{proof}

\begin{proposition}
    Soient \( r+1\) point \( A_0,\ldots, A_r\) dans \( \affE\). Le sous-espace affine engendré par les \( A_i\) est au plus de dimension \( r\).
\end{proposition}

\begin{proof}
    La direction de l'espace engendré \( \Aff\{ A_i \}\) est l'espace
    \begin{equation}
        \Span\{ \overrightarrow{ A_0A_i }_{i=1,\ldots, r} \}
    \end{equation}
    qui est engendré par \( r\) vecteurs et donc est au plus de dimension \( r\).
\end{proof}

En deux mots, la proposition suivante signifie que le barycentre des barycentres est le barycentre.
\begin{proposition}[Associativité des barycentres\cite{NOojwDf}]        \label{PropSFvjFZb}
    Soit \( I=\{ 0,1,\ldots, n \}\) et une partition \( I=J_0\cup\ldots\cup J_r\). Soient des points \( a_0,\ldots, a_n\in \affE\) et \( \lambda_0,\ldots, \lambda_m\) des nombres tels que \( \sum_i\lambda_i\neq 0\). Nous supposons que \( \mu_k=\sum_{i\in J_k}\lambda_i\neq 0\) pour tout \( k\), et enfin nous nommons \( b_k\) le barycentre de la famille \( \{ (a_i,\lambda_i),i\in J_k \}\).

    Alors le barycentre de la famille \( \{ (b_k,\mu_k) \}_{k=1,\ldots, n}\) est le barycentre de la famille \( \{ (a_i,\lambda_i) \}_{i\in I}\).
\end{proposition}

\begin{proof}
    Nous nommons \( b\) le barycentre des \( b_k\) pondérés par les \( \mu_k\), donc par définition
    \begin{subequations}
        \begin{align}
            0=\sum_{k=0}^r\mu_k\vect{ bb_k }\\
            &=\sum_{k}\sum_{i\in J_k}\lambda_i\vect{ bb_k }\\
            &=\sum_{k=0}^r\sum_{i\in J_k}\lambda_i(\vect{ ba_i }+\vect{ a_ib_k })\\
            &=\sum_{k=0}^r\sum_{i\in J_k}\lambda_i\vect{ ba_i }  +\sum_{k=0}^r\underbrace{\sum_{i\in J_k}\lambda_i\vect{ a_ib_k }}_{=0}\\
            &=\sum_{i\in I}\lambda_i\vect{ ba_i }.
        \end{align}
    \end{subequations}
    Donc \( b\) est bien barycentre des \( a_i\) avec les poids \( \lambda_i\).
\end{proof}

%---------------------------------------------------------------------------------------------------------------------------
\subsection{Enveloppe convexe}
%---------------------------------------------------------------------------------------------------------------------------

\begin{definition}      \label{DefNLYYooXUHFUY}
    Soit \( A\) une partie d'un espace vectoriel \( E\). L'\defe{enveloppe convexe}{enveloppe!convexe} de \( A\), notée \( \Conv(A)\)\nomenclature[G]{\( \Conv(A)\)}{enveloppe convexe} est l'intersection de tous les convexes contenant \( A\).
\end{definition}
L'enveloppe convexe est un convexe. En effet soit \( C\) un convexe contenant \( A\) et \( x,y\in\Conv(A)\); alors \( x\) et \( y \) sont dans \( C\) et par conséquent le segment \( [x,y]\) est inclus dans \( C\). Ce segment étant inclus dans tout convexe contenant \( A\), il est inclus dans \( \Conv(A)\).

\begin{proposition}[\cite{IQeaKlP}] \label{PropSVvAQzi}
    Soit \( C\) un convexe dans l'espace affine \( \affE\) et une famille de points pondérés \( \{ (a_i,\lambda_i) \}_{i=1,\ldots, r}\) dont tous les poids sont positifs (et non tous nuls). Alors le barycentre est aussi dans \( C\).

    En d'autre termes, un convexe est stable par barycentrage à poids positifs\footnote{Sauf si on prend tous les poids nuls; mais contre ce genre d'idées, on ne peut rien faire.}.
\end{proposition}
\index{convexité!barycentre}

\begin{proof}
    Nous prouvons par récurrence. D'abord pour \( r=2\). Le barycentre des points pondérés \( (a_1,\lambda_1)\), \( (a_2,\lambda_2)\) est le point \( b\) tel que
    \begin{equation}        \label{EqFWEErRX}
        \lambda_1\vect{ ba_1 }+\lambda_2\vect{ ba_2 }=0.
    \end{equation}
    Par définition, ce qui est noté \( \vect{ ab }\) n'est rien d'autre que \( b-a\); en déballant \eqref{EqFWEErRX}, nous trouvons
    \begin{equation}
        \lambda_1(a_1-b)+\lambda_2(a_2-b)=0
    \end{equation}
    et donc
    \begin{equation}
        b=\frac{ \lambda_1 }{ \lambda_1+\lambda_2 }a_1+\frac{ \lambda_2 }{ \lambda_1+\lambda_2 }a_2,
    \end{equation}
    qui est bien un point du segment \( [a_1,a_2]\) parce que c'est une combinaison à coefficients positifs de somme \( 1\).

    Nous passons maintenant à la vraie récurrence avec un ensemble de points pondérés
    \begin{equation}
        A_r=\{ (a_1,\lambda_1),\ldots, (a_r,\lambda_r) \}
    \end{equation}
    de masse totale non nulle; et en vous laissant deviner ce que va désigner \( A_{r-1}\). Si une des masses est nulle (disons \( \lambda_r\)), alors le barycentre de \( A_r\) est le même que celui de \( A_{r-1}\) et l'hypothèse de récurrence nous enseigne que ledit barycentre est dans \( C\). Nous supposons donc que \( \lambda_i\neq 0\) pour tout \( i\). Dans ce cas le théorème d'associativité des barycentres~\ref{PropSFvjFZb} dit que le barycentre de \( A_r\) est le barycentre entre le barycentre de \( A_{r-1}\) et \( (a_r,\lambda_r)\), qui sont deux points de \( C\) par hypothèse de récurrence.
\end{proof}

Si \( E\) est un espace vectoriel et si \( x_i\in E\) et \( \lambda_i\in \eR\), alors le barycentre des couples \( (x_i,\lambda_i)\) est le point \( g\) tel que \( \sum_i\lambda_i\vect{ gx_i }\), c'est-à-dire \( \sum_i\lambda_i(x_i-g)=0\) ou encore
\begin{equation}
    \sum_i\lambda_ix_i=\sum_i\lambda_ig.
\end{equation}
Donc quitte à diviser tous les \( \lambda_i\) par la somme, nous pouvons supposer que la somme des poids est \( 1\). C'est pourquoi lorsque nous parlerons de barycentre dans un espace vectoriel sans contexte affin, nous allons toujours supposer \( \sum_i\lambda_i=0\) et avoir le barycentre
\begin{equation}
    g=\sum_i\lambda_ix_i.
\end{equation}

\begin{proposition} \label{PropYHMTmZX}
    Soit \( E\), un espace vectoriel et \( A\subset E\). L'enveloppe convexe \( \Conv(A)\) est l'ensemble des barycentres de familles finies de points affublés de masses positives.
\end{proposition}

\begin{proof}
    Nous notons \( \mB\) l'ensemble des dits barycentres. Par la proposition~\ref{PropSVvAQzi}, ces barycentres sont dans l'enveloppe convexe et donc \( \mB\subset\Conv(A)\). A contrario, si nous prouvons que \( \mB\) était convexe, alors nous aurions \( \Conv(A)\subset\mB\) parce que l'enveloppe convexe est l'intersection des convexes contenant $A$.

    Soient \( a,b\in\mB\), c'est-à-dire que l'on a \( a_0,\ldots, a_n\) et \( b_0,\ldots, b_m\) dans \( A\) ainsi que les nombres strictement positifs \( \lambda_0,\ldots, \lambda_n\) et \( \mu_0,\ldots, \mu_m\) tels que
    \begin{subequations}
        \begin{align}
            a&=\sum_i\lambda_ia_i       & \sum_{i=1}^n\lambda_i&=1\\
            b&=\sum_j\mu_jb_j           &\sum_{j=1}^n\mu_j&=1
        \end{align}
    \end{subequations}
    Un point du segment \( \mathopen[ a , b \mathclose]\) est de la forme \( p=ta+(1-t)b\) avec \( t\in \mathopen[ 0 , 1 \mathclose]\). En développant,
    \begin{equation}
        p=\sum_{i=0}^n(t\lambda_i)a_i+\sum_{j=0}^m(1-t)\mu_jb_j.
    \end{equation}
    Cela est le barycentre de la famille \( \{ (a_i,\lambda_i),(b_j,\mu_j) \}\), parce que la somme des coefficients est bien \( 1\) :
    \begin{equation}
        \sum_i(t\lambda_i)+\sum_j(1-t)\mu_j=t+(1-t)=1.
    \end{equation}
\end{proof}


\begin{theorem}[Carathéodory\cite{KXjFWKA}] \label{ThoJLDjXLe}
    Dans un espace affine de dimension \( n\), l'enveloppe convexe\footnote{Définition~\ref{DefNLYYooXUHFUY}.} de \( A\) est l'ensemble des barycentres à coefficients positifs ou nuls de familles de \( n+1\) points.
\end{theorem}
\index{barycentre!enveloppe convexe}
\index{dimension!utilisation}
\index{théorème!Carathéodory}

\begin{proof}
    Soit \( x\in\Conv(A)\); on sait par la proposition~\ref{PropYHMTmZX} que \( x\) est barycentre de points de \( A\) avec des coefficients positifs :
    \begin{equation}    \label{EqWJDwOTH}
        x=\sum_{k=1}^p\lambda_kx_k
    \end{equation}
    avec \( \sum_k\lambda_k=1\). Nous supposons que \( p>n+1\) (sinon le théorème est réglé), et nous allons faire une récurrence à l'envers en montrant qu'on peut aussi écrire \( x\) sous forme d'un barycentre de strictement moins de \( p\) points.

    Étant donné que \( p-1>n\), la famille \( \{ x+i-x_1 \}_{i=2,\ldots, p}\) est liée et il existe donc \( \alpha_1,\ldots, \alpha_p\in \eR\) tels que \( \sum_{i=2}^p\alpha_i(x_i-x_1)=0\), c'est-à-dire telle que
    \begin{equation}
        \sum_{i=2}^p\alpha_ix_i=\sum_{i=2}^p\alpha_ix_1.
    \end{equation}
    Nous posons \( \alpha_1=-\sum_{i=2}^p\alpha_1\). Remarquons qu'alors \( \sum_{i=1}^p\alpha_ix_i=0\) parce que
    \begin{equation}
        \sum_{i=1}^p\alpha_ix_i=\alpha_1x_1+\sum_{i=2}^p\alpha_ix_i=\alpha_1x_1+\sum_{i=2}^p\alpha_ix_1=\sum_{i=1}^p\alpha_ix_1=0.
    \end{equation}
    Par conséquent ça ne coûte rien de récrire \eqref{EqWJDwOTH} sous la forme
    \begin{equation}
        x=\sum_{i=1}^p(\lambda_i+t\alpha_i)x_i.
    \end{equation}
    Les \( \alpha_i\) ne sont pas tous nuls, mais leur somme est nulle, donc il y en a au moins un négatif. Nous notons
    \begin{equation}
        \tau=\min\{ -\frac{ \lambda_i }{ \alpha_i }\tq \alpha_i<0 \},
    \end{equation}
    et \( J\) l'ensemble de \( i\) pour lesquels ce minimum est atteint. Nous considérons aussi le nombres \( \mu_i=\lambda_i+\tau\alpha_i\). Plusieurs remarques.
    \begin{enumerate}
        \item
            Si \( j\in J\), alors \( \mu_j=0\)
        \item
            Si \( \alpha_i>0\) alors \( \mu_i\geq 0\), mais si \( \alpha_i<0\) alors
            \begin{equation}
                \lambda_i+\tau\alpha_i\geq \lambda_i+(-\frac{ \lambda_i }{ \alpha_i })\alpha_i=0m
            \end{equation}
            donc \( \mu_i\geq 0\) quand même.
        \item
            \( \sum_{i=1}^p\mu_i=1\), toujours parce que \( \sum_{i=1}^p\alpha_i=0\).
    \end{enumerate}
    Avec tout ça, nous avons
    \begin{equation}
        \sum_{i\notin J}\mu_ix_i=\sum_{i=1}^p\mu_ix_i=x.
    \end{equation}
    Et voilà, nous avons écrit \( x\) comme un barycentre à coefficients positifs de moins de \( p\) éléments parce que \( J\) n'est pas vide.
\end{proof}

\begin{corollary}   \label{CorOFrXzIf}
    Dans un espace affine de dimension finie, l'enveloppe convexe d'un compact est compacte.
\end{corollary}

\begin{proof}
    Soit \( A\) une partie compacte de l'espace vectoriel \( E\), et \( \Conv(A)\) son enveloppe convexe. Nous allons montrer que toute suite dans \( \Conv(A)\) admet une sous-suite convergente en écrivant un point de \( \Conv(A)\) comme le théorème de Carathéodory~\ref{ThoJLDjXLe} nous le suggère. Pour cela nous considérons le simplexe
    \begin{equation}
        \Lambda=\left\{  \lambda\in \eR^{n+1}\tq \sum_{k=1}^{n+1}\lambda_k=1\text{ et } \lambda_k\geq 0\forall k   \right\}.
    \end{equation}
    Montrons en passant que \( \Lambda\) est compact. Si \( \lambda_k\in \Lambda\) est une suite, alors chacun des \( \lambda_k\) est un \( (n+1)\)-uple de nombres dans \( \mathopen[ 0 , 1 \mathclose]\) :
    \begin{equation}
        k\mapsto (\lambda_k)_i
    \end{equation}
    est une suite qui possède une sous-suite convergente. En passant \( n+1\) fois à une sous-suite, nous tombons sur une suite convergente vers \( \lambda\in\Lambda\), grâce à la convergence composante par composante. De plus pour chaque \( k\) nous avons \( \sum_{i=1}^{n+1}(\lambda_k)_i=1\), et en passant à la limite, la somme étant une application continue, \( \sum_{i}\lambda_i=1\).

    Considérons l'application
    \begin{equation}
        \begin{aligned}
            f\colon \Lambda\times A^{n+1}&\to \Conv(A) \\
            (\lambda,x)&\mapsto \sum_{k=1}^{n+1}\lambda_kx_k.
        \end{aligned}
    \end{equation}
    C'est une application continue parce qu'elle est bilinéaire en dimension finie; son image est contenue dans \( \Conv(A)\) par la proposition~\ref{PropSVvAQzi}, et elle est surjective par le théorème de Carathéodory~\ref{ThoJLDjXLe}. Bref, \( \Conv(A)=f(\Lambda\times A^{n+1})\) est donc l'image d'un compact par une application continue; elle est donc compacte par le théorème~\ref{ThoImCompCotComp}.
\end{proof}
Notons que sans le théorème de Carathéodory, peut être que le nombre de points utiles pour décomposer les différents \( a_k\) n'était pas borné; dans ce cas nous aurions du prendre une infinité de sous-suites et rien n'aurait été sûr.

%---------------------------------------------------------------------------------------------------------------------------
\subsection{Applications affines et barycentre}
%---------------------------------------------------------------------------------------------------------------------------

\begin{proposition}[\cite{ooXDCOooDkRJHF}]      \label{PROPooGSPZooRnVgiU}
    Une application \( f\colon \affE\to \affE'\) entre deux espaces affines est affine si et seulement si pour tout système \( \{ (A_i,\lambda_i) \}_{i=1,\ldots, k}\) de barycentre \( G\) et de poids total non nul, le point \( f(G)\) est barycentre du système \( \{ \big(f(A_i),\lambda_i\big) \}\).
\end{proposition}

\begin{proof}
    En deux parties.
    \begin{subproof}
        \item[Si \( f\) est affine]

            Par définition d'un barycentre,
            \begin{equation}
                \sum_i\lambda_i\vect{ GA_i }=0.
            \end{equation}
            Nous considérons un point arbitraire \( O\in\affE\) et nous écrivons \( A_i=O+x_i\), \( G=O+x_g\). Ensuite nous utilisons le lemme~\ref{LEMooXXTPooKYFGGM} pour le calcul suivant :
            \begin{subequations}
                \begin{align}
                    \sum_i\lambda_i\vect{ g(G)f(A_i) }&=\sum_i\lambda_iu_f(x_i-x_g)\\
                    &=u_f\big( \sum_i\lambda_i(x_i-x_g) \big)\\
                    &=u_f\big( \sum_i\lambda_i\vect{ GA_i } \big)\\
                    &=u_f(0)=0.
                \end{align}
            \end{subequations}
            Donc \( f(G)\) est bien le barycentre du nouveau système.

        \item[Si \( f\) conserve les barycentres]

            Nous définissons \( u\) par \( f(O+x)=f(O)+u(x)\). À priori, ce \( u\) dépend de \( O\) et n'est pas linéaire.
            \begin{subproof}
            \item[\( u\) est linéaire]

                Soient \( M,N\in\affE\) et les éléments \( x_m,x_n\in E\) tels que \( \vect{ OM }=x_m\) et \( \vect{ ON }=x_n\). Nous définissons enfin \( P\) par
                \begin{equation}
                    \vect{ OP }=\alpha \vect{ OM }+\beta\vect{ ON },
                \end{equation}
                et \( P=O+x_p\). En décomposant \( \vect{ MO }\) et \( \vect{ NO }\) par les relations de Chasles de la proposition~\ref{PROPooCOZCooCghwaR}\ref{ITEMooSDMIooUQiKeW} nous avons
                \begin{equation}
                    (\alpha+\beta-1)\vect{ PO }-\alpha\vect{ PM }-\beta\vect{ PN }
                \end{equation}
                et donc \( P\) est barycentre du système
                \begin{equation}
                    \big\{ (\alpha+\beta-1,O),(\alpha,M),(\beta,N) \}.
                \end{equation}
                Le point \( f(P)\) sera barycentre du système
                \begin{equation}
                    \Big\{ \big( \alpha+\beta-A,f(O) \big),\big( \alpha,f(M) \big), \big( \beta,f(N) \big) \}.
                \end{equation}
                Cela signifie que
                \begin{equation}
                    (\alpha+\beta-1)\vect{ f(P)f(O) }-\alpha\vect{ f(P)f(M) }-\beta\vect{ f(P)f(N) }=0.
                \end{equation}
                En y substituant \( \vect{ f(P)f(O) }=u(-x_p)\), \( \vect{ f(P)f(M) }=u(x_m-x_p)\) et \( \vect{ f(P)f(N) }=u(x_mm-x_p)\) ainsi que \( x_p=\alpha x_m+\beta x_b\) nous trouvons
                \begin{equation}
                    u(\alpha x_m+\beta x_n)=\alpha u(x_m)+\beta u(x_n).
                \end{equation}
                Donc \( u\) est linéaire.

            \item[\( u\) ne dépend pas du point \( O\)]

                Il n'est pas besoin de démonter cela parce que la définition~\ref{DEFooUAWZooXcMKve} ne le demande pas. Note : c'est le lemme~\ref{LEMooYJCDooOGAHkF} qui dit que c'est par ailleurs vrai.
            \end{subproof}
    \end{subproof}
\end{proof}

%+++++++++++++++++++++++++++++++++++++++++++++++++++++++++++++++++++++++++++++++++++++++++++++++++++++++++++++++++++++++++++
\section{Repères, coordonnées cartésiennes et barycentriques}
%+++++++++++++++++++++++++++++++++++++++++++++++++++++++++++++++++++++++++++++++++++++++++++++++++++++++++++++++++++++++++++

\begin{definition}
    On dit que les points \( A_0,\ldots, A_r\in \affE\) sont \defe{affinement indépendants}{indépendance!affine} si le sous-espace affine engendré est de dimension \( r\).
\end{definition}

\begin{proposition}[\cite{sZiwBQ}]  \label{PropGAneHg}
    Pour \( r+1\) points \( A_0,\ldots, A_r\) dans \( \affE\), les propriétés suivantes sont équivalentes.
    \begin{enumerate}
        \item
            Les \( A_i\) sont affinement indépendants.
        \item
            Pour tout \( i=0,\ldots, r\), le point \( A_i\) n'est pas dans \( \Aff\{ A_0,\ldots, \hat A_i,\ldots, A_r \}\).
        \item\label{ItemrAzkIl}
            Les points \( A_0,\ldots, A_{r-1}\) sont affinement indépendants et \( A_r\notin\Aff\{ A_0,\ldots, A_{r+1} \}\).
        \item
            Il existe \( i\) tel que les vecteurs \( \overrightarrow{ A_kA_i }\) (\( k\in i\)) sont linéairement indépendants.
        \item\label{ItemFBfcuq}
            Pour tout \( i\in\{ 1,\ldots, r \}\), les vecteurs \( \overrightarrow{ A_kA_i }\) (\( k\neq i\)) sont linéairement indépendants.
    \end{enumerate}
\end{proposition}

Notons à propos de la condition~\ref{ItemrAzkIl} que l'existence d'un \( i\) tel que \( A_i\notin\Aff\{ A_0,\ldots, \hat A_i,\ldots, A_r \}\) n'implique pas l'indépendance des \( r+1\) points. En effet dans \( \eR^2\) nous considérons les \( 4\) points \( A_0=(0,0)\), \( A_1=(1,0)\), \( A_2=(2,0)\) et \( A_3=(0,1)\). Évidemment le point \( A_3\) n'est pas dans l'espace engendré par les trois autres; il n'empêche que ces points ne sont pas affinement indépendants parce que la direction est de dimension \( 2\) au lieu de \( 3\).

\begin{definition}  \label{DefguuwEO}
    Soit \( \affE\) un espace affine de dimension \( n\) et \( \affF\) un sous-espace affine de dimension \( k\). Un \defe{repère affine}{repère!affine} de \( \affF\) est la donnée de \( k+1\) points affinement indépendants de \( \affF\).
\end{definition}
Si \( \{ A_0,\ldots, A_n \}\) est un repère affine, le point \( A_0\) est l'\defe{origine}{origine!repère affine}. C'est un choix complètement arbitraire; et c'est bien cet arbitraire qui nous amènera à considérer les coordonnées barycentriques au lieu des coordonnées cartésiennes.

Soit \( M\in \affE\); par définition nous avons
\begin{equation}
    M=A_0+\overrightarrow{ A_0M }.
\end{equation}
Mais nous savons que les vecteurs \( \overrightarrow{ A_0A_i }\) forment une base de \( E\), nous avons donc des nombres \( \lambda_i\) tels que
\begin{equation}
    \overrightarrow{ A_0M }=\sum_{i=1}^n\lambda_i\overrightarrow{ A_0A_i }.
\end{equation}
Les nombres \( \lambda_i\) ainsi construits sont les \defe{coordonnées cartésiennes}{coordonnées!cartésiennes!dans un espace affine} du point \( M\) dans le repère \( \{ A_0,\ldots, A_n \}\) d'origine \( A_0\).

À partir de ces coordonnées, le point \( M\in\affE\) se retrouve par la formule
\begin{equation}
    M=A_0+\sum_{i=1}^n\lambda_i\overrightarrow{ A_0A_i }.
\end{equation}

\begin{proposition}[\cite{MonCerveau}]      \label{PROPooIXVBooPpKsDE}
    La paire \( \big( O,\{ e_1,\ldots, e_n \} \big)\) est un repère cartésien de \( \affE\) si et seulement si \( \{ O,O+e_1,\ldots, O+e_n \}\) est un repère affine.
\end{proposition}

\begin{proof}
    En deux parties.
    \begin{subproof}
        \item[Sens direct]
            Vue la proposition \ref{PropGAneHg}, il suffit de prouver que les vecteurs $\overrightarrow{O(O+e_i)}$ sont linéairement indépendants. Mais \( \overrightarrow{O(O+e_i)}=e_i\), donc oui, ils sont linéairement indépendants.
        \item[Sens inverse]
            Il s'agit d'utiliser la même proposition \ref{PropGAneHg} qui est encore fonctionnelle parce qu'elle est une équivalence.
    \end{subproof}
\end{proof}

Soient \( (A,e_i)\) et \( (A',e'_i)\) deux repères cartésiens pour l'espace affine \( \affE\). Soit \( (a_{ij})\) la matrice de changement de base entre \( \{ e_i \}\) et \( \{ e'_i \}\) dans \( E\). Nous voudrions trouver les \( x_i\) en termes des \( x'_i\).

Pour cela nous considérons un point \( M\) dans \( \affE\) et nous l'écrivons dans les deux bases. Cela fournit l'égalité
\begin{equation}        \label{EqcYfuMg}
    A+\sum_ix_ie_i=A'+\sum_ix'_ie'_i.
\end{equation}
Nous considérons les coordonnées \( (a_i)\) de \( A'\) dans le repère \( (A,e_i)\), c'est-à-dire
\begin{equation}    \label{EqZNwPHE}
    A'=A+\sum_ia_ie_i.
\end{equation}
En substituant \( e'_i=\sum_ka_{jk}e_k\) et \eqref{EqZNwPHE} dans \eqref{EqcYfuMg} nous trouvons
\begin{equation}
    \sum_kx_ke_k=\sum_ka_ke_k+\sum_{jk}a_{jk}x'_je_k,
\end{equation}
et par conséquent
\begin{equation}
    x_k=a_k+\sum_ja_{jk}x'_j.
\end{equation}

Les coordonnées barycentriques sont données par la proposition suivante.
\begin{proposition}[\cite{sZiwBQ}]      \label{PROPooTIRXooLAipRa}
    Soient \( A_0,\ldots, A_r\) des points affinement indépendants dans \( \affE\) et \( \affF=\Aff\{ A_0,\ldots, A_r \}\). Tout point \( M\in\affF\) s'écrit de façon unique comme barycentre\footnote{Définition \ref{LemtEwnSH}.} des \( A_i\) affectés de poids \( \lambda_i\) tels que \( \sum_{i=0}^r\lambda_i=1\).
\end{proposition}

\begin{proof}
    Nous avons vu plus haut (définition~\ref{DefguuwEO}) que l'affine indépendance des points \( A_i\) assurait que \( (A_0,\ldots, A_r)\) était un repère de \( \affF\).

    En ce qui concerne l'existence de l'écriture de \( M\) comme barycentre, nous savons que les sous-espace affine sont exactement les ensembles de barycentres (proposition~\ref{PropBVbCOS}), c'est-à-dire que si on a des points dans un sous-espace affine, alors les barycentres de ces points est encore dans le sous-espace affine.

    L'unicité est comme suit. Si \( M\) est barycentre des \( A_i\) avec poids \( \lambda_i\), nous écrivons la caractérisation~\ref{ItemEgOQBX} du théorème~\ref{ThoIJVzxr} avec \( B=A_0\) :
    \begin{equation}
        \overrightarrow{ A_0M }=\sum_{i=1}^r\lambda_i\overrightarrow{ A_0A_i }
    \end{equation}
    où la somme à droite s'étend à priori de \( 0\) à \( r\), mais comme \( \overrightarrow{ A_0A_0 }=0\), nous l'avons limitée à \( 1\). Si \( M\) s'écrit comme barycentre de deux façons différentes, nous aurions
    \begin{equation}
        \overrightarrow{ A_0M }=\sum_{i=1}^r\lambda_i\overrightarrow{ A_0A_i }=\sum_{i=1}^r\mu_i\overrightarrow{ A_0A_i }
    \end{equation}
    avec \( \sum_i\lambda_i=\sum_i\mu_i=1\). Étant donné que les points \( A_0,\ldots, A_r\) forment un repère, les vecteurs \( \overrightarrow{ A_0A_i }\) sont linéairement indépendants (point~\ref{ItemFBfcuq} de la proposition~\ref{PropGAneHg}) et donc \( \lambda_i=\mu_i\) pour \( i=1,\ldots, r\). La condition de somme des points égale à \( 1\) impose alors immédiatement \( \lambda_0=\mu_0\).
\end{proof}

\begin{definition}
    Soit un espace affine \( \affE\) de dimension \( n\). Soient des points affinement indépendants \( A_1,\ldots, A_n \). Pour \( M\in\affE\), la proposition \ref{PROPooTIRXooLAipRa} indique qu'il existe un unique choix de \( \lambda_i\) tel que
    \begin{subequations}
        \begin{numcases}{}
            \sum_i\lambda_i=1\\
            \sum_i\lambda_i \overrightarrow{MA_i}=0.
        \end{numcases}
    \end{subequations}
    Ces \( \lambda_i\) sont les \defe{coordonnées barycentriques}{coordonnées barycentrique} de \( M\) dans le repère \( \{ A_i \}_{i=1,\ldots, n}\).
\end{definition}

\begin{normaltext}      \label{NORMooOGHBooMjmouu}
    Soit \( \eR^2\) et les points non alignés \( A\), \( B\), \( C\). Les coordonnées barycentriques \( (\alpha,\beta,\gamma)\) dans ce système correspondent à l'unique \( X\in \eR^2\) tel que
    \begin{equation}
        \alpha \vect{ XA }+\beta\vect{ XB }+\gamma\vect{ XC }=0.
    \end{equation}
\end{normaltext}

\begin{example}
    Soient les points \( A=(3,1)\), \( B=(-1,2)\) et \( C=(0,-1)\) dans \( \eR^2\). Nous allons montrer qu'il forment un repère affine de \( \eR^2\). L'espace engendré par ces trois points est l'espace des
    \begin{equation}
        A+\alpha\overrightarrow{ AB }+\beta\overrightarrow{ AC },
    \end{equation}
    et la direction correspondante est l'espace vectoriel donné par \( \alpha\overrightarrow{ AB }+\beta\overrightarrow{ AC }\) qui est de dimension deux. Donc l'espace affine engendré par \( A\), \( B\) et \( C\) est de dimension \( 2\).
\end{example}

\begin{example}
    Dans le repère \( (A,B,C)\), quel est le point de coordonnées barycentriques \( (\frac{1}{ 6 },\frac{1}{ 3 },\frac{1}{ 2 })\) ? D'abord nous vérifions que
    \begin{equation}
        \frac{1}{ 6 }+\frac{1}{ 3 }+\frac{1}{ 2 }=1.
    \end{equation}
    Ensuite nous cherchons \( X\in \eR^2\) tel que
    \begin{equation}
        \frac{1}{ 6 }\overrightarrow{ AX }+\frac{1}{ 3 }\overrightarrow{ BX }+\frac{1}{ 2 }\overrightarrow{ CX }=0,
    \end{equation}
    c'est-à-dire
    \begin{equation}
        \frac{1}{ 6 }\begin{pmatrix}
            x-3    \\
            y-1
        \end{pmatrix}+\frac{1}{ 3 }\begin{pmatrix}
            x+1    \\
            y-2
        \end{pmatrix}+\frac{1}{ 2 }\begin{pmatrix}
            x    \\
            y+1
        \end{pmatrix}=0.
    \end{equation}
    Nous trouvons immédiatement \( x=1/6\) et \( y=1/3\). Le point cherché est donc le point \( \begin{pmatrix}
        1/6    \\
        1/3
    \end{pmatrix}\).
\end{example}

\begin{lemma}[\cite{MonCerveau}]       \label{LEMooDUMVooFtfFOe}
    Une application affine \( f\colon \affE\to \affE\) qui préserve les points d'une base affine de $\affE$ est l'identité.
\end{lemma}

\begin{proof}
    Une base affine de \( \affE\) consiste en \( n+1\) points \( \{ A_0,\ldots, A_n \}\) affinement indépendants. Nous utilisons la proposition \ref{PROPooIXVBooPpKsDE} pour dire que \( \big( A_0, \{ \overrightarrow{A_0A_i} \}_{i=1,\ldots, n} \big)\) est un repère cartésien.
    
    En utilisant la formule du lemme \ref{LEMooYJCDooOGAHkF},
    \begin{equation}
        f(A_i)=f(A_0+\overrightarrow{A_0A_i})=f(A_0)+u(\overrightarrow{A_0A_i}).
    \end{equation}
    Donc $A_i=A_0+u(\overrightarrow{A_0A_i})$, ce qui signifie que
    \begin{equation}        \label{EQooAJYWooBTucpp}
        u(\overrightarrow{A_0A_i})=\overrightarrow{A_0A_i}
    \end{equation}
    
    Par ailleurs, tout point \( M\)\footnote{Même les points qui ne s'appellent pas «\( M \)» en fait.} de \( \affE\) peut être écrit sous la forme
    \begin{equation}
        M=A_0+\sum_i\lambda_i\overrightarrow{A_0A_i}.
    \end{equation}
    En appliquant \( M\), et en utilisant \eqref{EQooAJYWooBTucpp},
    \begin{equation}
        f(M)=f(A_0)+\sum_i\lambda_iu(\overrightarrow{A_0A_i})=A_0+\sum_i\lambda_i\overrightarrow{A_0A_i}=M.
    \end{equation}
    
    Donc tout point de \( \affE\) est fixé par \( f\), ce qui signifie que \( f\) est l'identité.
\end{proof}

%---------------------------------------------------------------------------------------------------------------------------
\subsection{Équation de droite}
%---------------------------------------------------------------------------------------------------------------------------

Soit \( \affE\) un espace affine de dimension trois muni d'un repère barycentrique. Une droite est donnée par trois nombres : \( D=D(a,b,c)\) est l'ensemble des points dont les coordonnées barycentriques (normalisées) \( (x,y,z)\) vérifient \( ax+by+cz=0\). C'est un espace de dimension un parce qu'il y a aussi la condition \( x+y+z=1\).

La droite \( D(1,1,1,)\) n'existe pas parce que ce serait \( x+y+z=0\), qui est incompatible avec \( x+y+z=1\).

Les droites \( D(a,b,c)\) et \( D(a',b',c')\) s'intersectent selon les solutions du système
\begin{subequations}
    \begin{numcases}{}
        x+y+z=1\\
        ax+by+cz=0\\
        a'x+b'y+c'z=0
    \end{numcases}
\end{subequations}
Donc deux droites affines ont un unique point d'intersection si et seulement si
\begin{equation}
    d=\begin{vmatrix}
        1    &   1    &   1    \\
        a    &   b    &   c    \\
        a'    &   b'    &   c'
    \end{vmatrix}\neq 0.
\end{equation}
Elles seront parallèles ou confondues si et seulement si \( d=0\).

%---------------------------------------------------------------------------------------------------------------------------
\subsection{Associativité, coordonnées barycentriques dans un triangle}
%---------------------------------------------------------------------------------------------------------------------------

\begin{lemma}[\cite{ooIXMCooHQKnee}]
    Soient trois points non alignés \( A\), \( B\), \( C\) ainsi que des nombres \( \alpha\), \( \beta\), \( \gamma\) tels que \( \alpha+\beta\neq 0 \) et \( \alpha+\beta+\gamma\neq 0\).

    Soit \( H\) le barycentre du système \( \{ (A,\alpha),(B,\beta) \}\) et \( G\) le barycentre de \( \{ (A,\alpha), (B,\beta),(C,\gamma) \}\).

    Alors \( G\) est barycentre de \( \{ (H,\alpha+\beta),(C,\gamma) \}\).
\end{lemma}

\begin{proof}
    Vues les définitions de \( H\) et \( G\) nous avons
    \begin{subequations}
        \begin{align}
            \alpha\vect{ HA }+\beta\vect{ HB }&=0\\
            \alpha\vect{ GA }+\beta\vect{ GB }+\gamma\vect{ GC }&=0.
        \end{align}
    \end{subequations}
    En utilisant les relations de Chasles nous introduisons \( H\) dans la seconde relation :
    \begin{subequations}
        \begin{align}
            \alpha(\vect{ GH }+\vect{ HA })+\beta(\vect{ GH }+\vect{ HB })+\gamma\vect{ GC }&=0\\
            (\alpha+\beta)\vect{ GH }+\underbrace{  \alpha\vect{ HA }+\beta\vect{ HB }   }_{=0}+\gamma\vect{ GC }&=0\\
            (\alpha+\beta)\vect{ GH }+\gamma\vect{ GC }&=0.
        \end{align}
    \end{subequations}
\end{proof}

Les coordonnées barycentriques dans un triangle (et plus généralement en fait) permettent de faire des projections.

\begin{proposition}     \label{PROPooBCUVooWKttiH}
    Soient trois points non alignés \( A\), \( B\), \( C\) ainsi qu'un point \( N\) de coordonnées barycentriques  \( (\alpha,\beta,\gamma) \) dans le système $(A,B,C)$. Si \( P\) est l'intersection \( (AN)\cap(BC)\) alors les coordonnées de \( P\) sont \( (0,\beta,\gamma)\).
\end{proposition}

\begin{proof}
    Un dessin de la situation :

    \begin{center}
        \input{auto/pictures_tex/Fig_GYODoojTiGZSkJ.pstricks}
    \end{center}

    Dire que les coordonnées de \( N\) sont \( (\alpha,\beta,\gamma)\) signifie que
    \begin{equation}
        \alpha\vect{ NA }+\beta\vect{ NB }+\gamma\vect{ NC }=0.
    \end{equation}
    Nous voudrions montrer que le point \( P\) est bien le point de coordonnées \( (0,\beta,\gamma)\). Soit donc le point \( P\) tel que
    \begin{equation}        \label{EQooYLGDooWqKKOM}
        \beta\vect{ PB }+\gamma\vect{ PC }=0
    \end{equation}
    et montrons que ce point est l'intersection \( (BC)\cap (NA)\).

    D'abord la relation \eqref{EQooYLGDooWqKKOM} nous dit immédiatement que \( P\) est sur la droite \( (BC)\). Ensuite, en utilisant les relations de Chasles pour introduire \( N\) :
    \begin{equation}
        \beta(\vect{ PN }+\vect{ NB })+\gamma(\vect{ PN }+\vect{ NC })=0.
    \end{equation}
    Nous remplaçons \( \beta\vect{ NB }+\gamma\vect{ NC }\) par \( -\alpha\vect{ NA }\) pour obtenir :
    \begin{equation}
        (\beta+\gamma)\vect{ PN }-\alpha\vect{ NA }=0.
    \end{equation}
    Cela montre que les vecteurs \( \vect{ PN }\) et \( \vect{ NA }\) sont colinéaires, et donc que \( P\), \( N\) et \( A\) sont alignés.
\end{proof}

%+++++++++++++++++++++++++++++++++++++++++++++++++++++++++++++++++++++++++++++++++++++++++++++++++++++++++++++++++++++++++++
\section{Applications affines sur \( \eR^n\)}
%+++++++++++++++++++++++++++++++++++++++++++++++++++++++++++++++++++++++++++++++++++++++++++++++++++++++++++++++++++++++++++

Soit \( v\in \eR^n\); nous notons \( \tau_v\colon \eR^n\to \eR^n\) la translation donnée par \( \tau_v(x)=x+v\). Le groupe de toutes les translations de \( \eR^n\) est noté \( T(n)\) et est isomorphe au groupe abélien \( (\eR^n,+)\).

Nous avons déjà discuté de la structure d'un espace vectoriel (en particulier \( \eR^n\)) comme espace affine en~\ref{NORMooXAJLooIupekj}.


\begin{lemma}\label{LEMooZZAIooOMiayy}
    Décomposition d'une application affine.
    \begin{enumerate}
        \item       \label{ITEMooSJBFooYHURto}
            Une application \( f\colon E\to E\) est affine si et seulement si il existe \( v\in E\) et une application linéaire \( \alpha\) sur \( E\) telle que \( f=\tau_v\circ\alpha\).
        \item       \label{ITEMooPYEOooKIesYm}
            Dans ce cas, le choix de \( (v,\alpha)\) est unique.
        \item       \label{ITEMooHIAUooRxfTqx}
            Si \( f\) est bijective, alors \( \alpha\) est bijective.
    \end{enumerate}
\end{lemma}

\begin{proof}
    Nous supposons d'abord que \( f\) est affine. Alors il existe une application linéaire \( u_f\) sur \( E\) telle que
    \begin{equation}
        f(M+x)=f(M)+u_f(x)=(\tau_{f(M)}+u_f)(x)
    \end{equation}
    pour tout \( x\) et \( M\). De plus l'application \( u_f\) ne dépend ni de \( M\) ni de \( x\) (c'est la proposition~\ref{PROPooALXYooHoMdqQ}\ref{ITEMooSKCYooHyRZYN}). En posant \( M=0\) nous avons :
    \begin{equation}
        f(x)=(\tau_{f(0)}\circ u_f)(x).
    \end{equation}

    Dans l'autre sens nous supposons avoir \( v\in E\) et \( \alpha\) linéaire sur \( E\) telles que
    \begin{equation}
        f(M)=(\tau_v\circ\alpha)(M).
    \end{equation}
    Notons qu'il y a un abus de notation entre \( \alpha\) qui est linéaire sur l'espace \emph{vectoriel} \( E\) et l'application \( \alpha\) qui est une application sur l'espace \emph{affine} \( E\). Cet abus est légitime parce que les deux espaces sont identiques en tant qu'ensembles. Ce qui est vraiment abuser par contre, c'est de se poser ce genre de questions.

    Nous avons :
    \begin{equation}
        \begin{aligned}[]
            f(M+x)=\tau_v\big( \alpha(M+x) \big)=\alpha(M+x)+v&=\alpha(M)+v+\alpha(x)
            \\&=(\tau_v\circ\alpha)(M)+\alpha(x)=f(M)+\alpha(x).
        \end{aligned}
    \end{equation}
    Donc la fonction \( f\) vérifie la définition~\ref{DEFooUAWZooXcMKve}. La partie \ref{ITEMooSJBFooYHURto} est prouvée.

    Pour prouver l'unicité de la partie \ref{ITEMooPYEOooKIesYm}, nous supposons que \( \tau_v\circ\alpha=\tau_w\circ \alpha\). En appliquant cela à \( 0\) nous trouvons \( v=w\). Nous avons donc \( \tau_v\circ \alpha=\tau_v\circ\beta\). Comme \( \tau_v\) est inversible, nous en déduisons \( \alpha=\beta\).

    Enfin le point \ref{ITEMooHIAUooRxfTqx} est relativement évident du fait que \( \tau_v\), elle, est surement bijective.
\end{proof}

\begin{corollary}       \label{CORooATCNooUwEPNI}
    Une application affine qui conserve l'origine est linéaire.
\end{corollary}

\begin{proof}
    Conserver l'origine demande de poser \( v=0\) dans l'expression du lemme~\ref{LEMooZZAIooOMiayy}.
\end{proof}

\begin{proposition}     \label{PROPooYRCJooIcmUVI}
    Soit une application affine \( f\colon \eR^n\to \eR^n\). L'ensemble des points fixes
    \begin{equation}
        \Fix(f)=\{ x\in \eR^n\tq f(x)=x \}
    \end{equation}
    est soit vide soit un sous-espaces affine de \( \eR^n\).
\end{proposition}

\begin{proof}
    Soit \( f=\tau_v\circ \alpha\); nous avons \( x\in\Fix(f)\) si et seulement si
    \begin{equation}
        x=\tau_v\big( \alpha(x) \big)=\alpha(x)+v,
    \end{equation}
    autrement dit, en considérant l'application linéaire \( \beta=\id-\alpha\), si et seulement si \( \beta(x)=v\). Nous écrivons \( \Fix(f)=\beta^{-1}(v)\). Supposons que ce soit non vide et considérons \( x_0\in\beta^{-1}(v)\). Nous avons
    \begin{subequations}
        \begin{align}
            \beta^{-1}(v)&=\{ x\in \eR^n\tq \beta(x)=\beta(x_0) \}\\
            &=\{ x\tq \beta(x-x_0)=0 \}\\
            &=\{ x\tq x-x_0\in \ker(\beta) \}\\
            &=\ker(\beta)+x_0\\
            &=\tau_{x_0}\big( \ker(\beta) \big).
        \end{align}
    \end{subequations}
    Mais comme \( \ker(\beta)\) est un sous-espace vectoriel, \( \beta^{-1}(v)\) est le translaté d'un sous-espace vectoriel, c'est-à-dire un sous-espace affine.
\end{proof}

%--------------------------------------------------------------------------------------------------------------------------- 
\subsection{Structure de groupe pour les applications affines}
%---------------------------------------------------------------------------------------------------------------------------

\begin{propositionDef}[\cite{MonCerveau}]      \label{PROPooBPKKooJRAMeT} \label{LEMooUBGZooBIlmAN}
    L'ensemble des applications affines bijectives de \( \eR^n\) forment un groupe pour la composition. Les lois de groupe sont données par les formules suivantes :
    \begin{enumerate}
        \item
            Le neutre est l'identité.
        \item       \label{ITEMooGUFRooMuhXds}
            Le produit est donné par
            \begin{equation}        \label{EQooMIFSooKIvPnW}
                (\tau_v\circ \alpha)(\tau_w\circ \beta)=\tau_{\alpha(w)+v}\circ \alpha\beta.
            \end{equation}
        \item       \label{ITEMooYOMSooRUDSdm}
            L'inverse est donné par
            \begin{equation}
                (\tau_v\circ\alpha)^{-1}=\tau_{-\alpha^{-1}(v)}\circ \alpha^{-1}.
            \end{equation}
    \end{enumerate}
    Ce groupe est noté \( \Aff(\eR^n)\)\nomenclature[R]{\( \Aff(\eR^n)\)}{Le groupe des applications affines bijectives de \( \eR^n\).}.
\end{propositionDef}

\begin{proof}
    Pour l'identité, oui, composer par l'identité est neutre.

    Le fait que la formule \eqref{EQooMIFSooKIvPnW} soit vraie est un simple calcul :
    \begin{equation}
        (\tau_v\circ\alpha)\circ(\tau_w\circ\beta)(x)=(\alpha\beta)(x)+\alpha(w)+v=\big( \tau_{\alpha(w)+v}\circ \alpha\beta\big)x.
    \end{equation}

    Le fait que la formule \eqref{EQooMIFSooKIvPnW} donne bien un produit pour tous les éléments de \( \Aff(\eR^n)\) est le lemme \ref{LEMooZZAIooOMiayy}.

    En ce qui concerne l'inverse, c'est un calcul :
    \begin{subequations}
        \begin{align}
            (\tau_{-\alpha^{-1}(v)}\alpha^{-1})(\tau_v\alpha)(x)&=(\tau_{-\alpha^{-1}(v)\alpha^{-1}})\big( \alpha(x)+v \big)\\
            &=\tau_{-\alpha^{-1}(v)}\big( x+\alpha^{-1}(v) \big)\\
            &=x.
        \end{align}
    \end{subequations}
\end{proof}

Si \( f\colon \eR^n\to \eR^n\) est une application affine, la proposition \ref{LEMooZZAIooOMiayy} affirme qu'il existe une application linéaire \( u\) telle que
\begin{equation}
    f(x+y)=f(x)+u(y).
\end{equation}
En écrivant cela pour \( x=0\),
\begin{equation}
    f(y)=f(0)+u(y),
\end{equation}
ou encore \( f=\tau_{f(0)}\circ u\).

\begin{proposition}  \label{PROPooTPFZooKtFxhg}
    L'ensemble \( \Aff(\eR^n)\) est isomorphe au produit semi-direct\footnote{Définition~\ref{DEFooKWEHooISNQzi}.}
    \begin{equation}
        \Aff(\eR^n)\simeq  T(n)\times_{\AD}\GL(n,\eR)
    \end{equation}
    où \( \AD\) est l'action adjointe, c'est-à-dire
    \begin{equation}
        \begin{aligned}
            \AD\colon \GL(n,\eR)&\to \Aut\big( T(n) \big) \\
            \alpha&\mapsto\big( \tau_v\mapsto \alpha\circ\tau_v\circ\alpha^{-1}\big).
        \end{aligned}
    \end{equation}
\end{proposition}

\begin{proof}
    L'application que nous allons montrer être un isomorphisme est \( \psi\) qui à \( f=\tau_v\circ\alpha\) fait correspondre le couple \( (\tau_v,\alpha)\in T(n)\times \GL(n,\eR)\).
    \begin{subproof}
        \item[Égalité d'ensembles]
            Il faut que \( \Aff(\eR^n)\) soit en bijection avec \( T(n)\times \GL(n,\eR)\). En effet si \( f\in \Aff(\eR^n)\), la décomposition \(f=\tau_v\circ\alpha \) est unique. D'abord en appliquant à \( 0\), \( f(0)=\tau_v\big( \alpha(v) \big)=v\). Donc \( v\) est fixé par la valeur de \( f(0)\). Ensuite \( \alpha=f\circ\tau_v^{-1}\), donc \( \alpha \) fixé.
        \item[L'action adjointe fonctionne]
            Il faut vérifier que \( \alpha\circ\tau_v\circ\alpha^{-1}\) est bien dans \( T(n)\). Pour cela, en agissant sur \( x\in \eR^n\) nous trouvons
            \begin{equation}
                \alpha\tau_v\alpha^{-1}(x)=\alpha\big( \alpha^{-1}(x)+v \big)=x+\alpha(v)=\tau_{\alpha(v)}(x).
            \end{equation}
            Le fait que \( \AD(\alpha)\) soit un automorphisme est toujours correct.
        \item[Morphisme]
            Il faut vérifier que l'application \( \psi\) est un morphisme de groupe. D'abord la loi de groupe sur \( \Aff(\eR^n)\) est donnée par
            \begin{equation}
                (\tau_v\circ \alpha)\circ(\tau_w\circ\beta)=\tau_{v+\alpha(w)}\circ(\alpha\circ\beta).
            \end{equation}
            Ensuite le loi de groupe de le produit semi-direct est donnée par
            \begin{equation}
                (\tau_v,\alpha)\cdot(\tau_w,\beta)=\big( \tau_v\AD(\alpha)\tau_w,\alpha\beta \big)=\big( \tau_v\tau_{\alpha(w)},\alpha\beta \big)=\big( \tau_{\alpha(w)+v},\alpha\beta \big).
            \end{equation}
            Nous avons donc bien
            \begin{equation}
                \psi\big( (\tau_v,\beta)\cdot(\tau_w,\beta) \big)=\psi(\tau_v,\beta)\circ\psi(\tau_w,\beta).
            \end{equation}
    \end{subproof}
\end{proof}

%+++++++++++++++++++++++++++++++++++++++++++++++++++++++++++++++++++++++++++++++++++++++++++++++++++++++++++++++++++++++++++
\section{Isométries}
%+++++++++++++++++++++++++++++++++++++++++++++++++++++++++++++++++++++++++++++++++++++++++++++++++++++++++++++++++++++++++++

\begin{definition}[Isométrie d'espace affine]       \label{DEFooZGKBooGgjkgs}
     Si \( \affE\) est un espace affine muni d'une distance \( d\), une isométrie de \( \affE\) est une application \( f\colon \affE\to \affE\) préservant \( d\).
\end{definition}
Notons que toutes les applications affines ne sont pas des isométries : par exemple les homothéties.

\begin{proposition}     \label{PROPooHSOGooBbFTYt}
    Si \( \affE\) est modelé sur un espace euclidien \( (E,\| . \|)\) alors la formule
    \begin{equation}
        d(A,B)=\| \vect{ AB } \|
    \end{equation}
    définit une distance sur \( \affE\).
\end{proposition}

\begin{proof}
    Étant donné ce qui est dit en~\ref{NORMooZANAooQdXqlh}, la formule a un sens parce qu'à \( A\) et \( B\) donnés dans \( \affE\), il est associé un unique vecteur \( \vect{ AB }\in E\).
\end{proof}

Nous parlons d'isométries affines ou linéaires dans le thème \ref{THMooVUCLooCrdbxm}.


\chapter{Espaces vectoriels (encore)}
% This is part of Mes notes de mathématique
% Copyright (c) 2008-2020
%   Laurent Claessens
% See the file fdl-1.3.txt for copying conditions.

%+++++++++++++++++++++++++++++++++++++++++++++++++++++++++++++++++++++++++++++++++++++++++++++++++++++++++++++++++++++++++++
\section{Déterminants}
%+++++++++++++++++++++++++++++++++++++++++++++++++++++++++++++++++++++++++++++++++++++++++++++++++++++++++++++++++++++++++++
\label{SecGYzHWs}

%---------------------------------------------------------------------------------------------------------------------------
\subsection{Formes multilinéaires alternées}
%---------------------------------------------------------------------------------------------------------------------------

%  Lire http://www.les-mathematiques.net/phorum/read.php?2,302266

\begin{definition}\index{déterminant!forme linéaire alternée}       \label{DEFooYWOBooUGJojy}
    Soit \( E\), un \( \eK\)-espace vectoriel. Une forme linéaire \defe{alternée}{forme linéaire!alternée}\index{alternée!forme linéaire} sur \( E\) est une application linéaire \( f\colon E\to \eK\) telle que \( f(v_1,\ldots, v_k)=0\) dès que \( v_i=v_j\) pour certains \( i\neq j\).
\end{definition}

\begin{lemma}   \label{LemHiHNey}
    Une forme linéaire alternée est antisymétrique. Si \( \eK\) est de caractéristique différente de \( 2\), alors une forme antisymétrique est alternée.
\end{lemma}

\begin{proof}
    Soit \( f\) une forme alternée; quitte à fixer toutes les autres variables, nous pouvons travailler avec une \( 2\)-forme et simplement montrer que \( f(x,y)=-f(y,x)\). Pour ce faire nous écrivons
    \begin{equation}
        0=f(x+y,x+y)=f(x,x)+f(x,y)+f(y,x)+f(y,y)=f(x,y)+f(y,x).
    \end{equation}

    Pour la réciproque, si \( f\) est antisymétrique, alors \( f(x,x)=-f(x,x)\). Cela montre que \( f(x,x)=0\) lorsque \( \eK\) est de caractéristique différente de deux.
\end{proof}

\begin{proposition}[\cite{GQolaof}] \label{ProprbjihK}
    Soit \( E\), un \( \eK\)-espace vectoriel de dimension \( n\), où la caractéristique de \( \eK\) n'est pas deux. L'espace des \( n\)-formes multilinéaires alternées sur \( E\) est de \( \eK\)-dimension \( 1\).
\end{proposition}
\index{groupe!permutation}
\index{groupe!et géométrie}
\index{espace!vectoriel!dimension}
\index{rang}
\index{déterminant}
\index{dimension!\( n\)-formes multilinéaires alternées}

\begin{proof}
    Soient \( \{ e_i \}\), une base de \( E\), une \( n\)-forme linéaire alternée \( f\colon E\to \eK\) ainsi que des vecteurs \( (v_1,\ldots, v_n)\) de \( E\). Nous pouvons les écrire dans la base
    \begin{equation}
        v_j=\sum_{i=1}^n\alpha_{ij}e_i
    \end{equation}
    et alors exprimer \( f\) par
    \begin{subequations}
        \begin{align}
            f(v_1,\ldots, v_n)&=f\big( \sum_{i_1=1}^n\alpha_{1i_1}e_{i_1},\ldots, \sum_{i_n=1}^n\alpha_{ni_n}e_{i_n} \big)\\
            &=\sum_{i,j}\alpha_{1i_1}\ldots \alpha_{ni_n}f(e_{i_1},\ldots, e_{i_n}).
        \end{align}
    \end{subequations}
    Étant donné que \( f\) est alternée, les seuls termes de la somme sont ceux dont les \( i_k\) sont tous différents, c'est-à-dire ceux où \( \{ i_1,\ldots, i_n \}=\{ 1,\ldots, n \}\). Il y a donc un terme par élément du groupe des permutations \( S_n\) et
    \begin{equation}
        f(v_1,\ldots, v_n)=\sum_{\sigma\in S_n}\alpha_{\sigma(1)1}\ldots \alpha_{\sigma(n)n}f(e_{\sigma(1)},\ldots, e_{\sigma(n)}).
    \end{equation}
    En utilisant encore une fois le fait que la forme \( f\) soit alternée, \( f=f(e_1,\ldots, e_n)\Pi\) où
    \begin{equation}
        \Pi(v_1,\ldots, v_n)=\sum_{\sigma\in S_n}\epsilon(\sigma)\alpha_{\sigma(1)1}\ldots \alpha_{\sigma(n)n}.
    \end{equation}
    Pour rappel, la donnée des \( v_i\) est dans les nombres \( \alpha_{ij}\).

    L'espace des \( n\)-formes alternées est donc \emph{au plus} de dimension \( 1\). Pour montrer qu'il est exactement de dimension \( 1\), il faut et suffit de prouver que \( \Pi\) est alternée. Par le lemme~\ref{LemHiHNey}, il suffit de prouver que cette forme est antisymétrique\footnote{C'est ici que joue l'hypothèse sur la caractéristique de \( \eK\).}.

    Soient donc \( v_1,\ldots, v_n\) tels que \( v_i=v_j\). En posant \( \tau=(1i)\) et \( \tau'=(2j)\) et en sommant sur \( \sigma\tau\tau'\) au lieu de \( \sigma\), nous pouvons supposer que \( i=1\) et \( j=2\). Montrons que \( \Pi(v,v,v_3,\ldots, v_n)=0\) en tenant compte que \( \alpha_{i1}=\alpha_{i2}\) :
    \begin{subequations}
        \begin{align}
            \Pi(v,v,v_3,\ldots, v_n)&=\sum_{\sigma\in S_n}\epsilon(\sigma)\alpha_{\sigma(1)1}\alpha_{\sigma(2)2}\alpha_{\sigma(3)3}\ldots \alpha_{\sigma(n)n}\\
            &=\sum_{\sigma\in S_n}\epsilon(\sigma\tau)\alpha_{\sigma\tau(1)1}\alpha_{\sigma\tau(2)2}\alpha_{\sigma\tau(3)3}\ldots \alpha_{\sigma\tau(n)n}&\text{où } \tau=(12)\\
            &=-\sum_{\sigma\in S_n}\epsilon(\sigma)\alpha_{\sigma(1)1}\alpha_{\sigma(2)2}\alpha_{\sigma(3)3}\ldots \alpha_{\sigma(n)n} \\
            &=-\Pi(v,v,v_3,\ldots, v_n).
        \end{align}
    \end{subequations}
\end{proof}

%---------------------------------------------------------------------------------------------------------------------------
\subsection{Déterminant d'une famille de vecteurs}
%---------------------------------------------------------------------------------------------------------------------------

Nous considérons un corps \( \eK\) et l'espace vectoriel \( E\) de dimension \( n\) sur \( \eK\).

\begin{definition}[Déterminant d'une famille de vecteurs\cite{MathAgreg}]\label{DEFooODDFooSNahPb}
    Le \defe{déterminant}{déterminant!d'une famille de vecteurs} de la famille de vecteurs \( (v_1,\ldots, v_n)\) dans la base \( B\) est l'élément de \( \eK\)
    \begin{equation}        \label{EQooOJEXooXUpwfZ}
        \det_{(e_1,\ldots, e_n)}(v_1,\ldots, v_n)=\sum_{\sigma\in S_n}\epsilon(\sigma)\prod_{i=1}^ne^*_{\sigma(i)}(v_i)
    \end{equation}
    où 
    \begin{itemize}
        \item 
            la somme porte sur le groupe symétrique, 
        \item
            le nombre \( \epsilon(\sigma)\) est la signature de la permutation \( \sigma\),
        \item
            les éléments \( \{ e_i \}\) forment la base canonique de \( \eK^n\).
        \item
            les éléments \( \{ e^*_i \}\) sont la base duale de \( \{ e_i \}\).
    \end{itemize}
    Nous le notons \( \det_{(e_1,\ldots, e_n)}(v_1,\ldots, v_n)\).
\end{definition}

\begin{normaltext}
    La base \( \{ e_i \}\) est la base canonique de \( \eK^n\), et l'élément \( e_k^*\) est la forme linéaire définie par
    \begin{equation}
        \begin{aligned}
            e_k^*\colon \eK^n&\to \eK \\
            \sum_ix_ie_i&\mapsto x_k. 
        \end{aligned}
    \end{equation}
    Il n'est pas sous-entendu que \( \eK^n\) ait un produit scalaire. Il n'est donc pas autorisé de dire que \( \{ e_i \}\) est une base orthonormée et que \( e^*_k(x)=\langle e_k, x\rangle \). Ce genre d'égalités sont vraies dans le cas \( \eK=\eR\), mais n'ont pas de sens en général.

    Le lemme \ref{LEMooEZFIooXyYybe} va un peu parler du cas où \( \eK^n\) est muni d'une base orthonormée.
\end{normaltext}

\begin{lemma}[\cite{MathAgreg}]     \label{LemJMWCooELZuho}
    Les propriétés du déterminant. Soit \( B\) une base de \( E\).
    \begin{enumerate}
        \item\label{ITEMooAHOHooDZgtSB}
            L'application \( \det_B\colon E^n\to \eK\) est \( n\)-linéaire.
        \item\label{ITEMooTXXBooBmDtzd}
            L'application \( \det_B\colon E^n\to \eK\) est \( n\)-linéaire est antisymétrique et alternée\footnote{Alternée, définition \ref{DEFooYWOBooUGJojy}. En caractéristique \( 2\), alternée n'est pas équivalent à symétrique.}.
        \item   \label{ITEMooNFJTooTqGoPr}
            Pour toute base, \( \det_B(B)=1\).
        \item   \label{ITEMooALRQooDvBzDQ}
            Le déterminant ne change pas si on remplace un vecteur par une combinaison linéaire des autres :
            \begin{equation}
                \det_B(v_1,\ldots, v_n)=\det_B\big( v_1+\sum_{s=2}^na_sv_s,v_2,\ldots, v_n \big).
            \end{equation}
        \item   \label{ITEMooQTTRooMbzqyW}
            Si on permute les vecteurs,
            \begin{equation}
                \det_B(v_1,\ldots, v_n)=\epsilon(\sigma)\det_B(v_{\sigma(1)},\ldots, v_{\sigma(n)}).
            \end{equation}
        \item   \label{ITEMooIPIDooTrerVF}
            Si \( B'\) est une autre base :
            \begin{equation}        \label{EqAWICooBLTTOY}
                \det_B=\det_B(B')\det_{B'}
            \end{equation}
        \item   \label{ITEMooXKTAooXynFTE}
            Nous avons aussi la formule \( \det_{B}(B')\det_{B'}(B)=1\).
        \item\label{ItemDWFLooDUePAf}
            Les vecteurs \( \{ v_1,\ldots, v_n \}\) forment une base si et seulement si \( \det_B(v_1,\ldots, v_n)\neq 0\).
    \end{enumerate}
\end{lemma}

\begin{proof}
    Point par point.
    \begin{subproof}
    \item[\ref{ITEMooAHOHooDZgtSB}]
            En posant \( v_1=x_1+\lambda x_2\) nous avons
            \begin{subequations}
                \begin{align}
                    \det_B(x_1+\lambda x_2,v_2,\ldots, v_n)&=\sum_{\sigma}\epsilon(\sigma)\prod_{i=1}^ne^*_{\sigma(i)}(v_i)\\
                    &=\sum_{\sigma}\epsilon(\sigma)\Big( e^*_{\sigma(1)}(x_1+\lambda x_2) \Big)\prod_{i=2}^ne^*_{\sigma(i)}(v_i).
                \end{align}
            \end{subequations}
            À partir de là, la linéarité de \( e^*_{\sigma(1)}\) montre que \( \det_B\) est linéaire en son premier argument. Pour les autres arguments, le même calcul tient.

        \item[\ref{ITEMooTXXBooBmDtzd}]

            Nous prouvons à présent que \( \det\) est alternée. Si votre corps est de caractéristique différente de deux, vous pouvez lire la proposition \ref{PROPooXNLDooGGkHpd}.

            Supposons \( v_k=v_l\), et considérons la permutation \( \beta=(k,l)\). Nous savons par la proposition \ref{PROPooZOWBooIMxxlj} que \( S_n=A_n\cup A_n\beta\). Cela nous permet de décomposer la somme sur \( S_n\) en deux parties :
            \begin{equation}        \label{EQooWFHQooTrTTWl}
                \sigma_{\sigma\in S_n}(-1)^{\sigma}\prod_i\epsilon_{\sigma(i)}^*(v_i)=\sum_{\sigma\in A_n}(-1)^{\sigma}\prod_i\epsilon_{\sigma(i)}^*(v_i)+\sum_{\sigma\in A_n}(-1)^{\sigma\beta}\prod_i\epsilon_{(\sigma\beta)(i)}^*(v_i).
            \end{equation}
            D'abord \( (-1)^{\sigma}=1\) et \( (-1)^{\sigma\beta}=-1\). Ensuite, pour un \( \sigma\in A_n\) donné, nous avons
            \begin{subequations}
                \begin{align}
                    \prod_i\epsilon^*_{(\sigma\beta)(i)}(v_i)&=\epsilon_{(\sigma\beta)(k)}^*(v_k)\epsilon^*_{(\sigma\beta)(l)}(v_l)\prod\stackrel{i\neq k}{i\neq l}\epsilon_{(\sigma\beta)(i)}^*(v_i)\\
                    &=\epsilon^*_{\sigma(l)}(v_k)\epsilon^*_{\sigma(k)}(v_l)\prod_{\substack{i\neq k\\i\neq l}}\epsilon^*_{\sigma(i)}(v_i)\\
                    &=\epsilon^*_{\sigma(l)}(v_l)\epsilon^*_{\sigma(k)}(v_k)\prod_{\substack{i\neq k\\i\neq l}}\epsilon^*_{\sigma(i)}(v_i)\\
                    &=\prod_i\epsilon^*_{\sigma(i)}(v_i).
                \end{align}
            \end{subequations}
            Donc les deux termes de la somme \eqref{EQooWFHQooTrTTWl} ne diffèrent que par un signe. Elle est donc nulle, et la forme déterminant est alternée.    
        
            La fonction \( \det\) est antisymétrique parce que alternée, voir le lemme \ref{LemHiHNey}.

        \item[\ref{ITEMooNFJTooTqGoPr}]
            Nous avons
            \begin{equation}
                \det_B(B)=\sum_{\sigma\in S_n}\epsilon(\sigma)\prod_{i=1}^n\underbrace{e_{\sigma(i)}^*(e_i)}_{=\delta_{\sigma(i),i}}.
            \end{equation}
            Si \( \sigma\) n'est pas l'identité, le produit contient forcément un facteur nul. Il ne reste de la somme que \( \sigma=\id\) et le résultat est \( 1\).
        \item[\ref{ITEMooALRQooDvBzDQ}]
            Vu que \( \det_B\) est linéaire en tous ses arguments,
            \begin{equation}
                \det_B\big( v_1+\sum_{s=2}^na_sv_s,v_2,\ldots, v_n \big)=\det_B(v_1,\ldots, v_n)+\sum_{s=2}^na_s\det_B(v_s,v_2,\ldots, v_n).
            \end{equation}
            Chacun des termes de la somme est nul parce qu'il y a répétition de \( v_s\) parmi les arguments alors que la forme est alternée.
        \item[\ref{ITEMooQTTRooMbzqyW}]
            Nous devons calculer \( \det_B(v_{\sigma(1)},\ldots, v_{\sigma(n)})\), et pour y voir plus clair nous posons \( w_i=v_{\sigma(i)}\). Alors :
            \begin{subequations}
                \begin{align}
                    \det_B(v_{\sigma(1)},\ldots, v_{\sigma(n)})&=\sum_{\sigma'}\epsilon(\sigma')\prod_{i=1}^ne^*_{\sigma'(i)}(w_i)\\
                    &=\sum_{\sigma'}\epsilon(\sigma')\prod_{i=1}^ne^*_{\sigma'(i)}(v_{\sigma(i)})\\
                    &=\sum_{\sigma'}\epsilon(\sigma')\prod_{i=1}^ne^*_{\sigma^{-1}\sigma'(i)}(v_i)\\
                    &=\sum_{\sigma'}\epsilon(\sigma\sigma')\prod_{i=1}^ne^*_{\sigma'(i)}(v_i)\\
                    &=\epsilon(\sigma)\det_B(v_1,\ldots, v_n).
                \end{align}
            \end{subequations}
            Justifications : nous avons d'abord modifié l'ordre des éléments du produit et ensuite l'ordre des éléments de la somme. Nous avons ensuite utilisé le fait que \( \epsilon\colon S_n\to \{ 0,1 \}\) était un morphisme de groupe (proposition~\ref{ProphIuJrC}).
        \item[\ref{ITEMooIPIDooTrerVF}]
            Étant donné que l'espace des formes multilinéaires alternées est de dimension \( 1\), il existe un \( \lambda\in \eK\) tel que \( \det_B=\lambda\det_{B'}\). Appliquons cela à \( B'\) :
            \begin{equation}
                \det_B(B')=\lambda\det_{B'}(B'),
            \end{equation}
            donc \( \lambda=\det_B(B')\).
        \item[\ref{ITEMooXKTAooXynFTE}]
            Il suffit d'appliquer l'égalité précédente à \( B\) en nous souvenant que \( \det_B(B)=1\).
        \item[\ref{ItemDWFLooDUePAf}]
            Si \( B'=\{ v_1,\ldots, v_n \}\) est une base alors \( \det_B(B')\neq 0\), sinon il n'est pas possible d'avoir \( \det_B(B')\det_{B'}(B)=1\).

            À l'inverse, si \( B'\) n'est pas une base, c'est que \( \{ v_1,\ldots, v_n \}\) est liée par la proposition \ref{PROPooVEVCooHkrldw}. Il y a donc moyen de remplacer un des vecteurs par une combinaison linéaire des autres. Le déterminant s'annule alors.
    \end{subproof}
\end{proof}

\begin{proposition}     \label{PROPooXNLDooGGkHpd}
    Si la caractéristique du corps de base n'est pas deux, le déterminant est antisymétrique et alterné.
\end{proposition}

\begin{proof}
    Si la caractéristique du corps de base n'est pas deux, une forme antisymétrique est alternée (lemme \ref{LemHiHNey}).

    Pour prouver que le déterminant est antisymétrique, remarquez que permuter \( v_k\) et \( v_l\) revient à calculer le nombre \( \det_B( v_{\sigma_{kl}(1)},\ldots, v_{\sigma_{kl}(n)} )\) au lieu de \( \det_B(v_1,\ldots, v_n)\). Cela revient à changer la somme \( \sum_{\sigma}\) en \( \sum_{\sigma\circ\sigma_{kl}}\). Cela ajoute \( 1\) à \( \epsilon(\sigma)\) vu que l'on ajoute une permutation.

    Donc le déterminant est antisymétrique. Nous en déduisons qu'il est alterné parce que, en permutant trivialement \( v_1\) et \( v_1\), nous obtenons \( \det_B(v_1,v_1)=-\det_B(v_1,v_1)\). Si le corps est de caractéristique différente de deux, cela implique que \( \det_B(v_1,v_1)=0\).
\end{proof}

D'après la proposition~\ref{ProprbjihK}, il existe une unique forme \( n\)-linéaire alternée égale à \( 1\) sur \( B\), et c'est \( \det_B\colon E^n\to \eK\).

%---------------------------------------------------------------------------------------------------------------------------
\subsection{Déterminant d'un endomorphisme}
%---------------------------------------------------------------------------------------------------------------------------

L'interprétation géométrique du déterminant en termes d'aires et de volumes est donnée après la théorème~\ref{ThoBVIJooMkifod}.

\begin{lemmaDef}       \label{LEMooQTRVooAKzucd}      \label{DefCOZEooGhRfxA}
    Si \( f\colon E\to E\) est un endomorphisme, et si les parties \( B\) et \( B'\) sont deux bases, alors 
    \begin{equation}
        \det_B\big( f(B) \big)=\det_{B'}\big( f(B') \big).
    \end{equation}
    Ce nombre, indépendant de la base choisie est nommé le \defe{déterminant}{déterminant!d'un endomorphisme} de \( f\) et est noté \( \det(f)\).
\end{lemmaDef}

\begin{proof}
    L'application
    \begin{equation}
        \begin{aligned}
            \varphi\colon E^n&\to \eK \\
            v_1,\ldots, v_n&\mapsto \det_B\big( f(v_1),\ldots, f(v_n) \big)
        \end{aligned}
    \end{equation}
    est \( n\)-linéaire et alternée; il existe donc \( \lambda\in \eK\) tel que \( \varphi=\lambda\det_B\). En appliquant cela à \( B\) :
    \begin{equation}
        \det_B\big( f(B) \big)=\lambda \det_B(B)=\lambda.
    \end{equation}
    Nous avons donc déjà prouvé que \( \lambda=\det_B\big( f(B) \big)\), c'est-à-dire
    \begin{equation}
        \det_B\big( f(v) \big)=\det_B\big( f(B) \big)\det_B(v).
    \end{equation}

    Nous allons maintenant introduire \( B'\) là où il y a du \( v\) en utilisant les formules \eqref{EqAWICooBLTTOY} :
    \begin{subequations}
        \begin{align}
            \det_B\big( f(v) \big)&=\det_B(B')\det_{B'}\big( f(v) \big)\\
            \det_B(v)=\det_B(B')\det_{B'}(v).
        \end{align}
    \end{subequations}
    Nous obtenons
    \begin{equation}
        \det_{B'}\big( f(v) \big)=\det_B\big( f(B) \big)\det_{B'}(v).
    \end{equation}
    Et on applique cela à \( v=B'\) :
    \begin{equation}
        \det_{B'}\big( f(B') \big)=\det_B\big( f(B) \big)\underbrace{\det_{B'}(B')}_{=1}.
    \end{equation}
\end{proof}

Couplé à la formule \eqref{EQooOJEXooXUpwfZ}, nous pouvons écrire la formule pratique à utiliser le plus souvent. 

\begin{lemma}       \label{LEMooEZFIooXyYybe}
    Soit un espace vectoriel euclidien\footnote{C'est-à-dire qu'il possède un produit scalaire, voir la définition \ref{DefLZMcvfj}.} \( E\) sur le corps \( \eK\). Si \( \{ e_i \}_{i=1,\ldots, n}\) est une base orthonormée de \( E\) et si \( f\colon E\to E\) est un endomorphisme, alors
    \begin{equation}        \label{EQooQAZLooZutFUz}
        \det(f)=\sum_{\sigma\in S_n}\epsilon(\sigma)\prod_{i=1}^n\langle e_{\sigma(i)}, f(e_i)\rangle.
    \end{equation}
\end{lemma}

\begin{proof}
    Nous utilisons la définition \ref{LEMooQTRVooAKzucd} du déterminant d'un endomorphisme \( \det(f)=\det_B\big( f(B) \big)\) en prenant la liste des vecteurs \( \{ e_i \}\) comme \( B\). En l'occurrence, le \( i\)\ieme\ vecteur de la famille \( B\) est \( f(e_i)\).

    Vu que la base est orthonormée, nous avons \( e^*_k(v)=\langle e_k, v\rangle \) et donc aussi
    \begin{equation}
        e^*_{\sigma(i)}(v_i)=\langle e_{\sigma(i)}^*, f(e_i)\rangle.
    \end{equation}
\end{proof}

Et si vous avez tout suivi, vous aurez remarqué que les produits scalaires impliqués dans la formule \eqref{EQooQAZLooZutFUz} sont les éléments de la matrice de \( f\) dans la base \( \{ e_i \}\) parce que \( \langle e_i, f(e_j)\rangle \) est la composante \( i\) de l'image de \( e_j\) par \( f\). Si la matrice est composée en mettant en colonne les images des vecteurs de base, le compte est bon.

\begin{proposition}     \label{PropYQNMooZjlYlA}
    Principales propriétés géométriques du déterminant d'un endomorphisme.
    \begin{enumerate}
        \item   \label{ItemUPLNooYZMRJy}
            Si \( f\) et \( g\)  sont des endomorphismes, alors \( \det(f\circ g)=\det(f)\det(g)\).
        \item       \label{ITEMooNZNLooODdXeH}
            L'endomorphisme \( f\) est un automorphisme\footnote{Endomorphisme inversible, définition~\ref{DEFooOAOGooKuJSup}.} si et seulement si \( \det(f)\neq 0\).\index{déterminant!et inversibilité}
        \item   \label{ITEMooZMVXooLGjvCy}
            Si \( \det(f)\neq 0\) alors \( \det(f^{-1})=\det(f)^{-1}\).
        \item       \label{ItemooPJVYooYSwqaE}
            L'application \( \det\colon \GL(E)\to \eK\setminus\{ 0 \}\) est un morphisme de groupe.
    \end{enumerate}
\end{proposition}

\begin{proof}
    Point par point.
    \begin{enumerate}
        \item
            Nous considérons l'application
            \begin{equation}
                \begin{aligned}
                    \varphi\colon E^n&\to \eK \\
                    v&\mapsto \det_B\big( f(v) \big).
                \end{aligned}
            \end{equation}
            Comme d'habitude nous avons \( \varphi(v)=\lambda\det_B(v)\). En appliquant à \( B\) et en nous souvenant que \( \det_B(B)=1\) nous avons
                $\det_B\big( f(B) \big)=\lambda$. Autrement dit :
                \begin{equation}
                    \lambda=\det(f).
                \end{equation}
            Calculons à présent \( \varphi\big( g(B) \big)\) : d'une part,
            \begin{equation}
                \varphi\big( g(B) \big)=\det_B\big( (f\circ g)(B) \big)
            \end{equation}
            et d'autre part,
            \begin{equation}
                \varphi\big( g(B) \big)=\lambda\det_B\big( g(B) \big)=\lambda\det(g)
            \end{equation}
            En égalisant et en reprenant la la valeur déjà trouvée de \( \lambda\),
            \begin{equation}
                \det\big(f\circ g)(B) \big)=\det(f)\det(g),
            \end{equation}
            ce qu'il fallait.
        \item
            Supposons que \( f\) soit un automorphisme. Alors si \( B\) est une base, \( f(B) \) est une base. Par conséquent \( \det(f)=\det_B\big( f(B) \big)\neq 0\) parce que \( f(B)\) est une base (lemme~\ref{LemJMWCooELZuho}\ref{ItemDWFLooDUePAf}).

            Réciproquement, supposons que \( \det(f)\neq 0\). Alors si \( B\) est une base quelconque nous avons \( \det_B\big( f(B) \big)\neq 0\), ce qui est uniquement possible lorsque \( f(B)\) est une base. L'application \( f\) transforme donc toute base en une base et est alors un automorphisme d'espace vectoriel.
        \item
            Vu que le déterminant de l'identité est \( 1\) et que \( f\) est inversible, \( 1=\det(f\circ f^{-1})=\det(f)\det(f^{-1})\).
    \end{enumerate}
\end{proof}

\begin{proposition}     \label{PROPooFKDXooKMSolt}
    Soient deux espaces vectoriels \( E\) et \( F\) de dimension finies \( n\) et \( m\) sur le corps \( \eK\) munis de bases \( \{e_i\}\) et \( \{f_{\alpha}\}\). À une matrice \( A\in \\eM(m\times n,\eK)\) nous associons l'application linéaire\footnote{Dont nous avons déjà beaucoup parlé entre autres dans la proposition \ref{PROPooCSJNooEqcmFm}.}
    \begin{equation}
        f_A(x)=\sum_{i\alpha}A_{\alpha i}x_if_{\alpha}.
    \end{equation}
    
    Alors, en ce qui concerne les déterminants\footnote{Définition \ref{LEMooQTRVooAKzucd} pour les applications linéaires et \ref{DEFooYCKRooTrajdP} pour les matrices.}, nous avons
    \begin{enumerate}
        \item
            \( \det(f_A)=\det(A)\)
        \item
            \( \det(f_{AB})=\det(f_A)\det(f_B)\)
    \end{enumerate}
\end{proposition}

\begin{proof}
    Nous devons étudier la formule
    \begin{equation}
        \det(f_A)=\sum_{\sigma\in S_n}\epsilon(\sigma)\prod_{i=1}^ne_{\sigma(i)}^*\big( f_A(e_i) \big).
    \end{equation}
    En premier lieu nous avons
    \begin{equation}
        f_A(e_i)=\sum_{jk}A_{jk}(e_i)_ke_j=\sum_jA_{ji}e_j.
    \end{equation}
    Nous avons alors
    \begin{equation}
        e_{\sigma(i)}^*\big( f_A(e_i) \big)=\sum_jA_{ji}\underbrace{e^*_{\sigma(i)}(e_j)}_{\delta_{j\sigma(i)}}=A_{\sigma(i)i}.
    \end{equation}
    Au final,
    \begin{equation}
        \det(f_A)=\sum_{\sigma}\epsilon(\sigma)\prod_{i=1}^nA_{\sigma(i)i}=\det(A^t)=\det(A)
    \end{equation}
    où la dernière égalité est autorisée par le lemme \ref{LEMooCEQYooYAbctZ}.

    Cela prouve la formule \( \det(f_A)=\det(A)\).

    En ce qui concerne la seconde formule, il s'agit de se souvenir de la proposition \ref{PROPooCSJNooEqcmFm} qui donne \( f_{AB}=f_A\circ f_B\), et ensuite de la proposition \ref{PropYQNMooZjlYlA}\ref{ItemUPLNooYZMRJy} qui donne \( \det(f_A\circ f_B)=\det(f_A)\det(f_B)\).
\end{proof}

%---------------------------------------------------------------------------------------------------------------------------
\subsection{Déterminant de Vandermonde}
%---------------------------------------------------------------------------------------------------------------------------

\begin{proposition}[\cite{fJhCTE}]  \label{PropnuUvtj}
    Le \defe{déterminant de Vandermonde}{déterminant!Vandermonde}\index{Vandermonde (déterminant)} est le polynôme en \( n\) variables donné par
    \begin{equation}
        V(T_1,\ldots, T_n)=\det\begin{pmatrix}
             1   &   1    &   \ldots    &   1    \\
             T_1   &   T_2    &   \ldots    &   T_n    \\
             \vdots   &   \ddots    &   \ddots    &   \vdots    \\
             T_1^{n-1}   &   T_2^{n-1}    &   \ldots    &   T_n^{n-1}
         \end{pmatrix}=\prod_{1\leq i<j\leq n}(T_j-T_i).
    \end{equation}
    Notez que l'inégalité du milieu est stricte (sinon d'ailleurs l'expression serait nulle).
\end{proposition}

\begin{proof}
    Nous considérons le polynôme
    \begin{equation}
        f(X)=V(T_1,\ldots, T_{n-1},X)\in \big( \eK[T_1,\ldots, T_{n-1}] \big)[X].
    \end{equation}
    %TODOooHLQLooWXznGe Mettre une référence vers la proposition qu'il faut à ce «par conséquent».
    C'est un polynôme de degré au plus \( n-1\) en \( X\) et il s'annule aux points \( T_1,\ldots, T_{n-1}\). Par conséquent il existe \( \alpha\in \eK[T_1,\ldots, T_{n-1}]\) tel que
    \begin{equation}    \label{EqeVxRwO}
        f=\alpha(X-T_{n-1})\ldots(X-T_1).
    \end{equation}
    Nous trouvons \( \alpha\) en écrivant \( f(0)\). D'une part la formule \eqref{EqeVxRwO} nous donne
    \begin{equation}    \label{EqblwWMj}
        f(0)=\alpha(-1)^{n-1}T_1\ldots T_{n-1}.
    \end{equation}
    D'autre par la définition donne
    \begin{subequations}
        \begin{align}
            f(0)&=\det\begin{pmatrix}
                 1   &   \cdots    &   1    &   1    \\
                 T_1      &       &   T_{n-1}    &   0    \\
                 \vdots   &       &   \vdots    &   \vdots    \\
                 T_1^{n-1}   &   \cdots    &   T_{n-1}^{n-1}    &   0
             \end{pmatrix}\\
             &=(-1)^{n-1}\det\begin{pmatrix}
                 T_1   &   \ldots    &   T_{n-1}    \\
                 \vdots   &   \ddots    &   \vdots    \\
                 T_1^{n-1}   &   \ldots    &   T_{n-1}^{n-1}
             \end{pmatrix}\\
             &=(-1)^{n-1}T_1\ldots T_{n-1}\det\begin{pmatrix}
                 1   &   \cdots    &   1    \\
                 \vdots   &   \ddots    &   \vdots    \\
                 T_1^{n-1}   &   \cdots    &   T_{n-1}^{n-1}
             \end{pmatrix}\\
             &=(-1)^{n-1}T_1\ldots T_{n-1}V(T_1,\ldots, T_{n-1})
        \end{align}
    \end{subequations}
    En égalisant avec \eqref{EqblwWMj}, nous trouvons \( \alpha=V(T_1,\ldots, T_{n-1})\), et donc
    \begin{equation}
        f=V(T_1,\ldots, T_{n-1})\prod_{j\leq n-1}(X-T_j)
    \end{equation}
    Enfin, une récurrence montre que
    \begin{subequations}
        \begin{align}
            V(T_1,\ldots, T_n)&=f(T_n)\\
            &=V(T_1,\ldots, T_{n-1})\prod_{j\leq n-1}(T_n-T_j)\\
            &=\prod_{k\leq n}\prod_{j\leq k-1}(T_k-T_j)\\
            &=\prod_{1\leq j<k\leq n}(T_i-T_j).
        \end{align}
    \end{subequations}
\end{proof}

\begin{example}
    Le déterminant de Vandermonde (proposition~\ref{PropnuUvtj}) est alterné, semi-symétrique et non symétrique. Le fait qu'il soit alterné est le fait qu'il soit un déterminant. Étant donné qu'il est alterné, il est semi-symétrique parce que sur \( A_n\), nous avons \( \epsilon=1\). Étant donné qu'il est alterné, il change de signe sous l'action des éléments impairs de \( S_n\) et n'est donc pas symétrique.
\end{example}

\begin{proposition}\index{action de groupe} \label{PropUDqXax}
    Un polynôme semi-symétrique \( f\in \eK[T_1,\ldots, T_n]\) se décompose de façon unique en
    \begin{equation}
        f=P+VQ
    \end{equation}
    où \( P\) et \( Q\) sont deux polynômes symétriques.
\end{proposition}
\index{groupe!permutation}
\index{polynôme!symétrique}

\begin{proof}

    Nous commençons par prouver l'unicité en montrant que si \( f=PVQ\) avec \( P\) et \( Q\) symétrique, alors \( P\) et \( Q\) sont donnés par des formules explicites en termes de \( f\).


    Si \( \sigma_1\) et \( \sigma_2\) sont deux permutations impaires de \( \{ 1,\ldots, n \}\), alors \( \sigma_1\cdot f=\sigma_2\cdot f\) parce que l'élément \( \sigma_2^{-1}\sigma_1\) est pair (proposition~\ref{ProphIuJrC}), de telle sorte que \( \sigma_2^{-1}\sigma_1\cdot f=f\). Nous posons donc \( g=\tau\cdot f\) où \( \tau\) est une permutation impaire quelconque -- par exemple une transposition.

    Vu que \( V\) est alternée et que \( \tau\) est une transposition nous avons
    \begin{equation}
        g=\tau\cdot f=P-VQ.
    \end{equation}
    Donc \( f+g=2P\) et \( f-g=2VQ\). Cela donne \( P\) et \( Q\) en termes de \( f\) et \( g\), et donc l'unicité.

    Attention : cela ne donne pas un moyen de prouver l'existence parce que rien ne prouve pour l'instant que \( f-g\) peut effectivement être écrit sous la forme \( VQ\), c'est-à-dire que \( f-g\) soit divisible par \( V\). C'est cela que nous allons nous atteler à démontrer maintenant.

    Nous commençons par prouver que \( f+g\) est symétrique et \( f-g\) alterné. Si \( \sigma\) est une transposition,
    \begin{equation}
        \sigma\cdot(f+g)=\sigma\cdot f+\sigma\tau\cdot f=g+f
    \end{equation}
    parce que \( \sigma\tau\) est pair. De la même façon,
    \begin{equation}
        \sigma\cdot(f-g)=g-f=\epsilon(\sigma)(f-g).
    \end{equation}
    Dans les deux cas nous concluons en utilisant le fait que toute permutation est un produit de transpositions (proposition~\ref{PropPWIJbu}) et que \( \epsilon\) est un homomorphisme.

    Soient maintenant deux entiers \( h<k\) dans \( \{ 1,\ldots, n \}\) et l'anneau
    \begin{equation}
        \big( \eK[T_1,\ldots, \hat T_k,\ldots, T_n] \big)[T_k].
    \end{equation}
    Cet anneau contient le polynôme \( T_k-T_h\) où \( T_k\) est la variable et \( T_h\) est un coefficient. Nous faisons la division euclidienne de \( f-g\) par  \( T_k-T_h\) parce que nous avons dans l'idée de faire arriver le déterminant de Vandermonde et donc le produit de toutes les différences \( T_k-T_h\) :
    \begin{equation}    \label{EqSHdgrG}
        f-g=(T_k-T_h)q+r
    \end{equation}
    où \( \deg_{T_k}r<1\), c'est-à-dire que \( r\) ne dépends pas de \( T_k\). Nous revoyons maintenant l'égalité \eqref{EqSHdgrG} dans \( \eK[T_1,\ldots, T_n]\) et nous y appliquons la transposition \( \tau_{kh}\). Nous savons que \( \tau_{kh}(f-g)=-(f-g)\) et \( \tau_{kh}(T_k-T_h)=-(T_k-T_h)\), et donc
    \begin{equation}    \label{EqVOhjKB}
        -(f-g)=-(T_k-T_h)\tau_{kh}\cdot   q+\tau_{kh}\cdot r
    \end{equation}
    où \(\tau_{kh}\cdot r\) ne dépend pas de \( T_h\). Nous appliquons à \eqref{EqVOhjKB} l'application
    \begin{equation}
        \begin{aligned}
            t\alpha\colon \eK[T_1,\ldots, T_n]&\to \eK[T_1,\ldots, \hat T_k,\ldots, T_n] \\
            \alpha(PT_1,\ldots, \hat T_k,\ldots, T_n)&=P(T_1,\ldots, T_h,\ldots, T_n).
        \end{aligned}
    \end{equation}
    Cette application vérifie \( \alpha\big( \tau_{kh}\cdot r \big)=\alpha(r)\) et nous avons
    \begin{equation}
        -\alpha(f-g)=\alpha(r).
    \end{equation}
    Puis en appliquant \( \alpha\) à la relation \( f-g=(T_k-T_h)q+r\), nous trouvons
    \begin{equation}
        \alpha(f-g)=\alpha(r),
    \end{equation}
    et par conséquent \( \alpha(r)=0\). Ici nous utilisons l'hypothèse de caractéristique différente de deux. Dire que \( \alpha(r)=0\), c'est dire que \( r\) est divisible par \( T_k-T_h\), mais \( r\) étant de degré zéro en \( T_k\), nous avons \( r=0\). Par conséquent \( T_k-T_h\) divise \( f-g\) pour tout \( h<k\), et nous pouvons définir un polynôme \( Q\) par
    \begin{equation}    \label{EqrnbgdA}
        f-g=2Q\prod_{h<k}\prod_{k\leq n}(T_k-T_h)=2Q(T_1,\ldots, T_n)V(T_1,\ldots, T_n),
    \end{equation}
    où nous avons utilisé la formule du déterminant de Vandermonde de la proposition~\ref{PropnuUvtj}.

    Étant donné que \( f+g\) est un polynôme symétrique, nous allons aussi poser \( f+g=2P\) avec \( P\) symétrique.

    Montrons à présent que \( Q\) est un polynôme symétrique. Soit \( \sigma\in S_n\); vu que nous savons déjà que \( f-g\) est alternée, nous avons
    \begin{equation}    \label{EqpSPEyq}
        \sigma\cdot (f-g)=\epsilon(\sigma)(f-g)=\epsilon(\sigma)2QV,
    \end{equation}
    Mais en appliquant \( \sigma\) à l'équation \eqref{EqrnbgdA},
    \begin{subequations}
        \begin{align}
            \sigma\cdot (f-g)&=2(\sigma\cdot V)(T_1,\ldots, ,T_n)(\sigma\cdot Q)(T_1,\ldots,T_n)\\
            &=2\epsilon(\sigma)V(T_1,\ldots, T_n)(\sigma\cdot Q)(T_1,\ldots, T_n).
        \end{align}
    \end{subequations}
    Nous égalisons cela avec \eqref{EqpSPEyq} et nous souvenant que l'anneau \( \eK[T_1,\ldots, T_n]\) est intègre par le théorème \ref{ThoBUEDrJ}. Ensuite nous simplifions par \( 2\epsilon(\sigma)V\) pour obtenir
    \begin{equation}
        Q=\sigma\cdot Q,
    \end{equation}
    c'est-à-dire que \( Q\) est symétrique.

    Au final nous avons \( f+q=2P\) et \( f-g=2VQ\) avec \( P\) et \( Q\) symétriques. En faisant la somme,
    \begin{equation}
        f=P+VQ.
    \end{equation}
\end{proof}

%---------------------------------------------------------------------------------------------------------------------------
\subsection{Déterminant de Gram}
%---------------------------------------------------------------------------------------------------------------------------

Si \( x_1,\ldots, x_r\) sont des vecteurs d'un espace vectoriel, alors le \defe{déterminant de Gram}{déterminant!Gram}\index{Gram (déterminant)} est le déterminant
\begin{equation}
    G(x_1,\ldots, x_r)=\det\big( \langle x_i, x_j\rangle  \big).
\end{equation}
Notons que la matrice est une matrice symétrique.

\begin{proposition}\label{PropMsZhIK}
    Si \( F\) est un sous-espace vectoriel de base \( \{ x_1,\ldots, x_n \}\) et si \( x\) est un vecteur, alors le déterminant de Gram est un moyen de calculer la distance entre \( x\) et \( F\) par
    \begin{equation}
        d(x,F)^2=\frac{ G(x,x_1,\ldots, x_n)}{ G(x_1,\ldots, x_n) }.
    \end{equation}
\end{proposition}

%---------------------------------------------------------------------------------------------------------------------------
\subsection{Déterminant de Cauchy}
%---------------------------------------------------------------------------------------------------------------------------

Soient des nombres \( a_i\) et \( b_i\) (\( i=1,\ldots, n\)) tels que \( a_i+b_j\neq 0\) pour tout couple \( (i,j)\). Le \defe{déterminant de Cauchy}{déterminant!de Cauchy}\index{Cauchy!déterminant} est
\begin{equation}
    D_n=\det\left( \frac{1}{ a_i+b_j } \right).
\end{equation}

\begin{proposition}[\cite{RollandRobertjyYDzY}] \label{ProptoDYKA}
    Le déterminant de Cauchy est donné par la formule
    \begin{equation}
        D_n=\frac{ \prod_{i<j}(a_j-a_i)\prod_{i<j}(b_j-b_i) }{ \prod_{ij}(a_i+b_j) }.
    \end{equation}
\end{proposition}

%---------------------------------------------------------------------------------------------------------------------------
\subsection{Matrice de Sylvester}
%---------------------------------------------------------------------------------------------------------------------------
\label{subsecSQBJfr}

La définition est pompée de \wikipedia{fr}{Matrice_de_Sylvester}{wikipédia}. Soient \( P\) et \( Q\) deux polynômes non nuls, de degrés respectifs \( m\) et \( n\) :
\begin{subequations}
    \begin{align}
        P(x)=p_0+p_1x+\cdots +p_nx^n\\
        Q(x)=q_0+q_1x+\cdots +q_mx^m.
    \end{align}
\end{subequations}
La \defe{matrice de Sylvester}{matrice!de Sylvester}\index{Sylvester (matrice)} associée à \( P\) et \( Q\) est la matrice carrée \( m+n\times m+n\) définie ainsi :
\begin{enumerate}
    \item
la première ligne est formée des coefficients de \( P\), suivis de 0 :
\begin{equation}
\begin{pmatrix} p_n & p_{n-1} & \cdots & p_1 & p_0 & 0 & \cdots & 0 \end{pmatrix} ;
\end{equation}
\item la seconde ligne s'obtient à partir de la première par permutation circulaire vers la droite ;
\item les $(m-2)$ lignes suivantes s'obtiennent en répétant la même opération ;
\item la ligne $(m+1)$ est formée des coefficients de \( Q\), suivis de 0 :
    \begin{equation}
    \begin{pmatrix} q_m & q_{m-1} & \cdots & q_1 & q_0 & 0 & \cdots & 0 \end{pmatrix} ;
    \end{equation}
    \item les $(m-1)$ lignes suivantes sont formées par des permutations circulaires.
\end{enumerate}

Ainsi dans le cas $n=4$ et $m=3$, la matrice obtenue est
\begin{equation}    \label{EqPEgtle}
S_{p,q}=\begin{pmatrix}
p_4 & p_3 & p_2 & p_1 & p_0 & 0 & 0 \\
0 & p_4 & p_3 & p_2 & p_1 & p_0 & 0 \\
0 & 0 & p_4 & p_3 & p_2 & p_1 & p_0 \\
q_3 & q_2 & q_1 & q_0 & 0 & 0 & 0 \\
0 & q_3 & q_2 & q_1 & q_0 & 0 & 0 \\
0 & 0 & q_3 & q_2 & q_1 & q_0 & 0 \\
0 & 0 & 0 & q_3 & q_2 & q_1 & q_0 \\
\end{pmatrix}.
\end{equation}
Le déterminant de la matrice de Sylvester associée à \( P\) et \( Q\) est appelé le \defe{résultant}{résultant} de \( P\) et \( Q\) et noté \( \res(P,Q)\)\nomenclature[A]{\( \res(P,Q)\)}{résultat des polynômes \( P\) et \( Q\)}.

Attention : si \( P\) est de degré \( n\) et \( Q\) de degré \( m\), il y a \( m\) lignes pour \( P\) et \( n\) pour \( Q\) dans le déterminant du résultant (et non le contraire).

\begin{lemma}[\cite{QQuRUzA}]       \label{LemBFrhgnA}
    Si \( P\) et \( Q\) sont deux polynômes de degrés \( n\) et \( m\) à coefficients dans l'anneau \( \eA\), alors pour tout \( \lambda\in \eA\),
    \begin{subequations}
        \begin{align}
            \res(\lambda P,Q)&=\lambda^m\res(P,Q)\\
            \res(P,\lambda Q)&=\lambda^n\res(P,Q).
        \end{align}
    \end{subequations}
\end{lemma}

\begin{proof}
    Cela est simplement un comptage du nombre de lignes. Il y a \( m\) lignes contenant les coefficients de \( P\); donc prendre \( \lambda P\) revient à multiplier \( m\) lignes dans un déterminant et donc le multiplier par \( \lambda^m\).
\end{proof}

L'équation de Bézout \eqref{EqkbbzAi}\index{théorème!Bézout!utilisation} peut être traitée avec une matrice de Sylvester. Soient \( P\) et \( Q\), deux polynômes donnés et à résoudre l'équation
\begin{equation}    \label{EqSsyXOo}
    xP+yQ=0
\end{equation}
par rapport aux polynômes inconnus \( x\) et \( y\) dont les degrés sont \( \deg(x)<\deg(Q)\) et \( \deg(y)<\deg(P)\). Si nous notons \( \tilde x\) et \( \tilde y\) la liste des coefficients de \( x\) et \( y\) (dans l'ordre décroissant de degré), nous pouvons récrire l'équation \eqref{EqSsyXOo} sous la forme
\begin{equation}
    S_{PQ}^t\begin{pmatrix}
        \tilde x    \\
        \tilde y
    \end{pmatrix}=0.
\end{equation}
Pour s'en convaincre, écrivons pour les polynômes de l'exemple \eqref{EqPEgtle} :
\begin{equation}
    \begin{pmatrix}
        p_4    &   0    &   0    &   q_3    &   0    &   0    &   0\\
        p_3    &   p_4    &   0    &   q_2    &   q_3    &   0    &   0\\
        p_2    &   p_3    &   p_4    &   q_1    &   q_2    &   q_3    &   0\\
        p_1    &   p_2    &   p_3    &   q_0    &   q_1    &   q_2    &   q_3\\
        p_0    &   p_1    &   p_2    &   0    &   q_0    &   q_1    &   q_2\\
        0    &   p_0    &   p_1    &   0    &   0    &   q_0    &   q_1\\
        0    &   0    &   p_0    &   0    &   0    &   0    &   q_0\\
    \end{pmatrix}\begin{pmatrix}
        x_2    \\
        x_1  \\
        x_0  \\
        y_3   \\
        y_2    \\
        y_1    \\
        y_0
    \end{pmatrix}=
    \begin{pmatrix}
        x_2p_4+y_2q_3    \\
        p_3x_2+p_4x_1+q_2y_3+q_3y_2  \\
          \\
           \\
        \vdots    \\
            \\

    \end{pmatrix}
\end{equation}
Nous voyons que sur la ligne numéro \( k\) (en partant du bas et en numérotant de à partir de zéro) nous avons les produits \( p_ix_j\) et \( q_iy_j\) avec \( i+j=k\). La colonne de droite représente donc bien les coefficients du polynôme \( xP+yQ\).


\begin{proposition} \label{PropAPxzcUl}
    Le résultant de deux polynômes est non nul si et seulement si les deux polynômes sont premiers entre eux.
\end{proposition}
\index{déterminant!résultant}

Un polynôme \( P\) a une racine double en \( a\) si et seulement si \( P\) et \( P'\) ont \( a\) comme racine commune, ce qui revient à dire que \( P\) et \( P'\) ne sont pas premiers entre eux.

Une application importante de ces résultats sera le théorème de Rothstein-Trager~\ref{ThoXJFatfu} sur l'intégration de fractions rationnelles.

\begin{example}
    Si nous prenons \( P=aX^2+bX+c\) et \( P'=2aX+b\) alors la taille de la matrice de Sylvester sera \( 2+1=3\) et
    \begin{equation}
        S_{P,P'}=\begin{pmatrix}
              a  &   b    &   c    \\
            2a    &   b    &   0    \\
            0    &   2a    &   b
        \end{pmatrix}.
    \end{equation}
    Le résultant est alors
    \begin{equation}
        \res(P,P')=-a(b^2-4ac).
    \end{equation}
    Donc un polynôme du second degré a une racine double si et seulement si \( b^2-4ac=0\). Cela est un résultat connu depuis longtemps mais qui fait toujours plaisir à revoir.
\end{example}

La matrice de Sylvester permet aussi de récrire l'équation de Bézout pour les polynômes; voir le théorème~\ref{ThoBezoutOuGmLB} et la discussion qui s'ensuit.

Une proposition importante du résultant est qu'il peut s'exprimer à l'aide des racines des polynômes.
\begin{proposition} \label{PropNDBOGNx}
    Si
    \begin{subequations}
        \begin{align}
        P(X)&=a_p\prod_{i=1}^p(X-\alpha_i)\\
        Q(X)&=b_q\prod_{j=1}^q(X-\beta_i)
        \end{align}
    \end{subequations}
    alors nous avons les expressions suivantes pour le résultant :
    \begin{equation}        \label{EqCFUumjx}
        \res(P,Q)=a_p^qb_q^p\prod_{i=1}^p\prod_{j=1}^q(\beta_j-\alpha_i)=b_q^p\prod_{j=1}^qP(\beta_j)=(-1)^{pq}a_p^q\prod_{i=1}^pQ(\alpha_i).
    \end{equation}
\end{proposition}

\begin{proof}
    Si \( P\) et \( Q\) ne sont pas premiers entre eux, d'une part la proposition~\ref{PropAPxzcUl} nous dit que \( \res(P,Q)=0\) et d'autre part, \( P\) et \( Q\) ont un facteur irréductible en commun, ce qui  signifie que nous devons avoir un des \( X-\alpha_i\) égal à un des \( X-\beta_j\). Autrement dit, nous avons \( \alpha_i=\beta_j\) pour un couple \( (i,j)\). Par conséquent tous les membres de l'équation \eqref{EqCFUumjx} sont nuls.

    Nous supposons donc que \( P\) et \( Q\) sont premiers entre eux. Nous commençons par supposer que les polynômes \( P\) et \( Q\) sont unitaires, c'est-à-dire que \( a_p=b_q=1\). Nous considérons alors l'anneau
    \begin{equation}
        \eA=\eZ[\alpha_1,\ldots, \alpha_p,\beta_1,\ldots, \beta_q].
    \end{equation}
    Dans cet anneau, l'élément \( \beta_j-\alpha_i\) est irréductible (tout comme \( X-Y\) est irréductible dans \( \eZ[X,Y]\)). Le résultant \( R=\res(P,Q)\) est un élément de \( \eA\) parce que tous leurs coefficients peuvent être exprimés à l'aide des \( \alpha_i\) et des \( \beta_j\). Dans \( \eA\), l'élément \( \beta_j-\alpha_i\) divise \( R\). En effet lorsque \( \beta_j=\alpha_i\), le déterminant définissant le résultant est nul, ce qui signifie que \( \beta_j-\alpha_i\) est un facteur irréductible de \( R\).

    Par conséquent il existe un polynôme \( T\in \eA\) tel que
    \begin{equation}
        R=\lambda(\alpha_1,\ldots, \beta_q)\prod_{i=1}^p\prod_{j=1}^r(\beta_j-\alpha_i).
    \end{equation}
    Comptons les degrés. Pour donner une idée de ce calcul de degré, voici comment se présente, au niveau des dimensions, le déterminant :
    \begin{equation}  \label{EqJCaATOH}
    \xymatrix{%
        \ar@{<->}[rrr]^{p+1}&&&& \ar@{<->}[r]^{q-1}  &\\
        a_p\ar@{.}[rrd] &a_{p-1}\ar@{.}[rr]  &  & a_0\ar@{.}[rrd] & 0\ar@{.}[r]&0&\ar@{<->}[d]^q \\
        0\ar@{.}[r]&0&a_p\ar@{.}[rr]&&a_1&a_0&\\
        \ar@{<->}[rrrrr]_{p+q}&&&&&&
       }
    \end{equation}
    si les \( a_i\) sont les coefficients de \( P\). Mais chacun des \( a_i\) est de degré \( 1\) en les \( \alpha_i\), donc le déterminant dans son ensemble est de degré \( q\) en les \( \alpha_i\), parce que \( R\) contient \( q\) lignes telles que \eqref{EqJCaATOH}. Le même raisonnement montre que \( R\) est de degré \( p\) en les \( \beta_j\). Par ailleurs le polynôme \( \prod_{i=1}^p\prod_{j=1}^r(\beta_j-\alpha_i)\) est de degré \( p\) en les \( \beta_j\) et \( q\) en les \( \alpha_i\). Nous en déduisons que \( T\) doit être un polynôme ne dépendant pas de \( \alpha_i\) ou de \( \beta_j\).

    Nous pouvons donc calculer la valeur de \( T\) en choisissant un cas particulier. Avec \( P(X)=X^p\) et \( Q(X)=X^q+1\), il est vite vu que \( R(P,Q)=1\) et donc que \( T=1\).

    Si les polynômes \( P\) et \( Q\) ne sont pas unitaires, le lemme~\ref{LemBFrhgnA} nous permet de conclure.

\end{proof}


%---------------------------------------------------------------------------------------------------------------------------
\subsection{Théorème de Kronecker}
%---------------------------------------------------------------------------------------------------------------------------

Nous considérons \( K_n\) l'ensemble des polynômes de \( \eZ[X]\)
\begin{enumerate}
    \item
        unitaires de degré \( n\),
    \item
        dont les racines dans \( \eC\) sont de modules plus petits ou égaux à \( 1\),
    \item
        et qui ne sont pas divisés par \( X\).
\end{enumerate}
Un tel polynôme s'écrit sous la forme
\begin{equation}
    P=X^n+\sum_{k=0}^{n-1}a_kX^k.
\end{equation}

\begin{theorem}[Kronecker\cite{KXjFWKA}]    \label{ThoOWMNAVp}
    Les racines des éléments de \( K_n\) sont des racines de l'unité.
\end{theorem}
\index{théorème!Kronecker}
\index{polynôme!à plusieurs indéterminées}
\index{résultant!utilisation}
\index{polynôme!symétrique}

\begin{proof}
    Vu que \( \eC\) est algébriquement clos
    nous pouvons considérer les racines \( \alpha_1,\ldots, \alpha_n\) de \( P\) dans \( \eC\). Nous les considérons avec leurs multiplicités.
%TODO : lorsqu'on aura démontré que \eC est algébriquement clos, il faudra le référentier ici.

    Soit \( R=X^n+\sum_{k=0}^{n-1}b_kX^k\) un élément de \( K_n\) dont nous notons \( \beta_1,\ldots, \beta_n\) les racines dans \( \eC\). Les relations coefficients-racines stipulent que
    \begin{equation}
        b_k=\sum_{1\leq i_1<\ldots <i_{n-k}\leq n}\prod_{j=1}^{n-k}\beta_{i_j}.
    \end{equation}
    En prenant le module et en se souvenant que \( | \beta_{l} |\leq 1\) pour tout \( l\), nous trouvons que
    \begin{equation}
        | b_k |\leq\binom{ n }{ n-k }.
    \end{equation}
    Mais comme \( b_k\in \eZ\), nous avons
    \begin{equation}
        b_k\in\big\{    -\binom{ n }{ n-k },-\binom{ n }{ n-k }+1,\ldots, 0,\cdots,\binom{ n }{ n-k }   \big\}
    \end{equation}
    qui est de cardinal \( \binom{ n }{ n-k }+1\). Nous avons donc
    \begin{equation}
        \Card(K_n)\leq\prod_{k=0}^{n-1}\big( 1+\binom{ n }{ n-k } \big)<\infty.
    \end{equation}
    La conclusion jusqu'ici est que \( K_n\) est un ensemble fini.

    Pour chaque \( k\in \eN^*\) nous considérons les polynômes
    \begin{subequations}
        \begin{align}
            P_k&=\prod_{i=1}^n(X-\alpha_i^k)\\
            Q_k&=X^k-Y\in \eZ[X,Y],
        \end{align}
    \end{subequations}
    et puis nous considérons le résultant \( R_k=\res_X(P,Q_k)\in \eZ[Y]\) :
    \begin{equation}
        R_k=\res_X(P,Q_k)=
        \begin{pmatrix}
            1&a_{n-1}&\cdots&a_0&0&\cdots&0&0&0\\
            0&1&a_{n-1}&\cdots&a_0&0&\cdots&0&0\\
            \vdots&\ddots&\ddots&\ddots&&\ddots&\\
            0&\cdots&0&1&a_{n-1}&\cdots&a_0&0&0\\
            0&\cdots&0&0&1&a_{n-1}&\cdots&a_0&0\\
            0&\cdots&0&0&0&1&a_{n-1}&\cdots&a_0\\
        \\
                    1&0&\ldots&0&-Y&0&\ldots&0&0\\
                    0&1&0&\ldots&0&-Y&0&\ldots&0\\
                &&\ddots&&&\ddots&\ddots\\
                    0&\cdots&0&1&0&\cdots&0&-Y&0\\
                    0&0&\cdots&0&1&0&\cdots&0&-Y
        \end{pmatrix}
    \end{equation}
    Cela est un polynôme en \( Y\) dont le terme de plus haut degré est \( (-1)^nY^n\). Les petites formules de la proposition~\ref{PropNDBOGNx} nous permettent d'exprimer \( R_k(Y)\) en termes des racines de \( P\) :
    \begin{equation}
        R_k(Y)=\prod_{i=1}^nQ_k(\alpha_i)=\prod_{i=1}^n(\alpha_i^k-Y)=(-1)^n\prod_{i=1}^n(Y-\alpha_i^k)=(-1)^nP_k(Y).
    \end{equation}
    Vu que \( P\in K_n\) nous savons que les \( \alpha_i\) ne sont pas tous nuls; donc \( P_k\in K_n\). Cependant nous avons vu que \( K_n\) est un ensemble fini; donc parmi les \( P_k\), il y a des doublons (et pas un peu)\quext{Ici dans \cite{KXjFWKA}, il déduit qu'on a un \( k\) tel que \( P_k=P_1=P\). Mois je vois pourquoi on a un \( k\) et un \( l\) tels que \( P_k=P_l\), mais pourquoi on peut en trouver un spécialement égal au premier ? Une réponse à cette question permettrait de solidement réduire la lourdeur de la suite de la preuve.}. Nous regardons même l'ensemble des \( P_{2^n}\) dans lequel nous pouvons en trouver deux les mêmes. Soit \( l>k\) tels que \( P_{2^k}=P_{2^l}\). Si \( \alpha\) est racine de \( P_{2^k}\), alors il est de la forme \( \alpha=\beta^{2^k}\) pour une certaine racines \( \beta\) de \( P\). Par conséquent
    \begin{equation}    \label{EqBEgJtzm}
        \alpha^{2^l/2^k}=\alpha^{2^{l-k}}
    \end{equation}
    est racine de \( P_{2^l}\). Notons que dans cette expression il n'y a pas de problèmes de définition d'exposant fractionnaire dans \( \eC\) parce que \( l>k\). Vu que \eqref{EqBEgJtzm} est racine de \( P_{2^l}\), il est aussi racine de \( P_{2^k}\). Donc
    \begin{equation}
        \big( \alpha^{2^{l-k}} \big)^{2^{l-k}}=\alpha^{2^{2(l-k)}}
    \end{equation}
    est racine de \( P_{2^l}\) et donc de \( P_{2^k}\). Au final nous savons que tous les nombres de la forme \( \alpha^{2^{n(l-k)}}\) sont racines de \( P_{2^k}\). Mais comme \( P_{2^k}\) a un nombre fini de racines, nous pouvons en trouver deux égales. Si nous avons
    \begin{equation}
        \alpha^{2^{n(l-k)}}=\alpha^{2^{m(l-k)}}
    \end{equation}
    pour certains entiers \( m>n\), alors
    \begin{equation}
        \alpha^{2^{n(l-k)}-2^{m(l-k)}}=1,
    \end{equation}
    ce qui prouver que \( \alpha\) est une racine de l'unité. Nous avons donc prouvé que toutes les racines de \( P_{2^k}\) sont des racines de l'unité et donc que les racines de \( P\) sont racines de l'unité.
\end{proof}

%+++++++++++++++++++++++++++++++++++++++++++++++++++++++++++++++++++++++++++++++++++++++++++++++++++++++++++++++++++++++++++ 
\section{Orientation}
%+++++++++++++++++++++++++++++++++++++++++++++++++++++++++++++++++++++++++++++++++++++++++++++++++++++++++++++++++++++++++++

%--------------------------------------------------------------------------------------------------------------------------- 
\subsection{Cas vectoriel}
%---------------------------------------------------------------------------------------------------------------------------

\begin{propositionDef}[\cite{BIBooEMJPooWLMWSd}]        \label{DEFooNVRHooEBHUSu}
    Soient deux bases \( \mB\) et \( \mB'\) d'un espace vectoriel réel \( E\). Nous définissons la relation \( \mB\sim \mB'\) si et seulement si \( \det_{\mB}(\mB')>0\)\footnote{Définition \ref{DEFooODDFooSNahPb}.}.

    Cela est une relation d'équivalence\footnote{Définition \ref{DefHoJzMp}.} sur l'ensemble des bases de \( E\), et les classes sont les \defe{orientations}{orientation} de \( E\).
\end{propositionDef}

\begin{proof}
    Tout est dans le lemme \ref{LemJMWCooELZuho}. D'abord quand \( \mB\) et \( \mB'\) sont des bases, \( \det_{\mB}(\mB')\neq 0\) ensuite, nous passons en revue les points qu'il faut pour être une relation d'équivalence.
    \begin{enumerate}
        \item
            \( \mB\sim \mB\) parce que \( \det_{\mB}(\mB)=1>0\).
        \item
            Vu que \( \det_{\mB}(\mB')=\frac{1}{ \det_{\mB'}(\mB) }\), les deux sont positifs en même temps ou pas du tout.
        \item
            Si \( \mB\sim \mB'\) et \( \mB'\sim\mB''\), alors en utilisant la formule
            \begin{equation}
                \det_{\mB}(\mB'')=\det_{\mB}(\mB')\det_{\mB'}(\mB''),
            \end{equation}
            nous voyons que \( \det_{\mB}(\mB'')>0\).
    \end{enumerate}
\end{proof}

\begin{lemma}
    Soit un espace vectoriel réel \( E\). L'ensemble des bases de \( E\) possède exactement deux orientations\footnote{Définition \ref{DEFooNVRHooEBHUSu}.}
\end{lemma}

\begin{proof}
    Nous considérons une base \( \mB=(e_1,\ldots, e_n\)\footnote{Nous notons \( (e_1,e_2)\) et non \( \{ e_1,e_2 \}\) parce que l'ordre est important.} à partir de laquelle nous définissons une autre base : \( \mB'=( -e_1,e_2,\ldots, e_n )\). Nous allons prouver que ces deux bases ne sont pas équivalentes, et que toute base de \( E\) est équivalente soit à \( \mB\) soit à \( \mB'\).

    \begin{subproof}
        \item[Au moins deux classes]
            Le fait que \( \det_{\mB}(\mB')=-1\) vient du fait que \( \det_{\mB}(\mB')=1\) et que l'application \( \det_{\mB}\) est \( n\)-linéaire; en multipliant par \( -1\) le premier argument, la valeur du déterminant est multipliée par \( -1\).

            Donc les bases \( \mB\) et \( \mB'\) ne sont pas équivalentes et il existe au moins deux classes.

        \item[Au plus deux classes]
            Nous montrons à présent que toute base est équivalente soit à \( \mB\) soit à \( \mB'\). Supposons que \( \mB''\) ne soit pas équivalente à \( \mB\), c'est-à-dire que \( \det_{\mB}(\mB'')<0\). Nous utilisons encore la formule \eqref{EqAWICooBLTTOY},
            \begin{equation}
                \underbrace{\det_{\mB}(\mB'')}_{<0}=\underbrace{\det_{\mB}(\mB')}_{<0}\det_{\mB'}(\mB''),
            \end{equation}
            et nous déduisons que \( \det_{\mB'}(\mB'')>0\).
    \end{subproof}
\end{proof}

\begin{normaltext}
    Vu qu'il n'y a que deux classes d'équivalence parmi les bases, nous pouvons utiliser le vocable «avoir la même orientation que» ou «avoir l'orientation contraire de». Ce n'est pas ambigu.
\end{normaltext}

\begin{proposition}[\cite{BIBooEMJPooWLMWSd}]
    Si \( \mB\) est une base de l'espace vectoriel \( E\) de dimension \( n\), et si \( \tau\) est une transposition\footnote{Définition \ref{DEFooXNAFooGTbTTJ}.} de \( S_n\), alors la base \( \tau(\mB)\) est de sens contraire.
\end{proposition}

\begin{proof}
    Le lemme \ref{LemJMWCooELZuho}\ref{ITEMooTXXBooBmDtzd} dit que \( \det_{\mB}\) est une forme anti-symétrique; donc
    \begin{equation}
        \det_{\mB}(\mB')=-\det_{\mB}\big( \tau(\mB)' \big).
    \end{equation}
    Si l'un est positif, l'autre est négatif. Elles ont donc des orientations contraires.
\end{proof}

\begin{corollary}
    Si \( \mB\) est une base de l'espace vectoriel \( E\) de dimension \( n\), et si \( \sigma\in S_n\), la base \( \sigma(\mB)\) a même orientation que \( \mB\) si et seulement si \( \sigma\in A_n\).
\end{corollary}

\begin{proof}
    Notons \( c_1\) la classe d'orientation de \( \mB\) et \( c_2\) l'autre classe. La permutation \( \sigma\) se décompose en produit de transpositions dont la parité est fixée (proposition \ref{PROPooKRHEooAxtmRv}). Posons \( \sigma=\tau_k\ldots \tau_1\).

    En posant \( \mB_0=\mB\) et \( \mB_{l+1}=\tau_{l+1}(\mB_l)\), pour tout \( l\), la base \( \mB_l\) est d'orientation contraire à celle de la base \( \mB_{l-1}\). Une base sur deux a l'orientation de \( \mB\) et l'autre sur deux a l'orientation contraire.

    Donc \( \sigma(\mB)\) a la même orientation que \( \mB\) si et seulement si \( k\) est pair. Mais \( \sigma\in A_n\) si et seulement si \( k\) est pair. C'est bon.
\end{proof}

\begin{propositionDef}[\cite{BIBooEMJPooWLMWSd}]        \label{PROPooNBAXooKNUrnk}
    Soit un espace vectoriel réel, et un endomorphisme \( f\) de \( E\). Deux définitions.
    \begin{enumerate}
        \item       \label{ITEMooOAXFooLIPHlW}
            L'endomorphisme \( f\) est \defe{direct}{endomorphisme direct} si sont déterminant est strictement positif.
        \item       \label{ITEMooNKYCooXTgKJA}
            L'endomorphisme \defe{préserve l'orientation}{endomorphisme préserve l'orientation} si il transforme toutes base de \( E\) en une base de même orientation.
    \end{enumerate}
    Un endomorphisme est direct si et seulement si il préserve l'orientation.
\end{propositionDef}

\begin{proof}
    En deux sens.
    \begin{subproof}
        \item[Direct implique préserve l'orientation]
            Soit une base \( \mB\) de \( E\) et un endomorphisme direct \( u\). D'abord, \( u\) est inversible du fait que son déterminant est non nul par la proposition \ref{PropYQNMooZjlYlA}\ref{ITEMooNZNLooODdXeH}. Donc \( u\) transforme une base en une base par le lemme \ref{LEMooDJSIooYcsvhO}.

            La définition \ref{LEMooQTRVooAKzucd} du déterminant de \( u\) est que
            \begin{equation}
                \det(u)=\det_{\mB}\big( u(\mB) \big)>0.
            \end{equation}
            Donc \( \mB\) et \( u(\mB)\) ont même orientation.
        \item[Préserve l'orientation implique direct]
            Le fait que \( u\) préserve l'orientation signifie en particulier qu'il transforme une base en une base et qu'il est inversible par le lemme \ref{LEMooDJSIooYcsvhO}.

            Donc si \( \mB\) est une base, \( u(\mB)\) est encore une base et nous avons, parce que \( \mB\) et \( u(\mB)\) ont même orientation,
            \begin{equation}
                0<\det_{\mB}\big( u(\mB) \big)=\det(u).
            \end{equation}
    \end{subproof}
\end{proof}

%--------------------------------------------------------------------------------------------------------------------------- 
\subsection{Cas affine}
%---------------------------------------------------------------------------------------------------------------------------

\begin{definition}      \label{DEFooOTFPooIVkHFP}
    Soit un espace affine \( \affE\) modelé sur \( E\). Les repères cartésiens\footnote{Définition \ref{DEFooQELZooEXvxgw}.} \( (O,\mB)\) et \( (O',\mB')\) ont \defe{même orientation}{orientation affine} si les bases \( \mB\) et \( \mB'\) ont même orientation.

    Les classes d'équivalence (il y en a deux) sont les orientations de \( \affE\).

Une application affine \( f\colon \affE\to \affE\) \defe{préserve l'orientation}{préserve l'orientation} si sa partie linéaire\footnote{Définition \ref{LEMooYJCDooOGAHkF}.} préserve l'orientation.
\end{definition}


%+++++++++++++++++++++++++++++++++++++++++++++++++++++++++++++++++++++++++++++++++++++++++++++++++++++++++++++++++++++++++++
\section{Hermitien, orthogonal, adjoint}
%+++++++++++++++++++++++++++++++++++++++++++++++++++++++++++++++++++++++++++++++++++++++++++++++++++++++++++++++++++++++++++


\begin{normaltext}      \label{NORMooWGEJooCtGtqZ}
    Une des choses à retenir de la définition de l'opérateur adjoint est que la notion de \( A^*\) dépend du produit scalaire considéré.
    
    Il se fait que le plus souvent, sur \( \eR^n\), nous considérons le produit scalaire usuel et la base canonique. De ce fait, les notions d'opérateur adjoint et d'opérateur transposés se confondent avec la notion de matrice transposée. Ce sont pourtant, en général, trois notions distinctes.
\end{normaltext}

\begin{propositionDef}[Définition de la transposée\cite{MonCerveau}]\label{DEFooROVNooFlTbSK}
    Soient deux espaces vectoriels euclidiens ou hermitiens \( E\) et \( F\) et une application linéaire \( A\colon E\to F\).
    \begin{enumerate}
        \item       \label{ITEMooRUZWooSZgGnf}
    Il existe une unique application linéaire \( B\colon F\to E\) telle que
    \begin{equation}        \label{EQooHWYKooFzAGgB}
        \langle Ax, y\rangle_F=\langle x, By\rangle_E
    \end{equation}
    pour tout \( x\in E\) et \( y\in F\).
\item   \label{ITEMooXXEUooPtfPKY}
 Si \( \{ e_i \}\) est une base orthonormée de \( E\) et \( \{ f_{\alpha} \}\) est une base orthonormée de \( F\), alors la matrice de \( A\) et \( B\) pour ces bases sont liées par
 \begin{equation}       \label{EQooUSNVooQtRNGL}
     B_{i\alpha}=A_{\alpha i}.
 \end{equation}
    \end{enumerate}
     L'application \( B\) ainsi définie set nommée \defe{adjoint}{adjoint} de \( A\) et sera notée \( B=A^*\). 
\end{propositionDef}

\begin{proof}
    Pour l'unicité, nous écrivons la condition avec \( x=e_j\) pour obtenir :
    \begin{equation}
        \langle Ae_j, y\rangle = \langle e_j, By\rangle =(By)_j
    \end{equation}
    c'est-à-dire que les coefficients \( B(y)_j\) de \( B(y)\) dans la base canonique sont fixés par la condition.

    Pour l'existence, il suffit de vérifier que poser
    \begin{equation}
        B(y)=\sum_j\langle Ae_j, y\rangle e_j
    \end{equation}
    fonctionne. Pour cela il faut utiliser la bilinéarité du produit scalaire et le fait que \( \langle x, e_j\rangle =x_j\). Nous avons :
    \begin{subequations}
        \begin{align}
            \langle x, B(y)\rangle &=\langle x, \sum_j\langle A(e_j), y\rangle e_j\rangle \\
            &=\sum_j\langle A(e_j), y\rangle \langle x, e_j\rangle \\
            &=\sum_j\langle A(x_je_j), y\rangle \\
            &=\langle A(x), y\rangle .
        \end{align}
    \end{subequations}

En ce qui concerne la matrice de l'application \( B\) ainsi définie, nous écrivons la condition \eqref{EQooHWYKooFzAGgB} avec \( y=e'_{\alpha}\) et \( x=e_i\), de telle sorte que
\begin{equation}
    A(x)=A(e_i)=\sum_{\beta}A_{\beta i}e'_{\beta}
\end{equation}
et
\begin{equation}
    B(y)=B(e'_{\alpha})=\sum_jB_{j\alpha}e_j.
\end{equation}
Alors nous avons :
\begin{equation}
    \sum_{\beta}A_{\beta i}\langle e'_{\beta}, e'_{\alpha}\rangle =\sum_j B_{j\alpha}\langle e_i, e_j\rangle ,
\end{equation}
donc
\begin{equation}
    A_{\alpha i}=B_{i \alpha}.
\end{equation}
\end{proof}

\begin{normaltext}
    À cause de l'expression \eqref{EQooUSNVooQtRNGL} pour la matrice de \( A^*\), cette application est souvent appelé \defe{transposé}{transposé} de \( A\) et noté \( A^t\). Nous savons, nous, que la transposée de \( A\) est une application \( A^t\colon F^*\to E^*\) donnée par la définition \ref{DefooZLPAooKTITdd}. Il nous arrivera donc d'écrire des égalités comme \( \langle Ax, y\rangle=\langle x, A^ty\rangle  \).
\end{normaltext}

\begin{proposition}     \label{PROPooSHZMooGwdfBd}
    En ce qui concerne le déterminant,
    \begin{equation}
        \det(A^*)=\det(A)^*
    \end{equation}
    où l'étoile à droite dénote la conjugaison complexe dans \( \eC\).
\end{proposition}

\begin{proof}
    Écrivons l'expression explicite \eqref{EQooOJEXooXUpwfZ} du déterminant. Le tout avec la base canonique :
    \begin{equation}
            \det(A)=\det_{(e_1,\ldots, e_n)}(Ae_1,\ldots, Ae_n)=\sum_{\sigma\in S_n}\epsilon(\sigma)\prod_{i=1}^ne_{\sigma(i)}^*(Ae_i).
    \end{equation}
    Mais nous pouvons développer :
    \begin{equation}
        e^*_{\sigma(i)}(Ae_i)=\langle e_{\sigma(i)}, Ae_i\rangle =\langle A^*e_{\sigma(i)}, e_i\rangle =\langle e_i, A^*e_{\sigma(i)}\rangle^*=e_i^*(A^*e_{\sigma(i)})^*.
    \end{equation}
    Notez que dans la dernière expression, les trois \( {}^*\) ont trois significations différentes. Par conséquent,
    \begin{equation}
        \det(A)=\sum_{\sigma\in S_n}\epsilon(\sigma)\prod_{i=1}^{n}e_i^*(A^*e_{\sigma(i)})^*.
    \end{equation}
    Mais \( e_i^*(A^*e_{\sigma(i)})=e_{\sigma(j)}^*(A^*e_{j})\) pour \( j=\sigma(i)\), donc le produit ne change pas si on déplace le \( \sigma\) :
    \begin{equation}
        \det(A)=\sum_{\sigma\in S_n}\epsilon(\sigma)\prod_{i=1}^{n}e_{\sigma(i)}^*(A^*e_{i})^*=\det(A^*)^*.
    \end{equation}
    Nous avons donc \( \det(A)=\det(A^*)^*\), c'est-à-dire \( \det(A)^*=\det(A^*)\). Pour information, la dernière étoile est la conjugaison complexe.
\end{proof}

\begin{proposition}[\cite{MonCerveau}]     \label{PROPooVPSYooRuoEFi}
    Si \( A\colon E_2\to E_3\) et \( B\colon E_1\to E_2\) sont des applications linéaires, alors
    \begin{equation}
        (AB)^*=B^*A^*
    \end{equation}
    où la «multiplication» est la composition.
\end{proposition}

\begin{proof}
    L'existence de \( (AB)^*\), de \( A^*\) et de \( B^*\) ne donne pas lieu à débat parce que la proposition \ref{DEFooROVNooFlTbSK} ne souffre pas de discussions. La propriété que \( (AB)^*\) est unique a avoir est que
    \begin{equation}
        \langle ABx, y\rangle =\langle x, (AB)^*y\rangle 
    \end{equation}
    pour tout \( x\in E_1\) et \( y\in E_3\). Or l'application \( B^*A^*\) possède également cette propriété parce que
    \begin{equation}
        \langle x, B^*A^*y\rangle =\langle Bx, A^*y\rangle =\langle ABx, y\rangle .
    \end{equation}
    La partie unicité de la proposition \ref{DEFooROVNooFlTbSK} nous impose donc d'accepter que les applications \( (AB)^*\) et \( B^*A^*\) sont en réalité les mêmes\footnote{Et ce même si vous croyez les avoir déjà vu ensemble dans la même pièce.}.
\end{proof}

\begin{normaltext}
    Un grand moment d'utilisation de la notion d'adjoint pour un opérateur non carré sera la définition d'une intégrale sur une variété; en particulier dans la proposition \ref{PROPooOAHWooAfxvyv}.
\end{normaltext}

\begin{definition}      \label{DEFooKEBHooWwCKRK}
    Un opérateur \( A\) est \defe{hermitien}{opérateur!hermitien} si \( A^*=A\). On dit aussi \defe{autoadjoint}{opérateur!autoadjoint}.
\end{definition}

\begin{normaltext}
    Le mot «hermitien» est réservé aux opérateurs sur des espaces hermitiens, c'est-à-dire des espaces vectoriels sur \( \eC\). Le mot «autoadjoint» par contre est plutôt utilisé dans le cadre d'opérateurs sur les espaces réels. En conséquence de quoi, ces deux mots sont synonymes, mais il est préférable d'utiliser «hermitien» lorsque l'espace vectoriel est sur \( \eC\) et «autoadjoint» lorsqu'il est sur \( \eR\).

    L'ensemble des opérateurs autoadjoints de \( E\) est noté \( \gS(E)\)\nomenclature[A]{\( \gS(E)\)}{Les opérateurs autoadjoints de $E$}. Cette notation provient du fait que dans \( \eR^n\) muni du produit scalaire usuel, les opérateurs autoadjoints sont les matrices symétriques.
\end{normaltext}

\begin{remark}
    Le fait d'être hermitien n'implique en rien le fait d'être inversible.
\end{remark}

\begin{lemma}
    Si \( E\) est un espace euclidien, un endomorphisme \( f\colon E\to E\) est autoadjoint si et seulement si pour tout \( x,y\in E\) nous avons \( \langle x, f(y)\rangle=\langle f(x), y\rangle  \).
\end{lemma}

\begin{proof}
    Dans le sens direct, nous avons
    \begin{equation}
        \langle f(x), y\rangle =\langle x, f^*(y)\rangle =\langle x, f(y)\rangle .
    \end{equation}
    La première égalité est la définition de \( f^*\) et la seconde est l'hypothèse \( f=f^*\).

    Dans l'autre sens, l'hypothèse est que l'endomorphisme \( f\) vérifie \( \langle x, f(y)\rangle =\langle f(x), y\rangle \). Mais la proposition \ref{DEFooROVNooFlTbSK}\ref{ITEMooRUZWooSZgGnf} spécifie que \( f^*\) est l'unique endomorphisme à satisfaire cette égalité. Donc \( f=f^*\).
\end{proof}

%---------------------------------------------------------------------------------------------------------------------------
\subsection{Opérateur orthogonal, matrice orthogonale}
%---------------------------------------------------------------------------------------------------------------------------

\begin{definition}      \label{DEFooYKCSooURQDoS}
    Un opérateur est \defe{orthogonal}{orthogonal!opérateur} lorsque \( A^*=A^{-1}\) où \( A^*\) est l'adjoint de \( A\) définit en~\ref{DEFooROVNooFlTbSK}.
\end{definition}

\begin{definition}      \label{DEFooUHANooLVBVID}
    Une matrice \( U\) est \defe{orthogonale}{matrice!orthogonale}\index{orthogonal!matrice} si \( U^t=U^{-1}\). Le \defe{groupe orthogonal}{groupe!orthogonal} noté \( \gO(n)\) est l'ensemble des matrices orthogonales \( n\times n\).
\end{definition}

\begin{lemma}       \label{LEMooSSALooSBFzJb}
    Soit un opérateur \( A\colon \eR^n\to \eR^n\) muni du produit scalaire usuel. Il est orthogonal si et seulement si sa matrice dans la base canonique est orthogonale\footnote{Définition~\ref{DEFooUHANooLVBVID}.}.
\end{lemma}

\begin{proof}
    Soit la base canonique \( \{ e_i \}_{i=1,\ldots, n}\) de \( \eR^n\). Nous avons
    \begin{equation}
        \langle AA^*e_i, e_j\rangle =\langle e_i, e_j\rangle =\delta_{ij},
    \end{equation}
    donc \( \big( (AA^*)e_i \big)_j=\delta_{ij}\), ou encore \( (AA^*)_{ij}=\delta_{ij}\), ce qui signifie que la matrice $AA^*$ est l'identité.
\end{proof}

\begin{proposition}[Thème~\ref{THMooVUCLooCrdbxm}]     \label{PropKBCXooOuEZcS}
    À propos de matrices orthogonales.
    \begin{enumerate}
        \item
            L'ensemble des matrices réelles orthogonales forme un groupe noté \( \gO(n,\eR)\)\nomenclature[B]{\( \gO(n,\eR)\)}{le groupe des matrices orthogonales}.
        \item
            Si \( A\) est une matrice orthogonale, alors \( \det(A)=\pm 1\).
        \item       \label{ITEMooOWMBooHUatNb}
            Le groupe \( \gO(n)\) est le groupe des isométries linéaires\footnote{Au sens où, parmi les applications linéaires, les isométries sont les éléments de \( \gO(n)\). À part ça, il y a aussi les translations, mais c'est une autre histoire qui vous sera contée une autre fois.} de \( \eR^n\).
    \end{enumerate}
\end{proposition}

\begin{proof}
    Si \( A\) et \( B\) sont orthogonales, alors
    \begin{equation}
        (AB)(AB)^t=ABB^tA^t=A\mtu A^t=\mtu.
    \end{equation}
    Vu que \( \mtu\) est orthogonale, nous avons bien un groupe.

    En ce qui concerne le déterminant, \( AA^t=\mtu\) donne \( \det(A)\det(A^t)=1\), mais la proposition~\ref{PROPooSHZMooGwdfBd} dit que \( \det(A)=\det(A^t)\), donc \( \det(A)^2=1\). D'où le fait que \( \det(A)=\pm 1\).

    D'autre part si \( A\) est une isométrie de \( \eR^n\) alors pour tout \( x,y\in \eR^n\) nous avons \( \langle Ax, Ay\rangle =\langle x, y\rangle \). En particulier,
    \begin{equation}
        \langle A^tAx, y\rangle =\langle x, y\rangle
    \end{equation}
    pour tout \( x,y\in \eR^n\). En prenant \( y=e_i\) nous trouvons
    \begin{equation}
        (A^tAx)_i=x_i,
    \end{equation}
    ce qui signifie que pour tout \( x\), \( A^tAx=x\), ou encore que \( A^tA\) est l'identité.

    Réciproquement si \( A^tA\) est l'identité nous avons
    \begin{equation}
        \langle x, y\rangle =\langle A^tAx, y\rangle =\langle Ax, Ay\rangle ,
    \end{equation}
    ce qui prouve que \( A\) est une isométrie.
\end{proof}

En ce qui concerne les valeurs propres des matrices de \( \gO(n)\) ainsi que leurs formes canoniques (avec des fonctions trigonométriques) pour \( \gO(3)\) et \( \SO(3)\), ce sera pour la proposition~\ref{PROPooVEJGooWnqtMm} et ce qui s'ensuit.

\begin{definition}      \label{DEFooJLNQooBKTYUy}
    Le sous-groupe des matrices orthogonales de déterminant \( 1\) est le groupe \defe{spécial orthogonal}{groupe!spécial orthogonal} noté \( \SO(n)\).
\end{definition}


% This is part of Mes notes de mathématique
% Copyright (c) 2008-2018, 2020
%   Laurent Claessens
% See the file fdl-1.3.txt for copying conditions.

%+++++++++++++++++++++++++++++++++++++++++++++++++++++++++++++++++++++++++++++++++++++++++++++++++++++++++++++++++++++++++++
\section{Topologie}
%+++++++++++++++++++++++++++++++++++++++++++++++++++++++++++++++++++++++++++++++++++++++++++++++++++++++++++++++++++++++++++

%---------------------------------------------------------------------------------------------------------------------------
\subsection{Boules et sphères}
%---------------------------------------------------------------------------------------------------------------------------

\begin{definition}
	Soit $(V,\| . \|)$, un espace vectoriel normé, $a\in V$ et $r>0$. Nous allons abondamment nous servir des ensembles suivants :
	\begin{enumerate}

		\item
			la \defe{boule ouverte}{boule!ouverte} $B(a,r)=\{ x\in V\tq \| x-a \|<r \}$;
		\item
			la \defe{boule fermée}{boule!fermée} $\bar B(a,r)=\{ x\in V\tq \| x-a \|\leq r \}$;
		\item
			la \defe{sphère}{sphère} $S(a,r)=\{ x\in V\tq \| x-a \|=r \}$.

	\end{enumerate}
\end{definition}
Les différences entre ces trois ensembles sont très importantes. D'abord, les \emph{boules} sont pleines tandis que la \emph{sphère} est creuse. En comparant à une pomme, la boule ouverte serait la pomme «sans la peau», la boule fermée serait «avec la peau» tandis que la sphère serait seulement la peau. Nous avons
\begin{equation}
	\bar B(a,r)=B(a,r)\cup S(a,r).
\end{equation}

\begin{definition}
	Une partie $A$ de $V$ est dite \defe{bornée}{borné!partie de $V$} s'il existe un réel $R$ tel que $A\subset B(0_V,R)$.
\end{definition}
Une partie est donc bornée si elle est contenue dans une boule de rayon fini.

\begin{example}
	Dans $\eR$, les boules sont  les intervalles ouverts et fermés tandis que la sphère est donnée par les points extrêmes des intervalles :
	\begin{equation}
		\begin{aligned}[]
			B(a,r)&=\mathopen] a-r , a+r \mathclose[,\\
			\bar B(a,r)&=\mathopen[ a-r , a+b \mathclose],\\
			S(a,r)&=\{ a-r,a+r \}.
		\end{aligned}
	\end{equation}
\end{example}

\begin{example}
	Si nous considérons $\eR^2$, la situation est plus riche parce que nous avons plus de normes. Essayons de voir les sphères de centre $(0,0)\in\eR^2$ et de rayon $r$ pour les normes $\| . \|_1$, $\| . \|_2$ et $\| . \|_{\infty}$.

	Pour la norme $\| . \|_1$, la sphère de rayon $r$ est donnée par l'équation
	\begin{equation}
		| x |+| y |=r.
	\end{equation}
	Pour la norme $\| . \|_2$, l'équation de la sphère de rayon $r$ est
	\begin{equation}
		\sqrt{x^2+y^2}=r,
	\end{equation}
	et pour la norme supremum, la sphère de rayon $r$ a pour équation
	\begin{equation}
		\max\{ | x |,| y | \}=r.
	\end{equation}
	Elles sont dessinées sur la figure~\ref{LabelFigLesSpheres}
\newcommand{\CaptionFigLesSpheres}{Les sphères de rayon $1$ pour les trois normes classiques.}
\input{auto/pictures_tex/Fig_LesSpheres.pstricks}
\end{example}

\begin{proposition}		\label{PropBoitPtLoin}
	Soient $V$ un espace vectoriel normé, $a$ dans $V$ et $x$ tel que $d(a,x)=r$, c'est-à-dire $x\in S(a,r)$. Dans ce cas, toute boule centrée en $x$ contient un point $P$ tel que $d(P,a)>r$ et un point $Q$ tel que $d(Q,a)<r$.
\end{proposition}

\begin{proof}
	Soit une boule de rayon $\delta$ autour de $x$. Le but est de trouver un point $P$ tel que $d(P,a)>r$ et $d(P,x)<\delta$. Pour cela, nous prenons $P$ sur la même droite que $x$ (en partant de $a$), mais juste «un peu plus loin», comme sur la figure suivante :

    \begin{center}
       \input{auto/pictures_tex/Fig_BoulePtLoin.pstricks}
    \end{center}

   Plus précisément, nous considérons le point
	\begin{equation}
		P=x+\frac{ v }{ N }
	\end{equation}
	où $v=x-a$ et $N$ est suffisamment grand pour que $d(x,P)$ soit plus petit que $\delta$. Cela est toujours possible parce que
	\begin{equation}
		d(P,x)=\| P-x \|=\frac{ \| v \| }{ N }
	\end{equation}
	peut être rendu aussi petit que l'on veut par un choix approprié de $N$. Montrons maintenant que $d(a,P)>d(a,x)$ :
	\begin{equation}
		\begin{aligned}[]
			d(a,P)&=\| a-x-\frac{ v }{ N }\| \\
			&=\| a-x+\frac{ a }{ N }-\frac{ x }{ N } \|\\
			&=\| \big( 1+\frac{1}{ N }(a-x) \big) \|\\
			&>\| a-x \|=d(a,x).
		\end{aligned}
	\end{equation}
	Nous laissons en exercice le soin de trouver un point $Q$ tel que $d(Q,a)<r$ et $d(Q,x)<\delta$.
\end{proof}


%---------------------------------------------------------------------------------------------------------------------------
\subsection{Ouverts, fermés, intérieur et adhérence}
%---------------------------------------------------------------------------------------------------------------------------

\begin{definition}
	Soit $(V,\| . \|)$ un espace vectoriel normé et $A$, une partie de $V$. Un point $a$ est dit \defe{intérieur}{intérieur!point} à $A$ s'il existe une boule ouverte centrée en $a$ et contenue dans $A$.

	On appelle \defe{l'intérieur}{intérieur!d'un ensemble} de $A$ l'ensemble des points qui sont intérieurs à $A$. Nous notons $\Int(A)$ l'intérieur de $A$.
\end{definition}
Notons que $\Int(A)\subset A$ parce que si $a\in\Int(A)$, nous avons $B(a,r)\subset A$ pour un certain $r$ et en particulier $a\in A$.

\begin{example}
	Trouver l'intérieur d'un intervalle dans $\eR$ consiste à «ouvrir là où c'est fermé».
	\begin{enumerate}

		\item
			$\Int\big(\mathopen[ 0 , 1 [\big)=\mathopen] 0 , 1 \mathclose[$.

			Prouvons d'abord que $\mathopen] 0,1  \mathclose[\subset\Int(\mathopen[ 0 , 1 [)$. Si $a\in\mathopen] 0 , 1 \mathclose[$, alors $a$ est strictement supérieur à $0$ et strictement inférieur à $1$. Dans ce cas, la boule de centre $a$ et de rayon $\frac{ \min\{ a,1-a \} }{ 2 }$ est contenue dans $\mathopen] 0 , 1 \mathclose[$ (voir figure~\ref{LabelFigIntervalleUn}). Cela prouve que $a$ est dans l'intérieur de $\mathopen[ 0 , 1 [$.

\newcommand{\CaptionFigIntervalleUn}{Trouver le rayon d'une boule autour de $a$. Une boule qui serait centrée en $a$ avec un rayon strictement plus petit à la fois de $a$ et de $1-a$ est entièrement contenue dans le segment $\mathopen] 0 , 1 \mathclose[$.}
\input{auto/pictures_tex/Fig_IntervalleUn.pstricks}

			Prouvons maintenant que $\Int\big( \mathopen[ 0 , 1 [ \big)\subset\mathopen] 0 , 1 \mathclose[$. Vu que l'intérieur d'un ensemble est inclus dans l'ensemble, nous savons déjà que $\Int\big( \mathopen[ 0 , 1 [ \big)\subset\mathopen[ 0 , 1 [$. Nous devons donc seulement montrer que $0$ n'est pas dans l'intérieur de $\mathopen[ 0 , 1 [$. C'est le cas parce que toute boule du type $B(0,r)$ contient le point $-r/2$ qui n'est pas dans $\mathopen[ 0 , 1 [$.

		\item
			$\Int\Big( \mathopen[ 0 , \infty [ \Big)=\mathopen] 0 , \infty \mathclose[$.
		\item
			$\Int\big( \mathopen] 2 , 3 \mathclose[ \big)=\mathopen] 2 , 3 \mathclose[$.

	\end{enumerate}

\end{example}

\begin{example}			\label{ExempleIntBoules}
	Les intérieurs des boules et sphères sont importantes à savoir.
	\begin{enumerate}
		\item
			$\Int\big( B(a,r) \big)=B(a,r)$. Si $x\in B(a,r)$, nous avons $d(a,x)<r$. Alors la boule $B\big(x,r-d(x,a)\big)$ est incluse à $B(a,r)$, et donc $x$ est dans l'intérieur de $B(a,r)$. Conseil : faire un dessin.
		\item
			$\Int\big( \bar B(a,r) \big)=B(a,r)$. Par le point précédent, la boule $B(a,r)$ est certainement dans l'intérieur de la boule fermée. Il reste à montrer que les points de $\bar B(a,r)$ qui ne sont pas dans $B(a,r)$ ne sont pas dans l'intérieur. Ces points sont ceux dont la distance à $a$ est \emph{égale} à $r$. Le résultat découle alors de la proposition~\ref{PropBoitPtLoin}.

		\item
			$\Int\big( S(a,r) \big)=\emptyset$. Si $x\in S(a,r)$, toute boule centrée en $a$ contient des points qui ne sont pas à distance $r$ de $a$.

			Notez que la sphère est un exemple d'ensemble non vide mais d'intérieur vide.
	\end{enumerate}
\end{example}


\begin{definition}
	Une partie $A$ de l'espace vectoriel normé $(V,\| . \|)$ est dite \defe{ouverte}{ouvert} si chacun de ses points est intérieur. La partie $A$ est donc ouverte si $A\subset\Int(A)$. Par convention, nous disons que l'ensemble vide $\emptyset$ est ouvert.

	Une partie est dite \defe{fermée}{fermé} si son complémentaire est ouvert. La partie $A$ est donc fermée si $V\setminus A$ est ouverte.
\end{definition}

Remarque : un ensemble $A$ est ouvert si et seulement si $\Int(A)=A$.

\begin{definition}
	Une partie $A$ de l'espace vectoriel normé $V$ est dite \defe{compacte}{compact} si elle est fermée et bornée.
\end{definition}

Nous verrons tout au long de ce cours que les ensembles compacts, et les fonctions définies sur ces ensembles ont de nombreuses propriétés agréables.

\begin{example}		\label{ExempleFermeIntevrR}
	En ce qui concerne les intervalles de $\eR$,
	\begin{itemize}
		\item $\mathopen] 1 , 2 \mathclose[$ est ouvert;
		\item $\mathopen[ 3,  4 \mathclose]$ est fermé;
		\item $\mathopen[ 5 , 6 [$ n'est ni ouvert ni fermé;
	\end{itemize}
	Les intervalles fermés de $\eR$ sont toujours compacts.
\end{example}

\begin{proposition}		\label{PropTopologieAx}
	Soit $V$ un espace vectoriel normé.
	\begin{enumerate}
		\item
			L'ensemble $V$ lui-même et le vide sont à la fois fermés et ouverts.
		\item
			Toute union d'ouverts est ouverte.
		\item
			Toute intersection \emph{finie} d'ouverts est ouverte.
		\item		\label{ItemPropTopologieAxiv}
			Le vide et $V$ sont les seules parties de $V$ à être à la fois fermées et ouvertes.
	\end{enumerate}
\end{proposition}

\begin{proof}
	L'ingrédient principal de cette démonstration est que si $a$ est un point d'un ouvert $\mO$, alors il existe une boule autour de $a$ contenue dans $\mO$ parce que $a$ doit être dans l'intérieur de $\mO$.
	\begin{enumerate}

		\item
			Nous avons déjà dit que, par définition, l'ensemble vide est ouvert. Cela implique que $V$ lui-même est fermé (parce que son complémentaire est le vide). De plus, $V$ est ouvert parce que toutes les boules sont inclues à $V$. Le vide est alors fermé (parce que son complémentaire est $V$).
		\item
			Soit une famille $(\mO_i)_{i\in I}$ d'ouverts\footnote{L'ensemble $I$ avec lequel nous «numérotons» les ouverts $\mO_i$ est \emph{quelconque}, c'est-à-dire qu'il peut être $\eN$, $\eR$, $\eR^n$ ou n'importe quel autre ensemble, fini ou infini.}, et l'union
			\begin{equation}
				\mO=\bigcup_{i\in I}\mO_i.
			\end{equation}
			Soit maintenant $a\in\mO$. Nous devons prouver qu'il existe une boule centrée en $a$ entièrement contenue dans $\mO$. Étant donné que $a\in\mO$, il existe $i\in I$ tel que $a\in\mO_i$ (c'est-à-dire que $a$ est au moins dans un des $\mO_i$). Par hypothèse l'ensemble $\mO_i$ est ouvert et donc tous ses points (en particulier $a$) sont intérieurs; il existe donc une boule $B(a,r)$ centrée en $a$ telle que $B(a,r)\subset\mO_i\subset\mO$.

		\item
			Soit une famille finie d'ouverts $(\mO_k)_{k\in\{ 1,\ldots,n \}}$, et $a\in\mO$ où
			\begin{equation}
				\mO=\bigcap_{k=1}^n\mO_k.
			\end{equation}
			Vu que $a$ appartient à chaque ouvert $\mO_k$, nous pouvons trouver, pour chacun de ces ouverts, une boule $B(a,r_k)$ contenue dans $\mO_k$. Chacun des $r_k$ est strictement positif, et nous n'en avons qu'un nombre fini, donc le nombre $r=\min\{ r_1,\ldots,r_n \}$ est strictement positif. La boule $B(a,r)$ est inclue dans toutes les autres (parce que $B(a,r)\subset B(a,r')$ lorsque $r\leq r'$), par conséquent
			\begin{equation}
				B(a,r)\subset\bigcap_{k=1}^nB(a,r_k)\subset\bigcap_{k=1}^n\mO_k=\mO,
			\end{equation}
			c'est-à-dire que la boule de rayon $r$ est une boule centrée en $a$ contenue dans $\mO$, ce qui fait que $a$ est intérieur à $\mO$.
		\item
			Nous acceptons ce point sans démonstration.
	\end{enumerate}
   % TODO : trouver et mettre une preuve du dernier point.

\end{proof}

La proposition dit que toute intersection \emph{finie} d'ouvert est ouverte. Il est faux de croire que cela se généralise aux intersections infinies, comme le montre l'exemple suivant :
\begin{equation}
	\bigcap_{i=1}^{\infty}\mathopen] -\frac{1}{ n } , \frac{1}{ n } \mathclose[=\{ 0 \}.
\end{equation}
Chacun des ensembles $\mathopen] -\frac{1}{ n } , \frac{1}{ n } \mathclose[$ est ouvert, mais le singleton $\{ 0 \}$ est fermé (pourquoi ?).

Nous reportons à la proposition~\ref{DefSupeA} la preuve du fait que tout ensemble borné de $\eR$ possède un infimum et un supremum.



\begin{definition}
	L'ensemble des ouverts de $V$ est la \defe{topologie}{topologie} de $V$. La topologie dont nous parlons ici est dite \defe{induite}{induite!topologie} par la norme $\| . \|$ de $V$ (parce que cette norme définit la notion de boule et qu'à son tour la notion de boule définit la notion d'ouverts). Un \defe{voisinage}{voisinage} de $a$ dans $V$ est un ensemble contenant un ouvert contenant $a$.
\end{definition}

Il existe de nombreuses topologies sur un espace vectoriel donné, mais certaines sont plus fameuses que d'autres. Dans le cas de $V=\eR^n$, la topologie \defe{usuelle}{topologie!usuelle sur $\eR^n$} est celle induite par la norme euclidienne. Lorsque nous parlons de boules, de fermés, de voisinages ou d'autres notions topologiques (y compris de convergence, voir plus bas) dans $\eR^n$, nous sous-entendons toujours la topologie de la norme euclidienne.

\begin{example}
	Les ensembles suivants sont des voisinages de $3$ dans $\eR$ :
	\begin{itemize}
		\item
			$\mathopen] 1 , 5 \mathclose[$;
		\item
			$\mathopen[ 0 , 10 \mathclose]$;
		\item
			$\eR$.
	\end{itemize}
	Les ensembles suivants ne sont pas des voisinages de $3$ dans $\eR$ :
	\begin{itemize}
		\item
			$\mathopen] 1 , 3 \mathclose[$;
		\item
			$\mathopen] 1 , 3 \mathclose]$;
		\item
			$\mathopen[ 0 , 5 [\setminus\{ 3 \}$.
	\end{itemize}
\end{example}

\begin{proposition}
	Dans un espace vectoriel normé,
	\begin{enumerate}
		\item
			toute intersection de fermés est fermée;
		\item
			toute union \emph{finie} de fermés est fermée.
	\end{enumerate}
\end{proposition}
Encore une fois, l'hypothèse de finitude de l'intersection est indispensable comme le montre l'exemple suivant :
\begin{equation}
	\bigcup_{n=1}^{\infty}\mathopen[ -1+\frac{1}{ n } , 1-\frac{1}{ n } \mathclose]=\mathopen] -1 , 1 \mathclose[.
\end{equation}
Chacun des intervalles dont on prend l'union est fermé tandis que l'union est ouverte.

\begin{definition}
	Soit $A$, une partie de l'espace vectoriel normé $V$. Un point $a\in V$ est dit \defe{adhérent}{adhérence} à $A$ dans $V$ si pour tout $\varepsilon>0$,
	\begin{equation}
		B(a,\varepsilon)\cap A\neq\emptyset.
	\end{equation}
	Nous notons $\bar A$ l'ensemble des points adhérents à $a$ et nous disons que $\bar A$ est l'adhérence de $A$. L'ensemble $\bar A$ sera aussi souvent nommé \defe{fermeture}{fermeture} de l'ensemble $A$.
\end{definition}
Un point peut être adhérent à $A$ sans faire partie de $A$, et nous avons toujours $A\subset\bar A$.

\begin{example}     \label{EXooICLBooJzQFNY}
	La terminologie «fermeture» de $A$ pour désigner $\bar A$ provient de deux origines.
	\begin{enumerate}
		\item
            L'ensemble $\bar A$ est le plus petit fermé contenant $A$. Cela signifie que si $B$ est un fermé qui contient $A$, alors $\bar A\subset A$. Cela est fondamentalement le sens de la définition~\ref{DEFooSVWMooLpAVZR}.
            % position 25804
            %Nous allons prouver cette affirmation dans l'exercice~\ref{exoGeomAnal-0008}.
		\item
			Pour les intervalles dans $\eR$, trouver $\bar A$ revient à fermer les extrémités qui sont ouvertes, comme on en a parlé dans l'exemple~\ref{ExempleFermeIntevrR}.
	\end{enumerate}
\end{example}

\begin{example}
	Dans $\eR$, l'infimum et le supremum d'un ensemble sont des points adhérents. En effet si $M$ est le supremum de $A\subset\eR$, pour tout $\varepsilon$, il existe un $a\in A$ tel que $a>M-\varepsilon$, tandis que $M>a$. Cela fait que $a\in B(M,\varepsilon)$, et en particulier que pour tout rayon $\varepsilon$, nous avons $B(M,\varepsilon)\cap A\neq\emptyset$.

	Le même raisonnement montre que l'infimum est également dans l'adhérence de $A$.
\end{example}

\begin{example}		\label{ParlerEncoredeF}
	Il ne faut pas conclure de l'exemple précédent qu'un point limite ou adhérent est automatiquement un minimum ou un maximum. En effet, si nous regardons l'ensemble formé par les points de la suite $x_n=(-1)^n/n$, le nombre zéro est un point adhérent et une limite, mais pas un infimum ni un maximum.
\end{example}

\begin{lemma}
	Si $B$ est une partie fermée de $V$, alors $B=\bar B$.
\end{lemma}

\begin{proof}
	Supposons qu'il existe $a\in\bar B$ tel que $a\notin B$. Alors il n'y a pas d'ouverts autour de $a$ qui soit contenu dans $\complement B$. Cela prouve que $\complement B$ n'est pas ouvert, et par conséquent que $B$ n'est pas fermé. Cela est une contradiction qui montre que tout point de $\bar B$ doit appartenir à $B$ lorsque $B$ est fermé.
\end{proof}

\begin{example}
	Au niveau des intervalles dans $\eR$, prendre l'adhérence consiste à «fermer là où c'est ouvert». Attention cependant à ne pas fermer l'intervalle en l'infini.
	\begin{enumerate}
		\item
			$\overline{ \mathopen[ 0 , 2 [ }=\mathopen[ 0 , 2 \mathclose]$.
		\item
			$\overline{ \mathopen] 3 , \infty \mathopen] }=\mathopen[ 3 , \infty [$.
	\end{enumerate}
\end{example}

\begin{proposition}
	Soit $V$ un espace vectoriel normé et $a\in V$. Les trois conditions suivantes sont équivalentes :
	\begin{enumerate}
		\item
			$a\in\bar A$;
		\item
			il existe une suite d'éléments $x_n$ dans $A$ qui converge vers $a$;
		\item
			$d(a,A)=0$.
	\end{enumerate}
\end{proposition}
Notez que dans cette proposition, nous ne supposons pas que $a$ soit dans $A$.

\begin{proposition}		\label{PropComleIntBar}
	Pour toute partie $A$ d'un espace vectoriel normé nous avons
	\begin{enumerate}
		\item
			$V\setminus\bar A=\Int(V\setminus A)$,
		\item
			$V\setminus\Int(A)=\overline{ V\setminus A }$.
	\end{enumerate}
\end{proposition}

En utilisant les notations du complémentaire (\ref{AppComplement}), les deux points de la proposition se récrivent
\begin{enumerate}
	\item
		$\complement \bar A=\Int(\complement A)$,
	\item\label{ItemLemPropComplementiv}
		$\complement\Int(A)=\overline{ \complement A }$.
\end{enumerate}

\begin{proof}
	Nous avons $a\in V\setminus\bar A$ si et seulement si $a\notin\bar A$. Or ne pas être dans $\bar A$ signifie qu'il existe un rayon $\varepsilon$ tel que la boule $B(a,\varepsilon)$ n'intersecte pas $A$. Le fait que la boule $B(a,\varepsilon)$ n'intersecte pas $A$ est équivalent à dire que $B(a,\varepsilon)\subset V\setminus A$. Or cela est exactement la définition du fait que $a$ est à l'intérieur de $V\setminus A$. Nous avons donc montré que $a\in V\setminus \bar A$ si et seulement si $a\in\Int(V\setminus A)$. Cela prouve la première affirmation.

	Pour prouver la seconde affirmation, nous appliquons la première au complémentaire de $A$ : $\complement(\overline{ \complement A })=\Int(\complement\complement A)$. En prenant le complémentaire des deux membres nous trouvons successivement
	\begin{equation}
		\begin{aligned}[]
			\complement\complement(\overline{ \complement A })&=\complement\Int(\complement\complement A),\\
			\overline{ \complement A }&=\complement\Int(A),
		\end{aligned}
	\end{equation}
	ce qu'il fallait démontrer.
\end{proof}

Attention à ne pas confondre $\complement \bar A$ et $\overline{ \complement A }$. Ces deux ensembles ne sont pas égaux. En effet, en tant que complément d'un fermé, l'ensemble $\complement \bar A$ est certainement ouvert, tandis que, en tant que fermeture, l'ensemble $\overline{ \complement A }$ est fermé. Pouvez-vous trouver des exemples d'ensembles $A$ tels que $\complement \bar A=\overline{ \complement A }$ ?

\begin{proposition}
	Soient $A$ et $B$ deux parties de l'espace vectoriel normé $V$.
	\begin{enumerate}
		\item
			Pour les inclusions, si $A\subset B$, alors $\Int(A)\subset\Int(B)$ et $\bar A\subset\bar B$.
		\item
			Pour les unions, $\overline{ A\cup B }=\overline{ A }\cup\overline{ B }$ et $\overline{ A\cap B }\subset\bar A\cap\bar B$.
		\item
			Pour les intersections, $\Int(A)\cap\Int(B)=\Int(A\cap B)$ et $\Int(A)\cup\Int(B)\subset\Int(A\cup B)$.
	\end{enumerate}
\end{proposition}

\begin{proof}
	\begin{enumerate}
		\item
			Si $a$ est dans l'intérieur de $A$, il existe une boule autour de $a$ contenue dans $A$. Cette boule est alors contenue dans $B$ et donc est une boule autour de $a$ contenue dans $B$, ce qui fait que $a$ est dans l'intérieur de $B$. Si maintenant $a$ est dans l'adhérence de $A$, toute boule centrée en $a$ contient un élément de $A$ et donc un élément de $B$, ce qui prouve que $a$ est dans l'adhérence de $B$.
		\item
			Nous avons $A\subset A\cup B$ et donc, en utilisant le premier point, $\bar A\subset\overline{ A\cup B }$. De la même manière, $\bar B\subset\overline{ A\cup B }$. En prenant l'union, $\bar A\cup\bar B\subset\overline{ A\cup B }$.

			Réciproquement, soit $a\in\overline{ A\cup B }$ et montrons que $a\in\bar A\cup\bar B$. Supposons par l'absurde que $a$ ne soit ni dans $\bar A$ ni dans $\bar B$. Il existe donc des rayons $\varepsilon_1$ et $\varepsilon_2$ tels que
			\begin{equation}
				\begin{aligned}[]
					B(a,\varepsilon_1)\cap A&=\emptyset,\\
					B(a,\varepsilon_2)\cap B&=\emptyset.
				\end{aligned}
			\end{equation}
			En prenant $r=\min\{ \varepsilon_1,\varepsilon_2 \}$, la boule $B(a,r)$ est inclue aux deux boules citées et donc n'intersecte ni $A$ ni $B$. Donc $a\notin\overline{ A\cup B }$, d'où la contradiction.

		\item
			Si nous appliquons le second point à $\complement A$ et $\complement B$, nous trouvons
			\begin{equation}
				\overline{ \complement A\cup\complement B }=\overline{ \complement A}\cup\overline{ \complement B}.
			\end{equation}
			En utilisant les propriétés du lemme~\ref{LemPropsComplement}, le membre de gauche devient
			\begin{equation}	\label{Eq2707CACBCAB}
				\overline{ \complement A\cup\complement B }=\overline{ \complement(A\cap B) }=\complement\Int(A\cap B),
			\end{equation}
			tandis que le membre de droite devient
			\begin{equation}		\label{Eq2707cAcBACAACB}
				\overline{ \complement A }\cup\overline{ \complement B }=\complement\Int(A)\cup\complement\Int(A)=\complement\Big( \Int(A)\cap\Int(B) \Big).
			\end{equation}
			En égalisant le membre de droite de \eqref{Eq2707CACBCAB} avec celui de \eqref{Eq2707cAcBACAACB} et en passant au complémentaire nous trouvons
			\begin{equation}
				\Int(A\cap B)=\Int(A)\cap\Int(B),
			\end{equation}
			comme annoncé.

			La dernière affirmation provient du fait que $\Int(A)\subset\Int(A\cup B)$ et de la propriété équivalente pour $B$.
	\end{enumerate}
\end{proof}

\begin{remark}
	Nous avons prouvé que $\overline{ A\cap B }\subset\bar A\cap\bar B$. Il arrive que l'inclusion soit stricte, comme dans l'exemple suivant. Si nous prenons $A=\mathopen[ 0 , 1 \mathclose]$ et $B=\mathopen] 1 , 2 \mathclose]$, nous avons $A\cap B=\emptyset$ et donc $\overline{ A\cap B }=\emptyset$. Par contre nous avons $\bar A\cap\bar B=\{ 1 \}$.
\end{remark}

\begin{definition}
	La \defe{frontière}{frontière} d'un sous-ensemble $A$ de l'espace vectoriel normé $V$ est l'ensemble des points $a\in V$ tels que
	\begin{equation}
		\begin{aligned}[]
			B(a,r)\cap A&\neq \emptyset,\\
			B(a,r)\cap \complement A&\neq \emptyset,
		\end{aligned}
	\end{equation}
	pour tout rayon $r$. En d'autres termes, toute boule autour de $a$ contient des points de $A$ et des points de $\complement A$. La frontière de $A$ se note $\partial A$\nomenclature[T]{$\partial A$}{La frontière de l'ensemble $A$}.
\end{definition}

\begin{proposition}		\label{PropDescFrpbsmI}
	La frontière d'une partie $A$ d'un espace vectoriel normé $V$ s'exprime sous la forme
	\begin{equation}
		\partial A=\bar A\setminus\Int(A).
	\end{equation}
\end{proposition}

\begin{proof}
	Le fait pour un point $a$ de $V$ d'appartenir à $\bar A$ signifie que toute boule centrée en $a$ intersecte $A$. De la même façon, le fait de ne pas appartenir à $\Int(A)$ signifie que toute boule centrée en $a$ intersecte $\complement A$.
\end{proof}

La description de la frontière donnée par la proposition~\ref{PropDescFrpbsmI} est celle qu'en pratique nous utilisons le plus souvent. Dans certains textes, elle est prise comme définition de la frontière.

\begin{lemma}
	La frontière de $A$ peut également s'exprimer des façons suivantes :
	\begin{equation}
		\partial A= \bar A\cap\complement\Int(A)=\bar A\cap\overline{ \complement A },
	\end{equation}
\end{lemma}

\begin{proof}
	En partant de $\partial A=\bar A\setminus \Int(A)$, la première égalité est une application de la propriété~\ref{ItemLemPropComplementiii} du lemme~\ref{LemPropsComplement}. La seconde égalité est alors la proposition~\ref{PropComleIntBar}.
\end{proof}

\begin{example}
	Dans $\eR$, la frontière d'un intervalle est la paire constituée des points extrêmes. En effet
	\begin{equation}
		\partial\mathopen[ a , b [=\overline{ \mathopen[ a , b [ }\setminus\Int\big( \mathopen[ a , b [ \big)=\mathopen[ a , b \mathclose]\setminus\mathopen] a , b \mathclose[=\{ a,b \}.
	\end{equation}

	Toujours dans $\eR$ nous avons
	\begin{equation}
		\partial\eR=\bar\eR\setminus\Int(\eR)=\eR\setminus\eR=\emptyset,
	\end{equation}
	et
	\begin{equation}
		\partial\eQ=\bar\eQ\setminus\Int(\eQ)=\eR\setminus\emptyset=\eR.
	\end{equation}
\end{example}

%TODO : prouver que la boule fermée est la fermeture de la boule ouverte.

\begin{example}
	Dans $\eR^n$, nous avons
	\begin{equation}
		\partial B(a,r)=\partial\bar B(a,r)=S(a,r).
	\end{equation}

    Cela est un boulot pour la proposition~\ref{PropBoitPtLoin}. Si \( x\in S(a,r)\) alors tout boule autour de \( x\) contient des points à distance strictement plus grande et plus petite que \( d(a,x)\), c'est-à-dire des points dans \( B(a,r)\) et hors de \( B(a,r)\). Cela prouve que les points de \( S(a,r)\) font partie de \( \partial B(a,r)\), c'est-à-dire que \( S(a,r)\subset \partial B(a,r)\); et idem pour \( \bar B(a,r)\).

Pour prouver l'inclusion inverse, soit \( x\in \partial B(a,r)\). Vu que toute boule autour de \( x\) contient des points intérieurs à \( B(a,r)\), pour tout \( \epsilon>0\), \( d(a,x)-\epsilon< r \), c'est-à-dire que \( d(a,x)\leq r\). De la même manière toute boule autour de \( x\) contient des points hors de \( B(a,r)\) signifie que pour tout \( \epsilon\), \( d(a,x)+\epsilon>r\) ou encore que \( d(a,x)\geq r\). Nous avons donc \( d(a,x)=r\).
\end{example}

\begin{remark}
    Il serait toutefois faux de croire que \( \partial A=\partial \bar A\) pour toute partie \( A\) de \( \eR^n\). En effet si \( A=\eR\setminus\{ 0 \}\) nous avons \( \partial A=\{ 0 \}\) et \( \bar A=\eR\), donc \( \partial \bar A=\emptyset\).
\end{remark}

%---------------------------------------------------------------------------------------------------------------------------
\subsection{Point isolé, point d'accumulation}
%---------------------------------------------------------------------------------------------------------------------------

\begin{definition}
	Soit $D$, une partie de $V$.
	\begin{enumerate}
		\item
			Un point $a\in D$ est dit \defe{isolé}{isolé!point dans un espace vectoriel normé} dans $D$ relativement à $V$ s'il existe un $\varepsilon>0$ tel que
			\begin{equation}
				B(a,\varepsilon)\cap D=\{ a \}.
			\end{equation}
		\item
			Un point $a\in V$ est un \defe{point d'accumulation}{accumulation!dans espace vectoriel normé} de $D$ si pour tout $\varepsilon>0$,
			\begin{equation}
				\Big( B(a,\varepsilon)\setminus\{ a \}\Big)\cap D\neq \emptyset.
			\end{equation}
	\end{enumerate}
\end{definition}

\newcommand{\CaptionFigAccumulationIsole}{L'ensemble décrit par l'équation \eqref{Eq2807BouleIso}. Le point $P$ est un point isolé de $D$, tandis que  les points $S$ et $Q$ sont des points d'accumulation.}
\input{auto/pictures_tex/Fig_AccumulationIsole.pstricks}

\begin{example}
	Considérons la partie suivante de $\eR^2$ :
	\begin{equation}	\label{Eq2807BouleIso}
		D=\{ (x,y)\tq x^2+y^2<1\}\cup\{ (1,1) \}.
	\end{equation}
	Comme on peut le voir sur la figure~\ref{LabelFigAccumulationIsole}, le point $P=(1,1)$ est un point isolé de $D$ parce qu'on peut tracer une boule autour de $P$ sans inclure d'autres points de $D$ que $P$ lui-même. Le point $Q=(-1,0)$ est un point d'accumulation de $D$ parce que toute boule autour de $Q$ contient des points de $D$.

    Le point $S$, étant un point intérieur, est un point d'accumulation : toute boule autour de $S$ intersecte $D$.

    Notez cependant que le point $Q$ lui-même n'est pas dans $D$ parce que l'inégalité qui définit $D$ est stricte.
\end{example}

\begin{remark}
    À propos de la position des points d'accumulation et des points isolés.
    \begin{enumerate}
        \item
            Les points intérieurs sont tous des points d'accumulation.
        \item
            Les points isolés ne sont jamais intérieurs.
        \item
            Certains points d'accumulation ne font pas partie de l'ensemble. Par exemple le point $1$ est un point d'accumulation de $E=\mathopen] 0 , 1 \mathclose[$.
        \item
            Les points de la frontière sont soit d'accumulation soit isolés.
    \end{enumerate}
\end{remark}

\begin{example}
	Tous les points de $\eR$ sont des points d'accumulation de $\eQ$ parce que dans toute boule autour d'un réel, on peut trouver un nombre rationnel.
\end{example}

\begin{remark}
	L'ensemble des points d'accumulation d'un ensemble n'est pas exactement son adhérence. En effet, un point isolé dans $A$ est dans l'adhérence de $A$, mais n'est pas un point d'accumulation de $A$.
\end{remark}

%--------------------------------------------------------------------------------------------------------------------------- 
\subsection{Des exemples}
%---------------------------------------------------------------------------------------------------------------------------


\begin{example} \label{ItemExoEVN3i}
    Nous considérons l'ensemble.
    \begin{equation}
A_1 = \{ (x, y ) \in \eR^2 \; | \; x^2 - 5x + 6 < y \leq 2 \}.
    \end{equation}
			Si un point $(x,y)\in\eR^2$ est tel que $x^2-5x+6<y$, alors dans une boule centrée en $(x,y)$ (de rayon $r_1$), l'inégalité reste vraie (parce que la fonction $x^2-5x+6-y$ est une fonction continue). De la même manière, si nous avons $y<2$ en $(x,y)$, alors nous avons encore l'inégalité dans une boule de rayon $r_2$. En prenant $r=\min\{ r_1,r_2 \}$, les deux inégalités restent vraies dans la boule de rayon $r$.

			Donc les points $(x,y)$ tels que $x^2 - 5x + 6 < y < 2$ sont dans l'intérieur de $A_1$.
			
			Pour les mêmes raisons, autour d'un point $(x,y)$ tel que $x^2-5x+6>y$, nous pouvons trouver une boule dans laquelle l'inégalité reste stricte. Ces points ne sont donc pas dans l'adhérence de $A_1$. Un point qui vérifie $x^2-5x+6= y= 2$ est par contre dans l'adhérence parce que dans toute boule, on pourra trouver un $x$ tel que $x^2-5x+6<y$, et un $y$. L'adhérence est donc donnée par les inéquations
			\begin{equation}
				\bar A_1\equiv x^2-5x+6\leq y\leq 2.
			\end{equation}
			
			La frontière est donnée par les points de l'adhérence qui ne sont pas dans l'intérieur de $A_1$. Attention : {\bf ne pas dire} que la frontière est alors donnée simplement en remplaçant les inégalités par des égalités : $\partial A_1\equiv x^2-5x+6= y= 2$. Quel est cet ensemble ?

			Trouver la frontière demande un peu plus de travail. Le point marqué sur la figure \ref{LabelFigAdhIntFr} est sur la frontière parce que toute boule intersecte l'intérieur et l'extérieur. Cela est dû au fait que, sur ce point, nous ayons $x^2-5x+6=y$ en même temps que $y<2$. Donc si on prend une boule assez petite, on conserve $y<2$, mais on obtient des points tels que $x^2-5x+6<y$. 

			En voyant le dessin, la chose à faire pour écrire la frontière est de trouver les deux points d'intersection entre la parabole et la droite horizontale. Ces points sont les points $(x,y)$ qui satisfont au système
			\begin{subequations}
				\begin{numcases}{}
					x^2-5x+6=y\\
					y=2.
				\end{numcases}
			\end{subequations}
			En substituant la seconde équation dans la première, il vient $x^2-5x+6=2$, ce qui nous donne à résoudre le polynôme du second degré $x^2-5x+4=0$. Les solutions sont $x=1$ et $x=4$, et les deux points d'intersection sont les points $P=(1,2)$ et $Q=(4,2)$. Les points de la frontière de $A_1$ sont donc donnés par 
			\begin{equation}
				\begin{aligned}[]
					\partial A_1&=\{ (x,y)\in\eR^2\tqs x^2-5x+6=y\text{ et } 1\leq x\leq 4 \}\\
						&\quad\cup\{ (x,y)\in\eR^2\tqs y=2\text{ et } 1\leq x\leq 4 \}.
				\end{aligned}
			\end{equation}
			
			\newcommand{\CaptionFigAdhIntFr}{En hachuré : l'intérieur; en trait plein : la frontière. L'adhérence est l'union des deux. Exemple\ref{ItemExoEVN3i}.}
			\input{auto/pictures_tex/Fig_AdhIntFr.pstricks}

			Notez que les points de la parabole qui sont sur la frontière ne font pas partie de l'ensemble $A_1$ lui-même, tandis que ceux de la frontière qui sont sur la droite horizontale en font partie sauf \( (4,2)\) et \( (1,2)\).


			L'intérieur de $A_1$ n'étant pas égal à $A_1$, cet ensemble n'est pas ouvert; de la même manière, vu que $\bar A_1\neq A_1$, l'ensemble n'est pas fermé. L'ensemble $A_1$ est par contre borné parce qu'il est contenu par exemple dans la boule de centre $(0,0)$ et de rayon $5$. Les points d'accumulation de \( A_1\) sont les points de sa fermeture.
\end{example}

\begin{example}\label{ItemExoEVN3ii} 
    Nous étudions
    \begin{equation}
    A_2 = \{ (x, y ) \in \eR^2 \; | \; x+ 1 < y < 2x \}.
    \end{equation}

    Pour les mêmes raisons que dans l'exemple \ref{ItemExoEVN3i} l'intérieur est donné par
			\begin{equation}
				\Int(A_2)\equiv x+1<y<2x;
			\end{equation}
			L'adhérence est donnée par
			\begin{equation}
				\overline{ A_2 }\equiv x+1\leq y\leq 2x,
			\end{equation}
			Pour la frontière, les deux droites dont il est question dans la définition de $A_2$ (les droites $y=x+1$ et $y=2x$) se coupent en $x=1$ (refaire soi-même le dessin de la figure \ref{LabelFigAdhIntFrDeux}). Lorsque $x<1$, les conditions $x+1<y$ et $y<2x$ sont incompatibles : aucun point de $A_2$ n'est dans la partie $x<1$ du plan. Lorsque $x>1$, alors les points situés \emph{entre} les deux droites font partie de $A_2$. La frontière est donc donnée par ces deux droites pour $x\geq 1$. 

			Étant donné que $\Int(A_2)=A_2$, cet ensemble est ouvert (et donc pas fermé par la proposition \ref{PropTopologieAx}\ref{ItemPropTopologieAxiv}). Il n'est par contre pas borné parce qu'il contient des point $(x,y)$ avec des $x$ arbitrairement grands.
\end{example}

\newcommand{\CaptionFigAdhIntFrDeux}{Notez que le point d'angle fait partie de la frontière, mais pas de l'ensemble. Exemple \ref{ItemExoEVN3ii}.}
\input{auto/pictures_tex/Fig_AdhIntFrDeux.pstricks}

% This is part of Mes notes de mathématique
% Copyright (c) 2008-2020
%   Laurent Claessens
% See the file fdl-1.3.txt for copying conditions.


%+++++++++++++++++++++++++++++++++++++++++++++++++++++++++++++++++++++++++++++++++++++++++++++++++++++++++++++++++++++++++++
\section{Valeur propre et vecteur propre}
%+++++++++++++++++++++++++++++++++++++++++++++++++++++++++++++++++++++++++++++++++++++++++++++++++++++++++++++++++++++++++++

%---------------------------------------------------------------------------------------------------------------------------
\subsection{Généralités}
%---------------------------------------------------------------------------------------------------------------------------

Nous savons qu'une application \emph{linéaire} $A\colon \eR^3\to \eR^3$ est complètement définie par la donnée de son action sur les trois vecteurs de base, c'est-à-dire par la donnée de
\begin{equation}
	\begin{aligned}[]
		Ae_1,&&Ae_2&&\text{et}&&Ae_3.
	\end{aligned}
\end{equation}
Nous allons former la matrice de $A$ en mettant simplement les vecteurs $Ae_1$, $Ae_2$ et $Ae_3$ en colonne. Donc la matrice
\begin{equation}		\label{EqExempleALin}
	A=\begin{pmatrix}
		3	&	0	&	0	\\
		0	&	1	&	0	\\
		0	&	1	&	0
	\end{pmatrix}
\end{equation}
signifie que l'application linéaire $A$ envoie le vecteur $e_1$ sur $\begin{pmatrix}
	3	\\
	0	\\
	0
\end{pmatrix}$, le vecteur $e_2$ sur $\begin{pmatrix}
	0	\\
	0	\\
	1
\end{pmatrix}$ et le vecteur $e_3$ sur $\begin{pmatrix}
	0	\\
	1	\\
	0
\end{pmatrix}$.
Pour savoir comment $A$ agit sur n'importe quel vecteur, on applique la règle de produit vecteur$\times$matrice :
\begin{equation}
	\begin{pmatrix}
		1	&	2	&	3	\\
		4	&	5	&	6	\\
		7	&	8	&	9
	\end{pmatrix}\begin{pmatrix}
		x	\\
		y	\\
		z
	\end{pmatrix}=
	\begin{pmatrix}
		x+2y+3z	\\
		4x+5y+6z	\\
		7x+8y+9z
	\end{pmatrix}.
\end{equation}

Une chose intéressante est de savoir quelles sont les directions invariantes de la transformation linéaire. Par exemple, on peut lire sur la matrice \eqref{EqExempleALin} que la direction $\begin{pmatrix}
	1	\\
	0	\\
	0
\end{pmatrix}$ est invariante : elle est simplement multipliée par $3$. Dans cette direction, la transformation est juste une dilatation. Afin de savoir si $v$ est un vecteur d'une direction conservée, il faut voir s'il existe un nombre $\lambda$ tel que $Av=\lambda v$, c'est-à-dire voir si $v$ est simplement dilaté.

L'équation $Av=\lambda v$ se récrit $(A-\lambda\mtu)v=0$, c'est-à-dire qu'il faut résoudre l'équation
\begin{equation}
	(A-\lambda\mtu)\begin{pmatrix}
		x	\\
		y	\\
		z
	\end{pmatrix}=
	\begin{pmatrix}
		0	\\
		0	\\
		0
	\end{pmatrix}.
\end{equation}
Nous savons qu'une telle équation ne peut avoir de solutions que si $\det(A-\lambda\mtu)=0$. La première étape est donc de trouver les $\lambda$ qui vérifient cette condition.

%---------------------------------------------------------------------------------------------------------------------------
\subsection{Dans le vif du sujet}
%---------------------------------------------------------------------------------------------------------------------------

\begin{definition}      \label{DefooMMKZooVcskCc}
    Soit un \( \eK\)-espace vectoriel \( E\) et un endomorphisme \( A\colon V\to V\). Un \defe{vecteur propre}{vecteur!propre} de \( A\) est un vecteur \( v \neq 0\) tel que \( Av=\lambda v\) pour un certain \( \lambda\in \eK\). Dans ce cas, \( \lambda\) est la \defe{valeur propre}{valeur!propre} de \( v\).

    L'\defe{espace propre}{espace!propre} de \( A\) pour la valeur \( \lambda\)\footnote{Nous laissons au lecteur le soin de vérifier que c'est bien un sous-espace vectoriel de \( E\).} est l'ensemble des vecteurs propres de \( A\) pour la valeur propre \( \lambda\) et zéro.
\end{definition}

\begin{definition}
    L'ensemble de valeurs propres de l'endomorphisme \( u\) est son \defe{spectre}{spectre d'un endomorphisme} et est noté \( \Spec(u)\).
\end{definition}

\begin{remark}
    Le nombre zéro peut être une valeur propre; c'est le vecteur zéro qui ne peut pas être vecteur propre. La matrice nulle est une matrice diagonalisable.
\end{remark}

\begin{lemma}       \label{LemjcztYH}
    Soit \( u\) un endomorphisme et \( E_{\lambda}(u)\)\nomenclature[A]{\( E_{\lambda}(u)\)}{Espace propre de \( u\)} ses espaces propres. La somme des \( V_{\lambda}\) est directe.
\end{lemma}

\begin{proof}
    Soit \( v_i\in V_{\lambda_i}\) un choix de vecteurs propres de \( u\). Si la somme n'est pas directe, nous pouvons considérer une combinaison linéaire des \( v_i\) qui soit nulle :
    \begin{equation}
        v_1+\cdots+v_p=0.
    \end{equation}
    Appliquons \( (A-\lambda_1\mtu)\) à cette égalité :
    \begin{equation}
        (\lambda_2-\lambda_1)v_1+\cdots+(\lambda_p-\lambda_1)v_p=0.
    \end{equation}
    En appliquant encore successivement les opérateurs \( (A-\lambda_i\mtu)\) nous réduisons le nombre de termes jusqu'à obtenir \( v_p=0\).
\end{proof}

\begin{proposition}[\cite{RombaldiO}]   \label{PropTVKbxU}
    Soit \( E\), un espace vectoriel sur un corps infini et \( (F_k)_{k=1,\ldots, r}\), des sous-espaces vectoriels propres\footnote{Définition~\ref{DefooMMKZooVcskCc}.} de \( E\) tels que \( \bigcup_{i=1}^rF_i=E\). Alors \( E=F_k\) pour un certain \( k\).

    Autrement dit, l'union finie de sous-espaces propres ne peut être égal à l'espace complet.
\end{proposition}

%+++++++++++++++++++++++++++++++++++++++++++++++++++++++++++++++++++++++++++++++++++++++++++++++++++++++++++++++++++++++++++
\section{Polynômes d'endomorphismes}
%+++++++++++++++++++++++++++++++++++++++++++++++++++++++++++++++++++++++++++++++++++++++++++++++++++++++++++++++++++++++++++
\label{SECooUEQVooLBrRiE}

Soit \( A\) un anneau commutatif et \( \eK\), un corps commutatif. L'injection canonique \( A\to A[X]\) se prolonge en une injection
\begin{equation}
   \eM(A)\to\eM\big( A[X] \big).
\end{equation}

%---------------------------------------------------------------------------------------------------------------------------
\subsection{Polynômes d'endomorphismes}
%---------------------------------------------------------------------------------------------------------------------------

Soit \( u\in\End(E)\) où \( E\) est un \( \eK\)-espace vectoriel. Nous considérons l'application
\begin{equation}    \label{EqOVKooeMJuv}
    \begin{aligned}
        \varphi_u\colon \eK[X]&\to \End(E) \\
        P&\mapsto P(u).
    \end{aligned}
\end{equation}
L'image de \( \varphi_u\) est un sous-espace vectoriel. En effet si \( A=\varphi_u(P)\) et \( B=\varphi_u(Q)\), alors \( A+B=\varphi_u(P+Q)\) et \( \lambda A=(\lambda P)(u)\). En particulier c'est un espace fermé.

Soit \( u\) un endomorphisme d'un \( \eK\)-espace vectoriel \( E\) et \( P\), un polynôme. Nous disons que \( P\) est un polynôme \defe{annulateur}{polynôme!annulateur} de \( u\) si \( P(u)=0\) en tant que endomorphisme de \( E\).

\begin{lemma}       \label{LemQWvhYb}
    Si \( P\) et \( Q\) sont des polynômes dans \( \eK[X]\) et si \( u\) est un endomorphisme d'un \( \eK\)-espace vectoriel \( E\), nous avons
    \begin{equation}
        (PQ)(u)=P(u)\circ Q(u).
    \end{equation}
\end{lemma}

\begin{proof}
    Si \( P=\sum_i a_iX^i\) et \( Q=\sum_j b_jX^j\), alors le coefficient de \( X^k\) dans \( PQ\) est
    \begin{equation}        \label{EqCoefGPyVcv}
        \sum_la_lb_{k-l}.
    \end{equation}
    Par conséquent \( (PQ)(u)\) contient \( \sum_la_lb_{k-l}u^k\). Par ailleurs \( P(u)\circ Q(u)\) est donné par
    \begin{equation}
        \sum_ia_iu^i\left( \sum_jb_ju^j \right)(x)=\sum_{ij}a_ib_ju^{i+j}(x).
    \end{equation}
    Le coefficient du terme en \( u^k\) est bien le même que celui donné par \eqref{EqCoefGPyVcv}.
\end{proof}

\begin{theorem}[Décomposition des noyaux ou lemme des noyaux]       \label{ThoDecompNoyayzzMWod}
    Soit \( u\) un endomorphisme du \( \eK\)-espace vectoriel \( E\). Soit \( P\in\eK[X]\) un polynôme tel que \( P(u)=0\). Nous supposons que \( P\) s'écrive comme le produit \( P=P_1\ldots P_n\) de polynômes deux à deux étrangers\footnote{Définition~\ref{DefDSFooZVbNAX}.}. Alors
    \begin{equation}
        E=\ker P_1(u)\oplus\ldots\oplus\ker P_n(u).
    \end{equation}
    De plus les projecteurs associés à cette décomposition sont des polynômes en \( u\).
\end{theorem}
\index{lemme!des noyaux}
Ce résultat est utilisé pour prouver que toute représentation est décomposable en représentations irréductibles, proposition~\ref{PropHeyoAN} ainsi que pour le théorème~\ref{ThoDigLEQEXR} qui dit que si le polynôme minimal d'un endomorphisme est scindé à racine simple alors il est diagonalisable.

\begin{proof}
    Nous posons
    \begin{equation}
        Q_i=\prod_{j\neq i}P_i.
    \end{equation}
    Par le lemme~\ref{LemuALZHn} ces polynômes sont étrangers entre eux et le théorème de Bézout (théorème~\ref{ThoBezoutOuGmLB}) donne l'existence de polynômes \( R_i\) tels que
    \begin{equation}
        R_1Q_1+\cdots+R_nQ_n=1.
    \end{equation}
    Si nous appliquons cette égalité à \( u\) et ensuite à \( x\in E\) nous trouvons
    \begin{equation}        \label{EqqVcpUy}
        \sum_{i=1}^n(R_iQ_i)(u)(x)=x,
    \end{equation}
    et en particulier si nous posons \( E_i=\Image\big(P_iQ_i(u)\big)\) nous avons
    \begin{equation}
        E=\sum_{i=1}^nE_i.
    \end{equation}
    Cette dernière somme n'est éventuellement pas une somme directe. Si \( i\neq j\), alors \( Q_iQ_j\) est multiple de \( P\) et nous avons, en utilisant le lemme~\ref{LemQWvhYb},
    \begin{equation}
        (R_iQ_i)(u)\circ (R_jQ_j)(u)=\big( R_iQ_iR_jQ_j \big)(u)=S_{ij}(u)\circ P(u)=0
    \end{equation}
    où \( S_{ij}\) est un polynôme.

    Nous pouvons voir \( E\) comme un \( \eK\)-module et appliquer le théorème~\ref{ThoProjModpAlsUR}. Les opérateurs \( R_iQ_i(u)\) ont l'identité comme somme et sont orthogonaux, et nous avons donc la décomposition en somme directe :
    \begin{equation}
        E=\bigoplus_{i=1}^nR_iQ_i(u)E.
    \end{equation}

    Afin de terminer la preuve, nous devons montrer que \( R_iQ_i(u)E=\ker P_i(u)\). D'abord nous avons
    \begin{equation}
        P_iR_iQ_i(u)=(R_iP)(u)=R_i(u)\circ P(u)=0,
    \end{equation}
    par conséquent \( \Image(R_iQ_i(u))\subset \ker P_i(u)\). Pour obtenir l'inclusion inverse, nous reprenons l'équation \eqref{EqqVcpUy} avec \( x\in\ker P_i(u)\). Elle se réduit à
    \begin{equation}
        (R_iQ_i)(u)x=x.
    \end{equation}
    Par conséquent \( x\in\Image\big( R_iQ_i(u) \big)\).
\end{proof}

\begin{corollary}   \label{CorKiSCkC}
    Soit \( E\), un \( \eK\)-espace vectoriel de dimension finie et \( f\), un endomorphisme semi-simple dont la décomposition du polynôme minimal \( \mu_f\) en facteurs irréductibles sur \( \eK[X]\) est \( \mu_f=M_1^{\alpha_1}\cdots M_r^{\alpha_r}\). Si \( F\) est un sous-espace stable par \( f\), alors
    \begin{equation}
        F=\bigoplus_{i=1}^r\ker M_i^{\alpha_i}(f)\cap F
    \end{equation}
\end{corollary}

\begin{proof}
    Nous posons \( E_i=\ker M_i^{\alpha_i}(f)\) et \( F_i=E_i\cap F\). Les polynômes \( M_i^{\alpha_i}\) sont deux à deux étrangers et \( \mu_f(f)=0\), donc le lemme des noyaux (\ref{ThoDecompNoyayzzMWod}) s'applique et
    \begin{equation}
        E=E_1\oplus\ldots\oplus E_r.
    \end{equation}
    Nous pouvons décomposer \( x\in F\) en termes de cette somme :
    \begin{equation}     \label{EqbBbrdi}
        x=x_1+\cdots +x_r
    \end{equation}
    avec \( x_i\in E_i\). Toujours selon le lemme des noyaux, les projections sur les espaces \( E_i\) sont des polynômes en \( f\). Par conséquent \( F\) est stable sous toutes ces projections \( \pr_i\colon E\to E_i\), et en appliquant \( \pr_i\) à \eqref{EqbBbrdi}, \( \pr_i(x)=x_i\). Vu que \( x\in F\), le membre de gauche est encore dans \( F\) et \( x_i\in E_i\cap F\). Nous avons donc
    \begin{equation}
        F\subset\bigoplus_{i=1}^rF_i.
    \end{equation}
    L'inclusion inverse est immédiate parce que \( F_i\subset F\) pour chaque \( i\).
\end{proof}

\begin{lemma}   \label{LemVISooHxMdbr}
    Si \( x\) est un vecteur propre de valeur propre \( \lambda\) pour l'endomorphisme \( u\) et si \( P\) est un polynôme, alors \( x\) est vecteur propre de \( u\) pour la valeur propre \( P(\lambda)\).
\end{lemma}

\begin{proof}
    C'est un simple calcul de \( P(u)x\) en ayant noté \( P(X)=\sum_{k=0}^nc_kX^n\) :
    \begin{equation}
        P(u)x=\sum_{k=0}^nc_ku^k(x)=\sum_{k=0}^nc_k\lambda^ku=P(\lambda)x.
    \end{equation}
\end{proof}

%---------------------------------------------------------------------------------------------------------------------------
\subsection{Polynôme minimal et minimal ponctuel}
%---------------------------------------------------------------------------------------------------------------------------

\begin{lemmaDef}        \label{DefooOHUXooNkPWaB}
    Soit un endomorphisme \( f\colon E\to E\) d'un \( \eK\)-espace vectoriel de dimension finie. Il existe un unique polynôme annulateur normalisé de degré minimum.

    Il est nommé le \defe{polynôme minimal}{polynôme!minimal} de \( f\) et il est noté \( \mu_f\) ou simplement \( \mu\) lorsque la dépendance en \( f\) est claire.
\end{lemmaDef}

\begin{proof}
    Pour l'unicité, soient \( P\) et \( Q\) deux polynômes annulateur de \( f\) de même degré \( N\) et ayant tous deux \( 1\) comme coefficient de \( x^N\). Alors \( P-Q\) est de degré \( N-1\) tout en étant encore annulateur.

    Pour l'existence, les endomorphismes \( \id\), \( f\), \( f^2\), \ldots ne peuvent pas être tous linéairement indépendants parce que la dimension de \( \End(E)\) est finie. Il existe donc un nombre \( N\) et des coefficients \( a_k\) tels que \( \sum_{k=0}^Na_kf^k=0\). Le polynôme \( P(X)=\sum_{k=0}^Na_kX^k\) est donc annulateur de \( f\).

    Une autre façon de le dire est que l'application linéaire \( \varphi\colon \eK[X]\to \End(E)\) donnée par \( \varphi(P)=P(f)\) est un endomorphisme d'un espace vectoriel de dimension infinie vers un espace vectoriel de dimension finie. Il ne peut donc pas être injectif et possède donc un noyau non réduit à zéro.
\end{proof}

\begin{remark}
    La preuve donnée ci-dessus montre que \( \deg(\mu)\leq \dim(E)^2\). Comme conséquence du théorème de Caley-Hamilton~\ref{ThoCalYWLbJQ} nous verrons qu'en réalité le degré du polynôme minimal est majoré par la dimension de l'espace.
\end{remark}

\begin{example}[Pas en dimension infinie]
    L'endomorphisme de dérivation
\end{example}


Dans la suite, l'endomorphisme \( f\) du \( \eK\)-espace vectoriel \( E\) de dimension \( n\) est fixé. Pour \( x\in E\) nous notons
\begin{equation}            \label{EqooOAYDooEpZELo}
    E_x=\{ P(f)x\tq P\in \eK[X] \}.
\end{equation}
Nous considérons le morphisme d'algèbres
\begin{equation}
    \begin{aligned}
        \varphi\colon \eK[X]&\to \End(E) \\
        P&\mapsto P(f)
    \end{aligned}
\end{equation}
et si \( x\in E\) est donné nous considérons le morphisme de \( \eK\)-espaces vectoriels
\begin{equation}
    \begin{aligned}
        \varphi_x\colon \eK[X]&\to E \\
        P&\mapsto P(f)x.
    \end{aligned}
\end{equation}
Les noyaux de ces applications sont des idéaux, entre autres par le lemme~\ref{LemQWvhYb}. Ils ont donc un unique générateur unitaire (chacun) par le théorème~\ref{ThoCCHkoU}. En termes de vocabulaire, l'ensemble
\begin{equation}
    \ker(\phi)=\{  Q\in\eK[X]\tq Q(f)=0  \}
\end{equation}
est l'\defe{idéal annulateur}{polynôme!annulateur} de \( f\) et un polynôme \( Q\) tel que \( Q(f)=0\) est une polynôme annulateur de \( f\).

\begin{definition}      \label{DEFooUICRooBGYhqQ}
    Le générateur unitaire de \( \ker(\varphi_x)\) est le \defe{polynôme minimal ponctuel}{polynôme!minimal!ponctuel} de \( f\) en \( x\). Il sera noté \( \mu_{f,x}\) ou \( \mu_x\) lorsque la dépendance en \( f\) est claire dans le contexte.
\end{definition}
Nous notons \( \mu\) le générateur unitaire du noyau de \( \varphi\) et \( \mu_x\) celui de \( \varphi_x\). Vu que \( \mu\in\ker(\varphi_x)\) pour tout \( x\) nous avons \( \mu_x\divides \mu\) pour tout \( x\).

\begin{example}[Pas en dimension infinie]       \label{ExooDTUJooIMqSKn}
    En dimension infinie, il n'y a pas toujours de polynôme annulateur. Si \( E\) est un espace vectoriel de dimension infine ayant une base dénombrable \( \{ e_i \}_{i\in \eN}\) alors l'opérateur donné par \( f(e_i)=e_{i+1}\) n'a pas de polynôme annulateur. Même pas ponctuel en quel que point que ce soir.

    De même l'opérateur donné par \( g(e_1)=0\) et \( g(e_i)=e_{i-1}\) si \( i\neq 1\) n'a pas de polynôme annulateur, mais il a un polynôme annulateur ponctuel évident en \( x=e_1\). L'exemple~\ref{ExooLRHCooMYLQTU} donnera un habillage à peine subtil à cet exemple.
\end{example}

\begin{proposition}     \label{PropAnnncEcCxj}
    Si \( P\) est un polynôme tel que \( P(f)=0\), alors le polynôme minimal \( \mu_f\) divise \( P\). Autrement dit, le polynôme minimal engendre l'idéal des polynômes annulateurs.
\end{proposition}

\begin{proof}
    L'ensemble \( \ker(\varphi)=\{ Q\in \eK[X]\tq Q(u)=0 \} \) est un idéal par le lemme \ref{LemQWvhYb}. Le polynôme minimal de \( u\) est un élément de degré plus bas dans \( I\) et par conséquent \( I=(\mu_u)\) par le théorème~\ref{ThoCCHkoU}. Nous concluons que \( \mu_u\) divise tous les éléments de \( I\).
\end{proof}

La proposition suivante permet de caractériser le polynôme minimal.
\begin{proposition}[\cite{ooEPEFooQiPESf}]      \label{PROPooVUJPooMzxzjE}
    Soit une application linéaire \( f\) sur un \( \eK\)-espace vectoriel. Il existe un unique polynôme unitaire\quext{À mon avis, «unitaire» manque dans \cite{ooEPEFooQiPESf}.} \( P\in \eK[X]\) tel que
    \begin{enumerate}
        \item
            \( P(f)=0\);
        \item
            l'application
            \begin{equation}        \label{EQooIBMDooVTaEhf}
                \begin{aligned}
                    \varphi\colon \frac{ \eK[X] }{ (P) }&\to \End(E) \\
                    \bar Q&\mapsto Q(f)
                \end{aligned}
            \end{equation}
            est injective.
    \end{enumerate}
\end{proposition}

\begin{proof}
    En ce qui concerne l'existence, il existe le polynôme minimal de \( f\) qui satisfait les conditions. Pour l'unicité nous travaillons maintenant.

    Supposons que l'application \eqref{EQooIBMDooVTaEhf} soit injective. Alors pour tout \( Q\in \eK[X]\) tel que \( Q(f)=0\) nous avons \( \bar Q=0\), c'est-à-dire \( Q=PR\) pour un certain \( R\in \eK[X]\). Autrement dit : \( P\) est un générateur unitaire de l'idéal annulateur de \( f\). Le théorème~\ref{ThoCCHkoU}\ref{ITEMooASHKooZqkiCH} nous dit alors que \( P=\mu\) parce que \( \mu\) est également générateur unitaire.
\end{proof}

\begin{lemma}[\cite{ooRJDSooXpVtMD}]\label{LemSYsJJj}
    Soit \( f\colon E\to E\) un endomorphisme de l'espace vectoriel \( E\). Il existe un élément \( x\in E\) tel que \( \mu_{f,x}=\mu_f\).
\end{lemma}

\begin{proof}
    Soit une décomposition en irréductibles du polynôme minimal \( \mu=P_1^{\alpha_1}\ldots P_r^{\alpha_r}\). Nous notons \( E_i=\ker\big( P_i^{\alpha_i}(f) \big)\). Les polynômes \( P_i\) sont étrangers deux à deux (un diviseur commun aurait a fortiori été un diviseur et aurait contredit l'irréductibilité). Le lemme des noyaux~\ref{ThoDecompNoyayzzMWod} nous donne la somme directe
    \begin{equation}
        E=\bigoplus_{i=1}^r\ker\big( P_i^{\alpha_i}(f) \big).
    \end{equation}
    Si \( x_i\in E_i\) alors \( \mu_{x_i}\) est une puissance de \( P_i\). En effet \( \mu_{x_i}\divides \mu\) et est donc un produit des puissances des \( P_j\). Or si \( (QP_j)(f)x_i=0\) alors \( (P_jQ)(f)x_i=0\), ce qui donne \( Q(f)x_i\in E_j\cap E_i=\{ 0 \}\). Donc \( \mu_{x_i}\) n'est pas de la forme \( QP_j\) pour \( j\neq i\). Nous en déduisons que \( \mu_{x_i}\) est une puissance de \( P_i\) dès que \( x_i\in E_i\). Nous choisissons \( x_i\in E_i\) tel que \( \mu_{x_i}=P_i^{\alpha_i}\).

    Nous posons enfin \( a=x_1+\cdots +x_r\); par définition du polynôme annulateur \( \mu_a\), nous avons
    \begin{equation}        \label{EqooVIGGooSfuvwB}
        0=\mu_a(f)a=\mu_a(f)x_1+\cdots +\mu_a(f)x_r.
    \end{equation}
    Mais \( m_a(f)x_j\in E_i\), et la somme des \( E_j\) est directe, donc l'annulation de la somme \eqref{EqooVIGGooSfuvwB} implique l'annulation de chacun des termes : \( \mu_a(f)x_i=0\) pour tout \( i\). Cela prouve que \( \mu_{x_i}\divides \mu_a\). Mais comme les \( \mu_{x_i}\) sont premiers deux à deux (parce que ce sont les \( P_i^{\alpha_i}\)), nous avons que le produit divise encore \( \mu_a\) :
    \begin{equation}
        \prod_{i=1}^r\mu_{x_i}\divides \mu_a,
    \end{equation}
    c'est-à-dire \( \mu\divides \mu_a\). Comme nous avons aussi \( \mu_a\divides \mu\), nous déduisons \( \mu_a=\mu\).
\end{proof}

\begin{definition}[Matrices, endomorphismes et vecteurs cycliques]      \label{DEFooFEIFooNSGhQE}
    Une matrice est \defe{cyclique}{cyclique!matrice}\index{matrice!cyclique} si elle est semblable à une matrice compagnon. Un endomorphisme \( f\colon E\to E\) est \defe{cyclique}{cyclique!endomorphisme}\index{endomorphisme!cyclique} s'il existe un vecteur \( x\in E\) tel que \( \{ f^k(x) \}_{k=0,\ldots, n-1} \) est une base de \( E\). Un vecteur ayant cette propriété est un \defe{vecteur cyclique}{vecteur!cyclique} pour \( f\).
\end{definition}

\begin{lemma}   \label{LemAGZNNa}
    Soit \( E\) un espace vectoriel de dimension finie et un endomorphisme cyclique\footnote{Voir la définition~\ref{DEFooFEIFooNSGhQE}.} \( f\) de \( E\). Soit un vecteur cyclique \( v\) de \( f\), alors le polynôme minimal de \( f\) est égal au polynôme minimal de \( f\) au point \( v\) : \( \mu_{f}=\mu_{f,v}\).
\end{lemma}

\begin{proof}
    Montrons que \( \mu_{f,v}\) est un polynôme annulateur de \( f\), ce qui prouvera que \( \mu_f\) divise \( \mu_{f,v}\) par la proposition~\ref{PropAnnncEcCxj}. Étant donné que \( v\) est cyclique, tout élément de \( E\) s'écrit sous la forme \( x=Q(f)v\). Prenons un polynôme \( P\) annulateur de \( f\) en \( v\) : \( P(f)v=0\). Nous montrons que \( P\) est alors un polynôme annulateur de \( f\). En effet, nous avons
    \begin{equation}
        P(f)x=\big( P(f)\circ Q(f) \big)v=\big( Q(f)\circ P(f) \big)v=0
    \end{equation}
    où nous avons utilisé le lemme~\ref{LemQWvhYb}.
\end{proof}

\begin{lemma}[\cite{ooRJDSooXpVtMD}]
    Soit \( a\in E\) tel que \( \mu_a=\mu\). Alors \( E_a\) est un sous-espace stable pour \( f\) pour lequel il existe un supplémentaire stable.
\end{lemma}

\begin{proof}
    Soit \( l=\deg(\mu)=\deg(\mu_a)\). L'espace \( E_a\) étant engendré par les \( f^k(a)\) nous savons que \( e_1=a\), \( e_2=f(a)\),\ldots, \( e_l=k^{l-1}(a)\) forment une base de \( E_a\). Nous pouvons la compléter en une base \( \{ e_1,\ldots, e_n \}\) de \( E\). Et nous posons\footnote{ici, comme presque partout, \( e^*_{l}\) est le dual de \( e_l\), c'est-à-dire l'application linéaire sur \( E\) donnée par \( e^*_l(e_i)=\delta_{li}\). }
    \begin{subequations}
        \begin{align}
            G&=\{ x\in E\tq e^*_l\big( f^k(x) \big)=0\,\forall k\geq 0 \}\\
            &=\bigcap_{k\geq 0}\ker\{ e^*_l\circ f^k \}\\
            &=\bigcap_{k=0}^{l-1}\ker(  e^*_l\circ f^k ).
        \end{align}
    \end{subequations}
    La dernière égalité est due au fait que \( l\) soit le degré de \( \mu\). Du coup \( f^l\) est une combinaison linéaire des \( f^i\) avec \( i\leq l-1\).

    Nous avons \( f(G)\subset G\) et de plus \( E_a\cap G=\{ 0 \}\) parce qu'un élément de \( E_a\) est une combinaison linéaire d'éléments de la forme \( f^j(a)\) (\( j\leq l\)). Après application de \( f^{l-j}\), ces éléments obtiennent une composante \( f^l(a)=e_l\). De plus \( G\) est un sous-espace vectoriel du fait que \( e^*_l\circ f^i\) est une application linéaire.

    Montrons enfin que \( \dim(G)=n-l\). Pour cela nous remarquons que \( G\) est une intersection d'hyperplans, et nous montrons que les équations définissant ces hyperplans sont linéairement indépendantes. Soit donc
    \begin{equation}        \label{EqooOHESooRtBUfc}
        \sum_{j=0}^{l-1}\lambda_j\big( e^*_l\circ f^j \big)=0
    \end{equation}
    et montrons que \( \lambda_j=0\) pour tout $j$ est l'unique solution. Soit \( x\in E\) et appliquons l'opération \eqref{EqooOHESooRtBUfc} au vecteur \( f^i(x)\); le résultat est zéro :
    \begin{equation}
        0=\sum_{j=0}^{l-1}\lambda_j(e^*_l\circ f^i\circ f^j)=(e^*_l\circ f^i)P(u)
    \end{equation}
    où nous avons posé \( P(X)=\sum_{j=0}^{l-1}\lambda_jX^j\). Appliquons cela à \( a\) : pour tout \( i\) nous avons
    \begin{equation}
        (e^*_l\circ f^i)\big( P(f)a \big)=0.
    \end{equation}
    Mais par définition de \( E_a\), l'élément \(P(f)a \) est dans \( E_a\). Nous en déduisons que
    \begin{equation}
        P(f)a\in G\cap E_a=\{ 0 \},
    \end{equation}
    c'est-à-dire que \( P\) est un polynôme annulateur de \( a\). Mais \( P\) est de degré \( l-1\) alors que le polynôme minimal de \( a\) est de degré \( l\). Par conséquent \( P=0\) et \( \lambda_j=0\) pour tout \( j\).
\end{proof}

\begin{definition}  \label{DEFooBOHVooSOopJN}
    Un endomorphisme d'un espace vectoriel est \defe{semi-simple}{semi-simple!endomorphisme} si tout sous-espace stable par \( u\) possède un supplémentaire stable.
\end{definition}

\begin{lemma}   \label{LemrFINYT}
    Si le polynôme minimal d'un endomorphisme est irréductible, alors il est semi-simple\footnote{Définition~\ref{DEFooBOHVooSOopJN}.}.
\end{lemma}

\begin{proof}
    Soit \( f\), un endomorphisme dont le polynôme minimal est irréductible et \( F\), un sous-espace stable par \( f\). Nous devons en trouver un supplémentaire stable. Si \( F=E\), il n'y a pas de problèmes. Sinon nous considérons \( u_1\in E\setminus F\) et
    \begin{equation}
        E_{u_1}=\{ P(f)u_1\tq P\in \eK[X] \},
    \end{equation}
    qui est un espace stable par \( f\).

    Montrons que \( E_{u_1}\cap F=\{ 0 \}\). Pour cela nous regardons l'idéal
    \begin{equation}
        I_{u_1}=\{ P\in \eK[X]\tq P(f)u_1=0 \}.
    \end{equation}
    Cela est un idéal non réduit à \( \{ 0 \}\) parce que le polynôme minimal de \( f\) par exemple est dans \( I_{u_1}\). Soit \( P_{u_1}\) un générateur unitaire de \( I_{u_1}\). Étant donné que \( \mu_f\in I_{u_1}\), nous avons que \( P_{u_1}\) divise \( \mu_f\) et donc \( P_{u_1}=\mu_f\) parce que \( \mu_f\) est irréductible par hypothèse.

    Soit \( y\in E_{u_1}\cap F\). Par définition il existe \( P\in\eK[X]\) tel que \( y=P(f)u_1\) et si \( y\neq 0\), ce la signifie que \( P\notin I_{u_1}\), c'est-à-dire que \( P_{u_1} \) ne divise pas \( P\). Étant donné que \( P_{u_1}\) est irréductible cela implique que \( P_{u_1}\) et \( P\) sont premiers entre eux (ils n'ont pas d'autre \( \pgcd\) que \( 1\)).

    Nous utilisons maintenant Bézout (théorème~\ref{ThoBezoutOuGmLB}) qui nous donne \( A,B\in \eK[X]\) tels que
    \begin{equation}
        AP+BP_{u_1}=1.
    \end{equation}
    Nous appliquons cette égalité à \( f\) et puis à \( u_1\):
    \begin{equation}
        u_1=A(f)\circ \underbrace{P(f)u_1}_{=y}+B(f)\circ \underbrace{P_{u_1}(u_1)}_{=0}=A(f)y.
    \end{equation}
    Mais \( y\in F\), donc \( A(f)y\in F\). Nous aurions donc \( u_1\in F\), ce qui est impossible par choix. Nous avons maintenant que l'espace \( E_{u_1}\oplus F\) est stable sous \( f\). Si cet espace est \( E\) alors nous arrêtons. Sinon nous reprenons le raisonnement avec \( E_{u_1}\oplus F\) en guise de \( F\) et en prenant \( u_2\in E\setminus(E_{u_1}\oplus F)\). Étant donné que \( E\) est de dimension finie, ce procédé s'arrête à un certain moment et nous aurons
    \begin{equation}
        E=F\oplus E_{u_1}\oplus\ldots\oplus E_{u_k}
    \end{equation}
    où chacun des \( E_{u_i}\) sont stables.
\end{proof}

\begin{theorem} \label{ThoFgsxCE}
    Un endomorphisme est semi-simple si et seulement si son polynôme minimal est produit de polynômes irréductibles distincts deux à deux.
\end{theorem}
\index{anneau!principal}

\begin{proof}

    Supposons que \( f\) soit semi-simple et que son polynôme minimal soit donné par \( \mu_f=M_1^{\alpha_1}\ldots M_r^{\alpha_r}\) où les \( M_i\) sont des polynômes irréductibles deux à deux distincts. Nous devons montrer que \( \alpha_i=1\) pour tout \( i\). Soit \( i\) tel que \( \alpha_i\geq 1\) et \( N\in \eK[X]\) tel que \( \mu_f=M^2N\) où l'on a noté \( M=M_i\). Nous étudions l'espace
    \begin{equation}
        F=\ker M(f)
    \end{equation}
    qui est stable par \( f\), et qui possède donc un supplémentaire \( S\) également stable par \( f\). Nous allons montrer que \( MN\) est un polynôme annulateur de \( f\).

    D'abord nous prenons \( x\in S\). Étant donné que \( F\) est le noyau de \( M(f)\),
    \begin{equation}
        M(f)\big( MN(f)x \big)=\mu_f(f)x=0,
    \end{equation}
    ce qui signifie que \( MN(f)x\in F\). Mais vu que \( S\) est stable par \( f\) nous avons aussi que \( MN(f)x\in S\). Finalement \( MN(f)x\in F\cap S=\{ 0 \}\). Autrement dit, \( MN(f)\) s'annule sur \( S\).

    Prenons maintenant \( y\in F\). Nous avons
    \begin{equation}
        MN(f)=N(f)\big( M(f)y \big)=0
    \end{equation}
    parce que \( y\in F=\ker M(f)\).

    Nous avons prouvé que \( MN(f)\) s'annule partout et donc que \( MN(f)\) est un polynôme annulateur de \( f\), ce qui contredit la minimalité de \( \mu_f=M^2N\).

    Nous passons au sens inverse. Soit \( m_f=M_1\ldots M_r\) une décomposition du polynôme minimal de l'endomorphisme \( f\) en irréductibles distincts deux à deux. Soit \( F\) un sous-espace vectoriel stable par \( f\). Nous notons
    \begin{equation}
        E_i=\ker(M_i(f))
    \end{equation}
    et \( f_i=f|_{E_i}\). Par le lemme~\ref{CorKiSCkC} nous avons
    \begin{equation}
        F=\bigoplus_{i=1}^r(F\cap E_i).
    \end{equation}
    Les espaces \( E_i\) sont stables par \( f\) et étant donné que \( M_i\) est irréductible, il est le polynôme minimal de \( f_i\). En effet, \( M_i\) est annulateur de \( f_i\), ce qui montre que le minimal de \( f_i\) divise \( M_i\). Mais \( M_i\) étant irréductible, \( M_i\) est le polynôme minimal. Étant donné que \( \mu_{f_i}=M_i\), l'endomorphisme \( f_i\) est semi-simple par le lemme~\ref{LemrFINYT}.

    L'espace \( F\cap E_i\) étant stable par l'endomorphisme semi-simple \( f_i\), il possède un supplémentaire stable que nous notons \( S_i\)~:
    \begin{equation}
        E_i=S_i\oplus(F\cap E_i).
    \end{equation}
    Étant donné que sur chaque \( S_i\) nous avons \( f|_{S_i}=f_i\), l'espace \( S=S_1\oplus\ldots\oplus S_r\) est stable par \( f\). Du coup nous avons
    \begin{subequations}
        \begin{align}
            E&=E_1\oplus\ldots\oplus E_r\\
            &=\big( S_1\oplus(F\cap E_1) \big)\oplus\ldots\oplus\big( S_r\oplus(F\cap E_r) \big)\\
            &=\big( \bigoplus_{i=1}^rS_i \big)\oplus\big( \bigoplus_{i=1}^rF\cap E_i \big)\\
            &=S\oplus F,
        \end{align}
    \end{subequations}
    ce qui montre que \( F\) a bien un supplémentaire stable par \( f\) et donc que \( f\) est semi-simple.
\end{proof}

\begin{example}[L'espace engendré par \( \mtu\), \( A\), \( A^2\),\ldots]
    Soit \( A\) une matrice, et
    \begin{equation}
        V=\Span\{A^k\tq k\in \eN \}.
    \end{equation}
    Nous montrons que \( \dim(V)\) est le degré du polynôme minimal de \( A\).

    D'abord l'idéal annulateur de \( A\) est engendré par le polynôme minimal\footnote{Proposition~\ref{PropAnnncEcCxj}.} que nous notons
        $\mu=\sum_{k=0}^pa_kX^k$.
    La partie \( \{ \mtu,\ldots, A^{p-1} \}\) est libre parce qu'une combinaison linéaire nulle de cela serait un polynôme annulateur en \( A\) de degré plus petit que \( p\). Donc \( \dim(V)\geq p\).

    La partie \( \{ \mtu,A,\ldots, A^p \}\) est liée à cause du polynôme minimal. Isoler \( A^p\) dans \( \mu(A)=0\) donne un polynôme \( f\) de degré \( p-1\) tel que \( A^p=f(A)\).

    Nous allons montrer à présent que la famille \( \{ \mtu,A,\ldots, A^{p-1} \}\) est génératrice (alors \( \dim(V)\leq p\)). Soit un entier \( q\geq p\)et de division euclidienne\footnote{Théorème~\ref{ThoDivisEuclide}.} \( np+r=q\) avec \( r<p\). Nous avons \( A^q=A^{np}A^r\). D'une part
    \begin{equation}
        A^{np}=(A^p)^n=f(A)^n
    \end{equation}
    est de degré \( n(p-1)\). Par conséquent
    \begin{equation}
        A^q=f(A)^nA^r
    \end{equation}
    qui est de degré \( n(p-1)+r=q-n\). Autrement dit il existe un polynôme \( g_1\) de degré \( q-n\) tel que \( A^q=g_1(A)\). Si \( q-n>p-1\) alors nous pouvons recommencer et obtenir un polynôme \( g_2\) de degré strictement inférieur à celui de \( g_1\) tel que \( A^q=g_2(A)\). Au bout du compte, il existe un polynôme \( g\) de degré au maximum \( p-1\) tel que \( A^q=g(A)\). Cela prouve que la partie \( \{ \mtu,A,\ldots, A^{p-1} \}\) est génératrice de \( V\).

    La dimension de \( V\) est donc \( p\), le degré du polynôme minimal.
\end{example}

\begin{proposition}     \label{PropooCFZDooROVlaA}
    Soit \( f\) un endomorphisme d'un espace vectoriel de dimension finie. Nous avons l'isomorphisme d'espace vectoriel
    \begin{equation}
        \eK[f]\simeq\frac{ \eK[X] }{ (\mu_f) }
    \end{equation}
    La dimension en est \( \deg(\mu_f)\).
\end{proposition}

\begin{proof}
    Notons avant de commencer que \( (\mu)\) est l'idéal engendré par \( \mu\). Les classes dont il est question dans le quotient \( \eK[X]/(\mu)\) sont
    \begin{equation}
        \bar P=\{ P+S\mu \}_{S\in \eK[X]}.
    \end{equation}
    Nous allons montrer que l'application suivante fournit l'isomorphisme :
    \begin{equation}
        \begin{aligned}
            \psi\colon \frac{ \eK[X] }{ (\mu) }&\to \eK[f] \\
            \bar P&\mapsto P(f).
        \end{aligned}
    \end{equation}
    \begin{subproof}
        \item[\( \psi\) est bien définie]
            Si \( Q\in \bar P\) alors \( Q=P+S\mu\) pour un certain \( S\in \eK[X]\). Du coup nous avons
            \begin{equation}
                \psi(\bar Q)=P(f)+(S\mu)(f).
            \end{equation}
            Mais \( \mu(f)=0\) donc le deuxième terme est nul. Donc \( \psi(\bar P)\) est bien définit.
        \item[Injectif]
            Si \( \psi(\bar P)=0\) nous avons \( P(f)=0\), ce qui signifie que \( P=S\mu\) pour un polynôme \( S\). Par conséquent \( P\in (\mu)\) et donc \( \bar P=0\).
        \item[Surjectif]
            Soit \( P\in \eK[X]\). L'élément \( P(f) \) de \( \eK[f]\) est dans l'image de \( \psi\) parce que c'est \( \psi(\bar P)\).
    \end{subproof}
    En ce qui concerne la dimension, le corolaire~\ref{CorsLGiEN} en parle déjà : une base est donné par les projections de \( 1,X,\ldots, X^{\deg(\mu_a)-1}\).
\end{proof}

%---------------------------------------------------------------------------------------------------------------------------
\subsection{Polynôme caractéristique}
%---------------------------------------------------------------------------------------------------------------------------

\begin{definition}  \label{DefOWQooXbybYD}
    Soit un anneau commutatif \( A\). Si \( u\in\eM(n,A)\), nous définissons le \defe{polynôme caractéristique de \( u\)}{polynôme!caractéristique}\index{caractéristique!polynôme} :
    \begin{equation}    \label{Eqkxbdfu}
        \chi_u(X)=\det(u-X\mtu_n).
    \end{equation}
    Nous définissons de même le polynôme caractéristique d'un endomorphisme \( u\colon E\to E\).
\end{definition}

\begin{remark}
    Quelques remarques à propos du signe\quext{Attention : je crois qu'il y a des incohérences dans le Frido à propos de ce choix}.
    \begin{itemize}
        \item
            Certains auteurs définissent le polynôme caractéristique par \( \det(X-u)\) au lieu de \( \det(u-X)\).
        \item
            Wikipédia francophone prend la définition \( \det(X-u)\) (donc inverse de la notre). Allez lire la page de discussion.
        \item
            Sur les wikipédias d'autre langues, ça varie.
        \item
            Un avantage de \( \det(u-X)\) est que \( \det(u)=\chi_u(0)\).
        \item
            Un avantage de \( \det(X-u)\) est qu'il est unitaire.
    \end{itemize}
\end{remark}

\begin{lemma}       \label{LemooWCZMooZqyaHd}
    Le polynôme caractéristique \( \chi_u\) est unitaire en dimension paire et a pour degré la dimension de l'espace vectoriel \( E\)..
\end{lemma}

\begin{theorem}     \label{ThoNhbrUL}
    Soit \( E\) un \(\eK\)-espace vectoriel de dimension finie \( n\) et un endomorphisme \( u\in\End(E)\). Alors
    \begin{enumerate}
        \item
            Le polynôme caractéristique divise \( (\mu_u)^n\) dans \(\eK[X]\).
        \item
            Les polynômes caractéristiques et minimaux ont mêmes facteurs irréductibles dans \(\eK[X]\).
        \item
            Les polynômes caractéristiques et minimaux ont mêmes racines dans \(\eK[X]\).
        \item
            Le polynôme caractéristique est scindé si et seulement si le polynôme minimal est scindé.
    \end{enumerate}
\end{theorem}

\begin{theorem} \label{ThoWDGooQUGSTL}
    Soit \( u\in\End(E)\) et \( \lambda\in\eK\). Les conditions suivantes sont équivalentes
    \begin{enumerate}
        \item\label{ItemeXHXhHi}
            \( \lambda\in\Spec(u)\)
        \item\label{ItemeXHXhHii}
            \( \chi_u(\lambda)=0\)
        \item\label{ItemeXHXhHiii}
            \( \mu_u(\lambda)=0\).
    \end{enumerate}
\end{theorem}

\begin{proof}
    \ref{ItemeXHXhHi} \( \Leftrightarrow\)~\ref{ItemeXHXhHii}. Dire que \( \lambda\) est dans le spectre de \( u\) signifie que l'opérateur \( u-\lambda\mtu\) n'est pas inversible, ce qui est équivalent à dire que \( \det(u-\lambda\mtu)\) est nul par la proposition~\ref{PropYQNMooZjlYlA}\ref{ItemUPLNooYZMRJy} ou encore que \( \lambda\) est une racine du polynôme caractéristique de \( u\).

    \ref{ItemeXHXhHii} \( \Leftrightarrow\)~\ref{ItemeXHXhHiii}. Cela est une application directe du théorème~\ref{ThoNhbrUL} qui précise que le polynôme caractéristique a les mêmes racines dans \(\eK\) que le polynôme minimal.
\end{proof}

\begin{example} \label{ExICOJcFp}
    Sur \( \eR^2\), nous considérons la matrice \( A=\begin{pmatrix}
        1    &   0    \\
        1    &   1
    \end{pmatrix}\) qui a pour polynôme caractéristique\footnote{Définition~\ref{DefOWQooXbybYD}.} le polynôme \( \chi_A=(X-1)^2\). Le nombre \( \lambda=1\) est une racine double de ce polynôme, et pourtant il n'y a qu'une seule dimension d'espace propre :
    \begin{equation}
        \begin{pmatrix}
            1    &   0    \\
            1    &   1
        \end{pmatrix}\begin{pmatrix}
            x    \\
            y
        \end{pmatrix}=\begin{pmatrix}
            x    \\
            y
        \end{pmatrix}
    \end{equation}
    entraine \( x=0\).

    Ici la multiplicité algébrique est différente de la multiplicité géométrique.
\end{example}

La proposition suivante donne une utilisation amusante de la notion de polynôme caractéristique\footnote{Définition~\ref{DefOWQooXbybYD}.}.
\begin{proposition}[\cite{ooNGUJooPphdsT}]
    Soit un espace vectoriel \( V\) de dimension finie pour lequel il existe un endomorphisme \( f\colon V\to V\) tel que \( (f\circ f)(v)=-v\) pour tout \( v\in V\). Alors la dimension de \( V\) est paire.
\end{proposition}

\begin{proof}
    Cherchons les valeurs propres de \( f\) en résolvant l'équation \( f(v)=\lambda v\). Nous appliquons \( f\) à cette égalité :
    \begin{equation}
        -v=\lambda f(v)=\lambda^2v.
    \end{equation}
    Donc \( \lambda\) ne peut pas être réel. Nous avons montré que \( f\) n'a pas de valeurs propres réelles. Or le polynôme caractéristique de \( f\) est de degré égal à la dimension. Si la dimension est impaire, le polynôme caractéristique est de degré impair, et possède donc une racine réelle. Autrement dit, l'absence de racines réelles au polynôme caractéristique indique une dimension paire.
\end{proof}

Une autre preuve possible est d'utiliser le déterminant : si la dimension de \( V\) est \( n\) nous avons :
\begin{equation}
    \det(f^2)=\det(-\id)=(-1)^n.
\end{equation}
Donc \( (-1)^n\) est positif, ce qui montre que \( n\) est pair.

\begin{proposition}[\cite{RombaldiO}]\label{PropNrZGhT}
    Soit \( f\), un endomorphisme de \( E\) et \( x\in E\). Alors
    \begin{enumerate}
        \item
            L'espace \( E_{f,x}\) est stable par \( f\).
        \item\label{ItemfzKOCo}
            L'espace \( E_{f,x}\) est de dimension
            \begin{equation}
                p_{f,x}=\dim E_{f,x}=\deg(\mu_{f,x})
            \end{equation}
            où \( \mu_{f,x}\) est le générateur unitaire de \( I_{f,x}\).
        \item   \label{ItemKHNExH}
            Le polynôme caractéristique de \( f|_{E_{f,x}}\) est \( \mu_{f,x}\).
        \item   \label{ItemHMviZw}
            Nous avons
            \begin{equation}
                \chi_{f|_{E_{f,x}}}(f)x=\mu_{f,x}(f)x=0.
            \end{equation}
    \end{enumerate}
\end{proposition}

\begin{proof}
    Le fait que \( E_{f,x}\) soit stable par \( f\) est classique. Le point~\ref{ItemHMviZw} est un une application du point~\ref{ItemKHNExH}. Les deux gros morceaux sont donc les points~\ref{ItemfzKOCo} et~\ref{ItemKHNExH}.

    Étant donné que \( \mu_{f,x}\) est de degré minimal dans \( I_{f,x}\), l'ensemble
    \begin{equation}
        B=\{ f^k(x)\tq 0\leq k\leq p_{f,x}-1 \}
    \end{equation}
    est libre. En effet une combinaison nulle des vecteurs de \( B\) donnerait un polynôme en \( f\) de degré inférieur à \( p_{f,x}\) annulant \( x\). Nous écrivons
    \begin{equation}
        \mu_{f,x}(X)=X^{p_{f,x}}-\sum_{i=0}^{p_{f,x}-1}a_iX^k.
    \end{equation}
    Étant donné que \( \mu_{f,x}(f)x=0\) et que la somme du membre de droite est dans \( \Span(B)\), nous avons \( f^{p_{f,x}}(x)\in\Span(B)\). Nous prouvons par récurrence que \( f^{p_{f,x}+k}(x)\in\Span(B)\). En effet en appliquant \( f^k\) à l'égalité
    \begin{equation}
        0=f^{p_{f,x}}(x)-\sum_{i=0}^{p_{f,x}-1}a_if^i(x)
    \end{equation}
    nous trouvons
    \begin{equation}
        f^{p_{f,x}+k}(x)=\sum_{i=0}^{p_{f,x}-1}a_if^{i+k}(x),
    \end{equation}
    alors que par hypothèse de récurrence le membre de droite est dans \( \Span(B)\). L'ensemble \( B\) est alors générateur de \( E_{f,x}\) et donc une base d'icelui. Nous avons donc bien \( \dim(E_{f,x})=p_{f,x}\).

    Nous montrons maintenant que \( \mu_{f,x}\) est annulateur de \( f\) au point \( x\). Nous savons que
    \begin{equation}
        \mu_{f,x}(f)x=0.
    \end{equation}
    En y appliquant \( f^k\) et en profitant de la commutativité des polynômes sur les endomorphismes (proposition~\ref{LemQWvhYb}), nous avons
    \begin{equation}
        0=f^k\big( \mu_{f,x}(f)x \big)=\mu_{f,x}(f)f^k(x),
    \end{equation}
    de telle sorte que \( \mu_{f,x}(f)\) est nul sur \( B\) et donc est nul sur \( E_{f,x}\). Autrement dit,
    \begin{equation}
        \mu_{f,x}\big( f|_{E_{f,x}} \big)=0.
    \end{equation}
    Montrons que \( \mu_{f,x}\) est même minimal pour \( f|_{E_{f,x}}\). Sot \( Q\), un polynôme non nul de degré \( p_{f,x}-1\) annulant \( f|_{E_{f,x}}\). En particulier \( Q(f)x=0\), alors qu'une telle relation signifierait que \( B\) est un système lié, alors que nous avons montré que c'était un système libre. Nous concluons que \( \mu_{f,x}\) est le polynôme minimal de \( f|_{E_{f,x}}\).
\end{proof}

Cette histoire de densité permet de donner une démonstration alternative du théorème de Cayley-Hamilton.
\begin{theorem}[Cayley-Hamlilton]   \label{ThoCalYWLbJQ}
    Le polynôme caractéristique est un polynôme annulateur.
\end{theorem}
\index{théorème!Cayley-Hamilton}

Une démonstration plus simple via la densité des diagonalisables est donnée en théorème~\ref{ThoHZTooWDjTYI}.
\begin{proof}
    Nous devons prouver que \( \chi_f(f)x=0\) pour tout \( x\in E\). Pour cela nous nous fixons un \( x\in E\), nous considérons l'espace \( E_{f,x}\) et \( \chi_{f,x}\), le polynôme caractéristique de \( f|_{E_{f,x}}\). Étant donné que \( E_{f,x}\) est stable par \( f\), le polynôme caractéristique de \( f|_{E_{j,x}}\) divise \( \chi_f\), c'est-à-dire qu'il existe un polynôme \( Q_x\) tel que
    \begin{equation}
        \chi_f=Q_x\chi_{f,x},
    \end{equation}
    et donc aussi
    \begin{equation}
        \chi_f(f)x=Q_x(f)\big( \chi_{f,x}(f)x \big)=0
    \end{equation}
    parce que la proposition~\ref{PropNrZGhT} nous indique que \( \chi_{f,x}\) est un polynôme annulateur de \( f|_{E_{f,x}}\).
\end{proof}

\begin{corollary}
    Le degré du polynôme minimal est majoré par la dimension de l'espace.
\end{corollary}

\begin{proof}
    Le polynôme minimal divise le polynôme caractéristique parce qu'il engendre l'idéal des polynômes annulateurs par la proposition \ref{PropAnnncEcCxj}. Or le degré du polynôme caractéristique est la dimension de l'espace par le lemme~\ref{LemooWCZMooZqyaHd}.
\end{proof}

\begin{example}[Calcul de l'inverse d'un endomorphisme]
    Le polynôme de Cayley-Hamilton donne un moyen de calculer l'inverse d'un endomorphisme inversible pourvu que l'on sache son polynôme caractéristique. En effet, supposons que
    \begin{equation}
        \chi_f(X)=\sum_{k=0}^na_kX^k.
    \end{equation}
    Nous aurons alors
    \begin{equation}
        0=\chi_f(f)=\sum_{k=0}^na_kf^k.
    \end{equation}
    Nous appliquons \( f^{-1}\) à cette dernière égalité en sachant que \( f^{-1}(0)=0\) :
    \begin{equation}
        0=a_0f^{-1}+\sum_{k=1}^na_kf^{k-1},
    \end{equation}
    et donc
    \begin{equation}
        u^{-1}=-\frac{1}{ \det(f) }\sum_{k=1}^na_kf^{k-1}
    \end{equation}
    où nous avons utilisé le fait que \( a_0=\chi_f(0)=\det(f)\).
\end{example}

\begin{proposition}\label{PropooBYZCooBmYLSc}
    Si \( (X-z)^l\) (\( l\geq 1\)) est la plus grande puissance de \( (X-z)\) dans le polynôme caractéristique d'un endomorphisme \( u\) alors
    \begin{equation}
        1\leq \dim(E_e)\leq l.
    \end{equation}
    C'est-à-dire que nous avons au moins un vecteur propre pour chaque racine du polynôme caractéristique.
\end{proposition}

\begin{proof}
    Si $(X-z)$ divise \( \chi_u\) alors en posant \( \chi_u=(X-z)P(X)\) nous avons
    \begin{equation}
        \det(u-X\mtu)=(X-z)P(X),
    \end{equation}
    ce qui, évalué en \( X=z\), donne \( \det(u-z\mtu)=0\). L'annulation du déterminant étant équivalente à l'existence d'un noyau non trivial, nous avons \( v\neq 0\) dans \( E\) tel que \( (u-z\mtu)v=0\). Cela donne \( u(v)=zv\) et donc que \( v\) est vecteur propre de \( u\) pour la valeur propre \( z\). Donc aussi \( \dim(E_z)\geq 1\).

    Si \( \dim(E_z)=k\) alors le théorème de la base incomplète~\ref{ThonmnWKs} nous permet d'écrire une base de \( E\) dont les \( k\) premiers vecteurs forment une base de \( E_z\). Dans cette base, la matrice de \( u\) est de la forme
    \begin{equation}
        \begin{pmatrix}
             z   &       &       &   *    \\
                &   \ddots    &       &   \vdots    \\
                &       &   z    &   *    \\
                &       &       &   *
         \end{pmatrix}
    \end{equation}
    où les étoiles représentent des blocs à priori non nuls. En tout cas il est vu sous cette forme que \( (X-z\mtu)^k\) divise \( \chi_u\).
\end{proof}

%+++++++++++++++++++++++++++++++++++++++++++++++++++++++++++++++++++++++++++++++++++++++++++++++++++++++++++++++++++++++++++
\section{Diagonalisation et trigonalisation}
%+++++++++++++++++++++++++++++++++++++++++++++++++++++++++++++++++++++++++++++++++++++++++++++++++++++++++++++++++++++++++++

Ici encore \( \eK\) est un corps commutatif.

%---------------------------------------------------------------------------------------------------------------------------
\subsection{Matrices semblables}
%---------------------------------------------------------------------------------------------------------------------------

\begin{definition}[matrices semblables] \label{DefCQNFooSDhDpB}
    Sur l'ensemble \( \eM_n(\eK)\) des matrices \( n\times n\) à coefficients dans \(\eK\) nous introduisons la relation d'équivalence \( A\sim B\) si et seulement s'il existe une matrice \( P\in\GL(n,\eK)\) telle que \( B=P^{-1}AP\). Deux matrices équivalentes en ce sens sont dites \defe{semblables}{semblables!matrices}.
\end{definition}

Le polynôme caractéristique\footnote{Définition~\ref{DefOWQooXbybYD}.} est un invariant sous les similitudes. En effet si \( P\) est une matrice inversible,
\begin{subequations}
    \begin{align}
        \chi_{PAP^{-1}}&=\det(PAP^{-1}-\lambda X)\\
        &=\det\big( P^{-1}(PAP^{-1}-\lambda X)P^{-1} \big)\\
        &=\det(A-\lambda X).
    \end{align}
\end{subequations}

La permutation de lignes ou de colonnes ne sont pas de similitudes, comme le montrent les exemples suivants :
\begin{equation}
    \begin{aligned}[]
        A&=\begin{pmatrix}
            1    &   2    \\
            3    &   4
        \end{pmatrix}&
        B&=\begin{pmatrix}
            2    &   1    \\
            4    &   3
        \end{pmatrix}.
    \end{aligned}
\end{equation}
Nous avons \( \chi_A=x^2-5x-2\) tandis que \( \chi_B=x^2-5x+2\) alors que le polynôme caractéristique est un invariant de similitude.

%---------------------------------------------------------------------------------------------------------------------------
\subsection{Endomorphismes nilpotents}
%---------------------------------------------------------------------------------------------------------------------------

La \defe{trace}{trace!matrice} d'une matrice \( A\in \eM(n,\eK)\) est la somme de ses éléments diagonaux :
\begin{equation}
    \tr(A)=\sum_{i=1}^nA_{ii}.
\end{equation}
Une propriété importante est son invariance cyclique.

\begin{lemma}   \label{LemhbZTay}
    Quelques propriétés de la trace.
    \begin{enumerate}
        \item
    Si \( A\) et \( B\) sont des matrices carrées, alors \( \tr(AB)=\tr(BA)\).
\item
    La trace est un invariant de similitude.
    \end{enumerate}
\end{lemma}

\begin{proof}
    C'est un simple calcul :
    \begin{equation}
            \tr(AB)=\sum_{ik}A_{ik}B_{ki}
            =\sum_{ik}A_{ki}B_{ik}
            =\sum_{ik}B_{ik}A_{ki}
            =\sum_i(BA)_{ii}
            =\tr(BA)
    \end{equation}
    où nous avons simplement renommé les indices \( i\leftrightarrow k\).

    En particulier, la trace est un invariant de similitude parce que \( \tr(ABA^{-1})=\tr(A^{-1} AB)=\tr(B)\) par l'invariance cyclique démontrée en \ref{LEMooUXDRooWZbMVN}\ref{ITEMooXDYQooAlnArd}.
\end{proof}
La trace étant un invariant de similitude, nous pouvons donc définir la \defe{trace}{trace!endomorphisme} comme étant la trace de sa matrice dans une base quelconque. Si la matrice est diagonalisable, alors la trace est la somme des valeurs propres.

\begin{lemma}[\cite{fJhCTE}]   \label{LemzgNOjY}
    L'endomorphisme \( u\in\End(\eC^n)\) est nilpotent si et seulement si \( \tr(u^p)=0\) pour tout \( p\).
\end{lemma}

\begin{proof}
    Supposons que \( u\) est nilpotent. Alors ses valeurs propres sont toutes nulles et celles de \( u^p\) le sont également. La trace étant la somme des valeurs propres, nous avons alors tout de suite \( \tr(u^p)=0\).

    Supposons maintenant que \( \tr(u^p)=0\) pour tout \( p\). Le polynôme caractéristique \eqref{Eqkxbdfu} est
    \begin{equation}    \label{EqfnCqWq}
        \chi_u=(-1)^nX^{\alpha}(X-\lambda_1)^{\alpha_1}\ldots (X-\alpha_r)^{\alpha_r}.
    \end{equation}
    où les \( \lambda_i\) (\( i=1,\ldots, r\)) sont les valeurs propres non nulles distinctes de \( u\).

    Il est vite vu que le coefficient de \( X^{n-1}\) dans \( \chi_u\) est \( -\tr(u)\) parce que le coefficient de \( X^{n-1}\) se calcule en prenant tous les $X$ sauf une fois \( -\lambda_i\). D'autre part le polynôme caractéristique de \( u^p \) est le même que celui de \( u\), en remplaçant \( \lambda_i\) par \( \lambda_i^p\); cela est dû au fait que si \( v\) est vecteur propre de valeur propre \( \lambda\), alors \( u^pv=\lambda^pv\).

    Par l'équation \eqref{EqfnCqWq}, nous voyons que le coefficient du terme \( X^{n-1}\) dans les polynôme caractéristique est
    \begin{equation}        \label{eqSoDSKH}
        0=\tr(u^p)=\alpha_1\lambda_1^p+\cdots +\alpha_r\lambda_r^p.
    \end{equation}
    Donc les nombres \( (\alpha_1,\ldots, \alpha_r)\) est une solution non triviale\footnote{Si \( \alpha_1=\ldots=\alpha_r=0\), alors les valeurs propres sont toutes nulles et la matrice est en réalité nulle dès le départ.} du système
    \begin{subequations}    \label{EqDpvTnu}
        \begin{numcases}{}
            \alpha_1X_1+\cdots +\lambda_rX_r=0\\
            \qquad\vdots\\
            \lambda^r_1X_1+\cdots +\lambda_r^rX_r=0.
        \end{numcases}
    \end{subequations}
    Cela sont les équations \eqref{eqSoDSKH} écrites avec \( p=1,\ldots, r\). Le déterminant de ce système est
    \begin{equation}
        \lambda_1\ldots\lambda_r\det\begin{pmatrix}
             1   &   \ldots    &   1    \\
             \lambda_1   &   \ldots    &   \lambda_1    \\
             \vdots   &       &   \vdots    \\
             \lambda_1^{r-1}   &   \ldots    &   \lambda_r^{r-1}
         \end{pmatrix}\neq 0,
    \end{equation}
    qui est un déterminant de Vandermonde (proposition~\ref{PropnuUvtj}) valant
    \begin{equation}
        0=\lambda_1\ldots\lambda_r\prod_{1\leq i\leq j\leq r}(\lambda_i-\lambda_j).
    \end{equation}
    Étant donné que les \( \lambda_i\) sont distincts et non nuls, nous avons une contradiction et nous devons conclure que \( (\alpha_1,\ldots, \alpha_r)\) était une solution triviale du système \eqref{EqDpvTnu}.
\end{proof}

\begin{proposition}[\cite{SVSFooIOYShq}]    \label{PropMWWJooVIXdJp}
    Soit un \( \eK\)-espace vectoriel \( E\). Un endomorphisme \( u\in\End(E)\) est nilpotent si et seulement s'il existe une base de \( E\) dans laquelle la matrice de \( u\) est strictement triangulaire supérieure.
\end{proposition}

\begin{proof}
    \begin{subproof}
       \item[\( \Rightarrow\)]
           Nous faisons la démonstration par récurrence sur la dimension de \( E\). Lorsque \( n=1\) nous avons \( u=(a)\) avec \( a\in \eK\). Vu que \( a^k=0\) pour un certain \( k\) nous avons \( a=0\) parce qu'un corps est toujours un anneau intègre\footnote{Lemme~\ref{LemAnnCorpsnonInterdivzer}.}.

           Lorsque \( \dim(E)=n\) nous savons que \( u\) a un noyau non réduit au vecteur nul (parce qu'il est nilpotent). Soit donc un vecteur non nul \( x\in\ker(u)\) et une base
           \begin{equation}
               \{ x,e_2,\ldots, e_n \}
           \end{equation}
           donnée par le théorème de la base incomplète~\ref{ThonmnWKs}. La matrice de \( u\) dans cette base s'écrit
           \begin{equation}
               \begin{pmatrix}
                       \begin{array}[]{c|c}
                           0&\begin{matrix}
                               * &   *    &   *
                           \end{matrix}\\
                           \hline
                           \begin{matrix}
                               0 \\
                               0 \\
                               0
                           \end{matrix}&
                           \begin{pmatrix}
                                &       &       \\
                                &   A    &       \\
                                &       &
                           \end{pmatrix}
                       \end{array}
               \end{pmatrix}.
           \end{equation}
           Un tout petit peu de calcul de produit de matrice montre que la matrice de \( u^k\) est de la forme
           \begin{equation}
               \begin{pmatrix}
                       \begin{array}[]{c|c}
                           0&\begin{matrix}
                               * &   *    &   *
                           \end{matrix}\\
                           \hline
                           \begin{matrix}
                               0 \\
                               0 \\
                               0
                           \end{matrix}&
                           \begin{pmatrix}
                                &       &       \\
                                &   A^k    &       \\
                                &       &
                           \end{pmatrix}
                       \end{array}
               \end{pmatrix}.
           \end{equation}
           Étant donné que \( u\) est nilpotente, la matrice \( A\) l'est aussi. L'hypothèse de récurrence dit alors que \( A\) est strictement triangulaire supérieure (ou en tout cas peut le devenir par un changement de base adéquat).

       \item[\( \Leftarrow\)]

            Lorsqu'une matrice est triangulaire supérieure stricte, elle applique
            \begin{equation}
                \Span\{ e_1,\ldots, e_k \}\to\Span\{ e_1,\ldots, e_{k-1} \}.
            \end{equation}
            Donc tout vecteur finit sur zéro si on lui applique \( u\) assez souvent.
    \end{subproof}
\end{proof}

\begin{proposition}[Thème~\ref{THEMEooPQKDooTAVKFH}]     \label{PROPooWTFWooXHlmhp}
    Soit \( E\) un espace de Banach (espace vectoriel normé complet). Si \( A\in\aL(E,E)\) est nilpotente, alors \( (\mtu-A)\) est inversible et son inverse est donné par
    \begin{equation}
        (\mtu-A)^{-1}=\sum_{k=0}^{\infty}A^k,
    \end{equation}
    où l'infini peut évidemment être remplacé par l'ordre de nilpotence de \( A\).
\end{proposition}

\begin{proof}
    En ce qui concerne la convergence de la somme, elle ne fait pas de doutes parce que \( A\) étant nilpotente, la somme contient seulement une quantité finie de termes non nuls.

    Montrons à présent que la somme est l'inverse de \( \mtu-A\) en multipliant terme à terme :
    \begin{equation}
        \sum_{k=0}^nA^k(\mtu-A)=\sum_{k=0}^n(A^k-A^{k+1})=\mtu-A^{n+1}.
    \end{equation}
    Par conséquent
    \begin{equation}
        \| \mtu-\sum_{k=0}^nA^k(\mtu-A) \|=\| A^{n+1} \|\to 0.
    \end{equation}
    La dernière limite est en réalité une égalité pour \( n\) assez grand.
\end{proof}

\begin{proposition}
    Soit \( A\in\GL(n,\eC)\). La suite \( (A^k)_{k\in \eZ}\) est bornée si et seulement si \( A\) est diagonalisable et \( \Spec(A)\subset \gS^1\).
\end{proposition}

\begin{proof}
    Si \( A\) est diagonalisable avec les valeurs propres \( \lambda_i\) de norme \( 1\) dans \( \eC\), alors \( A^k\) est la matrice diagonale avec les \( \lambda_i^k\) sur la diagonale. Cela reste borné pour toute valeur entière de \( k\).

    En ce qui concerne l'autre sens, nous supposons encore que
    \begin{equation}
        A=\begin{pmatrix}
            \lambda_1\mtu+N_1    &       &       \\
                &   \ddots    &       \\
                &       &   \lambda_s\mtu+N_s
        \end{pmatrix},
    \end{equation}
    et nous regardons un des blocs. Nous voulons prouver que \( N=0\) et que \( | \lambda |=1\).

    Nous commençons par regarder ce qu'implique le fait que \( (\lambda \mtu+N)^n\) reste borné pour \( n>0\). En notant \( r\) l'ordre de nilpotence de \( N\), nous avons le développement
    \begin{equation}
        (\lambda\mtu+N)^n=\sum_{k=0}^{r-1}\binom{ n }{ k }N^k\lambda^{n-k}.
    \end{equation}
    Par la proposition~\ref{PropMWWJooVIXdJp}, une matrice nilpotente s'écrit dans une base sous la forme
    \begin{equation}
        N=\begin{pmatrix}
             0   &   1    &       &       \\
                &   0    &   1    &       \\
                & &   \ddots   &   \ddots    &      \\
                &&       &   0    &   1     \\
                &&       &      &   0
         \end{pmatrix}
    \end{equation}
    et effectuer \( A^k\) revient à décaler la diagonale de \( 1\). Donc la famille
    \begin{equation}
        \{ \mtu,N,\ldots, N^{r-1} \}
    \end{equation}
    est libre. Par conséquent la suite \( (\lambda\mtu+N)^n\) restera bornée si et seulement si chacun des termes
    \begin{equation}    \label{EqXRDVDCM}
        \binom{ n }{ k }N^k\lambda^{n-k}
    \end{equation}
    reste borné. Le premier terme étant \( \lambda^n\mtu\), nous avons obligatoirement \( | \lambda |\leq 1\). Si \( | \lambda |<1\), alors le coefficient \( \binom{ n }{ k }\lambda^{n-k}\) tend vers zéro. Si \( | \lambda |=1\) par contre ce coefficient tend vers l'infini et la seule façon pour que \eqref{EqXRDVDCM} reste borné est que \( N=0\). Nous avons donc deux possibilités :
    \begin{itemize}
        \item \( | \lambda |<1\)
        \item \( | \lambda |=1\) et \( N=0\).
    \end{itemize}

    Nous nous tournons maintenant sur la contrainte que \( (\lambda\mtu+N)^n\) doive rester borné pour \( n<0\). Nous avons
    \begin{equation}
        \lambda\mtu+N=\lambda(\mtu+\lambda^{-1}N),
    \end{equation}
    et nous pouvons appliquer la proposition~\ref{PROPooWTFWooXHlmhp} à l'opérateur nilpotent \( -\lambda^{-1} N\) pour avoir
    \begin{equation}
        (\mtu+\lambda^{-1}N)^{-1}=\mtu+\sum_{k=1}^{\infty}(-\lambda)^{-1}N^k.
    \end{equation}
    Ceci pour dire que \( (\lambda\mtu+N)^{-1}=\lambda^{-1}(\mtu+\lambda^{-1}N')\) pour une autre matrice nilpotente \( N'\). Le travail déjà fait, appliqué à \( \lambda^{-1}\) et \( N'\), nous donne deux possibilités :
    \begin{itemize}
        \item \( | \lambda^{-1} |<1\)
        \item \( | \lambda^{-1} |=1\) et \( N'=0\).
    \end{itemize}
    La possibilité \( | \lambda^{-1} |<1\) est exclue parce qu'elle impliquerait \( | \lambda |>1\) qui avait déjà été exclu. Il ne reste donc que la possibilité \( | \lambda |=1\) et \( N=N'=0\).
\end{proof}

%---------------------------------------------------------------------------------------------------------------------------
\subsection{Endomorphismes diagonalisables}
%---------------------------------------------------------------------------------------------------------------------------

\begin{definition}  \label{DefCNJqsmo}
    Une matrice est \defe{diagonalisable}{diagonalisable} si elle est semblable\footnote{Définition~\ref{DefCQNFooSDhDpB}.} à une matrice diagonale.
\end{definition}

\begin{lemma}
    Une matrice triangulaire supérieure avec des \( 1\) sur la diagonale n'est diagonalisable que si elle est diagonale (c'est-à-dire si elle est la matrice unité).
\end{lemma}

\begin{proof}
    Si \( A\) est une matrice triangulaire supérieure de taille \( n\) telle que \( A_{ii}=1\), alors \( \det(A-\lambda\mtu)=(1-\lambda)^n\), ce qui signifie que \( \Spec(A)=\{ 1 \}\). Pour la diagonaliser, il faudrait une matrice \( P\in\GL(n,\eK)\) telle que \( \mtu=P^{-1}AP\), ce qui est uniquement possible si \( A=\mtu\).
\end{proof}

\begin{lemma}       \label{LemgnaEOk}
    Soit \( F\) un sous-espace stable par \( u\). Soit une décomposition du polynôme minimal
    \begin{equation}
        \mu_u=P_1^{n_1}\ldots P_r^{n_r}
    \end{equation}
    où les \( P_i\) sont des polynômes irréductibles unitaires distincts. Si nous posons \( E_i=\ker P_i^{n_i}\), alors
    \begin{equation}
        F=(F\cap E_1)\oplus\ldots \oplus(F\cap E_r).
    \end{equation}
\end{lemma}

\begin{theorem}     \label{ThoDigLEQEXR}
    Soit \( E\), un espace vectoriel de dimension \( n\) sur le corps commutatif \( \eK\) et \( u\in\End(E)\). Les propriétés suivantes sont équivalentes.
    \begin{enumerate}
        \item\label{ItemThoDigLEQEXRiv}
            L'endomorphisme \( u\) est diagonalisable.
        \item       \label{ItemThoDigLEQEXRi}
            Il existe un polynôme \( P\in\eK[X]\) non constant, scindé sur \(\eK\) dont toutes les racines sont simples tel que \( P(u)=0\).
        \item\label{ItemThoDigLEQEXRii}
            Le polynôme minimal \( \mu_u\) est scindé sur \(\eK\) et toutes ses racines sont simples\footnote{Le polynôme \emph{caractéristique}, lui, n'a pas spécialement ses racines simples; il peut encore être de la forme
            \begin{equation}
                \chi_u(X)=\prod_{i=1}^r(X-\lambda_i)^{\alpha_i},
        \end{equation}
        mais alors \( \dim(E_{\lambda_i})=\alpha_i\). }.
        \item\label{ItemThoDigLEQEXRiii}
            Tout sous-espace de \( E\) possède un supplémentaire stable par \( u\).
        \item       \label{ITEMooZNJFooEiqDYp}
            Dans une base adaptée, la matrice de \( u\) est diagonale et les éléments diagonaux sont ses valeurs propres.
    \end{enumerate}
\end{theorem}
\index{diagonalisable!et polynôme minimum scindé}

\begin{proof}
    Plein d'implications à prouver.
    \begin{subproof}
    \item[\ref{ItemThoDigLEQEXRi} implique~\ref{ItemThoDigLEQEXRii}] Étant donné que \( P(u)=0\), il est dans l'idéal des polynômes annulateurs de \( u\), et le polynôme minimal \( \mu_u\) le divise parce que l'idéal des polynômes annulateurs est généré par \( \mu_u\) par le théorème~\ref{ThoCCHkoU}.

    \item[\ref{ItemThoDigLEQEXRii} implique~\ref{ItemThoDigLEQEXRiv}] Étant donné que le polynôme minimal est scindé à racines simples, il s'écrit sous forme de produits de monômes tous distincts, c'est-à-dire
    \begin{equation}
        \mu_u(X)=(X-\lambda_1)\ldots(X-\lambda_r)
    \end{equation}
    où les \( \lambda_i\) sont des éléments distincts de \( \eK\). Étant donné que \( \mu_u(u)=0\), le théorème de décomposition des noyaux (théorème~\ref{ThoDecompNoyayzzMWod}) nous enseigne que
    \begin{equation}
        E=\ker(u-\lambda_1)\oplus\ldots\oplus\ker(u-\lambda_r).
    \end{equation}
    Mais \( \ker(u-\lambda_i)\) est l'espace propre \( E_{\lambda_i}(u)\). Donc \( u\) est diagonalisable.

\item[\ref{ItemThoDigLEQEXRiv} implique~\ref{ItemThoDigLEQEXRiii}] Soit \( \{ e_1,\ldots, e_n \}\) une base qui diagonalise \( u\), soit \( F\) un sous-espace de \( E\) un \( \{ f_1,\ldots, f_r \}\) une base de \( F\). Par le théorème \ref{ThoMGQZooIgrXjy}\ref{ITEMooCJQGooXwjsfm}, nous pouvons compléter la base de \( F\) par des éléments de la base \( \{ e_i \}\). Le complément ainsi construit est invariant par \( u\).

\item[\ref{ItemThoDigLEQEXRiii} implique~\ref{ItemThoDigLEQEXRiv}] En dimension un, tout endomorphisme est diagonalisable, nous supposons donc que \( \dim E=n\geq 2\). Nous procédons par récurrence sur le nombre de vecteurs propres connus de \( u\). Supposons avoir déjà trouvé \( p\) vecteurs propres \( e_1,\ldots, e_p\) de \( u\). Considérons \( H\), un hyperplan qui contient les vecteurs \( e_1,\ldots, e_p\). Soit \( F\) un supplémentaire de \( H\) stable par \( u\); par construction \( \dim F=1\) et si \( e_{p+1}\in F\), il doit être vecteur propre de \( u\).

\item[\ref{ItemThoDigLEQEXRiv} implique~\ref{ItemThoDigLEQEXRi}] Nous supposons maintenant que \( u\) est diagonalisable. Soient \( \lambda_1,\ldots, \lambda_r\) les valeurs propres deux à deux distinctes, et considérons le polynôme
    \begin{equation}
        P(x)=(X-\lambda_1)\ldots (X-\lambda_r).
    \end{equation}
    Alors \( P(u)=0\). En effet si \( e_i\) est un vecteur propre pour la valeur propre \( \lambda_i\),
    \begin{equation}
        P(u)e_i=\prod_{j\neq i}(u-\lambda_j)\circ(u-\lambda_i)e_i=0
    \end{equation}
    par le lemme~\ref{LemQWvhYb}. Par conséquent \( P(u)\) s'annule sur une base.

\item[\ref{ITEMooZNJFooEiqDYp} implique~\ref{ItemThoDigLEQEXRi}]
    Si la matrice \( A\) est diagonale alors le polynôme \( P=\prod_{i=1}^n(A-A_{ii}\mtu)\) est annulateur de \( A\).
        \item[\ref{ItemThoDigLEQEXRii} implique~\ref{ITEMooZNJFooEiqDYp}]
            le polynôme minimal de \( u\) s'écrit
            \begin{equation}
                \mu=(X-\lambda_1)\ldots(X-\lambda_r),
            \end{equation}
            et les espaces $E_i$ du lemme~\ref{LemgnaEOk} sont les espaces propres \( E_i=\ker(u-\lambda_i)\). Nous avons donc une somme directe
            \begin{equation}
                E=E_1\oplus\ldots\oplus E_r.
            \end{equation}
            Dans chacun des espaces propres, $u$ a une matrice diagonale avec la valeur propre correspondante sur la diagonale. Une base de \( E\) constituée d'une base de chacun des espaces propres est donc une base comme nous en cherchons.
    \end{subproof}
\end{proof}

\begin{corollary}       \label{CorQeVqsS}
    Si \( u\) est diagonalisable et si \( F\) est une sous-espace stable par \( u\), alors
    \begin{equation}
        F=\bigoplus_{\lambda}E_{\lambda}(u)\cap F
    \end{equation}
    où \( E_{\lambda}(u)\) est l'espace propre de \( u\) pour la valeur propre \( \lambda\). En particulier la restriction de \( u\) à \( F\), \( u|_F\) est diagonalisable.
\end{corollary}

\begin{proof}
    Par le théorème~\ref{ThoDigLEQEXR}, le polynôme \( \mu_u\) est scindé et ne possède que des racines simples. Notons le
    \begin{equation}
        \mu_u(X)=(X-\lambda_1)\ldots (X-\lambda_r).
    \end{equation}
    Les espaces \( E_i\) du lemme~\ref{LemgnaEOk} sont maintenant les espaces propres.

    En ce qui concerne la diagonalisabilité de \( u|_F\), notons que nous avons une base de \( F\) composée de vecteurs dans les espaces \( E_{\lambda}(u)\). Cette base de \( F\) est une base de vecteurs propres de \( u\).
\end{proof}

\begin{lemma}
    Soit \( E\) un \( \eK\)-espace vectoriel et \( u\in\End(E)\). Si \( \Card\big( \Spec(u) \big)=\dim(E)\) alors \( u\) est diagonalisable.
\end{lemma}

\begin{proof}
    Soient \( \lambda_1,\ldots, \lambda_n\) les valeurs propres distinctes de \( u\). Nous savons que les espaces propres correspondants sont en somme directe (lemme~\ref{LemjcztYH}). Par conséquent \( \Span\{ E_{\lambda_i}(u) \}\) est de dimension \( n\) est \( u\) est diagonalisable.
\end{proof}

Voici un résultat de diagonalisation simultanée. Nous donnerons un résultat de trigonalisation simultanée dans le lemme~\ref{LemSLGPooIghEPI}.
\begin{proposition}[Diagonalisation simultanée]     \label{PropGqhAMei}
    Soit \( (u_i)_{i\in I}\) une famille d'endomorphismes qui commutent deux à deux.
    \begin{enumerate}
        \item       \label{ItemGqhAMei}
            Si \( i,j\in I\) alors tout sous-espace propre de \( u_i\) est stable par \( u_j\). Autrement dit \( u_j\big(E_{\lambda}(u)\big)\subset E_{\lambda}(u)\).
        \item
            Si les \( u_i\) sont diagonalisables, alors ils le sont simultanément.
    \end{enumerate}
\end{proposition}
\index{diagonalisation!simultanée}

\begin{proof}
    Supposons que \( u_i\) et \( u_j\) commutent et soit \( x\) un vecteur propre de \( u_i\) : \( u_ix=\lambda x\). Nous montrons que \( u_jx\in E_{\lambda}(u)\). Nous avons
    \begin{equation}
        u_i\big( u_j(x) \big)=u_j\big( u_i(x) \big)=\lambda u_j(x).
    \end{equation}
    Par conséquent \( u_j(x)\) est vecteur propre de \( u_i\) de valeur propre \( \lambda\).

    Montrons maintenant l'affirmation à propos des endomorphismes simultanément diagonalisables. Si \( \dim E=1\), le résultat est évident. Nous supposons également qu'aucun des \( u_i\) n'est multiple de l'identité. Nous effectuons une récurrence sur la dimension.

    Soit \( u_0\) un des \( u_i\) et considérons ses valeurs propres deux à deux distinctes \( \lambda_1,\ldots, \lambda_r\). Pour chaque \( k\) nous avons
    \begin{equation}
        E_{\lambda_k}(u_0)\neq E,
    \end{equation}
    sinon \( u_0\) serait un multiple de l'identité. Par contre le fait que \( u_0\) soit diagonalisable permet de décomposer \( E\) en espaces propres de \( u_0\) :
    \begin{equation}
        E=\bigoplus_{k}E_{\lambda_k}(u_0).
    \end{equation}
    Ce que nous allons faire est de simultanément diagonaliser les \( (u_i)_{i\in I}\) sur chacun des \( E_{\lambda_k}\) séparément. Par le point~\ref{ItemGqhAMei}, nous avons \( u_i\colon E_{\lambda_k}(u_0)\to E_{\lambda_k}(u_0)\), et nous pouvons considérer la famille d'opérateurs
    \begin{equation}
        \left( u_i|_{E_{\lambda_k}(u_0)} \right)_{i\in I}.
    \end{equation}
    Ce sont tous des opérateurs qui commutent et qui agissent sur un espace de dimension plus petite. Par hypothèse de récurrence nous avons une base de \( E_{\lambda_k}(u_0)\) qui diagonalise tous les \( u_i\).
\end{proof}

\begin{example}     \label{ExewINgYo}
    Soit un espace vectoriel sur un corps \( \eK\). Un opérateur \defe{involutif}{involution} est un opérateur différent de l'identité dont le carré est l'identité. Typiquement une symétrie orthogonale dans \( \eR^3\). Le polynôme caractéristique d'une involution est \( X^2-1=(X+1)(X-1)\).

    Tant que \( 1\neq -1\), \( X^1-1\) est donc scindé à racines simples et les involutions sont diagonalisables (\ref{ThoDigLEQEXR}). Cependant si le corps est de caractéristique \( 2\), alors \( X^2-1=(X+1)^2\) et l'involution n'est plus diagonalisable.

    Par exemple si le corps est de caractéristique \( 2\), nous avons
    \begin{subequations}
        \begin{align}
            A&=\begin{pmatrix}
                1    &   1    \\
                0    &   1
            \end{pmatrix}\\
            A^1&=\begin{pmatrix}
                1    &   2    \\
                0    &   1
            \end{pmatrix}=\begin{pmatrix}
                1    &   0    \\
                0    &   1
            \end{pmatrix}.
        \end{align}
    \end{subequations}
    Ce \( A\) est donc une involution mais n'est pas diagonalisable.
\end{example}

%---------------------------------------------------------------------------------------------------------------------------
\subsection{Diagonalisation : cas complexe, pas toujours}
%---------------------------------------------------------------------------------------------------------------------------

Il n'est pas vrai qu'une matrice de \( \eM(n,\eC)\) soit toujours diagonalisable. En effet le théorème~\ref{ThoDigLEQEXR}\ref{ItemThoDigLEQEXRii} dit qu'une matrice est diagonalisable si et seulement si son polynôme minimal est scindé à racines simples. Certes sur \( \eC\) le polynôme minimal sera scindé, mais il ne sera pas spécialement à racines simples.

\begin{example}
    La matrice
    \begin{equation}
        A=\begin{pmatrix}
            0    &   1    \\
            0    &   0
        \end{pmatrix}
    \end{equation}
    a pour polynôme caractéristique \( \chi_A(X)=X^2\). Cela est également son polynôme minimal, et ce n'est pas à racine simple.

    Il est par ailleurs facile de voir que le seul espace propre de \( A\) est \( \Span\{ (1,0) \}\) (ici le span est sur \( \eC\)). Donc l'espace \( \eC^2\) ne possède pas de base de vecteurs propres de \( A\).
\end{example}

Ce qui est vrai, c'est que le polynôme caractéristique a des racines, et que ces racines correspondent à des vecteurs propres. Mais il n'y a pas toujours autant de vecteurs propres que la multiplicité des racines.

%---------------------------------------------------------------------------------------------------------------------------
\subsection{Trigonalisation : généralités}
%---------------------------------------------------------------------------------------------------------------------------

\begin{definition}[\cite{MQMKooPBfnZN}]
    Une matrice dans \( \eM(n,\eK)\) est \defe{trigonalisable}{matrice!trigonalisable} lorsqu'elle est semblable\footnote{Définition~\ref{DefCQNFooSDhDpB}.} à une matrice triangulaire supérieure.
\end{definition}

\begin{proposition}[Trigonalisation et polynôme caractéristique scindé] \label{PropKNVFooQflQsJ}
    Soit \( u\) un endomorphisme d'un espace vectoriel \( E\) sur le corps \( \eK\). Les faits suivants sont équivalents.
    \begin{enumerate}
        \item   \label{ItemZKDMooOrTHkwi}
            L'endomorphisme \( u\) est trigonalisable (auquel cas les valeurs propres sont sur la diagonale).
        \item   \label{ItemZKDMooOrTHkwii}
            Le polynôme caractéristique de \( u\) est scindé\footnote{Définition~\ref{DefCPLSooQaHJKQ}.}.
    \end{enumerate}
\end{proposition}
\index{trigonalisation!et polynôme caractéristique}

\begin{proof}
    \begin{subproof}
        \item[\ref{ItemZKDMooOrTHkwii}\( \Rightarrow\)\ref{ItemZKDMooOrTHkwi}]
            Nous avons par hypothèse que
            \begin{equation}
                \chi_u(X)=\prod_{i=1}^r(X-\lambda_i)^{\alpha_i}
            \end{equation}
            où les \( \lambda_i\) sont les valeurs propres de \( u\). Le théorème de Cayley-Hamilton~\ref{ThoCalYWLbJQ} dit que \( \chi_u(u)=0\), ce qui permet d'utiliser le théorème de décomposition des noyaux~\ref{ThoDecompNoyayzzMWod} :
            \begin{equation}
                E=\ker(X-\lambda_1)^{\alpha_1}\oplus\ldots\oplus\ker(X-\lambda_r)^{\alpha_r}.
            \end{equation}
            Les espaces \( F_{\lambda_i}(u)=\ker(X-\lambda_i)^{\alpha_i}\) sont les espaces caractéristiques de \( u\), ce qui fait que \( u-\lambda_i\mtu\) est nilpotent sur \( F_{\lambda_i}(u)\). L'endomorphisme \( u-\lambda_i\mtu\) est donc strictement trigonalisable supérieur sur son bloc\footnote{Proposition~\ref{PropMWWJooVIXdJp}.}. Cela signifie que \( u\) est triangulaire supérieure avec les valeurs propres sur la diagonale.

        \item[\ref{ItemZKDMooOrTHkwi}\( \Rightarrow\)\ref{ItemZKDMooOrTHkwii}]

            C'est immédiat parce que le déterminant d'une matrice triangulaire est le produit des éléments de sa diagonale.
    \end{subproof}
\end{proof}

\begin{remark}
    La méthode des pivots de Gauss\footnote{Le lemme~\ref{LemZMxxnfM}.} certes permet de trigonaliser n'importe quoi, mais elle ne correspond pas à un changement de base. Autrement dit, les pivots de Gauss ne sont pas de similitudes.

    C'est là qu'il faut bien avoir en tête la différence entre \emph{équivalence} et \emph{similarité}\footnote{Définition \ref{DefBLELooTvlHoB}.}. Lorsqu'on parle de changement de base, de matrice trigonalisable ou diagonalisable, nous parlons de similarité et non d'équivalence.
\end{remark}

%---------------------------------------------------------------------------------------------------------------------------
\subsection{Trigonalisation : cas complexe}
%---------------------------------------------------------------------------------------------------------------------------

La proposition~\ref{PropKNVFooQflQsJ} dit déjà que tous les endomorphismes sont trigonalisables sur \( \eC\). Nous allons aller plus loin et montrer que la trigonalisation peut être effectuée à l'aide d'une matrice unitaire.

Une démonstration alternative passant par le polynôme caractéristique sera présentée dans la remarque~\ref{RemXFZTooXkGzQg} utilisant la proposition~\ref{PropKNVFooQflQsJ}.
\begin{lemma}[Lemme de Schur complexe, trigonisation\cite{NormHKNPKRqV}]  \label{LemSchurComplHAftTq}
    Si \( A\in\eM(n,\eC)\), il existe une matrice unitaire \( U\) telle que \( UAU^{-1}\) soit triangulaire supérieure\footnote{«triangulaire supérieure» ne signifie pas «strictement triangulaire supérieure». Ici, il est possible que la diagonale soit non nulle; non seulement possible, mais même très probable en pratique.}.
\end{lemma}
\index{lemme!Schur complexe}
%TODO : Le lemme de Schur est souvent énoncé en disant que si p est une représentation irréductible, alors les seuls endomorphismes de V commutant avec tous les p(g) sont les multiples de l'identité. Quel est le lien avec ceci ?

\begin{proof}
    Étant donné que \( \eC\) est algébriquement clos, nous pouvons toujours considérer un vecteur propre \( v_1\) de \( A\), de valeur propre \( \lambda_1\). Nous pouvons utiliser un procédé de Gram-Schmidt pour construire une base orthonormée \( \{ v,u_2,\ldots, u_n \}\) de \( \eR^n\), et la matrice (unitaire)
    \begin{equation}
        Q=\begin{pmatrix}
             \uparrow   &   \uparrow    &       &   \uparrow    \\
             v   &   u_2    &   \cdots    &   u_n    \\
             \downarrow   &   \downarrow    &       &   \downarrow
         \end{pmatrix}.
    \end{equation}
    Nous avons \( Q^{-1}AQe_1=Q^{-1} Av=\lambda Q^{-1} v=\lambda e_1\), par conséquent la matrice \( Q^{-1} AQ\) est de la forme
    \begin{equation}
        Q^{-1}AQ=\begin{pmatrix}
            \lambda_1    &   *    \\
            0    &   A_1
        \end{pmatrix}
    \end{equation}
    où \( *\) représente une ligne quelconque et \( A_1\) est une matrice de \( \eM(n-1,\eC)\). Nous pouvons donc répéter le processus sur \( A_1\) et obtenir une matrice triangulaire supérieure (nous utilisons le fait qu'un produit de matrices orthogonales est une matrice orthogonale).
\end{proof}
En particulier les matrices hermitiennes, anti-hermitiennes et unitaires sont trigonalisables par une matrice unitaire, qui peut être choisie de déterminant \( 1\).

\begin{lemma}       \label{LEMooRCFGooPPXiKi}
    Soit \( A\in \eM(n,\eC)\) et une matrice unitaire \( U\) telle que \( A=UTU^{-1}\) où \( T\) est triangulaire.
    \begin{enumerate}
        \item
            En ce qui concerne les polynômes caractéristiques, \( \chi_A=\chi_T\).
        \item
            Pour les spectres, \( \Spec(A)=\Spec(T)\).
        \item
            Les valeurs propres de \( A\) sont les éléments diagonaux de \( T\).
    \end{enumerate}
\end{lemma}

\begin{proof}
    Vu que \( U\) commute évidemment avec \( \mtu\) nous avons
    \begin{equation}
        \chi_A(\lambda)=\det(A-\lambda \mtu)=\det(UTU^{-1}-\lambda\mtu)=\det\big( U(T-\lambda\mtu)U^{-1} \big).
    \end{equation}
    À ce niveau nous utilisons le fait que le déterminant soit multiplicatif~\ref{PropYQNMooZjlYlA} pour conclure :
    \begin{equation}
        \chi_A(\lambda)=\det\big( U(T-\lambda\mtu)U^{-1} \big)=\det(U)\det(T-\lambda\mtu)\det(U^{-1})=\det(T-\lambda\mtu)=\chi_T(\lambda).
    \end{equation}

    Pour les spectres, l'égalité des polynômes caractéristique implique l'égalité des spectres parce que les valeurs propres sont les racines du polynôme caractéristique par le théorème~\ref{ThoWDGooQUGSTL}.

    Les valeurs propres d'une matrice triangulaire sont les valeurs sur la diagonale.
\end{proof}

\begin{remark}
    Le lemme mentionne le fait que les valeurs propres de \( A\) sont les éléments diagonaux de \( T\). Mais attention : ceci ne dit rien au niveau des multiplicités géométriques. Un nombre peut être cinq fois sur la diagonale de \( T\) alors que l'espace propre correspondant pour \( A\) n'est que de dimension \( 1\). Exemple : la matrice
    \begin{equation}
        A=\begin{pmatrix}
            1    &   1    \\
            0    &   1
        \end{pmatrix}
    \end{equation}
    a deux \( 1\) sur la diagonale. Le nombre \( 1\) est bien une valeur propre de \( A\), mais le système
    \begin{equation}
        A\begin{pmatrix}
            x    \\
            y
        \end{pmatrix}=\begin{pmatrix}
            x    \\
            y
        \end{pmatrix}
    \end{equation}
    donne \( y=0\) et donc un espace propre de dimension seulement \( 1\).
\end{remark}

\begin{remark}  \label{RemXFZTooXkGzQg}
    Si \( \eK\) est algébriquement clos (comme \( \eC\) par exemple), alors tous les polynômes sont scindés et toutes les matrices sont trigonalisables\footnote{La proposition~\ref{PropKNVFooQflQsJ} montre cela, et le lemme de Schur complexe~\ref{LemSchurComplHAftTq} va un peu plus loin et précise que la trigonalisation peut être faite par une matrice unitaire.}. Un exemple un peu simple de cela est la matrice
    \begin{equation}
        u=\begin{pmatrix}
            0    &   -1    \\
            1    &   0
        \end{pmatrix}.
    \end{equation}
    Le polynôme caractéristique est \( \chi_u(X)=X^2+1\) et les valeurs propres sont \( \pm i\). Il est vite vu que dans la base
    \begin{equation}
        \{ \begin{pmatrix}
        i    \\
    1
\end{pmatrix}, \begin{pmatrix}
1    \\
i
\end{pmatrix}\}
    \end{equation}
    de \( \eC^2\), la matrice \( u\) se note \( \begin{pmatrix}
        i    &   0    \\
        0    &   -i
    \end{pmatrix}\).
\end{remark}

\begin{remark}  \label{RemREOSooGEDJWX}
    Cela nous donne une autre façon de prouver qu'une matrice nilpotente de \( \eM(n,\eC)\) ou \( \eM(n,\eR)\) est trigonalisable\cite{KDUFooVxwqlC}. D'abord dans \( \eM(n,\eC)\), toutes les matrices sont trigonalisables\footnote{Parce que le polynôme caractéristique est scindé, voir la proposition~\ref{PropKNVFooQflQsJ}..}, et les valeurs propres arrivent sur la diagonale. Mais comme les valeurs propres d'une matrice nilpotente sont zéro, elle est triangulaire stricte. Par ailleurs son polynôme caractéristique est alors \( X^n\).

    Ensuite si \( u\in \eM(n,\eR)\) nous pouvons voir \( u\) comme une matrice dans \( \eM(n,\eC)\) et y calculer son polynôme caractéristique qui sera tout de même \( X^n\). Ce polynôme étant scindé, la proposition~\ref{PropKNVFooQflQsJ} nous assure que \( u\) est trigonalisable. Une fois de plus, les valeurs propres étant sur la diagonale, elle est triangulaire supérieure stricte.
\end{remark}

\begin{corollary}   \label{CorUNZooAZULXT}
    Le polynôme caractéristique\footnote{Définition~\ref{DefOWQooXbybYD}.} sur \( \eC\) d'une matrice s'écrit sous la forme
    \begin{equation}
        \chi_A(X)=\prod_{i=1}^r(X-\lambda_i)^{m_i}
    \end{equation}
    où les \( \lambda_i\) sont les valeurs propres distinctes de \( A\) et \( m_i\) sont les multiplicités correspondantes.
\end{corollary}
\index{polynôme!caractéristique}

\begin{proof}
    Le lemme~\ref{LemSchurComplHAftTq} nous donne l'existence d'une base de trigonalisation; dans cette base les valeurs propres de \( A\) sont sur la diagonale et nous avons
    \begin{equation}
        \chi_A(X)=\det(A-X\mtu)=\det\begin{pmatrix}
            X-\lambda_1    &   *    &   *    \\
            0    &   \ddots    &   *    \\
            0    &   0    &   X-\lambda_r
        \end{pmatrix},
    \end{equation}
    qui vaut bien le produit annoncé.
\end{proof}

\begin{corollary}       \label{CORooTPDHooXazTuZ}
    Si \( A\in \eM(n,\eC)\) et \( k\in \eN\) alors
    \begin{equation}
        \Spec(A^k)=\{ \lambda^k\tq \lambda\in \Spec(A) \}.
    \end{equation}
\end{corollary}

\begin{proof}
    Par le lemme~\ref{LemSchurComplHAftTq} nous avons une matrice unitaire \( U\) et une triangulaire \( T\) telles que \( A=UTU^{-1}\). En passant à a puissance \( k\) nous avons aussi
    \begin{equation}
        A^k=UT^kU^{-1}.
    \end{equation}
    Donc le spectre de \( A^k\) est celui de \( T^k\) (lemme~\ref{LEMooRCFGooPPXiKi} et le fait qu'une puissance d'une matrice triangulaire est encore triangulaire). Or les éléments diagonaux de \( T^k\) sont les puissances \( k\)\ieme des éléments diagonaux de \( T\), qui sont les valeurs propres de \( A\).
\end{proof}

%---------------------------------------------------------------------------------------------------------------------------
\subsection{Diagonalisation : cas complexe, ce qu'on a}
%---------------------------------------------------------------------------------------------------------------------------

\begin{lemma}[Théorème spectral hermitien]      \label{LEMooVCEOooIXnTpp}
    Pour un opérateur hermitien\footnote{Définition~\ref{DEFooKEBHooWwCKRK}.},
    \begin{enumerate}
        \item
            le spectre est réel,
        \item
            deux vecteurs propres pour des valeurs propres distinctes sont orthogonales\footnote{Pour la forme \eqref{EqFormSesqQrjyPH}.}.
    \end{enumerate}
\end{lemma}
\index{spectre!matrice hermitienne}

\begin{proof}
    Soit \( v\) un vecteur de valeur propre \( \lambda\). Nous avons d'une part
    \begin{equation}
        \langle Av, v\rangle =\lambda\langle v, v\rangle =\lambda\| v \|^2,
    \end{equation}
    et d'autre part, en utilisant le fait que \( A\) est hermitien,
    \begin{equation}
        \langle Av, v\rangle =\langle v, A^*v\rangle =\langle v, Av\rangle =\bar\lambda\| v \|^2,
    \end{equation}
    par conséquent \( \lambda=\bar\lambda\) parce que \( v\neq 0\).

    Soient \( \lambda_i\) et \( v_i\) (\( i=1,2\)) deux valeurs propres de \( A\) avec leurs vecteurs propres correspondants. Alors d'une part
    \begin{equation}
        \langle Av_1, v_2\rangle =\lambda_1\langle v_1, v_2\rangle ,
    \end{equation}
    et d'autre part
    \begin{equation}
        \langle Av_1, v_2\rangle =\langle v_1, Av_2\rangle =\lambda_2\langle v_1, v_2\rangle .
    \end{equation}
    Nous avons utilisé le fait que \( \lambda_2\) était réel. Par conséquent, soit \( \lambda_1=\lambda_2\), soit \( \langle v_1, v_2\rangle =0\).
\end{proof}

\begin{remark}      \label{REMooMLBCooTuKFmz}
    Un opérateur de la forme \( A^*A\) est évidemment hermitien. De plus ses valeurs propres sont toutes positives parce que si \( A^*Ax=\lambda v\) alors
    \begin{equation}
        0\leq \langle Av, Av\rangle =\langle A^*Av, v\rangle =\lambda\langle v, v\rangle .
    \end{equation}
    Donc \( \lambda\geq 0\).
\end{remark}

\begin{definition}  \label{DefWQNooKEeJzv}
    Un endomorphisme est \defe{normal}{normal!endomorphisme}\index{matrice!normale} s'il commute avec son adjoint.
\end{definition}

Les opérateurs normaux comprennent évidemment les opérateurs hermitiens, mais également les anti-hermitiens, et ça c'est bien parce que c'est le cas de l'algèbre associée à \( \SU(2)\).

\begin{theorem}[Théorème spectral pour les matrices normales\footnote{Définition~\ref{DefWQNooKEeJzv}}\cite{LecLinAlgAllen,OMzxpxE,HOQzXCw}]\index{théorème!spectral!matrices normales}  \index{diagonalisation!cas complexe}  \label{ThogammwA}
    Soit \( A\in\eM(n,\eC)\) une matrice de valeurs propres \( \lambda_1,\ldots, \lambda_n\) (non spécialement distinctes). Alors les conditions suivantes sont équivalentes :
    \begin{enumerate}
        \item   \label{ItemJZhFPSi}
            \( A\) est normale,
        \item   \label{ItemJZhFPSii}
            \( A\) se diagonalise par une matrice unitaire,
        \item
            \( \sum_{i,j=1}^n| A_{ij} |^2=\sum_{j=1}^n| \lambda_j |^2\),
        \item
            il existe une base orthonormale de vecteurs propres de \( A\).
    \end{enumerate}
\end{theorem}

\begin{proof}
    Nous allons nous contenter de prouver~\ref{ItemJZhFPSi}\( \Leftrightarrow\)\ref{ItemJZhFPSii}.
    %TODO : le reste.

    Soit \( Q\) la matrice unitaire donnée par la décomposition de Schur (lemme~\ref{LemSchurComplHAftTq}) : \( A=QTQ^{-1}\). Étant donné que \( A\) est normale nous avons
    \begin{equation}
        QTT^*Q^{-1}=QT^*TQ^{-1},
    \end{equation}
    ce qui montre que \( T\) est également normale. Or une matrice triangulaire supérieure normale est diagonale. En effet nous avons \( T_{ij}=0\) lorsque \( i>j\) et
    \begin{equation}
        (TT^*)_{ii}=(T^*T)_{ii}=\sum_{k=1}^n| T_{ki} |^2=\sum_{k=1}^n| T_{ik} |^2.
    \end{equation}
    Écrivons cela pour \( i=1\) en tenant compte de \( | T_{k1} |^2=0\) pour \( k=2,\ldots, n\),
    \begin{equation}
        | T_{11} |^2=| T_{11} |^2+| T_{12} |^2+\cdots+| T_{1n} |^2,
    \end{equation}
    ce qui implique que \( T_{11}\) est le seul non nul parmi les \( T_{1k}\). En continuant de la sorte avec \( i=2,\ldots, n\) nous trouvons que \( T\) est diagonale.

    Dans l'autre sens, si \( A\) se diagonalise par une matrice unitaire, \( UAU^*=D\), nous avons
    \begin{equation}
        DD^*=UAA^*U^*
    \end{equation}
    et
    \begin{equation}
        D^*D=UA^*AU^*,
    \end{equation}
    qui ce prouve que \( A\) est normale.
\end{proof}

Tant que nous en sommes à parler de spectre de matrices hermitiennes\ldots Soit une matrice inversible \( A\in \GL(n,\eC)\). La matrice \( A^*A\) est hermitienne\footnote{Définition~\ref{DEFooKEBHooWwCKRK}.} et le théorème~\ref{LEMooVCEOooIXnTpp} nous assure que ses valeurs propres sont réelles. Par la remarque~\ref{REMooMLBCooTuKFmz}, ses valeurs propres sont même positives.

\begin{lemma}[\cite{ooLMMRooUXhOdx}]   \label{LEMooHUGEooVYhZdZ}
    Si \( A\) est une matrice carrée et inversible,
    \begin{equation}
        \Spec(A^*A)=\Spec(AA^*)
    \end{equation}
\end{lemma}

\begin{proof}
    Nous allons montrer l'égalité des polynômes caractéristiques. D'abord une simple multiplication montre que
    \begin{equation}
        (A^*A-\lambda\mtu)A^{-1}=A^{-1}(AA^*-\lambda\mtu).
    \end{equation}
    Nous prenons le déterminant de cette égalité en utilisant les propriétés~\ref{PropYQNMooZjlYlA}\ref{ItemUPLNooYZMRJy} et~\ref{ITEMooZMVXooLGjvCy} :
    \begin{equation}
        \det(A^*A-\lambda\mtu)\det(A^{-1})=\det(A^{-1})\det(AA^*-\lambda\mtu).
    \end{equation}
    En simplifiant par \( \det(A^{-1})\) (qui est non nul parce que \( A\) est inversible) nous obtenons l'égalité des polynômes caractéristiques et donc l'égalité des spectres.
\end{proof}

%---------------------------------------------------------------------------------------------------------------------------
\subsection{Diagonalisation : cas réel}
%---------------------------------------------------------------------------------------------------------------------------

\begin{lemma}[Lemme de Schur réel]  \label{LemSchureRelnrqfiy}
    Soit \( A\in\eM(n,\eR)\). Il existe une matrice orthogonale \( Q\) telle que \( Q^{-1}AQ\) soit de la forme
    \begin{equation}        \label{EqMtrTSqRTA}
        QAQ^{-1}=\begin{pmatrix}
            \lambda_1    &   *    &   *    &   *    &   *\\
            0    &   \ddots    &   \ddots    &   \ddots    &   \vdots\\
            0    &   0    &   \lambda_r    &   *    &   *\\
            0    &   0    &   0    &   \begin{pmatrix}
                a_1    &   b_1    \\
                c_1    &   d_1
            \end{pmatrix}&   *\\
            0    &   0    &  0     &   0    &   \begin{pmatrix}
                a_s    &   b_s    \\
                c_s    &   d_s
            \end{pmatrix}
        \end{pmatrix}.
    \end{equation}
    Le déterminant de \( A\) est le produit des déterminants des blocs diagonaux et les valeurs propres de \( A\) sont les \( \lambda_1,\ldots, \lambda_r\) et celles de ces blocs.
\end{lemma}
\index{lemme!Schur réel}

\begin{proof}
    Si la matrice \( A\) a des valeurs propres réelles, nous procédons comme dans le cas complexe. Cela nous fournit le partie véritablement triangulaire avec les valeurs propres \( \lambda_1,\ldots, \lambda_r\) sur la diagonale. Supposons donc que \( A\) n'a pas de valeurs propres réelles. Soit donc \( \alpha+i\beta \) une valeur propre (\( \beta\neq 0\)) et \( u+iv\) un vecteur propre correspondant où \( u\) et \( v\) sont des vecteurs réels. Nous avons
    \begin{equation}
        Au+iAv=A(u+iv)=(\alpha+i\beta)(u+iv)=\alpha u-\beta v+i(\alpha v+\beta v),
    \end{equation}
    et en égalisant les parties réelles et imaginaires,
    \begin{subequations}
        \begin{align}
            Au&=\alpha u-\beta v\\
            Av&=\alpha v+\beta u.
        \end{align}
    \end{subequations}
    Sur ces relations nous voyons que ni \( u\) ni \( v\) ne sont nuls. De plus \( u\) et \( v\) sont linéairement indépendants (sur \( \eR\)), en effet si \( v=\lambda u\) nous aurions \( Au=\alpha u-\beta\lambda u=(\alpha-\beta\lambda)u\), ce qui serait une valeur propre réelle alors que nous avions supposé avoir déjà épuisé toutes les valeurs propres réelles.

    Étant donné que \( u\) et \( v\) sont deux vecteurs réels non nuls et linéairement indépendants, nous pouvons trouver une base orthonormée \( \{ q_1,q_2 \}\) de \( \Span\{ u,v \}\). Nous pouvons étendre ces deux vecteurs en une base orthonormée \( \{ q_1,q_2,q_3,\ldots, q_n \}\) de \( \eR^n\). Nous considérons à présent la matrice orthogonale dont les colonnes sont formées de ces vecteurs : \( Q=[q_1\,q_2\,\ldots q_n]\).

    L'espace \( \Span\{ e_1,e_2 \}\) est stable par \( Q^{-1} AQ\), en effet nous avons
    \begin{equation}
        Q^{-1} AQe_1=Q^{-1} Aq_1=Q^{-1}(aq_1+bq_2)=ae_1+be_2.
    \end{equation}
    La matrice \( Q^{-1}AQ\) est donc de la forme
    \begin{equation}
        Q^{-1} AQ=\begin{pmatrix}
            \begin{pmatrix}
                \cdot    &   \cdot    \\
                \cdot    &   \cdot
            \end{pmatrix}&   C_1    \\
            0    &   A_1
        \end{pmatrix}
    \end{equation}
    où \( C_1\) est une matrice réelle \( 2\times (n-1)\) quelconque et \( A_1\) est une matrice réelle \( (n-2)\times (n-2)\). Nous pouvons appliquer une récurrence sur la dimension pour poursuivre.

    Notons que si \( A\) n'a pas de valeurs propres réelles, elle est automatiquement d'ordre pair parce que les valeurs propres complexes viennent par couple complexes conjuguées.

    En ce qui concerne les valeurs propres, il est facile de voir en regardant \eqref{EqMtrTSqRTA} que les valeurs propres sont celles des blocs diagonaux. Étant donné que \( QAQ^{-1}\) et \( A\) ont même polynôme caractéristique, ce sont les valeurs propres de \( A\).
\end{proof}

\begin{theorem}[Théorème spectral, matrice symétrique\cite{KXjFWKA}] \label{ThoeTMXla}
    Une matrice symétrique réelle,
    \begin{enumerate}
        \item       \label{ITEMooJWHLooSfhNSW}
            a un spectre contenu dans \( \eR\)
        \item       \label{ITEMooMWWRooXxGONW}
            est diagonalisable par une matrice orthogonale.
    \end{enumerate}
    Si \( M\) est une matrice symétrique réelle alors \( \eR^n\) possède une base orthonormée de vecteurs propres de \( M\).
\end{theorem}
\index{diagonalisation!cas réel}
\index{rang!diagonalisation}
\index{endomorphisme!diagonalisation}
\index{spectre!matrice symétrique réelle}
\index{théorème!spectral!matrice symétrique}

\begin{proof}
    Soit \( A\) une matrice réelle symétrique. Si \( \lambda\) est une valeur propre complexe pour le vecteur propre complexe \( v\), alors d'une part \( \langle Av, v\rangle =\lambda\langle v, v\rangle \) et d'autre part \( \langle Av, v\rangle =\langle v, Av\rangle =\bar\lambda\langle v, v\rangle \). Par conséquent \( \lambda=\bar\lambda\).

    Le lemme de Schur réel~\ref{LemSchureRelnrqfiy} donne une matrice orthogonale qui trigonalise \( A\). Les valeurs propres étant toutes réelles, la matrice \( QAQ^{-1}\) est même triangulaire (il n'y a pas de blocs dans la forme \eqref{EqMtrTSqRTA}). Prouvons que \( QAQ^{-1}\) est symétrique :
    \begin{equation}
        (QAQ^{-1})^t=(Q^{-1})^tA^tQ^t=QA^tQ^{-1}=QAQ^{-1}
    \end{equation}
    où nous avons utilisé le fait que \( Q\) était orthogonale (\( Q^{-1}=Q^t\)) et que \( A\) était symétrique (\( A^t=A\)). Une matrice triangulaire supérieure symétrique est obligatoirement une matrice diagonale.

    En ce qui concerne la base de vecteurs propres, soit \( \{ e_i \}_{i=1,\ldots, n}\) la base canonique de \( \eR^n\) et \( Q\) une matrice orthogonale e telle que \( A=Q^tDQ\) avec \( D\) diagonale. Nous posons \( f_i=Q^te_i\) et en tenant compte du fait que \( Q^t=Q^{-1}\) nous avons \( Af_i=Q^tDQQ^te_i=Q^t\lambda_i e_i=\lambda_if_i\). Donc les \( f_i\) sont des vecteurs propres de \( A\). De plus ils sont orthonormés parce que
    \begin{equation}
        \langle f_i, f_j\rangle =\langle Q^te_i, Q^te_j\rangle =\langle e_i, Q^tQe_j\rangle =\langle e_i, e_j\rangle =\delta_{ij}.
    \end{equation}
\end{proof}
Le théorème spectral pour les opérateurs autoadjoints sera traité plus bas parce qu'il a besoin de choses sur les formes bilinéaires, théorème~\ref{ThoRSBahHH}.
% et les choses sur la dégénérescences utilisent le théorème spectral, cas réel. Donc l'enchainement est très loumapotiste.

\begin{remark}  \label{RemGKDZfxu}
    Une matrice symétrique est diagonalisable par une matrice orthogonale. Nous pouvons en réalité nous arranger pour diagonaliser par une matrice de \( \SO(n)\). Plus généralement si \( A\) est une matrice diagonalisable par une matrice \( P\in\GL^+(n,\eR)\) alors elle est diagonalisable par une matrice de \( \GL^-(n,\eR)\) en changeant le signe de la première ligne de \( P\). Et inversement.

    En effet, si nous avons \( P^tDP=A\), alors en notant \( *\) les quantités qui ne dépendent pas de \( a\), \( b\) ou~\( c\),
    \begin{equation}
        \begin{aligned}[]
        \begin{pmatrix}
            a    &   *    &   *    \\
            b    &   *    &   *    \\
            c    &   *    &   *
        \end{pmatrix}
        \begin{pmatrix}
            \lambda_1    &       &       \\
                &   \lambda_2    &       \\
                &       &   \lambda_3
            \end{pmatrix}
            \begin{pmatrix}
                a    &   b    &   c    \\
                *    &   *    &   *    \\
                *    &   *    &   *
            \end{pmatrix}&=
        \begin{pmatrix}
            a    &   *    &   *    \\
            b    &   *    &   *    \\
            c    &   *    &   *
        \end{pmatrix}
        \begin{pmatrix}
            \lambda_1a    &   \lambda_1b    &   \lambda_1c    \\
            *    &   *    &   *    \\
            *    &   *    &   *
        \end{pmatrix}\\
        &=\begin{pmatrix}
            \lambda_1 a^2+*   &   \lambda_1ab+*    &   \lambda_1ac  +*  \\
            \ldots    &   \ldots    &   \ldots    \\
            \ldots    &   \ldots    &   \ldots
        \end{pmatrix}.
        \end{aligned}
    \end{equation}
    Nous voyons donc que si nous changeons les signes de \( a\), \( b\) et \( c\) en même temps, le résultat ne change pas.
\end{remark}

\begin{definition}[Matrice définie positive, opérateur définit positif]    \label{DefAWAooCMPuVM}
    Un opérateur sur un espace vectoriel sur \( \eC\) ou \( \eR\) est \defe{définit positif}{opérateur!définit positif} si toutes ses valeurs propres sont réelles et strictement positives.  Il est \defe{semi-définie positive}{semi-définie positive} si ses valeurs propres sont réelles positives ou nulles.
\end{definition}
Afin d'éviter l'une ou l'autre confusion, nous disons souvent \emph{strictement} définie positive pour positive.

\begin{normaltext}      \label{NORMooAJLHooQhwpvr}
    Nous nommons \( S^+(n,\eR)\) l'ensemble des matrices réelles symétriques \( n\times n\) et \( S^{++}(n,\eR)\) le sous-ensemble de \( S^+(n,\eR)\) des matrices strictement définies positives.
\end{normaltext}
    \nomenclature[B]{$ S^+(n,\eR)$}{matrices symétriques définies positives}
    \nomenclature[B]{$ S^{++}(n,\eR)$}{matrices symétriques strictement définies positives}

\begin{remark}
    Nous ne définissons pas la notion de matrice définie positive pour une matrice non symétrique.
\end{remark}

\begin{proposition}     \label{PropcnJyXZ}
    Soit $M$, une matrice symétrique. Nous avons
    \begin{enumerate}
        \item       \label{ITEMooTJVQooYmRkas}
            $\det M>0$ et $\tr(M)>0$ implique $M$ définie positive\footnote{Défintion~\ref{DefAWAooCMPuVM}.},
        \item
        $\det M>0$ et $\tr(M)<0$ implique $M$ définie négative,
    \item   \label{ItemluuFPN}
        $\det M<0$ implique ni semi-définie positive, ni définie négative
        \item
        $\det M=0$ implique $M$ semi-définie positive ou semi-définie négative.
    \end{enumerate}
\end{proposition}

\begin{proposition}     \label{PROPooUAAFooEGVDRC}
    Une application linéaire est définie positive\footnote{Définition~\ref{DefAWAooCMPuVM}.} si et seulement si sa matrice associée l'est.
\end{proposition}

Lorsqu'un énoncé parle d'une matrice symétrique, le premier réflexe est de la diagonaliser : considérer une matrice orthogonale \( T\) telle que \( T^tMT=D\) avec \( D\) diagonale. Et les valeurs propres sur la diagonale : \( D_{kl}=\delta_{kl}\lambda_k\). Les matrices symétriques définies positives ont cependant des propriétés même en dehors de leur base de diagonalisation.

\begin{lemma}   \label{LemWZFSooYvksjw}
    Soit une matrice symétrique \( M\).
    \begin{enumerate}
        \item       \label{ITEMooSKRAooOgHbGA}
           Elle est strictement définie positive si et seulement si \( \langle x, Mx\rangle >0\) pour tout \( x\) non nul dans \( \eR^n\).
        \item       \label{ITEMooMOZYooWcrewZ}
           Elle est semi-définie positive si et seulement si \( \langle x, Mx\rangle \geq 0\) pour tout \( x\) non nul dans \( \eR^n\).
       \item        \label{ITEMooRRMFooHSOHxZ}
           Si elle est seulement définie positive, alors \( \langle x, Mx\rangle \geq \lambda\| x \|^2\) dès que \( \lambda\geq 0\) minore toutes les valeurs propres.
    \end{enumerate}
\end{lemma}

\begin{proof}
    Démonstration en trois parties.
    \begin{subproof}
    \item[\ref{ITEMooSKRAooOgHbGA}]
    Soit \( \{ e_i \}_{i=1,\ldots, n}\) une base orthonormée de vecteurs propres de \( M\) dont l'existence est assurée par le théorème spectral~\ref{ThoeTMXla}. Nous nommons \( x_i\) les coordonnées de \( x\) dans cette base. Alors,
    \begin{equation}
        \langle x,Mx \rangle =\sum_{i,j}x_i\langle e_i, x_jMe_j\rangle =\sum_{i,j}x_ix_j\langle e_i, \lambda_je_j\rangle =\sum_{ij}x_ix_j\lambda_j\delta_{ij}=\sum_i\lambda_ix_i^2
    \end{equation}
    où les \( \lambda_i\) sont les valeurs propres de \( M\). Cela est strictement positif pour tout \( x\) si et seulement si tous les \( \lambda_i\) sont strictement positifs.
\item[\ref{ITEMooMOZYooWcrewZ}]

    Nous avons encore
    \begin{equation}
        \langle x, Mx\rangle =\sum_{i}\lambda_ix_i^2.
    \end{equation}
    Cela est plus grand ou égal à zéro si et seulement si tous les \( \lambda_i\) sont plus grands ou égaux à zéro.

\item[\ref{ITEMooRRMFooHSOHxZ}]

        Soit une matrice orthogonale \( T\) diagonalisant \( M\), c'est-à-dire telle que \( T^tMT=D\) avec \( D\) diagonale. Nous allons vérifier que
        \begin{equation}
            \langle Tx, Mtx\rangle \geq \lambda\| Tx \|^2
        \end{equation}
        pour tout \( x\). Vu que \( T\) est une bijection \footnote{Une matrice orthogonale a un déterminant $\pm 1$.}, cela impliquera le résultat pour tout \( x\). Si nous considérons la base de diagonalisation \( \{ e_k \}\) pour les valeurs propres \( \lambda_k\), nous avons le calcul
       \begin{subequations}
            \begin{align}
                \langle Tx, MTx\rangle &=\langle x, T^tMTx\rangle \\
                &=\langle x, Dx\rangle \\
                &=\sum_k\langle x, x_kDe_k\rangle \\
                &=\sum_k\lambda_kx_k \underbrace{\langle x, e_k\rangle }_{=x_k}\\
                &\geq \sum_k\lambda| x_k |^2\\
                &=\lambda\| x \|^2\\
                &=\lambda\| Tx \|^2.
            \end{align}
        \end{subequations}
        Au dernier passage nous avons utilisé le fait que \( T\) est une isométrie (proposition~\ref{PropKBCXooOuEZcS}).
    \end{subproof}
\end{proof}

Les personnes qui aiment les vecteurs lignes et colonnes écriront des inégalités comme
\begin{equation}
    x^tMx\geq x^tx.
\end{equation}
Tout à l'autre bout du spectre des personnes névrosées des notations, on trouvera des inégalités comme
\begin{equation}
    M(x\otimes x)\geq x\cdot x.
\end{equation}
Le penchant personnel de l'auteur de ces lignes est la notation avec le produit tensoriel. Si vous aimez ça, vous pouvez lire la section \ref{SECooUKRYooZjagcX} et en particulier ce qui suit \eqref{EQooUNRYooKBrXyK}.

La notation adoptée ici avec le produit scalaire \( \langle x, Mx\rangle \) est entre les deux. Elle a l'avantage de n'être pas technologique comme le produit tensoriel (si vous y mettez les pieds, vous devez savoir ce que vous faites), tout en évitant de se casser la tête à savoir qui est un vecteur ligne ou un vecteur colonne.

\begin{corollary}
    Une matrice symétrique strictement définie positive est inversible.
\end{corollary}

\begin{proof}
    Si \( Ax=0\) alors \( \langle Ax, x\rangle =0\). Mais dans le cas d'une matrice strictement définie positive, cela implique \( x=0\) par le lemme~\ref{LemWZFSooYvksjw}.
\end{proof}

\begin{lemma}
    Pour une base quelconque, les éléments diagonaux d'une matrice symétrique semi-définie positive sont positifs. Si la matrice est strictement définie positive, alors les éléments diagonaux sont strictement positifs.
\end{lemma}

\begin{proof}
    Il s'agit d'une application du lemme~\ref{LemWZFSooYvksjw}. Si \( A\) est définie positive et que \( \{ e_i \}\) est une base, alors
    \begin{equation}
        A_{ii}=\langle Ae_i, e_i\rangle \geq \lambda\| e_i \|^2=\lambda\geq 0.
    \end{equation}
    Si \( A\) est strictement définie positive, alors \( \lambda\) peut être choisi strictement positif.
\end{proof}

%+++++++++++++++++++++++++++++++++++++++++++++++++++++++++++++++++++++++++++++++++++++++++++++++++++++++++++++++++++++++++++
\section{Formes bilinéaires et quadratiques}
%+++++++++++++++++++++++++++++++++++++++++++++++++++++++++++++++++++++++++++++++++++++++++++++++++++++++++++++++++++++++++++

Plus à propos de formes bilinéaires dans le thème \ref{THEMEooOAJKooEvcCVn}.

%---------------------------------------------------------------------------------------------------------------------------
\subsection{Généralités}
%---------------------------------------------------------------------------------------------------------------------------

\begin{definition}[\cite{RUAoonJAym}]   \label{DefBSIoouvuKR}
    Soit un espace vectoriel \( E\) et \( \eF\) un corps de caractéristique différente de \( 2\). Une \defe{forme quadratique}{forme!quadratique} sur \( E\) est une application \( q\colon E\to \eF\) pour laquelle il existe une forme bilinéaire symétrique \( b\colon E\times E\to \eF\) satisfaisant \( q(x)=b(x,x)\) pour tout \( x\in E\).

    L'ensemble des formes quadratiques réelles sur \( E\) est noté \( Q(E)\)\nomenclature[B]{\( Q(E)\)}{formes quadratiques réelles sur \( E\)}.
\end{definition}

\begin{proposition} \label{PROPooZLXVooOsXCcB}
    Soit une forme bilinéaire \( b\) et la forme quadratique associée \( q\). Alors nous avons l'\defe{identité de polarisation}{identité de polarisation} :
    \begin{equation}    \label{EqMrbsop}
        b(x,y)=\frac{ 1 }{2}\big( q(x)+q(y)-q(x-y) \big).
    \end{equation}
\end{proposition}

\begin{proof}
    Il suffit de substituer dans le membre de droite \( q(x)=b(x,y)\) et d'utiliser la bilinéarité :
    \begin{subequations}
        \begin{align}
            q(x)+q(y)-q(x-y)&=b(x,x)+b(y,y)-b(x-y,x-y)\\
            &=b(x,x)+b(y,y)-b(x)+b(x,y)+b(y,x)-b(y,y)\\
            &=2b(x,y)
        \end{align}
    \end{subequations}
    où nous avons utilisé le fait que \( b\) est symétrique : \( b(x,y)=b(y,x)\).
\end{proof}

\begin{lemma}       \label{LEMooLKNTooSfLSHt}
    Si \( q\) est une forme quadratique, il existe une unique forme bilinéaire \( b\) telle que \( q(x)=b(x,x)\).
\end{lemma}

\begin{proof}
    L'existence n'est pas en cause : c'est la définition d'une forme quadratique. Pour l'unicité, étant donné une forme quadratique, la forme bilinéaire \( b\) doit forcément vérifier l'identité de polarisation de la proposition \ref{PROPooZLXVooOsXCcB}. Elle est donc déterminée par \( q\).
\end{proof}
Notons la division par \( 2\) qui est le pourquoi de la demande de la caractéristique différente de \( 2\) pour \( \eF\) dans la définition de forme quadratique.

%---------------------------------------------------------------------------------------------------------------------------
\subsection{Matrice associée à une forme bilinéaire}
%---------------------------------------------------------------------------------------------------------------------------

\begin{definition}      \label{DEFooAOGPooXWXUcN}
    Soit une forme bilinéaire \( b\colon E\times E\to \eK\) et une base quelconque \( \{ f_{\alpha} \}\) de \( E\). Nous définissons les nombres
    \begin{equation}    \label{EQooCUGFooRlKUtu}
        B_{\alpha\beta}=b(f_{\alpha},f_{\beta}),
    \end{equation}
    qui forment une matrice symétrique dans \( \eM(n,\eK)\). Cette matrice est la \defe{matrice associée}{matrice d'une forme bilinéaire} à la forme bilinéaire \( b\).

    La matrice d'une forme quadratique est celle associée à sa forme bilinéaire associée.
\end{definition}

\begin{lemma}       \label{LEMooDCIOooTlVZMR}
    Soit une forme bilinéaire \( b\colon E\times E\to \eK\) et une base quelconque \( \{ f_{\alpha} \}\) de \( E\). Nous notons \( B\) la matrice de \( b\) (definition \ref{DEFooAOGPooXWXUcN}) et \( q\) la forme quadratique associée.

    Alors nous avons
\begin{equation}        \label{EQooQFMWooVKVLMx}
    b(x,y)=\sum_{\alpha\beta}B_{\alpha\beta}x_{\alpha}y_{\beta}.
\end{equation}
et
\begin{equation}
    b(x,y)=\sum_{\alpha}x_{\alpha}\sum_{\beta}B_{\alpha\beta}y_{\beta}=\sum_{\alpha}x_{\alpha}(By)_{\alpha}=x\cdot By.
\end{equation}
où le point est le produit scalaire usuel (composante par composante).
\end{lemma}

\begin{proof}
    Si \( x=\sum_{\alpha}x_{\alpha}f_{\alpha}\) et \( y=\sum_{\beta}y_{\beta}f_{\beta}\) :

    En utilisant la convention \eqref{EQooAXRJooUwHbjB} et les choses autour (voir aussi \ref{SECooBTTTooZZABWA}),
    \begin{equation}
        b(x,y)=\sum_{\alpha}x_{\alpha}\sum_{\beta}B_{\alpha\beta}y_{\beta}=\sum_{\alpha}x_{\alpha}(By)_{\alpha}=x\cdot By.
    \end{equation}
\end{proof}

\begin{proposition}     \label{PROPooCIEUooODqfwm}
    Soit une forme quadratique \( q\colon E\to \eK\) et sa matrice\footnote{Matrice associée à une forme quadratique, définition \ref{DEFooAOGPooXWXUcN}.} \( (q_{ij})\in \eM(n,\eK)\). Nous avons
    \begin{subequations}        \label{SUBEQSooEHVXooJjKLqyiB}
        \begin{align}
            q(x)&=\sum_{i=1}^n\sum_{j=1}^nq_{ij}x_ix_j\\
            &=\sum_{i=1}^nq_{ii}x_i^2+2\sum_{1\leq i <j\leq n}q_{ij}x_ix_j.
        \end{align}
    \end{subequations}
\end{proposition}

\begin{normaltext}
    De nombreux auteurs préfèrent écrire des choses comme \( x^tBy\) ou \( xB^ty\) ou \( xBy^t\) et se poser de longues questions sur qui est un «vecteur colonne» et qui est un «vecteur ligne», et si la matrice \( B\) soit être transposée ou non. Toutes ces notations servent(?) à cacher un bête produit scalaire.
\end{normaltext}

\begin{normaltext}
    Notons que la matrice associée à une forme bilinéaire (ou quadratique associée) est uniquement valable pour une base donnée. Si nous changeons de base, la matrice change. Cependant lorsque nous travaillons sur \( \eR^n\), la base canonique est tellement canonique que nous allons nous permettre de parler de «la» matrice associée à une forme bilinéaire.
\end{normaltext}

%---------------------------------------------------------------------------------------------------------------------------
\subsection{Changement de base : matrice d'une forme bilinéaire}
%---------------------------------------------------------------------------------------------------------------------------

\begin{proposition}[Voir la section \ref{SECooBTTTooZZABWA}]     \label{PROPooLBIOooUpzxXA}
    
    Soit une forme bilinéaire\footnote{Définition~\ref{DEFooEEQGooNiPjHz}} \( b\colon V\times V\to \eK\) dont la matrice\footnote{Définition~\ref{EQooCUGFooRlKUtu}.} dans la base \( \{ e_i \}\) est \( A\) et celle dans la base \( \{ f_{\alpha} \}\) est \( B\). Nous supposons que les bases sont liées par \( f_{\alpha}=\sum_{i}Q_{i\alpha}e_i\). Alors
\begin{equation}        \label{EQooZUVTooKjqnJj}
    B=Q^tAQ.
\end{equation}
\end{proposition}

\begin{proof}
    Soit \( x,x'\in V\) de coordonnées \( (x_i)\) et \( (x'_i)\) dans la base \( \{ e_i \}\) et \( (y_{\alpha})\), \( (y'_{\alpha})\) dans la base \( \{ f_{\alpha} \}\). Par définition de la matrice associée à une forme bilinéaire,
    \begin{equation}
        b(x,x')=\sum_{ij}A_{ij}x_ix'_j=\sum_{\alpha\beta}B_{\alpha\beta}y_{\alpha}y'_{\beta}.
    \end{equation}
    En remplaçant les \( x_i\) et \( x'_i\) par leurs valeurs en fonction de \( y_{\alpha}\) et \( y'_{\beta}\) données par la proposition \ref{PROPooNYYOooHqHryX}, nous trouvons
    \begin{subequations}
        \begin{align}
            b(x,x')&=\sum_{ij\alpha\beta}A_{ij}Q_{i\alpha}y_{\alpha}Q_{j\beta}y'_{\beta}\\
            &=\sum_{\alpha\beta}(Q^tAQ)_{\alpha\beta}y_{\alpha}y'_{\beta}
        \end{align}
    \end{subequations}
    où \( Q^t\) désigne la transposée de la matrice \( Q\) :  \( Q^t_{ij}=Q_{ji}\). Vu que les nombres \( y_{\alpha}\) et \( y'_{\beta}\) sont arbitraires nous déduisons\footnote{Lemme~\ref{LEMooLXAHooPRyHaF}.} que \( B=Q^tAQ\).
\end{proof}

\begin{remark}      \label{REMooNEJLooSqgeih}
    Notons que cette «loi de transformation» n'est pas la même que celle pour une application linéaire\footnote{Proposition \ref{PROPooNZBEooWyCXTw}.}. Ici nous avons \( Q^t\) alors que pour les applications linéaires nous avions \( Q^{-1}\).

    Pour cette raison, tant que nous travaillons avec des bases orthonormées, c'est-à-dire tant que \( Q\) est orthogonale\footnote{Définition~\ref{DefMatriceOrthogonale}.}, nous pouvons confondre une application linéaire avec une application bilinéaire en passant par la matrice. Mais cette identification n'est pas du tout canonique : elle repose sur le fait que les bases soient orthonormées.

    Il en découle que la réduction des endomorphismes et la réduction des formes bilinéaires ne sont pas tout à fait les mêmes théories. Par exemple la pseudo-diagonalisation simultanée (corolaire~\ref{CorNHKnLVA}) est un résultat de réduction de forme bilinéaire et non d'endomorphismes.
\end{remark}


%--------------------------------------------------------------------------------------------------------------------------- 
\subsection{Réduction de Gauss}
%---------------------------------------------------------------------------------------------------------------------------

\begin{theorem}[Réduction de Gauss\cite{BIBooNUUEooJUjLpy,BIBooUULNooUtlrar}]     \label{THOooOMMFooKxqICS}
    Soit une forme quadratique non nulle \( q\) sur l'espace vectoriel \( E\) sur le corps \( \eK\). Il existe des formes linéaires linéairement indépendantes \(\{ l_i \}_{i=1,\ldots, n}\) et des coefficients \( \alpha_i\in \eK\) tels que 
        \begin{equation}
            q(x)=\sum_{i=1}^n\alpha_il_i(x)^2.
        \end{equation}
\end{theorem}

\begin{proof}
    Notre point de départ sont les formules \eqref{SUBEQSooEHVXooJjKLqyiB} pour la forme quadratique. Nous allons faire la preuve par récurrence sur la dimension de l'espace. Si \( n=1\), alors nous avons seulement
    \begin{equation}
        q(x)=\alpha x^2
    \end{equation}
    et donc le théorème est fait avec \( l(x)=x\).

    Nous supposons que le théorème est prouvé pour tout espace de dimension \( n\). Une forme quadratique pour un espace de dimension \( n+1\) s'écrit
    \begin{equation}
        q(x)=\sum_{i=1}^{n+1}m_{ii}x_i^2+2\sum_{1\leq i < j\leq n+1}m_{ij}x_ix_j.
    \end{equation}
    Vu que \( q\) est non nulle, un des \( m_{ij}\) est non nul. Nous allons diviser en plusieurs cas.
    \begin{itemize}
        \item
            \( m_{11}\neq 0\)
        \item
            \( m_{kk}\neq 0\) avec \( k\neq 1\)
        \item
            \( m_{12}\neq 0\) et \( m_{ii}=0\) pour tout \( i\).
        \item
            \( m_{kl}\neq 0\) avec \( (k,l)\neq (1,2)\) et \( m_{ii}=0\) pour tout \( i\).
    \end{itemize}
    Ces cas ne sont pas exclusifs, mais ils couvrent toutes les possibilités.

    \begin{subproof}
        \item[Si \( m_{11}\neq 0\)]
            Nous écrivons \( q\) sous la forme
            \begin{subequations}
                \begin{align}
                    q(x)&=m_{11}x_1^2+\sum_{i=2}^{n+1}m_{ii}x_i^2+2\sum_{i=1}^n\big( \sum_{j=i+1}^{n+1}m_{ij}x_ix_j \big)\\
                    &=m_{11}x_1^2+\sum_{i=2}^{n+1}m_{ii}x_i^2+2\sum_{j=2}^{n+1}m_{1j}x_1x_j+2\sum_{i=2}^n\sum_{j=i+1}^{n+1}(m_{ij}x_ix_j)\\
                    &=m_{11}x_1^2+2x_1\sum_{j=2}^{n+1}m_{1j}x_k+R(x_2,\ldots, x_{n+1})\\
                    &=m_{11}\left( x_1^2+2x_1\sum_{j=2}^{n+1}\frac{ m_{1j} }{ m_{11} }x_j \right)+R(x_2,\ldots, x_{n+1})\\
                    &=m_{11}\big( x_1^2+2x_1f(x_2,\ldots, x_{n+1}) \big)+R(x_2,\ldots, x_{n+1})\\
                    &=m_{11}\big( x_1+f(x_2,\ldots, x_{n+1}) \big)^2-f(x_2,\ldots, x_{n+1})+R(x_2,\ldots, x_{n+1})
                \end{align}
            \end{subequations}
            où 
            \begin{itemize}
                \item \( R\) est une forme quadratique de \( n-1\) variables;
                \item nous avons noté \( f(x_2,\ldots, x_{n+1})=\sum_{j=2}^{n+1}\frac{ m_{1j} }{ m_{11} }x_j\).
            \end{itemize}
            Maintenant, toute la partie \( -f(x_2,\ldots, x_{n+1})^2+R(x_2,\ldots, x_{n+1})\) est une forme quadratique de \( n\) variables. Par hypothèse de récurrence, il existe des coefficients \( \alpha_i\) et des formes linéairement indépendantes sur \( \eK^n\) \( l_i'(x_2,\ldots, x_{n+1})\) telles que
            \begin{equation}
                -f(x_2,\ldots, x_{n+1})^2+R(x_2,\ldots, x_{n+1})=\sum_{i=2}^{n+1}\alpha_il_i'(x_2,\ldots, x_{n+1})^2.
            \end{equation}
            En posant ensuite \( l_j(x_1,\ldots, x_{n+1})=l'_j(x_2,\ldots, x_{n+1})\), ainsi que \( l_1(x_1,\ldots, x_{n+1})=x_1+f(x_2,\ldots, x_{n+1})\), nous avons
            \begin{equation}
                q(x)=m_{11}l_1(x)^2+\sum_{j=2}^{n+1}\alpha_jl_j(x)^2.
            \end{equation}
            
        \item[Si \( m_{kk}\neq 0\) avec \( k\neq 1\)]

            Nous nommons \( k\) le plus petit entier pour lequel \( m_{kk}\neq 0\), et nous supposons que \( k\neq 1\), parce que nous avons déjà couvert ce cas. Dans ce cas, nous avons
            \begin{equation}
                q(x)=m_{kk}x_k^2+\sum_{j=k+1}^{n+1}m_{jj}x_j^2  +2\sum_{i=1}^n\big( \sum_{j=i+1}^{n+1}m_{ij}x_ix_j \big),
            \end{equation}
            et tout tourne comme dans le premier cas.
        \item[\( m_{ii}=0\) pour tout \( i\) et \( m_{12}\neq 0\)]
            Nous écrivons \( q\) en séparant les termes \( m_{1k}\) :
            \begin{subequations}
                \begin{align}
                    q(x)&=2\sum_{1\leq i<j\leq n+1}m_{ij}x_ix_j\\
                    &=2m_{12}x_1x_2+2\sum_{2\leq j\leq n+1}m_{1j}x_1x_j+2\sum_{2\leq i<j\leq n+1}m_{ij}x_ix_j\\
                    &=2m_{12}x_1x_2+2x_1\sum_{2\leq j\leq n+1}m_{1j}x_j+2\sum_{3\leq j\leq n+1}m_{2j}x_2x_j+2\sum_{3\leq i<j\leq n+1}m_{ij}x_ix_j\\
                    &=2m_{12}x_1x_2+x_1f(x_2,\ldots, x_{n+1})+x_2g(x_3,\ldots, x_{n+1})+T(x_3,\ldots, x_{n+1})      \label{SUBEQooLBXBooXoLyuw}
                \end{align}
            \end{subequations}
            où \( f\) et \( g\) sont linéaires et \( T\) est multilinéaire.

            À ce moment, nous tentons de factoriser toute la partie concernant \( x_1\) et \( x_2\). L'idée est d'utiliser ceci :
            \begin{equation}
                (x_1+g)(x_2+f)=x_1x_2+x_1f+x_2g+fg,
            \end{equation}
            mais en mettant les bons coefficients pour reproduire ce que nous avons dans \eqref{SUBEQooLBXBooXoLyuw} : 
            \begin{equation}
                (2m_{12}+2g)(x_1+\frac{ f }{ m_{12} })-\frac{ 2fg }{ m_{12} }=2m_{12}x_1x_2+2x_1f+2x_2g.
            \end{equation}
            Cela pour dire que
            \begin{equation}
                q(x)=2(m_{12}x_1+g)(x_2+\frac{ f }{ m_{12} })-\frac{ 2fg }{ m_{12} }+T
            \end{equation}
            où \(-2fg/m_{12}+T\) est une forme quadratique de \( x_3,\ldots, x_{n+1}\), c'est à dire de \( n-1\) variables.

            L'hypothèse de récurrence nous donne des formes linéaires \( (l_i)_{i=3,\ldots, n+1}\) telles que
            \begin{equation}
                \frac{ 2fg }{ m_{12} }+T=\sum_{i=3}^{n+1}\alpha_il_i(x)^2.
            \end{equation}
            Nous pouvons donc déjà écrire
            \begin{equation}
                q(x)=2l'_1(x)l'_2(x)+\sum_{i=3}^{n+1}\alpha_il_i(x)^2
            \end{equation}
            où
            \begin{itemize}
                \item Les forme \( l_i\) avec \( i\geq 3\) ne dépendent pas de \( x_1\) et \( x_2\), et sont donc indépendantes de \( l_1\) et \( l_2\).
                \item La forme \( l'_1\) ne dépend pas de \( x_2\),
                \item La forme \( l'_2\) ne dépend pas de \( x_1\).
            \end{itemize}
            Ce sont donc \( n+1\) formes linéaires indépendantes. Le seul problème résiduel est que les formes \( l'_1\) et \( l'_2\) arrivent en produit l'une de l'autre. Nous en définissons donc deux de plus :
            \begin{equation}
                \begin{aligned}[]
                    l_1(x)=\frac{ 1 }{2}(l'_1+l'_2)\\
                    l_2(x)=\frac{ 1 }{2}(l'_1-l'_2),
                \end{aligned}
            \end{equation}
            qui sont linéairement indépendantes l'une de l'autre et indépendantes des \( l_i\) (\( i\geq 3\)). Au final,
            \begin{equation}
                q(x)=l_1(x)^2+l_2(x)^2+\sum_{i=3}^{n+1}\alpha_il_i(x)^2.
            \end{equation}
        \item[Si \( m_{ii}=0\) et \( m_{12}=0\) et \( m_{kl}\neq 0\) avec \( k<l\)]
            Nous considérons la permutation
            \begin{equation}
                \begin{aligned}
                    \sigma\colon \{ 1,\ldots, n+1 \}&\to \{ 1,\ldots, n+1 \} \\
                    i&\mapsto \begin{cases}
                         1   &   \text{si } i=k\\
                         2   &   \text{si } i=l\\
                         k   &   \text{si } i=1\\
                         l   &   \text{si } i=2\\
                        i    &    \text{sinon,}
                    \end{cases}
                \end{aligned}
            \end{equation}
            c'est à dire que \( \sigma\) permute \( 1\) et \( k\) ainsi que \( 2\) et \( l\). Ensuite nous posons
            \begin{equation}
                \begin{aligned}
                    s\colon \eR^{n+1}&\to \eR^{n+1} \\
                    e_i&\mapsto e_{\sigma(i)}. 
                \end{aligned}
            \end{equation}
            Nous allons un peu considérer \( q\circ s\), pour changer : 
            \begin{equation}        \label{EQooLVAWooAirEzP}
                (q\circ s)(x)=\sum_{i,j}m_{ij}s(x)_is(x)_j=\sum_{ij}x_{\sigma(i)}x_{\sigma(j)}.
            \end{equation}
            parce que \( s(x)_i=x_{\sigma(i)}\). 

            Utilisons un petit abus de notation pour considérer
            \begin{equation}
                \begin{aligned}
                    \sigma\colon \{ 1,\ldots, n+1 \}\times \{ 1,\ldots, n+1 \}&\to \{ 1,\ldots, n+1 \}\times \{ 1,\ldots, n+1 \} \\
                    (i,j)&\mapsto \big(\sigma(i), \sigma(j)\big). 
                \end{aligned}
            \end{equation}
            Cela est une bijection; nous pouvons utiliser le lemme \ref{DEFooLNEXooYMQjRo} pour permuter les termes dans \eqref{EQooLVAWooAirEzP} :      
            \begin{subequations}
                \begin{align}
                    (q\circ s)(x)&=\sum_{ij}m_{\sigma(i)\sigma(j)}x_{\sigma\sigma(i)}x_{\sigma\sigma(j)}\\
                    &=\sum_{ij}a_{ij}x_ix_j     \label{EQooPCTCooFnMWat}
                \end{align}
            \end{subequations}
            où nous avons posé \( a_{ij}=m_{\sigma(i)\sigma(j)}\) et utilisé le fait que \( \sigma=\sigma^{-1}\). Le point intéressant de l'histoire est que dans \eqref{EQooPCTCooFnMWat}, \( a_{12}=m_{kl}\neq 0\). La forme \( q\circ s\) est donc dans le cas déjà traité et il existe des formes linéaires \( l'_i\) telles que
            \begin{equation}
                (q\circ s)(x)=\sum_{i=1}^{n+1}\alpha_il'_i(x)^2.
            \end{equation}
            En évaluant cela en \( s(x)\), et en tenant compte de \( s=s^{-1}\), nous trouvons
            \begin{equation}
                q(x)=\sum_i\alpha_i(l_i\circ s)(x)^2,
            \end{equation}
            de telle sorte que \( l_i=l'_i\circ s\) soit la réponse à notre théorème.
    \end{subproof}
\end{proof}

%--------------------------------------------------------------------------------------------------------------------------- 
\subsection{Orthogonalité}
%---------------------------------------------------------------------------------------------------------------------------

\begin{proposition}[\cite{BIBooUULNooUtlrar}]       \label{PROPooYXMMooYIuGRd}
    Soient un espace vectoriel \( (E,\eK)\) et une forme quadratique\footnote{Définition \ref{DefBSIoouvuKR}.} \( q\). Une base de \( E\) est \( q\)-orthogonale si et seulement si la matrice de \( q\) dans cette base est diagonale.
\end{proposition}

\begin{proof}
    La matrice de \( q\) est donnée par \( Q_{ij}=b(e_i,e_j)\). Donc oui, cette matrice est diagonale si et seulement si les \( e_i\) sont orthogonaux.
\end{proof}

\begin{proposition}
    Soit une forme quadratique \( q\). Si une base \( (e_i )\) de \( E\) est \( q\)-orthogonale, alors \( \mB=\{ e_i\tq q(e_i)=0 \}\) est une base de \( \ker(q)\).
\end{proposition}

\begin{proof}
    Nous considérons un vecteur de base \( e_j\), et nous montrons que \( q(e_j)=0\) si et seulement si \( e_j\in\ker(q)\). Nous savons par la proposition \ref{PROPooYXMMooYIuGRd} que la matrice de \( q\) dans la base \( (e_i)\) est diagonale et que les éléments diagonaux sont les \( q(e_i)\). Soit \( K=\{ i\tq q_(e_i)=0 \}\).
    \begin{subproof}
    \item[\( \Span\{ e_i \}_{i\in K}\subset\ker(q)\)]
        Si \( x=\sum_{i\in K}x_ie_i\), alors 
        \begin{equation}
            q(x)=b(x,x)=\sum_{i,j\in K}| x_i |^2b(e_i,e_j)=\sum_{i,j\in K}| x_i |^2\delta_{ij}q(e_i)=0
        \end{equation}
        parce que \( q(e_i)=0\) dès que \( i\in K\).
        \item{\( \ker(q)\subset\Span\{ e_i \}_{i\in K}\)}
            Soit \( x\in \ker(q)\) et écrivons-le sous la forme \( x=\sum_{i=1}^nx_ie_i\). Nous avons
            \begin{equation}
                0=q(x)=\sum_i| x_i |^2q(e_i).
            \end{equation}
            Mais \(    | x_i |^2\geq 0 \) et \( q(e_i)\geq 0\), donc si \( q(e_i)\neq 0\), alors \( x_i=0\). Donc les seules composantes non nulles de \( x\) sont celles sur lesquelles \( q\) s'annule. En d'autres termes \( x=\sum_ix_ie_i\in \Span\{ e_i \}_{i\in K}\).
    \end{subproof}
\end{proof}

\begin{theorem}[\cite{BIBooUULNooUtlrar}]       \label{THOooIDMPooIMwkqB}
    Toute forme quadratique sur un espace vectoriel de dimension finie admet une base formée de vecteurs \( 2\) à \( 2\) orthogonaux (pour la forme considérée).
\end{theorem}

%--------------------------------------------------------------------------------------------------------------------------- 
\subsection{Diagonalisation}
%---------------------------------------------------------------------------------------------------------------------------

Le théorème \ref{THOooIDMPooIMwkqB} a déjà donné une base orthogonale pour toute forme quadratique sur un espace vectoriel \( (E,\eK)\) de dimension finie. Dans le cas de \( \eR^n\), nous pouvons en donner une preuve basée sur le théorème spectral, c'est la proposition \ref{PROPooUKRUooGRIDHt}.

\begin{proposition}     \label{PROPooUKRUooGRIDHt}
    Soit une forme bilinéaire symétrique \( b\) sur un \( \eR^n\). Il existe une matrice orthogonale \( Q\) telle que 
    \begin{enumerate}
        \item
            \( D=Q^tbQ\) est diagonale
        \item
            \( D(x,y)=b(Qx,Qy)\) pour tout \( x,y\in E\).
    \end{enumerate}

    Il existe une base \( (f_i)_{i=1,\ldots, n}\) qui est \( b\)-orthogonale.

    Dans cet énoncé, nous mélangeons sans vergogne les formes et les matrices, en supposant qu'une base soit fixée\footnote{Autrement dit, si vous avez en tête d'utiliser cette proposition pour \( \eR^n\) c'est bon; mais sinon vous devez choisir une base et considérer toutes les matrices dans cette base.}. Par exemple
    \begin{equation}
        D(x,y)=\sum_{ij}D_{ij}x_iy_j.
    \end{equation}
\end{proposition}

\begin{proof}
    Pour la matrice diagonale, c'est le théorème spectral \ref{ThoeTMXla}\ref{ITEMooMWWRooXxGONW} qui joue parce que la matrice d'une forme bilinéaire symétrique est symétrique (c'est vu de la définition \eqref{EQooCUGFooRlKUtu}).

    Pour le reste c'est un calcul :
    \begin{subequations}
        \begin{align}
            D(x,y)&=\sum_{ijkl}Q^t_{ik}b_{kl}Q_{lj}x_iy_j\\
            &=\sum_{ijkl}b_{kl}(Q_{ki}x_i)(Q_{lj}y_j)\\
            &=\sum_{kl}b_{kl}(Qx)_k(Qy)_l\\
            &=b(Qx,Qy).
        \end{align}
    \end{subequations}
    Nous avons utilisé le produit matrice fois vecteur donné par \eqref{EQooQFVTooMFfzol}.

    En ce qui concerne l'existence d'une base \( b\)-orthogonale, vu que \( D\) est diagonale, nous avons, pour \( i\neq j\) que \( D(e_i,e_j)=0\). Donc en posant \( f_i=Qe_i\), nous trouvons
    \begin{equation}
        0=D(e_i,e_j)=b(Qe_i,Qe_j)=b(f_i,f_j).
    \end{equation}
    La base \( (Qe_i)_{i=1,\ldots, n}\) est donc \( b\)-orthogonale.
\end{proof}

%--------------------------------------------------------------------------------------------------------------------------- 
\subsection{Isométrie, forme quadratique et bilinéaire}
%---------------------------------------------------------------------------------------------------------------------------

\begin{example}
    La forme quadratique \( q(x)=x_1^2+x_2^2\) donne la norme euclidienne. La forme bilinéaire associée est \( b(x,y)=x_1y_1+x_2y_2\), qui est le produit scalaire usuel.
\end{example}

Il ne faudrait pas déduire trop vite que la formule \( \| x \|^2=q(x)\) donne une norme dès que \( q\) est non dégénérée. En effet \( q\) peut ne pas être définie positive. La forme \( q(x)=x_1^2-x_2^2\) prend des valeurs positives et négatives. A fortiori \( d(x,y)=q(x-y)\) ne donne pas toujours une distance.

\begin{definition}      \label{DEFooECTUooRxBhHf}
    Une \defe{isométrie}{isométrie!de forme quadratique} pour la forme quadratique \( q\) est une application bijective \( f\colon V\to V\) telle que 
    \begin{equation}
     q(x-y)=q\big( f(x)-f(y) \big).
    \end{equation}
     Dans les cas où \( q\) donne une distance, alors c'est une isométrie au sens usuel.
\end{definition}

\begin{definition}[Thème \ref{THMooVUCLooCrdbxm}]      \label{DEFooIQURooMeQuqX}
    Soit un espace vectoriel \( E\) muni d'une forme bilinéaire \( b\). Une \defe{isométrie}{isométrie (forme bilinéaire)} pour \( b\) est une bijection \( f\colon E\to E\) telle que
    \begin{equation}
        b\big( f(x),f(y) \big)=b(x,y)
    \end{equation}
    pour tout \( x,y\in E\).
\end{definition}

\begin{lemma}   \label{LemewGJmM}
    Soient \( q\) une forme quadratique et \( b\) la forme bilinéaire associée par le lemme~\ref{LEMooLKNTooSfLSHt}. Une application \( f\colon E\to E\) telle que \( f(0)=0\) est une isométrie pour \( b\) si et seulement si elle est une isométrie pour \( q\).
\end{lemma}

\begin{proof}
    Pour une application bijective \( f\colon E\to E\) telle que \( f(0)=0\), nous devons prouver l'équivalence des propriétés suivantes :
    \begin{enumerate}
        \item
            \( b\big( f(x),f(y) \big)=b(x,y)\) pour tout \( x,y\in E\);
        \item
            \( q\big( f(x)-f(y) \big)=q(x-y)\) pour tout \( x,y\in E\).
    \end{enumerate}

    Dans le sens direct, en posant \( x=y\) nous trouvons tout de suite \( q(f(x))=q(x)\); ensuite en utilisant la distributivité de \( b\),
    \begin{subequations}
        \begin{align}
            q\big( f(x)-f(y) \big)&=b\big( f(x)-f(y),f(x)-f(y) \big)\\
            &=q\big( f(x) \big)-2b\big( f(x),f(y) \big)+q\big( f(y) \big)\\
            &=q(x)+q(y)-2b(x,y)\\
            &=q(x-y).
        \end{align}
    \end{subequations}

    Dans l'autre sens, nous commençons par remarquer que l'hypothèse \( f(0)=0\) donne \( q(x)=q\big( f(x) \big)\). Ensuite nous utilisons l'identité de polarisation \eqref{EqMrbsop} :
    \begin{subequations}
        \begin{align}
            b\big( f(x),f(y) \big)&=\frac{ 1 }{2}\big[ q\big( f(x) \big)+q\big( f(y) \big)-q\big( f(x-y) \big) \big]\\
            &=\frac{ 1 }{2}\big[ q(x)+q(y)-q(x-y) \big]\\
            &=b(x,y).
        \end{align}
    \end{subequations}
\end{proof}


%+++++++++++++++++++++++++++++++++++++++++++++++++++++++++++++++++++++++++++++++++++++++++++++++++++++++++++++++++++++++++++ 
\section{Théorème de Sylvester}
%+++++++++++++++++++++++++++++++++++++++++++++++++++++++++++++++++++++++++++++++++++++++++++++++++++++++++++++++++++++++++++

\begin{definition}[Signature\cite{BIBooXOWGooAPWTfT}]       \label{DEFooWDCLooDkRYLK}
    Soit une forme quadratique\footnote{Définition \ref{DefBSIoouvuKR}.} \( Q\) sur un espace vectoriel \( E\) de dimension finie \( n\). L'\defe{indice d'inertie}{indice d'inertie} de \( q\) est le nombre
    \begin{equation}
        q=\max\{ \dim(F)\tq Q(v)<0\,\forall v\in F\setminus\{ 0 \} \}.
    \end{equation}
    Nous définissons aussi
    \begin{equation}
        p=\max\{ \dim(G)\tq Q(v)>0\,\forall v\in G\setminus\{ 0 \} \}.
    \end{equation}
    Le couple \( (p,q)\) est la \defe{signature}{signature!forme quadratique} de \( Q\).
\end{definition}
<++>

% Pour info, le théorème de Sylvester est bien énoncé dans
\begin{theorem}[de Sylvester\cite{BIBooXOWGooAPWTfT}]   \label{ThoQFVsBCk}
    Soit $Q$ une forme quadratique réelle de signature\footnote{Définition \ref{DEFooWDCLooDkRYLK}.} \( (p,q)\). Alors pour toute base orthonormée \( \{ e_i \}\) de \( \eR^{p+q}\) nous avons les points suivants.
    \begin{enumerate}
        \item
            Les nombres \( p\) et \( q\) sont donnée par 
    \begin{subequations}
        \begin{align}
            p&=\Card\{ i\tq Q(e_i)>0 \}\\
            q&=\Card\{ i\tq Q(e_i)<0 \}.
        \end{align}
    \end{subequations}
\item
    Le rang de \( Q\) est \( p+q\).
\item
    Si \( A\) est la matrice de \( Q\) dans une base, alors il existe une matrice inversible \( P\) telle que
    \begin{equation}
        P^tAP=\begin{pmatrix}
            -\mtu_q    &       &       \\
                &   \mtu_p    &       \\
                &       &   0
        \end{pmatrix}.
    \end{equation}
    \end{enumerate}
    <++>
\end{theorem}
\index{théorème!Sylvester}
\index{rang}

\begin{proposition}[\cite{BIBooXOWGooAPWTfT}]       \label{PROPooBWXMooLsgyKm}
    Deux formes quadratiques sont équivalentes si et seulement si elles ont même signature.
\end{proposition}

\index{matrice!semblables}
\index{forme!quadratique}

%--------------------------------------------------------------------------------------------------------------------------- 
\subsection{Invariance de la trace}
%---------------------------------------------------------------------------------------------------------------------------

\begin{proposition}[\cite{MonCerveau}]      \label{PROPooRMYQooWkEpJJ}
    Soit une application linéaire \( f\). Si la matrice de \( f\) dans une base est \( A\) et est \( B\) dans une autre base, alors
    \begin{equation}
        \trace(A)=\trace(B).
    \end{equation}
\end{proposition}

\begin{proof}
    Les matrices \( A\) et \( B\) sont liées par la proposition \ref{PROPooNZBEooWyCXTw} : \( B=Q^{-1}AQ\) où \( Q\) est la matrice qui lie les vecteurs des deux bases. L'invariance cyclique de la trace donnée en le lemme \ref{LEMooUXDRooWZbMVN} implique que
    \begin{equation}
        \trace(B)=\trace(Q^{-1}AQ)=\trace(QQ^{-1}A)=\trace(A).
    \end{equation}
\end{proof}



\input{59_EspacesVectos}
% This is part of Mes notes de mathématique
% Copyright (c) 2011-2020
%   Laurent Claessens, Carlotta Donadello
% See the file fdl-1.3.txt for copying conditions.

%+++++++++++++++++++++++++++++++++++++++++++++++++++++++++++++++++++++++++++++++++++++++++++++++++++++++++++++++++++++++++++
\section{Extension du corps de base}
%+++++++++++++++++++++++++++++++++++++++++++++++++++++++++++++++++++++++++++++++++++++++++++++++++++++++++++++++++++++++++++
\label{SECooAUOWooNdYTZf}

Nous avons discuté dans la section~\ref{SECooLQVJooTGeqiR} de ce qui arrive au corps lorsqu'on l'étend. Dans cette sections nous allons étudier ce qui arrive aux applications linéaires entre deux \( \eK\)-espaces vectoriels lorsque nous étendons le corps \( \eK\) en un corps \( \eL\).

Soit donc un corps \( \eK\) et deux \( \eK\)-espaces vectoriels \( E\) et \( F\), et entrons dans le vif du sujet\footnote{Le sujet étant le corps étendu.}. Soit \( \eK\) un corps (commutatif) et une extension \( \eL\) de \( \eK\). Soient \( E\) et \( F\), des \( \eK\)-espaces vectoriels de dimension finie.

%---------------------------------------------------------------------------------------------------------------------------
\subsection{Extension des applications linéaires}
%---------------------------------------------------------------------------------------------------------------------------


\begin{definition}[\cite{ooAFBYooYvTCCN}]
    L'espace vectoriel obtenu par \defe{extension du corps de base}{extension!corps de base} de \( E\) est l'espace vectoriel
    \begin{equation}
        E_{\eL}=\eL\otimes_{\eK}E.
    \end{equation}
    Ce dernier est le quotient \( \eL\otimes_{\eK}E=(\eL\times E)/\sim\) par la relation d'équivalence
    \begin{equation}
        (\lambda,v)\sim\big( a\lambda,\frac{1}{ a }v \big)
    \end{equation}
    pour tout \( a\in \eK\). Nous noterons \( [\lambda,v]\) ou \( \lambda\otimes v\) ou encore \( \lambda\otimes_{\eK}v\) la classe de \( (\lambda,v)\).
\end{definition}
Un élément de \( E_{\eL}\) est de la forme \( \sum_k[\lambda_k,v_k]\) avec \( \lambda_k\in \eL\) et \( v_k\in E\). Si \( f\colon E\to F\) est une applications linéaire nous définissons
\begin{equation}
    \begin{aligned}
        f_{\eL}\colon E_{\eL}&\to F_{\eL} \\
        [\lambda,v]&\mapsto [\lambda,f(v)].
    \end{aligned}
\end{equation}

\begin{remark}
    Si deux vecteurs de \( E_{\eL}\) sont linéairement indépendants pour \( \eK\), ils ne le sont pas spécialement pour \( \eL\). Par exemple si \( \eC\) est vu comme \( \eR\)-espace vectoriel, alors \( \{ 1,i \}\) est une partie libre. Mais dans \( \eC\) vu comme \( \eC\)-espace vectoriel, la partie \( \{ 1,i \}\) n'est pas libre.
\end{remark}

Nous définissons aussi l'injection canonique
\begin{equation}
    \begin{aligned}
        \iota\colon E&\to E_{\eL} \\
        v&\mapsto [1,v].
    \end{aligned}
\end{equation}

\begin{proposition}[\cite{ooEPEFooQiPESf}]      \label{PropooWECLooHPzIHw}
    Injectivité et surjectivité respectées.
    \begin{enumerate}
        \item
            L'application \( f_{\eL}\) est injective si et seulement si \( f\) est injective.
        \item
            L'application \( f_{\eL}\) est surjective si et seulement si \( f\) est surjective.
    \end{enumerate}
\end{proposition}

\begin{proof}
    Supposons pour commencer que \( f_{\eL}\) est injective.
    Le diagramme
    \begin{equation}
        \xymatrix{%
            E \ar[r]^-{f}\ar[d]_-{\tau}      &   F\ar[d]^{\tau}\\
            E_{\eL} \ar[r]_{f_{\eL}}  &   F_{\eL}
           }
    \end{equation}
    est un diagramme commutatif. En effet
    \begin{equation}
        (\tau\circ f)(v)=[1,f(v)]
    \end{equation}
    tandis que
    \begin{equation}
        (f_{\eL\circ\tau})(v)=f_{\eL}[1,v]=[1,f(v)].
    \end{equation}
    Donc si \( f(v)=0\) avec \( v\neq 0\) nous aurions \( (\tau\circ f)(v)=0\) et donc aussi \( (f_{\eL}\circ \tau)(v)=0\), alors que \( \tau(v)\neq 0\) dans \( E_{\eL}\).

    Réciproquement, nous supposons que \( f\) est injective et nous prouvons que \( f_{\eL}\) est injective. Par le lemme~\ref{LEMooDAACooElDsYb}\ref{ITEMooEZEWooZGoqsZ}, nous savons qu'il existe \( g\colon F\to E\) telle que \( f\circ g=\id|_F\). Nous en déduisons que \( f_{\eL}\circ g_{\eL}=\id|_{F_{\eL}}\) parce que si \( [\lambda,v]\in F_{\eL}\) alors
    \begin{equation}
        (f_{\eL}\circ g_{\eL})[\lambda,v]=f_{\eL}[\lambda,g(v)]=[\lambda,(f\circ g)(v)]=[\lambda,v].
    \end{equation}
    Notons que \( g\) est injective, donc \( g_{\eL}\) est injective et l'égalité \( f_{\eL}\circ g_{\eL}=\id|_{F_{\eL}} \) implique que \( f_{\eL}\) est également injective.
\end{proof}

\begin{proposition}[\cite{MonCerveau,ooYVQCooFBVEXo}] \label{PROPooMHARooUycAts}
    Soit \( \{ e_i \}_{i=1,\ldots, p}\) une base de \( E\). Alors \( \{ 1\otimes e_i \}_i\) est une base de \( E_{\eL}=\eL\otimes_{\eK}E\).
\end{proposition}

\begin{proof}
    L'espace vectoriel \( E\) peut être écrit comme somme directe \( E=\bigoplus_i\eK e_i\). Si \( \lambda\in \eL\) et \( k\in \eK\) nous avons
    \begin{equation}
        \lambda\otimes ke_i=\frac{ \lambda }{ k }\otimes e_i=\frac{ \lambda }{ k }(1\otimes e_i).
    \end{equation}
    Cela pour introduire que l'application
    \begin{equation}
        \begin{aligned}
            \psi\colon \eL\otimes_{\eK}E&\to \bigoplus_i\eL(1\otimes e_i) \\
            \sum_k \lambda_k\otimes v_k&\mapsto \oplus_i \sum_k(\lambda_k v_{ik})(1\otimes e_i)
        \end{aligned}
    \end{equation}
    où \( v_k=\sum_i v_{ik}e_i\) avec \( v_{ik}\in \eK\) est un isomorphisme de \( \eL\)-espaces vectoriels. La surjectivité est facile. En ce qui concerne l'injectivité, si
    \begin{equation}
        \sum_i\sum_k(\lambda_kv_{ik})(1\otimes e_i)=0
    \end{equation}
    alors les choses suivantes sont nulles également :
    \begin{equation}
        \sum_i\sum_k(\lambda_kv_{ik})(1\otimes e_i)=\sum_{ik}(\lambda_k\otimes v_{ik}e_i)=\sum_k(\lambda_k\otimes \sum_iv_{ik}e_i)=\sum_k(\lambda_k\otimes v_k).
    \end{equation}
    Le dernier est l'argument de \( \psi\). Le fait que ce soit nul implique que \( \psi\) est injective.
\end{proof}

\begin{remark}
    Nous n'avons pas dû prouver que chacun des \( \lambda_k\otimes v_k\) était nul. Et encore heureux, parce que cela pouvait très bien être faux, vu qu'il y a plusieurs façons de noter un élément de \( E_{\eL}\) sous la forme de tels termes.
\end{remark}

\begin{corollary}       \label{CORooTQGHooIKhNtr}
    La \( \eL\)-dimension de \( E_{\eL}\) est égale à la \( \eK\)-dimension de \( E\).
\end{corollary}

%---------------------------------------------------------------------------------------------------------------------------
\subsection{Projections}
%---------------------------------------------------------------------------------------------------------------------------

\begin{probleme}
    Nous allons définir \( \pr\colon \aL(E_{\eL},F_{\eL})\to \aL(E,F)\) en faisant appel à des bases et en prouvant que les choses définies ne dépendent pas des bases choisies. Il y a surement une façon plus «intrinsèque» de faire.
\end{probleme}


Nous savons que \( \eL\) est un \( \eK\)-espace vectoriel dans lequel nous pouvons voir \( \eK\) comme un sous-espace (lemme~\ref{LemooOLIIooXzdppM}). Dans cette optique nous choisissons dans \( \eL\) un supplémentaire de \( \eK\), c'est-à-dire un sous-espace vectoriel de \( \eL\) tel que
\begin{equation}
    \eL=\eK\oplus V.
\end{equation}
Nous avons alors naturellement une projection \( \pr\colon \eL\to \eK\).

Soit \( \{ e_i \}\) une base de \( E \) et \(\{ e_a \}\) une  de\( F\). Nous noterons également \( e_i\) et \( e_a\) les éléments \( \tau e_i\) et \( \tau e_a\) correspondants. Grâce à la proposition~\ref{PROPooMHARooUycAts}, ce sont des bases de \( E_{\eL}\) et \( F_{\eL}\). Si la fonction \( f\colon E_{\eL}\to F_{\eL}\) s'écrit dans ce ces bases comme
\begin{equation}
    f(e_i)=\sum_af_{ai}e_a
\end{equation}
alors nous définissons \( \pr(f)\) par
\begin{equation}        \label{EQooSAFRooJnfkLO}
    (\pr f)e_i=\sum_a\pr(f_{ai})e_a.
\end{equation}

\begin{proposition}[\cite{MonCerveau}]      \label{PROPooOEHTooHyjuZQ}
    L'application \( \pr\) définie en \eqref{EQooSAFRooJnfkLO} est indépendante du choix des bases.
\end{proposition}

\begin{proof}
    Notons que dans ce qui suit, les sommes sur \( a\) ou \( b\) et celles sur \( i\) ou \( j\) ne vont pas jusqu'au même indice (dimensions de \( E\) et \( F\)). De plus nous manipulons deux choses qui se notent \( \pr\). La première est la projection \( \pr\colon \eL\to \eK\) qui ne dépend que d'un choix de supplémentaire et que nous supposons fixée ici. D'autre part il y a \( \pr\colon E_{\eL}\to E\) qui dépend à priori des bases choisies.

    Nous choisissons de nouvelles bases qui sont liées aux anciennes bases par
    \begin{subequations}
        \begin{numcases}{}
            e'_b=\sum_aB_{ab}e_a\\
            e'_i=\sum_jA_{ji}e_j.
        \end{numcases}
    \end{subequations}
    Les matrices \( A\) et \( B\) sont dans \( \GL(\eK)\). Nous allons écrire l'opérateur \( \pr'\) qui correspond à ces bases et montrer que pour toute application linéaire \( f\colon E_{\eL}\to F_{\eL} \) nous avons \( \pr(f)=\pr'(f)\). Nous avons :
    \begin{subequations}
        \begin{align}
            f(e'_j)&=\sum_iA_{ji}f(e_i)\\
            &=\sum_a\sum_b\sum_iA_{ji}f_{ai}(B^{-1})_{ba}e'b\\
            &=\sum_b\Big( \sum_{ai}A_{ji}f_{ai}(B^{-1})_{ba} \Big)e'b,
        \end{align}
    \end{subequations}
    ce qui fait que
    \begin{equation}        \label{EQooUQNBooMWHRbD}
        (\pr'f)e'_j=\sum_b\Big( \pr\big( A_{ji}f_{ai}(B^{-1})_{ba} \big) \Big)e'_b.
    \end{equation}
    Nous calculons maintenant \( (\pr'f)e_j\) en substituant \( e_j=\sum_l(A^{-1})_{lj}e'_l\) et en utilisant \eqref{EQooUQNBooMWHRbD} et la linéarité de \( \pr'\) et la \( \eK\)-linéarité de \( \pr\colon \eL\to \eK\) :
    \begin{subequations}
        \begin{align}
            (\pr'f)\Big( \sum_l(A^{-1})_{lj}e'_l \Big)
            &=\sum_l(A^{-1})_{lj}\sum_b\sum_{ai}\pr\big(A_{li}f_{ai}(B^{-1})_{ba}\big)e_b\\
            &=\sum_a\pr(f_{aj})e_a\\
            &=(\pr f)e_j.
        \end{align}
    \end{subequations}
    Donc \( \pr=\pr'\).
\end{proof}

Note au passage comme toujours : il y a un abus systématique de notation entre \( e_i\in E\) et \( \tau(e_i)=1\otimes e_i\in E_{\eL}\).

\begin{remark}[\cite{MonCerveau}]       \label{REMooBEXGooLgpHzg}
    L'opération \( \pr\colon \aL(E_{\eL},F_{\eL})\to \aL(E,F)\) ne dépend pas des bases choisies un peu partout. Mais elle dépend de l'application \( pr\colon \eL\to \eK\) déjà construite. Et celle-là dépend du choix d'un supplémentaire $V$ qui fournit \( \eL=\eK\oplus V\).

    Si \( \pr(\lambda)=0\) pour un de ces choix, cela n'implique nullement que \( \lambda=0\). Penser à \( i\in \eC\) si la projection \( \pr\colon \eC\to \eR\) est l'application \( (x+iy)\mapsto x\) parallèle à l'axe des imaginaires.

    Par contre si \( \pr(\lambda)=0\) pour tout choix de \( V\), alors nous avons bien \( \lambda=0\). Dans la suit nous «fixons» un choix de \( V\) générique, et lorsque nous rencontrerons l'égalité \( \pr(\lambda)=0\) nous en déduirons \( \lambda=0\).
\end{remark}

\begin{proposition} \label{PROPooPWDKooFNFWRI}
    Si \( f\colon E\to F\) et si \( f_{\eL}e_j=\sum_a(f_{\eL})_{aj}e_a\) et si \( f(e_j)=\sum_af_{aj}e_a\) alors
    \begin{enumerate}
        \item
            \( \pr f_{\eL}=f\),
        \item       \label{ITEMooNMPYooXosGhI}
            \( (f_{\eL})_{ja}=f_{ja} \in \eK\).
    \end{enumerate}
\end{proposition}

\begin{proof}
    Nous avons
    \begin{equation}
        f_{\eL}(e_i)=\sum_a f_{ai}(1\otimes e_a)=\sum_a f_{ai}\tau(e_a),
    \end{equation}
    donc
    \begin{equation}
        (\pr f_{\eL})e_i=\sum_a\pr(f_{ai})e_a=\sum_af_{ai}e_a=f(e_i).
    \end{equation}
    Cela prouve que \( \pr f_{\eL}=f\).

    Par ailleurs,
    \begin{equation}        \label{EQooIOTFooNAdkit}
        f_{\eL}(\tau e_i)=f_{\eL}(1\otimes e_i)=1\otimes f(e_i)=\tau\big( f(e_i) \big)=\sum_af_{ai}\tau(e_a)
    \end{equation}
    alors que par définition,
    \begin{equation}        \label{EQooMYSCooPFWATG}
        f_{\eL}(\tau e_i)=\sum_a(f_{\eL})_{ai}\tau(e_a).
    \end{equation}
    Les éléments \( \tau(e_a)\) formant une base\footnote{Encore la proposition~\ref{PROPooMHARooUycAts}.}, la comparaison de \eqref{EQooIOTFooNAdkit} avec \eqref{EQooMYSCooPFWATG} donne \( (f_{\eL})_{ai}=f_{ai}\in \eK\).
\end{proof}

\begin{lemma}       \label{LEMooWZGSooONEnjZ}
    Soient
    \begin{enumerate}
        \item
            Une base \( \{ e_i \}\) de \( E\) et une application linéaire \( f\colon E\to F\);
        \item
            une base \( \{ e_a \}\) de \( F\) et une application linéaire \( g\colon G\to F\);
        \item
            une base \( \{ e_{\alpha} \} \) de \( G\) et une application linéaire \( \tilde h\colon G_{\eL}\to E_{\eL}\).
    \end{enumerate}
    Alors nous avons
    \begin{equation}
        \pr(f_{\eL}\circ \tilde h)=\pr(f_{\eL})\circ\pr(\tilde h).
    \end{equation}
\end{lemma}

\begin{proof}
    Pour écrire \( \pr(f_{\eL}\circ \tilde h)\) à partir de la définition \eqref{EQooSAFRooJnfkLO} nous commençons par écrire
    \begin{equation}
        (f_{\eL}\circ \tilde h)e_{\alpha}=\sum_a(f_{\eL}\circ \tilde h)_{a\alpha}e_a=\sum_{ai}(f_{\eL})_{ai}(\tilde h)_{i\alpha}e_a=\sum_a\Big( \sum_{i}f_{ai}(\tilde h)_{i\alpha} \Big)e_a
    \end{equation}
    où nous avons utilisé le fait que \( (f_{\eL})_{ai}=f_{ai}\). Donc, en utilisant la \( \eK\)-linéarité de \( \pr\),
    \begin{equation}        \label{EQooZGCGooQsCBQH}
        \pr(f_{\eL}\circ \tilde h)e_{\alpha}=\sum_a\sum_i\pr\Big( f_{ai}(\tilde h)_{i\alpha} \Big)e_a=\sum_a\sum_if_{ai}\pr\Big( (\tilde h)_{i\alpha} \Big)e_a.
    \end{equation}
    D'autre part,
    \begin{equation}
        \begin{aligned}[]
            \pr(f_{\eL})\circ \pr(\tilde h)e_{\alpha}&=\pr(f_{\eL})\sum_i\pr\Big( (\tilde h)_{i\alpha} \Big)e_i\\
            &=\sum_i\pr\Big( (\tilde h)_{i\alpha} \Big)\sum_af_{ai}e_a\\
            &=\sum_{ai}\pr\Big( (\tilde h)_{i\alpha} \Big)f_{ai}e_a,
        \end{aligned}
    \end{equation}
    et c'est égal à \eqref{EQooZGCGooQsCBQH}.
\end{proof}

\begin{remark}
    Nous n'avons en général pas \( \pr(xy)=\pr(x)\pr(y)\) pour tout \( x,y\in \eL\). Par exemple si \( \eK=\eR\) et \( \eL=\eC\) avec la projection canonique,
    \begin{equation}
        \pr(i\cdot i)=\pr(-1)=-1
    \end{equation}
    alors que \( \pr(i)=0\).
\end{remark}

\begin{proposition}
    Soient \( f\in\aL(E,F)\) et \( g\in\aL(F,E)\). Alors il existe \( h\colon G\to E\) tel que \( f\circ h=g\) si et seulement s'il existe \( \tilde g\colon G_{\eL}\to E_{\eL}\) tel que \( f_{\eL}\circ \tilde g=g_{\eL}\).
\end{proposition}

\begin{proof}
    Dans le sens direct, il suffit de poser \( \tilde h=h_{\eL}\).

    Dans le sens inverse, si nous avons \( \tilde h\colon G_{\eL}\to E_{\eL}\) tel que \( f_{\eL}\circ\tilde h=g_{\eL}\) alors en appliquant \( \pr\) des deux côtés et en utilisant le lemme~\ref{LEMooWZGSooONEnjZ},
    \begin{equation}
        \pr(f_{\eL})\circ\pr(\tilde h)=\pr(g_{\eL})
    \end{equation}
    c'est-à-dire
    \begin{equation}
        f\circ\pr(\tilde h)=g,
    \end{equation}
    c'est-à-dire que l'application \( \pr\tilde h\colon G\to E\) est la réponse à la proposition.
\end{proof}

%---------------------------------------------------------------------------------------------------------------------------
\subsection{Rang, polynôme minimal, polynôme caractéristique}
%---------------------------------------------------------------------------------------------------------------------------

\begin{proposition}[Stabilité du rang par extension des scalaires\cite{ooEPEFooQiPESf}]     \label{PROPooJFQDooZSsxMf}
    Si \( f\colon E\to F\) est linéaire alors nous avons
    \begin{equation}
        \rang(f)=\rang(f_{\eL}).
    \end{equation}
    où à droite nous considérons le rang de l'application \( \eL\)-linéaire \( f_{\eL}\colon E_{\eL}\to F_{\eL}\).
\end{proposition}

\begin{proof}
    Il existe un supplémentaire \( V\) tel que \( E=\ker(f)\oplus V\) avec \( \dim(V)=\rang(f)\). Nous pouvons factoriser \( f\) en
    \begin{equation}
        f=f_2\circ f_1
    \end{equation}
    avec \( f_1\colon E\to V\) est la projection parallèle à \( \ker(f)\) et est surjective (vers \( V\)) parce que \( \dim(V)=\rang(f)=\dim\big( \Image(f) \big)\). De plus \( f_2\colon V\to F\) est injective parce que si \( v\in V\) est tel que \( f_2(v)=0\) alors on aurait
    \begin{equation}
        f(v)=(f_2\circ f_1)(v)=f_2(v)=0.
    \end{equation}
    Cela donne \( v\in\ker(f)\cap V=\{ 0 \}\). Par la proposition~\ref{PropooWECLooHPzIHw}, les applications \( (f_1)_{\eL}\) et \( (f_2)_{\eL}\) sont respectivement surjective et injective.

    L'application \( (f_2)_{\eL}\colon V_{\eL}\to F_{\eL}\) est forcément surjective sur son image, donc
    \begin{equation}
        (f_2)_{\eL}\colon V_{\eL}\to \Image(f_{\eL})
    \end{equation}
    est un isomorphisme de \( \eL\)-espaces vectoriels. Nous avons alors les égalités
    \begin{equation}        \label{EQooWLOIooKlYWTL}
        \dim_{\eL}(V_{\eL})=\dim_{\eL}\big( \Image(f_{\eL}) \big)=\rang(f_{\eL}).
    \end{equation}
    Mais aussi, par les définitions posées plus haut,
    \begin{equation}        \label{EQooEVCGooAGjmoU}
        \dim(V)=\rang(f)=\dim\big( \Image(f) \big).
    \end{equation}
    Mais le corolaire~\ref{CORooTQGHooIKhNtr} nous dit que \( \dim_{\eL}(V_{\eL})=\dim_{\eK}(V)\). Donc il y a égalité des deux lignes \eqref{EQooWLOIooKlYWTL} et \eqref{EQooEVCGooAGjmoU} donne \( \rang(f)=\rang(f_{\eL})\).
\end{proof}

\begin{proposition}     \label{PROPooZAZFooUFdCUv}
    Nous avons
    \begin{enumerate}
        \item
            \( \det(f)=\det(f_{\eL})\)
        \item
            \( \chi_f=\chi_{f_{\eL}}\).
    \end{enumerate}
\end{proposition}

\begin{proof}
    Dès que l'on a des bases nous avons \( (f_{\eL})_{ai}=f_{ai}\) par la proposition~\ref{PROPooPWDKooFNFWRI}\ref{ITEMooNMPYooXosGhI}. Le nombre \( \det(f)\in \eK\) est un polynôme en les \( f_{ai}\). Entendons nous : il existe un polynôme indépendant de \( f\) et de \( \eK\) et de \( \eL\) donnant le déterminant de n'importe quelle matrice. Donc \( \det(f)=\det(f_{\eL})\).

    Même chose pour le polynôme caractéristique (définition~\ref{DefOWQooXbybYD}) : les coefficients de ce polynôme sont des polynômes en les \( f_{ai}\) qui sont indépendants de \( \eL\), de \( \eK\) et de \( f\).

    Notons que \( \chi_{f_{\eL}}\) est un polynôme à coefficients dans \( \eK\).
\end{proof}

La situation est très différente avec le polynôme minimal\footnote{Définition~\ref{DefCVMooFGSAgL}.}. Autant il existe une «recette» pour créer le polynôme caractéristique, il n'en n'existe pas pour le polynôme minimal (ou en tout cas, il ne suffit pas d'appliquer des polynômes en les coefficients de la matrice). La proposition suivante montre que le polynôme minimal est conservé par extension de corps, mais que pour le voir, il faut travailler plus.

\begin{proposition}[\cite{ooEPEFooQiPESf,MonCerveau}]      \label{PROPooXVZMooXcJrsJ}
    Soit \( \eL\) une extension du corps \( \eK\) et une application linéaire \( f\colon E\to F\) entre deux \( \eK\)-espaces vectoriels. Alors \( \mu_f=\mu_{f_{\eL}}\).
\end{proposition}

\begin{proof}
    Nous allons montrer que l'application
    \begin{equation}
        \begin{aligned}
            \tilde g\colon \frac{ \eL[X] }{ (\mu) }&\to \End(E_{\eL}) \\
            \bar P&\mapsto P(f_{\eL})
        \end{aligned}
    \end{equation}
    est bien définie et injective. La proposition~\ref{PROPooVUJPooMzxzjE} nous dira alors que \( \mu\) est le polynôme minimal de \( f_{\eL}\).

    Pour prouver que l'application \( \tilde g\) est bien définie, nous commençons par prouver que  \( P(f_{\eL})=P(f)_{\eL}\) :
    \begin{subequations}
        \begin{align}
            P(f_{\eL})\lambda\otimes v&=\sum_ka_kf_{\eL}^k\lambda\otimes v\\
            &=\lambda\otimes \sum_ka_kf^k(v)\\
            &=\lambda\otimes P(f)v\\
            &=P(f)_{\eL}\lambda\otimes v.
        \end{align}
    \end{subequations}
    Par conséquent \( \mu(f_{\eL})=0\) et l'application est bien définie.

    Sur \( \eL[X]/(\mu)\) nous considérons la base \( \{ 1,\bar X,\ldots, \bar X^{\deg(\mu)-1} \}\), et \( \End(E_{\eL})\) nous considérons une base qui commence\footnote{Théorème de la base incomplète~\ref{ThonmnWKs}\ref{ITEMooFVJXooGzzpOu}.} par \( \{ f_{\eL}^k \}_{k=0,\ldots, \deg(\mu)-1}\). Montrons tout de même que cette partie est libre (sinon le théorème de la base incomplète ne s'applique pas) : si \( \sum_k\lambda_kf_{\eL}^k=0\) alors
    \begin{equation}        \label{EQooSFHVooLxqUEl}
        \sum_k\pr\big( \lambda_k f_{\eL}^k\big)=0.
    \end{equation}
    Pour détailler ce que cela implique, nous calculons ceci :
    \begin{equation}
        (\lambda f_{\eL})(\tau e_i)=\lambda f_{\eL}(\tau e_i)=\sum_a \lambda f_{ia}e_a,
    \end{equation}
    par conséquent \( \pr(\lambda f_{\eL})e_i=\sum_a\pr(\lambda f_{ia})e_a\), et comme \( \pr\) est \( \eK\)-linéaire et que \( f_{ai}\in \eK\),
    \begin{equation}
        \pr(\lambda f_{\eL})e_i=\pr(\lambda)\sum_a f_{ai}e_a=\pr(\lambda)\pr(f_\eL)e_i=\pr(\lambda)f(e_i).
    \end{equation}
    Appliquer la projection \( \pr\) à l'équation \eqref{EQooSFHVooLxqUEl} donne alors \( \sum_k\pr(\lambda)_kf^k=0\). Mais comme les \( f^k\) sont linéairement indépendantes sur \( \eK\) nous avons pour tout \( k\) : \( \pr(\lambda_k)=0\) (égalité dans \( \eK\)). En nous souvenant de la remarque~\ref{REMooBEXGooLgpHzg} nous en déduisons \( \lambda_k=0\) dans \( \eL\).

    Dans les choix de bases faits, l'application \( \tilde g\) a la forme
    \begin{equation}
        \tilde g=\begin{pmatrix}
            \begin{matrix}
                1    &       &       \\
                    &   1    &       \\
                    &       &   1
            \end{matrix}\\
            \begin{matrix}
                *    &   *    &   *    \\
                *    &   *    &   *    \\
                *    &   *    &   *
            \end{matrix}
        \end{pmatrix},
    \end{equation}
    qui est injective.

    Vu que \( \tilde g\) est injective, \( \mu\) est le polynôme minimal de \( f_{\eL}\) et donc \( \mu=\mu_{\eL}\).
\end{proof}

%+++++++++++++++++++++++++++++++++++++++++++++++++++++++++++++++++++++++++++++++++++++++++++++++++++++++++++++++++++++++++++
\section{Frobenius et Jordan}
%+++++++++++++++++++++++++++++++++++++++++++++++++++++++++++++++++++++++++++++++++++++++++++++++++++++++++++++++++++++++++++

%---------------------------------------------------------------------------------------------------------------------------
\subsection{Matrice compagnon}
%---------------------------------------------------------------------------------------------------------------------------

\begin{definition}      \label{DEFooOSVAooGevsda}
    Soit le polynôme \( P=X^n-a_{n-1}X^{n-1}-\ldots-a_1X-a_0\) dans \( \eK[X]\). La \defe{matrice compagnon}{matrice!compagnon} de \( P\) est la matrice\nomenclature[A]{\( C(P)\)}{matrice compagnon} donnée par
    \begin{equation}
        C(P)=\begin{pmatrix}
            0    &   \cdots    &   \cdots    &   0    &   a_0\\
            1    &   0    &       &   \vdots    &   a_1\\
            0    &   \ddots    &   \ddots    &   \vdots    &   \vdots\\
            \vdots    &   \ddots    &   \ddots    &   0    &   a_{n-2}\\
            0    &   \cdots    &   0    &   1    &   a_{n-1}
        \end{pmatrix}
    \end{equation}
    si \( n\geq 2\) et par \( (a_0)\) si \( n=1\).

    Une matrice est dite compagnon si elle a cette forme.
\end{definition}

\begin{proposition}
    Si \( f\) est l'endomorphisme associé à la matrice \( C(P)\) nous avons
    \begin{equation}
        f(e_i)=\begin{cases}
            e_{i+1}    &   \text{si } i<n\\
            (a_0,\ldots, a_{n-1})    &    \text{si } i=n.
        \end{cases}
    \end{equation}
    De plus l'endomorphisme \( f\) vérifie \( P(f)e_1=0\).
\end{proposition}

\begin{lemma}[\cite{RapportArgreg2011}] \label{LemkVNisk}
    Un polynôme sur un corps commutatif est le polynôme caractéristique de sa matrice compagnon. En d'autres termes nous avons \( \chi_{C(P)}=P\).
\end{lemma}

\begin{proof}
    Nous notons \( f\) l'endomorphisme associé à \( C(P)\). La propriété \( P(f)e_1=0\) nous indique que le polynôme minimal ponctuel de \( f\) en \( e_1\) divise \( P\). L'ensemble des puissances de \( f\) appliquées à \( e_1\), \( \big( f^i(e_1) \big)_{i=1,\ldots, n-1}\) est libre, donc le polynôme minimal ponctuel en \( e_1\) est de degré \( n\) au minimum. En reprenant les notations du théorème~\ref{ThoCCHkoU}, nous avons \( I_{e_1}=(P)\) parce que \( P\) est de degré minimum dans \( I_{e_1}\) et \( \chi_f\in I_{e_1}\).

    Donc \( P\) divise \( \chi_f\) et est de degré égal à celui de \( \chi_f\). Étant donné qu'ils sont tous deux unitaires, ils sont égaux.
\end{proof}

\begin{remark}  \label{RemmQjZOA}
    Les matrices compagnons ne sont pas les seules dont le polynôme caractéristique est égal au polynôme minimal. En fait les matrices dont le polynôme caractéristique est égale au polynôme minimal sont denses dans les matrices. En effet une matrice dont le polynôme minimal n'est pas égal au polynôme caractéristique a un polynôme caractéristique avec une racine double. Il est possible, en modifiant arbitrairement peu la matrice de séparer la racine double en deux racines distinctes.
\end{remark}

%---------------------------------------------------------------------------------------------------------------------------
\subsection{Réduction de Frobenius}
%---------------------------------------------------------------------------------------------------------------------------

\begin{lemma}       \label{LEMooKUQDooKFeIYq}
    Soit un endomorphisme \( f\colon E\to E\) sur l'espace vectoriel de dimension finie \( n\). Nous notons \( \mu\) et \( \chi\) les polynômes minimal et caractéristique. Si \( f\) est cyclique, alors \( \mu=\chi\).
\end{lemma}
Le théorème~\ref{THOooGLMSooYewNxW} donnera une version plus complète de ce lemme.

\begin{proof}
    Soit \( v\) un vecteur cyclique de \( f\), c'est-à-dire que \( \{ f^k(v) \}_{k=0,\ldots, n-1}\) est libre. Donc si \( P\) est un polynôme de degré jusqu'à \( n-1\) nous ne pouvons pas avoir \( P(f)=0\) parce que, appliqué à \( v\), ce serait une combinaisons nulle non triviale des \( f^k(v)\). Donc le polynôme minimal est au minimum de degré \( n\). Mais le polynôme caractéristique est annulateur de degré \( n\) (Cayley-Hamilton~\ref{ThoCalYWLbJQ}), donc il est le polynôme minimal.
\end{proof}

\begin{theorem}[Réduction de Frobenius \cite{AutourFrobCompa,Vialivs,MoncetIVS}]        \label{THOooDOWUooOzxzxm}
    Soit \( E\), un \( \eK\)-espace vectoriel, et \( f\in \End(E)\). Alors il existe une suite de sous-espaces \( E_1,\ldots, E_r\) stables par \( f\) tels que
    \begin{enumerate}
        \item   \label{ItemmpwjnSs}
            \( E=\bigoplus_{i=1}^rE_i\);
        \item
            pour chaque \( E_i\), l'endomorphisme restreint \( f_i=f|_{E_i}\) est cyclique;
        \item
            si \( \mu_i\) est le polynôme minimal de \( f_i\) alors \( \mu_{i+1}\) divise \( \mu_i\);
    \end{enumerate}
    Une telle décomposition vérifie automatiquement \( \mu_1=\mu_f\) et \( \mu_1\cdots \mu_r=\chi_f\), et la suite \( (\mu_i)_{i=1,\ldots, r}\) ne dépend que de \( f\) et non du choix de la décomposition du point~\ref{ItemmpwjnSs}.
\end{theorem}
   \index{réduction!Frobénius}
   \index{Frobénius!réduction}

Les polynômes \( \mu_i\) sont les \defe{invariants de similitude}{invariant!de similitude} de l'endomorphisme \( f\).

\begin{proof}
    Nous commençons par montrer que si une telle décomposition existe, alors
    \begin{subequations}    \label{subEqzcGouz}
        \begin{align}
            \chi_f=\prod_{i=1}^r\mu_i  \label{EqTaxsvb}\\
            \mu_f=\mu_1
        \end{align}
    \end{subequations}
    où \( \chi_f\) est le polynôme caractéristique de \( f\) et \( \mu_f\) est le polynôme minimal. D'abord le polynôme caractéristique de \( f\) devra être égal au produit des polynômes caractéristique des \( f|_{E_i}\), mais ces derniers endomorphismes étant cycliques\footnote{Définition~\ref{DEFooFEIFooNSGhQE}.}, leurs polynôme caractéristiques sont égaux à leurs polynômes minimaux (lemme~\ref{LEMooKUQDooKFeIYq}). Cela prouve l'égalité \eqref{EqTaxsvb}. Ensuite tous les \( \mu_i\) doivent diviser le polynôme minimal, donc \( \ppcm(\mu_1,\ldots, \mu_r)\) divise \(\mu_f\). Cependant le polynôme minimal doit contenir une et une seule fois chacun des facteurs irréductibles du polynôme caractéristique, et chacun de ces facteurs sont dans les polynômes \( \mu_i\). Par conséquent \( \ppcm(\mu_1,\ldots, \mu_r)=\mu_f\). Mais par ailleurs \( \mu_1=\ppcm(\mu_1,\ldots, \mu_r)\) parce qu'on a supposé \( \mu_{i+1}\divides \mu_i\), donc \( \mu_1=\mu_f\).

    Soit \( d\), le degré du polynôme minimal de \( f\) et \( y\in E\) tel que \( \mu_f=\mu_{f,y}\) (voir lemme~\ref{LemSYsJJj}). Le plus petit espace stable sous \( f\) contenant \( y\) est
    \begin{equation}
        E_y=\Span\{ y,f(y),\ldots, f^{d-1}(y) \}.
    \end{equation}
    Nous notons \( e_i=f^{i-1}(y)\). Notons que les vecteurs donnés forment bien une base de \( E_y\) parce que si les \( e_i\) n'était pas linéairement indépendants, alors nous aurions des \( a_k\) tels que \( \sum_ka_ke_k=0\) et avec lesquels
    \begin{equation}
        \big( \sum_ka_kX^k \big)(f)y=0,
    \end{equation}
    ce qui contredirait la minimalité de \( \mu_{f,y}\).

    La difficulté du théorème est de trouver un complément de \( E_y\) qui soit également stable sous \( f\). Nous commençons par étendre\quext{Pour autant que j'aie compris, cette extension manque dans \cite{AutourFrobCompa}. Corrigez-moi si je me trompe.} \( \{ e_1,\ldots, e_d \}\) en une base \( \{ e_1,\ldots, e_n \}\) de \( E\). Ensuite nous allons montrer que
    \begin{equation}
        E=E_y\oplus F
    \end{equation}
    avec
    \begin{equation}
        F=\{ x\in E\tq  e^*_d\big( f^k(x) \big)=0\forall k\in \eN \}.
    \end{equation}
    Par construction, \( F\) est invariant sous \( f\). Montrons pour commencer que \( E_y\cap F=\{ 0 \}\). Un élément de \( E_y\) s'écrit
    \begin{equation}
        z=a_1e_1+\cdots +a_ke_k
    \end{equation}
    avec \( k\leq d\). Étant donné que \( f\) décale les vecteurs de base, nous avons \( e^*_d\big( f^{d-k}(z) \big)=a_k\). Du coup \( z\in F\) si et seulement si \( a_1=\ldots=a_d=0\), c'est-à-dire que \( E_y\cap F=\{ 0 \}\).

    Nous montrons maintenant que \( \dim F=n-d\). Pour cela nous considérons l'application
    \begin{equation}
        \begin{aligned}
            T\colon \eK[F]&\to E^* \\
            g&\mapsto e^*_d\circ g.
        \end{aligned}
    \end{equation}
    Cette application est injective. En effet un élément général de \( \eK[f]\) est
    \begin{equation}
        g=a_1\id+a_2f+\cdots +a_pf^{p-1}
    \end{equation}
    avec \( p\leq d\). Si \( T(g)=0\), alors nous avons en particulier
    \begin{equation}
        0=T(g)e_{_d-p+1}=e^*_d(a_1e_{d-p+1}+a_2e_{d-p+2}+\cdots +a_pe_d)=a_p.
    \end{equation}
    Donc \( a_p=0\) et en appliquant maintenant \( T(g)\) à \( e_{d-p}\) nous obtenons \( a_{p-1}=0\). Au final nous trouvons que \( g=0\) et donc que \( T\) est injective.

    Étant donné que \( \dim\eK[f]=d\) et que \( T\) est injective, \( \dim\Image(T)=d\). Nous regardons l'orthogonal de l'image :
    \begin{subequations}
        \begin{align}
            (\Image(T))^{\perp}&=\{ x\in E\tq T(g)x=0\forall g\in\eK[f] \}\\
            &=\{ x\in E\tq e^*_d\big( g(x) \big)=0\forall g\in \eK[f] \}\\
            &=F.
        \end{align}
    \end{subequations}
    Par conséquent \( F^{\perp}=\Image(T)\). Vu que \( \dim\Image(T)=d\), nous avons donc \( \dim F=n-d\) et il est établi que \( E=E_y\oplus F\).

    Nous avons donc trouvé \( F\), stable par \( f\) et tel que \( E=E_y\oplus F\). Nous devons maintenant nous assurer que cette décomposition tombe bien pour les polynômes minimaux. Si \( P_1\) est le polynôme minimal de \( f|_{E_yj}\), alors par le lemme~\ref{LemAGZNNa} nous avons \( P_1=\mu_{f,y}=\mu_f\) parce que \( f|_{E_y}\) est cyclique sur \( E_y\). Mettons \( P_2\), le polynôme minimal de \( f|_F\). Étant attendu que \( F\) est stable par \( f\), le polynôme \( P_2\) divise \( P_1\). En recommençant la construction sur \( F\), nous construisons un nouvel espace \( F'\) stable sous \( F\) et vérifiant \( \mu_{f|_{F'}}=P_2\), etc.

    Nous passons maintenant à la partie unicité du théorème. Soient deux suites \( F_1,\ldots, F_r\) et \( G_1,\ldots, G_s\) de sous-espaces stables par \( f\) et vérifiant
    \begin{enumerate}
        \item
            \( E=\bigoplus_{i=1}^rF_i\),
        \item
            \( f|_{F_i}\) est cyclique,
        \item
            \( \mu_{f|_{F_{i+1}}}\) divise \( \mu_{f|_{F_i}}\),
    \end{enumerate}
    et, \emph{mutatis mutandis}, les mêmes conditions pour la famille \( \{ G_i \}\). Nous posons \( P_i=\mu_{f_{F_i}}\) et \( Q_i=\mu_{f|_{G_i}}\). Nous allons montrer par récurrence que \( P_i=Q_i\) et \( \dim F_i=\dim G_i\). Il ne sera cependant pas garanti que \( F_i=G_i\). D'abord, \( P_1=Q_1\) parce qu'ils sont tous deux égaux à \( \mu_f\) par les relations \eqref{subEqzcGouz}. Nous supposons que \( P_i=Q_i\) pour \( i\leq 1\leq j-1\) et nous tentons de montrer que \( P_j=Q_j\).

    Nous avons
    \begin{equation}    \label{EqMrCtZO}
        P_j(f)=P_j(f)|_{F_1}\oplus\ldots\oplus P_j(f)|_{F_{j-1}}.
    \end{equation}
    En effet étant donné que \( P_{j+k}\) divise \( P_j\), nous avons\footnote{En vertu du lemme~\ref{LemQWvhYb}.} \( P_{j}(f)=A(f)\circ P_{j+k}(f)\), mais \( P_{j+k}(f)F_{j+k}=0\), donc \( P_j(f)F_{j+k}=0\). Les espaces \( G_i\) n'ayant à priori aucun rapport avec les polynômes \( P_i\), nous écrivons
    \begin{equation}    \label{EqJreLiO}
        P_j(f)=P_j(f)|_{G_1}\oplus\ldots\oplus P_j(f)|_{G_{j-1}}\oplus P_j(f)|_{G_j}\oplus\ldots\oplus P_j(f)|_{G_s}.
    \end{equation}
    Pour \( 1\leq i\leq j-1\), nous avons supposé \( P_i=Q_i\). Étant donné que \( f|_{F_i}\) est semblable à \( C_{_i}\) et \( f|_{G_i}\) est semblable à \( C_{Q_i}\), la matrice de \( f|_{E_i}\) est semblable à la matrice de \( f|_{G_i}\). En particulier,
    \begin{equation}
        \dim P_j(f)F_i=\dim P_j(f)G_i.
    \end{equation}
    En prenant les dimensions des images dans les égalités \eqref{EqMrCtZO} et \eqref{EqJreLiO}, nous trouvons que
    \begin{equation}
        P_j(f)|_{G_j}=\ldots=P_j(f)|_{G_s}=0.
    \end{equation}
    Par conséquent \( P_j\in I_{f|G_j}\) et donc \( P_j\) divise \( Q_j\), qui est générateur de \( I_{f|_{G_j}}\). La situation étant symétrique entre \( P\) et \( Q\), nous montrons de même que \( Q_j\) divise \( P_j\) et donc que \( P_j=Q_j\).

    Ceci achève la démonstration du théorème de réduction de Frobenius.

\end{proof}

\begin{remark}      \label{REMooPVLEooYDRXQI}
    Sous forme matricielle, ce théorème dit que toute matrice est semblable à une matrice de la forme bloc-diagonale
    \begin{equation}
        f=\begin{pmatrix}
            C_{\mu_1}    &       &       \\
                &   \ddots    &       \\
                &       &   C_{\mu_r}
        \end{pmatrix}
    \end{equation}
    où les \( C_{\mu_i}\) sont les matrices compagnon (définition~\ref{DEFooOSVAooGevsda}).

    En particulier, et ceci est très important, deux applications sont semblables si et seulement si elles ont même suite d'invariants de similitude.
\end{remark}


\begin{remark}
    Si nous travaillons sur \( \eR\), la réduite de Frobenius restera une matrice réelle, même si les valeurs propres sont complexes. En effet le procédé de Frobenius ne regarde absolument pas les valeurs propres, mais seulement les facteurs irréductibles du polynôme caractéristique. La réduite de Frobenius ne tente pas de résoudre ces polynômes, mais se contente d'en utiliser les matrices compagnon.

    La situation sera différente dans le cas de la forme normale de Jordan.
\end{remark}

%---------------------------------------------------------------------------------------------------------------------------
\subsection{Forme normale de Jordan}
%---------------------------------------------------------------------------------------------------------------------------

Il existe une preuve directe de la réduction de Jordan ne nécessitant pas la réduction de Frobenius\cite{LecLinAlgAllen}. Cette dernière passe par les espaces caractéristiques\footnote{Aussi appelés «espaces propres généralisés».} et est à mon avis plus compliquée que la démonstration de Frobenius elle-même. Nous allons donc nous contenter de donner la réduction de Jordan comme un cas particulier de Frobenius.

\begin{theorem}[Réduction de Jordan]        \label{ThoGGMYooPzMVpe}
    Soit \( E\) un espace vectoriel sur \( \eK\), et \( f\in\End(E)\) un endomorphisme dont le polynôme caractéristique \( \chi_f\) est scindé\footnote{C'est pour cette hypothèse que \( \eK=\eR\) n'est pas le bon cadre.}. Il existe une base de \( E\) dans laquelle la matrice de \( f\) s'écrit sous la forme
    \begin{equation}
        M=\begin{pmatrix}
            J_{n_1}(\lambda_1)    &       &       \\
                &   \ddots    &       \\
                &       &   J_{n_k}(\lambda_k)
        \end{pmatrix}
    \end{equation}
    où les \( \lambda_i\) sont les valeurs propres de \( f\) (avec éventuelle répétitions) et \( J_n(\lambda)\) représente le bloc \( n\times n\)
    \begin{equation}
        J_n(\lambda)=\begin{pmatrix}
            \lambda    &   1    &       &       &   \\
                &   \lambda    &   1    &       &   \\
                &       &   \lambda    &       &   \\
                &       &       &   \ddots    &   1\\
                &       &       &       &   \lambda
        \end{pmatrix}.
    \end{equation}
    En d'autres termes, \( J_n(\lambda)_{ii}=\lambda\) et \( J_n(\lambda)_{i-1,i}=1\).
\end{theorem}
\index{réduction!Jordan}
\index{Jordan!réduction}

\begin{proof}
    Nous commençons par le cas où \( f\) est nilpotente; nous notons \( M\) sa matrice. Dans ce cas la seule valeur propre est zéro et le polynôme caractéristique est \( X^m\) pour un certain \( m\). Nous savons par le lemme~\ref{LemkVNisk} que (la matrice de) \( f\) est semblable à sa matrice compagnon. En l'occurrence pour \( f\) nous avons
    \begin{equation}
        C_{X^m}=\begin{pmatrix}
             0   &       &       &  0     \\
             1   &   \ddots    &       &   \vdots    \\
                &   \ddots    &   \ddots    &    \vdots   \\
                &       &   1    &   0
         \end{pmatrix}.
    \end{equation}
    Ensuite le changement de base (qui est une similitude) \( (e_1,\ldots, e_n)\mapsto(e_n,\ldots, e_1)\) montre que \( C_{X^m}\) est semblable à un bloc de Jordan \( J_m(0)\).

    Supposons à présent que \( f\) ne soit pas nilpotente. Par l'hypothèse de polynôme caractéristique scindé, nous supposons que \( f\) a \( m\) valeurs propres distinctes et que son polynôme caractéristique est
    \begin{equation}
        \chi_f=(X-\lambda_1)^{l_1}\ldots (X-\lambda_m)^{l_m}.
    \end{equation}
    Le lemme des noyaux (théorème~\ref{ThoDecompNoyayzzMWod}) nous enseigne que
    \begin{equation}
        E=\bigoplus_{i=1}^m\underbrace{\ker(f-\mu_i\mtu)^{l_i}}_{F_i}.
    \end{equation}
    La restriction de \( f-\lambda_i\mtu\) à \( F_i\) est par construction un endomorphisme nilpotent, et donc peut s'écrire comme un bloc de Jordan avec des zéros sur la diagonale. En utilisant la décomposition
    \begin{equation}
        f|_{F_i}=(f-\lambda_i\mtu)|_{F_i}+\lambda_i\mtu_{F_i},
    \end{equation}
    nous voyons que \( f|_{F_i}\) s'écrit comme un bloc de Jordan avec \( \lambda_i\) sur la diagonale.
\end{proof}

\begin{remark}
    Nous pouvons calculer la forme normale de Jordan pour une matrice complexe ou réelle, mais dans les deux cas nous devons nous attendre à obtenir une matrice complexe parce que les valeurs propres d'une matrice réelle peuvent être complexes. Cependant nous demandons que le polynôme caractéristique de \( f\) soit scindé sur \( \eK\). En pratique, la décomposition de Jordan n'est garantie que sur les corps algébriquement clos, c'est-à-dire sur \( \eC\).

    La suite des invariants de similitude sur laquelle repose Frobenius, elle, est disponible sur tout corps, y compris \( \eR\).
\end{remark}

%+++++++++++++++++++++++++++++++++++++++++++++++++++++++++++++++++++++++++++++++++++++++++++++++++++++++++++++++++++++++++++
\section{Commutant et endomorphismes cycliques}
%+++++++++++++++++++++++++++++++++++++++++++++++++++++++++++++++++++++++++++++++++++++++++++++++++++++++++++++++++++++++++++

%---------------------------------------------------------------------------------------------------------------------------
\subsection{Endomorphisme cyclique}
%---------------------------------------------------------------------------------------------------------------------------

\begin{lemma}\label{LemSGmdnE}
    Si \( A\) est la matrice de l'endomorphisme \( f\) alors nous avons équivalence des propriétés suivantes :
    \begin{enumerate}
        \item
            La matrice \( A\) est cyclique.
        \item
            L'endomorphisme \( f\) est cyclique.
    \end{enumerate}
\end{lemma}

Si \( f\) est un endomorphisme de l'espace vectoriel \( E\) et si \( x\in E\), nous notons
\begin{equation}
    E_{f,x}=\Span\{ f^k(x)\tq k\in \eN \}.
\end{equation}

\begin{definition}
    Soit \( E\) un espace vectoriel de dimension finie sur un corps \( \eK\) et un endomorphisme \( f\colon E\to E\). Le \defe{commutant}{commutant} de \( f\) est l'ensemble des endomorphismes de \( E\) qui commutent avec \( f\) :
    \begin{equation}
        \comC(f)=\{ g\in\aL(E,E)\tq g\circ f=f\circ g \}.
    \end{equation}
\end{definition}
Il n'est pas très compliqué de vérifier que \( \comC(f)\) est un sous-espace vectoriel de \( \aL(E,E)\).

Notons l'inclusion évidente \( \eK[f]\subset \comC(f)\). L'inclusion inverse va un peu nous occuper durant les prochaines pages.

%---------------------------------------------------------------------------------------------------------------------------
\subsection{Commutant : cas diagonalisable}
%---------------------------------------------------------------------------------------------------------------------------

\begin{proposition}[\cite{ooKPTNooMmncYA}]      \label{PROPooRHHEooIRGmtl}
    Si \( f\) est diagonalisable, alors
    \begin{equation}        \label{EQooOTFLooPUKAos}
        \dim\big( \comC(f) \big)=\sum_{\lambda\in\Spec(f)}\dim(E_{\lambda})^2.
    \end{equation}
    où les \( E_{\lambda} \) sont les espaces propres de \( f\).
\end{proposition}

\begin{proof}
    D'abord su \( g\in\comC(f)\) alors \( E_{\lambda}\) est stable par \( g\). En effet si \( v\in E_{\lambda}\) alors \( f\big( g(v) \big)=g\big( f(v) \big)=g(\lambda v)=\lambda g(v)\), ce qui montre que \( g(v)\) est un vecteur propre de \( f\) pour la valeur propre \( \lambda\), et donc que \( g(v)\in E_{\lambda}\).

    Nous considérons ensuite l'application
    \begin{equation}
        \begin{aligned}
            \psi\colon \comC(f)&\to \End(E_1)\times \ldots\times \End(E_r) \\
            g&\mapsto  g|_{E_1}\times \ldots\times g|_{E_r}
        \end{aligned}
    \end{equation}
    qui est bien définie parce que \( g\) se restreint aux espaces propres de \( f\). Nous allons noter \( \psi(g)_{\lambda}\) la restriction de \( g\) à \( E_{\lambda}\).
    \begin{subproof}
    \item[\( \psi\) est injective]

        Supposons que \( g,h\in\comC(f)\) tels que \( \psi(g)=\psi(h)\). Vu que \( f\) est diagonalisable nous pouvons décomposer \( x\in  E\) en ses composantes sur les espaces propres\footnote{Théorème~\ref{ThoDigLEQEXR}\ref{ITEMooZNJFooEiqDYp}.} :
        \begin{equation}
            x=\sum_{\lambda\in\Spec(f)}x_{\lambda}
        \end{equation}
        avec \( x_{\lambda}\in E_{\lambda}\).  Nous avons alors
        \begin{equation}
            g(x)=\sum_{\lambda}g(x_{\lambda})=\sum_{\lambda}\psi(g)_{\lambda}(x_{\lambda}).
        \end{equation}
        Vu que nous avons \( \psi(g)_{\lambda}=\psi(h)_{\lambda}\), nous avons aussi
        \begin{equation}
            g(x)=\sum_{\lambda}\psi(g)_{\lambda}(x_{\lambda})=\sum_{\lambda}\psi(h)_{\lambda}(x_{\lambda})=\sum_{\lambda}h(x_{\lambda})=h(x).
        \end{equation}
        Cela prouve \( g=h\) et donc que \( \psi\) est injective.
    \item[\( \psi\) est surjective]
        Si nous avons pour chaque \( \lambda\in\Spec(f)\) un endomorphisme \( g_{\lambda}\) de \( E_{\lambda}\) alors en posant
        \begin{equation}
            g(x)=\sum_{\lambda\in\Spec(f)}g_{\lambda}(x_{\lambda})
        \end{equation}
        alors nous avons bien
        \begin{equation}
            \psi(g)=\big( g_{\lambda_1},\ldots, g_{\lambda_r} \big).
        \end{equation}
    \end{subproof}
    Nous pouvons donc conclure en écrivant
    \begin{equation}
        \dim\big( \comC(f) \big)=\sum_{\lambda\in\Spec(f)}\dim\big( \End(E_{\lambda}) \big)= \sum_{\lambda\in\Spec(f)}\dim(E_{\lambda})^2.
    \end{equation}
\end{proof}

\begin{remark}      \label{REMooUGFQooVzCOvV}
    Nous avons alors immédiatement
    \begin{equation}
        \dim\big( \comC(f) \big)\geq\dim(E)
    \end{equation}
    lorsque \( f\) est diagonalisable.
\end{remark}

En suivant la notation \eqref{EqooOAYDooEpZELo}, un endomorphisme est cyclique lorsqu'il existe \( x\in E\) tel que \( E_x=E\).

\begin{proposition}[\cite{ooKPTNooMmncYA}]      \label{PropooQALUooTluDif}
    Si \( f\) est un endomorphisme diagonalisable d'un espace vectoriel \( E\) de dimension \( n\). Nous avons équivalence entre les faits suivants.
    \begin{enumerate}
        \item\label{ITEMooSOYYooZVibjrii}
            Le polynôme minimal est égal au polynôme caractéristique : \( \mu_f=\chi_f\)
        \item\label{ITEMooSOYYooZVibjrvi}
            L'endomorphisme \( f\) est cyclique.
        \item\label{ITEMooSOYYooZVibjrv}
            \( \comC(f)=\eK[f]\).
        \item\label{ITEMooSOYYooZVibjriv}
            \( \dim\big( \comC(f) \big)=n\)
        \item\label{ITEMooSOYYooZVibjriii}
            L'endomorphisme \( f\) possède \( n\) valeurs propres distinctes.
        \item   \label{ITEMooSOYYooZVibjri}
            \( \dim\big( \eK[f] \big)=n\)
    \end{enumerate}
\end{proposition}

\begin{proof}
    Le point important de cette proposition sont les équivalences~\ref{ITEMooSOYYooZVibjrii}-\ref{ITEMooSOYYooZVibjrv}. Les autres sont des intermédiaires. En particulier, dans le cas diagonalisable, nous allons voir que le point~\ref{ITEMooSOYYooZVibjriii} est essentiellement une reformulation de~\ref{ITEMooSOYYooZVibjrii}.
    \begin{subproof}
        \item[\ref{ITEMooSOYYooZVibjriv} implique~\ref{ITEMooSOYYooZVibjriii}]
            Par la formule \eqref{EQooOTFLooPUKAos}, les espaces propres de \( f\) ont dimension \( 1\). Par conséquent \( f\) possède \( n\) valeurs propres distinctes.
        \item[\ref{ITEMooSOYYooZVibjriii} implique~\ref{ITEMooSOYYooZVibjri}]
            Le théorème~\ref{ThoDigLEQEXR} nous dit que le polynôme minimal est scindé à racines simples. Vu que \( f\) possède \( n\) valeurs propres distinctes, \( \mu\) est de degré \( n\).  Par l'isomorphisme \( \eK[f]=\eK[X]/(\mu)\) de la proposition~\ref{PropooCFZDooROVlaA} nous avons \(\dim\big( \eK[f] \big)= \deg(\mu)=n\) par la proposition~\ref{CorsLGiEN}.
        \item[\ref{ITEMooSOYYooZVibjri} implique~\ref{ITEMooSOYYooZVibjrii}]
            Par l'isomorphisme \( \eK[f]=\eK[X]/(\mu)\) de la proposition~\ref{PropooCFZDooROVlaA} et la proposition~\ref{CorsLGiEN} nous avons \(n=\dim\big( \eK[f] \big)= \deg(\mu)\). Vu que \( \chi\) est un polynôme annulateur (Caley-Hamilton~\ref{ThoCalYWLbJQ}), il est divisé par \( \mu\). Maintenant \( \mu\) et \( \chi\) sont des polynômes unitaires de degré \( n\) et \( \mu\) divise \( \chi\). Ils sont donc égaux.
        \item[\ref{ITEMooSOYYooZVibjrii} implique~\ref{ITEMooSOYYooZVibjrvi}]
            Le fait que $f$ soit diagonalisable permet d'utiliser le théorème~\ref{ThoDigLEQEXR} pour dire que \( \mu\) est scindé à racines simples. L'égalisation avec \( \chi \) nous permet de dire que \( f\) possède \( n\) valeurs propres distinctes. Soient \( \{ e_1,\ldots, e_n \}\) une base de diagonalisation, et prouvons que le vecteur \( v=e_1+\cdots +e_n\) est cyclique. Nous avons
            \begin{equation}
                f^k(v)=\sum_{i=1}^n\lambda_i^ke_i.
            \end{equation}
            Pour prouver que cette famille (avec \( k=0,\ldots, n-1\)) est libre\footnote{Ce sera alors une base parce que \( n\) vecteurs libres dans un espace de dimension \( n\) est toujours une base, proposition \ref{PROPooVEVCooHkrldw}.} nous en prenons une combinaison linéaire nulle et nous prouvons que les coefficients sont tous nuls. Soit donc
            \begin{equation}
                    0=\sum_{l=0}^{n-1}a_lf^l(v)=\sum_{l=0}^{n-1}a_l\sum_{i=1}^n\lambda_i^le_i=\sum_{i=1}^n\Big( \sum_{l=0}^{n-1}a_l\lambda_i^l \Big)e_i.
            \end{equation}
            Vu que cela est nul, nous avons pour tout \( i\) :
            \begin{equation}
                \sum_{l=0}^{n-1}a_l\lambda_i^l=0.
            \end{equation}
            En posant la matrice \( A_{ij}=\lambda_i^j\), cela revient à étudier le système \( \sum_j A_{ij}a_j=0\). Ce système n'a des solutions non nulles que si \( \det(A)= 0\); sinon il possède une unique solution et elle est \( a_j=0\) pour tout \( j\). Nous devons donc calculer le déterminant
            \begin{equation}
                \det\begin{pmatrix}
                    1&\lambda_1&\lambda_1^2&\cdots&\lambda_1^{n-1}\\
                    \vdots&\vdots&\vdots&&\vdots\\
                    1&\lambda_n&\lambda_n^2&\cdots&\lambda_n^{n-1}
                \end{pmatrix}.
            \end{equation}
            Il s'agit du déterminant de Vandermonde déjà étudié par la proposition~\ref{PropnuUvtj}. Nous avons \( \det(A)=\prod_{1\leq i<j\leq n}(\lambda_j-\lambda_i)\). Cela est bien non nul du fait que toutes les valeurs propres soient distinctes.
        \item[\ref{ITEMooSOYYooZVibjrvi} implique~\ref{ITEMooSOYYooZVibjrv}]
            Soit \( v\) un vecteur cyclique de \( f\). Un endomorphisme \( g\) donne lieu à un polynôme par le fait suivant : il existe des uniques \( a_k\) (\( k=0,\ldots, n-1\)) tels que
            \begin{equation}
                g(v)=\sum_{k=0}^{n-1}a_kf^k(v).
            \end{equation}
            Cela donne une application linéaire
            \begin{equation}
                \begin{aligned}
                    \psi\colon \comC(f)&\to \eK[f] \\
                    g&\mapsto P\tq P(f)v=g(v).
                \end{aligned}
            \end{equation}
            C'est une application injective parce que si \( \psi(g)=0\) alors \( g(v)=0\) et pour tout \( k\) nous avons \( g\big( f^k(v) \big)=f^k\big( g(v) \big)=0\). L'endomorphisme \( g\) s'annulant sur une base, est nul.
        \item[\ref{ITEMooSOYYooZVibjrv} implique~\ref{ITEMooSOYYooZVibjriv}]
            Si \( n_1,\ldots, n_r\) sont les dimensions des différents espaces propres de \( f\), nous avons les inégalités
            \begin{equation}
                \dim\big( \eK[f] \big)=\deg(\mu)\leq n=n_1+\cdots +n_r\leq n_1^2+\cdots +n_r^2=\dim\big( \comC(f) \big).
            \end{equation}
            Par hypothèse d'égalité entre le premier et le dernier terme de cette suite d'inégalités, toutes les inégalités sont des égalités et en particulier \( \dim\big( \comC(f) \big)=n\).
        \end{subproof}
            Nous avons fini de prouver toutes les équivalences demandées.
\end{proof}

\begin{example}
    Pour mieux comprendre pourquoi le fait d'avoir \( n\) valeurs propres distinctes est équivalent à être cyclique, notons que si deux valeurs propres sont identiques, alors un morceau de la matrice de \( f\) serait par exemple \( \begin{pmatrix}
          2  &   0    \\
        0    &   2
    \end{pmatrix}\), et dans ce cas n'importe quelle combinaison \( ae_i+be_j\) reste proportionnelle à elle-même après application de \( f\). Si nous avons des valeurs propres différentes par contre, nous avons par exemples dans \( \eR^2\) la matrice \( \begin{pmatrix}
        1    &   0    \\
        0    &   2
    \end{pmatrix}\) qui donne \( f(e_1+e_2)=e_1+2e_2\). La partie \( \{ e_1+e_2,e_1+2e_2 \}\) est une base.
\end{example}

%---------------------------------------------------------------------------------------------------------------------------
\subsection{Commutant : cas général}
%---------------------------------------------------------------------------------------------------------------------------

Nous considérons encore un espace vectoriel \( E\) de dimension finie \( n\) et un endomorphisme \( f\colon E\to E\). Nous notons \( \mu\) sont polynôme minimal et \( \mu_x\) le polynôme minimal ponctuel en \( x\).

\begin{lemma}[\cite{ooEFHLooXpAOFz,AutourFrobCompa,ooEPEFooQiPESf}]       \label{LEMooDFFDooJTQkRu}
    Nous avons
    \begin{equation}
        \dim\big( \comC(f) \big)\geq \dim(E)
    \end{equation}
\end{lemma}

\begin{proof}
    Si \( f\) est donnée, l'espace \( \comC(f)\) est l'espace des solutions de \( fg=gf\). Supposons avoir choisi une base de \( E\) et notons \( A\) la matrice de \( f\) et \( X\) celle de \( g\). L'équation est \( AX-XA=0\).
    \begin{subproof}
        \item[Si \( A\) est trigonalisable]
            Nous supposons avoir choisi la base de telle sorte que \( A\) soit triangulaire supérieure, et nous allons nous contenter de chercher les solutions \( X\) qui sont également triangulaires supérieure. S'il y en a déjà plus que \( n\), a fortiori le résultat sera vrai.

            Le produit de deux matrices triangulaires supérieures étant une matrice triangulaire supérieure, l'équation \( AX-XA\) contient, pour les coefficients de \( X\), \( n(n+1)/2\) équations. Mais il se fait que les termes diagonaux ne sont pas de vraies équations parce que
            \begin{equation}
                (AX-XA)_{kk}=\sum_i\big( A_{ki}X_{ik}-X_{ki}A_{ik} \big)=\sum_{k\leq i\leq k}(A_{ki}X_{ik}-X_{ki}A_{ik})=0.
            \end{equation}
            Nous avons donc au maximum
            \begin{equation}
                \frac{ n(x+1) }{2}-n
            \end{equation}
            équations linéairement indépendantes pour un minimum de \( n(n+1)/2\) inconnues. L'espace des solutions est donc de dimension au minimum \( n\).

            Cela a l'air d'être une majoration assez large, mais il existe des cas d'égalité.

        \item[Si \( A\) n'est pas trigonalisable]

            La preuve que nous donnons ici est valable même pour les endomorphismes trigonalisables.

            Nous considérons le résultat de Frobenius~\ref{THOooDOWUooOzxzxm}. Nous avons donc la structure suivante:
            \begin{itemize}
                \item
            une décomposition en somme directe \( E=E_1\oplus\ldots\oplus E_r\),
        \item
            les espaces \( E_i\) sont fixés par \( f\),
        \item
            les endomorphismes \( f_i=f|_{E_i}\) sont cycliques
        \item
            le polynôme minimal de \( f_i\) est \( \mu_i\) et \( \prod_{i=1}^r\mu_i=\chi_f\).
            \end{itemize}
            Les endomorphismes \( f_i^k\) commutent évidemment avec \( f_j\), et la partie \( \{ f_i^k \}_{k=0,\ldots, \deg(\mu_i)-1}\) est libre. Libre en tout cas en tant que partie de \( \End(E_i)\). Mais en prolongeant par \( 0\) sur \( E\), ça reste libre en tant que partie de \( \End(E)\).

            Bien entendu les \( f_j^k\) et les \( f_i^k\) (\( i\neq j\)) sont linéairement indépendants dans \( \End(E)\) parce qu'ils n'agissent pas sur les mêmes vecteurs. Donc les endomorphismes \( f_i^{k_i}\) avec \( k_i=0,\ldots, \deg(\mu_i)-1\) forment une partie libre de \( \End(E)\) composée d'endomorphismes qui commutent avec \( f\). Il y en a en tout
            \begin{equation}
                \sum_{i=1}^r\deg(\mu_i)=\deg(\chi_f)=n.
            \end{equation}
            Par conséquent \( \dim\big( \comC(f) \big)\geq \dim(E)\).
    \end{subproof}
\end{proof}

\begin{theorem}[\cite{ooRJDSooXpVtMD}]      \label{THOooGLMSooYewNxW}
    Soit un endomorphisme \( f\colon E\to E\) sur l'espace vectoriel de dimension finie \( n\). Nous notons \( \mu\) et \( \chi\) les polynômes minimal et caractéristique. Nous avons équivalence entre les faits suivants :
    \begin{enumerate}
        \item   \label{ITEMooLRXIooLWaYqJii}
            \( \mu=\chi\),
        \item   \label{ITEMooLRXIooLWaYqJi}
            \( f\) est cyclique,
        \item   \label{ITEMooLRXIooLWaYqJiii}
            \( \comC(f)=\eK[f]\).
    \end{enumerate}
\end{theorem}

\begin{proof}
    Plusieurs implications. Notons que~\ref{ITEMooLRXIooLWaYqJii} implique~\ref{ITEMooLRXIooLWaYqJii} a déjà été démontré par le lemme~\ref{LEMooKUQDooKFeIYq}.
    \begin{subproof}
        \item[\ref{ITEMooLRXIooLWaYqJii} implique~\ref{ITEMooLRXIooLWaYqJi}]
            Conformément à ce que nous permet le lemme~\ref{LemSYsJJj} nous choisissons\footnote{Dans toute la suite, nous devrions écrire \( \mu_f\) et \( \mu_{f,a}\) mais nous omettons d'indiquer explicitement la dépendance en \( f\).} \( a\in E\) de telle sorte à avoir \( \mu_a=\mu\). De plus pour \( x\in E\) nous considérons l'application
            \begin{equation}
                \begin{aligned}
                    \varphi_x\colon \eK[X]&\to E \\
                    P&\mapsto P(f)x.
                \end{aligned}
            \end{equation}
            Nous avons \( \varphi_a(P)=P(f)a\). Étant donné que\( E_{a}\) est engendré par les \( f^k(a)\) nous avons \( \varphi_a\big( \eK[X] \big)=E_a\). De plus l'application \( \varphi_a\) passe aux classes pour \( (\mu_a)\). Pour rappel, un élément de \( \eK[X]/(\mu_a)\) est de la forme
            \begin{equation}
                \bar P=\{ P+Q\mu_a \}_{Q\in \eK[X]}.
            \end{equation}
            Nous considérons donc l'application
            \begin{equation}
                \varphi_a\colon \frac{ \eK[X] }{ (\mu_a) }\to E_a
            \end{equation}
            et nous prouvons que c'est un isomorphisme d'espace vectoriel.
            \begin{subproof}
                \item[Linéaire]
                    Parce que \( (\lambda P+Q)(f)=(\lambda P)(f)+Q(f)\).
                \item[Injectif]
                    Si \( \varphi_a(\bar P)=0\) alors \( \varphi_a(P)=0\) (dans la deuxième, \( \varphi_a\) est l'application définie sur les polynômes et non sur les classes), ce qui montrer que \( P\) est annulateur de \( a\). Mais par définition~\ref{DEFooUICRooBGYhqQ} du polynôme minimal ponctuel, \( \mu_a\) est générateur de \( \ker(\varphi_a)\); donc il existe \( Q\in \eK[X]\) tel que \( P=Q\mu_a\). En d'autres termes, du point de vue du quotient, \( \bar P=0\).
                \item[Surjectif]
                    Si \( x\in E_a\) alors il existe des coefficients \( x_k\in \eK\) tels que \( x=\sum_{k=0}^{\deg(\mu_a)-1}x_kf^k(a)\), c'est-à-dire \( x=P(f)a=\varphi_a(P)\).
            \end{subproof}
            Mais par hypothèse et par choix de \( a\) nous avons \( \mu_a=\mu=\chi\), donc en fait \( E_a=\eK[X]/(\chi)\). Mais nous savons que \( \deg(\chi)=\dim(E)\) et que \( \dim\big( \eK[X]/P \big)=\deg(P)\) par la proposition~\ref{PropooCFZDooROVlaA}. Au final nous avons \( \dim(E_a)=\deg(\chi)=\dim(E)\). Et par conséquent \( E_a=E\). Cela prouver que \( a\) est un vecteur cyclique pour \( f\).

        \item[\ref{ITEMooLRXIooLWaYqJi} implique~\ref{ITEMooLRXIooLWaYqJiii}]
            Soit \( g\in \comC(f)\); nous devons prouver que \( g\) est un polynôme de \( f\). Par hypothèse nous avons un vecteur cyclique que nous notons \( v\). Nous avons un polynôme \( P\) (dépendant de \( g\)) tel que \( g(v)=P(f)v\). Nous allons voir que \( g=P(f)\). Soient \( y\in E\) et \( Q\) un polynôme tels que \( y=Q(f)v\); en notant que \( g\) commute avec \( P(f)\) nous avons
            \begin{equation}
                g(y)=g\big( Q(f)v \big)=Q(f)\big( g(v) \big)=Q(f)\big( P(f)v \big)=P(f)Q(f)v=P(f)y.
            \end{equation}
            Donc \( g=P(f)\).

        \item[\ref{ITEMooLRXIooLWaYqJiii} implique~\ref{ITEMooLRXIooLWaYqJii}]

            Nous avons les inégalités :
            \begin{equation}
                n\leq \dim\big( \comC(f) \big)=\dim\big( \eK[f] \big)=\deg(\mu)\leq \deg(\chi)=n.
            \end{equation}
            La première est le lemme~\ref{LEMooDFFDooJTQkRu}. Toutes les inégalités sont des égalités. En particulier \( \deg(\mu)=n\), ce qui signifie que \( \mu=\chi\) parce que \( \mu\) est un polynôme diviseur de \( \chi\), de même degré que \( \chi\) et unitaire tout comme \( \chi\).

    \end{subproof}
\end{proof}

\begin{corollary}[\cite{ooEPEFooQiPESf}]        \label{CORooAKQEooSliXPp}
    En suivant les notations sur les extensions de corps de base de la section~\ref{SECooAUOWooNdYTZf}, l'endomorphisme \( f\colon E\to F\) est cyclique si et seulement si l'endomorphisme \( f_{\eL}\colon E_{\eL}\to F_{\eL}\) est cyclique.
\end{corollary}

\begin{proof}
    Nous savons par le théorème~\ref{THOooGLMSooYewNxW} qu'un endomorphisme est cyclique si et seulement si son polynôme minimal est égal à son polynôme caractéristique. Or par les propositions~\ref{PROPooZAZFooUFdCUv} et~\ref{PROPooXVZMooXcJrsJ}, nous savons que ces polynômes sont identiques pour \( f\) et pour \( f_{\eL}\).
\end{proof}

\begin{theorem}[Similitude et extension de corps\cite{ooEPEFooQiPESf}]      \label{THOooHUFBooReKZWG}
    Les applications linéaires \( f,g\colon E\to E\) sont semblables si et seulement si \( f_{\eL}\) et \( g_{\eL}\) le sont.
\end{theorem}

\begin{proof}
    En ce qui concerne le sens direct, s'il existe \( m\in\GL(E)\) tel que \( f=mgm^{-1}\) alors il suffit d'appliquer le lemme~\ref{LEMooWZGSooONEnjZ} pour avoir \( f_{\eL}=m_{\eL}g_{\eL}m_{\eL}^{-1}\).

    Nous considérons les invariants de similitude de \( f\) du théorème~\ref{THOooDOWUooOzxzxm}. Il existe une unique suite de polynômes unitaires \( \mu_i\) ($i=1,\ldots, s$) tels que \( \mu_i\divides \mu_{i+1}\) et pour laquelle nous avons une décomposition \( E=E_1\oplus \ldots\oplus E_s\) pour laquelle \( f|_{E_i}\colon E_1\to E_i\) est cyclique et de polynôme minimal \( \mu\).

    Nous avons aussi \( E_{\eL}=(E_1)_{\eL}\oplus\ldots \oplus (E_s)_{\eL}\) et les \( (E_i)_{\eL}\) sont stables sous \( f_{\eL}\) qui y sera également cyclique (corolaire ~\ref{CORooAKQEooSliXPp}). De plus le polynôme minimal de \( f_{\eL}|_{(E_i)_{\eL}}\) est également \( \mu_i\).

    Autrement dit, la suite \( \mu_i\) est également la suite des invariants de similitude de \( f_{\eL}\). La remarque~\ref{REMooPVLEooYDRXQI} nous permet de conclure que \( f\) et \( g\) sont semblables si et seulement si \( f_{\eL}\) et \( g_{\eL}\) le sont.
\end{proof}

% This is part of Mes notes de mathématique
% Copyright (c) 2011-2020
%   Laurent Claessens, Carlotta Donadello
% See the file fdl-1.3.txt for copying conditions.

%+++++++++++++++++++++++++++++++++++++++++++++++++++++++++++++++++++++++++++++++++++++++++++++++++++++++++++++++++++++++++++ 
\section{Hyperplans et formes linéaires}
%+++++++++++++++++++++++++++++++++++++++++++++++++++++++++++++++++++++++++++++++++++++++++++++++++++++++++++++++++++++++++++

\begin{definition}      \label{DEFooEWDTooQbUQws}
    Si \( E\) est un espace vectoriel de dimension \( n\), un \defe{hyperplan}{hyperplan} de \( E\) est un sous-espace vectoriel de dimension \( n-1\).
\end{definition}

\begin{proposition}[\cite{ooDYWYooBJkHuh}]      \label{PROPooVYJUooAWDQrZ}
    À propos d'hyperplans et de formes linéaires sur un espace vectoriel \( E\) sur le corps \( \eK\).
    \begin{enumerate}
        \item
            Si \( \varphi\) est une forme linéaire non nulle, alors \( \ker(\varphi)\) est un hyperplan.
        \item
            Si \( H\) est un hyperplan de \( E\), il existe une forme linéaire dont \( H\) est le noyau :
            \begin{equation}
                H=\ker(\varphi).
            \end{equation}
    \end{enumerate}
\end{proposition}

\begin{proof}
    En deux parties.
    \begin{enumerate}
        \item
            Soit un supplémentaire \( A\) de \( H\). Nous considérons la restriction \( \varphi_A\colon A\to \eK\). Vu que les éléments non nuls de \( A\) sont hors de \( H\), nous avons \( \varphi(x)\neq 0\) dès que \( x\) est non nul dans \( A\). Cela implique que \( \varphi_A\) est surjective.

            D'autre part, \( \varphi_A\) est également injective : si \( \varphi_A(x)=\varphi_A(y)\), alors \( \varphi_A(x-y)=0\), ce qui signifie que \( x-y=0\) ou encore que \( x=y\).

            Donc \( \varphi_A\) est un isomorphisme de \( \eK\)-espaces vectoriels; nous en déduisons par le corolaire \ref{CORooXIPKooWThOsr} que \( A\) est de dimension \( 1\) sur \( \eK\), parce que \( \eK\) est de dimension \( 1\).

        \item
            Nous utilisons le théorème de la base incomplète \ref{ThonmnWKs}\ref{ITEMooJIJSooGuJMdt} pour considérer une base \( \{ e_i \}_{i=1,\ldots, n}\) de \( E\) telle que \( \Span\{ e_1,\ldots, e_{n-1} \}=H\). Nous pouvons alors considérer la forme linéaire définie par
            \begin{equation}
                \varphi(e_i)=\begin{cases}
                    0    &   \text{si }  i=1,\ldots, n-1\\
                    1    &    \text{si } i=n.
                \end{cases}
            \end{equation}
            Cette forme vérifie \( \ker(\varphi)=H\).
    \end{enumerate}
\end{proof}

\begin{proposition}[\cite{ooDSTAooKgSyCN}]
    Soit un espace vectoriel \( E\) de dimension finie \( n\geq 2\). Soit un sous-espace vectoriel \( V\) de \( E\) de dimension \( s\). Alors \( V\) est une intersection de \( n-s\) hyperplans de \( E\).
\end{proposition}

\begin{proof}
    Nous considérons une base de \( V\) que nous complétons\footnote{Théorème de la base incomplète, \ref{ThonmnWKs}\ref{ITEMooJIJSooGuJMdt}.} en une base de \( E\) : si \( x=\sum_{i=1}^nx_ie_i\), nous avons \( x\in V\) si et seulement si \( x_{s+1}=\ldots=x_n=0\). Nous considérons les formes linéaires
    \begin{equation}
        \begin{aligned}
            \varphi_i\colon E&\to \eR \\
            x&\mapsto x_i, 
        \end{aligned}
    \end{equation}
    et nous considérons les parties \( H_i=\ker(\varphi_i)\) qui sont de hyperplans par la proposition \ref{PROPooVYJUooAWDQrZ}. Les \( H_i\) avec \( s+1\leq i\leq n\) sont une famille de \( n-s\) hyperplans qui vérifient
    \begin{equation}
        V=\bigcap_{i=s+1}^n\ker(\varphi_i)
    \end{equation}
    parce que \( x\in \ker(\varphi_i)\) si et seulement si \( x_i=0\).

    Donc \( V\) peut être écrit comme intersection de \( n-s\) hyperplans de \( E\).
\end{proof}

\begin{proposition}[\cite{ooDSTAooKgSyCN}]      \label{PROPooRCLNooJpIMMl}
    Soit un \( \eK\)-espace vectoriel \( E\) de dimension finie \( n\geq 2\). Si \( H_i\) sont des hyperplans de \( E\), alors
    \begin{equation}
        \dim\Big( \bigcup_{i=1}^mH_i \Big)\geq n-m.
    \end{equation}
\end{proposition}

\begin{proof}
    N'oubliez pas de prouver que \( \bigcap_{i=1}^mH_i\) est un espace vectoriel. À part ça, nous faisons une petite récurrence.
    \begin{subproof}
        \item[Pour \( m=2\)]
            Nous savons déjà par la proposition \ref{PROPooQCIXooHIyPPq} que
            \begin{equation}
                \dim(H_1\cap H_2)=\dim(H_1)+\dim(H_2)-\dim(H_1\cap H_2).
            \end{equation}
            De plus \( \dim(H_1+H_2)\leq n\). En remplaçant, par les valeurs,
            \begin{subequations}
                \begin{align}
                    \dim(H_1\cap H_2)&=\dim(H_1)+\dim(H_2)-\dim(H_1\cap H_2)\\
                    &=n-1+n-1-\dim(H_1+H_2)\\
                    &\geq 2n-2-n\\
                    &=n-2.
                \end{align}
            \end{subequations}
            Donc \( \dim(H_1\cap H_2)\geq n-2\).

        \item[La récurrence]
            Nous calculons \( \dim(H_1\cap\ldots\cap H_m\cap H_{m+1})\) en commençant encore par la proposition \ref{PROPooQCIXooHIyPPq} :
            \begin{subequations}
                \begin{align}
                    \dim(H_1\cap \ldots\cap H_m\cap H_{m+1})&=\underbrace{\dim(H_1\cap\ldots\cap H_m)}_{\leq n-m}+\dim(H_{m+1})\\
                        &\qquad -\underbrace{\dim\big( (H_1\cap\ldots H_m)+H_{m+1} \big)}_{\leq n}\\
                    &\geq n-m+(n-1)-n\\
                    &=n-m-1.
                \end{align}
            \end{subequations}
            C'est bon pour la récurrence.
    \end{subproof}
\end{proof}

%---------------------------------------------------------------------------------------------------------------------------
\subsection{Trouver la matrice d'une symétrie donnée}
%---------------------------------------------------------------------------------------------------------------------------
\label{SubSecMtrSym}

Les notions de déterminants, produit scalaire et vectoriels\footnote{Définitions~\ref{LEMooQTRVooAKzucd},~\ref{DefVJIeTFj} et~\ref{DEFooTNTNooRjhuJZ}.} donnent une bonne intuition géométrique des matrices. Nous pouvons alors chercher les matrices de quelques symétriques dans \( \eR^2\) ou \( \eR^3\).

%///////////////////////////////////////////////////////////////////////////////////////////////////////////////////////////
\subsubsection{Symétrie par rapport à un plan}
%///////////////////////////////////////////////////////////////////////////////////////////////////////////////////////////

Comment trouver par exemple la matrice $A$ qui donne la symétrie autour du plan $z=0$ ? La définition d'une telle symétrie est que les vecteurs du plan $z=0$ ne bougent pas, tandis que les vecteurs perpendiculaires changent de signe. Ces informations vont permettre de trouver comment $A$ agit sur une base de $\eR^3$. En effet :
\begin{enumerate}

	\item
		Le vecteur $\begin{pmatrix}
			1	\\
			0	\\
			0
		\end{pmatrix}$ est dans le plan $z=0$, donc il ne bouge pas,

	\item
		le vecteur $\begin{pmatrix}
			0	\\
			1	\\
			0
		\end{pmatrix}$ est également dans le plan, donc il ne bouge pas non plus,

	\item
		et le vecteur $\begin{pmatrix}
			0	\\
			0	\\
			1
		\end{pmatrix}$ est perpendiculaire au plan $z=0$, donc il va changer de signe.

\end{enumerate}
Cela nous donne directement les valeurs de $A$ sur la base canonique et nous permet d'écrire
\begin{equation}
	A=\begin{pmatrix}
		1	&	0	&	0	\\
		0	&	1	&	0	\\
		0	&	0	&	-1
	\end{pmatrix}.
\end{equation}
Pour écrire cela, nous avons juste mit en colonne les images des vecteurs de base. Les deux premiers n'ont pas changé et le troisième a changé.

Et si maintenant on donne un plan moins facile que $z=0$ ? Le principe reste le même : il faudra trouver deux vecteurs qui sont dans le plan (et dire qu'ils ne bougent pas), et puis un vecteur qui est perpendiculaire au plan\footnote{Pour le trouver, penser au produit vectoriel.}, et dire qu'il change de signe.

Voyons ce qu'il en est pour le plan $x=-z$. Il faut trouver deux vecteurs linéairement indépendants dans ce plan. Prenons par exemple
\begin{equation}		\label{EqffudE}
	\begin{aligned}[]
		f_1&=\begin{pmatrix}
			0	\\
			1	\\
			0
		\end{pmatrix},&f_2&=\begin{pmatrix}
			1	\\
			0	\\
			-1
		\end{pmatrix}.
	\end{aligned}
\end{equation}
Nous avons
\begin{equation}
	\begin{aligned}[]
		Af_1&=f_1\\
		Af_2&=f_2.
	\end{aligned}
\end{equation}
Afin de trouver un vecteur perpendiculaire au plan, calculons le produit vectoriel :
\begin{equation}
	f_3=f_1\times f_2=\begin{vmatrix}
		e_1	&	e_2	&	e_3	\\
		0	&	1	&	0	\\
		1	&	0	&	-1
	\end{vmatrix}=-e_1-e_3=\begin{pmatrix}
		-1	\\
		0	\\
		-1
	\end{pmatrix}.
\end{equation}
Nous avons
\begin{equation}
	Af_3=-f_3.
\end{equation}
Afin de trouver la matrice $A$, il faut trouver $Ae_1$, $Ae_2$ et $Ae_3$. Pour ce faire, il faut d'abord écrire $\{ e_1,e_2,e_3 \}$ en fonction de $\{ f_1,f_2,f_3 \}$. La première des équations \eqref{EqffudE} dit que
\begin{equation}
	f_1=e_2.
\end{equation}
Ensuite, nous avons
\begin{equation}
	\begin{aligned}[]
		f_2&=e_1-e_3\\
		f_3&=-e_1-e_3.
	\end{aligned}
\end{equation}
La somme de ces deux équations donne $-2e_3=f_2+f_3$, c'est-à-dire
\begin{equation}
	e_3=-\frac{ f_2+f_3 }{ 2 }
\end{equation}
Et enfin, nous avons
\begin{equation}
	e_1=\frac{ f_2-f_3 }{ 2 }.
\end{equation}

Maintenant nous pouvons calculer les images de $e_1$, $e_2$ et $e_3$ en faisant
\begin{equation}
	\begin{aligned}[]
		Ae_1&=\frac{ Af_2-Af_3 }{ 2 }=\frac{1 }{2}\begin{pmatrix}
			0	\\
			0	\\
			-2
		\end{pmatrix}=\begin{pmatrix}
			0	\\
			0	\\
			-1
		\end{pmatrix},\\
		Ae_2&=Af_1=f_1=\begin{pmatrix}
			0	\\
			1	\\
			0
		\end{pmatrix},\\
		Ae_3&=-\frac{ f_2-f_3 }{ 2 }=-\frac{ 1 }{2}\begin{pmatrix}
			2	\\
			0	\\
			0
		\end{pmatrix}=\begin{pmatrix}
			-1	\\
			0	\\
			0
		\end{pmatrix}.
	\end{aligned}
\end{equation}
La matrice $A$ s'écrit maintenant en mettant les trois images trouvées en colonnes :
\begin{equation}
	A=\begin{pmatrix}
		0	&	0	&	-1	\\
		0	&	1	&	0	\\
		-1	&	0	&	0
	\end{pmatrix}.
\end{equation}

%///////////////////////////////////////////////////////////////////////////////////////////////////////////////////////////
\subsubsection{Symétrie par rapport à une droite}
%///////////////////////////////////////////////////////////////////////////////////////////////////////////////////////////

Le principe est exactement le même : il faut trouver trois vecteurs $f_1$, $f_2$ et $f_3$ sur lesquels on connaît l'action de la symétrie. Ensuite il faudra exprimer $e_1$, $e_2$ et $e_3$ en termes de $f_1$, $f_2$ et $f_3$.

Le seul problème est de trouver les trois vecteurs $f_i$. Le premier est tout trouvé : c'est n'importe quel vecteur sur la droite. Pour les deux autres, il faut un peu ruser parce qu'il faut impérativement qu'ils soient perpendiculaire à la droite. Pour trouver $f_2$, on peut écrire
\begin{equation}
	f_2=\begin{pmatrix}
		1	\\
		0	\\
		x
	\end{pmatrix},
\end{equation}
et puis fixer le $x$ pour que le produit scalaire de $f_2$ avec $f_1$ soit nul. S'il n'y a pas moyen (genre si $f_1$ a sa troisième composante nulle), essayer avec $\begin{pmatrix}
	x	\\
	1	\\
	0
\end{pmatrix}$. Une fois que $f_2$ est trouvé (il y a des milliards de choix possibles), trouver $f_3$ est super facile : prendre le produit vectoriel entre $f_1$ et $f_2$.

%///////////////////////////////////////////////////////////////////////////////////////////////////////////////////////////
\subsubsection{En résumé}
%///////////////////////////////////////////////////////////////////////////////////////////////////////////////////////////
La marche à suivre est

\begin{enumerate}

	\item
		Trouver trois vecteurs $f_1$, $f_2$ et $f_3$ sur lesquels on connaît l'action de la symétrie. Typiquement : des vecteurs qui sont sur l'axe ou le plan de symétrie, et puis des perpendiculaires. Pour la perpendiculaire, penser au produit scalaire et au produit vectoriel.

	\item
		Exprimer la base canonique $e_1$, $e_2$ et $e_3$ en termes de $f_1$, $f_2$, $f_3$.

	\item
		Trouver $Ae_1$, $Ae_2$ et $Ae_3$ en utilisant leur expression en termes des $f_i$, et le fait que l'on connaisse l'action de $A$ sur les $f_i$.

	\item
		La matrice s'obtient en mettant les images des $e_i$ en colonnes.
\end{enumerate}

%+++++++++++++++++++++++++++++++++++++++++++++++++++++++++++++++++++++++++++++++++++++++++++++++++++++++++++++++++++++++++++
\section{Théorème de Burnside}
%+++++++++++++++++++++++++++++++++++++++++++++++++++++++++++++++++++++++++++++++++++++++++++++++++++++++++++++++++++++++++++

\begin{lemma}       \label{LemwXXzIt}
    Soit \( P\), un polynôme sur \( \eK\). Une racine de \( P\) est une racine simple si et seulement si elle n'est pas racine de \( P'\).
\end{lemma}

\begin{theorem}     \label{ThoBurnsideoPuCtS}
    Toute représentation\footnote{Définition \ref{DEFooXVMSooXDIfZV}.} d'un groupe abélien d'exposant fini sur \( \eC^n\) a une image finie.
\end{theorem}

\begin{proof}
    Étant donné que \( G\) est d'exposant fini, il existe \( \alpha\in \eN^*\) tel que \( g^{\alpha}=e\) pour tout \( g\in G\). Le polynôme \( P(X)=X^{\alpha}-1\) est scindé à racines simples. En effet tout polynôme sur \( \eC\) est scindé. Le fait qu'il soit à racines simples provient du lemme~\ref{LemwXXzIt} parce que si \( a^{\alpha}=1\), alors il n'est pas possible d'avoir \( \alpha a^{\alpha-1}=0\).

    Par ailleurs \( P(g)=0\). Le fait que nous ayons un polynôme annulateur de \( g\) scindé à racines simples implique que \( g\) est diagonalisable (théorème~\ref{ThoDigLEQEXR}). Le fait que \( G\) soit abélien montre qu'il existe une base de \( \eC^n\) dans laquelle tous les éléments de \( G\) sont diagonaux. Nous devons par conséquent montrer qu'il existe un nombre fini de matrices de la forme
    \begin{equation}
        \begin{pmatrix}
            \lambda_1    &       &       \\
                &   \ddots    &       \\
                &       &   \lambda_n
        \end{pmatrix}.
    \end{equation}
    Nous savons que \( \lambda_i^{\alpha}=1\) parce que \( g^{\alpha}=\mtu\), par conséquent chacun des \( \lambda_i\) est une racine de l'unité dont il n'existe qu'un nombre fini.
\end{proof}

\begin{theorem}[Burnside\cite{fJhCTE,ooFBZQooXyHIWK}]\label{ThooJLTit}
    Un sous-groupe de \( \GL(n,\eC)\) est fini si et seulement s'il est d'exposant\footnote{Définition~\ref{DefvtSAyb}.} fini.
\end{theorem}
\index{exposant}
\index{racine!de l'unité}
\index{endomorphisme!diagonalisable}

\begin{proof}
    Soit \( G\) un sous-groupe de \( \GL(n,\eC)\). Si \( G\) est fini, l'ordre de ses éléments divise \( | G |\) (corolaire \ref{CorpZItFX} au théorème de Lagrange) et l'exposant est le PPCM qui est donc fini également. Le théorème est déjà démontré dans un sens.

    Dans l'autre sens, nous notons \( e<\infty\) l'exposant de \( G\), et nous allons prouver que l'ensemble \( G\) est fini. Nous commençons par remarquer que tous les éléments de \( G\) sont des racines du polynôme \( X^e-1\), et ensuite nous nous lançons dans le travail.

    \begin{subproof}
        \item[Générateurs]

            Le groupe \( G\) est une partie de \( \eM(n,\eC)\) dont nous considérons l'algèbre engendrée\footnote{Définition \ref{DefkAXaWY}.} \( \mG\). Soit \( C_1,\ldots, C_r\) une famille génératrice de \( \mG\) constituée d'éléments de \( G\) et la fonction
            \begin{equation}
                \begin{aligned}
                    \tau\colon G&\to \eC^r \\
                    A&\mapsto \big( \tr(AC_1),\ldots, \tr(AC_r) \big).
                \end{aligned}
            \end{equation}

        \item[\( \tau\) est injective] Soient \( A,B\in G\) tels que \( \tau(A)=\tau(B)\). Si \( C_i\) est un générateur de \( G\), nous avons \( \tr(AC_i)=\tr(BC_i)\) et par la linéarité de la trace, nous avons
            \begin{equation}    \label{EqnCYmKW}
                \tr(AM)=\tr(BM)
            \end{equation}
            pour tout \( M\in G\). Notons par ailleurs
            \begin{equation}
                N=AB^{-1}-\mtu,
            \end{equation}
            qui est diagonalisable parce que \( AB^{-1}\in G\) et donc est annulé par le polynôme \( X^e-1\) qui est scindé à racines simples. Du coup \( AB^{-1}\) est diagonalisable; posons \( PAB^{-1}P^{-1}=D\), alors \( P\big( AB^{-1}-\mtu \big)P^{-1}=D-\mtu\) qui est encore diagonale. Donc \( N\) est diagonalisable.

            Par ailleurs nous avons
            \begin{subequations}
                \begin{align}
                    \tr\big( (AB^{-1})^p \big)&=\tr\big( AB^{-1}(AB^{-1})^{p-1} \big)\\
                    &=\tr\big( BB^{-1}(AB^{-1})^{p-1} \big) &\text{\eqref{EqnCYmKW}}\\
                    &=\tr\big( (AB^{-1})^{p-1} \big).
                \end{align}
            \end{subequations}
            En continuant nous obtenons
            \begin{equation}
                \tr\big(  (AB^{-1})^p \big)=\tr(\mtu)=n.
            \end{equation}

            D'autre part,
            \begin{equation}
                N^k=(AB^{-1}-\mtu)^k=\sum_{p=0}^k{p\choose k}(-1)^{k-p}(AB^{-1})^p
            \end{equation}
            En prenant la trace, et en tenant compte du fait que \( \tr\big( (AB^{-1})^p \big)=n\),
            \begin{equation}
                \tr(N^k)=\sum_{p=0}^k{p\choose k}(-1)^{k-p}n=n(1-1)^k=0.
            \end{equation}
            Donc la trace de \( N^k\) est nulle et le lemme~\ref{LemzgNOjY} nous enseigne que \( N\) est alors nilpotente. Étant donné qu'elle est aussi diagonalisable, elle est nulle. Nous en concluons que \( AB^{-1}=\mtu\) et donc que \( A=B\). La fonction \( \tau\) est donc injective.

        \item[Nombre fini de valeurs]

            Les éléments de \( G\) sont annulés par \( X^e-1\) qui est un polynôme scindé à racines simples. Dons le polynôme minimal d'un élément de \( G\) est (a fortiori) scindé à racines simples et le théorème~\ref{ThoDigLEQEXR} nous assure alors que ces éléments sont diagonalisables. Du coup les valeurs propres des matrices de \( G\) sont des racines \( e\)ièmes de l'unité. Par conséquent les traces des éléments de \( G\) ne peuvent prendre qu'un nombre fini de valeurs : toutes les sommes de \( n\) racines \( e\)ièmes de l'unité. Mais vu que les \( C_i\) sont dans \( G\), nous avons
            \begin{equation}
                \Image(\tau)=\{ \tr(A)\tq A\in G \}^r,
            \end{equation}
            qui est un ensemble fini. Par conséquent \( G\) est fini parce que \( \tau\) est injective.
    \end{subproof}
\end{proof}

%---------------------------------------------------------------------------------------------------------------------------
\subsection{Théorème de Lie-Kolchin}
%---------------------------------------------------------------------------------------------------------------------------

Contrairement à ce que l'on peut parfois croire, il n'est pas vrai que toute matrice à coefficient réel est diagonalisable, même pas sur \( \eC\). La raison est qu'une telle matrice peut très bien avoir des valeurs propres multiples.

\begin{example} \label{ExBRXUooIlUnSx}
    Le théorème~\ref{ThoDigLEQEXR} nous donne une façon simple de trouver des matrices non diagonalisables sur \( \eC\) : il suffit que le polynôme minimal ne soit pas scindé à racines simples. Par exemple
    \begin{equation}
        A=\begin{pmatrix}
            1    &   1    \\
            0    &   1
        \end{pmatrix},
    \end{equation}
    dont le polynôme caractéristique est \( \chi_A=(1-X)^2\). Ce polynôme n'a manifestement pas des racines simples. Nous pouvons faire le calcul explicite pour montrer que \( A\) n'est pas diagonalisable. D'abord l'unique valeur propre de \( A\) est \( 1\) et nous pouvons sans peine résoudre
    \begin{equation}
        \begin{pmatrix}
            1    &   1    \\
            0    &   1
        \end{pmatrix}\begin{pmatrix}
            x    \\
            y
        \end{pmatrix}=\begin{pmatrix}
            x    \\
            y
        \end{pmatrix}
    \end{equation}
    qui revient au système
    \begin{subequations}
        \begin{numcases}{}
            x+y=x\\
            y=y.
        \end{numcases}
    \end{subequations}
    La première équation donne directement \( y=0\). Le seul espace propre est de dimension \( 1\) et est engendré par \( \begin{pmatrix}
        1    \\
        0
    \end{pmatrix}\).
\end{example}

La remarque~\ref{RemBOGooCLMwyb} donne un exemple un peu plus avancé, qui montre la multiplicité algébrique et géométrique d'une racine d'un polynôme caractéristique.

\begin{lemma}[Trigonalisation simultanée]   \label{LemSLGPooIghEPI}
    Une famille de matrices de \( \GL(n,\eC)\) commutant deux à deux est simultanément trigonalisable.
\end{lemma}
\index{trigonalisation!simultanée}

\begin{proof}
    Commençons par enfoncer une porte ouverte par la proposition~\ref{PropKNVFooQflQsJ} : toutes les matrices de \( \GL(n,\eC)\) sont trigonalisables parce que tous les polynômes sont scindés.

    Nous effectuons la démonstration par récurrence sur la dimension. Si \( n=1\) alors toutes les matrices sont triangulaires et nous ne nous posons pas de questions. Nous supposons donc \( n>1\).

    Soit la famille \( (A_i)_{i\in I}\) dans \( \GL(n,\eC)\) et \( A_0\) un de ses éléments. Nous nommons \( \lambda_1,\ldots, \lambda_r\) les valeurs propres distinctes de \( A_0\). Le théorème de décomposition primaire~\ref{ThoSpectraluRMLok} nous donne la somme directe d'espaces caractéristiques\footnote{Définition~\ref{DefFBNIooCGbIix}.}
    \begin{equation}
        E=F_{\lambda_1}(A_0)\oplus\ldots\oplus F_{\lambda_r}(A_0).
    \end{equation}
    Nous pouvons supposer que cette somme n'est pas réduite à un seul terme. En effet si tel était le cas, \( A_0\) serait un multiple de l'identité parce que \( A_0\) n'aurait qu'une seule valeur propre et les sommes dans la décomposition de Dunford~\ref{ThoRURcpW}\ref{ItemThoRURcpWiii} se réduisent à un seul terme (et \( p_i=\id\)). En particulier les dimensions des espaces \( F_{\lambda}(A_0)\) sont strictement plus petites que \( n\).

    Vu que tous les \( A_i\) commutent avec \( A_0\), les espaces \( F_{\lambda}(A_0)\) sont stables par les \( A_i\) et nous pouvons trigonaliser les \( A_i\) simultanément sur chacun des \( F_{\lambda}(A_0)\) en utilisant l'hypothèse de récurrence.
\end{proof}

\begin{theorem}[Lie-Kolchin\cite{PAXrsMn}]  \label{ThoUWQBooCvutTO}
    Tout sous-groupe connexe et résoluble de \( \GL(n,\eC)\) est conjugué à un groupe de matrices triangulaires.
\end{theorem}
\index{trigonalisation!simultanée}
\index{théorème!Lie-Kolchin}

\begin{proof}
    Soit \( G\) un sous-groupe connexe et résoluble de \( \GL(n,\eC)\).

    \begin{subproof}
        \item[Si sous-espace non trivial stable par \( G\)]

    Nous commençons par voir ce qu'il se passe s'il existe un sous-espace vectoriel non trivial \( V\) de \( \eC^n\) stabilisé par \( G\). Pour cela nous considérons une base de \( \eC^n\) dont les premiers éléments forment une base de \( V\) (base incomplète, théorème~\ref{ThonmnWKs}). Les éléments de \( G\) s'écrivent, dans cette base,
    \begin{equation}    \label{EqGOKTooEaGACG}
        \begin{pmatrix}
            g_1    &   *    \\
            0    &   g_2
        \end{pmatrix}.
    \end{equation}
    Les matrices \( g_1\) et \( g_2\) sont carrés. Nous considérons alors l'application \( \psi\) définie par
    \begin{equation}
        \begin{aligned}
            \psi\colon G&\to \GL(V) \\
            g&\mapsto g_1.
        \end{aligned}
    \end{equation}
    Cela est un morphisme de groupes parce que
    \begin{equation}
        \begin{pmatrix}
            g_1    &   *    \\
            0    &   g_2
        \end{pmatrix}\begin{pmatrix}
            h_1    &   *    \\
            0    &   h_2
        \end{pmatrix}=
        \begin{pmatrix}
            g_1h_1    &   *    \\
            0    &   g_2h_2
        \end{pmatrix},
    \end{equation}
    de telle sorte que \( \psi(gh)=\psi(g)\psi(h)\).

    Le groupe \( \psi(G)\) est connexe et résoluble. En effet \( \psi(G)\) est connexe en tant qu'image d'un connexe par une application continue (proposition~\ref{PropGWMVzqb}). Et il est résoluble en tant qu'image d'un groupe résoluble par un homomorphisme par la proposition~\ref{PropBNEZooJMDFIB}. Vu que \( \psi(G)\) est un sous-groupe résoluble et connexe de \( \GL(V)\) et que la dimension de \( V\) est strictement plis petite que celle de \( \eC^n\), une récurrence sur la dimension indique que \( \psi(G)\) est conjugué à un groupe de matrices triangulaires. C'est-à-dire qu'il existe une base de \( V\) dans laquelle toutes les matrices \( g_1\) (avec \( g\in G\)) sont triangulaires supérieures.

    On fait de même avec l'application \( g\mapsto g_2\), ce qui donne une base du supplémentaire de \( V\) dans laquelle les matrices \( g_2\) sont triangulaires.

    En couplant ces deux bases, nous obtenons une base de \( \eC^n\) dans laquelle toutes les matrices \eqref{EqGOKTooEaGACG} (c'est-à-dire toutes les matrices de \( G\)) sont triangulaires supérieures.

    \item[Sinon]

    Nous supposons à présent que \( \eC^n\) n'a pas de sous-espaces non triviaux stables sous \( G\). Nous posons \( m=\min\{ k\tq D^k(G)=\{ e \} \}\), qui existe parce que \( G\) et résoluble et que sa suite dérivée termine sur \( {e}\) (proposition~\ref{PropRWYZooTarnmm}).

\item[Si \( m=1\)]

    Si \( m=1\) alors \( G\) est abélien et il existe une base de \( G\) dans laquelle toutes les matrices de \( G\) sont triangulaires (lemme~\ref{LemSLGPooIghEPI}). Le premier vecteur d'une telle base serait stable par \( G\), mais comme nous avons supposé qu'il n'y avait pas de sous-espaces non triviaux stabilisés par \( G\), il faut déduire que ce vecteur stable est à lui tout seul non trivial, c'est-à-dire que \( n=1\). Dans ce cas, le théorème est démontré.

\item[Si \( m>1\)]

    Nous devons maintenant traiter le cas où \( m>1\). Nous posons \( H=D^{m-1}(G)\); cela est un sous-groupe normal et abélien de \( G\). Encore une fois le résultat de trigonalisation simultanée~\ref{LemSLGPooIghEPI} donne une base dans laquelle tous les éléments de \( H\) sont triangulaires. En particulier le premier élément de cette base est un vecteur propre commun à toutes les matrices de \( H\).

    Soit \( V\) le sous-espace engendré par tous les vecteurs propres communs de \( H\). Nous venons de voir que \( V\) n'est pas vide. Nous allons montrer que \( V\) est stable par \( G\). Soient \( h\in H\), \( v\in V\) et \( g\in G\) :
    \begin{equation}    \label{EqPMOBooVLIhrJ}
        h\big( g(v) \big)=g\underbrace{g^{-1}hg}_{\in H}(v)=g(\lambda v)=\lambda g(v)
    \end{equation}
    parce que \( v\) est vecteur propre de \( g^{-1} hg\). Ce que le calcul \eqref{EqPMOBooVLIhrJ} montre est que \( g(v)\) est vecteur propre de \( h\) pour la valeur propre \( \lambda\). Donc \( g(v)\in V\) et \( V\) est stabilisé par \( G\). Mais comme il n'existe pas d'espaces non triviaux stabilisés par \( G\), nous en déduisons que \( V=\eC^n\). Donc tous les vecteurs de \( \eC^n\) sont vecteurs propres communs de \( H\). Autrement dit on a une base de diagonalisation simultanée de \( H\).

\item[\( H\) est dans le centre de \( G\)]

    Montrons à présent que \( H\) est dans le centre de \( G\), c'est-à-dire que pour tout \( g\in G\) et \( h\in H\) il faut \( ghg^{-1}=h\). D'abord \( ghg^{-1}\) est une matrice diagonale (parce que elle est dans \( H\)) ayant les mêmes valeurs propres que \( h\). En effet si \( \lambda\) est valeur propre de \( ghg^{-1}\) pour le vecteur propre \( v\), alors
    \begin{subequations}
        \begin{align}
            (ghg^{-1})(v)&=\lambda v\\
            h\big( g^{-1} v \big)&=\lambda \big( g^{-1}v \big),
        \end{align}
    \end{subequations}
    c'est-à-dire que \( \lambda\) est également valeur propre de \( h\), pour le vecteur propre \( g^{-1} v\). Mais comme \( h\) a un nombre fini de valeurs propres, il n'y a qu'un nombre fini de matrices diagonales ayant les mêmes valeurs propres que \( h\). L'ensemble \( \AD(G)h\) est donc un ensemble fini. D'autre part, l'application \( g\mapsto g^{-1}hg\) est continue, et \( G\) est connexe, donc l'ensemble \( \AD(G)h\) est connexe. Un ensemble fini et connexe dans \( \GL(n,\eC)\) est nécessairement réduit à un seul point. Cela prouve que \( ghg^{-1}=h\) pour tout \( g\in G\) et \( h\in H\).

\item[Espaces propres stables pour tout \( G\)]

        Soit \( h\in H\) et \( W\) un espace propre de \( h\) (ça existe non vide parce que \( H\) est triangularisé, voir plus haut). Alors nous allons prouver que \( W\) est stable pour tous les éléments de \( G\). En effet si \( w\in W\) avec \( h(w)=\lambda w\) alors en permutant \( g\) et \( h\),
        \begin{equation}
            hg(w)=g(hw)=\lambda g(w),
        \end{equation}
        donc \( g(w)\) est aussi vecteur propre de \( h\) pour la valeurs propre \( \lambda\), c'est-à-dire que \( g(w)\in W\). Vu que nous supposons que \( \eC^n\) n'a pas d'espaces invariants non triviaux, nous devons conclure que \( W=\eC^n\), c'est-à-dire que \( H\) est composé d'homothéties. C'est-à-dire que pour tout \( h\in H\) nous avons \( h=\lambda_h\mtu\).

    \item[Contradiction sur la minimalité de \( m\)]

        Les éléments d'un groupe dérivé sont de déterminant \( 1\) parce que \( \det(g_1g_2g_1^{-1}g_2^{-1})=1\). Par conséquent pour tout \( h\), le nombre \( \lambda_h\) est une racine \( n\)\ieme de l'unité. Vu qu'il n'y a qu'une quantité finie de racines \( n\)\ieme de l'unité, le groupe \( H\) est fini et connexe et donc une fois de plus réduit à un élément, c'est-à-dire \( H=\{ e \}\). Cela contredit la minimalité de \( m\) et donc produit une contradiction. Nous devons donc avoir \( m=1\).

    \item[Conclusion]

        Nous avons vu que si \( \eC^n\) avait un sous-espace non trivial fixé par \( G\) alors le théorème était démontré. Par ailleurs si \( \eC^n\) n'a pas un tel sous-espace, soit \( m=1\) (et alors le théorème est également prouvé), soit \( m>1\) et alors on a une contradiction.

        Bref, le théorème est prouvé sous peine de contradiction.
    \end{subproof}
\end{proof}

%+++++++++++++++++++++++++++++++++++++++++++++++++++++++++++++++++++++++++++++++++++++++++++++++++++++++++++++++++++++++++++
\section{Retour sur les formes bilinéaires et quadratiques}
%+++++++++++++++++++++++++++++++++++++++++++++++++++++++++++++++++++++++++++++++++++++++++++++++++++++++++++++++++++++++++++


%--------------------------------------------------------------------------------------------------------------------------- 
\subsection{Dégénérescence d'une forme bilinéaire}
%---------------------------------------------------------------------------------------------------------------------------

Soit \( b\), une forme bilinéaire symétrique non dégénérée  sur l'espace vectoriel \( E\) de dimension \( n\) sur \( \eK\) où \( \eK\) est un corps de caractéristique différente de \( 2\). Nous notons \( q\) la forme quadratique associée.

\begin{definition}      \label{DEFooNUBFooLfCqaK}
    Une forme bilinéaire est \defe{non dégénérée}{forme!bilinéaire!non dégénérée} \( b(x,z)=0\) pour tout \( z\) implique \( x=0\).
\end{definition}

\begin{lemma}   \label{LemyKJpVP}
    Soit \( b\) une forme bilinéaire non dégénérée. Si \( x\) et \( y\) sont tels que \( b(x,z)=b(y,z)\) pour tout \( z\), alors \( x=y\).
\end{lemma}

\begin{proof}
    C'est immédiat du fait de la linéarité en le premier argument et de la non-dégénérescence : si \( b(x,z)-b(y,z)=0\) alors
    \begin{equation}
        b(x-y,z)=0
    \end{equation}
    pour tout \( z\), ce qui implique \( x-y=0\).
\end{proof}

\begin{proposition}     \label{PROPooQHHPooSqpgcb}
    Une forme bilinéaire est non-dénénérée\footnote{Définition \ref{DEFooNUBFooLfCqaK}.} si et seulement si sa matrice associée est inversible.
\end{proposition}

\begin{proof}
    Nous savons que la matrice associée est symétrique et qu'elle peut donc être diagonalisée (théorème~\ref{ThoeTMXla}). En nous plaçant dans une base de diagonalisation, nous devons prouver que la forme est non-dégénérée si et seulement si les éléments diagonaux de la matrice sont tous non nuls.

    Écrivons \( b(x,z)\) en choisissant pour \( z\) le vecteur de base \( e_k\) de composantes \( (e_k)_j=\delta_{kj}\) :
    \begin{equation}
            b(x,e_k)=\sum_{ij}x_i(e_k)_j
            =\sum_i b_{ik}x_i
            =b_{kk}x_k.
    \end{equation}
    Si \( b\) est dégénérée et si \( x\) est un vecteur non nul (disons que la composante \( x_i\) est non nulle) de \( E\) tel que \( b(x,z)=0\) pour tout \( z\in E\), alors \( b_{ii}=0\), ce qui montre que la matrice de \( b\) n'est pas inversible.

    Réciproquement si la matrice de \( b\) est inversible, alors tous les \( b_{kk}\) sont différents de zéro, et le seul vecteur \( x\) tel que \( b_{kk}x_k=0\) pour tout \( k\) est le vecteur nul.
\end{proof}

%--------------------------------------------------------------------------------------------------------------------------- 
\subsection{Isométries}
%---------------------------------------------------------------------------------------------------------------------------

Voici un théorème pas toujours bien énoncé dans les cours de physique qui font de la relativité. Au moment de «prouver» les transformations de Lorentz\footnote{Théorème \ref{THOooYHDWooWxVovH}.}, beaucoup oublient de justifier pourquoi elles devraient être linéaires.
\begin{theorem}[\cite{ooQFKAooFnllQU}]     \label{ThoDsFErq}
    Une isométrie d'une forme bilinéaire\footnote{Définition \ref{DEFooIQURooMeQuqX}.} non dégénérée est linéaire.
\end{theorem}

\begin{proof}
    Soient une forme bilinéaire non-dégénérée \( b\) sur l'espace vectoriel \( E\) ainsi qu'une isométrie $f$ pour icelle. Soit \( z\in E\); étant donné que \( f\) est bijective nous pouvons considérer l'élément \( f^{-1}(z)\in E\) et calculer
    \begin{subequations}
        \begin{align}
            b\big( f(x+y),z \big)&=b\big( f(x+y),f(f^{-1}(z)) \big)\\
            &=b(x+y,f^{-1}(z))\\
            &=b(x,f^{-1}(z))+b(y,f^{-1}(z))\\
            &=b(f(x),z)+b(f(y),z)\\
            &=b\big( f(x)+f(y),z \big),
        \end{align}
    \end{subequations}
    donc \( f(x+y)=f(x)+f(y)\) par le lemme~\ref{LemyKJpVP}.

    De la même façon on trouve \( b\big( f(\lambda x),z \big)=b\big( \lambda f(x),z \big)\) qui prouve que \( f(\lambda x)=\lambda f(x)\) et donc que \( f\) est linéaire.
\end{proof}

\begin{example}
    Une isométrie peut ne pas être linéaire quand la forme bilinéaire est dégénérée. Par exemple pour la forme bilinéaire sur \( \eR^2\) donnée par
    \begin{equation}
        b\big( (a,b),(x,y) \big)=ax,
    \end{equation}
    nous pouvons faire
    \begin{equation}
        f(x,y)=\begin{pmatrix}
            x    \\ 
            \lambda(x,y)    
        \end{pmatrix}
    \end{equation}
    où \( \lambda\) est n'importe quoi.
\end{example}

\ifbool{isGiulietta}{
\begin{remark}
    Des preuves alternatives.
    \begin{enumerate}
        \item
            En utilisant un peut plus d'indices et un peu plus de mots comme «tenseurs», peut être trouvée  \href{http://physics.stackexchange.com/questions/12664/proving-that-interval-preserving-transformations-are-linear}{ici}. Le fait que la preuve donnée soit tensorielle me fait penser que le résultat peut encore être généralisé.
        \item
            Et encore une autre preuve, utilisant des techniques de groupes de Lie sera la proposition~\ref{PROPooDVIWooAFDNPy}.
    \end{enumerate}
\end{remark}
}
{}

\begin{theorem}
    Soit un espace vectoriel \( E\) muni d'une forme quadratique \( q\). Soit une isométrie \( f\colon E\to E\) pour \( q\). Alors
    \begin{enumerate}
        \item
            si \( f(0)=0\), alors \( f\) est linéaire;
        \item
            si \( f(0)\neq 0\) alors \( f\) est affine\footnote{Définition \ref{DEFooUAWZooXcMKve}.}.
            
    \end{enumerate}
\end{theorem}

\begin{proof}
    Nous considérons la forme bilinéaire associée \( b\). Si \( f(0)=0\), nous savons par le lemme~\ref{LemewGJmM} que \( b\big( f(x),f(y) \big)=b(x,y)\). La proposition \ref{ThoDsFErq} nous dit alors que \( f\) est linéaire.


    Si \( f(0)\neq 0\), alors nous posons \( g(x)=f(x)-f(0)\) qui vérifie \( g(0)=0\) et
    \begin{equation}
        q\big( g(x)-g(y) \big)=q\big( f(x)-f(0)-f(y)+f(0) \big)=q(x-y).
    \end{equation}
    Nous pouvons donc appliquer le premier point à \( g\), déduire que \( g\) est linéaire et donc que \( f\) est affine. C'est la caractérisation du lemme \ref{LEMooZZAIooOMiayy} des fonctions affines.
\end{proof}

Nous pouvons maintenant particulariser tout cela au cas de \( \eR^n\) muni du produit scalaire usuel et de la norme associée pour voir quel résultat nous avons à peine prouvé.

\begin{lemma}[\cite{ooYPVPooYGSlNU}]        \label{LEMooJPYZooHETCqt}
    Une isométrie d'un espace vectoriel normé de dimension finie est bijective.
\end{lemma}

\begin{proof}
    Si \( f\colon E\to E\) est une isométrie, elle est linéaire par le théorème~\ref{ThoDsFErq}. Elle vérifie également \( \| f(x) \|=\| x \|\), et donc \( f(x)=0\) si et seulement si \( x=0\), c'est-à-dire que \( f\) est injective. Elle est alors bijective par le corolaire~\ref{CORooCCXHooALmxKk} du théorème du rang.
\end{proof}

Nous notons ici \( T(n)\) le groupe des translations sur \( \eR^n\). Un élément de \( T(n)\) est une translation \( \tau_v\) donnée par un vecteur \( v\) et agissant sur \( \eR^n\) par
\begin{equation}
    \begin{aligned}
        \tau_v\colon \eR^n&\to \eR^{n} \\
        x&\mapsto x+v.
    \end{aligned}
\end{equation}
Ce groupe est isomorphe au groupe abélien \( (\eR^n,+)\), et nous allons souvent identifier \( \tau_v\) à \( v\).

Vous savez par culture générale que les isométries de \( \eR^n\) pour le produit scalaire usuel sont les matrices orthogonales. En voici une petite généralisation (pensez à \( \eta=\mtu\) dans le cas du produit scalaire usuel).
\begin{proposition}     \label{PROPooSYQMooEnZFdp}
    Soit une forme bilinéaire \( b\) sur \( \eR^n\) de matrice symétrique \( \eta\). Si \( A\) est la matrice d'une application linéaire \( \eR^n\to \eR^n\) telle que
    \begin{equation}
        b(Ax,Ay)=b(x,y)
    \end{equation}
    pour tout \( x,y\in\eR^n\), alors
    \begin{equation}
        A^t\eta A=\eta.
    \end{equation}
\end{proposition}

\begin{proof}
    En suivant la formule générale \eqref{EQooQFMWooVKVLMx},
    \begin{equation}
            b(Ax,Ay)=\sum_{ij} \eta_{ij} (Ax)_i(Ay)_j=\sum_{ijkl}\eta_{ij}A_{ik}A_{jl}x_ky_l.
    \end{equation}
    En imposant que ce soit égal à \( \sum_{kl}\eta_{kl}\eta_{kl}x_ky_l\) pour tout \( x,y\) nous avons la contrainte
    \begin{equation}
        \sum_{ij}\eta_{ij}A_{ik}A_{jl}=\eta_{kl}
    \end{equation}
    qui signifie exactement \( A^t\eta A=\eta\).
\end{proof}

%---------------------------------------------------------------------------------------------------------------------------
\subsection{Pseudo-réduction simultanée}
%---------------------------------------------------------------------------------------------------------------------------

\begin{corollary}[Pseudo-réduction simultanée\cite{JMYQgLO}]  \label{CorNHKnLVA}
    Soient \( A,B\in \gS(n,\eR)\) avec \( A\) définie positive\footnote{Définition~\ref{DefAWAooCMPuVM}.}. Alors il existe \( Q\in \GL(n,\eR)\) telle que \( Q^tBQ\) soit diagonale et \( Q^tAQ=\mtu\).
\end{corollary}

\begin{proof}
    Nous allons noter \( x\cdot y\) le produit scalaire usuel de \( \eR^n\) et \( \{ e_i \}_{i=1,\ldots, n}\) sa base canonique.

    Vu que \( A\) est définie positive, l'expression \( \langle x, y\rangle =x\cdot Ay\) donne un produit scalaire sur \( \eR^n\). Nous avons donc deux produits scalaires sur \( \eR^n\), et nous allons travailler avec les deux.
    
    La proposition \ref{PropUMtEqkb} appliquée à l'espace euclidien \( (\eR^n,\langle ., .\rangle )\) dit qu'il existe une base de \( \eR^n\) orthonormée \( (f_i )_{i=1,\ldots, n}\) pour ce produit scalaire. Nous considérons l'application linéaire \( P\) définie par
    \begin{equation}
        Pe_i=f_i.
    \end{equation}

    Nous démontrons à présent que \( P^tAP=\mtu\). Pour cela, nous calculons
    \begin{subequations}
        \begin{align}
            \delta_{ij}&=\langle f_i, f_j\rangle    \label{SUBEQooGZDJooVMuWNn} \\
            &=f_i\cdot Af_j\\
            &=Pe_i\cdot APe_j       \\
            &=e_i\cdot P^tAPe_j     \label{SUBEQooQNVUooNbyIzM}\\
            &=(P^tAP)_{ij}.     \label{SUBEQooITBKooCEmqxx}
        \end{align}
    \end{subequations}
    Justifications :
    \begin{itemize}
        \item Pour \eqref{SUBEQooGZDJooVMuWNn}, la base \( (f_j)\) est orthonormée pour le produit scalaire \( \langle ., .\rangle \).
        \item Pour \eqref{SUBEQooQNVUooNbyIzM}, la proposition \ref{PROPooNARVooEuhweD} sur la transposée.
        \item Pour \eqref{SUBEQooITBKooCEmqxx}, la formule du produit scalaire usuel pour avoir les éléments de matrice, proposition \ref{PROPooZKWXooWmEzoA}.
    \end{itemize}
    La matrice \( P^tBP\) est une matrice symétrique, donc le théorème spectral~\ref{ThoeTMXla} nous donne une matrice \( R\in \gO(n,\eR)\) telle que \( R^tP^tBPR\) soit diagonale. En posant maintenant \( Q=PR\) nous avons la matrice cherchée.
\end{proof}

\begin{remark}
    Plusieurs remarques
    \begin{enumerate}
        \item

            Nous n'avons pas prouvé l'existence d'une matrice \( P\) telle que \( P^{-1}BP\) et \( P^{-1}AP\) soient diagonales. Au contraire, nous avons \( Q^tBQ\) et \( Q^tAQ\) qui sont diagonales. Tant que \( Q\) n'est pas orthogonales, ce n'est pas la même chose.

            Autrement dit, nous n'avons pas ici une réelle diagonalisation, parce que les matrices \( A\) et \( B\) ne sont pas semblables à des matrices diagonales. Voir les définitions~\ref{DefCNJqsmo} (diagonalisable) et~\ref{DefCQNFooSDhDpB} (semblable).

            C'est pour cela que nous parlons de \emph{pseudo}-diagonalisation.

        \item

            Dans le même ordre d'idée, la démonstration de la pseudo-diagonalisation simultanée parle clairement de formes bilinéaires, et non d'endomorphismes. Or en comparant les lois de transformations \eqref{ooWKTYooOJfclT} et \eqref{EQooZUVTooKjqnJj}, nous voyons bien que la réduction en passant par \( Q^tAQ\) est bien une réduction de forme bilinéaire et non une réduction d'endomorphismes.

        \item

            Nous avons prouvé la pseudo-réduction simultanée comme corolaire du théorème de diagonalisation des matrices symétriques~\ref{ThoeTMXla}. Il aurait aussi pu être vu comme un corolaire du théorème spectral~\ref{ThoRSBahHH} sur les opérateurs autoadjoints via son corolaire~\ref{CorSMHpoVK}.
    \end{enumerate}
\end{remark}

%---------------------------------------------------------------------------------------------------------------------------
\subsection{Topologie}
%---------------------------------------------------------------------------------------------------------------------------

La topologie considérée sur \( Q(E)\) est celle de la norme
\begin{equation}    \label{EqZYBooZysmVh}
    N(q)=\sup_{\| x \|_E=1}| q(x) |,
\end{equation}
qui du point de vue de \( S(n,\eR)\) est
\begin{equation}    \label{EQooJETQooIjxRWu}
    N(A)=\sup_{\| x \|_E=1}| x^tAx |.
\end{equation}
Notons que à droite, c'est la valeur absolue usuelle sur \( \eR\).

\begin{proposition} \label{PropFSXooRUMzdb}
    Soit \( \{ e_i \}\) une base de \( E\). L'application
    \begin{equation}
        \begin{aligned}
            \phi\colon Q(E)&\to S(n,\eR) \\
            q&\mapsto \big(   b(e_i,e_j)   \big)_{i,j}
        \end{aligned}
    \end{equation}
    où \( b\) est forme bilinéaire associée à \( q\) est une bijection linéaire et continue\footnote{Pour les topologies des normes \eqref{EqZYBooZysmVh} et \eqref{EQooJETQooIjxRWu}.}.
\end{proposition}

\begin{proof}
    Si \( \phi(q)=\phi(q')\); alors
    \begin{equation}
        q(x)=\sum_{i,j}\phi(q)_{ij}x_ix_j=\sum_{i,j}\phi(q')_{ij}x_ix_j=q'(x).
    \end{equation}
    Donc \( q=q'\). L'application \( \phi\) est donc injective

    De plus elle est surjective parce que si \( B\in S(n,\eR)\) alors la forme quadratique
    \begin{equation}
        q(x)=\sum_{i,j}B_{ij}x_ix_j
    \end{equation}
    a évidemment \( B\) comme matrice associée. L'application \( \phi\) est donc surjective.

    Notre application \( \phi\) est de plus linéaire parce que l'association d'une forme quadratique à la forme bilinéaire associée est linéaire.

    En ce qui concerne la continuité, nous la prouvons en zéro en considérant une suite convergente \( q_n\stackrel{Q(E)}{\longrightarrow}0\). C'est-à-dire que
    \begin{equation}
        \sup_{\| x \|=1}| q_n(x) |\to 0.
    \end{equation}
    Nous rappelons l'identité de polarisation :
    \begin{equation}
        b_n(x,y)=\frac{ 1 }{2}\big( q_n(x-y)-q(x)-q(y) \big).
    \end{equation}
    En ce qui concerne deux des trois termes, il n'y a pas de problèmes :
    \begin{equation}
        \big| \phi(q_n)_{ij} \big|=\big| b_n(e_i,e_j) \big|\leq\frac{ 1 }{2}\big| b_n(e_i-e_j) \big|+\frac{ 1 }{2}\big| q_n(e_i) \big|+\frac{ 1 }{2}\big| q_n(e_j) \big|.
    \end{equation}
    Si \( n\) est assez grand, nous avons tout de suite
    \begin{equation}
        \big| \phi(q_n)_{ij} \big|\leq \frac{ 1 }{2}\big| q_n(e_i-e_j) \big|+\epsilon.
    \end{equation}
    Nous définissons \( e_{ij}\) et \( \alpha_{ij}\) de telle sorte que \( e_i-e_j=\alpha_{ij}e_{ij}\) avec \( \| e_{ij} \|=1\). Si \( \alpha=\max\{ \alpha_{ij},1 \}\) alors nous avons
    \begin{equation}
        q_n(e_i-e_j)=\alpha_{ij}^2q_n(e_{ij})\leq \alpha^2q_n(e_{ij}).
    \end{equation}
    Il suffit maintenant de prendre \( n\) assez grand pour avoir \( \sup_{\| x \|=1}| q_n(x) |\leq \frac{ \epsilon }{ \alpha^2 }\) pour avoir
    \begin{equation}
        \big| \phi(q_n)_{ij} \big|\leq \frac{ \epsilon }{2}+\frac{ \epsilon }{ \alpha^2 }.
    \end{equation}
\end{proof}

%---------------------------------------------------------------------------------------------------------------------------
\subsection{Diagonalisation}
%---------------------------------------------------------------------------------------------------------------------------

\begin{proposition}\label{PropFWYooQXfcVY}
    Dans la base de diagonalisation de sa matrice associée, une forme quadratique a la forme
    \begin{equation}
        q(x)=\sum_i\lambda_ix_i^2
    \end{equation}
    où les \( \lambda_i\) sont les valeurs propres de la matrice associée à \( q\).
\end{proposition}

\begin{proof}
    Soit \( q\) une forme quadratique et \( b\) la forme bilinéaire associée. Si \( \{ f_i \}\) est une base de diagonalisation\footnote{Qui existe parce que la matrice est symétrique, théorème~\ref{ThoeTMXla}.} de la matrice de \( b\) alors dans cette base nous avons
\begin{equation}
    q(x)=b(x,x)=\sum_{ij}x_ix_jb(f_i,f_j)=\sum_i\lambda_ix_i^2
\end{equation}
où les \( \lambda_i\) sont les valeurs propres de la matrice de \( b\).
\end{proof}
Notons que si nous choisissons une autre base de diagonalisation, les \( \lambda_i\) ne changement pas (à part l'ordre éventuellement). Cela pour dire que nous nous permettrons de parler des \defe{valeurs propres}{valeur propre!d'une forme quadratique} d'une forme quadratique comme étant les valeurs propres de la matrice associée.

%--------------------------------------------------------------------------------------------------------------------------- 
\subsection{Isotropie}
%---------------------------------------------------------------------------------------------------------------------------

\begin{definition}[Isotropie]   \label{DefVKMnUEM}
    Un vecteur est \defe{isotrope}{isotrope (vecteur)} pour \( b\) s'il est perpendiculaire à lui-même; en d'autres termes, \( x\) est isotrope si et seulement si \( b(x,x)=0\). Un sous-espace \( W\subset E\) est \defe{totalement isotrope}{isotrope!totalement} si pour tout \( x,y\in W\), nous avons \( b(x,y)=0\).

    Le \defe{cône isotrope}{isotrope!cône} de \( b\) est l'ensemble de ses vecteurs isotropes :
    \begin{equation}
        C(b)=\{ x\in E\tq b(x,x)=0 \}.
    \end{equation}
\end{definition}
Nous introduisons quelques notations. D'abord pour \( y\in E\) nous notons
\begin{equation}
    \begin{aligned}
        \Phi_y\colon E&\to \eR \\
        x&\mapsto b(x,y)
    \end{aligned}
\end{equation}
et ensuite
\begin{equation}
    \begin{aligned}
        \Phi\colon E&\to E^* \\
        y&\mapsto \Phi_y.
    \end{aligned}
\end{equation}
\begin{definition}
    Le fait pour une forme bilinéaire \( b\) d'être dégénérée signifie que l'application \( \Phi\) n'est pas injective. Le \defe{noyau}{noyau!d'une forme bilinéaire} de la forme bilinéaire est celui de \( \Phi\), c'est-à-dire
    \begin{equation}
        \ker(b)=\{ z\in E\tq b(z,y)=0\,\forall y\in E \}.
    \end{equation}
    Autrement dit, \( \ker(b)=E^{\perp}\) où le perpendiculaire est pris par rapport à \( b\).
\end{definition}
Notons tout de même que nous utilisons la notation \( \perp\) même si \( b\) est dégénérée et éventuellement pas positive; c'est-à-dire même si la formule \( (x,y)\mapsto b(x,y)\) ne fournit pas un produit scalaire.

\begin{proposition}[\cite{RTzQrdx}]     \label{PropHIWjdMX}
    Soit \( b\) une forme bilinéaire et symétrique. Alors
    \begin{enumerate}
        \item
            \( \ker(b)\subset C(b)\) (cône d'isotropie, définition~\ref{DefVKMnUEM})
        \item
            si \( b\) est positive alors \( \ker(b)=C(b)\).
    \end{enumerate}
\end{proposition}

\begin{proof}
    \begin{enumerate}
        \item
            Si \( z\in\ker(b)\) alors pour tout \( y\in E\) nous avons \( b(z,y)=0\). En particulier pour \( y=z\) nous avons \( b(z,z,)=0\) et donc \( z\in C(b)\).
        \item
            Soit \( b\) positive et \( x\in C(b)\). Par l'inégalité de Cauchy-Schwarz (proposition~\ref{ThoAYfEHG}) nous avons
            \begin{equation}
                | b(x,y) |\leq \sqrt{   b(x,x)b(y,y) }=0.
            \end{equation}
            Donc pour tout \( y\) nous avons \( b(x,y)=0\).
    \end{enumerate}
\end{proof}

%---------------------------------------------------------------------------------------------------------------------------
\subsection{Inégalité de Minkowski}
%---------------------------------------------------------------------------------------------------------------------------

Ce qui est couramment nommé «inégalité de Minkowski» est la proposition~\ref{PropInegMinkKUpRHg} dans les espaces \( L^p\). Nous allons en donner ici un cas très particulier.

\begin{proposition} \label{PropACHooLtsMUL}
    Si \( q\) est une forme quadratique sur \( \eR^n\) et si \( x,y\in \eR^n\) alors
    \begin{equation}
        \sqrt{q(x+y)}\leq\sqrt{q(x)}+\sqrt{q(y)}.
    \end{equation}
\end{proposition}

\begin{proof}
    La proposition~\ref{PropFWYooQXfcVY} nous permet de «diagonaliser» la forme quadratique \( q\). Quitte à ne plus avoir une base orthonormale, nous pouvons renormaliser les vecteurs de base pour avoir
    \begin{equation}
        q(x)=\sum_ix_i^2.
    \end{equation}
    Le résultat n'est donc rien d'autre que l'inégalité triangulaire pour la norme euclidienne usuelle, laquelle est démontrée dans la proposition~\ref{PropEQRooQXazLz}.
\end{proof}

%---------------------------------------------------------------------------------------------------------------------------
\subsection{Ellipsoïde}
%---------------------------------------------------------------------------------------------------------------------------

\begin{lemma}   \label{LemYVWoohcjIX}
    Toute matrice peut être décomposée de façon unique en une partie symétrique et une partie antisymétrique. Cette décomposition est donnée par
\begin{equation}\label{subEqHIQooyhiWM}
    \begin{aligned}[]
            S&=\frac{ M+M^t }{ 2 },&A&=\frac{ M-M^t }{ 2 }
    \end{aligned}
\end{equation}
\end{lemma}

\begin{proof}
    L'existence est une vérification immédiate de \( S+A=M\) en utilisant \eqref{subEqHIQooyhiWM}. Pour l'unicité, si \( S+A=S'+A'\) alors \( S-S'=A-A'\). Mais \( S-S'\) est symétrique et \( A-A'\) est antisymétrique; l'égalité implique l'annulation des deux membres, c'est-à-dire \( S=S'\) et \( A=A'\).
\end{proof}

\begin{definition}  \label{DefOEPooqfXsE}
    Un \defe{ellipsoïde}{ellipsoïde} dans \( \eR^n\) centré en \( v\) est le lieu des points \( x\) vérifiant l'équation
    \begin{equation}\label{EqSNWooXfbTH}
        \langle x-v, M(x-v)\rangle =1
    \end{equation}
    où \( M\) est une matrice symétrique strictement définie positive\footnote{Définition~\ref{DefAWAooCMPuVM}.}.

    Lorsque nous parlons d'ellipsoïde \emph{plein}, il suffit de changer l'égalité en une inégalité.
\end{definition}

\begin{remark}
    Le fait que \( M\) soit symétrique n'est pas tout à fait obligatoire; la chose important est que toutes les valeurs propres soient strictement positives. En effet si \( M\) a toutes ses valeurs propres strictement positives, nous nommons \( S\) la partie symétrique de \( M\) et \( A\) la partie antisymétrique (lemme~\ref{LemYVWoohcjIX}). Alors pour tout \( x\in \eR^n\) nous avons
    \begin{equation}
        x^tAx=\langle x, Ax\rangle =\langle A^tx,x \rangle =-\langle Ax, x\rangle =-\langle x,Ax\rangle ,
    \end{equation}
    donc \( x^tAx=0\). L'équation \( x^tMx=1\) est donc équivalente à \( x^tSx=1\) (elles ont les mêmes solutions).

    De plus \( S\) reste strictement définie positive parce que pour tout \( x\in \eR^n\) nous avons
    \begin{equation}
        0<x^tMx=x^tSx.
    \end{equation}
\end{remark}

\begin{proposition}\label{PropWDRooQdJiIr}
    Si \( \ellE\) est un ellipsoïde centrée à l'origine, il existe une base de \( \eR^n\) dans laquelle son équation est :
    \begin{equation}
        \sum_{i=1}^n\frac{ x_i^2 }{ a_i^2 }=1.
    \end{equation}
\end{proposition}

\begin{proof}
    Nous avons une matrice symétrique strictement définie positive \( S\) telle que l'équation soit \( \langle x, Sx\rangle =1\). Le théorème spectral~\ref{ThoeTMXla} nous fournit une base orthonormale \( \{ e_i \}\) dans laquelle \( Se_i=\lambda_ie_i\) avec \( \lambda_i>0\). En substituant dans l'équation \( \langle x, Sx\rangle =1\) nous trouvons l'équation
    \begin{equation}
        \sum_i\lambda_ix_i^2=1.
    \end{equation}
    En posant \( a_i=\frac{1}{ \sqrt{\lambda_i} }\), nous trouvons le résultat.  Cette définition des \( a_i\) est toujours possible parce que \( \lambda_i>0\).
\end{proof}

\begin{corollary}   \label{CorKGJooOmcBzh}
    Un ellipsoïde plein centré en l'origine admet une équation de la forme \( q(x)\leq 1\) où \( q\) est une forme quadratique strictement définie positive.
\end{corollary}
Pour rappel de notation, l'ensemble des formes quadratiques strictement définies positives sur l'espace vectoriel \( E\) est noté \( Q^{++}(E)\).

\begin{proof}
    Soit \( \{ e_i \}\) une base de \( \eR^n\) telle que l'ellipsoïde \( \ellE\) ait pour équation
    \begin{equation}
        \sum_{i=1}^n\frac{ x_i^2 }{ a_i^2 }\leq 1.
    \end{equation}
    Nous considérons la forme quadratique
    \begin{equation}
        \begin{aligned}
            q\colon \eR^n&\to \eR \\
            x&\mapsto \sum_{i=1}^n\frac{ \langle x, e_i\rangle^2 }{ a_i^2 }.
        \end{aligned}
    \end{equation}
    Nous avons évidemment \( \ellE=\{ x\in \eR^n\tq q(x)\leq 1 \}\). De plus la forme \( q\) est strictement définie positive parce que dès que \( x\neq 0\), au moins un des produits scalaires \( \langle x, e_i\rangle \) est non nul et \( q(x)> 0\).
\end{proof}

%+++++++++++++++++++++++++++++++++++++++++++++++++++++++++++++++++++++++++++++++++++++++++++++++++++++++++++++++++++++++++++
\section{Théorème spectral autoadjoint}
%+++++++++++++++++++++++++++++++++++++++++++++++++++++++++++++++++++++++++++++++++++++++++++++++++++++++++++++++++++++++++++

\begin{theorem}[Théorème spectral autoadjoint] \label{ThoRSBahHH}
    Un endomorphisme autoadjoint d'un espace euclidien
    \begin{enumerate}
        \item
            est diagonalisable dans une base orthonormée,
        \item
            a son spectre réel.
    \end{enumerate}
\end{theorem}
\index{théorème!spectral!autoadjoint}
\index{diagonalisation!endomorphisme autoadjoint}

\begin{proof}
    Nous procédons par récurrence sur la dimension de \( E\), et nous commençons par \( n=1\)\footnote{Dans \cite{KXjFWKA}, l'auteur commence avec \( n=0\) mais moi je n'en ai \wikipedia{en}{Vacuous_truth}{pas le courage.}.}. Soit donc \( f\colon E\to E\) avec \( \langle f(x), y\rangle =\langle x, f(y)\rangle \). Étant donné que \( f\) est également linéaire, il existe \( \lambda\in \eR\) tel que \( f(x)=\lambda x\) pour tout \( x\in E\). Tous les vecteurs de \( E\) sont donc vecteurs propres de \( f\).

    Passons à la récurrence. Nous considérons \( \dim(E)=n+1\) et \( f\in\gS(E)\). Nous considérons la forme bilinéaire symétrique \( \Phi_f\) et la forme quadratique associée \( \phi_f\). Pour rappel,
    \begin{subequations}
        \begin{align}
        \Phi_f(x,y)=\langle x, f(y)\rangle \\
        \phi_f(x)=\Phi_f(x,x).
        \end{align}
    \end{subequations}
    Et nous allons laisser tomber les indices \( f\) pour noter simplement \( \Phi\) et \( \phi\). Étant donné que \( \overline{ B(0,1) }\) est compacte et que \( \phi\) est continue, il existe \( x_0\in\overline{ B(0,1) }\) tel que
    \begin{equation}
        \lambda=\phi(x_0)=\sup_{x\in\overline{ B(0,1) }}\phi(x).
    \end{equation}
    Notons aussi que \( \| x_0 \|=1\) : le maximum est pris sur le bord. Nous posons
    \begin{equation}
        g=\lambda\id-f
    \end{equation}
    ainsi que
    \begin{equation}
        \Phi_1(x,y)=\langle x, g(y)\rangle .
    \end{equation}
    Cela est une forme bilinéaire et symétrique parce que
    \begin{equation}
        \Phi_1(y,x)=\langle y, g(x)\rangle =\langle g(y), x\rangle =\langle x, g(y)\rangle =\Phi_1(x,y)
    \end{equation}
    où nous avons utilisé le fait que \( g\) était autoadjoint et la symétrie du produit scalaire. De plus \( \Phi_1\) est semi-définie positive parce que
    \begin{equation}
        \Phi_1(x,x)=\langle x, \lambda x-f(x)\rangle =\lambda\| x \|^2-\phi(x).
    \end{equation}
    Vu que \( \lambda\) est le maximum, nous avons tout de suite \( \Phi_1(x)\geq 0\) tant que \( \| x \|=1\). Et si \( x\) n'est pas de norme \( 1\), c'est le même prix parce qu'on se ramène à \( \| x \|=1\) en multipliant par un nombre positif. Attention cependant :
    \begin{equation}
        \Phi_1(x_0,x_0)=\lambda\| x_0 \|^2-\phi(x_0)=0.
    \end{equation}
    Donc \( \Phi_1\) a un noyau contenant \( x_0\) par la proposition~\ref{PropHIWjdMX}. Nous en déduisons que \( \Image(g)\neq E\) en effet, \( x_0\in\Image(g)^{\perp}\), mais nous avons la proposition~\ref{PropXrTDIi} sur les dimensions :
    \begin{equation}
        \dim E=\dim(\Image(g))+\dim( \Image(g)^{\perp}).
    \end{equation}
    Vu que \( \Image(g)^{\perp}\) est un espace vectoriel non réduit à \( \{ 0 \}\), la dimension de \( \Image(g)\) ne peut pas être celle de \( E\). L'endomorphisme \( g\) n'étant pas surjectif, il ne peut pas être injectif non plus parce que nous sommes en dimension finie; il existe donc \( e_1\in E\) tel que \( g(e_1)=0\) et tant qu'à faire nous choisissons \( \| e_1 \|=1\) (ici la norme est bien celle de l'espace euclidien considéré). Par définition,
    \begin{equation}
        f(e_1)=\lambda e_1,
    \end{equation}
    c'est-à-dire que \( \lambda\in\Spec(f)\). Et \( \phi\) étant une forme quadratique réelle nous avons \( \lambda\in \eR\).

    Nous posons à présent \( H=\Span\{ e_1 \}^{\perp}\). C'est un sous-espace stable par \( f\) parce que si \( x\in H\) alors
    \begin{equation}
        \langle e_1, f(x)\rangle =\langle f(e_1j),x\rangle =\lambda\langle e_1, x\rangle =0.
    \end{equation}
    Nous pouvons donc considérer la restriction de \( f\) à \( H\) : \( f_H\colon H\to H\). Cet endomorphisme est bilinéaire et symétrique sur l'espace \( H\) de dimension inférieure à celle de \( E\), donc la récurrence nous donne une base orthonormée
    \begin{equation}
        \{ e_2,\ldots, e_n \}
    \end{equation}
    de vecteurs propres de \( f_H\). De plus les valeurs propres sont réelles, toujours par récurrence. Donc
    \begin{equation}
        \Spec(f)=\{ \lambda \}\cup\Spec(f_H)\subset \eR.
    \end{equation}
    Notons pour être complet que si \( i\geq 2\) alors
    \begin{equation}
        \langle e_1, e_i\rangle =0
    \end{equation}
    parce que le vecteur \( e_i\) est par construction choisi dans l'espace \( H=e_1^{\perp}\). Nous avons donc bien une base orthonormée de \( E\) construite sur des vecteurs propres de \( f\).
\end{proof}

\begin{corollary}   \label{CorSMHpoVK}
    Soit \( E\) un espace vectoriel ainsi que \( \phi\) et \( \psi\) des formes quadratiques sur \( E\) avec \( \psi\) définie positive. Alors il existe une base \( \psi\)-orthonormale dans laquelle \( \phi\) est diagonale.
\end{corollary}

\begin{proof}
    Il suffit de considérer l'espace euclidien \( E\) muni du produit scalaire \( \langle x, y\rangle =\psi(x,y)\). Ensuite nous diagonalisons la matrice (symétrique) de \( \phi\) pour ce produit scalaire à l'aide du théorème~\ref{ThoRSBahHH}.
\end{proof}

\begin{definition}      \label{DefYNWUFc}
    Dans le cas de \( V=\eR^m\) nous avons un produit scalaire canonique. Soient $u$ et $v$, deux vecteurs de $\eR^m$. Le \defe{produit scalaire}{produit!scalaire!sur \( \eR^n\)} de $u$ et $v$, noté $\langle u, v\rangle $ ou $u\cdot v$ est le réel
	\begin{equation}		\label{EqDefProdScalsumii}
		\langle u, v\rangle =\sum_{k=1}^m u_kv_k=u_1v_1+u_2v_2+\cdots+u_mv_n.
	\end{equation}
\end{definition}

Calculons par exemple le produit scalaire de deux vecteurs de la base canonique : $\langle e_i, e_j\rangle $. En utilisant la formule de définition et le fait que $(e_i)_k=\delta_{ik}$, nous avons
\begin{equation}
	\langle e_i, e_j\rangle =\sum_{k=1}^m\delta_{ik}\delta_{jk}.
\end{equation}
Nous pouvons effectuer la somme sur $k$ en remarquant qu'à cause du $\delta_{ik}$, seul le terme avec $k=i$ n'est pas nul. Effectuer la somme revient donc à remplacer tous les $k$ par des $i$ :
\begin{equation}
	\langle e_i, e_j\rangle =\delta_{ii}\delta_{ji}=\delta_{ji}.
\end{equation}

Une des propriétés intéressantes du produit scalaire est qu'il permet de décomposer un vecteur dans une base, comme nous le montre la proposition suivante.

\begin{proposition}		\label{PropScalCompDec}
	Si nous notons $v_i$ les composantes du vecteur $v$, c'est-à-dire si $v=\sum_{i=1}^m v_ie_i$, alors nous avons $v_j=\langle v, e_j\rangle $.
\end{proposition}

\begin{proof}
	\begin{equation}		\label{Eqvejscalcomp}
		v\cdot e_j=\sum_{i=1}^m\langle v_ie_i, e_j\rangle =\sum_{i=1}^mv_i\langle e_i, e_j\rangle =\sum_{i=1}^mv_i\delta_{ij}
	\end{equation}
	En effectuant la somme sur $i$ dans le membre de droite de l'équation \eqref{Eqvejscalcomp}, tous les termes sont nuls sauf celui où $i=j$; il reste donc
	\begin{equation}
		v\cdot e_j=v_j.
	\end{equation}
\end{proof}

Le produit scalaire ne dépend en réalité pas de la base orthogonale choisie.

\begin{lemma}
	Si $\{ e_i \}$ est la base canonique, et si $\{ f_i \}$ est une autre base orthonormale, alors si $u$ et $v$ sont deux vecteurs de $\eR^m$, nous avons
	\begin{equation}
		\sum_i u_iv_j=\sum_iu'_iv'_j
	\end{equation}
	où $u_i$ sont les composantes de $u$ dans la base $\{ e_i \}$ et $u'_i$ sont celles dans la base $\{ f_i \}$.
\end{lemma}

\begin{proof}
	La preuve demande un peu d'algèbre linéaire. Étant donné que $\{ f_i \}$ est une base orthonormale, il existe une matrice $A$ orthogonale ($AA^t=\mtu$) telle que $u'_i=\sum_jA_{ij}u_j$ et idem pour $v$. Nous avons alors
	\begin{equation}
		\begin{aligned}[]
			\sum_iu'_iv'_j&=\sum_i\left( \sum_jA_{ij} u_j\right)\left( \sum_k A_{ik}v_k \right)\\
			&=\sum_{ijk}A_{ij}A_{ik}u_jv_k\\
			&=\sum_{jk}\underbrace{\sum_i(A^t)_{ji}A_{ik}}_{=\delta_{jk}}u_jv_k\\
			&=\sum_{jk}\delta_{jk}u_jv_k\\
			&=\sum_ku_jv_k.
		\end{aligned}
	\end{equation}
\end{proof}

Cette proposition nous permet de réellement parler du produit scalaire entre deux vecteurs de façon intrinsèque sans nous soucier de la base dans laquelle nous regardons les vecteurs.

Nous dirons que deux vecteurs sont \defe{orthogonaux}{orthogonal} lorsque leur produit scalaire est nul. Nous écrivons que $u\perp v$ lorsque $\langle u, v\rangle =0$.
\begin{definition}	\label{DefNormeEucleApp}
	La \defe{norme euclidienne}{norme!euclidienne!dans $\eR^m$} d'un élément de $\eR^m$ est définie par $\| u \|=\sqrt{u\cdot u}$.
\end{definition}

Cette définition est motivée par le fait que le produit scalaire $u\cdot u$ donne exactement la norme usuelle donnée par le théorème de Pythagore :
\begin{equation}
	u\cdot u=\sum_{i=1}^mu_iu_i=\sum_{i=1}^m u_i^2=u_1^2+u_2^2+\cdots+u_m^2.
\end{equation}

Le fait que $e_i\cdot e_j=\delta_{ij}$ signifie que la base canonique est \defe{orthonormée}{orthonormé}, c'est-à-dire que les vecteurs de la base canonique sont orthogonaux deux à deux et qu'ils ont tout $1$ comme norme.

\begin{lemma}\label{LemSclNormeXi}
	Pour tout $u\in\eR^m$, il existe un $\xi\in\eR^m$ tel que $\| u \|=\xi\cdot u$ et $\| \xi \|=1$.
\end{lemma}

\begin{proof}
	Vérifions que le vecteur $\xi=u/\| u \|$ ait les propriétés requises. D'abord $\| \xi \|=1$ parce que $u\cdot u=\| u \|^2$. Ensuite
	\begin{equation}
		\xi\cdot u=\frac{ u\cdot u }{ \| u \| }=\frac{ \| u \|^2 }{ \| u \| }=\| u \|.
	\end{equation}
\end{proof}

%+++++++++++++++++++++++++++++++++++++++++++++++++++++++++++++++++++++++++++++++++++++++++++++++++++++++++++++++++++++++++++
\section{Système d'équations linéaires : méthode de Gauss}
%+++++++++++++++++++++++++++++++++++++++++++++++++++++++++++++++++++++++++++++++++++++++++++++++++++++++++++++++++++++++++++

% TODO: Ajouter un texte sur les équations de plan, et pourquoi ax+by+cz+d=0 est perpendiculaire au vecteur (a,b,c).

Pour résoudre un système d'équations linéaires, on procède comme suit:
\begin{enumerate}
\item Écrire le système sous forme matricielle. \[\text{p.ex. } \begin{cases} 2x+3y &= 5 \\ x+2y &= 4 \end{cases} \Leftrightarrow \left(\begin{array}{cc|c} 2 & 3 & 5 \\ 1 & 2 & 4 \end{array}\right) \]
\item Se ramener à une matrice avec un maximum de $0$ dans la partie de gauche en utilisant les transformations admissibles:
\begin{enumerate}
\item Remplacer une ligne par elle-même + un multiple d'une autre;
\[\text{p.ex. } \left(\begin{array}{cc|c} 2 & 3 & 5 \\ 1 & 2 & 4 \end{array}\right)  \stackrel{L_1  - 2. L_2 \mapsto L_1'}{\Longrightarrow} \left(\begin{array}{cc|c} 0 & -1 & -3 \\ 1 & 2 & 4 \end{array}\right) \]
\item Remplacer une ligne par un multiple d'elle-même;
\[\text{p.ex. } \left(\begin{array}{cc|c} 0 & -1 & -3 \\ 1 & 2 & 4 \end{array}\right)  \stackrel{-L_1  \mapsto L_1'}{\Longrightarrow} \left(\begin{array}{cc|c} 0 & 1 & 3 \\ 1 & 2 & 4 \end{array}\right) \]
\item Permuter des lignes.
\[\text{p.ex. } \left(\begin{array}{cc|c} 0 & 1 & 3 \\ 1 & 0 & -2 \end{array}\right)  \stackrel{L_1  \mapsto L_2' \text{ et } L_2  \mapsto L_1'}{\Longrightarrow} \left(\begin{array}{cc|c} 1 & 0 & -2 \\ 0 & 1 & 3  \end{array}\right) \]
\end{enumerate}
\item Retransformer la matrice obtenue en système d'équations.
\[\text{p.ex. }  \left(\begin{array}{cc|c} 1 & 0 & -2 \\ 0 & 1 & 3  \end{array}\right) \Leftrightarrow \begin{cases} x &= -2 \\ y &= 3 \end{cases}  \]
\end{enumerate}

\begin{remark}
\begin{itemize}
\item Si on obtient une ligne de zéros, on peut l'enlever:
\[\text{p.ex. }  \left(\begin{array}{ccc|c} 3 & 4 & -2 & 2 \\ 4 & -1 & 3 & 0 \\ 0 & 0 & 0 & 0 \end{array}\right) \Leftrightarrow  \left(\begin{array}{ccc|c} 3 & 4 & -2 & 2 \\ 4 & -1 & 3 & 0 \end{array}\right) \]
\item Si on obtient une ligne de zéros suivie d'un nombre non-nul, le système d'équations n'a pas de solution:
\[\text{p.ex. }  \left(\begin{array}{ccc|c} 3 & 4 & -2 & 2 \\ 4 & -1 & 3 & 0 \\ 0 & 0 & 0 & 7 \end{array}\right) \Leftrightarrow  \begin{cases} \cdots \\ \cdots \\ 0x + 0y + 0z = 7 \end{cases} \Rightarrow \textbf{Impossible} \]
\item Si on a moins d'équations que d'inconnues, alors il y a une infinité de solutions qui dépendent d'un ou plusieurs paramètres:
\[\text{p.ex. }  \left(\begin{array}{ccc|c} 1 & 0 & -2 & 2 \\ 0 & 1 & 3 & 0 \end{array}\right) \Leftrightarrow  \begin{cases} x - 2z = 2 \\ y + 3z = 0 \end{cases} \Leftrightarrow  \begin{cases} x = 2 + 2\lambda \\ y = -3\lambda \\ z = \lambda \end{cases} \]
\end{itemize}
\end{remark}



\chapter{Espaces vectoriels normés}
% This is part of Le Frido
% Copyright (c) 2008-2019
%   Laurent Claessens
% See the file fdl-1.3.txt for copying conditions.

Plusieurs choses sur les espaces vectoriels normés (dont la définition \ref{DefNorme}) ont déjà été vues dans la section \ref{SECooWKJNooKOqpsx}. Voir aussi le thème \ref{THEMEooUJVXooZdlmHj}.

On fixe maintenant une définition largement utilisée dans la suite.
\begin{definition}      \label{DefAQIQooYqZdya}
	 Soient $U$ et $V$, deux ouverts d'un espace vectoriel normé. Une application $f$ de $U$ dans $V$ est un \defe{difféomorphisme}{difféomorphisme} si elle est bijective, différentiable et dont l'inverse $f^{-1}:V\to U $ est aussi différentiable.
\end{definition}

\begin{remark}
	Il n'est pas possible d'avoir une application inversible d'un ouvert de $\eR^m$ vers un ouvert de $\eR^n$ si $m\neq n$. Il n'y a donc pas de notion de difféomorphismes entre ouverts de dimensions différentes.
\end{remark}

%+++++++++++++++++++++++++++++++++++++++++++++++++++++++++++++++++++++++++++++++++++++++++++++++++++++++++++++++++++++++++++
\section{Norme opérateur}
%+++++++++++++++++++++++++++++++++++++++++++++++++++++++++++++++++++++++++++++++++++++++++++++++++++++++++++++++++++++++++++

La proposition suivante donne une norme (au sens de la définition~\ref{DefNorme}) sur $\aL(V,W)$ dès que \( V\) et \( W\) sont des espaces vectoriels normés.
\begin{propositionDef}[Norme opérateur\cite{ooTZRDooWmjBJi}, thème \ref{THEMEooOJJFooWMSAtL}]          \label{DefNFYUooBZCPTr}
    Soit une application linéaire \( T\colon V\to W\), et le nombre
	\begin{equation}
        \|T\|_{\aL}=\sup_{\substack{x\in V\\x\neq 0}}\frac{\|T(x)\|_{W}}{\|x\|_{V}}.
	\end{equation}
    \begin{enumerate}
        \item
            Si \( V\) est de dimension finie, alors \( \| T \|_{\aL}<\infty\).
        \item
            L'application \( T\mapsto\| T \|_{\aL}\) est une norme sur l'espace vectoriel des applications linéaires \( V\to W\).
        \item       \label{ITEMooUQPRooYQGZzu}
            Nous avons la formule
            \begin{equation}    \label{EqFZPooIoecGH}
                \| T \|_{\aL}=\sup_{x\in V}\frac{\|T(x)\|_{W}}{\|x\|_{V}} =\sup_{\|x\|_{V}=1}\|T(x)\|_{W}
            \end{equation}
    \end{enumerate}
    Le nombre \( \| T \|_{\aL}\) est la \defe{norme opérateur}{norme!d'application linéaire} de $T$. Nous disons que cette norme est \defe{subordonnée}{subordonnée!norme} aux normes choisies sur \( V\) et \( W\).
\end{propositionDef}
\index{norme!d'une application linéaire}

\begin{proof}
    Si \( V\) est de dimension finie alors l'ensemble $\{ \| x \|= 1 \}$ est compact par le théorème de Borel-Lebesgue~\ref{ThoXTEooxFmdI}. Alors la fonction
    \begin{equation}
        x\mapsto \frac{ \| T(x) \| }{ \| x \| }
    \end{equation}
    est une fonction continue sur un compact. Le corolaire~\ref{CorFnContinueCompactBorne} nous dit alors qu'elle est bornée. Le supremum est donc un nombre réel fini.

    Nous vérifions que l'application $\| . \|$ de $\aL(V,W)$ dans $\eR$ ainsi définie est effectivement une norme.
    \begin{enumerate}
        \item
            $\|T\|_{\aL}=0$ signifie que $\|T(x)\|=0$ pour tout $x$ dans $V$. Comme  $\|\cdot\|_W$ est une norme nous concluons que $T(x)=0_{n}$ pour tout $x$ dans $V$, donc $T$ est l'application nulle.
    \item
        Pour tout $a$ dans $\eR$ et tout  $T$ dans $\aL(V,W)$ nous avons
        \begin{equation}
            \|aT\|_{\mathcal{L}}=\sup_{\|x\|_{V}\leq 1}\|aT(x)\|_{W}=|a|\sup_{\|x\|_{V}\leq 1}\|T(x)\|_{W}=|a|\|T\|_{\mathcal{L}}.
        \end{equation}
    \item
        Pour tous $T_1$ et $T_2$ dans $\aL(V,W)$ nous avons
      \begin{equation}\nonumber
        \begin{aligned}
           \|T_1+ T_2\|_{\mathcal{L}}&=\sup_{\|x\|\leq 1}\|T_1(x)+T_2(x)\|\leq\\
     &\leq\sup_{\|x\|\leq 1}\|T_1(x)\| +\sup_{\|x\|\leq 1}\|T_2(x)\|\\
     &=\|T_1\|\|T_2\|.
        \end{aligned}
      \end{equation}
    \end{enumerate}


    Enfin nous prouvons la formule alternative \eqref{EqFZPooIoecGH}. Nous allons montrer que les ensembles sur lesquels ont prend le supremum sont en réalité les mêmes :
    \begin{equation}
        \underbrace{\left\{ \frac{ \| Ax \| }{ \| x \| }\right\}_{x\neq 0}}_{A}=\underbrace{\left\{ \| Ax \|\tq \| x \|=1 \right\}}_{B}.
    \end{equation}
    Attention : ce sont des sous-ensembles de réels; pas de sous-ensembles de \( \eM(\eR)\) ou des sous-ensembles de \( \eR^n\).

    Pour la première inclusion, prenons un élément de \( A\), et prouvons qu'il est dans \( B\). C'est-à-dire que nous prenons \( x\in V\) et nous considérons le nombre \( \| Ax \|/\| x \|\). Le vecteur \( y=x/\| x \|\) est un vecteur de norme $1$, donc la norme de \( Ay\) est un élément de \( B\), mais
    \begin{equation}
        \| Ay \|=\frac{ \| Ax \| }{ \| x \| }.
    \end{equation}
    Nous avons donc \( A\subset B\).

    L'inclusion \( B\subset A\) est immédiate.
\end{proof}

En d'autres termes, il y a autant de normes opérateur sur \( \aL(E,F)\) qu'il y a de paires de choix de normes sur \( E\) et \( F\). En particulier, cela donne lieu à toutes les normes \( \| A \|_p\) qui correspondent aux normes \( \| . \|_p\) sur \( \eR^n\).

\begin{example}     \label{EXooXPXAooYyBwMX}
    Voyons la norme opérateur subordonnée à la norme \( \| x \|_{\infty}=\max_i| x_i |\) sur \( \eC^n\). Par définition (et surtout par la propriété~\ref{DefNFYUooBZCPTr}\ref{ITEMooUQPRooYQGZzu}),
    \begin{equation}
        \| A \|_{\infty}=\sup_{\| x \|_{\infty}=1}=\| Ax \|_{\infty}.
    \end{equation}
    Vu que \( (Ax)_i=\sum_kA_{ik}x_k\), lorsque \( \| x \|_{\infty}\leq 1\) nous avons \( | (Ax)_i |\leq \sum_k| A_{ik} |\). Donc nous avons toujours
    \begin{equation}        \label{EQooPLCIooVghasD}
        \| A \|_{\infty}\leq \max_i\sum_{k}| A_{ik} |.
    \end{equation}
\end{example}

\begin{definition}
    La \defe{topologie forte}{topologie!forte} sur l'espace des opérateurs est la topologie de la norme opérateur.
\end{definition}
Lorsque nous considérons un espace vectoriel d'applications linéaires, nous considérons toujours\footnote{Sauf lorsque les événements nous forceront à trahir.} dessus la topologie liée à cette norme.

Il existe aussi la \defe{topologie faible}{topologie!faible} donnée par la notion de convergence\quext{Est-ce qu'on peut décrire cette topologie à partir de ses ouverts ? Facilement ?} \( A_i\to A\) si et seulement si \( A_ix\to Ax\) pour tout \( x\in E\).
    %TODO : il faut mettre au clair quelle est vraiment la topologie faible à partir des ouverts.

\begin{probleme}
    Je crois, mais demande confirmation, que la topologie faible est celle des semi-normes \( \{ p_v \}_{v\in E}\) données par \( p_v(A)=\| A \|\). En effet la notion de convergence associée par la proposition~\ref{PropQPzGKVk} est \( A_i\to A\) si et seulement si \( p_v(A_i-A)\to 0\). Cette condition signifie \( \| A_i(v)-A(v) \|\to 0\), c'est-à-dire \( A_i(v)\to A(v)\).

    Si le lecteur veut parler de cela au jury d'un concours, il est évident qu'il devra être capable d'ajouter des petits symboles au-dessus de toutes les flèches «\( \to\)» du paragraphe précédent pour indiquer pour quelles topologies sont les convergences dont on parle.
\end{probleme}

\begin{remark}
    Il faut noter que la topologie faible n'est pas une topologie métrique. Cela même si la condition \( A_ix\to Ax\), elle, est métrique vu qu'elle est écrite dans \( E\).

    Dans le cas où \( E\) est de dimension infinie, la topologie faible est réellement différente de la topologie forte. Nous verrons à la sous-section~\ref{subsecaeSywF} que dans le cas des projections sur un espaces de Hilbert, l'égalité
    \begin{equation}
        \sum_{i=1}^{\infty}\pr_{u_i}=\id
    \end{equation}
    est vraie pour la topologie faible, mais pas pour la topologie forte.
\end{remark}

%---------------------------------------------------------------------------------------------------------------------------
\subsection{Norme d'algèbre}
%---------------------------------------------------------------------------------------------------------------------------

\begin{definition}[Norme d'algèbre\cite{ooTZRDooWmjBJi}]  \label{DefJWRWQue}
    Si \( A\) est une algèbre\footnote{Définition~\ref{DefAEbnJqI}.}, une \defe{norme d'algèbre}{norme!d'algèbre} sur \( A\) est une norme telle que pour toute \( x,y\in A\),
    \begin{equation}
        \| xy \|\leq \| x \|\| y \|.
    \end{equation}
\end{definition}
La norme opérateur est une norme d'algèbre, comme nous le verrons dans le lemme \ref{LEMooFITMooBBBWGI}.

Un des intérêts d'utiliser une norme d'algèbre est que l'on a l'inégalité \( \| x^k \|\leq \| x \|^k \). Cela sera particulièrement utile lors de l'étude des séries entières, voir par exemple~\ref{secEVnZXgf}.

\begin{definition}[\cite{ooYLHAooCzQvoa}]      \label{DEFooEAUKooSsjqaL}
    Le \defe{rayon spectral}{rayon!spectral} d'une matrice carrée $A$, noté $\rho(A)$, est défini de la manière suivante :
    \begin{equation}    \label{EQooNVNOooNjJhSS}
        \rho(A)=\max_i|\lambda_i|
    \end{equation}
    où les $\lambda_i$ sont les valeurs propres de $A$.
\end{definition}

\begin{normaltext}
    Quelques remarques sur la définition du rayon spectral.
    \begin{itemize}
        \item
             Même si \( A\) est une matrice réelle, les valeurs propres sont dans \( \eC\). Donc dans \eqref{EQooNVNOooNjJhSS}, \( | \lambda_i |\) est le module dans \( \eC\) de \( \lambda_i\).
        \item
            Vu que les valeurs propres de \( A\) sont les racines de son polynôme caractéristique (théorème~\ref{ThoWDGooQUGSTL}), il y en a un nombre fini et le maximum est bien défini.
        \item
            La définition s'applique uniquement pour les espaces de dimension finie.
    \end{itemize}
\end{normaltext}

\begin{lemma}       \label{LEMooIBLEooLJczmu}
    Soient des espaces vectoriels normés \( E\) et \( F\), sur les corps \( \eR\) ou \( \eC\). Pour tout \( A\in \aL(E,F)\), et pour tout \( u\in E\) nous avons la majoration
    \begin{equation}
        \| Au \|\leq \| A \|\| u \|
    \end{equation}
    où la norme sur \( A\) est la norme opérateur subordonnée à la norme sur \( u\).
\end{lemma}

\begin{proof}
    Si \( u\in E\) alors, étant donné que le supremum d'un ensemble est plus grand ou égal à chacun de éléments qui le compose,
    \begin{equation}
        \| A \|=\sup_{x\in E}\frac{ \| Ax \| }{ \| x \| }\geq \frac{ \| Au \| }{ \| u \| },
    \end{equation}
    donc le résultat annoncé : \( \| Au \|\leq \| A \|\| u \|\).
\end{proof}

Le lemme suivant est valable en dimension infinie. Nous en toucherons un mot dans l'exemple \ref{EXooTQPEooRRdddt}.
\begin{lemma}       \label{LEMooWFNXooLyTyyX}
    Soient des espaces vectoriels normés \( E\) et \( F\). Soit \( x\in E\). Alors l'application d'évaluation
    \begin{equation}
        \begin{aligned}
            ev_x\colon \aL(E,F)&\to F \\
            f&\mapsto f(x) 
        \end{aligned}
    \end{equation}
    est continue.
\end{lemma}

\begin{proof}
    Si \( x=0\), alors par linéarité de \( f\) nous avons \( ev_0(f)=0\) pour tout \( f\). Donc d'accord pour la continuité.

    Soit une suite convergente \( f_k\stackrel{\aL(E,F)}{\longrightarrow}f\). Nous voulons prouver que \( ev_x(f_k)\stackrel{F}{\longrightarrow}ev_x(f)\), c'est-à-dire que
    \begin{equation}
        \lim_{k\to \infty} \| f_k(x)-f(x) \|=0.
    \end{equation}
    Par hypothèse si \( k\) est grand, alors \( \| f_k-f  \|_{\aL(E,F)}\leq \epsilon\), c'est-à-dire que\footnote{Définition \ref{DefNFYUooBZCPTr} de la norme sur \( \aL(E,F)\).}
    \begin{equation}
        \sup_{y\in E}\frac{ \| f_k(y)-f(y) \| }{ \| y \| }\leq \epsilon.
    \end{equation}
    En particulier pour notre \( x\) nous avons
    \begin{equation}
        \frac{ \| f_k(x)-f(x) \| }{ \| x \| }\leq \epsilon,
    \end{equation}
    c'est-à-dire \( \| f_k(x)-f(x) \|\leq \| x \|\epsilon\). Vu que \( \| x \|\) est une simple constante et que \( \epsilon\) est arbitraire, cela implique \( f_k(x)\to f(x)\).
\end{proof}

%--------------------------------------------------------------------------------------------------------------------------- 
\subsection{Matrices, spectre et norme}
%---------------------------------------------------------------------------------------------------------------------------

La lien entre la norme opérateur d'une matrice et son spectre sera entre autres utilisé pour étudier le conditionnement de problèmes numériques. Voir la définition \ref{DEFooBKQWooJuoCGX} et par exemple son lien avec la résolution numérique de systèmes linéaires dans la proposition \ref{PROPooGIXFooAhJkIs}.

\begin{proposition}[\cite{ooYLHAooCzQvoa}]      \label{PROPooKLFKooSVnDzr}
    Soit une matrice \( A\in \eM(n,\eC)\) de rayon spectral \( \rho(A)\). Soit une norme \( \| . \|\) sur \( \eC^n\) et la norme opérateur correspondante. Alors
    \begin{equation}
        \rho(A)\leq \| A^k \|^{1/k}
    \end{equation}
    pour tout \( k\in \eN\).
\end{proposition}

\begin{proof}
    Soit \( v\in \eC^n\) et \( \lambda\in \eC\) un couple vecteur-valeur propre. Nous avons \( \| Av \|=| \lambda |\| v \|\) et aussi
    \begin{equation}
        | \lambda |^k\| v \|=\| \lambda^kv \|=\| A^kv \|\leq \| A^k \|\| v \|.
    \end{equation}
    La dernière inégalité est due au fait que nous avons choisi sur \( \eM(n,\eC)\) la norme subordonnée à celle choisie sur \( \eC^n\), via le lemme~\ref{LEMooIBLEooLJczmu}. Nous simplifions par \( \| v \|\) et obtenons \( | \lambda |\leq \| A^k \|^{1/k}\). Étant donné que \( \rho(A)\) est la maximum de tous les \( \lambda\) possibles, la majoration passe au maximum :
    \begin{equation}
        \rho(A)\leq \| A^k \|^{1/k}.
    \end{equation}
\end{proof}

\begin{lemma}[La norme opérateur est une norme d'algèbre\cite{MonCerveau}]   \label{LEMooFITMooBBBWGI}
    Soient des espaces vectoriels normés \( E\), \( F\) et \( G\). Soient des opérateurs linéaires bornés \( B\colon E\to F\), \( A\colon F\to G\). Alors
    \begin{equation}
        \| AB \|\leq \| A \|\| B \|.
    \end{equation}
    C'est à dire que la norme opérateur est une norme d'algèbre\footnote{Définition \ref{DefJWRWQue}.}.
\end{lemma}

\begin{proof}

    Nous avons les (in)égalités suivantes :
    \begin{subequations}
        \begin{align}
            \| AB \|&=\sup_{x\in E}\frac{ \| ABx \|_G }{ \| x \|_E }\\
            &=\sup_{\substack{x\in E\\Bx\neq 0}}\frac{ \| ABx \| }{ \| x \| }\frac{ \| Bx \|_F }{ \| Bx \|_F }\\
            &=\sup_{\substack{x\in E\\Bx\neq 0}}\frac{ \| ABx \| }{ \| Bx \| }\frac{ \| Bx \| }{ \| x \| }\\
            &\leq\underbrace{\sup_{\substack{x\in E\\Bx\neq 0}}\frac{ \| ABx \| }{ \| Bx \| }}_{\leq\| A \|}\underbrace{\sup_{\substack{y\in E\\By\neq 0}}\frac{ \| Bx \| }{ \| y \| }}_{=\| B \|}\\
            &\leq \| A \|\| B \|.
        \end{align}
    \end{subequations}
    La dernière inégalité provient que dans \( \sup_{\substack{x\in E\\Bx\neq 0}}\| ABx \|/\| x \|\), le supremum est pris sur un ensemble plus petit que celui sur lequel porte la définition de la norme de \( A\) : seulement l'image de \( B\) au lieu de tout l'espace de départ de \( A\).
\end{proof}



\begin{proposition}     \label{PROPooJGNFooEwtNmJ}
    Soient deux espaces vectoriels normés \( E\) et \( V\). Soient des applications continues \( f,g\colon E\to \End(V)\). Alors l'application
    \begin{equation}
        \begin{aligned}
            \psi\colon E&\to \End(V) \\
            x&\mapsto f(x)\circ g(x) 
        \end{aligned}
    \end{equation}
    est continue.
\end{proposition}

\begin{proof}
    Soit une suite \( x_k\stackrel{E}{\longrightarrow}x\). Nous devons montrer que \( \psi(x_k)\stackrel{\End(V)}{\longrightarrow}\psi(x)\). Pour cela nous utilisons le lemme \ref{LEMooFITMooBBBWGI} qui indique que la norme opérateur est une norme d'algèbre. Nous avons :
    \begin{subequations}
        \begin{align}
            \| \psi(x_k)-\psi(x) \|&=\| f(x_k)\circ g(x_k)-f(x)\circ g(x) \|\\
            &\leq \| f(x_k)\circ g(x_k)-f(x_k)\circ g(x) \|+\| f(x_k)\circ g(x)-f(x)\circ g(x) \|\\
            &=\| f(x_k)\circ \big( g(x_k)\circ g(x) \big) \|+\| \big(f(x_k)-f(x)\big)\circ g(x) \|\\
            &\leq \| f(x_k) \|\| g(x_k)-g(x) \|+\| f(x_k)-f(x) \|\| g(x) \|.
        \end{align}
    \end{subequations}
    Pour \( k\to \infty\) nous avons \( \| f(x_k)\to \| f(x) \| \|\), \( \| f(x_k)-f(x) \|\to 0\) (parce que \( f\) est continue) et similaire avec \( g\). Donc le tout tend vers zéro.
\end{proof}

%--------------------------------------------------------------------------------------------------------------------------- 
\subsection{Rayon spectral}
%---------------------------------------------------------------------------------------------------------------------------

La chose impressionnante dans la proposition suivante est que \( \rho(A)\) est définit indépendamment du choix de la norme sur \( \eM(n,\eK)\) ou sur \( \eK\). Lorsque nous écrivons \( \| A \|\), nous disons implicitement qu'une norme a été choisie sur \( \eK\) et que nous avons pris la norme subordonnée sur \( \eM(n,\eK)\).
\begin{proposition}[\cite{ooETMNooSrtWet}]      \label{PROPooWZJBooTPLSZp}
    Soit \( A\) une matrice de \( \eM(n,\eR)\) ou \( \eM(n,\eC)\). Alors
    \begin{equation}
        \rho(A)\leq \| A \|.
    \end{equation}
\end{proposition}

\begin{proof}
    Nous devons séparer les cas suivant que le corps de base soit \( \eR\) ou \( \eC\).

    \begin{subproof}
        \item[Pour \( A\in \eM(n,\eC)\)]
            Soit \( \lambda\) une valeur propre de \( A\) telle que \( | \lambda |\) soit la plus grande. Nous avons donc \( \rho(A)=| \lambda |\). Soit un vecteur propre \( u\in \eC^n\) pour la valeur propre \( \lambda\). En prenant la norme sur l'égalité \( Au=\lambda u\), et en utilisant le lemme~\ref{LEMooIBLEooLJczmu},
            \begin{equation}
                | \lambda |\| u \|=\| Au \|\leq \| A \|\| u \|.
            \end{equation}
            Donc \( | \lambda |\leq \| A \|\) et \( \rho(A)\leq\| A \|\).

        \item[Pour \( A\in \eM(n,\eR)\)]

            L'endroit qui coince dans le raisonnement fait pour \( \eM(n,\eC)\) est que certes \( A\in \eM(n,\eR)\) possède une plus grande valeur propre en module et qu'un vecteur propre lui est associé. Mais ce vecteur propre est à priori dans \( \eC^n\), et non dans \( \eR^n\). Nous pouvons donc écrire \( Au=\lambda u\), mais pas \( \| Au \|=| \lambda |\| u \|\) parce que nous ne savons pas quelle norme prendre sur \( \eC^n\).

            Il n'est pas certain que nous ayons une norme sur \( \eC^n\) qui se réduit sur \( \eR^n\) à celle choisie implicitement dans l'énoncé. Nous allons donc ruser un peu.

            Soit une norme \( N\) sur \( \eC^n\)\footnote{Il y en a plein, par exemple celle du produit scalaire \( \langle x, y\rangle =\sum_kx_k\bar y_k\).}. Nous nommons également \( N\) la norme subordonnée sur \( \eM(n,\eC)\) et la norme restreinte sur \( \eM(n,\eR)\). Vu que \( N\) est une norme sur \( \eM(n,\eR)\) et que ce dernier est de dimension finie, le théorème~\ref{ThoNormesEquiv} nous indique que \( N\) est équivalente à \( \| . \|\). Il existe donc \( C>0\) tel que
            \begin{equation}        \label{EQooBNWMooNgnMxC}
                 N(B)\leq C\| B \|
            \end{equation}
            pour tout \( B\in \eM(n,\eR)\). Nous avons maintenant
            \begin{equation}
                \rho(A)^m\leq N(A^m)\leq C\| A^m \|\leq C\| A \|^m.
            \end{equation}
            Justifications
            \begin{itemize}
                \item Par la proposition~\ref{PROPooKLFKooSVnDzr}.
                \item Parce que \( A^m\in \eM(n,\eR)\) et la relation \eqref{EQooBNWMooNgnMxC}.
                \item Par itération du lemme~\ref{LEMooFITMooBBBWGI}.
            \end{itemize}

            Nous avons donc \( \rho(A)\leq C^{1/m}\| A \|\) pour tout \( m\in\eN\). En prenant \( m\to \infty\) et en tenant compte de \( C^{1/m}\to 1\) nous trouvons \( \rho(A)\leq \| A \|\).
    \end{subproof}
\end{proof}

\begin{lemma}[\cite{ooETMNooSrtWet}]        \label{LEMooGBLJooCPvxNl}
    Soit \( A\in \eM(n,\eK)\) avec \( \eK=\eR\) ou \( \eC\). Soit \( \epsilon>0\). Il existe une norme algébrique sur \( \eM(n,\eK)\) telle que
    \begin{equation}
        N(A)\leq \rho(A)+\epsilon.
    \end{equation}
\end{lemma}

\begin{proof}
    Soit par le lemme~\ref{LemSchurComplHAftTq} une matrice inversible \( U\) telle que \( T=UAU^{-1}\) soit triangulaire supérieure, avec les valeurs propres sur la diagonale. Notons que même si \( A\in \eM(n,\eR)\), les matrices \( U\) et \( T\) sont à priori complexes.

    Soit \( s\in \eR\) ainsi que les matrices
    \begin{equation}
        D_s=\diag(1,s^{-1},s^{-2},\ldots, s^{1-n})
    \end{equation}
    et \( T_s=D_sTD_s^{-1}\). Nous fixerons un choix de \( s\) plus tard.

    La norme que nous considérons est :
    \begin{equation}
        N(B)=\| (D_sU)B(D_sU)^{-1} \|_{\infty}
    \end{equation}
    où \( \| . \|_{\infty}\) est la norme sur \( \eM(,n\eK)\) subordonnée à la norme \( \| . \|_{\infty}\) sur \( \eK^n\) dont nous avons déjà parlé dans l'exemple~\ref{EXooXPXAooYyBwMX}. Cela est bien une norme parce que
    \begin{itemize}
        \item Nous avons \( \| B \|_{\infty}=0\) si et seulement si \( B=0\), et vu que \( (D_sU)\) est inversible nous avons \( (D_sU)B(D_sU)^{-1}=0\) si et seulement si \( B=0\).
        \item \( N(\lambda B)=| \lambda |N(B)\).
        \item Pour l'inégalité triangulaire :
            \begin{subequations}
                \begin{align}
             N(B+C)&=\| (D_sU)B(D_sU)^{-1}+(D_sU)C(D_sU)^{-1} \|_{\infty}\\
             &\leq  \| (D_sU)B(D_sU)^{-1}\|_{\infty} +\| (D_sU)C(D_sU)^{-1} \|_{\infty} \\
             &=N(B)+N(C).
                \end{align}
            \end{subequations}
    \end{itemize}

    En ce qui concerne la matrice \( A\) elle-même, nous avons
    \begin{equation}
        N(A)=\| (D_sU)A(D_sU)^{-1} \|_{\infty}=\| T_s \|_{\infty}.
    \end{equation}
    C'est le moment de se demander comment se présente la matrice \( T_s\). En tenant compte du fait que \( (D_s)_{ik}=\delta_{ik}s^{1-i}\) nous avons
    \begin{equation}
        (T_s)_{ij}=\sum_{kl}(D_s)_{ik}T_{kl}(D^{-1}_s)_{lj}=T_{ij}s^{j-i}.
    \end{equation}
    La matrice \( T\) est encore triangulaire supérieure avec les valeurs propres de \( A\) sur la diagonale. Les éléments au-dessus de la diagonale sont tous multipliés par au moins \( s\). Il est donc possible de choisir \( s\) suffisamment petit pour avoir\quext{Il me semble qu'il manque un module dans \cite{ooETMNooSrtWet}.}
    \begin{equation}        \label{EQooSIEIooTWAXQD}
        \sum_{j=i+1}^n| (T_s)_{ij} |<\epsilon
    \end{equation}
    Avec ce choix, la formule~\ref{EQooPLCIooVghasD} donne
    \begin{equation}
        N(T_s)\leq\max_i\sum_k| (T_s)_{ik} |\leq \epsilon+\rho(A).
    \end{equation}
    En effet le \( \epsilon\) vient de la somme sur toute la ligne sauf la diagonale (c'est-à-dire la partie \( k\neq i\)) et du choix \eqref{EQooSIEIooTWAXQD} pour \( s\). Le \( \rho(A)\) provient du dernier terme de la somme (le terme sur la diagonale) qui est une valeur propre de \( A\), donc majorable par \( \rho(A)\).

    Nous devons encore prouver que \( N\) est une norme algébrique. Pour cela nous allons montrer qu'elle est subordonnée à la norme
    \begin{equation}
        \begin{aligned}
            n\colon \eK^n&\to \eR^+ \\
            v&\mapsto \| (UD_s)v \|_{\infty}.
        \end{aligned}
    \end{equation}
    Cela sera suffisant pour avoir une norme algébrique par le lemme~\ref{LEMooFITMooBBBWGI}. La norme \( n\) sur \( \eK^n\) produit la norme suivante sur \( \eM(n,\eK)\) :
    \begin{equation}
        n(B)=\sup_{v\neq 0}\frac{ n(B) }{ n(v) }=\sup_{v\neq 0}\frac{ \| (UD_s)Bv \|_{\infty} }{ \| UD_sv \|_{\infty} }.
    \end{equation}
    Vu que \( UD_s\) est inversible nous pouvons effectuer le changement de variables \( v\mapsto (UD_s)^{-1} v\) pour écrire
    \begin{equation}
        n(B)=\sup_{v\neq 0}  \frac{  \| (UD_s)B(UD_s)^{-1}v \|_{\infty} }{ \| (UD_s)(UD_s)^{-1}v \|_{\infty} }=\sup_{v\neq 0}\frac{  \| (UD_s)B(UD_s)^{-1}v \|_{\infty} }{ \| v \|_{\infty} }=\| (UD_s)B(UD_s)^{-1} \|_{\infty}=N(B).
    \end{equation}
\end{proof}

\begin{proposition}     \label{PROPooYPLGooWKLbPA}
    Si \( A\in \eM(n,\eR)\) alors \( \rho(A)^m=\rho(A^m)\) pour tout \( m\in \eN\).
\end{proposition}

\begin{proof}
    La matrice \( A\) peut être vue dans \( \eM(n,\eC)\) et nous pouvons lui appliquer le corolaire~\ref{CORooTPDHooXazTuZ} :
    \begin{equation}        \label{EQooJJIYooDBacjn}
        \Spec(A^k)=\{ \lambda^k\tq \lambda\in\Spec(A) \}.
    \end{equation}
    À noter qu'il n'y a pas de magie : le spectre de la matrice réelle \( A\) est déjà défini en voyant \( A\) comme matrice complexe. Le spectre dont il est question dans \eqref{EQooJJIYooDBacjn} est bien celui dont on parle dans la définition du rayon spectral.

    Nous avons ensuite :
    \begin{subequations}
        \begin{align}
            \rho(A^k)&=\max\{ | \lambda |\tq \lambda\in\Spec(A^k) \}\\
            &=\max\{ | \lambda^k |\tq \lambda\in\Spec(A) \}\\
            &=\max\{ | \lambda |^k\tq\lambda\in\Spec(A) \}\\
            &=\rho(A)^k.
        \end{align}
    \end{subequations}
\end{proof}

\begin{proposition}[Bornée si et seulement si continue\cite{GKPYTMb}]       \label{PROPooQZYVooYJVlBd}
    Soient \( E\) et \( F\) des espaces vectoriels normés. Une application linéaire \( E\to F\) est bornée si et seulement si elle est continue.
\end{proposition}

\begin{proof}
    Nous commençons par supposer que \( A\) est bornée. Par le lemme~\ref{LEMooFITMooBBBWGI}, pour tout \( x,y\in E\), nous avons
    \begin{equation}
        \| A(x)-A(y) \|=\| A(x-y) \|\leq \| A \|\| x-y \|.
    \end{equation}
    En particulier si \( x_n\stackrel{E}{\longrightarrow}x\) alors
    \begin{equation}
        0\leq \| A(x_n)-A(x) \|\leq \| A \|\| x_n-x \|\to 0
    \end{equation}
    et \( A\) est continue en vertu de la caractérisation séquentielle de la continuité, proposition~\ref{PropFnContParSuite}.

    Nous supposons maintenant que \( \| A \|\) n'est pas borné : l'ensemble \( \{ \| A(x) \|\tq \| x \|=1 \}\) contient des valeurs arbitrairement grandes. Alors pour tout \( k\geq 1\) il existe \( x_k\in B(0,1)\) tel que \( \| A(x_k) \|>k\). La suite \( x_k/k\) tend vers zéro parce que \( \| x_k \|=1\), mais \( \| A(x_k) \|\geq 1\) pour tout \( k\). Cela montre que \( A\) n'est pas continue.
\end{proof}

\begin{definition}[\cite{ooAISYooXtUafT}]      \label{DEFooTLQUooJvknvi}
    Soient \( E\) et \( F\) deux espaces vectoriels normés.
    \begin{itemize}
        \item
            L'ensemble des applications linéaires \( E\to F\) est noté \( \aL(E,F)\).
        \item Un \defe{morphisme}{morphisme!espace vectoriel normé} est une application linéaire \( E\to F\) continue pour la topologie de la norme opérateur. Nous avons vu dans la proposition~\ref{PROPooQZYVooYJVlBd} que la continuité était équivalente à être bornée. L'ensemble des morphismes est noté \( \cL(E,F)\)\nomenclature[B]{\( \cL(E,F)\)}{applications linéaires bornées (continues)}.
        \item
            Un \defe{isomorphisme}{isomorphisme!espace vectoriel normé} est un morphisme continu inversible dont l'inverse est continu. Nous notons \( \GL(E,F)\) l'ensemble des isomorphismes entre \( E\) et \( F\).
    \end{itemize}
\end{definition}

Le point important de la définition~\ref{DEFooTLQUooJvknvi} est la continuité. En dimension infine, la continuité n'est par exemple pas équivalente à l'inversibilité (penser à \( e_k\mapsto ke_k\)).

Si \( V\) est un espace vectoriel normé, nous avons déjà défini son dual topologique \( V'\) comme étant l'ensemble des applications linéaires continues \( V\to \eC\) ou \( V\to \eR\) selon le corps de base de \( V\). C'est la définition \ref{DefJPGSHpn}.

\begin{proposition}
    Soient un espace vectoriel normé \( V\) et un élément \( v\in V\) vérifiant \( \| v \|=1\). Il existe une forme \( \varphi\in V'\) telle que \( \| \varphi \|=1\) et \( \varphi(v)=1\).
\end{proposition}

\begin{proof}
    Nous allons utiliser le théorème de la base incomplète \ref{THOooOQLQooHqEeDK}. Pour cela nous considérons \( I=V\) et la partie clairement génératrice \( G=\{ e_i=i \}_{i\in I}\) (si vous avez bien suivi, \( G=V\) en fait; rien de bien profond). Nous considérons ensuite \( I_0=\{ v \}\). Le théorème de la base incomplète nous donne l'existence de \( I_1\) tel que \( I_0\subset I_I\subset I\) et tel que \( B=\{ e_i \}_{i\in I_1}\) est une base.

    Tout cela pour dire que \( B=\{ e_i \}_{i\in I_1}\) est une base contenant \( v\). Nous allons aussi éventuellement redéfinir la norme de \( e_i\) pour avoir \( \| e_i \|=1\). Cette renormalisation n'affecte pas le fait que \( v\in B\).

    Nous passons maintenant à la définition de \( \varphi\colon V\to \eK\). Pour \( x\in V\) nous commençons par écrire
    \begin{equation}
        x=\sum_{j\in J}x_je_j
    \end{equation}
    et nous posons
    \begin{equation}
        \varphi(x)=\begin{cases}
            x_v    &   \text{si } v\in J\\
            0    &    \text{sinon. }
        \end{cases}
    \end{equation}
    Cette définition a un sens par la partie unicité de la proposition \ref{PROPooEIQIooXfWDDV} de décomposition d'un élément dans une base.

    Nous devons calculer la norme de \( \varphi\). Par la proposition \ref{DefNFYUooBZCPTr}\ref{ITEMooUQPRooYQGZzu} nous avons
    \begin{equation}        \label{EQooEFLLooOWPSev}
        \| \varphi \|=\sup_{\| x \|=1}| \varphi(x) |.
    \end{equation}
    Avec \( x=v\) nous avons \( \varphi(x)=1\) et donc \( \| \varphi \|\geq 1\).

    Nous devons encore montrer que \( \| \varphi \|\leq 1\). Un élément \( x\in V\) s'écrit toujours sous la forme
    \begin{equation}
        x=\sum_{i\in J}x_je_j
    \end{equation}
    pour un certain \( J\) fini dans \( I_1\) et pour certains \( x_j\in \eK\). Pour un tel \( x\) nous avons \( \varphi(x)=x_v\). Si \( |\varphi(x)|\geq 1\), alors \( | x_v |\geq 1\), mais alors
    \begin{equation}
        \| x \|\leq \sum_{j\in J}| x_j |\| e_j \|=\sum_{j\in J}| x_j |\geq | x_v |>1,
    \end{equation}
    ce qui fait que ce \( x\) ne participe pas au supremum \eqref{EQooEFLLooOWPSev}.

    Notons que \( \varphi\) est continue (et donc bien dans \( V'\)) parce qu'elle est bornée (proposition \ref{PROPooQZYVooYJVlBd}).
\end{proof}

%---------------------------------------------------------------------------------------------------------------------------
\subsection{Normes de matrices et d'applications linéaires}
%---------------------------------------------------------------------------------------------------------------------------
\label{subsecNomrApplLin}

\begin{theorem}[Norme matricielle et rayon spectral\cite{ooBCKVooVunKyT}]       \label{THOooNDQSooOUWQrK}
    La norme $2$ d'une matrice est liée au rayon spectral de la façon suivante :
    \begin{equation}
        \|A\|_2=\sqrt{\rho(A{^t}A)}
    \end{equation}
    ou plus généralement par \( \| A \|_2=\sqrt{\rho(A^*A)}\).
\end{theorem}

\begin{lemma}       \label{LEMooNESTooVvUEOv}
    Soit une matrice \( A\in \eM(n,\eR)\) qui est symétrique, strictement définie positive. Soient \( \lambda_{min}\) et \( \lambda_{max}\) les plus petites et plus grandes valeurs propres. Alors
    \begin{subequations}
        \begin{align}
            \| A \|_2=\lambda_{max}&&\text{ et }&&\|A^{-1}  \|_2=\frac{1}{ \lambda_{min} }.
        \end{align}
    \end{subequations}
\end{lemma}

\begin{proof}
    Soient les vecteurs \( v_1,\ldots, v_n\) formant une base orthonormée de vecteurs propres\footnote{Possible par le théorème spectral~\ref{ThoeTMXla}.} de \( A\). Nous notons \( v_{max}\) celui de \( \lambda_{max}\). Nous avons :
    \begin{equation}
        \| A \|_2\geq \| Av_{max} \|=| \lambda_{max} |\| v_{max} \|=| \lambda_{max} |=\lambda_{max}.
    \end{equation}
    Voilà l'inégalité dans un sens. Montrons l'inégalité dans l'autre sens. Soit \( x=\sum_ix_iv_i\) avec \( \| x \|_2=1\). Alors
    \begin{equation}
        \| Ax \|=\| \sum_ix_i\lambda_iv_i \|\leq\sqrt{ \sum_ix_i^2\lambda_i^2 }\leq \lambda_{max}\sqrt{ \sum_ix_i^2}=\lambda_{max}.
    \end{equation}

    En ce qui concerne l'affirmation pour la norme de \( A^{-1}\), il suffit de remarquer que ses valeurs propres sont les inverses des valeurs propres de \( A\).
\end{proof}

\begin{proposition} \label{PropMAQoKAg}
    La fonction
    \begin{equation}
        \begin{aligned}
            f\colon \eM(n,\eR)\times \eM(n,\eR)&\to \eR \\
            (X,Y)&\mapsto \tr(X^tY)
        \end{aligned}
    \end{equation}
    est un produit scalaire sur \( \eM(n,\eR)\).
\end{proposition}
\index{trace!produit scalaire sur \( \eM(n,\eR)\)}
\index{produit!scalaire!sur \( \eM(n,\eR)\)}

\begin{proof}
    Il faut vérifier la définition~\ref{DefVJIeTFj}.
    \begin{itemize}
        \item La bilinéarité est la linéarité de la trace.
        \item La symétrie de \( f\) est le fait que \( \tr(A^t)=\tr(A)\).
        \item L'application \( f\) est définie positive parce que si \( X\in \eM\), alors \( X^tX\) est symétrique définie positive, donc diagonalisable avec des nombres positifs sur la diagonale. La trace étant un invariant de similitude, nous avons \( f(X,X)=\tr(X^tX)\geq 0\). De plus si \( \tr(X^tX)=0\), alors \( X^tX=0\) (pour la même raison de diagonalisation). Mais alors \( \| Xu \|=0\) pour tout \( u\in E\), ce qui signifie que \( X=0\).
    \end{itemize}
\end{proof}

\begin{example}
	Soit $m=n$, un point $\lambda$ dans $\eR$ et $T_{\lambda}$ l'application linéaire définie par $T_{\lambda}(x)=\lambda x$. La norme de $T_{\lambda}$ est alors
\[
\|T_{\lambda}\|_{\mathcal{L}}=\sup_{\|x\|_{\eR^m}\leq 1}\|\lambda x\|_{\eR^n}= |\lambda|.
\]
Notez que $T_{\lambda}$ n'est rien d'autre que l'homothétie de rapport $\lambda$ dans $\eR^m$.
\end{example}

\begin{example}
	Toutes les isométries ont norme \( 1\). En effet si \( T\) est une isométrie, $\| Tx \|=\| x \|$. En ce qui concerne la norme de $T$ nous avons alors
	\begin{equation}
		\| T \|=\sup_{x\in\eR^2}\frac{ \| T(x) \| }{ \| x \| }=\sup_{x\in\eR^2}\frac{ \| x \| }{ \| x \| }=1.
	\end{equation}
\end{example}

\begin{example}
  Soit $m=n$, un point $b$ dans $\eR^m$ et $T_b$ l'application linéaire définie par $T_b(x)=b\cdot x$ (petit exercice : vérifiez qu'il s'agit vraiment d'une application linéaire).  La norme de $T_b$ satisfait les inégalités suivantes
 \[
\|T_b\|_{\mathcal{L}}=\sup_{\|x\|_{\eR^m}\leq 1}\|b\cdot x\|_{\eR^n}\leq \sup_{\|x\|_{\eR^m}\leq 1}\|b \|_{\eR^n}\|x\cdot x\|_{\eR^n}\leq\|b \|_{\eR^n},
\]
\[
\|T_b\|_{\mathcal{L}}=\sup_{\|x\|_{\eR^m}\leq 1}\|b\cdot x\|_{\eR^n}\geq \left\|b\cdot \frac{b}{\|b \|_{\eR^n}}\right\|_{\eR^n}=\|b \|_{\eR^n},
\]
donc $\|T_b\|_{\mathcal{L}}=\|b \|_{\eR^n}$.
\end{example}

\begin{proposition}
    Une application linéaire de \( \eR^m\) dans \( \eR^n\) est continue.
\end{proposition}

\begin{proof}
      Soit $x$ un point dans $\eR^m$. Nous devons vérifier l'égalité
      \begin{equation}
       \lim_{h\to 0_m}T(x+h)=T(x).
      \end{equation}
      Cela revient à prouver que $\lim_{h\to 0_m}T(h)=0$, parce que $T(x+h)=T(x)+T(h)$. Nous pouvons toujours majorer $\|T(h)\|_n$ par $\|T\|_{\mathcal{L}(\eR^m,\eR^n)}\| h \|_{\eR^m}$ (lemme~\ref{LEMooIBLEooLJczmu}). Quand $h$ s'approche de $ 0_m $ sa norme $\|h\|_m$ tend vers $0$, ce que nous permet de conclure parce que nous savons que de toutes façons, $\| T \|_{\aL}$ est fini.
\end{proof}

Note : dans un espace de dimension infinie, la linéarité ne suffit pas pour avoir la continuité : il faut de plus être borné (ce que sont toutes les applications linéaires \( \eR^m\to\eR^n\)). Voir la proposition~\ref{PROPooQZYVooYJVlBd}.

%---------------------------------------------------------------------------------------------------------------------------
\subsection{Application linéaire continue et bornée}
%---------------------------------------------------------------------------------------------------------------------------

Nous avons vu dans la proposition~\ref{PROPooQZYVooYJVlBd} que pour une application linéaire, être bornée est équivalent à être continue. Nous allons maintenant voir un certain nombre d'exemples illustrant ce fait.

\begin{example}[Une application linéaire non continue]  \label{ExHKsIelG}
    Soit \( V\) l'espace vectoriel normé des suites \emph{finies} de réels muni de la norme usuelle $\| c \|=\sqrt{\sum_{i=0}^{\infty}| c_i |^2}$ où la somme est finie. Nous nommons \( \{ e_k \}_{k\in \eN}\) la base usuelle de cet espace, et nous considérons l'opérateur \( f\colon V\to V\) donnée par \( f(e_k)=ke_k\). C'est évidemment linéaire, mais ce n'est pas continu en zéro. En effet la suite \( u_k=e_k/k\) converge vers \( 0\) alors que \( f(u_k)=e_k\) ne converge pas.
\end{example}

Cet exemple aurait pu également être donnée dans un espace de Hilbert, mais il aurait fallu parler de domaine.
%TODO : le faire, et regarder si Hilbet n'est pas la complétion de cet espace. Référencer à l'endroit qui définit l'espace vectoriel librement engendré. Ici ce serait par N.

%TODO : dire qu'une application bilinéaire sur RxR n'est pas une application linéaire sur R^2

\begin{example}[Une autre application linéaire non continue\cite{GTkeGni}]      \label{EXooDMVJooAJywMU}
    En dimension infinie, une application linéaire n'est pas toujours continue. Soit \( E\) l'espace des polynômes à coefficients réels sur \( \mathopen[ 0 , 1 \mathclose]\) muni de la norme uniforme. L'application de dérivation \( \varphi\colon E\to E\), \( \varphi(P)=P'\) n'est pas continue.

    Pour la voir nous considérons la suite \( P_n=\frac{1}{ n }X^n\). D'une part nous avons \( P_n\to 0\) dans \( E\) parce que \( P_n(x)=\frac{ x^n }{ n }\) avec \( x\in \mathopen[ 0 , 1 \mathclose]\). Mais en même temps nous avons \( \varphi(P_n)=X^{n-1}\) et donc \( \| \varphi(P_n) \|=1\).

    Nous n'avons donc pas \( \lim_{n\to \infty} \varphi(P_n)=\varphi(\lim_{n\to \infty} P_n)\) et l'application \( \varphi\) n'est pas continue en \( 0\). Elle n'est donc continue nulle part par linéarité.

    Nous avons utilisé le critère séquentiel de la continuité, voir la définition~\ref{DefENioICV} et la proposition~\ref{PropFnContParSuite}.
\end{example}

\begin{remark}  \label{RemOAXNooSMTDuN}
Cette proposition permet de retrouver l'exemple~\ref{ExHKsIelG} plus simplement. Si \( \{ e_k \}_{k\in \eN}\) est une base d'un espace vectoriel normé formée de vecteurs de norme \( 1\), alors l'opérateur linéaire donné par \( u(e_k)=ke_k\) n'est pas borné et donc pas continu.
\end{remark}

C'est également ce résultat qui montre que le produit scalaire est continu sur un espace de Hilbert par exemple.

\begin{example}     \label{EXooTQPEooRRdddt}
    Nous avons vu dans le lemme \ref{LEMooWFNXooLyTyyX} que pour un \( x\in E\) donné, l'application
    \begin{equation}
        \begin{aligned}
            ev_x\colon \aL(E,F)&\to F \\
            f&\mapsto f(x) 
        \end{aligned}
    \end{equation}
    est continue. Vu que \( ev_x\) est linéaire, la proposition \ref{PROPooQZYVooYJVlBd} nous indique que \( ev_x\) est bornée. Vérifions-le directement. Le calcul n'est pas très compliqué :
    \begin{equation}
        \| ev_x \|=\sup_{\| f \|=1}\| ev_x(f) \|=\sup_{\| f \|=1}\| f(x) \|\leq \sup_{\| f \|=1}\| x \|\| f \|=\| x \|
    \end{equation}
    où nous avons utilisé le lemme \ref{LEMooIBLEooLJczmu} en passant. Donc la norme de \( ev_x\) est majorée par \( \| x \|\).

    Elle est même égale à \( \| x \|\). En effet, pour chaque \( f\in \aL(E,F)\) tel que \(  \| f \|=1\), nous avons
    \begin{equation}
        \| ev_x \|\geq \| ev_x(f) \|=\| f(x) \|.
    \end{equation}
    En prenant \( f=\id\) nous trouvons \(  \| ev_x \|\geq \| x \|  \).
\end{example}

\begin{definition}      \label{DEFooKSDFooGIBtrG}
    Soit un espace vectoriel \( E\) sur le corps \( \eK\). Son \defe{dual topologique}{dual topologique}, noté \( E'\), est l'ensemble des formes linéaires continues de \( E\) vers \( \eK\).
\end{definition}

\begin{lemma}   \label{LemWWXVSae}
Soit \( F\) un espace de Banach et deux suites \( A_k\to A\) et \( B_k\to B\) dans \( \aL(F,F)\). Alors \( A_k\circ B_k\to A\circ B\) dans \( \aL(F,F)\), c'est-à-dire
\begin{equation}
    \lim_{n\to \infty} (A_kB_k)=\left( \lim_{n\to \infty} A_k \right)\left( \lim_{n\to \infty} B_k \right).
\end{equation}
\end{lemma}

\begin{proof}
    Il suffit d'écrire
    \begin{equation}
        \| A_kB_k-AB \|\leq \| A_kB_k-A_kB \|+\| A_kB-AB \|.
    \end{equation}
    Le premier terme tend vers zéro pour \( k\to\infty\) parce que
    \begin{subequations}
        \begin{align}
            \| A_kB_k-A_kB \|&=\| A_k(B_k-B) \|\\
            &\leq \| A_k \|\| B_k-B \|\to \| A \|\cdot 0\\
            &=0
        \end{align}
    \end{subequations}
    où nous avons utilisé la propriété fondamentale de la norme opérateur : la proposition~\ref{PROPooQZYVooYJVlBd}. Le second terme tend également vers zéro pour la même raison.
\end{proof}

\begin{proposition}[Distributivité de la somme infinie] \label{PropQXqEPuG}
    Soient \( E\) un espace normé, une suite \( (u_k)\) dans \( \GL(E)\) ainsi que \( a\in\GL(E)\). Pourvu que la série \( \sum_{n=0}^{\infty}u_k\) converge nous avons
    \begin{equation}
        \left( \sum_{k=0}^{\infty}u_k \right)a=\sum_{k=0}^{\infty}(u_ka).
    \end{equation}
\end{proposition}

\begin{proof}
    Par définition de la somme infinie,
    \begin{equation}
        \spadesuit=\left( \sum_{k=0}^{\infty}u_k \right)a=\left( \lim_{n\to \infty} \sum_{k=0}^nu_k \right)a.
    \end{equation}
    Le lemme~\ref{LemWWXVSae} appliqué à \( n\mapsto\sum_{k=0}^nu_k\) et à la suite constante \( a\) nous donne
    \begin{equation}    \label{EqOAoopjz}
        \spadesuit=\lim_{n\to \infty} \left( \sum_{k=0}u_ka \right),
    \end{equation}
    ce que nous voulions par distributivité de la somme finie : dans \eqref{EqOAoopjz}, le \( a\) est dans ou hors de la somme, au choix. L'important est qu'il soit dans la limite.
\end{proof}

\begin{proposition}[\cite{ooCUHNooNYIeGt}]      \label{PROPooQFTSooPFfbCc}
    Soient des espaces vectoriels normés \( V\) et \( W\) ainsi qu'une forme sesquilinéaire \( \phi\colon V\times W\to \eC\). Il y a équivalence des faits suivants.
    \begin{enumerate}
        \item
            \( \phi\) est continue.
        \item
            \( \phi\) est continue en \( (0,0)\)
        \item
            \( \phi\) est bornée
        \item
            Il existe \( C\geq 0\) telle que \( | \phi(x,y) |\leq C\| x \|\| y \|  \) pour tout \( (x,y)\in V\times W\).
    \end{enumerate}
    De plus la norme de \( \phi\) est alors donnée par
    \begin{equation}
        \| \phi \|=\min\{  C\geq 0\tq | \phi(x,y) |\leq C\| x \|\| y \|\forall (x,y)\in V\times W  \}.
    \end{equation}
\end{proposition}

On remarque tout de suite que la norme $\|.\|_\infty$ sur $\eR^2$ est la norme de l'espace produit $\eR\times\eR$. En outre cette définition nous permet de trouver plusieurs nouvelles normes dans les espaces $\eR^p$. Par exemple, si nous écrivons $\eR^4$ comme $\eR^2\times \eR^2$ on peut munir $\eR^4$ de la norme produit
\[
\|(x_1,x_2,x_3,x_4)\|_{\infty, 2}=\max\{\|(x_1,x_2)\|_\infty, \|(x_3,x_4)\|_2\}.
\]
Les applications de projection de l'espace produit $V\times W$ vers les espaces <<facteurs>>, $V$ $W$ sont notées $\pr_V$ et $\pr_W$ et sont définies par
\begin{equation}
	\begin{aligned}
		\pr_V\colon V\times W&\to V \\
		(v,w)&\mapsto v
	\end{aligned}
\end{equation}
et
\begin{equation}
	\begin{aligned}
		\pr_W\colon V\times W &\to W \\
		(v,w)&\mapsto w.
	\end{aligned}
\end{equation}
Les inégalités suivantes sont évidentes
\begin{equation}
	\begin{aligned}[]
		\|\pr_V(v,w)\|_V&\leq \|(v,w)\|_{V\times W} \\
		\|\pr_W(v,w)\|_W&\leq \|(v,w)\|_{V\times W}.
	\end{aligned}
\end{equation}
La topologie de l'espace produit est induite par les topologies des espaces <<facteurs>>. La construction est faite en deux passages : d'abord nous disons que une partie $A\times B$ de $V\times W$ est ouverte si $A$ et $B$ sont des parties ouvertes de $V$ et de $W$ respectivement.  Ensuite nous définissons que une partie quelconque de $V\times W$ est ouverte si elle est une intersection finie ou une réunion de parties ouvertes de $V\times W$ de la forme $A\times B$.

Ce choix de topologie donne deux propriétés utiles de l'espace produit
\begin{enumerate}
	\item
		Les projections sont des \defe{applications ouvertes}{application!ouverte}. Cela veut dire que l'image par $\pr_V$ (respectivement $\pr_W$) de toute partie ouverte de $V\times W$ est une partie ouverte de $V$ (respectivement $W$).
	\item
		Pour toute partir $A$ de $V$ et $B$ de $W$, nous avons $\Int (A\times B)=\Int A\times \Int B$.\label{PgovlABeqbAbB}
\end{enumerate}
Une propriété moins facile a prouver est que pour toute partie $A$ de $V$ et $B$ de $W$ nous avons  $\overline{A\times B}=\bar{A}\times \bar{B}$. Voir le lemme~\ref{LemCvVxWcvVW}.
% position 26329
%et l'exercice~\ref{exoGeomAnal-0009}.

Ce que nous avons dit jusqu'ici est valable pour tout produit d'un nombre fini d'espaces vectoriels normés. En particulier, pour tout $m>0$  l'espace  $\eR^m$ peut être considéré comme le produit de $m$ copies de $\eR$.

\begin{example}
	Si $V$ et $W$ sont deux espaces vectoriels, nous pouvons considérer le produit $E=V\times W$. Les projections $\pr_V$ et $\pr_W$\nomenclature{$\pr_V$}{projection de $V\times W$ sur $V$}, définies dans la section~\ref{sec_prod}, sont des applications linéaires.

	En effet, la projection $\pr_V\colon V\times W\to V$ est donnée par $\pr_V(v,w)=v$. Alors,
	\begin{equation}
		\begin{aligned}[]
			\pr_V\big( (v,w)+(v',w') \big)&=\pr_V\big( (v+v'),(w+w') \big)\\
			&=v+v'\\
			&=\pr_V(v,w)+\pr_V(v',w'),
		\end{aligned}
	\end{equation}
	et
	\begin{equation}
		\pr_V\big( \lambda(v,w) \big)=\pr_V\big( (\lambda v,\lambda w) \big)=\lambda v=\lambda\pr_V(v,w).
	\end{equation}
	Nous laissons en exercice le soin d'adapter ces calculs pour montrer que $\pr_W$ est également une projection.
\end{example}

\begin{proposition} \label{PropDXR_KbaLC}
    Si \( \mO\) est un voisinage de \( (a,b)\) dans \( V\times W\) alors \( \mO\) contient un ouvert de la forme \( B(a,r)\times B(b,r)\).
\end{proposition}

\begin{proof}
    Vu que \( \mO\) est un voisinage, il contient un ouvert et donc une boule
    \begin{equation}
        B\big( (a,b),r \big)=\{ (v,w)\in V\times W\tq \max\{ \| v-a \|,\| w-b \| \}< r \}.
    \end{equation}
    Évidemment l'ensemble \( B(a,r)\times B(b,r)\) est dedans.
\end{proof}

%---------------------------------------------------------------------------------------------------------------------------
\subsection{Suites}
%---------------------------------------------------------------------------------------------------------------------------

Nous allons maintenant parler de suites dans $V\times W$. Nous noterons $(v_n,w_n)$ la suite dans $V\times W$ dont l'élément numéro $n$ est le couple $(v_n,w_n)$ avec $v_n\in V$ et $w_n\in W$. La notions de convergence de suite découle de la définition de la norme via la proposition \ref{PROPooOSXCooJWXkWH}. Il se fait que dans le cas des produits d'espaces, la convergence d'une suite est équivalente à la convergence des composantes. Plus précisément, nous avons le lemme suivant.
\begin{lemma}		\label{LemCvVxWcvVW}
	La suite $(v_n,w_n)$ converge vers $(v,w)$ dans $V\times W$ si et seulement les suites $(v_n)$ et $(w_n)$ convergent séparément vers $v$ et $w$ respectivement dans $V$ et $W$.
\end{lemma}

\begin{proof}
	Pour le sens direct, nous devons étudier le comportement de la norme de $(v_n,w_n)-(v,w)$ lorsque $n$ devient grand. En vertu de la définition de la norme dans $V\times W$ nous avons
	\begin{equation}
		\Big\| (v_n,w_n)-(v,w) \Big\|_{V\times W}=\max\big\{ \| v_n-v \|_V,\| w_n-w \|_W \big\}.
	\end{equation}
	Soit $\varepsilon>0$. Par définition de la convergence de la suite $(v_n,w_n)$, il existe un $N\in\eN$ tel que $n>N$ implique
	\begin{equation}
		\max\big\{ \| v_n-v \|_V,\| w_n-w \|_W \big\}<\varepsilon,
	\end{equation}
	et donc en particulier les deux inéquations
	\begin{subequations}
		\begin{align}
			\| v_n-v \|&<\varepsilon\\
			\| w_n-w \|&<\varepsilon.
		\end{align}
	\end{subequations}
	De la première, il ressort que $(v_n)\to v$, et de la seconde que $(w_n)\to w$.

	Pour le sens inverse, nous avons pour tout $\varepsilon$ un $N_1$ tel que $\| v_n-v \|_V\leq\varepsilon$ pour tout $n>N_1$ et un $N_2$ tel que $\| w_n-w \|_W\leq\varepsilon$ pour tout $n>N_2$. Si nous posons $N=\max\{ N_1,N_2 \}$ nous avons les deux inégalités simultanément, et donc
	\begin{equation}
		\max\big\{ \| v_n-v \|_V,\| w_n-w \|_W \big\}<\varepsilon,
	\end{equation}
	ce qui signifie que la suite $(v_n,w_n)$ converge vers $(v,w)$ dans $V\times W$.
\end{proof}

\begin{proposition}[\cite{MonCerveau}]          \label{PROPooKDGOooDjWQct}
    Soit un espace \( E\) muni d'un produit scalaire à valeurs dans \( \eK\) (si \( \eK=\eC\) nous supposons le produit hermitien, mais ce n'est pas très important ici). Alors l'application
    \begin{equation}
        \begin{aligned}
            a\colon E\times E&\to \eK \\
            (x,y)&\mapsto \langle x, y\rangle
        \end{aligned}
    \end{equation}
    est continue.
\end{proposition}

\begin{proof}
    Nous ne disons pas que l'espace \( V\times V\) est muni d'un produit scalaire. Mais en tout cas c'est un espace métrique, et \( \eK\) l'est aussi. Donc \( a\) est une application entre deux espaces métriques et elle sera continue si et seulement si elle est séquentiellement continue (proposition~\ref{PropFnContParSuite}\ref{ItemWJHIooMdugfu}).

    Soit donc une suite convergente dans \( E\times E\), c'est-à-dire \( (x_k,y_k)\stackrel{E\times E}{\longrightarrow}(x,y)\). Nous devons démontrer que \( \langle x_k, y_k\rangle \stackrel{\eR}{\longrightarrow}\langle x, y\rangle \). Les majorations usuelles donnent
    \begin{subequations}
        \begin{align}
            \big| \langle x_k, y_k\rangle -\langle x, y\rangle  \big|&\leq \big| \langle x_k, y_k\rangle -\langle x, y_k\rangle  \big|+\big| \langle x, y_k\rangle -\langle x, y\rangle  \big|\\
            &=\big| \langle x_k-x, y_k\rangle  \big|+\big| \langle x, y_k-y\rangle  \big|.
        \end{align}
    \end{subequations}
    Nous savons du lemme~\ref{LemCvVxWcvVW} que les suites \( (x_k)\) et \( (y_k)\) sont séparément convergentes : \( x_k\stackrel{E}{\longrightarrow}x\) et \( y_k\stackrel{E}{\longrightarrow}y\). En utilisant l'inégalité de Cauchy-Schwarz~\ref{EQooZDSHooWPcryG} nous trouvons
    \begin{equation}
        \big| \langle x_k-x, y_k\rangle  \big|\leq \| x_k-x \|\| y_k \|.
    \end{equation}
    Nous avons \( \| x_k-x \|\to 0\) et \( \| y_k \|\to \| y \|\), et par la règle du produit de limites dans \( \eR\) nous avons que \( \big| \langle x_k-x, y_k\rangle  \big|\to 0\).
\end{proof}

\begin{remark}		\label{RemTopoProdPasRm}
	Il faut remarquer que la norme \eqref{EqNormeVxWmax} est une norme \emph{par défaut}. C'est la norme qu'on met quand on ne sait pas quoi mettre. Or il y a au moins un cas d'espace produit dans lequel on sait très bien quelle norme prendre : les espaces $\eR^m$. La norme qu'on met sur $\eR^2$ est
	\begin{equation}
		\| (x,y) \|=\sqrt{x^2+y^2},
	\end{equation}
	et non la norme «par défaut» de $\eR^2=\eR\times\eR$ qui serait
	\begin{equation}
		\| (x,y) \|=\max\{ | x |,| y | \}.
	\end{equation}
	Les théorèmes que nous avons donc démontré à propos de $V\times W$ ne sont donc pas immédiatement applicables au cas de $\eR^2$.

	Cette remarque est valables pour tous les espaces $\eR^m$. À moins de mention contraire explicite, nous ne considérons jamais la norme par défaut \eqref{EqNormeVxWmax} sur un espace $\eR^m$.
\end{remark}

Étant donné la remarque~\ref{RemTopoProdPasRm}, nous ne savons pas comment calculer par exemple la fermeture du produit d'intervalle $\mathopen] 0,1 ,  \mathclose[\times\mathopen[ 4 , 5 [$. Il se fait que, dans $\eR^m$, les fermetures de produits sont quand même les produits de fermetures.

\begin{proposition}		\label{PropovlAxBbarAbraB}
	Soit $A\subset\eR^m$ et $B\subset\eR^m$. Alors dans $\eR^{m+n}$ nous avons $\overline{ A\times B }=\bar A\times \bar B$.
\end{proposition}

%---------------------------------------------------------------------------------------------------------------------------
\subsection{Continuité du produit de matrices}
%---------------------------------------------------------------------------------------------------------------------------
\label{SUBSECooOAWAooFcyUfI}

Nous avons introduit des normes sur \( \eM(n,\eK)\), entre autres la norme opérateur de la définition~\ref{DefNFYUooBZCPTr}. Qui dit norme dit topologie. Il advient alors la question évidente : est-ce que des opérations aussi élémentaires que le produit de matrices sont continues pour ces topologies ?

Une façon simple de répondre à cela est d'introduire sur \( \eM(n,\eK)\) une nouvelle norme très simple : celle de \( \eK^n\). C'est la topologie par composante. Pour cette topologie, il est simple de voir que le produit matriciel est continu parce que les éléments de \( AB\) sont des polynômes en les éléments de \( A\) \( B\). Ensuite il suffit d'invoquer l'équivalence de toutes les normes (théorème~\ref{ThoNormesEquiv}).

Voyons comment montrer cela de façon plus directe (bien que le raisonnement précédent soit une démonstration qui devrait déjà avoir convaincu les plus sceptiques). La preuve suivante va donc s'amuser à bien préciser les topologies et caractérisations utilisées.

\begin{lemma}
    Si \( \| . \|\) est une norme algébrique sur \( \eM(n,\eK)\) (\( \eK\) est \( \eR\) ou \( \eC\)) alors l'application
    \begin{equation}
        \begin{aligned}
            p\colon \eM(n,\eK)\times \eM(n,\eK)&\to \eM(n,\eK) \\
            (A,B)&\mapsto AB
        \end{aligned}
    \end{equation}
    est continue.
\end{lemma}

\begin{proof}
    L'espace \( \eM(n,\eK)\times \eM(n,\eK)\) est métrique (définition~\ref{DefFAJgTCE}), donc la caractérisation séquentielle de la continuité (proposition~\ref{PropXIAQSXr}) s'applique. Nous considérons donc une suite \( (A_k,B_k)\) dans \( \eM(n,\eK)\times \eM(n,\eK)\) convergente vers \( AB\).

    Nous savons que la topologie sur \( \eM(n,\eK)\times \eM(n,\eK)\) est la topologie produit (lemme~\ref{DefFAJgTCE}) et que celle-ci donne la convergence composante par composante dès que nous avons convergence d'une suite; c'est la proposition~\ref{PROPooNRRIooCPesgO}. Nous avons donc \( A_k\stackrel{\eM(n,\eK)}{\longrightarrow}A\) et \( B_k\stackrel{\eM(n,\eK)}{\longrightarrow}B\).

    Voilà pour le contexte. Maintenant, la preuve de la continuité. Nous effectuons les majorations suivantes :
    \begin{subequations}
        \begin{align}
            \| p(A_k,B_K)-AB \|&\leq \| p(A_k,B_k)-p(A_k,B) \|+\| p(A_k,B)-AB \|\\
            &=\| A_Kb_k-A_kB \|+\| A_kB-AB \|\\
            &=\| A_k(B_k-B) \|+\| (A_k-A)B \|\\
            &\leq \underbrace{\| A_k \|}_{\to \| A \|}\underbrace{\| B_k-B \|}_{\to 0}+\underbrace{\| A_k-A \|}_{\to 0}\| B \|.
        \end{align}
    \end{subequations}
\end{proof}

%+++++++++++++++++++++++++++++++++++++++++++++++++++++++++++++++++++++++++++++++++++++++++++++++++++++++++++++++++++++++++++
\section{Applications multilinéaires}
%+++++++++++++++++++++++++++++++++++++++++++++++++++++++++++++++++++++++++++++++++++++++++++++++++++++++++++++++++++++++++++

\begin{definition}[Application multilinéaire]       \label{DefFRHooKnPCT}
    Une application $T: \eR^{m_1}\times \ldots \times\eR^{m_k}\to\eR^p $ est dite \defe{\( k\)-linéaire}{application!multilinéaire} si pour tout $X=(x_1, \ldots,x_k)$ dans $ \eR^{m_1}\times \ldots \times\eR^{m_k}$ les applications $x_i\mapsto T(x_1, \ldots, x_i,\ldots,x_k)$ sont linéaires pour tout $i$ dans $\{1,\ldots,k\}$, c'est-à-dire
	\begin{equation}
		\begin{aligned}[]
			T(\cdot,x_2, \ldots, x_i,\ldots,x_k)&\in \mathcal{L}(\eR^{m_1}, \eR^p),\\
			T(x_1,\cdot, \ldots, x_i,\ldots,x_k)&\in \mathcal{L}(\eR^{m_2}, \eR^p),\\
						& \vdots\\
			T(x_1, \ldots, x_i,\ldots,x_{k-1},\cdot)&\in \mathcal{L}(\eR^{m_k}, \eR^p).\\
		\end{aligned}
	\end{equation}
	En particulier lorsque $k=2$, nous parlons d'applications \defe{bilinéaires}{bilinéaire}. Vous pouvez deviner ce que sont les applications \emph{tri}linéaire ou \emph{quadri}linéaire.
\end{definition}

L'ensemble des applications $k$-linéaires de $ \eR^{m_1}\times \ldots \times\eR^{m_k}$ dans $\eR^p$ est noté $\mathcal{L}(\eR^{m_1}\times \ldots \times\eR^{m_k}, \eR^p)$ ou $\mathcal{L}(\eR^{m_1}, \ldots,\eR^{m_k}; \eR^p)$.

\begin{example}
  Soit $A$ une matrice avec $m$ lignes et $n$ colonnes. L'application bilinéaire de $\eR^m\times \eR^n$ dans $\eR$ associée à $A$ est définie par
\[
T_A(x,y)= x^TAy=\sum_{i,j}a_{i,j}x_i y_j, \qquad \forall x\in \eR^m, \, y \in \eR^n.
\]
\end{example}

Nous énonçons la proposition suivante dans le cas d'espaces vectoriels normés\footnote{Sans hypothèses sur la dimension.} parce que nous allons l'utiliser dans ce cas, mais le cas particulier \( E_i=\eR^{m_i}\) et \( F=\eR^p\) est important.
\begin{proposition} \label{PropUADlSMg}
    Soient des espaces vectoriels normés \( E_i\) et \( F\). Une application \( n\)-linéaire
    \begin{equation}
        T\colon E_1\times\ldots\times E_n\to F
    \end{equation}
    est est continue si et seulement s'il existe un réel $L\geq 0$ tel que
  \begin{equation}\label{limitatezza}
     \|T(x_1, \ldots,x_n)\|_F\leq L \|x_1\|_{F_1}\cdots\|x_n\|_{F_n}, \qquad \forall x_i\in E_i.
  \end{equation}
\end{proposition}

\begin{proof}
    Pour simplifier l'exposition nous nous limitons au cas $n=2$ et nous notons $T(x,y)=x*y$

    Supposons que l'inégalité \eqref{limitatezza} soit satisfaite.
    \begin{equation}\label{LimImplCont}
      \begin{aligned}
        \|x*y-x_0*y_0\|&=\|(x-x_0)*y-x_0*(y-y_0)\|\\
    &\leq \|(x-x_0)*y\|+\|x_0*(y-y_0)\|\\
    &\leq L\|x-x_0\|\|y\| + L\|x_0\|\|y-y_0\|.
      \end{aligned}
    \end{equation}
    Si $x\to x_0$ et $y\to y_0$  on voit que $T$ est continue en passant à la limite aux deux côtés de l'inégalité \eqref{LimImplCont}.

    Soit $T$ continue en $(0,0)$. Évidemment\footnote{Dans la formule suivante, les trois zéros sont les zéros de trois espaces différents.} $0*0=0$, donc il existe $\delta>0$ tel que si $x\in B_{E_1}(0,\delta)$ et $y\in B_{E_2}(0,\delta)$ alors $\|x*y\|\leq 1$. En particulier si \( (x,y)\in B_{E_1\times E_2}(0,\delta)\) nous sommes dans ce cas. Soient maintenant  $x\in E_1\setminus\{ 0 \}$  et $y\in E_2\setminus\{ 0\}$
    \begin{equation}
        x*y=\left(\frac{\|x\|}{\delta}\frac{\delta x}{\|x\|}\right)*\left(\frac{\|y\|}{\delta}\frac{\delta y}{\|y\|}\right)
    =\frac{\|x\|\|y\|}{\delta^2} \left(\frac{\delta x}{\|x\|}\right)*\left(\frac{\delta y}{\|y\|}\right).
     \end{equation}
    On remarque que $\delta x/\|x\|_m$ est dans la boule de rayon $\delta$ centrée en $0_m$ et que $\delta y/\|y\|_n$ est dans la boule de rayon $\delta$ centrée en $0_n$. On conclut
    \[
     x*y\leq \frac{\|x\|_m\|y\|_n}{\delta^2}.
    \]
    Il faut prendre $L=1/\delta^2$.
\end{proof}

La norme de \( T\) est alors définie comme la plus petite constante \( L\) qui fait fonctionner la proposition~\ref{PropUADlSMg}.
\begin{definition}  \label{DefKPBYeyG}
	La norme sur l'espace $\aL(E_1\times \cdots\times E_n, F)$ des applications $k$-linéaires et continues est
	\begin{equation}
        \|T\|_{E_1\times \ldots\times E_n}=\sup\{ \|T(u_1, \ldots,u_k)\|_{F}\,\vert\,\|u_i\|_{E_i}\leq 1, i=1,\ldots, k \}.
	\end{equation}
\end{definition}
Nous avons donc automatiquement
\begin{equation}    \label{EqYLnbRbC}
    \| T(u,v) \|\leq \| T \|\| u \|\| v \|.
\end{equation}
Et nous notons que cette norme est uniquement définie pour les applications linéaires continues. Ce n'est pas très grave parce qu'alors nous définissons \( \| T \|=\infty\) si \( T\) n'est pas continue. Cela pour retrouver le principe selon lequel on est continue si et seulement si on est borné.

\begin{proposition}\label{isom_isom}
  On définit les fonctions
  \begin{equation}
    \begin{array}{rccc}
      \omega_g: & \mathcal{L}(\eR^{m}\times\eR^{n}, \eR^p)&\to &\mathcal{L}(\eR^{m}, \mathcal{L}(\eR^{n}, \eR^p)),\\
      \omega_d: & \mathcal{L}(\eR^{m}\times\eR^{n}, \eR^p)&\to &\mathcal{L}(\eR^{n}, \mathcal{L}(\eR^{m}, \eR^p)),
    \end{array}
  \end{equation}
par
\[
\omega_g(T)(x)=T(x,\cdot), \qquad \forall x\in\eR^m,
\]
et
\[
\omega_d(T)(y)=T(\cdot, y), \qquad \forall y\in\eR^n.
\]
Les fonctions $\omega_g$ et $\omega_d$ sont des isomorphismes qui préservent les normes.
\end{proposition}

%+++++++++++++++++++++++++++++++++++++++++++++++++++++++++++++++++++++++++++++++++++++++++++++++++++++++++++++++++++++++++++
\section{Séries}
%+++++++++++++++++++++++++++++++++++++++++++++++++++++++++++++++++++++++++++++++++++++++++++++++++++++++++++++++++++++++++++
\label{SECooYCQBooSZNXhd}

Pour une définition plus générale de somme indexée par un ensemble infine, voir la définition \ref{DefIkoheE}.
\begin{definition}\label{DefGFHAaOL}
    Soit \( (a_k)\) une suite dans un espace vectoriel normé \( (V,\| . \| )\). La suite des \defe{sommes partielles}{somme!partielle} associée est la suite \( (s_k)\) définie par
    \begin{equation}
        s_k=\sum_{i=0}^ka_i
    \end{equation}
    La \defe{série}{série!dans un espace vectoriel normé} associée est la limite des sommes partielles
    \begin{equation}
        \sum_{n=0}^{\infty}a_k=\lim_{k\to \infty} \sum_{k=0}^na_k
    \end{equation}
    si elle existe.

    Si une telle limite existe nous disons que \( \sum_{k=0}^{\infty}a_k\) \defe{converge}{série convergente} dans \( V\). Si la limite de la suite des sommes partielles n'existe pas nous disons que la série \defe{diverge}{série divergente}.
\end{definition}

\begin{remark}
    Si la limite de la suite des sommes partielles n'existe pas dans \( V\), alors elle peut parfois exister dans des extensions de \( V\). Par exemple une série de rationnels convergeant vers \( \sqrt{2}\) dans \( \eR\) ne converge pas dans \( \eQ\). Autre exemple : avec une bonne topologie sur \( \bar \eR\), une série peut ne pas converger dans \( \eR\) mais converger vers \( \pm\infty\) dans \( \bar \eR\).
\end{remark}

Dans le cas des espaces de fonctions, nous avons une norme importante : la norme uniforme définie par \( \| f \|_{\infty}=\sup\{ f(x) \}\) où le supremum est pris sur l'ensemble de définition de \( f\).

\begin{lemma}       \label{LEMooHUZEooSyPipb}
    Soit une suite \( (a_k)\) dans un espace métrique complet\footnote{Définition \ref{DEFooHBAVooKmqerL}.} dont la série converge.
    
    \begin{enumerate}
        \item   \label{ITEMooPFSQooDhKFGL}
            Pour tout \( N\) nous avons
            \begin{equation}
                \sum_{k=0}^{\infty}a_k=\sum_{k=0}^Na_k+\sum_{k=N+1}^{\infty}a_k.
            \end{equation}
        \item       \label{ITEMooQNHMooUPjupB}
            La suite des queues de série converge vers \( 0\), c'est-à-dire que
            \begin{equation}
                \lim_{N\to \infty} \sum_{k=N}^{\infty}a_k=0.
            \end{equation}
    \end{enumerate}
\end{lemma}

\begin{proof}
    Voici un petit calcul :
    \begin{subequations}
        \begin{align}
            \lim_{n\to \infty} \sum_{k=0}^na_k&=\lim_{n\to \infty} \big( \sum_{k=0}^Na_k+\sum_{k=N+1}^{n}a_k \big)      \label{SUBEQooZRSHooSjismK}\\
            &=\lim_{n\to \infty} \sum_{k=0}^{N}a_k+\lim_{n\to \infty} \sum_{k=N+1}^{n}a_k       \label{SUBEQooTLVKooQfYXam}\\
            &=\sum_{k=0}^Na_k+\sum_{k=N+1}^{\infty}a_k.
        \end{align}
    \end{subequations}
    Justifications :
    \begin{itemize}
        \item Pour \eqref{SUBEQooZRSHooSjismK}. Pour chaque \( n\), la somme est finie et nous pouvons la décomposer. Si vous voulez vraiment couper les cheveux en quatre, vous devez fixer un \( \epsilon\), et un \( n\) de telle sorte à avoir \( n>N\), parce que \( N\) est fixé dans l'énoncé du lemme.
        \item Pour \eqref{SUBEQooTLVKooQfYXam}. Nous sommes dans un cas \( \lim_{n\to \infty}(u_n+v_n) \) où \( (u_n)\) est constante et où \( (u_n+v_n)\) converge. Nous pouvons donc permuter limite et somme\footnote{Pour rappel, la proposition \ref{PROPooICZMooGfLdPc} demande la convergence des deux suites pour fonctionner.}.
    \end{itemize}
    Voila que \ref{ITEMooPFSQooDhKFGL} est prouvé.

    Nous écrivons \( s_n=\sum_{k=0}^na_k\); l'hypothèse est que la suite \( (s_n)\) est une suite convergente dans un espace métrique. Elle est donc de Cauchy par la proposition \ref{PROPooZZNWooHghltd}.

    Soit \( \epsilon>0\). Il existe \( N\in \eN\) tel que pour tout \( p,q>N\), nous ayons \( \| s_p-s_q \|\leq \epsilon\). Soit \( p>N\). Pour tout \( n\geq 0\) nous avons
    \begin{equation}
        \epsilon>\| s_{p+n}-s_{p+1} \|=\| \sum_{k=p}^{p+n}a_k \|.
    \end{equation}
    En prenant la limite \( n\to \infty\) nous avons
    \begin{equation}
        \| \sum_{k=p}^{\infty}a_k \|\leq \epsilon.
    \end{equation}
    Nous avons donc démontré qu'il existe \( N\) tel que si \( p>N\), alors \( \| \sum_{k=p}^{\infty}a_k \|\leq \epsilon\). Cela signifie exactement que \( \lim_{n\to \infty} \sum_{k=n}^{\infty}a_k=0\).
\end{proof}

%--------------------------------------------------------------------------------------------------------------------------- 
\subsection{Les trois types de convergence}
%---------------------------------------------------------------------------------------------------------------------------

Trois notions de convergence à ne pas confondre :
\begin{enumerate}
    \item
        La convergence absolue,
    \item
        la convergence normale. C'est la même que la convergence absolue, mais dans le cas particulier d'un espace de fonctions muni de la norme uniforme.
    \item
        la convergence uniforme.
\end{enumerate}
Voici les définitions.


\begin{definition}[Convergence absolue] \label{DefVFUIXwU}
    Nous disons que la série \( \sum_{n=0}^{\infty}a_n\) dans l'espace vectoriel normé \( V\) \defe{converge absolument}{convergence absolue} si la série \( \sum_{n=0}^{\infty}\| a_n \|\) converge dans \( \eR\).
\end{definition}

\begin{definition}[Convergence normale] \label{DefVBrJUxo}
    Une série de fonctions \( \sum_{n\in \eN}u_n \) converge \defe{normalement}{convergence normale} si la série de nombres \( \sum_n\| u_n \|_{\infty}\) converge. C'est-à-dire si la série converge absolument pour la norme \( \| f \|_{\infty}\).
\end{definition}


\begin{definition}[Convergence uniforme]
    La somme \( \sum_nf_n\) \defe{converge uniformément}{convergence uniforme!série de fonctions} vers la fonction \( F\) si la suite des sommes partielles converge uniformément, c'est-à-dire si
    \begin{equation}        \label{EqLNCJooVCTiIw}
        \lim_{N\to \infty} \| \sum_{n=1}^Nf_n-F \|_{\infty}=0.
    \end{equation}
\end{definition}

\begin{proposition} \label{PropAKCusNM}
    Une série convergeant absolument dans un espace de Banach\footnote{Un espace vectoriel normé complet. Typiquement \( \eR\).} y converge au sens usuel.
\end{proposition}

\begin{proof}
    Soit \( (a_k)\) une suite dans un espace vectoriel normé complet dont la série converge absolument. Nous allons montrer que la suite des sommes partielles est de Cauchy. Cela suffira à montrer sa convergence par hypothèse de complétude.

    Nous avons
    \begin{equation}
        \| s_p-s_l \|=\| \sum_{k=l+1}^{p}a_k\|  \leq\sum_{k=l+1}^p\| a_k \|=\bar s_p-\bar s_l
    \end{equation}
    où \( \bar s_n=\sum_{k=0}^n \| a_k \|\) est la suite des sommes partielles de la série des normes (qui converge). Vu que la suite \( (\bar s_n)\) converge dans \( \eR\), elle y est de Cauchy par la proposition~\ref{PROPooTFVOooFoSHPg}. Donc il existe un \( N\) tel que \( p,l>N\) implique
    \begin{equation}
        \| s_p-s_l \|=\bar s_p-\bar s_l\leq \epsilon.
    \end{equation}
    Cela signifie que \( (s_n)\) est une suite de Cauchy et donc convergente.
\end{proof}

\begin{example}[Si l'espace n'est pas complet\cite{MonCerveau}]
    Dans un espace pas complet, il est possible de construire un série qui converge absolument sans converger au sens usuel.

    Nous allons trouver dans \( \eQ\) une série qui converge simplement vers \( \sqrt{ 2 }\) (et donc ne converge pas dans \( \eQ\)) mais absolument vers \( 4\).

    La base est que si \( A,B\in \eQ\) avec \( A<B\) il est possible de résoudre
    \begin{subequations}
        \begin{numcases}{}
            r_1+r_2=A\\
            | r_1 |+| r_2 |=B
        \end{numcases}
    \end{subequations}
    pour \( r_1,r_2\in \eQ\). Ce n'est pas très compliqué : la solution est \( r_1=(A+B)/2\) et \( r_2=(A-B)/2\).

    Nous considérons l'espace \( \eQ\) qui n'est pas complet dans \( \eR\). Soit une série \( (a_k)\) dans \( \eQ\) qui converge vers \( \sqrt{ 2 }\) (convergence dans \( \eR\)) nous nommons \( (s_k)\) la suite des ses sommes partielles. Soit aussi la suite \( (b_k)\) qui converge vers \( 4\) (zéro serait encore plus facile mais bon, juste pour faire un peu de généralité).

    Nous supposons que \( a_k<b_k\) pour tout \( k\) et que les deux suites sont constituées de rationnels positifs. Nous nommons \( (s_k)\) et \( (s'_k)\) les sommes partielles. En particulier \( s_n<s'_n\) et ce sont des suites croissantes.

    Nous savons comment trouver \( r_1,r_2\in \eQ\) tels que \( r_1+r_2=s_1\) et \( | r_1 |+| r_2 |=s'_1\). Par récurrence, si nous savons \( r_1,\ldots, r_k\) tels que
    \begin{subequations}
        \begin{numcases}{}
            r_1+\ldots +r_k=s_n\\
            |r_1|+\ldots +|r_k|=s'_n
        \end{numcases}
    \end{subequations}
    (avec, soit dit en passant \( k=2n\)), alors nous pouvons trouver des rationnels \( r_{k+1}\), \( r_{k+2}\) tels que
    \begin{subequations}
        \begin{numcases}{}
            r_1+\ldots +r_k+r_{k+1}+r_{k+2}=s_{n+1}\\
            |r_1|+\ldots +|r_k|+|r_{k+1}|+|r_{k+2}|=s'_{n+1},
        \end{numcases}
    \end{subequations}
    en effet il s'agit de résoudre
    \begin{subequations}
        \begin{numcases}{}
            r_{k+1}+r_{k+2}=s_{n+1}-r_1-\ldots-r_k=s_{n+1}-s_n>0\\
            | r_{k+1} |+| r_{k+2} |=s'_{n+1}-| r_1 | -\ldots -| r_k |=s'_{n+1}-s'_n>0.
        \end{numcases}
    \end{subequations}
    Cela se résout comme plus haut. Au final nous pouvons construire une suite \( (r_k)\) dans \( \eQ\) telle que
    \begin{equation}
        \sum_{k=0}^{2n}r_k=s_n
    \end{equation}
    et
    \begin{equation}
        \sum_{k=0}^{2n}| r_k |=s'_n.
    \end{equation}
\end{example}

\begin{remark}
    Nous savons que sur les espaces vectoriels de dimension finie toutes les normes sont équivalentes (théorème~\ref{DefEquivNorm}). La notion de convergence de série ne dépend alors pas du choix de la norme. Il n'en est pas de même sur les espaces de dimension infinie. Une série peut converger pour une norme mais pas pour une autre.
\end{remark}
Lorsque nous verrons la convergence de séries, nous verrons que la convergence normale est la convergence absolue pour la norme uniforme.

\begin{lemma}       \label{LemCAIPooPMNbXg}
    Si \( E\) et \( F\) sont des espaces de Banach\quext{Je crois qu'il ne faut pas que \( E\) soit complet.}, l'espace \( \aL(E,F)\) est également de Banach.
\end{lemma}

\begin{proof}
    Soit \( (u_n)\) une suite de Cauchy dans \( \aL(E,F)\); si \( x\in E\) il existe \( N\) tel que si \( l,m>N\) alors \( \| u_l-u_m \|<\epsilon\), c'est-à-dire que pour tout \( \| x \|=1\) on a \( \| u_l(x)-u_n(x) \|<\epsilon\). Cela signifie que \( u_n(x)\) est une suite de Cauchy dans l'espace complet \( F\). Cette suite converge et nous pouvons définir l'application \( u\colon E\to F\) par
    \begin{equation}
        u(x)=\lim_{n\to \infty} u_n(x).
    \end{equation}
    Il suffit maintenant de prouver que \( u\) est linéaire, ce qui est une conséquence directe de la linéarité de la limite :
    \begin{equation}
        u(\alpha x+\beta y)=\lim_{n\to \infty} \big( \alpha u_n(x)+\beta u_n(y) \big).
    \end{equation}
\end{proof}

\begin{proposition}  \label{PROPooYDFUooTGnYQg}
    Si une série converge dans un espace complet, la norme de son terme général converge vers $0$.
\end{proposition}

\begin{proof}
    Soit une suite \( (a_n)\) dont la série converge vers \( s\). Soit \( \epsilon>0\). La suite des sommes partielles \( (s_n)\) est de Cauchy et converge vers \( s\) : \( s_n\to s\). En particulier il existe un \( N\) tel que si \( n>N\), nous avons \( \| s_n-s_{n-1} \|<\epsilon\). Pour de telles valeurs de \( n\) nous avons :
    \begin{equation}
        \| a_n \|=\| s_n-s_{n-1} \|\leq \epsilon.
    \end{equation}
    Cela prouve que \( a_n\to 0\).
\end{proof}

Dans le même ordre d'idée nous avons la convergence des queues de suites.

\begin{lemma}       \label{LEMooFUCOooCOqLRj}
    Si \( \sum_{k=0}^{\infty}a_k\) est finie, alors
    \begin{equation}
        \lim_{n\to \infty} \sum_{k=n}^{\infty}a_k=0.
    \end{equation}
\end{lemma}

\begin{proposition}     \label{PROPooUEBWooUQBQvP}
    Si la série converge alors la somme est associative :
    \( \sum_k (a_k+b_k) = \sum_k a_k + \sum_k b_k \).
\end{proposition}

\begin{proof}
    Associativité. Supposons que \( \sum_ka_k\) et \( \sum_kb_k\) convergent tous deux. Alors nous avons pour tout \( N\) :
    \begin{equation}
        \sum_{k=0}^N(a_k+b_k)=\sum_{k=0}^Na_k+\sum_{k=0}^Nb_k.
    \end{equation}
    Mais si deux limites existent alors la somme commute avec la limite. C'est le cas pour la limite \( N\to \infty\), donc
    \begin{equation}
        \lim_{N\to \infty} \sum_{k=1}^{\infty}(a_k+b_k)=\lim_{N\to \infty} \sum_{k=0}^{\infty}a_k+\lim_{N\to \infty} \sum_{k=0}^{\infty}b_k.
    \end{equation}
\end{proof}

%+++++++++++++++++++++++++++++++++++++++++++++++++++++++++++++++++++++++++++++++++++++++++++++++++++++++++++++++++++++++++++
\section{Série réelle}
%+++++++++++++++++++++++++++++++++++++++++++++++++++++++++++++++++++++++++++++++++++++++++++++++++++++++++++++++++++++++++++
\label{secseries}

La notion de série formalise le concept de somme infinie\footnote{La définition d'une somme infinie est la définition \ref{DefHYgkkA}.}. L'absence de certaines propriétés de ces objets (problèmes de commutativité et même d'associativité) incite à la prudence et montre à quel point une définition précise est importante.


\subsection{Critères de convergence absolue}

Étant donné le terme général d'une série, il est souvent --dans les cas qui nous intéressent-- difficile de déterminer la somme de la série. L'exemple de la série géométrique est particulier\footnote{Voir l'exemple \ref{ExZMhWtJS}.}, puisqu'on connaît une formule pour chaque somme partielle, mais pour l'exemple des séries de Riemann il n'y a aucune formule simple pour un $\alpha$ général. D'où l'intérêt d'avoir des critères de convergence ne nécessitant aucune connaissance de l'éventuelle limite de la série.

\begin{lemma}[Critère de comparaison]   \label{LemgHWyfG}
Soient $\sum_i a_i$ et $\sum_j
b_j$ deux séries à termes positifs vérifiant
\begin{equation*}
  0 \leq a_i \leq b_i
\end{equation*}
alors
\begin{enumerate}
\item si $\sum_i a_i$ diverge, alors $\sum_j b_j$ diverge,
\item si $\sum_j b_j$ converge, alors $\sum_i a_i$ converge
  (absolument).
  \end{enumerate}
\end{lemma}

\begin{proposition}[Critère d'équivalence\cite{TrenchRealAnalisys}]
 Soient $\sum_i a_i$ et $\sum_j b_j$ deux séries à termes positifs. Supposons l'existence de la limite (éventuellement infinie) suivante
\begin{equation}
  \limite i \infty \frac{a_i}{b_i} = \alpha
\end{equation}
avec \( \alpha\in \eR\cup\{ +\infty \}\). Alors
\begin{enumerate}
\item si $\alpha \neq 0$ et $\alpha\neq \infty$, alors
  \begin{equation}
    \sum_i a_i \text{~converge} \ssi \sum_j b_j\text{~converge,}
  \end{equation}
\item si $\alpha = 0$ et $\sum_j b_j$ converge, alors $\sum_i a_i$
  converge (absolument),
\item si $\alpha = +\infty$ et $\sum_j b_j$ diverge, alors $\sum_i
  a_i$ diverge.
\end{enumerate}
\end{proposition}

\begin{proof}
\begin{enumerate}
    \item
        Le fait que la suite $a_n/b_n$ converge vers $\alpha$ signifie que tant sa limite supérieure que sa limite inférieure convergent vers $\alpha$. En particulier la suite $\frac{ a_n }{ b_n }$ est bornée vers le haut et vers le bas. À partir d'un certain rang $N$, il existe $M$ tel que
        \begin{equation}
            \frac{ a_n }{ b_n }<M
        \end{equation}
        et il existe $m$ tel que
        \begin{equation}
            \frac{ a_n }{ b_n }>m.
        \end{equation}
        Nous avons donc $a_n<Mb_n$ et $a_n>mb_n$. La série de $(a_n)$ converge donc si et seulement si la série de $(b_n)$ converge.
    \item
        Si $\alpha=0$, cela signifie que pour tout $\epsilon$, il existe un rang tel que $\frac{ a_n }{ b_n }<\epsilon$, et donc tel que $a_n<\epsilon b_k$. La suite de $(a_i)$ converge donc dès que la suite de $(b_i)$ converge.
    \item
        Pour tout $M$, il existe un rang dans la suite à partir duquel on a $\frac{ a_i }{ b_i }>M$, et donc $a_k>Mb_k$. Si la série de $(b_k)$ diverge, la série de $(a_k)$ doit également diverger.
\end{enumerate}
\end{proof}

\begin{proposition}[Critère du quotient\cite{KeislerElemCalculus}]     \label{PropOXKUooQmAaJX}
    Soit $\sum_i a_i$ une série. Supposons l'existence de la limite (éventuellement infinie) suivante
    \begin{equation}
      \limite i \infty \abs{\frac{a_{i+1}}{a_i}} = L
    \end{equation}
    avec \( L\in \eR\cup\{ +\infty \}\).  Alors
    \begin{enumerate}
    \item si $L < 1$, la série converge absolument,
    \item si $L > 1$, la série diverge,
    \item si $L = 1$ le critère échoue : il existe des exemples de convergence et des exemples de divergence.
    \end{enumerate}
\end{proposition}
\index{critère du quotient}

\begin{proof}
\begin{enumerate}
    \item
        Soit $b$ tel que $L<b<1$. À partir d'un certain rang $K$, on a $\left| \frac{ a_{i+1} }{ a_i } \right| <b$. En particulier,
        \begin{equation}
            | a_{K+1} |<b| a_K |,
        \end{equation}
        et pour $a_{K+2}$ nous avons
        \begin{equation}
            | a_{K+2} |<b| a_{K+1} |<b^2| a_K |.
        \end{equation}
        Au final,
        \begin{equation}
            | a_{K+n} |<b^n| a_K |.
        \end{equation}
        Étant donné que la série $\sum_{n\geq K}b^n$ converge (parce que $b<1$), la queue de suite $\sum_{i\geq K}a_i$ converge, et par conséquent la suite au complet converge.
    \item
        Si $L>1$, on a
        \begin{equation}
            | a_K |<| a_{K+1} |<| a_{K+2} |<\ldots
        \end{equation}
        Il est donc impossible que la suite $(a_i)$ converge vers zéro. La série ne peut donc pas converger.
    \item
        Par exemple la suite harmonique $a_n=\frac{1}{ n }$ vérifie $L=1$, mais la série ne converge pas. Par contre, la suite $a_n=\frac{ 1 }{ n^2 }$ vérifie aussi le critère avec $L=1$ tandis que la série $\sum_n\frac{1}{ n^2 }$ converge.
\end{enumerate}
\end{proof}


\begin{proposition}[Critère de la racine\cite{TrenchRealAnalisys}]
    Soit $\sum_i a_i$ une série, et considérons
    \begin{equation*}
      \limsup_{i \rightarrow \infty} \sqrt[i]{\abs{a_i}} = L
    \end{equation*}
    avec \( L\in \eR\cup\{ +\infty \}\). Alors
    \begin{enumerate}
    \item si $L < 1$, la série converge absolument,
    \item si $L> 1$, la série diverge,
    \item si $L = 1$ le critère échoue.
    \end{enumerate}
\end{proposition}

\begin{proof}
    \begin{enumerate}
        \item
            Si $L<1$, il existe un $r\in \mathopen] 0 , 1 \mathclose[$ tel que $| a_n |^{1/n}<r$ pour les grands $n$. Dans ce cas, $| a_n |<r^{n}$, et la série converge absolument parce que la série $\sum_nr^n$ converge du fait que $r<1$.
        \item
            Si $L>1$, il existe un $r>1$ tel que $| a_n |^{1/n}>r>1$. Cela fait que $| a_n |$ prend des valeurs plus grandes que $n$ pour une infinité de termes. Le terme général $a_n$ ne peut donc pas être une suite convergente. Par conséquent la suite diverge au sens où elle ne converge pas.

    \end{enumerate}
\end{proof}

%---------------------------------------------------------------------------------------------------------------------------
\subsection{Critères de convergence simple}
%---------------------------------------------------------------------------------------------------------------------------

Les critères de comparaison, d'équivalence, du quotient et de la racine sont des critères de convergence absolue. Pour conclure à une convergence simple qui n'est pas une convergence absolue, le critère d'Abel sera notre outil principal.

\subsubsection{Critère d'Abel}

\begin{proposition}[Critère d'Abel]
    Soit la série $\sum_i c_iz_i$ avec
    \begin{enumerate}
        \item $(c_i)$ est une suite réelle décroissante qui tend vers zéro,
        \item $(z_i)$ est une suite dans $\eC$ dont la suite des sommes partielles est bornée dans $\eC$, c'est-à-dire qu'il existe un $M>0$ tel que pour tout $n$,
        \begin{equation}
            \left| \sum_{i=1}^nz_i \right| \leq M.
        \end{equation}
        Alors la série $\sum_ic_iz_i$ est convergente.
    \end{enumerate}
\end{proposition}
Remarquons que ce critère ne donne pas de convergence absolue.

%---------------------------------------------------------------------------------------------------------------------------
\subsection{Quelques séries usuelles}
%---------------------------------------------------------------------------------------------------------------------------
\label{SUBSECooDTYHooZjXXJf}

\begin{example}[Série harmonique]       \label{EXooDVQZooEZGoiG}
    La \defe{série harmonique}{série!harmonique} est
    \begin{equation}
        \sum_{i=k}^\infty \frac{1}{ k }=+\infty.
    \end{equation}
\end{example}

\begin{example}[Série géométrique] \label{ExZMhWtJS}
    La \defe{série géométrique}{série!géométrique} de raison $q \in \eC$ est
    \begin{equation}    \label{EqZQTGooIWEFxL}
        \sum_{i=0}^\infty q^i.
    \end{equation}
    Étudions la somme partielle \( S_N=1+q+q^2+\cdots +q^{N}\). Nous avons évidemment $S_N-qS_N=1-q^{N+1}$ et donc
    \begin{equation}    \label{EqASYTiCK}
        S_N=\sum_{n=0}^Nq^n=\frac{ 1-q^{N+1} }{ 1-q }.
    \end{equation}
    La limite \( \lim_{N\to \infty} S_N\) existe si et seulement si \( | q |\leq 1\) et dans ce cas nous avons
    \begin{equation}    \label{EqRGkBhrX}
        \sum_{n=0}^{\infty}q^n=\frac{ 1 }{ 1-q }.
    \end{equation}
    La convergence est absolue.

    Si la somme commence en \( n=1\) au lieu de \( n=0\) alors
    \begin{equation}        \label{EqPZOWooMdSRvY}
        \sum_{n=1}^{\infty}q^n=\frac{1}{ 1-q }-1=\frac{ q }{ 1-q }.
    \end{equation}
\end{example}

Un cas particulier de la formule \eqref{EqASYTiCK} est le calcul de \( \sum_{j=1}^{N}q^{-j}\) bien utile lorsque l'on joue avec des nombres binaires (voir l'exemple~\ref{EXEMooRHENooGwumoA}). Nous avons
\begin{equation}        \label{EQooFMBAooEJkHWT}
    \sum_{j=1}^Nq^{-j}=\sum_{j=0}^Nq^{-j}-1=\frac{ 1-q^{-N} }{ q-1 }.
\end{equation}

\begin{example}[Série de Riemann]       \label{EXooCTYNooCjYQvJ}
    Pour $\alpha \in \eR$, la \defe{série de Riemann}{série!Riemann}
    \begin{equation}        \label{EqSerRiem}
        \sum_{i=1}^\infty \frac1{i^\alpha}
    \end{equation}
    converge (absolument, puisque réelle et positive) si et seulement si $\alpha > 1$, et diverge sinon.
\end{example}

\begin{example}[Série exponentielle] \label{ExIJMHooOEUKfj}
    La série exponentielle est la série (pour \( t\in \eR\))
    \begin{equation}
        \exp(t)=\sum_{k=0}^{\infty}\frac{ t^k }{ k! }.
    \end{equation}
    Nous montrons qu'elle converge pour tout \( t\in \eR\). Si \( a_k=t^k/k!\) alors \( \frac{ a_{k+1} }{ a_k }=\frac{ t }{ k }\) dont la limite \( k\to \infty\) est zéro (quel que soit \( t\)). En vertu du critère du quotient~\ref{PropOXKUooQmAaJX} la série exponentielle converge (absolument) pour tout \( t\in \eR\).

    Pour tout savoir de l'exponentielle et de ses variations, voir le thème~\ref{THEMEooKXSGooCsQNoY}.
\end{example}
\index{exponentielle!convergence}

\begin{example}[Série arithmético-géométrique\cite{QXuqdoo}]
    Une \defe{suite arithmético-géométrique}{suite!arithmético-géométrique} est une suite vérifiant pour tout \( n\) la relation
    \begin{equation}
        u_{n+1}=au_n+b
    \end{equation}
    avec \( a\) et \( b\) non nuls. Si elle possède une limite, cette dernière doit résoudre \( l=al+b\), et donc être donnée par
    \begin{equation}
        l=\frac{ b }{ 1-a }.
    \end{equation}

    Comportement amusant : la limite peut exister pour certains valeurs de \( a_0\) et pas pour d'autres. Mais elle ne dépend pas de \( a_0\) parmi ceux pour lesquelles la limite existe.

    Il n'est pas très compliqué de trouver le terme général de la suite en fonction de \( a\) et de \( b\). Il suffit de considérer la suite \( v_n=u_n-r\), et de remarquer que cette suite est géométrique :
    \begin{equation}
        v_{n+1}=av_n.
    \end{equation}
    Par conséquent \( v_n=a^nv_0\), ce qui donne pour la suite \( (u_n)\) la formule
    \begin{equation}
        u_n=a^n(u_0-r)+r.
    \end{equation}
\end{example}

\begin{lemma}[\cite{BIBooTIZHooGeFZri}]     \label{LEMooKDHPooPlFTIT}
    Nous avons :
    \begin{equation}
        \sum_{k=1}^N\frac{1}{ k(k+1) }=\frac{ N }{ N+1 }.
    \end{equation}
    et
    \begin{equation}
        \sum_{k=1}^{\infty}\frac{1}{ k(k+1) }=1.
    \end{equation}
\end{lemma}

\begin{proof}
    Nous posons
    \begin{subequations}
        \begin{align}
            f(n)&=\sum_{k=1}^n\frac{1}{ k(k+1) }\\
            g(n)&=\frac{ n }{ n+1 }
        \end{align}
    \end{subequations}
    et nous montrons par récurrence que \( f(n)=g(n)\). Pour \( n=1\) nous avons \( f(1)=g(1)=\frac{ 1 }{2}\).

    Nous supposons que \( f(n)=g(n)\) et nous prouvons que \( f(n+1)=g(n+1)\). Facile :
    \begin{subequations}
        \begin{align}
            f(n+1)&=f(n)+\frac{1}{ (n+1)(n+2) }\\
            &=\frac{ n }{ n+1 }+\frac{1}{ (n+1)(n+2) }\\
            &=\frac{ n(n+2)+1 }{ (n+1)(n+2) }\\
            &=\frac{ n^2+2n+1 }{ (n+1)(n+2) }\\
            &=\frac{ (n+1)^2 }{ (n+1)(n+2) }\\
            &=\frac{ n+1 }{ n+2 }\\
            &=g(n+1).
        \end{align}
    \end{subequations}
    En ce qui concerne la seconde formule, par définition\footnote{Définition d'une série, \ref{DefGFHAaOL}.}
    \begin{equation}
        \sum_{k=1}^{\infty}\frac{1}{ k(k+1) }=\lim_{n\to \infty} \sum_{k=1}^n\frac{1}{ k(k+1) }=\lim_{n\to \infty}\frac{ n }{ n+1 } =1.
    \end{equation}
\end{proof}

%--------------------------------------------------------------------------------------------------------------------------- 
\subsection{Séries alternées}
%---------------------------------------------------------------------------------------------------------------------------

\begin{theorem}[Critère des séries alternées\cite{ooXFPIooCLUvzV}]      \label{THOooOHANooHYfkII} 
    Si \( (a_n)_{n\in \eN}\) est une suite positive décroissante à limite nulle, alors
    \begin{enumerate}
        \item
            Si nous notons \( (S_n)\) la suite des sommes partielles, les sous-suites \( (S_{2n})\) et \( (S_{2n+1})\) sont adjacentes\footnote{Définition \ref{DEFooDMZLooDtNPmu}.}.
        \item
            La série \( \sum_n(-1)^na_n\) converge.
        \item       \label{ITEMooWEPWooXhLMYL}
            Si nous considérons le reste 
            \begin{equation}
                R_n=\sum_{k=n+1}^{\infty}(-1)^ka_k,
            \end{equation}
            nous avons
            \begin{subequations}
                \begin{align}
                    \signe(R_n)=(-1)^{n+1}\\
                    | R_n |\leq a_{n+1}.
                \end{align}
            \end{subequations}
    \end{enumerate}
\end{theorem}

\begin{proof}
    En termes de notations, nous allons écrire \( (S_n)\) la suite des sommes partielles de \( \sum_{k=0}^{\infty}(-1)^ka_k\). Nous notons \( (S_{2n})\) la suite des termes pairs de cette suite. C'est donc la suite \( n\mapsto S_{2n}\).
    Nous divisons en plusieurs morceaux.
    \begin{subproof}
        \item[\( S_{2n}\) est croissante]
            Nous avons simplement
            \begin{equation}
                S_{2n+2}-S_{2n}=a_{2n+2}-a_{2n+1}\leq 0.
            \end{equation}
        \item[\( (S_{2n+1})\) est décroissante]
            Même calcul.
        \item[Les suites \( (S_{2n})\) et \( S_{2n+1}\) sont adjacentes] Nous avons simplement
            \begin{equation}
                S_{2n+1}-S_{2n}=a_{2n+1}\to 0.
            \end{equation}
            Nous concluons par le théorème des suites adjacentes \ref{THOooZJWLooAtGMxD} que les sous-suites des termes pairs et impairs sont convergentes et convergent vers la même limite.
    \end{subproof}
    C'est le moment d'utiliser la proposition \ref{PROPooXOOCooGMqJNe} qui convaincra la lectrice que \( (S_n)\) converge vers la même limite, que nous notons \( S\). Le théorème des suites adjacentes nous dit encore que 
    \begin{equation}
        S_{2n+1}\leq S\leq S_{2n}
    \end{equation}
    et donc que \( R_{2n}=S-S_{2n}\leq 0\). Cela donne la majoration
    \begin{equation}
        | R_{2n} |=| S-S_n |=S_{2n}-S\leq S_{2n}-S_{2n+1}=a_{2n+1}.
    \end{equation}
    Nous faisons le même genre de majorations pour \( R_{2n+1}\).
\end{proof}

%---------------------------------------------------------------------------------------------------------------------------
\subsection{Moyenne de Cesaro}
%---------------------------------------------------------------------------------------------------------------------------

\begin{definition}
    Si \( (a_n)_{n\in \eN} \) est une suite dans \( \eR\) ou \( \eC\), alors sa \defe{moyenne de Cesaro}{moyenne!de Cesaro}\index{Cesaro!moyenne} est la limite (si elle existe) de la suite
    \begin{equation}
        c_n=\frac{1}{ n }\sum_{k=1}^na_k.
    \end{equation}
    En un mot, c'est la limite des moyennes partielles.
\end{definition}

\begin{lemma}       \label{LemyGjMqM}
    Si la suite \( (a_n)\) converge vers la limite \( \ell\) alors la suite admet une moyenne de Cesaro qui vaudra \( \ell\).
\end{lemma}

\begin{proof}
    Soit \( \epsilon>0\) et \( N\in \eN\) tel que \( | a_n-\ell |<\epsilon\) pour tout \( n>N\). En remarquant que
    \begin{equation}
        \frac{1}{ n }\sum_{k=1}^nk-\ell=\frac{1}{ n }\sum_{k=1}^n(a_k-\ell),
    \end{equation}
    nous avons
    \begin{subequations}
        \begin{align}
            | \frac{1}{ n }\sum_{k=1}^na_k-\ell |&\leq| \frac{1}{ n }\sum_{k=1}^N| a_k-\ell | |+\big| \frac{1}{ n }\sum_{k=N+1}^n\underbrace{| a_k-\ell |}_{\leq \epsilon} \big|\\
            &\leq \epsilon+\frac{ n-N-1 }{ n }\epsilon\\
            &\leq 2\epsilon.
        \end{align}
    \end{subequations}
    Dans ce calcul nous avons redéfinit \( N\) de telle sorte que le premier terme soit inférieur à \( \epsilon\).
\end{proof}

%---------------------------------------------------------------------------------------------------------------------------
\subsection{Écriture décimale d'un nombre}
%---------------------------------------------------------------------------------------------------------------------------

\begin{normaltext}      \label{NORMALooTZWYooPMgOIm}
    Soit \( b\geq 2\) un entier qui sera la base dans laquelle nous allons écrire les nombres. Nous considérons l'ensemble \( \eD_b\)\nomenclature[Y]{\( \eD_b\)}{l'ensemble de écritures décimales en base \( b\)} des suites dans \( \{ 0,1,\ldots, b-1 \}\) qui n'ont pas une queue de suite uniquement formée de \( b-1\). Autrement dit une suite \( (c_n)\) est dans \( \eD_b\) lorsque pour tout \( N\), il existe \( k>N\) tel que \( c_k\neq b-1\). Associé à cet ensemble nous considérons la fonction
    \begin{equation}    \label{EqXXXooOTsCK}
        \begin{aligned}
            \varphi_b\colon \eD_b&\to \mathopen[ 0 , 1 [ \\
                c&\mapsto \sum_{n=1}^{\infty}\frac{ c_n }{ b^n }.
        \end{aligned}
    \end{equation}
\end{normaltext}

\begin{lemma}
    La fonction \( \varphi_b\) est bien définie au sens où elle converge et prend ses valeurs dans \( \mathopen[ 0 , 1 [\).
\end{lemma}

\begin{proof}
    Tout se base sur la somme de la série géométrique \eqref{EqRGkBhrX} sous la forme
    \begin{equation}    \label{EqWZGooXJgwl}
        \sum_{k=0}^{\infty}\frac{1}{ b^k }=\frac{ b }{ b-1 }.
    \end{equation}
    La somme \eqref{EqXXXooOTsCK} est donc majorée par \( \sum_n\frac{ b-1 }{ b^n }\) qui converge.

    Pour prouver que l'image de \( \varphi_b\) est bien \( \mathopen[ 0 , 1 [\), nous savons qu'au moins un des \( c_n\) (en fait une infinité) est plus petit que \( b-1\), donc nous avons la majoration stricte\footnote{Notez que la somme \eqref{EqXXXooOTsCK} commence à un tandis que la série géométrique \eqref{EqWZGooXJgwl} commence à zéro.}
        \begin{equation}
            \varphi_b(c)<\sum_{n=1}^{\infty}\frac{ b-1 }{ b^n }=(b-1)\left( \sum_{n=1}^{\infty}\frac{1}{ b^n }-1 \right)=1
        \end{equation}
\end{proof}

Le fait d'introduire l'ensemble \( \eD\) au lieu de l'ensemble de toutes les suites est justifié par la proposition suivante. Elle explique pourquoi un nombre possède au maximum deux écritures décimales distinctes et que ces deux sont obligatoirement de la forme, par exemple en base \( 10\) :
\begin{equation}
    0.34599999999\ldots=0.34600000\ldots
\end{equation}
mais qu'un nombre commençant par \( 0.347\) ne peut pas être égal. C'est pour cela que dans la définition de \( \eD_b\) nous avons exclu les suites qui terminent par tout des \( b-1\).
\begin{proposition} \label{PropSAOoofRlQR}
    Soit la fonction
    \begin{equation}
        \begin{aligned}
            \varphi\colon \{ 0,\ldots, b-1 \}^{\eN}&\to \mathopen[ 0 , 1 [ \\
                x&\mapsto \sum_{n=1}^{\infty}\frac{ x_n }{ b^n }.
        \end{aligned}
    \end{equation}
    Si \( \varphi(x)=\varphi(y)\) et si \( n_0\) est le plus petit entier tel que \( x_{n_0}\neq y_{n_0}\) alors soit
    \begin{equation}
        x_{n_0}-y_{n_0}=1
    \end{equation}
    et \( x_n=0\), \( y_n=b-1\) pour tout \( n>n_0\), soit le contraire : \( y_{n_0}-x_{n_0}=1\) avec \( y_n=0\) et \( x_n=b-1\) pour tout \( n>n_0\).
\end{proposition}

\begin{proof}
    Nous nous basons sur la formule (facilement dérivable depuis \eqref{EqWZGooXJgwl}) suivante :
    \begin{equation}
        \sum_{k=n_0+1}^{\infty}\frac{1}{ b^k }=\frac{1}{ b^{n_0+1} }\frac{ b }{ b-1 }.
    \end{equation}
    Nous avons
    \begin{equation}
        0=\varphi(x)-\varphi(y)=\frac{ x_{n_0}-y_{n_0} }{ b^{n_0} }+\sum_{n=n_0+1}^{\infty}\frac{ x_n-y_n }{ b^n }\geq \frac{ x_{n_0}-y_{n_0} }{ b^{n_0} }-\sum_{n=n_0+1}^{\infty}\frac{ b-1 }{ b^n }=\frac{ x_{n_0}-y_{n_0}-1 }{ b^{n_0} }.
    \end{equation}
    Le dernier terme étant manifestement positif\footnote{C'est ici qu'intervient la subdivision entre le cas \( x_{n_0}-y_{n_0}=1\) ou le contraire. En effet si «ce dernier terme était manifestement \emph{négatif}», il aurait fallu majorer avec de \( 1-b\) au lieu de \( 1-b\).}, il est nul et nous avons \( x_{n_0}-y_{n_0}=1\).

    Nous avons donc maintenant
    \begin{equation}    \label{EqHWQoottPnb}
        0=\varphi(x)-\varphi(y)=\frac{1}{ b^{n_0} }+\sum_{n=n_0+1}^{\infty}\frac{ x_n-y_n }{ b^n }.
    \end{equation}
    Nous majorons la dernière somme de la façon suivante, en supposant que \( | x_n-y_n |\neq b-1\) pour un certain \( n>n_0\) :
    \begin{equation}
        \left| \sum_{n=n_0+1}^{\infty}\frac{ x_n-y_n }{ b^n } \right| \leq\sum_{n=n_0+1}^{\infty}\frac{ | x_n-y_n | }{ b^n }<\sum_{n=n_0+1}^{\infty}\frac{ b-1 }{ b^n }=\frac{1}{ b^{n_0} }.
    \end{equation}
    Étant donné cette inégalité stricte, l'équation \eqref{EqHWQoottPnb} ne peut pas être correcte (valoir zéro). Nous avons donc \( | x_n-b_n |=b-1\) pour tout \( n>n_0\). Donc pour chaque \( n>n_0\) nous avons soit \( x_n=0\) et \( y_n=b-1\), soit \( a_n=b-1\) et \( b_n=0\). Pour conclure il faut encore prouver que le choix doit être le même pour tout \( n\).

    Nous nous mettons dans le cas \( x_{n_0}-y_{n_0}=1\); dans ce cas nous avons bien l'égalité \eqref{EqHWQoottPnb} sans petites nuances de signes. Nous écrivons
    \begin{equation}
        \sum_{n=n_0+1}^{\infty}\frac{ x_n-y_n }{ b^n }=(b-1)\sum_{n=n_0+1}^{\infty}\frac{ (-1)^{s_n} }{ b^n }
    \end{equation}
    où \( s_n\) est pair ou impair suivant que \( x_n=0\), \( y_n=b-1\) ou le contraire. Si un des \( (-1)^{s_n}\) est pas \( -1\) alors nous avons l'inégalité stricte
    \begin{equation}
        (b-1)\sum_{n=n_0+1}^{\infty}\frac{ (-1)^{s_n} }{ b^n }>(b-1)\sum_{n=n_0+1}^{\infty}\frac{-1}{ b^n }=-\frac{1}{ b^{n_0} }.
    \end{equation}
    Dans ce cas il est impossible d'avoir \( \varphi(x)-\varphi(y)=0\). Nous en concluons que \( (-1)^{s_n}\) est toujours \( -1\), c'est-à-dire \( x_n-y_n=1-b\), ce qui laisse comme seule possibilité \( x_n=0\) et \( y_n=b-1\).
\end{proof}

\begin{theorem} \label{ThoRXBootpUpd}
    L'application \( \varphi_b\colon \eD_b\to \mathopen[ 0 , 1 [\) est bijective.
\end{theorem}

\begin{proof}
    En ce qui concerne l'injection, nous savons de la proposition~\ref{PropSAOoofRlQR} que si \( \varphi_b(x)=\varphi_b(y)\) pour \( x,y\in\{ 0,\ldots, b-1 \}^{\eN}\), alors soit \( x\) soit \( y\) a une queue de suite composée uniquement de \( b-1\), ce qui est exclu dans \( \eD_b\). Nous en déduisons que \( \varphi_b\) est bien injective en prenant \( \eD_b\) comme ensemble départ.

    La partie lourde est la surjectivité. Nous prenons \( x\in \mathopen[ 0 , 1 [\) et nous allons construire par récurrence une suite \( a\in \eD_b\) telle que \( \varphi_b(a)=x\). Si il existe \( a_1\in\{ 0,\ldots, b-1 \}\) tel que \( x=a_1/b\) alors nous prenons la suite \( (a_1,0,\ldots, )\) et nous avons évidemment \( \varphi(a)=x\). Sinon il existe \( a_1\in\{ 0,\ldots, b-1 \}\) tel que
        \begin{equation}
            \frac{ a_1 }{ b }<x<\frac{ a_1+1 }{ b }
        \end{equation}
        parce que les autres possibilités pour \( x\) sont dans l'ensemble \( \mathopen[ 0 , 1 \mathclose[\setminus\{ \frac{ k }{ b } \}_{k=0,\ldots, b-1}\) que nous subdivisons en
        \begin{equation}
        \mathopen] 0 , \frac{1}{ b } \mathclose[\cup\mathopen] \frac{1}{ b } , \frac{ 2 }{ b } \mathclose[\cup\ldots\cup\mathopen] \frac{ b-1 }{ b } , 1 \mathclose[.
        \end{equation}
        Pour la récurrence nous supposons avoir trouvé \( a_1,\ldots, a_n\) tels que
        \begin{equation}
            \sum_{k=1}^n\frac{ a_k }{ b^k }< x<\sum_{k=1}^{n-1}\frac{ a_k }{ b^k }+\frac{ a_n+1 }{ b^n }.
        \end{equation}
    Encore une fois s'il existe \( a_{n+1}\in\{ 0,\ldots, b-1 \}\) tel que \( \sum_{k=1}^{n+1}\frac{ a_k }{ b^k }=x\) alors nous prenons ce \( a_{n+1}\) et nous complétons la suite avec des zéros pour avoir \( \varphi(a)=x\). Sinon
%nous subdivisions l'intervalle \( \mathopen]  \frac{ a_n }{ b^n }, \frac{ a_n }{ b^n }+\frac{ a_n+1 }{ b^n } \mathclose[\) (auquel nous retranchons les \( b\) nombres déjà traités) en
 %       \begin{equation}
 %       \mathopen] \frac{ a_n }{ b^n } , \frac{ a_n }{ b^n }+\frac{1}{ b^{n+1} } \mathclose[ \cup \mathopen] \frac{ a_n }{ b^n }+\frac{1}{ b^{n+1} } , \frac{ a_n }{ b^n }+\frac{2}{ b^{n+1} } \mathclose[\cup\ldots\cup\mathopen] \frac{ a_n }{ b^n }+\frac{ b-1 }{ b^{n+1} } , \frac{ a_n }{ b^n }+\frac{ 1 }{ b^n } \mathclose[.
 %       \end{equation}
        , pour simplifier les notations nous notons \( x'=x-\sum_{k=1}^{n}\frac{ a_k }{ b^k }\) et nous avons
        \begin{equation}
            0<x'<\frac{ a_n+1 }{ b^n }.
        \end{equation}
        Le nombre \( x'\) est forcément dans un des intervalles
        \begin{equation}
                \mathopen] \frac{ s }{ b^{n+1} } , \frac{ s+1 }{ b^{n+1} } \mathclose[
        \end{equation}
        avec \( s\in\{ 0,\ldots, b-1 \}\). Nous prenons le \( s\) correspondant à \( x'\) comme \( a_{n+1}\). Dans ce cas nous avons
        \begin{equation}
            \sum_{k=1}^{n+1}\frac{ a_k }{ b^k }< x<\sum_{k=1}^{n+1}\frac{ a_k }{ b^k }+\frac{1}{ b^{n+1} }.
        \end{equation}
        Note : les deux inégalités sont strictes. La première parce que s'il y avait égalité, nous nous serions déjà arrêté en complétant avec des zéros. La seconde parce que
        \begin{equation}
            \sum_{k=n+2}^{\infty}\frac{ a_k }{ b^k }\leq \sum_{k=n+2}^{\infty}\frac{ b-1 }{ b^k }=\frac{1}{ b^{n+1} }
        \end{equation}
        où l'égalité n'est possible que si \( a_k=b-1\) pour tout \( k\geq n+2\). Dans ce cas nous aurions eu
        \begin{equation}
            x=\sum_{k=1}^{n}\frac{ a_k }{ b^k }+\frac{ a_{n+1}+1 }{ b^{n+1} }
        \end{equation}
        et nous aurions choisi le nombre \( a_{n+1}\) autrement et complété la suite par des zéros à partir de là. Notons que cela prouve au passage que la suite que nous sommes en train de construire est bien dans \( \eD_b\) parce qu'elle ne contiendra pas de queue de suite composée de \( b-1\).

        Ceci termine la construction par récurrence de la suite \( a\in \eD_b\). Par construction nous avons pour tout \( N\geq 1\),
        \begin{equation}
            \sum_{k=1}^N\frac{ a_k }{ b^k }\leq x\leq \sum_{k=1}^N\frac{ a_k }{ b^k }+\frac{1}{ b^{N+1} },
        \end{equation}
        autrement dit : \( \varphi_b(a_1,\ldots, a_N)\in B(x,\frac{1}{ b^{N+1} })\). Nous avons donc bien convergence
        \begin{equation}
            \lim_{N\to \infty} \varphi_b(a_1,\ldots, a_N)=x
        \end{equation}
        et l'application \( \varphi_b\) est surjective.
\end{proof}

L'application \( \varphi_b^{-1}\colon \mathopen[ 0 , 1 [\to \eD_b\) est la \defe{décomposition décimale}{décomposition décimale} en base \( b\) des nombres de \( \mathopen[ 0 , 1 [\).

Tout cela nous permet de montrer entre autres que \( \eR\) n'est pas dénombrable. Vu qu'il y a une bijection entre \( \mathopen[ 0 , 1 [\) et \( \eD_b\), il suffit de prouver que \( \eD_b\) est non dénombrable. De plus il suffit de démontrer que \( \eD_b\) est non dénombrable pour un entier \( b\geq 2\) donné.

\begin{proposition}[\cite{KZIoofzFLV}]  \label{PropNNHooYTVFw}
    Il n'existe pas de surjection \( \eN\to \eD_b\). Autrement dit \( \eD_b\) est non dénombrable.
\end{proposition}

\begin{proof}
    Nous prenons \( b\neq 2\) pour des raisons qui seront claires plus tard. Soit \( f\colon \eN\to \eD_b\). Pour \( i\in \eN\) nous notons
    \begin{equation}
        f(n)=(c_i^{(n)})_{i\geq 1},
    \end{equation}
    et nous définissons la suite
    \begin{equation}
        c_k=\begin{cases}
            0    &   \text{si } c_k^{(k)}\neq 0\\
            1    &    \text{si } c_k^{(k)}=0.
        \end{cases}
    \end{equation}
    Cela est une suite dans \( \eD_b\) parce que \( b\neq 2\) et que la suite ne contient que des \( 0\) et des \( 1\). Mais nous n'avons \( f(n)=c\) pour aucun \( n\in \eN\) parce que nous avons \( c_n\neq f(n)_n\).

    Si \( b=2\) alors nous savons que \( \eD_2\sim\mathopen[ 0 , 1 [\sim \eD_3\). Donc \( \eD_2\sim \eD_3\) et \( \eD_2\) ne peut pas plus être mis en bijection avec \( \eN\) que \( \eD_3\).
\end{proof}

\begin{remark}
    Le cas de la base \( b=2\) doit être fait à part parce que rien n'empêche d'avoir une queue de \( 1\). Il y a alors toutefois moyen de se débrouiller en construisant la suite \( c\) de façon plus subtile. Si \( b=2\) et \( n\in \eN\) alors \( f(n)\) est une suite de \( 0\) et \( 1\) contenant une infinité de \( 0\) (parce qu'il n'y a pas de queue de suite ne contenant que des \( 1\)). Nous construisons alors \( c\) de la façon suivante : d'abord nous recopions \( f(0)\) jusqu'à son \emph{deuxième} zéro que nous changeons en \( 1\); nommons \( n_0\) le rang de ce deuxième zéro. Ensuite nous recopions les éléments de \( f(1) \) à partir du rang \( n_0+1\) jusqu'au second zéro que nous changeons en \( 1\), etc.

    Le fait de prendre le deuxième zéro nous garantit que la suite \( c\) n'aura pas de queue de suite ne contenant que des \( 1\).

    Notons que cette construction s'adapte à tout \( b\); il suffit de prendre le second terme qui n'est pas \( b-1\) et le remplacer par \( b-1\).
\end{remark}

\begin{corollary}
    L'ensemble \( \mathopen[ 0 , 1 [\) n'est pas dénombrable.
\end{corollary}

\begin{proof}
    L'ensemble \( \mathopen[ 0 , 1 [\) est en bijection avec \( \eD_b\) que nous venons de prouver n'être pas dénombrable.
\end{proof}

%--------------------------------------------------------------------------------------------------------------------------- 
\subsection{Théorème de Banach-Steinhaus}
%---------------------------------------------------------------------------------------------------------------------------

\begin{lemma}[\cite{BIBooZUTUooNMvrdQ}]     \label{LEMooPIPLooMppGSO}
    Soient des espaces vectoriels normés \( X\) et \( Y\) ainsi qu'une application linéaire bornée \( T\colon X\to Y\). Pour tout \( a\in X\) et pour tout \( r>0\) nous avons
    \begin{equation}
        \sup_{x\in B(a,r)}\| Tx \|\geq r\| T \|
    \end{equation}
\end{lemma}

\begin{proof}
    Nous commençons avec \( a=0\). En utilisant la définition \ref{DefNFYUooBZCPTr} de la norme opérateur,
    \begin{equation}
        \| T \|=\sup_{x\in X}\frac{ \| Tx \| }{ \| x \| }=\sup_{x\in B(0,r)}\frac{ \| Tx \| }{ \| x \| }\leq \frac{1}{ r }\sup_{x\in B(0,r)}\| Tx \|.
    \end{equation}
    Donc
    \begin{equation}
        \sup_{x\in B(0,r)}\| Tx \|\geq r\| T \|.
    \end{equation}
    
    Il y a maintenant une astuce. Nous considérons un maximum :
    \begin{subequations}
        \begin{align}
            \max\{ \| T(a+x),\| T(a-x) \| \| \}&\geq \frac{ 1 }{2}\big( \| T(a+x) \|+\| T(a-x) \| \big) \label{SUBEQooPJPMooDkqRHs}\\
            &\geq \frac{ 1 }{2}\big( \| T(a+x)-T(a+x) \| \big)      \label{SUBEQooEZUUooVlKtfn}\\
            &=\frac{ 1 }{2}\| T(2x) \|\\
            &=\| Tx \|.
        \end{align}
    \end{subequations}
    Justifications :
    \begin{itemize}
        \item Pour \eqref{SUBEQooPJPMooDkqRHs}, la moyenne est plus petite que le maximum.
        \item Pour \eqref{SUBEQooEZUUooVlKtfn}, inégalité triangulaire : \( \| \alpha-\beta \|\leq \| \alpha \|+\| \beta \|\).
    \end{itemize}
    Si maintenant \( y\in B(a,r)\), nous avons \( y=a+x\) pour un certain \( x\in B(0,r)\), donc
    \begin{subequations}
        \begin{align}
            \sup_{y\in B(a,r)}\| Ty \|&=\sup_{x\in B(0,r)}\| T(a+x) \|\\
            &=\sup_{x\in B(0,r)}\max\{ \| T(a+x) \|, \| T(a-x) \| \}        \label{SUBEQooACJSooTHCAWs}\\
            &\geq \sup_{x\in B(0,r)}\| Tx \|\\
            &\geq r\| T \|.
        \end{align}
    \end{subequations}
    Pour \eqref{SUBEQooACJSooTHCAWs}, l'ensemble sur lequel nous prenons le supremum n'est pas modifié fondamentalement si nous regroupons les éléments deux à deux en prenant le maximum : les éléments exclus sont majorés.
\end{proof}

\begin{theorem}[Théorème de Banach-Steinhaus\cite{BIBooZUTUooNMvrdQ}]       \label{THOooJHVNooIDDxyT}
    Soient un espace de Banach\footnote{Définition \ref{DefVKuyYpQ}.} \( X\) et un espace vectoriel normé \( Y\). Soit une famille \( \mF\) d'opérateurs linéaire bornés. Si pour tout \( x\in  X\),
    \begin{equation}
        \sup_{T\in\mF}\| Tx \|<\infty,
    \end{equation}
    alors 
    \begin{equation}
        \sup_{T\in \mF}\| T \|<\infty.
    \end{equation}
\end{theorem}

\begin{proof}
    Nous supposons que \( \sup_{T\in\mF}\| T \|=\infty\), de telle sorte que nous pouvons choisir une suite \( (T_n)\) dans \( \mF\) telle que \( \| T_n \|\to \infty\). Cette suite peut diverger arbitrairement vite, et nous fixerons exactement cela plus tard.

    Soit par ailleurs une suite \( \alpha_n>0\) d'éléments petits et tels que \( \alpha_n\to 0\). Nous supposons que \( \sum_{n=0}^{\infty}\alpha_n<\infty\).

    Si \( a\in X\), le lemme \ref{LEMooPIPLooMppGSO} dit que
    \begin{equation}
        \sup_{x\in B(a,\alpha_n)}\| T_nx \|\geq \| T_n \|\alpha_n.
    \end{equation}
    En posant \( x_0=0\), nous construisons une suite \( (x_n)\) par récurrence en imposant
    \begin{enumerate}
        \item
            \( x_n\in B(x_{n=1}, \alpha_n)\)
        \item
            \( \| T_nx_n \|\geq \| T_n \|\alpha_n\).
    \end{enumerate}
    En utilisant une série télescopique et l'inégalité triangulaire \( \| x_k-x_{k+1} \|\leq \alpha_n\) à chaque étage,
    \begin{equation}
        \| x_p-x_q \|\leq \sum_{k=p}^q\alpha_k\leq \sum_{k=p}^{\infty}\alpha_k.
    \end{equation}
    Mais vu que la somme des \( \alpha_n\) converge, la suite des queues de somme converge vers zéro\footnote{Lemme \ref{LEMooHUZEooSyPipb}\ref{ITEMooQNHMooUPjupB}.} : \( \lim_{p\to \infty}\sum_{k=p}^{\infty}\alpha_n=0\). Cela implique que \( (x_n)\) est une suite de Cauchy\footnote{Proposition \ref{PROPooZZNWooHghltd}.}. Vu que \( X\) est de Banach, la suite \( (x_n)\) a une limite dans \( X\). Soit \( x\) cette limite.

    Nous avons \( \beta_n=\| x_n-x \|\to 0\). Il y aurait moyen de calculer \( \beta_n\) en fonction de \( \alpha_n\) (surtout si nous avions donné une forme explicite à \( \alpha_n\)), mais c'est sans importance ici. L'important est que c'est une suite qui tend vers zéro.

    Nous avons
    \begin{equation}
        x\in B(x_n,\beta_n),
    \end{equation}
    et donc il existe \( a_n\in B(0,\beta_n)\) tel que \( x=x_n+a_n\). Avec cela, pour chaque \( n\) nous avons :
    \begin{subequations}
        \begin{align}
            \| T_nx \|&=\| T_n(x_n+a_n) \|\\
            &\geq\| T_nx_n \|-\| T_na_n \|\\
            &\geq \| T_nx_n \|-\| T_n \|\beta_n     \label{SUBEQooPLVQooChVCLU}\\
        &\geq \| T_n \|\alpha_n-\| T_n \|\beta_n\\
        &=\| T_n \|(\alpha_n-\beta_n).
        \end{align}
    \end{subequations}
    Pour \ref{SUBEQooPLVQooChVCLU}, nous avons utilisé \( \| T_na_n \|\leq \| T_n \|\beta_n\). En résumé,
    \begin{equation}
        \| T_nx \|\geq \| T_n \|(\alpha_n-\beta_n).
    \end{equation}
    Il suffit de choisir \( \| T_n \|\) suffisamment rapidement croissant pour que\footnote{Le point important ici est que \( \alpha_n\) (et donc \( \beta_n\)) est choisi sans référence à \( \| T_n \|\).}
    \begin{equation}
       \| T_n \|(\alpha_n-\beta_n)\to \infty,
    \end{equation}
    et nous avons \( \| T_nx \|\to \infty\), qui est contraire aux hypothèses.
\end{proof}

\begin{theorem}[Théorème de Banach-Steinhaus\cite{KXjFWKA,VPvwAaQ}] \label{ThoPFBMHBN}
    Soit \( E\) un espace de Banach\footnote{Définition~\ref{DefVKuyYpQ}.} et \( F\) un espace vectoriel normé. Nous considérons une partie \( H\subset \aL_c(E,F)\) (espace des fonctions linéaires continues). Alors \( H\) est uniformément borné si et seulement s'il est simplement borné.
\end{theorem}
\index{théorème!Banach-Steinhaus}
\index{application!linéaire!théorème de Banach-Steinhaus}

\begin{proof}
    Si \( H\) est uniformément borné, il est borné; pas besoin de rester longtemps sur ce sens de l'équivalence. Supposons donc que \( H\) soit borné. Pour chaque \( k\in \eN^*\) nous considérons l'ensemble
    \begin{equation}
        \Omega_k=\{ x\in E\tq \sup_{f\in H}\| f(x) \|>k \}.
    \end{equation}

    \begin{subproof}
        \item[Les \( \Omega_k\) sont ouverts]

            Soit \( x_0\in \Omega_k\); nous avons alors une fonction \( f\in H\) telle que \(  \| f(x_0) \|>k \), et par continuité de \( f\) il existe \( \rho>0\) tel que \( \| f(x) \|>k\) pour tout \( x\in B(x_0,\rho)\). Par conséquent \( B(x_0,\rho)\subset \Omega_k\) et \( \Omega_k\) est ouvert par le théorème~\ref{ThoPartieOUvpartouv}.

        \item[Les \( \Omega_k\) ne sont pas tous denses dans \( E\)]

            Nous supposons que les ensembles \( \Omega_k\) soient tous dense dans \( E\). Le théorème de Baire~\ref{ThoBBIljNM} nous indique que \( E\) est un espace de Baire (parce que de Banach) et donc que
            \begin{equation}
                \overline{ \bigcap_{k\in \eN}\Omega_k }=E.
            \end{equation}
            En particulier l'intersection des \( \Omega_k\) n'est pas vide. Soit \( x_0\in \bigcap_{k\in \eN}\Omega_k\). Nous avons alors
            \begin{equation}
                \sup_{f\in H}\| f(x) \|=\infty,
            \end{equation}
            ce qui est contraire à l'hypothèse. Donc les ouverts \( \Omega_k\) ne sont pas tous denses dans \(E\).

        \item[La majoration]

            Il existe \( k\geq 0\) tel que \( \Omega_k\) ne soit pas dense dans \( E\), et nous voulons prouver que \( \{ \| f \|\tq f\in H \}\) est un ensemble borné. Soit donc \( k\geq 0\) tel que \( \Omega_k\) ne soit pas dense dans \( E\); il existe un \( x_0\in E\) et \( \rho>0\) tels que
            \begin{equation}
                B(x_0,\rho)\cap \Omega_k=\emptyset.
            \end{equation}
            Si \( x\in B(x_0,\rho)\) alors \( x\) n'est pas dans \( \Omega_k\) et donc
            \begin{equation}
                \sup_{f\in H}\| f(x) \|\leq k.
            \end{equation}
            Afin d'évaluer \( \| f \|\) nous devons savoir ce qu'il se passe avec les vecteurs sur une boule autour de \( 0\). Pour tout \( x\in B(0,\rho)\) et pour tout \( f\in H\), la linéarité de \( f\) donne
            \begin{equation}
                \| f(x) \|=\| f(x+x_0)-f(x_0) \|\leq \| f(x+x_0)+f(x_0) \|\leq 2k.
            \end{equation}
            Par continuité nous avons alors \( \| f(x) \|\leq 2k\) pour tout \( x\in \overline{ B(0,\rho) }\). Si maintenant \( x\in F\) vérifie \( \| x \|=1\) nous avons
            \begin{equation}
                \| f(x) \|=\frac{1}{ \rho }\| f(\rho x) \|\leq \frac{ 2k }{ \rho },
            \end{equation}
            et donc \( \| f \|\leq \frac{ 2k }{ \rho }\), ce qui montre que \( 2k/\rho\) est un majorant de l'ensemble \( \{ \| f \|\tq f\in H \}\).

    \end{subproof}

\end{proof}
Une application du théorème de Banach-Steinhaus est l'existence de fonctions continues et périodiques dont la série de Fourier ne converge pas. Ce sera l'objet de la proposition~\ref{PropREkHdol}.

%--------------------------------------------------------------------------------------------------------------------------- 
\subsection{Convergence forte}
%---------------------------------------------------------------------------------------------------------------------------

Lorsque nous avons une suite d'opérateurs linéaires, nous pouvons considérer la convergence d'une suite pour la norme opérateur : \( A_k\to A\) lorsque \( \| A_k-A \|\to 0\).

\begin{definition}[\cite{ooAGRZooTyUUVy}]       \label{DEFooNREQooElLvec}
    Soient un espace vectoriel \( E\) et un espace vectoriel normé \( V\). Nous disons que la suite d'opérateur \( T_k\colon E\to V\) \defe{converge fortement}{convergence forte} vers l'opérateur $T$ si pour tout \( x\in E\) nous avons
    \begin{equation}
        \| T_kx-Tx \|\to 0.
    \end{equation}
\end{definition}

Cette notion s'appelle \emph{forte} par opposition à la convergence \emph{faible} dont nous ne parlerons pas. Elle est cependant moins forte que la convergence en norme dont nous avons déjà parlé.

\begin{proposition}     \label{PROPooRFBLooUjSirP}
    Soient des espaces vectoriels normés \( E\) et \( F\) et une suite d'opérateurs \( T_k\colon E\to F\) convergeant vers \( T\)\footnote{Sans précisions, ce sera toujours la convergence en norme.}. Alors cette suite converge également fortement.
\end{proposition}

\begin{proof}
    Soit \( x\in E\) que nous supposons non nul. Soit \( \lambda\in \eC\) tel que \( x=\lambda y\) avec \( \| y \|=1\). Nous avons
    \begin{equation}
        \| T_kx-Tx \|=| \lambda |\| T_ky-Ty \|\leq | \lambda |\sup_{\| z \|=1}\| T_kz-Tz \|=| \lambda |\| T_k-T \|\to 0.
    \end{equation}
    La dernière étape est la convergence en norme \( T_k\to T\).
\end{proof}

\begin{proposition}
    Soient \( E\) et \( F\), des espaces vectoriels normés de dimension finie. Soit une suite \( (A_n)\) d'applications linéaires \( E\to F\). Si elle converge fortement vers \( A\), alors elle converge en norme vers \( A\).
\end{proposition}

\begin{proof}
    En plusieurs coups.
    \begin{subproof}
        \item[Si une sous-suite converge]
            Commençons par montrer que si \( (B_n)\) est une sous-suite de \( (A_n)\) qui converge vers \( B\), alors \( B=A\). Autrement dit, \( A\) est le seul candidat limite pour \( A_n\).

            Soit \( \| x \|=1\). Nous avons
            \begin{equation}
                \| B_nx-Bx \|\leq \| B_n-B \|\| x \|=\| B_n-B \|,
            \end{equation}
            mais pour la sous-suite \( (B_n)\) nous avons supposé \( \| B_n-B \|\to 0\). Donc \( \| B_nx-Bx \|\to 0\), ce qui signifie que \( B_nx\to Bx\). Mais par hypothèse, \( B_nx\to Ax\). Par unicité de la limite, \( Bx=Ax\) pour tout \( x\) de norme \( 1\). Pour les autres \( x\), c'est la linéarité qui conclu.

        \item[Utilisation de deux gros résultats]
        Par l'hypothèse de convergence, pour chaque \( x\) nous avons \( \sup_n\| A_nx \|<\infty\). Le théorème de Banach-Steinhaus \ref{THOooJHVNooIDDxyT} nous indique alors que l'ensemble \( \mF=\{ A_n \}_{n\in \eN}\) est borné. Il existe donc \( M > 0\) tel que \( \| A_n \|< M\) pour tout \( n\).

        Nous utilisons à présent l'hypothèse de dimension finie en disant que l'espace des applications linéaires \( E\to F\) est de dimension finie, de telle sorte que ses boules fermées soient compactes.

        Donc la suite \( (A_n)\) est contenue dans un compact.
        
        \item[Les sous-suite convergentes]

            La suite \( (A_n)\) est contenue dans un compact. Toutes ses sous-suites sont dans ce compact et possèdent donc une sous-suite convergente (théorème \ref{ThoBWFTXAZNH}). Toutes ces sous-sous-suites convergent nécessairement vers \( A\) par ce que nous avons dit dans la première étape de la preuve. Le lemme \ref{LEMooSJKMooKSiEGq} nous dit alors que \( A_n\to A\).
    \end{subproof}
\end{proof}

% This is part of Le Frido
% Copyright (c) 2008-2020
%   Laurent Claessens
% See the file fdl-1.3.txt for copying conditions.


%+++++++++++++++++++++++++++++++++++++++++++++++++++++++++++++++++++++++++++++++++++++++++++++++++++++++++++++++++++++++++++
\section{Sommes de familles infinies}
%+++++++++++++++++++++++++++++++++++++++++++++++++++++++++++++++++++++++++++++++++++++++++++++++++++++++++++++++++++++++++++
\label{SECooHHDXooUgLhHR}

%---------------------------------------------------------------------------------------------------------------------------
\subsection{Convergence commutative}
%---------------------------------------------------------------------------------------------------------------------------

\begin{definition}
    Soit \( x_k\) une suite dans un espace vectoriel normé \( E\). Nous disons que la suite \defe{converge commutativement}{convergence!commutative} vers \( x\in E\) si \( \lim_{n\to \infty}\| x_n-x \| =0\) et si pour toute bijection \( \tau\colon \eN\to \eN\) nous avons aussi
    \begin{equation}
        \lim_{n\to \infty} \| x_{\tau(k)}-x \|=0.
    \end{equation}
    La notion de convergence commutative est surtout intéressante pour les séries. La somme
    \begin{equation}
        \sum_{k=0}^{\infty}x_k
    \end{equation}
    converge commutativement vers \( x\) si \( \lim_{N\to \infty} \| x-\sum_{k=0}^Nx_k \|=0\) et si pour toute bijection \( \tau\colon \eN\to \eN\) nous avons
    \begin{equation}
        \lim_{N\to \infty} \| x-\sum_{k=0}^Nx_{\tau(k)} \|=0.
    \end{equation}
\end{definition}

Nous démontrons maintenant qu'une série converge réelle commutativement si et seulement si elle converge absolument.

\begin{proposition} \label{PopriXWvIY}
    Soit \( (a_i)_{i\in \eN}\) une suite absolument convergente\footnote{Définition \ref{DefVFUIXwU}.} dans \( \eC\). Alors elle converge commutativement.
\end{proposition}

\begin{proof}
    Soit \( \epsilon>0\). Nous posons \( \sum_{i=0}^\infty a_i=a\) et nous considérons \( N\) tel que
    \begin{equation}
        | \sum_{i=0}^Na_i-a |<\epsilon.
    \end{equation}
    Étant donné que la série des \( | a_i |\) converge, il existe \( N_1\) tel que pour tout \( p,q>N_1\) nous ayons \( \sum_{i=p}^q| a_i |<\epsilon\). Nous considérons maintenant une bijection \( \tau\colon \eN\to \eN \). Prouvons que la série \( \sum_{i=0}^{\infty}| a_{\tau(i)} |\) converge. Nous choisissons \( M\) de telle sorte que pour tout \( n>M\), \( \tau(n)>N_1\). Si \( s_k\) est la somme partielle de la suite \( ( a_{\tau(i)} )_{i\in \eN}\) et si \( M<p<q \) nous avons
    \begin{equation}
        | s_q-s_p |= | \sum_{i=p}^q a_{\tau(i)} | \leq \sum_{i=p}^q| a_{\tau(i)} |<\epsilon.
    \end{equation}
    Cela montre que \( (s_k)\) est une suite de Cauchy. Elle est alors convergente et nous en déduisons que la série
    \begin{equation}
        \sum_{i=0}^{\infty}a_{\tau(i)}
    \end{equation}
    converge. Nous devons montrer à présent qu'elle converge vers la même limite que la somme «usuelle» \( \lim_{N\to \infty} \sum_{i=0}^Na_i\).

    Soit \( n>\max\{ M,N \}\). Alors
    \begin{equation}
        \sum_{k=0}^na_{\tau(k)}-\sum_{k=0}^na_k=\sum_{k=0}^Ma_{\tau(k)}-\sum_{k=0}^Na_k+\underbrace{\sum_{M+1}^na_{\tau(k)}}_{<\epsilon}-\underbrace{\sum_{k=N+1}^na_k}_{<\epsilon}.
    \end{equation}
    Par construction les deux derniers termes sont plus petits que \( \epsilon\) parce que \( M\) et \( N\) sont les constantes de Cauchy pour les séries \( \sum a_{\tau(i)}\) et \( \sum a_i\). Afin de traiter les deux premiers termes, quitte à redéfinir \( M\), nous supposons que \( \{ 1,\ldots, N \}\subset \tau\{ 1,\ldots, M \}\); par conséquent tous les \( a_i\) avec \( i<N\) sont atteints par les \( a_{\tau(i)}\) avec \( i<M\). Dans ce cas, les termes qui restent dans la différence
    \begin{equation}
        \sum_{k=0}a_{\tau(k)}-\sum_{k=0}^Na_k
    \end{equation}
    sont des \( a_k\) avec \( k>N\). Cette différence est donc en valeur absolue plus petite que \( \epsilon\), et nous avons en fin de compte que
    \begin{equation}
        \left| \sum_{k=0}^na_{\tau(k)}-\sum_{k=0}^na_k \right| <\epsilon.
    \end{equation}
\end{proof}

\begin{proposition}     \label{PropyFJXpr}
    Soit \( \sum_{k=0}^{\infty}a_k\) une série réelle qui converge mais qui ne converge pas absolument. Alors pour tout \( b\in \eR\), il existe une bijection \( \tau\colon \eN\to \eN\) telle que \( \sum_{i=0}^{\infty}a_{\tau(i)}=b\).
\end{proposition}
Pour une preuve, voir \href{http://gilles.dubois10.free.fr/analyse_reelle/seriescomconv.html}{chez Gilles Dubois}.

Les propositions~\ref{PopriXWvIY} et~\ref{PropyFJXpr} disent entre autres qu'une série dans \( \eC\) est commutativement sommable si et seulement si elle est absolument sommable.

Soit \( (a_i)_{i\in I}\) une famille de nombres complexes indexée par un ensemble \( I\) quelconque. Nous allons nous intéresser à la somme \( \sum_{i\in I}a_i\).


Soit \( \{ a_i \}_{i\in I}\) des nombres positifs. Nous définissons la somme
\begin{equation}
    \sum_{i\in I}a_i=\sup_{ J\text{ fini}}\sum_{j\in J}a_j.
\end{equation}
Notons que cela est une définition qui ne fonctionne bien que pour les sommes de nombres positifs. Si \( a_i=(-1)^i\), alors selon la définition nous aurions \( \sum_i(-1)^i=\infty\). Nous ne voulons évidemment pas un tel résultat.

Dans le cas de familles de nombres réels positifs, nous avons une première définition de la somme.
\begin{definition}  \label{DefHYgkkA}
Soit \( (a_i)_{i\in I}\) une famille de nombres réels positifs indexés par un ensemble quelconque \( I\). Nous définissons
\begin{equation}
    \sum_{i\in I}a_i=\sup_{ J\text{ fini dans } I}\sum_{j\in J}a_j.
\end{equation}
\end{definition}

\begin{definition}  \label{DefIkoheE}
    Si \( \{ v_i \}_{i\in I}\) est une famille de vecteurs dans un espace vectoriel normé indexée par un ensemble quelconque \( I\). Nous disons que cette famille est \defe{sommable}{famille!sommable} de somme \( v\) si pour tout \( \epsilon>0\), il existe un \( J_0\) fini dans \( I\) tel que pour tout ensemble fini \( K\) tel que \( J_0\subset K\) nous avons
    \begin{equation}
        \| \sum_{j\in K}v_j-v \|<\epsilon.
    \end{equation}
\end{definition}
Notons que cette définition implique la convergence commutative.

\begin{example}
    La suite \( a_i=(-1)^i\) n'est pas sommable parce que quel que soit \( J_0\) fini dans \( \eN\), nous pouvons trouver \( J\) fini contenant \( J_0\) tel que \( \sum_{j\in J}(-1)^j>10\). Pour cela il suffit d'ajouter à \( J_0\) suffisamment de termes pairs. De la même façon en ajoutant des termes impairs, on peut obtenir \( \sum_{j\in J'}(-1)^i<-10\).
\end{example}

\begin{example}
    De temps en temps, la somme peut sortir d'un espace. Si nous considérons l'espace des polynômes \( \mathopen[ 0 , 1 \mathclose]\to \eR\) muni de la norme uniforme, la somme de l'ensemble
    \begin{equation}
        \{ 1,-1,\pm\frac{ x^n }{ n! } \}_{n\in \eN}
    \end{equation}
    est zéro.

    Par contre la somme de l'ensemble \( \{ 1,\frac{ x^n }{ n! } \}_{n\in \eN}\) est l'exponentielle qui n'est pas un polynôme.
\end{example}

%--------------------------------------------------------------------------------------------------------------------------- 
\subsection{Somme non dénombrables}
%---------------------------------------------------------------------------------------------------------------------------

Nous allons voir que les sommes non dénombrables ne sont pas intéressantes : si le nombre de valeurs non nulles parmi les \( (x_i)_{i\in I}\) est non dénombrable, alors la somme est infinie. La bonne généralisation de somme infinie dans le cas non dénombrable est l'intégrale qui viendra seulement avec la définition \ref{DefTVOooleEst} et la mesure de Lebesgue \ref{DefooYZSQooSOcyYN}.

\begin{lemma}       \label{LEMooYJCVooHajEbg}
    Si \( A\) est non dénombrable dans \( \eR\), alors il existe \( \delta>0\) tel que \( A\cap \{ | x |\geq \delta \}\) est non dénombrable.
\end{lemma}

\begin{proof}
    Nous y allons par l'absurde, et nous supposons que \( A\) ne contient pas zéro (sinon il faut ajouter zéro aux \( A_n\) ci-dessous, et ça alourdit les notations). Nous supposons donc que les parties
    \begin{equation}
        A_n=A\cap\{ | x |\geq \frac{1}{ n } \}
    \end{equation}
    sont dénombrables. Mais
    \begin{equation}
        A\subset \bigcup_{n=1}^{\infty}A_n.
    \end{equation}
    Une union dénombrable d'ensembles dénombrables est dénombrable\footnote{Proposition \ref{PROPooENTPooSPpmhY}.}. Vu qu'un ensemble non dénombrable ne peut être inclus à un ensemble dénombrable\footnote{Proposition \ref{PropQEPoozLqOQ}.}, nous avons une contradiction.
\end{proof}

\begin{lemma}       \label{LEMooQIMGooOUpZjk}
    Soit un ensemble \( I\) et une «suite» \( (x_i)_{i\in I}\) avec \( x_i\geq 0\) pour tout \( i\). Si l'ensemble
    \begin{equation}
        F=\{ i\in I\tq x_i>0 \}
    \end{equation}
    est non dénombrable, alors
    \begin{equation}
        \sum_{i\in I}x_i=\infty.
    \end{equation}
\end{lemma}

\begin{proof}
    Nous considérons l'ensemble des valeurs non nulles atteintes par \( x\) :
    \begin{equation}
        V=\{ x_i\tq i\in F \}.
    \end{equation}
    Il y a deux possibilités : soit \( V\) est dénombrable (ou fini), soit il est non dénombrable.

    \begin{subproof}
        \item[\( V\) est fini ou dénombrable]
            Dans ce cas, l'application \( x\colon F\to \mathopen[ 0 , \infty \mathclose[\) est une application d'un ensemble indénombrable vers un ensemble dénombrable. Le lemme \ref{LEMooGTOTooFbpvzU} nous indique qu'il existe \( y\in \eR\) tel que \( x^{-1}(y)\) est indénombrable et en particulier infini. La somme \( \sum_{i\in x^{-1}(y)}x_i\) est une somme indénombrable de termes tous égaux et strictement positifs. Elle est infinie.

            \item[\( V\) est indénombrable]
                La partie \( V\) de \( \eR\) est non dénombrable; elle est donc sujette au lemme \ref{LEMooYJCVooHajEbg} : il existe \( \delta>0\) tel que \( W=V\cap\{ x\geq \delta \}\) est indénombrable. Vu que \( x_i\geq \delta\) pour tout \( i\) dans \( x^{-1}(W)\) nous avons
                \begin{equation}
                    \sum_{i\in x^{-1}(W)x_i=\infty}.
                \end{equation}
    \end{subproof}
\end{proof}

%--------------------------------------------------------------------------------------------------------------------------- 
\subsection{Sommes dénombrables}
%---------------------------------------------------------------------------------------------------------------------------

Nous avons vu que les sommes non dénombrables ne sont pas intéressantes. La notion \ref{DefHYgkkA} de sommes est par contre réellement plus utile que la notion de somme sur \( \eN\) parce que \( \eN\) a un ordre. En effet une somme sur \( \eN\) peut être définie par les sommes partielles avec un ordre sans réelle discussions, alors que l'ordre de sommation sur \( \eZ\) est déjà plus discutable. Bref, nous allons voir maintenant quelque propriétés de la somme \ref{DefHYgkkA} dans le cas dénombrable.

\begin{example}     \label{EXooULLXooTDFYqf}
    Au sens de la définition~\ref{DefIkoheE} la famille
    \begin{equation}
        \frac{ (-1)^n }{ n }
    \end{equation}
    n'est pas sommable. En effet la somme des termes pairs est \( \infty\) alors que la somme des termes impairs est \( -\infty\). Quel que soit \( J_0\in \eN\), nous pouvons concocter, en ajoutant des termes pairs, un \( J\) avec \( J_0\subset J\) tel que \( \sum_{j\in J}(-1)^j/j\) soit arbitrairement grand. En ajoutant des termes négatifs, nous pouvons également rendre \( \sum_{j\in J}(-1)^j/j\) arbitrairement petit.
\end{example}

\begin{proposition} \label{PropVQCooYiWTs}
    Si \( (a_{ij})\) est une famille de nombres positifs indexés par \( \eN\times \eN\) alors
    \begin{equation}
        \sum_{(i,j)\in \eN^2}a_{ij}=\sum_{i=1}^{\infty}\Big( \sum_{j=1}^{\infty}a_{ij} \Big)
    \end{equation}
    où la somme de gauche est celle de la définition~\ref{DefHYgkkA}.
\end{proposition}
%TODO : cette proposition peut être vue comme une application de Fubini pour la mesure de comptage. Le faire et référentier ici.

\begin{proof}
    Nous considérons \( J_{m,n}=\{ 0,\ldots, m \}\times \{ 0,\ldots, n \}\) et nous avons pour tout \( m\) et \( n\) :
    \begin{equation}
        \sum_{(i,j)\in \eN^2}a_{ij}\geq \sum_{(i,j)\in J_{m,n}}a_{ij}=\sum_{i=1}^m\Big( \sum_{j=1}^na_{ij} \Big).
    \end{equation}
    Si nous fixons \( m\) et que nous prenons la limite \( n\to \infty\) (qui commute avec la somme finie sur \( i\)) nous trouvons
    \begin{equation}
        \sum_{(i,j)\in \eN^2}a_{ij}\geq =\sum_{i=1}^m\Big( \sum_{j=1}^{\infty}a_{ij} \Big).
    \end{equation}
    Cela étant valable pour tout \( m\), c'est encore valable à la limite \( m\to \infty\) et donc
    \begin{equation}
        \sum_{(i,j)\in \eN^2}a_{ij}\geq \sum_{i=1}^{\infty}\Big( \sum_{j=1}^{\infty}a_{ij} \Big).
    \end{equation}

    Pour l'inégalité inverse, il faut remarquer que si \( J\) est fini dans \( \eN^2\), il est forcément contenu dans \( J_{m,n}\) pour \( m\) et \( n\) assez grand. Alors
    \begin{equation}
        \sum_{(i,j)\in J}a_{ij}\leq \sum_{(i,j)\in J_{m,n}}a_{ij}=\sum_{i=1}^m\sum_{j=1}^na_{ij}\leq \sum_{i=1}^{\infty}\Big( \sum_{j=1}^{\infty}a_{ij} \Big).
    \end{equation}
    Cette inégalité étant valable pour tout ensemble fini \( J\subset \eN^2\), elle reste valable pour le supremum.
\end{proof}

La définition générale de la somme~\ref{DefIkoheE} est compatible avec la définition usuelle dans les cas où cette dernière s'applique.
\begin{proposition}[commutative sommabilité]\label{PropoWHdjw}
    Soit \( I\) un ensemble dénombrable et une bijection \( \tau\colon \eN\to I\). Soit \( (a_i)_{i\in I}\) une famille dans un espace vectoriel normé.  Si \( \sum_{i\in I}a_i\) existe, alors il est donné par
    \begin{equation}
        \sum_{i\in I}a_i=\lim_{N\to \infty} \sum_{k=0}^Na_{\tau(k)}.
    \end{equation}
\end{proposition}

\begin{proof}
    Nous posons \( a=\sum_{i\in I}a_i\). Soit \( \epsilon>0\) et \( J_0\) comme dans la définition. Nous choisissons
    \begin{equation}
        N>\max_{j\in J_0}\{ \tau^{-1}(j) \}.
    \end{equation}
    En tant que sommes sur des ensembles finis, nous avons l'égalité
    \begin{equation}
        \sum_{k=0}^Na_{\tau(k)}=\sum_{j\in J_0}a_j
    \end{equation}
    où \( J\) est un sous-ensemble de \( I\) contenant \( J_0\). Soit \( J\) fini dans \( I\) tel que \( J_0\subset J\). Nous avons alors
    \begin{equation}
        \| \sum_{k=0}^Na_{\tau(k)}-a \|=\| \sum_{j\in J}a_j-a \|<\epsilon.
    \end{equation}
    Nous avons prouvé que pour tout \( \epsilon\), il existe \( N\) tel que \( n>N\) implique \( \| \sum_{k=0}^na_{\tau(k)}-a\| <\epsilon\).
\end{proof}

La réciproque n'est pas vraie. Même en supposant que \( \lim_{N\to \infty} \sum_{n=0}^Na_n\) existe, il n'est pas forcé que \( \sum_{n\in\eN}a_n\) existe. Cela est une conséquence de l'exemple \ref{EXooULLXooTDFYqf}.

\begin{corollary}       \label{CORooBPILooWDXpUM}       % Il ne faut pas référentier ce corolaire qui est sans doute faux.
    Nous pouvons permuter une somme dénombrable et une fonction linéaire continue. C'est-à-dire que si \( f\) est une fonction linéaire continue sur l'espace vectoriel normé \( E\) et \( (a_i)_{i\in I}\) une famille sommable dans \( E\) alors
    \begin{equation}
        f\left( \sum_{i\in I}a_i \right)=\sum_{i\in I}f(a_i).
    \end{equation}
\end{corollary}

\begin{probleme}
    À mon avis, ce corolaire est faux parce qu'il manque l'hypothèse que la famille \( f(a_i)\) est sommable. Voir la proposition \ref{PROPooWLEDooJogXpQ}.
\end{probleme}

\begin{proof}
    En utilisant une bijection \( \tau\) entre \( I\) et \( \eN\) avec la proposition~\ref{PropoWHdjw} ainsi que le résultat connu à propos des sommes sur \( \eN\), nous avons
    \begin{subequations}
        \begin{align}
            f\left( \sum_{i\in I}a_i \right)&=f\left( \sum_{k=0}^{\infty}a_{\tau(k)} \right)\\
            &=\sum_{k=0}^{\infty}f(a_{\tau(k)}) \label{SUBEQooCVUTooPmnHER}\\
            &=\sum_{i\in I}f(a_i).
        \end{align}
    \end{subequations}
    Notons que le passage à \eqref{SUBEQooCVUTooPmnHER} n'est pas du tout une trivialité à deux francs cinquante. Il s'agit d'écrire la somme comme la limite des sommes partielles, et de permuter \( f\) avec la limite en invoquant la continuité, puis de permuter \( f\) avec la somme partielle en invoquant sa linéarité.

    Ah, tiens et tant qu'on y est-à-dire qu'il y a des choses évidentes qui ne le sont pas, oui, il existe des applications linéaires non continues, voir le thème~\ref{THEMEooYCBUooEnFdUg}.
\end{proof}

La proposition suivante nous enseigne que les sommes infinies peuvent être manipulée de façon usuelle.
\begin{proposition} \label{PropMpBStL}
    Soit \( I\) un ensemble dénombrable. Soient \( (a_i)_{i\in I}\) et \( (b_i)_{i\in I}\), deux familles de réels positifs telles que \( a_i<b_i\) et telles que \( (b_i)\) est sommable. Alors \( (a_i)\) est sommable.

    Si \( (a_i)_{i\in I}\) est une famille de complexes telle que \( (| a_i |)\) est sommable, alors \( (a_i)\) est sommable.
\end{proposition}

\begin{proposition}[\cite{MonCerveau}]     \label{PROPooWLEDooJogXpQ}
    Soit un espace vectoriel normé \( E\) et une famille sommable\footnote{Définition~\ref{DefIkoheE}.} \( \{ v_i \}_{i\in I}\) d'éléments de \( E\). Soit \( f\colon E\to \eC\) une application sur laquelle nous supposons
    \begin{enumerate}
        \item
            \( f\) est linéaire et continue;
        \item
            la partie \( \{ f(v_i)_{i\in I} \} \) est sommable.
    \end{enumerate}
    Alors nous pouvons permuter la somme et \( f\) :
    \begin{equation}        \label{EQooONHXooKqIEbY}
        f\big( \sum_{i\in I}v_i \big)=\sum_{i\in I}f(v_i).
    \end{equation}
\end{proposition}

\begin{proof}
    Soit \( \epsilon>0\); vu que les familles \( \{ v_i \}_{i\in I}\) et \( \{ f(v_i) \}_{i\in I}\) sont sommables, nous pouvons considérer les parties finies \( J_1\) et \( J_2\) de \( I\) telles que
    \begin{equation}
        \big\| \sum_{j\in J_1}v_j-\sum_{i\in I}v_i \big\|\leq \epsilon
    \end{equation}
    et
    \begin{equation}
        \big\| \sum_{j\in J_2}f(v_j)-\sum_{i\in I}f(v_i) \big\|\leq \epsilon
    \end{equation}
    Ensuite nous posons \( J=J_1\cup J_2\). Avec cela nous calculons un peu avec les majorations usuelles :
    \begin{equation}
        \| f(\sum_{i\in I}v_i) -\sum_{i\in I}f(v_i) \|\leq \| f(\sum_{i\in I}v_i)- f(\sum_{j\in J}v_j) \|+  \| f(\sum_{j\in J}v_j)-\sum_i\in If(v_i) \|.
    \end{equation}
    Le second terme est majoré par \( \epsilon\), tandis que le premier, en utilisant la linéarité de \( f\) possède la majoration
    \begin{equation}
        \| f(\sum_{i\in I}v_i)- f(\sum_{j\in J}v_j) \|=\| f(\sum_{i\in I}v_i-\sum_{j\in J}v_j) \|\leq \| f \| \| \sum_{i\in I}v_i- \sum_{j\in J}v_j\|\leq \epsilon\| f \|.
    \end{equation}
    Donc pour tout \( \epsilon>0\) nous avons
    \begin{equation}
        \| f(\sum_{i\in I}v_i) -\sum_{i\in I}f(v_i) \|\leq \epsilon(1+\| f \|).
    \end{equation}
    D'où l'égalité \eqref{EQooONHXooKqIEbY}.
\end{proof}

%+++++++++++++++++++++++++++++++++++++++++++++++++++++++++++++++++++++++++++++++++++++++++++++++++++++++++++++++++++++++++++ 
\section{Produit tensoriel d'espaces vectoriels}
%+++++++++++++++++++++++++++++++++++++++++++++++++++++++++++++++++++++++++++++++++++++++++++++++++++++++++++++++++++++++++++

Si vous êtes pressés, vous pouvez aller lire la définition \ref{DEFooKTVDooSPzAhH} de produit tensoriel d'espaces vectoriels. Mais si vous étiez vraiment pressés, vous ne seriez pas en train de lire des choses sur le produit tensoriel (il vous suffit de croire que \( x\otimes y\) n'est finalement que la concatenation de \( x\) et \( y\)).

\begin{definition}
    Soient un espace vectoriel \( V\) et un sous-espace \( N\). Le \defe{quotient}{quotient d'un espace vectoriel} de \( V\) par \( N\), noté \( V/N\) est l'ensemble des classes d'équivalence pour la relation \( x\sim y\) si et seulement si \( x-y\in N\).
\end{definition}

\begin{proposition}
    Soient un espace vectoriel \( V\) et un sous-espace vectoriel \( N\) de \( V\). Les définitions
    \begin{enumerate}
        \item
            \( [v]+[w]=[v+w]\)
        \item
            \( \lambda[v]=[\lambda v]\)
    \end{enumerate}
    ont un sens et définissent une structure d'espace vectoriel sur \( V/N\).
\end{proposition}

\begin{proof}
    Un élément général de la classe \( [v]\) est de la forme \( v+n\) avec \( n\in N\). Le calcul suivant montre que la somme fonctionne : 
    \begin{equation}
        [v+n_1]+[w+n_2]=[v+w+n_1+n_2]=[v+w]
    \end{equation}
    parce que \( n_1+n_2\in N\). De même,
    \begin{equation}
        \lambda[v+n]=[\lambda v+\lambda n]=[\lambda v]
    \end{equation}
    toujours parce que \( \lambda n\in N\).

    Notons que nous avons utilisé de façon on ne peut plus cruciale le fait que \( N\) soit un sous-espace vectoriel.
\end{proof}

\begin{proposition}
    Si \( \{ e_i \}\) est une base de \( V\) et si \( N\) est un sous-espace de \( V\), alors \( \{ [e_i] \}\) est une partie génératrice de \( V/N\).
\end{proposition}

\begin{proof}
    Si \( x=\sum_kx_ke_k\), alors \( [x]=\sum_kx_k[e_k]\), donc oui.
\end{proof}

%--------------------------------------------------------------------------------------------------------------------------- 
\subsection{Somme directe d'espaces vectoriels}
%---------------------------------------------------------------------------------------------------------------------------

Si \( V\) et \( W\) sont des espaces vectoriels, ce que nous notons \( V\oplus W\) n'est rien d'autre que l'espace vectoriel de l'ensemble \( V\times W\).

\begin{propositionDef}[\cite{ooXISFooTypogf}]
    Si \( V\) et \( W\) sont des espaces vectoriels sur le même corps \( \eK\), alors les définitions
    \begin{enumerate}
        \item
            \( (v_1,w_1)+(v_2,w_2)=(v_1+v_2,w_1+w_2)\)
        \item
            \( \lambda(v,w)=(\lambda v,\lambda w)\)
    \end{enumerate}
    donnent une structure d'espace vectoriel sur \( V\times W\). 

    Cet espace sera noté \( V\oplus W\) et est appelé \defe{somme directe}{somme directe} de \( V\) et \( W\).
\end{propositionDef}

\begin{proposition}[\cite{BIBooGTTEooGCUNkM}]       \label{PROPooCASNooEqisqa}
    Soient un espace vectoriel de dimension finie \( V\) et deux sous-espaces \( M_1\) et \( M_2\) satisfaisant
    \begin{enumerate}
        \item
            \( M_1\cap M_2=\{ 0 \}\),
        \item
            \( \dim(M_1)+\dim(M_2)\geq \dim(V)\).
    \end{enumerate}
    Alors \( V=M_1\oplus M_2\).
\end{proposition}

\begin{proof}
    Soient une base \( \{ e_i \}_{i\in I}\) de \( M_1\) et \( \{ f_{\alpha} \}\) de \( M_2\). Nous commençons par prouver que la partie \( B=\{ e_i \}\cup \{ f_{\alpha} \}\) est libre.

    Supposons en effet avoir des coefficients \( a_i\) et \( b_{\alpha}\) tels que
    \begin{equation}
        \sum_ia_ie_i+\sum_{\alpha}b_{\alpha}f_{\alpha}.
    \end{equation}
    Cela implique que \( \sum_ia_ie_i=-\sum_{\alpha}b_{\alpha}f_{\alpha}\). Or \( \sum_ia_ie_i\in M_1\) et \( -\sum_{\alpha}b_{\alpha}f_{\alpha}\in M_2\). Donc les éléments \( \sum_ia_ie_i\) et \( \sum_{\alpha}b_{\alpha}f_{\alpha}\) sont dans \( M_1\cap M_2=\{ 0 \}\). Nous avons alors les égalités
    \begin{equation}
        \sum_ia_ie_i=0
    \end{equation}
    et
    \begin{equation}
        \sum_{\alpha}b_{\alpha}f_{\alpha}=0.
    \end{equation}
    La première implique \( a_i=0\) pour tout \( i\) et la seconde implique \( b_{\alpha}=0\) pour tout \( \alpha\).

    Donc \( B\) est une partie libre de \( V\) contenant \( \dim(M_1)+\dim(M_2)\geq \dim(V)\) éléments. La proposition \ref{PROPooVEVCooHkrldw}\ref{ITEMooUUFCooIVtGgz} nous indique alors qu'en réalité \( \dim(M_1)+\dim(M_2)=\dim(V)\). Vu que \( B\) est une partie libre contenant \( \dim(V)\) éléments, c'est une base par la proposition \ref{PROPooVEVCooHkrldw}\ref{ITEMooSGGCooOUsuBs}.
\end{proof}

La proposition suivante est une version plus «pragmatique» de la proposition \ref{PropXrTDIi}.
\begin{proposition}[\cite{BIBooGTTEooGCUNkM}]       \label{PROPooNITTooCYcrrT}
    Soient un espace euclidien\footnote{Qui possède un produit scalaire, définition \ref{DefLZMcvfj}.} de dimension finie \( V\) ainsi qu'un sous-espace \( M\). Nous posons
    \begin{equation}
        M^{\perp}=\{ x\in V\tq x\cdot y=0\forall y\in M \}.
    \end{equation}
    Alors \( M\oplus M^{\perp}=V\).
\end{proposition}

\begin{proof}
    D'abord si \( x\in M\cap M^{\perp}\), alors \( x\cdot x=0\) et donc \( x=0\). Donc nous avons déjà \( M\cap M^{\perp}=\{ 0 \}\). Nous considérons une base \( \{b_1,\ldots, b_k\}\) de \( M\), et nous définissons l'application linéaire
    \begin{equation}
        \begin{aligned}
            f\colon V&\to \eR^k \\
            x&\mapsto (x\cdot b_1,\ldots, x\cdot b_k). 
        \end{aligned}
    \end{equation}
    Nous avons que \( M^{\perp}=\ker(f)\). Le théorème du rang \ref{ThoGkkffA} nous indique que
    \begin{equation}
        \dim(V)=\dim\big( \ker(f) \big)+\dim\big( \Image(f) \big)\leq \dim(M^{\perp})+k=\dim(M^{\perp})+\dim(M).
    \end{equation}
    Une justification : vu que \( f\) prend ses valeurs dans \( \eR^k\), la dimension de son image est majorée par \( k\).

    Nous en déduisons que 
    \begin{equation}
        \dim(M)+\dim(M^{\perp})\geq\dim(V),
    \end{equation}
    et la proposition \ref{PROPooCASNooEqisqa} nous permet de conclure que \( M\oplus M^{\perp}=V\).
\end{proof}

%--------------------------------------------------------------------------------------------------------------------------- 
\subsection{Les produits tensoriels}
%---------------------------------------------------------------------------------------------------------------------------

Nous allons procéder en deux temps. D'abord nous allons définir ce qu'est \emph{un} produit tensoriel entre deux espaces vectoriels \( V\) et \( W\), et nous allons montrer que tous les produits tensoriels possibles sont isomorphes. Ensuite nous allons montrer qu'un produit tensoriel existe en en construisant un. Voir la proposition \ref{PROPooIWZDooRRZNCf}.

\begin{definition}[\cite{ooWHNKooYVCiYc}]       \label{DEFooXKKQooAvWRNp}
    Soient deux espaces vectoriels \( V\) et \( W\). Un \defe{produit tensoriel}{produit tensoriel} de \( V\) et \( W\) est un couple \( (T,h)\) où \( T\) est un espace vectoriel et \( h\colon V\oplus W\to T\) est une application
    \begin{enumerate}
        \item
            bilinéaire\footnote{Définition \ref{DEFooEEQGooNiPjHz}.}
        \item
            surjective
        \item       \label{ITEMooJCNYooGvjjtL}
            telle que pour tout espace vectoriel \( U\) et toute applications bilinéaire \( f\colon V\oplus W\to U\), il existe une application linéaire \( g\colon T\to U\) telle que \( f=g\circ h\).
    \end{enumerate}
    La propriété \ref{ITEMooJCNYooGvjjtL} est appelée \defe{propriété universelle}{propriété universelle} du produit tensoriel.
\end{definition}

\begin{definition}  \label{DEFooPLHTooRiHjlE}
    Un \defe{morphisme}{morphisme de produits tensoriels} entre \( (T,h)\) et \( (T',h')\) est une application linéaire \( \psi\colon T\to T'\) telle que \( h'=\psi\circ h\).

    Nous parlons d'\defe{isomorphisme}{isomorphisme} si \( \psi\) a un inverse qui est également un morphisme.
\end{definition}

\begin{proposition}[\cite{ooWHNKooYVCiYc}]      \label{PROPooROPHooQXqNzZ}
    Si \( V\) et \( W\) sont des espaces vectoriels, tous les produits tensoriels entre \( V\) et \( W\) sont isomorphes entre eux au sens de la définition \ref{DEFooPLHTooRiHjlE}.

    Plus précisément, si \( (T,h)\) et \( (T',h')\) sont deux produits tensoriels de \( V\) et \( W\), alors 
    \begin{enumerate}
        \item
            il existe une unique unique application linéaire \( g\colon T\to T'\) telle que \( h'=g\circ h\),
        \item
            cette application \( g\) est inversible.
    \end{enumerate}
    En particulier, l'application \( g\) est un isomorphisme d'espaces vectoriels.
\end{proposition}

\begin{proof}
    Soient deux produits tensoriels \( (T,h)\) et \( (T',h')\). 

    \begin{subproof}
        \item[Existence]
    
    L'application \( h'\colon V\oplus W\to T'\) est bilinéaire, et \( (T,h)\) est un produit tensoriel. Donc il existe \( g\colon T\to T'\) tel que \( h'=g\circ h\). De même, il existe une application \( g'\colon T'\to T\) telle que \( h=g'\circ h\).

\item[Unicité]

    En ce qui concerne l'unicité, vu que \( h\colon V\oplus W\to T\) est surjective, la relation \( h'=g\circ h\) prescrit les valeurs de \( g\) sur tous les éléments de \( T\).

\item[Inversible]
    
    Ces deux applications \( g\) et \( g'\) vérifient $h'=gg'h$ et $h=g'gh$, et de plus \( h\colon V\oplus W\to T\) est surjective. Soient \( t\in T\) et \( x\in V\oplus W\) tel que \( t=h(x)\). Nous avons \( h(x)=g'gh(x)\). C'est-à-dire \( t=(g'\circ g)(t)\). De même dans l'autre sens, il existe \( x'\in V\oplus W\) tel que \( t=h'(x')\). En appliquant l'égalité \( h'=gg'h'\) à \( x'\), nous trouvons \( t=(g\circ g')(t)\).

    Tout cela pour dire que \( g'=g^{-1}\). Cette application \( g\) est donc un isomorphisme de produits tensoriels entre \( (T,h)\) et \( (T',h')\).
    \end{subproof}
    Au final, l'application \( g\colon T\to T'\) étant linéaire et inversible, elle est un isomorphisme d'espaces vectoriels.
\end{proof}

Tout cela est fort bien : nous avons unicité à isomorphisme près du produit tensoriel d'espaces vectoriels. Mais nous n'avons pas encore de certitudes à propos de l'existence d'un couple \( (T,h)\) vérifiant les propriétés demandées pour être un produit tensoriel.

Nous allons maintenant construire un produit tensoriel.

%--------------------------------------------------------------------------------------------------------------------------- 
\subsection{Le produit tensoriel}
%---------------------------------------------------------------------------------------------------------------------------

C'est le moment pour vous de relire la définition \ref{DEFooCPNIooNxsYMY} d'espace vectoriel librement engendré, et surtout le lemme \ref{LEMooLOPAooUNQVku} qui en donne une base.

\begin{definition}[\cite{ooWHNKooYVCiYc}]       \label{DEFooKTVDooSPzAhH}
    Soient deux espaces vectoriels \( V\) et \( W\) sur le corps commutatif\footnote{À part mention du contraire, tous les corps du Frido sont commutatifs.} \( \eK\). Dans \( F_{\eK}(V\times W)\) nous considérons les sous-espaces suivants:
    \begin{subequations}
        \begin{align}
            A_1&=\{ \delta_{(v_1,w)}+\delta_{(v_2,w)}-\delta_{(v_1+v_2,w)}\tq v_1,v_2\in V,w\in W  \}\\
            A_2&=\{ \delta_{(v,w_1)}+\delta_{(v,w_2)}-\delta_{(v,w_1+w_2)}\tq v\in V,w_1,w_2\in W  \} \label{SUBEQooSHBJooJLPVbK} \\
            A_3&=\{ \lambda\delta_{v,w}-\delta_{(\lambda v, w)}\tq v\in V,w\in W,\lambda\in \eK \}\\
            A_4&=\{ \lambda\delta_{v,w}-\delta_{(v,\lambda w)}\tq v\in V,w\in W,\lambda\in \eK \}.
        \end{align}
    \end{subequations}
    Nous considérons alors \( N=\Span(A_1,A_2,A_3,A_4)\) et le quotient
    \begin{equation}
        V\otimes_{\eK}W=F_{\eK}(V\times W)/N.
    \end{equation}
    Ce dernier espace vectoriel est le \defe{produit tensoriel}{produit tensoriel} de \( V\) par \( W\).
\end{definition}

\begin{remark}      \label{REMooSLEGooWEiutz}
    Quelque remarques.
    \begin{enumerate}
        \item
            Les éléments de \( V\otimes W\) ne s'écrivent pas tous sous la forme \( v\otimes w\). Certains ont vraiment besoin d'être écrits avec des sommes. En cela, la situation de \( V\otimes W\) est réellement différente de celle de \( V\times W\). Dans ce dernier, tous les éléments sont des couples.
        \item
            La classe de l'élément \( \delta_{(v,w)}\in F(V\times W)\) sera d'habitude noté \( v\otimes w\).
        \item
            Pour insister sur la notion de classe, nous allons aussi noter \( [x]\) la classe de \( x\in F(V\times W)\).
        \item       \label{ITEMooPVWHooMkgQoT}
            L'arithmétique dans \( V\otimes W\) est relativement simple. En ajoutant et soustrayant le même élément de \( A_3\) nous avons par exemple
            \begin{equation}
                (\lambda v)\otimes w=(\lambda v)\otimes w+\lambda (v\otimes w)-(\lambda v)\otimes w.
            \end{equation}
            Nous obtenons de cette façon
            \begin{equation}
                \lambda(v\otimes w)=(\lambda v)\otimes w=v\otimes (\lambda w),
            \end{equation}
            que nous noterons \( \lambda v\otimes w\) sans plus de précision.
    \end{enumerate}
\end{remark}

\begin{proposition}[\cite{ooWHNKooYVCiYc}]     \label{PROPooIWZDooRRZNCf}
    L'espace vectoriel \( V\times W\) muni de
    \begin{equation}
        \begin{aligned}
            h\colon V\oplus W&\to V\otimes W \\
            (v,w) &\mapsto v\otimes w 
        \end{aligned}
    \end{equation}
    est un produit tensoriel entre \( V\) et \( W\).
\end{proposition}

\begin{proof}
    Nous devons prouver les conditions de la définition \ref{DEFooXKKQooAvWRNp}. 
    
    \begin{subproof}
        \item[\( h\) est bilinéaire]

            Ce sont des calculs tels que faits dans la remarque \ref{REMooSLEGooWEiutz}\ref{ITEMooPVWHooMkgQoT} qui font le travail.

        \item[\(h \) est surjective]
    
            Un élément de \( V\otimes W\) est la classe d'un élément de \( F(V\times W)\), c'est-à-dire de la forme
            \begin{equation}
                \big[ \sum_{i\alpha}\delta_{(v_i,w_{\alpha})} \big]=\sum_{i\alpha}a_{i\alpha}v_i\otimes w_{\alpha}.
            \end{equation}
            Cet élément est dans l'image de \( h\) comme le montre le calcul suivant\footnote{Faites bien la distinction entre \( \delta_{v,w}\), \( (v,w)\) et \( v\otimes w\). Sachez dans quel ensemble se trouvent chacun de ces trois objets.} :
            \begin{equation}
                h\big( \sum_{i\alpha}(v_i,w_{\alpha}) \big)=\sum_{i\alpha}a_{i\alpha}h(v_i,w_{\alpha})=\sum_{i\alpha}v_i\otimes w_{\alpha}.
            \end{equation}

        \item[Propriété universelle]

            Soient un espace vectoriel \( U\) et une application linéaire \( f\colon V\oplus W\to U \). Nous devons trouver une application linéaire \( g\colon V\otimes W\to U\) telle que \( f=g\circ h\). Pour cela nous commençons par considérer l'application
            \begin{equation}
                \begin{aligned}
                    g\colon F(V\times W)&\to U \\
                    \delta_{(v,w)}&\mapsto f(v,w) 
                \end{aligned}
            \end{equation}
            définie sur tout \( F(V\times W)\) par linéarité sans encombres parce que les \( \delta_{v,w}\) forment une base par le lemme \ref{LEMooLOPAooUNQVku}.

            Nous démontrons que \( g(N)=0\) pour avoir le droit de passer \( g\) aux classes et le considérer comme application partant de \( V\otimes W\) au lieu de \( F(V\times W)\). Prenons par exemple
            \begin{subequations}
                \begin{align}
                    g\big( \delta_{(v_1,w)}+\delta_{(v_2,w)}-\delta_{(v_1+v_2,w)} \big)&=g( \delta_{(v_1,w)} )+g(\delta_{(v_2,w)})-g(\delta_{v_1+v_2,w})\\
                    &=f(v_1,w)+f(v_2,w)-f(v_1+v_2,w)\\
                    &=0
                \end{align}
            \end{subequations}
            par la bilinéarité de \( f\). Cela montre que \( g(A_1)=0\). Nous montrons de même que \( g(A_2)=g(A_3)=g(A_4)=0\), et enfin toujours par linéarité que \( g(N)=0\). Pour rappel, les éléments de \( N\) sont les combinaisons linéaires finies d'éléments de \( A_1\), \( A_2\), \( A_3\) et \( A_4\).

            Par passage aux classes, nous avons une application (que nous notons également \( g\))
            \begin{equation}
                g\colon F(V\times W)/N\to U
            \end{equation}
            vérifiant \( g(v\otimes w)=f(v,w)\). Mais comme \( h(v,w)=v\otimes w\), nous avons $g\circ h\colon V\oplus W\to U$ vérifiant \( g\circ h=f\).
    \end{subproof}
    L'espace vectoriel \( V\otimes W\) est donc un produit tensoriel.
\end{proof}

\begin{normaltext}
    Vu que \( V\otimes W\) est un produit tensoriel de \( V\) et \( W\), et vu qu'il y a unicité par la proposition \ref{PROPooROPHooQXqNzZ}, nous avons bien le droit de dire que \( V\otimes W\) est \emph{le} produit tensoriel. Cela justifie le titre.
\end{normaltext}

\begin{normaltext}
    Les prochains lemmes et propositions vont nous dire que l'application
    \begin{equation}
        \begin{aligned}
            \varphi\colon V^*\otimes W&\to \aL(V,W) \\
            \alpha\otimes w&\mapsto \big( v\mapsto \alpha(v)w \big) 
        \end{aligned}
    \end{equation}
    est un isomorphisme d'espaces vectoriels lorsque \( V\) est de dimension finie. Vu que nous aimons les énoncés très explicites, ça va être découpé en plusieurs morceaux, l'énoncé va devenir un peu long; mais c'est pour la bonne cause.
\end{normaltext}

\begin{lemma}       \label{LEMooOJEBooQruWEp}
    Soient deux espaces vectoriels \( V\) et \( W\) dont \( W\) est de dimension finie. Alors l'application définie par
    \begin{equation}
        \begin{aligned}
            \varphi\colon F(V^*\times W)&\to \aL(V,W) \\
            \delta_{(\alpha,w)}&\mapsto \big( v\mapsto \alpha(v)w \big) 
        \end{aligned}
    \end{equation}
    sur la base «canonique» de \( F(V^*\times W)\) passe aux classes.
\end{lemma}

\begin{proof}
    Avec les notations de la définition \ref{DEFooKTVDooSPzAhH} nous devons prouver que \( \varphi(N)=0\). Nous montrons que \( \varphi(A_4)=0\), et nous vous laissons faire les autres. Pour \( \lambda\in \eK\), \( \alpha\in V^*\) et \( w\in W\) en utilisant la linéarité de \( \varphi\) nous avons :
    \begin{subequations}
        \begin{align}
            \varphi\big( \lambda\delta_{(\alpha,w)}-\delta_{(\alpha,\lambda w)} \big)v&=\lambda\varphi(\delta_{(\alpha,w)})(v)-\varphi(\delta_{(\alpha,\lambda w)})(v)\\
            &=\lambda\alpha(v)w-\alpha(v)(\lambda w)\\
            &=0
        \end{align}
    \end{subequations}
    parce que \( \alpha(v)(\lambda w)=\lambda \alpha(v)w\) du fait que \( \eK\) est commutatif. La commutativité de \( \eK\) est ce qui permet de permuter le produit \( \lambda \alpha(v)\).

    Nous laissons à la lectrice le soin de prouver que \( \varphi(A_1)=\varphi(A_2)=\varphi(A_3)=0\).
\end{proof}

\begin{lemma}       \label{LEMooUQZHooWjIGsy}
    Si \( W\) est de dimension finie, alors \( \aL(V,W)\) muni de 
    \begin{equation}
        \begin{aligned}
            h'\colon V^*\oplus W&\to \aL(V,W) \\
            (\alpha,w)&\mapsto \big( v\mapsto \alpha(v)w \big) 
        \end{aligned}
    \end{equation}
    est un produit tensoriel\footnote{Définition \ref{DEFooXKKQooAvWRNp}.} de \( V^*\) par \( W\).
\end{lemma}

\begin{proof}
    Nous devons prouver que
    \begin{itemize}
        \item \( h\) est bilinéaire,
        \item \( h\) est surjective
        \item pour tout espace vectoriel \( U\), et pour toute application bilinéaire \( f\colon V^*\oplus W\to U\), il existe une application linéaire \( g\colon \aL(V,W)\to U\) tel que \( f=g\circ h\).
    \end{itemize}

    \begin{subproof}
        \item[Bilinéaire]
            Le fait que \( h\) soit bilinéaire est une simple vérification.
        \item[Surjective]
            L'espace \( W\) étant de dimension finie, nous pouvons en considérer une base \( \{ z_i \}_{i\in I}\). Soit \( \alpha\in \aL(V,W)\). Si \( v\in V\), l'élément \( \alpha(v)\) peut être décomposé dans la base \( \{ z_i \}\), ce qui définit des applications linéaires \( \alpha_i\colon V\to \eK\) par
            \begin{equation}
                \alpha(v)=\sum_{i\in I}\alpha_i(v)z_i.
            \end{equation}
            Notons que \( \alpha_i\in V^*\). En comparant avec la définition de \( h'\), nous voyons que
            \begin{equation}
                \alpha(v)=\sum_i h(\alpha_i,z_i)(v),
            \end{equation}
            c'est-à-dire \( \alpha=\sum_ih(\alpha_i,w_i)=h\big( \sum_i(\alpha_i,z_i) \big)\). Nous avons donc bien \( \alpha\in h(V^*\oplus W)\).
        \item[Propriété universelle]

            Soient un espace vectoriel \( U\) et une application bilinéaire \( f\colon V^*\oplus W\to U\). Pour \( \alpha\in\aL(V,W)\) nous définissons \( g(\alpha)\) comme suit. D'abord nous écrivons \( \alpha\) sous la forme
            \begin{equation}
                \alpha(v)=\sum_i\alpha_i(v)z_i,
            \end{equation}
            et nous posons 
            \begin{equation}
                g(\alpha)=\sum_if(\alpha_i,z_i).
            \end{equation}
            Avec cette définition, en posant \( w=\sum_iw_iz_i\), nous avons
            \begin{subequations}
                \begin{align}
                    (g\circ h')(\alpha,w)&=g\big( v\mapsto \alpha(v)w \big)\\
                    &=g\big( v\mapsto \sum_i\alpha(v)w_iz_i \big)\\
                    &=\sum_if(w_i\alpha,z_i)\\
                    &=\sum_if(\alpha,w_iz_i)\\
                    &=f(\alpha,\sum_iw_iz_i)\\
                    &=f(\alpha,w).
                \end{align}
            \end{subequations}
            Cela prouve que \( g\circ h=f\).
    \end{subproof}
\end{proof}

\begin{proposition}[\cite{ooNHIGooYlXxMf}]      \label{PROPooKJTCooVTXWAQ}
    Soient deux espaces vectoriels \( V\) et \( W\) dont \( V\) est de dimension finie. Alors l'application
    \begin{equation}
        \begin{aligned}
            \varphi\colon V^*\otimes W&\to \aL(V,W) \\
            \alpha\otimes w&\mapsto \big( v\mapsto \alpha(v)w \big) 
        \end{aligned}
    \end{equation}
    est bien définie\footnote{Au sens où il existe une fonction $\varphi$ définie sur tout $V^*\otimes W$ qui se réduit à cela pour les éléments de la forme $\alpha\otimes w$.} et est un isomorphisme d'espaces vectoriels.
\end{proposition}

\begin{proof}
    Le lemme \ref{LEMooUQZHooWjIGsy} donne une structure de produit tensoriel de \( V^*\) par \( W\) sur \( \aL(V,W)\). Rappelons les structures :
    \begin{equation}
        \begin{aligned}
            h\colon V^*\oplus W&\to V^*\otimes W \\
            (\alpha,w)&\mapsto \alpha\otimes w 
        \end{aligned}
    \end{equation}
    et
    \begin{equation}
        \begin{aligned}
            h'\colon V^*\oplus W&\to \aL(V,W) \\
            (\alpha,w)&\mapsto \big[ v\mapsto \alpha(v)w \big].
        \end{aligned}
    \end{equation}

    La proposition \ref{PROPooROPHooQXqNzZ} a déjà fait tout le boulot. La seule chose à faire est de vérifier qu'il existe une application \( \varphi\colon V^*\otimes W\to \aL(V,W)\) vérifiant simultanément les deux conditions suivantes :
    \begin{enumerate}
        \item       \label{ITEMooVNNSooNIXRoG}
            \( \varphi(\alpha\otimes w)=\big[ v\mapsto \alpha(v)w \big]\)
        \item 
            \( h'=\varphi\varphi\circ h\).
    \end{enumerate}
    La seconde condition assure que \( \varphi\) sera un isomorphisme d'espaces vectoriels.

    L'existence de \( \varphi\) vérifiant la condition \ref{ITEMooVNNSooNIXRoG} est un effet du lemme \ref{LEMooOJEBooQruWEp} qui donne une fonction sur \( F(V^*\times W)\) dont le \( \varphi\) qui nous concerne est un quotient. Il reste à voir que cette application vérifie \( h'=\varphi\circ h\).
    
    En nous rappellant que \( \alpha\otimes w=[\delta_{(\alpha,w)}]\) et en écrivant \( \varphi\) à la fois l'application et son passage au quotient,
    \begin{equation}
        (\varphi\circ h)(\alpha,w)=\varphi(\alpha\otimes w)=\varphi\big( [\delta_{(\alpha,w)}] \big)=\varphi(\delta_{(\alpha,w)}).
    \end{equation}
    En appliquant à \( v\in V\) nous avons:
    \begin{equation}
        (\varphi\circ h)(\alpha,w)v=\varphi(\delta_{(\alpha,w)})v=\alpha(v)w=h'(\alpha,w)v.
    \end{equation}
    Et voila. Nous avons \( \varphi\circ h=h'\).
\end{proof}

Une conséquence de la proposition \ref{PROPooKJTCooVTXWAQ} est que
\begin{equation}
    \dim(V\otimes W)=\dim(V)\dim(W)
\end{equation}
via le lemme \ref{LEMooJXFIooKDzRWR}\ref{ITEMooPMLWooNbTyJI}.

%--------------------------------------------------------------------------------------------------------------------------- 
\subsection{Bases}
%---------------------------------------------------------------------------------------------------------------------------

Voici un lemme entièrement dédié au principe «dans le Frido, on ne fait pas d'abus de notations, sauf pour la logique formelle et la théorie des ensembles, que nous admettons».
\begin{lemma}[\cite{MonCerveau}]        \label{LEMooXFIMooDkTSrq}
    Si \( \tau\colon V_1\to V_2\) est un isomorphisme d'espaces vectoriels, alors 
    \begin{equation}        \label{EQooEYUGooYYRZxD}
        \begin{aligned}
            \varphi\colon V_1\otimes W&\to V_2\otimes W \\
            v\otimes w&\mapsto \tau(v)\otimes W 
        \end{aligned}
    \end{equation}
    est un isomorphisme d'espaces vectoriels.
\end{lemma}

\begin{proof}
    Comme d'habitude, l'expression \eqref{EQooEYUGooYYRZxD} ne définit pas réellement \( \varphi\) parce que nous ne savons pas du tout si \( \{v\otimes w\tq v\in V,w\in W\}\) est plus ou moins une base de \( V\otimes W\)\footnote{Ne lisez pas la proposition \ref{PROPooTHDPooWgjUwk} qui dévoile toute l'intrigue.}. Ce que dit réellement ce lemme est qu'il existe une application \( V_1\otimes W\to V_2\otimes W\) qui est isomorphisme et qui se réduit à l'expression donnée dans le cas d'éléments de \( V_1\otimes W\) de la forme \( v\otimes w\).

    L'application
    \begin{equation}
        \begin{aligned}
            \varphi_0\colon F(V_1\times W)&\to F(V_2\times W) \\
            \delta{(v,w)}&\mapsto \delta_{\big( \tau(v),w \big)}
        \end{aligned}
    \end{equation}
    est un isomorphisme.

    Cette application passe aux classes, mais pas au sens où \( x\in [y]\) impliquerait \( \varphi_0(x)=\varphi_0(y)\); au sens où si \( x\in [y]\), alors \( \varphi_0(x)\in[\varphi_0(y)]\). Par exemple
    \begin{equation}
        \varphi_0\big( \lambda\delta_{(v,w)}-\delta_{(v,\lambda w)} \big)=\lambda\delta_{\big( \tau(v),w \big)}-\delta_{\big( \tau(v),w \big)}\in [0].
    \end{equation}
    Nous vous laissons le soin de vérifier les égalités correspondantes pour les autres parties de \( N\).

    Le passage au classes de \( \varphi_0\) signifie que l'on considère l'application
    \begin{equation}
        \begin{aligned}
            \varphi\colon V_1\otimes W&\to V_2\otimes W \\
            [x]&\mapsto [\varphi_0(x)] 
        \end{aligned}
    \end{equation}
    où vous aurez noté que la prise de classe à gauche n'est pas la même que celle à droite.

    Il faut prouver que ce \( \varphi\) est un isomorphisme. En ce qui concerne la linéarité,
    \begin{subequations}
        \begin{align}
            \varphi\big( [x]+[y] \big)&=\varphi\big( [x+y] \big)\\
            &=[\varphi_0(x+y)]\\
            &=[\varphi_0(x)+\varphi_0(y)]\\
            &=[\varphi_0(x)]+[\varphi_0(y)]\\
            &=\varphi([x])+\varphi([y]).
        \end{align}
    \end{subequations}
    Je vous laisse le reste de la linéarité. Et en ce qui concerne le fait que ce soit une bijection, allez-y.
\end{proof}

\begin{proposition}[\cite{ooNHIGooYlXxMf}]      \label{PROPooTHDPooWgjUwk}
    Soient des espaces vectoriels de dimension finie \( V\) et \( W\). Soient une base \( \{e_i\}\) de \( V\) et une base \( \{f_{\alpha}\}\) de \( W\).
    
    Alors :
    \begin{enumerate}
        \item       \label{ITEMooQCILooUncdGl}
            La partie \( \{e_i\otimes f_{\alpha}\}\) est une base de \( V\otimes W\).
        \item
    Au niveau des dimensions, \( \dim(V\otimes W)=\dim(V)\dim(W)\).
    \end{enumerate}
\end{proposition}

\begin{proof}
    Vu que \( V\) est de dimension finie, nous avons un isomorphisme d'espaces vectoriels \( V^*=V\), et même un isomorphisme d'espaces vectoriels
    \begin{equation}
        \begin{aligned}
            \tau\colon V&\to (V^*)^* \\
            \tau(v)\alpha&=\alpha(v).
        \end{aligned}
    \end{equation}
    Recopions l'isomorphisme de la proposition \ref{PROPooKJTCooVTXWAQ} en utilisant \( V^*\) au lieu de \( V\) :
    \begin{equation}
        \begin{aligned}
            \psi_0\colon (V^*)^*\otimes W&\to \aL(V^*,W) \\
           \tau(v)\otimes w &\mapsto \big( \alpha\mapsto \tau(v)(\alpha)w =\alpha(v)w \big).
        \end{aligned}
    \end{equation}
    En écrivant cela, nous avons tenu compte du fait que tout élément de \( (V^*)^*\) peut être écrit de façon univoque sous la forme \( \tau(v)\) pour un certain \( v\in V\).

    Vu que \( \tau\) est un isomorphisme, l'application suivante est encore un isomorphisme\footnote{Lemme \ref{LEMooXFIMooDkTSrq}.} :
    \begin{equation}        \label{EQooAEFRooPfmAnj}
        \begin{aligned}
            \psi\colon V\otimes W&\to \aL(V^*,W) \\
            v\otimes w&\mapsto \big( \alpha\mapsto \alpha(v)w \big). 
        \end{aligned}
    \end{equation}
    Nous avançons. Vu que nous avons un isomorphisme, nous pouvons faire passer des bases. Le lemme \ref{LEMooJXFIooKDzRWR} nous donne une base de \( \aL(V^*,W)\) en les éléments \( \beta_{i\alpha}\colon V^*\to W\) définies par
    \begin{equation}
        \beta_{ij}(\alpha)=\alpha(e_i)f_{\alpha}.
    \end{equation}
    Donc \( \{ \psi^{-1}(\beta_{i\alpha}) \}\) est une base de \( V\otimes W\).

    Pour \( a=\sum_ia_ie_i^*\) (base duale, définition \ref{DEFooTMSEooZFtsqa}) nous avons :
    \begin{equation}
        \psi(e_i\otimes f_{\alpha})a=a(e_i)f_{\alpha}=\beta_{i\alpha}(a).
    \end{equation}
    Cela prouve que \( \psi^{-1}(\beta_{i\alpha})=e_i\otimes f_{\alpha}\), et donc que ces \( e_i\otimes f_{\alpha}\) est une base de \( V\otimes W\).

    La formule concernant les dimensions est simplement la définition \ref{DEFooWRLKooArTpgh} de la dimension : le nombre d'éléments dans une base.
\end{proof}

\begin{example}
    Dans le produit tensoriel \( \eR\otimes \eR\), nous avons \( x\otimes 1=1\otimes x=x(1\otimes x)\) pour tout \( x\in \eR\). Et si \( x\geq 0\) nous avons aussi \( x\otimes 1=\sqrt{ x }\otimes \sqrt{ x }\).
\end{example}

%--------------------------------------------------------------------------------------------------------------------------- 
\subsection{Norme}
%---------------------------------------------------------------------------------------------------------------------------

Nous considérons des espaces vectoriels \( V\) et \( W\) de dimension finie. L'application \eqref{EQooAEFRooPfmAnj} donne un isomorphisme d'espaces vectoriels 
\begin{equation} 
    \begin{aligned}
        \psi\colon V\otimes W&\to \aL(V^*,W) \\
        v\otimes w&\mapsto \big( \alpha\mapsto \alpha(v)w \big). 
    \end{aligned}
\end{equation}
Et ça, c'est très bien, parce que nous connaissons une norme sur \( \aL(V^*,W)\) :  la norme opérateur \ref{DefNFYUooBZCPTr}.

\begin{definition}[\cite{MonCerveau}]      \label{DEFooEXXNooMgIpSV}
    Soient deux espaces vectoriels normés de dimension finie \( V\) et \( W\). Sur \( V\otimes W\) nous définissons, pour \( t\in V\otimes W\)
    \begin{equation}
        \| t \|=\| \psi(t) \|_{\aL(V^*,W)}.
    \end{equation}
\end{definition}
   
\begin{lemma}[\cite{MonCerveau}]        \label{LEMooQPXHooJWfpmk}
    La norme sur \( V\otimes W\) vérifie
    \begin{equation}
        \| v\otimes w \|=\| v \|\| w \|
    \end{equation}
    pour tout \( v\in V\) et \( w\in W\).
\end{lemma}

\begin{proof}
    C'est un simple(?) calcul :
    \begin{equation}
        \| v\otimes w \|=\| \psi(v\otimes w) \|=\| \alpha\mapsto \alpha(v)w \|=\sup_{\| \alpha \|=1}\| \alpha(v)w \|=\sup_{\| \alpha \|=1}| \alpha(v) |\| w \|.
    \end{equation}
    Étant donné que \( V\) est de dimension finie, \( \sup_{\| \alpha \|=1}| \alpha(v) |=\| v \|\)\quext{Cela est une des raisons pour lesquelles nous sommes en dimension finie : je ne sais pas si cette égalité est vraie en dimension inifinie.}. Nous avons donc
    \begin{equation}
        \| v\otimes w \|=\| v \|\| w \|.
    \end{equation}
\end{proof}

Le lemme suivant montre que \( \eR\otimes \eR\) n'est pas du tout \( \eR\times \eR=\eR^2\). Au contraire, \( \eR\otimes \eR\) est isomorphe à \( \eR\).
\begin{lemma}[\cite{MonCerveau}]        \label{LEMooVONEooQpPgcn}
    L'application
    \begin{equation}
        \begin{aligned}
            \varphi\colon \eR\otimes \eR&\to \eR \\
            1\otimes 1&\mapsto 1 
        \end{aligned}
    \end{equation}
    prolongée par linéarité est un isomorphisme isométrique.
\end{lemma}

\begin{proof}
    D'abord une base de \( \eR\) est \( \{ 1 \}\); donc une base de \( \eR\otimes \eR\) est \( \{ 1\otimes 1 \}\) par la proposition \ref{PROPooTHDPooWgjUwk}. Donc l'application proposée se prolonge par linéarité à tout \( \eR\otimes \eR\).

    Le fait que \( \varphi\) soit une bijection provient du fait que \( \varphi\) transforme une base en une base; si vous n'y croyez pas, la vérification de l'injectivité et de la surjectivité est facile.

    Pour que \( \varphi\) soit isométrique, nous faisons le calcul
    \begin{equation}
        \| \varphi(x\otimes y) \|=\| xy(1\otimes 1) \|=| xy |\| 1\otimes 1 \|=| xy |=\| x\otimes y \|.
    \end{equation}
    Nous avons utilisé la propriété \ref{DefNorme}\ref{ItemDefNormeii} d'une norme ainsi que le lemme \ref{LEMooQPXHooJWfpmk} pour la norme sur \( \eR\otimes \eR\).
\end{proof}

%---------------------------------------------------------------------------------------------------------------------------
\subsection{Applications bilinéaires, matrices et produit tensoriel}
%---------------------------------------------------------------------------------------------------------------------------
\label{SECooUKRYooZjagcX}

Soit \( E\), un espace vectoriel de dimension finie. Si \( \alpha\) et \( \beta\) sont deux formes linéaires sur un espace vectoriel \( E\), nous définissons \( \alpha\otimes \beta\) comme étant la \( 2\)-forme donnée par
\begin{equation}        \label{EQooUNRYooKBrXyK}
    (\alpha\otimes \beta)(u,v)=\alpha(u)\beta(v).
\end{equation}
Si \( a\) et \( b\) sont des vecteurs de \( E\), ils sont vus comme des formes sur \( E\) via le produit scalaire et nous avons
\begin{equation}
    (a\otimes b)(u,v)=(a\cdot u)(b\cdot v).
\end{equation}
Cette dernière équation nous incite à pousser un peu plus loin la définition de \( a\otimes b\) et de simplement voir cela comme la matrice de composantes
\begin{equation}
    (a\otimes b)_{ij}=a_ib_j.
\end{equation}
Cette façon d'écrire a l'avantage de ne pas demander de se souvenir qui est une vecteur ligne, qui est un vecteur colonne et où il faut mettre la transposée. Évidemment \( (a\otimes b)\) est soit \( ab^t\) soit \( a^tb\) suivant que \( a\) et \( b\) soient ligne ou colonne.

%---------------------------------------------------------------------------------------------------------------------------
\subsection{Application d'opérateurs}
%---------------------------------------------------------------------------------------------------------------------------

\begin{lemma}   \label{LemMyKPzY}
    Soient \( x,y\in E\) et \( A,B\) deux opérateurs linéaires sur \( E\) vus comme matrices. Alors
    \begin{equation}        \label{EqXdxvSu}
        (Ax\otimes By)=A(x\otimes y)B^t.
    \end{equation}
\end{lemma}

\begin{proof}
    Calculons la composante \( ij\) de la matrice \( (Ax\otimes By)\). Nous avons
    \begin{subequations}
        \begin{align}
            (Ax\otimes By)_{ij}&=(Ax)_i(By)_j\\
            &=\sum_{kl}A_{ik}x_kB_{jl}y_l\\
            &=A_{ik}(x\otimes y)_{kl}B_{jl}\\
            &=\big( A(x\otimes y)B^t \big)_{ij}.
        \end{align}
    \end{subequations}
\end{proof}

%+++++++++++++++++++++++++++++++++++++++++++++++++++++++++++++++++++++++++++++++++++++++++++++++++++++++++++++++++++++++++++
\section{Calcul différentiel dans un espace vectoriel normé}
%+++++++++++++++++++++++++++++++++++++++++++++++++++++++++++++++++++++++++++++++++++++++++++++++++++++++++++++++++++++++++++
\label{SecLStKEmc}

Quelques motivations pour la notion de différentielle sont données dans \ref{SEBSECooLPRQooJRQCFL}.

%---------------------------------------------------------------------------------------------------------------------------
\subsection{Définition de la différentielle}
%---------------------------------------------------------------------------------------------------------------------------

\begin{propositionDef}[\cite{MonCerveau}]      \label{DefDifferentiellePta}
    Soient deux espaces vectoriels normés \( E\) et \( F\) ainsi qu'une fonction \( f\colon \mU\to F\) où \( \mU\) est un ouvert de \( E\).

  Si il existe une une application linéaire \( T\in\aL(E,F)\) satisfaisant
  \begin{equation}	\label{EqCritereDefDiff}
      \lim_{\substack{h\to 0\\h\in E}}\frac{f(a+h)-f(a)-T(h)}{\|h\|_E}=0,
  \end{equation}
  alors il en existe une seule.

  Dans ce cas nous disons que $f$ est \defe{différentiable au point $a$}{application!différentiable} et l'application $T$ ainsi définie est appelée \defe{différentielle}{différentielle} de $f$ au point $a$, et nous la notons $df_a$.
\end{propositionDef}

\begin{proof}
    Soient deux applications linéaires \( T_1\), \( T_2\) satisfaisant la condition \eqref{EqCritereDefDiff}. Nous avons
    \begin{equation}
        \frac{ \| T_1(h)-T_2(h) \|_F }{ \| h \|_E }\leq \frac{ \| T_1(h)-f(a+h)+f(a) \| }{ \| h \| }+\frac{ \| f(a+h)-f(a)-T_2(h) \| }{ \| h \| }\to 0.
    \end{equation}
    Nous avons donc
    \begin{equation}
        \lim_{h\to 0} \frac{ \| (T_1-T_2)(h) \|_F }{ \| h \|_E }=0.
    \end{equation}
    Soit \( \epsilon>0\). Ce que signifie la limite est qu'il existe un \( r>0\) tel que pour tout \( u\in B_E(0,r)\), nous ayons
    \begin{equation}
        \frac{ \| (T_1-T_2)(u) \|_F }{ \| u \|_E }<\epsilon.
    \end{equation}
    Soit \( v\in E\). Nous considérons \( \lambda\in\eR\) tel que \( \lambda v\in B(0,r)\), par exemple \( \lambda<r/\| v \|\). Nous avons
    \begin{equation}
        \epsilon>\frac{ \| (T_1-T_2)(\lambda v) \|_F }{ \| \lambda v \|_E }=\frac{ \| (T_1-T_2)(v) \| }{ \| v \| }.
    \end{equation}
    Cela donne
    \begin{equation}
        \| (T_1-T_2)(v) \|<\| v \|\epsilon.
    \end{equation}
    Nous avons donc \( \| (T_1-T_2)(v) \|=0\), soit \( T_1(v)=T_2(v)\).
\end{proof}


L'application différentielle
\begin{equation}
    \begin{aligned}
        df\colon E&\to \aL(E,F) \\
        a&\mapsto df_a
    \end{aligned}
\end{equation}
est également très importante.

\begin{definition}      \label{DefJYBZooPTsfZx}
Une application \( f\colon E\to F\) est de \defe{classe \( C^1\)}{classe $C^1$} lorsque l'application différentielle \( df\colon E\to \aL(E,F)\) est continue. Voir aussi les définitions~\ref{DefPNjMGqy} pour les applications de classe \( C^k\).
\end{definition}

\begin{remark}      \label{RemATQVooDnZBbs}
    L'application norme étant continue, le critère du théorème~\ref{ThoWeirstrassRn} est en réalité assez général. Par exemple à partir d'une application différentiable\footnote{Définition~\ref{DefDifferentiellePta}.} \( f\colon X\to Y\)  nous pouvons considérer la fonction réelle
    \begin{equation}
        a\mapsto \|  df_a   \|
    \end{equation}
    où la norme est la norme opérateur\footnote{Définition~\ref{DefNFYUooBZCPTr}.}. Si \( f\) est de classe \( C^1\) alors cette application est continue et donc bornée sur un compact \( K\) de \( X\).
\end{remark}

%--------------------------------------------------------------------------------------------------------------------------- 
\subsection{Accroissements finis}
%---------------------------------------------------------------------------------------------------------------------------

\begin{lemma}       \label{LEMooYQZZooVybqjK}
    Soit une fonction \( f\colon E\to V\) (espaces vectoriels normés) différentiable en \( a\in E\). Alors il existe une fonction \( \alpha\colon E\to V\) telle que
    \begin{subequations}
        \begin{numcases}{}
            \lim_{h\to 0} \frac{ \alpha(h) }{ \| h \| }=0\\
            f(a+h)=f(a)+df_a(h)+\alpha(h).
        \end{numcases}
    \end{subequations}
\end{lemma}

\begin{proof}
    Il s'agit seulement de poser
    \begin{equation}
        \alpha(h)=f(a+h)-f(a)-df_a(h).
    \end{equation}
    Le fait que \( \alpha(h)/\| h \|\to 0\) est alors la définition de la différentiabilité de \( f\).
\end{proof}

%---------------------------------------------------------------------------------------------------------------------------
\subsection{(non ?) Différentiabilité des applications linéaires}
%---------------------------------------------------------------------------------------------------------------------------

Si \( E\) et \( F\) sont deux espaces vectoriels nous notons \( \aL(E,F)\)\nomenclature[Y]{\( \aL(E,F)\)}{Les applications linéaires de \( E\) vers \( F\)} l'ensemble des applications linéaires de \( E\) vers \( F\) et \( \cL(E,F)\)\nomenclature[Y]{\( \cL\)}{Les applications linéaires continues de \( E\) vers \( F\)} l'ensemble des applications linéaires continues de \( E\) vers \( F\). Ces espaces seront bien entendu, sauf mention du contraire, toujours munis de la norme opérateur de la définition~\ref{DefNFYUooBZCPTr}.

\begin{lemma}       \label{LemooXXUGooUqCjmp}
    Soit une application linéaire \( f\).
    \begin{enumerate}
        \item
            Si \( f\) est continue, alors elle est différentiable et \( df_a(u)=f(u)\) pour tout \( a\) et \( u\).
        \item
            Si \( f\) n'est pas continue, alors elle n'est pas différentiable.
    \end{enumerate}
\end{lemma}

\begin{proof}
    La linéarité de \( f\) donne :
    \begin{equation}
        f(a+h)-f(a)-f(h)=0,
    \end{equation}
    et donc prendre \( T=f\) dans la définition~\ref{DefDifferentiellePta} fait fonctionner la limite. De plus \( T\) est alors continue par hypothèse; elle est donc bien la différentielle de \( f\).

    Supposons que \( f\) ne soit pas continue, prenons une application linéaire continue \( T\), et calculons
    \begin{equation}        \label{EQooFLYMooEKTeOC}
        \frac{ f(a+h)-f(a)-T(h) }{ \| h \| }=\frac{ (f-T)(h) }{ \| h \| }=(f-T)(e_h)
    \end{equation}
    où \( e_h\) est le vecteur unitaire dans la direction de \( h\). Vu que \( f\) n'est pas continue et que \( T\) l'est, l'application \( f-T\) n'est pas continue. Elle n'est pas pas bornée par la proposition~\ref{PROPooQZYVooYJVlBd}. Il existe alors un vecteur \( h\) tel que \( \| (f-T)(e_h) \|>1\) (et même plus grand que ce qu'on veut).

    Donc la limite de \eqref{EQooFLYMooEKTeOC} pour \( h\to 0\) ne peut pas être nulle.
\end{proof}

\begin{lemma}   \label{LemLLvgPQW}
    Une application linéaire continue est de classe \(  C^{\infty}\).
\end{lemma}

\begin{proof}
    Soit \( a\in E\). Étant donné que \( f\) est linéaire et continue, elle est différentiable et
    \begin{equation}
        \begin{aligned}
            df\colon E&\to \cL(E,F) \\
            a&\mapsto f
        \end{aligned}
    \end{equation}
    est une fonction constante et en particulier continue; nous avons donc \( f\in C^1\). Pour la différentielle seconde nous avons \( d(df)_a=0\) parce que \( df(a+h)-df(a)=f-f=0\). Toutes les différentielles suivantes sont nulles.
\end{proof}

%---------------------------------------------------------------------------------------------------------------------------
\subsection{Dérivation en chaine et formule de Leibnitz}
%---------------------------------------------------------------------------------------------------------------------------

\begin{proposition} \label{PropOYtgIua}
    Soient \( f_i\colon U\to F_i\), des fonctions de classe \( C^r\) où \( U\) est ouvert dans l'espace vectoriel normé \( E\) et les \( F_i\) sont des espaces vectoriels normés. Alors l'application
    \begin{equation}
        \begin{aligned}
        f=f_1\times \cdots\times f_n\colon U&\to F_1\times \cdots\times F_n \\
    x&\mapsto \big( f_1(x),\ldots, f_n(x) \big)
        \end{aligned}
    \end{equation}
    est de classe \( C^r\) et
    \begin{equation}
    d^rf=d^rf_1\times\ldots d^rf_n.
    \end{equation}
\end{proposition}

\begin{proof}
    Soit \( x\in U\) et \( h\in E\). La différentiabilité des fonctions \( f_i\) donne
    \begin{equation}
        f_i(x+h)=f_i(x)+(df_i)_x(h)+\alpha_i(h)
    \end{equation}
    avec \( \lim_{h\to 0} \alpha_i(h)/\| h \|=0\). Par conséquent
    \begin{subequations}
        \begin{align}
            f(x+h)&=\big( \ldots, f_i(x)+(df_i)_x(h)+\alpha_i(h),\ldots \big)\\
            &= \big( \ldots,f_i(x),\ldots \big)+ \big( \ldots,(df_i)_x(h),\ldots \big)+ \big( \ldots,\alpha_i(h),\ldots \big).
        \end{align}
    \end{subequations}
    Mais la définition~\ref{DefFAJgTCE} de la norme dans un espace produit donne
    \begin{equation}
        \lim_{h\to 0} \frac{ \| \big( \alpha_1(h),\ldots, \alpha_n(h) \big) \| }{ \| h \| }=0,
    \end{equation}
    ce qui nous permet de noter \( \alpha(h)=\big( \alpha_1(h),\ldots, \alpha_n(h) \big)\) et avoir \( \lim_{h\to 0} \alpha(h)/\| h \|=0\). Avec tout ça nous avons bien
    \begin{equation}
        f(x+h)=f(x)+\big( (df_1)_x(h)+\cdots +(df_n)_x(h) \big)+\alpha(h),
    \end{equation}
    ce qui signifie que \( f\) est différentiable et
    \begin{equation}
        df_x=\big( df_1,\ldots, df_n \big).
    \end{equation}
\end{proof}

\begin{theorem}     \label{THOooIHPIooIUyPaf}
    Soient des espaces vectoriels normés \( E,V\) et \( W\). Nous considérons deux fonctions \( f\colon E\to V\) et \( g\colon V\to W\). Nous supposons que \( f\) est différentiable en \( a\in E\) et que \( g\) est différentiable en \( f(a)\in V\). 

    Nous supposons de plus que \( df_a\) est de norme finie\quext{Je ne suis pas totalement certain que cette hypothèse soit nécessaire, mais en tout cas, elle est utilisée.}.
    
    
    Alors \( g\circ f\colon E\to W\) est différentiable en \( a\) et
    \begin{equation}
        f(g\circ f)_a(u)=df_{f(a)}\big( df_a(u) \big),
    \end{equation}
    ou encore
    \begin{equation}
        f(g\circ f)_a=dg_{f(a)}\circ df_a.
    \end{equation}
\end{theorem}

\begin{proof}
    En utilisant le lemme \ref{LEMooYQZZooVybqjK} pour les fonctions \( f\) et \( g\), nous avons
    \begin{equation}        \label{EQooXNWZooJSPjRS}
        f(a+h)=f(a)+df_a(h)+\alpha(h)
    \end{equation}
    et
    \begin{equation}        \label{EQooIQZZooWPyMbE}
        g\big( f(a)+k \big)=g\big( f(a) \big)+dg_{f(a)}(k)+\beta(k).
    \end{equation}
    L'application \( dg_{f(a)}\circ df_a\) est une application linéaire, et est notre candidat différentielle. En suivant la définition \ref{DefDifferentiellePta}, nous allons calculer
    \begin{equation}
        \lim_{h\to 0} \frac{ (g\circ f)(a+h)-(g\circ f)(a)-(dg_{f(a)}\circ df_a)(h) }{ \| h \| }.
    \end{equation}
    Si cette limite existe et vaut zéro, alors nous aurons prouvé que le candidat différentielle est correct.

    Pour cela, nous emboîtons les formules \eqref{EQooXNWZooJSPjRS} et \eqref{EQooIQZZooWPyMbE} l'une dans l'autre pour avoir : 
    \begin{equation}
        g(a+h)=g\big( f(a)+df_a(h)+\alpha(h) \big)=g\big( f(a) \big)+dg_{f(a)}\big( df_a(h)+\alpha(h) \big)+\beta\big( df_a(h)+\alpha(h) \big).
    \end{equation}
    Vu que \( dg_{f(a)}\) est linéaire, le deuxième terme peut être coupé en deux et après recombinaisons,
    \begin{equation}
        (g\circ f)(a+h)-(g\circ f)(a)-(df_{f(a)}\circ df_a)(h)=dg_{f(a)}\big( \alpha(h) \big)+\beta\big( df_a(h)+\alpha(h) \big).
    \end{equation}
    Étant donné que \( dg_{f(a)}\) est linéaire,
    \begin{equation}
        \frac{ dg_{f(a)}\big(\alpha(h)\big) }{ \| h \| }=dg_{f(a)}\left( \frac{ \alpha(h) }{ \| h \| } \right)\to 0.
    \end{equation}
    Il nous reste à voir que
    \begin{equation}        \label{EQooUQNUooFgNyJp}
        \lim_{h\to 0} \frac{ \beta\big( df_a(h)+\alpha(h) \big) }{ \| h \| }
    \end{equation}
    existe au vaut zéro. Vu que \( df_a\) est linéaire, il existe \( M>0\) tel que\footnote{Ce \( M\) est par exemple la norme opérateur de \( df_a\), comme nous l'assure le lemme \ref{LEMooIBLEooLJczmu}. C'est pour ce passage-ci que nous avons supposé que \( df_a\) était de norme finie.} \( \| df_a(h) \|\leq M\| h \|\). D'autre part, vu que \( \alpha(h)/\| h \|\to 0\), nous avons \( \| \alpha(h) \|\leq \| h \|\) pour tout \( h\) suffisamment petit.

    Donc si \( h\) est assez petit, nous avons
    \begin{equation}        \label{EQooEQJBooSmacrD}
        \| df_a(h)+\alpha(h) \|\leq (M+1)\| h \|.
    \end{equation}
    Soit \( \epsilon>0\). Soit \( \delta>0\) tel que \( \| h \|\leq \delta\) implique \( \beta(h)/\| h \|\leq \epsilon\) et \eqref{EQooEQJBooSmacrD} en même temps. Soit \( r\) tel que \( (M+1)r<\delta\); et notons que \( r<\delta\). Nous considérons alors \( h\in B(0,r)\) et nous calculons :
    \begin{equation}
        \frac{ \beta\big( df_a(h)+\alpha(h) \big) }{ \| h \| }=\frac{ \beta\big( df_a(h)+\alpha(h) \big) }{ \| df_a(h)+\alpha(h) \| }\frac{ \| df_a(h)+\alpha(h) \| }{ \| h \| }\leq (M+1)\epsilon.
    \end{equation}
    La limite \eqref{EQooUQNUooFgNyJp} existe donc et vaut zéro.
\end{proof}

\begin{theorem}[Différentielle de fonctions composées\cite{SNPdukn}]    \label{ThoAGXGuEt}
    Soient \( E\), \( F\) et \( G\) des espaces vectoriels normés, \( U\) ouvert dans \( E\) et \( V\) ouvert dans \( F\). Soient des applications de classe \( C^r\) (\( r\geq 1\))
    \begin{subequations}
        \begin{align}
            f\colon U\to V\\
            g\colon V\to G.
        \end{align}
    \end{subequations}
    Alors l'application \( g\circ f\colon V\to G\) est de classe \( C^r\) et
    \begin{equation}\label{EqHFmezmr}
        d(g\circ f)_x=dg_{f(x)}\circ df_x.
    \end{equation}
\end{theorem}

\begin{proof}
    Nous nous fixons \( x\in U\). La fonction \( f\) est différentiable en \( x\in U\) et \( g\) en \( f(x)\), donc nous pouvons écrire
    \begin{equation}
        f(x+h)=f(x)+df_x(h)+\alpha(h)
    \end{equation}
    et
    \begin{equation}
        g\big( f(x)+u \big)=g\big( f(x) \big)+dg_{f(x)}(u)+\beta(u)
    \end{equation}
    où la fonction \( \alpha\) a la propriété que
    \begin{equation}
        \lim_{h\to 0} \frac{ \| \alpha(h) \| }{ \| h \| }=0;
    \end{equation}
    et la même chose pour \( \beta\). La fonction composée en \( x+h\) s'écrit donc
    \begin{equation}    \label{EqCXcfhfH}
        (g\circ f)(x+h)=g\big( f(x)+df_x(h)+\alpha(h) \big)=g\big( f(x) \big)+dg_{f(x)}\big( df_x(h)+\alpha(h) \big)+\beta\big( df_x(h)+\alpha(h) \big).
    \end{equation}
    Nous montrons que tous les «petits» termes de cette formule peuvent être groupés. D'abord si \( h\) est proche de \( 0\), nous avons
    \begin{equation}
        \frac{ \| df_x(h)+\alpha(h) \| }{ \| h \| }\leq\frac{ \| df_x \|\| h \| }{ \| h \| }+\frac{ \| \alpha(h) \| }{ \| h \| }.
    \end{equation}
    Si \( h\) est petit, le second terme est arbitrairement petit, donc en prenant n'importe que \( M>\| df_x \|\) nous avons
    \begin{equation}
        \frac{ \| df_x(h)+\alpha(h) \| }{ \| h \| }\leq M.
    \end{equation}
    Par ailleurs, nous avons
    \begin{equation}
        \frac{ \| \beta\big( df_x(h)+\alpha(h) \big) \| }{ \| h \| }=\frac{  \| \beta\big( df_x(h)+\alpha(h) \big) \|  }{ \| df_x(h)+\alpha(h) \| }\frac{  \| df_x(h)+\alpha(h) \|  }{ \| h \| }\leq M\frac{  \| \beta\big( df_x(h)+\alpha(h) \big) \|  }{   \| df_x(h)+\alpha(h) \| }.
    \end{equation}
    Vu que la fraction est du type \( \frac{ \beta( f(h)) }{ f(h) }\) avec \( \lim_{h\to 0} f(h)=0\), la fraction tend vers zéro lorsque \( h\to 0\). En posant
    \begin{equation}
        \gamma_1(h)=\beta\big( df_x(h)+\alpha(h) \big)
    \end{equation}
    nous avons \( \lim_{h\to 0} \gamma_1(h)/\| h \|=0\).

    L'autre candidat à être un petit terme dans \eqref{EqCXcfhfH} est traité en utilisant le lemme~\ref{LEMooFITMooBBBWGI} :
    \begin{equation}
        \| dg_{f(x)}\big( \alpha(h) \big) \|\leq \| dg_{f(x)} \|\| \alpha(h) \|.
    \end{equation}
    Donc
    \begin{equation}
        \frac{ \| dg_{f(x)}\big( \alpha(h) \big) \| }{ \| h \| }\leq \| dg_{f(x)} \|\frac{ \| \alpha(h) \| }{ \| h \| },
    \end{equation}
    ce qui nous permet de poser
    \begin{equation}
        \gamma_2(h)=dg_{f(x)}\big( \alpha(h) \big)
    \end{equation}
    avec \( \gamma_2\) qui a la même propriété que \( \gamma_1\). Avec tout cela, en posant \( \gamma=\gamma_1+\gamma_2\) nous récrivons
    \begin{equation}
        (g\circ f)(x+h)=g\big( f(x) \big)+dg_{f(x)}\big( df_x(h) \big)+\gamma(h)
    \end{equation}
    avec \( \lim_{h\to 0} \frac{ \gamma(h) }{ \| h \| }=0\). Tout cela pour dire que
    \begin{equation}
        \lim_{h\to 0} \frac{ (g\circ f)(x+h)-(g\circ f)(x)-\big( dg_{f(x)}\circ df_x \big)(h) }{ \| h \| }=0,
    \end{equation}
    ce qui signifie que
    \begin{equation}
        d(g\circ f)_x=dg_{f(x)}\circ df_x.
    \end{equation}
    Nous avons donc montré que si \( f\) et \( g\) sont différentiables, alors \( g\circ f\) est différentiable avec différentielle donnée par \eqref{EqHFmezmr}.

    Nous passons à la régularité. Nous supposons maintenant que \( f\) et \( g\) sont de classe \( C^r\) et nous considérons l'application
    \begin{equation}
        \begin{aligned}
            \varphi\colon L(F,G)\times L(E,F)&\to L(E,G) \\
            (A,B)&\mapsto A\circ B.
        \end{aligned}
    \end{equation}
    Montrons que l'application \( \varphi\) est continue en montrant qu'elle est bornée\footnote{Proposition~\ref{PROPooQZYVooYJVlBd}.}. Pour cela nous écrivons la norme opérateur
    \begin{equation}
        \| \varphi \|=\sup_{\| (A,B) \|=1}\| \varphi(A,B) \|=\sup_{\| (A,B) \|=1}\| A\circ B \|\leq\sup_{\| (A,B) \|=1}\| A \|\| B \|\leq 1.
    \end{equation}
    Justifications : d'une part la norme opérateur est une norme algébrique\footnote{Lemme \ref{LEMooFITMooBBBWGI}.}, et d'autre part la définition \ref{DefFAJgTCE} de la norme sur un espace produit pour la dernière majoration. L'application \( \varphi\) est donc continue et donc \(  C^{\infty}\) par le lemme~\ref{LemLLvgPQW}. Nous considérons également l'application
    \begin{equation}
        \begin{aligned}
        \psi\colon U&\to L(F,G)\times L(E,F) \\
        x&\mapsto \big( dg_{f(x)},df_x \big).
        \end{aligned}
    \end{equation}
    Vu que \( f\) et \( g\) sont \( C^1\), l'application \( \psi\) est continue. Ces deux applications \( \varphi\) et \( \psi\) sont choisies pour avoir
    \begin{equation}
        (\varphi\circ\psi)(x)=\varphi\big( dg_{f(x)},df_x \big)=dg_{f(x)}\circ df_x,
    \end{equation}
    c'est-à-dire \( \varphi\circ\psi=d(g\circ f)\). Les applications \( \varphi\) et \( \psi\) étant continues, l'application \( d(g\circ f)\) est continue, ce qui prouve que \( g\circ f\) est \( C^1\).

    Si \( f\) et \( g\) sont \( C^r\) alors \( dg\in C^{r-1}\) et \( dg\circ f\in C^{r-1}\) où il ne faut pas se tromper : \( dg\colon F\to L(F,G)\) et \( f\colon U\to F\); la composée est \( dg\circ f\colon x\mapsto dg_{f(x)}\in L(F,G)\).

    Pour la récurrence nous supposons que \( f,g\in C^{r-1}\) implique \( g\circ f\in C^{r-1}\) pour un certain \( r\geq 2\) (parce que nous venons de prouver cela avec \( r=1\) et \( r=2\)). Soient \( f,g\in C^r\) et montrons que \( g\circ f\in C^r\). Par la proposition~\ref{PropOYtgIua} nous avons
    \begin{equation}
        \psi=dg\circ f\times df\in C^{r-1},
    \end{equation}
    et donc \( d(g\circ f)=\varphi\circ\psi\in C^{r-1}\), ce qui signifie que \( g\circ f\in C^r\).
\end{proof}

\begin{proposition}[\cite{MonCerveau}]      \label{PROPooRCZOooSgvpSE}
    Soit une application \( f\colon E\to V\) de classe \( C^1\). Soit une application linéaire \( \varphi\colon V \to W\). Alors \( \varphi\circ f\) est de classe \( C^p\).
\end{proposition}

\begin{proof}
    Toute la preuve est un grand jeu de cohérence des espaces en présence, alors soyez attentifs et capable de dire précisément à quel espace appartient chacun de objets entrant en jeu.

    Nous posons \( V_0=V\) et \( V_{k+1}=\aL(E,V_k)\). Idem pour les espaces \( W_k\). Ensuite nous posons
    \begin{equation}
        \begin{aligned}
            \varphi_1\colon \aL(E,V)&\to \aL(E,W) \\
            \alpha&\mapsto \varphi\circ \alpha.
        \end{aligned}
    \end{equation}
    et
    \begin{equation}
        \begin{aligned}
            \varphi_k\colon \aL(E,V_{k-1})&\to \aL(E,W_{k-1}) \\
            \alpha&\mapsto \varphi_{k-1}\circ \alpha.
        \end{aligned}
    \end{equation}
    Notez la cohérence : si \( a\in E\), \( \alpha(a)\in V_{k-1}=\aL(E,V_{k-2})\), et donc
    \begin{equation}
        (\varphi_{k-1}\circ\alpha)(a)=\varphi_{k-1}\big( \alpha(a) \big).
    \end{equation}
    À droite nous avons \( \varphi_{k-1}\big( \alpha(a) \big)\in \aL(E,W_{k-2})=V_{k-1}\).

    De plus, \( \varphi\) est linéaire; ça se prouve par récurrence en partant de \( \varphi_1\) et en se basant sur le fait que \( \varphi\) est linéaire.

    C'est parti pour une récurrence.

    \begin{subproof}
        \item[Énoncé]
            Nous allons prouver par récurrence que
            \begin{equation}
                d^k(\varphi\circ f)=\varphi_k\circ d^kf.
            \end{equation}
            pour tout \( k\leq p\).
        \item[Initialisation]
 
            D'abord, \( f\) est de classe \( C^p\), donc différentiable et \( \varphi\) est linéaire donc différentiable. Donc la composée est différentiable et le théorème \ref{THOooIHPIooIUyPaf} nous donne la différentiabilité de \( \varphi\circ f\) ainsi que la formule
            \begin{equation}
                d(\varphi\circ f)_a(u)=d\varphi_{f(a)}\big( df_a(u) \big)=(\varphi\circ df_a)(u)=\varphi_1(df_a)(u).
            \end{equation}
            Donc \( d(\varphi\circ f)_a=\varphi_1(df_a)\), ce qui signifie 
            \begin{equation}
                d(\varphi\circ f)=\varphi_1\circ df.
            \end{equation}
            C'est bon pour \( k=1\).


        \item[La pas de récurrence]

            Vu que \( f\) est de classe \( C^p\), \( d^kf\) est encore différentiable. Vu que \( \varphi_k\) est encore linéaire, nous pouvons encore utiliser la règle de différentiation de fonctions composées sur l'application \( \varphi_k\circ d^kf\). Nous avons :
            \begin{equation}
                d^{k+1}(\varphi\circ f)_a(u)=d\big( d^k(\varphi\circ f) \big)_a(u)=d(\varphi_k\circ d^kf)_a(u).
            \end{equation}
            C'est le moment d'utiliser la formule de différentiation en chaine :
            \begin{equation}
                d^{k+1}(\varphi\circ f)_a(u)=\big( (d\varphi_k)_{d^kf_a}\circ d^{k+1}f_a \big)(u).
            \end{equation}
            Mais \( \varphi_k\) étant linéaire, \( (d\varphi_k)_{d^kf_a}=\varphi_k\), donc
            \begin{equation}
                d^{k+1}(\varphi\circ f)_a(u)=(\varphi_k\circ d^{k+1}f_a)(u).
            \end{equation}
            Donc, en oubliant l'application au vecteur \( u\),
            \begin{equation}
                d^{k+1}(\varphi\circ f)_a=\varphi_k\circ d^{k+1}f_a=\varphi_{k+1}\big( d^{k+1}f_a \big)=(\varphi_{k+1}\circ d^{k+1}f)(a).
            \end{equation}
            Nous avons donc bien
            \begin{equation}
                d^{k+1}(\varphi\circ f)=\varphi_{k+1}\circ d^{k+1}f.
            \end{equation}
    \end{subproof}
\end{proof}

\begin{lemma}       \label{LemooTJSZooWkuSzv}
    Si \( f\colon U\to V\) est un difféomorphisme\footnote{Définition~\ref{DefAQIQooYqZdya}} alors pour tout \( a\in U\), l'application \( df_a\) est inversible et
    \begin{equation}
        (df_a)^{-1}=(df^{-1})_{f(a)}.
    \end{equation}
\end{lemma}

\begin{proof}
    Il suffit d'apercevoir qu'en vertu de la règle de différentiation en chaine \eqref{EqHFmezmr},
    \begin{equation}
        (df_a)(df^{-1})_{f(a)}=d(f\circ f^{-1})_{f(a)}=\id.
    \end{equation}
\end{proof}

\begin{proposition}     \label{PROPooNONAooCyAtce}
    Soient des ouverts \( A\) de \( \eR^p\) et \( B\) de \( \eR^m\). Si il existe un difféomorphisme \( f\colon A\to B\), alors \( p=m\).
\end{proposition}

\begin{proof}
    Vu que \( f\) est un difféomorphisme, le lemme \ref{LemooTJSZooWkuSzv} fait son travail : l'application linéaire \( df_a\colon \eR^p\to \eR^m\) est inversible d'inverse \( df^{-1}_{f(a)}\colon \eR^m\to \eR^m\).

    Or une application linéaire ne peut pas être bijective entre espaces de dimensions différentes (finies). Donc \( p=m\).
\end{proof}

%--------------------------------------------------------------------------------------------------------------------------- 
\subsection{Différentiation de produit}
%---------------------------------------------------------------------------------------------------------------------------

Si nous avons deux application \( f\colon E\to V\) et \( g\colon E\to W\), alors nous voudrions considérer la fonction
\begin{equation}
    \begin{aligned}
        f\otimes g\colon E&\to V\otimes W \\
        a&\mapsto f(a)\otimes g(a). 
    \end{aligned}
\end{equation}
Le problème avec cette notation est que très souvent, les applications \( f\) et \( g\) sont des éléments d'espaces vectoriels. Si par exemple \( f\in \aL(E,V)\) et \( g\in \aL(E,W)\), nous avons \( f\otimes g\in \aL(E,V)\otimes \aL(E,W)\). Dans le Frido nous ne nous permettons pas de dire calmement que \( \aL(E,V)\otimes \aL(E,W)=\aL(E,V\otimes W)\). Et je ne vous dit même pas à quel point il n'est pas évident, si \( f\in C^{\infty}(E,V)\) et \( g\in  C^{\infty}(E,W)\) que nous aurions \( f\otimes g\in C^{\infty}(E,V)\otimes  C^{\infty}(E,W)= C^{\infty}(E,V\otimes W)\).

Tout cela pour dire que nous n'allons pas nous lancer dans des abus de notations. Non. Au lieu de cela, nous introduisons une notation. Pour rappel, dans tout le Frido, \( \Fun(A,B)\) désigne l'ensemble de toutes les application de \( A\) vers \( B\) sans suppositions de régularité. Pour les puristes, nous précisions que si \( f\in\Fun(A,B)\), nous supposons que \( f\) est définie sur tout \( A\). hum \ldots sauf mention du contraire.
\begin{definition}      \label{DEFooMVNDooFWFtRn}
    Si \( f\in \Fun(E,V)\) et \( g\in \Fun(E,W)\), alors nous définissons
    \begin{equation}
        \begin{aligned}
            f\tilde\otimes g\colon E&\to V\otimes W \\
            a&\mapsto f(a)\otimes g(a). 
        \end{aligned}
    \end{equation}
\end{definition}

\begin{proposition}     \label{PROPooCRVXooEGxdZl}
    Soient des applications continues \( f\colon E\to V\) et \( g\colon E\to W\) entre espaces vectoriels de dimension finies. Alors la fonction \( f\tilde\otimes g\colon E\to V\otimes W\) est continue.
\end{proposition}

\begin{proof}
    Soient \( a\in E\) ainsi qu'une suite \( x_k\to a\) dans \( E\). Nous voulons prouver que \( f\tilde\otimes g(x_k)\stackrel{V\otimes W}{\longrightarrow}f(a)\otimes g(a)\). Nous avons :
    \begin{equation}        \label{EQooSNXUooXrYOeY}
        \| f(x_k)\otimes g(x_k)-f(a)\otimes g(a) \|\leq \| f(x_k)\otimes g(x_k)-f(x_k)\otimes g(a) \|+\| f(x_k)\otimes g(a)-f(a)\otimes g(a) \|.
    \end{equation}
    Ensuite en utilisant la classe d'équivalence \eqref{SUBEQooSHBJooJLPVbK}, 
    \begin{equation}
        f(x_k)\otimes g(x_k)-f(x_k)\otimes g(a)=f(x_k)\otimes \big( g(x_k)-g(a) \big),
    \end{equation}
    et en ce qui concerne les normes,
    \begin{equation}
    \|   f(x_k)\otimes g(x_k)-f(x_k)\otimes g(a)\|  =\|f(x_k)\|  \|\otimes \big( g(x_k)-g(a) \big)\|.
    \end{equation}
    Mais par hypothèse, \( f(x_k)\to f(a)\) et \( g(x_k)\to g(a)\). Donc le tout tend vers zéro lorsque \( k\to \infty\).

    Le même raisonnement fonctionne avec le second terme de \eqref{EQooSNXUooXrYOeY}.
\end{proof}

Lorsque nous parlons de différentielle de produit de fonctions, nous voulons étudier la différentiabilité de \( f\tilde\otimes g\) sous l'hypothèse de différentiabilité de \( f\) et \( g\). Et aussi, si \( f\) et \( g\) sont de classe \( C^p\), est-ce que \( f\tilde\otimes g\) est également de classe \( C^p\) ?

Nous voudrions avoir une formule du type
\begin{equation}
    d(f\tilde\otimes g)=df\tilde\otimes g+f\tilde\otimes dg,
\end{equation}
mais ça ne colle pas au niveau des espaces. En effet, en évaluant cela en \( a\in E\), nous avons à gauche \( d(f\tilde\otimes g)_a\in\aL(E,V\otimes W)\), tandis qu'à droite nous avons \( df_a\otimes g(a)\in \aL(E,V)\otimes W\) et \( f(a)\otimes dg_a\in V\otimes \aL(E,W)\).

Nous pourrions bien entendu dire que \( V\otimes \aL(E,W)\) est isomorphe à \( \aL(E,V\otimes W)\) et hop voila, on n'en parle plus. Ce serait passer sur deux points importants. D'abord est-ce que \( V\otimes \aL(E,W)\) est vraiment isomorphe à \( \aL(E,V\otimes W)\) ? Et ensuite, l'isomorphisme implique une utilisation du théorème \ref{THOooIHPIooIUyPaf} qui est tout sauf une trivialité.

Bref, fidèle au principe fridesque de ne pas cacher des difficultés techniques sous des abus de notations, nous allons écrire les choses explicitement.

\begin{lemma}
    Si \( E\), \( V\) et \( W\) sont de dimension finie, les applications
    \begin{equation}        \label{EQooVWXRooCesUqH}
        \begin{aligned}
            \psi\colon \aL(E,V)\otimes W&\to \aL(E,V\otimes W) \\
            f\otimes w&\mapsto \Big( u\mapsto f(u)\otimes w \Big) 
        \end{aligned}
    \end{equation}
    et
    \begin{equation}
        \begin{aligned}
            \varphi\colon V\otimes \aL(E,W)&\to \aL(E,V\otimes W) \\
            v\otimes g&\mapsto \big( a\mapsto v\otimes g(a) \big). 
        \end{aligned}
    \end{equation}
    sont des isomorphismes d'espaces vectoriels.
\end{lemma}
Dans le meilleur des mondes, ces applications devraient être affublés d'indices \( V\) et \( W\).

\begin{proof}
    Nous donnons des détails à propos de \( \psi\). Pour \( \varphi\) c'est la même chose.
    \begin{subproof}
        \item[Linéaire]
            La formule \eqref{EQooVWXRooCesUqH} définit \( \psi\) en particulier sur une base de \( \aL(E,V)\otimes W\) par la proposition \ref{PROPooTHDPooWgjUwk}\ref{ITEMooQCILooUncdGl}. Ce que signifie réellement la formule \eqref{EQooVWXRooCesUqH} est que \( \psi\) est ainsi définie sur la base et est prolongée par continuité.
        \item[Injective]
            Si pour un \( f\) et un \( w\) fixé nous avons \( \psi(f\otimes w)=0\), alors il y a deux cas : soit \( w=0\) soit \( w\neq0\). Dans le premier cas, \( f\otimes w=0\), et dans le second cas, nous remarquons que 
            \begin{equation}
                0=\psi(f\otimes w)(a)=f(a)\otimes w
            \end{equation}
            pour tout \( a\in E\). Cela implique \( f(a)=0\) pour tout \( a\) et donc \( f=0\), ce qui signifie que \( f\otimes w=0\).
        \item[Bijective]
            En utilisant la proposition \ref{PROPooTHDPooWgjUwk} et le lemme \ref{LEMooJXFIooKDzRWR}\ref{ITEMooPMLWooNbTyJI}, nous avons égalité des dimensions entre \( \aL(E,V)\otimes W\) et \( \aL(E,V\otimes W)\).

            Une application linéaire injective entre deux espaces vectoriels de même dimension (finie) est une bijection.
    \end{subproof}
\end{proof}

\begin{proposition}     \label{PROPooZOAFooRMeBgI}
    Soient des espaces vectoriels normés de dimension finie. Soient \( f\colon E\to V\) et \( g\colon E\to W\) des fonctions de classe \( C^1\). Alors \( f\tilde\otimes g\colon E\to V\otimes W\) est de classe \( C^1\) nous avons les formules
    \begin{equation}        \label{EQooSUSCooBhZXFC}
        d(f\tilde\otimes g)_a(u)=df_a(u)\otimes g(a)+f(a)\otimes dg_a(u)
    \end{equation}
    ainsi que
    \begin{equation}        \label{EQooOCEEooUrsIDd}
        d(f\tilde\otimes g)=\psi\circ(df\tilde\otimes g)+\varphi\circ(f\tilde\otimes dg).
    \end{equation}
\end{proposition}

\begin{proof}
    Nous commençons par prouver que \( f\tilde\otimes g\) est différentiable en injectant le candidat \eqref{EQooSUSCooBhZXFC} dans la définition. Au numérateur nous avons :
    \begin{equation}        \label{EQooOMXSooYsAiKh}
        (f\tilde\otimes g)(a+h)-(f\tilde\otimes g)(a)-df_a(h)\otimes g(a)-f(a)\otimes dg_a(h).
    \end{equation}
    Le lemme \ref{LEMooYQZZooVybqjK} assure qu'il existe une fonction \( \alpha\colon E\to V\) telle que \( \lim_{h\to 0} \alpha(h)/\| h \|\) et \( f(a+h)+f(a)+df_a(h)+\alpha(h)\). Même chose pour \( g\). Nous avons donc
    \begin{equation}
        (f\tilde\otimes g)(a+h)=f(a+h)\otimes g(a+h)=\big( f(a)+df_a(h)+\alpha(h) \big)\otimes \big( g(a)+dg_a(h)+\beta(h) \big)
    \end{equation}
    qui se développe en \( 9\) termes. En effectuant les différences dans \eqref{EQooOMXSooYsAiKh}, nous nous retrouvons avec un numérateur qui vaut
    \begin{equation}
        f(a)\otimes \beta(h)+df_a(h)\otimes dg_a(h)+df_a(h)\otimes \beta(h)+\alpha(h)\otimes g(a)+\alpha(h)\otimes dg_a(h)+\alpha(h)\otimes \beta(h).
    \end{equation}
    Nous pouvons prouver terme à terme qu'en divisant par \( \| h \|\) nous avons une limite qui vaut zéro. Par exemple,
    \begin{equation}
        \lim_{h\to 0} \frac{ f(a)\otimes \beta(h) }{ \| h \| }
    \end{equation}
    se calcule en prenant la norme du numérateur et en utilisant le lemme \ref{LEMooQPXHooJWfpmk} :
    \begin{equation}
        \frac{ \| f(a)\otimes \beta(h) \| }{ \| h \| }=\frac{ \| f(a) \|\| \beta(h) \| }{ \| h \| }\to 0.
    \end{equation}
    Tous les termes contenant \( \alpha(h)\) ou \( \beta(h)\) se traitent de la même manière. Le dernier terme à traiter est
    \begin{equation}
        \lim_{h\to 0} \frac{ df_a(h)\otimes dg_a(h) }{ \| h \| }.
    \end{equation}
    En prenant la norme du numérateur, en utilisant encore le lemme \ref{LEMooQPXHooJWfpmk} et en utilisant le lemme \ref{LEMooIBLEooLJczmu}, nous avons
    \begin{equation}
        \| df_a(h)\otimes dg_a(h) \|=\| df_a(h) \|\| dg_a(h) \|\leq \| df_a \|\| dg_a \|\| h \|^2,
    \end{equation}
    donc
    \begin{equation}
        \lim_{h\to 0} \frac{ df_a(h)\otimes dg_a(h) }{ \| h \| }=0.
    \end{equation}
    Notons que l'utilisation du lemme \ref{LEMooIBLEooLJczmu} requière que \( df_a\) soit continue, ce qui n'est pas évident en dimension infinie : une application linéaire n'est pas spécialement continue. C'est donc ici que nous utilisons le fait que \( E\), \( V\) et \( W\) sont de dimension finie\quext{Il y a surement moyen de paufiner, et d'affaiblir cette hypothèse, mais je ne me lance pas là-dedans.}.

    Ceci prouve que \( f\tilde\otimes g\) est différentiable et nous donne la formule \eqref{EQooSUSCooBhZXFC} pour appliquer sa différentielle à un élément de \( E\). La formule \eqref{EQooOCEEooUrsIDd} est un corolaire : elle se vérifie en l'appliquant à \( a\) puis à \( u\).
    
    Pour terminer nous devons prouver que \( d(f\tilde\otimes g)\) est continue. Vu que \( f\) et \( g\) sont de classe \( C^1\), les applications \( f\), \( g\), \( df\) et \( dg\) sont continues. Les applications \( \psi\) et \( \varphi\) sont également continues parce que linéaires sur des espaces de dimensioy finie. La proposition \ref{PROPooCRVXooEGxdZl} appliquée à \( df\) et \( g\) montre que \( df\tilde\otimes g\) est continue. La composition avec \( \psi\) qui est linéaire conserve la continuité.

    Dons le membre de droite de \eqref{EQooOCEEooUrsIDd} est continu et \( f\tilde\otimes g\) est a une différentielle continue. Elle est donc de classe \( C^1\).
\end{proof}

Il est temps de démontrer le truc difficile, à savoir que si \( f\) et \( g\) sont de classe \( C^p\), alors \( f\tilde\otimes g\) est également de classe \( C^p\). 

\begin{proposition}     \label{PROPooAWZFooMlhoCN}
    Nous applellons \( P_k\) la propriété suivante :
    \begin{quote}
        Pour tout \( p\geq k\), pour tout espaces vectoriels normés \( E\), \( V\), \( W\) de dimension finies et pour toutes applications \( f\colon E\to V\) et \( g\colon E\to W\) de classe \( C^k\), la fonction \( f\tilde\otimes g\) est de classe \( C^k\).
    \end{quote}
    \begin{enumerate}
        \item       \label{ITEMooDQRYooAEdxrW}
            La propriété \( P_k\) est vraie pour tout \( k\).
        \item       \label{ITEMooUUIFooGDyTMM}
            Si \( f\colon E\to V\) et \( g\colon E\to W\) sont de classe \( C^p\), alors \( f\tilde\otimes g\colon E\to V\otimes W\) est de classe \( C^p\).
    \end{enumerate}
\end{proposition}

\begin{proof}
    Le gros de la preuve est le point \ref{ITEMooDQRYooAEdxrW}. Le point \ref{ITEMooUUIFooGDyTMM} est alors une utilisation de la propriété \( P_p\) avec \( p=k\).

    Pour \( k=0\). Si \( f\) et \( g\) sont de classe \( C^p\) avec \( p\geq k\), alors \( f\) et \( g\) sont a fortiori continues. La proposition \ref{PROPooCRVXooEGxdZl} montre alors que \( f\tilde\otimes g\) est continue.

    Bien que ce ne soit pas tout à fait nécessaire, nous prouvons que \( P_1\) est également vraie avant de passer à la récurrence. Si \( f\) et \( g\) sont de classe \( C^p\) avec \( p\geq 1\), alors elles sont de classe \( C^1\) et la proposition \ref{PROPooZOAFooRMeBgI} s'applique : \( f\tilde\otimes g\) est de classe \( C^1\).

    Nous faisons la récurrence en supposant que \( P_k\) est vraie, et en prouvant que \( P_{k+1}\) est vraie. Soit \( p\geq k+1\) ainsi que des applications \( f\colon E\to V\) et \( g\colon E\to W\) de classe \( C^{k+1}\). La proposition \ref{PROPooZOAFooRMeBgI} dit que \( f\tilde\otimes g\) est de classe \( C^1\) et que
    \begin{equation}
        d(f\tilde\otimes g)=\psi\circ(df\tilde\otimes g)+\varphi\circ(f\tilde\otimes dg).
    \end{equation}
    À droite, \( df\) et \( g\) sont de classe \( C^k\) parce que \( f\) et \( g\) sont de classe \( C^{k+1}\). Donc \( df\tilde\otimes g\) est de classe \( C^k\) par l'hypothèse de récurrence appliquée aux espaces \( \aL(E,V)\) et \( W\). La proposition \ref{PROPooRCZOooSgvpSE} nous assure alors que \( \psi\circ(df\tilde\otimes g)\) est de classe \( C^k\) également.

    Nous avons prouvé que \( d(f\tilde\otimes g)\) est de classe \( C^k\), donc \( f\tilde\otimes g\) est de classe \( C^{k+1}\). Cela nous fait la récurrence.
\end{proof}

%---------------------------------------------------------------------------------------------------------------------------
\subsection{Formule des accroissements finis}
%---------------------------------------------------------------------------------------------------------------------------

\begin{proposition} \label{PropDQLhSoy}
    Soient \( a<b\) dans \( \eR\) et deux fonctions
    \begin{subequations}
        \begin{align}
            f\colon \mathopen[ a , b \mathclose]\to E\\
            g\colon \mathopen[ a , b \mathclose]\to \eR
        \end{align}
    \end{subequations}
    continues sur \( \mathopen[ a , b \mathclose]\) et dérivables sur \( \mathopen] a , b \mathclose[\). Si pour tout \( t\in\mathopen] a , b \mathclose[\) nous avons \( \| f'(t) \|\leq g'(t)\) alors
        \begin{equation}
            \| f(b)-f(a) \|\leq g(b)-g(a).
        \end{equation}
\end{proposition}

\begin{proof}
    Soit \( \epsilon>0\) et la fonction
    \begin{equation}
        \begin{aligned}
            \varphi_{\epsilon}\colon \mathopen[ a , b \mathclose]&\to \eR \\
            t&\mapsto \| f(t)-f(a) \|-g(t)-\epsilon t.
        \end{aligned}
    \end{equation}
    Cela est une fonction continue réelle à variable réelle. En particulier pour tout \( u\in\mathopen] a , b \mathclose[\) la fonction \( \varphi_{\epsilon}\) est continue sur le compact \( \mathopen[ u , b \mathclose]\) et donc y atteint son minimum en un certain point \( c\in\mathopen[ u , b \mathclose]\); c'est le bon vieux théorème de Weierstrass~\ref{ThoWeirstrassRn}. Nous commençons par montrer que pour tout \( u\), ledit minimum ne peut être que \( b\). Pour cela nous allons montrer que si \( t\in\mathopen[ u , b [\), alors \( \varphi_{\epsilon}(s)<\varphi_{\epsilon}(t)\) pour un certain \( s>t\). Par continuité si \( s\) est proche de \( t\) nous avons
        \begin{equation}
            \left\|  \frac{ f(s)-f(t) }{ s-t }  \right\|-\frac{ \epsilon }{2}<\| f'(t) \|<g'(t)+\frac{ \epsilon }{2}=\frac{ g(s)-g(t) }{ s-t }+\frac{ \epsilon }{2}.
        \end{equation}
        Ces inégalités proviennent de la limite
        \begin{equation}
            \lim_{s\to t} \frac{ f(s)-f(t) }{ s-t }=f'(t),
        \end{equation}
        donc si \( s\) et \( t\) sont proches,
        \begin{equation}
            \left\| \frac{ f(s)-f(t) }{ s-t }-f'(t) \right\|
        \end{equation}
        est petit. Si \( s>t\) nous pouvons oublier des valeurs absolues et transformer l'inégalité en
        \begin{equation}
            \| f(s)-f(t) \|<g(s)-g(t)+\epsilon(s-t).
        \end{equation}
        Utilisant cela et l'inégalité triangulaire,
        \begin{subequations}
            \begin{align}
                \varphi_{\epsilon}(s)&\leq\| f(s)-f(t) \|+\| f(t)-f(a) \|-g(s)-\epsilon s\\
                &\leq g(s)-g(t)+\epsilon s-\epsilon t+\| f(t)-f(a) \|-g(s)-\epsilon s\\
                &=\varphi_{\epsilon}(t).
            \end{align}
        \end{subequations}
        Donc nous avons bien \( \varphi_{\epsilon}(s)<\varphi_{\epsilon}(t)\) avec l'inégalité stricte. Par conséquent pour tout \( u\in\mathopen] a , b \mathclose[\) nous avons \( \varphi_{\epsilon}(b)<\varphi_{\epsilon}(u)\) et en prenant la limite \( u\to a\) nous avons
        \begin{equation}
            \varphi_{\epsilon}(b)\leq \varphi_{\epsilon}(a).
        \end{equation}
        Cette inégalité donne immédiatement
        \begin{equation}
            \| f(b)-f(a) \|\leq g(b)-g(a)+\epsilon(b-a)
        \end{equation}
         pour tout \( \epsilon>0\) et donc
         \begin{equation}
            \| f(b)-f(a) \|\leq g(b)-g(a).
         \end{equation}
\end{proof}

\begin{theorem}[Théorème des accroissements finis]\label{ThoNAKKght}
    Soient \( E\) et \( F\) des espaces vectoriels normés, \( U \) ouvert dans \( E\) et une application différentiable \( f\colon U\to F\). Pour tout segment \( \mathopen[ a , b \mathclose]\subset U\) nous avons
    \begin{equation}
        \| f(b)-f(a) \|\leq\left( \sup_{x\in\mathopen[ a , b \mathclose]}\| df_x \| \right)\| b-a \|.
    \end{equation}
\end{theorem}
\index{théorème!accroissements finis}


\begin{proof}
    Nous prenons les applications
    \begin{equation}
        \begin{aligned}
            k\colon \mathopen[ 0 , 1 \mathclose]&\to E \\
            t&\mapsto f\big( (1-t)a+tb \big)
        \end{aligned}
    \end{equation}
    et
    \begin{equation}
        \begin{aligned}
            g\colon \mathopen[ 0 , 1 \mathclose]&\to \eR \\
            t&\mapsto t\sup_{x\in\mathopen[ a , b \mathclose]}\| df_x \|\| b-a \|.
        \end{aligned}
    \end{equation}
    Pour tout \( t\) nous avons \( g'(t)=M\| b-a \|\) où il n'est besoin de dire ce qu'est \( M\). D'un autre côté nous avons aussi
    \begin{equation}
        \begin{aligned}[]
            k'(t)&=\lim_{\epsilon\to 0}\frac{ f\big( (1-t-\epsilon)a+(t+\epsilon)b \big)-f\big( (1-t)a+tb \big) }{ \epsilon }\\
            &=\Dsdd{ f\big( (1-t)a+tb+\epsilon(b-a) \big)  }{\epsilon}{0}\\
            &=df_{(1-t)a+tb}(b-a)
        \end{aligned}
    \end{equation}
    où nous avons utilisé l'hypothèse de différentiabilité de \( f\) sur \( \mathopen[ a , b \mathclose]\) et donc en \( (1-t)a+tb\). Nous avons donc
    \begin{equation}
        \| k'(t) \|\leq \| b-a \|\| df_{(1-t)a+tb} \|\leq M\| b-a \|=g'(t)
    \end{equation}
    La proposition~\ref{PropDQLhSoy} est donc utilisable et
    \begin{equation}
        \| k(1)-k(0) \|=g(1)-g(0),
    \end{equation}
    c'est-à-dire
    \begin{equation}
        \| f(b)-f(a) \|=M\| b-a \|
    \end{equation}
    comme il se doit.
\end{proof}

\begin{proposition} \label{ProFSjmBAt}
    Soient \( E\) et \( F\) des espaces vectoriels normés, \( U \) ouvert dans \( E\) et une application \( f\colon U\to F\). Soient \( a,b\in U\) tels que \( \mathopen[ a , b \mathclose]\subset U\). Nous posons \( u=(b-a)/\| b-a \|\) et nous supposons que pour tout \( x\in\mathopen[ a , b \mathclose]\), la dérivée directionnelle
    \begin{equation}
        \frac{ \partial f }{ \partial u }(x)=\Dsdd{ f(x+tu) }{t}{0}
    \end{equation}
    existe. Nous supposons de plus que \( \frac{ \partial f }{ \partial u }(x)\) est continue en \( x=a\). Alors
    \begin{equation}
        \| f(b)-f(a) \|\leq\left( \sup_{x\in\mathopen[ a , b \mathclose]}\| \frac{ \partial f }{ \partial u }(x) \| \right)\| b-a \|.
    \end{equation}
\end{proposition}

\begin{proof}
    Nous posons évidemment
    \begin{equation}
        M=\sup_{x\in\mathopen[ a , b \mathclose]}\| \frac{ \partial f }{ \partial u }(x) \|
    \end{equation}
    et nous considérons les fonctions
    \begin{equation}
        k(t)=f\big( (1-t)a+tb \big)
    \end{equation}
    et
    \begin{equation}
        g(t)=tM\| b-a \|.
    \end{equation}
    Pour alléger les notations nous posons \( x=(1-t)a+tb\) et nous calculons avec un petit changement de variables dans la limite :
    \begin{equation}
        k'(t)=\Dsdd{  f\big( x+\epsilon(b-a) \big)  }{\epsilon}{0}=\| b-a \|\Dsdd{ f\big( x+\frac{ \epsilon }{ \| b-a \| }(b-a) \big) }{\epsilon}{0}=\| b-a \|\frac{ \partial f }{ \partial u }(x),
    \end{equation}
    et donc encore une fois nous avons
    \begin{equation}
        \| k'(t) \|\leq g'(t),
    \end{equation}
    ce qui donne
    \begin{equation}
        \| k(1)-k(0) \|=g(1)-g(0),
    \end{equation}
    c'est-à-dire
    \begin{equation}
        \| f(b)-f(a) \|\leq \sup_{x\in\mathopen[ a , b \mathclose]}\| \frac{ \partial f }{ \partial u }(x) \|\| b-a \|.
    \end{equation}
\end{proof}

\begin{theorem} \label{ThoOYwdeVt}
    Soient \( E,V\) deux espaces vectoriels normés, une application \( f\colon E\to V\), un point \( a\in E\) tel que pour tout \( u\in E\), la dérivée
    \begin{equation}
        \Dsdd{ f(x+tu) }{t}{0}
    \end{equation}
    existe pour tout \( x\in B(a,r)\) et est continue (par rapport à \( x\)) en \( x=a\). Nous supposons de plus que
    \begin{equation}
        \frac{ \partial f }{ \partial u }(a)=0
    \end{equation}
    pour tout \( u\in E\). Alors \( f\) est différentiable en \( a\) et
    \begin{equation}
        df_a=0
    \end{equation}
\end{theorem}

\begin{proof}
    Soit \( \epsilon>0\). Pourvu que \( \| h \|\) soit assez petit pour que \( a+h\in B(a,r)\), la proposition~\ref{ProFSjmBAt} nous donne
    \begin{equation}
        \| f(a+h)-f(a) \|\leq \sup_{x\in\mathopen[ a , a+h \mathclose]}\| \frac{ \partial f }{ \partial u }(x) \|  |h |
    \end{equation}
    où \( u=h/\| h \|\). Par continuité de \( \partial_uf(x)\) en \( x=a\) et par le fait que cela vaut \( 0\) en \( x=a\), il existe un \( \delta>0\) tel que si \( \| h \|<\delta\) alors
    \begin{equation}
        \| \frac{ \partial f }{ \partial u }(a+h) \|\leq \epsilon.
    \end{equation}
    Pour de tels \( h\) nous avons
    \begin{equation}
        \| f(a+h)-f(a) \|\leq \epsilon\| h \|,
    \end{equation}
    ce qui prouve que l'application linéaire \( T(u)=0\) convient parfaitement pour faire fonctionner la définition \ref{DefDifferentiellePta}.
%
%    Nous ne supposons plus que les dérivées directionnelles de \( f\) sont nulles en \( x=a\). Alors nous posons, pour \( x\in U\),
%    \begin{equation}    \label{EqCUgHXHy}
%        g(x)=f(x)-\Dsdd{ f(a+s(x-a)) }{s}{0}.
%    \end{equation}
%    Le fait que cette fonction soit bien définie est encore un coup de hypothèses sur les dérivées directionnelles de \( f\) qui sont bien définies autour de \( a\). Cette nouvelle fonction \( g\) satisfait à \( \frac{ \partial g }{ \partial v }(a)=0\) pour tout \( v\in E\) parce que
%    \begin{subequations}
%        \begin{align}
%            \frac{ \partial g }{ \partial v }(a)&=\Dsdd{ g(a+tv) }{t}{0}\\
%            &=\Dsdd{ f(a+tv)-\Dsdd{ f\big( a+s(tv) \big) }{s}{0} }{t}{0}\\
%            &=\frac{ \partial f }{ \partial v }(a)-\Dsdd{ t\frac{ \partial f }{ \partial v }(a) }{t}{0}\\
%            &=0.
%        \end{align}
%    \end{subequations}
%    Pour la dérivée par rapport à \( s\) nous avons effectué le changement de variables \( s\to ts\), ce qui explique la présence d'un \( t\) en facteur. La fonction \( g\) est donc différentiable en \( a\).
%
%
% Position 229262367
    % Attention : ce qui suit est faux. Mais il y a peut-être moyen d'adapter.
%\item[Dérivées non nulles]
%
%    Nous allons montrer que la fonction
%    \begin{equation}
%        l(x)=\Dsdd{ f\big( a+s(x-a) \big) }{t}{0}
%    \end{equation}
%    est différentiable en \( x=a\), de différentielle \( T(u)=l(u+a)\). Cela fournira la différentiabilité de \( f\) parce que \eqref{EqCUgHXHy} donnerait alors \( f\) comme somme de deux fonctions différentiables.
%
%    En premier lieu nous devons montrer que \( T\) ainsi définie est linéaire.
%
%    Notre but est donc de prouver que
%    \begin{equation}
%        \lim_{h \to 0}\frac{ \| l(x+h)-l(x)-l(h) \| }{ \| h \| }=0.
%    \end{equation}
%    Un premier pas est de calculer
%    \begin{subequations}
%        \begin{align}
%            l(x+h)-l(x)-l(h)&=\lim_{s\to 0}\frac{ f\big( s(x+h) \big)-f(0)-f(sx)+f(0)-f(sh)+f(0) }{ s }\\
%            &=\lim_{s\to 0}\frac{ f\big( s(x+h) \big)-f(sx)-f(sh)+f(0) }{ s }.
%        \end{align}
%    \end{subequations}
%    Ensuite nous étudions le numérateur en utilisant la proposition~\ref{ProFSjmBAt}:
%    \begin{subequations}
%        \begin{align}
%            \| f\big( s(x+h) \big)-f(sx)-f(sh)+f(0) \|&\leq  \| f\big( s(x+h) \big)-f(sx)\| + \|f(sh)-f(0) \|  \\
%            &\leq \sup_{z\in\mathopen[ sx , sx+sh \mathclose]}\| \frac{ \partial f }{ \partial h }(z) \|\| sh \|\\
%            &\quad +\sup_{z\in\mathopen[ 0 , sh \mathclose]}\| \frac{ \partial f }{ \partial h }(z) \|\| sh \|.
%        \end{align}
%    \end{subequations}
%    La division par \( s\) se passe bien et nous avons
%    \begin{subequations}
%        \begin{align}
%            \| l(x+h)-l(x)-l(h) \|&\leq \lim_{s\to 0}  \sup_{z\in\mathopen[ sx , sx+sh \mathclose]}\| \frac{ \partial f }{ \partial h }(z) \|\| h \|+ \sup_{z\in\mathopen[ 0 , sh \mathclose]}\| \frac{ \partial f }{ \partial h }(z) \|\| h \|\\
%            &=2\| h \|\| \frac{ \partial f }{ \partial h }(0) \|        \label{SubeqVMMoSDH}\\
%            &=2\| h \|^2\| \frac{ \partial f }{ \partial u }(0) \|
%        \end{align}
%    \end{subequations}
%    où nous avons posé \( u=h/\| h \|\). Pour l'égalité \eqref{SubeqVMMoSDH} nous avons utilisé la continuité de \( \frac{ \partial f }{ \partial h }(z)\) en \( z=0\). Du coup
%    \begin{equation}
%        \lim_{y\to 0} \frac{ \| f(x+h)-f(x)-f(h) \| }{ \| h \| }=\lim_{h\to 0} 2\| h \|\| \frac{ \partial f }{ \partial u }(0) \|=0.
%    \end{equation}
%    Cela prouve que \( l\) est bien différentiable en \( x=0\).
%
%    \end{subproof}
%
\end{proof}

%--------------------------------------------------------------------------------------------------------------------------- 
\subsection{Applications multilinéaires}
%---------------------------------------------------------------------------------------------------------------------------

Nous avons déjà parlé d'applications multilinéaires dans la définition \ref{DefFRHooKnPCT}.

\begin{lemma}[Leibnitz pour les formes bilinéaires\cite{SNPdukn}]\label{LemFRdNDCd}
    Si \( B\colon E\times F\to G\) est bilinéaire et continue, elle est \(  C^{\infty}\) et
    \begin{equation}    \label{EqXYJgDBt}
        dB_{(x,y)}(u,v)=B(x,v)+B(u,y).
    \end{equation}
\end{lemma}

\begin{proof}
    D'abord le membre de droite de \eqref{EqXYJgDBt} est une application linéaire et continue, donc c'est un bon candidat à être différentielle. Nous allons prouver que ça l'est, ce qui prouvera la différentiabilité de \( B\). Avec ce candidat, le numérateur de la définition \eqref{DefDifferentiellePta} s'écrit dans notre cas
    \begin{equation}
        B\big( (x,y)+(u,v) \big)-B(x,y)-B(x,v)-B(u,y)=B(u,v).
    \end{equation}
    Il reste à voir que
    \begin{equation}
        \lim_{ (u,v)\to (0,0) } \frac{ B(u,v) }{ \| (u,v) \| }=0
    \end{equation}
    Par l'équation \eqref{EqYLnbRbC} nous avons
    \begin{equation}
        \frac{ \| B(u,v) \| }{ \| (u,v) \| }\leq \frac{ \| B \|\| u \|\| v \| }{ \| u \| }=\| B \|\| v \|
    \end{equation}
    parce que \( \| (u,v) \|\geq \| u \|\). À partir de là il est maintenant clair que
    \begin{equation}
        \lim_{(u,v)\to (0,0)}\frac{ \| B(u,v) \| }{ \| (u,v) \| }=0,
    \end{equation}
    ce qu'il fallait.
\end{proof}

\begin{proposition}[Règle de Leibnitz\cite{SNPdukn}]
    Soient \( E,F_1,F_2\) des espaces vectoriels normés, \( U\) ouvert dans \( E\) et des applications de classe \( C^r\) (\( r\geq 1\))
    \begin{subequations}
        \begin{align}
            f_1\colon U\to F_1\\
            f_2\colon U\to F_2\\
        \end{align}
    \end{subequations}
    et \( B\in\cL(F_1\times F_2,G)\). Alors l'application
    \begin{equation}
        \begin{aligned}
            \varphi\colon U&\to G \\
            x&\mapsto B\big( f_1(x),f_2(x) \big)
        \end{aligned}
    \end{equation}
    est de classe \( C^r\) et
    \begin{equation}    \label{EqMNGBXWc}
        d\varphi_x(u)=\varphi\big( (df_1)_x(u),f_2(x) \big)+\varphi\big( f_1(x),(df_2)_x(u) \big).
    \end{equation}
\end{proposition}
\index{Leibnitz!applications entre espaces vectoriels normés}

\begin{proof}
    Par hypothèse \( B\) est continue (c'est la définition de l'espace \( \cL\)), et donc \(  C^{\infty}\) par le lemme~\ref{LemFRdNDCd}. Par ailleurs la fonction \( f_1\times f_2\) est de classe \( C^r\) parce que \( f_1\) et \( f_2\) le sont et parce que la proposition~\ref{PropOYtgIua} le dit. L'application composée \( B\circ(f_1\times f_2)\) est donc également de classe \( C^r\) par le théorème~\ref{ThoAGXGuEt}.

    Il ne nous reste donc qu'à prouver la formule~\ref{EqMNGBXWc}. En utilisant la différentielle du produit cartésien\footnote{Proposition~\ref{PropOYtgIua}.} nous avons
    \begin{equation}
        f\big( B\circ(f_1\times f_2) \big)_x(h)=dB_{(f_1\times f_2)(x)}\big( (df_1)_x(h),(df_2)_x(h) \big).
    \end{equation}
    Nous développons cela en utilisant le lemme~\ref{LemFRdNDCd} :
    \begin{subequations}
        \begin{align}
        d\big( B\circ(f_1\times f_2) \big)_x(h)&=dB_{\big( f_1(x),f_2(x) \big)}\big( (df_1)_x(h),(df_2)_x(h) \big)\\
        &=B\big( f_1(x),(df_2)_x(h) \big)+B\big( (df_1)_x(h),f_2(x) \big),
        \end{align}
    \end{subequations}
    comme souhaité.
\end{proof}

%---------------------------------------------------------------------------------------------------------------------------
\subsection{Différentielle partielle}
%---------------------------------------------------------------------------------------------------------------------------

\begin{definition}[Différentielle partielle]    \label{VJM_CtSKT}
    Soient \( E\), \( F\) et \( G\) des espaces vectoriels normés et une fonction \( f\colon E\times F\to G\). Nous définissons sa \defe{différentielle partielle}{différentielle!partielle} sur l'espace \( E\) par
    \begin{equation}
        \begin{aligned}
            d_1f_{(x_0,y_0)}\colon E&\to G \\
            u&\mapsto \Dsdd{ f(x_0+tu,y_0 }{t}{0} .
        \end{aligned}
    \end{equation}
    La différentielle \( d_2\) se définit de la même façon.
\end{definition}

\begin{proposition}[\cite{SNPdukn}] \label{PropLDN_nHWDF}
    Soient \( E_1\), \( E_2\) et \( F\) des espaces vectoriels normés, soit un ouvert \( U\subset E_1\times E_2\) et une fonction \( f\colon U\to F\).
    \begin{enumerate}
        \item   \label{ItemRDD_oPmXVi}
            Si \( f\) est différentiable alors les différentielles partielles existent et
            \begin{subequations}
                \begin{align}
                    d_1f_{(x_0,y_0)}(u)=df_{(x_0,y_0)}(u,0)\\
                    d_2f_{(x_0,y_0)}(v)=df_{(x_0,y_0)}(0,v)
                \end{align}
            \end{subequations}
            où \( u\in E_1\) et \( v\in E_2\).
        \item
            Si \( f\) est différentiable alors
            \begin{equation}
                df_{(x_0,y_0)}(u,v)=d_1f_{(x_,y_0)}(u)+d_2f_{(x_0,y_0)}(v).
            \end{equation}
    \end{enumerate}
\end{proposition}

\begin{proof}
    Nous posons \( \alpha=(x_0,y_0)\in U\) et
    \begin{equation}
        \begin{aligned}
            j_{\alpha}^{(1)}\colon E_1&\to E_1\times E_2 \\
            x&\mapsto (x,y_0).
        \end{aligned}
    \end{equation}
    C'est une fonction de classe \(  C^{\infty}\) et
    \begin{equation}
        (dj_{\alpha}^{(1)})_{x_0}(u)=\Dsdd{ j_{\alpha}^{(1)}(x_0+tu) }{t}{0}=\Dsdd{ (x_0+tu,y_0) }{t}{0}=(u,0).
    \end{equation}
    D'autre part
    \begin{subequations}
        \begin{align}
            (d_1f)_{\alpha}(u)&=\Dsdd{ f(x_0+tu,y_0) }{t}{0}\\
            &=\Dsdd{ (f\circ j_{\alpha}^{(1)})(x_0+tu) }{t}{0}\\
            &=\big( d(f\circ j_{\alpha}^{(1)}) \big)_{x_0}(u).
        \end{align}
    \end{subequations}
    À ce moment nous utilisons la règle des différentielles composées~\ref{ThoAGXGuEt} pour dire que
    \begin{equation}
        (d_1f)_{\alpha}(u)=df_{j_{\alpha}^{(1)}(x_0)}\circ (dj_{\alpha}^{(1)})_{x_0}(u)=df_{\alpha}(u,0).
    \end{equation}
    Voila qui prouve déjà le point~\ref{ItemRDD_oPmXVi}.

    Pour la suite nous considérons les fonctions
    \begin{equation}
        \begin{aligned}[]
            P_1(x,y)&=x,&&&J_1(u)&=(u,0),\\
            P_2(x,y)&=y,&&&J_2(v)&=(0,v)
        \end{aligned}
    \end{equation}
    et nous avons l'égalité évidente
    \begin{equation}
        J_1\circ P_1+J_2\circ P_2=\mtu
    \end{equation}
    sur \( E_1\times E_2\). En appliquant \( df_{\alpha}\) à cette dernière égalité, en appliquant à \( (u,v)\) et en utilisant la linéarité de \( df_{\alpha}\) nous trouvons
    \begin{subequations}
        \begin{align}
            df_{\alpha}(u,v)&=df_{\alpha}\big( (J_1\circ P_1)(u,v) \big)+df_{\alpha}\big( (J_2\circ P_2)(u,v) \big)\\
            &=df_{\alpha}(u,0)+df_{\alpha}(0,v)\\
            &=(d_1f)_{\alpha}(u)+(d_2f)_{\alpha}(v)
        \end{align}
    \end{subequations}
    où nous avons utilisé le point~\ref{ItemRDD_oPmXVi} pour la dernière égalité.
\end{proof}

%---------------------------------------------------------------------------------------------------------------------------
\subsection{L'inverse, sa différentielle}
%---------------------------------------------------------------------------------------------------------------------------

Si \( E\) est un espace de Banach, nous sommes intéressés à l'espace \( \GL(E)\) des endomorphismes inversibles de \( E\) sur \( E\). Cet ensemble est métrique par la formule usuelle
\begin{equation}
    \| T \|=\sup_{\| x \|=1}\| T(x) \|_E.
\end{equation}

\begin{proposition}[Thème~\ref{THEMEooPQKDooTAVKFH}]     \label{PropQAjqUNp}
    Soit \( E\) un espace de Banach (espace vectoriel normé complet). Si \( A\) est un endomorphisme de \( E\) satisfaisant  \( \| A \|<1\) pour la norme opérateur, alors \( (\mtu-A)\) est inversible et son inverse est donné par
    \begin{equation}
        (\mtu-A)^{-1}=\sum_{k=0}^{\infty}A^k.
    \end{equation}
\end{proposition}
\index{série!donnant \( (1-A)^{-1}\)}

\begin{proof}
    Étant donné que la norme opérateur est une norme algébrique (lemme~\ref{LEMooFITMooBBBWGI}), nous avons \( \| A^k \|\leq \| A \|^k\). Par conséquent la série \( \| A^k \|\) est majorée par la série géométrique qui converge\footnote{Voir l'exemple \ref{ExZMhWtJS}.}. Par conséquent \( \sum_{k}A^k\) est une série absolument convergente et donc convergente par la proposition~\ref{PropAKCusNM} et le fait que \( \aL(E)\) est complet (proposition~\ref{LemCAIPooPMNbXg}).

    Montrons à présent que la somme est l'inverse de \( \mtu-A\) en utilisant le produit terme à terme autorisé par la proposition~\ref{PropQXqEPuG} :
    \begin{equation}
        \sum_{k=0}^nA^k(\mtu-A)=\sum_{k=0}^n(A^k-A^{k+1})=\mtu-A^{n+1}.
    \end{equation}
    Par conséquent
    \begin{equation}
        \| \mtu-\sum_{k=0}^nA^k(\mtu-A) \|=\| A^{n+1} \|\leq \| A \|^{n+1}\to 0.
    \end{equation}
\end{proof}

\begin{theorem}[Inverse dans \( \GL(E)\)\cite{laudenbach2000calcul,SNPdukn}]    \label{ThoCINVBTJ}
    Soient \( E\) et \( F\) des espaces vectoriels normés.
    \begin{enumerate}
        \item
        L'ensemble \( \GL(E)\) est ouvert dans \( \End(E)\).
    \item
        L'application inverse
    \begin{equation}
        \begin{aligned}
        i\colon \GL(E,F)&\to \GL(F,E) \\
        u&\mapsto u^{-1}
        \end{aligned}
    \end{equation}
    est de classe \( C^{\infty}\) et
    \begin{equation}
        di_{u_0}(h)=-u_0^{-1}\circ h\circ u_0^{-1}
    \end{equation}
    pour tout \( h\in\End(E)\)
    \end{enumerate}
\end{theorem}
\index{différentielle!de $u\mapsto u^{-1}$}

\begin{proof}
Nous supposons que \( \GL(E,F)\) n'est pas vide, sinon ce n'est pas du jeu.
        \begin{subproof}

        \item[Cas de dimension finie]

            Si la dimension de \( E\) et \( F\) est finie, elles doivent être égales, sinon il n'y a pas de fonctions inversibles \( E\to F\). L'ensemble \( \GL(E,F)\) est donc naturellement \( \GL(n,\eR)\). Un élément de \( \eM(n,\eR)\) est dans \( \GL(n,\eR)\) si et seulement si son déterminant est non nul. Le déterminant étant une fonction continue (polynomiale) en les entrées de la matrice, l'ensemble \( \GL(n,\eR)\) est ouvert dans \( \eM(n,\eR)\).

            Même idée pour la régularité de la fonction \( i\colon \GL(n,\eR)\to \GL(n,\eR)\), \( X\mapsto X^{-1}\). Les entrées de \( X^{-1}\) sont les cofacteurs de \( X\) divisé par \( \det(X)\), et donc des polynômes en les entrées de \( X\) divisés par un polynôme qui ne s'annule pas sur \( \GL(n,\eR)\), et donc sur un ouvert autour de \( X\) et de \( X^{-1}\). Bref, tout est \(  C^{\infty}\).

            Le reste de la preuve parle de la dimension infinie.

        \item[Ouvert autour de l'identité]

        Nous commençons par prouver que \( B(\mtu,1)\subset \GL(E)\). Pour cela il suffit de remarquer que si \( \| u \|<1\) alors le lemme~\ref{PropQAjqUNp} nous donne un inverse de \( (1+u)\) en la personne de \( \sum_{k=0}^{\infty}(-u)^k\).

    \item[Ouvert en général]

        Soit maintenant \( u_0\in\GL(E)\). Si \( \| u \|<\frac{1}{ \| u_0^{-1} \| }\) alors \( \| u_0^{-1}u \|<1\), ce qui signifie que
        \begin{equation}
            \mtu+u_0^{-1}u
        \end{equation}
    est inversible. Mais \( u_0+u=u_0(\mtu+u_0^{-1}u)\), donc \( u_0+u\in\GL(E)\) ce qui signifie que
    \begin{equation}
    B\left( u_0,\frac{1}{ \| u_0^{-1} \| } \right)\subset \GL(E).
    \end{equation}

    \item[Différentielle en l'identité]

    Nous commençons par prouver que \( di_{\mtu}(u)=-u\). Pour cela nous posons
    \begin{equation}
        \alpha(h)=\sum_{k=2}^{\infty}(-1)^kh^k
    \end{equation}
    et nous calculons
    \begin{equation}
    di_{\mtu}(u)=\Dsdd{ i(\mtu+tu) }{t}{0}=\Dsdd{ \mtu-tu+\alpha(tu) }{t}{0}.
    \end{equation}
    Il suffit de prouver que \( \Dsdd{ \alpha(tu) }{t}{0}=0\) pour conclure que \( di_{\mtu}(u)=-u\). Pour cela, nous remarquons que \( \alpha(0)=0\) et donc que
    \begin{subequations}
        \begin{align}
        \Dsdd{ \alpha(tu) }{t}{0}&=\lim_{t\to 0} \frac{ \alpha(tu)-\alpha(0) }{ t }\\
        &=\lim_{t\to 0} \sum_{k=2}^{\infty}(-1)^k\frac{ (tu)^k }{ t }\\
        &=-\lim_{t\to 0} u\sum_{k=1}^{\infty}(-1)^kt^ku^k.
        \end{align}
    \end{subequations}
    La norme de ce qui est dans la limite est majorée par
    \begin{equation}
    \| u \|\sum_{k=1}^{\infty}\| tu \|^k=\| u \|\left( \frac{1}{ 1-\| tu \| }-1 \right),
    \end{equation}
    et cela tend vers zéro lorsque \( t\to\infty\). Nous avons utilisé la somme~\ref{EqRGkBhrX} de la série géométrique. Nous avons bien prouvé que \( di_{\mtu}(u)=-u\).

    \item[Différentielle en général]
    Soit maintenant \( u_0\in\GL(E)\) et \( h\in\End(E)\) tel que \( u_0+h\in \GL(E)\); par le premier point, il suffit de prendre \( \| h \|\) suffisamment petit. Vu que \( u_0+h=u_0(\mtu+u_0^{-1}h)\) nous avons
    \begin{equation}
        (u_0+h)^{-1}=(\mtu+u_0^{-1}h)^{-1}u_0^{-1}.
    \end{equation}
    Nous pouvons donc calculer
    \begin{equation}
        (u_0+h)^{-1}=\big( \mtu-u_0^{-1}h+\alpha(u_0^{-1}h) \big)u_0^{-1}=u_0^{-1}-u_0^{-1}hu_0^{-1}+\alpha(u_0^{-1}h)u_0^{-1},
    \end{equation}
    et ensuite
    \begin{equation}
        di_{u_0}(h)=\Dsdd{ i(u_0+th) }{t}{0}=\Dsdd{ u_0^{-1}-tu_0^{-1}hu_0^{-1}+\alpha(tu_0^{-1}h)u_0^{-1} }{t}{0},
    \end{equation}
    mais nous avons déjà vu que
    \begin{equation}
        \Dsdd{ \alpha(th) }{t}{0}=0,
    \end{equation}
    donc
    \begin{equation}
        di_{u_0}(h)=-u_0^{-1}hu_0^{-1}
    \end{equation}
    Cela donne la différentielle de l'application inverse.

    \item[Continuité de l'inverse]

        L'application \( i\) est continue parce que différentiable.
    \item[L'inverse est \(  C^{\infty}\)]

        Nous allons écrire la fonction inverse comme une composée. Soient les applications
        \begin{equation}
            \begin{aligned}
                B\colon \cL(F,E)\times \cL(F,E)&\to \cL\big( \cL(E,F),\cL(F,E) \big) \\
                B(\psi_1,\psi_2)(A)&= -\psi_1\circ A\circ\psi_2
            \end{aligned}
        \end{equation}
        et
        \begin{equation}
            \begin{aligned}
                \Delta\colon \cL(F,E)&\to \cL(F,E)\times \cL(F,E) \\
                \varphi&\mapsto (\varphi,\varphi)
            \end{aligned}
        \end{equation}
        Nous avons alors
        \begin{equation}
            di=B\circ\Delta\circ i.
        \end{equation}
        L'application \( \Delta\) est de classe \(  C^{\infty}\). Nous devons voir que \( B\) l'est aussi. Pour le voir nous commençons par prouver qu'elle est bornée :
        \begin{equation}
            \begin{aligned}[]
                \| B \|&=\sup_{\| \psi_1 \|,\| \psi_2 \|=1}\| B(\psi_1,\psi_2) \|_{\aL\big( L(E,F),L(F,E) \big)}\\
                &=\sup_{  \| \psi_1 \|,\| \psi_2 \|=1 }\sup_{\| A \|=1}\| \psi_1\circ A\circ\psi_2 \|_{L(F,E)}\\
                &\leq \sup_{\| \psi_1 \|,\| \psi_2 \|=1}\sup_{\| A \|=1}\| \psi_1 \|\| A \|\| \psi_2 \|\\
                &\leq 1.
            \end{aligned}
        \end{equation}
        Donc \( B\) est bien bornée et par conséquent continue. Une application bilinéaire continue est \(  C^{\infty}\) par le lemme~\ref{LemFRdNDCd}. La décomposition \( di=B\circ \Delta\circ i\) nous donne donc que \( i\in C^{\infty}\) dès que \( i\) est continue, ce que nous avions déjà montré.
        \end{subproof}
\end{proof}



%+++++++++++++++++++++++++++++++++++++++++++++++++++++++++++++++++++++++++++++++++++++++++++++++++++++++++++++++++++++++++++
\section{Exponentielle de matrice}
%+++++++++++++++++++++++++++++++++++++++++++++++++++++++++++++++++++++++++++++++++++++++++++++++++++++++++++++++++++++++++++
\label{secAOnIwQM}

\begin{proposition}     \label{PropPEDSooAvSXmY}
    Soit \( V\) un espace vectoriel de dimension finie et \( A\in\End(V)\). La série
    \begin{equation}
        \exp(A)=\mtu+A+\frac{ A^2 }{ 2 }+\frac{ A^3 }{ 3 }+\ldots =\sum_{k=1}^{\infty}\frac{ A^k }{ k! }.
    \end{equation}
    converge normalement dans \( \big( \End(V),\| . \|_{op} \big)\).  L'\defe{exponentielle}{exponentielle!de matrice} de la matrice \( A\) est cette matrice.
\end{proposition}

\begin{proof}
    Vu que la norme opérateur est une norme d'algèbre par le lemme~\ref{LEMooFITMooBBBWGI}, nous avons pour tout \( k\) la majoration \( \| A^k \|\leq \| A \|^k\). Nous avons donc
    \begin{equation}
        \sum_{k=0}^{\infty}\frac{ \| A^k \| }{ k! }\leq \sum_k\frac{ \| A \|^k }{ k! }.
    \end{equation}
    La dernière somme converge en vertu de la convergence de la série exponentielle donnée en exemple~\ref{ExIJMHooOEUKfj}.
\end{proof}

Étant donné que c'est une limite, il y a une question de convergence et donc de topologie. C'est pour cela que nous ne pouvions pas introduire l'exponentielle de matrice avant d'avoir introduit la norme des matrices. La convergence de la série pour toute matrice sera prouvée au passage dans la proposition~\ref{PropFMqsIE}.


La fonction exponentielle \(  x\mapsto e^{x}\) n'est pas un polynôme en \( x\), mais nous avons le résultat marrant suivant.
\begin{proposition} \label{PropFMqsIE}
    Si \( u\) est un endomorphisme, alors \( \exp(u)\) est un polynôme en \( u\)\footnote{Nan, mais j'te jure : \( \exp\) n'est pas un polynôme, mais $\exp(u)$ est un polynôme de \( u\).}.
\end{proposition}

\begin{proof}
    Nous considérons l'application
    \begin{equation}
        \begin{aligned}
            \varphi_u\colon \eK[X]&\to \End(E) \\
            P&\mapsto P(u)
        \end{aligned}
    \end{equation}
    Étant donné que l'image de \( \varphi_u\) est un fermé dans \( \End(E)\), il suffit de montrer que la série
    \begin{equation}
        \sum_{k=0}^{\infty}\frac{ \varphi_u(X)^k }{ k! }
    \end{equation}
    converge dans \( \End(E)\) pour qu'elle converge dans \( \Image(\varphi_u)\). Pour ce faire nous nous rappelons de la norme opérateur\footnote{Définition~\ref{DefNFYUooBZCPTr}.} et de la propriété fondamentale \( \| A^k \|\leq \| A \|^k\). En notant \( A=\varphi_u(X)\),
    \begin{equation}
        \left\| \sum_{k=n}^m\frac{ A^k }{ k! } \right\|\leq \sum_{k=n}^m\frac{ \| A^k \| }{ k! }\leq \sum_{k=n}^m\frac{ \| A \|^k }{ k! },
    \end{equation}
    ce qui est une morceau du développement de \(  e^{\| A \|}\). La limite \( n\to\infty\) est donc zéro par la convergence de l'exponentielle réelle. La suite des sommes partielles de  $e^{A}$ est donc de Cauchy. La série converge donc parce que nous sommes dans un espace vectoriel réel de dimension finie (\( \End(E)\)).
\end{proof}
% TODO : et tant qu'on y est, justifier la convergence de la série de l'exponentielle réelle.

\begin{normaltext}
    Pourquoi \( \exp(u)\) est-il un polynôme d'endomorphisme alors que \( \exp\) n'est pas un polynôme ? Lorsque nous disons que la fonction \( x\mapsto \exp(x)\) n'est pas un polynôme, nous sommes en train de localiser la fonction \( \exp\) à l'intérieur de l'espace de toutes les fonctions \( \eR\to \eR\), c'est-à-dire à l'intérieur d'un espace de dimension infinie. Au contraire lorsqu'on parle de \( \exp(u)\) et qu'on le compare aux endomorphismes \( P(u)\), nous sommes en train de repérer \( \exp(u)\) à l'intérieur de l'espace des matrices qui est de dimension finie. Il n'est donc pas étonnant que l'on parvienne moins à faire la distinction.

    Si par contre nous considérons \( \exp\) en tant qu'application \( \exp\colon \End(E)\to \End(E)\), ce n'est pas un polynôme.

    Si \( u\) et \( v\) sont des endomorphismes, nous aurons des polynômes \( P\) et \( Q\) tels que \( e^u=P(u)\) et \( e^v=Q(v)\); mais nous n'aurons en général évidemment pas \( P=Q\). En cela, \( \exp\) n'est pas un polynôme.
\end{normaltext}

%+++++++++++++++++++++++++++++++++++++++++++++++++++++++++++++++++++++++++++++++++++++++++++++++++++++++++++++++++++++++++++
\section{Espace dual}
%+++++++++++++++++++++++++++++++++++++++++++++++++++++++++++++++++++++++++++++++++++++++++++++++++++++++++++++++++++++++++++
\label{SECooKOJNooQVawFY}

\begin{definition}
    Soit un espace vectoriel normé \( (V,\| . \|)\) sur le corps \( \eC\) ou \( \eR\) (que nous nommons \( \eK\)). Son \defe{dual topologique}{dual topologique}, noté \( V'\) est l'ensemble des applications linéaires continues \( V\to \eK\).
\end{definition}

%---------------------------------------------------------------------------------------------------------------------------
\subsection{Topologies}
%---------------------------------------------------------------------------------------------------------------------------

Il est possible de mettre sur \( V'\) (au moins) deux topologies distinctes. La première est la topologie de la norme opérateur; rien de nouveau pour elle. La seconde est la topologie \( *\)-faible dont nous avons déjà un peu parlé dans la définition~\ref{DefHUelCDD}.

En termes de notations, nous allons noter les semi-normes de la topologie faible par
\begin{equation}
    p_x(\varphi)=| \varphi(x) |
\end{equation}
pour \( x\in V\) et \( \varphi\in V'\). À droite, les barres dénotent soit la valeur absolue (si \( \eK=\eR\)), soit le module (si \( \eK=\eC\)).

\begin{lemma}       \label{LEMooFMAUooQBIeTh}
    Soit \( \varphi\in V'\) et \( x\in V\). Alors
    \begin{equation}
        p_x(\varphi)\leq\frac{ \| \varphi \| }{ \| x \| }.
    \end{equation}
    Si \( \varphi_0\in V'\), si \( r>0\) et si \( x\in V\) nous avons aussi :
    \begin{equation}
        B(\varphi_0,r)\subset B_x(\varphi_0,\frac{ r }{ \| x \| }).
    \end{equation}
\end{lemma}

\begin{proof}
    En posant \( x'=x/\| x \|\) nous avons
    \begin{equation}
        p_x(\varphi)=| \varphi(x) |=\frac{1}{ \| x \| }| \varphi(x') |\leq \frac{1}{ \| x \| }\| \varphi \|.
    \end{equation}

    En ce qui concerne la seconde affirmation, si \( \varphi\in B(\varphi_0,r)\) alors en notant \( x'=x/\| x \|\) nous avons :
    \begin{equation}
        p_x(\varphi_0-\varphi)=| \varphi_0(x)-\varphi(x) |=\frac{1}{ \| x \| }| \varphi_0(x')-\varphi(x') |\leq\frac{1}{ \| x \| }\|\varphi_0-\varphi  \|\leq \frac{ r }{ \| x \| }.
    \end{equation}
    Donc \( \varphi\in B_x\big( \varphi_0,\frac{ r }{ \| x \| } \big)\).
\end{proof}

\begin{proposition}
    En ce qui concerne la convergence d'une suite \( (\varphi_k)\) dans \( V'\) mais si elle vérifie
    \begin{equation}
        \varphi_k\stackrel{\| . \|}{\longrightarrow}\varphi
    \end{equation}
    alors
    \begin{equation}
        \varphi_k\stackrel{*}{\longrightarrow}\varphi.
    \end{equation}
\end{proposition}

\begin{proof}
    Soit une suite \( (\varphi_k)\) dans \( V'\), convergente vers \( \varphi\) pour la topologie de la norme.  Soit \( x\in V\), et \( x'=x/\| x \|\). Nous avons
    \begin{equation}
        p_x(\varphi_k-\varphi)=\frac{1}{ \| x \| }| \varphi_k(x')-\varphi(x) |\leq\frac{1}{ \| x \| }\| \varphi_k-\varphi \|\to 0.
    \end{equation}
\end{proof}

\begin{lemma}       \label{LEMooEAVEooAFveHn}
    La translation dans \( V'\) est une opération continue pour la topologie de la norme opérateur et pour celle de la topologie \( *\).
\end{lemma}

\begin{proof}
    Soit une suite \( \varphi_k\) tendant vers \( 0\); nous devons prouver que \( \tau_{\sigma}(\varphi_k)\to \tau_{\sigma}(0)=\sigma\). Et ce, pour chacune des deux topologies.

    \begin{subproof}
        \item[Norme opérateur]

            L'hypothèse \( \varphi_k\stackrel{\| . \|}{\longrightarrow} 0\) signifie que \( \| \varphi_k \|\to 0\), c'est-à-dire que
            \begin{equation}
                \sup_{\| v \|=1}| \varphi_k(v) |\to 0.
            \end{equation}
            Nous avons alors
            \begin{equation}
                \| \tau_{\sigma}(\varphi_k)-\sigma \|=\sup_{\| v \|=1}| \tau_{\sigma}(\varphi_k)v-\sigma(v) |=\sup_{\| v \|=1}| \varphi_k(v) |\to 0.
            \end{equation}
            Donc d'accord pour \( \tau_{\sigma}(\varphi)\to \sigma\).

        \item[Topologie $*$]

            Nous supposons maintenant que \( \varphi_k\stackrel{*}{\longrightarrow}0\). Pour tout \( v\in V\) nous avons
            \begin{equation}
                p_v\big( \tau_{\sigma}(\varphi_k)-\sigma \big)=\big| \tau_{\sigma}(\varphi_k)v-\sigma(v) \big|=| \varphi_k(v) |=p_v(\varphi_k).
            \end{equation}
            Mais par hypothèse, \( p_v(\varphi_k)\to 0\).
    \end{subproof}
\end{proof}

Pour la suite, nous allons préfixer par \( N\) les concepts liés à la topologie de \( V'\) associée à la norme opérateur et par \( *\), les concepts de la topologie \( *\).

\begin{proposition}     \label{PROPooFGXAooFRWweD}
    Soit un espace vectoriel normé \( V\). Un \( *\)-ouvert et toujours un \( N\)-ouvert.
\end{proposition}

\begin{proof}
    Soit un \( *\)-ouvert \( \mO\) de \( V'\). Il existe donc \( x\in V\) et \( r>0\) tels que \( B_x(\varphi,r)\subset \mO\). Nous avons alors, en utilisant le lemme~\ref{LEMooFMAUooQBIeTh},
    \begin{equation}
        B(\varphi,r\| x \|)\subset B_x(\varphi,r)\subset \mO.
    \end{equation}
    Donc \( \mO\) est un \( N\)-ouvert.
\end{proof}

\begin{corollary}
    Soit un espace topologique \( X\). Si \( f\colon (V',*)\to X\) est continue, alors \( f\colon (V',\| . \|)\to X\) est continue.
\end{corollary}

\begin{proof}
    Soit un ouvert \( \mO\) de \( X\). Vu que \( f\) est \( *\)-continue, la partie \( f^{-1}(\mO)\) est un \( *\)-ouvert de \( V'\). Il est onc un \( N\)-ouvert de \( V'\) par la proposition~\ref{PROPooFGXAooFRWweD}.
\end{proof}

%---------------------------------------------------------------------------------------------------------------------------
\subsection{Réflexivité}
%---------------------------------------------------------------------------------------------------------------------------

Pour la suite nous notons \( V''\) le dual de \( (V',\| . \|)\). Certes en tant qu'ensembles, \( (V',*)\) et \( (V',\| . \|) \) sont identiques, mais comme ils n'ont pas la même topologie, les duaux ne sont pas les mêmes.

Bref, \( V''\) est l'ensemble des applications linéaires continues \( (V',\| . \|)\to \eC\). Et lorsque nous disons \( \eC\) ici, ça peut aussi bien être \( \eR\) selon le contexte.

De plus nous considérons que \( V''\) la norme opérateur qui dérive de la norme de \( V'\), laquelle dérive de la norme vectorielle sur \( V\).

\begin{propositionDef}      \label{PROPooMAQSooCGFBBM}
    Soit un espace vectoriel normé $V$ sur $\eR$ ou $\eC$. Nous considérons l'application
    \begin{equation}
        \begin{aligned}
            J\colon V&\to V'' \\
            J(x)\varphi&= \varphi(x).
        \end{aligned}
    \end{equation}
    \begin{enumerate}
        \item       \label{ITEMooNVVSooNFXgnE}
            L'application \( J\) est bien définie : \( J(x)\) est continue.
        \item       \label{ITEMooKURHooZZWpbu}
            L'application \( J\) est continue.
        \item       \label{ITEMooTFYVooKhMOjp}
             Elle est injective.
    \end{enumerate}

    Lorsque \( J\) est bijective, l'espace \( V\) est dit \defe{réflexif}{réflexif}.
\end{propositionDef}

\begin{proof}
    Point par point.
    \begin{subproof}
        \item[\ref{ITEMooNVVSooNFXgnE}]
            Nous commençons par montrer que \( J(x)\colon (V',\| . \|)\to \eC\) est continue pour chaque \( x\in V\). Soit une suite \( \varphi_k\stackrel{\| . \|}{\longrightarrow}0\). Nous avons :
            \begin{equation}
                J(x)\varphi_k=\varphi_k(x)\leq \| \varphi_k \|\| x \|\to 0
            \end{equation}
            où vous aurez noté l'utilisation du lemme~\ref{LEMooIBLEooLJczmu}.  Cela prouve que \( J(x)\) est continue et donc que \( J\) est bien à valeurs dans \( V''\).
        \item[\ref{ITEMooKURHooZZWpbu}]

            Soit une suite \( x_k\stackrel{V}{\longrightarrow}0\), et étudions \( \| J(x_k) \|\) pour la norme dans \( V''\). Nous posons \( x'_k=x_k/\| x_k \|\) et nous calculons (encore une fois, nous écrivons «\( \eC\)», mais ça pourrait être \( \eR\))
            \begin{equation}
                \| J(x_k) \|=\sup_{\| \varphi \|=1}| J(x_k)\varphi |_{\eC}=\sup_{\| \varphi \|=1}| \varphi(x_k) |=\| x_k \|\sup_{\| \varphi \|=1}| \varphi(x'_k) |\leq \| x_k \|\to 0.
            \end{equation}
            La dernière inégalité pourrait être sans doute une égalité\quext{Écrivez-moi si vous en êtes certain.}, mais nous n'en avons pas besoin ici.
    \end{subproof}
\end{proof}

%--------------------------------------------------------------------------------------------------------------------------- 
\subsection{Module de continuité}
%---------------------------------------------------------------------------------------------------------------------------

\begin{definition}
    Soient deux espaces topologiques normés \( X\) et \( Y\),  ainsi qu'une application \( f\colon X\to Y\). Le \defe{module de continuité}{module!de continuité} de \( f\) est la fonction
    \begin{equation}
        \begin{aligned}
            \omega_f\colon \eR^+&\to \eR^+\cup\{ \infty \} \\
            h&\mapsto\sup_{\substack{x,y\in X\\d_X(x,y)< h}} d_Y\big( f(x),f(y) \big).
        \end{aligned}
    \end{equation}
    Nous définissons aussi \( \omega_f(h)=0\) pour \( h\leq 0\).
\end{definition}

Notons que le module de continuité est une fonction croissante.

\begin{lemma}   \label{LemLUbgYeo}
    Soit \( f\in C^0\big( \mathopen[ 0 , 1 \mathclose],\eC \big)\) et \( \omega\) son module de continuité. Si \( \lambda\) et \( h\) sont strictement positifs avec \( \lambda h\in\mathopen[ 0 , 1 \mathclose]\) alors
    \begin{equation}
        \phi(\lambda h)\leq (\lambda+1)\omega(h).
    \end{equation}
\end{lemma}

\begin{proof}
    La fonction \( \omega\) est décroissante, et pour \( h,k>0\) nous avons \( \omega(h+k)\leq\omega(h)+\omega(k)\). Par récurrence pour tout \( k\in \eN\) nous avons
    \begin{equation}
        \omega(kh)\leq k\omega(h).
    \end{equation}
    En écrivant cela pour \( k=\lceil \lambda\rceil\), nous avons
    \begin{equation}
        \omega(\lambda h)\leq \omega(kh)\leq k\omega(h)\leq (\lambda+1)\omega(h).
    \end{equation}
\end{proof}

\begin{lemma}   \label{LemeERapq}
    Une fonction est uniformément continue\footnote{Définition \ref{DEFooYIPXooQTscbG}.} si et seulement si son module de continuité est continu en zéro\footnote{Dans ce lemme, nous avons deux espaces métriques, mais nous allons noter \( d\) la distance des deux côtés.}.
\end{lemma}

\begin{proof}
    Nous commençons par supposer que \( f\) est uniformément continue. Soit \( \epsilon>0\). Par uniforme continuité, il existe \( \delta>0\) tel que \( d\big( f(x),f(y) \big)\leq \epsilon\) dès que \( d(x,y)\leq \delta\). Si \( h\in B(0,\delta)\), alors
    \begin{equation}
        \omega_f(h)\leq \omega_f(\delta)=\sup_{\substack{x,y\in X\\d(x,y)\leq \delta}}d\big( f(x),f(y) \big)\leq \epsilon.
    \end{equation}
    Cela prouve que \( \lim_{h\to 0} \omega_f(h)=0\).

    Dans l'autre sens, si \( \epsilon>0\) est fixé, il suffit de prendre \( \delta\) tel que \( \omega_f(h)\leq \epsilon\) pour tout \( h\leq \delta\) pour faire fonctionner la définition de l'uniforme continuité.
\end{proof}

\begin{lemma}[\cite{ooCPZDooOqIIEz}]        \label{LEMooKPPSooPIncvn}
    Soient des espaces métriques \( E\) et \( E'\) et une suite de fonctions \( (f_i)_{i\geq 0}\) qui converge uniformément vers \( f\). Alors pour chaque \( \delta>0\) nous avons
    \begin{equation}
        \limsup_{i\to \infty}\omega_{f_i}(\delta)\leq \omega_f(\delta).
    \end{equation}
\end{lemma}

\begin{proof}
    Soient \( \delta>0\) ainsi que \( x,y\in E\) tels que \( \| x-y \|\leq \delta\). Pour chaque \( i\) nous avons
    \begin{subequations}
        \begin{align}
            | f_i(x)-f_i(y) |&\leq | f_i(x)-f(x) |+| f(x)-f(y) |+| f(y)-f_i(y) |\\
            &\leq | f(x)-f(y) |+2\| f_i-f \|_{\infty}\\
            &\leq \omega_f(\delta)+2\| f_i-f \|_{\infty}.
        \end{align}
    \end{subequations}
    Nous prenons le supremum de cela sur \( \{  x,y\in E\tq \| x-y \|\leq \delta \}\) pour obtenir :
    \begin{equation}
        \omega_{f_i}(\delta)\leq \omega_f(\delta)+2\| f_i-f \|_{\infty}.
    \end{equation}
    La tentation est grande à ce point de prendre la limite des deux côtés pour \( i\to \infty\). Cependant, rien ne nous permet de dire que la suite \( i\mapsto   \omega_{f_i}(\delta)  \) ait une limite. Nous pouvons cependant prendre la limite supérieures\footnote{Définition \ref{ooMVZAooVVCOnP}.} et obtenir
    \begin{equation}
        \limsup_{i\to \infty}\omega_{f_i}(\delta)\leq \omega_f(\delta).
    \end{equation}
\end{proof}

%+++++++++++++++++++++++++++++++++++++++++++++++++++++++++++++++++++++++++++++++++++++++++++++++++++++++++++++++++++++++++++
\section{Mini introduction aux nombres \texorpdfstring{\( p\)}{p}-adiques}
%+++++++++++++++++++++++++++++++++++++++++++++++++++++++++++++++++++++++++++++++++++++++++++++++++++++++++++++++++++++++++++

\subsection{La flèche d'Achille}\label{s:un}

C'est un grand classique que je donne ici juste comme introduction pour montrer que des séries infinies peuvent donner des nombres finis de manière tout à fait intuitive.

Achille tire une flèche vers un arbre situé à $\unit{10}{\meter}$ de lui. Disons que la flèche avance à une vitesse constante de $\unit{1}{\meter\per\second}$. Il est clair que la flèche mettra $\unit{10}{\second}$ pour toucher l'arbre. En $\unit{5}{\second}$, elle aura parcouru la moitié de son chemin. On le note :
\[
\text{temps}=5s+\ldots
\]
Reste \( \unit{5}{\meter}\) à faire. En $\unit{2.5}{\second}$, elle aura fait la moitié de ce chemin chemin, soit $2.5m=\frac{10}{4}m$. On le note :
\[
\text{temps}=\frac{10}{2}s+\frac{10}{4}s+
\]
Reste $2.5m$ à faire. La moitié de ce trajet, soit $\frac{10}{8}m$, est parcouru en $\frac{10}{8}s$; on le note encore, mais c'est la dernière fois !

\[
\text{temps}=\frac{10}{2}s+\frac{10}{4}s+\frac{10}{8}s+
\]
En continuant ainsi à regarder la flèche qui parcours des demi-trajets puis des moitiés de demi-trajets et encore des moitiés de moitiés de demi-trajets, et en sachant que le temps total est $10s$, on trouve :
\[
10\left( \frac{1}{2}+\frac{1}{4}+\frac{1}{8}+\frac{1}{16}+\ldots  \right)=10.
\]
On doit donc croire que la somme jusqu'à l'infini des inverses des puissances de deux vaut $1$ :
\[
   \sum_{n=1}^{\infty}\frac{1}{2^n}=1.
\]
Cela peut être démontré à la loyale.

\subsection{La tortue et Achille}

Maintenant qu'on est convaincu que des sommes infinies peuvent représenter des nombres tout à fait normaux, passons à un truc plus marrant.

Achille, qui marche peinard à $\unit{10}{\meter\per\hour}$, part avec $1m$ d'avance sur une tortue qui avance à $\unit{1}{\meter\per\hour}$. Le temps que la tortue arrive au point de départ d'Achille, Achille aura parcouru $10m$, et le temps que la tortue mettra pour arriver à ce point, eh bien, Achille ne sera déjà plus là : il sera à $100m$. Si la tortue tient bon pendant un temps infini, et si l'on est confiant en le genre de raisonnements faits à la section~\ref{s:un}, elle rattrapera Achille dans
\[
1m+10m+100m+1000m+\ldots
\]
Autant dire que ça ne risque pas d'arriver. Et pourtant, mettons en équations :
\begin{subequations}
    \begin{numcases}{}
        x_{\text{Achile}}(t)=1+10t\\
        x_{\text{tortue}}(t)=t.
    \end{numcases}
\end{subequations}
La tortue rejoint Achille au temps \( t\) tel que \( x_{\text{Achille}(t)}=x_{\text{tortue}}(t)\). Un mini calcul donne $t=-1/9$. Physiquement, c'est une situation logique. Peut-on en déduire une égalité mathématique du style de
\[
1+10+100+1000+\ldots=-\frac{1}{9}\; ???
\]
Là où les choses deviennent jolies, c'est quand on cherche à voir ce que peut bien être la valeur d'un hypothétique $x=1+10+100+1000+\ldots$. En effet, logiquement on devrait avoir
\begin{equation*}
\begin{split}
\frac{x}{10}&=\frac{1}{10}+1+10+100+\ldots\\
            &=\frac{1}{10}+x.
\end{split}
\end{equation*}
Reste à résoudre l'équation du premier degré : $\frac{x}{10}=x+\frac{1}{10}$. Ai-je besoin de donner la solution ?

%---------------------------------------------------------------------------------------------------------------------------
\subsection{Dans les nombres \texorpdfstring{\( p\)}{p}-adiques, c'est vrai}
%---------------------------------------------------------------------------------------------------------------------------

Nous nous proposons d'apprendre sur les nombres \( p\)-adiques juste ce qu'il faut pour montrer que l'égalité
\begin{equation}
    \sum_{k=0}^{\infty}10^k=-\frac{1}{ 9 }
\end{equation}
est vraie dans les nombres \( 5\)-adiques. Tout ce qu'il faut est sur \wikipedia{fr}{Nombre_p-adique}{wikipedia}.

Soit \( a\in \eN\) et \( p\), un nombre premier. La \defe{valuation}{valuation!$p$-adique} \( p\)-adique de \( a\) est l'exposant de \( p\) dans la décomposition de \( a\) en nombres premiers. On la note \( v_p(a)\). Pour un rationnel on définit
\begin{equation}
    v_p\left( \frac{ a }{ b } \right)=v_p(a)-v_p(b)
\end{equation}
La \defe{valeur absolue}{valeur absolue!$p$-adique} \( p\)-adique de \( r\in \eQ\) est
\begin{equation}
    | r |_p=p^{-v_p(r)}.
\end{equation}
Nous posons \( | 0 |_p=0\). De là nous considérons la distance
\begin{equation}
    d_p(x,y)=| x-y |_p.
\end{equation}

\begin{lemma}
    L'espace \( (\eQ,d_p)\) est un espace métrique\footnote{Définition~\ref{DefMVNVFsX}}.
\end{lemma}
\index{topologie!\( p\)-adique}

Nous considérons maintenant \( p=5\). Étant donné que \( a=5\cdot 2\) nous avons \( v_5(10)=1\) et
\begin{equation}
    v_5\left( \frac{1}{ 9 } \right)=v_5(1)-v_5(9)=0.
\end{equation}
Nous avons
\begin{equation}
    \sum_{k=0}^N10^k+\frac{1}{ 9 }=\frac{ 10^{N+1} }{ 9 }
\end{equation}
mais
\begin{equation}
    v_p\left( \frac{ 10^{N+1} }{ 9 } \right)=v_5(10^{N+1})-v_5(9)=N+1.
\end{equation}
Par conséquent
\begin{equation}
    d_5\big( \sum_{k=0}^N10^k,-\frac{1}{ 9 } \big)=| \frac{ 10^{N+1} }{ 9 } |_p=p^{-(N+1)}.
\end{equation}
En passant à la limite,
\begin{equation}
    \lim_{N\to \infty} d_5\big( \sum_{k=0}^N10^k,-\frac{1}{ 9 } \big)=0,
\end{equation}
ce qui signifie que\footnote{Voir la définition~\ref{DefGFHAaOL} de la convergence d'une série dans un espace métrique.}
\begin{equation}
    \sum_{k=0}^{\infty}10^k=-\frac{1}{ 9 }.
\end{equation}



\chapter{Analyse réelle}
% This is part of Mes notes de mathématique
% Copyright (c) 2008-2019
%   Laurent Claessens
% See the file fdl-1.3.txt for copying conditions.

%+++++++++++++++++++++++++++++++++++++++++++++++++++++++++++++++++++++++++++++++++++++++++++++++++++++++++++++++++++++++++++
\section{Intervalles}
%+++++++++++++++++++++++++++++++++++++++++++++++++++++++++++++++++++++++++++++++++++++++++++++++++++++++++++++++++++++++++++

\begin{definition}[Intervalle]
    Une partie \( I\) de \( \eR\) est un \defe{intervalle}{intervalle} si pour tout \( a,b\in I\) nous avons \( t\in I\) dès que \( a\leq t\leq b\).

    Un intervalle est \defe{ouvert}{intervalle!ouvert} s'il est de la forme \( \mathopen] a , b \mathclose[\) avec éventuellement \( a=-\infty\) ou \( b=+\infty\). Un intervalle est \defe{fermé}{intervalle!fermé} s'il est de la forme \( \mathopen[ a , b \mathclose]\) ou \( \mathopen] -\infty , b \mathclose]\) ou \( \mathopen[ a , +\infty [\) avec \( a,b\in \eR\).
\end{definition}

\begin{remark}
  L'ensemble $\eR$ ne contient pas $-\infty$ et $-\infty$. L'intervalle $[-\infty, 5]$ par exemple, n'est pas une partie de $\eR$.
\end{remark}

\begin{example}
    \begin{enumerate}
        \item
        Les ensembles \( \mathopen] 3 , 7 \mathclose[\) et \( \mathopen] -\infty , \pi \mathclose[\) sont des intervalles ouverts.
        \item
            Les ensembles \( \mathopen[ 10 , 15 \mathclose]\) et \( \mathopen[ -1 , +\infty [\) sont des intervalles fermés.
        \item
        L'ensemble \( \mathopen] -4 , -2 \mathclose[\cup\mathopen] 2 , 9 \mathclose[\) n'est pas un intervalle (il y a un «trou» entre \(- 2\) et \( 2\)).
        \item
            L'ensemble \( \eR\) lui-même est un intervalle; par convention, il est à la fois ouvert et fermé.
    \end{enumerate}
Un intervalle peut n'être ni ouvert ni fermé; par exemple \( \mathopen] 4 , 8 \mathclose]\). Cet intervalle est «ouvert en \( 4\) et fermé en \( 8\)» .
\end{example}

\begin{definition}[Fonction, domaine, image, graphe]
  Soient $X$ et $Y$ deux ensembles. Une \defe{fonction}{fonction} $f$ définie sur $X$ et à valeurs dans $Y$ est une correspondence qui associe à chaque élément $x$ dans $X$ {\bf au plus} un élément $y$ dans $Y$. On écrit $y= f(x)$.
  \begin{itemize}
  \item La partie de $X$ qui contient tous les $x$ sur lesquels $f$ peut opérer est dite \defe{domaine}{domaine} de $f$. Le domaine de $f$ est indiqué par $\Dom f$.
  \item L'élément de $y\in Y$ associé par $f$ à un élément $x\in \Dom f$ (c'est-à-dire $f(x) = y$)  est appellé \defe{image}{image} de $x$ par $f$. L'\defe{image}{fonction!image} de la fonction $f$ est la partie de $Y$ qui contient les images de tous les éléments de $\Dom f$. L'image de $f$ est indiquée par $\Im f$.
  \item Le \defe{graphe}{graphe} de $f$ est l'ensemble de toutes les couples $(x, f(x))$ pour $x\in \Dom f$. Le graphe de $f$ est une partie de l'ensemble noté $X\times Y$ et il est indiqué par $\Graph f$. Dans ce cours $X = \eR$ et $Y = \eR$, donc le graphe de $f$ est contenu dans le plan cartésien.
  \end{itemize}
\end{definition}

\begin{definition}[Fonction croissante, décroissante et monotone]
    Soit \( f\colon \eR\to \eR\) une fonction définie sur un intervalle \( I\subset \eR\).
    \begin{enumerate}
        \item
            Le fonction \( f\) est \defe{croissante}{fonction!croissante} sur \( I\) si pour tout \( x<y\) dans \( I\) nous avons \( f(x)\leq f(y)\). Elle est \emph{strictement} croissante si \( f(x)<f(y)\) dès que \( x<y\).
        \item
            Le fonction \( f\) est \defe{décroissante}{fonction!décroissante} sur \( I\) si pour tout \( x<y\) dans \( I\) nous avons \( f(x)\geq f(y)\). Elle est \emph{strictement} décroissante si \( f(x)>f(y)\) dès que \( x<y\).
        \item
            La fonction \( f\) est dite \defe{monotone}{fonction!monotone} sur \( I\) si elle est soit croissante soit décroissante sur \( I\).
    \end{enumerate}
\end{definition}

\begin{example}
    La fonction \( x\mapsto x^2\) est décroissante sur l'intervalle \( \mathopen] -\infty , 0 \mathclose]\) et croissante sur l'intervalle \( \mathopen[ 0 , \infty \mathclose[\). Elle n'est par contre ni croissante ni décroissante sur l'intervalle \( \mathopen[ -4 , 3 \mathclose]\).
\end{example}

%+++++++++++++++++++++++++++++++++++++++++++++++++++++++++++++++++++++++++++++++++++++++++++++++++++++++++++++++++++++++++++
\section{Application réciproque}
%+++++++++++++++++++++++++++++++++++++++++++++++++++++++++++++++++++++++++++++++++++++++++++++++++++++++++++++++++++++++++++

%---------------------------------------------------------------------------------------------------------------------------
\subsection{Définitions}
%---------------------------------------------------------------------------------------------------------------------------

Les définitions d'injection, surjection, bijection et d'application réciproque sont les définitions~\ref{DEFooBFCQooPyKvRK} et~\ref{DEFooTRGYooRxORpY}.

\begin{example}     \label{EXooCWYHooLEciVj}
    \begin{enumerate}
        \item
            La fonction \( x\mapsto x^2\) n'est pas une bijection de \( \eR\) vers \( \eR\) parce qu'il n'existe aucun \( x\) tel que \( x^2=-1\).
        \item
            La fonction
            \begin{equation}
                \begin{aligned}
                    f\colon \mathopen[ 0 , +\infty [&\to \mathopen[ 0 , +\infty [ \\
                    x&\mapsto x^2
                \end{aligned}
            \end{equation}
            est une bijection. Notez que c'est la même fonction que celle de l'exemple précédent. Seul l'intervalle sur laquelle nous nous plaçons a changé.
        \item
            La fonction
            \begin{equation}
                \begin{aligned}
                    f\colon \eR&\to \mathopen[ 0 , \infty \mathclose[ \\
                    x&\mapsto x^2
                \end{aligned}
            \end{equation}
            n'est pas une bijections parce qu'il existe plusieurs \( x\) pour lesquels \( f(x)=4\).
        \item
            Nous verrons un peu plus tard (\ref{PROPooXQYFooPxoEHE}) que l'application
            \begin{equation}
                \begin{aligned}
                    f\colon \mathopen[ 0 , \infty \mathclose[\to \mathopen[ 0 , \infty \mathclose[\\
                    x&\mapsto x^2
                \end{aligned}
            \end{equation}
            est une bijection.
    \end{enumerate}
    En conclusion : il est très important de préciser les domaines des fonctions considérées.
\end{example}

\begin{remark}
    Dire que la fonction \( f\colon I\to J\) est bijective, c'est dire que l'équation \( f(x)=y\) d'inconnue \( x\) peut être résolue de façon univoque pour tout \( y\in J\).
\end{remark}

\begin{remark}
  Toute fonction strictement monotone sur un intervalle $I$ est injective.
\end{remark}

\begin{example}
    Trouvons la fonction réciproque de la fonction affine \( f\colon \eR\to \eR\), \( x\mapsto 3x-2\). Si \( y\in \eR\) le nombre \( f^{-1}(y)\) est la valeur de \( x\) pour laquelle \( f(x)=y\). Il s'agit donc de résoudre
    \begin{equation}
        3x-2=y
    \end{equation}
    par rapport à \( x\). La solution est \( x=\frac{ y+2 }{ 3 }\) et donc nous écrivons
    \begin{equation}
        f^{-1}(y)=\frac{ y+2 }{ 3 }.
    \end{equation}
    Notons que dans les calculs, il est plus simple d'écrire «\( y\)» que «\( x\)» la variable de la fonction réciproque. Il est néanmoins (très) recommandé de nommer «\( x\)» la variable dans la réponse finale. Dans notre cas nous concluons donc
    \begin{equation}
        f^{-1}(x)=\frac{ x+2 }{ 3 }.
    \end{equation}
\end{example}

%---------------------------------------------------------------------------------------------------------------------------
\subsection{Graphe de la fonction réciproque}
%---------------------------------------------------------------------------------------------------------------------------

Par définition le graphe de la fonction \( f\) est l'ensemble des points de la forme \( (x,y)\) vérifiant \( y=f(x)\). Afin de déterminer le graphe de la bijection réciproque nous pouvons faire le raisonnement suivant.

        Le point \( (x_0,y_0)\) est sur le graphe de \( f\)

\noindent\( \Leftrightarrow\)

        La relation \( f(x_0)=y_0\) est vérifiée

\noindent\( \Leftrightarrow\)

        La relation \( x_0=f^{-1}(y_0)\) est vérifiée

\noindent\( \Leftrightarrow\)

        Le point \( (y_0,x_0)\) est sur le graphe de \( f^{-1}\).

\begin{Aretenir}
    Dans un repère orthonormal, le graphe de la bijection réciproque est obtenu à partir du graphe de \( f\) en effectuant une symétrie par rapport à la droite d'équation \( y=x\).
\end{Aretenir}

Le dessin suivant montre le cas de la courbe de la fonction carré comparé à celle de la racine carrée.
\begin{center}
   \input{auto/pictures_tex/Fig_CELooGVvzMc.pstricks}
\end{center}

%+++++++++++++++++++++++++++++++++++++++++++++++++++++++++++++++++++++++++++++++++++++++++++++++++++++++++++++++++++++++++++
\section{Limite de fonctions}
%+++++++++++++++++++++++++++++++++++++++++++++++++++++++++++++++++++++++++++++++++++++++++++++++++++++++++++++++++++++++++++

%---------------------------------------------------------------------------------------------------------------------------
\subsection{Définition}
%---------------------------------------------------------------------------------------------------------------------------

La définition générale de la limite est~\ref{DefYNVoWBx}. Dans le cas de fonctions \( \eR\to \eR\), elle peut s'écrire de façon plus efficace. La proposition suivante montre comment fonctionne la limite pour une fonction définie sur tout \( \eR\).

\begin{proposition}[Caractérisation de la limite]       \label{PropAJQQooQQClfp}
	Soit une fonction $f\colon \eR\to \eR$ définie sur \( \eR\) et $a\in \eR$. La fonction \( f\) admet la limite \( \ell\) pour \( x\to a\) si et seulement si il existe un réel $\ell$ tel que pour tout \( \epsilon>0\), il existe un \( \delta>0\) tel que
	\begin{equation}\label{EqDefLimiteFonction}
		0<| x-a |<\delta\Rightarrow| f(x)-\ell |<\varepsilon.
	\end{equation}
\end{proposition}

\begin{proof}
    Il s'agit de montrer l'équivalence avec la définition~\ref{DefYNVoWBx}. Nous allons faire un usage intensif de la remarque~\ref{RemQDRooKnwKk}\ref{ITEMooUIHJooXAFaJa}.
    \begin{subproof}
    \item[Sens direct]
        Soient \( \epsilon>0\) et \( V=B(\ell,\epsilon)\). Alors il existe un voisinage \( W\) de \( a\) dans \( \eR\) tel que
        \begin{equation}
            f\big( W\setminus\{ a \} \big)\subset V.
        \end{equation}
        Soit \( \delta\) tel que \( B(a,\delta)\subset W\). Nous avons encore
        \begin{equation}
            f\big( B(a,\delta)\setminus\{ a \} \big)\subset V.
        \end{equation}
        Soit maintenant \( x\in \eR\) tel que $0<| x-a |<\delta$. Cela signifie \( x\in B(a,\delta)\setminus\{ a \}\). Pour un tel \( x\) nous avons donc \( f(x)\in B(\ell,\epsilon)\), c'est-à-dire \( | f(x)-\ell |<\epsilon\).
    \item[Dans l'autre sens]
        Soient un voisinage \( V\) de \( \ell\) et \( \epsilon>0\) tel que \( B(\ell,\epsilon)\subset V\). Nous considérons \( \delta\) tel que \( 0<| x-a |<\delta\) implique \( | f(x)-\ell |<\epsilon\).

        Avec tout cela nous posons \( W=B(x,\delta)\), et nous avons
        \begin{equation}
            f\big( W\setminus\{ a \} \big)\subset B(\ell,\alpha)\subset V.
        \end{equation}
    \end{subproof}
\end{proof}

Si aucun nombre $\ell$ ne vérifie la condition de la définition, alors on dit que la fonction n'admet pas de limite en $a$. Lorsque $f$ possède la limite $\ell$ en $a$, nous notons
\begin{equation}
	\lim_{x\to a} f(x)=\ell.
\end{equation}

La proposition suivante a déjà été démontrée dans la proposition~\ref{PropFObayrf}. Nous en donnons ici une démonstration adaptée au cas \( \eR\to \eR\).

\begin{proposition}
	Soit une fonction $f\colon D\to \eR$. Si $a$ est un point d'accumulation de $D$ et s'il existe une limite de $f$ en $a$, alors il en existe une seule.
\end{proposition}


\begin{proof}
    Nous prouvons qu'il ne peut pas exister deux nombres $\ell\neq\ell'$ vérifiant tout les deux la condition \eqref{EqDefLimiteFonction}.

	Soient $\ell$ et $\ell'$ deux limites de $f$ au point $a$. Par définition, pour tout $\varepsilon$ nous avons des nombres $\delta$ et $\delta'$ tels que
	\begin{equation}	\label{EqsContf2307Right}
		\begin{aligned}[]
			| x-a |<\delta&\Rightarrow \big| f(x)-\ell \big|<\varepsilon\\
			| x-a |<\delta'&\Rightarrow \big| f(x)-\ell' \big|<\varepsilon
		\end{aligned}
	\end{equation}
	Pour fixer les idées, supposons que $\delta<\delta'$ (le cas $\delta\geq\delta'$ se traite de la même manière).

	Étant donné que $a$ est un point d'accumulation du domaine $D$ de $f$, il existe un $x\in D$ tel que $| x-a |<\delta$. Évidemment, nous avons aussi $| x-a |<\delta'$. Les conditions \eqref{EqsContf2307Right} signifient alors que ce $x$ vérifie en même temps
	\begin{equation}
		| f(x)-\ell |<\varepsilon,
	\end{equation}
	et
	\begin{equation}
		| f(x)-\ell' |<\varepsilon.
	\end{equation}
	Afin de prouver que $\ell=\ell'$, nous allons maintenant calculer $| \ell-\ell' |$ et montrer que cette distance est plus petite que tout nombre. Nous avons (voir remarque~\ref{RemTechniqueIneqs})
	\begin{equation}	\label{EqInesq2307ellellepr}
		| \ell-\ell' |=| \ell-f(x)+f(x)-\ell' |\leq | \ell-f(x) |+| f(x)-\ell' |<\varepsilon+\varepsilon.
	\end{equation}
	En résumé, pour tout $\varepsilon>0$ nous avons
	\begin{equation}
		| \ell-\ell' |<2\varepsilon,
	\end{equation}
	et donc $| \ell-\ell' |=0$, ce qui signifie que $\ell=\ell'$.
\end{proof}

\begin{remark}		\label{RemTechniqueIneqs}
	Les inégalités \eqref{EqInesq2307ellellepr} utilisent deux techniques très classiques en analyse qu'il convient d'avoir bien compris. La première est de faire
	\begin{equation}
		| A-B |=| A-C+C-B |.
	\end{equation}
	Il s'agit d'ajouter $-C+C$ dans la norme. Évidemment, cela ne change rien.

	La seconde technique est l'inégalité
	\begin{equation}
		| A+B |\leq| A |+| B |.
	\end{equation}
\end{remark}

\begin{example}
	Considérons la fonction $f(x)=2x$, et calculons la limite $\lim_{x\to 3} f(x)$. Vu que $f(3)=6$, nous nous attendons à avoir $\ell=6$. C'est ce que nous allons prouver maintenant. Pour chaque $\varepsilon>0$ nous devons trouver un $\delta>0$ tel que $| x-3 |<\delta$ implique $| f(x)-6 |<\varepsilon$. En remplaçant $f(x)$ par sa valeur en fonction de $x$ et avec quelques manipulations nous trouvons :
	\begin{equation}
		\begin{aligned}[]
			| f(x)-6 |&<\varepsilon\\
			| 2x-6 |&<\varepsilon\\
			2| x-3 |&<\varepsilon\\
			| x-3 |&<\frac{ \varepsilon }{2}
		\end{aligned}
	\end{equation}
	Donc dès que $| x-3 |<\frac{ \varepsilon }{2}$, nous avons $| f(x)-6 |<\varepsilon$. Nous posons donc $\delta=\frac{ \varepsilon }{2}$.

	Plus généralement, nous avons $\lim_{x\to a} f(x)=2a$, et cela se prouve en étudiant $| f(x)-2a |$ exactement de la même manière.
\end{example}

%---------------------------------------------------------------------------------------------------------------------------
\subsection{Quelques règles de calcul}
%---------------------------------------------------------------------------------------------------------------------------

\begin{proposition}	\label{PropLimEstLineraure}
	La limite est une opération linéaire, c'est-à-dire que si $f$ et $g$ sont des fonctions qui admettent des limites en $a$ et si $\lambda$ est un nombre réel,
	\begin{enumerate}

		\item
			$\lim_{x\to a} (\lambda f)(x)=\lambda\lim_{x\to a} f(x)$,
		\item
			$\lim_{x\to a} (f+g)(x)=\lim_{x\to a} f(x)+\lim_{x\to a} g(x)$.
	\end{enumerate}
\end{proposition}
En combinant les deux propriétés de la proposition~\ref{PropLimEstLineraure}, nous pouvons écrire
\begin{equation}
	\lim_{x\to a} (\lambda f+\mu g)(x)=\lambda\lim_{x\to a} f(x)+\mu\lim_{x\to a} g(x).
\end{equation}
pour toutes fonctions $f$ et $g$ admettant une limite en $a$ et pour tout réels $\lambda$ et $\mu$.

En plus d'être linéaire, la limite possède les deux propriétés suivantes.
\begin{proposition}     \label{PROPooDQFIooMMwxxJ}
	Si $f$ et $g$ sont deux fonctions qui admettent une limite en $a$, alors
	\begin{equation}
		\lim_{x\to a} (fg)(x)=\lim_{x\to a} f(x)\cdot\lim_{x\to a} g(x).
	\end{equation}
	Si de plus $\lim_{x\to a} g(x)\neq 0$, alors
	\begin{equation}
		\lim_{x\to a} \frac{ f(x) }{ g(x) }=\frac{ \lim_{x\to a} f(x) }{ \lim_{x\to a} g(x) }.
	\end{equation}
\end{proposition}

\begin{theorem}     \label{ThoLimLinMul}
    Si
    \begin{equation} \label{Eqhypmullimlin}
      \lim_{x\to a}f(x)=b,
    \end{equation}
    alors
    \begin{equation} \label{Eqbutmultlim}
      \lim_{x\to a}(\lambda f)(x)=\lambda b
    \end{equation}
    pour n'importe quel $\lambda\in\eR$.
\end{theorem}

\begin{proof}
Soit $\epsilon>0$. Afin de prouver la propriété \eqref{Eqbutmultlim}, il faut trouver un $\delta$ tel que pour tout $x$ dans $[a-\delta,a+\delta]$, on ait $| (\lambda f)(x)- \lambda b |\leq\epsilon$. Cette dernière inégalité est équivalente à $|\lambda|| f(x)-b |\leq\epsilon$. Nous devons donc trouver un $\delta$ tel que
\begin{equation}
| f(x)-b |\leq\frac{ \epsilon }{ | \lambda | }.
\end{equation}
soit vraie pour tout $x$ dans $[a-\delta,a+\delta]$. Mais l'hypothèse \eqref{Eqhypmullimlin} dit précisément qu'il existe un $\delta$ tel que pour tout $x$ dans $[a-\delta,a+\delta]$ on ait cette inégalité.
\end{proof}

\begin{theorem}     \label{ThoLimLin}
    Si
    \begin{subequations}
    \begin{align}
        \lim_{x\to a}f(x)&=b_1\\
        \lim_{x\to a}g(x)&=b_2,
    \end{align}
    \end{subequations}
    alors
    \begin{equation}
        \lim_{x\to a}(f+g)(x)=b_1+b_2.
    \end{equation}
\end{theorem}

\begin{proof}
    Soit $\epsilon>0$. Par hypothèse, il existe $\delta_1$ tel que
    \begin{equation}    \label{Eqfbunepsdeux}
      | f(x)-b_1 |\leq \frac{ \epsilon }{ 2 }
    \end{equation}
    dès que $| x-a |\leq\delta_1$. Il existe aussi $\delta_2$ tel que
    \begin{equation}    \label{Eqgbdeuxepsdeux}
      | g(x)-b_2 |\leq \frac{ \epsilon }{ 2 }.
    \end{equation}
    dès que $| x-a |\leq \delta_2$. Tu notes l'astuce de prendre $\epsilon/2$ dans la définition de limite pour $f$ et $g$. Maintenant, ce qu'on voudrait c'est un $\delta$ tel que l'on ait $| (f+g)(x)-(b_1+b_2) |\leq \epsilon$ dès que $| x-a |\leq \delta$. Moi je dit que $\delta=\min\{ \delta_1,\delta_2 \}$ fonctionne. En effet, en utilisant l'inégalité $| a+b |\leq | a |+| b |$, nous trouvons :
    \begin{align}
    | (f+g)(x)-(b_1+b_2) |=| (f(x)-b_1)+(g(x)-b_2) |
            \leq | f(x)-b_1 |+| g(x)-b_2 |.     \label{Eqfplusgfbun}
    \end{align}
    Comme on suppose que $| x-a |\leq\delta$, on a évidemment $| x-a |\leq\delta_1$, et donc l'équation \eqref{Eqfbunepsdeux} tient. Mais si $| x-a |\leq\delta$, on a aussi $| x-a |\leq\delta_2$, et donc l'équation  \eqref{Eqfbunepsdeux} tient également. Chacun des deux termes de \eqref{Eqfplusgfbun} est donc plus petits que $\epsilon/2$, et donc le tout est plus petit que $\epsilon$, ce qu'il fallait montrer.

\end{proof}

Voici une formule qui résume la linéarité de la limite :
\begin{equation}    \label{EqLimLinRes}
    \lim_{x\to a}[\alpha f(x)+\beta g(x)]=\alpha\lim_{x\to a}f(x)+\beta\lim_{x\to a}g(x).
\end{equation}

\begin{proposition}[\cite{TrenchRealAnalisys}]      \label{PROPooOUPNooTrClHw}
    Soient des fonctions \( f,g\colon \eR\to \eR\) telles que \( \lim_{x\to a} f(x)=\ell\) et \( \lim_{x\to a} g(x)=\ell'\neq 0\). Alors
    \begin{equation}
        \lim_{x\to a} \frac{ f(x) }{ g(x) }=\frac{ \ell }{ \ell' }.
    \end{equation}
\end{proposition}

\begin{proof}
    Nous avons :
    \begin{equation}
        \left| \frac{ f(x) }{ g(x) }-\frac{ \ell }{ \ell' } \right| =\frac{ | \ell'f(x)-g(x)\ell | }{ |g(x)\ell| }.
    \end{equation}
    Soit \( s\), un minorant de \( | g(x) |\) sur un voisinage de \( a\); vu que la limite en \( a\) est \( \ell'\neq 0\), nous pouvons prendre par exemple \( s=\ell'/2\) : \( | g(x) |>\ell'/2\) sur \( B(a,\delta)\) dès que \( \delta\) est assez petit. Nous considérons \( x\in B(a,\delta)\). Avec cela nous avons :
    \begin{subequations}
        \begin{align}
            \left| \frac{ f(x) }{ g(x) }-\frac{ \ell }{ \ell' } \right| &=\frac{ | \ell'f(x)-g(x)\ell | }{ |g(x)\ell| }\\
            &\leq \frac{ 2 }{ | \ell'} |\left( \frac{ | \ell'f(x)-g(x)\ell | }{ | \ell' | } \right) \\
            &\leq \frac{ 2 }{ | \ell' |^2 }\big( | \ell'f(x)-\ell\ell' |+| \ell\ell'-g(x)\ell | \big)\\
            &=\frac{ 2 }{ | \ell' |^2 }\big( | \ell' | |f(x)-\ell |+| \ell | |\ell'-g(x) | \big).
        \end{align}
    \end{subequations}
    Soient \( \epsilon>0\) et \( \delta\) tel que \( | f(x)-\ell |<\epsilon\) et \( | g(x)-\ell' |<\epsilon\) pour tout \( x\in B(a,\delta)\). Avec cela nous avons
    \begin{equation}
        \left| \frac{ f(x) }{ g(x) }-\frac{ \ell }{ \ell' } \right| \leq\frac{ 2 }{ | \ell' |^2\big( | \ell' |+| \ell | \big) }\epsilon.
    \end{equation}
    D'où la limite attendue.
\end{proof}

\begin{lemma}       \label{LemLimMajorableVois}
    Si $\lim_{x\to a}f(x)=b$ avec $a$, $b\in\eR$, alors il existe un $\delta>0$ et un $M>0$ tels que
    \[
        (| x-a |\leq\delta)\Rightarrow | f(x) |\leq M.
    \]
\end{lemma}

Ce que signifie ce lemme, c'est que quand la fonction $f$ admet une limite finie en un point, alors il est possible de majorer la fonction sur un intervalle autour du point.

\begin{proof}
    Cela va être démontré par l'absurde. Supposons qu'il n'existe pas de $\delta$ ni de $M$ qui vérifient la condition. Dans ce cas, pour tout $\delta$ et pour tout $M$, il existe un $x$ tel que $| x-a |\leq\delta$ et $| f(x) |> M$. Cela est valable pour tout $M$, donc prenons par exemple $b+1000$. Donc
    \begin{equation}
    \forall\delta>0,\exists x\text{ tel que } | x-a |\leq\delta\text{ et }| f(x) |>b+1000.
    \end{equation}
    Cela signifie qu'aucun $\delta$ ne peut convenir dans la définition de $\lim_{x\to a}f(x)=b$, ce qui contredit les hypothèses.
\end{proof}

Dans le même ordre d'idée, on peut prouver que si la limite de la fonction en un point est positive, alors elle est positive autour ce ce point. Plus précisément, nous avons la
\begin{proposition} \label{PropoLimPosFPos}
    Si $f$ est une fonction telle que $\lim_{x\to a}f(x)>0$, alors il existe un voisinage de $a$ sur lequel $f$ est positive.
\end{proposition}

\begin{proof}
    Supposons que $\lim_{x\to a}f(x)=y_0$. Par la définition de la limite fait que si pour tout $x$ dans un voisinage autour de $a$, on ait $| f(x)-a |<\epsilon$. Cela est valable pour tout $\epsilon$, pourvu que le voisinage soit assez petit. Si je choisis un voisinage pour lequel $| f(x)-a |<\frac{ y_0 }{ 2 }$, alors sur ce voisinage, $f$ est positive.
\end{proof}

\begin{theorem}     \label{Tholimfgabab}
    Si
    \begin{align}
        \lim_{x\to a}f(x)&=b_1&\text{et}&&\lim_{x\to a}g(x)=b_2,
    \end{align}
    alors
    \begin{equation}
        \lim_{x\to a}(fg)(x)=b_1b_2.
    \end{equation}
\end{theorem}

\begin{proof}
    Soit $\epsilon>0$, et tentons de trouver un $\delta$ tel que $| f(x)g(x)-b_1b_2 |\leq \epsilon$ dès que $| x-a |\leq \delta$. Nous avons
    \begin{equation}    \label{EqfgbunbdeuxMin}
    \begin{split}
    | f(x)g(x)-b_1b_2 |&=|  f(x)g(x)-b_1b_2 +f(x)b_2-f(x)b_2 |\\
            &=\left|   f(x)\big( g(x)-b_2 \big)+b_2\big( f(x)-b_1 \big)    \right|\\
            &\leq \left|  f(x)\big( g(x)-b_2 \big)  \right|+\left|  b_2\big( f(x)-b_1 \big)    \right|\\
            &= | f(x) | | g(x)-b_2  |+| b_2 | |f(x)-b_1 |.
    \end{split}
    \end{equation}
    À la première ligne se trouve la subtilité de la démonstration : on ajoute et on enlève\footnote{Comme exercice, tu peux essayer de refaire la démonstration en ajoutant et enlevant $g(x)b_1$ à la place.} $f(x)b_2$. Maintenant nous savons par le lemme~\ref{LemLimMajorableVois} que pour un certain $\delta_1$, la quantité $| f(x) |$ peut être majoré par un certain $M$ dès que $| x-a |\leq \delta_1$. Prenons donc un tel $\delta_1$ et supposons que $| x-a |\leq \delta_1$. Nous savons aussi que pour n'importe quel choix de $\epsilon_2$ et $\epsilon_3$, il existe des nombres $\delta_2$ et $\delta_3$ tels que $| f(x)-b_1 |\leq \epsilon_2$ et $| g(x)-b_1 |\leq \epsilon_3$ dès que $| x-a |\leq\delta_2$ et $| x-a |\leq\delta_3$. Dans ces conditions, la dernière expression \eqref{EqfgbunbdeuxMin} se réduit à
    \begin{equation}
    | f(x)g(x)-b_1b_2 |\leq M\epsilon_2+| b_2 |\epsilon_3.
    \end{equation}
    Pour terminer la preuve, il suffit de choisir $\epsilon_2$ et $\epsilon_3$ tels que $M\epsilon_2+| b_2 |\epsilon_3\leq\epsilon$, et puis prendre $\delta=\min\{ \delta_1,\delta_2,\delta_3 \}$.

    Remettons les choses dans l'ordre. L'on se donne $\epsilon$ au départ. La première chose est de trouver un $\delta_1$ qui permet de majorer $|f(x)|$ par $M$ selon le lemme~\ref{LemLimMajorableVois}, et puis choisissons $\epsilon_2$ et $\epsilon_3$ tels que $M\epsilon_2+| b_2 |\epsilon_3\leq\epsilon$. Ensuite nous prenons, en vertu des hypothèses de limites pour $f$ et $g$, les nombres $\delta_2$ et $\delta_3$ tels que $| f(x)-b_1 |\leq \epsilon_2$ et $| g(x)-b_2 |\leq \epsilon_3$ dès que $| x-a |\leq \delta_2$ et $| x-a |\leq \delta_3$.

    Si avec tout ça on prend $\delta=\min\{ \delta_1,\delta_2,\delta_3 \}$, alors la majoration et les deux inégalités sont valables en même temps et au final
    \[
      | f(x)g(x)-b_1b_2 |\leq M\epsilon_2+b_2\epsilon_3\leq \epsilon,
    \]
    ce qu'il fallait prouver.

\end{proof}

À l'aide de ces petits résultats, nous pouvons déjà calculer pas mal de limites. Nous pouvons déjà par exemple calculer les limites de tous les polynômes en tous les nombres réels. En effet, nous savons la limite de la fonction $f(x)=x$. La fonction $x\mapsto x^2$ n'est rien d'autre que le produit de $f$ par elle-même. Donc
\[
  \lim_{x\to a}x^2=\big( \lim_{x\to a}x\big)\cdot\big( \lim_{x\to a}x \big)=a^2.
\]
De la même façon, nous trouvons facilement que
\begin{equation}
 \lim_{x\to a}x^n=a^n.
\end{equation}


%--------------------------------------------------------------------------------------------------------------------------- 
\subsection{Limite en l'infini}
%---------------------------------------------------------------------------------------------------------------------------

Non, sur \( \eR\) nous n'allons pas ajouter \( \infty\) avec la topologie d'Alexandrov de la définition \ref{PROPooHNOZooPSzKIN}. Nous n'allons pas considérer \( \hat \eR=\eR\cup\{ \infty \}\).

\begin{definition}[Droite réelle achevée\cite{ooDZRQooPpOXhY}]       \label{DEFooRUyiBSUooALDDOa}
    Nous considérons l'ensemble
    \begin{equation}
        \bar \eR=\eR\cup\{ +\infty,-\infty \}
    \end{equation}
    où \( +\infty\) et \( -\infty\) ne sont pas des éléments de \( \eR\).

    Nous mettons sur \( \bar\eR\) la relation d'ordre en prenant celle de \( \eR\) à laquelle nous ajoutons les règles
    \begin{enumerate}
        \item
            \( -\infty<x\) pour tout \( x\in\eR\cup\{ +\infty \}\)
        \item
            \( +\infty>x\) pour tout \( x\in \eR\cup\{-\infty  \}\).
    \end{enumerate}

    Nous mettons une topologie sur \( \bar\eR\) en donnant la base\footnote{Base de topologie, définition \ref{DefQELfbBEyiB}.} suivante :
    \begin{itemize}
        \item \( \mathopen] a , b \mathclose[\),
        \item \( \mathopen] a , +\infty \mathclose]\),
        \item \( \mathopen[ -\infty , b \mathclose[\)
    \end{itemize}
    pour tout réels \( a\) et \( b\).
\end{definition}

\begin{normaltext}
    En principe, la notation «\( \infty\)» est réservée à l'infini du compactifié d'Alexandrov, et pour les infinis de la droite réelle achevée, il faudrait bien écrire «\( +\infty\)» et «\( -\infty\)». Cependant, nous allons souvent écrire \( \lim_{x\to \infty} \) au lieu de \( \lim_{x\to +\infty} \).
\end{normaltext}

\begin{lemma}
    La topologie sur \( \eR\) induite de celle sur \( \bar \eR\) est la topologie usuelle.
\end{lemma}

\begin{lemma}       \label{LEMooFCIXooJuHFqk}
    Si \( n\geq 1\) nous avons 
    \begin{equation}        \label{EQooRRFEooLYcuRP}
        \lim_{x\to +\infty} x^n = +\infty
    \end{equation}
    et
    \begin{equation}
        \lim_{x\to +\infty} \frac{1}{ x^n }=0.
    \end{equation}
\end{lemma}

\begin{proof}
    Si \( V\) est un voisinage de \( +\infty\), alors nous devons montrer qu'il existe un voisinage \( W\) de \( +\infty\) tel que \( x^n\in V\) pour tout \( x\in W\).   

    Un ouvert est une union d'éléments de la base de topologie\footnote{C'est la proposition \ref{PropMMKBjgY}.}; en regardant la liste donnée dans la définition \ref{PropMMKBjgY}, nous voyons que \( V\) contient au moins une partie de la forme \( \mathopen] R , +\infty \mathclose]\). Nous supposons que \( R>1\).

    Si \( x>R>1\), alors nous avons \( x^n>x\) et donc
    \begin{equation}
        x^n> x>R,
    \end{equation}
    ce qui signifie \( x\in V\).

    En prenant \( W=\mathopen] R , +\infty \mathclose]\), nous avons bien \( W^n\subset V\). Cela prouve \eqref{EQooRRFEooLYcuRP}.

    En ce qui concerne la seconde limite, la démonstration est du même type. Remarquez seulement que vous n'avez pas formellement le droit d'utiliser la proposition \ref{PROPooOUPNooTrClHw} en invoquant \( \frac{1}{ +\infty }=0\).
\end{proof}

%---------------------------------------------------------------------------------------------------------------------------
\subsection{Limite en des nombres}
%---------------------------------------------------------------------------------------------------------------------------

Nous posons la définition suivante.
\begin{definition}      \label{DefInfNombre}
Lorsque $a\in\eR$, on dit que la fonction $f$ \defe{tend vers l'infini quand $x$ tend vers $a$}{} si
\[
  \forall M\in\eR,\exists \delta\tq (| x-a |\leq \delta )\Rightarrow f(x)\geq M\text{ quand }x\in\dom f.
\]
\end{definition}
Cela signifie que l'on demande que dès que $x$ est assez proche de $a$ (c'est-à-dire dès que $| x-a |\leq\delta$), alors $f(x)$ est plus grand que $M$, et que l'on peut trouver un $\delta$ qui fait ça pour n'importe quel $M$. Une autre façon de le dire est que pour toute hauteur $M$, on peut trouver un intervalle de largeur $\delta$ autour de $a$\footnote{C'est-à-dire un intervalle de la forme $[a-\delta,a+\delta]$.} tel que sur cet intervalle, la fonction $f$ est toujours plus grande que $M$.

Montrons sur un dessin pourquoi je disais que la fonction $x\to 1/x$ n'est pas de ce type.


Le problème est qu'il n'existe par exemple aucun intervalle autour de $0$ sur lequel $f$ serait toujours plus grande que $10$. En effet n'importe quel intervalle autour de $0$ contient au moins un nombre négatif. Or quand $x$ est négatif, $f$ n'est certainement pas plus grande que $10$. Nous y reviendrons.

Pour l'instant, montrons que la fonction $f(x)=1/x^2$ est une fonction qui vérifie la définition~\ref{DefInfNombre}.  Avant de prendre n'importe quel $M$, prenons par exemple $100$. Nous avons besoin d'un intervalle autour de zéro sur lequel $f$ est toujours plus grande que $100$. C'est vite vu que $f(0.1)=f(-0.1)=100$, donc l'intervalle $[-\frac{ 1 }{ 10 },\frac{1}{ 10 }]$ est le bon. Partout dans cet intervalle, $f$ est plus grande que $100$. Partout ? Ben non : en $x=0$, la fonction n'est même pas définie, donc c'est un peu dur de dire qu'elle est plus grande que $100$. C'est pour cela que nous avons ajouté la condition « quand $x\in\dom f$ » dans la définition de la limite.

Prenons maintenant un $M\in\eR$ arbitraire, et trouvons un intervalle autour de $0$ sur lequel $f$ est toujours plus grande que $M$. La réponse est évidemment l'intervalle de largeur $1/\sqrt{M}$, c'est-à-dire
\[
  \left[ -\frac{ 1 }{ \sqrt{M} },\frac{ 1 }{ \sqrt{M} } \right].
\]

\subsection{Limites quand tout va bien}
%--------------------------------------

D'abord définissons ce qu'on entend par la limite d'une fonction en un point quand il n'y a aucun infini en jeu.
\begin{definition}      \label{DefLimPointSansInfini}
 On dit que la fonction $f$ \defe{tend vers $b$ quand $x$ tend vers $a$}{} si
\[
  \forall \epsilon>0,\exists\delta\tq (| x-a |\leq\delta)\Rightarrow | f(x)-b |\leq \epsilon\text{ quand }x\in\dom f.
\]
Dans ce cas, nous notons
\begin{equation}
\lim_{x\to a}f(x)=b.
\end{equation}
\end{definition}

Commençons par un exemple très simple : prouvons que $\lim_{x\to 0}x=0$. C'est donc $a=b=0$ dans la définition. Prenons $\epsilon>0$, et trouvons un intervalle autour de zéro tel que partout dans l'intervalle, $x\leq \epsilon$. Bon ben c'est clair que $\delta=\epsilon$ fonctionne.

Plus compliqué maintenant, mais toujours sans surprises.

\begin{proposition}
\[
  \lim_{x\to 0}x^2=0.
\]

\end{proposition}

\begin{proof}
Soit $\epsilon>0$. On veut un intervalle de largeur $\delta$ autour de zéro tel que $x^2$ soit plus petit que $\epsilon$ sur cet intervalle. Cette fois-ci, le $\delta$ qui fonctionne est $\delta=\sqrt{\epsilon}$. En effet un élément de l'intervalle $[-\delta,\delta]$ est un $r$ de valeur absolue plus petite ou égale à $\delta$ :
\[
| r |\leq\delta=\sqrt{\epsilon}.
\]
En prenant le carré de cette inégalité on a :
\[
  r^2\leq\epsilon,
\]
ce qu'il fallait prouver.
\end{proof}

Calculer et prouver des valeurs de limites, mêmes très simples, devient vite de l'arrachage de cheveux à essayer de trouver le bon $\delta$ en fonction de $\epsilon$ si on n'a pas quelques théorèmes généraux. Heureusement nous en avons déjà quelques uns : \ref{PropLimEstLineraure}, \ref{PROPooDQFIooMMwxxJ}, \ref{ThoLimLinMul}, \ref{ThoLimLin}, \ref{PROPooOUPNooTrClHw}.

\begin{proposition}[\cite{MonCerveau}]      \label{PROPooWXBAooAEweSF}
    Soit \( f\colon \eR^2\to \eR\) une application continue dont la variable \( y\) varie dans un compact \( I\) de \( \eR\). Alors la fonction
    \begin{equation}
        \begin{aligned}
            d\colon \eR&\to \eR \\
            x&\mapsto \sup_{y\in I} f(x,y)
        \end{aligned}
    \end{equation}
    est continue.
\end{proposition}

\begin{proof}
    Soit \( x_0\) fixé. Prouvons que \( d\) est continue en \( x_0\). Nous notons \( y_0\) la valeur de \( y\) qui réalise le maximum (par le théorème~\ref{ThoMKKooAbHaro} et le fait que les fonctions projection soient continues, lemme~\ref{LEMooHAODooYSPmvH}). Soit aussi \( \epsilon>0\) tellement fixé que même avec un tourne vis hydraulique, il ne bougerait pas. Nous considérons \( \delta\) tel que si \( \| (x,y)-(x_0,y_0) \|\leq \delta\) alors \( \| f(x,y)-f(x_0,y_0) \|<\epsilon\).

    Si \( | x-x_0 |<\delta\) alors pour \( y\) assez proche de \( y_0\) nous avons \( \| (x,y)-(x_0,y_0) \|\leq \delta\), et donc \( \| f(x,y)-f(x_0,y_0) \|\leq \epsilon \). Cela montre qu'il existe \( \delta\) tel que \( | x-x_0 |\leq \delta\) implique \( d(x)\geq d(x_0)-\epsilon\).

    Nous devons encore trouver un \( \delta\) tel que si \( | x-x_0 |\leq \delta\) alors \( d(x)\leq d(x_0)+\epsilon\). Supposons que non. Alors pour tout \( \delta\) il existe un \( x\) tel que \( | x-x_0 |\leq \delta\) et \( d(x)> d(x_0)+\epsilon\). Cela nous donne une suite \( x_i\to x_0\).

    Pour chaque \( x_i\) nous notons \( y_i\) la valeur de \( y\) qui réalise le supremum correspondant. La suite \( (y_i)\) étant contenue dans un compact nous supposons prendre une sous-suite de \( (x_i)\) telle que la suite \( (y_i)\) converge. Nous nommons \( a\) la limite (et non \( y_0\) parce que nous ne savons pas si \( y_i\to y_0\)). Pour chaque \( i\) nous avons
    \begin{equation}
        f(x_i,y_i)>\sup_{y\in I}f(x_0,y)+\epsilon.
    \end{equation}
    En prenant la limite et en utilisant la continuité de \( f\),
    \begin{equation}
        f(x_0,a)>\sup_{y\in I} f(x_0,y)+\epsilon,
    \end{equation}
    ce qui est impossible.
\end{proof}

%---------------------------------------------------------------------------------------------------------------------------
\subsection{Limites de fonctions}
%---------------------------------------------------------------------------------------------------------------------------

\begin{definition}\label{def_limite}
	Soit $f\colon D\subset\eR^m\to \eR$ une fonction et $a$ un point d'accumulation de $D$.  On dit que $f$ possède une \defe{limite}{limite!fonction de plusieurs variables} s'il existe un élément $\ell\in\eR$ tel que
	\begin{equation}		\label{Eq2807CondiionLimifnm}
		\forall\varepsilon>0,\,\exists\delta>0\tq 0<\| x-a \|<\delta\Rightarrow | f(x)-\ell |<\varepsilon.
	\end{equation}

	Pour une fonction $f\colon D\subset\eR^m\to \eR^n$, la définition est la même, sauf que nous remplaçons la valeur absolue par la norme dans $\eR^n$. Nous disons donc que $\ell$ est la limite de $f$ lorsque $x$ tend vers $a$, et nous notons $\lim_{x\to a} f(x)=\ell$ lorsque pour tout $\varepsilon>0$, il existe un $\delta>0$ tel que
	\begin{equation}		\label{EqDefLimRpRn}
		0<\| x-a \|_{\eR^m}<\delta\Rightarrow\,\| f(x)-\ell \|_{\eR^n}<\varepsilon.
	\end{equation}
\end{definition}

\begin{remark}
	Dans l'équation \eqref{EqDefLimRpRn}, nous avons explicitement écrit les normes $\| . \|_{\eR^m}$ et $\| . \|_{\eR^n}$. Dans la suite nous allons le plus souvent noter $\| . \|$ sans plus de précision. Il est important de faire l'exercice de bien comprendre à chaque fois de quelle norme nous parlons.
\end{remark}

\begin{remark}
	Il est important de remarquer à quel point les définitions~\ref{def_limite}, et les caractérisations~\ref{PropHOCWooSzrMjl},~\ref{PropAJQQooQQClfp} sont analogues. En réalité, la définition fondamentale est la définition de la limite dans les espaces vectoriels normés; les deux autres sont des cas particuliers, adaptés à $\eR$ et $\eR^m$. Il en sera de même pour les définitions de fonctions continues : il y aura une définition pour la continuité de fonctions entre espaces vectoriels normés, et ensuite une définition pour les fonctions de $\eR^m$ dans $\eR^n$ qui en sera un cas particulier.
\end{remark}

Tentons de comprendre ce que signifie qu'un nombre $\ell$ \emph{ne soit pas} la limite de $f$ lorsque $x\to a$. Il s'agit d'inverser la condition \eqref{Eq2807CondiionLimifnm}. Le nombre $\ell$ n'est pas une limite de $f$ pour $x\to a$ lorsque
\begin{equation}		\label{EqCaractNonLim}
	\exists\varepsilon>0\tq\,\forall\delta>0,\,\exists x\tq 0<\| x-a \|<\delta\text{ et }\| f(x)-\ell \|>\varepsilon,
\end{equation}
c'est-à-dire qu'il existe un certain seuil $\varepsilon$ tel qu'on a beau s'approcher aussi proche qu'on veut de $a$ (distance $\delta$), on trouvera toujours un $x$ tel que $f(x)$ n'est pas $\varepsilon$-proche de $\ell$.

\begin{lemma}[Unicité de la limite]
	Si $\ell$ et $\ell'$ sont deux limites de $f(x)$ lorsque $x$ tend vers $a$, alors $\ell=\ell'$.
\end{lemma}

\begin{proof}
	Soit $\varepsilon>0$. Nous considérons $\delta$ tel que $\| f(x)-\ell \|<\varepsilon$ pour tout $x$ tel que $\| x-a \|<\delta$. De la même manière, nous prenons $\delta'$ tel que $\| x-a \|<\delta'$ implique $\| f(x)-\ell' \|<\varepsilon$. Pour les $x$ tels que $\| x-a \|$ est plus petit que $\delta$ et $\delta'$ en même temps, nous avons
	\begin{equation}
		\| \ell-\ell' \|=\| \ell-f(x)+f(x)-\ell' \|\leq\| \ell-f(x) \|+\| f(x)-\ell' \|<2\varepsilon,
	\end{equation}
	et donc $\| \ell-\ell' \|=0$ parce que c'est plus petit que $2\varepsilon$ pour tout $\varepsilon$.
\end{proof}

%--------------------------------------------------------------------------------------------------------------------------- 
\subsection{Limite à gauche et à droite}
%---------------------------------------------------------------------------------------------------------------------------

Si \( a\) est à l'intérieur du domaine de \( f\), nous savons ce que signifie \( \lim_{x\to a} f(x)\). Nous donnons également une définition des limites à gauche et à droite.

\begin{definition}
    Soient \( D\subset \eR\) et une fonction \( f\colon D\to \eR\). Si \( a\in \Adh(D)\) nous définissons la \defe{limite à droite}{limite à droite} de \( f\) en \( a\) par
    \begin{equation}        \label{EQooQKHLooMoSXVe}
        \lim_{x\to a^+} f(x)=\lim_{x\to a} \tilde f(x)
    \end{equation}
    où \( \tilde f\) est la fonction \( f\) restreinte à \( D\cap\{ x\tq x>a \}\). La limite \eqref{EQooQKHLooMoSXVe} est souvent écrite sous la forme condensée
    \begin{equation}
        \lim_{\substack{x\to a\\x>a}}f(x).
    \end{equation}

    Pour la limite à gauche c'est un peu la même chose :
    \begin{equation}
        \lim_{x\to a^-} f(x)=\lim_{\substack{x\to a\\x<a}}f(x).
    \end{equation}
\end{definition}

\begin{proposition}[\cite{ooOMWZooZvUFiG}]      \label{PROPooGDDJooDCmydE}
    Soit une fonction \( f\colon D\to \eR\) où \( D\) est une partie de \( \eR\). Si \( a\in \Adh(D)\) alors la limite \( \lim_{x\to a} f(x)\) existe si et seulement si les limites à gauche et à droite existent et sont égales. Dans ce cas nous avons égalité :
    \begin{equation}
        \lim_{x\to a} f(x)=\lim_{x\to a^+} f(x)=\lim_{x\to a^-} f(x).
    \end{equation}
\end{proposition}

\begin{normaltext}
    Quelques remarques à propos de la proposition \ref{PROPooGDDJooDCmydE}.
    \begin{enumerate}
        \item
            
    Cette proposition ne se généralise pas du tout aux dimensions supérieures. Dans \( \eR^2\) par exemple, il ne faudrait pas croire que si les limites suivant toutes les directions existent alors la limite existe.
\item
    Cette proposition est souvent utilisée pour calculer des limites dans lesquelles arrivent des valeurs absolues. Par exemple durant la démonstration de la proposition \ref{PROPooCNDHooKRwils}.
    \end{enumerate}
\end{normaltext}

%+++++++++++++++++++++++++++++++++++++++++++++++++++++++++++++++++++++++++++++++++++++++++++++++++++++++++++++++++++++++++++ 
\section{Limite en compactifié d'Alexandrov}
%+++++++++++++++++++++++++++++++++++++++++++++++++++++++++++++++++++++++++++++++++++++++++++++++++++++++++++++++++++++++++++

Nous considérons l'espace topologique localement compact \( \eR\), et son compactifié d'Alexandrov défini en \ref{PROPooHNOZooPSzKIN}. Nous avons donc un point supplémentaire noté \( \infty\). Ce point n'est ni du côté des grands nombres positifs, ni du côté des grands nombres négatifs. Il n'est ni \( +\infty\) ni \( -\infty\).

\begin{proposition}
    Dans cet espace topologique \( \hat \eR=\eR\cup\{ \infty \}\),
    \begin{equation}
        \lim_{x\to 0} \frac{1}{ x }=\infty.
    \end{equation}
\end{proposition}

\begin{proof}
    Soit un voisinage \( V\) de \( \infty\) dans \( \hat \eR\). Il s'écrit \( V=K^c\cup\{ \infty \}\) pour un certain compact de \( \eR\). Le théorème \ref{ThoXTEooxFmdI} nous assure que \( K\) est borné. Donc il existe \( R>0\) tel que \( K\subset B(0,R)\). Pour \( x\in B(0,1/R)\) nous avons
    \begin{equation}
        | \frac{1}{ x } |>R,
    \end{equation}
    et donc \( 1/x\in K^c\). Donc aussi \( \frac{1}{ x }\in V\).
\end{proof}

De la même façon, dans \( \eC\cup\{ \infty \}\) nous avons
\begin{equation}
    \lim_{z\to 0} \frac{1}{ z }=\infty.
\end{equation}

\begin{normaltext}
    Je vous laisse deviner la topologie à considérer sur \( \bar \eR=\eR\cup\{ +\infty,-\infty \}\). Dans cet espace topologique la limite \( \lim_{x\to 0} \frac{1}{ x }\) n'existe pas.
\end{normaltext}

%+++++++++++++++++++++++++++++++++++++++++++++++++++++++++++++++++++++++++++++++++++++++++++++++++++++++++++++++++++++++++++
\section{Continuité}
%+++++++++++++++++++++++++++++++++++++++++++++++++++++++++++++++++++++++++++++++++++++++++++++++++++++++++++++++++++++++++++

\begin{normaltext}
    Nous allons considérer trois approches différentes de la continuité. La première sera de définir la continuité de fonctions de $\eR$ vers $\eR$ au moyen du critère usuel. Ensuite, nous définiront la continuité des applications entre n'importe quels espaces métriques, et nous montrerons que les deux définitions sont équivalentes dans le cas des fonctions sur $\eR$ à valeurs réelles.

    Enfin, un peu plus tard nous verrons que la continuité peut également être vue en termes de limites. Encore une fois nous verrons que dans le cas de fonctions de $\eR$ vers $\eR$ cette troisième approche est équivalentes aux deux premières.
\end{normaltext}

La définition de fonction continue est la définition~\ref{DefOLNtrxB}. Dans le cas d'une fonction \( f\colon \eR\to \eR\), elle devient ceci.
\begin{proposition}      \label{PROPooVNGEooPwbxXP}
    La fonction \( f\colon \eR\to \eR\) est \defe{continue en $a$}{continue sur \( \eR\)} si et seulement si
    \begin{equation}
        \forall \epsilon>0,\exists \delta\text{ tel que } \big(| x-a |\leq\delta\big)\Rightarrow | f(x)-f(a) |\leq \epsilon.
    \end{equation}
\end{proposition}

Nous allons maintenant étudier quelques conséquences de la continuité sur \( \eR\).

\begin{enumerate}
\item D'abord on voit que la continuité n'a été définie qu'en un point. On peut dire que la fonction $f$ est continue \emph{en tel point donné}, mais nous n'avons pas dit ce qu'est une fonction continue \emph{dans son ensemble}.

\item
    Le théorème \ref{ThoESCaraB} nous précise que si $I$ est un intervalle de $\eR$, la fonction $f$ est continue sur $I$ si et seulement si elle est continue en chaque point de $I$.

\item Comme la définition de $f$ continue en $a$ fait intervenir $f(x)$ pour tous les $x$ pas trop loin de $a$, il faut au moins déjà que $f$ soit définie sur ces $x$. En d'autres termes, dire que $f$ est continue en $a$ demande que $f$ existe sur un intervalle autour de $a$.

Ceci couplé à la définition précédente laisse penser qu'il est surtout intéressant d'étudier les fonctions qui sont continues sur un intervalle.

\item L'intuition comme quoi une fonction continue doit pouvoir être tracée sans lever la main correspond aux fonctions continues sur des intervalles. Au moins sur l'intervalle où elle est continue, elle est traçable en un morceau.
\end{enumerate}

\begin{example}
    Il est très possible d'être continue en un seul point. Par exemple la fonction
    \begin{equation}
        f(x)=x(1-\mtu_{\eQ}(x))
    \end{equation}
    où \( \mtu_{\eQ}\) est la fonction indicatrice de \( \eQ\) dans \( \eR\).
\end{example}

\begin{proposition}     \label{PROPooUBUAooNIxjfg}
    Si \( f\colon \eR\to \eR\) est continue au point \( a\in \eR\) et si \( f(a)\neq 0\), alors il existe un voisinage de \( a\) sur lequel \( f\) ne s'annule pas.
\end{proposition}

\begin{proof}
    Si \( f \) s'annulait sur tout voisinage de \( a\) (mais pas ne \( a\) lui-même), nous aurions, pour tout \( n\) un réel
    \begin{equation}
        x_n\in B\big( a,\frac{1}{ n } \big)\setminus\{ a \}
    \end{equation}
    tel que \( f(x_n)=0\). Cela donnerait une suite \( x_n\to a\) avec \( f(x_n)\to 0\), ce qui contredit la continuité de \( f\) en \( a\) en vertu de la proposition \ref{PropFnContParSuite}\ref{ItemWJHIooMdugfu} sur la continuité séquentielle en un point.
\end{proof}

Notons que ce résultat se généralise beaucoup : si \( f\) est continue et pas égale à \( r\) en \( a\), alors elle continue à n'être pas égale à \( r\) dans un voisinage de \( a\).

%--------------------------------------------------------------------------------------------------------------------------- 
\subsection{Opération sur la continuité}
%---------------------------------------------------------------------------------------------------------------------------

Nous allons démontrer maintenant une série de petits résultats qui permettent de simplifier la démonstration de la continuité de fonctions.
\begin{theorem}
Si la fonction $f$ est continue au point $a$, alors la fonction $\lambda f$ est également continue en $a$.
\end{theorem}

\begin{proof}
Soit $\epsilon>0$. Nous avons besoin d'un $\delta>0$ tel que pour chaque $x$ à moins de $\delta$ de $a$, la fonction $\lambda f$ soit à moins de $\epsilon$ de $(\lambda f)(a)=\lambda f(a)$. Étant donné que la fonction $f$ est continue en $a$, on sait déjà qu'il existe un $\delta_1$ (nous notons $\delta_1$ afin de ne pas confondre ce nombre dont on est sûr de l'existence avec le $\delta$ que nous sommes en train de chercher) tel que
\[
  (| x-a |\leq \delta_1)\Rightarrow | f(x)-f(a) |\leq \epsilon_1.
\]
Hélas, ce $\delta_1$ n'est pas celui qu'il faut faut parce que nous travaillons avec $\lambda f$ au lieu de $f$, ce qui fait qu'au lieu d'avoir $| f(x)-f(a) |$, nous avons $| \lambda f(x)-\lambda f(a) |=| \lambda |\cdot | f(x)-f(a) |$.  Ce que $\delta_1$ fait avec $(\lambda f)$, c'est
\[
  (| x-a |\leq\delta_1)\Rightarrow  | (\lambda f)(x)- (\lambda f)(a)|\leq | \lambda |\epsilon_1.
\]
Ce que nous apprend la continuité de $f$, c'est que pour chaque choix de $\epsilon_1$, on a un $\delta_1$ qui fait cette implication. Comme cela est vrai pour chaque choix de $\epsilon_1$, essayons avec $\epsilon_1=\epsilon/| \lambda |$ pour voir ce que ça donne. Nous avons donc un $\delta_1$ qui fait
\[
  (| x-a |\leq\delta_1)\Rightarrow  | (\lambda f)(x)- (\lambda f)(a)|\leq | \lambda |\epsilon_1=\epsilon.
\]
Ce $\delta_1$ est celui qu'on cherchait.
\end{proof}

\begin{theorem}
Si $f$ et $g$ sont deux fonctions continues en $a$, alors la fonction $f+g$ est également continue en $a$.
\end{theorem}

\begin{proof}
La continuité des fonctions $f$ et $g$ au point $a$ fait en sorte que pour tout choix de $\epsilon_1$ et $\epsilon_2$, il existe $\delta_1$ et $\delta_2$ tels que
\[
  (| x-a |\leq \delta_1)\Rightarrow | f(x)-f(a) |\leq \epsilon_1.
\]
et
\[
  (| x-a |\leq \delta_2)\Rightarrow | g(x)-g(a) |\leq \epsilon_2.
\]
La quantité que nous souhaitons analyser est $| f(x)+g(x)-f(a)-g(a) |$. Tout le jeu de la démonstration de la continuité est de triturer cette expression pour en tirer quelque chose en termes de $\epsilon_1$ et $\epsilon_2$. Si nous supposons avoir pris $| x-a |$ plus petit en même temps que $\delta_1$ et que $\delta_2$, nous avons
\[
| f(x)+g(x)-f(a)-g(a) |\leq| f(x)-g(x) |+| g(x)-g(a) |\leq\epsilon_1+\epsilon_2
\]
en utilisant la formule générale $| a+b |\leq | a |+| b |$. Maintenant si on choisit $\epsilon_1$ et $\epsilon_2$ tels que $\epsilon_1+\epsilon_2<\epsilon$, et les $\delta_1$, $\delta_2$ correspondants, on a
\[
| f(x)+g(x)-f(a)-g(a) |\leq\epsilon,
\]
pourvu que $| x-a |$ soit plus petit que $\delta_1$ et $\delta_2$. Le bon $\delta$ à prendre est donc le minimum de $\delta_1$ et $\delta_2$ qui eux-mêmes sont donnés par un choix de $\epsilon_1$ et $\epsilon_2$ tels que $\epsilon_1+\epsilon_2\leq\epsilon$.
\end{proof}

Pour résumer ces deux théorèmes, on dit que si $f$ et $g$ sont continues en $a$, alors la fonction $\alpha f+\beta g$ est également continue en $a$ pour tout $\alpha$, $\beta\in\eR$.

Parmi les propriétés immédiates de la continuité d'une fonction, nous avons ceci qui est souvent bien utile.

\begin{corollary}   \label{CorNNPYooMbaYZg}
Si la fonction $f$ est continue en $a$ et si $f(a)>0$, alors $f$ est positive sur un intervalle autour de $a$.
\end{corollary}

\begin{proof}
Prenons $\epsilon<f(a)$ et voyons\footnote{ici, nous insistons sur le fait que nous prenons $\epsilon$ \emph{strictement} plus petit que $f(a)$.} ce que la continuité de $f$ en $a$ nous offre : il existe un $\delta$ tel que
\[
  (| x-a |\leq \delta)\Rightarrow | f(x)-f(a) |\leq\epsilon < f(a).
\]
Nous en retenons que sur un intervalle (de largeur $\delta$), nous avons $| f(x)-f(a) |\leq f(a)$. Par hypothèse, $f(a)>0$, donc si $f(x)<0$, alors la différence $f(x)-f(a)$ donne un nombre encore plus négatif que $-f(a)$, c'est-à-dire que $| f(x)-f(a) |>f(a)$, ce qui est contraire à ce que nous venons de démontrer. D'où la conclusion que $f(x)>0$.
\end{proof}

\subsection{La fonction la moins continue du monde}
%--------------------------------------------------

Parmi les exemples un peu sales de fonctions non continues, il y a celle-ci :
\[
  \chi_{\eQ}(x)=
\begin{cases}
    1 \text{ si }x\in\eQ\\
    0 \text{ sinon.}
\end{cases}
\]
Par exemple, $\chi_{\eQ}(0)=1$, et\footnote{Pour prouver que $\sqrt{2}$ n'est pas rationnel, c'est pas trop compliqué, mais pour prouver que $\pi$ ne l'est pas non plus, il faudra encore manger de la soupe.} $\chi_{\eQ}(\pi)=\chi_{\eQ}(\sqrt{2})=0$. Malgré que $\chi_{\eQ}(0)=1$, il n'existe \emph{aucun} voisinage de $1$ sur lequel la fonction reste proche de $1$, parce que tout voisinage va contenir au moins un irrationnel. À chaque millimètre, cette fonction fait une infinité de bonds !

Cette fonction n'est donc continue nulle part.

À partir de là, nous pouvons construire la fonction suivante qui n'est continue qu'en un point :
\[
  f(x)=x\chi_{\eQ}(x)=
\begin{cases}
x\text{ si }x\in\eQ\\
0\text{ sinon.}
\end{cases}
\]
Cette fonction est continue en zéro. En effet, prenons $\delta>0$; il nous faut un $\epsilon$ tel que $| x |\leq\epsilon$ implique $f(x)\leq \delta$ parce que $f(0)=0$. Bon ben prendre simplement $\epsilon=\delta$ nous contente. Cette fonction est donc très facilement continue en zéro.

Et pourtant, dès que l'on s'écarte un tant soit peu de zéro, elle fait des bons une infinité de fois par millionième de millimètre ! Cette fonction est donc la plus discontinue du monde en tous les points saut un (zéro) où elle est une fonction continue !

\subsection{Approche topologique}
%--------------------------------

Nous avons vu que sur tout ensemble métrique, nous pouvons définir ce qu'est un ouvert : c'est un ensemble qui contient une boule ouverte autour de chacun de ses points. Quand on est dans un ensemble ouvert, on peut toujours un peu se déplacer sans sortir de l'ensemble.

Le théorème suivant est une très importante caractérisation des fonctions continues (de $\eR$ dans $\eR$) en termes de topologie, c'est-à-dire en termes d'ouverts.

\begin{theorem}     \label{ThoContInvOuvert}
Si $I$ est un intervalle ouvert contenu dans $\dom f$, alors $f$ est continue sur $I$ si et seulement si pour tout ouvert $\mO$ dans $\eR$, l'image inverse $f|_I^{^{-1}}(\mO)$ est ouvert.
\end{theorem}

Par abus de langage, nous exprimons souvent cette condition par « une fonction est continue si et seulement si l'image inverse de tout ouvert est un ouvert ».

\begin{proof}

Dans un premier temps, nous allons transformer le critère de continuité en termes de boules ouvertes, et ensuite, nous passerons à la démonstration proprement dite. Le critère de continuité de $f$ au point $x$ dit que
\begin{equation}        \label{EqDEfCOntAn}
  \forall \delta>0,\exists\,\epsilon>0\text{ tel que }\big( | x-a |< \epsilon \big)\Rightarrow| f(x)-f(a) |<\delta.
\end{equation}
Cette condition peut être exprimée sous la forme suivante :
\[
  \forall \delta>0,\exists\epsilon\text{ tel que } a\in B(x,\epsilon)\Rightarrow f(a)\in B\big( f(x),\delta \big),
\]
ou encore
\begin{equation}        \label{EqRedefContBoules}
  \forall \delta>0,\exists\epsilon\text{ tel que } f\big( B(x,\epsilon) \big)\subset B\big( f(x),\delta \big).
\end{equation}
Jusque ici, nous n'avons fait que du jeu de notations. Nous avons exprimé en termes de topologie des inégalités analytiques. La condition \eqref{EqRedefContBoules} est le plus souvent utilisée comme définition de la continuité d'une fonction en \( x\), lorsque le contexte ne demande pas de définitions plus générales. Si tel est le choix, il faut pouvoir retrouver \eqref{EqDEfCOntAn} à partir de \eqref{EqRedefContBoules}.

Passons maintenant à la démonstration proprement dite du théorème.

D'abord, supposons que $f$ est continue sur $I$, et prenons $\mO$, un ouvert quelconque. Le but est de prouver que $f|_I^{-1}(\mO)$ est ouvert. Pour cela, nous prenons un point $x_0\in f|_I^{-1}(\mO)$ et nous allons trouver un ouvert autour ce point contenu dans $f|_I^{-1}(\mO)$. Nous écrivons $y_0=f(x_0)$. évidemment, $y_0\in\mO$, donc on a une boule autour de $y_0$ qui est contenue dans $\mO$, soit donc $\delta>0$ tel que
\[
  B(y_0,\delta)\subset\mO.
\]
Par hypothèse, $f$ est continue en $x_0$, et nous pouvons donc y appliquer le critère \eqref{EqRedefContBoules}. Il existe donc $\epsilon>0$ tel que
\[
  f\big( B(x_0,\epsilon) \big)\subset B\big( f(x_0),\delta \big)\subset\mO.
\]
Cela prouve que $B(x_0,\epsilon)\subset f|_I^{-1}(\mO)$.

Dans l'autre sens, maintenant. Nous prenons $x_0\in I$ et nous voulons prouver que $f$ est continue en $x_0$, c'est-à-dire que pour tout $\delta$ nous cherchons un $\epsilon$ tel que $f\big( B(x_0,\epsilon) \big)\subset B\big( f(x_0),\delta \big)$. Oui, mais $B\big( f(x_0),\delta \big)$ est ouverte, donc par hypothèse, $f|_I^{-1}\Big( B\big( f(x_0),\delta \big) \Big)$ est ouvert, inclus dans $I$ et contient $x_0$. Donc il existe un $\epsilon$ tel que
\[
  B(x_0,\epsilon)\subset f|_I^{-1}\Big( B\big( f(x_0),\delta \big) \Big),
\]
et donc tel que
\[
  f\big( B(x_0,\epsilon) \big)\subset B\big( f(x_0),\delta \big),
\]
ce qu'il fallait prouver.
\end{proof}

\begin{lemma}   \label{LemConncontconn}
L'image d'un ensemble connexe par une fonction continue est connexe.
\end{lemma}

\begin{proof}
Nous allons encore faire la contraposée. Soit $A$ une partie de $\eR$ telle que $f(A)$ ne soit pas connexe. Nous allons prouver que $A$ elle-même n'est pas connexe. Dire que $f(A)$ n'est pas connexe, c'est dire qu'il existe $\mO_1$ et $\mO_2$, deux ouverts disjoints qui recouvrent $f(A)$. Je prétends que $f^{-1}(\mO_1)$ et $f^{-1}(\mO_2)$ sont ouverts, disjoints et qu'ils recouvrent $A$.
\begin{itemize}
\item Ces deux ensembles sont ouverts parce qu'ils sont images inverses d'ouverts par une fonction continue (théorème~\ref{ThoContInvOuvert}).
\item Si $x\in f^{-1}(\mO_1)\cap f^{-1}(\mO_2)$, alors $f(x)\in \mO_1\cap\mO_2$, ce qui contredirait le fait que $\mO_1$ et $\mO_2$ sont disjoints. Il n'y a donc pas d'éléments dans l'intersection de $f^{-1}(\mO_1)$ et de $f^{-1}(\mO_2)$.
\item Si $f^{-1}(\mO_1)$ et $f^{-1}(\mO_2)$ ne recouvrent pas $A$, il existe un $x$ dans $A$ qui n'est dans aucun des deux. Dans ce cas, $f(x)$ est dans $f(A)$, mais n'est ni dans $\mO_1$, ni dans $\mO_2$, ce qui contredirait le fait que ces deux derniers recouvrent $f(A)$.
\end{itemize}
Nous déduisons que $A$ n'est pas connexe. Et donc le lemme.
\end{proof}

\begin{theorem}[Théorème des valeurs intermédiaires]        \label{ThoValInter}
    Soit $f$, une fonction continue sur $[a,b]$, et supposons que $f(a)<f(b)$. Alors pour tout $y$ tel que $f(a)\leq y\leq f(b)$, il existe un \( x\in\mathopen[ a , b \mathclose]\) tel que $f(x)=y$.
\end{theorem}
\index{connexité!théorème des valeurs intermédiaires}
\index{théorème!valeurs intermédiaires}

\begin{proof}
Nous savons que $[a,b]$ est connexe parce que c'est un intervalle (proposition~\ref{PropInterssiConn}). Donc $f\big( [a,b] \big)$ est connexe (lemme~\ref{LemConncontconn}) et donc est un intervalle (à nouveau la proposition~\ref{PropInterssiConn}). Étant donné que $f\big( [a,b] \big)$ est un intervalle, il contient toutes les valeurs intermédiaires entre n'importe quels deux de ses éléments. En particulier toutes les valeurs intermédiaires entre $f(a)$ et $f(b)$.
\end{proof}

\begin{corollary}       \label{CorImInterInter}
L'image d'un intervalle par une fonction continue est un intervalle.
\end{corollary}

\begin{proof}
Soient \( I\) un intervalle, \( \alpha<\beta\in f(I)\) et \( \gamma\in\mathopen] \alpha , \beta \mathclose[\). Nous considérons \(a,b\in I\) tels que \( \alpha=f(a)\) et \( \beta=f(b)\). Par le théorème des valeurs intermédiaires \ref{ThoValInter}, il existe \( t\in\mathopen] a , b \mathclose[\) tel que \( f(t)=\gamma\). Par conséquent \( \gamma\in f(I)\).
\end{proof}

\begin{corollaryDef}[Existence de la racine carrée]
    Si \( x\geq 0\) alors il existe un unique \( y\geq 0\) tel que \( y^2=x\). Ce nombre est noté \( \sqrt{x}\) et est nommé \defe{racine carrée}{racine carrée} de \( x\).
\end{corollaryDef}

\begin{proof}
    La fonction \( f\colon t\mapsto t^2\) est continue et strictement croissante. Nous avons \( f(0)=0\) et\footnote{Faites deux cas suivant \( x\geq 1\) ou non si vous le voulez, moi je prends \( x+1\).} \( f(x+1)>x\). Donc le théorème des valeurs intermédiaires~\ref{ThoValInter} nous assure qu'il existe un unique \( y\in\mathopen[ 0 , x+1 \mathclose]\) tel que \( f(y)=x\).
\end{proof}

\subsection{Continuité de la racine carrée, invitation à la topologie induite}
%-----------------------------------------

Pourquoi nous intéresser particulièrement à cette fonction ? Parce qu'elle a une sale condition d'existence : son domaine de définition n'est pas ouvert. Or dans tous les théorèmes de continuité d'approche topologique que nous avons vus, nous avons donné des conditions \emph{pour tout ouvert}. Nous nous attendons donc a avoir des difficultés avec la continuité de $\sqrt{x}$ en zéro.

Prenons $I$, n'importe quel intervalle ouvert dans $\eR^+$, et voyons que la fonction
\begin{equation}
\begin{aligned}
 f\colon \eR^+&\to \eR^+ \\
   x&\mapsto \sqrt{x}
\end{aligned}
\end{equation}
est continue sur $I$. Remarque déjà que si $I$ est un ouvert dans $\eR^+$, il ne peut pas contenir zéro. Avant de nous lancer dans notre propos, nous prouvons un lemme qui fera tout le travail\footnote{C'est toujours ingrat d'être un lemme : on fait tout le travail et c'est toujours le théorème qui est nommé.}.

\begin{lemma}
Soit $\mO$, un ouvert dans $\eR^+$. Alors $\mO^2=\{ x^2\tq x\in\mO \}$ est également ouvert .
\end{lemma}

\begin{proof}
Un élément de $\mO^2$ s'écrit sous la forme $x^2$ pour un certain $x\in\mO$. Le but est de trouver un ouvert autour de $x^2$ qui soit contenu dans $\mO^2$. Étant donné que $\mO$ est ouvert, on a une boule centrée en $x$ contenue dans $\mO$. Nous appelons $\delta$ le rayon de cette boule :
\[
  B(x,\delta)\subset\mO.
\]
Étant donné que cet ensemble est connexe, nous savons par le lemme~\ref{LemConncontconn} que $B(x,\delta)^2$ est également connexe (parce que la fonction $x\mapsto x^2$ est continue). Son plus grand élément est $(x+\delta)^2=x^2+\delta^2+2x\delta>x^2+\delta^2$, et son plus petit élément est $(x-\delta)^2=x^2+\delta^2-2x\delta$.

Ce qui serait pas mal, c'est que ces deux bornes entourent $x^2$; de cette façon elles définiraient un ouvert autour de $x^2$ qui soit dans $\mO^2$. Hélas, c'est pas gagné que $x^2+\delta^2-2x\delta$ soit plus petit que $x^2$.

Heureusement, en fait c'est vrai parce que d'une part, du fait que $\mO\subset\eR^+$, on a $x>0$, et d'autre part, pour que $\mO$ soit positif, il faut que $\delta<x$. Donc on a évidemment que $\delta<2x$, et donc que
\[
  x^2+\delta^2-2x\delta=x^2+\delta\underbrace{(\delta-2x)}_{<0}<x^2.
\]
Donc nous avons fini : l'ensemble
\[
  B(x,\delta)^2=]x^2+\delta^2-2x\delta,x^2+\delta^2+2x\delta[\subset\mO^2
\]
est un intervalle qui contient $x^2$, et donc qui contient une boule ouverte centrée en~$x^2$.

\end{proof}

Maintenant nous pouvons nous attaquer à la continuité de la racine carrée sur tout ouvert positif en utilisant le théorème~\ref{ThoContInvOuvert}. Soit $\mO$ n'importe quel ouvert de $\eR$, et prouvons que $f|_I^{-1}(\mO)$ est ouvert. Par définition,
\begin{equation}
  f|_I^{-1}(\mO)=\{ x\in I\tq \sqrt{x}\in\mO \}.
\end{equation}
Maintenant c'est un tout petit effort que de remarquer que $f|_I^{-1}(\mO)=\mO^2\cap I$. De là, on a gagné parce que $\mO^2$ et $I$ sont des ouverts. Or l'intersection de deux ouverts est ouvert.

Nous n'en avons pas fini avec la fonction $\sqrt{x}$. Nous avons la continuité de la racine carrée pour tous les réels strictement positifs. Il reste à pouvoir dire que la fonction est continue en zéro malgré qu'elle ne soit pas définie sur un ouvert autour de zéro.

Il est possible de dire que la racine carrée est continue en $0$, malgré qu'elle ne soit pas définie sur un ouvert autour de $0$\ldots en tout cas pas un ouvert au sens que tu as en tête. Nous allons rentabiliser un bon coup notre travail sur les espaces métriques.

Nous pouvons définir la notion de boule ouverte sur n'importe quel espace métrique $A$ en disant que
\[
  B(x,r)=\{ y\in A\tq d(x,y)<r \}.
\]
\begin{definition}      \label{DefContMetrique}
Soit $f\colon A\to B$, une application entre deux espaces métriques. Nous disons que $f$ est \defe{continue}{continue!sur espace métrique} au point $a\in A$ si $\forall \delta>0$, $\exists\epsilon>0$ tel que
\begin{equation}
  f\big( B(a,\epsilon) \big)\subset B\big( f(a),\delta \big).
\end{equation}
\end{definition}
Tu reconnais évidemment la condition \eqref{EqRedefContBoules}. Nous l'avons juste recopiée. Tu remarqueras cependant que cette définition généralise immensément la continuité que l'on avait travaillé à propos des fonctions de $\eR$ vers $\eR$. Maintenant tu peux prendre n'importe quel espace métrique et c'est bon.

Nous n'allons pas faire un tour complet des conséquences et exemples de cette définition. Au lieu de cela, nous allons juste montrer en quoi cette définition règle le problème de la continuité de la racine carrée en zéro.

La fonction que nous regardons est
\begin{equation}
\begin{aligned}
f \colon \eR^+&\to \eR^+ \\
   x&\mapsto \sqrt{x}.
\end{aligned}
\end{equation}
Mais cette fois, nous ne la voyons pas comme étant une fonction dont le domaine est une partie de $\eR$, mais comme fonction dont le domaine est $\eR^+$ vu comme un espace métrique en soi. Quelles sont les boules ouvertes dans $\eR^+$ autour de zéro ? Réponse : la boule ouverte de rayon $r$ autour de zéro dans $\eR^+$ est :
\[
  B(0,r)_{\eR^+}=\{ x\in\eR^+\tq d(x,0)<r \}=[0,r[.
\]
Cet intervalle est un ouvert. Aussi incroyable que cela puisse paraitre !

Testons la continuité de la racine carrée en zéro dans ce contexte. Il s'agit de prendre $A=\eR^+$, $B=\eR^+$ et $a=0$ dans la définition~\ref{DefContMetrique}. Nous avons que $B(\sqrt{0},\delta)=B(0,\delta)=[0,\delta[$ pour la topologie de $\eR^+$.

Il s'agit maintenant de trouver un $\epsilon$ tel que $f\big( B(0,\epsilon) \big)\subset [0,\delta[$. Par définition, nous avons que
\[
  f\big( B(0,\epsilon) \big)=[0,\sqrt{\epsilon}[,
\]
le problème revient dont à trouver $\epsilon$ tel que $\sqrt{\epsilon}\leq\delta$. Prendre $\epsilon<\delta^2$ fait l'affaire.


Donc voilà. Au sens de la \href{http://fr.wikipedia.org/wiki/Topologie_induite}{topologie propre} à $\eR^+$, nous pouvons dire que la fonction racine carrée est partout continue.

%---------------------------------------------------------------------------------------------------------------------------
\subsection{Prolongement par continuité}
%---------------------------------------------------------------------------------------------------------------------------

%///////////////////////////////////////////////////////////////////////////////////////////////////////////////////////////
\subsubsection{Discussion avec mon ordinateur}
%///////////////////////////////////////////////////////////////////////////////////////////////////////////////////////////

Voici un extrait de ce peut donner Sage. Nous lui donnons la fonction
\begin{equation}    \label{EqyEHTBZ}
    f(x)=\frac{ x+4 }{ 3x^2+10x-8 }.
\end{equation}
Cette fonction est faite exprès pour que le dénominateur s'annule en \( -4\). En fait \( 3x^2+10x-8=(x+4)(3x-2)\), et la fraction peut se simplifier en
\begin{equation}
    f(x)=\frac{1}{ 3x-2 }.
\end{equation}
Et avec cela nous écririons \( f(-4)=-\frac{1}{ 14 }\). Voyons comment cela passe dans Sage.

\begin{verbatim}
----------------------------------------------------------------------
| Sage Version 5.2, Release Date: 2012-07-25                         |
| Type "notebook()" for the browser-based notebook interface.        |
| Type "help()" for help.                                            |
----------------------------------------------------------------------
sage: f(x)=(x+4)/(3*x**2+10*x-8)
sage: f(-4)
---------------------------------------------------------------------------
ValueError                                Traceback (most recent call last)
ValueError: power::eval(): division by zero
\end{verbatim}
Il produit donc une erreur de division par zéro. Cela n'est pas étonnant. Pourtant si on lui demande, il est capable de simplifier. En effet :
\begin{verbatim}
sage: f.simplify_full()
x |--> 1/(3*x - 2)
sage: f.simplify_full()(-4)
-1/14
\end{verbatim}

Nous considérons la question suivante : étant donné une fonction \( f\) définie sur \( I\setminus\{ x_0 \}\), est-il possible de définir \( f\) en \( x_0\) de telles façon à ce qu'elle soit continue ?

\begin{example}
    La fonction
    \begin{equation}
        \begin{aligned}
            f\colon \eR\setminus\{ 0 \}&\to \eR \\
            x&\mapsto \frac{1}{ x }
        \end{aligned}
    \end{equation}
    n'est pas définie pour \( x=0\) et il n'y a pas moyen de définir \( f(0)\) de telle sorte que \( f\) soit continue parce que \( \lim_{x\to 0} \frac{1}{ x }\) n'existe pas.
\end{example}

%///////////////////////////////////////////////////////////////////////////////////////////////////////////////////////////
\subsubsection{Limite et prolongement}
%///////////////////////////////////////////////////////////////////////////////////////////////////////////////////////////

Reprenons l'exemple de la fonction \eqref{EqyEHTBZ} que mon ordinateur refusait de calculer en zéro :
\begin{equation}
f(x)=\frac{ x+4 }{ 3x^2+10x-8 }=\frac{ x+4 }{ (x+4)\left( x-\frac{ 2 }{ 3 } \right) }.
\end{equation}
Cette fonction a une condition d'existence en $x=-4$. Et pourtant, tant que $x\neq 4$, cela a un sens de simplifier les $(x+4)$ et d'écrire
\[
  f(x)=\frac{ 1 }{ x-\frac{ 2 }{ 3 } }=\frac{ 3 }{ 3x-2 }.
\]
Étant donné que pour toute valeur de $x$ différente de $-4$, la fonction $f$ s'exprime de cette façon, nous avons que
\[
  \lim_{x\to -4}f(x)=\lim_{x\to -4}\left(\frac{ 3 }{ 3x-2 }\right).
\]
Oui, mais la fonction\footnote{Cette fonction $g$ n'est pas $f$ parce que $g$ a en plus l'avantage d'être définie en $-4$.} $g(x)=3/(3x-2)$ est continue en $-4$ et donc sa limite vaut sa valeur. Nous en déduisons que
\[
  \lim_{x\to -4}f(x)=-\frac{ 3 }{ 14 }.
\]
Que dire maintenant de la fonction ainsi définie ?
\begin{equation}
\tilde f(x)=
\begin{cases}
f(x)&\text{si }x\neq -4\\
-3/14&\text{si }x=-4.
\end{cases}
\end{equation}
Cette fonction est continue en $-4$ parce qu'elle y est égale à sa limite. Les étapes suivies pour obtenir ce résultat sont :
\begin{itemize}
\item Repérer un point où la fonction n'existe pas,
\item calculer la limite de la fonction en ce point, et en particulier vérifier que cette limite existe, ce qui n'est pas toujours le cas,
\item définir une nouvelle fonction qui vaut partout la même chose que la fonction originale, sauf au point considéré où l'on met la valeur de la limite.
\end{itemize}
C'est ce qu'on appelle \defe{prolonger la fonction par continuité}{prolongement!par continuité} parce que la fonction résultante est continue. La prolongation de $f$ par continuité est donc en général définie par
\begin{equation}
\tilde f(x)=
\begin{cases}
f(x)            &\text{si }f(x)\\
\lim_{y\to x}f(y)   &\text{si }f(x)
\end{cases}
\end{equation}
Dans le cas que nous regardions,
\[
    f(x)=\frac{ x+4 }{ 3x^2+10x-8 },
\]
le prolongement par continuité est donné par
\begin{equation}
\tilde f =\frac{ 3 }{ 3x-2 }.
\end{equation}
Remarquons que cette fonction n'est toujours pas définie en $x=2/3$.

%---------------------------------------------------------------------------------------------------------------------------
\subsection{Prolongement par continuité}
%---------------------------------------------------------------------------------------------------------------------------

\begin{propositionDef}[Prolongement par continuité]
    Soit \( f\colon I\setminus\{ x_0 \}\to \eR\) telle que \( \lim_{x\to x_{0}} f(x)=\ell\in \eR\). La fonction
    \begin{equation}
        \begin{aligned}
            \tilde f\colon I&\to \eR \\
            \tilde f(x)&=\begin{cases}
                f(x)    &   \text{si } x\neq x_0\\
                \ell    &    \text{si } x=x_0
            \end{cases}
        \end{aligned}
    \end{equation}
    est une fonction continue sur \( I\) et est appelée le \defe{prolongement par continuité}{prolongement!par continuité} de \( f\) en \( x_0\).
\end{propositionDef}
Vous noterez que dans cet énoncé nous demandons \( \ell\in \eR\). Les cas \( \ell=\pm\infty\) sont donc exclus.

\begin{normaltext}
    Le lemme~\ref{LEMooUAFBooAwiXxj} donnera un autre gros morceau de prolongement par continuité. Là, ce ne sera pas juste une valeur qui manquera, mais carrément la majorité des valeurs; mais par contre, ce ne sera pas vraiment de la prolongation par continuité, mais de la prolongation par Cauchy-continuité.
\end{normaltext}

\begin{example}
    La fonction
    \begin{equation}
        \begin{aligned}
            f\colon \eR\setminus\{ -3,2 \}&\to \eR \\
            x&\mapsto  \frac{ x^2+2x-3 }{ (x+3)(x-2) }
        \end{aligned}
    \end{equation}
    admet pour limite \( \lim_{x\to -3} f(x)=\frac{ 4 }{ 5 }\). Son prolongement par continuité en \( x=-3\) est donné par
    \begin{equation}
        \tilde f(x)=\frac{ x-1 }{ x-2 }.
    \end{equation}
    Notons que les fonctions \( f\) et \( \tilde f\) ne sont pas identiques : l'une est définie pour \( x=-3\) et l'autre pas. Lorsqu'on fait le calcul
    \begin{equation}
        \frac{ x^2+2x-3 }{ (x+3)(x-2) }=\frac{ (x-1)(x+3) }{ (x+3)(x-2) }=\frac{ x-1 }{ x-2 },
    \end{equation}
    la simplification n'est pas du tout un acte anodin. Le dernier signe «\( =\)» est discutable parce que les deux dernières expressions ne sont pas égales pour tout \( x\); elles ne sont égales «que» pour les \( x\) pour lesquels les deux expressions existent.
\end{example}

Les fonctions trigonométriques donneront quelques exemples intéressants de prolongements par continuité. Voir l'exemple~\ref{ExQWHooGddTLE}. Et une avec la fonction logarithme dans l'exemple~\ref{EXooAGEOooQdQkrS}.

%---------------------------------------------------------------------------------------------------------------------------
\subsection{Théorème de la bijection}
%---------------------------------------------------------------------------------------------------------------------------

\begin{proposition} \label{PropOARooUuCaYT}
    Une fonction monotone et surjective d'un intervalle $I$ sur un autre intervalle $J$ est continue sur $I$.
\end{proposition}

\begin{proposition}
    Soient \( f\colon I\to J\) une bijection et \( f^{-1}\colon J\to I\) sa réciproque. Alors pour tout \( x_0\in I\) nous avons
    \begin{equation}    \label{EqHQRooNmLYbF}
        f^{-1}\big( f(x_0) \big)=x_0
    \end{equation}
    et pour tout \( y_0\in J\) nous avons
    \begin{equation}    \label{EqIYTooQPvZDr}
        f\big( f^{-1}(y_0) \big)=y_0.
    \end{equation}
\end{proposition}

\begin{proof}
    Nous prouvons la relation \eqref{EqHQRooNmLYbF} et nous laissons \eqref{EqIYTooQPvZDr} comme exercice au lecteur.

    Soit \( x_0\in I\). Posons \( y_0=f(x_0)\). La définition de l'application réciproque est que pour \( y\in J\), \( f^{-1}(y)\) est l'unique élément \( x\) de \( I\) tel que \( f(x)=y\). Donc \( f^{-1}(y_0)\) est l'unique élément de \( I\) dont l'image est \( y_0\). C'est donc \( x_0\) et nous avons \( f^{-1}(y_0)=x_0\), c'est-à-dire
    \begin{equation}
        f^{-1}\big( f(x_0) \big)=x_0.
    \end{equation}
\end{proof}

\begin{theorem}[Théorème de la bijection] \label{ThoKBRooQKXThd}
    Soit $I$ un intervalle et $f$ une fonction continue et strictement monotone de $I$ dans \( \eR\). Nous avons alors :
    \begin{enumerate}
        \item
            $f(I)$ est un intervalle de \( \eR\) ;
        \item       \label{ITEMooMAWXooZXmVwA}
            La fonction \( f\colon I\to f(I)\) est bijective
        \item
            La fonction \( f^{-1}\colon f(I)\to I\) est strictement monotone de même sens que $f$ ;
        \item \label{ItemEJZooKuFoeFiv}
            La fonction \( f\colon I\to f(I)\) est un homéomorphisme, c'est-à-dire que \( f^{-1}\colon f(I)\to I\) est continue.
    \end{enumerate}
\end{theorem}

\begin{proof}

    Prouvons les choses point par point.

    \begin{enumerate}
    \item

        Supposons pour fixer les idées que \( f\) est monotone croissante\footnote{Traitez en tant qu'exercice le cas où $ f$ est décroissante.}.

        Soient \( a< b\) dans \( f(I)\). Par définition il existe \( x_1,x_2\in I\) tels que \( a=f(x_1)\) et \( b=f(x_2)\). La fonction \( f\) est continue sur l'intervalle \( \mathopen[ x_1 , x_2 \mathclose]\) et vérifie \( f(x_1)<f(x_2)\). Donc le théorème des valeurs intermédiaires~\ref{ThoValInter} nous dit que pour tout \( t\) dans \( \mathopen[ f(x_2) , f(x_2) \mathclose]\), il existe un \( x_0\in\mathopen[ x_1 , x_2 \mathclose]\) tel que \( f(x_0)=t\). Cela montre que toutes les valeurs intermédiaires entre \( a\) et \( b\) sont atteintes par \( f\) et donc que \( f(I)\) est un intervalle.

    \item

    Nous prouvons maintenant que \( f\) est bijective en prouvant séparément qu'elle est surjective et injective.

    \begin{subproof}

        \item[\( f\) est surjective]

            Une fonction est toujours surjective depuis un intervalle \( I\) vers l'ensemble \(\Im f \).

        \item[\( f\) est injective]

            Soit \( x\neq y\) dans \( I\); pour fixer les idées nous supposons que \( x<y\). La stricte monotonie de \( f\) implique que \( f(x)<f(y)\) ou que \( f(x)>f(y)\). Dans tous les cas \( f(x)\neq f(y)\).

    \end{subproof}

    La fonction \( f\) est donc bijective.

\item

    Comme d'accoutumée nous supposons que \( f\) est croissante. Soient \( y_1<y_2\) dans \( f(I)\); nous devons prouver que \( f^{-1}(y_1)\leq f^{-1}(y_2)\). Pour cela nous considérons les nombres \( x_1,x_2\in I\) tels que \( f(x_1)=y_1\) et \( f(x_2)=y_2\). Nous allons en prouver la contraposée en supposant que \( f^{-1}(y_1)>f^{-1}(y_2)\). En appliquant \( f\) (qui est croissante) à cette dernière inégalité il vient
    \begin{equation}
        f\big( f^{-1}(y_1) \big)\geq f\big( f^{-1}(y_2) \big),
    \end{equation}
    ce qui signifie
    \begin{equation}
        y_1\geq y_2
    \end{equation}
    par l'équation \eqref{EqIYTooQPvZDr}.

\item

    La fonction \( f^{-1}\colon f(I)\to I\) est une fonction monotone et surjective, donc continue par la proposition~\ref{PropOARooUuCaYT}.

    \end{enumerate}
\end{proof}

\begin{example}
    La fonction
    \begin{equation}
        \begin{aligned}
            f\colon \mathopen[ 2 , 3 \mathclose]&\to \mathopen[ 4 , 9 \mathclose] \\
            x&\mapsto x^2
        \end{aligned}
    \end{equation}
    est une bijection. Sa réciproque est la fonction
    \begin{equation}
        \begin{aligned}
            f^{-1}\colon \mathopen[ 4 , 9 \mathclose]&\to \mathopen[ 2 , 3 \mathclose] \\
            x&\mapsto \sqrt{x}.
        \end{aligned}
    \end{equation}
\end{example}

%+++++++++++++++++++++++++++++++++++++++++++++++++++++++++++++++++++++++++++++++++++++++++++++++++++++++++++++++++++++++++++
\section{Limite et continuité}
%+++++++++++++++++++++++++++++++++++++++++++++++++++++++++++++++++++++++++++++++++++++++++++++++++++++++++++++++++++++++++++
\label{SecLimiteFontion}

Voir les remarques dans l'index thématique~\ref{THEMEooGVCCooHBrNNd} pour comprendre la place et la portée de ce qui va venir à propos de limite et de continuité.

\begin{theorem}[Limite et continuité]           \label{ThoLimCont}
La fonction $f$ est continue au point $a$ si et seulement si $\lim_{x\to a}f(x)=f(a)$.
\end{theorem}

\begin{proof}
    Nous commençons par supposer que $f$ est continue en $a$, et nous prouvons que $\lim_{x\to a}f(x)=a$. Soit $\epsilon>0$; ce qu'il nous faut c'est un $\delta$ tel que $| x-a |\leq\delta$ implique $| f(x)-f(a) |\leq\epsilon$. La caractérisation \ref{PROPooVNGEooPwbxXP} de la continuité donne l'existence d'un $\delta$ comme il nous faut.

    Dans l'autre sens, c'est-à-dire prouver que $f$ est continue au point $a$ sous l'hypothèse que $\lim_{x\to a}f(x)=f(a)$, la preuve se fait de la même façon.
\end{proof}

Nous en déduisons que si nous voulons gagner quelque chose à parler de limites, il faut prendre des fonctions non continues. En effet, si une fonction est continue en un point, la limite ne donne aucune nouvelle information que la valeur de la fonction elle-même en ce point.

Prenons une fonction qui fait un saut. Pour se fixer les idées, prenons celle-ci :
\begin{equation}    \label{EqnCtOEL}
f(x)=
\begin{cases}
2x&\text{si }x\in]\infty,2[\\
x/2&\text{si }x\in[2,\infty[
\end{cases}
\end{equation}
Essayons de trouver la limite de cette fonction lorsque $x$ tend vers $2$. Étant donné que $f$ n'est pas continue en $2$, nous savons déjà que $\lim_{x\to 2}f(x)\neq f(2)$. Donc ce n'est pas $1$. Cette limite ne peut pas valoir $4$ non plus parce que si je prends n'importe quel $\epsilon$, la valeur de $f(2+\epsilon)$ est très proche de $2$, et donc ne peut pas s'approcher de $4$. En fait, tu peux facilement vérifier que \emph{aucun nombre ne vérifie la condition de limite pour $f$ en $2$}. Nous disons que la limite n'existe pas.

Il ne faudrait pas en déduire trop vite que si une fonction n'est pas continue en \( a\), alors la limite \( x\to a\) n'existe pas. Ce que dit le théorème~\ref{ThoLimCont} est que si une fonction n'est pas continue en \( a\), alors sa limite (si elle existe) ne vaut pas \( f(a)\).

\begin{example}[Un exemple de continuité Thème~\ref{THEMEooGVCCooHBrNNd}]     \label{EXooKREUooLeuIlv}
    Soit la fonction
    \begin{equation}        \label{EQooSYSWooSGsUfR}
        f(x)=\begin{cases}
            x    &   \text{si } x\neq 0\\
            4    &    \text{si } x=0.
        \end{cases}
    \end{equation}
    Cette fonction n'est pas continue en \( x=0\), et pourtant la limite existe : \( \lim_{x\to 0} f(x)=0\). Faisons cela en détail pour nous assurer de ce qu'il se passe.

    Considérons l'ouvert \( \mathopen] 3 , 5 \mathclose[\). L'image réciproque de cet ouvert par \( f\) est la partie \( \mathopen] 3 , 5 \mathclose[\cup\{ 0 \}\) qui n'est pas ouvert. Donc la fonction \( f\) n'est pas continue comme fonction \( \eR\to \eR\).

    Considérons pour comprendre la restriction \( f\colon \mathopen[ -1 , 1 \mathclose]\to \eR\). L'image inverse de \( \mathopen] 3 , 5 \mathclose[\) par cette fonction est \( \{ 0 \}\) qui n'est pas un ouvert.

    Plus généralement tant qu'on considère des restrictions de \( f\) sur des parties contenant un voisinage de \( 0\), la fonction ne peut pas être continue\footnote{Les plus acharnés se demanderont ce qu'il se passe pour la restriction de \( f\) à la partie \( \{ 0 \}\) munie de la topologie induite de $\eR$.}.

    Voyons ce qui en est de la continuité ponctuelle de \( f\) en \( x=0\). La définition~\ref{DefOLNtrxB} est celle de la continuité en un point; elle dit que \( f\) sera continue en \( 0\) si \( f(0)=4\) est une limite de \( f\). Nous voila parti vers la définition~\ref{DefYNVoWBx}.

Soit le voisinage \( V=\mathopen] 3 , 5 \mathclose[\) de \( f(0)\). Quel que soit le voisinage \( W\) de \( 0\) dans \( \eR\), il existe un \( \epsilon>0\) tel que \( W\subset B(0,\epsilon)\). Nous avons alors
    \begin{equation}
        f\big( W\setminus \{ a \} \big)\subset f\big( B(0,\epsilon)\setminus\{ 0 \} \big).
    \end{equation}
    Mais le nombre \( \epsilon/2\) fait partie de \( f\big( B(0,\epsilon)\setminus\{ 0 \} \big)\) et n'est pas dans \( V\). Donc \( f(0)\) n'est pas une limite de \( f\) en zéro. Cette fonction n'est donc pas continue en zéro.
\end{example}

\begin{example}[Même exemple, limite]
    Nous avons vu que, pour la fonction \eqref{EQooSYSWooSGsUfR}, le nombre \( 4\) n'est pas une limite de \( f\) en zéro. Nous montrons à présent que \( 0\) est une limite (et même la seule par la proposition~\ref{PropFObayrf} que nous ne rappellerons plus à chaque fois) de \( f\).

    Montrons que \( 0\) est une limite de \( f\) en zéro, c'est-à-dire que \( \lim_{x\to 0} f(x)=0\).

    Nous suivant la définition~\ref{DefYNVoWBx}. Soit un voisinage \( V\) de \( 0\) dans \( \eR\). Il existe \( \delta\) tel que \( B(0,\delta)\subset V\). En posant \( \epsilon=\delta\) et en définissant \( W=B(0,\epsilon)\) nous avons
    \begin{equation}
        f\big( B(0,\epsilon)\setminus\{ 0 \} \big)=B(0,\epsilon)\setminus\{ 0 \}\subset  B(0,\delta)\subset V.
    \end{equation}
    Donc \( 0\) est une limite de \( f\) en zéro.
\end{example}

Nous avons déjà vu par le corolaire~\ref{CorFHbMqGGyi} qu'une suite croissante et bornée était convergente. Il en va de même pour les fonctions.
\begin{proposition}[\cite{MonCerveau}] \label{PropMTmBYeU}
    Si la fonction réelle \( f\colon I=\mathopen[ a , b [\to \eR\) est croissante et bornée, alors la limite
    \begin{equation}
        \lim_{x\to b} f(x)
    \end{equation}
    existe et est finie.
\end{proposition}

\begin{proof}
    Commençons par prouver que si \( (x_n)\) est une suite dans \( I\) convergent vers \( b\), alors \( f(x_n)\) est une suite convergente. Dans un second temps nous allons prouver que si \( (x_n)\) et \( (x'_n)\) sont deux suites qui convergent vers \( b\), alors les suites convergentes \( f(x_n)\) et \( f(x'_n)\) convergent vers la même limite. Alors le critère séquentiel de la limite d'une fonction conclura (proposition~\ref{PROPooJYOOooZWocoq}).

    Nous pouvons extraire de \( x_n\) une sous-suite croissante \( (x_{\alpha(n)})\). Alors la suite \( f\big( x_{\alpha(n)} \big)\) est une suite croissante et majorée, donc convergente par le corolaire~\ref{CorFHbMqGGyi}\footnote{En gros nous sommes en train de dire que toute la théorie des fonctions convexes est un vulgaire corolaire de Bolzano-Weierstrass.}. Nommons \( \ell\) la limite et montrons qu'elle est aussi limite de \( f\) sur la suite originale.

    Pour tout \( \epsilon>0\), il existe \( K\) tel que si \( n>K\) alors \( \big| f\big( x_{\alpha(n)} \big)-\ell \big|<\epsilon\). Soit \( K'\) tel que pour tout \( n>K'\) nous ayons \( x_n>x_{\alpha(K')}\). Cela est possible parce que la suite est bornée par \( b\) et converge vers \( b\) : il suffit de prendre \( K'\) de telle sorte que \( | x_n-b |\leq | x_{\alpha(n)}-b |\). Si \( n>K'\) alors \( x_n>x_{\alpha(K)}\) et
    \begin{equation}
        f(x_n)\geq f(x_{\alpha(n)})\geq \ell-\epsilon;
    \end{equation}
    en résumé si \( n>K\) alors \( | f(x_n)-\ell |<\epsilon\). Cela prouve que \( f(x_n)\to\ell\).

    Soit maintenant une autre suite \( (x'_n)\) qui converge également vers \( b\). Comme nous venons de le voir la suite \( f(x'_n)\) est convergente et nous nommons \( \ell'\) la limite. Si nous considérons \( (x''_n)\) la suite «alternée» (\( x_1,x'_1,x_2,x'_2,\cdots\)) alors nous avons encore une suite qui converge vers \( b\) et donc \( f(x''_n)\to \ell'\).

    Mais étant donné que \( f(x_n)\) et \( f(x'_n)\) sont des sous-suites, elles doivent converger vers la même valeur. Donc \( \ell=\ell'=\ell''\).
\end{proof}

%TODO : écrire un truc sur la limite à gauche et la limite pour la topologie induite.

%---------------------------------------------------------------------------------------------------------------------------
\subsection{Règles simples de calcul}
%---------------------------------------------------------------------------------------------------------------------------

Les opérations simples passent à la limite, sauf la division pour laquelle il faut faire attention au dénominateur.
\begin{proposition}     \label{PropOpsSimplesLimites}
    Soient \( f\) et \( g\) deux fonctions telles que \( \lim_{x\to a} f(x)=\alpha\) et \( \lim_{x\to a} g(x)=\beta\). Alors
    \begin{enumerate}
        \item
            \( \lim_{x\to a} f(x)+g(x)=\alpha+\beta\),
        \item
            \( \lim_{x\to a} f(x)g(x)=\alpha\beta\),
        \item
            s'il existe un voisinage de \( a\) sur lequel \( g\) ne s'annule pas, alors \( \lim_{x\to a} \frac{ f(x) }{ g(x) }=\frac{ \alpha }{ \beta }\).
    \end{enumerate}
\end{proposition}

Le résultat suivant est pratique pour le calcul des limites.
\begin{proposition}     \label{PropChmVarLim}
Quand la limite existe, nous avons
\[
  \lim_{x\to a}f(x)=\lim_{\epsilon\to 0}f(a+\epsilon),
\]
ce qui correspond à un «changement de variables» dans la limite.
\end{proposition}

\begin{proof}
Si $A=\lim_{x\to a}f(x)$, par définition,
\begin{equation}        \label{EqCondFaplusespLim}
\forall\epsilon'>0,\,\exists\delta\text{ tel que }| x-a |\leq\delta\Rightarrow| f(x)-A |\leq\epsilon'.
\end{equation}
La seule subtilité de la démonstration est de remarquer que si $| x-a |\leq\delta$, alors $x$ peut être écrit sous la forme $x=a+\epsilon$ pour un certain $| \epsilon |\leq\delta$. En remplaçant $x$ par $a+\epsilon$ dans la condition~\ref{EqCondFaplusespLim}, nous trouvons
\begin{equation}
\forall\epsilon'>0,\,\exists\delta\text{ tel que }| \epsilon |\leq\delta\Rightarrow| f(x+\epsilon)-A |\leq\epsilon',
\end{equation}
ce qui signifie exactement que $\lim_{\epsilon\to 0}f(x+\epsilon)=A$.
\end{proof}

Il y a une petite différence de point de vue entre $\lim_{x\to a}f(x)$ et $\lim_{\epsilon\to 0}f(a+\epsilon)$. Dans le premier cas, on considère $f(x)$, et on regarde ce qu'il se passe quand $x$ se rapproche de $a$, tandis que dans le second, on considère $f(a)$, et on regarde ce qu'il se passe quand on s'éloigne un tout petit peu de $a$. Dans un cas, on s'approche très près de $a$, et dans l'autre on s'en éloigne un tout petit peu. Le contenu de la proposition~\ref{PropChmVarLim} est de dire que ces deux points de vue sont équivalents.

% Il y a des techniques de calcul de limites décrites sur le site
% http://bernard.gault.free.fr/terminale/limites/limite.html

%---------------------------------------------------------------------------------------------------------------------------
\subsection{Prolongement des rationnels vers les réels}
%---------------------------------------------------------------------------------------------------------------------------

Si \( f\colon \eQ\to \eR\) est une fonction continue pour la topologie induite, est-ce qu'on peut la prolonger en une fonction continue sur \( \eR\) ? La réponse est hélas non.

\begin{example}[\cite{BIBooOFHOooWZGRPw}]       \label{EXooWZNCooQkKdtJ}
    Vu que \( \sqrt{ 2 }\) est irrationnel\footnote{Proposition \ref{PropooRJMSooPrdeJb}. Le fait que $\sqrt{ 2 }$ existe dans \( \eR\) est la proposition \ref{PROPooUHKFooVKmpte}.}, ceci définit bien une fonction sur \( \eQ\) :
    \begin{equation}
        \begin{aligned}
            f\colon \eQ&\to \eR \\
            q&\mapsto \begin{cases}
                0    &   \text{si } q<\sqrt{ 2 }\\
                1    &    \text{si }q>\sqrt{ 2 }.
            \end{cases}
        \end{aligned}
    \end{equation}
    Cela est une fonction continue sur \( \eQ\). En effet, soient \( q\in \eQ\) et \( \epsilon>0\). Nous prenons \( \delta>0\) tel que \( \sqrt{ 2 }\) ne soit pas dans \( B(q,\delta)\). Alors si \( p\in B_{\eQ}(q,\delta)\) nous avons \( f(q)=f(p)\) et donc 
    \begin{equation}
        | f(p)-f(q) |<\epsilon.
    \end{equation}

    Il n'est cependant pas possible de la prolonger en une fonction continue sur \( \eR\).
\end{example}

Pour qu'une fonction sur \( \eQ\) puisse être prolongée en une fonction continue sur \( \eR\), il faut un peu plus que la continuité. Il fait la Cauchy-continuité que nous définissons pas plus tard qu'immédiatement.

\begin{definition}[\cite{BIBooOFHOooWZGRPw}]        \label{DEFooXXOGooXblyKP}
    Soient \( X\) et \( Y\) deux espaces métriques. Une application \( f\colon X\to Y\) est dite \defe{Cauchy-continue}{Cauchy-continue} si pour toute suite de Cauchy \( (x_n)\) dans \( X\), la suite \( \big( f(x_n) \big)\) est de Cauchy dans \( Y\).
\end{definition}

En terme de prolongement continu, nous avons ce lemme qui demande à une fonction d'être Cauchy continue. Vous pouvez comparer avec le principe de prolongement analytique \ref{ThoAVBCewB} qui donne un énoncé similaire pour un prolongement analytique.
\begin{lemma}[\cite{MonCerveau, BIBooUNVDooGfFtGp,BIBooERJSooURHjMX}]        \label{LEMooUAFBooAwiXxj}
    Soit une fonction Cauchy continue \footnote{Définition \ref{DEFooXXOGooXblyKP}; nous en avons discuté dans l'exemple~\ref{EXooWZNCooQkKdtJ}.} \( f\colon \eQ\to \eR\).
    \begin{enumerate}
        \item
            La limite \( \lim_{q\to x} f(q)\) existe pour tout \( x\in \eR\).
        \item
            Il existe un unique prolongement continu \( \tilde f\colon \eR\to \eR\).
        \item
            Ce prolongement est donné par
            \begin{equation}
            \tilde f(x)=\begin{cases}
                f(x)    &   \text{si } x\in \eQ\\
                \lim_{q\to x} f(q)    &    \text{sinon }
            \end{cases}
        \end{equation}
    \end{enumerate}
\end{lemma}

\begin{proof}
    Imprégniez vous bien de la la définition~\ref{DefYNVoWBx} de la limite avant de commencer.

    \begin{subproof}

    \item[Unicité]

        Prouvons rapidement l'unicité avant l'existence parce que c'est facile.

        L'unicité du prolongement est la proposition~\ref{PropCJGIooZNpnGF} à propos de fonctions continues égales sur une partie denses. La densité de \( \eQ\) dans \( \eR\), si vous la cherchez est la proposition~\ref{PropooUHNZooOUYIkn}.

        \item[Candidat limite]
        Soit \( x\in \eR\). Vu que \( x\in \bar \eQ\), nous pouvons chercher à savoir si \( \lim_{q\to x} f(q) \) existe. Si elle existe, elle sera unique.

        Soit une suite \( (q_i)\) d'éléments de \( \eQ\) qui converge vers \( x\) dans \( \eR\) (i.e. pour la topologie de \( \eR\)). Les nombres réels \( f(q_i)\) forment une suite dans \( \eR\). La suite \( (q_i)\) étant convergente, elle est de Cauchy\footnote{Théorème \ref{THOooNULFooYUqQYo}\ref{ITEMooUUFCooIVtGgz}.}.

        Vu que \( f\) est supposée Cauchy-continue, la suite \(\big( f(q_i) \big)\) est de Cauchy dans \( \eR\), et elle est donc convergente.
    \item[C'est bien la limite]

        Nous prouvons à présent que le nombre réel \( \lim_{i\to \infty} f(q_i)\) (dont l'existence vient d'être prouvée) vérifie bien la définition de la limite \( \lim_{q\to x}f(q)\).

        Soit un voisinage \( V\) de \( \lim f(q_i)\) dans \( \eR\). Nous devons trouver un voisinage \( W\) de \( x\) dans \( \eR\) tel que
        \begin{equation}
            f\big( W\cap\eQ\setminus\{ x \} \big)\subset V.
        \end{equation}
        Pour cela nous considérons \( \epsilon>0\) tel que \( B\big( \lim f(q_i),\epsilon \big)\subset V\). Vu que \( f\) est continue sur \( \eQ\), il existe \( \delta\) tel que
        \begin{equation}
            | p-q |<2\delta\Rightarrow\,| f(p)-f(q) |<\epsilon.
        \end{equation}
        Nous posons \( W=B(x,\delta)\).

        Soit \( q\in W\cap\eQ\setminus\{ x \}\). Nous nous proposons de majorer la quantité $| f(q)-\lim f(q_i) |$ par un multiple de \( \epsilon\).

        Pour cela nous considérons \( k\) suffisamment grand pour que \( | f(q_k)-\lim f(q_i)  |<\epsilon\). Et de plus, vu que \( q_i\to x\) nous considérons \( k\) suffisamment grand pour que \( | q_k-x |<\delta\). L'indice \( k\) est choisi pour vérifier les deux conditions en même temps.

        Nous écrivons alors la majoration suivante :
        \begin{equation}
                | f(q)-\lim f(q_i) |\leq | f(q)-f(q_k) |+| f(q_k)+\lim f(q_i) |.
        \end{equation}
        Le second terme est majoré par \( \epsilon\). Pour le premier terme, \( q\in B(x,\delta)\) et \( q_k\in B(x,\delta)\), donc \( | q-q_k |\leq 2\delta\), ce qui implique \( | f(q)-f(q_k) |<\epsilon\).

        Au final, \( | f(q)-\lim f(q_i) |\leq 2\epsilon\). En reprenant tout le travail avec \( \epsilon/2\) au lieu de \( \epsilon\) nous trouvons \( f(q)\in B\big( \lim f(q_i),\epsilon \big)\subset V\).

    \item[Intermède]

        Jusqu'à présent, nous avons prouvé que
        \begin{equation}        \label{EQooUJJKooTYRNDo}
            \lim_{q\to x} f(q)
        \end{equation}
        existe et vaut
        \begin{equation}        \label{EQooNSYCooTmECjs}
            \lim f(q_i)
        \end{equation}
        lorsque \( (q_{i})\) est une suite quelconque de rationnels qui converge vers \( x\). Nous l'écrivons pour la référentier plus tard :
        \begin{equation}        \label{EQooSGCMooKtpVMy}
            \lim_{q\to x} f(q)=\lim f(q_i).
        \end{equation}
        La limite \eqref{EQooUJJKooTYRNDo} est une limit de fonction définie sur \( \eQ\subset \eR\) en un pour adhérent à l'ensemble de définition de \( f\). La limite \eqref{EQooNSYCooTmECjs} est une limite usuelle d'une suite dans \( \eR\).

    \item[Le prolongement]

        Nous posons
        \begin{equation}
            \tilde f(x)=\begin{cases}
                f(x)    &   \text{si } x\in \eQ\\
                \lim_{q\to x} f(q)    &    \text{sinon }
            \end{cases}
        \end{equation}
        et nous allons prouver que \( \tilde f\) est une fonction continue sur \( \eR\).

    \item[Continuité]

        Soit \( a\in \eR\); nous allons montrer la continuité de \( \tilde f\) en \( a\). Nous fixonx bien entendu \( \epsilon>0\), et nous nous acharnons à majorer la quantité \( | \tilde f(x)-\tilde f(a) |\).

        Vu que \( f\) est continue sur \( \eQ\) nous considérons \( \delta'\) tel que (dans \( \eQ\)) \( 0<| q-q' |<\delta'\) implique \( | f(q)-f(q') |<\epsilon\).

        \begin{subproof}
            \item[\( a\in \eQ\), \( x\in \eQ\)]
                Alors \( \tilde f(x)=f(x)\) et \( \tilde f(a)=f(a)\). Par la continuité de \( f\) sur \( \eQ\), il existe un \( \delta\) tel que \( 0<| x-a |<\delta\) implique \( | f(x)-f(a) |<\epsilon\).

            \item[\( a\in \eQ\), \( x\) irrationnel]

                Nous considérons une suite de rationnels \( q_k\to x\) (vous penserez à l'utilisation du lemme \ref{LemooRTGFooYVstwS}). Nous avons la majoration
                \begin{equation}        \label{EQooPDEMooHlwTcm}
                    | \tilde f(x)-\tilde f(a) |=| \lim_{q\to x} f(q)-f(a) |\leq | \lim_{q\to x} f(q)-f(q_k) |+| f(q_k)-f(a) |.
                \end{equation}
                Nous considérons \( \delta<\delta'\) et \( k\) suffisament grand pour que \( | q_k-x |<\delta'-\delta\). Avec ces choix,
                \begin{equation}
                    | q_k-a |\leq | q_k-x |+| x-a |\leq \delta'.
                \end{equation}
                Enfin nous prenons également \( k\) suffisament grand pour avoir \( | \lim_{q\to x} f(q)-f(q_k) |\leq \epsilon\).

                Les inégalités \eqref{EQooPDEMooHlwTcm} peuvent alors être prolongées pour avoir
                \begin{equation}
                    | \tilde f(x)-\tilde f(a) |\leq 2\epsilon.
                \end{equation}
                
            \item[\( a\) irrationnel, \( x\in \eQ\)]

                Nous faisons encore la majoration
                \begin{equation}
                    | \tilde f(x)-\tilde f(a) |=| f(x)-\lim_{q\to a} f(q) |\leq | f(x)-f(q_k) |+| f(q_k)-\lim_{q\to a} f(a) |.
                \end{equation}
                Nous prenons \( \delta<\delta'/2\) et nous choisissons \( k\) assez grand pour que \( | q_k-a |<\delta'/2\). De ces choix il ressort que
                \begin{equation}
                    | q_k-x |\leq | q_k-a |+| a-x |\leq \frac{ \delta' }{2}+\frac{ \delta' }{2}\leq \delta'.
                \end{equation}
                Donc \( | f(x)-f(q_k) |<\epsilon\). De plus, pour \( k\) assez grand, \( | f(q_k)-\lim_{q\to a} f(q) |\leq \epsilon\).

            \item[\( a\) et \( x\) irrationnels]

                Nous avons
                \begin{equation}
                    | \tilde f(x)-\tilde f(a) |=| \lim_{q\to x} f(q)-\lim_{r\to a} f(r) |,
                \end{equation}
                et nous considérons des suites de rationnels \( q_k\to x\) et \( r_i\to a\). De plus nous considérons \( \delta<\delta'/4\), et \( k,i\) suffisament grands pour avoir \( | q_k-x |\leq \delta'/4\) et \( | r_i-a |<\delta'/4\). Avec tout cela nous avons
                \begin{equation}
                    | q_k-r_i |\leq | q_k-x |+| x-a |+| a-r_i |\leq 3\delta'/4<\delta'.
                \end{equation}
                Enfin, en choisissant \( i\) et \( k\) de telle sorte à avoir \( | \lim_{q\to x} f(q)-f(q_k) |\leq \epsilon\) et \( | f(r_i)-\lim_{r\to a} f(r) |<\epsilon\) nous avons les majorations
                \begin{subequations}
                    \begin{align}
                        | \tilde f(x)-\tilde f(a) |&=| \lim_{q\to x} f(q)-\lim_{r\to a} f(r) |\\
                        &\leq | \lim_{q\to x} f(x)-f(q_k) |+| f(q_k)-f(r_i) |+| f(r_i)-\lim_{r\to q} f(r) |\\
                        &\leq 3\epsilon.
                    \end{align}
                \end{subequations}
        \end{subproof}

    \end{subproof}
\end{proof}

\begin{proposition}     \label{PROPooXWHYooFiVYfi}
    Soient des fonctions continues \( f,g\colon \eR\to \eR\). Si \( f\) et \( g\) sont égales sur \( \eQ\), alors elles sont égales sur \( \eR\).
\end{proposition}

\begin{proof}
    Sinon nous pouvons utiliser les propriétés fondamentales des réels et de la continuité. Soit \( x\in \eR\); nous voulons montrer que \( f(x)=g(x)\). En prenant par exemple le lemme \ref{LemooRTGFooYVstwS}, il existe une suite \( q_i\) de rationnels telle que \( q_i\stackrel{\eR}{\longrightarrow}x\). 
    
    Par ailleurs, \( f\) et \( g\) sont continues sur \( \eR\) et donc en chaque point de \( \eR\) (théorème \ref{ThoESCaraB}). Par la caractérisation séquentielle \ref{PropFnContParSuite} de la continuité, nous avons
    \begin{equation}
        f(x)=\lim_{i\to \infty} f(q_i)=\lim_{i\to \infty} g(q_i)=g(x).
    \end{equation}
\end{proof}

\begin{proposition}[\cite{MonCerveau}]      \label{PROPooTNIAooNAJDzL}
    Soit une fonction strictement croissante \( f\colon \eQ\to \eR\). Alors la prolongation continue \( \tilde f\colon \eR\to \eR\) est également strictement croissante.
\end{proposition}

\begin{proof}
    Soient \( x,y\in \eR\) avec \( x<y\). Notons \( d=y-x\). Nous considérons des suites de rationnels \( x_k\to x\) et \( y_l\to y\) telles que pour tout \( k\), \( x_k\in B(x,d/3)\) et \( y_k\in B(y,d/3)\). En particulier, \( x_k<y_l\) pour tout \( k\) et \( l\).

    Soient des rationnels \( q\) et \( q'\) tels que pour tout \( k\),
    \begin{equation}
        x_k<q<q'<y_k.
    \end{equation}
    Pour trouver de tels rationnels, il suffit de les chercher dans \( \mathopen] x+\frac{ d }{ 3 } , y-\frac{ d }{ 3 } \mathclose[\). Cet intervalle étant de longueur \( d/3\), il contient des rationnels.

    Vue la croissance de \( f\) sur \( \eQ\), nous avons, pour tout \( k\) :
    \begin{equation}
        f(x_k)<f(q)<f(q')<f(y_k),
    \end{equation}
    et à la limite :
    \begin{equation}
        \tilde f(x)\leq f(q)<f(q')\leq \tilde f(y).
    \end{equation}
    Notez que les inégalités strictes se changent en inégalités larges au passage à la limite. D'où l'utilisé de prendre \emph{deux} rationnels entre \( x_k\) et \( y_k\) pour maintenir une inégalité stricte entre \(\tilde f(x)\) et \( \tilde f(y)\).
\end{proof}


% Copyright (c) 2008-2020
%   Laurent Claessens
% See the file fdl-1.3.txt for copying conditions.

%+++++++++++++++++++++++++++++++++++++++++++++++++++++++++++++++++++++++++++++++++++++++++++++++++++++++++++++++++++++++++++
\section{Espace des fonctions continues}
%+++++++++++++++++++++++++++++++++++++++++++++++++++++++++++++++++++++++++++++++++++++++++++++++++++++++++++++++++++++++++++

\begin{definition}
    Soit \( I\), un intervalle de \( \eR\). L'\defe{oscillation}{oscillation!d'une fonction} sur \( I\) est le nombre
    \begin{equation}
        \omega_f(I)=\sup_{x\in I}f(x)-\inf_{x\in I}f(x).
    \end{equation}
\end{definition}
    Pour chaque \( x\) fixé, la fonction
    \begin{equation}
        x\mapsto \omega_f\big( B(x,\delta) \big)
    \end{equation}
    est une fonction positive, croissante et a donc une limite (pour \( \delta\to 0\)). Nous notons \( \omega_f(x)\) cette limite qui est l'\defe{oscillation}{oscillation!d'une fonction en un point} de \( f\) en ce point. Une propriété immédiate est que \( f\) est continue en \( x_0\) si et seulement si \( \omega_f(x_0)=0\).

    \begin{lemma}       \label{LemuaPbtQ}
    L'ensemble des points de discontinuité d'une fonction \( f\colon \eR\to \eR\) est une réunion dénombrable de fermés.
\end{lemma}

\begin{proof}
    Soit \( D\) l'ensemble des points de discontinuité de \( f\). Nous avons
    \begin{equation}
        D=\bigcup_{n=1}^{\infty}\{ x\tq \omega_f(x)\geq \frac{1}{ n } \}.
    \end{equation}
    Il nous suffit donc de montrer que pour tout \( \epsilon\), l'ensemble
    \begin{equation}
        \{ x\tq \omega_f(x)<\epsilon \}
    \end{equation}
    est ouvert. Soit en effet \( x_0\) dans cet ensemble. Il existe \( \delta\) tel que \( \omega_f\big( B(x_0,\delta) \big)<\epsilon\). Si \( x\in B(x_0,\delta)\), alors si on choisit \( \delta'\) tel que \( B(x,\delta')\subset B(x_0,\delta)\), nous avons \( \omega_f\big( B(x,\delta') \big)<\epsilon\), ce qui justifie que \( \omega_f(x)<\epsilon\) et donc que \( x\) est également dans l'ensemble considéré.
\end{proof}

\begin{theorem}
    L'ensemble des points de discontinuité d'une limite simple de fonctions continues est de première catégorie.
\end{theorem}

\begin{proof}
    Soit \( (f_n)\) une suite de fonctions qui converge simplement vers \( f\). Nous devons écrire l'ensemble des points de discontinuité de \( f\) comme une union dénombrable d'ensembles tels que sur tout intervalle \( I\), aucun de ces ensembles n'est dense. Nous savons déjà par le lemme~\ref{LemuaPbtQ} que l'ensemble des points de discontinuité  de \( f\) est donné par
    \begin{equation}
        D=\bigcup_{n=1}^{\infty}\{ x\tq \omega_f(x)\geq \frac{1}{  n } \}.
    \end{equation}
    Nous essayons donc de prouver que pour tout \( \epsilon\), l'ensemble
    \begin{equation}
        F=\{ x\tq \omega_f(x)\geq \epsilon \}
    \end{equation}
    est nulle part dense. Soit
    \begin{equation}
        E_n=\bigcap_{i,j>n}\{ x\tq | f_i(x)-f_j(x) |<\epsilon \}.
    \end{equation}
    Nous montrons que cet ensemble est fermé en étudiant le complémentaire. Soit \( x\notin E_n\); alors il existe un couple \( (i,j)\) tel que
    \begin{equation}
        | f_i(x)-f_j(x) |>\epsilon.
    \end{equation}
    Par continuité, cette inégalité reste valide dans un voisinage de \( x\). Donc il existe un voisinage de \( x\) contenu dans \( \complement E_n\) et \( E_n\) est donc fermé.

    De plus nous avons \( E_n\subset E_{n+1}\) et \( \bigcup_nE_n=\eR\). Ce dernier point est dû au fait que pour tout \( x\), il existe \( N\) tel que \( i,j>N\) implique \( | f_i(x)-f_j(x) |\leq \epsilon\). Cela est l'expression du fait que la suite \( \big( f_n(x) \big)_{n\in \eN}\) est de Cauchy.

    Soit \( I\), un intervalle fermé de \( \eR\). Nous voulons trouver un intervalle \( J\subset I\) sur lequel \( f\) est continue. Nous écrivons \( I\) sous la forme
    \begin{equation}
        I=\bigcup_{n=1}^{\infty}(E_n\cap I).
    \end{equation}
    Tous les ensembles \( J_n=E_n\cap I\) ne peuvent être nulle part dense en même temps (à cause du théorème de Baire~\ref{ThoQGalIO}). Il existe donc un \( n\) tel que \( J_n\) contienne un ouvert \( J\). Le but est de montrer que \( f\) est continue sur \( J\). Pour ce faire, nous n'allons pas simplement majorer \( | f(x)-f(x_0) |\) par \( \epsilon\) lorsque \( | x-x_0 |\) est petit. Nous allons majorer l'oscillation de \( f\) sur \( B(x_0,\delta)\) lorsque \( \delta\) est petit. Pour cela nous prenons \( x_0\) et \( x\) dans \( J\) et nous écrivons
    \begin{equation}
        | f(x)-f(x_0) |\leq | f(x)-f_n(x) |+| f_n(x)-f_n(x_0) |.
    \end{equation}
    À ce niveau nous rappelons que \( n\) est fixé par le choix de \( J\), dans lequel \( \epsilon\) est déjà inclus. Nous choisissons évidemment \( | x-x_0 |\leq \delta\) de telle sorte que le second terme soit plus petit que \( \epsilon\) en vertu de la continuité de \( f_n\). Pour le premier terme, pour tout \( i,j\geq n\) nous avons
    \begin{equation}
        | f_i(x)-f_j(x) |<\epsilon.
    \end{equation}
    Si nous posons \( j=n\) et \( i\to\infty\), en tenant compte du fait que \( f_i\to f\) simplement,
    \begin{equation}
        | f(x)-f_n(x) |\leq \epsilon.
    \end{equation}
    Nous avons donc obtenu \( | f(x)-f_n(x_0) |\leq 2\epsilon\). Cela signifie que dans un voisinage de rayon \( \delta\) autour de \( x_0\), les valeurs extrêmes prises par \( f(x) \) sont \( f_n(x_0)\pm 4\epsilon\). Nous avons donc prouvé que pour tout \( \epsilon\), il existe \( \delta\) tel que
    \begin{equation}
        \omega_f\big( \mathopen[ x_0-\delta , x_0+\delta \mathclose] \big)\leq 4\epsilon.
    \end{equation}
    De là nous concluons que
    \begin{equation}
        \lim_{\delta\to 0}\omega_f\big( \mathopen[ x_0-\delta , x_0+\delta \mathclose] \big)=0,
    \end{equation}
    ce qui signifie que \( f\) est continue en \( x_0\).
\end{proof}

\begin{example}
    Une fonction discontinue sur \( \eQ\) et continue ailleurs. La fonction
    \begin{equation}
        f(x)=\begin{cases}
            0    &   \text{si } x\notin \eQ\\
            \frac{1}{ q }    &    \text{si } x=p/q
        \end{cases}
    \end{equation}
    où par «\( x=p/q\)» nous entendons que \( p/q\) est la fraction irréductible.

    Cette fonction est discontinue sur \( \eQ\) parce que si \( q\in \eQ\) alors \( f(q)\neq 0\) alors que dans tous voisinage de \( q\) il existe un irrationnel sur qui la fonction vaudra zéro.

    Montrons que \( f\) est continue sur les irrationnels. Si \( x_0\notin \eQ\) alors \( f(x_0)=0\). Mais si on prend un voisinage suffisamment petit de \( x_0\), nous pouvons nous arranger pour que tous les rationnels aient un dénominateur arbitrairement grand. En effet si nous nous fixons un premier rayon \( r_0>0\) alors il existe un nombre fini de fractions de la forme \( 1\), \( \frac{ k }{2}\), \( \frac{ k }{ 3 }\),\ldots, \( \frac{ k }{ N }\) dans \( B(x_0,r_0)\). Il suffit maintenant de choisir \( 0<r\leq r_0\) tel que ces fractions soient toutes hors de \( B(x_0,r)\). Dans cette boule nous avons \( f<\frac{1}{ N }\). Du coup \( f\) est continue en \( x_0\).
\end{example}

\begin{definition}[Point périodique\cite{TMCHooOaTrJL}]
    Soit \( f\colon I\to I\) une application d'un ensemble \( I\) dans lui-même. Si \( x\in I\) vérifie \( f^n(x)=x\) et \( f^k(x)\neq x\) pour \( k=1,\ldots, n-1\) alors on dit que \( x\) est un point \( n\)-périodique.
\end{definition}

\begin{lemma}       \label{LemAONBooGZBuYt}
    Soit \( I\) un segment\footnote{définition~\ref{DefLISOooDHLQrl}. Un segment est un intervalle fermé borné.} de \( \eR\) et une fonction continue \( f\colon I\to I\). Si \( K\) est un segment fermé avec \( K\subset f(I)\) alors il existe un segment fermé \( L\subset I\) tel que \( K=f(L)\).
\end{lemma}

\begin{proof}
    Mentionnons immédiatement que \( f\) est continue sur \( I\) qui est compact\footnote{Par le lemme~\ref{LemOACGWxV}.}. Par conséquent tous les nombres dont nous allons parler sont finis parce que \( f\) est bornée par le théorème~\ref{ThoMKKooAbHaro}.

    Soit \( K=\mathopen[ \alpha , \beta \mathclose]\). Si \( \alpha=\beta\) alors le segment \( L=\{ a \}\) convient. Nous supposons donc que \( \alpha\neq \beta\) et nous considérons \( a,b\in I\) tels que \( \alpha=f(a)\) et \( \beta=f(b)\). Vu que \( a\neq b\) nous supposons \( a<b\) (le cas \( a>b\) se traite de façon similaire).

    Nous posons
    \begin{equation}
        A=\{ x\in\mathopen[ a , b \mathclose]\tq f(x)=\alpha \}.
    \end{equation}
    C'est un ensemble borné par \( a\) et \( b\). De plus il est fermé; ce dernier point n'est pas tout à fait évident parce que \( f\) n'est pas définit sur \( \eR\) mais sur \( I\) qui est fermé, le corolaire~\ref{CorNNPYooMbaYZg} n'est donc pas immédiatement utilisable. Prouvons donc que \( Z=\{ x\in \eR\tq f(x)=\alpha \}\) est fermé. Si \( x_0\) est hors de \( Z\) alors soit \( x_0\) est dans \( I\) soit il est hors de \( I\). Dans ce second cas, le complémentaire de \( I\) étant ouvert, on a un voisinage de \( x_0\) hors de \( I\) et par conséquent hors de \( Z\). Si au contraire \( x_0\in I\) alors il y a (encore) deux cas : soit \( x_0\in\Int(I)\) soit \( x_0\) est sur le bord de \( I\). Dans le premier cas, le théorème des valeurs intermédiaires\footnote{Théorème~\ref{ThoValInter}.} fonctionne. Pour le second cas, nous supposons \( x_0=\max(I)\) (le cas \( x_0=\min(I)\) est similaire). Le théorème des valeurs intermédiaires dit que sur \( \mathopen[ x_0-\epsilon , x_0 \mathclose]\), \( f\neq \alpha\) et en même temps, sur \( \mathopen] x_0 , x_0+\epsilon \mathclose]\), nous sommes en dehors du domaine. Au final \( \{ f(x)=\alpha \}\) est fermé et \( A\) est alors fermé en tant que intersection de deux fermés.

    L'ensemble \( A\) étant non vide (\( a\in A\)), il possède donc un maximum que nous nommons \( u\) :
    \begin{equation}
        u=\max(A).
    \end{equation}
    Nous posons aussi
    \begin{equation}
        B=\{ x\in \mathopen[ u , b \mathclose]\tq f(x)=\beta \}
    \end{equation}
    qui est encore fermé, borné et non vide. Nous pouvons donc définir
    \begin{equation}
        v=\min(B).
    \end{equation}
    Nous prouvons maintenant que \( f\big( \mathopen[ u , v \mathclose] \big)=\mathopen[ \alpha , \beta \mathclose]\). D'abord \( f\big( \mathopen[ u , v \mathclose] \big)\) est un intervalle compact\footnote{Corolaire \ref{CorImInterInter} et théorème~\ref{ThoImCompCotComp}.} contenant \( f(u)=\alpha\) et \( f(v)=\beta\). Par conséquent \( \mathopen[ \alpha , \beta \mathclose]\subset f\big( \mathopen[ u , v \mathclose] \big)\). Pour l'inclusion inverse supposons \( t\in \mathopen[ u , v \mathclose]\) tel que \( f(t)>\beta\). Vu que \( f(a)=\alpha\) et \( \alpha<\beta\) le théorème des valeurs intermédiaires il existe \( t_0\in \mathopen[ a , t \mathclose]\) tel que \( f(t_0)=\beta\). Cela donne \( t_0<v\) et donc contredit la minimalité de \( v\) dans \( B\). Nous en déduisons que \( f\big( \mathopen[ u , v \mathclose] \big)\) ne contient aucun élément plus grand que \( \beta\). Même jeu pour montrer que ça ne contient aucun élément plus petit que \( \alpha\).

    En définitive, le segment \( L=\mathopen[ u , v \mathclose]\) fonctionne.
\end{proof}

Lorsque \( I_2\subset f(I_1)\) nous notons \( I_1\to I_2\) ou, si une ambiguïté est à craindre, \( I_1\stackrel{f}{\longrightarrow}I_2\). Cette flèche se lit «recouvre».
\begin{lemma}[\cite{PAXrsMn,TMCHooOaTrJL}]      \label{LemSSPXooMkwzjb}
    Soient les segments \( I_0,\ldots, I_{n-1}\) tels que nous ayons le cycle
    \begin{equation}
        I_0\to I_1\to\ldots\to I_{n-1}\to I_0.
    \end{equation}
    Alors \( f^n\) admet un point fixe \( x_0\in I_0\) tel que \( f^k(x_0)\in I_k\) pour tout \( k=0,\ldots, n-1\).
\end{lemma}

\begin{proof}
    Nous prouvons les cas \( n=1\) et \( n=2\) séparément.
    \begin{subproof}
    \item[\( n=1\)]
        Nous avons \( I_0\to I_0\), c'est-à-dire que $I_0\subset f(I_0)$. Si \( I_0=\mathopen[ a , b \mathclose]\) alors nous posons \( a=f(\alpha)\) et \( b=f(\beta)\) pour certains \( \alpha,\beta\in I_0\). Nous posons ensuite \( g(x)=f(x)-x\).

        Dans un premier temps, \( g(\alpha)=a-\alpha\leq 0\) parce que \( a=\in(I_0)\) et \( \alpha\in I_0\). Pour la même raison, \( g(\beta)=b-\beta\geq 0\). Le théorème des valeurs intermédiaires donne alors \( t_0\in \mathopen[ \alpha , \beta \mathclose]\subset I_0\) tel que \( g(t_0)=0\). Nous avons donc \( f(t_0)=t_0\).
    \item[\( n=2\)]
        Nous avons \( I_0\to I_1\to I_0\). Vu que \( I_1\subset f(I_0)\), le lemme~\ref{LemAONBooGZBuYt} donne un segment \( J_1\subset I_0\) tel que \( f(J_1)=I_1\). Mézalors
        \begin{equation}
            J_1\subset I_0\subset f(I_1)=f^2(J_1).
        \end{equation}
        Nous avons donc \( J_1\stackrel{f^2}{\longrightarrow}J_1\) et par le cas \( n=1\) traité plus haut, la fonction \( f^2 \) a un point fixe \( x_0\) dans \( J_1\). De plus
        \begin{equation}
            f(x_0)\in f(J_1)=I_1,
        \end{equation}
        le point \( x_0\) est donc bien celui que nous cherchions.
    \item
        Cas général. Nous avons
        \begin{equation}
            I_0\to I_1\to\ldots\to I_{n-1}\to I_0.
        \end{equation}
        Vu que \( I_1\subset f(I_0)\), il existe \( J_1\subset I_0\) tel que \( f(J_1)=I_1\). Mais
        \begin{equation}
            I_2\subset f(I_1)=f^2(J_1),
        \end{equation}
        donc il existe \( J_2\subset J_1\) tel que \( I_2=f^2(J_2)\). En procédant encore longtemps ainsi nous construisons les ensembles \( J_1,\ldots, J_{n-1}\) tels que
        \begin{equation}
            J_{n-1}\subset J_{n-2}\subset\ldots\subset J_1\subset J_0
        \end{equation}
        tels que \( I_k=f^k(J_k)\) pour tout \( k=1,\ldots, n-1\). La dernière de ces inclusions est \( I_{n-1}=f^{n-1}(J_{n-1})\), mais \( I_{n-1}\to I_0\), c'est-à-dire que
        \begin{equation}
            I_0\subset f(I_{n-1})=f^n(J_{n-1}),
        \end{equation}
        et il existe \( J_n\subset J_{n-1}\) tel que \( I_0\subset f^n(J_n)\). Mais comme \( J_n\subset J_0\) nous avons en particulier \( J_n\subset f^n(J_n)\).

        Cela donne un point fixe \( x_0\in J_n\) pour \( f^n\). Par construction nous avons \( J_n\subset J_{n-1}\subset\ldots\subset J_1\subset J_0\) et donc \( x_0\in J_k\) pour tout \( k\). En  particulier
        \begin{equation}
            f^k(x_0)\in f^k(J_k)=I_k
        \end{equation}
        pour tout \( k\).
    \end{subproof}
\end{proof}

\begin{theorem}[Théorème de Sarkowski\cite{PAXrsMn,TMCHooOaTrJL}]
    Soit \( I\), un segment de \( \eR\) et une application continue \( f\colon I\to I\). Si \( f\) admet un point \( 3\)-périodique, alors \( f\) admet des points \( n\)-périodiques pour tout \( n\geq 1\).
\end{theorem}

\begin{proof}
    Soit \( a\in I\) un point \( 3\)-périodique pour \( f\) et notons \( b=f(a)\), \( c=f(b)\). Les points \( b\) et \( c\) sont également des points \( 3\)-périodiques. Quitte à renommer, nous pouvons supposer que \( a\) est le plus petit des trois. Il reste deux possibilités : \( a<b<c\) et \( a<c<b\). Nous traitons d'abord le premier cas.

    Supposons \( a<b<c\). Nous posons \( I_0=\mathopen[ a , b \mathclose]\) et \( I_1=\mathopen[ b , c \mathclose]\). Nous avons immédiatement \( I_1\subset f(I_0)\) et comme \( f(b)=c\) et \( f(c)=a\), \( f(I_1)\) recouvre \( \mathopen[ a , c \mathclose]\) et donc recouvre en même temps \( I_1\) et \( I_2\). Nous avons donc \( I_0\to I_1\), \( I_1\to I_0\) et \( I_1\to I_1\).
    \begin{subproof}
    \item[Un point \( 1\)-périodique]
        Nous avons \( I_1\to I_1\) qui prouve que \( f\) a un point fixe dans \( I_1\). C'est le cas \( n=1\) du lemme~\ref{LemSSPXooMkwzjb}. Voilà un point \( 1\)-périodique.
    \item[Un point \( 2\)-périodique]
        Nous avons \( I_0\to I_1\to I_0\). Par conséquent, le lemme~\ref{LemSSPXooMkwzjb} dit que \( f^2\) a un point fixe \( x_0\in I_0\) tel que \( f(x_0)\in I_1\). Montrons que \( f(x_0)\neq x_0\). Pour avoir \( x_0=f(x_0)\), il faudrait \( x_0\in I_0\cap I_1=\{ b \}\). Mais \( b\) est un point \( 3\)-périodique, donc ne vérifiant certainement pas \( f^2(b)=b\). Nous en déduisons que \( f(x_0)\neq x_0\) et donc que \( x_0\) est \( 2\)-périodique.
    \item[Un point \( 3\)-périodique]
        On en a par hypothèse.
    \item[Un point \( n\)-périodique pour \( n\geq 4\)]
        Nous avons le cyle
        \begin{equation}
            I_0\to \underbrace{I_1\to I_1\to\ldots\to I_1}_{\text{n-1} fois}\to I_0.
        \end{equation}
        Le lemme donne alors un point fixe \( x\in I_0\) pour \( f^n\) tel que \( f^k(x)\in I_1\) pour \( k=1,\ldots, n-1\). Est-ce possible que \( x=b\) ? Non parce que \( f^2(b)=a\in I_0\) alors que \( f^2(x)\in I_1\). Mais \( I_0\cap I_1=\{ b \}\).

        Par conséquent la relation \( f^k(x)\in I_1\) exclu d'avoir \( f^k(x)=x\), et le point \( x\) est bien \( n\)-périodique.
    \end{subproof}

    Passons au cas \( a<c<b\). Alors nous posons \( I_0=\mathopen[ a , c \mathclose]\) et \( I_1=\mathopen[ c , b \mathclose]\). Encore une fois \( f(I_0)\) contient \( a\) et \( b\), donc \( I_0\to I_0\) et \( I_0\to I_1\). Mais en même temps \( f(I_1)\) contient \( a\) et \( c\), donc \( I_1\to I_0\).

    Nous pouvons donc refaire comme dans le premier cas, en inversant les rôles de \( I_0\) et \( I_1\). En particulier nous pouvons considérer le cycle
    \begin{equation}
        I_1\to I_0\to I_0\to\ldots\to I_0\to I_1.
    \end{equation}
\end{proof}

%+++++++++++++++++++++++++++++++++++++++++++++++++++++++++++++++++++++++++++++++++++++++++++++++++++++++++++++++++++++++++++
\section{Uniforme continuité}		\label{SecUnifContinue}
%+++++++++++++++++++++++++++++++++++++++++++++++++++++++++++++++++++++++++++++++++++++++++++++++++++++++++++++++++++++++++++

\begin{definition}
	Une partie $A\subset\eR^m$ est dite \defe{bornée}{bornée!partie de $\eR^m$} s'il existe un $M>0$ tel que $A\subset B(0,M)$. Le \defe{diamètre}{diamètre} de la partie $A$ est\nomenclature[T]{$\Diam(A)$}{Diamètre de la partie $A$} le nombre
	\begin{equation}
		\Diam(A)=\sup_{x,y\in A}\| x-y \|\in\mathopen[ 0 , \infty \mathclose].
	\end{equation}
\end{definition}
Lorsque $A$ est borné, il existe un $M$ tel que $\| x \|\leq M$ pour tout $x\in A$.

\begin{lemma}
	Si $A$ est une partie non vide de $\eR^m$, alors $\Diam(A)=\Diam(\bar A)$.
\end{lemma}
Nous n'allons pas donner de démonstrations de ce lemme.


Si $(x_n)$ est une suite et $I$ est un sous-ensemble infini de $\eN$, nous désignons par $x_I$ la suite des éléments $x_n$ tels que $n\in I$. Par exemple la suite $x_{\eN}$ est la suite elle-même, la suite $x_{2\eN}$ est la suite obtenue en ne prenant que les éléments d'indice pair.

Les suites $x_I$ ainsi construites sont dites des \defe{sous-suites}{sous-suite} de la suite $(x_n)$.


Pour une fonction $f\colon D\subset\eR^m\to \eR$, la continuité au point $a$ signifie que pour tout $\varepsilon>0$,
\begin{equation}
	\exists\delta>0\tq 0<\| x-a \|<\delta\Rightarrow | f(x)-f(a) |<\varepsilon.
\end{equation}
Le $\delta$ qu'il faut choisir dépend évidemment de $\varepsilon$, mais il dépend en général aussi du point $a$ où l'on veut tester la continuité. C'est-à-dire que, étant donné un $\varepsilon>0$, nous pouvons trouver un $\delta$ qui fonctionne pour certains points, mais qui ne fonctionne pas pour d'autres points.

Il peut cependant également arriver qu'un même $\delta$ fonctionne pour tous les points du domaine. Dans ce cas, nous disons que la fonction est uniformément continue sur le domaine.

\begin{definition}
	Une fonction $f\colon D\subset\eR^m\to \eR$ est dite \defe{uniformément continue}{continue!uniformément} sur $D$ si
	\begin{equation}	\label{EqConditionUnifCont}
		\forall\varepsilon>0,\,\exists\delta>0\tq\,\forall x,y\in D,\,\| x-y \|\leq\delta \Rightarrow| f(x)-f(a) |<\varepsilon.
	\end{equation}
\end{definition}

Il est intéressant de voir ce que signifie le fait de \emph{ne pas} être uniformément continue sur un domaine $D$. Il s'agit essentiellement de retourner tous les quantificateurs de la condition \eqref{EqConditionUnifCont} :
\begin{equation}	\label{EqConditionPasUnifCont}
	\exists\varepsilon>0\tq\forall\delta>0,\,\exists x,y\in D\tq \| x-y \|<\delta\text{ et }\big| f(x)-f(y) \big|>\varepsilon.
\end{equation}
Dans cette condition, les points $x$ et $y$ peuvent être fonction du $\delta$. L'important est que pour tout $\delta$, on puisse trouver deux points $\delta$-proches dont les images par $f$ ne soient pas $\varepsilon$-proches.

\begin{example}
	Prenons la fonction $f(x)=\frac{1}{ x }$, et demandons nous pour quel $\delta$ nous sommes sûr d'avoir
	\begin{equation}
		| f(a+\delta)-f(a) |=\left| \frac{1}{ a+\delta }-\frac{1}{ a } \right| <\varepsilon.
	\end{equation}
	Pour simplifier, nous supposons que $a>0$. Nous calculons
	\begin{equation}
		\begin{aligned}[]
			\frac{ 1 }{ a }-\frac{1}{ a+\delta }&<	\varepsilon\\
			\frac{ \delta }{ a(a+\delta) }&<\varepsilon\\
			\delta&<\varepsilon a^2+\varepsilon a\delta\\
			\delta(1-\varepsilon a)&<\varepsilon a^2\\
			\delta&<\frac{ \varepsilon a^2 }{ 1-\varepsilon a }.
		\end{aligned}
	\end{equation}
	Notons que, à $\varepsilon$ fixé, plus $a$ est petit, plus il faut choisir $\delta$ petit. La fonction $x\mapsto\frac{1}{ x }$ n'est donc pas uniformément continue. Cela correspond au fait que, proche de zéro, la fonction monte très vite. Une fonction uniformément continue sera une fonction qui ne montera jamais très vite.
\end{example}

\begin{proposition}
	Quelques propriétés des fonctions uniformément continues.
	\begin{enumerate}
		\item
			Toute application uniformément continue est continue;
		\item
			la composée de deux fonctions uniformément continues est uniformément continue;
	\end{enumerate}
\end{proposition}
Nous verrons qu'une application lipschitzienne est uniformément continue (proposition~\ref{PROPooVZSAooUneOQK}).

Une fonction peut être uniformément continue sur un domaine et pas sur un autre. Le théorème suivant donne une importante indication à ce sujet.
\begin{theorem}[Heine]\index{théorème!Heine}\index{Heine (théorème)}		\label{ThoHeineContinueCompact}
    Soit \( K\) un compact de \( \eR^n\). Une fonction continue \( f\colon \eR^n\to \eR^m\) est uniformément continue sur \( K\).
\end{theorem}

La démonstration qui suit est valable pour une fonction \( f\colon \eR^n\to \eR^m\) et utilise le fait que le produit cartésien de compacts est compact. Dans le cas de fonctions sur \( \eR\), nous pouvons modifier la démonstration pour ne pas utiliser ce résultat; voir plus bas.
%TODO : trouver où se trouve la preuve du produit de compacts et la référentier ici.
\begin{proof}
	Nous allons prouver ce théorème par l'absurde. Nous commençons par écrire la condition \eqref{EqConditionPasUnifCont} qui exprime que $f$ n'est pas uniformément continue sur le compact \( K\) :
	\begin{equation}
		\exists\varepsilon>0\tq\forall\delta>0,\,\exists x,y\in K\tqs \| x-y \|<\delta\text{ et }\big| f(x)-f(y) \big|>\varepsilon.
	\end{equation}
	En particulier (en prenant $\delta=\frac{1}{ n }$ pour tout $n$), pour chaque $n$ nous pouvons trouver $x_n$ et $y_n$ dans $K$ qui vérifient simultanément les deux conditions suivantes :
	\begin{subequations}
		\begin{numcases}{}
			\| x_n-y_n \|<\frac{1}{ n }\\
			\big| f(x_n)-f(y_n) \big|>\varepsilon.	\label{EqCond3107fxfyepsppt}
		\end{numcases}
	\end{subequations}
    Nous insistons que c'est le même $\varepsilon$ pour chaque $n$. L'ensemble $K$ étant compact, l'ensemble \( K\times K \) est compact (théorème~\ref{THOIYmxXuu}) et nous pouvons trouver une sous-suite convergente \emph{du couple} \( (x_n,y_n)\) dans \( K\times K\). Quitte à passer à ces sous-suites, nous  nous supposons que \( (x_n,y_n)\) converge dans \( K\times K\) et en particulier que les suites $(x_n)$ et $(y_n)$ sont convergentes. Étant donné que pour chaque $n$ elles vérifient $\| x_n-y_n \|<\frac{1}{ n }$, les limites sont égales :
	\begin{equation}
		\lim x_n=\lim y_n=x.
	\end{equation}
	L'ensemble $K$ étant fermé, la limite $x$ est dans $K$. Par continuité de $f$, nous avons finalement
	\begin{equation}
		\lim f(x_n)=\lim f(y_n)=f(x),
	\end{equation}
	mais alors
	\begin{equation}
		\lim_{n\to\infty}\big| f(x_n)-f(y_n) \big|=0,
	\end{equation}
	ce qui est en contradiction avec le choix \eqref{EqCond3107fxfyepsppt}.

	Tout ceci prouve que $f(K)$ est bornée supérieurement et que $f$ atteint son supremum (qui est donc un maximum). Le fait que $f(K)$ soit borné inférieurement se prouve en considérant la fonction $-f$ au lieu de $f$.

\end{proof}

\begin{remark}
    Nous pouvons ne pas utiliser le fait que le produit de compacts est compact. Cela est particulièrement commode lorsqu'on considère des fonctions de \( \eR\) dans \( \eR\) parce que dans ce cadre nous ne pouvons pas supposer connue la notion de produit d'espace topologiques.

    Pour choisir les sous-suites \( (x_n)\) et \( (y_n)\), il suffit de prendre une sous-suite convergente de \( (x_n)\) et d'invoquer le fait que \( \| x_n-y_n \|\leq \frac{1}{ n }\). Les suites \( (x_n)\) et \( (y_n)\) étant adjacentes\footnote{Définition \ref{DEFooDMZLooDtNPmu}.}, la convergence de \( (x_n)\) implique la convergence de \( (y_n)\) vers la même limite.

    Il est donc un peu superflus de parler de la convergence du couple \( (x_n,y_n)\).
\end{remark}

\begin{proposition}[Heine\cite{ooNDDIooKLdIWH}]     \label{PROPooBWUFooYhMlDp}
    Toute application continue d'un espace métrique compact dans un espace métrique quelconque est uniformément continue.
\end{proposition}

%+++++++++++++++++++++++++++++++++++++++++++++++++++++++++++++++++++++++++++++++++++++++++++++++++++++++++++++++++++++++++++
\section{Fonctions sur un compact}
%+++++++++++++++++++++++++++++++++++++++++++++++++++++++++++++++++++++++++++++++++++++++++++++++++++++++++++++++++++++++++++

Par le théorème des valeurs intermédiaires \ref{ThoValInter}, l'image d'un intervalle par une fonction continue est un intervalle, et nous avons l'importante propriété suivante des fonctions continues sur un compact.

Le théorème suivant est un cas particulier du théorème~\ref{ThoMKKooAbHaro}.
\begin{theorem}
    Si $f$ est une fonction continue sur l'intervalle compact $[a,b]$. Alors $f$ est bornée sur $[a,b]$ et elle atteint ses bornes.
\end{theorem}

\begin{proof}
    Étant donné que $[a,b]$ est un intervalle compact, son image est également un intervalle compact, et donc est de la forme $[m,M]$. Ceci découle du théorème~\ref{ThoImCompCotComp} et le corolaire~\ref{CorImInterInter}. Le maximum de $f$ sur $[a,b]$ est la borne $M$ qui est bien dans l'image (parce que $[m,M]$ est fermé). Idem pour le minimum $m$.
\end{proof}

%+++++++++++++++++++++++++++++++++++++++++++++++++++++++++++++++++++++++++++++++++++++++++++++++++++++++++++++++++++++++++++ 
\section{Polynômes}
%+++++++++++++++++++++++++++++++++++++++++++++++++++++++++++++++++++++++++++++++++++++++++++++++++++++++++++++++++++++++++++

L'algèbre des polynômes sur un anneau est définie en \ref{DEFooFYZRooMikwEL}. Si \( P\in A[X]\) et si \( \alpha\in A\) nous avons également défini l'évaluation de \( P\) en \( \alpha\); c'est la définition \ref{DEFooNXKUooLrGeuh}. Dans le cadre de l'analyse, lorsque nous considérons des polynômes, nous allons complètement confondre le polynôme avec la fonction qu'il définit.

%--------------------------------------------------------------------------------------------------------------------------- 
\subsection{Polynômes sur les réels}
%---------------------------------------------------------------------------------------------------------------------------

\begin{proposition}     \label{PROPooJKYJooFqbQMr}
    Tout polynôme à coefficients réels de degré impair possède une racine réelle.
\end{proposition}

\begin{proof}
    Nous mettons le plus haut degré en facteur :
    \begin{equation}
        P(x)=\sum_{k=0}^na_kx^k=x^n\sum_{k=0}^n\frac{ a_k }{ x^{n-k} }.
    \end{equation}
    Le terme \( k=0\) vaut \( a_nx^n\) tandis que les autres sont de la forme (à coefficient près) \( \frac{1}{ x^l }\) pour un \( l\geq 1\). Lorsque \( x\to \infty\), chacun de ces termes s'annule (lemme \ref{LEMooFCIXooJuHFqk}). Nous avons donc
    \begin{equation}
        \lim_{x\to \infty} P(x)=+\infty,
    \end{equation}
    et de même, \( n\) étant impair, \( \lim_{x\to -\infty} P(x)=-\infty\). Le théorème des valeurs intermédiaires \ref{ThoValInter} nous donne alors l'existence d'un réel sur lequel \( P\) s'annule.
\end{proof}

%--------------------------------------------------------------------------------------------------------------------------- 
\subsection{Polynômes sur les complexes}
%---------------------------------------------------------------------------------------------------------------------------

Nous allons parler de comportement asymptotique de polynômes définis sur \( \eC\). La topologique que nous considérons est celle de la compactification en un point décrite en \ref{PROPooHNOZooPSzKIN}.

Le lemme suivant donne une caractérisation de la limite en l'infini dans le compactifié \( \hat \eC\). Dans beaucoup de cas, cette caractérisation est prise comme la définition de la limite. Hélas, dans le Frido nous sommes des extrémistes et nous ne parvenons pas à dire le mot «limite» si il n'y a pas une topologie.
\begin{lemma}[\cite{MonCerveau}]        \label{LEMooERABooQjLBzW}
    Nous considérons la compatification en un point d'Alexandrov\footnote{Définition \ref{PROPooHNOZooPSzKIN}.}. Soit une fonction \( f\colon \eC\to \eC\). Nous avons \( \lim_{z\to \infty} f(z)=\infty\) si et seulement si pour tout \( M>0\), il existe \( R>0\) tel que \( | z |>R\) implique \( | f(z) |>M\).
\end{lemma}

\begin{proof}
    Souvenons-nous que, en général\footnote{Définition \ref{DefYNVoWBx}.}, nous avons
    \begin{equation}
        \lim_{x\to a} f(x)=b
    \end{equation}
    si pour tout voisinage \( V\) de \( b\), il existe un voisinage \( W\) de \( a\) tel que \( z\in W\setminus\{ a \}\) implique \( f(z)\in V\).

    Précisons encore un point de notation. Si \( K\) est une partie de \( \eC\), nous notons \( K^c\) son complémentaire dans \( \eC\), pas dans \( \hat  \eC\).

    Ceci étant dit, nous passons à la preuve.
    \begin{subproof}
        \item[Sens direct]
            Nous supposons que \( \lim_{z\to \infty} f(z)=\infty\). Soit \( M>0\); nous considérons le voisinage \( V=\overline{ B(0,M) }^c\cup\{ \infty \}\). Par définition de la limite, il existe un voisinage \( W\) de \( \infty\) tel que \( z\in W\Rightarrow f(z)\in V\setminus\{ \infty \}=\overline{ B(0,M) }^c\). Ce voisinage est de la forme \( K^c\cup\{ \infty \}\). Vu que \( K\) est compact, il est borné et il existe \( R>0\) tel que \( K\subset B(0,R)\).

            Avec tout cela nous avons la chaine suivante d'implications :
            \begin{equation}
                | z |>R\Rightarrow z\in K^c\Rightarrow z\in W\Rightarrow f(z)\in V\setminus\{ \infty \}=\overline{ B(0,M) }^c\Rightarrow | f(z) |>M.
            \end{equation}
            C'est bien la propriété que nous voulions.
        \item[Sens réciproque]
            Soit un voisinage \( V\) de \( \infty\). Nous avons \( V=K^c\cup\{ \infty \}\) où \( K\) est compact dans \( \eC\). Il existe \( M>0\) tel que \( K\subset B(0,M)\).

            Par hypothèse, il existe \( R\) tel que \( | z |>R\Rightarrow | f(z) |>M\). Soit \( W=\overline{ B(0,R) }^c\cup\{ \infty \}\). Nous avons la chaine
            \begin{equation}
                z\in W\Rightarrow| z |>R\Rightarrow| f(z) |>M\Rightarrow f(z)\in K^c\Rightarrow f(z)\in V.
            \end{equation}
    \end{subproof}
\end{proof}

\begin{proposition}[\cite{MonCerveau}]     \label{PROPooPWVWooGuftxZ}
    Soit le polynôme
    \begin{equation}
        \begin{aligned}
            P\colon \eC&\to \eC \\
            z&\mapsto \sum_{i=0}^na_iz^i 
        \end{aligned}
    \end{equation}
    où nous sous-entendons que \( a_n\neq 0\). La fonction \( z\mapsto | P(z) |\) est équivalente\footnote{Définition \ref{DEFooWDSAooKXZsZY}.} en l'infini à la fonction
    \begin{equation}
        \begin{aligned}
            w\colon \eC&\to \eR^+ \\
            z&\mapsto | a_nz^n |. 
        \end{aligned}
    \end{equation}
\end{proposition}

\begin{proof}
    Nous voudrions prouver qu'il existe une fonction \( \alpha\colon \eC\to \eR\) telle que 
    \begin{subequations}     \label{EQooGXWZooDJZNzE}
        \begin{numcases}{}
        | \sum_{i=0}^na_iz^i |=\big( 1+\alpha(z) \big)| a_nz^n |.
         \lim_{z\to \infty} \alpha(z)=0.
        \end{numcases}
    \end{subequations}
    Nous trouvons un candidat pour être une telle fonction en isolant simplement \( \alpha(z)\) de cette égalité. Nous trouvons
    \begin{equation}
        \alpha(z)=\big| \sum_{i=0}^n\frac{ a_i }{ a_n }z^{i-n} \big|-1.
    \end{equation}
    Elle vérifie immédiatement \eqref{EQooGXWZooDJZNzE}. Le point qui fait intervenir la topologie de  est de vérifier que \( \lim_{z\to \infty} \alpha(z)=0\). Le terme \( i=0\) de la somme vaut \( 1\). Il suffit donc de montrer que pour \( i\neq 0\) nous avons
    \begin{equation}
        \lim_{z\to \infty} \frac{1}{ z^{n-i} }=0.
    \end{equation}
    Soit \( \epsilon>0\). Nous devons prouver qu'il existe un voisinage \( V\) de \( \infty\) dans \( \hat \eC\) tel que
    \begin{equation}
        | \frac{1}{ z^{n-i} }-0 |\leq \epsilon
    \end{equation}
    pour tout \( z\in V\).
    
    En utilisant la proposition \ref{PROPooXLARooYSDCsF} nous avons déjà
    \begin{equation}
        | \frac{1}{ z^{n-i} } |=\frac{1}{ | z^{n-i} | }=\frac{1}{ | z |^{n-i} }.
    \end{equation}
    Soit \( R>0\) tel que \( \frac{1}{ R }<\epsilon\). Nous considérons le voisinage \( \{ | z |>R \}\cup \{ \infty \}\) de \( \infty\). Dans ce voisinage, nous avons
    \begin{equation}
        \frac{1}{ | z |^{n-i} }\leq \frac{1}{ | z | }\leq \frac{1}{ R }<\epsilon.
    \end{equation}
    Et voila.
\end{proof}

Le lemme suivant parle de polynôme sur \( \eC\). Vous pouvez l'adapter à \( \hat \eR\) et \( \bar \eR\).
\begin{lemma}       \label{LEMooYZVGooXZvBAc}
    Si \( P\colon \eC\to \eC\) est un polynôme, alors \( | P |\) atteint une borne inférieure globale.
\end{lemma}

\begin{proof}
    Nous savons, par l'équivalence de fonctions prouvée dans la proposition \ref{PROPooPWVWooGuftxZ} que \( \lim_{z\to \infty} P(z)=\infty\). Soit \( a>0\) dans \( \eR\). Par le lemme \ref{LEMooERABooQjLBzW} il existe un \( R>a\) tel que \( | z |>R\Rightarrow | f(z) |>| f(a) |\).

    La fonction \( | P |\) est continue sur le compact \( \overline{ B(0,R) }\). Soit \( z_0\) le point de minimum\footnote{Théorème de Weierstrass \ref{ThoWeirstrassRn}.} de \( | P |\) sur \( \overline{ B(0,R) }\).

    Nous devons prouver que \( z_0\) donne même un minimum global. Vu que \( a\in\overline{ B(0,R) }\) nous avons
    \begin{equation}
        | f(z_0) |\leq | f(a) |.
    \end{equation}
    Si \( z\in \overline{ B(0,R) }^c\), nous avons
    \begin{equation}
        | f(z) |>| f(a) |\geq | f(z_0) |.
    \end{equation}
    Donc ce \( z_0\) est un minimum sur \( B(0,R)\) et sur \( \overline{ B(0,R) }^c\). Bref, un minimum global.
\end{proof}

\begin{lemma}       \label{LEMooTTOYooXaukuH}
    Soit le polynôme
    \begin{equation}
        \begin{aligned}
            P\colon \eC&\to \eC \\
            z&\mapsto \sum_{i=0}^na_iz^i. 
        \end{aligned}
    \end{equation}
    La fonction \( P\) est équivalente à \( a_0+a_1z\) en \( z=0\).
\end{lemma}

\begin{proof}
    En posant \( g(z)=a_0+a_1z\), nous devons trouver une fonction \( \alpha\) telle que
    \begin{equation}        \label{EQooZFJBooVAYVBv}
        P(z)=\big( 1+\alpha(z) \big)g(z).
    \end{equation}
    Si \( a_0\neq 0\), il existe un voisinage de \( z=0\) sur lequel la fonction
    \begin{equation}        \label{EQooVCOVooAKWJxF}
        \alpha(z)=\frac{ z^2\sum_{i=2}^na_iz^{i-2} }{ a_0+a_1z }
    \end{equation}
    existe. Il n'y a aucun problème à ce que \( \alpha(z)\to 0\) pour \( z\to 0\)\footnote{En remarquant toutefois que c'est une limite à deux dimensions. Sachez la définir.}, et un simple calcul\footnote{En fait, la formule \eqref{EQooVCOVooAKWJxF} est obtenue en isolant \( \alpha(z)\) dans \eqref{EQooZFJBooVAYVBv}.} donne \eqref{EQooVCOVooAKWJxF}.

    Si par contre \( a_0=0\), nous faisons le calcul intermédiaire suivant :
    \begin{equation}
        \alpha(z)g(z)=P(z)-g(z)=z^2\sum_{i=2}^na_iz^{i-2},
    \end{equation}
    et donc, en isolant \( \alpha(z)\) et en simplifiant par \( z\), nous voyons que la fonction \( \alpha\) définie par
    \begin{equation}
        \alpha(z)=\frac{z}{ a_1 }\sum_{i=2}^na_iz^{i-2}
    \end{equation}
    fonctionne.
\end{proof}

\begin{proposition}[\cite{ooRIPVooMlBiAH,MonCerveau}]       \label{PROPooLBBLooQwEiHr}
    Soient \( a,b\in \eR\).
    \begin{enumerate}
        \item       \label{ITEMooSPSWooKLtqzZ}
            L'équation \( z^2=a+bi\) a une solution dans \( \eC\).
        \item       \label{ITEMooQOJDooWjfGXv}
            Pour tout \( l\), l'équation \( z^{2^l}=a+bi\) a une solution dans \( \eC\).
    \end{enumerate}
    Nous ne disons pas que ces solutions sont uniques\footnote{Comme vous en conviendrez en pensant à \( z^2=1\) qui a déjà les solutions \( 1\) et \( -1\).}.
\end{proposition}

\begin{proof}
    Pour prouver \ref{ITEMooSPSWooKLtqzZ}, l'équation \( z^2=a+bi\) a pour solution \( \pm\xi\) où
        \begin{equation}
            \xi=\sqrt{ \frac{ 1 }{2}a+\frac{ 1 }{2}\sqrt{ a^2+b^2 } }+i\signe(b)\sqrt{ -\frac{ 1 }{2}a+\frac{ 1 }{2}\sqrt{ a^2+b^2 } }.
        \end{equation}
        Nous n'avons en fait pas besoin de montrer que \( \pm\xi\) sont toutes deux des solutions, ni que ce sont les seules. Un calcul direct montre que \( \xi^2=a+bi\) et nous sommes content.

    Pour \ref{ITEMooQOJDooWjfGXv}, nous faisons une récurrence sur \( l\). Nous savons que
        \begin{equation}
            z^{2^{k+1}}=(z^{2^k})^2.
        \end{equation}
        Soit \( \xi\in \eC\) tel que \( \xi^{2^k}=a+bi\); un tel \( \xi\) existe par hypothèse de récurrence. Alors si \( z\) est tel que \( z^2=\xi\), nous avons 
        \begin{equation}
            z^{2^{k+1}}=a+bi.
        \end{equation}
\end{proof}

Le théorème de d'Alembert possède de nombreuses démonstrations. En voici une qui à ma connaissance est celle demandant le moins d'analyse; une démonstration à base de théorie de Galois peut être trouvée dans \cite{rqrNyg,ooPSLMooAVODjn}. Si vous lisez ces lignes pour savoir qu'un polynôme de degré \( n\) possède au \emph{maximum} \( n\) racines, ce n'est pas ici qu'il faut regarder, mais le corolaire \ref{CORooUGJGooBofWLr}.
\begin{theorem}[d'Alembert\cite{ooRIPVooMlBiAH}]   \label{THOooIRJYooBiHRyW}
    Tout polynôme non constant à coefficients complexes admet au moins une racine complexe.
\end{theorem}

\begin{proof}
    Nous effectuons une preuve tout à la fois par l'absurde et par récurrence en supposant que le polynôme
    \begin{equation}
        \begin{aligned}
            f\colon \eC&\to \eC \\
            z&\mapsto z^n+a_1z^{n-1}+\ldots+a_n 
        \end{aligned}
    \end{equation}
    n'a pas de racines dans \( \eC\), et que \( n\) soit le plus petit entier pour lequel un tel polynôme existe. Nous notons
    \begin{equation}
        n=2^km
    \end{equation}
    où \( m\) est impair.

    Le lemme \ref{LEMooYZVGooXZvBAc} donne un point \( z_0\) qui réalise le minimum global de \( | f |\) sur $\eC$. Nous posons \( g(z)=f(z_0+z)\) et nous définissons ses coefficients \( A_i\) par
    \begin{equation}
        g(z)=\sum_{i=0}^nA_iz^i.
    \end{equation}
    Nous avons \( A_n=1\) et \( | A_0 |=| f(z_0) |\). Soit \( A_r\) le premier à être non nul parmi les \( A_1\), \( A_2\), \ldots.
    \begin{subproof}
        \item[Si \( r<n\)]
            Par hypothèse de récurrence, il existe \( \xi\in \eC\) tel que \( \xi^r=-A_1/A_r\). Nous avons
            \begin{equation}
                g(t\xi)=A_0+\frac{ -A_rt^rA_0 }{ A_r }+t^{r+1}\sum_{i=r+1}^nA_i\xi^it^{i-r-1}.
            \end{equation}
            En notant \( P(t)\) le dernier polynôme, nous pouvons écrire cela sous forme compacte :
            \begin{equation}
                g(t\xi)=A_0-t^rA_0+t^{r+1}P(t).
            \end{equation}
            Vu que
            \begin{equation}
                \lim_{t\to 0} \frac{ t^{r+1}P(t) }{ t^r| A_0 | }=\lim_{t\to 0} tP(t)=0,
            \end{equation}
            il existe \( t_0>0\) tel que
            \begin{equation}
                | t_0^{r+1}P(t_0) |<| A_0t_0r |.
            \end{equation}
            Nous choisissons de plus \( t_0<1\), de telle sorte que \( 1-t^r>0\). Avec cela nous avons
            \begin{equation}
                | g(t\xi) |\leq | A_0 |(1-t^r)+| t^{r+1}P(t) |=| A_0 |\underbrace{-t^r| A_0 |+| t^{r+1}P(t) |}_{<0}<| A_0 |.
            \end{equation}
            Or \( | A_0 |\) était un minimum global de \( | g |\). Contradiction.

        \item[Si \( r=n\)]

            Dans ce cas,
            \begin{equation}
                g(z)=f(z_0+z)=A_0+z^n,
            \end{equation}
            et nous rappelons que \( n=2^km\) où \( m\) est impair. Nous allons trouver une contradiction dans les quatre cas \( \real{A_0}>0\), \( \real(A_0)<0\), \( \imag(A_0)>0\) et \( \imag(A_0)<0\). Bien entendu ces cas se recouvrent largement, mais en toute généralité, nous avons besoin des quatre.
            \begin{subproof}
                \item[Si \( \real(A_0)>0\)]
                    La proposition \ref{PROPooLBBLooQwEiHr} nous permet de considérer \( v\in \eC\) tel que \( v^{2^k}=-1\). Nous avons alors
                    \begin{equation}
                        g(tv)=A_0+(tv)^n=A_0+t^n(v^{2^k})^m=A_0+t^n(-1)^m=A_0-t^n
                    \end{equation}
                    parce que \( m\) est impair. Nous avons \( \imag\big( g(tv) \big)=\imag(A_0)\). Si \( t\) est assez petit pour que \( t^n<| \real(A_0) |\) nous avons aussi \( |\real\big( g(tv) \big)|<| \real(A_0) |\). Donc
                    \begin{equation}
                        | g(tv) |^2=| \real\big( g(tv) \big) |^2+| \real\big( g(tv) \big) |^2<| \real(A_0) |^2+| \imag(A_0) |^2=| A_0 |^2.
                    \end{equation}
                    Donc \( | g(tv) |<| A_0 |\). Contradiction.
                \item[Si \( \real(A_0)<0\)]
                    Nous prenons \( v=1\), et même histoire.
                \item[Si \( \imag(A_0)<0\)]
                    Nous prenons \( w\in \eC\) tel que
                    \begin{equation}
                        w^{2^k}=i(-1)^{\frac{ 1 }{2}(m-1)}.
                    \end{equation}
                    Là, il y a un peu d'arrachage de cheveux pour bien voir les cas. La difficulté est que les puissances de \( i\) alternent entre \( 1\), \( -1\), \( i\) et \( -i\). Vu que \( m\) est impair, nous avons un \( l\) tel que \( m=2l+1\). Nous subdivisons les cas \( l\) pair et \( l\) impair.
                    \begin{subproof}
                        \item[Si \( l\) est pair]
                            Alors d'une part \( \frac{ 1 }{2}(m-1)=l\) est pair et donc 
                            \begin{equation}
                                (-1)^{\frac{ 1 }{2}(m-1)}=1.
                            \end{equation}
                            Et d'autre part, \( i^{2l+1}=(-1)^li=i\). En tout,
                            \begin{equation}
                                i^m(-1)^{\frac{ 1 }{2}(m-1)}=i.
                            \end{equation}
                        \item[Si \( l\) est impair]
                            Alors \( \frac{ 1 }{2}(m-1)=l\) et \( (-1)^{\frac{ 1 }{2}(m-1)}=-1\). Mais en même temps, \( i^{2l+1}=-i\), ce qui donne encore une fois
                            \begin{equation}
                                i^m(-1)^{\frac{ 1 }{2}(m-1)}=i.
                            \end{equation}
                    \end{subproof}
                    Bref, que \( l\) soit pair ou impair, nous avons \( i^m(-1)^{\frac{ 1 }{2}(m-1)}=i\).
            \end{subproof}
            Nous avons donc \( \real\big( g(tw) \big)=\real(A_0)\) et \( \imag\big( g(tw) \big)<\imag(A_0)\). Encore contradiction.
                \item[Si \( \imag(A_0)=0\)]
                    Même chose que ce que nous venons de faire, mais avec
                    \begin{equation}
                        w^{2^k}=-i(-1)^{\frac{ 1 }{2}(m-1)}.
                    \end{equation}
    \end{subproof}
\end{proof}

\begin{corollary}
    Le corps $\eC$ est algébriquement clos.
\end{corollary}

\begin{corollary}[\cite{MonCerveau}]       \label{CORooKKNWooWEQukb}
    Tout polynôme de degré \( 3\) à coefficients réels possède au moins une racine réelle.
\end{corollary}

\begin{proof}
    Soient les racines \( \lambda_1\), \( \lambda_2\) et \( \lambda_3\) du polynôme en question. Toutes trois sont dans \( \eC\). Supposons que \( \lambda_1\) ne soit pas réelle. Alors \( \lambda_2\) ou \( \lambda_3\) doit être égale à \( \bar\lambda_1\). Disons \( \lambda_2\). Nous avons donc les racines \( \lambda_1\), \( \bar\lambda_1\) et \( \lambda_3\). Le polynôme se factorise alors en
    \begin{equation}        \label{EQooELMMooNbpBgg}
        a(X-\lambda_1)(X-\bar\lambda_1)(X-\lambda_3).
    \end{equation}
    Le coefficient \( a\) doit être réel parce qu'il est le coefficient du terme en \( X^3\) (réel par hypothèse). Si \( \lambda_3\) n'est pas réel, alors ce polynôme ne peut pas avoir des coefficients réels. Entre autres parce que terme indépendant est \( a| \lambda_1 |^2\lambda_3\), qui est réel si et seulement si \( \lambda_3\) est réel\footnote{Notez l'utilisation du lemme~\ref{LEMooONLNooXLNbtB}.}.
\end{proof}
Tant que vous y êtes, vous pouvez voir que le polynôme \eqref{EQooELMMooNbpBgg} est à coefficient réels si et seulement si \( a\in \eR\) et \( \lambda_3\in \eR\).

\begin{example}     \label{EXooIPLOooSNfiWg}
    Toute application linéaire \( \eR^3\to \eR^3\) a un vecteur propre. En effet si \( R\colon \eR^3\to \eR^3\) est linéaire, son polynôme caractéristique \( \chi_R\) est de degré \( 3\). Le corolaire \ref{CORooKKNWooWEQukb} indique qu'un tel polynôme possède au moins une racine réelle.
    Une telle racine est une valeur propre de \( R\) par le théorème \ref{ThoWDGooQUGSTL}.
\end{example}

\begin{definition}
    Si \( \lambda\in\eK\) est une racine de \( \chi_u\), l'ordre de l'annulation est la \defe{multiplicité algébrique}{multiplicité!valeur propre!algébrique} de la valeur propre \( \lambda\) de \( u\). À ne pas confondre avec la \defe{multiplicité géométrique}{multiplicité!valeur propre!géométrique} qui sera la dimension de l'espace propre.
\end{definition}

\begin{proposition}
    Un polynôme irréductible à coefficients réels est soit de degré un soit de degré \( 2\) avec un discriminant négatif.
\end{proposition}

\begin{proof}
    Soit un polynôme \( P\) à coefficients réels de degré plus grand que \( 1\). Alors le théorème de d'Alembert-Gauss (théorème~\ref{THOooIRJYooBiHRyW}) implique l'existence d'une racine \( \alpha \in \eC \). Si $\alpha$ est un réel, $P$ est réductible. Si \( \alpha\) n'est pas réel, alors conjugué complexe \( \bar \alpha\) est également une racine. Par conséquent les polynômes \( (X-\alpha)\) et \( (X-\bar \alpha)\) divisent \( P\) dans \( \eC[X]. \).

    Ces deux polynômes sont premiers entre eux parce que
    \begin{equation}
        a(X-\alpha)+b(X-\bar \alpha)=0
    \end{equation}
    implique \( a=b=0\). Par conséquent le produit
    \begin{equation}
        X^2-(\alpha+\bar \alpha)X+\alpha\bar\alpha
    \end{equation}
    divise également \( P\). Ce dernier est un polynôme à coefficients réels de degré \( 2\). Donc tout polynôme de degré \( 3\) ou plus est réductible.
\end{proof}

\begin{proposition}     \label{PROPooLXGSooXmVcVG}
    Si \( E\) est un espace vectoriel sur \( \eC\), tout endomorphisme possède au moins une valeur propre.
\end{proposition}

\begin{proof}
    Soit un endomorphisme \( u\) sur \( E\). Le théorème \ref{ThoWDGooQUGSTL} dit que \( \lambda\in \eC\) est une valeur propre si et seulement si \( \lambda\) est une racine du polynôme caractéristique \( \chi_u\). Or ce polynôme possède au moins une racine dans \( \eC\) par le théorème de d'Alembert \ref{THOooIRJYooBiHRyW}.
\end{proof}

%+++++++++++++++++++++++++++++++++++++++++++++++++++++++++++++++++++++++++++++++++++++++++++++++++++++++++++++++++++++++++++
\section{Dérivée : exemples introductifs}
%+++++++++++++++++++++++++++++++++++++++++++++++++++++++++++++++++++++++++++++++++++++++++++++++++++++++++++++++++++++++++++

%---------------------------------------------------------------------------------------------------------------------------
\subsection{La vitesse}
%---------------------------------------------------------------------------------------------------------------------------

Lorsqu'un mobile se déplace à une vitesse variable, nous obtenons la \emph{vitesse instantanée} en calculant une vitesse moyenne sur des intervalles de plus en plus petits. Si le mobile a un mouvement donné par $x(t)$, la vitesse moyenne entre $t=2$ et $t=5$ sera
\[
  v_{\text{moy}}(2\to 5)=\frac{ x(5)-x(2) }{ 5-2 }.
\]
Plus généralement, la vitesse moyenne entre $2$ et $2+\Delta t$ est donnée par
\[
  v_{\text{moy}}(2\to 2+\Delta t)=\frac{ x(2+\Delta t)-x(2) }{ \Delta t }.
\]
Cela est une fonction de $\Delta t$. Oui, mais je te rappelle qu'on a dans l'idée de calculer une vitesse instantanée, c'est-à-dire de voir ce que vaut la vitesse moyenne sur un intervalle très {\small très} {\footnotesize très} {\scriptsize très} {\tiny petit}. La notion de limite semble toute indiquée pour décrire mathématiquement l'idée physique de vitesse instantanée.

Nous allons dire que la vitesse instantanée d'un mobile est la limite quand $\Delta t$ tend vers zéro de sa vitesse moyenne sur l'intervalle de temps $\Delta t$, ou en formule :
\begin{equation}		\label{Eqvinstlimite}
	v(t_0)=\lim_{\Delta t\to 0}\frac{ x(t_0)-x(t_0+\Delta t) }{ \Delta t }.
\end{equation}

%---------------------------------------------------------------------------------------------------------------------------
\subsection{La tangente à une courbe}
%---------------------------------------------------------------------------------------------------------------------------

Passons maintenant à tout autre chose, mais toujours dans l'utilisation de la notion de limite pour résoudre des problèmes intéressants. Comment trouver l'équation de la tangente à la courbe $y=f(x)$ au point $(x_0,f(x_0))$ ?

Essayons de trouver la tangente au point $P$ donné de la courbe donnée à la figure~\ref{LabelFigTangenteQuestion}.

\newcommand{\CaptionFigTangenteQuestion}{Comment trouver la tangente à la courbe au point $P$ ?}
\input{auto/pictures_tex/Fig_TangenteQuestion.pstricks}

La tangente est la droite qui touche la courbe en un seul point sans la traverser. Afin de la construire, nous allons dessiner des droites qui touchent la courbe en $P$ et un autre point $Q$, et nous allons voir ce qu'il se passe quand $Q$ est très proche de $P$. Cela donnera une droite qui, certes, touchera la courbe en deux points, mais en deux points \emph{tellement proches que c'est comme si c'étaient les mêmes}. Tu sens que la notion de limite va encore venir.

%Pour rappel cette figure TangenteDetail est générée par phystricksRechercheTangente.py
\newcommand{\CaptionFigTangenteDetail}{Traçons d'abord une corde entre le point $P$ et un point $Q$ un peu plus loin.}
\input{auto/pictures_tex/Fig_TangenteDetail.pstricks}

Nous avons placé le point, sur la figure~\ref{LabelFigTangenteDetail}, le point $P$ en $a$ et le point $Q$ un peu plus loin $x$. En d'autres termes leurs coordonnées sont
\begin{align}
	P=\big(a,f(a)\big)&& Q=\big(x,f(x)\big).
\end{align}
En regardant par exemple la figure~\ref{LabelFigTangenteDetail}, le coefficient directeur de la droite qui passe par ces deux points est donné par
\begin{equation}
	\frac{ f(x)-f(a) }{ x-a },
\end{equation}
et bang ! Encore le même rapport que celui qu'on avait trouvé à l'équation \eqref{Eqvinstlimite} en parlant de vitesses. Si tu regardes la figure~\ref{LabelFigLesSubFigures}, tu verras que réellement en faisant tendre $x$ vers $a$ on obtient la tangente.

\newcommand{\CaptionFigLesSubFigures}{Recherche de la tangente par approximations successives.}
\input{auto/pictures_tex/Fig_LesSubFigures.pstricks}
%See also the subfigure~\ref{LabelFigLesSubFiguressssubZ}
%See also the subfigure~\ref{LabelFigLesSubFiguressssubO}
%See also the subfigure~\ref{LabelFigLesSubFiguressssubT}
%See also the subfigure~\ref{LabelFigLesSubFiguressssubTh}
%See also the subfigure~\ref{LabelFigLesSubFiguressssubF}
%See also the subfigure~\ref{LabelFigLesSubFiguressssubFi}

%---------------------------------------------------------------------------------------------------------------------------
\subsection{L'aire en dessous d'une courbe}		\label{SubSecAirePrimInto}
%---------------------------------------------------------------------------------------------------------------------------

Encore un exemple. Nous voudrions bien pouvoir calculer l'aire en dessous d'une courbe. Nous notons $S_f(x)$ l'aire en dessous de la fonction $f$ entre l'abscisse $0$ et $x$, c'est-à-dire l'aire bleue de la figure~\ref{LabelFigNOCGooYRHLCn}. % From file NOCGooYRHLCn
\newcommand{\CaptionFigNOCGooYRHLCn}{L'aire en dessous d'une courbe. Le rectangle rouge d'aire $f(x)\Delta x$ approxime l'augmentation de l'aire lorsqu'on passe de $x$ à $x+\Delta x$.}
\input{auto/pictures_tex/Fig_NOCGooYRHLCn.pstricks}

Si la fonction $f$ est continue et que $\Delta x$ est assez petit, la fonction ne varie pas beaucoup entre $x$ et $x+\Delta x$. L'augmentation de surface entre $x$ et $x+\Delta x$ peut donc être approximée par le rectangle de surface $f(x)\Delta x$. Ce que nous avons donc, c'est que quand $\Delta x$ est très petit,
\begin{equation}
	S_f(x+\Delta x)-S_f(x)=f(x)\Delta x,
\end{equation}
c'est-à-dire
\begin{equation}
	f(x)=\lim_{\Delta x\to 0}\frac{  S_f(x+\Delta x)-S_f(x)}{ \Delta x }.
\end{equation}
Donc, la fonction $f$ est la dérivée de la fonction qui représente l'aire en dessous de $f$. Calculer des surfaces revient donc au travail inverse de calculer des dérivées.

Nous avons déjà vu que calculer la dérivée d'une fonction n'est pas très compliqué. Aussi étonnant que cela puisse paraitre, il se fait que le processus inverse est très compliqué : il est en général extrêmement difficile (et même souvent impossible) de trouver une fonction dont la dérivée est une fonction donnée.

Une fonction dont la dérivée est la fonction $f$ s'appelle une \defe{primitive}{primitive} de $f$, et la fonction qui donne l'aire en dessous de la fonction $f$ entre l'abscisse $0$ et $x$ est notée
\begin{equation}
	S_f(x)=\int_0^xf(t)dt.
\end{equation}
Nous pouvons nous demander si, pour une fonction $f$ donnée, il existe une ou plusieurs primitives, c'est-à-dire s'il existe une ou plusieurs fonctions $F$ telles que $F'=f$. La réponse viendra\ldots
%TODO : faire la référence

%+++++++++++++++++++++++++++++++++++++++++++++++++++++++++++++++++++++++++++++++++++++++++++++++++++++++++++++++++++++++++++
\section{Dérivation de fonctions réelles}
%+++++++++++++++++++++++++++++++++++++++++++++++++++++++++++++++++++++++++++++++++++++++++++++++++++++++++++++++++++++++++++
\label{seccontetderiv}

On considère dans la suite une fonction $f : A \to \eR$, où $a \in A \subset \eR$ ; cependant, les notions de continuité et de dérivabilité se généralisent immédiatement au cas de fonctions à valeurs vectorielles ; la notion de continuité se généralise au cas des fonctions à plusieurs variables (la notion de dérivabilité est remplacée par celle de différentiabilité dans ce cadre).

\begin{definition}      \label{DEFooOYFZooFWmcAB}
    La fonction $f$ est \defe{dérivable}{dérivable} en \( a\) si $a \in
  \operatorname{int} A$ et si
  \begin{equation*}
    \lim_{x\to a} \frac{f(x)-f(a)}{x-a}
  \end{equation*}
  existe. On note alors cette quantité $f'(a)$, c'est le nombre
  dérivé de $f$ en $a$. La \defe{fonction dérivée}{fonction dérivée} de $f$ est
  \begin{equation}
      \begin{aligned}
          f'\colon A'&\to \eR \\
          a&\mapsto f(a)
      \end{aligned}
  \end{equation}
  définie sur l'ensemble noté $A'$ des points $a$ où $f$ est dérivable.
\end{definition}

\begin{example}
      Montrons que la fonction $f : \eR \to \eR : x\mapsto x$ est continue et dérivable. Exceptionnellement (bien qu'on sache que la dérivabilité implique la continuité), montrons ces deux assertions séparément.
      \begin{description}
      \item[Continuité] Pour prouver la continuité au point $a \in \eR$ nous devons montrer que
     \begin{equation}
       \limite x a x = a
     \end{equation}
     c'est-à-dire
     \begin{equation}
       \forall \epsilon > 0, \exists \delta > 0 :  \forall x \in \eR \abs{x-a} <
       \delta \Rightarrow \abs{x-a} < \epsilon
     \end{equation}
     ce qui est clair en prenant $\delta = \epsilon$.

      \item[Dérivabilité] Soit $a \in \eR$. Calculons la limite du quotient différentiel
        \begin{equation}
          \limite[x\neq a]{x}{a} \frac{x-a}{x-a} = \limite[x\neq a]x a 1 = 1
        \end{equation}
        ce qui prouve que $f$ est dérivable et que sa dérivée vaut $1$ en
        tout point $a$ de $\eR$.
      \end{description}

     On a donc montré que la fonction $x \mapsto x$ est continue, dérivable, et que sa dérivée vaut $1$ en tout point $a$ de son domaine.

\end{example}

\begin{proposition} \label{PropSFyxOWF}
    Une fonction dérivable sur un intervalle est continue sur cet intervalle.
\end{proposition}

\begin{proof}
    Soit \( I\) un intervalle sur lequel la fonction \( f\) est dérivable, et soit \( x_0\in I\). Nous allons prouver la continuité de \( f\) en \( x_0\). Le fait que la limite
    \begin{equation}
        f'(x_0)=\lim_{h\to 0} \frac{ f(x_0+h)-f(x_0) }{ h }
    \end{equation}
    existe implique a fortiori que
    \begin{equation}
        \lim_{h\to 0} f(x_0+h)-f(x_0)=0.
    \end{equation}
    Cela signifie la continuité de \( f\) en vertu du critère~\ref{ThoLimCont}.
\end{proof}

\begin{theorem} \label{THOooFFOZooCYGets}
  Toute fonction dérivable en un point est continue en ce point.
\end{theorem}

\begin{proof}
    Soient \( f\colon \eR\to \eR\) et \( a\in \eR\). Nous supposons que \( f\) n'est pas continue en \( a\) et nous allons en déduire qu'elle n'est pas non plus dérivable en \( a\). Pour cela nous considérons le lien entre limite et continuité donné dans le théorème \ref{ThoLimCont}. Nier que \( f\) est continue en \( a\) revient à dire qu'il existe un voisinage \( V\) de \( f(a)\) tel que
    \begin{equation}
        \forall r>0,\,\exists \epsilon<r \tq f(a+\epsilon)\notin V.
    \end{equation}
    Si \( B\big( f(a),R \big)\subset V\)\footnote{Existence par la définition de la topologie métrique \ref{ThoORdLYUu}.}, et si \( r=1/n\), nous construisons une suite \( \epsilon_n\to 0\) telle que
    \begin{equation}
        | f(a+\epsilon_n)-f(a) |>R.
    \end{equation}
    Avec cela nous avons
    \begin{equation}
        \frac{ | f(a+\epsilon_n)-f(a) | }{ \epsilon_n }>\frac{ R }{ \epsilon_n }\to \infty.
    \end{equation}
    Donc la fonction \( f\) ne peut pas être dérivable en \( a\).
\end{proof}

\begin{remark}
     La réciproque du théorème précédent n'est pas vraie : il existent bien des fonctions qui sont continues à un point $x_0$ mais qui ne sont pas dérivables en $x_0$. La fonction valeur absolue, $x\mapsto |x|$, par exemple est continue sur tout $\eR$ mais elle n'est pas dérivable en $0$.
\end{remark}

Si \( f\) est une fonction dérivable, il peut arriver que la fonction dérivée \( f'\) soit elle-même dérivable. Dans ce cas nous notons \( f''\) ou \( f^{(2)}\) la dérivée de la fonction \( f'\). Cette fonction $f''$ est la \defe{dérivée seconde}{dérivée!seconde} de \( f\). Elle peut encore être dérivable; dans ce cas nous notons \( f^{(3)}\) sa dérivée, et ainsi de suite. Nous définissons \( f^{(n)}=(f^{(n-1)})'\) la dérivée \( n\)\ieme de \( f\). Nous posons évidemment $f^{(0)}=f$.

%---------------------------------------------------------------------------------------------------------------------------
\subsection{Exemples}
%---------------------------------------------------------------------------------------------------------------------------

\begin{example}
    Commençons par la fonction $f(x)=x$. Dans ce cas nous avons
    \begin{equation}
        \frac{ f(x)-f(a) }{ x-a }=\frac{ x-a }{ x-a }=1.
    \end{equation}
    La dérivée est donc $1$.
\end{example}

\begin{proposition}
    La dérivé de la fonction $x\mapsto x$ vaut $1$, en notations compactes : $(x)'=1$.
\end{proposition}

\begin{proof}
    D'après la définition de la dérivée, si $f(x)=x$, nous avons
    \begin{equation}
        f(x)=\lim_{\epsilon\to 0}\frac{ (x+\epsilon) -x }{\epsilon} =\lim_{\epsilon\to 0}\frac{ \epsilon }{\epsilon} =1,
    \end{equation}
    et c'est déjà fini.
\end{proof}

%///////////////////////////////////////////////////////////////////////////////////////////////////////////////////////////
\subsubsection{La fonction carré}
%///////////////////////////////////////////////////////////////////////////////////////////////////////////////////////////

Prenons ensuite $f(x)=x^2$. En utilisant le produit remarquable $(x^2-a^2)=(x-a)(x+a)$ nous trouvons
\begin{equation}
	\frac{ f(x)-f(a) }{ x-a }=x+a.
\end{equation}
Lorsque $x\to a$, cela devient $2a$. Nous avons par conséquent
\begin{equation}
	f'(x)=2x.
\end{equation}

\begin{lemma}           \label{LemDeccCarr}
    Si $f(x)=x^2$, alors $f'(x)=2x$.
\end{lemma}

\begin{proof}
    Utilisons la définition, et remplaçons $f$ par sa valeur :
    \begin{subequations}
        \begin{align}
            f'(x)   &=\lim_{\epsilon\to 0}\frac{ f(x+\epsilon)-f(x) }{ \epsilon }\\
                &=\lim_{\epsilon\to 0}\frac{ (x+\epsilon)^2-x^2 }{ \epsilon }\\
                &=\lim_{\epsilon\to 0}\frac{ x^2+2x\epsilon+\epsilon^2-x^2 }{ \epsilon }\\
                &=\lim_{\epsilon\to 0}\frac{\epsilon(2x+\epsilon)}{ \epsilon }\\
                &=\lim_{\epsilon\to 0}(2x+\epsilon)\\
                &=2x,
        \end{align}
    \end{subequations}
    ce qu'il fallait prouver.
\end{proof}


%///////////////////////////////////////////////////////////////////////////////////////////////////////////////////////////
\subsubsection{La fonction racine carré}
%///////////////////////////////////////////////////////////////////////////////////////////////////////////////////////////

Considérons maintenant la fonction $f(x)=\sqrt{x}$. Nous avons
\begin{equation}
	\begin{aligned}[]
		\frac{ f(x)-f(a) }{ x-a }&=\frac{ \sqrt{x}-\sqrt{a} }{ x-a }\\
		&=\frac{ (\sqrt{x}-\sqrt{a})(\sqrt{x}+\sqrt{x}) }{ (x-a)(\sqrt{x}+\sqrt{x}) }\\
		&=\frac{1}{ \sqrt{x}+\sqrt{x} }.
	\end{aligned}
\end{equation}
Lorsque $x\to 0$, nous obtenons
\begin{equation}
	f'(a)=\frac{1}{ 2\sqrt{a} }.
\end{equation}
Notons que la dérivée de $f(x)=\sqrt{x}$ n'existe pas en $x=0$. En effet elle serait donnée par le quotient
\begin{equation}
	f'(0)=\lim_{x\to 0} \frac{ \sqrt{x}-\sqrt{0} }{ x }=\lim_{x\to 0} \frac{ \sqrt{x} }{ x }=\lim_{x\to 0} \frac{1}{ \sqrt{x} }.
\end{equation}
Mais si $x$ devient très petit, la dernière fraction tend vers l'infini.

%--------------------------------------------------------------------------------------------------------------------------
\subsection[Interprétation géométrique : tangente]{Interprétation géométrique de la dérivée : tangente}
%--------------------------------------------------------------------------------------------------------------------------

Considérons le \defe{graphe}{graphe} de la fonction $f$ sur $I$, c'est-à-dire l'ensemble
\begin{equation}
	\big\{ \big( x,f(x) \big)\tq x\in I \big\}.
\end{equation}
Le nombre
\begin{equation}
	\frac{ f(x)-f(a) }{ x-a }
\end{equation}
est la pente de la droite qui joint les points $\big( x,f(x) \big)$ et $\big( a,f(a) \big)$, voir la figure ~\ref{LabelFigGWOYooRxHKSm}. % From file GWOYooRxHKSm
\newcommand{\CaptionFigGWOYooRxHKSm}{Le coefficient directeur de la corde entre $a$ et $x$.}
\input{auto/pictures_tex/Fig_GWOYooRxHKSm.pstricks}

Étant donné que $f'(a)$ est le coefficient directeur de la tangente au point $\big( a,f(a) \big)$, l'équation de la tangente est
\begin{equation}		\label{EqTgfaen}
	y-f(a)=f'(a)(x-a).
\end{equation}

%--------------------------------------------------------------------------------------------------------------------------
\subsection[Interprétation géométrique : approximation affine]{Interprétation géométrique de la dérivée : approximation affine}
%--------------------------------------------------------------------------------------------------------------------------

Le fait que la fonction $f$ soit dérivable au point $a\in I$ signifie que
\begin{equation}
	\lim_{x\to a} \frac{ f(x)-f(a) }{ x-a }=\ell
\end{equation}
pour un certain nombre $\ell$. Cela peut être récrit sous la forme
\begin{equation}
	\lim_{x\to a} \frac{ f(x)-f(a) }{ x-a }-\ell=0,
\end{equation}
ou encore
\begin{equation}
	\lim_{x\to a} \frac{ f(x)-f(a)-\ell(x-a) }{ x-a }=0.
\end{equation}
Introduisons la fonction
\begin{equation}
	\alpha(t)=\frac{ f(a+t)-f(a)-t\ell }{ t }.
\end{equation}
Cette fonction est faite exprès pour que
\begin{equation}		\label{EqIntermsaxaama}
	\alpha(x-a)=\frac{ f(x)-f(a)-\ell(x-a) }{ x-a };
\end{equation}
par conséquent $\lim_{x\to a} \alpha(x-a)=0$. Nous récrivons l'équation \eqref{EqIntermsaxaama} sous la forme
\begin{equation}        \label{EqCodeDerviffxam}
	f(x)-f(a)-\ell(x-a)=(x-a)\alpha(x-a).
\end{equation}
Le second membre tend vers zéro lorsque $x$ tend vers $a$ avec une «vitesse au carré» : c'est le produit de deux facteurs tous deux tendant vers zéro. Si $x$ n'est pas très loin de $a$, il n'est donc pas une mauvaise approximation de dire
\begin{equation}
	f(x)-f(a)-\ell(x-a)\simeq 0,
\end{equation}
c'est-à-dire
\begin{equation}		\label{Eqfxsimesfa}
	f(x)\simeq f(a)+f'(a)(x-a).
\end{equation}
Nous avons retrouvé l'équation \eqref{EqTgfaen}. La manipulation que nous venons de faire revient donc à dire que la fonction $f$, au voisinage de $a$, est bien approximée par sa tangente.

L'équation \eqref{Eqfxsimesfa} peut être aussi écrite sous la forme
\begin{equation}		\label{EqfxdxSimeqfxfpx}
	f(x+\Delta x)\simeq f(x)+f'(x)\Delta x
\end{equation}
qui est une approximation d'autant meilleure que $\Delta x$ est petit.

%---------------------------------------------------------------------------------------------------------------------------
\subsection{Développement limité au premier ordre}
%---------------------------------------------------------------------------------------------------------------------------

Si une fonction est dérivable en \( a\) alors elle peut être approximée «au premier ordre» par une formule simple qui sera généralisé pour des dérivées d'ordre supérieurs avec les séries de Taylor, théorème~\ref{ThoTaylor}.
\begin{proposition}[Développement limité au premier ordre]  \label{PropUTenzfQ}
    Si \( f\colon \eR\to \eR\) est une fonction dérivable, alors is existe une fonction \( \alpha\colon \eR\to \eR\) telle que
    \begin{equation}
        f(a+h)=f(a)+hf'(a)+\alpha(h)
    \end{equation}
    et
    \begin{equation}
        \lim_{h\to 0} \frac{ \alpha(h) }{ h }=0.
    \end{equation}
\end{proposition}
\index{développement!limité!premier ordre}

\begin{proof}
    La fonction \( f\) étant dérivable en \( a\) nous avons l'existence de la limite suivante :
    \begin{equation}
        f'(a)=\lim_{h\to 0} \frac{ f(a+h)-f(a) }{ h },
    \end{equation}
    ce qui revient à dire qu'en définissant la fonction \( \beta\) par
    \begin{equation}
        f'(a)=\frac{ f(a+h)-f(a) }{ h }+\beta(h)
    \end{equation}
    alors \( \beta(h)\to 0\) lorsque \( h\to 0\). En multipliant par \( h\) et en nommant \( \alpha(h)=h\beta(h)\) nous trouvons le résultat :
    \begin{equation}
        f(a+h)=f(a)+hf'(a)+\alpha(h)
    \end{equation}
    avec
    \begin{equation}
        \lim_{h\to 0} \frac{ \alpha(h) }{ h }=\lim_{h\to 0} \beta(h)=0.
    \end{equation}
\end{proof}

%+++++++++++++++++++++++++++++++++++++++++++++++++++++++++++++++++++++++++++++++++++++++++++++++++++++++++++++++++++++++++++
\section{Règles de calcul}
%+++++++++++++++++++++++++++++++++++++++++++++++++++++++++++++++++++++++++++++++++++++++++++++++++++++++++++++++++++++++++++

D'abord une dérivée facile, qui sera utile pour démontrer la formule de dérivation d'un quotient.
\begin{lemma}
    Nous avons :
    \begin{equation}
        \left( \frac{1}{ x } \right)'=-\frac{1}{ x^2 }.
    \end{equation}
\end{lemma}

\begin{proof}
    En posant \( f(x)=1/x\), nous avons le calcul
    \begin{equation}
        \frac{ f(x+\epsilon)-f(x) }{ \epsilon }=\frac{ \frac{1}{ x+\epsilon }-\frac{1}{ x } }{ \epsilon }=\frac{ x-(x+\epsilon) }{ \epsilon x(x+\epsilon) }=\frac{ -1 }{ x(x+\epsilon) }.
    \end{equation}
    Nous trouvons le résultat en passant à la limite et en tenant compte de la proposition \ref{PROPooOUPNooTrClHw} sur la limite d'un quotient.
\end{proof}


\begin{proposition}[\cite{ooRCDWooONrayj,ooVNAOooAuQSse,ooOGGJooCGQgDO}]     \label{PROPooOUZOooEcYKxn}
    Nous avons les règles suivantes.
    \begin{enumerate}
        \item       \label{ITEMooTFNPooYngHnD}
            Si \( f,g\colon \eR\to \eR\) sont dérivables en \( a\in \eR\), alors \( f+g\) est dérivable en \( a\) et
            \begin{equation}
                (f+g)'(a)=f'(a)+g'(a).
            \end{equation}
        \item       \label{ITEMooIPLRooOZXqMg}
            Si \( f\colon \eR\to \eR\) est dérivable en \( a\in \eR\) et si \( \lambda\in \eR\), alors \( (\lambda f)\) est dérivable en \( a\) et 
            \begin{equation}
                (\lambda f)'(a)=\lambda f'(a).
            \end{equation}
        \item   \label{ITEMooMQERooBCqnvS}
            Si \( f,g\colon \eR\to \eR\) sont dérivables en \( a\in \eR\), alors \( fg\) est dérivable en \( a\) et
    		\begin{equation}
    			(fg)'(a)=f'(a)g(a)+f(a)g'(a).
    		\end{equation}
    		Cette formule est appelée \defe{règle de Leibnitz}{Leibnitz}.
        \item   \label{ITEMooLYZCooVUPTyh}
            Soient deux intervalles \( I,J\) dans \( \eR\). Soient des fonctions \( f\colon I\to J\) et \( g\colon J\to \eR\). Soit encore \( a\in I\). SI \( f\) est dérivable en \( a\) et si \( g\) est dérivable en \( f(a)\), alors \( g\circ f\) est dérivable en \( a\) et
            \begin{equation}
                (g\circ f)'(a)= g'\big( f(a) \big)f'(a).
            \end{equation}
        \item      \label{ITEMooMUNQooLiKffz}
            Soient \( f,g\colon I\to \eR\) des fonction sur un intervalle ouvert \( I\). Soit \( a\in I\); supposons que \( g(a)\neq 0\). Alors la fonction \( \frac{ f }{ g }\) est dérivable en \( a\) et
            \begin{equation}
                \left( \frac{ f }{ g } \right)'(a)=\frac{ f'(a)g(a)-f(a)g'(a) }{ g(a)^2 }.
            \end{equation}
    \end{enumerate}
    En particulier, la dérivation est une opération linéaire sur l'espace des fonctions infiniement dérivables.
\end{proposition}

\begin{proof}
    Point par point.
    \begin{subproof}
        \item[Pour \ref{ITEMooTFNPooYngHnD}]
        \item[Pour \ref{ITEMooIPLRooOZXqMg}]
            Écrivons la définition de la dérivée avec $(\lambda f)$ au lieu de $f$, et calculons un petit peu :
            \begin{subequations}
                \begin{align}
                    (\lambda f)'(x) &=\lim_{\epsilon\to 0}\frac{ (\lambda f)(x+\epsilon)-(\lambda f)(x) }{ \epsilon }\\
                            &=\lim_{\epsilon\to 0}\frac{ \lambda \big( f(x+\epsilon) \big)-\lambda f(x) }{ \epsilon }\\
                            &=\lim_{\epsilon\to 0}\lambda \frac{ f(x+\epsilon) -f(x) }{ \epsilon }\\
                            &=\lambda \lim_{\epsilon\to 0}\frac{ f(x+\epsilon) -f(x) }{ \epsilon }\\
                            &=\lambda f'(x).
                \end{align}
            \end{subequations}
        \item[Pour \ref{ITEMooMQERooBCqnvS}, règle de Leibnitz]

            La définition de la dérivée dit que
            \begin{equation}        \label{Eqfgrimeepsfgx}
                (fg)'(x)=\lim_{\epsilon\to 0}\frac{f(x+\epsilon)g(x+\epsilon)-f(x)g(x)}{\epsilon}.
            \end{equation}
            La subtilité est d'ajouter au numérateur la quantité $-f(x)g(x+\epsilon)+f(x)g(x+\epsilon)$, ce qui est permis parce que cette quantité est nulle\footnote{Nous avons déjà faut le coup d'ajouter et enlever la même chose durant la démonstration du théorème~\ref{Tholimfgabab}. C'est une technique assez courante en analyse.}. Le numérateur de \eqref{Eqfgrimeepsfgx} devient donc
            \begin{equation}
                \begin{aligned}[]
            f(x+\epsilon)g(x+\epsilon)&-f(x)g(x+\epsilon)+f(x)g(x+\epsilon)-f(x)g(x) \\
                        &= g(x+\epsilon)\big( f(x+\epsilon)-f(x) \big)+f(x)\big( g(x+\epsilon)-g(x) \big),
                \end{aligned}
            \end{equation}
            où nous avons effectué deux mises en évidence. Étant donné que nous avons deux termes, nous pouvons couper la limite en deux :
            \begin{equation}
                \begin{aligned}[]
                    (fg)'(x)    &=\lim_{\epsilon\to 0}g(x+\epsilon)\frac{ f(x+\epsilon)-f(x) }{\epsilon}            &+\lim_{\epsilon\to 0}f(x)\frac{ g(x+\epsilon)-g(x) }{\epsilon}\\
                            &=\lim_{\epsilon\to 0}g(x+\epsilon)\lim_{\epsilon\to 0}\frac{ f(x+\epsilon)-f(x) }{\epsilon}    &+f(x)\lim_{\epsilon\to 0}\frac{ g(x+\epsilon)-g(x) }{\epsilon},
                \end{aligned}
            \end{equation}
            où nous avons utilisé le théorème~\ref{Tholimfgabab} pour scinder la première limite en deux, ainsi que la propriété \eqref{Eqbutmultlim} pour sortir le $f(x)$ de la limite dans le second terme. Maintenant, dans le premier terme, nous avons évidemment\footnote{Pas tout à fait évidemment : selon le théorème~\ref{ThoLimCont}, \emph{limite et continuité}, il faut que $g$ soit continue.} $\lim_{\epsilon\to 0}g(x+\epsilon)=g(x)$. Les limites qui restent sont les définitions classiques des dérivées de $f$ et $g$ au point~$x$ :
            \begin{equation}
                (fg)'(x)=g(x)f'(x)-f(x)g'(x),
            \end{equation}
            ce qu'il fallait démontrer.

        \item[Pour \ref{ITEMooLYZCooVUPTyh}]
            Nous posons \( b=f(a)\) et nous considérons la fonction suivante :
            \begin{equation}
                \begin{aligned}
                    u\colon J&\to \eR \\
                    y&\mapsto u(y)=\begin{cases}
                        \frac{ g(y)-g(b) }{ y-b }    &   \text{si } y\neq b\\
                        g'(b)    &    \text{si } y=b.
                    \end{cases}
                \end{aligned}
            \end{equation}
            Vu que \( g\) est dérivable en \( b\), la seconde ligne existe et \( u\) est continue en \( y=b=f(a)\). C'est la définition de la dérivée. 

            Mais \( f\) est continue en \( a\), donc \( u\circ f\) est également continue en \( a\), et nous avons
            \begin{equation}
                \lim_{x\to a} (u\circ f)(x)=u\big( f(a) \big)=u(b)=g'(b).
            \end{equation}
            En récrivant la définition de \( u\) en \( f(x)\), l'expression suivante est une fonction continue de \( x\) :
            \begin{equation}
                u\big( f(x) \big)=\begin{cases}
                    \frac{ g\big( f(x) \big)-g(b) }{ f(x)-b }    &   \text{si } f(x)\neq b\\
                    g'(b)    &    \text{si } y=b.
                \end{cases}
            \end{equation}
            Si \( f(x)\neq b\) nous avons :
            \begin{equation}        \label{EQooKHQZooJdbmlT}
                g\big( f(x) \big)-g(b)=u\big( f(x) \big)\big( f(x)-b \big).
            \end{equation}
            Si par contre \( f(x)=b\), en réalité, l'égalité \eqref{EQooKHQZooJdbmlT} est encore valable parce qu'elle se résume à \( 0=0\). Nous divisons par \( x-a\) et nous avons l'égalité
            \begin{equation}
                \frac{ g\big( f(x) \big)-f\big( f(a) \big) }{ x-a }=u\big( f(x) \big)\frac{ f(x)-f(a) }{ x-a }
            \end{equation}
            qui est valable sur \( I\setminus\{ a \}\).

            Il ne s'agit pas maintenant de prendre la limite \( x\to a\) des deux côtés, parce que la limite du membre de gauche est précisément ce que ce théorème s'efforce de prouver exister. Nous montrons que la limite du membre de gauche existe en montrant que celle de droite existe. 
            
            D'une part, \( u\circ f\) est continue et
            \begin{equation}
                \lim_{x\to a} u\big( f(x) \big)=u\big( f(a) \big)=u(b)=g'(b).
            \end{equation}
            D'autre par, \( f\) est dérivable en \( a\), donc
            \begin{equation}
                \lim_{x\to a} \frac{ f(x)-f(a) }{ x-a }=f'(a).
            \end{equation}
            Tout cela pour dire qu'à droite, la limite existe et vaut \( g'(b)f'(a)\). Donc nous avons l'existence de la limite que nous définissant \( (g\circ f)'(a)\), et la valeur
            \begin{equation}
                \lim_{x\to a} \frac{ g\big( f(x) \big)-f\big( f(a) \big) }{ x-a }= g'\big( f(a) \big)f'(a).
            \end{equation}
            Le résultat est prouvé.
        \item[Pour \ref{ITEMooMUNQooLiKffz}]
            Nous considérons la fonction
            \begin{equation}
                \begin{aligned}
                    i\colon \eR\setminus\{ 0 \}&\to \eR \\
                    x&\mapsto \frac{1}{ x }. 
                \end{aligned}
            \end{equation}
            La fonction \( g\) est dérivable en \( a\), la fonction \( i\) est dérivable en \( g(a)\). Donc par le théorème de dérivation des fonctions composées\footnote{Proposition \ref{PROPooOUZOooEcYKxn}\ref{ITEMooLYZCooVUPTyh}.}, la fonction \( i\circ f\) est dérivable en \( a\) et
            \begin{equation}
                (i\circ g)'(a)=i'\big( g(a) \big)g'(a)=-\frac{ g'(a) }{ g(a)^2 }.
            \end{equation}
            
            Pour le quotient, nous utilisons la formule de la dérivée du produit sur \( \frac{ f }{ g }(x)=f(x)\frac{1}{ g(x) } \) pour dire que \( f/g\) est dérivable en \( a\) et
            \begin{equation}
                    \left( \frac{ f }{ g } \right)'(a)=f'(a)\frac{1}{ g(a) }+f(a)\left( \frac{1}{ g } \right)'(a)
                    =\frac{ f'(a) }{ g(a) }-\frac{ f(a)g'(a) }{ g(a)^2 }
                    =\frac{ f'(a)g(a)-f(a)g'(a) }{ g(a)^2 },
            \end{equation}
            ce qu'il fallait démontrer.
    \end{subproof}
\end{proof}

\begin{remark}
    Nous ne pouvons pas dire que la dérivée est une opération linéaire sur l'espace des fonctions dérivables. Certes la proposition \ref{PROPooOUZOooEcYKxn} implique entre autres que l'ensemble des fonctions dérivables est un espace vectoriel. Mais la dérivée d'une fonction dérivable n'est pas spécialement dérivable.
\end{remark}

\begin{remark}
    La formule \( (1/u)'=-u'/u^2\) ne peut pas être vue comme un cas particulier de \( (u^{\alpha})'=\alpha u^{\alpha-1}\) (proposition \ref{PROPooSGLGooIgzque}) parce que cette formule est utilisée dans la démonstration de la formule générale.
\end{remark}


Pour les fonctions à valeurs dant \( \eR^n\), nous posons la définition suivante.
\begin{definition}
    Soit une fonction \( f\colon \eR\to \eR^n\) dont les composantes \( f_i\colon \eR\to \eR\) sont dérivables. Nous définissons la fonction \( f'\) par
    \begin{equation}
        f'(x)=\sum_if'_i(x)e_i,
    \end{equation}
    c'est-à-dire une dérivation composante par composante.
\end{definition}

Cette définition est celle pour une fonction \( \eR\to \eR^n\), et elles est facile. Très différente est la situation d'une fonction \( \eR^n\to \eR\) dans laquelle il faudra introduire la notion de différentielle\footnote{Ce sera pour la définition \ref{DefDifferentiellePta}.}.

Par rapport à la dérivation, les produits scalaire et vectoriel vérifient une règle de Leibnitz. 
\begin{proposition}     \label{PROPooFKKHooQZGXhE}
    Soit $I$ un intervalle de $\eR$. Si $u$ et $u$ sont dans $C^1(I,\eR^3)$, alors
    \begin{equation}		\label{EqFormLeibProdscalVect}
        \begin{aligned}[]
            \frac{ d }{ dt }\big( u(t)\cdot v(t) \big)&=\big( u'(t)\cdot v(t) \big)+\big( u(t)\cdot v'(t) \big)\\
            \frac{ d }{ dt }\big( u(t)\times v(t) \big)&=\big( u'(t)\times v(t) \big)+\big( u(t)\times v'(t) \big).
        \end{aligned}
    \end{equation}
\end{proposition}

Nous faisons la preuve pour le produit scalaire; sans doute que le produit vectoriel sera la même chose.
\begin{proof}
    Nous considérons des fonctions dérivables \( f,g\colon \eR\to \eR^3\), et nous posons \( \varphi(t)=f(t)\cdot g(t)\). En ce qui concerne la dérivée de la fonction \( f\cdot g\colon \eR\to \eR\), nous devons étudier la limite
    \begin{equation}        \label{EQooGRFKooNHceiW}
        \lim_{\epsilon\to 0}\frac{ \varphi(t+\epsilon)-\varphi(t) }{ \epsilon }=\lim_{\epsilon\to 0}\frac{ f(t+\epsilon)\cdot g(t+\epsilon)-f(t)\cdot g(t) }{ \epsilon }.
    \end{equation}
    La fonction \( f\) étant dérivable, la proposition \ref{PropUTenzfQ} nous donne une fonction \( \alpha\colon \eR\to \eR^3\) telle que
    \begin{equation}
        f(t+\epsilon)=f(t)+\epsilon f'(t)+\epsilon\alpha(\epsilon)
    \end{equation}
    et \( \lim_{\epsilon\to 0}\alpha(\epsilon)=0\). En substituant cela dans le numérateur de \eqref{EQooGRFKooNHceiW} nous calculons un peu :
    \begin{subequations}
        \begin{align}
            f(t+\epsilon)\cdot g(t+\epsilon)-f(t)\cdot g(t)&=\big( f(t)+\epsilon f'(t)+\epsilon \alpha(\epsilon) \big)\cdot\big( g(t)+\epsilon g'(t)+\epsilon\beta(\epsilon) \big)\\
            &\quad - f(t)\cdot g(t)\\
            &=\epsilon f(t)\cdot g'(t)+\epsilon^2 f'(t)\cdot \beta(\epsilon)\\
            &\quad+\epsilon\alpha(\epsilon)\cdot g(t)+\alpha(\epsilon)\epsilon^2\cdot g'(t)+\epsilon^2\alpha(\epsilon)\cdot \beta(\epsilon).
        \end{align}
    \end{subequations}
    En divisant cela par \( \epsilon\) et en prenant la limite \( \epsilon\to 0\), et nous reste
    \begin{equation}
        f(t)\cdot g'(t)+f'(t)\cdot g(t).
    \end{equation}
\end{proof}

%---------------------------------------------------------------------------------------------------------------------------
\subsection{Dérivée de la réciproque}
%---------------------------------------------------------------------------------------------------------------------------

\begin{proposition}[\cite{XGIooNMtKqx}] \label{PropMRBooXnnDLq}
    Soit \( f\colon I\to J=f(I)\) une fonction bijective, continue et dérivable\footnote{Définition~\ref{DEFooOYFZooFWmcAB}.}. Soient \( x_0\in I\) et \( y_0=f(x_0)\). Si \( f'(x_0)\neq 0\) alors la fonction réciproque \( f^{-1}\) est dérivable en \( y_0\) et sa dérivée est donnée par
    \begin{equation}
        (f^{-1})'(y_0)=\frac{1}{ f'(x_0) }.
    \end{equation}
\end{proposition}

\begin{proof}
    Pour rappel, une fonction dérivable est toujours continue (proposition~\ref{PropSFyxOWF}).

    Prouvons que \( f^{-1}\) est dérivable au point \( b=f(a)\in J\). Étant donné que \( f\) est dérivable en \( a\), nous avons
    \begin{equation}\label{EqJEWooSjQrfk}
        f'(a)=\lim_{x\to a} \frac{ f(x)-f(a) }{ x-a }.
    \end{equation}
    Par ailleurs, étant donnée la continuité de \( f^{-1}\) donnée par la proposition~\ref{ThoKBRooQKXThd}\ref{ItemEJZooKuFoeFiv}, nous avons
    \begin{equation}
        \lim_{\epsilon\to 0} f^{-1}(b+\epsilon)=f^{-1}(b)=a.
    \end{equation}
    Nous pouvons donc remplacer dans \eqref{EqJEWooSjQrfk} tous les \( x\) par \( f^{-1}(b+\epsilon)\) et prendre la limite \( \epsilon\to 0\) au lieu de \( x\to a\) :
    \begin{equation}
        \begin{aligned}[]
            f'(a)&=\lim_{\epsilon\to 0}\frac{ f\big( f^{-1}(b+\epsilon) \big)-f(a) }{ f^{-1}(b+\epsilon)-a }\\
            &=\lim_{\epsilon\to 0}\frac{ b+\epsilon-f(a) }{ f^{-1}(b+\epsilon)-f^{-1}(b) }\\
            &=\lim_{\epsilon\to 0}\frac{ \epsilon }{ f^{-1}(b+\epsilon)-f^{-1}(b) }\\
            &=\frac{1}{ \lim_{\epsilon\to 0}\frac{ f^{-1}(b+\epsilon)-f^{-1}(b) }{ \epsilon } }\\
            &=\frac{1}{ (f^{-1})'(b) }.
        \end{aligned}
    \end{equation}
    Nous avons utilisé le fait que \( f(a)=b\) et \( a=f^{-1}(b)\).
\end{proof}

\begin{proposition}[\cite{BIBooWTHJooTyNuub}]      \label{PROPooSGTBooFxUuXK}
    Soit \(f \) une fonction dérivable et strictement monotone de l'intervalle \( I\) sur l'intervalle \( J\)  (f est alors une bijection de $I$ vers $J$). Si \( f'\)  ne s'annule par sur \( I\) alors
    \begin{enumerate}
        \item
            la fonction \( f\) est une bijection de \( I\) vers \( J\),
        \item
            la fonction \( f^{-1}\) est dérivable sur \( J\),
        \item
            et nous avons la formule
            \begin{equation}        \label{EQooELIHooDxUFxH}
                (f^{-1})'=\frac{1}{ f'\circ f^{-1} }.
            \end{equation}
    \end{enumerate}
\end{proposition}
\index{réciproque!dérivabilité}

\begin{normaltext}
 Très souvent on préfère retenir la formule
    \begin{equation}\label{EqWWAooBRFNsv}
      (f^{-1})'(y_0) = \frac{1}{f'\left((f^{-1})(y_0)\right)}
    \end{equation}

    Elle est très simple à retrouver : il suffit d'écrire
    \begin{equation}
        f^{-1}\big( f(x) \big)=x
    \end{equation}
    puis de dériver les deux côtés par rapport à \( x\) en utilisant la règle de dérivation des fonctions composées :
    \begin{equation}
        (f^{-1})'\big( f(x) \big)f'(x)=1.
    \end{equation}
\end{normaltext}

\begin{example}[difféomorphisme entre \( \eR\) et un ouvert borné]      \label{EXooGKPNooZtmJen}
    Nous cherchons à construire une application dérivable et d'inverse dérivable entre \( \eR\) (en entier) et un ouvert borné de \( \eR\). Il serait tentant de prendre l'application arc tangente
    \begin{equation}
        \begin{aligned}
        \arctan\colon \eR&\to \left] -\frac{ \pi }{2} , \frac{ \pi }{2} \right[ \\
            x&\mapsto \arctan(x)
        \end{aligned},
    \end{equation}
    mais elle ne sera définie que dans le théorème~\ref{THOooUSVGooOAnCvC}.

    Nous posons
    \begin{equation}
        f(x)=\begin{cases}
            2+\frac{1}{ x-2 }    &   \text{si } x\leq 1\\
            \frac{1}{ x }    &    \text{si } x>1.
        \end{cases}
    \end{equation}
    Cela est continue en \( x=1\) : il suffit de calculer les deux valeurs. En ce qui concerne la dérivabilité en \( x=1\), nous devons faire
    \begin{equation}
        \lim_{\epsilon\to 0}\frac{ f(1+\epsilon)-f(1) }{ \epsilon }.
    \end{equation}
    La limite à gauche est égale à la dérivée de \( x\mapsto 2+\frac{ 1 }{ x-2 }\) en \( x=1\) et la limite à droite est égale à la dérivée de \( x\mapsto 1/x\) en \( x=1\). Dans les deux cas nous trouvons \( -1\).

    \begin{center}
        \input{auto/pictures_tex/Fig_LMHMooCscXNNdU.pstricks}
    \end{center}

    Nous voyons vite que cette fonction est strictement décroissante; et un calcul de limite nous dit qu'il s'agit d'une bijection dérivable
    \begin{equation}
        f\colon \eR\to \mathopen] 0 , 2 \mathclose[.
    \end{equation}
    La proposition~\ref{PropMRBooXnnDLq} s'applique et la bijection réciproque est également dérivable (donc continue aussi).
\end{example}

\begin{probleme}
Si vous connaissez un autre exemple, plus simple, de difféomorphisme \( f\colon \eR\to \mathopen] a , b \mathclose[\), faites-le moi savoir. Ne pas utiliser d'exponentielle (vous pensiez à bricoler quelque chose à partir de la primitive de \( x\mapsto  e^{-x^2}\) ?) ni de fonctions trigonométriques.
\end{probleme}

\begin{example}
    Nous aimerions donner le logarithme comme exemple, mais l'exponentielle ne sera définie que dans longtemps à partir des séries entières. Allez voir l'exemple~\ref{ExZLMooMzYqfK} pour le logarithme comme inverse de l'exponentielle.
\end{example}


% This is part of Mes notes de mathématique
% Copyright (c) 2006-2019
%   Laurent Claessens, Carlotta Donadello
% See the file fdl-1.3.txt for copying conditions.

%+++++++++++++++++++++++++++++++++++++++++++++++++++++++++++++++++++++++++++++++++++++++++++++++++++++++++++++++++++++++++++
\section{Dérivation et croissance}
%+++++++++++++++++++++++++++++++++++++++++++++++++++++++++++++++++++++++++++++++++++++++++++++++++++++++++++++++++++++++++++

Supposons une fonction dont la dérivée est positive. Étant donné que la courbe est « collée » à ses tangentes, tant que les tangentes montent, la fonction monte. Or, une tangente qui monte correspond à une dérivée positive, parce que la dérivée est le coefficient directeur de la tangente.

Ce résultat très intuitif peut être prouvé rigoureusement. C'est la tache à laquelle nous allons nous atteler maintenant.

\begin{proposition} \label{PropGFkZMwD}
    Si $f$ et $f'$ sont des fonctions continues sur l'intervalle $[a,b]$ et si $f'$ est strictement positive sur $[a,b]$, alors $f$ est croissante sur $[a,b]$.

    De la même manière, si $f'$ est strictement négative sur $[a,b]$, alors $f$ est décroissante sur $[a,b]$.
\end{proposition}

\begin{proof}
    Nous n'allons prouver que la première partie. La seconde partie se prouve en considérant $-f$ et en invoquant alors la première\footnote{Méditer cela.}. Prenons $x_1$ et $x_2$ dans $[a,b]$ tels que $x_1<x_2$. Par hypothèse, pour tout $x$ dans $[x_1,x_2]$, nous avons
    \begin{equation}
        f'(x)=\lim_{\epsilon\to 0}\frac{ f(x+\epsilon)-f(x) }{\epsilon} >0.
    \end{equation}
    Maintenant, la proposition~\ref{PropoLimPosFPos} dit que quand une limite est positive, alors la fonction dans la limite est positive sur un voisinage. En appliquant cette proposition à la fonction
    \begin{equation}
        r(\epsilon)=\frac{ f(x+\epsilon)-f(x) }{ \epsilon },
    \end{equation}
    dont la limite en zéro est positive, nous trouvons que $r(\epsilon)>0$ pour tout $\epsilon$ pas trop éloigné de zéro. En particulier, il existe un $\delta>0$ tel que $\epsilon<\delta$ implique $r(\epsilon)>0$; pour un tel $\epsilon$, nous avons donc
    \begin{equation}
        r(\epsilon)=\frac{ f(x+\epsilon)-f(x) }{ \epsilon }>0.
    \end{equation}
    Étant donné que $\epsilon>0$, nous avons que $f(x+\epsilon)-f(x)>0$, c'est-à-dire que $f$ est strictement croissante entre $x$ et $x+\epsilon$.

    Jusqu'ici, nous avons prouvé que la fonction $f$ était strictement croissante dans un voisinage autour de chaque point de $[a,b]$. Cela n'est cependant pas encore tout à fait suffisant pour conclure. Ce que nous voudrions faire, c'est de dire, c'est prendre un voisinage $]a,m_1[$ autour de $a$ sur lequel $f$ est croissante. Donc, $f(m_1)>f(a)$. Ensuite, on prend un voisinage $]m_1,m_2[$ de $m_1$ sur lequel $f$ est croissante. De ce fait, $f(m_2)>f(m_1)>f(a)$. Et ainsi de suite, nous voulons construire des $m_3$, $m_4$,\ldots jusqu'à arriver en $b$. Hélas, rien ne dit que ce processus va fonctionner. Il faut trouver une subtilité. Le problème est que les voisinages sur lesquels la fonction est croissante sont peut-être de plus en plus petits, de telle sorte à ce qu'il faille une infinité d'étapes avant d'arriver à bon port (en $b$).

    Heureusement, nous pouvons drastiquement réduire le nombre d'étapes en nous souvenant du théorème de Borel-Lebesgue~\ref{ThoBOrelLebesgue}. Nous notons par $\mO_x$, un ouvert autour de $x$ tel que $f$ soit strictement croissante sur $\mO_x$. Un tel voisinage existe. Cela fait une infinité d'ouverts tels que
    \begin{equation}
        [a,b]\subseteq\bigcup_{x\in[a,b]}\mO_x.
    \end{equation}
    Ce que le théorème dit, c'est qu'on peut en choisir un nombre fini qui recouvre encore $[a,b]$. Soient $\{ \mO_{x_1},\ldots,\mO_{x_n} \}$, les heureux élus, que nous supposons pris dans l'ordre : $x_1<x_2<\ldots<x_n$. Nous avons
    \begin{equation}
        [a,b]\subseteq\bigcup_{i=1}^n\mO_i.
    \end{equation}
    Quitte à les rajouter à la collection, nous supposons que $x_1=a$ et que $x_n=b$. Maintenant nous allons choisir encore un sous-ensemble de cette collection d'ouverts. On pose $\mA_1=\mO_{x_1}$. Nous savons que $\mA_1$ intersecte au moins un des autres $\mO_{x_i}$. Cette affirmation vient du fait que $[a,b]$ est connexe (proposition~\ref{PropInterssiConn}), et que si $\mO_{x_1}$ n'intersectait personne, alors
    \begin{equation}
        \begin{aligned}[]
            \mO_{x_1}&&\text{et}&&\bigcup_{i=2}^n\mO_{x_i}
        \end{aligned}
    \end{equation}
    forment une partition de $[a,b]$ en deux ouverts disjoints, ce qui n'est pas possible parce que $[a,b]$ est connexe. Nous nommons $\mA_2$, un des ouverts $\mO_{x_i}$ qui intersecte $\mA_1$. Disons que c'est $\mO_k$. Notons que $\mA_1\cup\mA_2$ est un intervalle sur lequel $f$ est strictement croissante. En effet, si $y_{12}$ est dans l'intersection, $f(a)<f(y_{12})$ parce que $f$ est strictement croissante sur $\mA_1$, et pour tout $x>y_{12}$ dans $\mA_2$, $f(x)>f(y_{12})$ parce que $f$ est strictement croissante dans $\mA_2$.

    Maintenant, nous éliminons de la liste des $\mO_{x_i}$ tous ceux qui sont inclus dans $\mA_1\cup\mA_2$. Dans ce qu'il reste, il y en a automatiquement un qui intersecte $\mA_1\cup\mA_2$, pour la même raison de connexité que celle invoquée plus haut. Nous appelons cet ouvert $\mA_3$, et pour la même raison qu'avant, $f$ est strictement croissante sur $\mA_1\cup\mA_2\cup\mA_3$.

    En recommençant suffisamment de fois, nous finissons par devoir prendre un des $\mO_{x_i}$ qui contient $b$, parce qu'au moins un des $\mO_{x_i}$ contient $b$. À ce moment, nous avons fini la démonstration.
\end{proof}

Il est intéressant de noter que ce théorème concerne la croissance d'une fonction sous l'hypothèse que la dérivée est positive. Il nous a fallu très peu de temps, en utilisant la positivité de la dérivée, pour conclure qu'autour de tout point, la fonction était strictement croissante. À partir de là, c'était pour ainsi dire gagné. Mais il a fallu un réel travail de topologie très fine\footnote{et je te rappelle que nous avons utilisé la proposition~\ref{PropInterssiConn}, qui elle même était déjà un très gros boulot !} pour conclure. Étonnant qu'une telle quantité de topologie soit nécessaire pour démontrer un résultat essentiellement analytique dont l'hypothèse est qu'une limite est positive, n'est-ce pas ?

Une petite facile, maintenant.
\begin{proposition}
    Si $f$ est croissante sur un intervalle, alors $f'\geq 0$ à l'intérieur de cet intervalle, et si $f$ est décroissante sur l'intervalle, alors $f'\leq 0$ à l'intérieur de l'intervalle.
\end{proposition}

Note qu'ici, nous demandons juste la croissance de $f$, et non sa \emph{stricte} croissance.

\begin{proof}
    Soit $f$, une fonction croissante sur l'intervalle $I$, et $x$ un point intérieur de $I$. La dérivée de $f$ en $x$ vaut
    \begin{equation}
        f'(x)=\lim_{\epsilon\to 0}\frac{ f(x+\epsilon)-f(x) }{\epsilon},
    \end{equation}
    mais, comme $f$ est croissante sur $I$, nous avons toujours que $f(x+\epsilon)-f(x)\geq0$ quand $\epsilon>0$, et $f(x+\epsilon)-f(x)\leq0$ quand $\epsilon<0$, donc cette limite est une limite de nombres positifs ou nuls, qui est donc positive ou nulle. Cela prouve que $f'(x)\geq 0$.
\end{proof}

%---------------------------------------------------------------------------------------------------------------------------
\subsection{Théorèmes de Rolle et des accroissements finis}
%---------------------------------------------------------------------------------------------------------------------------

\begin{theorem}[Théorème de Rolle\cite{ooNRTLooCpjVdc,ooFQESooWuxtpx}]       \label{ThoRolle}
    Soit $f$, une fonction continue sur $[a,b]$ et dérivable sur $]a,b[$. Si $f(a)=f(b)$, alors il existe un point $c\in]a,b[$ tel que $f'(c)=0$.
\end{theorem}
\index{théorème!Rolle}

\begin{proof}
    Étant donné que $[a,b]$ est un intervalle compact, l'image de $[a,b]$ par $f$ est un intervalle compact, soit $[m,M]$ (théorème~\ref{ThoImCompCotComp}). Si $m=M$, alors le théorème est évident : c'est que la fonction est constante, et la dérivée est par conséquent nulle. Supposons que $M> f(a)$ (il se peut que $M=f(a)$, mais alors si $f$ n'est pas constante, il faut avoir $m<f(a)$ et le reste de la preuve peut être adaptée).

    Comme $M$ est dans l'image de $[a,b]$ par $f$, il existe $c\in ]a,b[$ tel que $f(c)=M$. Considérons maintenant la fonction
    \begin{equation}
        \tau(x) =\frac{ f(c+x)-f(c) }{ x }.
    \end{equation}
    Par définition, $\lim_{x\to 0}\tau(x)=f'(c)$. Par hypothèse, si $u<c$,
    \begin{equation}
        \tau(u-c) = \frac{ f(u)-f(c) }{ u-c }>0
    \end{equation}
    parce que $u-c<0$ et $f(u)-f(c)<0$. Par conséquent, $\lim_{x\to 0}\tau(x)\geq 0$. Nous avons aussi, pour $v>c$,
    \begin{equation}
        \tau(v-c) = \frac{ f(v)-f(c) }{ v-c }<0
    \end{equation}
    parce que $v-c>0$ et $f(v)-f(c)<0$. Par conséquent, $\lim_{x\to 0}\tau(x)\leq 0$. Mettant les deux ensemble, nous avons $f'(c)=\lim_{x\to 0}\tau(x)=0$, et $c$ est le point que nous cherchions.
\end{proof}

Voici une généralisation du théorème de Rolle, dans le cas où nous n'aurions pas deux points sur lesquels la fonction est identique, mais deux points en lesquels la limite de la fonction est identique. Typiquement, lorsque les points en question sont \( \pm\infty\).
\begin{theorem}[Généralisation de Rolle\cite{ooNRTLooCpjVdc}]           \label{THOooXDTBooFeSZoK}
    Soient \( -\infty\leq a<b\leq +\infty\). Soit une fonction dérivable \( f\colon \mathopen] a , b \mathclose[\to \eR\) telle que
    \begin{equation}
        \lim_{x\to a} f(x)=\lim_{x\to b} f(x)=\ell
    \end{equation}
    avec \( \ell\in \bar \eR\). Alors il existe \( x\in \mathopen] a , b \mathclose[\) tel que \( f'(x)=0\).
\end{theorem}

\begin{proof}
    Soit un difféomorphisme\footnote{Définition~\ref{DefAQIQooYqZdya}.} strictement croissant \( \varphi\colon \eR\to \mathopen] \alpha , \beta \mathclose[\). Pour cela vous pouvez bricoler à partir de l'exemple~\ref{EXooGKPNooZtmJen}.
        Mais n'utilisez pas la fonction arc tangente, parce qu'elle n'est définie qu'au théorème~\ref{THOooUSVGooOAnCvC}.

    Nous posons \( a'=\varphi(a)\), \( b'=\varphi(b)\) et
    \begin{equation}
    g= \varphi\circ f\circ \varphi^{-1}\colon \mathopen] a' , b' \mathclose[\to \mathopen] \alpha , \beta \mathclose[.
    \end{equation}
    Cela est une fonction dérivable et continue sur \( \mathopen[ a' , b' \mathclose]\) en posant \( g(a')=g(b')=\varphi(\ell)\).

    Donc il existe \( c'\in\mathopen] a' , b' \mathclose[\) tel que \( g'(c')=0\). En posant \( c=\varphi^{-1}(c')\) nous avons \( c\in \mathopen] a , b \mathclose[\) et, en utilisant de nombreuses fois la règle de dérivation des fonctions composées~\ref{PROPooOUZOooEcYKxn}\ref{ITEMooLYZCooVUPTyh},
    \begin{subequations}
        \begin{align}
            f'(c)&=f'\big( \varphi^{-1}(c') \big)\\
            &=(\varphi^{-1})'\Big( (g\circ \varphi)\big( \varphi^{-1}(c') \big) \Big)(g\circ\varphi)'\big( \varphi^{-1}(c') \big)\\
            &=(\varphi^{-1})'\big( g(c') \big)\underbrace{g'(c')}_{=0}\varphi'\big( \varphi^{-1}(c') \big)\\
            &=0.
        \end{align}
    \end{subequations}
\end{proof}

Une autre généralisation de Rolle, avec des dérivées d'ordre supérieur.
\begin{proposition}[\cite{ooWCFFooRMBEJl}]      \label{PROPooCPCAooJjOZNy}
Soit un intervalle ouvert \( I\subset \eR\) contenant \( a,b\) (\( a\neq b\)). Soit une fonction \( f\in C^{k+1}(I,\eR)\). Si \( f(a)=f(b)\) et si \( f^{(j)}(a)=0\) pour \( j=1,\ldots, n\), alors il existe \( x\in \mathopen] a , b \mathclose[\) tel que \( f^{(n+1)}(c)=0\).
\end{proposition}

\begin{proof}
    Le théorème de Rolle \ref{ThoRolle} nous dit qu'il existe \( c_1\in \mathopen] a , b \mathclose[\) tel que \( f'(c_1)=0\). Mais alors \( f'(a)=f'(x_1)=0\), et le théorème de Rolls appliqué à \( f'\) donne \( c_2\in \mathopen] a , c_1 \mathclose[\) tel que \( f''(c_2)=0\). Continuant ainsi \( n\) fois, il existe \( c\in \mathopen] a ,b\mathclose[\) tel que \( f^{(n+1)}(c)=0\).
\end{proof}

Le théorème suivant est le théorème des \defe{accroissements finis}{théorème!accroissements finis!dans $\eR$}.
\begin{theorem}[Accroissements finis]       \label{ThoAccFinis}
    Soit $f$, une fonction continue sur $[a,b]$ et dérivable sur $]a,b[$.
        \begin{enumerate}
            \item       \label{ITEMooFZONooXJqLyX}
               Il existe au moins un réel $c\in]a,b[$ tel que
                   \begin{equation}
                       f'(c)=\frac{ f(b)-f(a) }{ b-a }
                   \end{equation}
                   Autrement dit, la tangente en \( c\) est parallèle à la corde entre \( a\) et \( b\).
               \item       \label{ITEMooXRQKooDBFpdQ}
               Nous avons la majoration
               \begin{equation}
                   \big| \frac{ f(b)-f(a) }{ b-a } \big| \leq \sup_{x\in\mathopen[ a , b \mathclose]}| f'(x) |  | b-a |.
               \end{equation}
        \end{enumerate}
\end{theorem}

\begin{proof}
    Considérons la fonction
    \begin{equation}
        \tau(x)=f(x)-\big( \frac{ f(b)-f(a) }{ b-a }x + f(a) - a\frac{ f(b)-f(a) }{ b-a } \big),
    \end{equation}
    c'est-à-dire la fonction qui donne la distance entre $f$ et le segment de droite qui lie $(a,f(a))$ à $(b,f(b))$. Par construction, $\tau(a)-\tau(b)=0$, donc le théorème de Rolle s'applique à $\tau$ pour laquelle il existe donc un $c\in]a,b[$ tel que $\tau'(c)=0$.

    En utilisant les règles de dérivation, nous trouvons que la dérivée de $\tau$ vaut
    \begin{equation}
        \tau'(x)= f'(x)-\frac{ f(b)-f(a) }{ b-a },
    \end{equation}
    donc dire que $\tau'(c)=0$ revient à dire que $f(b)-f(a)=(b-a)f'(c)$, ce qu'il fallait démontrer.

    La majoration est une conséquence immédiate, parce que le supremum de \( | f'(x) |\) est forcément plus grand que \( | f'(c) |\).
\end{proof}

Une généralisation pour une fonction sur un intervalle \( \mathopen] a , b \mathclose[\) où \( a\) et \( b\) peuvent être infinis.
\begin{theorem}[Généralisation des accroissements finis] \label{THOooRIIBooOjkzMa}
    Soient \( -\infty\leq a<b\leq +\infty\) et \( f,g\) des fonctions continues sur \( \mathopen[ a , b \mathclose]\) et dérivables sur \( \mathopen] a , b \mathclose[\).

        Si \( a=-\infty\) :
        \begin{itemize}
            \item Nous demandons la continuité sur \( \mathopen] -\infty , b \mathclose]\) et la dérivabilité sur \( \mathopen] -\infty , b \mathclose[\).
            \item
                Nous notons \( f(a)\) la limite \( \lim_{x\to -\infty} f(x)\), et nous supposons qu'elle est finie.
        \end{itemize}

        Mêmes conventions si \( b=+\infty\).

    Alors il existe \( c\in \mathopen] a , b \mathclose[\) tel que
        \begin{equation}
            \big( f(b)-f(a) \big)g'(c)=\big( g(b)-g(a) \big)f'(c).
        \end{equation}

\end{theorem}


\begin{proof}
    Nous posons
    \begin{equation}
        h(t)=\big( g(b)-g(a) \big)f(t)-\big( f(b)-f(a) \big)g(t).
    \end{equation}
    Nous avons \( \lim_{t\to a} h(t)=\lim_{t\to b} h(t)\), de telle sorte que le théorème de Rolle généralisé~\ref{THOooXDTBooFeSZoK} s'applique et il existe \( c\in \mathopen] a , b \mathclose[\) tel que \( h'(c)=0\). Pour ce \( c\) nous avons
    \begin{equation}
        0=h'(c)=\big( f(b)-f(a) \big)g'(c)-\big( g(b)-g(a) \big)f'(c),
    \end{equation}
   et donc
    \begin{equation}
        \big( f(b)-f(a) \big)g'(c)=\big( g(b)-g(a) \big)f'(c).
    \end{equation}
\end{proof}

%---------------------------------------------------------------------------------------------------------------------------
\subsection{Règle de l'Hospital}
%---------------------------------------------------------------------------------------------------------------------------

\begin{proposition}[Règle de l'Hospital pour \( \frac{ 0 }{ 0 }\)\cite{ooHQARooDfptJC}]     \label{PROPooBZHTooHmyGsy}
Soient des fonctions \( f,g\) dérivables sur \( \mathopen] a , b \mathclose[\) et dont la limite en \( a\) est nulle. Si \( g'\) ne s'annule pas sur \( \mathopen] a , b \mathclose[\) et si
    \begin{equation}
        \lim_{x\to a^+} \frac{ f'(x) }{ g'(x) }=\ell
    \end{equation}
    alors
    \begin{equation}        \label{EQooJHWYooLGdbPH}
        \lim_{x\to a^+} \frac{ f(x) }{ g(x) }=\ell.
    \end{equation}
    Ici \( \ell\in \bar \eR\), et les hypothèses garantissent l'existence de la limite \eqref{EQooJHWYooLGdbPH}.
\end{proposition}

\begin{proof}
Soit \( x\in\mathopen] a , b \mathclose[\). Les fonctions \( f\) et \( g\) sont dérivables sur \( \mathopen] a , x \mathclose[\) et continues sur \( \mathopen[ a , x \mathclose]\), de telle sorte que le théorème~\ref{THOooRIIBooOjkzMa} s'applique et nous avons \( c_x\in \mathopen] a , x \mathclose[\) tel que
    \begin{equation}        \label{EQooMALUooNagavh}
        \big( f(x)-f(a) \big)g'(c_x)=\big( g(x)-g(a) \big)f'(c_x).
    \end{equation}
    Nous nous souvenons de ce que signifient les notations dans le théorème : les notations \( f(a)\), \( f(x)\), \( g(a)\) et \( g(x)\) désignent en réalité les limites. Donc dans \eqref{EQooMALUooNagavh}, nous avons \( f(a)=g(a)=0\).

D'autre part nous avons \( g(x)\neq g(a)\), sinon le théorème de Rolle~\ref{THOooXDTBooFeSZoK} annulerait \( g'\) quelque part dans \( \mathopen] a , x \mathclose[\). Nous pouvons donc récrire \eqref{EQooMALUooNagavh} sous la forme
    \begin{equation}        \label{EQooUCLVooFgAfwC}
        \frac{ f(x) }{ g(x) }=\frac{ f'(c_x) }{ g'(c_x) }.
    \end{equation}
Mais \( \lim_{x\to a^+} c_x=a\) parce que \( c_x\in\mathopen] a , x \mathclose[\). Donc la limite du membre de droite de \eqref{EQooUCLVooFgAfwC} lorsque \( x\to a^+\) existe et vaut \( \ell\). La même limite à gauche doit alors exister et valoir la même valeur.
\end{proof}

\begin{proposition}[L'Hospital pur \( \frac{ \infty }{ \infty }\)]      \label{PROPooTJVCooMeUhIy}
    Soit \( f\) et \( g\) deux fonctions
    \begin{enumerate}
        \item
            dérivables sur \( \mathopen] a , b \mathclose[\),
        \item
                dont les limites en \( a\) sont toutes deux \( \infty\),
            \item
            \( g'\neq 0\) sur \( \mathopen] a , b \mathclose[\).
        \item
            \begin{equation}        \label{EQooVFYCooMjOGtI}
                \lim_{x\to a^+} \frac{ f'(x) }{ g'(x) }=\ell\in \bar \eR.
            \end{equation}
    \end{enumerate}
    Alors
    \begin{equation}
        \lim_{x\to a^+} \frac{ f(x) }{ g(x) }=\ell.
    \end{equation}
    Cette dernière égalité signifie «la limite existe et vaut \( \ell\)».
\end{proposition}

\begin{proof}
Soit un intervalle \( \mathopen] x , y \mathclose[\) strictement inclus dans \( \mathopen] a , b \mathclose[\) avec \( x,y\in\eR\). Par le théorème des accroissements finis généralisés~\ref{THOooRIIBooOjkzMa} il existe \( c\in \mathopen] x , y \mathclose[\) tel que
    \begin{equation}
        \frac{ f(x)-f(y) }{ g(x)-g(y) }=\frac{ f'(c) }{ g'(c) }.
    \end{equation}
    Notons que le dénominateur à gauche n'est pas nul à cause de Rolle et de l'hypothèse que \( g'\) ne s'annule pas sur \( \mathopen[ x , y \mathclose]\). Nous isolons \( f(x)\) :
    \begin{equation}        \label{EQooDFXNooJhdUca}
        f(x)=\frac{ f'(c) }{ g'(c) }\Big( g(x)-g(y) \Big)+f(y).
    \end{equation}
Avant de diviser par \( g(x)\) nous devons prendre quelques précautions. Soit \( V\), un voisinage de \( \ell\)\footnote{Vous savez ce que signifie un «voisinage de \( \infty\)» ? Allez voir la sous-section~\ref{SUBSECooKRRUooSlZSmM}.}. Vu la limite \eqref{EQooVFYCooMjOGtI}, il existe \( y\in \mathopen] a , b \mathclose[\) tel que
    \begin{equation}
        \frac{ f'(t) }{ g'(t) }\in V
    \end{equation}
pour tout \( t\in \mathopen] a , y \mathclose[\). Nous utilisons ici avec subtilité le fait que ces intervalles sont une base de la topologie autour de \( \infty\). Maintenant \( f(y)\) et \( g(y)\) sont fixés et sont des nombres réels. Vu que \( \lim_{x\to a} g(x)=0\) nous pouvons choisir \( r<y\) tel que nous ayons simultanément
    \begin{enumerate}
        \item
        \( g(x)\neq 0\) sur \( \mathopen] a , r \mathclose[\),
        \item
            \begin{equation}
                | \frac{ g(y) }{ g(x) } |\leq \epsilon
            \end{equation}
            et
            \begin{equation}
                | \frac{ f(y) }{ g(x) } |\leq \epsilon
            \end{equation}
        pour tout \( x\in \mathopen] a , r \mathclose[\).
    \end{enumerate}
Nous sommes maintenant armés de \( y\) et \( r\) satisfaisant tout cela et nous pouvons traiter avec la formule \eqref{EQooDFXNooJhdUca} en ne la considérant que pour \( x\in \mathopen] a , r \mathclose[\). Soit \( x\in \mathopen] a , r \mathclose[\); il existe \( c_x\in \mathopen] a , x \mathclose[\) tel que
    \begin{equation}        \label{EQooNEZQooYGJmFW}
        \frac{ f(x) }{ g(x) }=\frac{ f'(c_x) }{ g'(c_x) }\left( 1-\frac{ g(y) }{ g(x) } \right)+\frac{ f(y) }{ g(x) }.
    \end{equation}
    Nous avons :
    \begin{enumerate}
        \item
            \( \lim_{x\to a^+} c_x=a\),
        \item
            \begin{equation}
                \lim_{x\to a^+}\frac{ f'(c_x) }{ g'(c_x) }=\lim_{x\to a^+} \frac{ f'(x) }{ g'(x) }=\ell,
            \end{equation}
        \item
            \begin{equation}
                \lim_{x\to a^+} \left( 1-\frac{ g(y) }{ g(x) } \right)=1,
            \end{equation}
        \item
            \( \lim_{x\to a^+} \frac{ f(y) }{ g(x) }=0\).
    \end{enumerate}
    Donc chaque partie du membre de droite de \eqref{EQooNEZQooYGJmFW} a une limite bien déterminée pour \( x\to a^+\). Les règles de calcul s'appliquent et nous avons
    \begin{equation}
        \lim_{x\to a^+} \frac{ f(x) }{ g(x) }=\ell\times 1+0=\ell.
    \end{equation}
\end{proof}

%---------------------------------------------------------------------------------------------------------------------------
\subsection{Dérivée et primitive}
%---------------------------------------------------------------------------------------------------------------------------

\begin{corollary}       \label{CORooEOERooYprteX}
Soit $f$ une fonction dérivable sur $[a,b]$ telle que $f'(x) = 0$ pour tout $x \in [a,b]$. Alors $f$ est constante sur $[a,b]$.
\end{corollary}

\begin{proof}
    Si $f$ n'était pas constante sur $[a,b]$, il existerait un $x_1\in ]a,b[$ tel que $f(a)\neq f(x_1)$, et dans ce cas, il existerait, par le théorème des accroissements finis~\ref{ThoAccFinis}, un $c\in]a,x_1[$ tel que
    \begin{equation}
        f'(c)=\frac{ f(x_1)-f(a) }{ x_1-a }\neq 0,
    \end{equation}
    ce qui contredirait les hypothèses.
\end{proof}

\begin{corollary}   \label{CorNErEgcQ}
    Soient $f$ et $g$, deux fonctions dérivables sur $[a,b]$ telles que
    \begin{equation}
        f'(x) = g'(x)
    \end{equation}
    pour tout $x \in [a,b]$. Alors existe un réel $C$ tel que $f (x) = g (x) + C$ pour tout $x\in [a,b]$.
\end{corollary}

\begin{proof}
    Considérons la fonction $h(x)=f(x)-g(x)$, dont la dérivée est, par hypothèse, nulle. L'annulation de la dérivée entraine par le corolaire~\ref{CorNErEgcQ} que $h$ est  constante. Si $h(x)=C$, alors $f(x)=g(x)+C$, ce qu'il fallait prouver.
\end{proof}

\begin{definition}  \label{DefXVMVooWhsfuI}
    Soit \( I\) un intervalle ouvert de \( \eR\) et une fonction \( f\colon I\to \eR\). La fonction \( F\colon I\to \eR\) est une \defe{primitive}{primitive!fonction} de \( f\) si \( F\) est dérivable sur \( I\) et si \( F'(x)=f(x)\) pour tout \( x\) dans \( I\).
\end{definition}

Exprimé en termes des primitives, le corolaire~\ref{CorNErEgcQ} signifie que
\begin{corollary}  \label{CorZeroCst}
    Si $F$ et $G$ sont deux primitives de la même fonction $f$ sur un intervalle, alors il existe une constante $C$ pour laquelle $F(x)=G(x)+C$.
\end{corollary}
Cela signifie qu'il n'y a, en réalité, pas des milliards de primitives différentes à une fonction. Il y en a essentiellement une seule, et puis les autres, ce sont juste les mêmes, mais décalées d'une constante.

\begin{remark}
    L'hypothèse de se limiter à un intervalle est importante parce que si on considère la fonction sur deux intervalles disjoints, nous pouvons choisir la constante indépendamment dans l'un et dans l'autre. Par exemple la fonction
    \begin{equation}
        F(x)=\begin{cases}
            \ln(x)+1    &   \text{si } x>0\\
            \ln(x)-7    &    \text{si } x<0
        \end{cases}
    \end{equation}
    est une primitive de \( \frac{1}{ x }\) sur l'ensemble \( \eR\setminus\{ 0 \}\).

    Certains ne s'en privent pas. Le logiciel \href{https://www.sagemath.org}{ Sage } par exemple fait ceci :
    \begin{verbatim}
sage: f(x)=1/x
sage: F=f.integrate(x)
sage: A=F(x)-F(-x)
sage: A.full_simplify()
I*pi
    \end{verbatim}
    En réalité lorsque \( x>0\), Sage définit \( \ln(-x)=\ln(x)+i\pi\). Cela a une certaine logique parce que \( \ln(-1)=i\pi\) (du fait que \(  e^{i\pi}=-1\)), mais si on ne le sait pas, ça peut étonner.
\end{remark}

\begin{normaltext}
    Il existe plusieurs primitives à une fonction donnée. En physique, la constante arbitraire est souvent fixée par une condition initiale, comme nous le verrons dans la section~\ref{SecMRUAsecondeGGdQoT}.
\end{normaltext}

%+++++++++++++++++++++++++++++++++++++++++++++++++++++++++++++++++++++++++++++++++++++++++++++++++++++++++++++++++++++++++++
\section{Fonctions de plusieurs variables}
%+++++++++++++++++++++++++++++++++++++++++++++++++++++++++++++++++++++++++++++++++++++++++++++++++++++++++++++++++++++++++++

La physique, et les sciences en général, regorgent de fonctions à plusieurs variables.
\begin{description}
    \item[Accélération centripète]\footnote{Appelez la «centrifuge» si vous voulez; ça ne me fait ni chaud ni froid.}  Si une masse $m$ tourne sur un cercle, elle subira une accélération dirigée vers l'intérieur égale à
        \begin{equation}
            F(v,r)=\frac{ mv^2 }{ r }
        \end{equation}
        où $r$ est le rayon du cercle et $v$ est la vitesse.
    \item[Pression dans un gaz] Si on a $n$ moles de gaz dans un volume $V$ a une température $T$, alors la pression sera donnée par la fonction de trois variables
        \begin{equation}
            p=\frac{ nRT }{ V }
        \end{equation}
        où $R$ est la constante des gaz parfaits.
\end{description}

En mathématique, on peut inventer de nombreuses fonctions de plusieurs variables. La fonction
\begin{equation}
    f(x,y)=x^2+xy\cos(x^2+y^3)
\end{equation}
est définie sur $\eR^2$. La fonction
\begin{equation}
    f(x,y,z)=\frac{ x+y-2z }{ 1-x^2-y^2-z^2 }
\end{equation}
est définie sur $\eR^3$ moins la sphère unité $\{ x^2+y^2+z^2=1 \}$.

La plus grande partie de ce cours est consacrée à l'étude des fonctions de plusieurs variables. Nous allons maintenant donner quelques indications sur comment <<dessiner>> une telle fonction. Vous connaissez déjà la définition de graphe pour une fonction $f$ d'une seule variable à valeurs dans $\eR$ : c'est l'ensemble des points du plan de la forme $(x, f(x))$. Vous voyez que cet ensemble n'est pas vraiment un gros morceau de $\eR^2$ parce que son intérieur est vide : il y a une seule valeur de $f$ qui correspond au point $x$, donc une boule de $\eR^2$ centrée en $(x, f(x))$ de n'importe quel rayon contient toujours des points qui ne font pas partie du graphe de $f$.

%La première chose qu'on a envie de dire est que un tel graphe est une courbe dans $\eR^2$ mais cela n'est pas toujours vrai. Le graphe de la fonction cosinus est bien une courbe dans dans le plan, mais le graphe de la fonction tangente est une réunion infinie de courbes. Ce qui est vrai est que le graphe d'une fonction d'une variable est \emph{localement} une courbe si la fonction n'est pas trop mal choisie. % exemple?

Nous voulons donner une définition assez générale pour le graphe d'une fonction
\begin{definition}
  Soit $f$ une fonction de $\eR^m$ dans $\eR^n$. Le \defe{graphe}{graphe!fonction} de $f$ est la partie de $\eR^m\times \eR^n$ de la forme
  \begin{equation}
    \Graph f= \{ (x,y)\in \eR^m\times \eR^n \,|\, y=f(x)\}.
  \end{equation}
\end{definition}

Cette définition se spécialise de la façon suivante dans les cas communs. Soit $f$ une fonction de $\eR^m$ dans $\eR$. Le graphe de $f$ est la partie de $\eR^m\times \eR$ donné par
  \begin{equation}
    \Graph f= \{ (x,y)\in \eR^m\times \eR \,|\, y=f(x)\}.
  \end{equation}
  Et pour les fonctions \( \eR^2\to \eR\) :
\begin{equation}
	\Graph f= \{ (x,y,z)\in\eR^2\tq z=f(x,y) \}.
\end{equation}
C'est cette définition qu'il faut garder à l'esprit lorsqu'on travaille sur des dessins en trois dimensions.

Si $f$ est une fonction de deux variables indépendantes $x$ et $y$ à valeurs dans $\eR$, alors un point dans le graphe de $f$ est un point $(x,y,z)\in\eR^3$ tel que
\begin{equation}
	z=f(x,y),
\end{equation}
ou encore, un point de la forme
\begin{equation}
	\big( x,y,f(x,y) \big).
\end{equation}
%Si $g$ est une fonction d'une variable $x$ à valeurs dans $\eR^2$, alors un point dans le graphe de $g$ prend la forme $(x,g_1(x), g_2(x))$, où $g_1$ et $g_2$ sont les composantes de $g$.  Dans les deux cas le graphe est un sous-ensemble de $\eR^3$.

%Nous considérons d'abord le cas d'une fonction $f$  de deux variables $x$ et $y$ à valeurs dans $\eR$. L'espace $\eR^3$ a trois dimensions, cela veut dire que il faut fixer trois paramètres indépendants pour désigner un point de manière unique (voir, au cours d'une deuxième lecture de ces notes, la section sur les coordonnées cylindriques et sphériques,~\ref{sec_coord}). Le graphe d'une fonction comme $f$ est un sous-ensemble de $\eR^3$ où l'un des trois paramètres est d'office la valeur de $f$, donc il est décrit par seulement deux paramètres $x$ et $y$. Son intérieur est alors vide et, si $f$ est une fonction <<suffisamment gentille>>, $\Graph f$ est localement une surface dans $\eR^3$.

Nous avons parfois besoin de donner des représentations graphiques d'une fonction. Nous pouvons, par exemple, penser à la fonction que associe à un point de la Terre son altitude. Lorsqu'on part pour une promenade en montagne on a envie de connaitre le graphe de cette fonction qui correspond en fait à la surface de la montagne. Bien sur nous ne voulons pas amener avec nous un modèle en 3D de la montagne donc il nous faut une méthode efficace pour projeter le graphe de $f$ sur le plan $x$-$y$ tout en gardant les informations fondamentales. Pour cela nous avons besoin de deux définitions (à ne pas confondre !)
\begin{definition}
	Soit $f$ une fonction de $\eR^2$ dans $\eR$ et soit $c$ dans $\eR$.  La \defe{$z$-section}{section!de graphe} de $\Graph f$ à la hauteur $c$ est donné par
\[
S^z_c=\{ (x,y,c)\in \eR^3\,|\, f(x,y)=c\}.
\]
\end{definition}

\begin{definition}\label{def_niveau}
	Soit $f$ une fonction de $\eR^n$ dans $\eR$ et soit $c$ dans $\eR$. La \defe{courbe de niveau}{courbe de niveau} de $f$ à la hauteur $c$ est l'ensemble
    \begin{equation}
        N_c=\{ (x1,\ldots, x_n)\in \eR^n\,|\, f(x1,\ldots, x_n)=c\}.
    \end{equation}
\end{definition}
On peut représenter la fonction $f$ d'une façon très précise en traçant quelques-unes de ses courbes de niveau.  Dans la suite on pourra considérer aussi les $x$-sections et les $y$-sections du graphe d'une fonction de deux variables. La $x$-section de $\Graph f$ à la hauteur $a$ est
\[
S^x_a=\{(a,y,z)\in\eR^3\,|\, f(a,y)=z\}.
\]
Comme vous avez peut être déjà compris, $S^x_a$ est le graphe de la fonction de $y$ qu'on obtient de $f$ en fixant $x=a$. Cette fonction est appelée $x$-section de $f$ pour $x=a$.

Certaines surfaces dans $\eR^3$ sont le graphe d'une fonction.

\begin{example}
	Quelques graphes importants.
  \begin{description}
    \item[Un plan non vertical] Tout plan dans $\eR^3$ peut être décrit par une équation de la forme
\[
a(x-x_0)+ b(y-y_0) + c(z-z_0) = r,
\]
où, $(x_0, y_0, z_0)$ est vecteur dans $\eR^3$, et $a$, $b$, $c$ et $r$ sont des nombres réels. Si $c\neq 0$ alors le plan n'est pas vertical et on peut dire que il est le graphe de la fonction
\[
P(x,y)= \frac{r+cz_0 -a(x-x_0)-b(y-y_0)}{c},
\]
quitte à choisir des nouvelles constantes $s$, $t$, $q$,
\[
P(x,y)=sx +ty +q.
\]
    \item[Un paraboloïde elliptique] Pour tous $\alpha$ et $\beta$ dans $\eR$ les  graphes des fonctions
\[
PE_1(x,y)=\frac{x^2}{\alpha^2}+\frac{y^2}{\beta^2}
\]
ou de la fonction
\[
PE_2(x,y)=-\frac{x^2}{\alpha^2}-\frac{y^2}{\beta^2}
\]
sont des paraboloïdes elliptiques. Le premier est contenu dans le demi-espace $z\geq 0$, l'autre dans $z\leq 0$. Le nom de cette surface vient de la forme de ses sections. En fait toutes  sections $S^z_c$ sont des ellipses, alors que les sections $S^x_a$ et $S^y_b$ sont des paraboles.
    \item[Un paraboloïde hyperbolique (selle)]  Pour tous $\alpha$ et $\beta$ dans $\eR$ les  graphes des fonctions
\[
PH_1(x,y)=\frac{x^2}{\alpha^2}-\frac{y^2}{\beta^2}
\]
ou de la fonction
\[
PH_2(x,y)=-\frac{x^2}{\alpha^2}+\frac{y^2}{\beta^2}
\]
sont des paraboloïdes hyperboliques. Remarquez que les  sections $S^z_c$ de ce graphe sont des hyperboles, alors que les sections $S^x_a$ et $S^y_b$ sont des paraboles.
    \item[Une demi-sphère] La fonction $S^+=\sqrt{R^2-x^2-y^2}$ a pour graphe la demi-sphère supérieure centrée en l'origine et de rayon $R$.
Le dernier de ces exemples nous signale une chose très importante : une sphère entière n'est pas le graphe d'une fonction de $x$ et $y$. Par contre, une demi-sphère est bien le graphe de la fonction $f(x,y)=\sqrt{1-x^2-y^2}$.

L'équation que nous utilisons  pour d'écrire une sphère de rayon $R$ centrée en l'origine est
\[
x^2+y^2+z^2=R^2
\]
Donc, à  chaque point  $(x,y)$ dans le disque $x^2+y^2\leq R^2$ (notez que ce disque est contenu dans la section $S^z_0$), on peut associer deux valeurs de $z$ : $z_1=\sqrt{R^2-x^2-y^2}$ et  $z_2=-\sqrt{R^2-x^2-y^2}$. Par définition, une fonction n'associe qu'un seul valeur à chaque point de son domaine, d'où l'impossibilité de décrire cette sphère comme le graphe d'une fonction de $x$ et $y$.

  \end{description}
\end{example}

Considérons la fonction $Sp: \eR^3\to \eR$ qui associe à $(x,y,z)$ la valeur $x^2+y^2+z^2$. La sphère de rayon $R$ centrée en l'origine est l'ensemble de niveau $N_{R^2}$ de $Sp$. L'ensemble de niveau $N_{0}$ de $Sp$ est l'origine, et tous les ensembles de niveau de hauteur négative sont vides. La même chose est vraie pour les ellipsoïdes centrées en l'origine avec les axes $x$, $y$ et $z$ comme axes principaux et comme longueurs de demi-axes $a$, $b$ et $c$. Voici la fonction dont il sont les ensembles de niveau
\[
El(x,y,z)= \frac{x^2}{a^2}+\frac{y^2}{b^2}+\frac{z^2}{c^2}.
\]
\begin{example}
	Des ensembles de niveau importants.
  \begin{description}
    \item[Tout graphe]
	    Le graphe de toute fonction $f$  de $\eR^2$ dans $\eR$ peut être considéré comme l'ensemble de niveau zéro de la fonction $F(x,y,z)=z-f(x,y)$.

    \item[Hyperboloïdes]
	    Les hyperboloïdes, comme les ellipsoïdes, sont une famille d'ensemble de niveau. En particulier, nous considérons des hyperboloïdes dont l'axe de symétrie est l'axe des $z$ et qui sont symétriques par rapport un plan $x$-$y$.  Une fois que les paramètres  $a$, $b$ et $c$ sont fixés la fonction que nous intéresse est
\[
Hyp(x,y,z)= \frac{x^2}{a^2}+\frac{y^2}{b^2}-\frac{z^2}{c^2}.
\]
Les ensembles de niveau $N_d$ pour $d>0$ sont connexes, on les appelle \emph{hyperboloïdes à une feuille}. L'ensemble de niveau $N_0$ est \emph{cône (elliptique)}, les deux moitiés du cône se touchent en l'origine. Enfin, les ensembles de niveau $N_d$ pour $d<0$ ne sont  pas connexes et pour cette raison on les appelle \emph{hyperboloïdes à deux feuilles}.
  \end{description}
\end{example}

%---------------------------------------------------------------------------------------------------------------------------
\subsection{Graphes de fonctions à plusieurs variables}
%---------------------------------------------------------------------------------------------------------------------------

Le \defe{graphe}{graphe!fonction de deux variables} d'une fonction de deux variables $f\colon D\subset\eR^2\to \eR$ est l'ensemble
\begin{equation}
    \Big\{   \big( x,y,f(x,y) \big) \tq (x,y)\in D \Big\}\subset\eR^3.
\end{equation}
Ce graphe est une surface dans $\eR^3$.

\begin{example}     \label{ExempleTroisDxxyy}

    Tracer le graphe de la fonction
    \begin{equation}
        (x,y)\mapsto x^2+y^2.
    \end{equation}
    Le plus simple est de demander à Sage de nous fournir une représentation 3D
    \begin{verbatim}
----------------------------------------------------------------------
| Sage Version 4.6.1, Release Date: 2011-01-11                       |
| Type notebook() for the GUI, and license() for information.        |
----------------------------------------------------------------------
sage: f(x,y)=x**2+y**2
sage: plot3d(f,(x,-3,3),(y,-3,3))
    \end{verbatim}

    Voici ce que cela donne\footnote{En vrai, ce que Sage donne est un objet qu'on peut même faire bouger.} : (à regarder avec des lunettes bleues et rouges) :
    \begin{center}
            \includegraphics[width=15cm]{pictures_bitmap/coupe.png}
    \end{center}
    À part que l'ordinateur l'a dit, est-ce qu'on peut comprendre pourquoi le graphe de la fonction $x^2+y^2$ ressemble à un bol ? En coordonnées cylindriques, le graphe s'écrit
    \begin{equation}
        z=r^2.
    \end{equation}
    Donc il se fait que plus on s'éloigne du point $(0,0)$ dans le plan $XY$, plus le graphe va monter. Et il monte à quelle vitesse ? Il monte à la vitesse $r^2$. Il s'agit donc de dessiner la fonction $z=r^2$ dans le plan et de la «faire tourner».

\end{example}

%---------------------------------------------------------------------------------------------------------------------------
\subsection{Courbes de niveau}
%---------------------------------------------------------------------------------------------------------------------------

Une technique utile pour se faire une idée de la forme d'une fonction en trois dimensions est de tracer les \defe{courbes de niveau}{courbe de niveau}. La courbe de niveau de hauteur $h$ est la courbe dans le plan donnée par l'équation
\begin{equation}
    f(x,y)=h.
\end{equation}

\begin{example}

    Dessinons par exemple les courbes de niveau de la fonction
    \begin{equation}
        f(x,y)=x+y+2.
    \end{equation}
    La courbe de niveau $h$ est donnée par l'équation $x+y+2=h$, c'est-à-dire
    \begin{equation}
        y(x)=-x+h-2.
    \end{equation}
    Par conséquent la courbe de niveau de hauteur $0$ est $y=-x-2$, celle de hauteur $5$ est $y=-x+3$, etc.

    Nous pouvons également nous aider de Sage pour ce faire :
    \begin{verbatim}
----------------------------------------------------------------------
| Sage Version 4.6.1, Release Date: 2011-01-11                       |
| Type notebook() for the GUI, and license() for information.        |
----------------------------------------------------------------------
sage: f(x,y)=x+y+2
sage: var('h')
h
sage: niveau(h,x)=solve(f(x,y)==h,y)[0].rhs()
sage: g1(x)=niveau(1,x)
sage: g1
x |--> -x - 1
    \end{verbatim}
    Ici la fonction \verb+g1+ est la courbe de niveau $1$.

    Si on veut faire tracer une courbe de niveau, Sage peut le faire :
    \begin{verbatim}
        sage: implicit_plot(f(x,y)==1,(x,-3,3),(y,-4,4))
    \end{verbatim}
    Cela tracera la courbe de niveau $h=1$ dans la partie du plan $x\in\mathopen[ -3 , 3 \mathclose]$ et $y\in\mathopen[ -4,4 ,  \mathclose]$.

\end{example}

Il est bien entendu possible de créer automatiquement $50$ courbes de niveau et de demander de les tracer toutes sur le même graphe.
\lstinputlisting{tex/frido/courbeNiveau.py}

Le résultat est :

\begin{center}
        \includegraphics[width=8cm]{pictures_bitmap/niveauCercles.png}
\end{center}
Notez que les courbes sont censées être des cercles : les axes $X$ et $Y$ n'ont pas la même échelle.

%Vous trouverez sur \href{http://www.sagenb.org/home/pub/23/}{cette page} tout ce qu'il vous faudra pour créer des courbes de niveau avec Sage.
%TODO : cette adresse n'est plus valide. Il faut retrouver où elle est passés sur sagemath.org (sagenb est devenu sagemath)

\begin{example}
    Un exemple plus riche en enseignements est celui de la fonction
    \begin{equation}
        f(x,y)=x^2-y^2.
    \end{equation}
    La courbe de niveau $h$ est donnée par l'équation $x^2-y^2=h$.

    Commençons par $h=0$. Dans ce cas nous avons $(x+y)(x-y)=0$ et par conséquent les courbes de niveau de hauteur zéro sont les deux droites $x+y=0$ et $x-y=0$.

    Voyons ensuite la courbe de niveau $h=1$. Cela est l'équation $x^2-y^2=1$, c'est-à-dire
    \begin{equation}
        y(x)=\pm\sqrt{x^2-1}.
    \end{equation}
    C'est une fonction qui n'est définie que pour $| x |\geq 1$. Avec $x=1$ nous avons $y=1$. Ensuite, lorsque $x$ grandit, $y$ grandit également, mais la courbe ne peut pas croiser la courbe de niveau $h=0$. Donc, suivant les notations de la figure~\ref{LabelFigCQIXooBEDnfK}, la courbe de niveau «part» de $P$ et doit monter sans croiser les diagonales.

 % les figures CQIXooBEDnfK et KGQXooZFNVnW sont générées par le script MBFDooRFPyNW

%The result is on figure~\ref{LabelFigCQIXooBEDnfK}. % From file CQIXooBEDnfK
\newcommand{\CaptionFigCQIXooBEDnfK}{La courbe de niveau $h=1$ de $x^2-y^2$. Notez qu'elle est en deux morceaux.}
\input{auto/pictures_tex/Fig_CQIXooBEDnfK.pstricks}

%The result is on figure~\ref{LabelFigKGQXooZFNVnW}. % From file KGQXooZFNVnW
\newcommand{\CaptionFigKGQXooZFNVnW}{La courbe de niveau $x^2-y^2=-1$.}
\input{auto/pictures_tex/Fig_KGQXooZFNVnW.pstricks}

    En ce qui concerne la courbe de niveau $h=-1$, elle correspond à la courbe $y=\pm\sqrt{1+x^2}$ qui est définie pour tous les $x\in\eR$. Le même raisonnement que précédemment nous amène à la figure~\ref{LabelFigKGQXooZFNVnW}.

\end{example}

Une autre façon de voir les courbes de niveau est de dire que la courbe de niveau de hauteur $h$ est la projection dans le plan $XY$ de la section du graphe de $f$ par le plan $z=h$.

On peut également définir le graphe de fonctions de trois (ou plus) variables. Le graphe de la fonction $f\colon D\subset\eR^3\to \eR$ est l'ensemble
\begin{equation}
    \big\{ \big( x,y,z,f(x,y,z) \big)\tq (x,y,z)\in D \big\}\subset \eR^4.
\end{equation}
De tels graphes ne peuvent pas être représentés sur une feuille de papier. Il est toutefois possible de définir les ensembles de niveaux :
\begin{equation}
    E_h=\big\{ (x,y,z)\in D\tq  f(x,y,z)=h\big\}.
\end{equation}
Ce sont des surfaces dans $\eR^3$ que l'on peut dessiner.

\begin{example}
    Les surfaces de niveau de la fonction $f(x,y,z)=x^2+y^2+z^2$ sont des sphères. Il n'y a pas de surfaces de niveau pour les «hauteurs» négatives.
\end{example}

\begin{example}
    Considérons la fonction $f(x,y,z)=x^2+y^2-z^2$. En coordonnées cylindriques, cette fonction s'écrit
    \begin{equation}
        f(r,\theta,z)=r^2-z^2.
    \end{equation}
    La surface de niveau $0$ est donnée par l'équation $r=| z |$. Cela fait un cercle à chaque hauteur, dont le rayon grandit linéairement avec la hauteur; le tout est donc un cône. C'est d'ailleurs le cône obtenu par rotation de la courbe de niveau $h=0$ que nous avions obtenu pour la fonction $x^2-y^2$.

    En ce qui concerne les ensembles de niveau positifs, ils sont donnés par
    \begin{equation}
        z=\pm\sqrt{x^2+y^2-h}.
    \end{equation}
    Notez qu'ils ne sont pas définis pour $r\geq h$. Cela pose un petit problème quand on veut le tracer à l'ordinateur :
    \begin{verbatim}
----------------------------------------------------------------------
| Sage Version 4.6.1, Release Date: 2011-01-11                       |
| Type notebook() for the GUI, and license() for information.        |
----------------------------------------------------------------------
sage: var('x,y')
(x, y)
sage: f(x,y)=sqrt(x**2+y**2-3)
sage: F=plot3d(f(x,y),(x,-5,5),(y,-5,5))
sage: G=plot3d(-f(x,y),(x,-5,5),(y,-5,5))
sage: F+G
    \end{verbatim}
Le résultat est\footnote{Encore une fois : ça donne mieux à l'écran, et vous pouvez le faire bouger; je vous encourage à le faire !} :
    \begin{center}
            \includegraphics[width=15cm]{pictures_bitmap/AdSmauvais.png}
    \end{center}
    On voit qu'il y a un grand trou au centre correspondant aux $z$ proches de zéro. Or d'après l'équation, il n'en est rien : en $z=0$ il y a bel et bien tout un cercle. Afin d'obtenir une meilleur image, il faut demander de tracer avec un maillage plus fin :
    \begin{verbatim}
----------------------------------------------------------------------
| Sage Version 4.6.1, Release Date: 2011-01-11                       |
| Type notebook() for the GUI, and license() for information.        |
----------------------------------------------------------------------
sage: var('x,y')
(x, y)
sage: f(x,y)=sqrt(x**2+y**2-3)
sage: F=plot3d(f(x,y),(x,-5,5),(y,-5,5),plot_points=300)
sage: G=plot3d(-f(x,y),(x,-5,5),(y,-5,5),plot_points=300)
sage: F+G
    \end{verbatim}
    Le temps de calcul est un peu plus long, mais le résultat est meilleur :
    \begin{center}
            \includegraphics[width=15cm]{pictures_bitmap/AdSbon.png}
    \end{center}
\end{example}

%+++++++++++++++++++++++++++++++++++++++++++++++++++++++++++++++++++++++++++++++++++++++++++++++++++++++++++++++++++++++++++
%\section{Calcul de limites}
%+++++++++++++++++++++++++++++++++++++++++++++++++++++++++++++++++++++++++++++++++++++++++++++++++++++++++++++++++++++++++++

%Incidemment, le lemme~\ref{Def_diff2} nous donne une nouvelle technique pour calculer des limites à plusieurs variables, similaire à celle du développement asymptotique expliquée dans la section~\ref{SecTaylorR}.

%En effet, la formule \eqref{def_diff2} nous permet d'écrire $f(x)$ sous la forme
%\begin{equation}
%	f(x)=f(a)+df(a).(x-a)+\sigma_f(a,x)\| x-a \|
%\end{equation}
%où la fonction $\sigma_f$ satisfait $\lim_{x\to a}\sigma_f(a,x)=0$. Ici, $x$ et $a$ sont des éléments de $\eR^m$.

%+++++++++++++++++++++++++++++++++++++++++++++++++++++++++++++++++++++++++++++++++++++++++++++++++++++++++++++++++++++++++++
\section{Limites à plusieurs variables}
%+++++++++++++++++++++++++++++++++++++++++++++++++++++++++++++++++++++++++++++++++++++++++++++++++++++++++++++++++++++++++++
\label{SecLimVarsPlus}

Prenons une fonction $f\colon \eR^n\to \eR$. Nous disons que
\begin{equation}
    \lim_{x\to x_0}f(x)=l\in\eR
\end{equation}
lorsque $\forall \epsilon>0$, $\exists\delta$ tel que $\| x-x_0 \|\leq\delta$ implique $| f(x)-l |\leq \epsilon$.

Remarquez qu'ici, $x\in\eR^n$, et sachez distinguer $\| . \|$, la norme dans $\eR^n$ de $| . |$ qui est la valeur absolue dans $\eR$. Une autre façon d'exprimer cette définition est que l'ensemble des valeurs atteintes par $f$ dans une boule de rayon $\delta$ autour de $x_0$ n'est pas très loin de $l$. Nous définissons donc
\begin{equation}
    E_{\delta}=\{ f(x)\tq x\in B(x_0,\delta) \}.
\end{equation}
Notez que si $f$ n'est pas définie en $x_0$, il n'y a pas de valeurs correspondantes au centre de la boule dans $E_{\delta}$. Ceci est évidemment la situation générique lorsqu'il y a une indétermination à lever dans le calcul de la limite. Nous avons alors que
\begin{equation}
    \lim_{x\to x_0}f(x)=l
\end{equation}
lorsque $\forall\epsilon>0$, $\exists\delta$ tel que
\begin{equation}        \label{Eqvmoinsrapplimdeux}
    \sup\{ | v-l |\tq v\in E_{\delta} \}\leq\epsilon.
\end{equation}
Une façon classique de montrer qu'une limite n'existe pas, est de prouver que, pour tout $\delta$, l'ensemble $E_{\delta}$ contient deux valeurs constantes. Si par exemple $0\in E_{\delta}$ et $1\in E_{\delta}$ pour tout $\delta$, alors aucune valeur de $l$ (même pas $l=\pm\infty$) ne peut satisfaire à la condition \eqref{Eqvmoinsrapplimdeux} pour toute valeur de $\epsilon$.

Nous laissons à la sagacité de l'étudiant le soin d'adapter tout ceci pour le cas $\lim_{x\to x_0}f(x)=\pm\infty$.

La proposition suivante semble évidente, mais nous sera tellement
utile qu'il est préférable de l'expliciter~:
\begin{proposition}     \label{PROPooPOAQooPmxEtb}
    Soient $f : D \to \eR$ une fonction de domaine \( D\), \( a\in\Adh(D)\) et un voisinage \( V\) de \( a\). Nous supposons que \( V\cap D\) s'écrive comme une intersection finie :
  \begin{equation*}
    V\cap D = \bigcup_{i=1}^k A_i
  \end{equation*}
  telle que $a \in \Adh A_i$ pour tout $i \leq k$. Alors, la limite
  \begin{equation}      \label{EQooEXELooHccCGw}
    \limite x a f(x)
  \end{equation}
  existe et vaut $b\in \eR$ si et seulement si chacune des limites
  \begin{equation}
    \limite[x \in A_i] x a f(x)
  \end{equation}
  existe et vaut $b$.
\end{proposition}

\begin{proof}On sait déjà que si la limite de $f : D \to \eR$
    existe, alors toute restriction à $A_i$ admet la même limite\footnote{C'est une conséquence de la caractérisation séquentielle de la continuité \ref{fContEstSeqCont}.}. Il suffit donc de prouver la réciproque.

    Fixons provisoirement un entier \( i\) entre \( 1\) et \( k\) ainsi que \( \epsilon>0\). Vu que \( \lim_{\substack{x\to a\\x\in A_i}} f(x)=b\), il existe \( \delta_i>0\) tel que si \( x\in A_i\) et si \( 0<| x-a |<\delta_i\), alors
    \begin{equation}        \label{EQooUCAIooRpUgnE}
        | f(x)-f(a) |<\epsilon.
    \end{equation}
    Quitte à prendre \( \delta_i\) un peu plus petit, nous supposons que \( V\subset B(a,\delta_i)\).

    Nous posons $\delta = \min\{\delta_i\}_{i=1,\ldots, k}$, et nous considérons \( x\in D\) tel que \( 0<| x-a |<\delta\). Nous avons alors
  \begin{enumerate}
      \item \( x\in V\cap D\),
      \item il existe \( i\) tel que \( x\in A_i\).
  \end{enumerate}
  Ce \( x\) est donc un élément de \( A_i\) vérifiant \( 0<| x-a |<\delta\leq \delta_i\). Il vérifie donc \eqref{EQooUCAIooRpUgnE} : \( | f(x)-f(a) |<\epsilon\). 

  Cela prouve la limite \eqref{EQooEXELooHccCGw}.
\end{proof}

\begin{example}
  \begin{enumerate}
  \item Pour qu'une fonction $f : \eR \to \eR$ admette une limite en $a \in \eR$, il faut et il suffit qu'elle y admette une limite à droite et une limite à gauche qui soient égales.

  Cela est une application de la proposition \ref{PROPooPOAQooPmxEtb} avec \( \eR=\mathopen] -\infty , a \mathclose[\cup\mathopen] a , \infty \mathclose[\).

  \item Une suite $(x_k)$ admet une limite si et seulement si les
    sous-suites $(x_{2k})$ et $(x_{2k+1})$ convergent vers la même
    limite. Ceci n'est pas une application directe de la proposition,
    mais la teneur est la même.
  \end{enumerate}
\end{example}

\begin{lemma}[\cite{MonCerveau}]
    Soient deux espaces vectoriels \( E\) et \( F\). Soit une fonction \( f\colon \eR\to F\) telle que \( \lim_{t\to 0} f(t)=y\in F\). Nous posons 
    \begin{equation}
        \begin{aligned}
            \varphi\colon E&\to F \\
            h&\mapsto f(\| h \|). 
        \end{aligned}
    \end{equation}
    Alors \( \varphi\) admet une limite pour \( h\to 0\) et elle est donnée par
    \begin{equation}
        \lim_{h\to 0} \varphi(h)=\lim_{t\to 0} f(t).
    \end{equation}
\end{lemma}

\begin{proof}
    Soit \( \epsilon>0\). Il existe \( \delta\) tel que si \( t<\delta\) alors \( \| f(t)-y \|_F<\epsilon\). Si \( \| h \|<\delta\) nous avons
    \begin{equation}
        \| \varphi(h)-y \|=\| f(\| h \|)-y \|<\epsilon.
    \end{equation}
    Donc c'est bon.
\end{proof}

Voici, dans le même ordre d'idée, un autre résultat qui permet de réduire le nombre de variables dans une limite lorsque la fonction ne dépend pas de certaines variables.

\begin{lemma}[\cite{MonCerveau}]        \label{LEMooYLIHooFBQyzC}
    Soit \( f\colon \eR^2\to \eR\) telle qu'il existe une fonction \( g\colon \eR\to \eR\) vérifiant
    \begin{enumerate}
        \item
            \( f(x,y)=g(x)\),
        \item
            \( \lim_{t\to a} g(t)=\ell\).
    \end{enumerate}
    Alors
    \begin{equation}
        \lim_{\substack{(x,y)\to(a,b)\\x\neq a}} f(x,y)=\ell.
    \end{equation}
\end{lemma}

\begin{proof}
    Soit \( \epsilon>0\). Par hypothèse sur la limite de \( g\) en \( a\), il existe \( \delta>0\) tel que \( 0<| t-a |<\delta\) implique \( | g(t)-\ell |<\epsilon\).

    Attention : passage subtil\footnote{Je rejette déjà en bloc et d'un revers de main toute tentative de dire «la limite épointée, c'est mieux». Voir aussi l'exemple \ref{EXooHSYNooBZhDbE}.}.  Si \( 0<\| (x,y)-(a,b) \|<\delta\), alors nous avons évidemment aussi \( | x-a |<\delta\), mais pas spécialement \( 0<| x-a |<\delta\) comme le requis pour utiliser la limite de \( g\).
    
    Dans le calcul de la limite restreinte à \( x\neq a\), les points qui interviennent sont les valeurs de \( (x,y)\) dans \( B\big( (a,b),\delta \big)\setminus\{ x=a \}\). Or pour celles-là nous avons bien \( 0<| x-a |<\delta\). Le calcul suivant fonctionne donc :
    \begin{equation}
        | f(x,y)-\ell |=| g(x)-\ell |<\epsilon.
    \end{equation}
\end{proof}

\begin{example}     \label{EXooHSYNooBZhDbE}
    Pourquoi prendre la limite \( (x,y)\to (a,b)\) avec \( x\neq a\) dans l'énoncé du lemme \ref{LEMooYLIHooFBQyzC} ? Imaginons la fonction
    \begin{equation}
        g(x)=\begin{cases}
            0    &   \text{si } x\neq 0\\
            1    &    \text{si } x=0.
        \end{cases}
    \end{equation}
    Dans ce cas, le graphe de la fonction \( f(x,y)=g(x)\) est tout plat sauf la ligne \( x=0\) qui est en hauteur. Nous avons donc \( f(0,t)=1\) pour tout \( t\) et donc nous n'avons pas \( \lim_{(x,y)\to (0,0)} f(x,y)=0\) :  tout voisinage de \( (0,0)\) contient des points \( (x,y)\) tels que \( f(x,y)=1\) et des points \( (x,y)\) tels que \( f(x,y)=0\).
\end{example}

Il existe de nombreuses façons de calculer des limites à plusieurs variables. Plus nous connaîtrons de mathématiques, plus nous aurons de techniques à notre disposition. Nous allons tout de suite voir quelques méthodes. Voir le thème~\ref{THEMEooLTCIooGDIPnF} pour plus de techniques et d'exemples.

%--------------------------------------------------------------------------------------------------------------------------- 
\subsection{Caractérisation de la limite par les suites}
%---------------------------------------------------------------------------------------------------------------------------

\begin{example}		\label{ExFNExempleMethodeTrigigi}
	Considérons la fonction
	\begin{equation}
		f(x,y)=\frac{ xy }{ x^2+y^2 },
	\end{equation}
	et remarquons que, quelle que soit la valeur de $y$, cette fonction est nulle lorsque $x=0$. De la même manière, nous voyons que si $x=y$, alors la fonction vaut\footnote{En fait ce que nous sommes en train de faire est de poser $\theta=\pi/2$ et $\theta=\pi/4$ dans \eqref{Eq2807fpolairerhodeuxcossin}.} $\frac{ 1 }{2}$.

	Il est impossible que la fonction ait une limite en $(0,0)$ parce qu'on ne peut pas trouver un $\ell$ dont on s'approche à la fois en suivant la ligne $x=0$ et la ligne $x=y$.

	Deux autres chemins avec encore deux autres valeurs sont dessinés sur la figure~\ref{LabelFigMethodeChemin}.

    Cet exemple pourra être formalisé en utilisant le théorème~\ref{ThoLimSuite}. Voir l'exemple~\ref{EXooNBTYooFyKRTB}.
\end{example}

\begin{theorem}[Caractérisation de la limite par les suites]		\label{ThoLimSuite}
	Une fonction $f\colon D\subset\eR^m\to \eR^n$ admet une limite $\ell$ en un point d'accumulation $a$ de $D$ si et seulement si pour toute suite $(x_n)$ dans $D\setminus\{ a \}$ convergente vers $a$, la suite $\big( f(x_n) \big)$ dans $\eR^n$ converge vers $\ell$.
\end{theorem}

\begin{proof}
	Supposons d'abord que la fonction ait une limite $\ell$ lorsque $x\to a$, et considérons une suite $(x_n)$ dans $D\setminus\{ a \}$ convergente vers $a$. Nous devons montrer que la suite $y_n=f(x_n)$ converge vers $\ell$, c'est-à-dire que si nous choisissons $\varepsilon>0$ nous devons montrer qu'il existe un $N$ tel que $n>N$ implique $\| y_n-\ell  \|=\| f(x_n)-\ell \|<\varepsilon$.

	Nous avons deux hypothèses. La première est la convergence de la fonction et la seconde est la convergence de la suite $(x_n)$. L'hypothèse de convergence de la fonction nous dit que (le $\varepsilon$ a déjà été choisi dans le paragraphe précédent)
	\begin{equation}
		\exists\delta\tq\,0<\| x-a \|<\delta\Rightarrow\| f(x)-\ell \|<\varepsilon.
	\end{equation}
	Une fois choisi ce $\delta$ qui «va avec» le $\varepsilon$ qui a été choisi précédemment, la définition de la convergence de la suite nous enseigne que
	\begin{equation}
		\exists N\tq n>N\Rightarrow\| x_n-a \|<\delta.
	\end{equation}
	Récapitulons ce que nous avons fait. Nous avons choisi un $\varepsilon$, et puis nous avons construit un $N$. Lorsque $n>N$, nous avons $\| x_n-a \|<\delta$. Mais alors, par construction de ce $\delta$, nous avons $\| f(x_n)-\ell \|<\varepsilon$. Au final, $n>N$ implique bien $\| y_n-\ell \|<\varepsilon$, ce qu'il nous fallait.

	Nous supposons maintenant que la fonction $f$ \emph{ne} converge \emph{pas} vers $\ell$, et nous allons construire une suite d'éléments $x_n$ qui converge vers $a$ sans que $(y_n)=f(x_n)$ ne converge vers $\ell$. La fonction $f$ vérifie la condition \eqref{EqCaractNonLim}. Nous prenons donc un $\varepsilon$ tel que $\forall \delta$, il existe un $x$ qui vérifie \emph{en même temps} les deux conditions
	\begin{subequations}
		\begin{numcases}{}
			0<\| x-a \|<\delta\\
			\| f(x)-\ell \|>\varepsilon.
		\end{numcases}
	\end{subequations}
	Un tel $x$ existe pour tout choix de $\delta$. Choisissons un $n$ arbitraire et $\delta=\frac{1}{ n }$. Nous nommons $x_n$ le $x$ correspondant à ce choix de $n$. La suite $(x_n)$ ainsi construite converge vers $a$ parce que
	\begin{equation}
		\| x_n-a \|<\delta_n=\frac{1}{ n },
	\end{equation}
	donc dès que $n$ est grand, $\| x_n-a \|$ est petit. Mais la suite $y_n=f(x_n)$ ne converge pas vers $\ell$ parce que
	\begin{equation}
		\| f(x_n)-\ell \|>\varepsilon
	\end{equation}
	pour tout $n$. La suite $y_n$ ne s'approche donc jamais à moins d'une distance $\varepsilon$ de $\ell$.
\end{proof}

\begin{example} \label{EXooNBTYooFyKRTB}
    Reprenons l'exemple \ref{ExFNExempleMethodeTrigigi}. Considérons les deux suites $x_n=(0,\frac{1}{ n })$ et $y_n=(\frac{1}{ n },\frac{1}{ n })$. Ce sont deux suites dans $\eR^2$ qui tendent vers $(0,0)$. Si la fonction $f$ convergeait vers $\ell$, alors nous aurions au moins
    \begin{subequations}\label{Eq3007Lixxyyell}
        \begin{align}
            \lim f(x_n)&=\ell\\
            \lim f(y_n)&=\ell,
        \end{align}
    \end{subequations}
    mais nous savons que pour tout $n$, $f(x_n)=f(0,\frac{1}{ n })=0$ et $f(y_n)=f(\frac{1}{ n },\frac{1}{ n })=\frac{1}{ 2 }$. Il n'y a donc aucun nombre $\ell$ qui vérifie les deux équations \eqref{Eq3007Lixxyyell} parce que $\lim f(x_n)=0$ et $\lim f(y_n)=\frac{ 1 }{2}$.
\end{example}

%---------------------------------------------------------------------------------------------------------------------------
\subsection{Règle de l'étau}
%---------------------------------------------------------------------------------------------------------------------------

Une première façon de calculer la limite d'une fonction est de la «coincer» entre deux fonctions dont nous connaissons la limite. 
\begin{theorem}[Règle de l'étau\cite{BIBooNKECooNNYQvB}]		\label{ThoRegleEtau}
	Soit $\mO$, un ouvert de $\eR^m$ contenant le point $a$. Soient $f$, $g$ et $h$, trois fonctions définies sur $\mO$ (éventuellement pas en $a$ lui-même). Supposons que pour tout $x\in\mO$ (à part éventuellement $a$), nous ayons les inégalités
	\begin{equation}
		g(x)\leq f(x)\leq h(x).
	\end{equation}
	Supposons de plus que
	\begin{equation}
		\lim_{x\to a} g(x)=\lim_{x\to a} h(x)=\ell.
	\end{equation}
	Alors la limite $\lim_{x\to a} f(x)$ existe et vaut $\ell$.
\end{theorem}

Nous insistons sur le fait que les deux fonctions entre lesquelles nous coinçons $f$ doivent tendre vers \emph{la même} valeur.

Cette méthode est très pratique lorsqu'on a des fonctions trigonométriques qui se factorisent parce qu'elles sont toujours majorables par $1$; voir l'exemple~\ref{EXooSPFDooSluUGV}.

\begin{example}
	Prouver la continuité en $(0,0)$ de la fonction
	\begin{equation}
		f(x,y)=\begin{cases}
			\frac{ x | y | }{ \sqrt{x^2+y^2} }	&	\text{si }(x,y)\neq (0,0)\\
			0	&	 \text{sinon.}
		\end{cases}
	\end{equation}
	Considérons une suite $(x_n,y_n)\in\eR^2$ qui tend vers $(0,0)$. Étant donné que $\frac{ | y | }{ \sqrt{x^2+y^2} }<1$ pour tout $x$ et $y$, nous avons
	\begin{equation}
		0\leq | f(x_n,y_n) |=\left| \frac{ x_n | y_n | }{ \sqrt{x_n^2+y_n^2} } \right| \leq | x_n |\to 0.
	\end{equation}
	Donc nous avons
	\begin{equation}
		\lim_{(x,y)\to(0,0)}f(x,y)=0=f(0,0),
	\end{equation}
	ce qui prouve que la fonction est continue en $(0,0)$ par la proposition~\ref{PropFnContParSuite}. Nous avons utilisé la règle de l'étau (théorème~\ref{ThoRegleEtau}).
\end{example}

\begin{normaltext}
    Nous notons \( f\sim g\)\nomenclature[Y]{\( f\sim g\)}{fonctions ayant des limites équivalentes} pour \( x\to a\) lorsque \( \lim_{x\to a} \frac{ f(x) }{ g(x) }=1\).

    Cela signifie que \( f\) et \( g\) tendent vers la même limite, à la même vitesse.
\end{normaltext}

%---------------------------------------------------------------------------------------------------------------------------
\subsection{Méthode des chemins}
%---------------------------------------------------------------------------------------------------------------------------

Lorsque la limite n'existe pas, il y a une façon en général assez simple de le savoir, c'est la \defe{méthode des chemins}{méthode!des chemins}.

\newcommand{\CaptionFigMethodeChemin}{Sur toute la droite $y=-x$, la fonction vaut $-1/2$, tandis que sur toute la droite $y=x/2$, elle vaut $\frac{2}{ 5 }$. Il est donc impossible que la fonction ait une limite en $(0,0)$, parce que dans toute boule autour de zéro, il y aura toujours un point de chacune de ces deux droites.}
\input{auto/pictures_tex/Fig_MethodeChemin.pstricks}

C'est la proposition suivante qui va faire une grosse partie du travail.  
\begin{proposition}[\cite{MonCerveau}]     \label{PROPooSAFIooWvmSiT}
	Soit $f\colon D\subset\eR^m\to \eR^n$ et $a$ un point d'adhérence de $D$. Alors nous avons
	\begin{equation}
		\lim_{x\to a} f(x)=\ell
	\end{equation}
	si et seulement si pour toute fonction $\gamma\colon \eR\to \eR^m$ telle que $\lim_{t\to 0} \gamma(t)=a$, nous avons
	\begin{equation}
		\lim_{t\to 0} (f\circ\gamma)(t)=\ell.
	\end{equation}
\end{proposition}

\begin{proof}
    En deux parties.
    \begin{subproof}
        \item[Sens direct]
            Soit une fonction \( \gamma\colon \eR\to \eR^m\) telles que \( \lim_{t\to 0} \gamma(t)=a\). Par le théorème \ref{ThoLimSuite}, il suffit de montrer que \( (f\circ\gamma)(t_n)\to\ell\) pour toute suite \( t_n\to 0\) dans \( \eR\).

            Nous savons que la suite \( n\mapsto \gamma(t_n)\) est une suite qui converge vers \( a\). Mais l'hypothèse \( \lim_{x\to a} f(x)=\ell\) implique que pour toute suite \( x_n\to a\) nous avons \( f(x_n)\to \ell\). Cela est en particulier vrai pour la suite \( n\mapsto \gamma(t_n)\). Donc :
            \begin{equation}
                \lim_{n\to \infty} f\big( \gamma(t_n) \big)=\ell,
            \end{equation}
            ce qu'il fallait prouver.
        \item[Réciproque]

            Pour les mêmes raisons de caractérisation séquentielle que précédemment, il faut prouver que \( \lim_{n\to \infty} f(x_n)=\ell\) pour tout suite \( x_n\to a\).
            
            \begin{subproof}
                \item[Un chemin]

                    Soit la fonction \( \gamma\colon \eR\to \eR^m\) affine par morceaux et telle que 
                    \begin{equation}
                        \gamma\left( \frac{1}{ n } \right)=x_n.
                    \end{equation}
                    Nous prolongeons \( \gamma\) par \( \gamma(t)=a\) pour \( t\leq 0\).

                \item[\( \gamma(t)\to a\)]

                    Nous montrons que \( \lim_{t\to 0} \gamma(t)=a\). Soient \( \epsilon>0\) et \( N\) tel que \( x_n\in B(a,\epsilon)\) pour tout \( n\geq N\). Si \( t<\frac{1}{ N }\) alors \( t\in\mathopen[ \frac{1}{ k+1 } , \frac{1}{ k } \mathclose]\) pour un certain \( k>N\). Donc
                    \begin{equation}
                        \gamma(t)\in\mathopen[ \gamma(\frac{1}{ k+1 }) , \gamma(\frac{1}{ k }) \mathclose]
                    \end{equation}
                    et donc \( \gamma(t)\in\mathopen[ x_{k+1} , x_k \mathclose]\) parce que \( \gamma\) est formé de ces segments de droites. Mais comme \( B(a,\epsilon)\) est convexe\footnote{C'est la proposition \ref{PROPooUQLUooDQfYLT}.}, nous avons 
                    \begin{equation}
                        \gamma(t)\in\mathopen[ x_{k+1} , x_k \mathclose]\subset B(a,\epsilon).
                    \end{equation}
                    Nous avons donc bien \( \lim_{t\to 0} \gamma(t)=a\).
                \item[Conclusion]

                    L'hypothèse nous donne alors \( \lim_{t\to 0} (f\circ \gamma)(t)=\ell\). En particulier le critère de la caractérisation séquentielle de la limite dit que
                    \begin{equation}
                        \lim_{n\to \infty} f\big( \gamma(\frac{1}{ n }) \big)=\ell,
                    \end{equation}
                    ce qui signifie \( \lim_{n\to \infty} f(x_n)=\ell\).
            \end{subproof}
    \end{subproof}
\end{proof}

\begin{corollary}	\label{CorMethodeChemin}
	Soient $f\colon D\subset\eR^m\to \eR^n$ et $a$ un point d'accumulation de $D$. Si nous avons deux fonctions $\gamma_1,\gamma_2\colon \eR\to \eR^m$ telles que
	\begin{equation}
		\lim_{t\to 0} \gamma_1(t)=\lim_{t\to 0} \gamma_2(t)=a
	\end{equation}
	tandis que
	\begin{equation}
		\lim_{t\to 0} (f\circ \gamma_1)(t)\neq\lim_{t\to 0} (f\circ \gamma_2)(t),
	\end{equation}
	ou bien que l'une des deux limites n'existe pas, alors la limite de $f(x)$ lorsque $x\to a$ n'existe pas.
\end{corollary}

\begin{corollary}	\label{CorMethodeChemoinNegatif}
	Soient $f\colon D\subset\eR^m\to \eR^n$ et $a$ un point d'accumulation de $D$. Si il existe une fonction $\gamma\colon \eR\to \eR^m$ avec $\gamma(0)=a$ telle que la limite $\lim_{t\to 0} (f\circ\gamma)(t)$ n'existe pas, alors la limite $\lim_{x\to a} f(x)$ n'existe pas.
\end{corollary}

En ce qui concerne le calcul de limites, la méthode des chemins peut être utilisé de trois façons :
\begin{enumerate}
	\item
		Dès que l'on trouve une fonction $\gamma\colon \eR\to \eR^m$ telle que $\lim_{t\to 0} (f\circ \gamma)(t)=\ell$, alors nous savons que \emph{si la limite $\lim_{x\to a} f(x)$ existe}, alors cette limite vaut $\ell$.
	\item
		Dès que l'on a trouvé deux fonctions $\gamma_i$ qui tendent vers $a$, mais dont les limites de $\lim_{t\to 0} (f\circ\gamma_i)(t)$ sont différentes, alors la limite $\lim_{x\to a} f(x)$ n'existe pas.
	\item
		Dès qu'on trouve une chemin le long duquel il n'y a pas de limite, alors la limite n'existe pas (corolaire~\ref{CorMethodeChemoinNegatif}).
\end{enumerate}
La méthode des chemins ne permet donc pas de de calculer une limite quand elle existe. Elle permet uniquement de la «deviner», ou bien de prouver que la limite n'existe pas.

\begin{example}
	Soit à calculer
	\begin{equation}	\label{Eq3007ExempleLimiche}
		\lim_{(x,y)\to(0,0)}\frac{ x-y }{ x+y }.
	\end{equation}
	Si nous prenons le chemin $\gamma_1(t)=(t,t)$, nous avons bien $\lim_{t\to 0} \gamma_1(t)=(0,0)$, et nous avons
	\begin{equation}
		\lim_{t\to 0} (f\circ\gamma_1)(t)=\lim_{t\to 0} \frac{ t-t }{ t+t }=0.
	\end{equation}
	Donc si la limite \eqref{Eq3007ExempleLimiche} existait, elle vaudrait obligatoirement $0$. Mais si nous considérons $\gamma_2(t)=(0,t)$, nous avons
	\begin{equation}
		(f\circ\gamma_2)(t)=\frac{ -t }{ t }=-1,
	\end{equation}
	donc si la limite existe, elle doit obligatoirement valoir $-1$. Ne pouvant être égale à $0$ et à $-1$ en même temps, la limite \eqref{Eq3007ExempleLimiche} n'existe pas.
\end{example}

%+++++++++++++++++++++++++++++++++++++++++++++++++++++++++++++++++++++++++++++++++++++++++++++++++++++++++++++++++++++++++++
\section{Dérivée directionnelle}
%+++++++++++++++++++++++++++++++++++++++++++++++++++++++++++++++++++++++++++++++++++++++++++++++++++++++++++++++++++++++++++

Nous sommes capables de dériver une fonction de deux variables $f(x,y)$ par rapport à $x$ et par rapport à $y$. C'est-à-dire que nous sommes capables de donner la variation de la fonction lorsqu'on bouge le long des axes horizontal et vertical. Il est évidemment souhaitable de parler de la variation de la fonction lorsqu'on se déplace le long d'autre droites.

Soit donc $u=\begin{pmatrix}
    u_1    \\
    u_2
\end{pmatrix}$ un vecteur unitaire (c'est-à-dire $u_1^2+u_2^2=1$), et considérons la fonction de une variable
\begin{equation}
    \begin{aligned}
        \varphi\colon \eR&\to \eR \\
        t&\mapsto f(a+tu_1,b+tu_2).
    \end{aligned}
\end{equation}
La fonction $\varphi$ n'est rien d'autre que la fonction $f$ vue le long de la droite de direction donnée par le vecteur $u$. Nous pouvons aussi l'écrire $\varphi(t)=f(p+tu)$.

Soit $f\colon \eR^2\to \eR$ une fonction de deux variables et soit $(a,b)\in\eR^2$. La façon la plus naturelle de définir une dérivée à deux variables est de considérer les \defe{dérivées partielles}{dérivée!partielle} définies par
\begin{equation}
    \begin{aligned}[]
        \frac{ \partial f }{ \partial x }(a,b)&=\lim_{x\to a} \frac{ f(x,b)-f(a,b) }{ x-a }\\
        \frac{ \partial f }{ \partial y }(a,b)&=\lim_{y\to b} \frac{ f(a,y)-f(a,b) }{y-b}.
    \end{aligned}
\end{equation}
Ces nombres représentent la façon dont le nombre $f(x,y)$ varie lorsque soit seul $x$ varie soit seul $y$ varie. Les dérivées partielles se calculent de la même façon que les dérivées normales. Pour calculer $\partial_xf$, on fait «comme si» $y$ était une constante, et pour calculer $\partial_yf$, on fait comme si $x$ était une constante.

%---------------------------------------------------------------------------------------------------------------------------
                    \subsection{Dérivée partielle et directionnelles}
%---------------------------------------------------------------------------------------------------------------------------

Soit une fonction $f:A\subset \mathbb{R}^n \rightarrow \mathbb{R}^m$. Si $n\neq 1$, la notion de \emph{dérivée} de la fonction $f$ n'a plus de sens puisqu'on ne peut plus parler de pente de \emph{la} tangente au graphe de $f$ en un point. On introduit alors quelques notions qui feront, en dimension quelconque, le même travail que la dérivée en dimension un : les dérivées directionnelles et la différentielle. Nous allons voir qu'en dimension un, la différentielle coïncide avec la dérivée.

\begin{definition}
Soit $f$ une application de $U\subset\eR^m$ dans $\eR$, $a$ un point dans $U$ et $v$ un vecteur de $\eR^m$. On dit que $f$ admet une \defe{dérivée suivant le vecteur $v$ au point $a$}{dérivée!directionnelle} si la fonction $t\mapsto f(a+tv)$ admet une dérivée en $t=0$. La  dérivée de $f$ suivant le vecteur $v$ au point $a$ est alors cette dérivée, et $f$ est dite dérivable suivant $v$ en $a$,
\[
\partial_v f(a)=\lim_{
  \begin{subarray}{l}
    t\to 0\\ t\neq 0
  \end{subarray}
 }\frac{f(a+tv)-f(a)}{t}.
\]
\end{definition}

\begin{definition}
  La fonction $f:U\subset\eR^m\to \eR^n$ de composantes $(f_1,\ldots, f_n)$, est dite \defe{dérivable suivant $v$ au point $a$}{} si toute ses composantes $f_i$, $i=1,\ldots, n$ sont dérivables suivant $v$ au point $a$. Dans ce cas, nous écrivons
  \begin{equation}
	\partial_v f(a)=\left(\partial_v f_1(a), \ldots, \partial_v f_n(a)\right).
  \end{equation}
\end{definition}

Géométriquement, il s'agit du taux de variation instantané de $f$ en $a$ dans la direction du vecteur $u$, c'est-à-dire de la pente de la tangente dans la direction du vecteur $u$ au graphe de $f$ au point $(a, f(a))$.

\begin{remark}
    De nombreuses sources parlent de de dérivée \defe{dans la direction} du vecteur $v$ en définissant (avec une certaine raison) une \defe{direction}{direction} dans $\eR^m$ comme étant un vecteur de norme $1$.

    Ces personnes ne définissent alors \( \partial_uf\) que pour \( \| u \|=1\). Pourquoi ? Le but de la dérivée directionnelle dans la direction $u$ est de savoir à quelle vitesse la fonction monte lorsque l'on se déplace en suivant la direction $u$. Cette information n'aura un caractère « objectif » que si l'on avance à une vitesse donnée. En effet, si on se déplace deux fois plus vite, la fonction montera deux fois plus vite. Par convention, on demande alors d'avancer à vitesse \( 1\).

    Ici, pour être plus souple en termes de notations et de manipulations, nous définissons \( \partial_uf\) pour tout \( u\) (non nul). Nous devons cependant garder en tête que le nombre \( (\partial_vf)(a)\) ne peut pas être interprété comme étant une «vitesse de croissance de \( f\) en \( a\)» de façon trop sérieuse.
\end{remark}

\subsubsection*{Cas particulier où $n=2$:} $a = (a_1, a_2)$, $u =
(u_1,u_2)$ et
$$\frac{\partial f}{\partial u}(a_1, a_2) = \lim_{t\rightarrow
0}\frac{f(a_1+tu_1,a_2+tu_2) - f(a_1, a_2)}{t}$$

Un cas particulier des dérivées directionnelles est la dérivée partielle. Si nous considérons la base canonique $e_i$ de $\eR^n$, nous notons
\begin{equation}
    \frac{ \partial f }{ \partial x_i }=\frac{ \partial f }{ \partial e_i }.
\end{equation}
Dans le cas d'une fonction à deux variables, nous avons donc les deux dérivées partielles
\begin{equation}
    \begin{aligned}[]
        \frac{ \partial f }{ \partial x }(a)&&\text{et}&&\frac{ \partial f }{ \partial y }(a)
    \end{aligned}
\end{equation}
qui correspondent aux dérivées directionnelles dans les directions des axes. Ces deux nombres représentent de combien la fonction $f$ monte lorsqu'on part de $a$ en se déplaçant dans le sens des axes $X$ et $Y$.

%///////////////////////////////////////////////////////////////////////////////////////////////////////////////////////////
                    \subsubsection{Quelques propriétés et notations}
%///////////////////////////////////////////////////////////////////////////////////////////////////////////////////////////

\begin{enumerate}
\item
 $\forall \alpha \in \mathbb{R}$,
si $v = \alpha\,u$, alors $\frac{\partial f}{\partial v}(a) =
\alpha\,\frac{\partial f}{\partial u}(a)$.
\item Si on prend $u=e_j$ le $j$ème vecteur de la base canonique de
$\mathbb{R}^n$, alors
$$\frac{\partial f}{\partial e_j}(a) = \frac{\partial f}{\partial
x_j}(a)$$ c'est-à-dire que la dérivée de $f$ au point $a$ dans la
direction $e_j$ est la dérivée partielle de $f$ par rapport à sa
$j$ème variable.

\item
Une fonction peut être dérivable dans certaines directions
mais pas dans d'autres (rappelez vous que si la limite à droite est
différente de la limite à gauche, la limite n'existe pas).

\item
Même si une fonction est dérivable en un point dans toutes les
directions, on n'est pas sûr qu'elle soit continue en ce point. La
dérivabilité directionnelle n'est donc pas une notion suffisante
pour assurer la continuité. C'est pourquoi on introduit le concept
de \emph{différentiabilité}.
\end{enumerate}

\begin{proposition}
Soit $u$ un vecteur de norme $1$ dans $\eR^m$ et soit $v=\lambda u$, avec $\lambda$ dans $\eR$. La fonction $f$ est dérivable suivant $v$ au point $a$ si et seulement si $f$ est dérivable suivant $u$ au point $a$, en outre
\[
\partial_v f(a)=\lambda\partial_u f(a).
\]
\end{proposition}
\begin{proof}
  \begin{equation}
    \begin{aligned}
  \partial_v f(a)=&\lim_{\begin{subarray}{l}
     t\to 0\\ t\neq 0
    \end{subarray}}\frac{f(a+tv)-f(a)}{t}=\lim_{\begin{subarray}{l}
     t\to 0\\ t\neq 0
    \end{subarray}}\frac{f(a+t\lambda u)-f(a)}{t}=\\
&=\lambda \lim_{\begin{subarray}{l}
    t\to 0\\ t\neq 0
  \end{subarray}}\frac{f(a+t\lambda u)-f(a)}{\lambda t}=\lambda \partial_u f(a).
    \end{aligned}
  \end{equation}
\end{proof}
\begin{definition}
Soit $f$ une application de $U\subset\eR^m$ dans $\eR$. On appelle \defe{dérivées partielles de $f$ au point $a$}{dérivée!partielle} les dérivées de $f$ suivant les vecteurs de base $e_1,\ldots,e_m $ au point $a$, si elles existent.
\end{definition}
Si $m=2,3$ on peut utiliser la notation $f_x$, $\partial_x$  ou $\partial_1$ pour la dérivée partielle suivant $e_1$, $f_y$, $\partial_y$  ou $\partial_2$  pour la dérivée partielle suivant $e_2$ et $f_z$,  $\partial_z$  ou $\partial_3$  pour la dérivée partielle suivant $e_3$. En général, nous écrivons $\partial_i$ pour noter la la dérivée partielle suivant $e_i$.

Des exemples faisons intervenir les fonctions trigonométriques, exponentielles et logarithme sont les exemples~\ref{EXooETZYooYsKPDJ},~\ref{EXooGMRIooUucRez}.

La fonction d'une seule variable qu'on obtient à partir de $f$ en fixant les $p-1$ variables  $x_1,\ldots, x_{i-1}, x_{i+1}, \ldots, x_p$ et qui associe à $x_i$ la valeur $f(x_1,\ldots, x_{i-1}, x_i, x_{i+1}, \ldots, x_p)$, est appelée $x_i$-ème \defe{section}{section} de $f$ en $x_1,\ldots, x_{i-1}, x_{i+1}, \ldots, x_p$. L'$i$-ème dérivée partielle de $f$ au point $a=(x_1,\ldots,x_m)$ est la dérivée de l'$i$-ème section de $f$ au point $x_i$. En pratique, pour calculer les dérivées partielles d'une fonction on fait une dérivation par rapport à la variable choisie en considérant les  autres variables comme des constantes.

\begin{example}
	Considérons la fonction $f(x,y)=2xy^2$. Lorsque nous calculons $\partial_xf(x,y)$, nous faisons comme si $y$ était constant. Nous avons donc $\partial_xf(x,y)=2y^2$. Par contre lors du calcul de $\partial_yf(x,y)$, nous prenons $x$ comme une constante. La dérivée de $y^2$ par rapport à $y$ est évidemment $2y$, et par conséquent, $\partial_yf(x,y)=4xy$.
\end{example}

\begin{definition}
  Soit $f$ une application de $U\subset\eR^m$ dans $\eR$ et $u$ un vecteur de $\eR^m$. La fonction $f$ est \defe{dérivable sur $U$ suivant le vecteur $u$}{}, si $f$ est dérivable  suivant le vecteur $u$ en tout point de $U$.
\end{definition}

Pour les fonctions d'une seule variable la dérivabilité en un point $a$ implique la continuité en $a$. Cela n'est pas vrai pour les fonctions de plusieurs variables : il existe des fonctions $f$  qui sont dérivables suivant tout vecteur au point $a$ sans pour autant être continue en $a$.

  \begin{example}
    Considérons la fonction $f:\eR^2\to \eR$
    \begin{equation}
      f(x,y)=\left\{
      \begin{array}{ll}
        \frac{x^2y}{x^4+y^2} \qquad&\textrm{si } (x,y)\neq (0,0),\\
        0     & \textrm{sinon}.
      \end{array}
      \right.
    \end{equation}
Pour voir que $f$ n'est pas continue en $(0,0)$ il suffit de calculer la limite de $f$ restreinte à la parabole $y=x^2$
\[
\lim_{x\to 0} f(x,x^2)=\frac{1}{2} \neq 0.
\]
Pourtant la fonction $f$ est dérivable en $(0,0)$ dans toutes les directions. En effet, soit $v=(v_1,v_2)$. Si $v_2\neq 0$, alors
\[
\partial_v f(a)=\lim_{\begin{subarray}{l}
			t\to 0\\ t\neq 0
  		\end{subarray}}
  		\frac{t^3v_1^2v^2}{t^5 v_1^4+ t^3v_2^2}=\frac{v_1^2}{v_2},
\]
tandis que si $v_2=0$, alors la valeur de $f(tv_1, 0)$  est $0$ pour tout $t$ et $v_1$, donc la dérivée partielle de $f$ par rapport à $x$ en l'origine existe et est nulle.
\end{example}

\begin{example}
    Pour une fonction réelle à variable réelle, la dérivabilité entraine la continuité. Il n'en va pas de même pour les fonctions à plusieurs variables, comme le montre l'exemple suivant :
    \begin{equation}
        f(x,y)=\begin{cases}
            0    &   \text{si } x=0\\
            \frac{ y }{ x }\sqrt{x^2+y^2}    &    \text{sinon.}
        \end{cases}
    \end{equation}
    Nous avons tout de suite
    \begin{equation}
        \frac{ \partial f }{ \partial y }(0,0)=0.
    \end{equation}
    De plus si \( u_x\neq 0\) nous avons
    \begin{equation}
            \frac{ \partial f }{ \partial u }(0,0)=\frac{ u_y }{ u_x }\| u \|.
    \end{equation}
    Donc toutes les dérivées directionnelles de \( f\) en \( (0,0)\) existent alors que la fonction n'y est manifestement pas continue. En effet sous forme polaire,
    \begin{equation}
        f(r,\theta)=\frac{ r\sin(\theta) }{ \cos(\theta) },
    \end{equation}
    et quelle que soit la valeur de \( r\), en prenant \( \theta\) suffisamment proche de \( \pi/2\), la fraction peut être arbitrairement grande.

    Nous verrons par la proposition~\ref{diff1} que la différentiabilité d'une fonction implique sa continuité.
\end{example}

\begin{theorem}[Accroissement finis pour les dérivées suivant un vecteur]\label{val_medio_1}		\index{accroissements finie!dérivée partielle}
    Soit $U$ un ouvert dans $\eR^m$ et soit $f:U\to\eR^n$ une fonction. Soient $a$ et $b$ deux points distincts dans $U$, tels que le segment\footnote{Définition~\ref{DefLISOooDHLQrl}.} $[a,b]$ soit contenu dans $U$. Soit $u$ le vecteur
	\[
		u=\frac{b-a}{\|b-a\|_m}.
	\]
	Si $\partial_u f(x)$ existe pour tout $x$ dans $[a,b]$ on a
	\[
		\|f(b)-f(a)\|_n\leq \sup_{x\in[a,b]}\|\partial_uf(x)\|_n\|b-a\|_m.
	\]
\end{theorem}

\begin{proof}
	Nous considérons la fonction $g(t)=f\big( (1-t)a-tb \big)$. Elle décrit la droite entre $a$ et $b$ parce que $g(0)=a$ et $g(1)=b$. En ce qui concerne la dérivée,
	\begin{equation}
		\begin{aligned}[]
			g'(t)&=\lim_{h\to 0} \frac{ g(t+h)-g(t) }{ h }\\
			&=\lim_{h\to 0} \frac{ f\big( (1-t-h)a-(t+h)b \big) }{ h }\\
			&=\lim_{h\to 0} \frac{ f\big( a+(t+h)(b-a) \big)-f\big( a+t(b-a) \big) }{ h }\\
			&=\frac{ \partial f }{ \partial u }\big( a+t(b-a) \big)\| b-a \|.
		\end{aligned}
	\end{equation}
	Le dernier facteur $\| b-a \|$ apparaît pour la normalisation du vecteur $u$. En effet dans la limite, il apparaît $h(b-a)$, ce qui donnerait la dérivée le long de $b-a$, tandis que $u$ vaut $(b-a)/\| b-a \|$.

	Par le théorème des accroissements finis pour $g$, il existe $t_0\in\mathopen] 0 , 1 \mathclose[$ tel que
	\begin{equation}
		g(1)=g(0)+g'(t_0)(1-0).
	\end{equation}
	Donc
	\begin{equation}
		\| g(1)-g(0) \|\leq\sup_{t_0}\| g'(t_0) \|=\sum_{t_0\in\mathopen] 0 , 1 \mathclose[}\left\| \frac{ \partial f }{ \partial u }(a+t_0(b-a)) \right\|\| b-a \|.
	\end{equation}
	Mais lorsque $t_0$ parcours $\mathopen] 0 , 1 \mathclose[$, le point $a+t_0(b-a)$ parcours le segment $\mathopen] a , b \mathclose[$, d'où le résultat.
\end{proof}

\begin{corollary}
	Dans les mêmes hypothèses, si $n=1$, alors il existe $\bar x $ dans $]a,b[$ tel que
	\[
		f(b)-f(a)=\partial_uf(\bar x)\|b-a\|_m.
	\]
\end{corollary}

\begin{definition}
    Le nombre
    \begin{equation}
        \lim_{t\to 0} \frac{ f\big( a+tu_1,b+tu_2 \big)-f(a,b) }{ t }
    \end{equation}
    est la \defe{dérivée directionnelle}{dérivée!directionnelle} de $f$ dans la direction de $u$ au point $(a,b)$. Il sera noté
    \begin{equation}
        \frac{ \partial f }{ \partial u }(a,b),
    \end{equation}
    ou plus simplement $\partial_uf(a,b)$.
\end{definition}

Lorsque $f$ est différentiable, la dérivée directionnelle est donnée par
\begin{equation}        \label{EqDerDirnablau}
    \frac{ \partial f }{ \partial u }(p)=\nabla f(p)\cdot u.
\end{equation}

%---------------------------------------------------------------------------------------------------------------------------
\subsection{Gradient : direction de plus grande pente}
%---------------------------------------------------------------------------------------------------------------------------

Étant donné que $u$ est de norme $1$, l'inégalité de Cauchy-Schwarz donne
\begin{equation}
    \big| \nabla f(a,b)\cdot \begin{pmatrix}
        u_1    \\
        u_2
    \end{pmatrix}\big|\leq \| \nabla f(a,b) \|.
\end{equation}
Donc
\begin{equation}
    -\| \nabla f(p) \|\leq \nabla f(p)\cdot u\leq\| \nabla f(p) \|.
\end{equation}
La norme de la dérivée directionnelle (qui est la valeur absolue du nombre au centre) est donc «coincée» entre $-\| \nabla f(p) \|$ et $\| \nabla f(p) \|$. Prenons par exemple
\begin{equation}
    u=\frac{ \nabla f(p) }{ \| \nabla f(p) \| }.
\end{equation}
Dans ce cas, nous avons exactement
\begin{equation}
    \nabla f(p)\cdot u=\| \nabla f(p) \|,
\end{equation}
qui est la valeur maximale que la dérivée directionnelle peut prendre.

La direction du gradient est donc la direction suivant laquelle la dérivée directionnelle est la plus grande. Pour la même raison, la dérivée directionnelle est la plus petite dans le sens opposé au gradient.

En termes bien clairs : lorsqu'on veut aller le plus vite possible au ski, on prend la direction du gradient de la piste de ski. C'est dans cette direction que ça descend le plus vite. Dans quelle direction vont les débutants ? Ils vont perpendiculairement à la pente (ce qui ennuie tout le monde, mais c'est un autre problème). Les débutants vont donc dans la direction perpendiculaire au gradient. Prenons donc $u\perp \nabla f(p)$ et calculons la dérivée directionnelle de $f$ dans la direction $u$ en utilisant la formule~\ref{EqDerDirnablau} :
\begin{equation}
    \frac{ \partial f }{ \partial u }(p)=\nabla f(p)\cdot u=0
\end{equation}
parce que nous avons choisi $u\perp \nabla f(p)$. Nous voyons donc que les débutants en ski ont eu la bonne intuition que la direction dans laquelle la piste ne descend pas, c'est la direction perpendiculaire au gradient.

C'est aussi pour cela que l'on a tendance à faire du zig-zag à vélo lorsqu'on monte une pente très forte et qu'on est fatigué. C'est toujours pour cela que les routes de montagne font de longs lacets. La montée est moins rude en suivant une direction proche d'être perpendiculaire au gradient !

\begin{theorem}
    Le gradient des fonctions suit à peu près les mêmes règles que les dérivées. Soient $f$ et $g$ deux fonctions différentiables. Nous avons entre autres
    \begin{enumerate}
        \item
            $\nabla(f+g)=\nabla f+\nabla g$;
        \item
            $\nabla(fg)(a,b)=g(a,b)\nabla f(a,b)+f(a,b)\nabla g(a,b)$;
        \item
            Dès que $g(a,b)\neq 0$, nous avons
            \begin{equation}
                \nabla\frac{ f }{ g }=\frac{ g(a,b)\nabla f(a,b)-f(a,b)\nabla g(a,b) }{ g(a,b)^2 }.
            \end{equation}
    \end{enumerate}
\end{theorem}

%---------------------------------------------------------------------------------------------------------------------------
\subsection{Gradient : orthogonal au plan tangent}
%---------------------------------------------------------------------------------------------------------------------------

Vu que le gradient d'une fonction est la direction de plus grande pente et que le plan tangent est le plan de plus petite pente, quoi de plus naturel que de penser que le gradient est orthogonal au plan tangent ?

\begin{lemma}
    Soit \( \phi\colon \eR^n\to \eR\) une fonction de classe \( C^1\) et la partie
    \begin{equation}
        \Gamma=\{ x\in \eR^n\tq \phi(x)=C \}
    \end{equation}
    pour une certaine constante \( C\).

    Soit \( x_0\in \Gamma\). Le gradient de \( \phi\) en \( x_0\) est orthogonal au plan tangent à \( \Gamma\) en \( x_0\).
\end{lemma}

\begin{proof}
    Un vecteur tangent à \( \Gamma\) en \( x_0\) est de la forme \( \gamma'(0)\) où \( \gamma\colon \eR \to \Gamma\) vérifie \( \gamma(0)=x_0\). Vu que \( \phi\) est constante sur \( \Gamma\) nous avons
    \begin{equation}
        \Dsdd{ \phi\big( \gamma(s) \big) }{s}{0}=0,
    \end{equation}
    ce qui donne
    \begin{equation}
        \sum_i\frac{ \partial \phi }{ \partial x_i }\big( \gamma(0) \big)\gamma_i'(0)=0,
    \end{equation}
    ce qui signifie exactement \( \langle (\nabla\phi)(x_0), \gamma'(0)\rangle=0\). Le vecteur \( (\nabla\phi)(x_0)\) est donc perpendiculaire à tout vecteur tangent de \( \Gamma\) en \( x_0\).
\end{proof}

%+++++++++++++++++++++++++++++++++++++++++++++++++++++++++++++++++++++++++++++++++++++++++++++++++++++++++++++++++++++++++++
\section{Dérivée directionnelle de fonctions composées}		\label{SecDerDirFnComp}
%+++++++++++++++++++++++++++++++++++++++++++++++++++++++++++++++++++++++++++++++++++++++++++++++++++++++++++++++++++++++++++

Nous savons déjà comment dériver les fonctions composées de $\eR$ dans $\eR$. Si nous avons deux fonctions $f\colon \eR\to \eR$ et $u\colon \eR\to \eR$, nous formons la composée $\varphi=f\circ u\colon \eR\to \eR$ dont la dérivée vaut
\begin{equation}
	\varphi'(a)=f'\big( u(a) \big)u'(a).
\end{equation}

Considérons maintenant le cas un peu plus compliqué des fonctions $f\colon \eR\to \eR$ et $u\colon \eR^2\to \eR$, et de la composée
\begin{equation}
	\begin{aligned}
		\varphi\colon \eR^2&\to \eR \\
		\varphi(x,y)&= f\big( u(x,y) \big).
	\end{aligned}
\end{equation}
Afin de calculer la dérivée partielle de $\varphi$ par rapport à $x$, nous admettons que pour tout $a$, $b$ et $t$, il existe $c\in\mathopen[ a , a+t \mathclose]$ tel que
\begin{equation}
	u(a+t,b)=u(a,b)+t\frac{ \partial u }{ \partial x }(c,b).
\end{equation}
Cela est une généralisation immédiate du théorème~\ref{ThoAccFinis}. Nous devons calculer
\begin{equation}		\label{EqPremPasDiffxvp}
	\frac{ \partial \varphi }{ \partial x }(a,b)=\lim_{t\to 0} \frac{ \varphi(a+t,b)-\varphi(a,b) }{ t }=\lim_{t\to 0} \frac{ f\big( u(a+t,b) \big)-g\big( u(a,b) \big) }{ t }.
\end{equation}
Étant donné l'hypothèse que nous avons faite sur $u$, nous avons
\begin{equation}
	f\big( u(a+t,b) \big)=f\big( u(a,b)+t\frac{ \partial u }{ \partial x }(c,b) \big).
\end{equation}
En utilisant le théorème des accroissements finis pour $f$, nous avons un point $d$ entre $u(a,b)$ et $u(a,b)+t\frac{ \partial u }{ \partial x }(c,b)$ tel que
\begin{equation}
	f\big( u(a,b)+t\frac{ \partial u }{ \partial x }(c,b) \big)=f\big( u(a,b) \big)+t\frac{ \partial u }{ \partial x }(c,b)f'(d).
\end{equation}
Le numérateur de \eqref{EqPremPasDiffxvp} devient donc
\begin{equation}
	t\frac{ \partial u }{ \partial x }(c,b)f'(d).
\end{equation}
Certes les points $c$ et $d$ sont inconnus, mais nous savons que $c$ est entre $a$ et $a+t$ ainsi que $d$ se situe entre $u(a,b)$ et $u(a,b)+t\frac{ \partial u }{ \partial x }(c,b)$. Lorsque nous prenons la limite $t\to 0$, nous avons donc $\lim_{t\to 0} c=a$ et $\lim_{t\to 0} d=u(a,b)$. Nous avons alors
\begin{equation}
	\lim_{t\to 0} \frac{ t\frac{ \partial u }{ \partial x }(c,b)f'(d) }{ t }=\frac{ \partial u }{ \partial x }(a,b)f'\big( u(a,b) \big).
\end{equation}
La formule que nous avons obtenue (de façon pas très rigoureuse) est
\begin{equation}
	\frac{ \partial  }{ \partial x }f\big( u(x,y) \big)=\frac{ \partial u }{ \partial x }(x,y)f'\big( u(x,y) \big).
\end{equation}

Prenons maintenant un cas un peu plus compliqué où nous voudrions savoir les dérivées partielles de la fonction $\varphi$ donnée par
\begin{equation}
	\varphi(x,y,z)=f\big( u(x,y),v(x,y,z) \big)
\end{equation}
où $f\colon \eR^2\to \eR$, $u\colon \eR^2\to \eR$ et $v\colon \eR^3\to \eR$.

Commençons par la dérivée partielle par rapport à $z$. Étant donné que $\varphi$ ne dépend de $z$ que via la seconde entrée de $f$, il est normal que seule la dérivée partielle de $f$ par rapport à sa seconde entrée arrive dans la formule :
\begin{equation}
	\frac{ \partial \varphi }{ \partial z }(x,y,z)=\frac{ \partial f }{ \partial v }\big( u(x,y),v(x,y,z) \big)\frac{ \partial v }{ \partial z }(x,y,z).
\end{equation}
La dérivée partielle par rapport à $y$ demande de tenir compte en même temps de la façon dont $f$ varie avec sa première entrée et la façon dont elle varie avec sa seconde entrée; cela nous fait deux termes :
\begin{equation}
	\frac{ \partial \varphi }{ \partial y }(x,y,z)=\frac{ \partial f }{ \partial u }\big( u(x,y),v(x,y,z) \big)\frac{ \partial u }{ \partial y }(x,y)+\frac{ \partial f }{ \partial v }\big( u(x,y),v(x,y,z) \big)\frac{ \partial v }{ \partial y }(x,y,z).
\end{equation}


Cette formule a une interprétation simple. Lançons un caillou du sommet d'une falaise. Son mouvement est une chute libre avec une vitesse initiale horizontale :
\begin{subequations}
	\begin{numcases}{}
		x(t)=v_0t\\
		y(t)=h_0-\frac{ gt^2 }{ 2 }
	\end{numcases}
\end{subequations}
où $v_0$ est la vitesse initiale horizontale et $h_0$ est la hauteur de la falaise. Si nous sommes intéressés à la distance entre le caillou et le bas de la falaise (point $(0,0)$), le théorème de Pythagore nous dit que
\begin{equation}
	d(t)=\sqrt{x^2(t),y^2(t)}.
\end{equation}
Pour trouver la variation de la distance par rapport au temps il faut savoir de combien la distance varie lorsque $x$ varie et multiplier par la variation de $x$ par rapport à $t$, et puis faire la même chose avec $y$.

\begin{theorem}		\label{ThoDerDirFnComp}
    Soit $g\colon \eR^m\to \eR^n$ une fonction différentiable\quext{Je ne suis pas certain que l'hypothèse de différentiabilité soit obligatoire.} en $a$, et $f\colon \eR^n\to \eR^p$ une fonction différentiable en $g(a)$. Si nous définissons $\varphi(x)=(f\circ g)(x)$, alors pour tout $i=1,\ldots,m$, nous avons
	\begin{equation}
		\frac{ \partial \varphi }{ \partial x_i }(a)=\sum_{k=1}^n\frac{ \partial f }{ \partial y_k }\big( g(a) \big)\frac{ \partial g }{ \partial x_i }
	\end{equation}
	où $\frac{ \partial f }{ \partial y_k }$ dénote la dérivée partielle de $f$ par rapport à sa $k$-ième variable.
\end{theorem}

Donnons un exemple d'utilisation de cette formule. Si
\begin{equation}
	\begin{aligned}[]
		g\colon \eR^2\to \eR^3\\
		f\colon \eR^3\to \eR,
	\end{aligned}
\end{equation}
nous avons $\varphi\colon \eR^2\to \eR$. Les dérivées partielles de $\varphi$ sont données par les formules
\begin{equation}
	\frac{ \partial \varphi }{ \partial x }(x,y)=\frac{ \partial f }{ \partial x_1 }\big( g(x,y) \big)\frac{ \partial g_1 }{ \partial x }(x,y)+\frac{ \partial f }{ \partial x_2 }\big( g(x,y) \big)\frac{ \partial g_2 }{ \partial y }(x,y)+\frac{ \partial f }{ \partial x_3 }\big( g(x,y) \big)\frac{ \partial g_3 }{ \partial x }(x,y)
\end{equation}
et
\begin{equation}
	\frac{ \partial \varphi }{ \partial y }(x,y)=\frac{ \partial f }{ \partial x_1 }\big( g(x,y) \big)\frac{ \partial g_1 }{ \partial y }(x,y)+\frac{ \partial f }{ \partial x_2 }\big( g(x,y) \big)\frac{ \partial g_2 }{ \partial y }(x,y)+\frac{ \partial f }{ \partial x_3 }\big( g(x,y) \big)\frac{ \partial g_3 }{ \partial y }(x,y)
\end{equation}
Notez que les dérivées de $\varphi$ et des composantes de $g$ sont calculées en $(x,y)$, tandis que celles de $f$ sont calculées en $g(x,y)$.

% This is part of Mes notes de mathématique
% Copyright (c) 2006-2019
%   Laurent Claessens, Carlotta Donadello
% See the file fdl-1.3.txt for copying conditions.

%+++++++++++++++++++++++++++++++++++++++++++++++++++++++++++++++++++++++++++++++++++++++++++++++++++++++++++++++++++++++++++
\section{Formes différentielles}
%+++++++++++++++++++++++++++++++++++++++++++++++++++++++++++++++++++++++++++++++++++++++++++++++++++++++++++++++++++++++++++
\label{SecFormDiffRappel}

Nous parlerons de formes différentielles exactes et fermées dans la section~\ref{DefEFKQmPs}.

%---------------------------------------------------------------------------------------------------------------------------
\subsection{Décomposition dans la base duale}
%---------------------------------------------------------------------------------------------------------------------------

\begin{definition}      \label{DEFooMGXSooWioKie}
	Soit $U$, un ouvert dans $\eR^n$. Une $1$-\defe{forme différentielle}{forme!différentielle} $\omega$ sur $U$ est une application
	\begin{equation}
		\begin{aligned}
				\omega\colon U&\to (\eR^n)^* \\
				x&\mapsto \omega_x.
			\end{aligned}
		\end{equation}
\end{definition}

\begin{remark}
	L'ensemble des $1$-formes différentielles forment un espace vectoriel avec les définitions
	\begin{equation}
		\begin{aligned}[]
			(\lambda\omega)_x(v)&=\lambda\omega_x(v)\\
			(\omega+\mu)_x(v)&=\omega_x(v)+\mu_x(v).
		\end{aligned}
	\end{equation}
\end{remark}

Nous connaissons la de $(\eR^n)^*$ définie en \ref{DEFooTMSEooZFtsqa}. Nous allons noter ces formes par $dx_i$ :
\begin{equation}        \label{EQooITHKooDzigPY}
	\begin{aligned}[]
		e^*_1&=dx_1\colon v\mapsto v_1	\\
			&\vdots			\\
		e^*_n&=dx_n\colon v\mapsto v_n
	\end{aligned}
\end{equation}
Toute forme différentielle s'écrit
\begin{equation}
  \omega_x = \sum_{i=0}^n a_i(x) d x_i
\end{equation}
où $a_1,\ldots,a_n$ sont les composantes de $\omega$ dans la base usuelle, et sont des fonctions à valeurs réelles.

\begin{lemma}
    Une $1$-forme différentielle est \defe{continue}{continue!forme différentielle} si les fonctions $a_i$ sont continues. La forme sera $C^k$ quand les $a_i$ seront $C^k$.
\end{lemma}

Pour un vecteur $v = (v_1,\ldots,v_n)$ on a donc par définition de $d x_i$
\begin{equation}
  \omega_x (v) = \sum_{i=0}^n a_i(x) v_i.
\end{equation}
Ces fonctions $a_i$ peuvent être trouvées en appliquant $\omega$ aux éléments de la base canonique de $\eR^n$ :
\begin{equation}
	a_j(x)=\omega_x(e_j)
\end{equation}
parce que $\omega_x(e_j)=\sum_ia_i(x)dx_i(e_i)=\sum_ia_i(x)\delta_{ij}=a_j(x)$.

%---------------------------------------------------------------------------------------------------------------------------
\subsection{L'isomorphisme musical}
%---------------------------------------------------------------------------------------------------------------------------

Si $G$ est un champ de vecteur sur $\eR^n$, et si $x\in\eR^n$, nous pouvons définissons
\begin{equation}		\label{EqDefBemol}
	\begin{aligned}[]
		G^{\flat}_x\colon \eR^n&\to \eR \\
			v&\mapsto \langle G(x), v\rangle
	\end{aligned}
\end{equation}

Pour chaque $x$, l'application $G_x^{\flat}$ est une forme sur $\eR^n$, c'est-à-dire une application linéaire de $\eR^n$ vers $\eR$. Nous écrivons que
\begin{equation}
	G_x^{\flat}\in\big( \eR^n \big)^*.
\end{equation}

Nous pouvons ainsi déterminer le développement de $G^{\flat}$ dans la base des $dx_i$ en faisant le calcul
\begin{equation}
	G_x^{\flat}(e_i)=\langle G(x), e_i\rangle =G_i(x),
\end{equation}
donc les composantes de $G^{\flat}$ dans la base $dx_i$ sont exactement les composantes de $G$ dans la base $e_i$ :
\begin{equation}
	G^{\flat}_x=G_1(x)dx_1+\cdots+G_n(x)dx_n.
\end{equation}

La construction inverse existe également. Si $\omega$ est une $1$-forme différentielle, nous pouvons définir le champ de vecteur $\omega^{\sharp}$ par la formule (implicite)
\begin{equation}
	\omega_x(v)=\langle \omega^{\sharp}(x), v\rangle
\end{equation}
pour tout $v\in\eR^n$. Par définition, $(\omega^{\sharp})^{\flat}=\omega$.

\begin{lemma}
    En composantes nous avons :
	\begin{equation}
		\omega^{\sharp}(x)=\big( a_1(x),\ldots,a_n(x) \big).
	\end{equation}
	Si $G$ est un champ de vecteurs, alors $(G^{\flat})^{\sharp}=G$.
\end{lemma}

%+++++++++++++++++++++++++++++++++++++++++++++++++++++++++++++++++++++++++++++++++++++++++++++++++++++++++++++++++++++++++++
\section{Différentielle}
%+++++++++++++++++++++++++++++++++++++++++++++++++++++++++++++++++++++++++++++++++++++++++++++++++++++++++++++++++++++++++++

Nous avons déjà donné une définition abstraite de la différentielle dans la définition \ref{DefDifferentiellePta}. Nous en voyons maintenant quelques motivations dans le cas de fonctions sur \( \eR^2\) ou \( \eR^n\).

%---------------------------------------------------------------------------------------------------------------------------
\subsection{Exemples introductifs}
%---------------------------------------------------------------------------------------------------------------------------
\label{SEBSECooLPRQooJRQCFL}

La notion de dérivée est associée à la recherche de la droite tangente à une courbe. Reprenons rapidement le cheminement. La dérivée de $f\colon \eR\to \eR$ au point $a$ est un nombre $f'(a)$, qui définit donc une application linéaire dont le coefficients angulaire est $f'(a)$, et que nous notons $df_a$ :
\begin{equation}
    \begin{aligned}
        df_a\colon \eR&\to \eR \\
        u&\mapsto f'(a)u.
    \end{aligned}
\end{equation}
La droite donnée par l'équation
\begin{equation}
    y(a+u)=f'(a)u
\end{equation}
est parallèle à la tangente en $a$. Pour trouver la tangente, il suffit de la décaler de la hauteur qu'il faut. L'équation de la droite tangente au graphe de $f$ au point $\big( a,f(a) \big)$ devient
\begin{equation}        \label{EqDiffRapTgDer}
    y(x)=f(a)+f'(a)(x-a)=f(a)+df_a(x-a).
\end{equation}
Nous nous proposons de généraliser cette formule au cas de la recherche du plan tangent à une surface.

\begin{example}
    Considérons $f(x,y)=x^2y+y^2 e^{x}$. Les dérivées partielles sont
    \begin{equation}
        \begin{aligned}[]
            \frac{ \partial f }{ \partial x }&=2xy+y^2e^x\\
            \frac{ \partial f }{ \partial y }&=x^2+2ye^x.
        \end{aligned}
    \end{equation}
\end{example}

Cet exemple était l'exemple facile où tout se passe bien.

\begin{example}
    Les choses sont moins simples lorsqu'on considère la fonction suivante :
    \begin{equation}
        f(x,y)=\begin{cases}
            \frac{ xy }{ x^2+y^2 }    &   \text{si }(x,y)\neq(0,0)\\
            0    &    \text{si }(x,y)=(0,0).
        \end{cases}
    \end{equation}
    On voit que pour tout $x$ et tout $y$, nous avons $f(x,0)=f(0,y)=0$. Donc cette fonction est nulle sur les axes horizontaux et verticaux. Nous avons en particulier
    \begin{equation}
        \begin{aligned}[]
            \frac{ \partial f }{ \partial x }(0,0)&=0\\
            \frac{ \partial f }{ \partial y }(0,0)&=0.
        \end{aligned}
    \end{equation}
    Donc ces dérivées partielles existe.

    Il n'est par contre pas question de dire que cette fonction «va bien» autour du point $(0,0)$. En effet si nous regardons sa valeur sur la droite diagonale $y=x$, nous avons
    \begin{equation}
        f(x,x)=\frac{ x^2 }{ 2x^2 }=\frac{ 1 }{2}.
    \end{equation}
    Par conséquent si nous suivons la fonction le long de la droite $y=x$, la hauteur vaut $\frac{ 1 }{2}$ en permanence, sauf juste en $(0,0)$ où la fonction fait un grand plongeon !
    \begin{verbatim}
    sage: var('x,y')
    (x, y)
    sage: f(x,y)=(x*y)/(x**2+y**2)
    sage: plot3d(f,(x,-2,2),y(-2,2))
    \end{verbatim}

    D'ailleurs elle fait un plongeon le long de toutes les droites (sauf verticale et horizontale). En effet si nous regardons la fonction le long de la droite $y=mx$, nous avons
    \begin{equation}
        f(x,mx)=\frac{ mx^2 }{ x^2+m^2x^2 }=\frac{ m }{ 1+m^2 }.
    \end{equation}
    La fonction est donc \emph{constante} sur chacune de ces droites. Il n'est donc pas question de dire que cette fonction est «dérivable» en $(0,0)$, vu qu'elle fait des grands sauts dans presque toutes les directions.
\end{example}

Nous devons donc trouver mieux que les dérivées partielles pour étudier le comportement des fonctions un peu problématiques.

%---------------------------------------------------------------------------------------------------------------------------
\subsection{Différentielle}
%---------------------------------------------------------------------------------------------------------------------------

Nous nous souvenons de l'équation \eqref{EqCodeDerviffxam} qui nous dit que pour une fonction d'une variable la dérivabilité signifiait qu'il existait un nombre $\ell$ et une fonction $\alpha$ tels que
\begin{equation}
    f(x)=f(a)+\ell(x-a)+(x-a)\alpha(x-a)
\end{equation}
et $\lim_{t\to 0} \alpha(t)=0$.

En nous inspirant de cela, nous comprenons peut-être un peu le pourquoi de la définition \ref{DefDifferentiellePta}.

\begin{normaltext}
    L'objet $df_a$ est \emph{en soi} une application $df_a\colon \eR^m\to \eR^n$. Nous notons $df_a(u)$\nomenclature{$df_a(u)$}{Application de la différentielle de $f$ sur le vecteur $u$} la valeur de $df_a$ sur le vecteur $u\in\eR^m$. En particulier, l'application \( df\) est une forme différentielle au sens de la définition~\ref{DEFooMGXSooWioKie}.
\end{normaltext}

\begin{normaltext}
    Les propositions~\ref{PropExistDiffUn} et~\ref{PropExistDiffDeux} vont montrer qu'en étudiant bien les dérivées partielles, nous pouvons conclure à la différentiabilité d'une fonction.
    Attention cependant, nous verrons dans l'exemple~\ref{Exemple0046Diff} que l'existence des dérivées directionnelles partielles ne permettait pas de conclure à la différentiabilité.
\end{normaltext}

%--------------------------------------------------------------------------------------------------------------------------- 
\subsection{Matrice de la différentielle}
%---------------------------------------------------------------------------------------------------------------------------

La différentielle est une application linéaire. Elle possède donc une matrice lorsque des bases sont fixées.
\begin{proposition}     \label{PROPooBMROooThgLuU}
    Soient une application différentiable \( f\colon \eR^m\to \eR^n\) et \( a\in \eR^m\). Dans les bases canoniques de \( \eR^m\) et \( \eR^n\), la matrice de \( df_a\) est
    \begin{equation}
        (df_a)_{ij}=\frac{ \partial f_i }{ \partial x_j }(a).
    \end{equation}
\end{proposition}

\begin{proof}
    Le lien entre matrice et application linéaire est fait dans la proposition \ref{PROPooGXDBooHfKRrv}. Dans le cas des bases canoniques de \( \eR^m\) et \( \eR^n\) nous savons qu'extraire une composante revient à prendre le produit scalaire. Nous avons donc
    \begin{equation}
        (df_a)_{ij}=\big( df_a(e_j) \big)_i=df_a(e_j)\cdot e_i.
    \end{equation}
    La linéarité de la dérivation donne alors
    \begin{equation}
        (df_a)_{ij}=df_a(e_j)\cdot e_i=\Dsdd{ f(a+te_j) }{t}{0}\cdot e_i=\Dsdd{ f_i(a+te_j) }{t}{0}=\frac{ \partial f_i }{ \partial x_j }(a).
    \end{equation}
\end{proof}

%--------------------------------------------------------------------------------------------------------------------------- 
\subsection{Quelques propriétés}
%---------------------------------------------------------------------------------------------------------------------------

\begin{lemma}       \label{LEMooZSNMooCfjzOB}
    La différentielle d'une application linéaire est l'application elle-même. Plus précisément : soit une application linéaire \( f\colon E\to F\). Alors nous avons, pour tout \( a\in E\) et \( u\in E\) :
    \begin{equation}
        df_a(u)=f(u).
    \end{equation}
\end{lemma}

\begin{proof}
    Pour rappel, toujours bon à avoir en tête : \( df\colon E\to \aL(E,F)\). En posant \( T(u)=f(u)\) nous avons :
    \begin{equation}
        \lim_{h\to 0} \frac{ \| f(a+h)-f(a)- T(h) \|_F }{ \| h \|_E }=0
    \end{equation}
    parce que le numérateur est nul pour tout \( h\).
\end{proof}

\begin{lemma}[\cite{MonCerveau}]       \label{LEMooDDUZooLwXkRp}
    Soit une fonction \( g\colon \eR\to \eR\) de classe \(  C^{\infty}\). Nous posons
    \begin{equation}
        \begin{aligned}
            f\colon \eR^2&\to \eR \\
            (x,y)&\mapsto g(x). 
        \end{aligned}
    \end{equation}
    Alors \( f\) est de classe \(  C^{\infty}\) sur \( \eR^2\).
\end{lemma}

\begin{proof}
    Le problème lorsqu'il faut démontrer qu'une fonction est de classe \(  C^{\infty}\), c'est que \( d^kf\) sera une application de \( \eR^2\) vers un espace qui est un terrible emboîtement de \( \aL(\eR^2,\ldots)\). Pour traiter cette difficulté, nous considérons les espaces suivants: \( V_0=\eR\) et par récurrence \( V_{k+1}=\aL(\eR^2,V_k)\). 

    Et nous considérons également les éléments
    \begin{equation}
        \begin{aligned}
            \alpha_1\colon \eR^2&\to \eR \\
            (u,v)&\mapsto u 
        \end{aligned}
    \end{equation}
    et plus généralement \( \alpha_k\in V_k\) donné par
    \begin{equation}
        \begin{aligned}
            \alpha_k\colon \eR^2&\to V_{k-1} \\
            (u,v)&\mapsto u\alpha_{k-1}. 
        \end{aligned}
    \end{equation}
    Notons que dans l'expression \( u\alpha_{k-1}\), il s'agit d'un produit entre un scalaire \( u\in \eR\) et un vecteur \( \alpha_{k+1}\in V_{k-1}\).

    Nous prouvons maintenant par récurrence que \( d^{k}f_{(a,b)}=g^{(k)}(a)\alpha_k\), en utilisant directement la définition.

    \begin{subproof}
        \item[Initialisation]

    Pour \( k=1\), nous calculons
    \begin{equation}
        \frac{ |f(a+h_1,b+h_2)-f(a,b)-g'(a)\alpha_1(h)| }{ \| h \| }=\frac{ |g(a+h_1)-g(a)-g'(a)h_1| }{ \| h \| }
    \end{equation}
    Notre but est de calculer la limite de cela lorsque \( h\stackrel{\eR^2}{\longrightarrow}0\) avec \( h\neq 0\). L'hypothèse sur la dérivabilité de \( g\) nous indique que si \( 0<| t |<\delta\), alors
    \begin{equation}        \label{EQooQLWNooLRKhUv}
        \frac{ | g(a+t)-g(a)-tg'(a) | }{ | t | }<\epsilon.
    \end{equation}
    Nous considérons donc la boule épointée de \( \eR^2\) de rayon \( \delta\) : \( B=B\big( (0,0),\delta \big)\setminus\{(0,0)\}\), et nous considérons \( h\in B\). Deux cas sont à distinguer : \( h_1=0\) et \( h_1\neq 0\). 

    Si \( h_1=0\), alors
    \begin{equation}
        \frac{ |g(a+h_1)-g(a)-g'(a)h_1| }{ \| h \| }=0.
    \end{equation}
    Sinon nous avons \( 0<h_1\leq\| h \|<\delta\) et donc
    \begin{equation}
        \frac{ |g(a+h_1)-g(a)-g'(a)h_1| }{ \| h \| }\leq\frac{ |g(a+h_1)-g(a)-g'(a)h_1| }{ | h_1 | }<\epsilon
    \end{equation}
    par la relation \eqref{EQooQLWNooLRKhUv}. Nous avons donc bien
    \begin{equation}
        \lim_{h\to 0}\frac{ |f(a+h_1,b+h_2)-f(a,b)-g'(a)\alpha_1(h)| }{ \| h \| }=0.
    \end{equation}

\item[Récurrence]

    Nous supposons que \( d^kf_{a,b}=g^{(k)}(a)\alpha_k\), et nous devons prouver que \( d^kf\) est différentiable et que \( d^{k+1}f_{(a,b)}=g^{(k+1)}(a)\alpha_{k+1}\). Pour cela nous introduisons tout dans la définition de la différentielle pour voir ce qui arrive.

    Nous avons :
    \begin{equation}
        \frac{ d^kf_{(a+h_1,b+h_2)}-d^kf_{(a,b)}-g^{(k+1)}(a)\alpha_{k+1}(h_1,h_2) }{ \| h \| }=
        \frac{ g^{(k)}(a+h_1)\alpha_k-g^{(k)}(a)\alpha_k-g^{(k+1)}(a)h_1\alpha_k }{ \| h \| }.
    \end{equation}
    Cela est, pour chaque \( h\neq 0\), un élément \( V_k\), mais le coefficient \( \alpha_k\) se factorise de telle sorte que nous devons seulement calculer la limite (si elle existe)
    \begin{equation}
        \lim_{h\to 0} \frac{ g^{(k)}(a+h_1)-g^{(k)}(a)-h_1g^{(k+1)}(a) }{ \| h \| }.
    \end{equation}
    Le même jeu de séparation entre \( h_1=0\) et \( h_1\neq 0\) que dans le cas \( k=1\) nous permet de déduire que cette limite existe et vaut zéro, grace à la définition de \( g^{(k+1)}\).
    \end{subproof}

    Nous avons donc prouvé que \( f\) est différentiable autant que fois que souhaité. Elle est donc de classe \(  C^{\infty}\) comme annoncé.
\end{proof}

%---------------------------------------------------------------------------------------------------------------------------
\subsection{Différentielle, dual et forme différentielle}
%---------------------------------------------------------------------------------------------------------------------------

%///////////////////////////////////////////////////////////////////////////////////////////////////////////////////////////
\subsubsection{Dans la base duale}
%///////////////////////////////////////////////////////////////////////////////////////////////////////////////////////////

Nous avons déjà parlé en \eqref{EQooITHKooDzigPY} de la base \( \{ dx_i \}_{i=1,\ldots, n}\) des formes différentielles sur \( \eR^n\).

\begin{proposition}
    La forme de base \( dx_i\) est la différentielle de la fonction de projection
    \begin{equation}
        \begin{aligned}
            \pr_i\colon \eR^n&\to \eR \\
            v&\mapsto v_i.
        \end{aligned}
    \end{equation}
    Autrement dit nous avons
    \begin{equation}
        d(\pr_i)_a=dx_i
    \end{equation}
    pour tout \( i\) et pour tout \( a\).
\end{proposition}

\begin{proof}
    Le quotient
    \begin{equation}
        \frac{ \pr_i(a+h)-\pr_i(a)-dx_i(h) }{ \| h \| }
    \end{equation}
    est toujours nul. La limite est a fortiori nulle.
\end{proof}

Nous avons donc \( (d\pr_i)_a=dx_i\) pour tout \( a\). Notons que les fonctions \( dx_i\) et \( \pr_i\) sont les mêmes. Cela justifie la notation «\( dx_i\)» pour les formes différentielles de base, parce que ce sont les différentielles des fonctions «coordonnées» que nous pouvons noter \( x_i\).

Étant donné une fonction \( f\), il est légitime de nous demander comment (si elle existe) la différentielle se décompose en chaque point dans la base duale. C'est-à-dire fixer les fonctions \( a_i\) en termes des dérivées de \( f\) pour avoir
\begin{equation}
    df_a=\sum_{i=1}^n\frac{ \partial f }{ \partial x_i }(a)dx_i.
\end{equation}
C'est ce que nous allons faire dans le corolaire~\ref{CORooXURPooQMKvBl}.
:5cnext

\begin{example}
    Si $F\colon \eR^2\to \eR$ est une fonction $C^2$, sa différentielle est la forme
    \begin{equation}
        dF=\frac{ \partial F }{ \partial x }dx+\frac{ \partial F }{ \partial y }dy.
    \end{equation}
    Si nous nommons $f$ et $g$ les fonctions $\partial_xF$ et $\partial_yF$, nous avons donc
    \begin{equation}
        Df=fdx+gdy,
    \end{equation}
    qui vérifie
    \begin{equation}
        \partial_yf=\partial_xg,
    \end{equation}
    parce que $\frac{ \partial f }{ \partial y }=\frac{ \partial^2F  }{ \partial x\partial y }=\frac{ \partial^2F  }{ \partial y\partial x }=\frac{ \partial g }{ \partial x }$. Ce que nous avons donc prouvé, c'est que
\end{example}

\begin{lemma}
    Si $fdx+gdy$ est la différentielle d'une fonction de classe $C^2$ sur \( \eR^2\), alors $\partial_yf=\partial_xg$.
\end{lemma}

%---------------------------------------------------------------------------------------------------------------------------
\subsection{Ce n'est pas la différentielle extérieure}
%---------------------------------------------------------------------------------------------------------------------------

Il existe une notion de différentielle extérieure, mais ce n'est pas celle-là que nous utilisons la majorité du temps. En particulier si \( E\) et \( F\) sont des espaces vectoriels normés, lorsque \( f\colon E\to F\) est une fonction, \( df\) est une application
\begin{equation}
    df\colon E\to \aL(E,F)
\end{equation}
et la différentielle seconde est la différentielle de cette application-là. Chose faisable parce que \( \aL(E,F)\) est un espace vectoriel on ne peut plus respectable.

Soit $D\subset\eR^n$. Par définition de la différentielle extérieure d'une $1$-forme, nous avons une formule de Leibnitz
\begin{equation}
    d(f\omega)=df\wedge\omega+fd\omega.
\end{equation}
En particulier,
\begin{equation}
    d(fdx)=df\wedge dx+f\underbrace{d(dx)}_{=0}=\frac{ \partial f }{ \partial x }dx\underbrace{dx\wedge dx}_{=0}+\frac{ \partial f }{ \partial y }dy\wedge dx.
\end{equation}

Attention : la différentielle extérieure n'est pas la différentielle usuelle. Certes dans le cas d'une \( 0\)-forme (c'est-à-dire d'une fonction), les deux notions coïncident, mais ça ne va pas plus loin. La différentielle extérieure vérifie \( d^2\omega=0\) pour tout \( \omega\), y compris pour les fonctions : si \( \omega=df\) alors \( d\omega=0\).

%TODO : donner la définition et quelques exemples de différentielle extérieure.


Nous mentionnerons la différentielle extérieure dans le cas de
\begin{enumerate}
    \item
        Théorème de Stockes~\ref{ThoATsPuzF}.
\end{enumerate}

%--------------------------------------------------------------------------------------------------------------------------- 
\subsection{Fonctions composées}
%---------------------------------------------------------------------------------------------------------------------------

Cette façon de voir la différentielle nous permet de jeter un nouveau regard sur la formule de différentiation des fonctions composées. Soient
\begin{equation}
    \begin{aligned}[]
        f\colon \eR^p&\to \eR^n\\
        g\colon \eR^n&\to \eR,
    \end{aligned}
\end{equation}
et $h\colon \eR^p\to \eR$ définie par
\begin{equation}
    h(u)=h\big( f(u) \big)=(g\circ f)(u).
\end{equation}
Nous allons noter $x$ les coordonnées de $\eR^p$, $a$ un point de $\eR^p$ et $u$, un vecteur de $\eR^p$ accroché au point $a$. Pour $\eR^n$, les notations seront que les coordonnées sont $y$, $b$ est un point de $\eR^n$ et $v$ est un vecteur «accroché» au point $b$.

Nous avons
\begin{equation}
    dg_b(v)=\sum_{i=1}^n\frac{ \partial g }{ \partial y_i }(b)dy_i(v).
\end{equation}
Ici $dy_i(v)$ signifie la $i$ème composante de $v$. C'est simplement $v_i$. Cette formule étant valable pour tout point $b\in\eR^n$ et pour tout vecteur $v$, nous pouvons l'écrire en particulier pour
\begin{subequations}
    \begin{numcases}{}
        b=f(a)\\
        v=df_a(u).
    \end{numcases}
\end{subequations}
Cela donne
\begin{equation}        \label{Eqdgfadfau}
    dg_{f(a)}\big( df_a(u) \big)=\sum_{i=1}^n\frac{ \partial g }{ \partial y_i }\big( f(a) \big)dy_i\big( df_a(u) \big).
\end{equation}
Mais
\begin{equation}
    df_a(u)=\sum_{j=1}^p\frac{ \partial f }{ \partial x_j }(a)dx_j(u),
\end{equation}
donc la $i$ème composante de ce vecteur est
\begin{equation}
     \big( df_a(u)\big)_i=\sum_{j=1}^p\frac{ \partial f_i }{ \partial x_j }(a)dx_j(u).
\end{equation}
En remplaçant $dy_i\big( df_a(u) \big)$ par cela dans l'expression \eqref{Eqdgfadfau}, nous trouvons
\begin{equation}
    dg_{f(a)}\big( df_a(u) \big)=\sum_{i=1}^n\frac{ \partial g }{ \partial y_i }\big( f(a) \big)\sum_{j=1}^p\frac{ \partial f_i }{ \partial x_j }(a)dx_j(u).
\end{equation}
Nous pouvons vérifier que cela est la différentielle de $g\circ f$ au point $a$ appliquée au vecteur $u$. En effet
\begin{equation}
    d(g\circ f)_a(u)=\sum_{j=1}^p\frac{ \partial (g\circ f) }{ \partial x_j }(a)dx_j(u),
\end{equation}
tandis que, par la dérivation de fonctions composées,
\begin{equation}        \label{EqDerCompofg}
    \frac{ \partial (g\circ f) }{ \partial x_j }(a)=\sum_{i=1}^n\frac{ \partial g }{ \partial y_i }\big( f(a) \big)\frac{ \partial f_i }{ \partial x_j }(a).
\end{equation}
Au final, ce que nous avons prouvé est que
\begin{equation}        \label{EQooWSIYooBmsBDU}
    d(g\circ f)_a(u)=dg_{f(a)}\big( df_a(u) \big).
\end{equation}

%---------------------------------------------------------------------------------------------------------------------------
\subsection{Continuité, dérivabilité et différentiabilité}
%---------------------------------------------------------------------------------------------------------------------------

Le théorème suivant reprend les principales propriétés d'une fonction différentiable. Il est à ne pas confondre avec le théorème \ref{THOooBEAOooBdvOdr} qui dira que si les dérivées partielles sont continues sur un voisinage de $a$, alors $f$ est différentiable en $a$.
\begin{proposition}\label{diff1}\label{ThoRapPropDiffSi}
    Si $f$ est différentiable au point $a\in \eR^n$ alors
    \begin{enumerate}
        \item
            elle est continue en \( a\),
        \item
            elle admet une dérivée dans toutes les directions de \( \eR^m\),
\item  toutes les dérivées directionnelles $\partial_uf(a)$ existent et nous avons l'égalité
\begin{equation}        \label{EqDiffPartRap}
    \begin{aligned}
        df_a\colon \eR^n&\to \eR^m \\
        u&\mapsto df_a(u)=\frac{ \partial f }{ \partial u }(a)=\sum_i \frac{ \partial f }{ \partial x_i }(a)u_i,
    \end{aligned}
\end{equation}
si les $u^i$ sont les composantes de $u$ dans la base canonique de $\eR^n$.

    \end{enumerate}
\end{proposition}
\index{application!différentiable}

La dernière égalité sera de temps en temps utilisée sous la forme
\begin{equation}    \label{EqOWQSoMA}
    df_a(u)=\Dsdd{ f(a+tu) }{t}{0}.
\end{equation}

\begin{proof}
  La limite
\[
\lim_{h\to 0_m}\frac{\|f(a+h)-f(a)-T(h)\|_n}{\|h\|_m}=0,
\]
implique que
 \[
\lim_{h\to 0_m}\|f(a+h)-f(a)-T(h)\|_n=0.
\]
Comme $T$ est dans $\mathcal{L}(\eR^m,\eR^n)$, on a $\lim_{h\to 0}T(h)=0$, d'où la continuité de $f$ au point $a$.

Si $u$ est un vecteur non nul, la différentiabilité de $f$ au point $a$ implique
\[
\lim_{t\to 0}\frac{\|f(a+tu)-f(a)-T(tu)\|_n}{\|tu\|_m}=0,
\]
par la linéarité de $T$ et par l'égalité $\|tu\|_m=|t|\|u\|_m$ on obtient
\[
\lim_{t\to 0}\frac{f(a+tu)-f(a)}{|t|}= T(u).
\]
Donc $f$ est dérivable suivant le vecteur $u$ et $\partial_uf(a)=T(u)=df_a(u)$.
\end{proof}

\begin{corollary}       \label{CORooXURPooQMKvBl}
    Si \( f\) est différentiable, alors la forme différentielle \( df_a\) se décompose en
    \begin{equation}
        df_af=\sum_i(\partial_if)(a)dx_i.
    \end{equation}
\end{corollary}

\begin{proof}
    Vue la définition des formes \( dx_i\) nous pouvons remplacer \( u_i\) par \( dx_i(u)\) dans l'égalité \eqref{EqDiffPartRap} et écrire
    \begin{equation}
        df_a(u)=\sum_i(\partial_if)(a)dx_i(u)
    \end{equation}
    et donc écrire l'égalité demandée.
\end{proof}

Le lemme suivant regroupe quelques égalités avec lesquelles nous allons souvent travailler. Il explique comment sont liés les dérivées directionnelles, les dérivées partielles et la différentielle.
\begin{lemma}		\label{LemdfaSurLesPartielles}
	Si $f\colon \eR^m\to \eR^n$ est une fonction différentiable, alors
	\begin{equation}
        df_a(u)=\frac{ \partial f }{ \partial u }(a)=\Dsdd{ f(a+tu) }{t}{0}=\sum_{i=1}^mu_i\frac{ \partial f }{ \partial x_i }(a)=\nabla f(a)\cdot u
	\end{equation}
	pour tout vecteur $u\in\eR^m$
\end{lemma}

\begin{proof}
La première égalité est la proposition~\ref{diff1}, et la seconde est seulement la définition de la dérivée directionnelle avec des notations un peu plus snob. En particulier nous avons
\begin{equation}
    df_a(e_i)=\frac{ \partial f }{ \partial x_i }(a).
\end{equation}
Pour le reste c'est la linéarité de la différentielle qui joue : le vecteur $u$ peut être écrit de façon unique comme combinaison linéaire des vecteurs de base
\[
u=\sum_{i=1}^{m}u_i e_i, \qquad  u_i\in\eR,\, \forall i\in\{1,\ldots, m\}.
\]
Alors, la linéarité de $df_a$ nous donne
\begin{equation}
     df_a(u)= df_a\left(\sum_{i=1}^{m}u_i e_i\right)
=\sum_{i=1}^{m}u_i \left(df_ae_i\right)
=\sum_{i=1}^{m}u_i \frac{ \partial f }{ \partial x_i }(a).
 \end{equation}
Le lien avec le gradient est la définition du produit scalaire \eqref{DefYNWUFc}.
\end{proof}

La formule $df_a(u)=\Dsdd{ f(a+tu) }{t}{0}$ est bien utile pour calculer des différentielles, mais elle ne permet pas de prouver que \( f\) est différentiable. Autrement dit, même si le calcul de la dérivée \( \Dsdd{ f(a+tu) }{t}{0}\) donne un résultat pour tout \( u\), nous ne pouvons pas en déduire que \( f\) est différentiable au point \( a\).


\begin{proposition} \label{PropExistDiffUn}
    Soient $f$ une fonction de $x$ et $y$ et un point $(a,b)\in\eR^2$. Si les nombres $\partial_xf(a,b)$ et $\partial_yf(a,b)$ existent et s'il existe une fonction $\alpha\colon \eR\to \eR$ telle que
    \begin{equation}        \label{eqCritDifffabsrt}
        \begin{aligned}[]
            f(x,y)=f(a,b)&+\frac{ \partial f }{ \partial x }(a,b)(x-a)+\frac{ \partial f }{ \partial y }(a,b)(y-b)\\
            &+\| (x,y)-(a,b) \| \alpha\Big( \| (x,y)-(a,b) \| \Big)
        \end{aligned}
    \end{equation}
    et
    \begin{equation}
        \lim_{t\to 0} \alpha(t)=0,
    \end{equation}
    alors $f$ est différentiable en $(a,b)$.
\end{proposition}


\begin{normaltext}
    Dans cet énoncé nous avons écrit $d\big( (x,y),(a,b) \big)$ la distance entre $(x,y)$ et $(a,b)$, c'est-à-dire le nombre $\sqrt{(x-a)^2+(y-b)^2}$. Afin d'écrire l'équation \eqref{eqCritDifffabsrt} sous forme plus compacte, nous introduisons le vecteur
    \begin{equation}
        \nabla f(a,b)=\begin{pmatrix}
            \frac{ \partial f }{ \partial x }(a,b)    \\
            \frac{ \partial f }{ \partial y }(a,b).
        \end{pmatrix}
    \end{equation}
    L'équation \eqref{eqCritDifffabsrt} devient alors
    \begin{equation}        \label{EqdiffComp}
        f(X)=f(P)+\nabla f(a,b)\cdot (X-P)+\| X-P \|\alpha\big( \| X-P \| \big).
    \end{equation}
    Le vecteur $(\nabla f)(a,b)$ est appelé le \defe{gradient}{gradient} de $f$ au point $(a,b)$.
\end{normaltext}

\begin{remark}
    Nous avons introduit la notation \( \nabla f\) pour le gradient d'une fonction \( f\). Nous allons par la suite introduire \( \nabla\cdot F\) pour la divergence du champ de vecteurs \( F\) et \( \nabla\times F\) pour son rotationnel.

    Toutes les formules pour \( \nabla f\), \( \nabla\cdot F\) et \( \nabla\times F\) peuvent facilement être mémorisées en pensant à \( \nabla\) comme étant le vecteur
    \begin{equation}        \label{EQooQKGQooOPeFoo}
        \nabla=\begin{pmatrix}
            \partial_x    \\ 
            \partial_y    \\ 
            \partial_z    
        \end{pmatrix}.
    \end{equation}
    Nous allons ici cependant seulement penser à \eqref{EQooQKGQooOPeFoo} comme un moyen mnémotechnique; nous ne donnons pas de définitions à «\( \nabla\)» tout seul.
\end{remark}

\begin{proposition} \label{PropExistDiffDeux}
    Soit $f$ une fonction de deux variables admettant des dérivées partielles $\partial_xf(x,y)$ et $\partial_yf(x,y)$ qui sont elles-mêmes des fonctions continues de $x$ et $y$. Alors la fonction $f$ est différentiable partout.
\end{proposition}

\begin{proposition}
    Si $f$ est différentiable en $(a,b)$ alors pour tout vecteur \( u\), la fonction
    \begin{equation}
        \begin{aligned}
            \varphi\colon \eR&\to \eR \\
            t&\mapsto   f(a+tu_1,b+tu_2)
        \end{aligned}
    \end{equation}
    est dérivable en $0$ et on a
    \begin{equation}
        \varphi'(0)=\nabla f(p)\cdot u
    \end{equation}
    où nous avons noté $p=(a,b)$.
\end{proposition}

\begin{proof}
    Récrivons la formule \eqref{EqdiffComp} sous la forme
    \begin{equation}
        f(x)=f(p)+\nabla f(p)\cdot (x-p)+\| x-p \|\alpha(\| x-p \|).
    \end{equation}
    Cela étant vrai pour tout $x$, nous l'écrivons en particulier pour $x=p+tu$ où $t$ est un réel et $u$ est le vecteur unitaire choisi. Nous avons donc
    \begin{equation}
        f(p+tu)=f(p)+t\nabla f(p)\cdot u+\| tu \|\alpha(\| tu \|).
    \end{equation}
    En utilisant le fait que $u$ est unitaire, $\| tu \|=| t |\| u \|=| t |$. La dérivée de $\varphi$ en $0$ est alors donnée par
    \begin{equation}
        \lim_{t\to 0} \frac{ f(p+tu)-f(p) }{ t }=\lim_{t\to 0} \nabla f(p)\cdot u+\alpha(| t |).
    \end{equation}
    Lorsque nous prenons la limite, le membre de gauche devient $\varphi'(0)$ tandis que dans le membre de droite, le second terme disparaît. Nous avons finalement
    \begin{equation}
        \varphi'(0)=\nabla f(p)\cdot u
    \end{equation}
\end{proof}

%---------------------------------------------------------------------------------------------------------------------------
\subsection{Calcul de valeurs approchées}
%---------------------------------------------------------------------------------------------------------------------------

Si nous remplaçons les accroissements $x-a$ et $y-b$ par $h$ et $k$, le critère de différentiabilité s'écrit
\begin{equation}
    \begin{aligned}[]
        f(a+h,b+k)=f(a,b)+\frac{ \partial f }{ \partial x }(a,b)h&+\frac{ \partial f }{ \partial y }(a,b)k\\
        &+\sqrt{h^2+k^2}\alpha\big( \sqrt{h^2+k^2} \big).
    \end{aligned}
\end{equation}
Le dernier terme du membre de droite tend vers zéro à une vitesse double lorsque $h$ et $k$ tendent vers zéro : d'une part parce que $\sqrt{h^2+k^2}$ tend vers zéro et d'autre part parce que $\alpha\big( \sqrt{h^2+k^2} \big)$ tend vers zéro. Nous avons donc la «bonne» approximation
\begin{equation}        \label{EqFormApproxfxyab}
    f(x,y)\simeq f(a,b)+\frac{ \partial f }{ \partial x }(a,b)(x-a)+\frac{ \partial f }{ \partial y }(a,b)(y-b).
\end{equation}
lorsque $(x,y)$ n'est pas trop loin de $(a,b)$. Cette expression est évidemment une généralisation immédiate de l'équation \eqref{EqfxdxSimeqfxfpx}. Elle exprime que l'on peut obtenir des informations sur la valeur d'une fonction en $(x,y)$ si on peut calculer la fonction et ses dérivées en un point $(a,b)$ non loin de $(x,y)$.

Cette formule peut aussi être vue sous la forme suivante, plus pratique dans certains calculs :
\begin{equation}        \label{EqFormApproxfxyabDF}
    f(a+\Delta x,b+\Delta y)\simeq f(a,b)+\Delta x\frac{ \partial f }{ \partial x }(a,b)+\Delta y\frac{ \partial f }{ \partial y }(a,b).
\end{equation}

\begin{example}
    Prenons la fonction $f(x,y)=\cos(x)\sin(y)$ et calculons une approximation de
    \begin{equation}
        f\big( \frac{ \pi }{ 3 }+0.01,\frac{ \pi }{ 2 }+0.03 \big).
    \end{equation}
    D'abord les dérivées partielles sont
    \begin{equation}
        \begin{aligned}[]
            \frac{ \partial f }{ \partial x }(x,y)=-\sin(x)\sin(y)\\
            \frac{ \partial f }{ \partial y }(x,y)=\cos(x)\cos(y).
        \end{aligned}
    \end{equation}
    Nous allons utiliser l'approximation
    \begin{equation}
        f\big( \frac{ \pi }{ 3 }+0.01,\frac{ \pi }{ 2 }+0.03 \big)\simeq f\big( \frac{ \pi }{ 3 },\frac{ \pi }{2} \big)+0.01\frac{ \partial f }{ \partial x }\big( \frac{ \pi }{ 3 },\frac{ \pi }{2} \big)+0.03\frac{ \partial f }{ \partial y }\big( \frac{ \pi }{ 3 },\frac{ \pi }{2} \big).
    \end{equation}
    Nous avons
    \begin{equation}
        \begin{aligned}[]
            \frac{ \partial f }{ \partial x }\big( \frac{ \pi }{ 3 },\frac{ \pi }{2} \big)&=-\sin\frac{ \pi }{ 3 }\sin\frac{ \pi }{ 2 }=-\frac{ \sqrt{3} }{2}\\
            \frac{ \partial f }{ \partial y }\big( \frac{ \pi }{ 3 },\frac{ \pi }{2} \big)&=\cos\frac{ \pi }{ 3 }\cos\frac{ \pi }{ 2 }=0.
        \end{aligned}
    \end{equation}
    Par conséquent
    \begin{equation}
        f\big( \frac{ \pi }{ 3 }+0.01,\frac{ \pi }{ 2 }+0.03 \big)\simeq \frac{ 1 }{2}-0.01\frac{ \sqrt{3} }{2}=\frac{ 1 }{2}-\frac{ \sqrt{3} }{ 200 }.
    \end{equation}

    \begin{verbatim}
sage: var('x,y')
(x, y)
sage: f(x,y)=cos(x)*sin(y)
sage: a=f(pi/3+0.01,pi/2+0.03)
sage: numerical_approx(a)
0.491093815387986
sage: b=1/2-sqrt(3)/200
sage: numerical_approx(b)
0.491339745962156
sage: numerical_approx(a-b)
-0.000245930574169814
    \end{verbatim}
    Cela fait une erreur de l'ordre du dix millième.

\end{example}

\begin{remark}
    Les esprits les plus critiques diront que cette vérification pas Sage n'en est pas une parce que Sage a certainement utilisé un algorithme d'approximation qui se base sur la même idée que ce que nous venons de faire, et que par conséquent le fait qu'il obtienne le même résultat que nous est un peu tautologique.

    Ils n'auront pas tort. Cependant, le code source de Sage est disponible publiquement\footnote{Voir \url{http://www.sagemath.org}}; vous pouvez aller le lire et vérifier qu'il y a effectivement une \emph{preuve} que le résultat fourni par Sage possède une bonne dizaine de décimales correctes.

    Cette disponibilité publique du code source est une des nombreuses différences fondamentales entre Sage et votre calculatrice\footnote{et les autres logiciels de type fenêtre, pomme ou feuille d'érable.}. Dois-je vous rappeler qu'un des principes fondamentaux de l'éthique scientifique est que les résultats et les méthodes utilisés doivent être absolument ouverts à la vérification et à la critique de tous ?
\end{remark}

\begin{equation}        \label{Eqdfpunfpdu}
    df_p(u)=\nabla f(p)\cdot u.
\end{equation}

%--------------------------------------------------------------------------------------------------------------------------- 
\subsection{Différentielle et tangente}
%---------------------------------------------------------------------------------------------------------------------------

La notion de dérivée partielle (ou de dérivée suivant un vecteur) pour une fonction de plusieurs variables n'est pas une  généralisation de la notion de dérivée en une variable d'espace. En fait, du point de vue géométrique, la dérivée de la fonction $g:\eR\to\eR$ au point $a$ est la pente de la ligne droite tangente au graphe de $g$ au point $(a, g(a))$. Cette ligne, d'équation $r(x)=g'(a)x+g(a)$, est la meilleure approximation affine du graphe de $g$ au point $a$, comme à la figure~\ref{LabelFigTangentSegment}.
\newcommand{\CaptionFigTangentSegment}{Tangentes au graphe d'une fonction d'une variable}
\input{auto/pictures_tex/Fig_TangentSegment.pstricks}

Le graphe d'une fonction $f$ de $\eR^2$ dans $\eR$ est une surface de deux paramètres dans $\eR^3$. Si l'approximation affine d'une telle surface au point $(x,y,f(x,y))$ existe, alors elle est un plan tangent. En dimension plus haute, le graphe de la fonction $f:\eR^m\to\eR$ est une surface de $m$ paramètres dans $\eR^{m+1}$ et son approximation affine (si elle existe) est un hyperplan de $\eR^m$.

Nous allons voir que si $f$ prend ses valeurs dans $\eR^n$ l'approximation affine de $f$ au point $a$ est l'élément de $ f(a)+\mathcal{L}(\eR^m,\eR^n)$ qui ressemble le plus à $f$ au voisinage de $a$. Plus précisément, on utilise les définitions suivantes.
\begin{definition}
  Soient $f$ et $g$ deux applications d'un ouvert $U$ de $\eR^m$ dans $\eR^n$. On dit que $g$ est \defe{tangente}{application!tangente} à $f$ au point $a\in U$ si $f(a)=g(a)$ et
\[
\lim_{\begin{subarray}{l}
    x\to a\\ x\neq a
  \end{subarray}}\frac{\|f(x)-g(x)\|_n}{\|x-a\|_m}=0.
\]
\end{definition}
La relation de tangence est une relation d'équivalence. Nous sommes particulièrement intéressés par le cas où $f$ admet une application  affine tangente au point $a$.


\newcommand{\CaptionFigDifferentielle}{Interprétation géométrique de la différentielle.}
\input{auto/pictures_tex/Fig_Differentielle.pstricks}
En ce qui concerne l'interprétation géométrique, si nous regardons la figure~\ref{LabelFigDifferentielle}, et d'ailleurs aussi en voyant la définition~\ref{EqCritereDefDiff}, la fonction est différentiable et la différentielle est \( T\) s'il existe une fonction \( \alpha\) telle que
\begin{equation}
    f(a+u)-f(a)-T(u)=\alpha(u)
\end{equation}
où la fonction \( \alpha\) satisfait
\begin{equation}		\label{EqPresqueTa}
	\lim_{u\to 0} \frac{ \| \alpha(u)\| }{\| u \|}=0
\end{equation}
C'est cela qui fait écrire \( f(a+u)-f(a)-df_a(u)=o(\| u \|)\) à ceux qui n'ont pas peur de la notation \( o\).

La différentielle $df_a$ est donc la partie linéaire de l'application affine qui approxime au mieux la fonction $f$ autour du point $a$. La notion de différentielle est la vraie généralisation du concept de dérivée pour fonctions de plusieurs variables, en outre elle nous permet d'expliciter la relation qui associe au vecteur $u$ la dérivée $\partial_u f(a)$, pour $f$ et $a$ fixés.

\begin{remark}
	Si on remplace les normes $\|\cdot\|_m$  et $\|\cdot\|_n$ par d'autres normes, l'existence et la valeur de la différentielle de $f$ au point $a$ ne sont pas remises en cause. En effet, soient  $\|\cdot\|_M$  une norme sur $\eR^m$ et $\|\cdot\|_N$ une norme sur $\eR^n$. Par le théorème~\ref{ThoNormesEquiv}, ces normes sont équivalentes à $\| . \|_m$ et $\| . \|_n$ respectivement; il existe donc des constantes $k,\, K,\, l,\,L >0$ telles que  pour tout vecteur $u$ de $\eR^m$ et tout vecteur $v$ de $\eR^n$
\[
k\|u\|_M\leq \|u\|_m\leq K\|u\|_M,
\]
\[
l\|v\|_N\leq \|v\|_n\leq L\|v\|_N.
\]
Les éléments de $\mathcal{L}(\eR^m, \eR^n)$ sont les mêmes et on a
\begin{equation}
  \begin{aligned}
 & \frac{l}{K}  \frac{\|f(a+h)-f(a)-T(h)\|_N}{\|h\|_M}\leq \frac{\|f(a+h)-f(a)-T(h)\|_n}{\|h\|_m}\leq\\
&\leq\frac{L}{k} \frac{\|f(a+h)-f(a)-T(h)\|_N}{\|h\|_M}.
  \end{aligned}
\end{equation}
Il est donc possible, pour démontrer la différentiabilité ou pour calculer la différentielle, d'utiliser le critère \eqref{EqCritereDefDiff} avec une norme au choix. Parfois c'est utile.
\end{remark}

%---------------------------------------------------------------------------------------------------------------------------
                    \subsection{Prouver qu'une fonction n'est pas différentiable}
%---------------------------------------------------------------------------------------------------------------------------

Chacun des points du théorème~\ref{ThoRapPropDiffSi} est en soi un critère pour montrer qu'une fonction n'est pas différentiable en un point.

%///////////////////////////////////////////////////////////////////////////////////////////////////////////////////////////
                    \subsubsection{Continuité}
%///////////////////////////////////////////////////////////////////////////////////////////////////////////////////////////


Le premier critère à vérifier est donc la continuité. Si une fonction n'est pas continue en un point, alors elle n'y sera pas différentiable. Pour rappel, la continuité en $a$ se teste en vérifiant si $\lim_{x\to a}f(x)=f(a)$.

%///////////////////////////////////////////////////////////////////////////////////////////////////////////////////////////
                    \subsubsection{Linéarité}
%///////////////////////////////////////////////////////////////////////////////////////////////////////////////////////////

Un second test est la linéarité de la dérivée directionnelle par rapport à la direction : l'application $u\mapsto\frac{ \partial f }{ \partial u }(a)$ doit être linéaire, sinon $df_a$ n'existe pas.

\begin{example}     \label{Exemple0046Diff}
Examinons la fonction
\begin{equation}
    \begin{aligned}
        f\colon \eR^2&\to \eR \\
        (x,y)&\mapsto \begin{cases}
    \frac{ xy^2 }{ x^2+y^4 }    &   \text{si }(x,y)\neq (0,0)\\
    0   &    \text{sinon}.
\end{cases}
    \end{aligned}
\end{equation}
Prenons $u=(u_1,u_2)$ et calculons la dérivée de $f$ dans la direction de $u$ au point~$(0,0)$ :
\begin{equation}
    \begin{aligned}[]
        \frac{ \partial f }{ \partial u }(0,0)
            &=\lim_{t\to 0}\frac{ f(tu_1,tu_2)-f(0,0) }{ t }\\
            &=\lim_{t\to 0}\frac{1}{ t }\left( \frac{ tu_1t^2u_2 }{ t^2u_1^2+t^4u_2^4 } \right)\\
            &=\lim_{t\to 0}\left( \frac{ u_1u_2^2 }{ u_1^2+t^2u_2^4 } \right)\\
            &=\begin{cases}
    \frac{ u_2^2 }{ u_1 }   &   \text{si }u_1\neq 0\\
    0   &    \text{si }u_1=0.
\end{cases}
    \end{aligned}
\end{equation}
Cette application n'est pas linéaire par rapport à $u$. En effet, notons
\begin{equation}
    \begin{aligned}
        A\colon \eR^n&\to \eR \\
        u&\mapsto \frac{ \partial f }{ \partial u }(0,0),
    \end{aligned}
\end{equation}
et vérifions que pour tout $u$ et $v$ dans $\eR^n$ et $\lambda\in\eR$, nous ayons $A(\lambda u)=\lambda A(u)$ et $A(u+v)=A(u)+A(v)$. Le premier fonctionne parce que
\begin{equation}
    A(\lambda u)=A(\lambda u_1,\lambda u_2)=\frac{ \lambda^2 u_2^2 }{ \lambda u_1 }=\lambda\frac{ u_2^2 }{ u_1 }=\lambda A(u).
\end{equation}
Mais nous avons par exemple
\begin{equation}
    A\big( (0,1)+(2,3) \big)=A(2,4)=\frac{ 16 }{ 2 }=8,
\end{equation}
tandis que
\begin{equation}
    A(0,1)+A(2,3)=0+\frac{ 9 }{ 2 }\neq 8.
\end{equation}
La fonction $f$ n'est donc pas différentiable en $(0,0)$, parce que la candidate différentielle, $df_{(0,0)}(u)=\frac{ \partial f }{ \partial u }(0,0)$, n'est même pas linéaire.

\end{example}

Voici une autre façon de traiter la fonction de l'exemple~\ref{Exemple0046Diff}.

\begin{example} \label{ExeFHmCLII}
    La figure~\ref{LabelFigFWJuNhU} représente le domaine d'une fonction $f\colon \eR^2\to \eR$, et sur chacune des parties, elle est définie différemment.
    \newcommand{\CaptionFigFWJuNhU}{La fonction de l'exemple~\ref{ExeFHmCLII}.}
\input{auto/pictures_tex/Fig_FWJuNhU.pstricks}

L'expression de $f$ est ici
\begin{equation}
  f(x,y) =
  \begin{cases}
      xy & \text{si } x < 0 \text{ et } y > 0\\
      x-y & \text{si } x \geq 0 \text{ et } y \geq 0\\
      x^2y & \text{si } x > 0 \text{ et } y < 0\\
    x+y & \text{sinon.}
  \end{cases}
\end{equation}

On note que les deux axes forment une zone à problèmes. La zone hors
des axes est un ouvert sur lequel $f$ est différentiable car composée
de polynômes. Analysons chacun des points de la forme $(a,b)$ dans la
zone à problèmes (c'est-à-dire si $ab = 0$).

\subparagraph{Si $a = 0$ et $b > 0$} Un tel point $(0,b)$ est sur
l'axe verticale, dans la moitié supérieure. Pour calculer la limite de
$f$ en ce point, on peut restreindre notre étude au demi-plan ouvert
$y > 0$, ce qui revient à comparer la limite
\begin{equation*}
  \limite[y>0\\x\geq 0] {(x,y)} {(0,b)} f(x,y) =   \limite[y>0\\x\geq
  0] {(x,y)} {(0,b)} x-y = 0 - b = -b
\end{equation*}
avec la limite
\begin{equation*}
  \limite[y>0\\x<0] {(x,y)} {(0,b)} f(x,y) =   \limite[y>0\\x<0]
  {(x,y)} {(0,b)} xy = 0 b = 0
\end{equation*}
qui sont différentes puisque $b$ est supposé non nul.

\conclusion $f$ n'est pas continue en un point du type $(0,b)$ avec $b
> 0$.

\subparagraph{Si $a = 0$ et $b < 0$} Un tel point $(0,b)$ est sur
l'axe verticale, dans la moitié inférieure. Pour calculer la limite de
$f$ en ce point, on peut restreindre notre étude au demi-plan ouvert
$y < 0$, ce qui revient à comparer la limite
\begin{equation*}
  \limite[y<0\\x\geq 0] {(x,y)} {(0,b)} f(x,y) =   \limite[y<0\\x\geq
  0] {(x,y)} {(0,b)} x^2 y = 0^2 b = 0
\end{equation*}
avec la limite
\begin{equation*}
  \limite[y<0\\x<0] {(x,y)} {(0,b)} f(x,y) =   \limite[y<0\\x<0]
  {(x,y)} {(0,b)} x+y = 0 + b = b
\end{equation*}
qui sont différentes puisque $b$ est supposé non nul.

\conclusion $f$ n'est pas continue en un point du type $(0,b)$ avec $b
< 0$.

\subparagraph{Si $a > 0$ et $b = 0$} Un tel point $(a,0)$ est sur
l'axe horizontal, dans la moitié droite. Pour calculer la limite de
$f$ en ce point, on peut restreindre notre étude au demi-plan ouvert
$x > 0$, ce qui revient à comparer la limite
\begin{equation*}
  \limite[x>0\\y \geq 0] {(x,y)} {(a,0)} f(x,y) =   \limite[x>0\\y \geq
  0] {(x,y)} {(a,0)} x-y = a - 0 = a
\end{equation*}
avec la limite
\begin{equation*}
  \limite[x>0\\y < 0] {(x,y)} {(a,0)} f(x,y) =   \limite[x>0\\y < 0]
  {(x,y)} {(a,0)} x^2y = a^2 0 = 0
\end{equation*}
qui sont différentes puisque $a$ est supposé non nul.

\conclusion $f$ n'est pas continue en un point du type $(a,0)$ avec $a
> 0$.

\subparagraph{Si $a < 0$ et $b = 0$} Un tel point $(a,0)$ est sur
l'axe horizontal, dans la moitié gauche. Pour calculer la limite de
$f$ en ce point, on peut restreindre notre étude au demi-plan ouvert
$x < 0$, ce qui revient à comparer la limite
\begin{equation*}
  \limite[x<0\\y> 0] {(x,y)} {(a,0)} f(x,y) =   \limite[x<0\\y>
  0] {(x,y)} {(a,0)} x y = a 0 = 0
\end{equation*}
avec la limite
\begin{equation*}
  \limite[x<0\\y\leq 0] {(x,y)} {(a,0)} f(x,y) =   \limite[x<0\\y\leq0]
  {(x,y)} {(a,0)} x+y = a + 0 = a
\end{equation*}
qui sont différentes puisque $a$ est supposé non nul.

\conclusion $f$ n'est pas continue en un point du type $(a,0)$ avec $a
< 0$.

\subparagraph{Si $a = 0$ et $b = 0$} Le cas du point $(0,0)$ est
particulier, puisque il est adhérent aux quatre composantes du
domaine où la fonction est définie différemment. Pour étudier la
continuité, il faut donc étudier quatre limites. Ces limites ont déjà
été étudiées ci-dessus et valent toutes $0$, ce qui prouve la
continuité de $f$ en $(0,0)$.

En ce qui concerne la différentiabilité, on sait qu'il est nécessaire
que toutes les dérivées directionnelles existent. Calculons la dérivée
dans la direction $(0,1)$ (au point $(0,0)$)~:
\begin{equation*}
  \limite[t\neq0] t 0 \frac{f((0,0) + t(0,1)) - f(0,0)}{t} =%
  \limite[t\neq0] t 0 \frac{f(0,t)}{t} = \ldots
\end{equation*}
qu'on sépare en deux cas, car $f(0,t)$ possède une formule différente
si $t < 0$ ou si $t \geq 0$~:
\begin{equation*}
  \limite[t\neq0] t 0 \frac{f(0,t)}{t} = %
  \begin{arrowcases}
    \limite[t<0] t 0 \frac{f(0,t)}{t} = \limite[t<0] t 0 \frac{0+t}{t} = 1\\
    \limite[t\geq0] t 0 \frac{f(0,t)}{t} = \limite[t\geq0] t 0
    \frac{0-t}{t} = -1
  \end{arrowcases}
\end{equation*}
ce qui prouve que la limite n'existe pas, donc que la dérivée
directionnelle n'existe pas, et finalement que la fonction n'est pas
différentiable.

\conclusion La fonction donnée est continue hors des axes et au point
$(0,0)$, mais discontinue partout ailleurs sur les axes. Elle est
différentiable hors des axes, mais ne l'est pas sur les axes.

\end{example}

%///////////////////////////////////////////////////////////////////////////////////////////////////////////////////////////
                    \subsubsection{Cohérence des dérivées partielles et directionnelle}
%///////////////////////////////////////////////////////////////////////////////////////////////////////////////////////////

Dans la pratique, nous pouvons calculer $\partial_uf(a)$ pour une direction $u$ générale, et puis en déduire $\partial_xf$ et $\partial_yf$ comme cas particuliers en posant $u=(1,0)$ et $u=(0,1)$. Une chose incroyable, mais pourtant possible est qu'il peut arriver que
\begin{equation}
    \frac{ \partial f }{ \partial u }(a)\neq \sum_i\frac{ \partial f }{ \partial x_i }(a)u^i.
\end{equation}
Ceci se produit lorsque $f$ n'est pas différentiable en $a$. En voici un exemple.

%///////////////////////////////////////////////////////////////////////////////////////////////////////////////////////////
                    \subsubsection{Un candidat dans la définition (marche toujours)}
%///////////////////////////////////////////////////////////////////////////////////////////////////////////////////////////

Lorsqu'une fonction est donné, un candidat différentielle au point $(a_1,a_2)$ est souvent assez simple à trouver en un point :
\begin{equation}
    T(u_1,u_2)=\frac{ \partial f }{ \partial x }(a_1,a_2)u_1+\frac{ \partial f }{ \partial y }(a_1,a_2)u_2.
\end{equation}
L'application $T$ est la candidate différentielle en ce sens que si la différentielle existe, alors elle est égale à $T$. Ensuite, il faut vérifier si
\begin{equation}        \label{EqLimDefDiff}
    \lim_{(x,y)\to (a_1,a_2)} \frac{f(x,y) - f(a_1,a_2) - T\big( (x,y)-(a_1,a_2) \big)}{\| (x,y)-(a_1,a_2) \|}=0
\end{equation}
ou non. Si oui, alors la différentielle existe et $df_{(a,b)}(u)=T(u)$, sinon\footnote{y compris si la limite \eqref{EqLimDefDiff} n'existe même pas.}, la différentielle n'existe pas.

Attention : dans la ZAP, les dérivées partielles $\partial_xf$ et $\partial_yf$ ne peuvent en général pas être calculées en utilisant les règles de calcul (c'est bien pour ça que la ZAP est une zone à problèmes). Il faut d'office utiliser la définition
\begin{equation}
    \frac{ \partial f }{ \partial x }(a_1,a_2)=\lim_{t\to 0}\frac{ f(a_1+t,a_2)-f(a_1,a_2) }{ t },
\end{equation}
et la définition correspondante pour $\partial_yf$.

\subsubsection*{Conclusion}
Soient $f:A\subset \eR^n \rightarrow \eR^m$, et $a\in int\,A$. Si $f$ est différentiable en $a$, $$ (df_a (e_j))_i = d(f_i)_a(e_j) =\frac{\partial f_i}{\partial x_j}(a)= [Jac(f)_{|a}]_{ij}$$ et la matrice de l'application linéaire $df_a$ est la matrice jacobienne $m\times n$ de $f$ en $a$ notée $Jac(f)_{|a}$.

% This is part of Le Frido
% Copyright (c) 2006-2019
%   Laurent Claessens, Carlotta Donadello
% See the file fdl-1.3.txt for copying conditions.


\begin{proposition}[Règles de calculs]  \label{PROPooBWZFooTxKavX}
    Soient $f$ et $g$ des fonctions
  différentiables en $g(a)$ et $a$ respectivement, alors la composée
  $f\circ g$ est différentiable en $a$ et
  \begin{equation*}
    d (f\circ g)_a = d f_{g(a)} \circ d g_a
  \end{equation*}
  et de plus les jacobiennes correspondantes vérifient
  \begin{equation*}
      J_{f\circ g}(a) = J_f\big( g(a) \big)J_g(a)
  \end{equation*}
  où le membre de droite est le produit (non-commutatif !) des deux matrices.
\end{proposition}

\begin{corollary}[Chain rule] Si $f : \eR^p \to \eR$ et $g : \eR \to
  \eR^p$, alors
  \begin{equation*}
    (f\circ g)^\prime(t) = \sum_{i=1}^p \pder f {x_i}(g(t)) g_i^\prime(t).
  \end{equation*}
\end{corollary}

\begin{remark}
    Quelques remarques à propos de la règle de dérivation en chaine.
  \begin{enumerate}
  \item Si $p = 1$, on retrouve la règle usuelle de dérivation de
    fonctions composées.

  \item
      Si $g$ est à plusieurs variables, cette règle permet de déterminer les dérivées partielles de $f \circ g$, puisqu'une dérivée partielle peut être vue comme dérivée usuelle par rapport à une seule variable (voir \ref{deriveepartielles}).

  \item Si $f$ est à valeurs vectorielles, cette formule permet de
    retrouver la jacobienne de $f \circ g$ puisqu'il suffit de traiter
    chaque composante de $f$ séparément.
  \end{enumerate}
\end{remark}


%--------------------------------------------------------------------------------------------------------------------------- 
\subsection{Gradient}
%---------------------------------------------------------------------------------------------------------------------------

 \begin{definition}
	 Soit $f$ une fonction différentiable de $\eR^m$ dans $\eR$. On appelle \defe{gradient}{gradient} de $f$ la fonction $\nabla f : \eR^m\to \eR^m$\nomenclature{$\nabla f$}{gradient de la fonction $f$} de composantes
\[
\partial_{1}f,\ldots,\partial_{m}f.
\]
Soit $f$ une fonction de $\eR^m$ dans $\eR^n$, $f(a)=(f_1(a),\ldots,f_n(a))^T$. On appelle \defe{matrice jacobienne}{matrice!jacobienne} de $f$ la fonction $J(f) : \eR^m\to \eR^m\times\eR^n$ définie par
\begin{equation}
a\mapsto  \begin{pmatrix}
    \partial_{1}f_1(a) &\ldots&\partial_{m}f_1(a)\\
\vdots&\ddots&\vdots\\
\partial_{1}f_n (a)&\ldots&\partial_{m}f_n(a)\\
  \end{pmatrix}
\end{equation}
\end{definition}

%---------------------------------------------------------------------------------------------------------------------------
\subsection{Linéarité}
%---------------------------------------------------------------------------------------------------------------------------

La proposition suivante signifie que différentiation est une opération linéaire sur l'ensemble des fonctions différentiables.
\begin{proposition}		\label{PropDiffLineaire}
  Soient $f$ et $g$ deux fonctions de $U\subset\eR^m$ dans $\eR^n$ différentiables au point $a\in U$, et soit $\lambda$ dans $\eR$. Alors les fonctions $f+g$ et $\lambda f$ sont différentiables au point $a$ et on a
  \begin{equation}
    \begin{aligned}
 &     d(f+g)(a)=df(a)+dg(a), \\
& d(\lambda f)(a)=\lambda df(a),
    \end{aligned}
\end{equation}
\end{proposition}
\begin{proof}
  \begin{equation}
    \begin{aligned}
     & \lim_{h\to 0_m}\frac{\left\|\left(f(a+h)+g(a+h)\right)-\left(f(a)+g(a)\right)-df(a).h-dg(a).h\right\|_n}{\|h\|_m}\leq\\
&\lim_{h\to 0_m}\frac{\|f(a+h)-f(a)-df(a).h\|_n}{\|h\|_m}+\lim_{h\to 0_m}\frac{\|g(a+h)-g(a)-dg(a).h\|_n}{\|h\|_m}=0.
    \end{aligned}
  \end{equation}
  De même on démontre la  propriété $d(\lambda f)(a)=\lambda df(a)$.
\end{proof}

%+++++++++++++++++++++++++++++++++++++++++++++++++++++++++++++++++++++++++++++++++++++++++++++++++++++++++++++++++++++++++++ 
\section{Produit}
%+++++++++++++++++++++++++++++++++++++++++++++++++++++++++++++++++++++++++++++++++++++++++++++++++++++++++++++++++++++++++++

Soient $f$ et $g$ deux fonctions de $\eR^m$ dans $\eR^n$. Nous notons $f\cdot g$ la fonction de $\eR^n$ dans $\eR$ donnée par le produit scalaire point par point, c'est-à-dire
\begin{equation}
	(f\cdot g)(x)=f(x)\cdot g(x)
\end{equation}
pour tout $x\in\eR^m$. Le point dans le membre de droite est le produit scalaire dans $\eR^n$. Le cas particulier $n=1$ revient au produit usuel de fonctions :
\begin{equation}
	(fg)(x)=f(x)g(x).
\end{equation}

\begin{lemma}		\label{LemDiffProsuid}
	Si $f$ et $g$ sont des fonctions différentiables sur $\eR^m$ à valeurs dans $\eR$, alors la fonction produit $fg$ est également différentiable et
	\begin{equation}		\label{EqDifffgProd}
        (dfg)_a=g(a)fd_a+f(a)dg_a
	\end{equation}
	au sens où pour chaque $u$ dans $\eR^m$,
	\begin{equation}
        (dfg)_a(u)=g(a)df_a(u)+f(a)dg_a(u).
	\end{equation}
\end{lemma}

\begin{proof}
	Ce que nous devons faire pour vérifier la formule~\ref{EqDifffgProd}, c'est de vérifier le critère \eqref{EqCritereDefDiff} en remplaçant $f$ par $fg$ et $T(h)$ par $g(a)df(a).h+f(a)dg(a).h$.

	Ce que nous avons au numérateur est
	\begin{equation}
		\begin{aligned}[]
			\clubsuit&=(fg)(a+h)-(fg)(a)-g(a)df(a).h-f(a)dg(a).h\\
				&=f(a+h)g(a+h)-f(a)g(a)-g(a)df(a).h-f(a)dg(a).h.
		\end{aligned}
	\end{equation}
	Maintenant, nous allons faire apparaitre $\big( f(a+h)-f(a)-df(a) \big)g(a+h)$ en ajoutant et soustrayant ce qu'il faut pour conserver $\clubsuit$ :
	\begin{equation}
		\begin{aligned}[]
			\clubsuit&=\big( f(a+h)-f(a)-df(a).h \big)g(a+h)\\
					&\quad +f(a)g(a+h)+g(a+h)df(a).h\\
					&\quad -f(a)g(a)-g(a)df(a).h-f(a)dg(a).h.
		\end{aligned}
	\end{equation}
	Nous mettons maintenant $f(a)$ et $fd(a).h$ en évidence là où c'est possible :
	\begin{equation}
		\begin{aligned}[]
			\clubsuit&=\big( f(a+h)-f(a)-df(a).h \big)g(a+h)\\
				&\quad+f(a)\big( g(a+h)-g(a)-dg(a).h \big)\\
				&\quad+\big( g(a+h)-g(a) \big)df(a).h.
		\end{aligned}
	\end{equation}
    Nous devons maintenant considérer la limite
	\begin{equation}
		\lim_{h\to 0}\frac{ \| \clubsuit \| }{ \| h \| }.
	\end{equation}
    Étant donné que $f$ et $g$ sont différentiables, les deux premiers termes sont nuls :
    \begin{equation}
        \begin{aligned}[]
            \lim_{h\to 0}\frac{ \big( f(a+h)-f(a)-df(a).h \big)}{\| h \|}g(a+h)=0\\
            \lim_{h\to 0} f(a)\frac{ \big( g(a+h)-g(a)-dg(a).h \big)}{\| h \|}=0.
        \end{aligned}
    \end{equation}
    En ce qui concerne le troisième terme, en utilisant la norme d'une application linéaire, nous avons
	\begin{equation}
		\lim_{h\to 0} \frac{ \| df(a).h \| }{ \| h \| }\leq\sup_{h\in\eR^m}\frac{ \| df(a).h \| }{ \| h \| }=\| df(a) \|,
	\end{equation}
    et par conséquent
    \begin{equation}
        \begin{aligned}[]
            0&\leq\lim_{h\to 0} \| g(a+h)-g(a) \|\frac{ \| df(a).h \|\| h \| }{ \| h \| }\\
            &\leq \lim_{h\to 0} \| g(a+h)-g(a) \|\| df(a) \|=0
        \end{aligned}
    \end{equation}
    parce que $g$ est continue (la limite du premier facteur est nulle tandis que la norme de $df(a)$ est un nombre constant). Nous avons donc bien prouvé que la formule \eqref{EqDifffgProd} est la différentielle de $fg$ au point $a$.
\end{proof}


Ce résultat se généralise pour des fonctions $f$ et $g$ de $\eR^m$ dans $\eR^n$ dans la proposition suivante qui généralise tout en même temps la proposition \ref{PROPooFKKHooQZGXhE}.
\begin{proposition}
	Soient $f$ et $g$ deux fonctions de $U\subset\eR^m$ dans $\eR^n$ différentiables au point $a\in U$. Alors la fonction $f\cdot g$ est différentiable  au point $a$ et on a
	\begin{equation}
        d(f\cdot g)(a)=g(a)\cdot df(a)+f(a)\cdot dg(a)
	\end{equation}
	au sens où
	\begin{equation}		\label{Eqdfcdotgexpl}
		d(f\cdot g)_a(u)=g(a)\cdot\big( df_a(u) \big)+f(a)\cdot\big( dg_a(u) \big)
	\end{equation}
	pour tout $u\in\eR^m$.
\end{proposition}

\begin{proof}
	La preuve du cas $n=1$ est déjà faite; c'est la formule \eqref{EqDifffgProd}. Pour le cas général $n\geq 2$, nous passons au composantes en nous rappelant que
	\begin{equation}
		(f\cdot g)(a)=\sum_{i=1}^nf_i(a)g_i(a)=\sum_{i=1}^n(f_ig_i)(a).
	\end{equation}
	En utilisant la linéarité de la différentiation, nous nous réduisons donc au cas des produits $f_ig_i$ qui sont des fonctions de $\eR^m$ dans $\eR$ :
	\begin{equation}
		\begin{aligned}[]
			d(f\cdot g)(a)&=d\left( \sum_{i=1}^n f_ig_i \right)(a)\\
			&=\sum_{i=1}^n\big( df_i(a)g_i(a)+f_i(a)dg_i(a) \big)\\
			&=g(a)\cdot df(a)+f(a)\cdot dg(a).
		\end{aligned}
	\end{equation}
	Ceci termine la preuve.
\end{proof}

%--------------------------------------------------------------------------------------------------------------------------- 
\subsection{Difficulté d'ordre supérieur}
%---------------------------------------------------------------------------------------------------------------------------

\begin{normaltext}
    Il serait tentant de faire une récurrence sur le lemme \ref{LemDiffProsuid} pour dire que si \( f\) et \( g\) sont de classe \( C^p\), alors le produit \( fg\) est également de classe \( C^p\), parce que la formule de \( d(fg)\) contient des produits de fonctions de classe \( C^p\) et \( C^{p-1}\).

    Le problème est que le lemme \ref{LemDiffProsuid} est énoncé et prouvé pour des fonctions à valeurs dans \( \eR\), alors que déjà la formule 
    \begin{equation}
        d(fg)=gdf+fdg
    \end{equation}
    contient le produit de \( g\colon E\to \eR \) par \( df\colon E\to \aL(E,\eR)\). Lorsque nous montons dans les différentielles, la situation empire, et les produits dont sont composés les formules sont réellement à définir\ldots
\end{normaltext}

Oublions un instant les questions de régularité, et calculons sans ménagement, pour voir ce qu'il se passe. Nous considérons un espace vectoriel \( E\) ainsi que des des fonctions \( f\colon E\to \eR\) et \( g\colon E \to V\) où \( V\) est un autre espace vectoriel.

Nous avons
\begin{equation}
    d(fg)_a(u)=df_a(u)g(a)+f(a)dg_a(u).
\end{equation}
Les deux termes sont des produits \( \eR\times V\to \eR\). Montons un coup :
\begin{equation}
    d(gdf)_a(u)=\Dsdd{ (gdf)(a+tu) }{t}{0}=\Dsdd{ g(a+tu)df_{a+tu} }{t}{0}=dg_a(u)df_a+g(a)(d^2f)_a(u).
\end{equation}
Un autre pour voir comment ça se passe plus haut :
\begin{equation}
    d(dfdg)_a(u)=\Dsdd{ (dfdg)(a+tu) }{t}{0}=\Dsdd{ df_{a+tu}dg_{a+tu} }{t}{0}=(d^2f)_a(y)dg_a+df_a(d^2g)_a(u).
\end{equation}
Là déjà vous noterez que nous sommes passés par le produit
\begin{equation}
    df_{a+tu}df_{a+tu}
\end{equation}
qui pour chaque \( t\) est un produit \( \aL(E,\eR)\times \aL(E,V)\) que nous n'avons pas réellement défini.

En continuant le calcul ainsi nous trouvons par exemple
\begin{equation}
    \begin{aligned}[]
        (d^3fg)_a(u)&=d^3f_a(u)g(a)+d^2f_adg_a(u)\\
            &\quad +d^2f_a(u)dg_a+df_ad^2g_a(u)\\
            &\quad +d^2f_a(u)dg_a+df_a(d^2g)_a(u)\\
            &\quad +df_a(u)d^2g_a+f(a)(d^3g)_a(u).
    \end{aligned}
\end{equation}
Vous noterez que cette formule contient trois termes que nous aurions eu envie de noter \( d^2fdg\). Or ces trois termes ne sont pas identiques : deux sont \( d^2f_a(u)dg_a\) et un est \( (d^2f)_adg_a(u)\).

%--------------------------------------------------------------------------------------------------------------------------- 
\subsection{Solution : produit tensoriel}
%---------------------------------------------------------------------------------------------------------------------------

Afin de donner un sens à tous les produits, nous allons passer par les produits tensoriel. Nous avons déjà le théorème \ref{PROPooAWZFooMlhoCN} qui fait pratiquement tout.

\begin{proposition}[\cite{MonCerveau}]      \label{PROPooWNCGooHbmcVb}
    Soient des fonctions \( f\colon \eR^n\to \eR\) et \( g\colon \eR^n\to \eR\) de classe \( C^p\). Alors \( fg\) est de classe \( C^p\).
\end{proposition}

\begin{proof}
    Nous considérons l'application
    \begin{equation}
        \begin{aligned}
            \varphi\colon \eR\otimes \eR&\to \eR \\
            1\otimes 1&\mapsto 1 
        \end{aligned}
    \end{equation}
    dont nous avons déjà parlé dans le lemme \ref{LEMooVONEooQpPgcn}. En utilisant la notation \( \tilde\otimes\) de la définition \ref{DEFooMVNDooFWFtRn}, nous avons
    \begin{equation}
        fg=\varphi\circ(f\tilde\otimes g).
    \end{equation}
    La proposition \ref{PROPooAWZFooMlhoCN} nous dit que \( f\tilde\otimes g\colon \eR^n\to \eR\otimes \eR\) est de classe \( C^p\). Vu que \( \varphi\) est un isomorphisme d'espaces vectoriels, la proposition \ref{PROPooRCZOooSgvpSE} nous dit que \( \varphi\circ(f\tilde\otimes g)\) est encore de classe \( C^p\).

    Et voila.
\end{proof}

%--------------------------------------------------------------------------------------------------------------------------- 
\subsection{Formes bilinéaires}
%---------------------------------------------------------------------------------------------------------------------------

Nous avons aussi une formule importante pour la différentielle des formes bilinéaires.
  \begin{lemma}\label{bilin_diff}
    Toute application bilinéaire
    \begin{equation}
	    \begin{aligned}
		    B\colon \eR^m\times\eR^n&\to \eR^p \\
		    B(a_1,a_2)&=a_1 \star a_2
	    \end{aligned}
    \end{equation}
    est différentiable en tout point $(a_1,a_2)$ de $\eR^m\times\eR^n$, et on a
\[
dB(a_1,a_2).(h_1,h_2)=h_1\star a_2 + a_1\star h_2.
\]
  \end{lemma}
  \begin{proof}
    \begin{equation}
      \begin{aligned}
  & \frac{\|B(a_1+h_1,a_2+h_2)-B(a_1,a_2)-(h_1\star a_2 + a_1\star h_2)\|_p}{\|(h_1,h_2)\|_{\eR^m\times\eR^n}} = \\
&= \frac{\|(a_1+h_1)\star(a_2+h_2)-a_1\star a_2-(h_1\star a_2 + a_1\star h_2)\|_p}{\|(h_1,h_2)\|_{\eR^m\times\eR^n}}=\spadesuit
 \end{aligned}
    \end{equation}
on rajoute et on enlève la quantité $(a_1+h_1)\star a_2$ dans le numérateur, et on obtient
   \begin{equation}
      \begin{aligned}
%&= \frac{\|(a_1+h_1)\star(a_2+h_2)-(a_1+h_1)\star a_2 +(a_1+h_1)\star a_2- a_1\star a_2-}{\|(h_1,h_2)\|_{\eR^m\times\eR^n}}\\
%&\hspace{7cm}\frac{-(h_1\star a_2 + a_1\star h_2)\|_p}{\quad}=\\
&\spadesuit= \frac{\|(a_1+h_1)\star h_2+h_1\star a_2-(h_1\star a_2 + a_1\star h_2)\|_p}{\|(h_1,h_2)\|_{\eR^m\times\eR^n}}=\\
&= \frac{\|h_1\star h_2\|_p}{\|(h_1,h_2)\|_{\eR^m\times\eR^n}}\leq C\frac{\|h_1\|_m\|h_2\|_n}{\|(h_1,h_2)\|_{\eR^m\times\eR^n}}\leq\\
&\leq C\frac{\|(h_1,h_2)\|^2_{\eR^m\times\eR^n}}{\|(h_1,h_2)\|_{\eR^m\times\eR^n}}= C\|(h_1,h_2)\|_{\eR^m\times\eR^n}.
      \end{aligned}
    \end{equation}
Si on prend la limite de cette expression pour $(h_1,h_2)\to (0_m,0_n)$ on obtient $0$, donc la preuve est complète. À noter, que dans l'avant-dernier passage on a utilisé la continuité des applications linéaires $\pr_m:\eR^m\times\eR^n\to \eR^m$ et $\pr_n: \eR^m\times\eR^n\to \eR^n$ qui à chaque point $(a_1,a_2)$ de $\eR^m\times\eR^n$ associent $a_1$ et $a_2$ respectivement.
\end{proof}

\begin{proposition}     \label{PropEKLTooSvZjdW}
    Soit \( V\) et \( W\) deux espaces vectoriels et \( \varphi\colon V\to W\) un isomorphisme. Soit \( f\colon \eC\to V\) une application telle que \(\varphi\circ f\colon \eC\to W\) soit différentiable.

    Alors \( f\) est différentiable et \( df=\varphi^{-1}\circ d(\varphi\circ f)\).
\end{proposition}

\begin{proof}
    Si \( T\) est la différentielle de \( \varphi\circ f\) au point \( z\) nous avons
    \begin{equation}
        \lim_{\substack{h\to 0\\h\in \eC}}\frac{ (\varphi\circ f)(z+h)-(\varphi\circ f)(z)+T(h) }{ h }=0.
    \end{equation}
    En appliquant \( \varphi\) aux deux membres, et en permutant avec la limite (parce que \( \varphi\) est continue),
    \begin{equation}
        \varphi\lim_{h\to 0} \frac{ f(z+h)-f(z)+\varphi^{-1} T(h) }{ h }=0,
    \end{equation}
    ce qui signifie que \( f\) est différentiable et que \( df=\varphi^{-1}\circ T=\varphi^{-1}\circ d(\varphi\circ f)\).
\end{proof}

%+++++++++++++++++++++++++++++++++++++++++++++++++++++++++++++++++++++++++++++++++++++++++++++++++++++++++++++++++++++++++++ 
\section{Différentielle de fonction composée}
%+++++++++++++++++++++++++++++++++++++++++++++++++++++++++++++++++++++++++++++++++++++++++++++++++++++++++++++++++++++++++++

Une importante règles de différentiation est la règle de différentiation d'une fonction composée (\emph{chain rule} dans les livres anglais et américains). Cette règle généralise la règle de dérivation pour fonctions de $\eR$ dans $\eR$. 

Cette règle a déjà été donnée dans le théorème \ref{THOooIHPIooIUyPaf}, mais si vous avez seulement envie d'entendre parler de \( \eR^n\), vous pouvez lire le lemme \ref{Def_diff2} suivit de la proposition \ref{PropDiffCompose}.

Le lemme suivant est essentiellement une reformulation du lemme \ref{LEMooYQZZooVybqjK}.
\begin{lemma}\label{Def_diff2}
  Soit $U$ un ouvert de $\eR^m$. La fonction $f: U\to\eR^n$ est différentiable au point $a$ dans $U$, si et seulement s'il existe une fonction $\sigma_f: U\times U\to \eR^n$ telle que
  \begin{subequations}		\label{SubEqsDiff2}
	  \begin{align}
  		\sigma_f(a,a)&=\lim_{x\to a} \sigma_f(a,x)=0\\
		 f(x)&=f(a)+T(x-a)+\sigma_f(a,x)\|x-a\|_m,   \label{def_diff2}
	  \end{align}
  \end{subequations}
pour une certaine application linéaire $T\in\mathcal{L}(\eR^m,\eR^n)$.
\end{lemma}

\begin{proof}
	Si les conditions \eqref{SubEqsDiff2} sont satisfaites alors $T$ est la différentielle de $f$ en $a$. En effet, dans ce cas nous avons
	\begin{equation}
		f(a+h)=f(a)+T(h)+\sigma_f(a,a+h)\| h \|,
	\end{equation}
	et la condition \eqref{EqCritereDefDiff} devient
	\begin{equation}
		\lim_{h\to 0} \frac{ \| \sigma_f(a,a+h) \|\| h \| }{ \| h \| }=\lim_{h\to 0} \| \sigma_f(a,a+h)\| =0
	\end{equation}


Si $f$ est différentiable au point $a$ il suffit de prendre $T=df(a)$ et
\[
\sigma_f(a,x)=\frac{f(x)-f(a)-df(a).(x-a)}{\|x-a\|_m}.
\]
\end{proof}

\begin{remark}
	La fonction $\sigma_f(a,x)\| x-a \|_m$ est ce qui avait été appelle $\epsilon(h)$ sur la figure~\ref{LabelFigDifferentielle}.
\end{remark}

\begin{proposition}		\label{PropDiffCompose}
Soient $U$ un ouvert de $\eR^m$ et $V$ un ouvert de $\eR^n$. Soient $f: U\to V$  et $g: V \to \eR^p$ deux fonctions différentiables respectivement au point $a$ dans $U$ et $b=f(a)$ dans $V$. Alors la fonction composée $g\circ f: U\to \eR^p $ est différentiable au point $a$ et
\begin{equation}	\label{EqDiffCompose}
    d(g\circ f)_a=dg_{f(a)}\circ df_a.
\end{equation}
\end{proposition}

\begin{proof}
 En tenant compte du lemme~\ref{Def_diff2} on peut écrire
 \begin{subequations}
	 \begin{align}
		f(a+h)-f(a)&=df_a(h)+\sigma_f(a,a+h)\|h\|_m,	&&\forall h\in U-a,\\
		g(b+k)-g(b)&=dg_b(k)+\sigma_g(b,b+k)\|k\|_n,	&&\forall k\in V-b.
	 \end{align}
 \end{subequations}
On sait que $f(a)=b$ et que $f(a+h)$ est  un élément de $V$ et $f(a+h)=f(a)+k$ pour $k=df(a).h+\sigma_f(a,a+h)\|h\|_m$.  Par substitution dans la deuxième équation on obtient
\begin{equation}
	\begin{aligned}
		g\big(f(a+h)\big)& - g\big(f(a)\big)\\
        &=dg_{f(a)}\Big(df_a(h)+\sigma_f(a,a+h)\|h\|_m\Big)\\
		&\quad+\sigma_g\left(f(a), f(a+h)\right)\left\| df_a(h)+\sigma_f(a,a+h)\|h\|_m\right \|_n\\
		&=g\circ f (a+h) - g\circ f (a)\\
        &= dg_{f(a)}\circ df_a(h) \\
        &\quad +\|h\|_m\Big[ dg_{f(a)}\sigma_f(a,a+h)\\
		&\quad+\sigma_g\left(f(a), f(a+h)\right)\big\| df_a\frac{h}{\|h\|_m}+\sigma_f(a,a+h)\big \|_n\Big],
	\end{aligned}
\end{equation}
donc
\begin{equation}
	(g\circ f) (a+h) - (g\circ f) (a) = dg_f(a)\circ df_a(h) + S(a,a+h) \|h\|_m
\end{equation}
où $S$ représente le contenu du dernier grand crochet. Il ne reste plus qu'à prouver que $S(a,a+h)$ est $o(\|h\|_m)$. En tenant compte du fait que $\sigma_f(a,a+h)$ et $\sigma_g\left(f(a), f(a+h)\right)$ sont $o (\|h\|_m)$,
\begin{equation}
  \begin{aligned}
      & \lim_{h\to 0_m} \frac{S(a,a+h)}{\|h\|_m}= \lim_{h\to 0_m}\frac{dg_{f(a)}\sigma_f(a,a+h)}{\|h\|_m}+ \\
& + \lim_{h\to 0_m}\frac{\sigma_g\left(f(a), f(a+h)\right)\left\| df_a\frac{h}{\|h\|_m}+\sigma_f(a,a+h)\right \|_n}{\|h\|_m} = 0.
  \end{aligned}
\end{equation}
\end{proof}

\begin{remark}
    Note : la formule \eqref{EqDiffCompose} est à comprendre de la façon suivante. Si $u\in\eR^m$, alors
    \begin{equation}
        d(g\circ f)_a(u)=\underbrace{dg_{f(a)}}_{\in\aL(\eR^n,\eR^p)}\Big( \underbrace{df_a(u)}_{\in\eR^n} \Big)\in\eR^p.
    \end{equation}
\end{remark}

Le lemme suivant sert à prouver les théorèmes \ref{ThoLDpRmXQ} et \ref{ThoUJMhFwU}. Il est fondamentalement la raison de la formule définissant l'intégrale d'une forme sur un chemin (définition \ref{DEFooRMHGooFtMEPB}).
\begin{lemma}[\cite{MonCerveau}]        \label{LEMooKNBVooQSowos}
    Soient un espace vectoriel normé \( F\) de dimension finie, ainsi que \( E\), un espace vectoriel normé. Nous considérons un chemin de classe \( C^1\)
    \begin{equation}
        \gamma\colon \mathopen[ a , b \mathclose]\to E
    \end{equation}
    et une application de classe \( C^1\)
    \begin{equation}
        f\colon E\to F.
    \end{equation}
    Si \( g=f\circ\gamma\), alors
    \begin{equation}
        g'(t)=(df)_{\gamma(t)}\big( \gamma'(t) \big)
    \end{equation}
    pour tout \( t\in \mathopen[ a , b \mathclose]\).
\end{lemma}
 
\begin{proof}
    Nous écrivons la dérivée de \( g\) de la façon suivante : 
    \begin{subequations}        \label{SUBEQSooWKRYooGyVgNl}
        \begin{align}
            g'(t)&=\Dsdd{ g(t+s) }{s}{0}\\
            &=\Dsdd{ (f\circ \gamma)(t+s) }{s}{0}\\
            &=d(f\circ \gamma)_t(1) \label{SUBEQooHUZVooWECnVZ}\\
            &=(df)_{\gamma(t)}\big( d\gamma_t(1) \big) \label{SUBEQooOAYLooIzltaY}\\
        \end{align}
    \end{subequations}
    Justifications :
    \begin{itemize}
        \item Pour \eqref{SUBEQooHUZVooWECnVZ} : les formules \eqref{LemdfaSurLesPartielles}. Notez que le \( 1\) à qui s'applique la différentielle de \( f\circ\gamma\) est le vecteur de \( \eR\) qui est multiplié par \( s\) dans l'expression \( (f\circ\gamma)(t+s)\).
        \item
            Pour \eqref{SUBEQooOAYLooIzltaY} : la différentiation de fonction composées de la proposition \ref{PropDiffCompose}.
    \end{itemize}
    Mais
    \begin{equation}
        d\gamma_t(1)=\Dsdd{ \gamma(t+s) }{s}{0}=\gamma'(t).
    \end{equation}
    En remettant au bout de \eqref{SUBEQSooWKRYooGyVgNl}, nous obtenons le résultat.
\end{proof}

%+++++++++++++++++++++++++++++++++++++++++++++++++++++++++++++++++++++++++++++++++++++++++++++++++++++++++++++++++++++++++++ 
\section{Autres trucs sur la différentielle}
%+++++++++++++++++++++++++++++++++++++++++++++++++++++++++++++++++++++++++++++++++++++++++++++++++++++++++++++++++++++++++++

%---------------------------------------------------------------------------------------------------------------------------
  \subsection{Différentielle et dérivées partielles}
%---------------------------------------------------------------------------------------------------------------------------

\begin{proposition}		\label{Diff_totale}
 Soit $U$ un ouvert dans $\eR^m$ et $a$ un point dans $U$. Soit $f$ une application de $U$ dans $\eR^n$. Si toutes les dérivées partielles de $f$ existent sur \( U\) et sont continues au point $a$ alors $f$ est différentiable au point $a$.
\end{proposition}
\begin{proof}
 On se limite au cas $m=2$.  Pour rendre les calculs plus simples on utilise ici la norme $\|\cdot\|_\infty$ dans l'espace $\eR^2$, mais comme on a vu plus en haut, cela ne peut pas avoir des conséquences sur la différentiabilité de $f$. Si la différentielle de $f$ au point $a$ existe alors elle est définie par la formule
\[
    df_a(v)=\frac{ \partial f }{ \partial x }(a)v_1+\frac{ \partial f }{ \partial y }(a)v_2
\]
pour tout $v$ dans $\eR^m$.

On commence par prouver le résultat en supposant que les dérivées partielles de $f$ au point $a$ sont nulles. La différentiabilité de $f$ signifie que pour toute constante  $\varepsilon> 0$ il y a une constante $\delta>0$ telle que si $\|v\|_\infty\leq \delta $ alors
\[
\frac{\|f(a_1+v_1, a_2+v_2)-f(a_1, a_2)\|_n}{\|v\|_\infty}\leq \varepsilon.
\]
On écrit alors
\begin{equation}
  \begin{aligned}
   & \|f(a_1+v_1, a_2+v_2)-f(a_1, a_2)\|_n=\\
&=\|f(a_1+v_1, a_2+v_2)-f(a_1+v_1, a_2)+f(a_1+v_1, a_2)-f(a_1, a_2)\|_n\leq\\
&\leq \|f(a_1+v_1, a_2+v_2)-f(a_1+v_1, a_2)\|_n+\|f(a_1+v_1, a_2)-f(a_1, a_2)\|_n.
  \end{aligned}
\end{equation}
Comme la dérivée partielle $\partial_x f$ est nulle au point $a$  on sait que  pour toute constante  $\varepsilon> 0$ il y a une constante $\delta_1>0$ telle que si $|v_1|\leq \delta_1 $ alors
\[
\|f(a_1+v_1, a_2)-f(a_1, a_2)\|_n\leq \varepsilon |v_1|.
\]
Pour l'autre terme on a, par la proposition~\ref{val_medio_1},
\begin{equation}
   \|f(a_1+v_1, a_2+v_2)-f(a_1+v_1, a_2)\|_n\leq \sup\{\|\partial_yf(x)\|_n\,\vert\, x\in S\}|v_2|.
\end{equation}
où $S$ est le segment d'extrémités  $(a_1+v_1, a_2)$ et $ (a_1+v_1, a_2+v_2)$. Comme la  dérivée partielle $\partial_y f$ est continue et nulle au point $a$ on sait que  pour toute constante  $\varepsilon> 0$ il existe une constante $\delta_2>0$ telle que si $\|(u_1,u_2)\|_\infty\leq \delta_2 $ alors $\|\partial_yf(a_1+u_1,a_2+u_2)\|_n\leq \varepsilon$. Si on choisit $\delta = \min\{\delta_1,\,\delta_2\}$ le segment $S$ est contenu dans la boule de rayon $\delta$ centrée au point $a$ et on obtient
\[
 \|f(a_1+v_1, a_2+v_2)-f(a_1, a_2)\|_n\leq \varepsilon |v_1|+\varepsilon |v_2|\leq 2\varepsilon \|v\|_\infty.
\]
Cela prouve que \( f\) est différentiable en \( (a_1,a_2)\) et que la différentielle est nulle :
\begin{equation}
    df_{(a_1,a_2)}=0.
\end{equation}

Dans le cas général, où les dérivées partielles de $f$ au point $a$ ne sont pas spécialement nulles, on peut considérer la fonction\footnote{Vous verrez dans la discussion à propos de la fonction \eqref{EqCJVooJOuXdN} pourquoi cette fonction ne fonctionne pas dans le cas de la dimension infinie.}
\begin{equation}    \label{EqXHVooJeQKrB}
    g(x,y)=f(x,y)-\partial_1 f(a)x-\partial_2 f(a)y,
\end{equation}
qui a dérivées partielles nulles au point $a$. La fonction $g$ est donc différentiables. La fonction $f$ est maintenant la somme de $g$ et de la fonction linéaire et continue $(x,y)\mapsto \partial_1 f(a)x-\partial_2 f(a)y$. On verra dans la prochaine section que la somme de deux fonctions différentiables est une fonction différentiable. Par conséquent, la fonction $f$ est différentiable.
\end{proof}

\begin{remark}
    En dimension infinie, il n'est pas vrai que l'existence et la continuité de toutes les dérivées partielles en un point implique la différentiabilité en ce point. Pour donner un exemple, nous allons continuer l'exemple~\ref{ExHKsIelG}
    avec la fonction~\ref{EqCJVooJOuXdN} sur un espace de Hilbert.

    En dimension infinie nous aurons le théorème~\ref{ThoOYwdeVt} qui donnera quelque chose de moins fort.
\end{remark}

Étant donné que pour tout vecteur $u$ dans $\eR^m$ on a $\partial_uf(a)=\nabla f(a)\cdot u$, le gradient de $f$ nous donne la direction dans laquelle la croissance de $f$ est maximale. Soit $C$ une colline et soit $f$ la fonction que a chaque point $(x,y)$ de la Terre associe son altitude. Si nous voulons monter la colline le plus vite possible nous n'avons qu'a suivre la direction $\nabla f$ à chaque point. Elle est la projection sur le plan $x$-$y$ de la direction de pente maximale. Au contraire, la direction $-\nabla f$ est la direction de croissance minimale.

La matrice jacobienne calculé au point $a$ est la matrice associée canoniquement à l'application linéaire $df_a:\eR^m\to\eR^n$.

%---------------------------------------------------------------------------------------------------------------------------
\subsection{Plan tangent}
%---------------------------------------------------------------------------------------------------------------------------

On a dit au début de cette section que si $f$ est une fonction de $\eR^2$ dans $\eR$ alors le graphe de $f$ est une surface à deux paramètres et que l'application affine tangente au graphe de $f$ au point $(a, f(a))$ est un plan. Maintenant on sait que ce plan est celui d'équation
\begin{equation}
	T_a(x,y)=f(a_1,a_2)+\frac{ \partial f }{ \partial x }(a_1,a_2)(x-a_1)+\frac{ \partial f }{ \partial y }(a_1,a_2)(y-a_2).
\end{equation}
Le plan tangent au graphe de $f$ au point $a$ est le graphe de cette fonction $T_a$.

\begin{remark}
	Il existe cependant des fonctions différentiables dont les dérivées partielles ne sont pas continues. La construction d'un tel exemple est cependant délicate, et nous le ferons pas ici. Retenez cependant que si dans un exercice vous obtenez que les dérivées partielles ne sont pas continues, vous ne pouvez pas immédiatement en conclure que la fonction ne sera pas différentiable.
\end{remark}

%---------------------------------------------------------------------------------------------------------------------------
\subsection{Calcul de différentielles}
%---------------------------------------------------------------------------------------------------------------------------

\begin{normaltext} \label{deriveepartielles}
En pratique, ayant une formule pour la fonction $f$, nous la dérivons par rapport à la variable $x_i$ en utilisant les règles usuelle de dérivation en considérant que les autres ($x_j$ avec $j \neq i$) sont des constantes.
\end{normaltext}

\begin{example}Pour $f(x,y) = xy + x^2$, les dérivées partielles
  s'écrivent
  \begin{equation*}
    \frac{\partial f}{\partial x} = y + 2x \quad\text{et}\quad \frac{\partial f}{\partial y} = x
  \end{equation*}
\end{example}


Des \emph{règles de calcul} sont d'application. En particulier, quand
ces opérations existent, les sommes, différences, produits, quotients
et compositions d'applications différentiables sont différentiables.

Toute application linéaire est différentiable, et sa différentielle en
tout point est égale à l'application elle-même\footnote{Lemme \ref{LEMooZSNMooCfjzOB}.}. En particulier, les
\Defn{projections canoniques}, c'est-à-dire les applications du type
$(x,y,z) \mapsto y$, sont linéaires donc différentiables.

\begin{example}
Les cas suivants sont faciles :
  \begin{enumerate}
  \item En combinant les projections canoniques avec les règles de
    calculs, on obtient que toute fonction polynomiale à $n$ variables
    est différentiable comme application de $\eR^n$ dans $\eR$.

  \item Toute fonction rationnelle, du type $f(x) \pardef
    \frac{P(x)}{Q(x)}$ où $P$ et $Q$ sont des polynômes, est
    différentiable en tout point $a$ tel que $Q(a) \neq 0$.

  \item Pour une fonction d'une variable $f : D \subset \eR \to
    \eR$, le caractère différentiable et le caractère dérivable
    coïncident. De plus, on a
    \begin{equation*}
      d f_a(u) = f'(a) u.
    \end{equation*}
  \end{enumerate}
\end{example}

%---------------------------------------------------------------------------------------------------------------------------
                    \subsection{Notes idéologiques quant au concept de plan tangent}
%---------------------------------------------------------------------------------------------------------------------------
\label{ssecConceptPlanTag}

Notons $G$, le graphe d'une fonction $f$, c'est-à-dire
\begin{equation}
    G=\{ (x,y,z)\in\eR^3\tq z=f(x,y) \}.
\end{equation}
Première affirmation : si $\gamma\colon \eR\to G$ est une courbe telle que $\gamma(0)=\big( a,f(a) \big)$, alors $\gamma'(0)\in\eR^n$ est dans le plan tangent à $G$ au point $\big( a,f(a) \big)$.

Plus fort : tous les éléments du plan tangent sont de cette forme.

Le plan tangent à $G$ en un point $x\in G$ est donc constitué des vecteurs vitesse de tous les chemins qui passent par $x$.

Prenons maintenant $S$, une courbe de niveau de $G$, c'est-à-dire
\begin{equation}
    S=\{ (x,y)\in\eR^2\tq f(x,y)=C \}.
\end{equation}
Si nous prenons un chemin dans $G$ qui est, de plus, contraint à $S$, c'est-à-dire tel que $\gamma(t)\in S$, alors $\gamma'(0)$ sera tangent à $G$ (ça, on le savait déjà), mais en plus, $\gamma'(0)$ sera tangent à $S$, ce qui est logique.

La morale est que si vous prenez un chemin qui se ballade dans n'importe quoi, alors la dérivée du chemin sera un vecteur tangent à ce n'importe quoi.

En outre, si $\gamma(t)\in S$ et $\gamma(0)=a$, alors
\begin{equation}
    \scal{\nabla f(a)}{\gamma'(0)}=0,
\end{equation}
c'est-à-dire que le vecteur tangent à la courbe de niveau est perpendiculaire au gradient. Cela est intuitivement logique parce que la tangente à la courbe de niveau correspond à la direction de \emph{moins} grande pente.

%---------------------------------------------------------------------------------------------------------------------------
                    \subsection{Gradient et recherche du plan tangent}
%---------------------------------------------------------------------------------------------------------------------------

Nous avons maintenant en main les concepts utiles pour trouver l'équation du plan tangent à une surface.

De la même manière que la tangente à une courbe était la droite de coefficient directeur donné par la dérivée, maintenant, le plan tangent à une surface est le plan dont les vecteurs directeurs sont les dérivées partielles :

La généralisation de l'équation \eqref{EqDiffRapTgDer} est
\begin{equation}        \label{EqDefPlanTag}
    T_a(x)=f(a)+\sum_i\frac{ \partial f }{ \partial x_i }(a)(x-a)^i
\end{equation}

Nous introduisons aussi souvent l'opérateur différentiel abstrait \defe{nabla}{nabla}, noté $\nabla$ et qui est donné par le vecteur
\begin{equation}
    \nabla=\left( \frac{ \partial  }{ \partial x_1 },\ldots,\frac{ \partial  }{ \partial x_n } \right).
\end{equation}
Les égalités suivantes sont juste des notations, sommes toutes logiques, liées à $\nabla$ :
\begin{equation}
    \nabla f=\left( \frac{ \partial f }{ \partial x_1 },\ldots,\frac{ \partial f }{ \partial x_n } \right),
\end{equation}
et
\begin{equation}        \label{EqDefGradient}
    \nabla f(a) = \left(\frac{\partial f}{\partial x_1}(a), \frac{\partial f}{\partial x_2}(a), \ldots, \frac{\partial f}{\partial x_n}(a)\right).
\end{equation}
Ce dernier est un élément de $\eR^n$ : chaque entrée est un nombre réel.

\begin{definition}
Le vecteur gradient de $f$ au point $a$ est le vecteur donné par la formule \eqref{EqDefGradient}.
\end{definition}
La notation $\nabla$ permet d'écrire la différentielle sous forme un peu plus compacte. En effet, la formule \eqref{EqDiffPartRap} peut être notée
\begin{equation}
    df_a(u)=\scal{\nabla f(a)}{u}.
\end{equation}

En utilisant ce produit scalaire, l'équation \eqref{EqDefPlanTag} peut se récrire
\begin{equation}
    T_a(x)=f(a)+\sum_i\frac{ \partial f }{ \partial x_i }(a)(x-a)^i=f(a)+\scal{\nabla f(a)}{x-a}.
\end{equation}

Afin d'éviter les confusions, il est parfois souhaitable de bien mettre les parenthèses et noter $(\nabla f)(a)$ au lieu de $\nabla f(a)$.

\begin{proposition}
$\nabla f(a)\,\bot \,S_a$
\end{proposition}


\begin{equation}        \label{EqPlanTgSansNabla}
    z=f(a)+\sum_i\frac{ \partial f }{ \partial f }(a)(x-a)^i.
\end{equation}

\subsubsection*{Cas particulier où $n=2$:}
Le plan $T_a$ avec $a=(a_1,a_2)$ a pour équation dans $\eR^3$:
\begin{equation}        \label{EqPlanTgEnDimDeux}
    z = f(a_1,a_2) + \frac{\partial f}{\partial x}(a_1,a_2)\,(x-a_1)+ \frac{\partial f}{\partial y}(a_1,a_2)\,(y-a_2).
\end{equation}

\begin{definition}
  Soit $f : \eR^n \to\eR$ une fonction différentiable en un point
  $a$. Le \emph{plan tangent} au graphe de $f$ en $(a,f(a))$ est
  l'ensemble des points
  \begin{equation*}
    \begin{split}
      T_af &= \{ (x,z) \in \eR^n \times \eR \tq z = f(a) + d f_a (x-a)\}\\
      &= \{ (x,z) \in \eR^n \times \eR \tq z = f(a) + \scalprod{\nabla f(a)}{x-a}\}
    \end{split}
  \end{equation*}
\end{definition}

Nous avons vu que, de la même façon qu'en deux dimensions nous avions l'approximation \eqref{Eqfxsimesfa} d'une fonction par sa tangente, en trois dimensions nous avons l'approximation suivante d'une fonction de deux variables :
\begin{equation}
    f(x,y)\simeq f(a,b)+\frac{ \partial f }{ \partial x }(a,b)(x-a)+\frac{ \partial f }{ \partial y }(a,b)(y-b)
\end{equation}
lorsque $(x,y)$ n'est pas trop loin de $(a,b)$. Cela signifie que le graphe de $f$ ressemble au graphe de la fonction $T_{(a,b)}$ donnée par
\begin{equation}
    T_{(a,b)}(x,y)=f(a,b)+\frac{ \partial f }{ \partial x }(a,b)(x-a)+\frac{ \partial f }{ \partial y }(a,b)(x-a).
\end{equation}
En notations compactes :
\begin{equation}
    T_p(x)=f(p)+\nabla f(p)\cdot (x-p).
\end{equation}
Le graphe de la fonction $T_p$ sera le \defe{plan tangent}{plan!tangent} au graphe de $f$ au point $p$. L'équation du plan tangent sera donc
\begin{equation}
    z-f(p)=\nabla f(p)\cdot (x-p).
\end{equation}

\begin{remark}
    Lorsque nous utilisons la notation vectorielle, la lettre «$x$» désigne le vecteur $(x,y)$. Il faut être attentif. Dans un cas $x$ est un vecteur dans l'autre c'est une composante d'un vecteur.
\end{remark}

%---------------------------------------------------------------------------------------------------------------------------
\subsection{Projection orthogonale}
%---------------------------------------------------------------------------------------------------------------------------

Le théorème suivant n'est pas indispensablissime parce qu'il est le même que le théorème de la projection sur les espaces de Hilbert\footnote{Théorème~\ref{ThoProjOrthuzcYkz}}. Cependant la partie existence est plus simple en se limitant au cas de dimension finie.
\begin{theorem}[Théorème de la projection]  \label{ThoWKwosrH}
    Soit \( E\) un espace vectoriel réel ou complexe de dimension finie, \( x\in E\), et \( C\) un sous-ensemble fermé convexe de \(E\).
    \begin{enumerate}
        \item
            Les deux conditions suivantes sur \( y\in E\) sont équivalentes:
    \begin{enumerate}
        \item   \label{zzETsfYCSItemi}
            \( \| x-y \|=\inf\{ \| x-z \|\tq z\in C \}\),
        \item\label{zzETsfYCSItemii}
            pour tout \( z\in C\), \( \real\langle x-y, z-y\rangle \leq 0\).
    \end{enumerate}
\item
    Il existe un unique \( y\in E\), noté \( y=\pr_C(x)\) vérifiant ces conditions.
    \end{enumerate}
\end{theorem}
%TODO : il y a surement un endroit plus adapté pour mettre ce théorème.

\begin{proof}
    Nous commençons par prouver l'existence et l'unicité d'un élément dans \( C\) vérifiant la première condition. Ensuite nous verrons l'équivalence.

    \begin{subproof}
        \item[Existence]

            Soit \( z_0\in C\) et \( r=\| x-z_0 \|\). La boule fermée \( \overline{ B(x,r) }\) est compacte\footnote{C'est ceci qui ne marche plus en dimension infinie.} et intersecte \( C\). Vu que \( C\) est fermé, l'ensemble \( C'=C\cap\overline{ B(x,r) }\) est compacte. Tous les points qui minimisent la distance entre \( x\) et \( C\) sont dans \( C'\); la fonction
            \begin{equation}
                \begin{aligned}
                     C'&\to \eR \\
                    z&\mapsto d(x,z)
                \end{aligned}
            \end{equation}
            est continue sur un compact et donc a un minimum qu'elle atteint\footnote{Théorème~\ref{ThoMKKooAbHaro}.}. Un point \( P\) réalisant ce minimum prouve l'existence d'un point vérifiant la première condition.

        \item[Unicité]
            Soient \( y_1\) et \( y_2\), deux éléments de \( C\) minimisant la distance avec \( x\), et soit \( d\) ce minimum. Nous avons par l'identité du parallélogramme \eqref{EqYCLtWfJ} que
            \begin{equation}
                \| y_1-y_2 \|^2=-4\left\| \frac{ y_1+y_2-x }{2} \right\|^2+2\| y_1-x \|^2+2\| y_2-x \|^2\leq -4d+2d+2d=0.
            \end{equation}
            Par conséquent \( y_1=y_2\).

        \item[\ref{zzETsfYCSItemi}\( \Rightarrow\)~\ref{zzETsfYCSItemii}]

            Soit \( z\in C\) et \( t\in \mathopen] 0 , 1 \mathclose[\); nous notons \( P=\pr_Cx\). Par convexité le point \( z=ty+(1-t)P\) est dans \( C\), et par conséquent,
                \begin{equation}
                    \| x-P \|^2\leq\| x-tz-(1-t)P \|^2=\| (x-P)-t(z-P) \|^2.
                \end{equation}
                Nous sommes dans un cas \( \| a \|^2\leq | a-b |^2\), qui implique \( 2\real\langle a, b\rangle \leq \| b \|^2\). Dans notre cas,
                \begin{equation}
                    2\real\langle x-P , t(z-P)\rangle \leq t^2\| z-P \|^2.
                \end{equation}
                En divisant par \( t\) et en faisant \( t\to 0\) nous trouvons l'inégalité demandée :
                \begin{equation}
                    2\real\langle x-P, z-P\rangle \leq 0.
                \end{equation}

        \item[\ref{zzETsfYCSItemii}\( \Rightarrow\)~\ref{zzETsfYCSItemi}]

            Soit un point \( P\in C\) vérifiant
            \begin{equation}
                \real\langle x-P, z-P\rangle \leq 0
            \end{equation}
            pour tout \( z\in C\). Alors en notant \( a=x-P\) et \( b=P-z\),
            \begin{equation}
                \begin{aligned}[]
                \| x-z \|^2=\| x-P+P-z \|^2&=\| a+b \|^2\\
                &=\| a \|^2+\| b \|^2+2\real\langle a, b\rangle \\
                &=\| a \|^2+\| b \|^2-2\real\langle x-P, z-P\rangle \\
                &\geq \| b \|^2,
                \end{aligned}
            \end{equation}
            ce qu'il fallait.
    \end{subproof}
\end{proof}

%+++++++++++++++++++++++++++++++++++++++++++++++++++++++++++++++++++++++++++++++++++++++++++++++++++++++++++++++++++++++++++
\section{Jacobienne}
%+++++++++++++++++++++++++++++++++++++++++++++++++++++++++++++++++++++++++++++++++++++++++++++++++++++++++++++++++++++++++++

\subsection{Rappels et définitions}

Dans cette section nous considérons des fonctions $f : D \to \eR^m$
où $D \subset \eR^n$, et un point $a \in \Int D$ où $f$ est
différentiable.
\begin{remark}
  La définition de continuité (resp. différentiabilité) pour une
  fonction à valeurs vectorielles est celle introduite précédemment,
  et on remarque que pour avoir la continuité
  (resp. différentiabilité) de $f$ en un point, il faut et il suffit
  de chacune des composantes de $f = (f_1,\ldots, f_m)$, vues
  séparément comme fonctions à $n$ variables et à valeurs réelles,
  soit continue (resp. différentiable) en ce point.
\end{remark}

\begin{definition}
    La \defe{jacobienne}{matrice!jacobienne} de $f$ en $a$ est la matrice de l'application linéaire donnée par la différentielle. Elle a de nombreuses notations
  \begin{equation}
      J_f(a) = \frac{ \partial (f_1,\ldots, f_m) }{ \partial x_1,\ldots, x_m }=
    \begin{pmatrix}
      \pder {f_1} {x_1}(a) & \ldots &\pder {f_1} {x_n}(a)\\
      \vdots& & \vdots\\
      \pder {f_m} {x_1}(a) & \ldots &\pder {f_m} {x_n}(a)
    \end{pmatrix}
  \end{equation}
  Autrement dit, c'est la matrice composée de l'ensemble des dérivées partielles de $f$. Le \defe{jacobien}{jacobien} de \( f\) au point \( a\) est le déterminant de cette matrice.

  Si $m = 1$, cette matrice ne contient qu'une ligne ; c'est donc un vecteur appelé le \defe{gradient}{gradient} de $f$ au point $a$ et noté $\nabla f(a)$.
\end{definition}

\begin{remark}
  \begin{enumerate}
  \item Si la fonction est supposée différentiable, calculer la
    jacobienne revient à connaitre la différentielle. En effet, par
    linéarité de la différentielle et par définition des dérivées
    partielles, nous avons
    \begin{equation*}
      d f_a (u) =%
      \begin{pmatrix}
        \pder {f_1} {x_1}(a) & \ldots &\pder {f_1} {x_n}(a)\\
        \vdots& & \vdots\\
        \pder {f_m} {x_1}(a) & \ldots &\pder {f_m} {x_n}(a)
      \end{pmatrix}
      \begin{pmatrix}u_1\\\vdots\\u_n\end{pmatrix}
    \end{equation*}
    où $u = (u_1, \ldots, u_n)$ et où le membre de droite est un
    produit matriciel

  \item Remarquons que la jacobienne peut exister en un point donné
    sans que la fonction soit différentiable en ce point !
  \end{enumerate}
\end{remark}


\begin{normaltext}
    Le théorème de différentiation de fonctions composées \ref{PropDiffCompose} peut également se lire au niveau des matrices jacobiennes. La matrice jacobienne de $g\circ f$ au point $a$ est le produit matriciel des matrices jacobiennes de $f$ et de $f$. Plus précisément, nous avons
    \begin{equation}
        J_{g\circ f}(a)=J_g\big( f(a) \big)J_f(a).
    \end{equation}
    Remarquez que nous considérons la matrice jacobienne de $g$ au point $f(a)$.

    Dans la cas particulier où $m=1$ et $f$ est une fonction d'un intervalle $I$ dans $\eR^n$, dérivable au point $a$, on a que la fonction composée $g\circ f$ est dérivable au point $a$ si $g$ est différentiable et alors
    \[
    (g\circ f)'(a)= dg\left(f(a)\right).f'(a).
    \]
    En fait, pour les fonctions d'une seule variable la dérivabilité coïncide avec la différentiabilité.
\end{normaltext}

% This is part of Mes notes de mathématique
% Copyright (c) 2006-2019
%   Laurent Claessens, Carlotta Donadello
% See the file fdl-1.3.txt for copying conditions.

%++++++++++++++++++++++++++++++++++++++++++++++++++++++++++++++++++++++++++++++++++++++++
\section{Fonctions de classe \texorpdfstring{$ C^1$}{C1}}
%++++++++++++++++++++++++++++++++++++++++++++++++++++++++++++++++++++++++++++++++++++++++++++++++++++++++++++++++++++++++++++++

Soit $f$ une fonction différentiable de $U$, ouvert de $\eR^m$, dans $\eR^n$. L'application différentielle de $f$ est une application  de $\eR^m$ dans $\aL(\eR^m, \eR^n)$
\begin{equation}
    \begin{aligned}
        df\colon \eR^m&\to \aL(\eR^m,\eR^n) \\
        a&\mapsto df_a 
    \end{aligned}
\end{equation}
Nous savons que $\aL(\eR^m, \eR^n)$ est un espace vectoriel normé avec la définition~\ref{DefDQRooVGbzSm}. Si $T$ est un élément dans $\aL(\eR^m, \eR^n)$ alors la norme de $T$ est définie par
\[
\|T\|_{\aL(\eR^m, \eR^n)}=\sup_{x\in\eR^m} \frac{\|T(x)\|_n}{\|x\|_m}=\sup_{\begin{subarray}{l}
    x\in\eR^m\\
\|x\|_m\leq 1
  \end{subarray}} \|T(x)\|_n.
\]

Lorsqu'il existe un $M>0$ tel que $\| df(a) \|_{\aL(\eR^m,\eR^n)}<M$ pour tout $a$ dans $U$, nous disons que la différentielle de $f$ est \defe{bornée}{bornée!différentielle} sur $U$.

\begin{definition}
	La fonction $f$ est dite \defe{de classe $\mathcal{C}^1$}{fonction!de classe  $\mathcal{C}^1$} de $U\subset\eR^m$  dans $\eR^n$ si son application différentielle $df$ est continue de $\eR^m$ dans $\aL(\eR^m, \eR^n)$. Nous écrivons $f\in\mathcal{C}^1(U,\eR^n)$\nomenclature{$C^1(U,\eR^n)$}{Les applications une fois continument dérivables}.
\end{definition}

\begin{proposition}		\label{PropDerContCun}
	Une fonction \( f\colon U\to \eR^n\) où \( U\) est ouvert dans \( \eR^m\) est de classe \( C^1\) si et seulement si les dérivées partielles de $f$ existent et sont continues.
\end{proposition}

\begin{proof}
	Supposons que les dérivées partielles de $f$ existent et sont continues. Nous savons alors déjà par la proposition~\ref{Diff_totale} que la fonction $f$ est différentiable et qu'elle s'exprime sous la forme
	\[
		df_a(h)=\sum_{i=1}^{m}\partial_if (a)h_i, \qquad \forall a \in U,\,\forall h\in\eR^m.
	\]
	Pour montrer que $df$ est continue, nous devons montrer que la quantité $\| df(x)-df(a) \|_{\aL(\eR^m,\eR^n)}$ peut être rendue arbitrairement petite si $\| x-a \|_m$ est rendu petit. Nous avons
	\begin{equation}
		\begin{aligned}
			\| df_x-df_a \|_{\aL}&=\sup_{\| h \|=1}\| df_x(h)-df_a(h) \|\\
			&=\sup_{\| h \|_m=1}\left\|\sum_{i=1}^{m}\left(\partial_if (x)-\partial_if (a)\right)h_i\right\|_n\leq\\
			&\leq\sup_{\| h \|_m=1}\sum_{i=1}^{m} \left\|\left(\partial_if (x)-\partial_if (a)\right)\right\|_n|h_i|\leq\\
			&\leq\sup_{\| h \|_m=1} \|h\|_\infty\sum_{i=1}^{m} \left\|\left(\partial_if (x)-\partial_if (a)\right)\right\|_n\\
			&\leq \sum_{i=1}^m\| \partial_if(x)-\partial_if(a) \|.
		\end{aligned}
	\end{equation}
	Dans ce calcul, nous avons utilisé le fait que si $\| h \|_m\leq 1$, alors $\| h \|_{\infty}\leq 1$. Étant donné la continuité de $\partial_if$, la dernière ligne peut être rendue arbitrairement petite lorsque $x$ est proche e $a$.

Supposons maintenant que $f$ soit dans $\mathcal{C}^1(U,\eR^n)$. Alors
\[
\left\|\partial_if (x)-\partial_if (a)\right\|_n= \left\|df(x).e_i-df(a).e_i\right\|_n \leq  \left\|df(x)-df(a)\right\|_{\aL(\eR^m,\eR^n)},
\]
la continuité de $df$ implique donc celle de $\partial_i f$ pour tout $i$ dans $\{1,\ldots,m\}$.
\end{proof}
\begin{proposition}
  Soient $U$ un ouvert de $\eR^m$ et $V$ un ouvert de $\eR^n$. Soient $f: U\to V$  dans $\mathcal{C}^1(U,V)$ et $g: V \to \eR^p$ dans $\mathcal{C}^1(V,\eR^n)$.  Alors la fonction composée $g\circ f: U\to \eR^p $ est dans $\mathcal{C}^1(U,\eR^p)$.
\end{proposition}
\begin{proof} On fixe $a$ dans $U$
  \begin{equation}
    \begin{aligned}
     \big\|d(g\circ f)(x)&-d(g\circ f)(a)\big\|_{\aL(\eR^m,\eR^p)}\\
     &=\left\|dg(f(x))\circ df(x)-dg(f(a))\circ df(a)\right\|_{\aL(\eR^m,\eR^p)}\leq\\
&\leq \left\|\left(dg(f(x))-dg(f(a))\right)\circ df(x)\right\|_{\aL(\eR^m,\eR^p)}+\\
&\quad+ \left\|dg(f(a))\circ \left(df(x)-df(a)\right)\right\|_{\aL(\eR^m,\eR^p)}\leq\\
&\leq \left\|dg(f(x))-dg(f(a))\right\|_{\aL(\eR^n,\eR^p)}\left\| df(x)\right\|_{\aL(\eR^m,\eR^n)}+\\
&\quad+ \left\|dg(f(a))\right\|_{\aL(\eR^n,\eR^p)}\left\| df(x)-df(a)\right\|_{\aL(\eR^n,\eR^p)}.\\
    \end{aligned}
  \end{equation}
On peut conclure en passant à la limite $x\to a$ parce que les fonctions $f$, $g$, $df$ et $dg$ sont continues, de telle sorte que
\begin{equation}
	\begin{aligned}[]
		\lim_{x\to a} dg\big( f(x) \big)=dg\big( f(a) \big)\\
		\lim_{x\to a} df(x)=df(a).
	\end{aligned}
\end{equation}
\end{proof}

\begin{remark}
  On peut prouver le même résultat en utilisant la continuité de l'application bilinéaire
\begin{equation}
  \begin{array}{rccc}
    \circ : & \mathcal{C}^1(U,V)\times\mathcal{C}^1(V,\eR^p)  & \to & \aL(U, \eR^p)\\
& (T,S)& \mapsto & T\circ S.
  \end{array}
\end{equation}
\end{remark}


%+++++++++++++++++++++++++++++++++++++++++++++++++++++++++++++++++++++++++++++++++++++++++++++++++++++++++++++++++++++++++++
\section{Différentielle et dérivée complexe}
%+++++++++++++++++++++++++++++++++++++++++++++++++++++++++++++++++++++++++++++++++++++++++++++++++++++++++++++++++++++++++++
\label{SECooJWNOooOgMiWR}

\begin{normaltext}
    Nous commençons par donner quelques éléments à propos de dérivée et de différentielle pour des fonctions \( \eC\to \eC\) parce que les séries entières vont souvent être des fonctions complexes. Le gros du chapitre sur les fonctions holomorphes est le chapitre~\ref{ChapICHIooXbLccl}.
\end{normaltext}

Nous identifions \( \eR^2\) à \( \eC\) par l'application
\begin{equation}
    \begin{aligned}
        \varphi\colon \eR^2&\to \eC \\
        (x,y)&\mapsto x+iy.
    \end{aligned}
\end{equation}
Dans cette partie, nous désignons par \( \Omega\) un ouvert de \( \eC\).

\begin{definition}      \label{DEFooVJVXooKlnFkh}
    Une fonction \( f\colon \Omega\to \eC\) est \defe{$\eC$-dérivable}{dérivable!au sens complexe} si la limite
    \begin{equation}
        \lim_{\substack{h\to0\\h\in \eC}} \frac{ f(a+h)-f(a) }{ h }
    \end{equation}
    existe. Dans ce cas, cette limite est la dérivée de \( f\) et est notée \( f'\).
\end{definition}

\begin{definition}  \label{DefMMpjJZ}
    Soit \( \Omega\) un ouvert dans \( \eC\). Une fonction \( f\colon \Omega\to \eC\) est \defe{holomorphe}{holomorphe}\index{fonction!holomorphe} si elle est \( \eC\)-dérivable sur \( \Omega\).
\end{definition}

\begin{definition}      \label{DEFooQSMCooOoWVZk}
    Si \( K\) est un compact de \( \eC\), nous disons qu'une fonction est \defe{holomorphe}{holomorphe!sur un compact} sur \( K\) si il existe un ouvert contenant \( K\) sur lequel \( f\) est holomorphe.

    Et si \( f\) n'est réellement définie que sur \( K\), elle est holomorphe sur \( K\) si il existe une extension holomorphe de \( f\) vers un ouvert contenant \( K\).
\end{definition}

\begin{definition}
    Une matrice de la forme
    \begin{equation}
        \begin{pmatrix}
            \alpha    &   \beta    \\
            -\beta    &   \alpha
        \end{pmatrix}
    \end{equation}
    avec \( \alpha,\beta\in \eR\) est une \defe{similitude}{matrice!de similitude}\index{similitude}.
\end{definition}

\begin{lemma}       \label{LEMooJNFEooZCbJMo}
    En tant qu'application linéaire \( \eC\to \eC\), l'opération de multiplication par \( \alpha+\beta i\) est la matrice
    \begin{equation}
        \begin{pmatrix}
            \alpha    &   -\beta    \\
            \beta    &   \alpha
        \end{pmatrix}.
    \end{equation}
\end{lemma}

\begin{proof}
    Cela est vite remarqué en calculant explicitement \( (\alpha+\beta i)(u_1+iu_2)\).
\end{proof}

\begin{lemma}
    Une application \( A\colon \eC\to \eC\) est \( \eC\)-linéaire si et seulement si elle est une similitude en tant qu'application \( \eR^2\to \eR^2\).

    Dans ce cas, il existe \( z_0\in \eC\) tel que \( A(z)=z_0z\) pour tout \( z\in \eC\).
\end{lemma}

\begin{proof}
    Commençons par considérer l'application \( A\) sur \( \eR^2\). Elle est en particulier une application \( \eR\)-linéaire et par conséquent il existe une matrice \( \begin{pmatrix}
        \alpha    &   \beta    \\
        \gamma    &   \delta
    \end{pmatrix}\) telle que
    \begin{equation}
        A\begin{pmatrix}
            x    \\
            y
        \end{pmatrix}=\begin{pmatrix}
            \alpha    &   \beta    \\
            \gamma    &   \delta
        \end{pmatrix}\begin{pmatrix}
            x    \\
            y
        \end{pmatrix}.
    \end{equation}
    Nous voulons maintenant imposer la \( \eC\)-linéarité, c'est-à-dire que nous voulons
    \begin{equation}
        A\big( (a+bi)(x+iy) \big)=(a+bi)A(x+iy)
    \end{equation}
    pour tout \( a,b,x,y\in \eR\). À gauche nous avons
    \begin{equation}
        A\big( ax-by+i(bx+ay) \big)
    \end{equation}
    et à droite nous avons
    \begin{equation}
        (a+bi)\big( \alpha x+\beta y+i(\gamma x+\delta y) \big).
    \end{equation}
    En égalant les deux expressions nous obtenons les équations
    \begin{subequations}
        \begin{numcases}{}
            \beta b=-b\gamma\\
            -\alpha b+\beta a =a\beta -b\delta\\
            \delta b=b\alpha\\
            -\gamma b+\delta a=b\beta+a\delta,
        \end{numcases}
    \end{subequations}
    dont nous tirons immédiatement que \( \gamma=-b\beta\) et \( \delta=\alpha\). La matrice de \( A\) est donc de la forme demandée.

    Inversement nous devons prouver que la fonction
    \begin{equation}        \label{EqOEWYooMaHCNb}
        f(x+iy)=\alpha x+\beta y+i(-\beta x+\alpha y)
    \end{equation}
    est \( \eC\)-linéaire, c'est-à-dire qu'elle vérifie \( f(z_0z)=z_0f(z)\) pour tout \( z_0,z\in \eC\). Cela est un simple calcul que nous confions à Sage : le code suivant affiche «\( 0\)».
    \lstinputlisting{tex/frido/code_sage3.py}

    Pour conclure, notons que la fonction \eqref{EqOEWYooMaHCNb} est la fonction de multiplication par \( \alpha-i\beta\).
\end{proof}

\begin{normaltext}      \label{NORMooMKNDooBeoGRN}
    Soient une fonction \( f\colon \eC\to \eC\) et l'isomorphisme canonique \( \varphi\colon \eC\to \eR^2\). La fonction \( f\) définit une la fonction
    \begin{equation}
        F=\varphi^{-1} \circ f\circ \varphi\colon \eR^2\to \eR^2.
    \end{equation}
    Cela est la fonction \( \eR^2\to \eR^2\) associée à \( f\). Il serait tentant de croire que tout ce qui est vrai pour \( F\) est également vrai pour \( f\). Eh bien non.

    Par exemple, \( F\) peut être différentiable sans que \( f\) le soit. La proposition suivant donne une condition sur \( dF\) pour que \( f\) soit différentiable.
\end{normaltext}

\begin{proposition}     \label{PropKJUDooJfqgYS}
    Une fonction \( f\colon \Omega\to \eC\) est $\eC$-dérivable en \( a\in\Omega\) si et seulement si elle est différentiable en \( a\) et si \( df_a\) est une similitude.

    Plus précisément avec les notations de~\ref{NORMooMKNDooBeoGRN}, la fonction \( f\) est $\eC$-dérivable (donc holomorphe) au point \( z_0=x_0+iy_0\) si et seulement si la fonction \( F\) est différentiable en \( (x_0,y_0)\) et si la matrice de \( dF\) est de la forme
    \begin{equation}        \label{EQooWZGKooLDEHGr}
        dF=\begin{pmatrix}
            \alpha    &   \beta    \\
            -\beta    &   \alpha
        \end{pmatrix},
    \end{equation}
    c'est-à-dire si \( dF_{(x_0,y_0)}\) fournit une application \( \eC\)-linéaire.

    Dans ce cas, le lien entre \( \eC\)-dérivée et différentielle est donné par
    \begin{equation}        \label{EqPAEFooYNhYpz}
        (df_{z_0})(z)=f'(z_0)z.
    \end{equation}
\end{proposition}

\begin{proof}
    Nous décomposons \( f\) en parties réelles et imaginaires :
    \begin{equation}
        f(x+iy)=P(x,y)+iQ(x,y)
    \end{equation}
    où \( P\) et \( Q\) sont des fonctions réelles. La jacobienne de \( F\) est la matrice
    \begin{equation}
        \begin{pmatrix}
            \frac{ \partial P }{ \partial x }    &   \frac{ \partial P }{ \partial y }    \\
            \frac{ \partial Q }{ \partial x }    &   \frac{ \partial Q }{ \partial y }
        \end{pmatrix},
    \end{equation}
    et la condition dont nous parlons s'écrit comme le système
    \begin{subequations}    \label{EqFDUrXBP}
        \begin{numcases}{}
            \frac{ \partial P }{ \partial x }=\frac{ \partial Q }{ \partial y }\\
            \frac{ \partial P }{ \partial y }=-\frac{ \partial Q }{ \partial x}.
        \end{numcases}
    \end{subequations}
    Si \( F\) est différentiable en \( (x_0,y_0)\) alors nous avons
    \begin{equation}        \label{EqwlVfiR}
        F\big( (x_0,y_0)+(h,k) \big)=F(x_0,y_0)+dF_{(x_0,y_0)}\begin{pmatrix}
            h    \\
            k
        \end{pmatrix}+s(| h |+| k |)
    \end{equation}
    où \( s\) est une fonction vérifiant \( \lim_{t\to 0} \frac{ s(t) }{ t }=0\). Soit
    \begin{equation}
        dF_{(x_0,y_0)}=\begin{pmatrix}
            \alpha    &   \beta    \\
            -\beta    &   \alpha
        \end{pmatrix}.
    \end{equation}
    Si nous posons \( \sigma=\alpha-i\beta\) et \( w=h+ik\), l'équation \eqref{EqwlVfiR} s'écrit dans \( \eC\) sous la forme
    \begin{equation}        \label{EqYFmoiM}
        f(z_0+w)=f(z_0)+\sigma w+s(|w|),
    \end{equation}
    ce qui implique que \( f\) est $\eC$-dérivable en \( z_0\).

    Supposons maintenant que \( f\) soit $\eC$-dérivable en \( z_0\). Alors nous avons
    \begin{equation}
        f'(z_0)=\lim_{w\to 0} \frac{ f(z_0+w)-f(z_0) }{ w }=\sigma\in \eC,
    \end{equation}
    ce qui se récrit sous la forme
    \begin{equation}
        \lim_{w\to 0} \frac{ f(z_0+w)-f(z_0)-\sigma w }{ w }=0.
    \end{equation}
    Si nous posons \( z_0=x_0+iy_0\), \( w=h+ik\) et \( \sigma=\alpha-i\beta\) nous avons
    \begin{equation}
        \lim_{(h,k)\to (0,0)} \left| \frac{ F\big( (x_0,y_0)+(h,k) \big)-F(x_0,y_0)-\begin{pmatrix}
            \alpha    &   \beta    \\
            -\beta    &   \alpha
        \end{pmatrix}\begin{pmatrix}
            h    \\
            k
        \end{pmatrix}}{ | w | } \right| =0,
    \end{equation}
    ce qui signifie que \( F\) est différentiable et que sa différentielle est la matrice
    \begin{equation}    \label{EqMLtbLD}
       \begin{pmatrix}
           \alpha &   \beta    \\
           -\beta &   \alpha
       \end{pmatrix}.
    \end{equation}

    La matrice \eqref{EqMLtbLD} est, vue dans \( \eR^2\), la matrice de multiplication dans \( \eC\) par \( \alpha-i\beta=f'(z_0)\). En d'autre termes, dans \( \eC\) nous avons
    \begin{equation}
        df_{z_0}(z)=f'(z_0)z,
    \end{equation}
    et en particulier la différentielle est donnée par
    \begin{equation}        \label{EqPropZOkfmO}
        df_{z_0}=f'(z_0)dz.
    \end{equation}
\end{proof}

\begin{example}[Une application \( C^{\infty}\) mais pas \( \eC\)-dérivable]
    Nous considérons la fonction
    \begin{equation}
        \begin{aligned}
            f\colon \eC&\to \eC \\
            x+iy&\mapsto x.
        \end{aligned}
    \end{equation}
    Vu que c'est une application linéaire, elle est différentiable une infinité de fois et sa différentielle est elle-même. C'est donc une application \( C^{\infty}\).

    Elle n'est cependant pas \( \eC\)-dérivable. En effet le quotient différentiel est, pour \( \epsilon\in \eC\) :
    \begin{equation}
        \frac{ f(x+iy+\epsilon_x+i\epsilon_y)-f(x+iy) }{ \epsilon }=\frac{ \epsilon_x }{ \epsilon }.
    \end{equation}
    Cela n'a pas de limite lorsque \( \epsilon\to 0\). Pour voir cela nous invoquons la méthode des chemins du corolaire~\ref{CorMethodeChemin} avec les chemins \( \epsilon_1(t)=t\) et \( \epsilon_2(t)=it\). Dans le premier cas, le quotient différentiel vaut \( 1\) pour tout \( t\), tandis que dans le second il vaut zéro pour tout \( t\).
\end{example}

%---------------------------------------------------------------------------------------------------------------------------
\subsection{Quelques règles de calcul}
%---------------------------------------------------------------------------------------------------------------------------

\begin{lemma}       \label{LEMooVDXOooUyFHXZ}
    Si \( f\) et \( g\) sont deux fonctions holomorphes sur un ouvert \( \Omega\subset \eC\) et si \( g\) ne s'annule pas sur \( \Omega\), alors \( f/g\) est holomorphe sur \( \Omega\).
\end{lemma}


%++++++++++++++++++++++++++++++++++++++++++++++++++++++++++++++++++++++++++++++++++++++++++++++++++++++++++++++++++++++
\section{Théorèmes des accroissements finis}		\label{SecThoAccrsFinis}
%++++++++++++++++++++++++++++++++++++++++++++++++++++++++++++++++++++++++++++++++++++++++++++++++++++++++++++++++++++++

Nous avons déjà démontré (lemme~\ref{LemdfaSurLesPartielles}) que si $f$ est différentiable au point $x$ alors  $df_x(u)=\partial_uf(x)$. Une importante conséquence est le théorème des accroissements finis
\begin{theorem}[Accroissements finis, inégalité de la moyenne]\label{val_medio_2}
   Soit $U$ un ouvert dans $\eR^m$ et soit $f:U\to\eR^n$ une fonction différentiable. Soient $a$ et $b$ deux points dans $U$, $a\neq b$, tels que le segment $[a,b]$ soit contenu dans $U$. Alors
   \begin{equation}
        \|f(b)-f(a)\|_n\leq \sup_{x\in[a,b]}\|df(x)\|_{\aL(\eR^m,\eR^n)}\|b-a\|_m.
   \end{equation}
\end{theorem}
\index{application!différentiable}
\index{inégalité!de la moyenne}
\index{théorème!accroissements finis!forme générale}

\begin{proof}
 On utilise le théorème~\ref{val_medio_1} et le fait que
\[
\|\partial_u f(x)\|_n\leq \|df(x)\|_{\aL(\eR^m,\eR^n)}\|u\|_m,
\]
pour tout $u$ dans $\eR^m$.
\end{proof}

La proposition suivante est une application fondamentale du théorème des accroissements finis~\ref{val_medio_2}.
\begin{proposition}		\label{PropAnnulationEtConstance}
	Soit $U$ un ouvert connexe par arcs de $\eR^m$ et une fonction $f\colon U\to \eR^n$. Les conditions suivantes sont équivalentes :
	\begin{enumerate}
		\item\label{ItemPropCstDiffZeroi}
			$f$ est constante;
		\item\label{ItemPropCstDiffZeroii}
			$f$ est différentiable et $df(a)=0$ pour tout $a\in U$;
		\item\label{ItemPropCstDiffZeroiii}
			les dérivées partielles $\partial_1f,\ldots,\partial_mf$ existent et sont nulles sur $U$.
	\end{enumerate}
\end{proposition}
\index{connexité!par arc!fonction différentiable}
\index{différentiabilité}

\begin{proof}
	Nous allons démonter les équivalences en plusieurs étapes. D'abord~\ref{ItemPropCstDiffZeroi} $\Rightarrow$~\ref{ItemPropCstDiffZeroii}, puis~\ref{ItemPropCstDiffZeroii} $\Rightarrow$~\ref{ItemPropCstDiffZeroiii}, ensuite~\ref{ItemPropCstDiffZeroiii} $\Rightarrow$~\ref{ItemPropCstDiffZeroii} et enfin~\ref{ItemPropCstDiffZeroii} $\Rightarrow$~\ref{ItemPropCstDiffZeroi}.

	Commençons par montrer que la condition~\ref{ItemPropCstDiffZeroi} implique la condition~\ref{ItemPropCstDiffZeroii}. Si $f(x)$ est constante, alors la condition \eqref{EqCritereDefDiff} est vite vérifiée en posant $T(h)=0$.

	Afin de voir que la condition~\ref{ItemPropCstDiffZeroii} implique la condition~\ref{ItemPropCstDiffZeroiii}, remarquons d'abord que la différentiabilité de $f$ implique que les dérivées partielles existent (proposition~\ref{diff1}) et que nous avons l'égalité $df(a).u=\sum_iu_i\partial_if(a)$ pour tout $u\in\eR^m$ (lemme~\ref{LemdfaSurLesPartielles}). L'annulation de $\sum_iu_i\partial_if(a)$ pour tout $u$ implique l'annulation des $\partial_if(a)$ pour tout $i$.

	Prouvons maintenant que la propriété~\ref{ItemPropCstDiffZeroiii} implique la propriété~\ref{ItemPropCstDiffZeroii}. D'abord, par la proposition~\ref{Diff_totale}, l'existence et la continuité des dérivées partielles $\partial_if(a)$ implique la différentiabilité de $f$. Ensuite, la formule $df(a).u=\sum_i u_i\partial_if(a)$ implique que $df(a)=0$.


	Il reste à montrer que~\ref{ItemPropCstDiffZeroii} implique la condition~\ref{ItemPropCstDiffZeroi}, c'est-à-dire que l'annulation de la différentielle implique la constance de la fonction. C'est ici que nous allons utiliser le théorème des accroissements finis. En effet, si $a$ et $b$ sont des points de $U$, le théorème~\ref{val_medio_2} nous dit que
	\begin{equation}
		\|f(b)-f(a)\|_n\leq \sup_{x\in[a,b]}\|df(x)\|_{\aL(\eR^m,\eR^n)}\|b-a\|_m.
	\end{equation}
	Mais $\| df(x) \|=0$ pour tout $x\in U$, donc ce supremum est nul et $f(b)=f(a)$, ce qui signifie la constance de la fonction.
\end{proof}

%\begin{proof}
%  \begin{itemize}
%  \item Le théorème~\ref{val_medio_2} nous dit que si la différentielle de $f$ est nulle alors $f$ est constante sur chaque segment contenu dans $U$. Cela nous dit que $f$ est constante sur chaque boule contenue dans $U$, donc $f $ est localement constante. Il est possible de démontrer que toute fonction localement constante sur un connexe est constante.
%\item Si toutes les dérivées partielles $\partial_1 f, \ldots, \partial_m f $ existents et sont identiquement nulles sur $U$ alors $f$ est différentiable et sa différentielle est identiquement nulle. On utilise la première partie de la preuve pour conclure.
%  \end{itemize}
%\end{proof}

%+++++++++++++++++++++++++++++++++++++++++++++++++++++++++++++++++++++++++++++++++++++++++++++++++++++++++++++++++++++++++++
\section{Fonctions Lipschitziennes}
%+++++++++++++++++++++++++++++++++++++++++++++++++++++++++++++++++++++++++++++++++++++++++++++++++++++++++++++++++++++++++++


\begin{definition}      \label{DEFooQHVEooDbYKmz}
    Soient \( (E,d_E)\) et \( (F,d_F)\) deux espaces métriques\footnote{Pour rappel, les espaces métriques sont définis par la définition~\ref{DefMVNVFsX} et le théorème~\ref{ThoORdLYUu}; je précise que nous ne supposons pas que \( E\) soit vectoriel; en particulier il peut être un ouvert de \( \eR^n\).}, \( f\colon E\to F\) une application et un réel \( k\) strictement positif. Nous disons que \( f\) est \defe{Lipschitzienne}{Lipschitzienne} de constante $k$ sur \( E\) si pour tout \( x,y\in E\),
    \begin{equation}
        d_F\big( f(x)-f(y) \big)\leq kd_E(x,y).
    \end{equation}
\end{definition}
%TODO : faire la chasse aux endroits où cette définition devrait être référencée.
Soit \( f\) une fonction \( k\)-Lipschitzienne. Si \( y\in \overline{ B(x,\delta)}\) alors \( \| x-y \|\leq\delta\) et donc \( \big\| f(x)-f(y) \big\|\leq k\delta\). Cela signifie que la condition Lipschitz pour s'énoncer en termes de boules fermées par
\begin{equation}    \label{EqDZvtUbn}
    f\big( \overline{ B(x,\delta) } \big)\subset \overline{  B\big( f(x),k\delta \big) }
\end{equation}
tant que \( \overline{ B(x,\delta) } \) est contenue dans le domaine sur lequel \( f\) est Lipschitz.

\begin{proposition}
  Soit  $U$ un ouvert convexe  de $\eR^m$, et soit $f:U\to \eR^n$ une fonction différentiable. La fonction $f$ est Lipschitzienne sur $U$ si et seulement si $df$ est bornée sur $U$.
\end{proposition}
\begin{proof}
	Le fait que l'application différentielle $df$ soit bornée signifie qu'il existe un $M>0$ dans $\eR$ tel que $\|df_a\|_{\aL(\eR^m,\eR^n)}\leq M$, pour tout $a$ dans $U$. Si cela est le cas, alors le théorème~\ref{val_medio_2} et la convexité\footnote{La convexité de $U$ sert à assurer que la droite reliant $a$ à $b$ est contenue dans $U$; c'est ce que nous utilisons dans la démonstration du théorème~\ref{val_medio_2}.} de $U$ impliquent évidemment que $f$ est de Lipschitz de constante plus petite ou égale à $M$.

	Inversement, si $f$ est Lipschitz de constante $k$, alors pour tout $a$ dans $U$ et $u$ dans $\eR^m$ on a
	\[
		\left\|\frac{f(a+tu)-f(a)}{t}\right\|_n\leq k \|u\|_m,
	\]
	En passant à la limite pour $t\to 0$ on a
	\[
		\|\partial_u f(a)\|_n=\|df_a(u)\|_n\leq k \|u\|_m,
	\]
	donc la norme de $df_a$ est majorée par $k$ pour tout $a$ dans $U$.
\end{proof}

Notez cependant qu'une fonction peut être Lipschitzienne sans être différentiable.

\begin{proposition} \label{PropFZgFTEW}
    Une fonction Lipschitzienne \( f\colon \eR\to \eR\) est continue.
\end{proposition}

\begin{proof}
    Nous utilisons la caractérisation de la continuité donnée par le théorème~\ref{ThoESCaraB}. Prouvons donc la continuité en \( a\in \eR\). Pour tout \( x\) nous avons
    \begin{equation}
        \big| f(x)-f(a) \big|\leq k| x-a |.
    \end{equation}
    Si \( \epsilon>0\) est donné, il suffit de prendre \( \delta<\frac{ \epsilon }{ k }\) pour avoir
    \begin{equation}
        \big| f(x)-f(a) \big|\leq k\frac{ \epsilon }{ k }=\epsilon.
    \end{equation}
    Donc \( f\) est continue en \( a\).
\end{proof}

\begin{definition}      \label{DefJSFFooEOCogV}
    Une fonction
    \begin{equation}
        \begin{aligned}
            f\colon \eR^n\times \eR^m&\to \eR^p \\
            (t,y)&\mapsto f(t,y)
        \end{aligned}
    \end{equation}
    est \defe{localement Lipschitz}{Lipschitz!localement} en \( y\) au point \( (t_0,y_0)\) s'il existe des voisinages \( V\) de \( t_0\) et \( W\) de \( y_0\) et un nombre \( k>0\) tels que pour tout \( (t,y)\in V\times W\) on ait
    \begin{equation}
        \big\| f(t_0,y_0)-f(t,y) \big\|\leq k\| y-y_0 \|.
    \end{equation}
    La fonction est localement Lipschitz sur un ouvert \( U\) de \( \eR^n\times \eR^m\) si elle est localement Lipschitz en chaque point de \( U\).
\end{definition}

\begin{normaltext}      \label{NORMooYNRAooBgobcK}
    Autrement dit, une fonction est localement Lipschitzienne en sa deuxième variable lorsque tout point admet un voisinage sur lequel elle est Lipschitzienne.
\end{normaltext}

\begin{proposition} \label{PROPooVZSAooUneOQK}
    Une application Lipschitz\footnote{Définition~\ref{DEFooQHVEooDbYKmz}.} est uniformément continue.
\end{proposition}

\begin{proposition}     \label{PropGIBZooVsIqfY}
    Si \( f\) et \( g\) sont deux fonctions localement Lipschitz alors \( f+g\) l'est.
\end{proposition}

\begin{proof}
    Il s'agit d'un simple calcul avec une majoration standard :
    \begin{subequations}
        \begin{align}
            \| (f+g)(t_0,y_0)-(f+g)(t,y) \|&\leq \|  f(t_0,y_0)-f(t,y)  \|+\| g(t_0,y_0)-g(t,y) \|\\
            &\leq k_f\| y-y_0 \|+k_g\| y-y_0 \|\\
            &=(k_f+k_g)\| y-y_0 \|.
        \end{align}
    \end{subequations}
\end{proof}

\begin{lemma}   \label{LemCFZUooVqZmpc}
    La fonction donné par
    \begin{equation}
        f(t, (x,y) )=xy
    \end{equation}
    est localement Lipschitz en tout point.
\end{lemma}

\begin{proof}
    Nous avons la majoration classique
    \begin{equation}
        | f\big(t,(x_0,y_0)\big)-f\big( t,(x,y) \big) |=| x_0y_0-xy |\leq| x_0y_0-x_0y |+| x_0y-xy |\leq | x_0 || y_0-y |+| y || x_0-x |.
    \end{equation}
    Vu que nous parlons de fonction \emph{localement Lipschitzienne}, nous pouvons majorer \( | y |\) et \( | x_0 |\) par un même nombre \( k\) dans un voisinage de \( (x_0,y_0)\). Cela donne
    \begin{equation}
        | f\big(t,(x_0,y_0)\big)-f\big( t,(x,y) \big) |\leq k\big( | y_0-y |+| x_0-x | \big)\leq \sqrt{2}k\| \begin{pmatrix}
            x_0-x    \\
            y_0-y
        \end{pmatrix}\|.
    \end{equation}
    Nous avons utilisé l'équivalence de norme de la proposition~\ref{PropLJEJooMOWPNi}\ref{ItemABSGooQODmLNi}.
\end{proof}

%++++++++++++++++++++++++++++++++++++++++++++++++++++++++++++++++++++++++++++++++++++++++
\section{Différentielles d'ordre supérieur}		\label{SecDiffOrdSup}
%++++++++++++++++++++++++++++++++++++++++++++++++++++++++++++++++++++++++++++++++++++++++++++++++++++++++++++++++++++++++++++++

\begin{definition}
    Soient deux espaces vectoriels normés \( V\) et \( W\) ainsi qu'une application \( f\colon V\to W\). Nous disons que \( f\) est de classe \( C^1\) si elle est différentiable et si sa différentielle \( df\colon V\to \aL(V,W)\) est une application continue.

    Nous disons que \( f\) est de classe \( C^k\) si sa différentielle est de classe \( C^{k-1}\).
\end{definition}

Soient deux espaces vectoriels normés \( V\) et \( W\) ainsi qu'une application \( f\colon V\to W\). La différentielle est une application \( df\colon V\to \aL(V,W)\). Pour être clair, la différentielle seconde consiste à différentier \( df_x\) par rapport à \( x\). C'est à dire que la différentielle seconde est une application \( d(df)\colon V\to \aL\big( V,\aL(V,W) \big)\).

Et c'est là que commencent les problèmes. Les différentielles successives font intervenir des emboîtements de plus en plus profonds d'espaces comme \( d^2f\colon V\to \aL\Big( V,\aL\big( V,\aL(V,W) \big) \Big)\).

Nous introduisons maintenant quelque notations et lemmes pour traiter ces problèmes. Soient des espaces vectoriels normés \( V\) et \( W\) Nous introduisons le produit suivant\cite{MonCerveau} :
\begin{equation}
    \begin{aligned}
        \cdot\colon W\times \aL\big( V,\aL(V,\eR) \big)   &\to \aL\big( V,\aL(V,W) \big) \\
        \big( (w\cdot \psi)(u) \big)v&=\big( \psi(u)v \big)w. 
    \end{aligned}
\end{equation}
Dans la suite, pour économiser des parenthèses et des maux de tête, nous allons noter \( \psi(u,v)\) le nombre \( \psi(u)v\). Il n'est cependant pas question de dire que \( \psi\) est une application bilinéaire. En effet, identifier \( \aL\big( V,\aL(V,W) \big)\) à l'espace des applications bilinéaires \( V\times V\to W\) ne sert pas à grand chose pour l'instant parce qu'une telle identification a le prix de devoir prouver que toutes les propriétés des différentielles passent à travers l'identification, tâche qui est à priori (conservation de la difficulté) de la même nature que celle à laquelle nous nous attachons à présent.

\begin{lemma}[\cite{MonCerveau}]        \label{LEMooHCUSooXYHuBR}
    Soient deux espaces vectoriels normés \( V\) et \( E\) ainsi que \( \psi\in\aL\big( V,\aL(V,\eR) \big)\). Pour tout \( a\in E\) nous avons
    \begin{equation}
        \| a\cdot \psi \|_{\aL\big( V,\aL(V,E) \big)}= \| \psi \|_{\aL\big( V,\aL(V,\eR) \big)}\| a \|_E.
    \end{equation}
\end{lemma}

\begin{proof}
    Il s'agit seulement d'un calcul :
    \begin{subequations}
        \begin{align}
            \| a\cdot \psi \|&=\sup_{\| v \|=1}\| (a\cdot \psi)(v) \|_{\aL(V,E)}\\
            &=\sup_{\| v \|=1}\sup_{\| w \|=1}\| (a\cdot\psi)(v)w \|_E\\
            &=\sup_{\| v \|=1}\sup_{\| w \|=1}\| \big( \psi(v)w \big)a \|_E  \label{SUBEQooDVSVooFkgDQb}   \\
            &=\sup_{\| v \|=1}\sup_{\| w \|=1}| \psi(v)w |\,\| a \|_{E} \label{SUBEQooBJQDooDyZMOy}   \\
            &=\sup_{\| v \|=1}\| \psi(v) \|\,\| a \|\\
            &=\| \psi \|\| a \|.
        \end{align}
    \end{subequations}
    Notez que dans \eqref{SUBEQooDVSVooFkgDQb}, \( | \psi(v)w |\) est un simple réel; c'est pourquoi nous le retrouvons hors de la norme \( \| . \|_E\) dans \eqref{SUBEQooBJQDooDyZMOy}, muni d'une simple valeur absolue.
\end{proof}

\begin{lemma}[\cite{MonCerveau}]
    Soient \( \psi\in\aL\big( V,\aL(V,\eR) \big)\) et une fonction continue \( f\colon V\to E\). Alors la fonction
    \begin{equation}
        \begin{aligned}
            g\colon V&\to \aL\big( V,\aL(V,E) \big) \\
            x&\mapsto f(x)\cdot \psi 
        \end{aligned}
    \end{equation}
    est continue.   
\end{lemma}

\begin{proof}
    Pour des raisons de notations, nous allons écrire \( g_x\) pour \( g(x)\). Cela étant dit nous considérons \( a\in E\), une suite \( x_k\stackrel{E}{\longrightarrow}a \) et nous calculons :
    \begin{subequations}
        \begin{align}
            \| g_a-g_{x_k} \|&=\sup_{\| u \|=1}\| g_a(u)-g_{x_k}(u) \|_{\aL(V,E)}\\
            &=\sup_{\| u \|=1}\sup_{\| v \|=1}\| g_a(u)v-g_{x_k}(u)v \|_{E}\\
            &=\sup_{\| u \|=1}\sup_{\| v \|=1}\| f(a)\psi(u)v-f(x_k)\psi(u)v \|\\
            &=\sup_{\| u \|=1}\sup_{\| v \|=1}\| f(a)-f(x_k) \|\big| \psi(u)v \big|\\
            &=\big| f(a)-f(x_k) \big|\sup_{u,v}| \psi(u)v |\\
            &=\| f(a)-f(x_k) \|\,\| \psi \|.
        \end{align}
    \end{subequations}
    En prenant la limite \( k\to \infty\), et en considérant que \( f\) est continue en \( a\), nous obtenons
    \begin{equation}
        \lim_{k\to \infty} \| g_a-g_{x_k} \|=0.
    \end{equation}
\end{proof}

\begin{lemma}[\cite{MonCerveau}]
    Soient une application différentiable \( f\colon V\to E\) ainsi que \( \psi\in\aL\big( V,\aL(V,\eR) \big)\). Soit
    \begin{equation}
        \begin{aligned}
            g\colon V&\to \aL\big(V, \aL(V,E) \big) \\
            x&\mapsto f(x)\cdot \psi. 
        \end{aligned}
    \end{equation}
    Alors \( g\) est différentiable et pour tout \( a\in V\) nous avons
    \begin{equation}
        dg_a(h)=df_a(h)\cdot \psi.
    \end{equation}
\end{lemma}

Notons que nous n'avons pas \( dg_a=df_a\cdot \psi\). En effet, \( df_a\in \aL( V,W )\), de telle sorte que \( df_a\cdot\psi\in \aL\big( V,\aL\big( V,\aL(V,W) \big) \big)\). Les espaces ne s'emboîtent pas dans le bon ordre.

\begin{proof}
    Il s'agit de vérifier que \( h\mapsto df_a(h)\cdot \psi\) vérifie la condition de la définition \ref{DefDifferentiellePta}. En utilisant le fait que \( (a+b)\cdot \psi = a\cdot \psi+b\cdot \psi\) ainsi que le lemme \ref{LEMooHCUSooXYHuBR} nous écrivons
    \begin{subequations}
        \begin{align}
            \frac{ \| f(a+h)\cdot \psi-f(a)\cdot\psi-df_a(h)\cdot \psi\|  }{ \| h \| }&=\| \frac{ f(a+h)-f(a)-df_a(h) }{ h  }\cdot \psi\| \\
            &=\| \frac{ f(a+h)-f(a)-df_a(h) }{ h } \|\| \psi \|.
        \end{align}
    \end{subequations}
    Vu que \( f\) est différentiable en \( a\) et que \( df_a\) est la différentielle, nous avons bien
    \begin{equation}
        \lim_{h\to 0}  \| \frac{ f(a+h)-f(a)-df_a(h) }{ h } \|\| \psi \|=0.
    \end{equation}
\end{proof}


%--------------------------------------------------------------------------------------------------------------------------- 
\subsection{Espaces d'applications multilinéaires et identifications}
%---------------------------------------------------------------------------------------------------------------------------

Si \( V\) et \( E\) sont des espaces vectoriels de dimensions finies, la différentielle de \( f\colon V\to E\) est une application \( df\colon V\to \aL(V,E)\). La différentielle seconde est une application \( d(df)\colon V\to \aL\big( V,\aL(V,E) \big)\) et ainsi de suite.

Une grande difficulté de la manipulation des différentielles d'ordre supérieurs provient de cet emboîtement d'espaces d'applications linéaires. Nous nous attaquons à présent à la description de ces espaces emboîtés\footnote{Toutes les constructions, tous les énoncés et les preuves qui suivent sont de l'invention personnelle de l'auteur de ces lignes. Je n'ai trouvé nulle part une source qui s'attaque réellement à le récurrence.}.

Pour la suite, nous considérerons des espaces vectoriels normés \( V\) et \( E\) de dimension finie. Nous notons \( \aL^n(V,E)\) l'espace des applications multilinéaires \( V^n\to E\).

Nous définissons aussi par récurrence
\begin{subequations}
    \begin{numcases}{}
        E_0=E\\
        E_{k+1}=\aL(V,E_k).
    \end{numcases}
\end{subequations}

\begin{lemma}[\cite{MonCerveau,BIBooLWVSooOZSJBR}]      \label{LEMooSMZQooJBVySP}
    À propos de dimensions,
    \begin{enumerate}
        \item       \label{ITEMooUWEBooSzFseN}
         $\dim(E_n)=\dim(V)^n\dim(E)$.
     \item       \label{ITEMooFMKQooFSMpgF}
        \( \dim\big( \aL^n(V,E) \big)=\dim(E)\dim(V)^n\).
    \end{enumerate}
\end{lemma}

\begin{proof}
    Nous faisons \ref{ITEMooUWEBooSzFseN} par récurrence. D'abord \( \dim(E_0)=\dim(E)\) et ensuite
    \begin{equation}
        \dim(E_{k+1})=\dim\aL(V,E_k)=\dim(V)\dim(E_k)=\dim(V)^{n+1}\dim(E).
    \end{equation}
    
    Pour \ref{ITEMooFMKQooFSMpgF}, si \( \{ e_i \}\) est une base de \( V\), un élément \( \omega\in \aL^n(V,E)\) est déterminé par les valeurs de \( \omega(e_{i_1},\ldots, e_{i_n})\) qui peuvent être n'importe quel vecteur de \( E\). Donc la dimension est \( \dim(V)^n\dim(E)\).
\end{proof}

\begin{lemma}[\cite{MonCerveau}]
    Soit \( n\in \eN\) nous définissons par récurrence
    \begin{equation}
        \begin{aligned}
            \psi_{n,0}\colon E_n&\to E_n \\
            \alpha&\mapsto \alpha. 
        \end{aligned}
    \end{equation}
    et
    \begin{equation}
        \begin{aligned}
            \psi_{n,k}\colon E_n&\to \aL^k(V,E_{n-k}) \\
            \psi_{n,k}(\alpha)(v_1,\ldots, v_k)&=\big( \psi_{n,k-1}(\alpha)(v_1,\ldots, v_{k-1}) \big)v_k. 
        \end{aligned}
    \end{equation}
    L'application
    \begin{equation}
        \psi_{n,n}\colon E_n\to \aL^n(V,E) 
    \end{equation}
    est un isomorphisme isométrique.
\end{lemma}

\begin{proof}
    Nous allons démontrer par récurrence sur \( k\) que tous les \( \psi_{n,k}\) sont des isomorphismes isométriques. Pour \( k=1\) c'est évident parce que \( \psi_{n,1}\) est l'identité.

    \begin{subproof}
        \item[Injective]
            Soient \( \alpha\in E_n\) tels que \( \psi_{n,k+1}(\alpha)=O\). Cela signifie que pour tout \( v_1,\ldots, v_{k+1}\) nous avons
            \begin{equation}
                \psi_{n,k}(\alpha)(v_1,\ldots, v_k)v_{k+1}=0.
            \end{equation}
            c'est à dire \( \psi_{n,k}(\alpha)(v_1,\ldots, v_{k}=0)\). Vu que \( \psi_{n,k}\) est injective (hypothèse de récurrence), nous avons \( \alpha=0\).

        \item[Surjective]
            Soit \( \omega\in \aL^{k+1}(V,E_{n-k})\); nous cherchons \( \alpha\in E_n\) tel que \( \psi_{n,k+1}(\alpha)=\omega\). Cette condition s'exprime
            \begin{equation}        \label{EQooACUOooKCKame}
                \psi_{n,k+1}(\alpha)(v_1,\ldots, v_{k+1})=\psi_{n,k}(\alpha)(v_1,\ldots, v_k)v_{k+1}=\stackrel{!}{=}\omega(v_1,\ldots, v_{k+1}).
            \end{equation}
            Notez que
            \begin{equation}
                \psi_{n,k}(\alpha)\in \aL^k(V,E_{n-k})=\aL^k\big( V,\aL(V,E_{n-k-1}) \big).
            \end{equation}
            En considérant \( \sigma\in \aL^k\big( V,\aL(V,E_{n-k-1}) \big)\) donné par
            \begin{equation}
                \sigma(v_1,\ldots, v_k)v_{k+1}=\omega(v_1,\ldots, v_{k+1}),
            \end{equation}
            il existe (hypothèse de récurrence sur \( k\)) un \( \alpha\in E_n\) tel que \( \psi_{n,k}(\alpha)=\sigma\).

            Pour ce \( \alpha\), la condition \eqref{EQooACUOooKCKame} est satisfaite.

        \item[Isométrique]
            Encore une fois par récurrence. Soit \( \alpha\in E_n\). Nous avons
            \begin{subequations}
                \begin{align}
                    \| \psi_{n,k}(\alpha) \|_{\aL^k(V,E_{n-k})}&=\sup_{\| v_i \|=1}\| \psi_{n,k}(\alpha)(v_1,\ldots, v_k) \|_{E_{n-k}}\\
                    &=\sup_{\| v_i \|=1}\| \psi_{n,k-1}(\alpha)(v_1,\ldots, v_{k-1})v_k \|_{E_{n-k}}\\
                    &=\sup_{\substack{\| v_i \|=1\\i=1,\ldots, k-1}}\| \psi_{n,k-1}(v_1,\ldots, v_{k-1}) \|_{\aL(V,E_{n-k-1})}\\
                    &=\| \psi_{n,k-1}(\alpha) \|\\
                    &=\| \alpha \|.
                \end{align}
            \end{subequations}
            La dernière égalité est l'hypothèse de récurrence. Notez la subtile utilisation du lemme \ref{LEMooQLVAooICaPvR} qui permet de donner un sens aux supremums, grace au fait que \( \{ v\in V\tq \| v \|=1 \}\) est compact.
    \end{subproof}
\end{proof}

Nous allons maintenant construire l'application inverse de \( \psi_{n,n}\). Pour cela nous commençons par introduire une espèce de projection; pour \( u\in V\) nous définissons
\begin{equation}
    \begin{aligned}
        \pr_u\colon \aL^n(V,E)&\to \aL^{n-1}(V,E) \\
        \pr_u(\omega)(v_1,\ldots, v_{n-1})&=\omega(u,v_1,\ldots, v_{n-1}). 
    \end{aligned}
\end{equation}

\begin{lemma}[\cite{MonCerveau}]
    Nous définissons les \( \phi_k\) par récurrence. D'abord
    \begin{equation}
        \begin{aligned}
            \phi_1\colon \aL^n(V,E)&\to \aL\big( V,\aL^{n-1}(V,E) \big) \\
            \phi_1(\omega)u&=\pr_u(\omega),
        \end{aligned}
    \end{equation}
    et ensuite
    \begin{equation}
        \begin{aligned}
            \phi_k\colon \aL^n(V,E)&\to \aL^{n-k}(V,E)_{k} \\
            \phi_k(\omega)u&=\phi_{k-1}\big( \pr_u(\omega) \big).
        \end{aligned}
    \end{equation}
    Les applications \( \phi_k\) sont bijectives.
\end{lemma}

\begin{proof}
    Nous prouvons par récurrence. 
    \begin{subproof}
        \item[Injective, \( k=1\)]
            Soit \( \omega\in \aL^n(V,E)\) tel que \( \phi_1(\omega)=0\). Pour tout \( u\in V\) nous avons \( \phi_1(\omega)u=0\), ce qui signifie que \( \pr_u(\omega)=0\) ou encore que pour tout \( v_1,\ldots, v_{n-1}\in V\) nous avons \( \omega(u,v_1,\ldots, v_{n-1})=0\). Nous avons donc bien \( \omega=0\).
        \item[Surjective, \( k=1\)]
            Soit \( \alpha\in\aL\big( V,\aL^{n-1}(V,E) \big)\). Nous allons construire \( \omega\in \aL^n(V,E)\) tel que \( \phi_1(\omega)=\alpha\). Nous posons
            \begin{equation}
                \omega(v_0,\ldots, v_{n-1})=\alpha(v_0)(v_1,\ldots, v_{n-1}).
            \end{equation}
            Avec lui nous avons bien \( \pr_{v_0}(\omega)=\alpha(v_0)\).
        \item[Injective, \( k=k\)]
            Nous supposons que \( \phi_k(\omega)=0\), c'est à dire que pour tout \( u\in V\) nous avons 
            \begin{equation}
                \phi_{k-1}\big( \pr_u(\omega) \big)=0.
            \end{equation}
            Cela implique \( \pr_u(\omega)=0\) parce que \( \phi_{k-1}\) est injective par hypothèse de récurrence. Nous déduisons que \( \omega=0\), et que \( \phi_k\) est injective.

        \item[Surjective, \( k=k\)]
            Nous allons montrer que \(   \phi_k\colon \aL^n(V,E)\to \aL^{n-k}(V,E)_k  \) est une application linéaire injective entre deux espaces de même dimension.

            Il s'agit d'utiliser le lemme \ref{LEMooSMZQooJBVySP}. D'abord
            \begin{equation}
                \dim\big( \aL^{n-k}(V,E)_k \big)=\dim(V)^k\dim\big( \aL^{n-k}(V,E) \big)=\dim(V)^n\dim(E).
            \end{equation}
            Ensuite \( \dim\big( \aL^n(V,E) \big)=\dim(V)^n\dim(E)\). Le compte est bon.

            Le théorème du rang (formule \eqref{EQooUEOQooLySRiE}) avec \( \dim\big( \ker(\phi_k) \big)=0\) nous dit que le rang de \( \phi_k\) est maximal et donc que \( \phi_k\) est surjective.
    \end{subproof}
\end{proof}

\begin{lemma}
    Nous avons \( \psi_{n,n}=\phi_n^{-1}\).
\end{lemma}

\begin{proof}
    Pour nous échauffer nous posons \( \omega\in\aL^n(V,\eR)\), et nous calculons
    \begin{subequations}        \label{EQooPBQIooUValDA}
        \begin{align}
            \psi_{n,2}\big( \phi_n(\omega) \big)(v_1,v_2)&=\Big( \psi_{n,1}\big( \phi_n(\omega) \big)v_1 \Big)v_2\\
            &=\big( \phi_n(\omega)v_1 \big)v_2\\
            &=\phi_{n-1}\big( \pr_{v_1}(\omega) \big)\\
            &=\phi_{n-2}\big( \pr_{v_2}\pr_{v_1}(\omega) \big).
        \end{align}
    \end{subequations}
    Cela étant dit, nous allons prouver ceci par récurrence :
    \begin{equation}
        \psi_{n,k}\big( \phi_n(\omega) \big)(v_1,\ldots, v_k)=\phi_{n-k}\Big( \prod_{i=1}^k\pr_{v_i}(\omega) \Big).
    \end{equation}
    Notez l'ordre du produit des projections. En ce qui concerne cet ordre, pour fixer les idées voici un exemple :
    \begin{equation}
        \pr_{v_2}\pr_{v_1}(\omega)=\big( \pr_{v_1}(\omega) \big)(v_2)=\omega(v_1,v_2).
    \end{equation}
    
    Faisons maintenant la récurrence.
    \begin{subproof}
        \item[Pour \( k=2\)]
            C'est le calcul \eqref{EQooPBQIooUValDA}.
        \item[Pour \( k+1\)]
            C'est encore un calcul, en faisant attention à l'ordre dans lequel viennent les projections :
            \begin{subequations}
                \begin{align}
                    \psi_{n,k+1}\big( \phi_n(\omega) \big)(v_1,\ldots, v_{k+1})&=\Big( \psi_{n,k}\big( \phi_n(\omega) \big)(v_1,\ldots, v_k) \Big)v_{k+1}\\
                    &=\Big( \phi_{n-k}\big( \prod_{i=1}^k\pr_{v_i}(\omega) \big) \Big)v_{k+1}\\
                    &=\phi_{n-k-1}\big( \pr_{v_{k+1}}\prod_{i=1}^k\pr_{v_i}(\omega) \big)\\
                    &=\phi_{n-(k+1)}\big( \prod_{i=1}^{k+1}\pr_{v_i}(\omega) \big).
                \end{align}
            \end{subequations}
    \end{subproof}
    Il ne reste qu'à écrire la formule démontrée avec \( k=n\) :
    \begin{equation}
        \psi_{n,n}\big( \phi_n(\omega) \big)(v_1,\ldots, v_n)=\phi_{n-n}\big( \prod_{i=1}^n\pr_{v_i(\omega)} \big)=\omega(v_1,\ldots, v_n).
    \end{equation}
    Donc nous avons bien que \( \psi_{n,n}\big( \phi_n(\omega) \big)=\omega\).
\end{proof}

%---------------------------------------------------------------------------------------------------------------------------
\subsection{Identification des espaces d'applications multilinéaires (vieux)}
%---------------------------------------------------------------------------------------------------------------------------
\label{SUBSECooXBGUooXYFZjy}

\begin{probleme}        \label{PROBooDLAWooLqHDPs}
    Les choses dites en \ref{SUBSECooXBGUooXYFZjy} sont obsolètes. Il faut soit les retirer soit les déplacer.
\end{probleme}

La différentielle de la différentielle de $f$ est notée
\[
d(df)(a)=d^2f(a),
\]
et est une application de $U$ dans $\aL(\eR^m,\aL(\eR^m, \eR^n) )$. Comme on a vu dans la proposition~\ref{isom_isom}, l'espace $\aL(\eR^m,\aL(\eR^m, \eR^n) )$ est isométriquement isomorphe à l'espace $\aL(\eR^m\times\eR^m, \eR^n )$. On verra comment cette propriété  est utilisé dans l'exemple~\ref{bilin_2diff}.


Soient \( V\) et \( W\) deux espaces vectoriel normés de dimension finie et \( \mO\) un ouvert autour de \( x\in V\). D'une part l'espace des applications linéaires \( \aL(V,W)\) est lui-même un espace vectoriel normé de dimension finie, et on peut identifier \(  \aL\big( V,\aL^{(k)}(V,W) \big)\)\nomenclature[Y]{\( \aL^{(n)}(V,W)\)}{L'espace des applications \( n\)-linéaires \( V^n\to W\)} avec \( \aL^{(k+1)}(V,W)\), ce qui nous permet de dire que la \( k\)\ieme\ différentielle est une application
\begin{equation}
    d^kf\colon \mO\to \aL^{(k)}(V,W).
\end{equation}
Plus précisément, l'identification se fait de la façon suivante : si \( \omega\in \aL\big( V,\aL^{(k)}(V,W) \big)\), alors \( \omega\) vu dans \( \aL^{(k+1)}(V,W)\) est définie par
\begin{equation}
    \omega(u_1,\ldots, u_{k+1})=\omega(u_1)(u_2,\ldots, u_{k+1}).
\end{equation}
Cela étant posé nous pouvons donner des définitions.

%---------------------------------------------------------------------------------------------------------------------------
\subsection{Fonctions différentiables plusieurs fois}
%---------------------------------------------------------------------------------------------------------------------------

\begin{definition}[\cite{ZCKMFRg}]  \label{DefPNjMGqy}
    La fonction \( f\colon \mO\subset V\to W\) est
    \begin{enumerate}
        \item
            de classe \( C^0\) si elle est continue,
        \item
            de classe \( C^1\) si \( df\colon \mO\to \aL(V,W)\) est continue,
        \item
            de classe \( C^k\) si \( d^kf\colon \mO\to \aL^{(k)}(V,W)\) est continue,
        \item
            de classe \(  C^{\infty}\) si \( f\) est dans \( \bigcap_{k=0}^{\infty}C^k(V,W)\).
    \end{enumerate}
\end{definition}
\index{application!différentiable}
\index{application!de classe \( C^k\)}

\begin{definition}
    Un \defe{\( C^k\)-difféomorphisme}{difféomorphisme!de classe $C^k$} est une application inversible de classe \( C^k\) dont l'inverse est également de classe \( C^k\).
\end{definition}

\begin{example}\label{bilin_2diff}
	Soit $B:\eR^m\times \eR^m\to\eR^n$ une application bilinéaire. On définit $f:\eR^m\to\eR^n$ par $f(x)=B(x,x)$. Le lemme~\ref{bilin_diff} nous dit que $B$ est différentiable. Cela implique la différentiabilité de $f$. Pour trouver la différentielle de la fonction $f$, nous écrivons $f=B\circ s$ où $s\colon \eR^m\to \eR^m\times\eR^m$ est l'application $s(x)=(x,x)$. En utilisant la règle de différentiation de fonctions composées,
	\begin{equation}
		df(a)=dB\big( s(a) \big)\circ ds(a).
	\end{equation}
	Mais $ds(a).u=(u,u)$ parce que $s(a+h)-s(a)-(h,h)=0$. Par conséquent,
	\begin{equation}		\label{EqdBsaExp}
		df(a).u=dB\big( s(a) \big)(u,u)=B(u,a)+B(a,u)
	\end{equation}
	où nous avons utilisé la formule du lemme~\ref{bilin_diff}. La formule \eqref{EqdBsaExp} peut être écrite sous la forme compacte
	\begin{equation}
		df(a)=B(\cdot,a)+B(a,\cdot)
	\end{equation}
    La fonction $df(a)$ ainsi écrite est linéaire par rapport à $a$, donc différentiable. En outre elle coïncide avec sa différentielle, comme on a vu dans le lemme \ref{LEMooZSNMooCfjzOB}, au sens que la différentielle de $df$ au point $a$ sera l'application que à chaque $x$ dans $\eR^m$ associe l'application linéaire $B(x,\cdot)+B(\cdot, x)$. On voit bien que $d^2f$ au point $a$ est une application de $\eR^m$ vers l'espace des applications linéaires $\aL(\eR^m, \eR^n)$. On peut utiliser d'autre part l'isomorphisme des espaces $\aL(\eR^m,\aL(\eR^m, \eR^n) )$ et $\aL(\eR^m\times\eR^m, \eR^n )$ et dire que, une fois que $a$ est fixé, l'application $d^2f(a)$ est une application bilinéaire sur $\eR^m\times\eR^m$. On écrit alors $d^2f(a)(x,y)=B(x,y)+B(y,x)$.
\end{example}

%---------------------------------------------------------------------------------------------------------------------------
\subsection{Différentielle seconde, fonction de classe \texorpdfstring{$ C^2$}{C2}}
%---------------------------------------------------------------------------------------------------------------------------

Une condition nécessaire et suffisante pour l'existence de la différentielle seconde est la suivante
\begin{proposition}
   Soit $U$ un ouvert de $\eR^m$ et  $f:U\subset\eR^m\to \eR^n$ une fonction. La fonction $f$ est deux fois différentiable au point $a$ si et seulement si les dérivées partielles $\partial_1 f, \ldots, \partial_m f $ sont différentiables en $a$.
\end{proposition}
Cela veut dire, en particulier, que $f$ est deux fois différentiable si et seulement si ses dérivées partielles secondes, $\partial_i\partial_j f$, pour tout couple d'indices $i,j$  dans $\{1,\ldots, m\}$, existent et sont continues. Pour les différentielles d'ordre supérieur on a la proposition suivante.

La différentielle seconde dans l'exemple ~\ref{bilin_2diff} est symétrique, c'est-à-dire que $d^2f(a)(x_1,x_2)=d^2f(a)(x_2,x_1)$. En fait toute différentielle seconde est symétrique.


\begin{theorem}[Schwarz]\label{Schwarz}
 Soit $U$ un ouvert de $\eR^m$ et  $f:U\subset\eR^m\to \eR^n$ une fonction de classe $\mathcal{C}^2$. Alors, pour tout couple $i,j$ d'indices dans $\{1,\ldots, m\}$ et pour tout point $a$ dans $U$, on a
\[
\frac{\partial^2 f}{\partial  x_i\partial x_j}(a)=\frac{\partial^2 f}{\partial  x_j\partial x_i}(a).
\]
\end{theorem}
\begin{proof}
  Pour simplifier nous nous limitons ici au cas $m=2$. Soit $(h,g)$ un vecteur fixé dans $\eR^2$. Pour tout  $v=(x,y)$ dans $\eR^2$ on note
  \begin{equation}
    \begin{array}{c}
      \Delta_h f(v)=f(v+he_1) -f(v) = f(x+h,y)-f(x,y),\\
      \Delta_g f(v)=f(v+ge_2) -f(v) = f(x,y+g)-f(x,y),\\
    \end{array}
  \end{equation}
Nous avons
\begin{equation}
  \begin{array}{c}
   \Delta_g   \Delta_h f(v)=\left(f(x+h,y+g)-f(x,y+g)\right)-\left(f(x+h,y)-f(x,y)\right),\\
   \Delta_h   \Delta_g f(v)=\left(f(x+h,y+g)-f(x+h,y)\right)-\left(f(x,y+g)-f(x,y)\right),
  \end{array}
\end{equation}
donc,
\begin{equation}
  \frac{1}{g} \Delta_g  \left(\frac{1}{h} \Delta_h f(v)\right) = \frac{1}{h} \Delta_h \left(\frac{1}{g} \Delta_g f(v)\right).
\end{equation}
On utilise alors le théorème des accroissements finis~\ref{ThoAccFinis}
\begin{equation}
\frac{1}{h} \Delta_h f(v)=\frac{1}{h}\big(f(x+h,y)-f(x,y)\big)=\frac{1}{h}\partial_1f(x+t_1h,y )h=\partial_1f(x+t_1h, y),
\end{equation}
pour un certain $t_1$ dans $]0,1[$. De même on obtient
\[
\frac{1}{g} \Delta_g f(v)= \partial_2 f(x, y+t_2g),
\]
pour un certain $t_2$ dans $]0,1[$. Alors
 \begin{equation}
  \frac{1}{g} \Delta_g  \big(\partial_1f(x+t_1h, y)\big) = \frac{1}{h} \Delta_h \big(\partial_2 f(x, y+t_2g)\big).
\end{equation}
En appliquant encore une fois le théorème des accroissements finis on a
 \begin{equation}
  \partial_2\partial_1f(x+t_1h, y+s_1g) = \partial_1\partial_2 f(x+s_2h, y+t_2g).
\end{equation}
Il suffit maintenant de passer à la limite pour $(h,g) \to (0,0)$ et de se souvenir du fait que $f$ est $\mathcal{C}^2$ seulement si ses dérivées partielles secondes sont continues pour avoir $\partial_2\partial_1f(v)=\partial_1\partial_2 f(v)$.
\end{proof}
Si $f$ est deux fois différentiable $d^2f(a)$ est l'application bilinéaire associée avec la matrice symétrique
\begin{equation}
 H_f(a)= \begin{pmatrix}
    \partial^2_1f(a)& \ldots& \partial_1\partial_m f(a)\\
    \vdots& \ddots& \vdots\\
    \partial_1\partial_m f(a)&\ldots&\partial^2_1f(a),
  \end{pmatrix}
\end{equation}
Cette matrice est dite la matrice \defe{hessienne}{hessienne} de $f$.

\begin{example}
  Montrons qu'il n'existe pas de fonctions $f$ de classe $\mathcal{C}^2$ telles que
  \begin{subequations}
      \begin{numcases}{}
  \partial_xf(x,y)= 5\sin x\\
  \partial_y(x,y)=6x+y.
      \end{numcases}
  \end{subequations}
  Ceci est vite fait en appliquant le théorème de Schwarz,~\ref{Schwarz}; ce que nous trouvons est
\[
\partial_y (\partial_xf)= 0\neq \partial_x(\partial_yf)= 6.
\]
Donc, l'existence d'une fonction $f$ de classe $\mathcal{C}^2$ telle que $\partial_x(x,y)= 5\sin x$ et $\partial_yf(x,y)=6x+y$ serait en contradiction avec le théorème.
\end{example}

Soit une fonction de classe \( C^2\) \( f\colon V\to \eR\) où \( V\) est un espace vectoriel de dimension \( n<\infty\). Nous avons
\begin{subequations}
    \begin{align}
        f&\colon V\to \eR\\
        df&\colon V\to \aL(V,\eR)\\
        d^2f&\colon V\to \aL\Big( V,\aL(V,\eR) \Big),
    \end{align}
\end{subequations}
avec, en suivant les différentes formules du lemme~\ref{LemdfaSurLesPartielles},
\begin{equation}
        df_a(u)=\Dsdd{ f(v+tu) }{t}{0}
\end{equation}
et
\begin{equation}
    (d^2f)_a(u)=\Dsdd{ df_{v+tu} }{t}{0}
\end{equation}
pour tout \( a,u\in V\). Notons que dans le deuxième cas, il s'agit d'une limite dans \( \aL(V,\eR)\). Si \( \dim(V)=n\), alors \( \dim\aL(V,\eR)=n\) et avec un choix de base, nous pouvons trouver une matrice \( n\times n\) pour \( (d^2f)_a\).

Soit une base \( \{ e_i \}\) de \( V\) et la base duale \( \{ e_i^* \}\) de \( \aL(V,\eR)\). Nous allons chercher la matrice de \( (d^2f)_a\) pour ces bases. L'élément de matrice
\begin{equation}
    \big[ (d^2f)_a \big]_{ij}
\end{equation}
est la composante \( e_j^*\) de \( (d^2f)_a\) appliqué à \( e_i\). Trouver cette composante \( e_j^*\) revient à appliquer l'élément \( (d^2f)_ae_i\) de \( \aL(V,\eR)\) à \( e_j\). Le calcul est donc :
\begin{subequations}
    \begin{align}
        \big[ (d^2f)_{a} \big]_{ij}&=\big( (d^2f)_ae_i \big)(e_j)\\
        &=\Dsdd{ df_{a+te_i}(e_j) }{t}{0}       \label{SUBEQooDRZFooAuuaad}\\
        &=\Dsdd{    \Dsdd{ f(a+te_i+se_j) }{s}{0}    }{t}{0}\\
        &=\frac{ \partial^2f }{ \partial x_i\partial x_j }(a).
    \end{align}
\end{subequations}
Attention : le passage à \eqref{SUBEQooDRZFooAuuaad} n'est pas une trivialité. Le fait est que si \( t\mapsto A(t)\) est une application continue \( \eR\to \aL(V,\eR)\) alors
\begin{equation}
    \lim_{t\to 0} \big( A(t)v \big)=\big( \lim_{t\to 0} A(t) \big)v.
\end{equation}

Donc la matrice de \( d^2f  \) est la matrice des dérivées secondes. Il s'agit d'une matrice symétrique par le théorème de Schwarz~\ref{Schwarz}.

\begin{normaltext}      \label{NORMooZAOEooGqjpLH}
    Si \( a\in v\), nous pouvons aussi voir \( (d^2f)_a\) comme une forme bilinéaire sur \( V\) grâce à la proposition~\ref{isom_isom}. Si \( u,v\in V\) nous notons
    \begin{equation}
        (d^2f)_a(u,v)=(d^2f)_a(u)v.
    \end{equation}
    À droite, il s'agit de la définition réelle de \( d^2f\) sans abus de notations, et à gauche, il s'agit d'une notation. Cette application bilinéaire \( (d^2f)_a\in \aL^{(2)}(V,\eR)\) a pour matrice symétrique la matrice des dérivées secondes calculées en \( a\).
\end{normaltext}

\begin{example} \label{ExZHZYcNH}
    Voyons comment la différentielle seconde fonctionne entre deux espaces vectoriels. Soient deux espaces vectoriels de dimension finie \( V\) et \( W\). Pour que les choses soient claires, nous avons :
    \begin{subequations}
        \begin{align}
            f&\colon V\to W\\
            df&\colon V\to \aL(V,W)\\
            d^2f&\colon V\to \aL\Big( V,\aL(V,W) \Big).
        \end{align}
    \end{subequations}
    Si \( a\in V\), alors \( (d^2f)_a\) est une application \( V\to \aL(V,W)\). Il faut donc l'appliquer à \( u\in V\) et ensuite à \( v\in V\) pour obtenir un élément de \( W\) :
    \begin{subequations}
        \begin{align}
            (d^2f)_a(u)v&=\Dsdd{ df_{a+tu} }{t}{0}v\\
            &=\Dsdd{ df_{a+tu}(v) }{t}{0}\\
            &=\Dsdd{ \Dsdd{ f(a+tu+sv) }{s}{0} }{t}{0}\\
            &=\frac{ \partial^2f }{ \partial u\partial v }(a).
        \end{align}
    \end{subequations}


    Par conséquent nous voyons
    \begin{equation}\label{EqQHINNtD}
        \begin{aligned}
            d^2f\colon V&\to \aL^{(2)}(V,W) \\
            d^2f_a(u,v)&=\frac{ \partial^2f  }{ \partial u\partial v }(a).
        \end{aligned}
    \end{equation}

    Dans le cas d'une fonction \( f\colon \eR\to \eR\), nous avons une seule direction et par linéarité de \eqref{EqQHINNtD} par rapport à \( u\) et \( v\), nous avons
    \begin{equation}        \label{EQooSOCGooIiNGmG}
        d^2f_a(u,v)=f''(a)uv
    \end{equation}
    où les produits sont des produits usuels dans \( \eR\) et \( f''\) est la dérivée seconde usuelle.
\end{example}

Tout ceci est un peu résumé dans la proposition suivante.
\begin{proposition}     \label{PROPooFWZYooUQwzjW}
    Soit une fonction \( f\colon \eR^n\to \eR\) de classe \( C^2\). Alors en désignant par \( H_af\) sa matrice hessienne au point \( a\) nous avons
    \begin{equation}
        (d^2f)_a(u,v)=\frac{ \partial^2f }{ \partial u\partial v }(a)=\langle (H_af)u, v\rangle
    \end{equation}
    pour tout \( u,v\in \eR^n\).
\end{proposition}

\begin{proof}
    La première égalité est l'équation \eqref{EQooSOCGooIiNGmG} déjà faite. Pour la seconde, il faut se rappeler du lien entre dérivée partielle et dérivée directionnelle, donné en le lemme~\ref{LemdfaSurLesPartielles}. En particulier ici nous avons
    \begin{equation}
        \frac{ \partial^2f }{ \partial u\partial v }=\sum_{kl}\frac{ \partial^2f }{ \partial x_k\partial x_l  }(a)u_kv_l=\langle (H_af)u, v\rangle .
    \end{equation}
\end{proof}

En particulier, la matrice hessienne \( H_af\) est symétrique et donc diagonalisable (théorème spectral~\ref{ThoeTMXla}). Si \( e_i\) est un vecteur propre unitaire pour la valeur propre \( \lambda_i\) nous avons
\begin{equation}
    (d^2f)_a(e_i,e_i)=\langle (H_af)e_i, e_i\rangle =\lambda_i\langle e_i, e_i\rangle =\lambda.
\end{equation}

Enfin pour celles qui aiment les notations matricielles de tout poil, il y a cette façon-ci d'écrire :
\begin{equation}
    (d^2f)_a(\alpha,\beta)=\begin{pmatrix}
        \alpha    &   \beta
    \end{pmatrix}\begin{pmatrix}
        \partial^2_xf(a)    &   \partial^2_{xy}f(a)    \\
        \partial^2_{xy}f(a)    &   \partial^2_yf(a)
    \end{pmatrix}\begin{pmatrix}
        \alpha    \\
        \beta
    \end{pmatrix}.
\end{equation}

%---------------------------------------------------------------------------------------------------------------------------
\subsection{Ordre supérieur}
%---------------------------------------------------------------------------------------------------------------------------

Intuitivement, une fonction est de classe \( C^p\) si elle est \( p\) fois continûment différentiable. Nous posons la définition suivante.

\begin{definition}
    Une fonction \( f\colon E\to F\) est \defe{de classe \( C^0\)}{classe \( C^0\)} si elle est continue.

    Nous disons que la fonction \( f\colon E\to F\) est \defe{de classe \( C^p\)}{classe \( C^p\)} si elle est différentiable et si sa différentielle \( df\colon E\to \aL(E,F)\) est de classe \( C^{p-1}\).
\end{definition}

Ce qui est terrible avec les différentielles d'ordre supérieurs, c'est l'emboîtement des espaces. Pour une fonction \( f\colon E\to F\), nous allons souvent poser
\begin{subequations}
    \begin{align}
        V_0&=F\\
        V_{k+1}&=\aL(E,V_k),
    \end{align}
\end{subequations}
de telle sorte à avoir
\begin{equation}
    df\colon E\to \aL(E,F)=V_1
\end{equation}
et 
\begin{equation}
    d^2f\colon E\to \aL(E,V_1)=V_2,
\end{equation}
ce qui donne en général
\begin{equation}
    d^kf\colon E\to \aL(E,v_{k-1})=V_k.
\end{equation}

La proposition suivante lie une bonne fois pour toute dérivée et différentielle dans le cadre de fonctions \( \eR\to \eR\).
\begin{proposition}[\cite{MonCerveau}]      \label{PROPooCNDHooKRwils}
    Une fonction \( f\colon \eR\to \eR\) est de classe \( C^p\) si et seulement si elle est \( p\) fois continûment dérivable.
\end{proposition}

\begin{proof}
    Nous commençons par poser un certain nombre de notations. Comme souvent nous posons \( V_0=\eR\) et
    \begin{equation}
        V_{k+1}=\aL(\eR,V_k).
    \end{equation}
    De plus nous considérons \( M_1\in \aL(\eR,\eR)\) donnée par \( M_1(t)=t\). Et par récurrence
    \begin{equation}
        M_{k+1}(t)=tM_{k}.
    \end{equation}
    Nous avons \( M_1\in V_1\) et \( M_k\colon \eR\to V_{k-1}\), c'est-à-dire \( M_k\in V_k\).

    Les formules que nous allons prouver sont que d'une part,
    \begin{equation}
        df_a=f'(a)M_1.
    \end{equation}
    et que d'autre part, plus généralement,
    \begin{equation}
        (d^kf)_a=f^{(k)}(a)M_k.
    \end{equation}

    En plusieurs parties et par récurrence.
    \begin{subproof}
    \item[Si \( f\) est continûment dérivable, alors \( f\) est \( C^1\) ]
        Le candidat différentielle serait \( df_a(h)=hf'(a)\). Vérifions cela directement dans la définition :
        \begin{equation}        \label{EQooCPWKooWdgbED}
            \frac{ f(a+h)-f(a)-f'(a)h }{ \| h \| }=\frac{ f(a+h)-f(a) }{ \| h \| }-1_h'f(a).
        \end{equation}
        où nous avons noté \( 1_h\) le vecteur unité dans la direction de \( h\), c'est-à-dire \( 1_h=h/\| h \|\). Vu que \( h\in \eR\), c'est simplement
        \begin{equation}
            1_h=\begin{cases}
                1    &   \text{si } h>0\\
                -1    &    \text{si }h<0
            \end{cases}
        \end{equation}
        et nous ne définissons pas \( 1_h\) si \( h=0\).
        
        C'est le moment de prendre la limite de \eqref{EQooCPWKooWdgbED} pour \( h\to 0^+\) et \( h\to 0^-\) séparément. Lorsque \( h\to 0^+\), nous avons \( \| h \|=h\) et \( 1_h=h\). Vu que \( f\) est supposée dérivable, la limite du quotient existe et vaut \( f'(a)\). Donc le tout a une limite nulle :
        \begin{equation}       
            \lim_{h\to 0^+} \frac{ f(a+h)-f(a)-f'(a)h }{ \| h \| }=\lim_{h\to 0^+}\frac{ f(a+h)-f(a) }{  h  }-'f(a)=0.
        \end{equation}
        En ce qui concerne la limite \( h\to 0^-\), nous avons \( \| h \|=-h\) et \( 1_h=-1\), et à nouveau une limite nulle. La proposition \ref{PROPooGDDJooDCmydE} nous permet alors de conclure que la limite existe et est nulle. Les limites à gauche et à droite étant nulles, la limite existe et est nulle par la proposition \ref{PROPooGDDJooDCmydE}.

    \item[Si \( f^{(p)}\) est continue alors \( d^pf\) aussi]
        Nous passons à la récurrence de notre preuve. Sous l'hypothèse que \( f^{(p)}\) existe et est continue, nous allons montrer que \( d^pf\) existe, est continue et vaut
        \begin{equation}
            (d^pf)_a=f^{(p)}(a)M_p.
        \end{equation}
        Soit \( k<p\) tel que ce soit bon (pour \( k=1\) c'est déjà fait). Nous avons \( (d^kf)_a=f^{(k)}(a)M_k\), et pour prouver que \( (d^{k+1}f)_a=f^{(k+1)}(a)M_{k+1}\) nous l'y mettons dans la définition de la différentielle. Nous avons :
        \begin{equation}
            \frac{ (d^kf)_{a+h}-(d^kf)_a-f^{(k+1)}(a)M_{k+1}(h) }{ \| h \| }=\frac{ f^{(k)}(a+h)M_k-f^{(k)}(a)M_k-hf^{(k+1)}(a)M_k }{ \| h \| }.
        \end{equation}
        La limite \( h\to 0\) est une limite dans \( V_k\), et elle se traite comme précédemment. Elle vaut zéro parce que \( f^{(k+1)}\) est la dérivée de \( f^{(k)}\). Cela justifie les faits que \( d^kf\) est différentiable en \( a\) et que la différentielle est donné par la formule voulue.

        Par hypothèse, \( k+1\leq p\), donc \( f^{(k+1)}\) est continue. Par conséquent l'application \( a\mapsto f^{(k+1)}(a)M_{k+1}\) est continue.

    \item[Si \( f\) est de classe \( C^1\) alors \( f'\) existe et est continue]
        Dire que \( f\) est de classe \( C^1\) revient à dire que la différentielle \( df\colon \eR\to \aL(\eR,\eR)\) existe et est continue. Soyons conscient que \( df_a(\epsilon)=\epsilon df_a(1)\) et calculons
        \begin{equation}
            \frac{ f(a+\epsilon)-f(a)-df_a(\epsilon) }{ \epsilon }=\frac{ f(a+\epsilon)-f(a) }{ \epsilon }-df_a(1).
        \end{equation}
        La définition de la différentielle est que la limite de cela pour \( \epsilon\to 0\) soit nulle. Cela implique que la limite suivante existe et vaut
        \begin{equation}
            \lim_{\epsilon\to 0}\frac{ f(a+\epsilon)-f(a) }{ \epsilon }=df_a(1).
        \end{equation}
        Nous avons prouvé que \( f'(a)=df_a(1)\).

        La fonction \( a\mapsto df_a\) est continue. Pouvons-nous en déduire que \( f'\) est continue ? Nous avons
        \begin{equation}
            f'=ev_1\circ df
        \end{equation}
        où \( ev_1\) est l'application d'évaluation dont le lemme \ref{LEMooWFNXooLyTyyX} a déjà donné la continuité. Donc \( f'\) est continue comme composée d'applications continues.
    \item[\( f\) est \( C^p\). Récurrence]
        Nous supposons que \( f\) est de classe \( C^p\), et nous allons montrer par récurrence que \( f^{(k)}\) existe et est continue pour tout \( k\leq p\). Posons exactement l'énoncé de notre récurrence.

        Pour \( k=1\) c'est fait. Nous supposons que la formule soit correcte pour un certain \( k\leq p\) et nous y allons pour \( k+1\). Nous avons
        \begin{subequations}        \label{SUBEQSooUPLAooQhueCl}
            \begin{align}
            \frac{ (d^kf)(a+h)-(d^kf)(a)-f^{(k+1)}(a)M_{k+1}(h) }{ \| h \| }&=\frac{ \big[ f^{(k)}(a+h)-f^k(a)-hf^{(k+1)}(a) \big]M_k  }{ \| h \| }\\
                &=\big[ \frac{ f^{(k)}(a+h)-f^{(k)}(a) }{ \| h \| }-1_hf^{(k+1)}(a) \big]M_k.
            \end{align}
        \end{subequations}
        où nous avons aussi tenu compte que \( M_{k+1}(h)=hM_k\).

        C'est le moment de calculer séparément les limites \( h\to 0^+\) et \( h\to 0^-\). Cela fonctionne comme toutes les autres fois.
    \end{subproof}
\end{proof}

Soit une fonction \( f\colon \eR^n\to \eR\) différentiable \( l\) fois. L'application
\begin{equation}
    d^lf\colon \eR^n\to \aL\Big( \eR^n,\aL\big( \eR^n,\aL(\ldots \big) \Big)
\end{equation}
au point \( x\) appliquée à \( v^{(1)}\) appliquée au point \( v^{(2)}\), \ldots, appliquée à \( v^{(l)}\) est notée
\begin{equation}        \label{EQooITOLooQllUlJ}
    (d^lf)_x(v^{(1)},\ldots ,v^{(l)})\in \eR.
\end{equation}

\begin{proposition}     \label{PROPooQKZIooXTvkIr}
    Soit une fonction \( f\colon \eR^n\to \eR\) différentiable \( l\) fois. Avec la notation \eqref{EQooITOLooQllUlJ} nous avons
    \begin{equation}
        (d^lf)_x(v^{(1)},\ldots v^{(l)})=\sum_{k_1,\ldots, k_l}v^{(1)}_{k_1}\ldots v_{k_l}^{(l)}\frac{ \partial^lf }{ \partial x_{k_1}\ldots \partial x_{k_l} }(x).
    \end{equation}
\end{proposition}

\begin{proof}
    La preuve se fait par récurrence sur \( l\), en sachant que la formule est déjà vraie pour \( l=1\) et \( l=2\). Si la formule est valable pour \( l\), nous avons
    \begin{subequations}
        \begin{align}
            (d^{l+1}f)_x(v^{(1)},\ldots, v^{(l+1)})&=\Dsdd{ (d^l)_{x+tv^{(l+1)}}(v^{(1)},\ldots, v^{(l)}) }{t}{0}\\
            &=\sum_{k_1\ldots k_l}v_{k_1}^{(1)}\ldots v_{k_l}^{(l)}\Dsdd{   \frac{ \partial^lf }{ \partial x_1\ldots \partial x_l }(x+tv^{l+1})   }{t}{0}\\
            &=\sum_{k_1\ldots k_l}v_{k_1}^{(1)}\ldots v_{k_l}^{(l)}\sum_i\frac{ \partial  }{ \partial x_i }\frac{ \partial^lf }{ \partial x_{k_1}\ldots \partial x_k }(x).
        \end{align}
    \end{subequations}
    Cela donne le résultat attendu.
\end{proof}

\begin{normaltext}
    La formule de la proposition~\ref{PROPooQKZIooXTvkIr} nous permet d'écrire de jolies formules comme
    \begin{equation}        \label{EQooXRWWooMoKoOB}
        (d^3f)_x(h,h,h)=\sum_{ijk}h_ih_jh_k(\partial^3_{ijk}f)(x).
    \end{equation}
\end{normaltext}

\begin{proposition}[Dérivées partielles et fonctions \( C^k\)] \label{PropDYKooHvrfGw}
    Soit $U$ un ouvert de $\eR^m$ et  $f:U\subset\eR^m\to \eR^n$. La fonction $f$ est de classe $C^k$ si et seulement si les dérivées partielles $\partial_1 f, \ldots, \partial_m f $ existent et sont de classe $C^{k}$.
\end{proposition}

\begin{proposition}[\cite{MonCerveau}]
    Soient des espaces vectoriels \( E\),  \( V\) et \( W\) de dimension fine, et une fonction \( f\colon E\to V\) de classe \( C^p\). Si \( \varphi\colon V\to W\) est linéaire, alors
    \begin{equation}
        \varphi\circ f\colon E\to W
    \end{equation}
    est de classe \( C^p\).
\end{proposition}

\begin{proof}
    En utilisant le théorème de différentiation de fonctions composées \ref{THOooIHPIooIUyPaf},
    \begin{equation}
        f(\varphi\circ f)_a(u)=d\varphi_{f(a)}df_a(u),
    \end{equation}
    et donc, parce que \( \varphi\) est linéaire,
    \begin{equation}
        d(\varphi\circ f)_a=\varphi\circ df_a.
    \end{equation}
    Nous pouvons exprimer cela de façon un peu différente en posant \( \varphi_1\colon \aL(E,V)\to \aL(E,W)\),
    \begin{equation}
        \varphi_1(\alpha)(a)=(\varphi\circ \alpha)(a).
    \end{equation}
    Cela nous permet d'écrire \( \varphi\circ df_a=(\varphi_1\circ df)(a)\) et donc
    \begin{equation}        \label{EQooUJPWooTzgSJx}
        d(\varphi\circ f)=\varphi_1\circ df
    \end{equation}
    où \( \varphi_1\) est encore une application linéaire. Une récurrence semble possible. Nous posons \( V_0=V\) et \( W_0=W\) puis
    \begin{subequations}
        \begin{align}
            V_{k+1}&=\aL(E,V_k)\\
            W_{k+1}&=\aL(E,W_k)
        \end{align}
    \end{subequations}
    et
    \begin{equation}
        \begin{aligned}
            \varphi_k\colon \aL(E,V_{k-1})&\to \aL(E,W_{k-1}) \\
            g&\mapsto \varphi_{k-1}\circ g.
        \end{aligned}
    \end{equation}
    Avec tout cela, nous prétendons que \( d^k(\varphi\circ f)=\varphi_k\circ d^kf\) avec \( \varphi_k\) linéaire.

    \begin{subproof}
        \item[\( \varphi_k\) est linéaire]
            Soient \( \alpha_1,\alpha_2\in \aL(E,V_{k-1})\), ainsi que \( \lambda,\mu\in \eK\). Nous avons, en utilisant la linéarité de \( \varphi_{k-1}\) :
            \begin{subequations}
                \begin{align}
                    \varphi_k(\lambda\alpha_1+\mu\alpha_2)(a)&=\varphi_{k-1}\big( (\lambda\alpha_1+\mu\alpha_2)(a) \big)\\
                    &=\varphi_{k-1}\big(\lambda \alpha_1(a)\big)+\mu\varphi_{k-1}\big( \alpha_2(a) \big)\\
                    &=\lambda\varphi_k(\alpha_1)a+\mu\varphi_k(\alpha_2)(a).
                \end{align}
            \end{subequations}
            Donc \( \varphi_k\) est linéaire pour tout \( k\).
        \item[La relation]
            La relation 
            \begin{equation}
                d^k(\varphi\circ f)=\varphi_k\circ d^kf
            \end{equation}
            se démontre par récurrence, chaque pas étant justifié de la même manière que \eqref{EQooUJPWooTzgSJx}.
    \end{subproof}
\end{proof}


% This is part of Mes notes de mathématique
% Copyright (c) 2006-2020
%   Laurent Claessens, Carlotta Donadello
% See the file fdl-1.3.txt for copying conditions.

%+++++++++++++++++++++++++++++++++++++++++++++++++++++++++++++++++++++++++++++++++++++++++++++++++++++++++++++++++++++++++++ 
\section{Suites et séries : généralités}
%+++++++++++++++++++++++++++++++++++++++++++++++++++++++++++++++++++++++++++++++++++++++++++++++++++++++++++++++++++++++++++
\label{SECooTDZNooJvjPks}

%--------------------------------------------------------------------------------------------------------------------------- 
\subsection{Convergence uniforme}
%---------------------------------------------------------------------------------------------------------------------------

%///////////////////////////////////////////////////////////////////////////////////////////////////////////////////////////
\subsubsection{Critère de Cauchy uniforme}
%///////////////////////////////////////////////////////////////////////////////////////////////////////////////////////////

\begin{definition}[\cite{TrenchRealAnalisys}]
    Soient un espace (dont la nature n'est pas très importante) \( \Omega\), une partie \( A\) de \( \Omega\) et un espace normé \( V\). Lorsque \( g\) est une fonction \( g\colon \Omega\to V\), nous notons
    \begin{equation}
    \| g \|_A=\sup_{x\in A}\| g(x) \|
    \end{equation}
    C'est la norme supremum limitée à la partie \( A\).

    Nous disons qu'une suite de fonctions \( (f_n)\) définies sur un ensemble \( A\) \defe{converge uniformément sur \( A\)}{convergence!uniforme} vers la fonction \( f\) si
    \begin{equation}
        \lim_{n\to \infty} \| f_n-f \|_A=0.
    \end{equation}
\end{definition}

\begin{proposition}[Critère de Cauchy uniforme\cite{LCbyNWQ}]   \label{PropNTEynwq}
    Soit \( X\) un espace topologique et \( (Y,d)\) un espace topologique complet. La suite de fonctions \( f_n\colon X\to Y\) converge uniformément sur \( A\) si et seulement si pour tout \( \epsilon>0\) il existe \( N\in \eN\) tel que si \( k,l>N\) alors
    \begin{equation}
        d\big( f_k(x),f_l(x) \big)\leq \epsilon
    \end{equation}
    pour tout \( x\in X\).
\end{proposition}
\index{Cauchy!critère!uniforme}
\index{critère!Cauchy!uniforme}
Grosso modo, cela dit que si qu'une suite de Cauchy pour la norme uniforme est une suite uniformément convergente. Le fait que la suite converge fait partie du résultat et n'est pas une hypothèse. Ce critère sera utilisé pour montrer que \( \big( C(K),\| . \|_{\infty} \big)\) est complet, proposition~\ref{PropSYMEZGU}.

\begin{proof}
    Si \( f_n\stackrel{unif}{\longrightarrow}f\) alors le critère est satisfait; c'est dans l'autre sens que la preuve est intéressante.

    Soit donc une suite de fonctions satisfaisant au critère et montrons qu'elle converge uniformément. Pour tout \( x\in X\) la suite \( n\mapsto f_n(x)\) est de Cauchy dans l'espace complet \( Y\); nous avons donc convergence ponctuelle \( f_n\to f\). Nous devons prouver que cette convergence est uniforme. Soit \( \epsilon>0\) et \( N\in \eN\) tel que si \( k,l>N\) alors
    \begin{equation}
        d\big( f_k(x),f_l(x) \big)\leq \epsilon
    \end{equation}
    pour tout \( x\in X\). Si nous nous fixons un tel \( k\) et un \( x\in A\) nous considérons l'inégalité
    \begin{equation}
        d\big( f_k(x),f_l(x) \big)\leq \epsilon
    \end{equation}
    qui est vraie pour tout \( l\). En passant à la limite \( l\to\infty\) (limite qui commute avec la fonction distance par définition de la topologie) nous avons
    \begin{equation}
        d\big( f_k(x),f(x) \big)\leq \epsilon.
    \end{equation}
    Cette inégalité étant valable pour tout \( x\in X\), cela signifie que \( f_n\stackrel{unif}{\longrightarrow}f\).
\end{proof}

\begin{theorem}[Limite uniforme de fonctions continues]			\label{ThoUnigCvCont}
    Soit \( A\), un ensemble mesuré et \( f_n\colon A\to \eR^n\), une suite de fonctions continues convergeant uniformément vers \( f\). Si les fonctions \( f_n\) sont toutes continues en \( x_0\in A\), alors \( f\) est continue en \( x_0\).
\end{theorem}

\begin{proof}
    Soit \( \epsilon>0\). Si \( x\in A\) nous avons, pour tout \( n\), la majoration
    \begin{subequations}
        \begin{align}
            \| f(x)-f(x_0) \|&\leq \| f(x)-f_n(x) \|+\| f_n(x)-f_n(x_0) \|+\| f_n(x_0)-f(x_0) \|\\
            &\leq\| f_n(x)-f_n(x_0) \|+2\| f_n-f \|_{\infty}.
        \end{align}
    \end{subequations}
    Grâce à l'uniforme convergence, nous considérons \(N\in \eN\) tel que \( \| f_n-f \|\leq \epsilon\) pour tout \( n\geq N\). Pour de tels \( n\), nous avons
    \begin{equation}
        \| f(x)-f(x_0) \|\leq 2\epsilon+\| f_n(x)-f_n(x_0) \|.
    \end{equation}
    La continuité de \( f_n\) nous fournit un \( \delta>0\) tel que \( \| f_n(x_0)-f_n(x) \|<\epsilon\) dès que \( \| x-x_0 \|<\delta\). Pour ce \( \delta\), nous avons alors \( \| f(x)-f(x_0) \|<\epsilon\).

    Donc lorsque \( \| x-x_0 \|<\delta\) et \( n\geq N\) nous avons
    \begin{equation}
        \| f(x)-f(x_0) \|\leq 3\epsilon,
    \end{equation}
    où vous remarquerez qu'il n'y a plus de dépendance en \( n\). Cela prouve la continuité de \( f\) en \( x_0\).
\end{proof}

%///////////////////////////////////////////////////////////////////////////////////////////////////////////////////////////
\subsubsection{Complétude avec la norme uniforme}
%///////////////////////////////////////////////////////////////////////////////////////////////////////////////////////////

\begin{proposition}[Limite uniforme de fonctions continues]\label{PropCZslHBx}
    Soit \( X\) un espace topologique et \( (Y,d)\) un espace métrique. Si une suite de fonctions \( f_n\colon X\to Y\) continues converge uniformément, alors la limite est séquentiellement continue\footnote{Si \( X\) est métrique, alors c'est la continuité usuelle par la proposition~\ref{PropFnContParSuite}.}.
\end{proposition}

\begin{proof}
    Soit \( a\in X\) et prouvons que \( f\) est séquentiellement continue en \( a\). Pour cela nous considérons une suite \( x_n\to a\) dans \( X\). Nous savons que \( f(x_n)\stackrel{Y}{\longrightarrow}f(x)\). Pour tout \(k\in \eN\), tout \( n\in \eN\) et tout \( x\in X\) nous avons la majoration
    \begin{subequations}
        \begin{align}
            \big\| f(x_n)-f(x) \big\|&\leq \big\| f(x_n)-f_k(x_n) \big\|+\big\| f_k(x_n)-f_k(x) \big\|+\big\| f_k(x)-f(x) \big\|\\
            &\leq 2\| f-f_k \|_{\infty}+\big\| f_k(x_n)-f_k(x) \big\|.
        \end{align}
    \end{subequations}
    Soit \( \epsilon>0\). Si nous choisissons \( k\) suffisamment grand la premier terme est plus petit que \( \epsilon\). Et par continuité de \( f_k\), en prenant \( n\) assez grand, le dernier terme est également plus petit que \( \epsilon\).
\end{proof}

\begin{proposition} \label{PropSYMEZGU}
    Soit \( X\) un espace topologique métrique \( (Y,d)\) un espace espace métrique complet. Alors les espaces
    \begin{enumerate}
        \item
            \( \big( C^0_b(X,Y),\| . \|_{\infty} \big)\) des fonctions continues et bornées \( X\to Y\),
        \item
            \( \big( C^0_0(X,Y),\| . \|_{\infty} \big)\) des fonctions continues et s'annulant à l'infini
        \item
            \( \big( C^k_0(X,Y),\| . \|_{\infty} \big)\) des fonctions de classe \( C^k\) et s'annulant à l'infini
    \end{enumerate}
    sont complets.
\end{proposition}

\begin{proof}
    Soit \( (f_n)\) une suite de Cauchy dans \( C(X,Y)\), c'est-à-dire que pour tout \( \epsilon>0\) il existe \( N\in \eN\) tel que si \( k,l>N\) nous avons \( \| f_k-f_l \|_{\infty}\leq \epsilon\). Cette suite vérifie le critère de Cauchy uniforme~\ref{PropNTEynwq} et donc converge uniformément vers une fonction \( f\colon X\to Y\). La continuité (ou l'aspect \( C^k\)) de la fonction \( f\) découle de la convergence uniforme et de la proposition~\ref{PropCZslHBx} (c'est pour avoir l'équivalence entre la continuité séquentielle et la continuité normale que nous avons pris l'hypothèse d'espace métrique).

    Si les fonctions \( f_k\) sont bornées ou s'annulent à l'infini, la convergence uniforme implique que la limite le sera également.
\end{proof}
    Notons que si \( X\) est compact, les fonctions continues sont bornées par le théorème~\ref{ThoImCompCotComp} et nous pouvons simplement dire que \( C^0(X,Y)\) est complet, sans préciser que nous parlons des fonctions bornées.


\begin{lemma}       \label{LemdLKKnd}
    Soient un espace topologique compact \( A\) et un espace complet \( B\). L'ensemble des fonctions continues de \( A\) vers \( B\) muni de la norme uniforme est complet.

    Dit de façon courte : \( \big( C(A,B),\| . \|_{\infty} \big)\) est complet.
\end{lemma}

\begin{proof}
    Soit \( (f_k)\) une suite de Cauchy de fonctions dans \( C(A,B)\). Pour chaque \( x\in A \) nous avons
    \begin{equation}
        \| f_k(x)-f_l(x) \|_B\leq \| f_k-f_l \|_{\infty},
    \end{equation}
    de telle sorte que la suite \( (f_k(x))\) est de Cauchy dans \( B\) et converge donc vers un élément de \( B\). La suite de Cauchy \( (f_k)\) converge donc ponctuellement vers une fonction \( f\colon A\to B\). Nous devons encore voir que cette fonction est continue; ce sera l'uniformité de la norme qui donnera la continuité. En effet soit \( x_n\to x\) une suite dans \( A\) qui converge vers \( x\in A\). Pour chaque \( k\in \eN\) nous avons
    \begin{equation}
        \| f(x_n)-f(x) \|\leq \| f(x_n)-f_k(x_n) \|  +\| f_k(x_n)-f_k(x) \|+\| f_k(x)-f(x) \|.
    \end{equation}
    En prenant \( k\) et \( n\) assez grands, cette expression peut être rendue aussi petite que l'on veut; le premier et le troisième terme par convergence ponctuelle \( f_k\to f\), le second terme par continuité de \( f_k\). La suite \( f(x_n)\) est donc convergente vers \( f(x)\) et la fonction \( f\) est continue.
\end{proof}

\begin{probleme}
Il serait sans doute bon de revoir cette preuve à la lumière du critère de Cauchy uniforme~\ref{PropNTEynwq}.
\end{probleme}


\begin{normaltext}[\cite{ooXYZDooWKypYR}]
    Le théorème de Stone-Weierstrass indique que les polynômes sont denses pour la topologie uniforme dans les fonctions continues. Donc il existe des limites uniformes de fonctions \( C^{\infty}\) qui ne sont même pas dérivables. Les espaces de type \( C^p\) munis de \( \| . \|_{\infty}\) ne sont donc pas complets sans quelques hypothèses. Voir la proposition~\ref{PropSYMEZGU} et le thème~\ref{THMooOCXTooWenIJE}.
\end{normaltext}

\begin{theorem}[Théorème de Dini\cite{JIFGuct}] \label{ThoUFPLEZh}
    Soient un espace métrique complet \( D\) et une suite de fonctions \( f_n\in C(D,\eR)\) telle que
    \begin{enumerate}
        \item
            \( f_n\to g\) ponctuellement,
        \item
            \( g\in C(D,\eR)\),
        \item
            la suite \( (f_n)\) est croissante, c'est-à-dire que pour tout \( x\in D\) et pour tout \( n\geq 0\) nous avons \( f_{n+1}(x)\geq f_n(x)\).
    \end{enumerate}
    Alors la convergence est uniforme.
\end{theorem}
\index{convergence!uniforme!théorème de Dini}
\index{compacité!théorème de Dini}
\index{théorème!Dini}

\begin{proof}
    Soit \( x\in D\) et \( \epsilon>0\). Il existe \( N(x)\in \eN\) tel que
    \begin{equation}
        g(x)-\epsilon\leq f_{N(x)}\leq g(x).
    \end{equation}
    De plus \( g\) et \( f_{N(x)}\) sont des fonctions continues, donc il existe \( \eta(x)\) tel que si \( y\in B\big( x,\eta(x) \big)\) alors
    \begin{subequations}
        \begin{align}
            g(y)&\in B\big( g(x),\epsilon \big) \label{subEqXKjgKgv}\\
            f_{N(x)}(y)&\in B\big( f_{N(x)}(x),\epsilon \big)   \label{subEqHTiYZLd}.
        \end{align}
    \end{subequations}
    Si \( n\geq N(x)\) et si \( y\in B(x,\eta(x))\) alors nous avons les majorations
    \begin{equation}
            g(y)\geq f_n(y)
            \geq f_{N(x)}(y)
            \geq f_{N(x)}(x)-\epsilon
            \geq g(x)-2\epsilon
            \geq g(y)-3\epsilon.
    \end{equation}
    Justifications :
    \begin{multicols}{2}
        \begin{enumerate}
            \item
                Les deux premières inégalités sont la croissance de la suite.
            \item
                La suivante est \eqref{subEqHTiYZLd}.
            \item
                Ensuite il y a le choix de \( N(x)\).
            \item
                Et enfin il y a \eqref{subEqXKjgKgv}.
        \end{enumerate}
    \end{multicols}
    Nous retenons que si \( x\in D\) et si \( n\geq N(x)\) alors
    \begin{equation}    \label{EqJCMktdj}
        g(y)\geq f_n(y)\geq g(y)-3\epsilon
    \end{equation}
    pour tout \( y\in B(x,\eta(x))\).

    Nous utilisons maintenant la compacité de \( D\). Pour chaque \( x\in D\) nous pouvons considérer la boule ouverte \( B\big( x,\eta(x) \big)\); ces boules recouvrent \( D\). Nous en extrayons un sous-recouvrement fini, c'est-à-dire un ensemble fini d'éléments \( x_1\),\ldots, \( x_K\) tels que
    \begin{equation}
        D=\bigcup_{k=1}^K B\big(x_k,\eta(x_k)\big).
    \end{equation}
    Si à ce moment vous ne comprenez pas pourquoi c'est une égalité au lieu d'une inclusion, il faut lire l'exemple~\ref{ExKYZwYxn}. Considérons
    \begin{equation}
        n\geq N=\max\{ N(x_1),\ldots, N(x_K) \}.
    \end{equation}
    Pour tout \( y\in D\) il existe \( k\in\{ 1,\ldots, K \}\) tel que \( y\in B\big( x_k,\eta(x_k) \big)\), et vu que \( n\geq N(x_k)\) nous reprenons la majoration \eqref{EqJCMktdj} :
    \begin{equation}
        g(y)\geq f_n(y)\geq g(y)-3\epsilon.
    \end{equation}
    Pour le \( n\) choisi nous avons ces inégalités pour tout \( y\in D\), c'est-à-dire que nous avons \( \| f_n-g \|\leq 3\epsilon\) et donc la convergence uniforme.
\end{proof}

\begin{proposition}[\cite{MonCerveau}]      \label{PROPooFWVIooCzXojO}
    Soient une suite de fonctions continues \( u_i\colon \eR\to \eR\) et une fonction continue \( u\) telle que \( u_i\to u\) simplement. Alors la convergence est uniforme sur tout compact.
\end{proposition}

\begin{proof}
    Soit un compact \( K\); nous notons \( \| . \|\) la norme uniforme sur \( K\). Supposons que la limite ne soit pas uniforme, c'est-à-dire qu'il existe un \( \epsilon>0\) tel que 
    \begin{equation}
        \| u_i-u \|> 2\epsilon
    \end{equation}
    pour tout \( i\). Cela permet de considérer pour tout \( i\) un élément \( x_i\in K\) tel que\footnote{Notez l'inégalité stricte, obetenue en considérant $2\epsilon$ plus haut.}
    \begin{equation}
        \| u_i(x_i)-u(x_i) \|> \epsilon.
    \end{equation}
    Pour cela, il faut noter que \( K\) est compact et que la fonction \( x\mapsto \| u_i(x)-u(x) \|\) est continue sur \( K\). Elle est donc bornée et atteint son maximum (c'est le théorème de Weierstrass \ref{ThoWeirstrassRn}).

    La suite \( i\mapsto x_i\) est une suite dans un compact, et quitte à prendre une sous-suite, nous supposons qu'elle converge vers \( a\in K\) (ça, c'est Bolzano-Weierstrass \ref{LemMGQqgDG}).

    La convergence ponctuelle \( u_i\to u\), prise en \( a\), dit qu'il existe un \( N\) tel que \( | u_i(a)-u(a) |<\epsilon\) pour tout \( i\geq N\). Pour un tel \( i\), nous avons aussi
    \begin{equation}
        | u_i(x)-u(x) |<\epsilon
    \end{equation}
    sur un voisinage de \( a\), parce que \( u_i-u\) est continue. Mais tout voisinage de \( a\) contient un élément \( x_j\) pour lequel
    \begin{equation}
        | u_i(x_j)-u(x_j) |>\epsilon.
    \end{equation}
    Contradiction.
\end{proof}

%--------------------------------------------------------------------------------------------------------------------------- 
\subsection{Série de fonctions}
%---------------------------------------------------------------------------------------------------------------------------

Les séries de fonctions sont des cas particuliers de suites. 

\begin{definition}      \label{DEFooYEIUooCAgrxI}
    Si \( (f_n)\) est une suite de fonctions, nous définissons la somme des \( f_n\) de la façon suivante :
    \begin{equation}
        \sum_{n=1}^{\infty}f_n=\lim_{N\to \infty} \sum_{n=1}^{N}f_n.
    \end{equation}
    Le membre de droite est une définition de la notation introduite dans le membre de gauche.
\end{definition}
Avant de vous lancer, relisez une bonne fois les définitions de convergence absolue (définition \ref{DefVFUIXwU}) et de convergence uniforme (équation \ref{EqLNCJooVCTiIw}).

\begin{lemma}
    Soient des fonctions \( u_n\colon \Omega\to \eC\). Si il existe une suite réelle positive \( (a_n)_{n\in \eN}\) telle que
    \begin{enumerate}
        \item
            pour tout \( z\in \Omega\) et pour tout \( n\in \eN\) nous avons \( | u_n(z) |\leq a_n\) (c'est-à-dire \( a_n\geq \| u_n \|_{\infty}\)),
        \item
            la somme \( \sum_{n}a_n\) converge,
    \end{enumerate}
    alors la série de fonctions \( \sum_{n=0}^{\infty}u_n\) converge normalement\footnote{Définition~\ref{DefVBrJUxo}.}.
\end{lemma}

\begin{proof}
    Découle du lemme de comparaison~\ref{LemgHWyfG}.
\end{proof}

\begin{theorem}				\label{ThoSerCritAbel}
	Soit $\sum_{k=1}^{\infty}g_k(x)$, une série de fonctions complexes où $g_k(x)=\varphi_k(x)\psi_k(x)$. Supposons que
	\begin{enumerate}

		\item
			$\varphi_k\colon A\to \eC$ et $| \sum_{k=1}^K\varphi_k(x) |\leq M$ où $M$ est indépendant de $x$ et $K$,
		\item
			$\psi_k\colon A\to \eR$ avec $\psi_k(x)\geq 0$ et pour tout $x$ dans $A$, $\psi_{k+1}(x)\leq \psi_k(x)$, et enfin supposons que $\psi_k(x)$ converge uniformément vers $0$.

	\end{enumerate}
	Alors $\sum_{k=1}^{\infty}g_k$ est uniformément convergente.
\end{theorem}

\begin{theorem}		\label{ThoAbelSeriePuiss}
	Si la série de puissances (réelle) converge en $x=x_0+R$, alors elle converge uniformément sur $\mathopen[ x_0-R+\epsilon , x_0+R \mathclose]$ ($\epsilon>0$) vers une fonction continue.
\end{theorem}


\begin{proposition}     \label{PropUEMoNF}
    Soit \( (u_n)\) une suite de fonctions continues \( u_n\colon \Omega\subset\eC\to \eC\). Si la série \( \sum_nu_n\) converge normalement alors la somme est continue.
\end{proposition}

\begin{proof}
    Nous posons \( u(z)=\lim_{N\to \infty} \sum_{n=0}^N u_n(z)\), et nous vérifions que la fonction ainsi définie sur \( \Omega\) est continue. Soit \( z\in \Omega\). Prouvons la continuité de \( u\) au point \( z\). Pour tout \( z'\) dans un voisinage de \( z\) nous avons
    \begin{subequations}
        \begin{align}
            \big| u(z)-u(z') \big|&=\left| \sum_{n=0}^{N}u_n(z)-\sum_{n=0}^{N}u_n(z')+\sum_{n=N+1}^{\infty}u_n(z)-\sum_{n=N+1}^{\infty}u_n(z') \right| \\
            &\leq \left| \sum_{n=0}^N u_n(z)-\sum_{n=0}^Nu_n(z') \right| +\sum_{n=N+1}^{\infty}| u_n(z) |+\sum_{n=N+1}^{\infty}| u_n(z') |.
        \end{align}
    \end{subequations}
    Étant donné que les sommes partielles sont continues, en prenant \( N\) suffisamment grand, le premier terme peut être rendu arbitrairement petit. Si \( N\) est suffisamment grand, le second terme est également petit. Par contre, cet argument ne tient pas pour le troisième terme parce que nous souhaitons une majoration pour tout \( z'\) dans une boule autour de \( z\). Nous devons donc écrire
    \begin{equation}
        \sum_{n=N}^{\infty}| u_n(z) |\leq \sum_{n=N+1}^{\infty}\| u_n \|_{\infty}.
    \end{equation}
    Ce dernier est arbitrairement petit lorsque \( N\) est grand. Notons que nous avons utilisé l'hypothèse de convergence normale.
\end{proof}

La même propriété, avec la même démonstration, tient dans le cas d'espaces vectoriels normés.

\begin{proposition} \label{PropOMBbwst}
    Soient \( E\) et \( F\), deux espaces vectoriels normés, \( \Omega\) une partie ouverte de \( E\) et une suite de fonctions \( u_n\colon \Omega\to F\) convergeant normalement sur \( \Omega\), c'est-à-dire que \( \sum_n\| u_n \|_{\infty}\) converge, la norme \( \| . \|_{\infty} \) devant être comprise comme la norme supremum sur \( \Omega\). Alors la fonction \( u=\sum_nu_n\) est continue sur \( \Omega\).
\end{proposition}

\begin{proof}
    Soit \( x,x'\in \Omega\) en supposant que \( \| x-x' \|\) est petit. Soit encore \( \epsilon>0\). Nous allons montrer la continuité en \( x\). Pour cela nous savons que pour tout \( N\) l'inégalité suivante est correcte :
    \begin{equation}
        \| u(x)-u(x') \|\leq \left\|  \sum_{n=0}^Nu_n(x)-\sum_{n=0}^{N}u_n(x') \right\|+\sum_{n=N+1}^{\infty}\| u_n(x) \|+\sum_{n=N+1}^{\infty}\| u_n(x') \|.
    \end{equation}
    Les deux derniers termes sont majorés par \( \sum_{n=N+1}^{\infty}\| u_n \|_{\infty}\) qui, par hypothèse, peut être rendu aussi petit que souhaité en choisissant \( N\) assez grand. Nous choisissons donc un \( N\) tel que ces deux termes soient plus petits que \( \epsilon\). Ce \( N\) étant fixé, la fonction \( \sum_{n=0}^{N}u_n\) est continue et nous pouvons choisir \( x'\) assez proche de \( x\) pour que le premier terme soit majoré par \( \epsilon\).
\end{proof}

\begin{theorem}			\label{ThoSerUnifCont}
	Si les $g_k$ sont continues et si $\sum g_k$ converge uniformément, alors $\sum g_k$ est continue.
\end{theorem}

Le corolaire suivant permet de considérer des séries de fonctions indexées par exemple par \( \eZ\) plutôt que par \( \eN\).
\begin{corollary}
    Une famille dénombrable de fonctions continues convergeant normalement converge vers une fonction continue.
\end{corollary}

\begin{proof}
    Soit \( I\) dénombrable. Considérons une famille de fonctions continues \( (f_n)_{n\in I}\) telles que la famille \( (\| f_i \|_{\infty})_{i\in I}\) soit sommable. Le proposition~\ref{PropoWHdjw} nous permet d'utiliser une bijection entre \( I\) et \( \eN\). Le théorème~\ref{PropUEMoNF} s'applique alors.
\end{proof}

\begin{theorem}[Critère de Weierstrass]\index{critère!Weierstrass!série de fonctions}		\label{ThoCritWeierstrass}
	Soit une suite de fonctions $f_k\colon A\to \eC$ telles que $| f_k(x) |\leq M_k\in\eR$, $\forall x\in A$. Si $\sum_{k=1}^{\infty}M_k$ converge, alors $\sum_{k=1}^{\infty}f_k$ converge absolument et uniformément.
\end{theorem}

\begin{proof}
    La convergence normale est facile : l'hypothèse dit que \( \| f_k \|_{\infty}\leq M_k\), et donc que
    \begin{equation}
        \sum_{k=1}^{\infty}\| f_k \|_{\infty}\leq \sum_kM_k<\infty.
    \end{equation}

    La convergence uniforme est à peine plus subtile. Nous nommons \( F\) la fonction somme. Pour tout \( x\) et pour tout \( N\), nous avons
    \begin{subequations}
        \begin{align}
            \left\| \sum_{n=1}^Nf_n(x)-F(x) \right\|&=\| \sum_{n=N}^{\infty}f_n(x) \|\\
            &\leq\sum_{n=N}^{\infty}\| f_k(x) \|\\
            &\leq \sum_{n=N}^{\infty}\| f_n \|_{\infty}.
        \end{align}
    \end{subequations}
    La convergence normale étant assurée, la série \( \sum_{n_1}^{\infty}\| f_n \|_{\infty}\) est finie, ce qui implique que la queue de somme \( \sum_{n=N}^{\infty}\| f_n \|_{\infty}\) tend vers zéro lorsque \( N\to \infty\). Pour tout \( \epsilon\), il existe donc un \( N\) (non dépendant de \( x\)) tel que
    \begin{equation}
        \| \sum_{n=1}^Nf_n(x)-F(x) \|\leq \epsilon.
    \end{equation}
    En prenant le supremum sur \( x\in A\) nous trouvons la convergence uniforme.
\end{proof}

\begin{remark}
    Il n'y a pas de critère correspondant pour les suites. Il n'est pas vrai que si \( \lim_{n\to \infty}\| f_n \| \) existe, alors \( \lim_{n\to \infty} f_n\) existe, comme le montre l'exemple
    \begin{equation}
        f_n(x)=\begin{cases}
            1    &   \text{si } x\in\mathopen[ 0 , 1 \mathclose]\text{ et } n\text{ est pair}\\
            1    &    \text{si } x\in\mathopen[ 1 , 2 \mathclose]\text{ et } n\text{ est impair}\\
             0   &    \text{sinon.}
        \end{cases}
    \end{equation}
\end{remark}

%+++++++++++++++++++++++++++++++++++++++++++++++++++++++++++++++++++++++++++++++++++++++++++++++++++++++++++++++++++++++++++ 
\section{Permuter limite et dérivée}
%+++++++++++++++++++++++++++++++++++++++++++++++++++++++++++++++++++++++++++++++++++++++++++++++++++++++++++++++++++++++++++

Pour permuter différentielle et limite, ce sera le théorème \ref{ThoLDpRmXQ}.

\begin{theorem}[\cite{TrenchRealAnalisys,ooCPZDooOqIIEz}]     \label{THOooXZQCooSRteSr}
    Soient une suite de fonctions \( f_i\colon \eR\to \eR\), une fonction \( f\colon \eR\to \eR\) et une fonction \( g\colon \eR\to \eR\) telles que
    \begin{enumerate}
        \item
            \( f_i\) est de classe \( C^1\) pour tout \( i\),
        \item
            \( f_i\to f\) simplement,
        \item
            \( f_i'\to g\) uniformément sur tout compact.
    \end{enumerate}
    Alors
    \begin{enumerate}
        \item
            \( f\) est de classe \( C^1\),
        \item
            \( f'=g\),
        \item
            \( f_i\to f\) uniformément sur tout compact.
    \end{enumerate}
\end{theorem}

\begin{proof}
  On commence par démonter la convergence uniforme sur les compacts des~$f_i$.
  Soit~$K \subseteq \eR$ un intervalle compact contenant~$x$.
  Il suffit de montrer que la suite~$(f_i)_i$ restreinte à $K$ est une suite de
  Cauchy pour la norme uniforme.  Soit~$\epsilon > 0$.
  On note~$\omega_i$ le module de continuité de~$f_i'$.
  Par le lemme~\ref{LEMooKPPSooPIncvn}, il existe~$N \geq 0$ tel que pour tout \( \delta>0\) nous ayons
  \begin{equation}\label{eq:1}
    \omega_i(\delta) \leq \omega_g(\delta) + \epsilon.
  \end{equation}
  
  Soit~$y\in K$, $n \in \eN$ et~$\alpha_n = \frac{x-y}{n+1}$.
  Pour tout~$i \geq 0$,
  \begin{equation}
    f_i(y) = f_i(x) + \sum_{k=0}^n \left( f_i(x+(k+1)\alpha_n) - f_i(x+k\alpha_n) \right),
  \end{equation}
  c'est une somme télescopique.
  Par le théorème des accroissements finis, il existe pour tout~$0\leq k\leq n$
  un réel~$u_{n,i,k} \in [k,(k+1)\alpha_n]$ tel que
  \begin{equation}
    f_i(x+(k+1)\alpha_n) - f_i(x+k\alpha_n) = \alpha_n f'_i(x+ u_{n,i,k}),
  \end{equation}
  de sorte que
  \begin{equation}
    f_i(y) = f_i(x) + \alpha_n \sum_{k=0}^n  f'_i(x+ u_{n,i,k}).
  \end{equation}
  Et pour tout~$i,j \geq 0$, on obtient
  \begin{align}
    \left| f_i(y) - f_j(y) \right| &\leq \left| f_i(x) - f_j(x) \right|
                                     + \alpha_n \sum_{k=0}^n \left| f'_i(x+ u_{n,i,k}) - f'_j(x+u_{n,j,k}) \right|.
  \end{align}
  Or
  \begin{align}
    \left| f'_i(x+ u_{n,i,k}) - f'_j(x+u_{n,j,k}) \right| &\leq  \left| f'_i(x+ u_{n,i,k}) - f'_i(x+u_{n,j,k}) \right| +  \left| f'_i(x+ u_{n,j,k}) - f'_j(x+u_{n,j,k}) \right| \\
                                                          &\leq \omega_i(\alpha_n) + \|f'_i-f'_j\|,
  \end{align}
  et il suit que pour tout~$i,j \geq N$,
  \begin{align}
    \left| f_i(y) - f_j(y) \right| & \leq  \left| f_i(x) - f_j(x) \right| + |x-y| \left(\omega_i(\alpha_n) + \|f'_i+f'_j\|_K \right)\\
    &\leq \left| f_i(x) - f_j(x) \right| + |x-y| \left(\omega_g(\alpha_n) + \epsilon + \|f'_i+f'_j\|_K \right), \label{eq:2}
  \end{align}
  où la dernière inégalité vient de~\eqref{eq:1}.
  Comme~$g$ est continue (limite uniforme de fonctions continues) sur un
  compact, elle est uniformément continue et
  $\lim_{\delta\to 0} \omega_g(\delta) = 0$.
  Donc l'inégalité~\eqref{eq:2}, avec~$n\to \infty$ devient
  \begin{equation}\label{eq:3}
  \left| f_i(y) - f_j(y) \right| \leq \left| f_i(x) - f_j(x) \right| + |x-y|
    \left(\epsilon + \|f'_i+f'_j\|_K \right).
  \end{equation}

  Comme~$K$ est borné, $\left| x-y \right|$ est borné par une quantité~$M$ ne
  dépendant que de~$K$.
  En prenant dans~\eqref{eq:3} le supremum par rapport à~$y$, on obtient
  \begin{equation}
    \|f_i-f_j\|_K \leq  \left| f_i(x) - f_j(x) \right| + |x-y|
    \left(\epsilon + \|f'_i+f'_j\|_K \right).
  \end{equation}
  
  Étant donné
  que~$\left( f_i(x) \right)_i$ est une suite de Cauchy, que~$\left( f'_i
  \right)_i$ est une suite de Cauchy pour la norme uniforme sur~$K$,
  il suit que~$(f_i|_K)_i$ est une suite de Cauchy. Donc elle converge uniformément vers une
  limite~$f|_K$.
  
  \bigskip
  Montrons maintenant que la limite~$f$ de la suite~$(f_i)_i$ est dérivable
  et que~$f'=g$.
  Soit~$y\in \eR$ et soit~$K$ un voisinage compact de~$y$. On reprend les
  notations précédentes. Pour tout~$\delta \in \eR$ suffisamment petit
  $y+\delta \in K$, et pour tout~$i \geq 0$ on a alors
  \begin{subequations}
      \begin{align}
    \Big| \frac{1}\delta \big( f(y+\delta) &- f(y) \big) - g(y) \Big|\\
    &\leq  \frac1\delta \left| f(y)-f_i(y) \right| + \frac1\delta \left| f(y+\delta) - f_i(y+\delta)  \right|
      + \left| \frac1\delta \left(f_i(y+\delta) - f_i(y)\right) - g(y) \right|\\
    &\leq \frac2\delta \|f_i - f\|_K  + \left| \frac1\delta \left(f_i(y+\delta) - f_i(y)\right) - g(y) \right|.
      \end{align}
  \end{subequations}
  Par le théorème des accroissements finis, il existe~$u\in[y-|\delta|, y+|\delta|]$ tel que
  \begin{equation}
    \frac1\delta \left(f_i(y+\delta) - f_i(y)\right) = f'_i(u).
  \end{equation}
  Il suit,
  \begin{align}
    \left| \frac{1}\delta \left( f(y+\delta) - f(y) \right) - g(y) \right| &\leq \frac2\delta \|f_i - f\|_K +
                                                                             \left| f'_i(u) - f'_i(y) \right| + \|f_i'-g\| \\
    &\leq \frac2\delta \|f_i - f\|_K + \omega_i(\delta) + \|f_i'-g\|.
  \end{align}
  En prenant~$i \to \infty$, on obtient
  \begin{equation}
    \left| \frac{1}\delta \left( f(y+\delta) - f(y) \right) - g(y) \right| \leq \omega_g(\delta).
  \end{equation}
  Or~$\lim_{\delta\to 0}\omega_g(\delta) = 0$, donc~$f$ est dérivable en~$y$
  et~$f'(y) = g(y)$.
\end{proof}

\begin{theorem}		\label{ThoSerUnifDerr}
	Soit $U\subset\eR^n$ ouvert, $f_k\colon U\to \eR$ et $f_k$ de classe $C^1$. Supposons que $f_k$ converge simplement vers $f$ et que $\partial_if_k$ converge uniformément sur tout compact  vers une fonction $g_i$ pour $i=1,\ldots,n$. Alors $f$ est de classe $C^1$ et $\partial_if=g_i$. De plus, $f_k$ converge vers $f$ uniformément.
\end{theorem}
\index{permuter!dérivée et limite}

%+++++++++++++++++++++++++++++++++++++++++++++++++++++++++++++++++++++++++++++++++++++++++++++++++++++++++++++++++++++++++++
\section{Densité des polynômes}
%+++++++++++++++++++++++++++++++++++++++++++++++++++++++++++++++++++++++++++++++++++++++++++++++++++++++++++++++++++++++++++

%---------------------------------------------------------------------------------------------------------------------------
\subsection{Théorème de Stone-Weierstrass}
%---------------------------------------------------------------------------------------------------------------------------

Voir le thème~\ref{THEooPUIIooLDPUuq}.

Note : le lemme~\ref{LemYdYLXb} est utilisé dans la démonstration du théorème~\ref{ThoWmAzSMF}; c'est pour cela que nous l'avons isolé.

\begin{lemma}       \label{LemYdYLXb}
    Il existe une suite de polynômes sur \( \mathopen[ 0 , 1 \mathclose]\) convergeant uniformément vers la fonction racine carrée.
\end{lemma}

\begin{proof}
    Nous donnons cette suite par récurrence :
    \begin{subequations}
        \begin{align}
            P_0(t)&=0\\
            P_{n+1}(t)&=P_n(t)+\frac{ 1 }{2}\big( t-P_n(t)^2 \big).
        \end{align}
    \end{subequations}
    Nous commençons par montrer que pour tout \( t\in \mathopen[ 0 , 1 \mathclose]\), \( P_n(t)\in\mathopen[ 0 , \sqrt{t} \mathclose]\). Pour \( P_0\), c'est évident. Ensuite nous avons
    \begin{subequations}
        \begin{align}
            P_{n+1}(t)-\sqrt{t}&=P_n(t)-\sqrt{t}+\frac{ 1 }{2}(t-P_n(t)^2)\\
            &=\big( P_n(t)-\sqrt{t} \big)\left( 1-\frac{ 1 }{2}\frac{ t-P_n(t)^2 }{ P_n(t)-\sqrt{t} } \right)\\
            &=\big( P_n(t)-\sqrt{t} \big)\left( 1-\frac{ \sqrt{t}+P_n(t) }{2} \right)\\
            &\leq 0
        \end{align}
    \end{subequations}
    parce que \( \sqrt{t} \leq 1\) et \( P_n(t)\leq 1\) par hypothèse de récurrence.

    Nous savons au passage que \( P_n(t)\) est une suite réelle croissante parce que \( t-P_n(t)^2\geq t-(\sqrt{t})^2=0\). La suite \( P_n(t)\) est donc croissante et majorée par \( \sqrt{t}\); elle converge donc. Les candidats limites sont déterminés par l'équation
    \begin{equation}
        \ell=\ell+\frac{ 1 }{2}(t-\ell^2),
    \end{equation}
    dont les solutions sont \( \ell=\pm\sqrt{t}\). La suite étant positive, nous avons une convergence ponctuelle de \( P_n\) vers la racine carrée. Cette suite étant une suite croissante de fonctions continues sur un compact, convergeant ponctuellement vers une fonction continue, la convergence est uniforme par le théorème de Dini~\ref{ThoUFPLEZh}.
\end{proof}

\begin{lemma}           \label{LemUuxcqY}
    Soit \( K\), un compact de \( \eR\) et \( f_n\) une suite de fonctions sur \( K\) convergeant uniformément vers \( f\). Soit \( g\colon X\to K\) une fonction depuis un espace topologique \( K\). Alors \( f_n\circ g\) converge uniformément vers \( f\circ g\).
\end{lemma}

\begin{proof}
    En effet, pour tout \( x\in X\) nous avons
    \begin{equation}
        \| (f_n\circ g)-(f\circ g) \|_{\infty}=\sup_{x\in X} \| f_n\big( g(x) \big)-f\big( g(x) \big) \|\leq \| f_n-f \|_{\infty}.
    \end{equation}
    Par conséquent, si \( \epsilon\>0\) est donné, il suffit de choisir \( n\) de telle sorte à avoir \( \| f_n-f \|_{\infty}<\epsilon\) et nous avons \( \| (f_n\circ g)-(f\circ g) \|_{\infty}\leq \epsilon\).
\end{proof}

\begin{definition}
    Nous disons qu'une algèbre \( A\) de fonctions sur un espace \( X\) \defe{sépare les points}{sépare!les points} de \( X\) si pour tout \( x_1\neq x_2\) il existe \( g\in A\) telle que \( g(x_1)\neq g(x_2)\).
\end{definition}

Nous pouvons maintenant énoncer et démontrer une forme nettement plus générale du théorème de Stone-Weierstrass. Le théorème \ref{ThoWmAzSMF} le donne pour \( C(X,\eC)\) et le théorème \ref{THOooMDILooGPXbTW} le donne pour \( C(X,\eR)\).

\begin{theorem}[Stone-Weierstrass\cite{MGecheleSW}] \label{THOooMDILooGPXbTW}
    Soient \( X\), un espace compact et Hausdorff. Soit \( A\), une sous-algèbre de \( C(X,\eR)\) contenant une fonction constante non nulle. Alors \( A\) est dense dans \( \Big( C(X,\eR),\| . \|_{\infty}\Big)\) si et seulement si \( A\) sépare les points de \(X\).
\end{theorem}
\index{théorème!Stone-Weierstrass}

\begin{proof}
    Nous allons écrire la démonstration en plusieurs étapes (dont la première est le lemme~\ref{LemYdYLXb}). Nous commençons par la première partie, sur les réels.

    \begin{description}
        \item[Première étape] Pour tout \( x\neq y\in X\) et pour tout \( \alpha,\beta\in \eR\), il existe une fonction \( f\in A\) telle que \( f(x)=\alpha\) et \( f(y)=\beta\).

            En effet, vu que \( A\) sépare les points nous pouvons considérer une fonction \( g\in A\) telle que \( g(x)\neq g(y)\) et ensuite poser
            \begin{equation}
                f(z)=\alpha+\frac{ \alpha-\beta }{ g(y)-g(x) }\big( g(z)-g(x) \big).
            \end{equation}
            Les constantes faisant partie de \( A\), cette fonction \( f\) est encore dans \( A\).

        \item[Seconde étape] Pour tout \( n\)-uples de fonctions \( f_1,\ldots, f_n\) dans \( \bar A\), les fonctions \( \min(f_1,\ldots, f_n)\) et \( \max(f_1,\ldots, f_n)\) sont dans \( \bar A\).

            Nous le démontrons pour \( n=2\); le reste allant évidemment par récurrence. Soient \( f,g\in \bar A\). Étant donné que
            \begin{subequations}
                \begin{align}
                    \max(f,g)&=\frac{ f+g }{2}+\frac{ | f-g | }{2}\\
                    \min(f,g)&=\frac{ f+g }{2}-\frac{ | f-g | }{2},
                \end{align}
            \end{subequations}
            if suffit de montrer que si \( f\in\bar A\) alors \( | f |\in \bar A\). Si \( f\) est nulle, c'est évident; supposons que \( f\neq 0\) et posons \( M=\| f \|_{\infty}\neq 0\). Pour tout \( x\in X\) nous avons
            \begin{equation}
                \frac{ f(x)^2 }{ M^2 }\in \mathopen[ 0 , 1 \mathclose].
            \end{equation}
            Nous considérons alors la suite
            \begin{equation}
                h_n=P_n\circ\frac{ f^2 }{ M^2 }
            \end{equation}
            où \( P_n\) est une suite de polynômes convergent uniformément vers la racine carrée (voir lemme~\ref{LemYdYLXb}). Le lemme~\ref{LemUuxcqY} nous assure que \( h_n\) converge uniformément vers \( \frac{ | f | }{ M }\) dans \( C(X,\eR)\). Étant donné que \( \bar A\) est également une algèbre, \( h_n\) est dans \( \bar A\) pour tout \( n\) et la limite s'y trouve également (pour rappel, la fermeture \( \bar A\) est celle de la topologie de la convergence uniforme).

        \item[Troisième étape] Soit \( \epsilon>0\), \( f\in C(X,\eR)\) et \( x\in X\). Il existe une fonction \( g_x\in \bar A\) telle que
            \begin{subequations}
                \begin{numcases}{}
                    g_x(x)=f(x)\\
                    g_x(y)\leq f(y)+\epsilon
                \end{numcases}
            \end{subequations}
            pour tout \( y\in X\).

            Soit \( z\in X\setminus\{ x \}\) et une fonction \( h_z\) telle que \( h_z(x)=f(x)\) et \( h_z(z)=f(z)\). Une telle fonction existe par une des étapes précédentes. Étant donné que \( f\) et \( h_z\) sont continues, il existe un voisinage ouvert \( V_z\) de \( z\) sur lequel
            \begin{equation}
                h_z(y)\leq f(y)+\epsilon
            \end{equation}
            pour tout \( y\in V_z\). Nous pouvons sélectionner un nombre fini de points \( z_1,\ldots, z_n\) tels que les ouverts \( V_{z_1},\ldots, V_{z_n}\) recouvrent \( X\) (parce que \( X\) est compact, de tout recouvrement par des ouverts, nous extrayons un sous recouvrement fini.). Nous posons
            \begin{equation}
                g_x=\min(h_{z_1},\ldots, h_{z_n})\in \bar A.
            \end{equation}
            Si \( y\in X\), nous sélectionnons le \( i\) tel que \( h_{z_i}(y)\leq f(y)+\epsilon\) et nous avons
            \begin{equation}
                g_x(y)\leq h_{z_i}(y)\leq f(y)+\epsilon.
            \end{equation}

        \item[Étape \wikipedia{fr}{Final_Doom}{finale}] Soit \( \epsilon>0\) et \( f\in C(X,\eR)\). Pour chaque \( x\in X\) nous considérons une fonction \( g_x\in \bar A\) telle que
            \begin{subequations}
                \begin{numcases}{}
                    g_x(x)=f(x)\\
                    g_x(y)\leq f(y)+\epsilon
                \end{numcases}
            \end{subequations}
            pour tout \( y\in X\). Les fonctions \( f\) et \( g_x\) sont continues, donc il existe un voisinage ouvert \( W_x\) de \( x\) sur lequel
            \begin{equation}
                g_x(y)\geq f(y)-\epsilon.
            \end{equation}
            De ces \( W_x\) nous extrayons un sous recouvrement fini de \( X\) : \( W_{x_1},\ldots, W_{x_m}\) et nous posons
            \begin{equation}
                \varphi=\max(g_{x_1},\ldots, g_{x_n})\in \bar A.
            \end{equation}
            Si \( y\in X\), il existe un \( i\) tel que
            \begin{equation}
                \varphi(y)\geq g_{x_i}(y)\geq f(y)-\epsilon.
            \end{equation}
            La première inégalité est le fait que \( \varphi\) est le maximum des \( g_{x_k}\), et la seconde est le choix de \( i\). Donc pour tout \( y\in X\) nous avons
            \begin{equation}        \label{EqJMxHaF}
                f(y)-\epsilon\leq \varphi(y)\leq f(y)+\epsilon.
            \end{equation}
            La première inégalité est ce que l'on vient de faire. La seconde est le fait que pour tout \( i\) nous ayons \( g_{x_i}(y)\leq f(y)+\epsilon\); le fait que \( \varphi\) soit le maximum sur les \( i\) ne change pas l'inégalité.

            Le fait que les inégalités \eqref{EqJMxHaF} soient vraies pour tout \( y\in X\) signifie que \( \| \varphi-f \|_{\infty}\leq \epsilon\), et donc que \( f\in \Adh\big( \Adh(A) \big)=\Adh(A)\).
    \end{description}

    Tout cela prouve que \( C(X,\eR)\subset \Adh(A)\). L'inclusion inverse est le fait que \( C(X,\eR)\) est fermé pour la norme \( \| . \|_{\infty}\), étant donné qu'une limite uniforme de fonctions continues est continue.

    Nous pouvons maintenant nous tourner vers l'énoncé concernant \( C(X,\eC)\).
\end{proof}

\begin{theorem}[Stone-Weierstrass\cite{MonCerveau}] \label{ThoWmAzSMF}
    Soit \( X\), un espace compact et Hausdorff. Soit une sous-algèbre \( A\) stable par conjugaison\footnote{Pour tout \( g\in A\), nous avons \( \bar g\in A\).} \( A\) de \( C(X,\eC)\) contenant une fonction constante non nulle. Alors \( A\) est dense dans \( \Big( C(X,\eC),\| . \|_{\infty}\Big)\) si et seulement si \( A\) sépare les points de \(X\).

    Entendons-nous bien : ici \( A\) et \( C(X,\eC)\) sont des algèbres à coefficients dans \( \eC\).
\end{theorem}

\begin{proof}
    La preuve de cette version dans \( C(X,\eC)\) va bien entendu fortement reposer sur le cas dans \( C(X,\eR)\) que nous venons de prouver. Soit donc \( A\), une sous-algèbre vérifiant les hypothèses.
    \begin{subproof}
        \item[\( \real(A)\subset A\)]
            Nous prouvons que si \( f\in A\), alors \( \real(f)\in A\). En effet, vu que \( A\) est stable par conjugaison, si \( f\in A\), alors \( \bar f\in A\) et \( f+\bar f=2\real(f)\in A\).
    \end{subproof}
    Nous posons
    \begin{equation}
        A_1=\{ \real(g)\tq g\in A \}.
    \end{equation}
    \begin{subproof}
        \item[\( A_1\) est une sous-algèbre de \( A\)]
            Le fait que les élément de \( A_1\) soient dans \( A\) est déjà fait. Pour le produit, si \( g_1,h_1\in A_1\), alors il existe \( h,h\in A\) tels que \( g_1=\real(g)\) et \( h_1=\real(h)\). Nous avons
            \begin{equation}
                (g_1+ig_2)(h_1+ih_2)=g_1h_2-g_2h_2+i(g_1h_2+g_2h_1)\in A.
            \end{equation}
            La partie réelle de cela est dans \( A_1\), donc
            \begin{equation}        \label{EQooYAGUooJVpaEa}
                g_1h_2-g_2h_1\in A_1.
            \end{equation}
            Mais comme \( g_1+ig_2\in A\), nous avons aussi \( g_1-ig_2\in A \) et donc
            \begin{equation}
                (g_1-ig_2)(h_1+ih_2)=g_1h_1+g_2h_2+i(g_1h_2-g_2h_1)\in A.
            \end{equation}
            La partie réelle de cela est dans \( A_1\). Donc
            \begin{equation}
                g_1h_2+g_2h_1\in A_1.
            \end{equation}
            En comparant avec \eqref{EQooYAGUooJVpaEa}, nous avons \( g_1h_1\in A_1\).
        \item[\( A_1\) sépare les points de \( X\)]
            Soient \( x,y\in X\) ainsi que \( f\in A\) séparant les points \( x\) et \( y\), c'est-à-dire
            \begin{equation}
                f(x)\neq f(y).
            \end{equation}
            Supposons \( f_1(x)=f_2(y)\). Vu que \( f\) sépare, si ce ne sont pas les parties réelles, ce sont les parties imaginaires. C'est-à-dire que  \( f_2(x)\neq f_2(y)\). Mais d'autre part, \( if=f_2+if_1\in A\),  donc en réalité \( f_2\in A_1\) également.
    \end{subproof}
    Le partie \( A_1\) dans \( C(X,\eR)\) vérifie les hypothèses de Stone-Weierstrass réel \ref{THOooMDILooGPXbTW}, donc \( A_1\) est dense dans \( C(X,\eR)\). Le même raisonnement montre que \( A_2\) est également dense dans \( C(X,\eR)\)\quext{Il me semble même que \( A_1=A_2\) et qu'il y a un raccourci possible dans cette preuve en exploitant ce fait. Écrivez-moi pour dire ce que vous en pensez.}

    Soit maintenant le vif de la preuve : \( f\in C(X,\eC)\) avec \( f=u+iv\), les fonctions \( u\) et \( v \) étant dans \( C(X,\eR)\). Nous avons des suites \( u_{k}\stackrel{unif}{\longrightarrow}u\) et \( v_k\stackrel{unif}{\longrightarrow}v\) pour des suites \( (u_k) \) et \( (v_k)\) dans \( C(X,\eR)\).

    Par le même genre de raisonnements que nous avons déjà fait, nous nous convainquons que \( u_k+iv_k\in A\) pour chaque \( k\). Nous avons
    \begin{equation}
        \| u_k+iv_k-u-iv \|_{\infty}\leq \| u_k-u \|_{\infty}+\| v_k-v \|_{\infty}
    \end{equation}
    En prenant \( k\) assez grand, les deux termes peuvent être rendus plus petit que \( \epsilon\).
\end{proof}

\begin{corollary}[\cite{MonCerveau}]        \label{CORooNIUJooLDrPSv}
    Soit \( B\), la boule fermée de centre \( 0\) et de rayon \( 1\) dans \( \eR^n\). La partie \( C^{\infty}(B,\eR^n)\) est dense dans \( \big( C(B,B),\| . \|_{\infty} \big)\).
\end{corollary}

\begin{proof}
    Soit \( f \in C(B,B)\) et \( \epsilon>0\). La fonction donnant la composante \( i\) est une fonction \( f_i\in C(B,\eR)\) et il existe donc, par le théorème de Stone-Weierstrass~\ref{ThoWmAzSMF}, une fonction \( g_i\in  C^{\infty}(B,\eR)\) telle que \( \| g_i-f_i \|_{\infty}\leq \epsilon\).

    La fonction \( g\) dont les composantes sont les \( g_i\) ainsi construits vérifie \( \| g-f \|_{\infty}\leq n\epsilon\).
\end{proof}

Attention toutefois que rien n'assure que les fonctions construites par le corolaire~\ref{CORooNIUJooLDrPSv} prennent leurs valeurs dans \( B\).

Le théorème suivant est un des énoncés les plus classiques de Stone-Weierstrass. Il découle évidemment du théorème général~\ref{ThoWmAzSMF} (encore qu'il faut alors bien comprendre qu'il faut traiter la fonction \( x\mapsto \sqrt{x}\) séparément). Il en existe cependant une preuve indépendante.

\begin{theorem}     \label{ThoGddfas}   
    Soit \( f\), une fonction continue de l'intervalle compact \( \mathopen[ a , b \mathclose]\) à valeurs dans \( \eR\). Alors pour tout \( \epsilon>0\), il existe un polynôme \( P\) tel que \( \| P-f \|_{\infty}<\epsilon\).

    Autrement dit, les polynômes sont denses dans \( C\mathopen[ a , b \mathclose]\) pour la norme uniforme.
\end{theorem}
\index{théorème!Stone-Weierstrass}

%+++++++++++++++++++++++++++++++++++++++++++++++++++++++++++++++++++++++++++++++++++++++++++++++++++++++++++++++++++++++++++
\section{Primitive de fonction continue}
%+++++++++++++++++++++++++++++++++++++++++++++++++++++++++++++++++++++++++++++++++++++++++++++++++++++++++++++++++++++++++++

\begin{proposition}[\cite{MQKDooSuEGxk}]    \label{PropQACVooBnHtRJ}
    Soit un intervalle compact \( K\) de \( \eR\) et une suite \( (f_n)\) de fonctions continues sur \( K\) telles que \( f_n\stackrel{unif}{\longrightarrow}f\). Si chacune des fonctions \( f_n\) a une primitive sur \( K\) alors \( f\) également.
\end{proposition}

\begin{proof}
    Soit \( x_0\in K\) et les primitives \( F_n\) choisies\footnote{Les fonctions \( F_n\) étant dérivables sont continues.} pour avoir \( F_n'f_n\) et \( F_n(x_0)=0\). Nous allons voir que \( (F_n)\) est une suite de Cauchy dans \( \big( K,\| . \|_{\infty} \big)\). Soient \( n,m\in \eN\) et \( x\in K\). Nous avons
    \begin{subequations}
        \begin{align}
            \| F_n-F_m \|_{\infty}&\leq \| F_n(x)-F_m(x) \|\\
            &=\| (F_n-F_m)(x) \|\\
            &\leq \| F'_n-F'_m \|_{[x,x_0]}\| x-x_0 \|
        \end{align}
    \end{subequations}
    où nous avons utilisé le théorème des accroissements finis~\ref{ThoNAKKght}. Vu que \( x\in K\) et que \( K\) est borné, \( \| x-x_0 \|\) est majoré par \( \diam(K)\) et
    \begin{subequations}
        \begin{align}
            \| F_n-F_m \|_K\leq \| f_n-f_m \|_K\diam(K).
        \end{align}
    \end{subequations}
    Vu que \( (f_n) \) est de Cauchy, si \( n\) et \( m\) sont assez grands, cela tend vers zéro. La suite \( (F_n)\) converge donc vers une certaine fonction \( F\).

    Le théorème~\ref{ThoSerUnifDerr} nous permet de permuter la limite et la dérivée pour conclure que \( F'=f\) et donc que \( f\) a une primitive sur \( K\).
\end{proof}

\begin{proposition}[\cite{MQKDooSuEGxk}]        \label{PropKKGAooDQYGKg}
    Soit un intervalle ouvert \( I\) de \( \eR\) et une fonction \( f\colon I\to \eR\) qui admet une primitive sur tout compact de \( I\). Alors \( f\) a une primitive sur \( I\).
\end{proposition}
\index{primitive!de fonction continue}

\begin{proof}
    Nous considérons une suite exhaustive\footnote{Voir le lemme~\ref{LemGDeZlOo}.} de compacts \( K_n\) pour \( I\) et \( x_0\in K_0\). Nous considérons aussi \( F_n\) la primitive de \( f\) sur \( K_n\) telle que \( F_n(x_0)=0\) (possible parce que \( x_0\in K_n\) pour tout \( n\)). Les fonctions \( F_n\) sont des restrictions les unes des autres, et nous pouvons définir
    \begin{equation}
        \begin{aligned}
            F\colon I&\to \eR \\
            x&\mapsto F_n(x)\text{ si } x\in K_n.
        \end{aligned}
    \end{equation}
    Nous avons évidemment \( F(x_0)=0\) et nous allons prouver que \( F\) est une primitive de \( f\) sur \( I\). Soit \( x\in I\) vu que \( I\) est ouvert, nous pouvons choisir \( n_0\) tel que \( x\in\Int(K_{n_0})\). Les fonctions \( F\) et \( F_{n_0}\) sont égales sur \( K_n\) et donc sur un ouvert autour de \( x\). Par conséquent \( F\) est dérivable en \( x\) et \( F'(x)=F'_{n_0}(x)=f(x)\).
\end{proof}

\begin{theorem}    \label{ThoEXXyooCLwgQg}
    Soit \( I\) un intervalle ouvert de \( \eR\). Une fonction continue sur \( I\) admet une primitive\footnote{Définition~\ref{DefXVMVooWhsfuI}.} sur \( I\).
\end{theorem}

\begin{proof}
    Sur chaque compact de \( I\), la fonction \( f\) est limite uniforme de polynômes\footnote{Si tu veux te passer de Stone-Weierstrass, tu peux prouver que toute fonction continue sur un compact est limite uniforme de fonctions affines par morceaux, par exemple. Voir \cite{MQKDooSuEGxk}.} (théorème de Stone-Weierstrass~\ref{ThoGddfas}). Donc \( f\) est primitivable sur tout compact de \( I\) (proposition~\ref{PropQACVooBnHtRJ}) et donc sur \( I\) par la proposition~\ref{PropKKGAooDQYGKg}.
\end{proof}

%+++++++++++++++++++++++++++++++++++++++++++++++++++++++++++++++++++++++++++++++++++++++++++++++++++++++++++++++++++++++++++ 
\section{La fonction puissance}
%+++++++++++++++++++++++++++++++++++++++++++++++++++++++++++++++++++++++++++++++++++++++++++++++++++++++++++++++++++++++++++

Si \( x\) et \( y\) sont des réels, définir \( x^y\) n'est pas une mince affaire. Pour l'instant nous savons déjà définir \( x^n\) lorsque \( x\in \eR\) et \( n\in \eN\). Voir la définition \ref{DEFooGVSFooFVLtNo} et le thème \ref{THEMEooBSBLooWcaQnR}.

Pour la suite nous notons
\begin{subequations}
    \begin{numcases}{}
        f_{\alpha}(x)=x^{\alpha}\\
        g_{a}(x)=a^x
    \end{numcases}
\end{subequations}
pour autant que ces fonctions sont définies\footnote{L'objet des pages suivantes est de déterminer pour quelles valeurs de $a$, $\alpha$ et $ x$ nous pouvons trouver des définitions raisonnables pour ces fonctions.}.

%--------------------------------------------------------------------------------------------------------------------------- 
\subsection{Sur les naturels}
%---------------------------------------------------------------------------------------------------------------------------

\begin{definition}      \label{DEFooKEBIooZtPkac}
    La fonction puissance définie sur \( \eN\) s'étend à \( \eZ\) de la façon suivante :
    \begin{equation}
        x^{-n}=\frac{1}{ x^n }
    \end{equation}
    pour \( n\geq 0\). Cela donne donne donc \( x^n\) pour \( x\in \eR\) et \( n\in \eZ\) a l'exception de \( x=0\) lorsque \( n<0\).
\end{definition}

Nous étudions quelques propriétés de cette fonction pour \( n>0\) fixé.

\begin{proposition}     \label{PROPooXQYFooPxoEHE}
    Soit \( n\in \eN\setminus\{ 0 \}\); nous posons \( f_n(x)=x^n\).

    Si \( n\) est pair,
    \begin{equation}
        f_n\colon \mathopen[ 0 , \infty \mathclose[\to \mathopen[ 0 , \infty \mathclose[
    \end{equation}
    est bijective.

    Si \( n\) est impair,
    \begin{equation}
        f_n\colon \eR \to \eR
    \end{equation}
    est bijective.

    Toutes les fonctions \( f_n\) sont continues sur \( \eR\).
\end{proposition}

\begin{proof}
    En plusieurs morceaux, pas spécialement dans l'ordre auquel on s'attend.
    \begin{subproof}
        \item[Continuité]

            Soit \( x\in \eR\). En vertu de~\ref{ThoLimCont} nous allons prouver que \( \lim_{\epsilon\to 0}f_n(x+\epsilon)=f_n(x)\). Pour cela nous utilisons la formule du binôme~\ref{PropBinomFExOiL} avec \( x,h>0\) :
            \begin{equation}
                f_n(x+h)=(x+h)^n=\sum_{k=0}^n{n\choose k}x^{n-k}h^k.
            \end{equation}
            Nous fixons \( x_0\in \eR\). Calcul :
            \begin{subequations}
                \begin{align}
                    | f_n(x_0+h)-f_n(x) |&=| \sum_{k=1}^n{n\choose k}x_0^{n-k}h^k |\\
                    &\leq \sum_{k=1}^n{n\choose k}| x_0 |^{n-k} |h|^k\\
                    &=h\sum_{k=1}^n{n\choose k}| x_0 |^{n-k}| h |^{k-1}\\
                    &\leq h\sum_{k=1}^n{n\choose k}| x_0 |^{n-k}.
                \end{align}
            \end{subequations}
            Justifications :
            \begin{itemize}
                \item
                   Le terme \( k=0\) est égal à \( x^n=f_n(x)\) parce que \( {n\choose 0}=1\).
               \item
                   Dans la somme nous avons majoré \( | h |\) par \( 1\), opération justifiée par le fait que nous ayons dans l'idée de faire \( h\to 0\).
            \end{itemize}
            Nous avons donc
            \begin{equation}
                \lim_{h\to 0} | f_n(x_0+h)-f_n(x) | \leq\lim_{h\to 0}  h\sum_{k=1}^n{n\choose k}| x_0 |^{n-k}=0.
            \end{equation}
            D'où la continuité de \( f_n\) en tout point \( x_0\in \eR\).

        \item[Pour \( n\) pair ou impair, bijection sur les positifs]
            Ceci sera déjà le résultat complet pour les \( n\) pairs, et a moitié du résultat pour les \( n\) impairs.
            \begin{subproof}
                \item[Stricte croissance]
                    Soit \( n\neq 0\) dans \( \eN\). Nous commençons par prouver que \( f_n\) est strictement croissante sur \( \mathopen[ 0 , \infty \mathclose[\). Nous repartons de la formule du binôme, mais cette fois, nous séparons les termes \( k=0\) et \( k=n\) des autres (si \( n=1\), il y a un peu de réécriture) en tenant compte de \( {n\choose 0}={n\choose n}=1\) :
                        \begin{equation}
                            f_n(x+h)=x^n+h^n+\sum_{k=1}^{n-1}{n\choose k}x^{n-k}h^k>x^n=f_n(x).
                        \end{equation}
                        Vous noterez que l'inégalité est stricte même si \( n=1\).

                        Vu que nous avons stricte monotonie, le théorème~\ref{ThoKBRooQKXThd}\ref{ITEMooMAWXooZXmVwA} nous dit que
                        \begin{equation}
                            f_n\colon \mathopen[ 0 , \infty \mathclose[\to f_n\big( \mathopen[ 0 , \infty \mathclose[ \big)
                        \end{equation}
                        est une bijection.
                    \item[Bijection]

                        Nous prouvons que \( f_n\big( \mathopen[ 0 , \infty \mathclose[ \big)=\mathopen[ 0 , \infty \mathclose[\). Si \( x>0\) alors \( f_n(x)>0\), cela prouve une inclusion.

                            Pour l'autre inclusion nous savons que \( f_n(x)>x\) dès que \( x>1\). Donc \( \lim_{x\to \infty} f_n(x)=\infty\). Si \( y\in \mathopen[ 0 , \infty \mathclose[\), alors il existe \( x_0\) tel que \( f_n(x_0)>y\). Étant donné que \( f_n(0)=0\) et que nous avons déjà prouvé que \( f_n\) était continue (proposition~\ref{PROPooXQYFooPxoEHE}), le théorème des valeurs intermédiaires~\ref{ThoValInter} nous indique l'existence de \( x_1\in \mathopen[ 0 , x_0 \mathclose[\) tel que \( f_n(x_1)=y\).

            \end{subproof}

            Nous avons prouvé que pour tout \( n\), la fonction
            \begin{equation}        \label{EQooYWHGooJWMTUI}
                f_n\colon \mathopen[ 0 , \infty \mathclose[\to \mathopen[ 0 , \infty \mathclose[
            \end{equation}
            est une bijection.

        \item[Pour \( n\) impair]

            Nous montrons à présent que si \( n\) est impair, alors
            \begin{equation}        \label{EQooTSLJooMAAUXH}
                f_n\colon \mathopen] -\infty , 0 \mathclose]\to \mathopen] -\infty , 0 \mathclose]
            \end{equation}
            est une bijection.

            Tout se base sur le fait que si \( x>0\) alors \( f_n(-x)=-f_n(x)\). Le fait que \eqref{EQooYWHGooJWMTUI} soit injective et surjective montre alors tout de suite le fait que \eqref{EQooTSLJooMAAUXH} soit également injective et surjective.
    \end{subproof}
\end{proof}

Vous noterez que la continuité de \( f_n\) démontrée dans la proposition \ref{PROPooXQYFooPxoEHE} est indépendant de la proposition \ref{LEMooUAFBooAwiXxj} qui sera invoquée plus tard pour définir \( a^x\) lorsque \( a>0\) dans \( \eR\).

%--------------------------------------------------------------------------------------------------------------------------- 
\subsection{Sur les rationnels, racines}
%---------------------------------------------------------------------------------------------------------------------------

L'existence, pour tout réel \( a\geq 0\), d'une réel \( r\) tel que \( r^2=a\) est déjà faite en la proposition \ref{PROPooUHKFooVKmpte}.

\begin{definition}[Exposant rationnels]        \label{DEFooJWQLooWkOBxQ}
    La proposition \ref{PROPooXQYFooPxoEHE} nous dit entre autres que pour tout \( n\in \eN\), la fonction
    \begin{equation}
        \begin{aligned}
            f_n\colon \mathopen[ 0 , \infty \mathclose[&\to \mathopen[ 0 , \infty \mathclose[ \\
            x&\mapsto x^n 
        \end{aligned}
    \end{equation}
    est bijective. Nous définissions alors, pour \( a\in \mathopen[ 0 , \infty \mathclose[\),
    \begin{equation}
        a^{1/n}=f_n^{-1}(a).
    \end{equation}
    Autrement dit, le nombre \( a^{1/n}\) est l'unique solution positive de
    \begin{equation}
        x^n=a.
    \end{equation}
\end{definition}

\begin{normaltext}      \label{NORMooDUNZooUNdUKg}
    Nous ne définissons pas \( a^{1/n}\) pour \( a<0\), du moins pas encore. Vu que \( f_3\) est bijective sur \( \eR\), il serait tentant de définir \( (-1)^{1/3}=f_3^{-1}(-1)=-1\).

    Cela causera un certain nombre de problèmes plus tard vu que nous aurons envie de deux choses en même temps :
    \begin{itemize}
        \item d'une part \( \ln(-1)=i\pi\),
        \item d'autre part, \( a^x= e^{x\ln(a)}\).
    \end{itemize}
    De cette façon, nous devrions avoir
    \begin{equation}
        (-1)^{1/3}= e^{i\pi /3},
    \end{equation}
    qui est un nombre complexe non réel. Voici un exemple de ce que ça donne avec Sage :
    \lstinputlisting{tex/sage/sageSnip019.sage}
\end{normaltext}

\begin{definition}[Racince]     \label{DEFooPOELooPouwtD}
    Pour \( n\in \eN\) nous définissons \( \sqrt[n]{ x }=f_n^{-1}(x)\). Lorsque \( n\) est pair, la fonction \( x\mapsto\sqrt[n]{ x }\) n'est définie que sur \( \eR^+\), et lorsque \( n\) est impair, elle est définie sur tout \( \eR\).
\end{definition}

\begin{normaltext}      \label{NORMooYPRNooWCjEgR}
    Notons que les fonctions \( x\mapsto \sqrt[3]{ x }\) et \( x\mapsto x^{1/3}\) ne sont pas les mêmes : la première est définie sur tout \( \eR\) et donne des valeurs réelles tandis que la seconde n'est (pour l'instant) définie que sur les positifs, et donnera (quand on l'aura définie par l'exponentielle) des nombres complexes sur les négatifs.

    En suivant cette convention, c'est-à-dire en réservant la notation \( \sqrt{  }\) pour l'inverse de \( f_2\), nous ne devrions pas écrire des choses comme «\( \sqrt{ -1 }=i\)», mais plutôt «\( (-1)^{1/2}=i \)». En effet, \( \sqrt{ -1 }\) n'est pas défini et ne sera jamais défini alors que \( (-1)^{1/2}\) n'est pas encore défini, mais sera défini par 
    \begin{equation}
        (-1)^{1/2}= e^{\frac{ 1 }{2}\ln(-1)}= e^{i\pi/2}=i.
    \end{equation}
\end{normaltext}

En résumé, nous avons les fonctions suivantes :
\begin{enumerate}
    \item
        \( \sqrt[n]{  }\colon \eR\to \eR\) si \( n\) est impair,
    \item
        \( \sqrt[n]{  }\colon \mathopen[ 0 , \infty \mathclose[\to \mathopen[ 0 , \infty \mathclose[ \) si \( n\) est pair,
    \item
        \( x^{1/n}\colon \mathopen[ 0 , \infty \mathclose[\to \mathopen[ 0 , \infty \mathclose[\) pour tout \( n\in \eN\).
\end{enumerate}
Cependant nous n'hésiterons pas à utiliser la notation \( \sqrt{ x }\) pour \( x^{1/2}\) même lorsque \( x\) est négatif, parce c'est une notation très pratique. Il faut garder en tête que cette façon de faire est incohérente parce qu'elle inciterait à penser que \( \sqrt[3]{-1  }= e^{i\pi/3}\) au lieu de \( \sqrt[3]{-1  }=-1\).

Pour toute la suite de cette section, nous allons considérer \( a^x\) uniquement pour \( a>0\).

\begin{definition}
    Pour \( m,n\in \eN\) nous définissons 
    \begin{equation}        \label{EQooZFOAooTsMbub}
        a^{m/n}=(a^m)^{1/n},
    \end{equation}
    ce qui définit la fonction puissance sur \( \eQ^+\). Enfin nous posons
    \begin{equation}        \label{DEFooTUCVooXikxRh}
        a^{-q}=\frac{1}{ a^q }
    \end{equation}
    lorsque \( q\in \eQ^+\).

    Et avec tout ça, lorsque \( a>0\) nous avons défini \( a^q\) pour tout \( q\in \eQ\).
\end{definition}

Nous allons souvent noter la définition \eqref{EQooZFOAooTsMbub} sous la forme
\begin{equation}        \label{EQooZIKKooVfjkZo}
    f_{m/n}(x)^n=x^m.
\end{equation}

\begin{lemma}[\cite{MonCerveau}]        \label{LEMooIDLJooZALNaD}
    Pour \( a>0\) et \( p,q\in \eZ\) nous avons :
    \begin{equation}
        a^{p/q}=(a^p)^{1/q}=(a^{1/q})^p.
    \end{equation}
\end{lemma}

\begin{proof}
    Nous divisons la preuve en fonction de la positivité du numérateur et du dénominateur.
    \begin{subproof}
        \item[Numérateur et dénominateurs positifs]
                
            Nous commençons avec \( p,q\in \eN\). La première égalité est la définition \ref{DEFooJWQLooWkOBxQ}. Pour la seconde, la définition de \( (a^p)^{1/q}\) est d'être le \( x>0\) tel que
            \begin{equation}
                x^q=a^p.
            \end{equation}
            La définition de \( a^{1/q}\) est d'être le \( y>0\) tel que
            \begin{equation}
                y^q=a.
            \end{equation}
            Ce \( y\) vérifie donc aussi \( y^{pq}=a^p\) et donc \( (y^p)^q=a^p\). Autrement dit, \( y^p=x\), c'est-à-dire exactement
            \begin{equation}
                (a^{1/q})^p=(a^p)^{1/q}.
            \end{equation}
            Le lemme est prouvé dans le cas où \( p,q\in \eN\).

        \item[Numérateur et dénominateur négatifs]

            Si \( p\) et \( q\) sont tous les deux négatifs, nous remarquons que \( p/q=(-p)/(-q)\) et nous sommes dans le même cas qu'avant.

        \item[Numérateur négatif, dénominateur positif]

            Pour simplifier les notations nous supposons toujours \( p,q\in \eN\) mais nous considérons \( a^{(-p)/q}\). Nous avons d'une part :
            \begin{equation}
                a^{(-p)/q}=a^{-(p/q)}=\frac{1}{ a^{p/q} }=\frac{1}{ (a^{1/q})^p }=(a^{1/q})^{-p}.   
            \end{equation}
            Dans ce calcul, nous avons utilisé au dénominateur le résultat dans le cas positif. 

            Et d'autre part nous avons :
            \begin{equation}
                (a^{-p})^{1/q}=\left( \left( \frac{1}{ a } \right)^p \right)^{1/q}=\left( \left( \frac{1}{ a } \right)^{1/q} \right)^p=\left( \frac{1}{ a^{1/q} } \right)^p=\frac{1}{ (a^{1/q})^p }=(a^{1/q})^{-p}
            \end{equation}
            où nous avons utilisé le résultat avec \( 1/a\) en guise de \( a\).

        \item[Numérateur positif, dénominateur négatif]

            Nous traitons maintenant \( a^{p/(-q)}\). Nous avons d'une part
            \begin{equation}
                a^{p/(-q)}=a^{-(p/q)}=\frac{1}{ a^{p/q} }=\frac{1}{ (a^p)^{1/q} }=(a^p)^{-(1/q)}=(a^p)^{1/(-q)}.
            \end{equation}
            Et d'autre part :
            \begin{equation}
                a^{p/(-q)}=\frac{1}{ a^{p/q} }=\frac{1}{ (a^{1/q})^p }=\left( \frac{1}{ a^{1/q} } \right)^p=\left( a^{-(1/q)} \right)^p=(a^{1/(-q)})^p.
            \end{equation}
    \end{subproof}
\end{proof}

Le lemme suivant montre que la définition sur \( \eQ^-\) est cohérente avec celle sur \( \eQ^+\), au sens où finalement nous retrouvons que \( a^{m/n}\) vérifie \( x^n=a^m \) quel que soient les signes de \( m\) et \( n\).
\begin{lemma}[\cite{MonCerveau}]
    Le nombre \( y=a^{-m/n}\) vérifie l'équation \( y^{-n}=a^m\)
\end{lemma}

\begin{proof}
    Nous posons \( x=a^{m/n}\), c'est-à-dire \( x^n=a^m\). Nous avons, par définition \( y=a^{-m/n}=\frac{1}{ x }\). Alors
    \begin{equation}
        y^{-n}=\frac{1}{ \left( \frac{1}{ x } \right)^n }=x^n=a^m,
    \end{equation}
    donc c'est bon.
\end{proof}

\begin{lemma}[\cite{MonCerveau}]        \label{LEMooJYGUooHhLASp}
    Pour \( a>0\) et \( q,q'\in \eQ\) nous avons
    \begin{equation}
        a^qa^{q'}=a^{q+q'}.
    \end{equation}
\end{lemma}

\begin{proof}
    Nous mettons \( q\) et \( q'\) au même dénominateur. Soient \( q=s/c\) et \( q'=r/c\) avec \( s,r\in \eZ\) et \( c\in \eN\). En utilisant les égalités du lemme \ref{LEMooIDLJooZALNaD} nous trouvons
    \begin{equation}
        a^{s/c}a^{r/c}=(a^{1/c})^s(a^{1/c})^r=(a^{1/c})^{s+r}=a^{(s+r)/c}=a^{q+q'}.
    \end{equation}
\end{proof}

\begin{lemma}[\cite{MonCerveau}]        \label{LEMooXJXUooLoiTMo}
    La fonction puissance prend les valeurs suivantes.
    \begin{enumerate}
        \item
            Si \( a=1\) alors \( a^q=1\) pour tout \( q\in \eQ\).
        \item       \label{ITEMooKZCGooKskUQx}
            Si \( a>1\) alors 
            \begin{itemize}
                \item \( a^q>1\) si \( q>0\)
                \item \( a^q<1\) si \( q<0\)
                \item \( a^0=1\).
            \end{itemize}
        \item
            Si \( a<1\) alors 
            \begin{itemize}
                \item \( a^q<1\) si \( q>0\)
                \item \( a^q>1\) si \( q<0\)
                \item \( a^0=1\).
            \end{itemize}
    \end{enumerate}
\end{lemma}

\begin{proof}
    Si \( a=1\) alors \( a^k=1\) pour tout \( k\in \eN\). Ensuite, pour \( m,n\in \eN\), \( a^{n/m}\) est solution de \( x^m=a^n=1\), donc \( x=1\). En ce qui concerne les puissances négatives, \( 1/1=1\).

    Si \( a>1\) alors \( a^k>1\) pour tout \( k\in \eN\). De plus pour \( q>0\) nous avons \( q=m/n\) avec \( m,n\in \eN\). Alors \( a^{m/n}\) est solution de \( x^m=a^n>1\). Or pour \( x\leq 1\) nous avons \( x^m\leq 1\), donc la solution à \( x^m=a^n\) vérifie forcément \( x>1\).

    Toujours avec \( a>1\), si \( q<0\) nous posons \( q=-q'\) avec \( q'>0\). Alors
    \begin{equation}
        a^q=q^{-q'}=\frac{1}{ a^{q'} }.
    \end{equation}
    Mais \( a^{q'}>1\), donc l'inverse est inférieur à \( 1\).

    En ce qui concerne les cas \( a<1\), ils sont obtenus en posant \( b=1/a\) et en calculant
    \begin{equation}
        a^q=\left( \frac{1}{ b } \right)^q=\frac{1}{ b^q }=b^{-q}.
    \end{equation}
\end{proof}

\begin{proposition}[\cite{MonCerveau}]\label{PROPooVXKBooQPPjMn}
    Si \( a>1\) et si \( M>0\), il existe \( n\in \eN\) tel que \( a^n>M\).
\end{proposition}

\begin{proof}
    Soit \( a=1+h\). Alors en utilisant la formule du binôme, 
    \begin{equation}
        a^n=(1+h)^n=\sum_{k=0}^n{n\choose k}h^{n-k}.
    \end{equation}
    Tous les termes de la somme sont strictement positifs. Prenons le terme \( k=n-1\). Il vaut
    \begin{equation}
        {n\choose n-1}h=nh.
    \end{equation}
    Donc \( a^n\geq nh\), donc oui, cela peut être rendu arbitrairement grand avec \( n\) sans toucher à \( a\).
\end{proof}

\begin{proposition}[\cite{MonCerveau}]      \label{PROPooGCBZooTcyGtO}
    Pour \( a>0\) nous considérons la fonction
    \begin{equation}
        \begin{aligned}
            g_a\colon \eQ&\to \eR \\
            q&\mapsto a^q.
        \end{aligned}
    \end{equation}
    \begin{enumerate}
        \item
            Si \( a\in \mathopen] 0 , 1 \mathclose[\) alors \( g_a\) est décroissante et
            \begin{subequations}
                \begin{align}
                    \lim_{q\to \infty} g_a(q)=0,&& \lim_{q\to -\infty} g_a(q)=\infty.
                \end{align}
           \end{subequations}
       \item      \label{ITEMooGOEVooKVoVpZ}
            Si \( a>1\)  alors \( g_a\) est croissante et
            \begin{subequations}
                \begin{align}
                    \lim_{q\to \infty} g_a(q)=\infty,&& \lim_{q\to -\infty} g_a(q)=0.
                \end{align}
           \end{subequations}
    \end{enumerate}
\end{proposition}

\begin{proof}
    Nous prouvons le cas \( a>1\). L'autre cas s'en déduit en posant \( b=1/a\). Pour la croissance, soient \( q\in \eQ\) et \( r>0\) dans \( \eQ\). En utilisant le lemme \ref{LEMooJYGUooHhLASp}, nous avons
    \begin{equation}
        a^{q+r}=a^qa^r>a^q
    \end{equation}
    parce que \( a^r>1\) par le lemme \ref{LEMooXJXUooLoiTMo}. 

    En ce qui concerne la limite \( q\to \infty\), la fonction \( g_a\) est croissante et non bornée par la proposition \ref{PROPooVXKBooQPPjMn}. Donc sa limite est \( \infty\).

    Pour la limite \( q\to -\infty\), nous avons
    \begin{equation}
        \lim_{q\to -\infty} a^q=\lim_{q\to \infty} a^{-q}=\lim_{q\to \infty} \frac{1}{ a^q }=0.
    \end{equation}
\end{proof}

\begin{proposition}[\cite{MonCerveau}]      \label{PROPooIIDGooTRtlUD}
    Soit \( a>0\). Nous avons
    \begin{equation}
        \lim_{q\to 0} a^q=1.
    \end{equation}
    Notons que cette limite est une limite dans \( \eQ\) parce que nous n'avons même pas encore défini \( a^x\) lorsque \( x\) est irrationnel.
\end{proposition}

\begin{proof}
    Nous notons, comme à l'accoutumée, \( g_a(x)=a^x\). Soit une suite \( x_k\to 0\) (avec \( x_k\neq 0\) pour tout \( k\)). En définissant \( y_k\) par \( x_k=1/y_k\) nous savons que \( a^{1/y_k}\) est la solution de \( x^{y_k}=a\).

    Nous posons \( t_k=a^{x_k}\) et notre but est de prouver que \( t_k\to 1\). Pour tout \( k\) nous avons la relation
    \begin{equation}
        t_k^{y_k}=a.
    \end{equation}
    Soit \( s>1\). Il existe un \( M>0\) tel que \( y_k>M\) implique \( s^{y_k}>a\) (proposition \ref{PROPooVXKBooQPPjMn}). Donc dès que \( y_k>M\) nous avons \( t_k<s\).

    De la même manière, si \( r<1\), il existe un \( R>0\) tel que \( y_k>R\) implique \( r^{y_k}<a\). Donc dès que \( y_k>R\) nous avons \( t_k>r\).

    Soit donc un voisinage \( \mathopen] r , s \mathclose[\) de \( 1\) (avec \( r<1\) et \( s>1\)). Nous avons les nombres \( M\) et \( R\) correspondant et nous posons \( L=\max\{ M,R \}\). Soit \( K\) tel que \( k>K\) implique \( y_k>L\). Alors pour \( k>K\) nous avons aussi \( t_k<s\) et \( t_k>r\), c'est-à-dire \( t_k\in \mathopen] r , s \mathclose[\).

        Cela prouve que \( t_k\to 1\).

        Donc pour toute suite \( x_k\to 0\) nous avons \( g_a(x_k)\to 1\). Par le critère séquentiel de la limite (proposition \ref{PROPooJYOOooZWocoq}) nous avons \( \lim_{x\to 0} g_a(x)=1\).
\end{proof}

\begin{lemma}       \label{LEMooKDBPooLQwxMD}
    Soit \( a>0\). La fonction
    \begin{equation}
        \begin{aligned}
            g_a\colon \eQ&\to \eR \\
            x&\mapsto a^x 
        \end{aligned}
    \end{equation}
    est continue.
\end{lemma}

\begin{proof}
    Soient \( x\in \eQ\) et une suite \( x_k\to 0\) (toujours dans \( \eQ\)) et utilisons le lemme \ref{LEMooJYGUooHhLASp} :
    \begin{equation}
        a^{x+x_k}=a^xa^{x_k}.
    \end{equation}
    Cela est, dans \( \eR\), le produit entre une constante (\( a^x\)) et une suite. La limite est donc le produit de cette constante et la limite de la suite (si elle existe). Par la proposition \ref{PROPooIIDGooTRtlUD} nous avons la limite \( a^{x_k}\to 1\), et donc
    \begin{equation}
        \lim_{k\to \infty} a^{x+x_k}=a^x,
    \end{equation}
    ce qui prouve la continuité (caractérisation séquentielle, proposition \ref{PropFnContParSuite}) de \( g_a\).
\end{proof}

\begin{proposition}     \label{PROPooQRFSooVzYdJM}
    Soit \( a>0\) dans \( \eR\). La fonction
    \begin{equation}
        \begin{aligned}
            g_a\colon \eQ&\to \eR \\
            q&\mapsto a^q 
        \end{aligned}
    \end{equation}
    est Cauchy-continue.
\end{proposition}

\begin{proof}
    En quelque étapes.
    \begin{subproof}
        \item[Pour \( a>1\)]
            Avant de nous lancer dans la preuve directe, nous prouvons une petite formule. Soit \( \epsilon>0\). Vu que, par la proposition \ref{PROPooIIDGooTRtlUD}, \( \lim_{q\to 0} g_a(q)=1\), il existe \( \delta>0\) tel que \( 0<q<\delta\) implique \( | 1-g_a(q) |<\epsilon\).

            Soient maintenant \( p,q\in \eQ\) tels que \( | p-q |<\delta\). En utilisant de plus la définition \eqref{DEFooTUCVooXikxRh} et la formule du lemme \ref{LEMooJYGUooHhLASp},
            \begin{subequations}
                \begin{align}
                    | g_a(q)-g_a(p) |=| g_a(q) |\left| 1-\frac{ g_a(p) }{ g_a(q) } \right| =| g_a(p) | | 1-g_a(p-q) |\leq | g_a(q) |\epsilon.
                \end{align}
            \end{subequations}
            
            Nous y allons pour la preuve directe. Soit une suite de Cauchy \( (q_n)\) dans \( \eQ\). Nous devons prouver que la suite \( n\mapsto g_a(q_n)\) est de Cauchy dans \( \eR\). Soit \( \epsilon>0\).

            La suite \((q_n)\) étant de Cauchy dans \( \eQ\), elle l'est également dans \( \eR\), elle est bornée parce que convergente vu que \( \eR\) est complet\footnote{Théorème \ref{THOooNULFooYUqQYo}.}. Vu que \( g_a\) est croissante\footnote{Proposition \ref{PROPooGCBZooTcyGtO}\ref{ITEMooGOEVooKVoVpZ}.} et que \( (q_n)\) est bornée, il existe \( M\) tel que \( | g_a(q_n) |\leq M\) pour tout \( n\).

            Nous considérons \( \delta\) tel que \( 0<q<\delta\) implique \( | 1-g_a(q) |\leq \epsilon\), ainsi que \( N\) tel que \( i,j>N\) implique \( | q_i-q_j |\leq \delta\) (là nous utilisons le fait que \( (q_n)\) est de Cauchy). Pour de tels \( N, i,j\) nous avons
            \begin{equation}
                | g_a(q_i)-g_a(q_j) |\leq M\epsilon.
            \end{equation}
            Donc la suite \( g_a(g_n)\) est de Cauchy.

        \item[Pour \( a=1\)]
            La fonction \( g_a\) est constante.
        \item[Pour \( 0\leq a<1\)]
            J'imagine que ça se fait comme \( a>1\), mais en renversant quelque inégalités\quext{Je n'ai pas essayé. Faites-le et écrivez-moin pour me dire si ça marche.}.
    \end{subproof}
\end{proof}

\begin{normaltext}
    L'ingrédient magique qui fait fonctionner la proposition \ref{PROPooQRFSooVzYdJM} est le fait que \( g_a(x+y)=g_a(x)g_a(y)\) couplé au fait que \( \lim_{q\to 0} g_a(q)=1\). 
    C'est cela qui débloque la situation pour étendre la fonction puissance de \( \eQ\) vers \( \eR\) en utilisant le lemme \ref{LEMooUAFBooAwiXxj}. 

    Le chemin suivit par \cite{BIBooXUZHooOHWxiF} pour étendre la fonction puissance de \( \eQ\) vers \( \eR\) est un peu différent : il définit \( a^x=\sup\{ a^q\tq q<x,q\in \eQ \}\). La preuve que cette définition donne \( x\mapsto a^x\) continue sur \( \eR\) repose, elle aussi, essentiellement sur le fait que \( \lim_{q\to 0} a^q=1\).

    Il y a donc une certaine justice.
\end{normaltext}

\begin{propositionDef}[Fonction puissance\cite{MonCerveau}]  \label{DEFooOJMKooJgcCtq}
    Si \( a>0\) et \( x\in \eR\), la fonction
    \begin{equation}
        \begin{aligned}
            g_a\colon \eQ&\to \eR \\
            x&\mapsto a^x. 
        \end{aligned}
    \end{equation}
    est Cauchy-continue par la proposition \ref{PROPooQRFSooVzYdJM}. Si \( x\in \eR\) nous définissons
    \begin{equation}
        a^x=\tilde g_a(x)
    \end{equation}
    où \( \tilde g_a\) est l'extension de \( g_a\) donnée par le lemme \ref{LEMooUAFBooAwiXxj}.

    Nous allons la noter \( g_a\) également, et écrire \( a^x\) la valeur de \( g_a\) même lorsque \( x\) n'est pas un rationnel.
\end{propositionDef}

\begin{proposition}[\cite{MonCerveau}]      \label{PROPooVADRooLCLOzP}
    Quelques propriétés de la fonction puissance. 
    \begin{enumerate}
        \item       \label{ITEMooQHYRooJIewyp}
            Pour \( a>0\), la fonction \( g_a\colon x\mapsto a^x\) est continue sur \( \eR\).
        \item       \label{ITEMooIZBLooSGtWIp}
            Pour \( a>1\), la fonction \( g_a\colon x\mapsto a^x\) est croissante.
        \item       \label{ITEMooSCJBooNVJZah}
            Pour \( a>0\) et \( x,y\in \eR\) nous avons
            \begin{equation}        \label{EQooEWIHooDRAQGR}
                a^xa^y=a^{x+y}.
            \end{equation}
            En particulier, 
            \begin{equation}
                a^{-x}=\frac{1}{ a^x }.
            \end{equation}
    \end{enumerate}
\end{proposition}

\begin{proof}
    La continuité de \( x\mapsto a^x\) est par construction. Le point \ref{ITEMooQHYRooJIewyp} est fait.

    Pour le point \ref{ITEMooIZBLooSGtWIp}, lorsque \( a>1\), la fonction \( f_a\colon \eQ\to \eR\) est croissante (proposition \ref{PROPooGCBZooTcyGtO}). Donc par la proposition \ref{PROPooTNIAooNAJDzL}, la fonction \( x\mapsto a^x\) est croissante sur \( \eR\).

    Et enfin pour le point \ref{ITEMooSCJBooNVJZah}, il faut faire un peu plus attention. Soient des suites \( x_i\to x\) et \( y_i\to y\) dans \( \eQ\). Calculons :
    \begin{subequations}        \label{SUBEQSooMPNLooPoyjwJ}
        \begin{align}
            a^xa^y&=(\lim_ia^{x_i})a^y      \label{SUBEQooOCIOooZcewMo} \\
            &=\lim_i\big( a^{x_i}a^y \big)   \label{SUBEQooEKQXooPLqzcG}\\
            &=\lim_i\big( \lim_ka^{x_i}a^{y_k} \big)    \label{SUBEQooZEXDooRytDvS}\\
            &=\lim_i\big( \lim_k a^{x_i+y_k} \big)     \label{SUBEQooSYNBooIQZJzl}\\
            &=\lim_ia^{x_i+y}                           \label{SUBEQooKHKCooGwaPDQ}\\
            &=a^{x+y}.                                  \label{SUBEQooMZBFooSoSgKU}
        \end{align}
    \end{subequations}
    Justifications :
    \begin{itemize}
        \item Pour \ref{SUBEQooOCIOooZcewMo}. Définition de \( a^x\) lorsque \( x\in \eR\).
        \item Pour \ref{SUBEQooEKQXooPLqzcG}. Nous entrons le nombre \( a^y\) dans la limite. Entrer un facteur dans une limite convergente dans \( \eR\) est un acte anodin.
        \item Pour \ref{SUBEQooZEXDooRytDvS}. Définition de \( a^y\), et renter le nombre réel \( a^{x_i}\) dans la limite sur \( k\).
        \item Pour \ref{SUBEQooSYNBooIQZJzl}. Utilisation du lemme \ref{LEMooJYGUooHhLASp}, valable pour \( x_i,y_k\in \eQ\).
        \item Pour \ref{SUBEQooKHKCooGwaPDQ}. Pour \( i\) fixé, la suite \( k\mapsto x_i+x_k\) est une suite de rationnels qui converge vers le réel \( x_i+y\). Par définition \ref{DEFooOJMKooJgcCtq} de la fonction puissance nous avons alors \( \lim_ka^{x_i+y_k}=a^{x_i+y}\).
        \item Pour \ref{SUBEQooMZBFooSoSgKU}. La suite de réels \( i\mapsto x_i+y\) converge dans \( \eR\) vers le réel \( x+y\). Par la continuité de \( t\mapsto a^t\) (ça fait partie du lemme \ref{LEMooUAFBooAwiXxj} définissant la fonction puissance sur \( \eR\)) nous avons \( \lim_ia^{x_i+y}=a^{x+y}\).
    \end{itemize}

    Vous remarquerez que les limites sur \( k\) et sur \( i\) ne s'enlèvent pas tout à fait avec la même justification. Nous aurions pu invoquer la continuité sur \( \eR\) de \( t\mapsto a^t\) pour les deux limites. Mais cette continuité, dans le cas d'une suite purement constituée de rationnels, est la définition de la prolongation vers \( \eR\).
\end{proof}

\begin{lemma}       \label{LEMooIPLLooCgpCIn}
    Soient \( a,b>0\). Si \( 1<x<y\) alors
    \begin{equation}
        a-b<ay-bx.
    \end{equation}
\end{lemma}

\begin{proof}
    Nous posons \( y=x+s\) avec \( s>0\). Alors
    \begin{equation}
        ay-bx=a(x+s)-bx=(a-b)x+as>(a-b)x>a-b
    \end{equation}
    parce que \( as>0\) et \( x>1\).
\end{proof}

\begin{proposition}[\cite{MonCerveau}]      \label{PROPooJXHFooCDwxCS}
    Pour \( q>0\) dans \( \eQ\), la fonction
    \begin{equation}
        \begin{aligned}
            f_{q}\colon \eQ^+&\to \eR \\
                x&\mapsto x^{q} 
        \end{aligned}
    \end{equation}
    est strictement croissante.
\end{proposition}

\begin{proof}
    Division selon la généralité de \(q\).
    \begin{subproof}
        \item[Si \( q\) est entier positif]
            Soit \( q=n\in \eN\). Si \( s>0\) alors l'inégalité \( (x+s)^n>x^n\) découle du binôme de Newton de la proposition \ref{PropBinomFExOiL}.
        \item[Si \( q\) est rationnel]
            Soient un rationnel \( q=m/n\) et un nombre strictement positif \( s\). Nous avons, par la définition \ref{DEFooJWQLooWkOBxQ} sous la forme \eqref{EQooZIKKooVfjkZo} :
            \begin{equation}
                f_{m/n}(x+s)^n=(x+s)^m>x^m=f_{m/n}(x)^n.
            \end{equation}
            Nous avons utilisé la stricte croissance de \( x\mapsto x^m\). Cela donne
            \begin{equation}
                f_{m/n}(x+s)^n>f_{m/n}(x)^n.
            \end{equation}
            En utilisant encore la stricte croissance de \( x\mapsto x^n\), nous avons le résultat.
    \end{subproof}
\end{proof}

\begin{corollary}       \label{CORooYWNNooLwKmiD}
    Soient \( 1<b<a\) dans \( \eR\) et des rationnels strictement positifs \( p<q\). Alors
    \begin{equation}
        a^p-b^p<a^q-b^q
    \end{equation}
\end{corollary}

\begin{proof}
    Nous notons \( q=p+r\) avec \( r>0\) dans \( \eQ\). Par la proposition \ref{PROPooJXHFooCDwxCS},
    \begin{equation}
        a^r>b^r.
    \end{equation}
    Cela nous permet d'utilise le lemme \ref{LEMooIPLLooCgpCIn} pour écrire
    \begin{equation}
        a^p-b^p<a^pa^r-b^pb^r=a^q-b^q.
    \end{equation}
\end{proof}

\begin{proposition}[\cite{MonCerveau}]      \label{PROPooKWRGooMTbRdU}
    Soient \( a,b>0\) et \( \alpha\in \eR\). Nous avons :
    \begin{equation}
        a^{\alpha}b^{\alpha}=(ab)^{\alpha}.
    \end{equation}
\end{proposition}

\begin{proof}
    Nous supposons que c'est bon pour \( \alpha\in \eN\) et \( \alpha\in \eZ\). Pour les autres, nous donnons plus de détails.
    \begin{subproof}
        \item[\( \eQ^+\)]
            Soit \( q=m/n\) avec \( m,n\in \eN\). Si \( a^{m/n}=x\) et \( b^{m/n}=y\), alors
            \begin{subequations}
                \begin{align}
                    x^n&=a^m    \label{EQooGNMAooQJMNsL}\\
                    y^n&=b^m
                \end{align}
            \end{subequations}
            par \eqref{EQooZIKKooVfjkZo}. Nous multiplions \eqref{EQooGNMAooQJMNsL} par \( y^n\) à gauche et par \( b^m\) à droite : \( x^ny^n=a^mb^m\). En tenant compte du résultat pour \( \eN\), nous avons
            \begin{equation}
                (xy)^n=(ab)^m,
            \end{equation}
            ce qui signifie que le nombre \( xy\) est \( (ab)^{m/n}\).
        \item[Pour \( \eQ^-\)]
            Soit \( q\in \eQ^+ \), nous avons le calcul
            \begin{equation}
                a^{-q}b^{-q}=\frac{1}{ a^qb^q }=\frac{1}{ (ab)^q }=(ab)^{-q}.
            \end{equation}
        \item[Pour \( \eR\)]
            Soit une suite de rationnels \( \alpha_i\to \alpha\). Nous avons
            \begin{equation}
                a^{\alpha}b^{\alpha}=\big( \lim_ia^{\alpha_i} \big)\big( \lim_j b^{\alpha_j} \big)=\lim_i\big( a^{\alpha_i}b^{\alpha_i}\big)=\lim_i(ab)^{\alpha_i}=(ab)^{\alpha}.
            \end{equation}
            Justifications :
            \begin{itemize}
                \item la proposition \ref{PROPooIQOAooJPMoDD} pour le produit des limites,
                \item le résultat dans \( \eQ\) que nous venons de prouver,
                \item la définition de \( (ab)^{\alpha}\) comme limite de \( (ab)^{\alpha_i}\).
            \end{itemize}
    \end{subproof}
\end{proof}
    
Pour rappel, la proposition suivantes, dans le cas de \( \alpha\in \eQ^+\) est la proposition \ref{PROPooJXHFooCDwxCS}.
\begin{proposition}[\cite{MonCerveau}]      \label{PROPooRXLNooWkPGsO}
    Pour \( \alpha>0\), la fonction
    \begin{equation}
        \begin{aligned}
            f_{\alpha}\colon\mathopen] 0 , \infty \mathclose[&\to \eR \\
                x&\mapsto x^{\alpha} 
        \end{aligned}
    \end{equation}
    est strictement croissante.

    Aussi, la fonction
    \begin{equation}
        \begin{aligned}
        f_{\alpha}\colon  \mathopen] -\infty , 0 \mathclose[  &\to \eR \\
                x&\mapsto x^{\alpha} 
        \end{aligned}
    \end{equation}
    est strictement décroissante.
\end{proposition}

\begin{proof}
    Nous rappellons que le cas \( \alpha\in \eQ^+\) est déjà traité par la proposition \ref{PROPooJXHFooCDwxCS}. Soient \( x\in \mathopen] 0 , \infty \mathclose[\) et \( s>0\). Nous allons montrer que \( f_{\alpha}(x+s)-f_{\alpha}(x)>0\). Pour cela nous décomposons en plusieurs cas.
    \begin{subproof}
        \item[\( x>1\)]
            Par la proposition \ref{PROPooFGBOooHiZqbs}, nous considérons une suite strictement croissante de rationnels strictement positifs \( \alpha_i\to \alpha\). Pour tout \( i\) nous avons \( \alpha_i>\alpha_0\).

            En utilisant la stricte croissance de \( f_{\alpha_0}\) et le lemme \ref{LEMooXJXUooLoiTMo}\ref{ITEMooKZCGooKskUQx}, nous avons les inégalités \( 1<x^{\alpha_0}<(x+s)^{\alpha_0}\), et en particulier
            \begin{equation}
                0<(x+s)^{\alpha_0}-x^{\alpha_0}.
            \end{equation}
            De plus nous avons \( 1<x<x+s\) et \( \alpha_0<\alpha_i\) pour tout \( i\). Donc le corolaire \ref{CORooYWNNooLwKmiD} s'applique et nous avons, pour tout \( i\) :
            \begin{equation}
                0<(x+s)^{\alpha_0}-x^{\alpha_0}<(x+s)^{\alpha_i}-x^{\alpha_i}.
            \end{equation}
            C'est le moment de passer à la limite \( i\to \infty\). La seconde inégalité devient non stricte, mais la première reste :
            \begin{equation}
                0<(x+s)^{\alpha_0}-x^{\alpha_0}\leq(x+s)^{\alpha}-x^{\alpha}.
            \end{equation}
            Nous avons donc bien la stricte croissance de \( f_{\alpha}\) sur \( \mathopen] 1 , \infty \mathclose[\).
        \item[\( x\leq 1\)]
            Nous choisissons encore \( \alpha_i\to \alpha\) strictement croissante dans \( \eQ\). Pour chaque \( i\), nous avons encore
            \begin{equation}
                (x+s)^{\alpha_i}-x^{\alpha_i}>0.
            \end{equation}
            Le passage à la limite change l'inégalité stricte en inégalité large, et ne permet donc pas de conclure immédiatement. Nous devons donc ruser. Soit \( k\in \eN\) tel que \( k(x+s)>1\) et \( kx>1\) (existence parce que \( \eR\) est archimédien, proposition \ref{ThoooKJTTooCaxEny}). Nous avons :
            \begin{equation}
                \big( k(x+s) \big)^{\alpha}-(kx)^{\alpha}>0
            \end{equation}
            par la partie «\( x>1\)» que nous venons de prouver. Grâce à la proposition \ref{PROPooKWRGooMTbRdU} nous pouvons factoriser \( k^{\alpha}\) :
            \begin{equation}
                0<\big( k(x+s) \big)^{\alpha}-(kx)^{\alpha}=k^{\alpha}\big( (x+s)^{\alpha}-x^{\alpha} \big).
            \end{equation}
            Vu que \( k^{\alpha}>0\), cela implique \( (x+s)^{\alpha}-x^{\alpha}>0\), ce qu'il fallait.
    \end{subproof}
Nous avons fini de prouver que la fonction \( f_{\alpha}\) était strictement croissante sur \( \mathopen] 0 , \infty \mathclose[\). En ce qui concerne la fonction \( f_{\alpha}\) sur \( \mathopen] -\infty , 0 \mathclose[\), nous avons, pour \( x>0\) que
    \begin{equation}
        f_{\alpha}(-x)=\frac{1}{ f_{\alpha}(x) },
    \end{equation}
    et donc stricte décroissance.
\end{proof}

Nous prouvons à présent que \( f_{\alpha}\) est localement injective; nous en avons besoin pour prouver la continuité. Or cette continuité est nécessaire à prouver que \( f_{\alpha}\) est localement bijective. Donc nous ne pouvons pas énoncer la bijectivité ici. Ce sera la proposition \ref{PROPooEXGKooCqzLor}.

\begin{proposition}     \label{PROPooHKTKooCUEBjh}
    Soient \( \alpha\in \eR\) et \( x\in \eR\setminus\{ 0 \}\). Il existe un voisinage \( V\) de \( x\) sur lequel
    \begin{equation}
            f_{\alpha}\colon V \to f_{\alpha}(V) 
    \end{equation}
    est injective.
\end{proposition}

\begin{proof}
Soit \( x>0 \); nous considérons un voisinage \( V\) de \( x\) inclu à \( \mathopen] 0 , \infty \mathclose[\). Soit \( y\in V\); pour fixer les idées nous supposons \( y<x\). Par la stricte croissance de \( f_{\alpha}\) sur \( \mathopen] 0 , \infty \mathclose[\) (proposition \ref{PROPooRXLNooWkPGsO}), nous avons \( f_{\alpha}(y)<f_{\alpha}(x)\) et en particulier \( f_{\alpha}(x)\neq f_{\alpha}(y)\).

Le cas \( x<0\) se traite de façon analogue, avec la stricte décroissance de \( f_{\alpha}\) sur \( \mathopen] -\infty , 0 \mathclose[\).
\end{proof}
Notons que les voisinages sur lesquels \( f_{\alpha}\) est injective sont assez grands. Ils peuvent être toute une demi-droite, si l'on veut.

\begin{lemma}   \label{LEMooQTNKooLVEytN}
    Soient \( \alpha>0\), une suite de rationnels strictement décroissante \( \alpha_i\to \alpha\) ainsi que les fonctions
    \begin{equation}
        \begin{aligned}
        f_{\alpha_i}\colon \mathopen] 1 , \infty \mathclose[&\to \eR \\
            x&\mapsto x^{\alpha_i}. 
        \end{aligned}
    \end{equation}
    La famille \( \{ f_{\alpha_i} \}_{i\in \eN}\) est équicontinue\footnote{Définition \ref{DEFooSGMVooASNbxo}.}.
\end{lemma}

\begin{proof}
    Soient \( x>1\), et \( \alpha>0\). Nous allons montrer que \( \{ f_{\alpha_i} \}\) est équicontinue en \( x\). Soit \( s\) tel que \( 1<x<x+s\); le corolaire \ref{CORooYWNNooLwKmiD} nous enseigne que 
    \begin{equation}
        (x+s)^p-x^p<(x+s)^q-x^q
    \end{equation}
    dès que \( p<q\). En particulier, \( f_p\) étant croissante par la proposition \ref{PROPooRXLNooWkPGsO},
    \begin{equation}
        0<(x+s)^{\alpha_i}-x^{\alpha_i}<(x+s)^{\alpha_0}-x^{\alpha_0}.
    \end{equation}
    Soit \( \epsilon>0\) et \( \delta\) tel que \( s<\delta\) implique \( | (x+s)^{\alpha_0}-x^{\alpha_0} |<\epsilon\). Alors nous avons aussi, pour de tels \( \sigma\) et \( s\) :
    \begin{equation}
        |(x+s)^{\alpha_i}-x^{\alpha_i}|<|(x+s)^{\alpha_0}-x^{\alpha_0}|<\epsilon.
    \end{equation}
    En procédant de même\ pour \( s<0\), nous trouvons bien que
    \begin{equation}
        | y^{\alpha_i}-x^{\alpha_i} |\leq \epsilon
    \end{equation}
    pour tout \( y\in B(x,\delta)\).

    Cela signifie que \( \{ f_i \}\) est équicontinue.
\end{proof}


\begin{proposition}[\cite{MonCerveau}]      \label{PROPooUQNZooSSHLqr}
    Soit \( \alpha>0\) dans \( \eR\). La fonction
    \begin{equation}
        \begin{aligned}
            f_{\alpha}\colon \eR&\to \eR \\
            x&\mapsto x^{\alpha} 
        \end{aligned}
    \end{equation}
    est continue (sauf pour \( x=0\) si \( \alpha<0\)).
\end{proposition}

\begin{proof}
    Nous allons subdiviser quelque cas.
    \begin{subproof}
        \item[Pour \( \alpha\in \eN\)]
            Nous supposons que ce cas va bien.
        \item[Pour \( \alpha\in \eQ^+\)]
            Soit \( q=m/n\) avec \( m,n\in \eN\). Soit aussi \( \epsilon>0\). Nous avons :
            \begin{subequations}
                \begin{align}
                    f_{m/n}(x)^n&=x^m\\
                    f_{m/n}(x+\epsilon)^n&=(x+\epsilon)^m.      \label{SUBEQooGNCSooWAeRcL}
                \end{align}
            \end{subequations}
            L'équation \eqref{SUBEQooGNCSooWAeRcL} s'écrit aussi bien sous la forme
            \begin{equation}
                f_n\big( f_{m/n}(x+\epsilon) \big)=(x+\epsilon)^m.
            \end{equation}
            En prenant la limite,
            \begin{equation}
                \lim_{\epsilon\to 0}\big[ f_n\big( f_{m/n}(x+\epsilon) \big) \big]=x^m=f_{m/n}(x)^n.
            \end{equation}
            Vu que \( f_n\) est continue, nous pouvons la permuter avec la limite dans le membre de gauche tout en écrivant \( f_{m/n}(x)^n=f_n\big( f_{m/n}(x) \big)\) dans le membre de droite :
            \begin{equation}
                f_n\big[ \lim_{\epsilon\to 0}f_{m/n}(x+\epsilon) \big]=f_n\big( f_{m/n}(x) \big).
            \end{equation}
            La fonction \( f_n\) étant injective dans un voisinage autour de \( x\) (proposition \ref{PROPooHKTKooCUEBjh}),
            \begin{equation}
                \lim_{\epsilon\to 0}f_{m/n}(x+\epsilon)=f_{m/n}(x),
            \end{equation}
            ce qui est la continuité de \( f_{m/n}\) en \( x\).

        \item[Pour \( \alpha\in \eR^+\)]

        Nous prouvons séparément le cas \( x<1\) et le cas \( x\geq 1\). Commençons par \( x\in \mathopen] 1 , \infty \mathclose[\).

            Soit une suite \( \alpha_i\to \alpha\) strictement décroissante dans \( \eQ^+\). Le lemme \ref{LEMooQTNKooLVEytN} nous dit que l'ensemble de fonctions  \( \{ f_{\alpha_i}\colon \mathopen] 1 , \infty \mathclose[\to \eR \}_{i\in \eN}\) est équicontinu. La convergence simple \( f_{\alpha_i}\to f_{\alpha}\) étant par définition, la proposition \ref{PROPooICNNooAMjcut} nous dit que la fonction \( f_{\alpha}\colon \mathopen] 1 , \infty \mathclose[\to \eR\) est continue.

            Soit maintenant \( x\in \mathopen] 0 , 1 \mathclose]\). Il existe \( k\in \eN\) tel que \( kx>1\), \( k(x/2)>1\) et \( k^{\alpha}>1\) (si vous pensez bien, seule la première condition est utile).

            Nous considérons \( \epsilon\) tel que \( x+\epsilon>x/2\); de toutes façons nous comptions faire \( \epsilon\to 0\). Nous avons :
            \begin{equation}
                \big| (x+\epsilon)^{\alpha}-x^{\alpha} \big|\leq k^{\alpha}\big| (x+\epsilon)^{\alpha}-x^{\alpha} \big|=\big| [k(x+\epsilon)]^{\alpha}-(kx)^{\alpha} \big|.
            \end{equation}
            Nous prenons le \( \delta\) qui correspond à \( \epsilon\) en \( kx\) dans la continuité de \( f_{\alpha}\) déjà démontée pour \( kx>1\). Alors si \( \epsilon<\delta\) nous avons
            \begin{equation}
                \big| (x+\epsilon)^{\alpha}-x^{\alpha} \big|\leq\epsilon.
            \end{equation}
        \item[Pour \( \alpha\in \eR^{-}\)]

            Si \( \alpha>0\), la fonction \( f_{-\alpha}\) est donnée par
            \begin{equation}
                f_{-\alpha}(x)=\frac{1}{  f_{\alpha}(x) }
            \end{equation}
            et est donc continue (sauf en \( x=0\) où elle n'existe pas).
    \end{subproof}
\end{proof}

\begin{proposition}     \label{PROPooEXGKooCqzLor}
    Soit \( \alpha>0\). La fonction
    \begin{equation}
        \begin{aligned}
            f_{\alpha}\colon \mathopen[ 0 , \infty \mathclose[&\to \mathopen[ 0 , \infty \mathclose[ \\
                x&\mapsto x^{\alpha} 
        \end{aligned}
    \end{equation}
    est bijective.
\end{proposition}

\begin{proposition}[\cite{MonCerveau}]     \label{PROPooDWZKooNwXsdV}
    Soient \( a>0\) ainsi que \( x,y\in \eR\). Alors
    \begin{equation}
        (a^x)^y=(a^y)^x=a^{xy}.
    \end{equation}
\end{proposition}

\begin{proof}
    Nous découpons en fonction de la nature de \( x\) et \( y\). 

    \begin{subproof}
        \item[\( x\) rationnel, \( y\) naturel]
            Si \( q\in \eQ\) et \( n\in \eN\) alors la formule
            \begin{equation}
                (a^q)^n=a^{nq}
            \end{equation}
            découle seulement d'une récurrence sur la formule \ref{EQooEWIHooDRAQGR}.

        \item[ \( x,y\in \eQ\)]
            Soient \( y=m/n\) avec \( n\in \eZ\), \( m\in \eN\) et \( q\in \eQ\). Nous avons, en utilisant la partie déjà démontrée et le lemme \ref{LEMooIDLJooZALNaD},
            \begin{equation}
                (a^q)^{p}=(a)^{m/n}=\big( (a^q)^m \big)^{1/n}=(a^{mq})^{1/n}=a^{mq/n}=a^{pq}.
            \end{equation}
        \item[\( x,y\) irrationnels]

            Soient des suites des rationnels \( x_i\to x\) et \( y_i\to y\). En utilisant les définitions,
            \begin{equation}        \label{EQooXITUooHYNSPU}
                (a^x)^y=\lim_i(a^x)^{y_i}=\lim_i\big( \lim_j a^{x_j} \big)^{y_i}.
            \end{equation}
            Fixons un \(i\) pour commencer. Nous avons, par la continuité de \( f_{y_i}\) (proposition \ref{PROPooUQNZooSSHLqr})
            \begin{equation}
                \big( \lim_ja^{x_j} \big)^{y_i}=f_{y_i}\big( \lim_ja^{x_j} \big)=\lim_j\big( f_{y_i}(a^{x_j}) \big)=\lim_ja^{x_jy_i}.
            \end{equation}
            Nous avons utilisé le résultat déjà démontré dans le cas des rationnels. La suite \( j\mapsto x_jy_i\) est une suite dans \( \eQ\) qui converge vers le réel \( xy_i\), donc la limite sur \( j\) redonne la fonction puissance :
            \begin{equation}        \label{EQooWORSooFoRBod}
                \big( \lim_ja^{x_j} \big)^{y_i}=\lim_ja^{x_jy_i}=a^{xy_i}.
            \end{equation}
            Le résultat découle maintenant de la prise de limite dans \eqref{EQooXITUooHYNSPU} qui revient à prendre la limite \( i\to \infty\) de l'expression dans \eqref{EQooWORSooFoRBod} :
            \begin{equation}
                (a^x)^y=\lim_i\big( \lim_j a^{x_j} \big)^{y_i}=\lim_ia^{xy_i}=a^{xy}.
            \end{equation}
    \end{subproof}
\end{proof}

Le lemme suivant montre en gros que \( x^y\) croît plus rapidement en \( y\) qu'en \( x\). 
\begin{lemma}       \label{LemLJOSooEiNtTs}
     Pour tout \( \alpha>0\) et \( a<1\) nous avons la limite
     \begin{equation}
         \lim_{n\to \infty} n^{\alpha}a^n=0
     \end{equation}
\end{lemma}

\begin{proof}
    Soit \( k\in \eN\) plus grand que \( \alpha\).
    Soit la suite numérique \( s_n=n^ka^n\). Tous ses termes sont positifs et
    \begin{equation}
        \frac{ s_n }{ s_{n+1} }=\left( \frac{ n }{ n+1 } \right)^k\frac{1}{ a }.
    \end{equation}
    Étant donné que \( n/n+1\to 1\) et que \( a<1\), il existe un certain rang à partir duquel la suite \( (s_n)\) est décroissante. Deux conclusions :
    \begin{itemize}
        \item Elle est majorée par une constante \( M\).
        \item Elle est convergente par le lemme~\ref{LemSuiteCrBorncv}.
    \end{itemize}
    Soit \( l\) tel que \( ka^l<1\) et \( n>l\) alors
    \begin{equation}
        s_{n+l}=(n+l)^ka^{n+l}\leq kn^ka^na^l=ka^ls_n\leq ka^lM.
    \end{equation}
    La majoration est due au fait que dans \( (n+l)^k\) nous avons \( k\) termes tous plus petits que \( n^k\). De la même façon,
    \begin{equation}
        s_{2n+2l}\leq ka^{2l}s_{2n}\leq ka^{2l}M.
    \end{equation}
    En posant \( \varphi(i)=in+il\) nous avons
    \begin{equation}
        s_{\varphi(i)}\leq ka^iM,
    \end{equation}
    qui est une sous-suite convergente vers \( 0\). Or si une suite est convergente (ce qui est le cas de \( (s_n)\)), toutes les sous-suites convergent vers la même limite. Nous en concluons que \( s_n\to 0\).
\end{proof}

\begin{normaltext}
    Une conséquence est que si vous voulez choisir un mot de passe fort, la longueur du mot est plus importante que la taille de l'alphabet choisit : il est plus efficace de choisir une combinaison longue qu'une combinaisons mélangeant des lettres, chiffres et symboles spéciaux.
    
    Exemple : si vous choisissez un mot de passe contenant majuscules, minuscules, chiffres et symboles spéciaux complètement mélangés (ne mentez pas, vous ne le faites pas), mais que vous ne le choisissez que de taille \( 6\), vous avez \( 72^6\) possibilités (en supposant un jeu de 10 symboles spéciaux).

    Eh bien, en seulement \( 8\) lettres minuscules, vous avez plus de possibilités : \( 26^8>72^6\).

    De nombreux sites font l'erreur de considérer que
    \begin{itemize}
        \item « ggzxzheaiynshunxuydajkwyohgqxz » est un mot de passe faible,
        \item «azerty.2019A» est un mot de passe fort.
    \end{itemize}
    Il n'en est rien. Le premier est considérablement meilleur que le second, même si le second, très superficiellement, mélange les lettres majuscules, minuscules, chiffres et symboles spéciaux.

    Voila voila. La prochaine fois qu'un site vous refusera un mot de passe de 30 lettres minuscules mélangées, vous saurez pourquoi il n'y a rien qui marche en informatique, et en particulier pourquoi la sécurité générale de nos systèmes d'informations est désastreuse.
\end{normaltext}


%--------------------------------------------------------------------------------------------------------------------------- 
\subsection{Dérivation de la fonction puissance (première)}
%---------------------------------------------------------------------------------------------------------------------------

Nous n'allons pas complètement résoudre la question de la dérivation de la fonction \( x\mapsto a^x\); il faudrait des logarithmes, et nous ne les avons pas encore défini. Le logarithme sera introduit comme fonction inverse de l'exponentielle en \ref{DEFooELGOooGiZQjt}.

\begin{proposition}[\cite{BIBooXUZHooOHWxiF}]       \label{PROPooMXCDooBffXbl}
    Soit la fonction puissance
    \begin{equation}
        \begin{aligned}
            g_a\colon \eR&\to \eR \\
            x&\mapsto a^x.
        \end{aligned}
    \end{equation}
    \begin{enumerate}
        \item
            La fonction \( g_a\) est dérivable.
        \item
            La dérivée vérifie l'équation
            \begin{equation}        \label{EQooNIUJooPqDnax}
                g_a'(x)=g_a'(0)g_a(x).
            \end{equation}
    \end{enumerate}
\end{proposition}

\begin{proof}
    La fonction \( g_a\) est continue par \ref{PROPooVADRooLCLOzP}\ref{ITEMooQHYRooJIewyp}. La proposition \ref{ThoEXXyooCLwgQg} nous dit donc que la fonction \( g_a\) admet une primitive sur \( \eR\). Nous notons \( F\) une telle primitive.

    Soit \( x\in \eR\). En posant \( F_x(t)=F(x+t)\), nous avons une primitive de \( t\mapsto a^xa^t\). En effet
    \begin{equation}
        F_x'(t)=F'(x+t)\Dsdd{ x+t }{t}{0}=a^{a+t}=a^xa^t.
    \end{equation}
    Par ailleurs la fonction \( t\mapsto a^xF(t)\) est également une primitive de \( t\mapsto a^xa^t\). Donc il existe un nombre \( C(x)\) tel que
    \begin{equation}
        F_x(t)=F(x+t)=a^xF(t)+C(x).
    \end{equation}
    
    Le nombre \( F(1)-F(0)\) est un nombre sans histoires. Nous avons :
    \begin{subequations}        \label{SUBALIGNooVARJooIcPEHN}
        \begin{align}
            g_a(x)\big( F(1)-F(0) \big)&=g_a(x)F(1)-g_a(x)F(0)\\
            &=F_x(1)-C(x)-F_x(0)+C(x)\\
            &=F_x(1)-F_x(0)\\
            &=F(1+x)-F(x).
        \end{align}
    \end{subequations}
    La fonction \( F\) étant dérivable, nous en déduisons que \( g_a\) est dérivable.

    Vu que nous n'avons aucune idée de la forme de \( F\), nous ne pouvons pas tirer beaucoup d'informations d'une dérivation des membres de gauche et de droite de \eqref{SUBALIGNooVARJooIcPEHN}.

    En ce qui concerne la formule, nous écrivons la fameuse équation fonctionnelle\footnote{Pour rappel, proposition \ref{PROPooVADRooLCLOzP}\ref{ITEMooSCJBooNVJZah}.}
    \begin{equation}
        g_a(x+y)=g_a(x)g_a(y)
    \end{equation}
    Nous fixons \( x\) et dérivons par rapport à \( y\) en \( y=y_0\) :
    \begin{equation}
        g_a'(x+y_0)=g_a(x)g_a'(y_0).
    \end{equation}
    En posant \( y_0=0\) nous trouvons le résultat demandé.
\end{proof}

\begin{normaltext}
    La démonstration donnée dans \cite{BIBooXUZHooOHWxiF} s'assure d'abord de l'existence d'une intégrale (lemme \ref{LEMooWKSWooPptdEm}), pose ensuite  \( A=\int_0^1g_a(t)dt\) et fait le calcul suivant :
    \begin{equation}
        Ag_a(x)=\int_{0}^1g_a(x)g_a(t)dt=\int_{0}^1g_a(x+t)dt=\int_x^{x+1}g_a(t)dt.
    \end{equation}
    Vu que le membre de droite est une fonction dérivable de \( x\), nous concluons que \( g_a\) est dérivable. Cela demande donc toute la théorie de l'intégration pour prouver la \emph{dérivabilité} d'une fonction.

    La démonstration donnée ici est à peine mieux. Elle utilise l'existence d'une primitive et donc tout le théorème de Stone-Weierstrass \ref{ThoGddfas}.

    Dans les deux cas, je trouve que la situation n'est pas fameuse. Si vous êtes capable de montrer l'existence de la limite
    \begin{equation}
        \lim_{\epsilon\to 0}\frac{ a^{\epsilon}-1 }{ \epsilon }
    \end{equation}
    sans recourir à autre chose que des astuces sur les limites, je suis preneur. Ou, au contraire, si vous avez un argument pour dire que c'est impossible, dites-le moi également. Écrivez-moi.
\end{normaltext}



Nous posons une définition
\begin{definition}      \label{DEFooPJKMooOfZzgy}
    Soit \( a>0\). Nous nommons l'\defe{équation fonctionnelle}{équation fonctionnelle} l'équation
    \begin{subequations}        \label{EQooULHBooByFVec}
        \begin{numcases}{}
            f(x+y)=f(x)f(y)\\
            f(1)=a
        \end{numcases}
    \end{subequations}
    pour la fonction inconnue \( f\colon \eR\to \eR\).
\end{definition}



\begin{definition}      \label{DEFooXMQTooSbZzqJ}
    Soit \( a>0\). Nous nommons l'\defe{équation exponentielle}{équation exponentielle} l'équation
    \begin{subequations}        \label{EQooGDBYooUoAFPW}
        \begin{numcases}{}
            y'=y\\
            y(1)=a
        \end{numcases}
    \end{subequations}
    pour la fonction inconnue \( y\colon \eR\to \eR\).
\end{definition}

\begin{proposition}[Unicité de l'exponentielle] \label{PropDJQSooYIwwhy}
    Si elle existe, la solution au problème
    \begin{subequations}
        \begin{numcases}{}
            y'=y\\
            y(0)=1
        \end{numcases}
    \end{subequations}
    est unique.
\end{proposition}
\index{exponentielle!unicité}

\begin{proof}
    Soient \( y\) et \( g\) deux solutions et considérions la fonction \( h(x)=g(x)y(-x)\). Un calcul immédiat donne
    \begin{equation}
        h'(x)=0
    \end{equation}
    et donc \( h\) est constante. Vu que \( h(0)=1\) nous avons \( g(x)y(-x)=1\) pour tout \( x\), c'est-à-dire
    \begin{equation}
        g(x)=\frac{1}{ y(-x) }=y(x).
    \end{equation}
\end{proof}

\begin{normaltext}
    Nous savons qu'il existe une unique solution de l'équation exponentielle avec \( a=1\). Avec la relation
    \begin{equation}
        g_a'(x)=g_a(x)g_a'(0),
    \end{equation}
    de la proposition \ref{PROPooMXCDooBffXbl}, nous n'en sommes pas loin. Il faut encore savoir si il existe un \( a>0\) tel que \( g_a'(0)=1\). Notre culture générale nous dit qu'un tel réel existe et est la fameuse constante \( e\).

    Nous nous attelons maintenant à la tâche de montrer l'existence de la chose.
\end{normaltext}

%--------------------------------------------------------------------------------------------------------------------------- 
\subsection{Équation fonctionnelle}
%---------------------------------------------------------------------------------------------------------------------------

Il n'est un secret pour personne (proposition \ref{PROPooVADRooLCLOzP}\ref{ITEMooSCJBooNVJZah}) que la fonction
\begin{equation}
    \begin{aligned}
        g_a\colon \eR&\to \eR \\
        x&\mapsto a^x 
    \end{aligned}
\end{equation}
vérifie l'équation fonctionnelle \eqref{EQooULHBooByFVec}. Nous pouvons nous demander à quel point cette propriété caractérise la fonction puissance.


\begin{proposition}[\cite{BIBooXUZHooOHWxiF}]       \label{PROPooJDPEooYTDVtU}
    Encore plusieurs résultats sur la fonction \( g_a\) avec \( a>0\).
    \begin{enumerate}
        \item       \label{ITEMooZJUEooVoqKul}
            La fonction \( g_a\) vérifie l'équation fonctionnelle.
        \item       \label{ITEMooCSQXooUDyiMq}
            La dérivée vérifie \( g_a'(0)\neq 0\).
        \item       \label{ITEMooCKIHooNuDwrk}
            Pour tout \( a\), en posant \( \alpha=1/g'_a(0)\) nous avons
            \begin{equation}
                g_{a^{\alpha}}'(0)=1.
            \end{equation}
        \item       \label{ITEMooQQFRooWtlViJ}
            Il existe un unique \( e>0\) tel que 
            \begin{equation}
                g_e'=g_e.
            \end{equation}
        \item       \label{ITEMooERTLooWLjlnZ}
            Pour la valeur de \( e\) donnée en \ref{ITEMooQQFRooWtlViJ}, la fonction \( g_e\) vérifie l'équation exponentielle \eqref{EQooGDBYooUoAFPW} 
            \begin{subequations}
                \begin{numcases}{}
                    g_e'=g_e\\
                    g_e(1)=e.
                \end{numcases}
            \end{subequations}
    \end{enumerate}
\end{proposition}

\begin{proof}
    Un point à la fois.
    \begin{subproof}
        \item[Pour \ref{ITEMooZJUEooVoqKul}]
            Le fait que \( g_a\) vérifie l'équation fonctionnelle est la proposition \ref{PROPooVADRooLCLOzP}\ref{ITEMooSCJBooNVJZah}.

        \item[Pour \ref{ITEMooCSQXooUDyiMq}]

            La formule \eqref{EQooNIUJooPqDnax} de la proposition \ref{PROPooMXCDooBffXbl} nous assure que
            \begin{equation}        \label{EQooSCDJooTvnjEp}
                g_a'(x)=g_a(x)g_a'(0).
            \end{equation}
            Donc \( g_a'(0)=0\) impliquerait que \( g_a=0\), ce qui n'est pas le cas.

        \item[Pour \ref{ITEMooCKIHooNuDwrk}]
            Par ailleurs la proposition \ref{PROPooDWZKooNwXsdV} nous permet d'écrire
            \begin{equation}
                g_a(\alpha x)=g_{a^{\alpha}}(x).
            \end{equation}
            En dérivant des deux côtés,
            \begin{equation}        \label{EQooIHCDooGSEGNm}
                \alpha g_a'(\alpha x)=g'_{a^{\alpha}}(x).
            \end{equation}
            En posant donc \( \alpha=g'_a(0)\) et en évaluant \eqref{EQooIHCDooGSEGNm} en \( x=0\) nous trouvons le résultat.

        \item[Pour \ref{ITEMooQQFRooWtlViJ}, existence]

            Pour les valeurs de \( \alpha\) données par le point \ref{ITEMooCKIHooNuDwrk}, nous avons \( g'_{a^{\alpha}}(0)=1\), et l'équation \eqref{EQooSCDJooTvnjEp} nous donne alors
            \begin{equation}
                g_{a^{\alpha}}(x)=g_{a^{\alpha}}(x).
            \end{equation}
            Comme de plus \( g_{a^{\alpha}}(0)=1\), cette fonction vérifie bien l'équation exponentielle.
        \item[Pour \ref{ITEMooQQFRooWtlViJ}, unicité]

            Si \( a\) et \( b\) font en sorte que \( g_a'=g_a\) et \( g_b'=g_b\), alors nous avons aussi \( g_a'(0)=g_b'(0)=1\) à cause de \eqref{EQooSCDJooTvnjEp}. Donc \( g_a\) et \( g_b\) vérifient l'équation de la proposition \ref{PropDJQSooYIwwhy} dont la solution est unique. Donc \( g_a=g_b\).

            Pour tout \( x\) nous avons \( g_a(x)=g_b(x)\). En particulier pour \( x=1\) nous avons \( a=b\).
    \end{subproof}
\end{proof}

\begin{proposition}[\cite{BIBooXUZHooOHWxiF}]    \label{PROPooGBUPooWtWaFI}
    Soit \( a>0\). Nous considérons l'équation fonctionnelle \ref{DEFooPJKMooOfZzgy} et l'équation exponentielle \ref{DEFooXMQTooSbZzqJ} pour une fonction \( f\colon \eR\to \eR\).
    \begin{enumerate}
        \item       \label{ITEMooYHAVooWzJqBj}
            Si \( f\) vérifie l'équation fonctionnelle, alors
            \begin{equation}
                f(q)=a^q
            \end{equation}
            pour tout \( q\in \eQ\).
        \item       \label{ITEMooQHOMooNVzSxn}
            Si \( f\) vérifie l'équation fonctionnelle et est monotone, alors \( f=g_a\).
        \item       \label{ITEMooCNXOooZcrxeB}
            Si \( f\) vérifie l'équation fonctionnelle et est continue, alors \( f=g_a\).
    \end{enumerate}
\end{proposition}

\begin{proof}
    En beaucoup de parties. Nous commençons par prouver \ref{ITEMooYHAVooWzJqBj}. Nous supposons que \( f\colon \eR\to \eR\) est une fonction vérifiant l'équation fonctionnelle\footnote{Bien que ce ne soit pas strictement nécessaire ici, nous rappelons qu'une telle fonction existe par la proposition \ref{PROPooJDPEooYTDVtU}.}.
    \begin{subproof}
        \item[\( f(x)\geq 0\)]
            Quel que soit \( x\in \eR\) nous avons
            \begin{equation}
                f(x)=f(\frac{ x }{2}+\frac{ x }{2})=f(\frac{ x }{2})^2\geq 0.
            \end{equation}
            Vous noterez que cet argument ne fonctionne pas si \( f\) est à valeurs dans \( \eC\) au lieu de \( \eR\).
        \item[Pour \( n\in \eN\)]
            Soit \( n\in \eN\). Je vous laisse rédiger la récurrence correctement, mais l'idée est que \( f(1)=a\) et ensuite que
            \begin{equation}
                f(n+1)=f(n)f(1)=f(1)^nf(1)=f(1)^{n+1}.
            \end{equation}
        \item[Pour \( m\in \eZ\)]
            Nous avons d'une part que \( f(-m+m)=f(0)=1\), mais d'autre part que \( f(-m+m)=f(-m)f(m)\). Donc \( 1=f(-m)f(m)\); et nous concluons que
            \begin{equation}
                f(-m)=\frac{1}{ f(m) }.
            \end{equation}
        \item[Pour \( q=1/n\)]
            Nous savons que \( f(1)=a\), mais \( 1=\frac{1}{ n }+\ldots\frac{1}{ n }\) (avec \( n\) termes), donc
            \begin{equation}
                a=f(\frac{1}{ n }+\ldots \frac{1}{ n })=f(\frac{1}{ n })^n.
            \end{equation}
            Cela implique que \( f(\frac{1}{ n })^n=a\). La proposition \ref{PROPooXQYFooPxoEHE} indique qu'il existe un unique \( x>0\) tel que \( x^n=a\). Vu que nous savons déjà que \( f\) est partout positive\footnote{C'est ici que l'hypothèse de fonction à valeurs dans \( \eR\) est cruciale. Pour \( f\colon \eR\to \eC\) ceci ne fonctionne pas, et de loin.}, cette contrainte fixe \( f(1/n)\) et la définition \ref{DEFooJWQLooWkOBxQ} nous permet d'écrire
            \begin{equation}
                f(\frac{1}{ n })=a^{1/n}.
            \end{equation}
        \item[Pour \( q\in \eQ\)]
            Nous posons \( q=m/n\). Le nombre \( q\) peut être écrit sous la forme \( \frac{ m }{ n }=\frac{1}{ n }+\ldots +\frac{1}{ n }\) avec \( m\) termes. Donc
            \begin{equation}
                f(\frac{ m }{ n })=f(\frac{1}{ n }+\ldots \frac{1}{ n })=f(\frac{1}{ n })^{m}=(a^{1/n})^m=a^{m/n}
            \end{equation}
            où nous avons utilisé le lemme \ref{LEMooIDLJooZALNaD}.
    \end{subproof}
    La preuve de \ref{ITEMooYHAVooWzJqBj} est terminée.

    \begin{subproof}
        \item[Démonstration de \ref{ITEMooQHOMooNVzSxn}]
            Nous faisons maintenant la preuve de \ref{ITEMooQHOMooNVzSxn}. Nous supposons que \( f\) vérifie l'équation fonctionnelle et qu'elle est monotone. Pour fixer les idées, nous supposons qu'elle est monotone croissante\footnote{Si \( f\) est monotone décroissante, soit vous adaptez la preuve, soit vous essayez de voir si on ne peut pas recycler le cas croissant en l'appliquant à \( -f\).}.
        
            Nous considérons les parties\footnote{Il du meilleur gout de citer le lemme \ref{LemooHLHTooTyCZYL} pour dire qu'ils sont non vides.}
            \begin{subequations}
                \begin{align}
                    A=\{ q\in \eQ\tq q<x \}\\
                    B=\{ q\in \eQ\tq q>x \}.
                \end{align}
            \end{subequations}
            Vu que \( f\) est croissante, nous avons \( f(x)\geq f(q)\) pour tout \( q\in A\) et \( f(x)\leq f(q)\) pour tout \( q\in B\). En passant au supremum et à l'infimum,
            \begin{equation}
                \sup_{q\in A}f(q)\leq f(x)\leq \inf_{q\in B}f(q).
            \end{equation}
            Mais il existe dans \( A\) une suite strictement croissante convergente \( q_i\) vers \( x\) (parce que \( x=\sup(A)\)), donc
            \begin{equation}
                a^x=\lim_ia^{q_i}
            \end{equation}
            par la définition \ref{DEFooOJMKooJgcCtq}. Et de même, il existe une suite \( r_i\) décroissante dans \( B\) telle que \( x=\lim r_i\). Cette suite donne aussi
            \begin{equation}
                a^x=\lim_i a^{r_i}.
            \end{equation}
            Nous avons donc l'encadrement
            \begin{equation}
                a^x\leq f(x)\leq a^x,
            \end{equation}
            qui implique que \( f(x)=a^x\).

        \item[Démonstration de \ref{ITEMooCNXOooZcrxeB}]

            Nous ne supposons plus que \( f\) est monotone. Au lieu de cela nous supposons qu'elle est continue. Nous avons déjà vu en \ref{ITEMooYHAVooWzJqBj} que \( f=g_a\) sur \( \eQ\). Mais par hypothèse \( f\) est continue et par la proposition \ref{PROPooVADRooLCLOzP}, \( g_a\) est continue. La proposition \ref{PROPooXWHYooFiVYfi} conclu que \( f=g_a\) sur \( \eR\).
    \end{subproof}
\end{proof}

\begin{proposition}[\cite{BIBooXUZHooOHWxiF}]
    Si \( y\) vérifie l'équation exponentielle, alors elle est continue, monotone et vérifie l'équation fonctionnelle.
\end{proposition}

%--------------------------------------------------------------------------------------------------------------------------- 
\subsection{Dérivation de la fonction puissance (seconde)}
%---------------------------------------------------------------------------------------------------------------------------

La proposition suivante donne la dérivée de \( x\mapsto x^q\) pour tout \( q\in \eQ\). La formule donnée est encore valable pour \( x\mapsto x^{\alpha}\) pour tout \( \alpha\in \eR\), mais elle demandera plus de théorie pour être démontrée, voir la proposition \ref{PROPooKIASooGngEDh}.
\begin{proposition}[\cite{MonCerveau}]     \label{PROPooSGLGooIgzque}
    Pour tout \( \alpha\in \eQ\), si \( f_{\alpha}(x)=x^{\alpha}\) alors
    \begin{equation}
        f'_{\alpha}(x)=\alpha x^{\alpha-1}.
    \end{equation}
    En particulier, \( f_{\alpha}\) est de classe \(  C^{\infty}\) sur \( \eR\setminus\{ 0 \}\).
\end{proposition}

\begin{proof}
    Petit à petit.
    \begin{subproof}
        \item[Naturel]
            Nous prouvons que \( (x^n)'=nx^{n-1}\) par récurrence en utilisant la règle de Leibnitz de la propositon  \ref{PROPooOUZOooEcYKxn}\ref{ITEMooMQERooBCqnvS}.

            D'abord pour \( n=1\) nous avons \( f_1(x)=x\) et donc
            \begin{equation}
                f_1'(x)=\lim_{\epsilon\to 0}\frac{ (x+\epsilon)-x }{ \epsilon }=1.
            \end{equation}
            Supposons que \( f_k'(x)=kx^{k-1}\) pour un certain \( k\in \eN\). Nous prouvons que \( f_{k+1}'(x)=(k+1)x^{k}\).  Nous avons
            \begin{equation}
                x^{k+1}=xx^k.
            \end{equation}
            En utilisant la règle de Leibnitz et l'hypothèse de récurrence,
            \begin{equation}
                    \big( x^{k+1} \big)'=(x)'x^k+x\big( x^k \big)'
                    =x^k+x\big( kx^{k-1} \big)
                    =x^k+kx^k
                    =(k+1)x^k,
            \end{equation}
            ce qu'il fallait démontrer.

        \item[Rationnel positif]
            Soit donc \( \alpha=p/q\) avec \( p,q\in \eN\). Le lemme \ref{LEMooIDLJooZALNaD} nous permet d'écrire \( f_{p/q}(x)=x^{p/q}=(x^p)^{1/q}\). Cela donne
            \begin{equation}
                f_{p/q}(x)^q=x^p.
            \end{equation}
            Nous dérivons cette relation par rapport à \( x\) en utilisant à la fois la règle pour les entiers et la règle des fonctions composées\footnote{Proposition \ref{PROPooOUZOooEcYKxn}\ref{ITEMooLYZCooVUPTyh}.} :
            \begin{equation}
                qf_{p/q}'(x)^{q-1}f'_{p/q}(x)=px^{p-1}.
            \end{equation}
            En isolant \( f_{p/q}'(x)\) dans cette expression et en utilisant le fait que \( \frac{ x^a }{ x^b }=x^{a-b}\), nous trouvons le résultat.

        \item[Rationnels négatifs]

            Soit \( \alpha=-p/q\) avec \( p,q\in \eN\). Nous avons \( x^{-p/q}=\frac{1}{ f_{p/q}(x) }\). En utilisant la proposition \ref{PROPooOUZOooEcYKxn}\ref{ITEMooMUNQooLiKffz} et le point déjà prouvé sur les rationnels positifs,
            \begin{equation}
                f'_{p/q}=-\frac{ f'_{-p/q} }{ f_{p/q}^2 }=-\frac{ (-p/q)x^{-p/q-1} }{ x^{-2p/q} }=(p/q)x^{p/q-1}.
            \end{equation}
            Notez l'utilisation de la proposition \ref{PROPooDWZKooNwXsdV} au dénominateur.

        \item[Irrationnel]

            Ah ah ! On vous a bien eu. Les irrationnels, c'est pour la proposition \ref{PROPooKIASooGngEDh}.
    \end{subproof}
    En ce qui concerne le fait que la fonction \( f_{\alpha}\) est de classe \(  C^{\infty}\) sur \( \eR\setminus\{ 0 \}\), c'est simplement une récurrence. Attention : si le rationnel \( \alpha\) est négatif, \( f_{\alpha}(0)\) n'est pas défini. Mais, lorsque \( \alpha\) est positif non entier, à partir d'un certain ordre, les dérivées font intervenir \( x^{\beta}\) avec \( \beta<0\). D'où la restriction à \( \eR\setminus\{ 0 \}\) du domaine sur lequel \( f_{\alpha}\) est de classe \(  C^{\infty}\).

    Si \( \alpha\) est positif entier, alors \( f_{\alpha}\) est de classe \(  C^{\infty}\) sur tout \( \eR\) parce que toutes les dérivées sont nulles à partir d'un certain ordre.
\end{proof}

%--------------------------------------------------------------------------------------------------------------------------- 
\subsection{Hölder}
%---------------------------------------------------------------------------------------------------------------------------

Notre étude de la fonction puissance permet de démontrer quelques inégalités de Hölder. Voir le thème \ref{THEMEooUJVXooZdlmHj}.

\begin{theorem}[\cite{BIBooWAUMooIUnSOs}]       \label{THOooPPDPooJxTYIy}
    Si \( 1\leq p <  \infty\), alors pour tout \( x\in \eR^n\) nous avons
    \begin{equation}
        \| x \|_{\infty}\leq \| x \|_p\leq n^{1/p}\| x \|_{\infty}.
    \end{equation}
\end{theorem}

\begin{proof}
    Vu que la fonction \( t\mapsto | t |^p\) est croissante pour les \( t \) positifs\footnote{Parce que \( p\geq 1\) et la proposition \ref{PROPooRXLNooWkPGsO}.}, pour chaque \( i\) nous avons
    \begin{equation}
        | x_i |\leq \left( \sum_k| x_k |^p \right)^{1/p}=\| x \|_p.
    \end{equation}
    Cela montre que
    \begin{equation}
        \| x \|_{\infty}=\max\{ | x_i | \}\leq \| x \|_p.
    \end{equation}
    
    D'autre part, pour chaque \( i\) nous avons \( | x_i |\leq \| x \|_{\infty}\), donc
    \begin{equation}
        \| x \|_p\leq \big( n\| x \|_{\infty} \big)^{1/p}=n^{1/p}\| x \|_{\infty}.
    \end{equation}
\end{proof}

Le corolaire suivant donne une façon de majorer une norme \( \ell^p\) par une norme \( \ell^q\) moyennant un facteur. Notons cependant que l'inégalité de Hölder de la proposition \ref{PROPooQZTNooGACMlQ} est plus précise. Ce corolaire est suffisant pour prouver l'équivalence des normes \( \ell^p\).
\begin{corollary}       \label{CORooEZGHooACHOiB}
    Soient \( p\geq 1\) et \( q\leq \infty\). Pour tout \( x\in \eR^n\) nous avons
    \begin{equation}
        \| x \|_p\leq n^{1/p} \| x \|_q.
    \end{equation}
\end{corollary}

\begin{proof}
    Il suffit d'utiliser les deux inégalités du théorème \ref{THOooPPDPooJxTYIy}. D'abord la seconde avec \( p\), et ensuite la première avec \( q\).
\end{proof}

%--------------------------------------------------------------------------------------------------------------------------- 
\subsection{Vers les complexes}
%---------------------------------------------------------------------------------------------------------------------------

Nous avons déjà vu la proposition \ref{PROPooJDPEooYTDVtU} qui dit essentiellement que si une fonction continue \( f\colon \eR\to \eR\) vérifie \( f(x+y)=f(x)f(y)\), alors \( f(x)=a^x\). Comme indiqué durant la preuve, cette proposition (et en particulier sa preuve) ne fonctionne pas pour les fonctions à valeurs complexes. L'endroit où cela coinçait est que la contrainte
\begin{equation}
    f(\frac{1}{ n })^n=a
\end{equation}
n'implique pas grand chose lorsque \( f\) est à valeurs complexes.

Nous allons maintenant attaquer ce problème.


\begin{lemma}       \label{LEMooDEGEooXheixp}
    Soit \( \alpha\in \eC\). Si elle existe, la solution au problème
    \begin{subequations}
        \begin{numcases}{}
            y'=\alpha y\\
            y(0)=1
        \end{numcases}
    \end{subequations}
    pour \( y\colon \eR\to S^1\) est unique.
\end{lemma}

\begin{proof}
    Soient deux solutions \( y_1\) et \( y_2\). Nous posons \( h(x)=y_1(x)y_2(-x)\). Une dérivation donne
    \begin{equation}
        h'(x)=y_1'(x)y_2(-x)-y_1(x)y'_2(-x).
    \end{equation}
    En y substituant \( y'_1(x)=\alpha y_1(x)\) et \( y'_2(-x)=\alpha y_2(x)\) nous trouvons \( h'(x)=0\). Donc \( h\) est constante et nous avons
    \begin{equation}        \label{EQooTWBQooBLLKSt}
        y_1(x)y_2(-x)=1
    \end{equation}
    pour tout \( x\). Notons que cette identité est encore valable avec \( y_1=y_2\). Nous avons en particulier les égalités \( y_1(x)y_1(-x)=1\) et \( y_2(x)y_2(-x)=1\), et nous notons au passage que \( y_1(x)\) et \( y_2(x)\) ne s'annulent pas.

    En substituant dans \eqref{EQooTWBQooBLLKSt} la valeur \( y_2(-x)=\frac{1}{ y_2(x) }\) nous trouvons
    \begin{equation}
        \frac{ y_1(x) }{ y_2(x) }=1,
    \end{equation}
    ce qui signifie \( y_1(x)=y_2(x)\).
\end{proof}

Dans la proposition suivante, \( S^1\) désigne l'ensemble des nombres complexes de norme \( 1\), dont un paramétrage est donnée dans la proposition \ref{PROPooZEFEooEKMOPT} :
\begin{equation}
    S^1=\{  e^{ix}\tq x\in \eR \}=\{  e^{ix}\tq x\in \mathopen[ 0 , 2\pi \mathclose[ \}.
\end{equation}

\begin{proposition}[\cite{MonCerveau}]      \label{PROPooVJLYooOzfWCd}
    Soit une fonction continue \( f\colon \eR\to S^1\) vérifiant
    \begin{equation}        \label{EQooHANKooHirpTL}
        f(x+y)=f(x)f(y).
    \end{equation}
    Alors
    \begin{enumerate}
        \item
            \( f\) est dérivable,
        \item
            \( f\) satisfait au système
            \begin{subequations}
                \begin{numcases}{}
                    f'(x)=f'(0)f(x)\\
                    f(0)=1,
                \end{numcases}
            \end{subequations}
        \item
            il existe \( m\in \eR\) tel que \( f(x)= e^{imx}\).
    \end{enumerate}
\end{proposition}

\begin{proof}
    Pour chaque \( m\in \eR\), la fonction
    \begin{equation}
        g_m(x)= e^{imx}
    \end{equation}
    vérifie évidemment toutes les conditions. Le but de cette démonstration est de montrer que les conditions imposées à \( f\) la déterminent de façon univoque (à part ce \( m\)).

    La condition \eqref{EQooHANKooHirpTL} nous dit que \( f(0)=1\). Soit une primitive \( F\) de \( f\). Il existe \( s>0\) tel que \( F(s)>F(0)\) parce que \( F'=f\) et \( f(0)=1\).

    Soit \( a\in \eR\). La fonction \( G_a\) donnée par \( G_a(x)=F(x+a)\) est une primitive de \( x\mapsto f(x)f(a)\). Donc \( G_a(x)=f(a)F(x)\). Cela dit nous avons
    \begin{equation}
        f(x)\big( F(s)-F(0) \big)=f(x)F(s)-f(x)F(0)=G_1(x)-G_0(x).
    \end{equation}
    Le membre de droite est évidemment dérivable, et \( F(s)-F(0)\neq 0\). Donc \( f\) est dérivable.

    Nous dérivons maintenant la relation \( f(x+y)=f(x)f(y)\) par rapport à \( y\) en \( y=0\). Cela donne
    \begin{equation}
        f'(x)=f'(0)f(x).
    \end{equation}
    Donc il existe \( \alpha\in \eC\) tel que \( f'(x)=\alpha f(x)\). 

    Jusqu'ici nous avons prouvé qu'il existe \( \alpha\in \eC\) tel que
    \begin{subequations}
        \begin{numcases}{}
            f'(x)=\alpha f(x)\\
            f(0)=1.
        \end{numcases}
    \end{subequations}
    Or le lemme \ref{LEMooDEGEooXheixp} donne l'unicité de la solution à ce système, et il ne faut pas chercher loin : la solution est
    \begin{equation}
        f(x)= e^{\alpha x}.
    \end{equation}
    
    Pour avoir \( f(x)\in S^1\), nous devons de plus imposer que \( \alpha\) soit imaginaire pur. Donc, en posant \( \alpha=im\), nous avons \( m\in \eR\) tel que \( f(x)= e^{imx}\).
\end{proof}

%+++++++++++++++++++++++++++++++++++++++++++++++++++++++++++++++++++++++++++++++++++++++++++++++++++++++++++++++++++++++++++
\section{Polynômes de Taylor}
%+++++++++++++++++++++++++++++++++++++++++++++++++++++++++++++++++++++++++++++++++++++++++++++++++++++++++++++++++++++++++++
\label{AppSecTaylorR}

\begin{definition}      \label{DEFooCLGZooRuEkTe}
    Soit un ouvert \( \Omega\subset \eC\). Une fonction \( f\colon \Omega\to \eC\) est \defe{$\eC$-analytique}{analytique!au sens complexe} sur \( \Omega\) si pour tout \( z_0\in\Omega\), il existe une suite complexe \( (c_n)\) et \( r>0\) tels que
    \begin{equation}
        f(z)=\sum_{n=0}^{\infty} c_n(z-z_0)^n
    \end{equation}
    pour tout \( z\in B(z_0,r)\).
\end{definition}

\begin{definition}
    Soit une fonction \( f\colon \eR\to \eR\). Si il existe, nous définissons le \( n\)\ieme\ \defe{polynôme de Taylor}{polynôme de Taylor} de $f$ au point \( a\in \eR\) par
    \begin{equation}
        P_n(x)=\sum_{k=0}^n\frac{ f^{(k)}(a) }{ k! }(x-a)^k.
    \end{equation}
    Et la \defe{série de Taylor}{série de Taylor} de \( f\) est la limite :
    \begin{equation}
        T(x)=\sum_{k=0}^{\infty}\frac{ f^{(k)}(a) }{ k! }(x-a)^k
    \end{equation}
    dans la mesure où la somme converge.
\end{definition}

Tant que \( f\) est \( n\) fois dérivable, le polynôme \( P_n\) existe et vérifie \( P_n(a)=f(a)\). Nous ne pouvons rien en dire de plus pour l'instant. En particulier, si \( f\) est de classe \(  C^{\infty}\) il ne faudrait pas croire que
\begin{equation}
    \lim_{n\to \infty} P_n(x)=f(x)
\end{equation}
pour tout \( x\) dans un voisinage de \( a\). Autrement dit, même si toutes les dérivées de \( f\) existent, la série entière \( T\) n'est pas garantie de
\begin{itemize}
    \item 
        un rayon de convergence\footnote{Définition \ref{DefZWKOZOl}.} plus grand que zéro,
    \item
        et même avec un grand rayon de convergence, que la limite soit les valeurs de \( f\).
\end{itemize}

\begin{normaltext}      \label{NORMooADIZooUmevqk}
    Il n'est pas très compliqué de construire une fonction \( f\) telles que \( f(0)=0\) et telle que \( f^{(k)}(0)=0\) pour tout \( k\), sans pour autant que \( f\) soit nulle partout (voir les fonctions plateaux \ref{subsecOSYAooXXCVjv}). Les polynômes de Taylor d'une telle fonction sont tous identiquement nuls.

    Ceci pour dire qu'en posant
    \begin{equation}
        T(x)=\sum_{n=0}^{\infty}\frac{ f^{(k)}(0) }{ k! }x^k,
    \end{equation}
    nous n'avons aucune garantie de \( T=f\), même pas sur le rayon de convergence de la série entière définissant \( P\). Et nous n'avons pas de garanties d'avoir un rayon de convergence plus grand que \( 0\).

    Notons toutefois que les polynômes étant denses pour la norme supremum parmi les fonctions continues\footnote{Théorème \ref{ThoGddfas}.}, pour tout compact, il existe une suite de polynômes qui converge uniformément uniformément \( f\). Mais ces polynômes ne sont pas spécialement ceux de Taylor.
\end{normaltext}

\begin{normaltext}
    Ce que nous venons de dire en \ref{NORMooADIZooUmevqk} n'est pas vrai pour les fonctions analytiques\footnote{Définition \ref{DEFooCLGZooRuEkTe} qu'il faut écrire pour \( \eR\) au lieu de $\eC$}. Une fonction analytique \( f\colon \Omega\to \eR\) s'écrit, autour de \( 0\), sous la forme
    \begin{equation}
        f(x)=\sum_{n=0}^{\infty}a_nx^n.
    \end{equation}
    Demander \( f^{(k)}(0)=0\) pour tout \( k\) implique \( a_n=0\) pour tout \( n\), et donc \( f=0\) sur un voisinage de \( 0\).

    La condition d'analycité est donc très rigide.
\end{normaltext}

Le théorème de Taylor que nous démontrons à présent n'est pas un résultat que va dans le sens de \( \lim_{n\to \infty} P_n(x)=f(x)\). C'est un résultat qui dit juste que \( \lim_{x\to a} P_n(x)=f(a)\), et que la limite va d'autant plus vite que \( n\) est grand.

Le théorème de Taylor généralise le développement limité au premier ordre de la proposition \ref{PropUTenzfQ}.

\begin{normaltext}
    Lorsque le contexte n'est pas ambigu, nous notons simplement \( P_n\) le polynôme d'ordre \( n\) de \( f\) au point \( a\). De même nous notons le reste
    \begin{equation}
        R_n(x)=f(x)-P_n(x).
    \end{equation}
\end{normaltext}

\begin{proposition}[\cite{ooSZKEooLejXAh}]      \label{PROPooUYCMooQjeXpn}
    Soit une fonction \( f\) qui est \( n\) fois dérivable sur l'intervalle ouvert \( I\subset \eR\) contenant \( a\). Alors
    \begin{equation}
        \lim_{x\to a} \frac{ R_n(x) }{ (x-a)^n }=\lim_{x\to a} \frac{ f(x)-P_n(x) }{ (x-a)^n }=0
    \end{equation}
    où \( P_n\) est le \( n\)\ieme\ polynôme de Taylor de \( f\) autour de \( x=a\).
\end{proposition}

\begin{proof}
    Pour tout \( k=0,\ldots, n\) nous avons \( f^{(k)}(a)=P_n^{(k)}(a)\) et donc
    \begin{equation}
        R_n^{(k)}(a)=0
    \end{equation}
    pour \( k=0,\ldots, n\). En posant d'autre par \( s(x)=(x-a)^n\) nous avons \( s^{(k)}(a)=0\) pour tout \( k=0,\ldots, n-1\). Par conséquent la règle de l'Hospital de la proposition \ref{PROPooBZHTooHmyGsy} s'applique au quotient \( R_n(x)/s(x)^n\). En l'utilisant \( n\) fois,
    \begin{equation}
        \lim_{x\to a} \frac{ R_n(x) }{ s(x) }=\lim_{x\to a} \frac{ R^{(k)}(x) }{ k!(x-a)^0 }=\frac{ 0 }{ k! }=0.
    \end{equation}
\end{proof}

Nous démontrons à présent que le polynôme de Taylor est le seul à avoir la propriété de la proposition \ref{PROPooUYCMooQjeXpn}.
\begin{proposition}[\cite{ooSZKEooLejXAh}]
    Soit \( f\), une fonction \( n\) fois dérivable sur l'intervalle \( I\subset \eR\) contenant \( 0\). Soit un polynôme \( Q\) de degré \( n\) (ou moins) tel que
    \begin{equation}    \label{EQooXPTIooOZqBaD}
        \lim_{x\to a} \frac{ f(x)-Q(x) }{ (x-a)^n  }=0.
    \end{equation}
    Alors \( Q\) est le polynôme de Taylor de degré \( n\) pour \( f\) en \( a\) ci-après simplement noté \( P_n\).
\end{proposition}

\begin{proof}
    D'après la proposition \ref{PROPooUYCMooQjeXpn}, la fonction \( f-P_n\) vérifie la même limite que \( f-Q\). DOnc \( P_n-Q\) vérifie également la limite
    \begin{equation}        \label{EQooYBHHooEgZxtk}
        \lim_{x\to 0} \frac{ (P_n-Q)(x) }{ x^n }=0.
    \end{equation}
    Nous notons \( P_n(x)=\sum_{k=0}^na_kx^k\) et \( Q(x)=\sum_{k=0}^nb_kx^k\). La relation \eqref{EQooYBHHooEgZxtk} donne en particulier 
    \begin{equation}
        \lim_{x\to 0} (P_n-Q)(x)=0
    \end{equation}
    qui donne \( a_0-b_0=0\). Nous continuons par récurrence en supposant que \( a_i=b_i\) pour \( i=0,\ldots, k\). Alors
    \begin{equation}
        0=\lim_{x\to 0} \frac{ (P_n-Q)(x) }{ x^{k+1} }=\lim_{x\to 0} \sum_{l=k+1}^n(a_l-b_l)x^{l-(k+1)}.
    \end{equation}
    Le seul terme non nul à droite est celui vérifiant \( l-(k+1)=0\). Et ce terme donne l'équation
    \begin{equation}
        a_{k+1}-b_{k+1}=0,
    \end{equation}
    c'est-à-dire \( a_{k+1}=b_{k+1}\). La récurrence continue ainsi jusqu'à \( k=n\), et nous pouvons conclure que \( Q=P_n\).
\end{proof}
L'intérêt de cette proposition est que si l'on trouve, par n'importe quel moyen, un polynôme \( Q\) vérifiant la condition \eqref{EQooXPTIooOZqBaD}, alors nous savons que c'est le polynôme de Taylor.

\begin{theorem}[Théorème de Taylor\cite{TrenchRealAnalisys,ooCNZAooJEcgHZ}]		\label{ThoTaylor}
Soit $I\subset$ un intervalle non vide et non réduit à un point de $\eR$ ainsi que $a\in I$. Soit une fonction $f\colon I\to \eR$ telle que $f^{(n)}(a)$ existe. Alors il existe une fonction $\alpha$ définie sur $I$ et à valeurs dans $\eR$ vérifiant les deux conditions suivantes :
\begin{subequations}		\label{SubEqsDevTauil}
	\begin{align}
		f(x)&= \sum_{k=0}^n\frac{ f^{(k)}(a) }{ k! }(x-a)^k +\alpha(x)(x-a)^{n},	\\	\label{subeqfTepseqb}
		\lim_{t\to a}\alpha(t)&=0
	\end{align}
\end{subequations}
pour tout \( x\in I\). Ici $f^{(k)}$ dénote la $k$-ième dérivée de $f$ (en particulier, $f^{(0)}=f$, $f^{(1)}=f'$).\nomenclature{$f^{(n)}$}{La $n$-ième dérivée de la fonction $f$}
\end{theorem}

\begin{proof}
    Si \( R_n(x)=f(x)-P_n(x)\), il suffit de poser
    \begin{equation}
        \alpha(x)=R_n(x)(x-a)^n
    \end{equation}
    et d'utiliser la proposition \ref{PROPooUYCMooQjeXpn}.
\end{proof}

\begin{remark}
    Quelque remarques.
    \begin{enumerate}
        \item
            La formule \eqref{subeqfTepseqb} est une égalité, et non une approximation. Ce qui serait une approximation serait de récrire la formule dans le terme contenant $\alpha$.
        \item
            Nous avons l'égalité \eqref{subeqfTepseqb} uniquement sur \( I\). Pour les \( x\) hors de \( I\), le polynôme existe évidemment, mais nous n'avons pas spécialement de fonction \( \alpha\), et d'ailleurs la fonction \( f\) n'est pas spécialement définie.
    \end{enumerate}
\end{remark}

\begin{normaltext}
    Les conditions \eqref{SubEqsDevTauil} sont souvent aussi énoncées sous la forme qu'il existe une fonction \( \alpha\) telle que
    \begin{subequations}    \label{SUBEQooPYABooKpDgdu}
        \begin{numcases}{}
            \lim_{t\to 0} \frac{ \alpha(t) }{ t^n }=0\\
            f(a+h)=f(a)+hf'(a)+\frac{ h^2 }{2}f''(a)+\cdots+ \frac{ h^n }{ n! }f^{(n)}(a)+\alpha(h).
        \end{numcases}
    \end{subequations}
\end{normaltext}

Le théorème suivant donne une expression pas tout à fait explicite, mais pas mal quand même pour le reste de Taylor.
\begin{theorem}     \label{THOooSIGRooJTLvlV}
Soient un intervalle ouvert \( I\subset \eR\) ainsi que \( a\in I\). Soit encore une fonction de classe \( C^{k+1}\) sur $I$. Pour tout \( x\in I\), il existe un \( c\in \mathopen] a , x \mathclose[\) tel que l'égalité
    \begin{equation}        \label{EQooQFMFooBVpGzy}
        f(x)=\sum_{k=0}^n\frac{ f^{(k)}(a) }{k!}(x-a)^k+\frac{ f^{(n+1)}(c) }{ (n+1)! }(x-a)^{n+1}
    \end{equation}
    soit vérifiée.
\end{theorem}

\begin{proof}
    Pour les besoins de la preuve, nous allons démontrer la formule \eqref{EQooQFMFooBVpGzy} pour un \( b\in I\) au lieu de \( x\). C'est juste que nous allons écrire \( b\) au lieu de \( x\) parce que nous aurons besoin de la notation \( x\) dans le courant de la preuve.

    Nous posons
    \begin{equation}
        R(x)=f(x)-P_n(x)=f(x)-\sum_{k=0}^n\frac{ f^{(k)}(a) }{ k! }(x-a)^k.
    \end{equation}
    Cela vérifie \( R(a)=f(a)-f(a)=0\) et même
    \begin{equation}
        R^{(j)}(a)=0
    \end{equation}
    pour tout \( j=1,\ldots, n\). Nous posons encore
    \begin{equation}
        F(x)=R(x)-\frac{ R(b) }{ (b-a)^{n+1} }(x-a)^{n+1}.
    \end{equation}
    Nous avons \( F^{(j)}(a)=0\) pour tout \( j=0,\ldots, n\) ainsi que
    \begin{equation}
        F(b)=R(b)-\frac{ R(b) }{ (b-a)^{n+1} }(b-a)^{n+1}=0.
    \end{equation}
    et aussi
    \begin{equation}
        F(a)=R(a)-0=0.
    \end{equation}
Bref, la fonction \( F\) vérifie les conditions de la généralisation \ref{PROPooCPCAooJjOZNy} du lemme de Rolle. Il existe donc \( c\in\mathopen] a , b \mathclose[\) tel que \( F^{(n+1)}(c)=0\). Mais vu que \( P_n^{(n+1)}(x)=0\), nous avons \( R^{(n+1)}(x)=f^{(n+1)}(x)\), de telle sorte que
    \begin{equation}
        f^{(n+1)}(c)=\frac{ R(b)(n+1)! }{ (b-a)^{n+1} }.
    \end{equation}
    En injectant cela dans la définition de \( F\)
    \begin{equation}
        F(x)=R(x)-\frac{ f^{(n+1)}(c) }{ (n+1)! }(x-a)^{n+1}.
    \end{equation}
    En évaluant en \( x=b\), et en nous souvenant que \( F(b)=0\), nous trouvons
    \begin{equation}
        0=R(b)-\frac{ f^{(n+1)}(c) }{ (n+1)! }(b-a)^{n+1}
    \end{equation}
    qui est ce que nous voulions prouver.
\end{proof}

Voici un énoncé pour les fonctions à plusieurs variables.
\begin{theorem}[\cite{ooLZSZooIOILHY}]      \label{THOooTDFRooEkChgi}
    Si \( f\colon E\to \eR\) est une application \( n\) fois différentiable en \( a\in E\) alors il existe une fonction \( \epsilon\colon \eR\to \eR\) telle que
    \begin{subequations}
        \begin{numcases}{}
            f(a+h)=f(a)+df_a(h)+\frac{ 1 }{2}(d^2f)_a(h,h)+\ldots +\\
            \qquad+\ldots+\frac{1}{ n! }(d^nf)_a(h,\ldots, h)+\| h \|^n\epsilon(\| h \|)\\
            \lim_{t\to 0} \epsilon(t)=0.
        \end{numcases}
    \end{subequations}
\end{theorem}

%---------------------------------------------------------------------------------------------------------------------------
\subsection{Fonctions «petit o» }
%---------------------------------------------------------------------------------------------------------------------------

Nous voulons formaliser l'idée d'une fonction qui tend vers zéro « plus vite » qu'une autre. Nous disons que $f\in o\big(\varphi(x)\big)$ si
\begin{equation}
    \lim_{x\to 0} \frac{ f(x) }{ \varphi(x) }=0.
\end{equation}
En particulier, nous disons que $f\in o(x)$ lorsque $\lim_{x\to 0} f(x)/x=0$.


En termes de notations, nous définissons l'ensemble $o(x)$\nomenclature{$o(x)$}{fonction tendant rapidement vers zéro} l'ensemble des fonctions $f$ telles que
\begin{equation}
	\lim_{x\to 0} \frac{ f(x) }{ x }=0.
\end{equation}
Plus généralement si $g$ est une fonction telle que $\lim_{x\to 0} g(x)=0$, nous disons $f\in o(g)$ si
\begin{equation}
	\lim_{x\to 0} \frac{ f(x) }{ g(x) }=0.
\end{equation}
De façon intuitive, l'ensemble $o(g)$ est l'ensemble des fonctions qui tendent vers zéro «plus vite» que $g$.

Nous pouvons donner un énoncé alternatif au théorème~\ref{ThoTaylor} en définissant $h(x)=\epsilon(x+a)x^n$. Cette fonction est définie exprès pour avoir
\begin{equation}
	h(x-a)=\epsilon(x)(x-a)^n,
\end{equation}
et donc
\begin{equation}
	\lim_{x\to 0} \frac{ h(x) }{ x^n }=\lim_{x\to 0} \epsilon(x-a)=\lim_{x\to a}\epsilon(x)=0.
\end{equation}
Donc $h\in o(x^n)$.

Le théorème dit donc qu'il existe une fonction $\alpha\in o(x^n)$ telle que
\begin{equation}
	f(x)=T^a_{f,n}(x)+\alpha(x-a).
\end{equation}
pour tout $x\in I$.

\begin{remark}
    À titre personnel, l'auteur de ces lignes déconseille d'utiliser cette notation qui est un peu casse-figure pour qui ne la maîtrise pas bien.
\end{remark}

\begin{example}
    Le développement en série du cosinus sera traité dans la proposition \ref{PROPooNPYXooTuwAHP}.
\end{example}

\begin{proposition}[Ordre deux sur \( \eR^n\)\cite{MonCerveau}]         \label{PROPooTOXIooMMlghF}
    Soit un ouvert \( \Omega\) de \( \eR^n\) et \( a\in \Omega\) ainsi qu'une fonction \( f\colon \Omega\to \eR\) de classe \( C^2\). Alors il existe une fonction \( \alpha\colon \eR^n\to \eR\) telle que
    \begin{subequations}
        \begin{numcases}{}
            f(a+h)=f(a)+df_a(h)+\frac{ 1 }{2}(d^2f)_a(h,h)+\| h \|^2\alpha(h)\\
            \lim_{h\to 0} \alpha(h)=0.
        \end{numcases}
    \end{subequations}
    Ici, la notation \( (d^2f)_a(h,h)\) réfère à ce qui est expliqué en~\ref{NORMooZAOEooGqjpLH}.
\end{proposition}

\begin{proof}
    Dans la suite nous considérons \( t\) et \( h\) tels que toutes les expressions suivantes aient un sens, c'est-à-dire que tous les trucs comme \( a+th\) restent dans \( \Omega\). Pour \( h\in \eR^n\) nous nommons \( e_h\) le vecteur unitaire dans la direction de \( h\), c'est-à-dire \( e_h=h/\| h \|\) et nous posons
    \begin{equation}
        k_h(t)=f(a+te_h).
    \end{equation}
    et nous lui appliquons Taylor~\ref{ThoTaylor} à l'ordre deux : il existe une fonction \( \beta_h\) telle que
    \begin{equation}        \label{EQooETDFooAmiRcV}
        k_h(x)=k_h(0)+xk_h'(0)+\frac{ x^2 }{2}k''_h(0)+x^2\beta_h(x).
    \end{equation}
    avec \( \lim_{x\to 0} \beta_h(x)=0\).

    En ce qui concerne les dérivées de \( k_h\) nous avons
    \begin{equation}
        k'_h(0)=df_a(e_h)
    \end{equation}
    et
    \begin{equation}
        k_h''(0)=(d^2f)_{a}(e_h,e_h).
    \end{equation}
    Il est maintenant temps d'écrire \( f(a+h)=k(\| h \|)\) et de substituer les dérivées de \( k\) par les différentielles de \( f\) dans \eqref{EQooETDFooAmiRcV} :
    \begin{equation}        \label{EQooUSUGooYPscxV}
            f(a+h)=k(\| h \|)=f(a)+df_a(h)+\frac{ 1 }{2}(d^2f)_a(h,h)+\| h^2 \|\beta_{h}(\| h \|).
    \end{equation}
    Il reste à voir que la fonction \( \alpha\colon h\mapsto \beta_h(\| h \|)\) tend vers zéro pour \( h\to 0\). En prenant la limite \( h\to 0\) dans \eqref{EQooUSUGooYPscxV}, il est manifeste que la limite du membre de gauche existe et vaut \( f(a)\). Donc la limite du membre de droite doit exister et valoir également \( f(a)\). Nous en déduisons que la limite de
    \begin{equation}
        df_a(h)+\frac{ 1 }{2}(d^2f)_a(h,h)+\| h \|^2\beta_h(\| h \|)
    \end{equation}
    existe et vaut zéro. La limite des deux premiers termes existe et vaut zéro, donc la limite du troisième existe et vaut zéro :
    \begin{equation}
        \lim_{h\to 0} \| h \|^2\beta_h(\| h \|)=0.
    \end{equation}
\end{proof}

\begin{proposition}     \label{PROPooWWMYooPOmSds}
Soit un ouvert \( \Omega\) de \( \eR^n\) et une fonction \( f\colon \Omega\to \eR\) de classe \( C^1\) et deux fois différentiable sur \( \mathopen] x , x+h \mathclose[\). Alors il existe \( \theta\in \mathopen] 0 , 1 \mathclose[\) tel que
    \begin{equation}
        f(x+h)=f(x)+df_x(h)+\frac{ 1 }{2}(d^2f)_{x+\theta h}(h,h).
    \end{equation}
\end{proposition}

%---------------------------------------------------------------------------------------------------------------------------
\subsection{Autres formulations}
%---------------------------------------------------------------------------------------------------------------------------

\begin{example}		\label{ExempleUtlDev}
	Une des façons les plus courantes d'utiliser les formules \eqref{SubEqsDevTauil} est de développer $f(a+t)$ pour des petits $t$ en posant $x=a+t$ dans la formule :
	\begin{equation}	\label{EqDevfautouraeps}
		f(a+t)=f(a)+f'(a)t+f''(a)\frac{ t^2 }{ 2 }+\epsilon(a+t)t^2
	\end{equation}
	avec $\lim_{t\to 0} \epsilon(a+t)=0$. Ici, la fonction $T$ dont on parle dans le théorème est $T_{f,2}^a(a+t)=f(a)+f'(a)t+f''(a)\frac{ t^2 }{2}$.

	Lorsque $x$ et $y$ sont deux nombres «proches\footnote{par exemple dans une limite $(x,y)\to(h,h)$.}», nous pouvons développer $f(y)$ autour de $f(x)$ :
	\begin{equation}		\label{Eqfydevfx}
		f(y)=f(x)+f'(x)(y-x)+f''(x)\frac{ (y-x)^2 }{ 2 }+\epsilon(y-x)(y-x)^2,
	\end{equation}
	et donc écrire
	\begin{equation}
		f(x)-f(y)=-f'(x)(y-x)-f''(x)\frac{ (y-x)^2 }{ 2 }-\epsilon(y-x)(y-x)^2.
	\end{equation}
	De cette manière nous obtenons une formule qui ne contient plus que $y$ dans la différence $y-x$.
\end{example}

%---------------------------------------------------------------------------------------------------------------------------
\subsection{Formule et reste}
%---------------------------------------------------------------------------------------------------------------------------

\begin{proposition}     \label{PropDevTaylorPol}
    Soient $f\colon I\subset\eR\to \eR$ et $a\in\Int(I)$. Soit un entier $k\geq 1$. Si $f$ est $k$ fois dérivable en $a$, alors il existe un et un seul polynôme $P$ de degré $\leq k$ tel que
    \begin{equation}
        f(x)-P(x-a)\in o\big( | x-a |^k \big)
    \end{equation}
    lorsque $x\to a$, $x\neq a$. Ce polynôme  est donné par
    \begin{equation}
        P(h)=f(a)+f'(a)h+\frac{ f''(a) }{ 2! }h^2+\cdots+\frac{ f^{(k)}(a) }{ k! }h^k.
    \end{equation}
    Notons encore deux façons alternatives d'écrire le résultat. Si \( f\in C^k\) il existe une fonction \( \alpha\) telle que \( \lim_{t\to 0} \alpha(t)=0\) et
    \begin{equation}
        f(x)=\sum_{n=0}^k\frac{ f^{(n)}(a) }{ n! }(x-a)^n+(x-a)^n\alpha(x-a).
    \end{equation}
    Si \( f\in C^{k+1}\) alors
    \begin{equation}        \label{EquQtpoN}
        f(x)=\sum_{n=0}^k\frac{ f^{(n)}(a) }{ n! }(x-a)^n+(x-a)^{n+1}\xi(x-a)
    \end{equation}
    où \( \xi\) est une fonction telle que \( \xi(t)\) tend vers une constante lorsque \( t\to 0\).
\end{proposition}

La proposition suivant donne une intéressante façon de trouver le reste d'un développement de Taylor.
\begin{proposition}     \label{PropResteTaylorc}
Soient $I$, un intervalle dans $\eR$ et $f\colon I\to \eR$ une fonction de classe $C^k$ sur $I$ telle que $f^{(k+1)}$ existe sur $I$. Soient $a\in\Int(I)$ et $x\in I$. Alors il existe $c\in\mathopen] x , a \mathclose[$ tel que
\begin{equation}
    f(x)=\sum_{k=0}^n\frac{ f^{(k)}(a) }{ k! }(x-a)^k+\frac{ f^{(n+1)}(c) }{ (n+1)! }(x-a)^{n+1}.
\end{equation}
\end{proposition}

%---------------------------------------------------------------------------------------------------------------------------
\subsection{Reste intégral}
%---------------------------------------------------------------------------------------------------------------------------

Comme son nom l'indique, le «reste intégral» demande de savoir les intégrales. La formule du reste intégral sera donc pour après la définition des intégrales, proposition~\ref{PropAXaSClx}.

%+++++++++++++++++++++++++++++++++++++++++++++++++++++++++++++++++++++++++++++++++++++++++++++++++++++++++++++++++++++++++++
\section{Développement limité autour de zéro}
%+++++++++++++++++++++++++++++++++++++++++++++++++++++++++++++++++++++++++++++++++++++++++++++++++++++++++++++++++++++++++++

Dans cette sections nous supposons toujours que les fonctions sont définies sur un intervalle ouvert de $\eR$, $I$, contenant \( 0\).

%---------------------------------------------------------------------------------------------------------------------------
\subsection{Généralités}
%---------------------------------------------------------------------------------------------------------------------------

\begin{definition}
    Soit \( f\colon I\to 0\) une fonction définie sur un ouvert \( I\) autour de zéro. Nous disons que \( f\) admet un \defe{développement limité}{développement!limité!en zéro} autour de \( 0\) à l'ordre \( n\) s'il existe une fonction \( \alpha\colon I\to \eR\) telle que
    \begin{subequations}
        \begin{numcases}{}
            f(x)=P_n(x)+x^n\alpha(x)\\
            \lim_{x\to 0} \alpha(x)=0
        \end{numcases}
    \end{subequations}
    où \( P(x)=a_0+a_1x+\cdots +a_nx^n\) est une polynôme de degré \( n\). Le polynôme \( P_n\) est appelé la \defe{partie régulière}{partie!régulière} du développement.
\end{definition}
La fonction \( \alpha\) est appelé le \defe{reste}{reste!d'un développement limité} du développement et sera parfois noté \( \alpha_f\). Lorsque \( P\) est la partie régulière d'un développement limité de \( f\) nous notons parfois \( f\sim P\).

\begin{proposition}[Troncature]
    Si \( f\) admet un développement limité d'ordre \( n\) alors il admet également un développement limité d'ordre \( n'\) pour tout \( n'<n\). Ce dernier s'obtient en tronquant le polynôme d'ordre \( n\) à l'ordre \( n'\).
\end{proposition}

\begin{proposition}[Unicité]
    Si \( f\) admet une développement limité alors ce dernier est unique : il existe un unique polynôme \( P_n\) d'ordre \( n\) et une unique fonction \( \alpha\) vérifiant simultanément les deux conditions
    \begin{subequations}
        \begin{numcases}{}
            f(x)=P_n(x)+x^n\alpha(x),\\
            \lim_{x\to 0} \alpha(x)=0.
        \end{numcases}
    \end{subequations}
\end{proposition}

\begin{example} \label{ExTHGooCBcnAy}
    En ce qui concerne les séries géométriques de raison \( x\) nous savons les formules
    \begin{equation}
        1+x+x^2+\cdots +x^n=\frac{ 1-x^{n+1} }{ 1-x }
    \end{equation}
    et
    \begin{equation}
        1+x+x^2+x^3+\cdots=\frac{ 1 }{ 1-x }
    \end{equation}
    pour tout \( x\in\mathopen] -\infty , 1 \mathclose[\). Comparant les deux, il est naturel d'essayer de prendre \( 1+x+x^2+\cdots +x^n\) comme développement limité de la fonction \( f(x)=\frac{1}{ 1-x }\). Pour voir si cela fonctionne, il faut vérifier si «le reste» est bien de la forme \( x^n\alpha(x)\) avec \( \lim_{x\to 0} \alpha(x)=0\).

    Le reste en question est donné par
    \begin{equation}
        \frac{1}{ 1-x }-1-x-x^2-\ldots-x^n=\frac{1}{ 1-x }-\frac{ 1-x^{n+1} }{ 1-x }=\frac{ x^{n+1} }{ 1-x }=x^n\frac{ x }{ 1-x }.
    \end{equation}
    En posant \( \alpha(x)=\frac{ x }{ 1-x }\) nous avons donc bien
    \begin{equation}
        f(x)=\frac{1}{ 1-x }=1+x+x^2+\cdots +x^n+x^n\frac{ x }{ 1-x }
    \end{equation}
    et \( \lim_{x\to 0} \frac{ x }{ 1-x }=0\). Cela est le développement limité de \( f\) à l'ordre \( n\) autour de \( 0\).
\end{example}

La formule des accroissements finis est un cas particulier de développement fini. Supposons que \( f\) soit dérivable en \( 0\). En effet nous pouvons facilement trouver la fonction \( \alpha\) qui convient. Sachant que \( f(0)+xf'(0)\) donne l'approximation affine de \( f\) autour de \( 0\), nous cherchons \( \alpha\) en écrivant
\begin{equation}
    f(x)=f(0)+xf'(0)+x\alpha(x).
\end{equation}
Cela nous pousse à définir
\begin{equation}    \label{EqDCFooKozKrt}
    \alpha(x)=\frac{ f(x)-f(0) }{ x }-f'(0).
\end{equation}
Notons que cette fonction n'est pas définie en \( x=0\), mais cela n'a pas d'importance : seule la limite \( \lim_{x\to 0} \alpha(x)\) nous intéresse. Par définition de la dérivée,
\begin{equation}
    \lim_{x\to 0} \alpha(x)=\lim_{x\to 0} \frac{ f(x)-f(0) }{ x }-f'(0)=0.
\end{equation}

En conclusion si \( f\) est dérivable, son développement limité à l'ordre \(  1\) est donné par
\begin{equation}
    f(x)=f(0)+xf'(0)+x\alpha(x)
\end{equation}
où \( \alpha(x)\) est donnée par la formule \eqref{EqDCFooKozKrt}.

%---------------------------------------------------------------------------------------------------------------------------
\subsection{Formule de Taylor-Young}
%---------------------------------------------------------------------------------------------------------------------------

Plus généralement nous avons la proposition suivante qui donne le développement limité de toute fonction dérivable \( n\) fois.

\begin{proposition}[Formule de Taylor-Young]    \label{PropVDGooCexFwy}
    Soit \( f\) une fonction \( n\) fois dérivable sur un intervalle \( I\) contenant \( 0\). Alors il existe une fonction \( \alpha\colon I\to \eR\) telle que
    \begin{equation}        \label{EQooBKZDooTqYyIB}
        f(x)=f(0)+f'(0)x+\frac{ f''(0) }{ 2 }x^2+\frac{ f^{(3)}(0) }{ 3! }x^3+\cdots +\frac{ f^{(n)}(0) }{ n! }x^n+x^n\alpha(x)
    \end{equation}
    et
    \begin{equation}
        \lim_{x\to 0} \alpha(x)=0.
    \end{equation}
\end{proposition}

Cette proposition nous permet de calculer facilement des développements limités tant que nous sommes capables de calculer les dérivées successives de la fonction à développer. Dans l'exemple \ref{ExTHGooCBcnAy} nous avons dû utiliser des astuces et des formules pour déterminer le développement limité de \( \frac{1}{ 1-x }\). Au contraire la formule \eqref{EQooBKZDooTqYyIB} nous permet de trouver le polynôme en appliquant mécaniquement une formule simple.

\begin{example}
    Utilisation de la formule \eqref{EQooBKZDooTqYyIB} pour déterminer le développement limité de la fonction
    \begin{equation}
        f(x)=\frac{1}{ 1-x }.
    \end{equation}
    Il faut calculer les dérivées successives de \( f\) :
    \begin{subequations}
        \begin{align}
            f(x)&=\frac{1}{ 1-x }\\
            f'(x)&=\frac{ 1 }{ (1-x)^2 }\\
            f''(x)&=\frac{ 2 }{ (1-x)^3 }
        \end{align}
    \end{subequations}
    Avec ces résultats, nous devinons que
    \begin{equation}
        f^{(n)}(x)=\frac{ n! }{ (1-x)^{n+1} }.
    \end{equation}
    Pour en être sûr nous le prouvons par récurrence. La dérivée de \(\frac{ n! }{ (1-x)^{n+1} } \) est donnée par
    \begin{equation}
        \frac{ n!(n+1)(1-x)^n }{ (1-x)^{2n+2} }=\frac{(n+1)! }{ (1-x)^{n+2} }.
    \end{equation}
    Évaluées en \( x=0\), les dérivées successives de \( f\) sont \( f(0)=0\), \( f'(0)=1\), \( f''(0)=2\),\ldots,\( f^{(n)}(0)=n!\). Utilisant la formule \eqref{EQooBKZDooTqYyIB} nous avons
    \begin{equation}
        f(x)=1+x+x^2+\cdots +x^n+x^n\alpha(x),
    \end{equation}
    conformément à ce que nous avions déjà trouvé.
\end{example}

\begin{example}     \label{EXooFLBJooYfuRsG}
    Soient \( r\in \eQ\) et la fonction donnée par
    \begin{equation}
        f(x)=(1+x)^r.
    \end{equation}
    Nous notons \( I\) le domaine de cette fonction : c'est \( \eR\) si \( r>0\) ou \( \mathopen[ -1 , \infty \mathclose]\) si \( r<0\). Si par contre \( r=0\), la fonction est constante et le domaine est \( I=\eR\).

    En ce qui concerne les dérivées\footnote{Nous utilisons la proposition \ref{PROPooSGLGooIgzque}.} : \( f'(x)=r(1+x)^{r-1}\) et plus généralement
    \begin{equation}
        f^{(k)}(x)=r(r-1)\ldots (r-k+1)(1+x)^{r-k}
    \end{equation}
    si \( k>0\). Pour \( k=0\) nous avons \( f^{(k)}(0)=1\). Le développement de Taylor-Young est alors
    \begin{equation}
      (1+x)^r=1+\sum_{k=1}^n\frac{r(r-1)\ldots (r-k+1)}{ k! }x^k+x^n\alpha(x).
    \end{equation}
    
    Notons que que si \( r\) est un entier, pour \( k=r\), le produit au numérateur s'annule et le développement s'arrête. 
    
    Dans le développement de \( (1+x)^{r}\), nous reconnaissons la formule de \( \binom{ k }{r}\), sauf que nous ne pouvons pas l'écrire avec cette notation lorsque \( r\) n'est pas entier.
\end{example}
Cet exemple fonctionnera encore avec \( r\in \eR\) au lieu de \( r\in \eQ\), mais il faudra la proposition \ref{PROPooKUULooKSEULJ} pour la dérivée

\begin{remark}
  Pour alléger la notation et ne pas écrire \(\ldots +x^n\alpha(x)\) nous pouvons aussi écrire
    \begin{equation}
         f(x)\sim 1+x+x^2+\cdots +x^n,
    \end{equation}
    mais il est interdit d'écrire
    \begin{equation}
         f(x)= 1+x+x^2+\cdots +x^n
    \end{equation}
    en mettant un signe d'égalité entre une fonction et son développement limité\footnote{Il faut cependant être très prudents avec la notation abrégée. Elle pourrait nous faire oublier des informations importantes, voir les développements des fonctions trigonométriques pour un exemple.}.
\end{remark}

Notons cependant que la proposition~\ref{PropVDGooCexFwy} ne donne pas de moyen simple de trouver la fonction \( \alpha\). Si la fonction $f$ est très régulière dans l'intervalle $I$ on a le résultat suivant.

\begin{proposition}[Reste dans la forme de Lagrange]
    Si la fonction $f$ est dérivable $n+1$ fois dans $I$ alors il existe $\bar x$ dans l'intervalle \( \mathopen[ 0 , x \mathclose]\) tel que
  \begin{equation}
    f(x) = P_n(x) + \frac{1}{(n+1)!} f^{n+1}(\bar x) x^{n+1}.
  \end{equation}
\end{proposition}

%---------------------------------------------------------------------------------------------------------------------------
\subsection{Règles de calcul}
%---------------------------------------------------------------------------------------------------------------------------

    Les règles suivantes permettent de calculer les développements limités des fonctions qu'on peut écrire comme combinaison de fonctions dont nous savons déjà le développement.

    Il est toujours possible de calculer le développement limité d'une fonction par la formule de Taylor-Young (proposition \ref{PropVDGooCexFwy}). Les règles suivantes peuvent nous economiser de l'effort et du temps.

%///////////////////////////////////////////////////////////////////////////////////////////////////////////////////////////
\subsubsection{Linéarité des développements limités}
%///////////////////////////////////////////////////////////////////////////////////////////////////////////////////////////

L'opération qui consiste à prendre le développement limité d'une fonction est une opération linéaire : connaissant les développements limités de \( f\) et de \( g\), il suffit de les sommer pour obtenir celui de \( f+g\). De m\^eme, si $\lambda$ est une constante, le développement limité de $\lambda f$ est le développement limité de $f$ fois $\lambda$.
\begin{proposition}
Soient $\lambda$ et $\mu$ dans $\eR$.  Si \( f\) et \( g\) sont deux fonctions acceptant des développements limités d'ordre \( n\)
    \begin{subequations}    \label{EqJPPooCHihNn}
        \begin{align}
            f(x)&=P(x)+x^n\alpha_f(x)\\
            g(x)&=Q(x)+x^n\beta(x)
        \end{align}
    \end{subequations}
    avec \( \lim_{x\to 0} \alpha(x)=\lim_{x\to 0} \beta(x)=0\), alors la fonction \( \lambda f+\mu g\) admet le développement limité
    \begin{equation}    \label{EqCJFooVpyCtz}
        (f+g)(x)=(\lambda P+\mu Q)(x)+(\lambda \alpha+\mu\beta)(x).
    \end{equation}
\end{proposition}
\begin{remark}
  La forme explicite du reste ne nous interesse pas. Dans la pratique on écrira toujours $(f+g)(x)=(P+Q)(x)+\alpha(x)$, où on appelle $\alpha$ une fonction apportune telle que $\lim_{x\to 0} \alpha(x)=0$.
\end{remark}
\begin{proof}
    Vu les définitions \eqref{EqJPPooCHihNn} des polynômes \( P\), \( Q\) et des restes \( \alpha\) et \( \beta\), l'égalité \eqref{EqCJFooVpyCtz} est une conséquence de la linéarité de la dérivation et de la proposition~\ref{PropVDGooCexFwy}

    De plus \( P+Q\) est un polynôme de degré \( n\) dès que \( P\) et \( Q\) sont des polynômes de degré \( n\), et
    \begin{equation}
        \lim_{x\to 0} (\lambda \alpha+\mu\beta)(x)=\lim_{x\to 0} \lambda\alpha(x)+\lim_{x\to 0} \mu\beta(x)=0.
    \end{equation}
    Par conséquent \(\lambda \alpha+\mu\beta\) est la fonction de reste de \( \lambda f+\mu g\).
\end{proof}

\begin{example} \label{ExKPBooJmdFvY}
    Calculer le développement de la fonction
    \begin{equation}
        f(x)=3\sqrt[3]{1+x}+ e^{-2x}.
    \end{equation}
    Le développement de \( \sqrt[3]{1+x}\) est donné par la formule de l'exemple \ref{EXooFLBJooYfuRsG} avec \( \alpha=\frac{1}{ 3 }\). Nous avons donc dans un premier temps
    \begin{subequations}
        \begin{align}
            \sqrt[3]{1+x}&=1+\frac{ 1 }{ 3 }x+\frac{ \frac{1}{ 3 }\left( \frac{1}{ 3 }-1 \right) }{ 2 }x^2+\frac{ \frac{1}{ 3 }\left( \frac{1}{ 3 }-1 \right)\left( \frac{1}{ 3 }-2 \right) }{ 6 }x^3+x^3\alpha(x)\\
            &=1+\frac{1}{ 3 }x-\frac{1}{ 9 }x^2+\frac{ 5 }{ 81 }x^3+x^3\alpha(x).
        \end{align}
    \end{subequations}
    Nous avons alors
    \begin{subequations}
        \begin{align}
            3\sqrt[3]{1+x}+ e^{-2x}&=3\Big[  1+\frac{1}{ 3 }x-\frac{1}{ 9 }x^2+\frac{ 5 }{ 81 }x^3+x^3\alpha(x)\Big]+1-2x+2x^2-\frac{ 4 }{ 3 }x^3+x^3\beta(x)\\
            &=4-x+\frac{ 5 }{ 3 }x^2-\frac{ 31 }{ 27 }x^3+x^3\big( \alpha(x)+\beta(x) \big).
        \end{align}
    \end{subequations}

\end{example}

La condition \( \lim_{x\to 0} \alpha(x)=0\) signifie que l'approximation qui consiste à remplacer \( f(x) \) par le polynôme n'est pas une trop mauvaise approximation lorsque \( x\) est petit. Cela ne signifie rien de plus. En particulier si \( x\) est grand, l'approximation polynomiale peut-être (et est souvent) très mauvaise.

À ce propos, notez qu'un polynôme tend toujours vers \( \pm\infty\) lorsque \( x\) est grand. Une approximation polynomiale d'une fonction bornée est donc toujours (très) mauvaise pour les grandes valeurs de \( x\).

À titre d'exemple nous avons tracé sur la figure~\ref{LabelFigWUYooCISzeB} la fonction
\begin{equation}
    f(x)=3\sqrt[3]{x+1}+ e^{-2x}
\end{equation}
et ses développements limités d'ordre \( 1\) à \( 3\). Il est particulièrement visible que l'approximation est assez bonne pour la partie gauche du graphe sur laquelle la fonction est bien croissante, alors qu'elle est franchement mauvaise sur la droite où le graphe ressemble plutôt à une constante\footnote{Pouvez-vous cependant dire que vaut \( \lim_{x\to \infty} f(x)\) ?}.

\newcommand{\CaptionFigWUYooCISzeB}{Les développements limités d'ordre de plus en plus grand de la fonction de l'exemple~\ref{ExKPBooJmdFvY}. La fonction est en bleu et les «approximations» sont en rouge.}
\input{auto/pictures_tex/Fig_WUYooCISzeB.pstricks}

%///////////////////////////////////////////////////////////////////////////////////////////////////////////////////////////
\subsubsection{Développement limité d'un quotient}
%///////////////////////////////////////////////////////////////////////////////////////////////////////////////////////////

\begin{proposition}     \label{PROPooMANAooXhuanS}
    Si \( P_f\) est le polynôme du développement limité de \( f\) à l'ordre \( n\) et \( P_g\) celui de \( g\), alors nous obtenons le développement limité de \( f/g\) à l'ordre \( n\) en effectuant la division selon les puissances croissantes de \( P_f\) par \( P_g\).
\end{proposition}
Attention : il s'agit bien de faire une division selon les puissances croissantes, et non une divisions euclidienne. La division euclidienne de \( A\) par \( B\) consiste à écrire \( A=BQ+R\) avec le reste \( R\) de degré le plus \emph{petit} possible. Ici nous voulons avoir un reste de degré le plus \emph{grand} possible.

%///////////////////////////////////////////////////////////////////////////////////////////////////////////////////////////
\subsubsection{Développement limité d'une fonction composée}
%///////////////////////////////////////////////////////////////////////////////////////////////////////////////////////////


\begin{proposition}
    Soient \( f\) et \( g\) des fonctions admettant des développements limités d'ordre $n$ au voisinage de $0$. Nous supposons que \( \lim_{x\to 0} g(x)=0\). Alors la composée \( f\big( g(x) \big)\) admet un développement limité d'ordre $n$ au voisinage de $0$ qui s'obtient en substituant le développement de \( g\) à chaque <<\(x \)>> du développement de \( f\), et en supprimant tous les termes de degré plus élevé que $n$.
\end{proposition}

%+++++++++++++++++++++++++++++++++++++++++++++++++++++++++++++++++++++++++++++++++++++++++++++++++++++++++++++++++++++++++++
\section{Développement ailleurs qu'à l'origine}
%+++++++++++++++++++++++++++++++++++++++++++++++++++++++++++++++++++++++++++++++++++++++++++++++++++++++++++++++++++++++++++

Il est intéressant de développer une fonction au voisinage de zéro lorsque nous nous intéressons à son comportement pour les \( x\) pas très grands. Il est toutefois souvent souhaitable de savoir le comportement d'une fonction au voisinage d'autres valeurs que zéro.

Pour développer la fonction \( f\) autour de \( x_0\), nous considérons la fonction \( h\mapsto f(x_0+h)\) que nous développons autour de zéro (pour \( h\)). L'objectif est de trouver une polynôme \( P\) et une fonction \( \alpha\) tels que
\begin{subequations}
    \begin{numcases}{}
        f(x)=P(x)+(x-x_0)^n\alpha(x)\\
        \lim_{x\to x_0} \alpha(x)=0.
    \end{numcases}
\end{subequations}
En pratique, le développement limité à l'ordre $n$ d'une fonction autour d'un point $x_0$ quelconque à l'intérieur de son domaine prend la forme suivante, qui généralise la formule de Taylor-Young vue dans la proposition~\ref{PropVDGooCexFwy}
\begin{proposition}[Formule de Taylor-Young, cas général]
    Soit \( f\) une fonction \( n\) fois dérivable sur un intervalle \( I\) contenant \(x_0\). Alors il existe une fonction \( \alpha\colon I\to \eR\) telle que
    \begin{equation}    \label{EqTJRooUbsyzJ}
      \begin{aligned}
        f(x)=f(x_0)+&f'(x_0)(x-x_0)+\frac{ f''(x_0) }{ 2 }(x-x_0)^2+\\
        &+\frac{ f^{(3)}(x_0) }{ 3! }(x-x_0)^3+\cdots +\frac{ f^{(n)}(x_0) }{ n! }(x-x_0)^n+(x-x_0)^n\alpha(x-x_0)
      \end{aligned}
    \end{equation}
    et
    \begin{equation}
        \lim_{t\to 0} \alpha(t)=0.
    \end{equation}
\end{proposition}


%+++++++++++++++++++++++++++++++++++++++++++++++++++++++++++++++++++++++++++++++++++++++++++++++++++++++++++++++++++++++++++
\section{Développement au voisinage de l'infini}
%+++++++++++++++++++++++++++++++++++++++++++++++++++++++++++++++++++++++++++++++++++++++++++++++++++++++++++++++++++++++++++

Il est souvent utile de connaitre le comportement d'une fonction pour les grandes valeurs de \( x\) et de déterminer ses asymptotes éventuelles. La technique que nous allons utiliser consiste à poser \( x=\frac{1}{ h }\) et de développer la fonction ``auxiliaire'' $g(h) = f(1/h)$ autour de \( h=0\). La limite avec \( h\to 0^+\) donnera le comportement pour \( x\to \infty\) et la limite \( h\to 0^-\) donnera le comportement pour \( x\to -\infty\).

Dans le cas d'une développement autour de \( \pm\infty\) nous ne parlons plus de développement \emph{limité} mais de \defe{développement asymptotique}{développement!asymptotique}.

%--------------------------------------------------------------------------------------------------------------------------- 
\subsection{La fonction puissance : remarques pour la suite}
%---------------------------------------------------------------------------------------------------------------------------

Il y a encore de nombreuses choses à dire sur la fonction puissance. Pour savoir lesquelles, voir le thème \ref{THEMEooBSBLooWcaQnR}.

%+++++++++++++++++++++++++++++++++++++++++++++++++++++++++++++++++++++++++++++++++++++++++++++++++++++++++++++++++++++++++++
\section{Fonctions réelles de deux variables réelles}
%+++++++++++++++++++++++++++++++++++++++++++++++++++++++++++++++++++++++++++++++++++++++++++++++++++++++++++++++++++++++++++

Une \textbf{fonction réelle de 2 variables réelles} est une fonction $f : A \subset \eR^2 \to \eR : (x,y) \mapsto z = f(x,y)$.

Le \textbf{graphe de $f$}, noté $\Graphe f$, est un sous-ensemble de $\eR^3$:\[\Graphe f = \{(x,y,z) \in \eR^3 \mid (x,y) \in A \text{ et } z = f(x,y)\}\]

Les \textbf{courbes de niveau} de la fonction $f$ sont obtenues en posant $f(x,y)=\lambda$.

%---------------------------------------------------------------------------------------------------------------------------
\subsection{Limites de fonctions à deux variables}
%---------------------------------------------------------------------------------------------------------------------------

Ici nous n'allons pas entrer dans tous les détails, mais simplement mentionner les quelques techniques les plus courantes.

\begin{theorem}		\label{ThoLimiteCompose}
	Soient deux fonctions $f\colon \eR^n\to \eR^p$ et $g\colon \eR^p\to \eR^q$. Si $a$ est un point adhérent au domaine de $g\circ f$ et si
	\begin{equation}
		\begin{aligned}[]
			\lim_{x\to a}f(x)&=b\\
			\lim_{y\to b}g(y)&=c,
		\end{aligned}
	\end{equation}
	alors
	\begin{equation}
		\lim_{x\to a}(g\circ f)(x)=c.
	\end{equation}
\end{theorem}

Les techniques usuelles sont
\begin{enumerate}

	\item
		La règle de l'étau. Cette technique demande un peu plus d'imagination parce qu'il faut penser à un «truc» différent pour chaque exercice. En revanche, la justification est facile : il y a un théorème qui dit que ça marche.

	\item
		Lorsqu'on applique la règle de l'étau, penser à
		\begin{equation}
			| x |=\sqrt{x^2}\leq\sqrt{x^2+y^2}.
		\end{equation}
		Cela permet de majorer le numérateur. Attention : ce genre de majoration fonctionne seulement au numérateur : agrandir le dénominateur ferait diminuer la fraction.

	\item
		Il n'est pas vrai que
		\begin{equation}
			| x |=\sqrt{x^2}\leq\sqrt{x^4}\leq\sqrt{x^4+2y^4}.
		\end{equation}
		En effet, si $x$ est petit, alors $x^2>x^4$, et non le contraire.

\end{enumerate}

Une technique très efficace pour les limites $(x,y)\to (0,0)$ est le passage aux coordonnées polaires. Il s'agit de poser
\begin{subequations}
	\begin{numcases}{}
		x=r\cos(\theta)\\
		y=r\sin(\theta)
	\end{numcases}
\end{subequations}
et puis de faire la limite $r\to 0$.

Si la limite obtenue {\bf ne dépend pas de $\theta$}, alors c'est la limite cherchée. Voici quelque exemples.

\begin{example}
	Calculer les limites suivantes :
	\begin{enumerate}

		\item
			$\lim_{(x,y)\to(0,0)}\frac{ x-y }{ x+y }$
		\item
			$\lim_{(x,y)\to(0,0)}\frac{ (xy)^2 }{ (x+y)^2+(x-y)^2 }$
		\item
			$\lim_{(x,y)\to(0,0)}\frac{ xy^3 }{ x^2+y^2 }$
		\item
			$\lim_{(x,y)\to(0,0)}\frac{ x\sin(y) }{ \sqrt{x^2+y^2} }$
	\end{enumerate}
    
    Tentez de les faire par vous-même avant de regarder la solution qui suit.
	\begin{enumerate}
		\item
			Ici la méthode des chemins pour est particulièrement éclairante. Regardons d'abord la fonction sur la droite $x=y$. Nous avons
			\begin{equation}
				f(x,y)=\frac{ x-x }{ 2x }=0.
			\end{equation}
			Donc la fonction est nulle sur toute la ligne.

			Si nous regardons maintenant la ligne verticale $x=0$, nous avons
			\begin{equation}
				f(0,y)=\frac{ -y }{ y }=-1,
			\end{equation}
			donc la fonction vaut $-1$ sur toute la ligne verticale.
       %TODO : refaire la figure
%Regardez la figure \ref{LabelFigExoHuitUnINGE}

		\item

		\item
			Regardons la technique des coordonnées polaires. Nous remplaçons $x$ par $r\cos(\theta)$ et $y$ par $r\sin(\theta)$ :
			\begin{equation}
				f(r,\theta)=\frac{ r^4\cos(\theta)\sin^3(\theta) }{ r^2 }=r^2\cos(\theta)\sin^3(\theta).
			\end{equation}
			Cette fonction tend vers zéro quand $r\to 0$. Nous avons donc 
			\begin{equation}
				\lim_{(x,y)\to(0,0)}f(x,y)=0.
			\end{equation}

			Pour cet exercice nous pouvons aussi utiliser la règle de l'étau en écrivant d'abord
			\begin{equation}
				0\leq | f(x,y) |\leq\frac{ | x | |y^3 | }{ | x^2+y^2 | }.
			\end{equation}
			Mais on a $| x |\leq\sqrt{x^2+y^2}$, $| y |\leq\sqrt{x^2+y^2}$ et $| x^2+y^2 |=\big( \sqrt{x^2+y^2} \big)^2$, donc
			\begin{equation}
				0\leq| f(x,y) |\leq \frac{ \sqrt{x^2+y^2}\big( \sqrt{x^2+y^2} \big)^3 }{ \big( \sqrt{x^2+y^2} \big)^2 }=\big( \sqrt{x^2+y^2} \big)^2\to 0.
			\end{equation}

		\item
			En passant aux polaires, nous avons
			\begin{equation}
				f(r,\theta)=\frac{ r\cos\theta\sin\big( r\sin\theta \big) }{ r }=\cos(\theta)\sin\big( r\sin\theta \big).
			\end{equation}
			La limite de cette dernière fonction lorsque $r\to 0$ vaut zéro.

			Une autre façon de procéder consiste à multiplier et diviser par $y$ de telle façon à faire apparaitre $\sin(y)/y$ dont nous connaissons la limite :
			\begin{equation}
				f(x,y)=\frac{ \sin(y) }{ y }\cdot\frac{ xy }{ \sqrt{x^2+y^2} }.
			\end{equation}
			La limite du premier facteur est $1$, tandis que le second peut être traité de façon classique en prenant la valeur absolue et en majorant $| x |$ par $\sqrt{x^2+y^2}$.
			
	\end{enumerate}

	%\newcommand{\CaptionFigExoHuitUnINGE}{Sur toute la ligne rouge, la fonction vaut zéro, tandis que sur la ligne bleue elle vaut $-1$. Au point $(0,0)$, les deux sont inconciliables. Donc la limite n'existe pas.}
	%\input{auto/pictures_tex/Fig_ExoHuitUnINGE.pstricks}

\end{example}


%---------------------------------------------------------------------------------------------------------------------------
\subsection{Dérivées partielles}
%---------------------------------------------------------------------------------------------------------------------------

La \defe{dérivée partielle}{dérivée!partielle} par rapport à $x$ au point $(x,y)$ est notée
\begin{equation}
	\frac{\partial f}{\partial x}(x,y)
\end{equation}
et se calcule en dérivant $f$ par rapport  à $x$ en considérant que $y$ est constante.

De la même manière, la dérivée partielle par rapport à $y$ au point $(x,y)$ est notée
\begin{equation}
	\frac{\partial f}{\partial y}(x,y)
\end{equation}
et se calcule en dérivant $f$ par rapport  à $y$ en considérant que $x$ est constante.

Pour les dérivées partielles secondes,
\begin{itemize}
\item $f''_{xx} (x,y) = (f'_x)'_x = \frac{\partial^2 f}{\partial x^2}(x,y) = \frac{\partial}{\partial x}(\frac{\partial f}{\partial x})$.
\item $f''_{yy} (x,y) = (f'_y)'_y = \frac{\partial^2 f}{\partial y^2}(x,y) = \frac{\partial}{\partial y}(\frac{\partial f}{\partial y})$.
\item $f''_{xy} (x,y) = (f'_x)'_y  = (f'_y)'_x = f''_{yx} (x,y) \text{ ou } \frac{\partial^2 f}{\partial x \partial y}(x,y) = \frac{\partial}{\partial x}(\frac{\partial f}{\partial y})  = \frac{\partial}{\partial y}(\frac{\partial f}{\partial x}) =\frac{\partial^2 f}{\partial y \partial x}(x,y)$.
\end{itemize}

%---------------------------------------------------------------------------------------------------------------------------
\subsection{Différentielle et accroissement}
%---------------------------------------------------------------------------------------------------------------------------

La \defe{différentielle totale}{différentielle!totale} de $f$ au point $(a,b)$ est donnée, quand elle existe (!), par la formule
\begin{equation}
	df(a,b) = \frac{\partial f}{\partial x}(a,b)dx + \frac{\partial f}{\partial y}(a,b) dy.
\end{equation}

De la même façon que la formule des accroissements finis disait que $f(x+a)\simeq f(x)+af'(x)$, en deux dimensions nous avons que l'\defe{accroissement}{accroissement} approximatif de $f$ au point $(a,b)$ pour des accroissements $\Delta x$ et $\Delta y$ est
\begin{equation}
	f(x+\Delta x,y+\Delta y)=f(x,y)+\Delta x\frac{ \partial f }{ \partial x }(x,y)+\Delta y\frac{ \partial f }{ \partial y }(x,y).
\end{equation}

%TODO : pour l'index, l'expression régulière suivante aide :
% grep "defe{[A-Za-z ]*}{[A-Z]" *.tex
Le \defe{plan tangent}{plan!tangent} au graphe de $f$ au point $\big(a,b,f(a,b)\big)$ est
\begin{equation}
	T_{(a,b)}(x,y) = f(a,b) + \frac{\partial f}{\partial x}(a,b) (x-a) + \frac{\partial f}{\partial y}(a,b) (y-b)
\end{equation}
essayez d'écrire l'équation de la droite tangente au graphe de $f(x)$ au point $x=a$ en termes de la dérivée de $f$, et comparez votre résultat à cette formule.

Un des principaux théorèmes pour tester la différentiabilité d'une fonction est le suivant.

\begin{theorem}		\label{ThoProuverDiffable}
	Soit une fonction $f\colon \eR^m\to \eR^p$. Si les dérivées partielles existent dans un voisinage de $a$ et donc continues en $a$, alors $f$ est différentiable en $a$.
\end{theorem}
Le plus souvent, nous prouvons qu'une fonction est différentiable en calculant les dérivées partielles et en montrant qu'elles sont continues.

\textbf{Dérivation implicite:} Soit $F(x,f(x)) = 0$ la représentation implicite d'une fonction $y=f(x)$ alors \[y' = f'(x) = - \frac{F'_x}{F'_y}.\]


%+++++++++++++++++++++++++++++++++++++++++++++++++++++++++++++++++++++++++++++++++++++++++++++++++++++++++++++++++++++++++++
\section{Les fonctions à valeurs vectorielles}
%+++++++++++++++++++++++++++++++++++++++++++++++++++++++++++++++++++++++++++++++++++++++++++++++++++++++++++++++++++++++++++

Jusqu'à présent nous avons vu des fonctions de plusieurs variables qui prenaient leurs valeurs dans $\eR$. Nous allons maintenant voir ce qu'il se passe lorsque les fonctions prennent leurs valeurs dans $\eR^3$.

Une fonction d'une variable est dite \defe{à valeurs vectorielles}{fonction!valeurs vectorielles} lorsque
\begin{equation}
    \begin{aligned}
        f\colon I\subset \eR&\to \eR^3 \\
        f(x)&=\begin{pmatrix}
            f_1(x)    \\
            f_2(x)    \\
            f_3(x)
        \end{pmatrix}.
    \end{aligned}
\end{equation}
Les fonctions $f_i\colon \eR\to \eR$ sont les \defe{composantes}{composante} de $f$. Ce que nous avons raconté à propos des dérivées passe facilement :
\begin{equation}
    \frac{ f(a+\epsilon)-f(a) }{ \epsilon }=
    \begin{pmatrix}
        \frac{ f_1(a+\epsilon)-f_1(a) }{ \epsilon }    \\
        \frac{ f_2(a+\epsilon)-f_2(a) }{ \epsilon }    \\
        \frac{ f_3(a+\epsilon)-f_3(a) }{ \epsilon }
    \end{pmatrix}.
\end{equation}
En particulier dès que les fonctions $f_i$ sont dérivables, nous avons
\begin{equation}
    f'(a)=\begin{pmatrix}
        f_1'(a)    \\
        f_2'(a)    \\
        f_3'(a)
    \end{pmatrix}
\end{equation}
comme dérivée de la fonction. Cette dérivée est un vecteur.

\begin{example}
    Si
    \begin{equation}
        f\colon x\in\eR\mapsto \begin{pmatrix}
            x^2 e^{x}    \\
            \cos(x^2)    \\
            x^3+x
        \end{pmatrix},
    \end{equation}
    alors
    \begin{equation}
        f'(x)=\begin{pmatrix}
            2xe^x+x^2e^x    \\
            -2x\sin(x^2)    \\
            3x^2+1
        \end{pmatrix}.
    \end{equation}
\end{example}

%+++++++++++++++++++++++++++++++++++++++++++++++++++++++++++++++++++++++++++++++++++++++++++++++++++++++++++++++++++++++++++
\section{Fonctions vectorielles de plusieurs variables}
%+++++++++++++++++++++++++++++++++++++++++++++++++++++++++++++++++++++++++++++++++++++++++++++++++++++++++++++++++++++++++++

Ce sont les fonctions de la forme
\begin{equation}
    \begin{aligned}
        f\colon \eR^3&\to \eR^3 \\
        \begin{pmatrix}
            x    \\
            y    \\
            z
        \end{pmatrix}&\mapsto \begin{pmatrix}
            f_1(x,y,z)\\
            f_2(x,y,z)\\
            f_3(x,y,z)
        \end{pmatrix}.
    \end{aligned}
\end{equation}

En ce qui concerne les dérivées, tout se passe comme avant. Si les dérivées partielles des composantes $f_i$ existent au point $a\in\eR^3$, alors
\begin{equation}
    \begin{aligned}[]
        \frac{ \partial f }{ \partial x }(a)&=\begin{pmatrix}
            \partial_xf_1(a)    \\
            \partial_xf_2(a)    \\
            \partial_xf_3(a)    \\
        \end{pmatrix},&
        \frac{ \partial f }{ \partial y }(a)&=\begin{pmatrix}
            \partial_yf_1(a)    \\
            \partial_yf_2(a)    \\
            \partial_yf_3(a)    \\
        \end{pmatrix},&
        \frac{ \partial f }{ \partial z }(a)&=\begin{pmatrix}
            \partial_zf_1(a)    \\
            \partial_zf_2(a)    \\
            \partial_zf_3(a)    \\
        \end{pmatrix}.
    \end{aligned}
\end{equation}

%+++++++++++++++++++++++++++++++++++++++++++++++++++++++++++++++++++++++++++++++++++++++++++++++++++++++++++++++++++++++++++
\section{Limites à plusieurs variables}
%+++++++++++++++++++++++++++++++++++++++++++++++++++++++++++++++++++++++++++++++++++++++++++++++++++++++++++++++++++++++++++

\begin{proposition}	\label{PropLimParcompos}
	Soit $f\colon D\subset\eR^m\to \eR^n$. Nous avons
	\begin{equation}
		\lim_{x\to a} f(x)=\ell
	\end{equation}
	si et seulement si
	\begin{equation}
		\lim_{x\to a} f_i(x)=\ell_i
	\end{equation}
	pour tout $i\in\{ 1,\ldots,n \}$ où $f_i(x)$ dénote la $i$-ème composante de $f(x)$ et $\ell_i$ la $i$-ème composante de $\ell\in\eR^n$.
\end{proposition}
Cette proposition revient à dire que la convergence d'une fonction est équivalente à la convergence de chacune de ses composantes.

\begin{proof}
	L'élément clef de la preuve est le fait que pour tout vecteur $u\in\eR^p$, nous ayons l'inégalité
	\begin{equation}	\label{Equilequnorme}
		| u_i |\leq\sqrt{\sum_{k=1}^p| u_k |^2}=\| u \|.
	\end{equation}
	La norme (dans $\eR^p$) d'un vecteur est plus grande ou égale à la valeur absolue de chacune de ses composantes.

	Supposons que nous ayons une fonction dont chacune des composantes a une limite en $a$ : $\lim_{x\to a} f_i(x)=\ell_i$. Montrons que dans ce cas la fonction $f$ tend vers $\ell$. Si nous considérons $\varepsilon>0$, par définition de la limite de chacune des fonctions $f_i$, il  existent des $\delta_i$ tels que
	\begin{equation}
		\| x-a \|_{\eR^m}<\delta_i\Rightarrow | f_i(x)-\ell_i |<\varepsilon.
	\end{equation}
	Notez que la norme à gauche est une norme dans $\eR^m$ et que celle à droite est une simple valeur absolue dans $\eR$. Considérons $\delta=\min\{ \delta_i \}_{i=1,\ldots n}$. Si $\| x-a \|<\delta$, alors
	\begin{equation}
		\| f(x)-\ell \|=\sqrt{\sum_{i=1}^n| f_i(x)-\ell_i |^2}<\sqrt{\sum_{i=1}^n\varepsilon^2}=\sqrt{n\varepsilon^2}=\sqrt{n}\varepsilon.
	\end{equation}
	Nous voyons qu'en choisissant les $\delta_i$ tels que $| f_i(x)-\ell_i |<\varepsilon$, nous trouvons $\| f(x)-\ell \|<\sqrt{n}\varepsilon$. Afin d'obtenir $\| f(x)-\ell \|<\varepsilon$, nous choisissons donc les $\delta_i$ de telle manière a avoir $| f_i(x)-\ell_i |<\varepsilon/\sqrt{n}$.

	Nous avons donc prouvé que la limite composante par composante impliquait la limite de la fonction. Nous devons encore prouver le sens inverse.

	Supposons donc que $\lim_{x\to a} f(x)=\ell$, et prouvons que nous ayons $\lim_{x\to a} f_i(x)=\ell_i$ pour chaque $i$. Soit $\varepsilon>0$ et $\delta>0$ tel que $\| x-a \|<\delta$ implique $\| f(x)-\ell \|<\varepsilon$. Avec ces choix, nous avons
	\begin{equation}
		| f_i(x)-\ell_i |\leq\| f(x)-\ell \|<\varepsilon
	\end{equation}
	où nous avons utilisé la majoration \eqref{Equilequnorme} avec $f(x)-\ell$ en guise de $u$.
\end{proof}

De même, pour la continuité nous avons la proposition suivante :
\begin{proposition}
	Soit une fonction $f\colon D\subset\eR^m\to \eR^n$ et $a\in D$. La fonction $f$ est continue en $a$ si et seulement si chacune de ses composantes l'est, c'est-à-dire si et seulement si chacune des fonctions $f_i\colon D\to \eR$ est continue en $a$.
\end{proposition}
Essayez de prouver cette proposition directement par la définition de la continuité, en suivant pas à pas la démonstration de la proposition~\ref{PropLimParcompos}.

\begin{proposition}		\label{Propfaposfxposcont}
	Soit $f\colon \eR^m\to \eR$ et $a$, un point du domaine de $f$ telle que $f(a)>0$. Alors il existe un rayon $r$ tel que $f(x)>0$ pour tout $x$ dans $B(a,r)$.
\end{proposition}
Cette proposition signifie que si la fonction est strictement positive en un point, alors elle restera strictement positive en tous les points «pas trop loin».

\begin{proof}
	Prenons $\varepsilon=f(a)/2$ dans la définition de la continuité. Il existe donc un rayon $\delta$ tel que pour tout $x$ dans $B(a,\delta)$,
	\begin{equation}
		| f(x)-f(a) |\leq \frac{ f(a) }{2},
	\end{equation}
	en d'autres termes, $f(x)\in B\big( f(a),\frac{ f(a) }{ 2 } \big)$. évidemment aucun nombre négatif ne fait partie de cette dernière boule lorsque $f(a)$ est strictement positif.
\end{proof}

\begin{corollary}		\label{CorfneqzOuvert}
	Si $f\colon \eR^m\to \eR$ est une fonction continue, alors l'ensemble
	\begin{equation}
		A=\{ x\in\eR^m\tqs f(x)\neq 0 \}
	\end{equation}
	est ouvert.
\end{corollary}

\begin{proof}
	Soit $x\in A$. Si $x>0$ (le cas $x<0$ est laissé en exercice), alors il existe une boule autour de $x$ sur laquelle $f$ reste strictement positive (proposition~\ref{Propfaposfxposcont}). Cette boule est donc contenue dans $A$. Étant donné qu'autour de chaque point de $A$ nous pouvons trouver une boule contenue dans $A$, ce dernier est ouvert.
\end{proof}

\begin{example} \label{ExBNOQEWe}
    Soit  $GL_n(\eR)$ l'ensemble des matrices $n \times n$ inversibles.   Nous allons montrer que $GL_n(\eR)$ est un ouvert de $ \eR^{n^2}$. L'identification entre les vecteurs et les matrices consiste simplement à «déplier» la matrice pour en faire un vecteur. Par exemple, en dimension deux,
	\begin{equation}
		\begin{pmatrix}
			1	&	2	\\
			3	&	4
		\end{pmatrix}\mapsto
		\begin{pmatrix}
			1	\\
			2	\\
			3	\\
			4
		\end{pmatrix}\in\eR^4.
	\end{equation}
	En dimension $3$,
	\begin{equation}
		\begin{aligned}[]
			\begin{pmatrix}
				1	&	2	&	3	\\
				4	&	5	&	6	\\
				7	&	8	&	9
			\end{pmatrix}
			\mapsto
			\begin{pmatrix}
				1	\\
				2	\\
				3	\\
				4	\\
				5	\\
				6	\\
				7	\\
				8	\\
				9
			\end{pmatrix}\in\eR^9.
		\end{aligned}
	\end{equation}

	Une matrice est inversible si et seulement si son déterminant est non nul. Or le déterminant est un polynôme en les composantes de la matrice. En dimension deux, nous avons
	\begin{equation}
		\det\begin{pmatrix}
			a	&	b	\\
			c	&	d
		\end{pmatrix}=ad-bc,
	\end{equation}
	mais en écriture «dépliée», nous pouvons aussi bien écrire
	\begin{equation}
		\det\begin{pmatrix}
			a	\\
			b	\\
			c	\\
			d
		\end{pmatrix}=ad-bc.
	\end{equation}
	En dimension $3$, le déterminant est donc un polynôme des $9$ variables qui apparaissent dans le vecteur «déplié». En général, dans $\eR^{n^2}$, nous considérons donc le polynôme $\det\colon \eR^{n^2}\to \eR$ qui à un vecteur $X\in\eR^{n^2}$ fait correspondre le déterminant de la matrice obtenue en «repliant» le vecteur $X$.

	Donc dans $\eR^{n^2}$, l'ensemble des matrices inversibles est donné par l'ensemble des vecteurs sur lesquels le polynôme $\det$ ne s'annule pas, c'est-à-dire
	\begin{equation}
		\{ X\in\eR^{n^2}\tqs \det(X)\neq 0 \}.
	\end{equation}
	Mais le déterminant est un polynôme, et donc une fonction continue. Cet ensemble est par conséquence ouvert par le corolaire~\ref{CorfneqzOuvert}.
\end{example}



La proposition suivante montre que la limite peut «passer à travers» les fonctions continues.
\begin{proposition}[limite de fonction composée]		\label{PropLimCompose}
	Soit $f\colon \eR^n\to \eR^q$ et $g\colon \eR^m\to \eR^n$ telles que
	\begin{subequations}
		\begin{align}
			\lim_{x\to a} g(x)&= p		\label{EqLimCompHypa}\\
			\lim_{y\to p} f(y)&= q		\label{EqLimCompHypb}
		\end{align}
	\end{subequations}
	Alors nous avons $\lim_{x\to a} (f\circ g)(x)=q$.
\end{proposition}

\begin{proof}
	Comme presque toute preuve à propos de limite ou de continuité, nous commençons par choisir $\varepsilon>0$. Nous devons montrer qu'il existe un $\delta$ tel que $\| x-a \|\leq \delta$ implique $\| f\big( g(x) \big)-q \|\leq \varepsilon$.

	La limite \eqref{EqLimCompHypb} impose l'existence d'un $\tilde\delta$ tel que $\| y-p \|\leq\tilde\delta$ implique $\| f(y)-q \|\leq\varepsilon$, tandis que la limite \eqref{EqLimCompHypa} donne un $\delta$ tel que $\| x-a \|\leq\delta$ implique $\| g(x)-p \|\leq\tilde\delta$ (nous avons pris $\tilde\delta$ en guise de $\varepsilon$ dans la définition de la limite pour $g$).

	Avec ces choix, si $\| x-a \|\leq \delta$, alors $\| g(x)-p \|\leq\tilde\delta$, et par conséquent,
	\begin{equation}
		\| f\big( g(x) \big)-q \|\leq\varepsilon,
	\end{equation}
	ce que nous voulions.
\end{proof}

De façon pragmatique, la proposition~\ref{PropLimCompose} nous fournit une formule pour les limites de fonctions composée :
\begin{equation}		\label{Eqlimfgvomp}
	\lim_{x\to a} (f\circ g)(x)=\lim_{y\to \lim_{x\to a} g(x)}f(y)
\end{equation}
lorsque $f$ est continue.

\begin{remark}
	La formule \eqref{Eqlimfgvomp} ne peut pas être utilisée à l'envers. Il existe des cas où $\lim_{x\to a} (g\circ f)(x)=q$, et $\lim_{x\to a} f(x)=p$ sans pour autant avoir $\lim_{y\to q} g(y)=q$. Par exemple
	\begin{subequations}
		\begin{align}
			g(x)&=\begin{cases}
				2	&	\text{si }x\geq0\\
				0	&	 \text{si }x<0\\
			\end{cases}\\
			f(x)&=| x |.
		\end{align}
	\end{subequations}
	Nous avons $(g\circ f)(x)=2$ pour tout $x$, ainsi que $\lim_{x\to 0} f(x)=0$, mais la limite $\lim_{y\to 0} g(y)$ n'existe pas.
\end{remark}

%+++++++++++++++++++++++++++++++++++++++++++++++++++++++++++++++++++++++++++++++++++++++++++++++++++++++++++++++++++++++++++
\section{Champs de vecteurs}
%+++++++++++++++++++++++++++++++++++++++++++++++++++++++++++++++++++++++++++++++++++++++++++++++++++++++++++++++++++++++++++

Un champ de vecteur est une fonction $f\colon \eR^3\to \eR^3$. Géométriquement, il s'agit simplement de mettre un vecteur en chaque point de l'espace. Cela arrive très souvent en physique.

\begin{example}
    Si un fluide (eau, gaz) coule dans un tube, en tout point le point a une vitesse, qui sera un vecteur généralement dirigé le long du tube.
\end{example}

\begin{example}
    La force d'attraction de la Terre sur une masse $m$ située au point $r=(x,y,z)$ est donnée par
    \begin{equation}
        F(r)=-G\frac{ Mmr }{ \| r \|^3 }.
    \end{equation}
    Dans cette expression, tant $r$ que $F(r)$ sont des vecteurs. Nous l'avons représenté sur la figure~\ref{LabelFigSQNPooPTrLRQ}. % From file SQNPooPTrLRQ
\newcommand{\CaptionFigSQNPooPTrLRQ}{Le champ de gravitation de la Terre.}
\input{auto/pictures_tex/Fig_SQNPooPTrLRQ.pstricks}

    L'application
    \begin{equation}
        \begin{aligned}
            F\colon \eR^3&\to \eR^3 \\
            r&\mapsto F(r)
        \end{aligned}
    \end{equation}
    est le champ gravitationnel de la Terre.

\end{example}

%---------------------------------------------------------------------------------------------------------------------------
\subsection{Matrice jacobienne}
%---------------------------------------------------------------------------------------------------------------------------

La \defe{matrice jacobienne}{jacobien} de la fonction $f\colon \eR^3\to \eR^3$ au point $a\in\eR^3$ est la matrice dont les colonnes sont les vecteurs $\frac{ \partial f }{ \partial x }(a)$, $\frac{ \partial f }{ \partial y }(a)$ et $\frac{ \partial f }{ \partial z }(a)$, c'est-à-dire
\begin{equation}
    J_f(a)=\begin{pmatrix}
        \frac{ \partial f_1 }{ \partial x }(a)   &   \frac{ \partial f_1 }{ \partial y }(a)    &   \frac{ \partial f_1 }{ \partial z }(a)    \\
        \frac{ \partial f_2 }{ \partial x }(a)   &   \frac{ \partial f_2 }{ \partial y }(a)    &   \frac{ \partial f_2 }{ \partial z }(a)    \\
        \frac{ \partial f_3 }{ \partial x }(a)   &   \frac{ \partial f_3 }{ \partial y }(a)    &   \frac{ \partial f_3 }{ \partial z }(a)
    \end{pmatrix}.
\end{equation}

\begin{example}
    Si
    \begin{equation}
        f(x,y,z)=\begin{pmatrix}
            xy e^{z}    \\
            x^2+\cos(yz)    \\
            xyz
        \end{pmatrix},
    \end{equation}
    alors
    \begin{equation}
        J_f(x,y,z)=\begin{pmatrix}
            ye^z    &   xe^z    &   xye^z    \\
            2x    &   -z\sin(yz)    &   -y\sin(yz)    \\
            yz    &   xz    &   xy
        \end{pmatrix}.
    \end{equation}
\end{example}

%+++++++++++++++++++++++++++++++++++++++++++++++++++++++++++++++++++++++++++++++++++++++++++++++++++++++++++++++++++++++++++
\section{Divergence, rotationnel et l'opérateur nabla}
%+++++++++++++++++++++++++++++++++++++++++++++++++++++++++++++++++++++++++++++++++++++++++++++++++++++++++++++++++++++++++++

Nous avons déjà vu le gradient d'une fonction $f\colon \eR^3\to \eR$
\begin{equation}        \label{EqDefNablaf}
    \nabla f(x,y,z)=\begin{pmatrix}
        \partial_xf(x,y,z)    \\
        \partial_yf(x,y,z)    \\
        \partial_zf(x,y,z)
    \end{pmatrix}
\end{equation}
Afin de définir la divergence et le rotationnel, nous introduisons $\nabla$ sous une forme un peu plus abstraite comme le «vecteur»
\begin{equation}
    \nabla=\begin{pmatrix}
        \partial_x    \\
        \partial_y    \\
        \partial_z
    \end{pmatrix}.
\end{equation}
Vue comme ça, la formule \eqref{EqDefNablaf} est claire.

Si $F$ est un champ de vecteurs, nous introduisons la \defe{divergence}{divergence} de $F$ par
\begin{equation}
    \nabla\cdot F=\frac{ \partial F_x }{ \partial x }+\frac{ \partial F_y }{ \partial y }+\frac{ \partial F_z }{ \partial z }.
\end{equation}
Cela est une fonction. Et nous introduisons le rotationnel du champ de vecteur $F$ par
\begin{equation}
    \begin{aligned}[]
        \nabla\times F&=\begin{vmatrix}
              e_x  &   e_y    &   e_z    \\
            \partial_x    &   \partial_y    &   \partial_z    \\
            F_x    &   F_y    &   F_z
        \end{vmatrix}\\
        &=
        \left( \frac{ \partial F_z }{ \partial y }-\frac{ \partial F_y }{ \partial z } \right)e_x
        -\left( \frac{ \partial F_z }{ \partial x }-\frac{ \partial F_x }{ \partial z } \right)e_y
        +\left( \frac{ \partial F_y }{ \partial x }-\frac{ \partial F_x }{ \partial y } \right)e_z.
    \end{aligned}
\end{equation}
Cela est un champ de vecteur. En utilisant le symbole complètement antisymétrique \( \epsilon_{ijk}\), le rotationnel d'un champ de vecteur peut s'écrire
\begin{equation}
    \nabla\times F=\sum_{ijk}\epsilon_{ijk}\partial_i F_j e_k.
\end{equation}

Le gradient, la divergence et le rotationnel consistent à appliquer simplement à $\nabla$ est trois produits qu'on peut effectuer sur un vecteur:
\begin{enumerate}
    \item
        Le produit d'un vecteur par un scalaire multiplie chacune des composantes :
        \begin{equation}
            \begin{pmatrix}
                \partial_x    \\
                \partial_y    \\
                \partial_z
            \end{pmatrix}f
            =\begin{pmatrix}
                \partial_xf    \\
                \partial_yf    \\
                \partial_zf
            \end{pmatrix}.
        \end{equation}
    \item
        Le produit scalaire d'un vecteur avec un autre vecteur donne lieu à la divergence :
        \begin{equation}
            \begin{pmatrix}
                \partial_x    \\
                \partial_y    \\
                \partial_z
            \end{pmatrix}\cdot
            \begin{pmatrix}
                F_x    \\
                F_y    \\
                F_z
            \end{pmatrix}=
            \frac{ \partial F_x }{ \partial x }+\frac{ \partial F_y }{ \partial y }+\frac{ \partial F_z }{ \partial z }.
        \end{equation}
    \item
        Le produit vectoriel de deux vecteurs :
        \begin{equation}
            \begin{pmatrix}
                \partial_x    \\
                \partial_y    \\
                \partial_z
            \end{pmatrix}\times\begin{pmatrix}
                F_x    \\
                F_y    \\
                F_z
            \end{pmatrix}=
            \begin{vmatrix}
                e_x    &   e_y    &   e_z    \\
                \partial_x    &   \partial_y    &   \partial_z    \\
                F_x    &   F_y    &   F_z
            \end{vmatrix}.
        \end{equation}
\end{enumerate}
Ces trois opérations joueront un rôle central en électromagnétisme dans les équations de Maxwell.

\begin{example}
    Soit $F(x,y,z)=x e_x+xy e_y+e_z$, c'est-à-dire
    \begin{equation}
        F(x,y,z)=\begin{pmatrix}
            x    \\
            xy    \\
            1
        \end{pmatrix}.
    \end{equation}
    Son rotationnel est donné par
    \begin{equation}
        \nabla\times F=\begin{vmatrix}
            e_x    &   e_y    &   e_z    \\
            \frac{ \partial  }{ \partial x }    &   \frac{ \partial  }{ \partial y }    &   \frac{ \partial  }{ \partial y }    \\
            x    &   xy    &   1
        \end{vmatrix}=
        (0-0)e_x-(0-0)e_y+(y-0)e_z=ye_z=\begin{pmatrix}
            0    \\
            0    \\
            y
        \end{pmatrix}.
    \end{equation}
\end{example}

Afin d'étudier comment se comporte la composition de ces opérateurs, nous aurons besoin de ce lemme que nous n'énoncerons pas précisément.
\begin{lemma}       \label{LemPermDerrxyz}
    Si $f\colon \eR^3\to \eR$ est une fonction de classe $C^2$, alors on peut permuter l'ordre des dérivées:
    \begin{equation}
        \begin{aligned}[]
            \frac{ \partial  }{ \partial x }\left( \frac{ \partial f }{ \partial y } \right)&=\frac{ \partial  }{ \partial y }\left( \frac{ \partial f }{ \partial x } \right)\\
            \frac{ \partial  }{ \partial x }\left( \frac{ \partial f }{ \partial z } \right)&=\frac{ \partial  }{ \partial z }\left( \frac{ \partial f }{ \partial x } \right)\\
            \frac{ \partial  }{ \partial z }\left( \frac{ \partial f }{ \partial y } \right)&=\frac{ \partial  }{ \partial y }\left( \frac{ \partial f }{ \partial z } \right)
        \end{aligned}
    \end{equation}
\end{lemma}
La fonction
\begin{equation}
    (x,y,z)\mapsto\frac{ \partial  }{ \partial x }\left( \frac{ \partial f }{ \partial y } \right)(x,y,z)
\end{equation}
sera notée
\begin{equation}
    \frac{ \partial^2f }{ \partial x\partial y }.
\end{equation}

Il y a deux propriétés importantes :
\begin{theorem}
    Soit $f\colon \eR^3\to \eR$ une fonction de classe $C^2$. Alors
    \begin{equation}
        \nabla\times(\nabla f)=0.
    \end{equation}
    Si $F\colon \eR^3\to \eR^3$ est un champ de vecteurs de classe $C^2$, alors
    \begin{equation}
        \nabla\cdot(\nabla\times F)=0.
    \end{equation}
\end{theorem}

\begin{proof}
    Ce sont seulement deux calculs qui manipulent les définitions. Pour le premier, la divergence de $f$ est le champ de vecteurs
    \begin{equation}
        \nabla f=\frac{ \partial f }{ \partial x }e_x+\frac{ \partial f }{ \partial y }e_y+\frac{ \partial f }{ \partial z }e_z.
    \end{equation}
    En mettant ce champ dans la définition du rotationnel,
    \begin{equation}
        \begin{aligned}[]
            \nabla\times(\nabla f)=\begin{vmatrix}
                 e_x   &   e_y    &   e_z    \\
                 \frac{ \partial  }{ \partial x }    &   \frac{ \partial  }{ \partial y }    &   \frac{ \partial  }{ \partial z }    \\
                 \frac{ \partial f }{ \partial x }    &   \frac{ \partial f }{ \partial y }    &   \frac{ \partial f }{ \partial z }
            \end{vmatrix}
            &=\left[ \frac{ \partial  }{ \partial y }\left( \frac{ \partial f }{ \partial z } \right)-\frac{ \partial  }{ \partial z }\left( \frac{ \partial f }{ \partial y } \right) \right]e_x\\
            &\quad-\left[ \frac{ \partial  }{ \partial x }\left( \frac{ \partial f }{ \partial z } \right)-\frac{ \partial  }{ \partial z }\left( \frac{ \partial f }{ \partial x } \right) \right]e_y\\
            &\quad+\left[ \frac{ \partial  }{ \partial x }\left( \frac{ \partial f }{ \partial y } \right)-\frac{ \partial  }{ \partial y }\left( \frac{ \partial f }{ \partial x } \right) \right]e_z.
        \end{aligned}
    \end{equation}
    En utilisant le lemme~\ref{LemPermDerrxyz}, chacun des termes fait zéro.

    La seconde propriété se démontre en utilisant le même type de calcul.
\end{proof}

\begin{remark}
    Il n'y a pas de propriétés du même style pour la combinaison $\nabla\times(\nabla\cdot F)$ pour le rotationnel de la divergence. En effet la divergence d'un champ de vecteur est une fonction, et il n'y a pas de rotationnel pour une fonction.
\end{remark}

\notbool{isBook}
{
\begin{center}
            \includegraphics[width=10cm]{pictures_bitmap/501-curling-with-gradients.png}\\
        \url{http://spikedmath.com/501.html}{Spiked math}, \href{http://creativecommons.org/licenses/by-nc-sa/2.5/ca/}{licence Creative Commons by-nc 2.5}.
\end{center}
}{}

%+++++++++++++++++++++++++++++++++++++++++++++++++++++++++++++++++++++++++++++++++++++++++++++++++++++++++++++++++++++++++++
\section[Interprétation de la divergence]{Interprétation géométrique et physique de la divergence}
%+++++++++++++++++++++++++++++++++++++++++++++++++++++++++++++++++++++++++++++++++++++++++++++++++++++++++++++++++++++++++++

En physique, on dit qu'un champ de vecteurs à divergence nulle est \defe{incompressible}{incompressible!champ de vecteur}. Nous allons essayer de comprendre pourquoi. Lorsqu'un fluide incompressible se déplace, il faut qu'en chaque point il y ait autant de fluide qui rentre que de fluide qui sort. Nous allons voir sur quelques exemples que la divergence d'un champ de vecteurs est le «bilan de masse» d'un fluide qui se déplace selon le champ de vecteurs.

Si en un point la divergence est positive, cela signifie qu'il y a une perte de masse et si la divergence est négative, cela signifie qu'il y a une accumulation de masse.

Prenons par exemple un fluide qui se déplace selon le champ de vitesse montré à figure~\ref{LabelFigBEHTooWsdrys}. % From file BEHTooWsdrys
\newcommand{\CaptionFigBEHTooWsdrys}{Le champ de vecteurs $F(x,y)=\frac{1}{ x }(1,0)$.}
\input{auto/pictures_tex/Fig_BEHTooWsdrys.pstricks}

Étant donné que la vitesse diminue lorsque $x$ avance, il y a une accumulation de fluide. Regardez en effet la quantité de fluide qui rentre dans le rectangle par rapport à la quantité de fluide qui en sort. Ce champ de vecteurs a pour équation :
\begin{equation}
    F(x,y)=\frac{1}{ x }\begin{pmatrix}
        1    \\
        0
    \end{pmatrix}=\begin{pmatrix}
        1/x    \\
        0
    \end{pmatrix}.
\end{equation}
Sa divergence vaut donc
\begin{equation}
    (\nabla\cdot F)(x,y)=\frac{ \partial F_x }{ \partial x }(x,y)+\underbrace{\frac{ \partial F_y }{ \partial y }(x,y)}_{=0}=-\frac{1}{ x^2 }.
\end{equation}
Cette divergence étant négative, il y a bien accumulation de fluide en tout point, et d'autant plus que $x$ est petit.

\begin{example}     \label{ExamDivFrot}

    Prenons le champ de vecteurs tournant
    \begin{equation}
        F(x,y)=\frac{1}{ \sqrt{x^2+y^2} }\begin{pmatrix}
            y    \\
            -x
        \end{pmatrix}
    \end{equation}
    représenté à la figure~\ref{LabelFigYQVHooYsGLHQ}. Cela est un vecteur qui est constamment perpendiculaire au rayon.


\newcommand{\CaptionFigYQVHooYsGLHQ}{Le champ de vecteurs $F(x,y)=(y,-x)$.}
\input{auto/pictures_tex/Fig_YQVHooYsGLHQ.pstricks}

    Un fluide dont la vitesse serait donné par ce champ de vecteur se contente de tourner. Intuitivement il ne devrait pas y avoir de divergence parce qu'il n'y a aucune accumulation de fluide. En effet,
    \begin{equation}
        \nabla\cdot F(x,y)=\frac{ -2xy }{ (x^2+y^2)^2 }+\frac{ 2xy }{ (x^2+y^2)^2 }=0.
    \end{equation}
\end{example}

\begin{example}
    Prenons le cas du champ de force de gravitation :
    \begin{equation}
        F(x,y,z)=\frac{1}{ (x^2+y^2+z^2)^{3/2} }\begin{pmatrix}
            x    \\
            y   \\
            z
        \end{pmatrix}.
    \end{equation}
    Nous pouvons rapidement remarquer que $\nabla\cdot F=0$. Est-ce que cela peut se comprendre sur le dessin de la figure~\ref{LabelFigZGUDooEsqCWQ} ? % From file ZGUDooEsqCWQ
\newcommand{\CaptionFigZGUDooEsqCWQ}{Le champ de vecteur de la gravité. Nous avons tracé, sur les deux cercles la même densité de vecteurs, c'est-à-dire le même nombre de vecteurs par unité de surface.}
\input{auto/pictures_tex/Fig_ZGUDooEsqCWQ.pstricks}

    Essayons de voir combien de fluide entre dans la zone bleue et combien en sort. D'abord, il est certain que les vecteurs qui sortent sont plus courts que ceux qui rentrent, ce qui voudrait dire qu'il y a plus de fluide qui rentre. Mais on voit également que le \emph{nombre} de vecteurs qui sortent est plus grand parce que la seconde sphère est plus grande et qu'il y a un vecteur en chaque point de la sphère.

    Intuitivement nous pouvons dire que la quantité qui rentre dans la sphère de rayon $r_1$ donnée par la taille des vecteurs entrants multiplié par la surface de la sphère, c'est-à-dire
    \begin{equation}        \label{EqQpinormeVecto}
        4\pi r_1^2\| F(x,y,z) \|,
    \end{equation}
    mais $\| F(x,y,z) \|=\frac{1}{ r_1^2 }$, donc la quantité de fluide entrant est $4\pi$. La quantité de fluide sortant sera la même.

    Cela explique deux choses
    \begin{enumerate}
        \item
            Pourquoi les forces de gravitation et électromagnétiques sont en $1/r^2$; c'est parce que nous vivons dans un monde avec trois dimensions d'espace. En étudiant très précisément le champ de gravitation, certains physiciens espèrent trouver des déviations expérimentales par rapport à la règle du \( 1/r^2\); cela \emph{pourrait} être un signe que l'espace contient des dimensions supplémentaires.
        \item
            Pourquoi il y a un $4\pi$ comme coefficient dans beaucoup d'équations en électromagnétisme; en particulier dans certaines anciennes unités de flux.
    \end{enumerate}

\end{example}

\begin{remark}
    Nous allons voir plus loin comment s'assurer que l'équation \eqref{EqQpinormeVecto} représente bien la «quantité de fluide» qui rentre dans la zone délimitée
\end{remark}

%+++++++++++++++++++++++++++++++++++++++++++++++++++++++++++++++++++++++++++++++++++++++++++++++++++++++++++++++++++++++++++
\section{Quelques formules de Leibnitz}
%+++++++++++++++++++++++++++++++++++++++++++++++++++++++++++++++++++++++++++++++++++++++++++++++++++++++++++++++++++++++++++

La divergence étant une combinaison de dérivées, il n'est pas tellement étonnant que la divergence de produits donne lieux à des formules en deux termes. Si $f$ est une fonction et si $F$ et $G$ sont des champs de vecteurs, nous avons la proposition suivante.

\begin{proposition}     \label{PROPooDMWEooNaJBCM}
    Si \( F\) et \( G\) sont des champs de vecteurs dont toutes les dérivées partielles existent, alors    
    \begin{enumerate}
        \item
            $\nabla\cdot(fF)f\nabla\cdot F+F\cdot\nabla f$
        \item
            $\nabla\cdot(F\times G)=G\cdot\nabla\times F-F\cdot\nabla\times G$
        \item       \label{ITEMooFDJIooKTnvKj}
            $\nabla\times(fF)=f\nabla\times F+\nabla f\times F$.
    \end{enumerate}
\end{proposition}



\chapter{Analyse sur des groupes}
% This is part of Mes notes de mathématique
% Copyright (c) 2011-2013,2016-2019
%   Laurent Claessens
% See the file fdl-1.3.txt for copying conditions.

%+++++++++++++++++++++++++++++++++++++++++++++++++++++++++++++++++++++++++++++++++++++++++++++++++++++++++++++++++++++++++++
\section{Action de groupe et connexité}
%+++++++++++++++++++++++++++++++++++++++++++++++++++++++++++++++++++++++++++++++++++++++++++++++++++++++++++++++++++++++++++

Sources : \cite{MneimneLie} et \wikipedia{fr}{Matrice_normale}{wikipédia}.

\begin{theorem}     \label{ThojrLKZk}
    Soit \( G\) un groupe topologique localement compact et dénombrable à l'infini\footnote{Cela signifie qu'il est une réunion dénombrable de compacts} agissant continument et transitivement sur un espace topologique localement compact \( E\). Alors l'application
    \begin{equation}
        \begin{aligned}
            \varphi\colon G/G_x&\to E \\
            [g]&\mapsto g\cdot x
        \end{aligned}
    \end{equation}
    est un homéomorphisme.
\end{theorem}

\begin{lemma}       \label{LemkLRAet}
    Si \( G\) et \( H\) sont des groupes topologiques tels que $G/H$ et \( H\) sont connexes\footnote{Définition~\ref{DefIRKNooJJlmiD}.}, alors \( G\) est connexe.
\end{lemma}

\begin{proof}
    Soit \( f\colon G\to \{ 0,1 \}\) une fonction continue. Considérons l'application
    \begin{equation}
        \begin{aligned}
            \tilde f\colon G/H&\to \{ 0,1 \} \\
            [g]&\mapsto f(g).
        \end{aligned}
    \end{equation}
    D'abord nous montrons qu'elle est bien définie. En effet si \( h\in H\) nous aurions \( \tilde f([gh])=f(gh)\), mais étant donné que \( H\) est connexe, l'ensemble \( gH\) est également connexe; la fonction continue \( f\) est donc constante sur \( gH\). Nous avons donc \( f(gh)=f(g)\).

    Étant donné que \( G/H\) est également connexe, la fonction \( \tilde f\) doit être constante. Si \( g_1\) et \( g_2\) sont deux éléments du groupe, nous avons \( f(g_1)=\tilde f([g_1])=\tilde f([g_2])=f(g_2)\). Nous en déduisons que \( f\) est constante et que \( G\) est connexe.
\end{proof}

\begin{theorem}
    Le groupe \( \SO(n)\) est connexe, le groupe \( \gO(n)\) a deux composantes connexes.
\end{theorem}

\begin{proof}
    La seconde assertion découle de la première parce que les matrices de déterminant \( 1\) et celles de déterminant \( -1\) ne peuvent pas être reliées par un chemin continu tandis que l'application
    \begin{equation}
        M\mapsto \begin{pmatrix}
            -1    &       &       \\
                &   1    &       \\
                &       &   1
        \end{pmatrix}M
    \end{equation}
    est un homéomorphisme entre les matrices de déterminant \( 1\) et celles de déterminants \( -1\). Montrons donc que \( G=\SO(n)\) est connexe par arcs pour \( n\geq 2\) en procédant par récurrence sur la dimension.

    Nous acceptons le résultat pour $G=\SO(2)$. Notons que nous en avons besoin pour prouver que la sphère \( S^{n-1}\) est connexe.

    Le groupe \( \SO(n)\) agit, par définition, de façon transitive sur la sphère \( S^{n-1}\). Soit \( a\in S^{n-1}\), nous avons
    \begin{subequations}
        \begin{align}
            G\cdot a&=S^{n-1}\\
            G_a&\simeq \SO(n-1)
        \end{align}
    \end{subequations}
    où \( G_a\) est le fixateur de \( a\) dans \( G\). Pour montrer le second point, nous considérons \( \{ e_i \}\), la base canonique de \( \eR^n\) et \( M\in G\) telle que \( Ma=e_1\). Le fixateur de \( e_1\) est évidemment isomorphe à \( \SO(n-1)\) parce qu'il est constitué des matrices de la forme
    \begin{equation}
        \begin{pmatrix}
             1   &   0    &   \ldots    &   0    \\
             0   &   a_{11}    &   \ldots    &   a_{1,n-1}    \\
             \vdots   &   \vdots    &   \ddots    &   \vdots    \\
             0   &   a_{n-1,1}    &   \ldots    &   a_{n-1,n-1}
         \end{pmatrix}
    \end{equation}
    où \( (a_{ij})\in \SO(n-1)\). L'application
    \begin{equation}
        \begin{aligned}
            \alpha\colon G_{e_1} &\to G_{a} \\
            A&\mapsto M^{-1}A M
        \end{aligned}
    \end{equation}
    est un isomorphisme entre \( G_a\) et \( \SO(n-1)\). Le théorème~\ref{ThojrLKZk} nous montre alors que, en tant qu'espaces topologiques,
    \begin{equation}
        G/G_a=S^{n-1}.
    \end{equation}
    L'hypothèse de récurrence montre que \( G_a=\SO(n-1)\) est connexe tandis que nous savons que \( S^{n-1}\) est connexe. Le lemme~\ref{LemkLRAet} conclut que \( G=\SO(n)\) est connexe.
\end{proof}

\begin{lemma}       \label{LemIbrsFT}
    Une bijection continue entre un espace compact et un espace séparé est un homéomorphisme.
\end{lemma}

\begin{proposition}
    Les groupes \( \gU(n)\) et \( \SU(n)\) sont connexes.
\end{proposition}

\begin{proof}
    Soit \( G(n)\) le groupe \( \SU(n)\) ou \( \gU(n)\). Ce groupe opère transitivement sur la sphère complexe
    \begin{equation}
        S_{\eC}^{n-1}=\{ z\in \eC^n\tq \langle z, z\rangle=\sum_k| z_k |^2 =1 \}.
    \end{equation}
    Cet ensemble est le même que \( S^{2n-1}\) parce que \( |z_k|=x_k^2+y_k^2\). Nous avons une bijection continue entre \( S^{n-1}\) et \( S^{n-1}_{\eC}\) et donc un homéomorphisme (lemme~\ref{LemIbrsFT}). Soit \( a\in S^{n-1}_{\eC}\), nous avons
    \begin{subequations}
        \begin{align}
            G\cdot a&=S^{n-1}_{\eC}\\
            G_a&\simeq G(n-1).
        \end{align}
    \end{subequations}
    La seconde ligne est un isomorphisme de groupe et un homéomorphisme. Il est donné de la façon suivante. D'abord le fixateur de \( e_1\) dans \( G(n)\) est donné par les matrices de la forme
    \begin{equation}
        \begin{pmatrix}
             1   &   0    &   \ldots    &   0    \\
             0   &   a_{11}    &   \ldots    &   a_{1,n-1}    \\
             \vdots   &   \vdots    &   \ddots    &   \vdots    \\
             0   &   a_{n-1,1}    &   \ldots    &   a_{n-1,n-1}
         \end{pmatrix}
    \end{equation}
    où \( (a_{ij})\in G(n-1)\). Par ailleurs si \( M\) est une matrice de \( G(n)\) telle que \( Ma=e_1\), nous avons l'homéomorphisme
    \begin{equation}
        \begin{aligned}
            \alpha\colon G_{e_1}&\to G_a \\
            A&\mapsto M^{-1} AM.
        \end{aligned}
    \end{equation}
    Encore une fois, cela est un homéomorphisme par le lemme~\ref{LemIbrsFT}. Par composition nous avons \( G_a\simeq G(n-1)\) et un homéomorphisme
    \begin{equation}
        G(n)/G_a=S^{n-1}_{\eC}.
    \end{equation}
    Le groupe \( G_a\) et l'ensemble \( S^{n-1}_{\eC}\) étant connexes, le groupe \( G(n)\) est connexe par le lemme~\ref{LemkLRAet}.
\end{proof}

\begin{lemma}[\cite{PAXrsMn}]
    Si \( G\) est un sous-groupe connexe de \( \GL(n,\eC)\) alors son groupe dérivé\footnote{Définition~\ref{DEFooBNLPooShKYXa}.} l'est également.
\end{lemma}
\index{groupe dérivé!de \( \GL(n,\eC)\)}

\begin{proof}
    Soit \( S_m\) l'ensemble des produits de \( m\) commutateurs de \( G\) :
    \begin{equation}
        S_m=\{ g_1,\ldots, g_m\,\text{où les } g_i\text{ sont des commutateurs} \}.
    \end{equation}
    La partie \( S_m\) est l'image de \( G\) par l'application continue
    \begin{equation}
        \begin{aligned}
            \underbrace{G\times \ldots\times G}_{ 2m\text{ facteurs}}&\to G \\
            (g_1,h_1,g_2,h_2,\ldots, g_m,h_m)&\mapsto [g_1,h_1]\ldots [g_m,h_m]
        \end{aligned}
    \end{equation}
    En tant qu'image d'un connexe par une application continue, \( S_m\) est connexe par la proposition~\ref{PropGWMVzqb}. Vu que les \( S_m\) ont l'identité en commun, le groupe dérivé
    \begin{equation}
        D(G)=\bigcup_{m=1}^{\infty}S_m
    \end{equation}
    est également connexe.
\end{proof}

%+++++++++++++++++++++++++++++++++++++++++++++++++++++++++++++++++++++++++++++++++++++++++++++++++++++++++++++++++++++++++++
\section{Espaces de matrices}
%+++++++++++++++++++++++++++++++++++++++++++++++++++++++++++++++++++++++++++++++++++++++++++++++++++++++++++++++++++++++++++

L'ensemble des matrices est un espace vectoriel. Nous identifions $\eM(n,\eR)$ avec $ \eR^{n^2}$; plus précisément, nous identifions une matrice
\begin{equation}
    A = (a_{i,j})_{1\leq i \leq n, 1 \leq j \leq n}
\end{equation}
avec le vecteur $x = (x_1, x_2, \dots, x_{n^2}) \in \eR^{n^2}$, où $ a_{i,j} = x_{(n-1)i + j}$.

%---------------------------------------------------------------------------------------------------------------------------
\subsection{Dilatations et transvections}
%---------------------------------------------------------------------------------------------------------------------------

Soit un corps commutatif \( \eK\) et \( n\geq 2\).

\begin{theoremDef}[\cite{PAXrsMn}]     \label{ThoooAZKDooNDcznv}
    Soit une application linéaire \( u\colon E\to E\) dont les points fixes forment un hyperplan noté \( H\) d'équation \( H=\ker(f)\) avec \( f\in E^*\).
    \begin{enumerate}
        \item     \label{ITEMooGTKRooQSPNoI}
            Les affirmations suivantes sont équivalentes :
            \begin{enumerate}
                \item  \label{ITEMooZHYRooFGKaifi}
                    \( \det(u)\neq 1\)
                \item       \label{ooXKLWooTfUMzV}
                    L'application \( u\) est diagonalisable et a une valeur propre qui vaut \( \det(u)\neq 1\).
                \item       \label{ooMZPTooCLylbh}
                    \( \Image(u-\id)\nsubseteq H\).
                \item   \label{ITEMooZHYRooFGKaifiv}
                    Il existe une base de \( E\) dans laquelle la matrice de \( u\) est \( \diag(1,\ldots, 1,\lambda)\) avec \( \lambda\neq 1\).
            \end{enumerate}
        \item       \label{ITEMooMSJXooUsLCHx}
            Les affirmation suivantes sont équivalentes :
            \let\oldthenumii\theenumi
            \renewcommand{\theenumii}{\roman{enumii}}
            \begin{enumerate}
                \item       \label{ITEMooRTIEooOoWCFsa}
                    Il existe \( a\in H\) tel que pour tout \( x\in E\), \( u(x)=x+f(x)a\).
                \item       \label{ITEMooRTIEooOoWCFsb}
                    Dans une base adaptée, la matrice de \( u\) est donnée par
                    \begin{equation}        \label{EQooFXBDooTgZwMv}
                        \begin{pmatrix}
                             1   &       &       &       \\
                                &   \ddots    &       &       \\
                                &       &   1    &   1    \\
                                &       &       &   1
                         \end{pmatrix}.
                    \end{equation}
            \end{enumerate}
            \let\theenumii\oldtheenumii
        \item
            Les conditions~\ref{ITEMooZHYRooFGKaifi}-\ref{ITEMooZHYRooFGKaifiv} sont respectées si et seulement si les conditions~\ref{ITEMooRTIEooOoWCFsa}-\ref{ITEMooRTIEooOoWCFsb} ne sont pas respectées (elles sont les négations l'une de l'autre.).
    \end{enumerate}
    Un endomorphisme qui est soit l'identité soit respecte les conditions~\ref{ITEMooGTKRooQSPNoI} est une \defe{dilatation}{dilatation}. Un endomorphisme qui est soit l'identité soit qui vérifie les conditions~\ref{ITEMooMSJXooUsLCHx} est une \defe{transvection}{transvection} (dans les deux cas il faut que les points fixes forment un hyperplan).
\end{theoremDef}

Notons que selon cette terminologie, l'application \( x\mapsto \lambda x\) n'est pas une dilatation mais un produit de dilatations.

\begin{proof}
    Nous allons prouver plein d'implications \ldots
    \begin{subproof}
    \item[\ref{ITEMooZHYRooFGKaifi} implique~\ref{ooXKLWooTfUMzV}]
        Le théorème de la base incomplète (voir remarque~\ref{REMooYGJEooEcZQKa}) permet de considérer une base \( \{ e_1,\ldots, e_n \}\) de \( E\) telle que \( \{ e_1,\ldots, e_{n-1} \} \) soit une base de \( H\). Dans cette base, la matrice de \( u\) est de la forme suivante (les cases non remplies sont nulles et les étoiles correspondent à des valeurs inconnues mais pas spécialement nulles) :
        \begin{equation}        \label{EqooPQOEooGUyIwa}
        \begin{pmatrix}
             1   &       &       &   *    \\
                &   \ddots    &       &   \vdots    \\
                &       &   1    &   *    \\
                &       &       &   \lambda
         \end{pmatrix}
        \end{equation}
        Le fait que le déterminant de \( u\) ne soit pas \( 1\) implique que \( \lambda\neq 1\). Par conséquent le polynôme caractéristique
        \begin{equation}
            \chi_u(X)=(1-X)^{n-1}(\lambda-X)
        \end{equation}
        possède une racine \( \lambda\neq 1\), et donc \( u\) possède un vecteur propre \( v\) pour cette valeur\footnote{Proposition~\ref{PropooBYZCooBmYLSc}.}. Le vecteur \( v\) est linéairement indépendant de \( \{ e_1,\ldots, e_{n-1} \}\) (parce que vecteur propre de valeur propre différente). Par conséquent l'ensemble \( \{ e_1,\ldots, e_{n-1},v \}\) est une base par la proposition \ref{PROPooVEVCooHkrldw}. Cela est une base de vecteurs propres et donc une base de diagonalisation\footnote{Nous pourrions en dire à peine plus et prouver le point~\ref{ITEMooZHYRooFGKaifiv}, mais cela ne servirait à rien parce que nous voulons prouver les équivalences et qu'il faudra quand même prouver que~\ref{ooMZPTooCLylbh} implique~\ref{ITEMooZHYRooFGKaifiv}.}.
    \item[\ref{ooXKLWooTfUMzV} implique~\ref{ooMZPTooCLylbh}]
        Nous nommons maintenant \( \{ e_1,\ldots, e_{n} \}\) la base de diagonalisation. Nous avons \( u(e_n)=\lambda e_n\) avec \( \det(u)=\lambda\neq 1\). Nous avons
        \begin{equation}
            (u-\id)(e_n)=(\lambda-1)e_n\notin H,
        \end{equation}
        ce qui prouver que l'image de \( e_n\) par \( u-\id\) n'est pas dans \( H\).
    \item[\ref{ooMZPTooCLylbh} implique~\ref{ITEMooZHYRooFGKaifiv}]
        Reprenons une base \( \{ e_1,\ldots, ,e_n \}\) donnant la matrice \eqref{EqooPQOEooGUyIwa}. Il existe \( x\in E\) tel que \( u(x)-x\) n'est pas dans \( H\), c'est-à-dire tel que \( u\big( u(x)-x \big)\neq u(x)-x\). Nous en déduisons que
        \begin{equation}
            u^2(x)-2u(x)+x\neq 0
        \end{equation}
        ou encore que
        \begin{equation}
            (X-1)^2(u)x\neq 0.
        \end{equation}
        C'est-à-dire que \( (X-1)^2\) n'est pas un polynôme annulateur de \( u\). Or ce serait le cas si \( X-1\) était le polynôme minimal (proposition~\ref{PropAnnncEcCxj}). Le polynôme caractéristique étant \( (X-1)^{n-1}(X-\lambda)\) (et étant annulateur\footnote{Théorème de Cayley-Hamilton~\ref{ThoCalYWLbJQ}.}), le polynôme minimal est de la forme
        \begin{equation}
            \mu_u(X)=\begin{cases}
                (X-1)(X-\lambda)    &   \text{si } \lambda\neq 1\\
                X-1    &    \text{si } \lambda=1.
            \end{cases}
        \end{equation}
        Dans notre cas nous venons de voir que ce n'est pas \( X-1\) et donc c'est \( (X-1)(X-\lambda)\) avec \( \lambda\neq 1\).

        Nous devons trouver une base de diagonalisation \ldots Supposons
        \begin{equation}
            u(e_n)=\sum_{k=1}^{n-1}a_ke_k+\lambda e_n,
        \end{equation}
        dans lequel nous venons de prouver que \( \lambda\neq 1\), et cherchons
        \begin{equation}
            e'_n=\sum_{j=1}^np_je_j
        \end{equation}
        de telle sorte à avoir \( u(e'_n)=\lambda e_n\). Nous avons
        \begin{equation}
                u(e'_n)=\sum_{j=1}^{n-1}p_ju(e_j)+p_nu(e_n) =\sum_{j=1}^{n-1}(p_j+p_na_j)e_j+p_n\lambda e_n.
        \end{equation}
        En égalisant à \( \lambda\sum_{j=1}^np_je_j\), il vient
        \begin{equation}
            p_j+p_na_j=\lambda p_j
        \end{equation}
        pour tout \( j=1,\ldots, n-1\) et la condition triviale \( p_n\lambda=\lambda p_n\) pour \( j=n\). Nous en déduisons que le choix
        \begin{equation}
            p_j=\frac{ p_na_j }{ \lambda-1 }
        \end{equation}
        fonctionne (parce que \( \lambda\neq 1\) comme nous l'avons démontré plus haut). En bref, il suffit de poser
        \begin{equation}
            e'_n=\sum_{j=1}^{n-1}\frac{ p_na_j }{ \lambda-1 }e_j+p_ne_n
        \end{equation}
        avec \( p_n\) au choix pour avoir une base \( \{ e_1,\ldots, e_{n-1},e'_n \}\) de diagonalisation de \( u\) avec \( \lambda\neq 1\) comme dernière valeur propre.
    \item[\ref{ITEMooZHYRooFGKaifiv} implique~\ref{ITEMooZHYRooFGKaifi}] Évident \ldots encore qu'il faut invoquer l'invariance du déterminant par changement de base.
    \end{subproof}
    Nous avons terminé la première série d'équivalences. Nous continuons avec la seconde.
       \begin{subproof}
        \item[\ref{ITEMooRTIEooOoWCFsa} implique~\ref{ITEMooRTIEooOoWCFsb}]
            Nous prenons \( e_{n-1}=a\) et nous complétons en une base de \( H\). Pour \( e_n\) il suffit de prendre n'importe quel vecteur \( v\) tel que \( f(v)\neq 0\) (qui existe parce que \( f=0\) est seulement un hyperplan), et de le normaliser.

            Dans cette base, la matrice de \( u\) a la forme désirée parce que \( u(e_n)=e_n+f(e_n)a=e_n+e_{n-1}\) du fait que \( e_{n-1}=a\) et \( f(e_n)=1\).
        \item[\ref{ITEMooRTIEooOoWCFsb} implique~\ref{ITEMooRTIEooOoWCFsa}]
            Soit \( \{ e_1,\ldots, e_n \}\) cette base. En prenant \( a=e_{n-1}\) et en posant \( x=\sum_kx_ke_k\) nous avons
            \begin{equation}
                u(x)=\sum_{k=1}^{n-1}x_ke_k+x_n(e_{n-1}+e_n)=x+x_ne_{n-1}=x_na.
            \end{equation}
            Mais vu que \( f(x)=\sum_if_ix_i\), et que \( f(e_i)=0\) pour tout \( i=1,\ldots, n-1\) nous avons \( f(x)=f_nx_n\). Il n'y a cependant pas de raisons d'avoir \( f_n=1\). Cependant en définissant
            \begin{equation}
                e'_i=\frac{1}{ f_n }e_i
            \end{equation}
            nous avons bien \( u(e'_n)=\frac{1}{ f_n }(e_{n-1}+e_n)=e'_{n-1}+e'_n\). Donc dans cette base nous avons encore la matrice de \( u\) de la forme
            \begin{equation}
                \begin{pmatrix}
                     1   &       &       &       \\
                        &   \ddots    &       &       \\
                        &       &   1    &   1    \\
                        &       &       &   1
                 \end{pmatrix},
            \end{equation}
            mais cette fois avec \( f(e'_n)=1\).
    \end{subproof}
    Nous avons terminé avec la seconde série d'équivalences. Il nous reste à prouver que la première est équivalente à la négation de la seconde.
    \begin{subproof}
        \item[non~\ref{ooMZPTooCLylbh} implique~\ref{ITEMooRTIEooOoWCFsa}]
            Considérons \( x_0\in E\) tel que \( f(x_0)=1\) et posons \( a=u(x_0)-x_0\in\Image(u-\id)\). Par la négation de~\ref{ooMZPTooCLylbh} nous avons \( a\in H\). De plus \( x_0\notin H\) (sinon \( f(x_0)=0\)) donc \( u(x_0)\neq x_0\) et \( a\neq 0\).

            Nous montrons que ce choix de \( a\) fonctionne : \( u(x)=x+f(x)a\) pour tout \( x\in E\). Nous faisons cela séparément pour \( x\in H\) et pour \( x=x_0\).

            Si \( h\in H\) alors \( u(h)=h\) et \( f(h)=0\) donc \( h+f(h)a=h=u(h)\). Si \( x=x_0\) alors \( u(x_0)=a+x_0\) (cela est la définition de \( a\)) et\( x_0+f(x_0)a=x_0+a\).
        \item[\ref{ITEMooRTIEooOoWCFsb} implique non~\ref{ITEMooZHYRooFGKaifi}]
           Dans une base adaptée nous avons
           \begin{equation}
               \begin{pmatrix}
                    1    &       &       &       \\
                        &   \ddots    &       &       \\
                        &       &   1    &   1    \\
                        &       &       &   1
                \end{pmatrix},
           \end{equation}
           et donc \( \det(u)=1\), ce qui contredit~\ref{ITEMooZHYRooFGKaifi}.
    \end{subproof}
\end{proof}

\begin{remark}
    Nous notons \( E_{ij}\) la matrice qui possède uniquement \( 1\) en position \( (i,j)\). C'est-à-dire que \( \big( E_{ij} \big)_{kl}=\delta_{ik}\delta_{jl}\). Soit \( H\) l'hyperplan des points fixes de \( f\). Dans une base contenant une base de $H$, la matrice d'une transvection a pour forme type :
    \begin{equation}        \label{EqooZAKHooBjKlTd}
        T_{ij}(\lambda)=\mtu+\lambda E_{ij}
    \end{equation}
    avec \( i\neq j\) et \( \lambda\in \eK\), et une dilatation a pour forme type la matrice diagonale
    \begin{equation}
        D_i(\alpha)=\mtu+(\alpha-1)E_{ii}
    \end{equation}
    avec \( \alpha\in \eK^*\).

    Bien entendu, en choisissant une base quelconque, les matrices des dilatations et des translations peuvent avoir des formes différentes.
\end{remark}

\begin{lemma}       \label{LemooTQJXooGoIxsI}
    Quelques manipulations de lignes et de colonnes pour les matrices.
    \begin{enumerate}
        \item       \label{ITEMooRWANooPAVjkm}
            La multiplication à gauche par \( T_{ij}(\lambda)\) revient à effectuer le remplacement de ligne
            \begin{equation}
                L_i\to L_i+\lambda L_j.
            \end{equation}
        \item       \label{ITEMooHPSMooWBrSXP}
            La multiplication à droite par \( T_{ij}(\lambda)\) revient à effectuer le remplacement de colonne
            \begin{equation}
                C_j\to C_j+\lambda C_i.
            \end{equation}
        \item       \label{ITEMooXUGFooKcbrxs}
            La multiplication à gauche par \( T_{ij}(1)T_{ji}(-1)T_{ij}(1)\) revient à la substitution de lignes
            \begin{subequations}
                \begin{numcases}{}
                    L_i\to L_j\\
                    L_j\to -L_i.
                \end{numcases}
            \end{subequations}
    \end{enumerate}
\end{lemma}
Note qu'il n'est pas possible d'inverser deux lignes à l'aide de transvections sans changer un signe parce que les transvections sont de déterminant \( 1\) alors que l'inversion de lignes change le signe du déterminant.

\begin{proof}
    Point par point.
    \begin{subproof}
        \item[Pour~\ref{ITEMooRWANooPAVjkm}]
            Nous devons prouver que
            \begin{equation}
                \big( T_{ij}(\lambda)A \big)_{kl}=\begin{cases}
                    A_{kl}    &   \text{si } k\neq i\\
                    A_{il}+\lambda A_{jl}    &    \text{si } k=i.
                \end{cases}
            \end{equation}
            Un peu de calcul matriciel avec utilisation modérée des indices donner :
            \begin{subequations}
                \begin{align}
                    \big( T_{ij}(\lambda)A \big)_{kl}&=\sum_s\big( T_{ij}(\lambda) \big)_{ks}A_{sl}\\
                    &=\sum_s\delta_{ks}A_{sl}+\lambda\delta_{ik}\delta_{js}A_{sl}\\
                    &=A_{kl}+\lambda\delta_{ik}A_{jl}.
                \end{align}
            \end{subequations}
        \item[Pour~\ref{ITEMooHPSMooWBrSXP}] C'est la même chose.
        \item[Pour~\ref{ITEMooXUGFooKcbrxs}] Si nous appliquons successivement ces trois matrices (de droite à gauche) nous effectuons les substitutions :
            \begin{equation}
                \begin{aligned}[]
                \begin{cases}
                    L'_i=L_i+L_j\\
                    L'_j=L_j
                \end{cases}
                \text{suivit de }
                \begin{cases}
                    L''_i=L'_i\\
                    L''_j=L'_j-L'_i
                \end{cases}
                \text{et de}
                \begin{cases}
                    L'''_i=L''_i+L''_j\\
                    L'''_j=L''_j.
                \end{cases}
                \end{aligned}
            \end{equation}
            En effectuant ces substitutions,
            \begin{equation}
                L'''_i=L''_i+L''_j=L'_i+(L'_j-L'_i)=L'_j=L_j
            \end{equation}
            et
            \begin{equation}
                L'''_j=L''_j=L'_j-L'_i=L_j-(L_i+L_j)=-L_i,
            \end{equation}
            ce qu'il fallait.
    \end{subproof}
\end{proof}

\begin{proposition}[\cite{ooUWTWooPKySTQ}]      \label{PropooFDNRooWFfUDd}
    Soient \( n\geq 2\) et \( \eK\) un corps commutatif.
    \begin{enumerate}
        \item
            Si \( A\in \GL(n,\eK)\), il existe des transvections \( U_1,\ldots, U_r\), \( V_1,\ldots, V_s\) telles que
                \begin{equation}        \label{EQooKSQVooIpkdIE}
                    A=U_1\ldots U_r\,D_n\big( \det(A) \big)\,V_1\ldots V_s.
                \end{equation}
        \item       \label{ITEMooLRYXooSoKRiA}
            L'ensemble des transvections engendre le groupe spécial linéaire \( \SL(n,\eK)\).
        \item
            L'ensemble des transvections et des dilatations engendre le groupe linéaire \( \GL(n,\eK)\).
    \end{enumerate}
\end{proposition}


\begin{proof}
    Nous allons montrer que toutes les matrices de \( \SL(n,\eK)\) peuvent être écrites comme produits de matrices de la forme \eqref{EqooZAKHooBjKlTd}. Cela montrera qu'étant donné un endomorphisme \( f\) et une base \emph{pas spécialement liée à $f$}, il est possible décrire la matrice de \( f\) comme produit de transvections dont les hyperplans invariants sont «contenus» dans cette base. Cela suffit à prouver que les transvections engendrent \( \SL(n,\eK)\) grâce au lemme~\ref{LemFUIZooBZTCiy}.

    Toutes les transvections ont un déterminant égal à \( 1\). Donc le groupe engendré par les transvections est inclus dans \( \SL(2,\eK)\). Soit \( A\in\GL(n,\eK)\); nous allons utiliser le pivot de Gauss pour la diagonaliser. Étant donné que \( A\) est inversible, sa première colonne n'est pas nulle. Si \( A_{i1}\neq 0\) alors une multiplication à gauche par \( L_{1i}\big(   (A_{11}-1)/A_{i1}  \big)\) effectue la substitution
    \begin{equation}
        L_1\to L_1-\frac{ A_{11}-1 }{ A_{i1} }L_i
    \end{equation}
    qui met un \( 1\) en la position \( (1,1)\). Notons que si la première colonne est de la forme
    \begin{equation}
        \begin{pmatrix}
            s    \\
            0    \\
            \vdots    \\
            0
        \end{pmatrix}
    \end{equation}
    avec \( s\neq 0\) alors il faut plutôt faire les substitutions \( L_2\to L_2+L_1\) et ensuite \( L_1\to L_1-\frac{1}{ s }L_2\) pour obtenir le même résultat. En effectuant le pivot avec \( A_{11}\), une suite d'opérations sur les lignes et les colonnes donnent
    \begin{equation}
        M_1\ldots M_pAN_1\ldots N_q=\begin{pmatrix}
            1    &   0    \\
            0    &   A_1
        \end{pmatrix}
    \end{equation}
    où \( A_1\in\GL(n-1,\eK)\) et \( \det(A_1)=\det(A)\). En continuant de la sorte nous arrivons sur une matrice diagonale\footnote{Attention : les opérations sur les lignes et le colonnes ne sont pas des opérations de similitude. Il n'est pas question de prétendre ici que toutes les matrices de $ \GL(n,\eK)$ sont diagonales, voir la définition~\ref{DefBLELooTvlHoB}.}
    \begin{equation}
        M_1\ldots M_{p'}AN_1\ldots N_{q'}=
        \begin{pmatrix}
             1   &       &       &       \\
                &   \ddots    &       &       \\
                &       &   1    &       \\
                &       &       &   \alpha
         \end{pmatrix}
    \end{equation}
    avec \( \alpha=\det(A)\). En d'autres termes nous avons prouvé qu'il existe des transvections \( U_1,\ldots, U_r\) et \( V_1,\ldots, V_s\) telles que
    \begin{equation}        \label{EQooZYYFooQGCgxU}
        A=U_1\ldots U_r\,D_n\big( \det(A) \big)\,V_1\ldots V_s.
    \end{equation}
    Cela prouve que les transvections et les translations engendrent \( \GL(n,\eK)\). Si \( A\in \SL(n,\eK)\) alors \( D_n\big( \det(A) \big)=1\) et l'équation \eqref{EQooZYYFooQGCgxU} est un produit de transvections.
\end{proof}

\begin{proposition}
    Le groupe \( \GL(n,\eR)\) est engendré par les endomorphismes inversibles diagonalisables.
\end{proposition}

\begin{proof}
    Par la proposition~\ref{PropooFDNRooWFfUDd}, le groupe \( \GL(n,\eR)\) est engendré par les dilatations et les transvections. Il suffit donc de montrer qu'à leur tour, ces deux types d'endomorphismes sont engendrés par les endomorphismes inversibles et diagonalisables.

    Les dilatations sont diagonalisables et inversibles. C'est bon pour elles.
    
    Soit une transvection \( u\), et une base \( \{ e_i \}_{i=1,\ldots, n}\) dans laquelle \( u\) est de la forme \eqref{EQooFXBDooTgZwMv}. Nous considérons l'endomorphisme \( d\colon E\to E\) défini par \( d(e_k)=ke_k\). Cet endomorphisme est diagonalisable parce que son polynôme minimal, \( \mu_d=\prod_{k=1}^n(X-k)\), est scindé à racines simples (voir le théorème~\ref{ThoDigLEQEXR}).

    Nous avons évidemment \( u= d^{-1}\circ(d\circ u) \) où \( d^{-1}\) est diagonalisable et inversible. Voyons que \( d\circ u\) est également diagonalisable en montrant que \( \mu_d\) est son polynôme minimal (qui est scindé à racines simples).

    Il suffit de montrer que \( \mu_d(d\circ u)(e_k)=0\) pour tout \( k\). Ainsi \( \mu_d\) sera un polynôme annulateur de \( d\circ u\) de degré \( n\), et donc minimal.
    \begin{subproof}
        \item[Si \( k\leq n-1\)]
            Alors \( u(e_k)=e_k\) et \( (d\circ u-n)e_k=(k-n)e_k\). En tout :
            \begin{equation}
                \mu_d(d\circ u)(e_k)=(d\circ u-1)(d\circ u-2)\ldots (d\circ u-n)e_k=(k-1)(k-2)\ldots (k-n)e_k=0
            \end{equation}
            parce que dans le produit des \( k-i\), il y en a forcément un de nul.
        \item[Si \( k=n\)]
            Dans un premier temps,
            \begin{equation}
                (d\circ u-n)e_n=d(e_n+e_{n-1})-ne_n=ne_n+(n-1)e_{n-1}-ne_n=(n-1)e_{n-1}.
            \end{equation}
            Ensuite
            \begin{subequations}
                \begin{align}
                    \big( d\circ u-(n-1) \big)e_{n-1}&=d(e_{n-1})-(n-1)e_{n-1}\\
                    &=d(e_{n-1})-(n-1)e_{n-1}\\
                    &=(n-1)e_{n-1}-(n-1)e_{n-1}\\
                    &=0
                \end{align}
            \end{subequations}
    \end{subproof}
    Le polynôme \( \mu_d\) est donc un polynôme scindé à \( n\) racines simples annulateur de \( d\circ u\), qui est alors diagonalisable et inversible (parce que \( u\) et \( d\) le sont).

    Donc sous la forme \( u=d^{-1}(du)\), la transvection \( u\) est écrite comme produit de diagonalisables inversibles.
\end{proof}

\begin{proposition}[\cite{LoFdlw}]
    Soient \( n\geq 3\) et \( \eK\) un corps de caractéristique différente de \( 2\). Alors
    \begin{enumerate}
        \item
            le groupe dérivé de \( D(\GL(n,\eK))\) est \(\SL(n,\eK)\);  \index{groupe!dérivé!de \( \GL(n,\eK)\)}
        \item
            le groupe dérivé de \( \SL(n,\eK)\) est \( \SL(n,\eK)\).\index{groupe!dérivé!de \( \SL(n,\eK)\)}
    \end{enumerate}
\end{proposition}
La preuve utilise le fait que les transvections engendrent \( \SL(n,\eK)\) et que les transvections avec les dilatations engendrent \( \GL(n,\eK)\). Voir la proposition~\ref{PropooFDNRooWFfUDd}.
%TODO : faire une preuve de cela. C'est dans l'écrit d'algèbre de 2013.

%---------------------------------------------------------------------------------------------------------------------------
\subsection{Connexité  de certains groupes}
%---------------------------------------------------------------------------------------------------------------------------

\begin{lemma}           \label{LEMooIPOVooZJyNoH}
    Le groupe \( \gO(n,\eR)\) n'est pas connexe.
\end{lemma}

\begin{proof}
    La non connexité par arcs est facile parce que les éléments de déterminant \( 1\) ne peuvent pas être reliés aux éléments de déterminant \( -1\) par un chemin continu restant dans \( \gO(n)\) à cause du théorème des valeurs intermédiaires~\ref{ThoValInter}.

    En ce qui concerne la connexité, il faut en dire un peu plus.

    Les éléments de \( \gO(n,\eR)\) ont des déterminants égaux à \( 1\) ou à \( -1\). Ces deux parties sont des ouverts (pour la topologie induite de \( \eM(n,\eR)\)). En effet soit \( A\in\SO(n,\eR)\) (la partie contenant les déterminants \( 1\); ce que l'on va dire tient pour l'autre partie). Alors, vu que le déterminant est une fonction continue sur \( \eM(n,\eR)\) il existe un voisinage \( \mO\) de \( A\) dans \( \eM(n,\eR)\) dans lequel le déterminant reste entre \( \frac{ 1 }{2}\) et \( \frac{ 3 }{2}\) (c'est la définition de la continuité avec \( \epsilon=1/2\)). L'ensemble \( \mO\cap\gO(n,\eR)\) est par définition un ouvert de \( \gO(n,\eR)\) et ne contient que des éléments de déterminant \( 1\).

    La partie \( \gO(n,\eR)\) de \( \eM(n,\eR)\) est donc non-connexe selon la définition~\ref{DefIRKNooJJlmiD}.
\end{proof}

\begin{lemma}       \label{LEMooQMXHooZQozMK}
    Les groupes \( \gU(n)\) et \( \SU(n)\) sont connexes par arcs.
\end{lemma}

\begin{proof}
    Soient \( A\), une matrice unitaire et \( Q\) une matrice unitaire qui diagonalise \( A\). Étant donné que les valeurs propres arrivent par paires complexes conjuguées,
    \begin{equation}
        QAQ^{-1}=\begin{pmatrix}
            e^{i\theta_1}    &       &       &       &   \\
            &    e^{-i\theta_1}    &       &       &   \\
            &       &    \ddots    &       &   \\
            &       &       &    e^{i\theta_r}    &   \\
            &       &       &       &        e^{-i\theta_r}
        \end{pmatrix}.
    \end{equation}
    Le chemin \( U(t)\) obtenu en remplaçant \( \theta_i\) par \( t\theta_i\) avec \( t\in\mathopen[ 0 , 1 \mathclose]\) joint \( QAQ^{-1}\) à l'identité. Par conséquent \( Q^{-1}U(t)Q\) joint \( A\) à l'unité.
\end{proof}

\begin{proposition}     \label{PROPooYKMAooCuLtyh}
    Le groupe \( \SO(n)\) est connexe.
\end{proposition}

\begin{theorem}
    Les matrices \wikipedia{fr}{Endomorphisme_normal}{normales}\footnote{Définition~\ref{DefWQNooKEeJzv}.} forment un espace connexe par arc.
\end{theorem}

\begin{proof}
    Soient \( A\) une matrice normale et \( U\) une matrice unitaire qui diagonalise \( A\). Nous considérons \( U(t)\), un chemin qui joint \( \mtu\) à \( U\) dans \( \gU(n)\). Pour chaque \( t\), la matrice
    \begin{equation}
        A(t)=U(t)^{-1} AU(t)
    \end{equation}
    est normale. Nous avons donc trouvé un chemin dans les matrices normales qui joint \( A\) à une matrice diagonale. Il est à présent facile de la joindre à l'identité.

    Toutes les matrices normales étant connexes à l'identité, l'ensemble des matrices normales est connexe.
\end{proof}

\begin{proposition}     \label{PROPooALQCooLZCKrH}
    Le groupe \( \SL(n,\eK)\) est connexe par arcs.
\end{proposition}

\begin{proof}
    Soit \( A\in \SL(n,\eK)\); par la proposition~\ref{PropooFDNRooWFfUDd}\ref{ITEMooLRYXooSoKRiA} nous pouvons écrire
    \begin{equation}
        A=\prod_{c\in X}T_c(\lambda_c)
    \end{equation}
    où \( X\) est une partie de l'ensemble des couples \( (i,j)\) dans \( \{ 1,\ldots, n \}\). En posant
    \begin{equation}
        \begin{aligned}
            \varphi\colon \mathopen[ 0 , 1 \mathclose]&\to \SL(n,\eK) \\
            t&\mapsto \prod_{c\in X}T_c(t\lambda_c)
        \end{aligned}
    \end{equation}
    nous avons une application continue de \( A\) vers \( \mtu\), dont pour tout \( t\) la matrice \( \varphi(t)\) est inversible de déterminant\( 1\).

    Donc tous les éléments de \( \SL(n,\eK)\) peuvent être reliés à \( \mtu\). Donc \( \SL(n,\eK)\) est connexe par arcs.
\end{proof}

\begin{proposition}[\cite{ooGKOIooXKUQKk}]\label{PROPooVJNIooMByUJQ}
    Le groupe \( \GL(n,\eC)\) est connexe par arcs.
\end{proposition}

\begin{proof}
    Soient \( A\in\GL(n,\eC)\) et sa décomposition \eqref{EQooKSQVooIpkdIE}. Comme fait précédemment, chacune des transvections peut être reliée à \( \mtu\) par un chemin continu dans \( \SL(n,\eC)\). En ce qui concerne le facteur de translation,  nous ne pouvons pas simplement prendre le chemin donné par \( t\mapsto D_n\big( t\det(A) \big)\) parce que le résultat n'est pas inversible en \( t=0\).

    Vu que \( C^*\) il existe une application continue \( \alpha\colon \mathopen[ 0 , 1 \mathclose]\to \eC^*\) telle que \( \alpha(0)=\det(A)\in \eC^*\) et \( \alpha(1)=1\). Il suffit alors de prendre \( D_n\big( \alpha(t) \big)\) et nous avons un chemin continu de \( A\) vers \( \mtu\) restant dans \( \GL(n,\eC)\).
\end{proof}

\begin{proposition} \label{PROPooBIYQooWLndSW}
    Le groupe \( \GL(n,\eR)\) a exactement deux composantes connexes par arcs.
\end{proposition}
\index{connexité!le groupe \( \GL^+(n,\eR)\)}

\begin{proof}
    Nous notons \( \GL^+(n,\eR)\) et \( \GL^-(n,\eR)\) les parties de \( \GL(n,\eR)\) formées des applications de déterminant \( \pm1\) respectivement. Vu le théorème des valeurs intermédiaires (théorème~\ref{ThoValInter}), il n'existe pas d'applications continues dans \( \GL(n,\eR)\) reliant \( \GL^+(n,\eR)\) à \( \GL^-(n,\eR)\) tout en restant dans les applications de déterminant non nul\footnote{Si \( \varphi\colon \mathopen[ 0 , 1 \mathclose]\to \GL(n,\eR)\) est le chemin, la fonction à mettre dans le théorème des valeurs intermédiaires est la fonction \( f\colon \mathopen[ 0 , 1 \mathclose]\to \eR\) \(t\mapsto \det\big( \varphi(t) \big)\).}.

    Montrons que \( \GL^{\pm}(n,\eR)\) sont connexes par arcs. Si \( A\in\GL^+(n,\eR)\) alors grâce à la décomposition \eqref{EQooKSQVooIpkdIE}, il existe un chemin continu de \( A\) vers \( D_n\big( \det(A) \big)\). Vu que \( \eR^{\pm}\) sont connexes par arc, il est possible de relier \( D_n\big( \det(A) \big)\) à \( D_n(\pm 1)\) par un chemin continu.
\end{proof}

%---------------------------------------------------------------------------------------------------------------------------
\subsection{Densité}
%---------------------------------------------------------------------------------------------------------------------------

\begin{proposition}     \label{PropDigDensVxzPuo}
    Les matrices diagonalisables sont denses dans \( \eM(n,\eC)\).
\end{proposition}
\index{densité!matrices diagonalisables dans \( \eM(n,\eC)\)}

\begin{proof}
    D'après le lemme de Schur~\ref{LemSchurComplHAftTq}, une matrice de \( \eM(n,\eC)\) est de la forme
    \begin{equation}
        A=Q\begin{pmatrix}
            \lambda_1    &   *    &   *    \\
              0  &   \ddots    &   *    \\
            0    &   0    &   \lambda_n
        \end{pmatrix}Q^{-1}.
    \end{equation}
    Les valeurs propres sont sur la diagonale. La matrice est diagonalisable si les éléments de la diagonales sont tous différents. Il suffit maintenant de considérer \( n\) suites \( (\epsilon^{(r)}_k)_{k\in\eN}\) convergentes vers zéro telles que pour chaque \( k\) les nombres \( \lambda_r+\epsilon^{(r)}_k\) soient tous différents. La suite de matrices
    \begin{equation}
        A_k=Q\begin{pmatrix}
            \lambda_1+\epsilon^{(1)}_k    &   *    &   *    \\
              0  &   \ddots    &   *    \\
              0    &   0    &   \lambda_n+\epsilon^{(n)}_k
        \end{pmatrix}Q^{-1}.
    \end{equation}
    est alors diagonalisable pour tout \( k\) et nous avons \( \lim_{k\to \infty} A_k=A\).
\end{proof}

\begin{proposition} \label{PropQGUPooVudelJ}
    Les matrices inversibles sont denses dans l'ensemble des matrices. C'est-à-dire que \( \GL(n,\eR)\) est dense dans \( \eM(n,\eR)\).
\end{proposition}
\index{densité!de \( \GL(n,\eR)\) dans \( \eM(n,\eR)\)}

\begin{proof}
    Soit \( A\in \eM(n,\eR)\); le lemme de Schur réel~\ref{LemSchureRelnrqfiy} nous permet d'écrire
    \begin{equation}
        A=Q
        \begin{pmatrix}
            \lambda_1    &       &       &       &   \\
                &   \ddots    &       &       &   \\
                &       &  \lambda_r     &       &   \\
                &  &     &   \begin{pmatrix}
                    a    &   b    \\
                    c    &   d
                \end{pmatrix}&          \\
                &       &       &       &   \ddots
        \end{pmatrix}
        Q^{-1}
    \end{equation}
    avec \( Q\) orthogonale.

    Pour définir \( A_k\) nous remplaçons \( \lambda_i\) par \( \lambda_i+\epsilon^{(i)_k}\) de façon à avoir \( \epsilon^{(i)}_k\to 0\) et \( \lambda_i+\epsilon^{(i)}_k\neq 0\). En ce qui concerne les blocs, ceux dont le déterminant est non nul, nous n'y touchons pas, et ceux dont le déterminant est nul, nous remplaçons \( a\) par \( a+\epsilon_k\).

    Avec cela, \( QA_kA^{-1}\) est une suite dans \( \GL(n,\eR)\) qui converge vers \( A\).
\end{proof}

\begin{proposition}     \label{PROPooZUHOooQBwfZq}
    Si \( A\in\eM(n,\eC)\) alors
    \begin{equation}
        e^{\tr(A)}=\det( e^{A}).
    \end{equation}
\end{proposition}

\begin{proof}
    Ici, \( e^A\) est l'exponentielle soit d'endomorphisme soit de matrice définie par la proposition~\ref{PropPEDSooAvSXmY}.

    Le résultat est un simple calcul pour les matrices diagonalisable. Si \( A\) n'est pas diagonalisable, nous considérons une suite de matrices diagonalisables \( A_k\) dont la limite est \( A\) (proposition~\ref{PropDigDensVxzPuo}). La suite
    \begin{equation}
        a_k= e^{\tr(A_k)}
    \end{equation}
    converge vers \(  e^{\tr(A)}\) tandis que la suite
    \begin{equation}
        b_k=\det( e^{A_k})
    \end{equation}
    converge vers \( \det( e^{A})\). Mais nous avons \( a_k=b_k\) pour tout \( k\); les limites sont donc égales.
\end{proof}

\begin{corollary}       \label{CORooOKKSooHrsYOs}
    Si \( A\in\eM(n,\eC)\) alors
    \begin{equation}
        \Dsdd{ \det( e^{tX}) }{t}{0}=\tr(X).
    \end{equation}
\end{corollary}

\begin{proof}
    Nous écrivons la proposition \ref{PROPooZUHOooQBwfZq} pour \( tX\) au lieu de \( X\); pour chaque \( t\) nous avons
    \begin{equation}
        \det\big(  e^{tA} \big)= e^{\tr(tA)}= e^{t\tr(A)}.
    \end{equation}
    La dérivation par rapport à \( t\) en \( t=0\) donne le résultat attendu.
\end{proof}

\begin{theorem}[Cayley-Hamilton\cite{QATooFIHVMw,MOSooRVRrHw}]  \label{ThoHZTooWDjTYI}
    Tout endomorphisme d'un espace vectoriel de dimension finie sur un corps commutatif quelconque annule son propre polynôme caractéristique
\end{theorem}
\index{théorème!Cayley-Hamilton}
% position EYRooJkxiFf

Une autre démonstration est donnée en le théorème~\ref{ThoCalYWLbJQ}.
\begin{proof}
    La preuve est divisée en plusieurs étapes.
    \begin{subproof}
        \item[Endomorphisme diagonalisable]
            Soit \( u\) un endomorphisme sur un espace vectoriel \( V\) de dimension \( n\) sur un corps \( \eK\) et \( \chi_u\) sont polynôme caractéristique. Nous savons que si \( \lambda\) est une valeur propre de \( u\) alors \( \chi_u(\lambda)=0\) le théorème~\ref{ThoWDGooQUGSTL}\ref{ItemeXHXhHii}. En combinant avec le lemme~\ref{LemVISooHxMdbr}, si \( x\) est vecteur propre pour la valeur propre \( \lambda\) de \( u\) nous avons
            \begin{equation}
                \chi_u(u)x=\chi_u(\lambda)x=0.
            \end{equation}
            Donc tant que \( u\) possède une base de vecteurs propres nous avons \( \chi_u(u)=0\).

        \item[Le cas complexe]

            Nous nous restreignons à présent (et provisoirement) au cas \( \eK=\eC\), ce qui nous donne \( u\in \eM(n,\eC)\). Les matrices diagonalisables sont denses dans \( \eM(n,\eC)\) par la proposition~\ref{PropDigDensVxzPuo}. Si \( A\in \eM(n,\eC)\) nous considérons une suite de matrices diagonalisables \( A_k\stackrel{\eM(n,\eC)}{\longrightarrow}A\). Pour chaque \( k\) nous avons par le point précédent
            \begin{equation}
                \chi_{u_k}(u_k)=0.
            \end{equation}
            Chacune des composantes de \( \chi_{u_k}(u_k)\) est un polynôme en les composantes de \( u_k\), ce qui légitime le passage à la limite :
            \begin{equation}
                \chi_u(u)=0.
            \end{equation}
            Le théorème est établi pour toutes les matrices de \( \eM(n,\eC)\) et donc aussi pour tous les sous-corps de \( \eC\) comme \( \eR\) ou \( \eZ\).

        \item[La cas général]

            Par définition, \( \chi_u(X)=\det(u-X\mtu)\); les coefficients de \( X\) sont des polynômes à coefficients entiers en les composantes de \( u\). En substituant \( u\) à \( X\) nous obtenons une matrice dont chacune des entrées est un polynôme à coefficients entiers en les coefficients de \( u\). Pour chaque \( i\) et \( j\) entre \( 1\) et \( n\) il existe donc un polynôme \( P_{ij}\in \eZ(X_1,\ldots, X_{n^2})\) tel que
            \begin{equation}
                \chi_u(u)_{ij}=P(u_{11},\ldots, u_{nn}).
            \end{equation}
            Ces polynômes ne dépendent pas de \( u\) ni du corps sur lequel on travaille. Notre but est maintenant de prouver que \( P_{ij}=0\).

            Étant donné que le cas complexe (et a fortiori entier) est déjà prouvé nous savons que pour tout \( u\in \eM(n,\eZ)\) nous avons \( P(u_{11},\ldots, u_{nn})=0\). La proposition~\ref{PropTETooGuBYQf} nous donne effectivement \( P=0\), en conséquence de quoi l'endomorphisme \( \chi_u(u)\) est nul.

    \end{subproof}
\end{proof}

\begin{example}
    Pour montrer que chaque composante \( \chi_u(u)\) est bien un polynôme à coefficients entiers en les coefficients de \( u\), voyons l'exemple \( 2\times 2\) : \( u=\begin{pmatrix}
        a    &   b    \\
        c    &   d
    \end{pmatrix}\). D'abord
    \begin{equation}
        \chi_u(X)=\det\begin{pmatrix}
            a-X    &   b    \\
            c    &   d-X
        \end{pmatrix}=X^2-(a+d)X+ad-cb.
    \end{equation}
    Le coefficient de \( X^2\) est \( 1\), celui de \( X\) est \( -a-d\) et le terme indépendant est \( ad-cb\); tout trois sont des polynômes à coefficients entiers en \( a,b,c,d\). Après substitution de \( X\) par \( u\),
    \begin{equation}
        \chi_u(u)_{ij}=(u^2)_{ij}-(a+d)u_{ij}+ad-cb.
    \end{equation}
    Cela est bien un polynôme à coefficients entiers en les entrées de la matrice \( u\).
\end{example}

%---------------------------------------------------------------------------------------------------------------------------
\subsection{Racine carrée d'une matrice hermitienne positive}
%---------------------------------------------------------------------------------------------------------------------------

\begin{propositionDef}     \label{PropVZvCWn}
    Si \( A\in \eM(n,\eC)\) est une matrice hermitienne\footnote{Définition~\ref{DEFooKEBHooWwCKRK}.} positive, alors il existe une unique matrice hermitienne positive \( R\) telle que \( A=R^2\). De plus \( R\) est un polynôme (de \( \eR[X]\)) en \( A\).


    La matrice \( R\) ainsi définie est la \defe{racine carrée de}{matrice!racine carrée}\index{racine!carré de matrice!hermitienne positive} de \( A\), et est notée \( \sqrt{A}\)\nomenclature[A]{\( \sqrt{A}\)}{racine d'une matrice hermitienne positive}.
\end{propositionDef}
\index{matrice!semblable}
\index{polynôme!d'endomorphisme}
\index{endomorphisme!diagonalisable}
\index{matrice!hermitienne!racine carrée}
\index{racine!carré!de matrice hermitienne}


\begin{proof}
    \begin{subproof}
    \item[Existence]
        Étant donné que \( A \) est hermitienne, elle est diagonalisable par une matrice unitaire (proposition~\ref{ThogammwA}), et ses valeurs propres sont réelles et positives (parce que \( A\) est positive). Soit donc \( P\) une matrice unitaire telle que
        \begin{equation}
            P^*AP=\begin{pmatrix}
                \alpha_1    &       &       \\
                    &   \ddots    &       \\
                    &       &   \alpha_n
            \end{pmatrix}
        \end{equation}
        avec \( \alpha_i>0\). Si on pose
        \begin{equation}
            R=P\begin{pmatrix}
                \sqrt{\alpha_1}    &       &       \\
                    &   \ddots    &       \\
                    &       &   \sqrt{\alpha_n}
            \end{pmatrix}P^*,
        \end{equation}
        alors \( R^2=A\) parce que \( P^*P=\mtu\).
    \item[Hermitienne positive]
        La matrice \( R\) est hermitienne parce que, avec un peu de notation raccourcie, \( R=P^*\sqrt{\alpha}P\) et \( R^*=P^*\sqrt{\alpha}P\). D'autre part, elle est positive parce que ses valeurs propres sont les \( \sqrt{\alpha_i}\) qui sont positives.

    \item[Polynôme]
        Nous montrons maintenant que la matrice \( R\) est un polynôme en \( A\). Pour cela nous considérons un polynôme \( Q\) tel que \( A(\alpha_i)=\sqrt{\alpha_i}\) pour tout \( i\). Soit \( \{ e_i \}\) une base de diagonalisation de \( A\) : \( Ae_i=\alpha_ie_i\). Alors c'est encore une base de diagonalisation de \( Q(A)\). En effet si \( Q=\sum_ka_kX^k\), alors
        \begin{equation}
            Q(A)e_i=(\sum_ka_kA^k)e_i=(\sum_ka_k\alpha_i^k)e_i=Q(\alpha_i)e_i=\sqrt{\alpha_i}e_i.
        \end{equation}
        Les valeurs propres de \( Q(A)\) sont donc \( \sqrt{\alpha_i}\). Nous savons maintenant que \( Q(A)\) a la même base de diagonalisation de \( A\) (et donc la même matrice unitaire \( P\) qui diagonalise), c'est-à-dire que
        \begin{equation}
            Q(A)=P^*\begin{pmatrix}
                \sqrt{\alpha_1}    &       &       \\
                    &   \ddots    &       \\
                    &       &   \sqrt{\alpha_n}
            \end{pmatrix}=R.
        \end{equation}
        Donc oui, \( R\) est un polynôme en \( A\).

        Notons que ce \( Q\) n'est pas du tout unique; il existe une infinité de polynômes envoyant \( n\) nombres donnés sur \( n\) nombres donnés.

    \item[Unicité]
        Soit \( S\) une matrice hermitienne positive telle que \( R^2=S^2=A\). D'abord \( S\) commute avec \( A\) parce que
        \begin{equation}
            SA=S^3=S^2S=AS.
        \end{equation}
        Donc \( S\) commute aussi avec \( Q(A)=R\). Étant donné que \( S\) et \( R\) commutent et sont diagonalisables, ils sont simultanément diagonalisables par le corolaire~\ref{CorQeVqsS}. Soient \( D_R=PRP^*\) et \( D_S=PSP^*\) les formes diagonales de \( R\) et \( S\) dans une base de simultanée diagonalisation. Les carrés des valeurs propres de \( R\) et \( S\) étant identiques (ce sont les valeurs propres de \( A\)) et les valeurs propres de \( R\) et \( S\) étant positives, nous déduisons que \( D_R=D_S\) et donc que \( R=P^*D_RP=P^*D_SP=S\).
    \end{subproof}
\end{proof}

Une des applications usuelles de cette proposition est la décomposition polaire.

%---------------------------------------------------------------------------------------------------------------------------
\subsection{Racine carrée d'une matrice symétrique positive}
%---------------------------------------------------------------------------------------------------------------------------

\begin{lemma}[\cite{JJdQPyK}]   \label{LemTLlTAAf}
    Le groupe orthogonal \( \gO(n,\eR) \) est compact.
\end{lemma}

\begin{proof}
    Nous avons \( O(n)=f^{-1}\big( \{ \mtu_n \} \big)\) où \( f\) est l'application continue \( A\mapsto A^tA\). En tant qu'image inverse d'un fermé par une application continue, le groupe \( O(n)\) est fermé.

    De plus il est borné parce que tous les coefficients d'une matrice orthogonale sont \( \leq 1\), donc \( \| A \|_{\infty}\) pour tout \( A\in O(n)\).
\end{proof}

\begin{proposition} \label{PropPEMDqVT}
    Une matrice symétrique semi (ou pas) définie positive admet une unique racine carrée symétrique. Le spectre de la racine carrée est la racine carrée du spectre de la matrice de départ.
\end{proposition}

\begin{proof}
    Ceci est une phrase pour que les titres se mettent bien.
    \begin{subproof}
        \item[Existence]
            Soit \( T\) une matrice symétrique et \( Q\) une matrice orthogonale qui diagonalise\footnote{Théorème~\ref{ThoeTMXla}.} \( T\) : \( QTQ^{-1}=D\) avec \( D=\diag(\lambda_i)\) et \( \lambda_i\geq 0\). En posant \( R=Q^{-1}\sqrt{D}Q\), il est vite vérifié que \( R^2=T\) et que \( R\) est symétrique. En ce qui concerne le spectre, \( R\) a pour valeurs propres les \( \sqrt{\lambda_i}\).
        \item[Unicité]

            Soit \( R\) une matrice symétrique de \( T\) : \( R^2=T\). Du coup \( R\) et \( T\) commutent : \( RT=R^3=TR\). Par conséquent les espaces propres de \( T\) sont stables sous \( R\). Soit \( E_{\lambda} \) l'un d'eux de dimension \( d\), et \( T_F\), \( R_F\) les restrictions de \( T\) et \( R\) à \( E_{\lambda}\). L'application \( T_F\) est une homothétie et \( R_F^2=T_F=\lambda\mtu\). Mais \( R_F\) est encore une matrice symétrique définie positive, donc nous pouvons considérer une base \( \{ e_1,\ldots, e_d \}\) de \( E_{\lambda}\) qui diagonalise \( R_F\) avec les valeurs propres \( \mu_i\); nous avons donc en même temps
            \begin{subequations}
                \begin{align}
                    R_f^2(e_i)&=\mu_i^2 e_i\\
                    T_F(e_i)&=\lambda e_i,
                \end{align}
            \end{subequations}
            de telle sorte que \( \mu_i^2=\lambda\). Mais les valeurs propres de \( R_F\) sont positives, sont \( \mu_i=\sqrt{\lambda}\) pour tout \( i\). En conclusion \( R_F\) est univoquement déterminé par la donnée de \( T\). Vu que cela est valable pour tous les espaces propres de \( T\) et que ces espaces propres engendrent tout \( E\), l'opérateur \( R\) est déterminé de façon univoque par \( T\).
    \end{subproof}
\end{proof}
Notons que nous n'avons démontré l'unicité qu'au sein des matrices symétriques.

%---------------------------------------------------------------------------------------------------------------------------
\subsection{Décomposition polaires : cas réel}
%---------------------------------------------------------------------------------------------------------------------------

Nous rappelons que \( S^{++}(n,\eR)\) est l'ensemble des matrices symétriques strictement définies positives définies en~\ref{NORMooAJLHooQhwpvr}.

\begin{lemma}   \label{LemMGUSooPqjguE}
    La partie \( S^+(n,\eR)\) est fermée dans \( \eM(n,\eR)\).
\end{lemma}

\begin{proof}
    En effet si \( S_k\) est une suite de matrices symétriques convergeant dans \( \eM(n,\eR)\) vers la matrice \( A\), les suites \( (S_k)_{ij}\) et \( (S_k)_{ji}\) des composantes \( ij\) et \( ji\) sont des suites égales, et donc leurs limites sont égales\footnote{Ici nous utilisons le critère de convergence composante par composante et le fait que nous ne sommes pas trop inquiétés par la norme que nous choisissons parce que toutes les normes sont équivalentes par le théorème~\ref{ThoNormesEquiv}.}. Donc la limite est symétrique.

    En ce qui concerne le spectre, le théorème~\ref{ThoeTMXla} nous permet de diagonaliser : \( S_k=Q_kD_kQ_k^{-1}\) où les \( D_k\) sont des matrices diagonales remplies de nombres positifs ou nuls. Vu que \( O(n)\) est compact\footnote{Lemme~\ref{LemTLlTAAf}.}, nous avons une sous-suite \( Q_{\varphi(k)}\) convergente : \( Q_{\varphi(k)}\to Q\). Pour chaque \( k\), nous avons
    \begin{equation}
        S_{\varphi(k)}=Q_{\varphi(k)}D_{\varphi(k)}Q^{-1}_{\varphi(k)},
    \end{equation}
    dont la limite existe et vaut \( A\). Vu que pour tout \( k\), \( D_{\varphi(k)}=Q^{-1}_{\varphi(k)}S_{\varphi(k)}Q_{\varphi(k)}\) et que le produit matriciel est continu, la suite \( k\mapsto D_{\varphi(k)}\) est une suite convergente dans \( \eM(n,\eR)\). Nous notons \( D\) sa limite qui est encore une matrice diagonale contenant des nombres positifs ou nuls sur la diagonale.
    \begin{equation}
        A=\lim_{k\to \infty } S_{\varphi(k)}=QDQ^{-1},
    \end{equation}
    et donc le spectre de \( A\) est la limite de ceux des matrices \( D_{\varphi(k)}\). Chacun étant positif, la limite est positive. Donc \( A\in S^+(n,\eR)\).
\end{proof}

\begin{lemma}   \label{LemZKJWqIP}
    La fermeture de l'ensemble des matrices symétriques strictement définies positives est l'ensemble des matrices définies positives : \( \overline{ S^{++}(n,\eR) }=S^+(n,\eR)\).
\end{lemma}
\index{densité!de \( S^+(n,\eR)\) dans \( S^{++}(n,\eR)\)}

\begin{proof}
    Le lemme~\ref{LemMGUSooPqjguE} nous a à peine dit que \( S^+(n,\eR)\) était fermé. Nous devons prouver que pour tout élément de \( S^+(n,\eR)\), il existe une suite \( (S_k)\) dans \( S^{++}(n,\eR)\) convergeant vers \( S\).

    Si \( S\in S^+(n,\eR)\) alors nous avons la diagonalisation
    \begin{equation}
        S=QDQ^{-1} =Q
        \begin{pmatrix}
            \lambda_1    &       &       \\
                &   \ddots    &       \\
                &       &   \lambda_n
        \end{pmatrix}
        Q^{-1}
    \end{equation}
    où \( \lambda_i\geq 0\) pour tout \( i\). Nous définissons
    \begin{equation}
        D_k=
        \begin{pmatrix}
            \lambda_1+\epsilon^{(1)}_k    &       &       \\
                &   \ddots    &       \\
                &       &   \lambda_n+\epsilon^{(n)}_k
        \end{pmatrix}
    \end{equation}
    où \( \epsilon^{i}_k\) est une suite convergent vers \( 0\) telle que \( \lambda_i+\epsilon^{(i)}_n>0\) pour tout \( n\). Typiquement si \( \lambda_i>0\) alors \( \epsilon^{(i)}_k=0\) et sinon \( \epsilon^{(i)}_k=1/k\).

    Pour tout \( k\) nous avons \( QD_kQ^{-1}\in S^{++}(n,\eR)\) et de plus \( QD_kQ^{-1}\to QDQ=S\).
\end{proof}

\begin{theorem}[Décomposition polaire de matrices symétriques définies positives\cite{JJdQPyK,AABkVai,WWBTooITOwEn}] \label{ThoLHebUAU}
   En ce qui concerne les matrices inversibles :
   \begin{equation}
       \begin{aligned}
           f\colon O(n,\eR)\times S^{++}(n,\eR)&\to \GL(n,\eR) \\
           (Q,S)&\mapsto SQ
       \end{aligned}
   \end{equation}
   est un homéomorphisme\footnote{Cela est en réalité en difféomorphisme, voir la remarque~\ref{RemBJCBooGLiRmG}.}.

   En ce qui concerne les matrices en général :
   \begin{equation}
       \begin{aligned}
           g\colon O(n,\eR)\times S^+(n,\eR)&\to \eM(n,\eR) \\
           (Q,S)&\mapsto SQ
       \end{aligned}
   \end{equation}
   est une surjection mais pas une injection.

   De plus les mêmes conclusions tiennent si nous regardons \( (Q,S)\mapsto QS\) au lieu de \( SQ\).
\end{theorem}
\index{groupe!linéaire!décomposition polaire}
\index{endomorphisme!décomposition!polaire}
\index{décomposition!polaire}

%TODO : prouver le difféomorphisme.
%TODO : je crois qu'on doit pouvoir prouver que les éléments de la décomposition polaire sont des polynômes en M.

\begin{proof}
    Nous commençons par prouver les résultats concernant les matrices inversibles.
    \begin{subproof}
        \item[Existence et unicité]

            Si \( M=SQ\), alors \( MM^t=SQQ^tS^t=S^2\), donc \( S\) doit être une racine carrée symétrique de la matrice définie positive \( MM^t\). La proposition~\ref{PropPEMDqVT} nous dit que ça existe et que c'est unique. Donc \( S\) est univoquement déterminé par \( M\). Maintenant avoir \( Q=MS^{-1}\) est obligatoire (unicité) et fonctionne :
            \begin{equation}
                Q^tQ=(S^{-1})^tM^tMS^{-1}=S^{-1}S^2S^{-1}=\mtu,
            \end{equation}
            donc \( Q\) ainsi défini est orthogonale.

            Notons que ceci ne fonctionne pas lorsque \( M\) n'est pas inversible parce qu'alors \( S\) n'est pas inversible.

        \item[Homéomorphisme]

            Le fait que \( f\) soit continue n'est pas un problème : c'est un produit de matrices. Nous devons vérifier que \( f^{-1}\) est continue. Soit une suite convergente \( M_k\to M\) dans \( \GL(n,\eR)\). Si nous nommons \( (Q_k,S_k)\) la décomposition polaire de \( M_k\) et \( (Q,S)\) celle de \( M\), nous devons prouver que \( Q_k\to Q\) et \( S_k\to S\). En effet dans ce cas nous aurions
            \begin{equation}    \label{EqJIkoaJv}
                \lim_{k\to \infty} f^{-1}(M_k)=\lim_{k\to \infty} (Q_k,S_k)=(Q,S)=f^{-1}(M).
            \end{equation}

            Étant donné que \( O(n)\) est compact (lemme~\ref{LemTLlTAAf}), la suite \( (Q_k)\) admet une sous-suite convergente (Bolzano-Weierstrass, théorème~\ref{ThoBWFTXAZNH}) que nous nommons
            \begin{equation}
                Q_{\varphi(k)}\to F\in O(n).
            \end{equation}
            Vu que la suite \( (M_k)\) converge, sa sous-suite converge vers la même limite : \( M_{\varphi(k)}\to M\) et vu que pour tout \( k\) nous avons \( S_k=M_kQ_k^{-1}\),
            \begin{equation}
                S_{\varphi(k)}\to G=MF^{-1}.
            \end{equation}
            Vu que chacune des matrices \( S_{\varphi(k)}\) est symétrique définie positive, la limite est symétrique et semi-définie positive\footnote{Lemme~\ref{LemZKJWqIP}}. Donc \( G\in S^+(n,\eR)\cap \GL(n,\eR)\) parce que de plus \( M\) et \( F\) étant inversibles, \( G\) est inversible. En ce qui concerne la sous-suite nous avons
            \begin{equation}
                M_{\varphi(k)}=S_{\varphi(k)}Q_{\varphi(k)}\to GF=M
            \end{equation}
            où \( F\in O(n)\) et \( G\in S^+(n,\eR)\). Par unicité de la décomposition polaire de \( M\) (partie déjà démontrée), nous avons \( G=S\) et \( F=Q\).

            Nous avons prouvé que toute sous-suite convergente de \( Q_k\) a \( Q\) pour limite. Donc la suite elle-même converge\footnote{Proposition~\ref{PropHNylIAW}, pas difficile.} vers \( Q\). Donc \( Q_k\to Q\). Du coup vu que \( S_k=M_kQ_k^{-1}\) est un produit de suites convergentes, \( S_k\) converge également, vers \( S\) :  \( S_k\to S\).

            Au final l'application \( f^{-1}\) est bien continue parce que les égalités \eqref{EqJIkoaJv} ont bien lieu.
    \end{subproof}

    Nous passons maintenant à la preuve dans le cas des matrices en général.

    Soit \( A\in \eM(n,\eR)\); par densité (lemme~\ref{PropQGUPooVudelJ}), il existe une suite \( (A_k)\) dans \( \GL(n,\eR)\) telle que \( A_k\to A\). Pour chacun des \( k\) nous appliquons la décomposition polaire déjà prouvée : \( A_k=Q_kS_k\). D'abord \( (Q_k)\) est une suite dans le compact\footnote{Lemme~\ref{LemTLlTAAf}.} \( \gO(n,\eR)\) et accepte donc une sous-suite convergente. Quitte à redéfinir la suite de départ, nous supposons pour alléger les notations que \( Q_k\to Q\in \gO(n,\eR)\). Vu que \( Q_k\) est inversible,
    \begin{equation}
        S_k=Q^{-1}_kA_k
    \end{equation}
    Le produit matriciel étant continu nous avons \( S_k\to S\) dans \( \eM(n,\eR)\). Mais \( S^+(n,\eR)\) étant fermé (lemme~\ref{LemMGUSooPqjguE}) nous avons aussi \( S\in S^+(n,\eR)\).
\end{proof}

\begin{remark}  \label{RemBJCBooGLiRmG}
    Pour démontrer que \( f\) est différentiable, nous devons utiliser le théorème d'inversion locale~\ref{ThoXWpzqCn}; cela est fait dans la proposition~\ref{PropWCXAooDuFMjn}.
\end{remark}

\begin{corollary}       \label{CorAWYBooNCCQSf}
    Toute matrice peut être écrite sous la forme \( Q_1DQ_2\) où \( Q_1\) et \( Q_2\) sont orthogonales et \( D\) est diagonale.
\end{corollary}

\begin{proof}
    Si \( A\in\eM(n,\eR)\) alors la décomposition polaire~\ref{ThoLHebUAU} nous donne \( A=SQ\) où \( S\) est symétrique définie positive et \( Q\) est orthogonale. La matrice \( S\) peut ensuite être diagonalisée par le théorème~\ref{ThoeTMXla} : \( S=RDR^{-1}\) où \( D\) est diagonale et \( R\) est orthogonale. Avec ces deux décompositions en main, \( A=SQ=RDR^{-1}Q\). La matrice \( R^{-1}Q\) est orthogonale.
\end{proof}

%---------------------------------------------------------------------------------------------------------------------------
\subsection{Enveloppe convexe}
%---------------------------------------------------------------------------------------------------------------------------

\begin{definition}
    Sur \( C\) est un ensemble convexe, un point \( x\in C\) est un \defe{point extrémal}{extrémal!point dans un convexe} si \( C\setminus\{ x \}\) est encore convexe.
\end{definition}

\begin{theorem}[\cite{KXjFWKA}] \label{ThoBALmoQw}
    Soit \( E\) un espace euclidien de dimension \( n\geq 1\) et \( \aL(E)\) l'espace des opérateurs linéaires sur \( E\) sur lequel nous considérons la norme subordonnée\footnote{Définition~\ref{DefNFYUooBZCPTr}.} à celle sur \( E\). L'ensemble des points extrémaux de la boule unité fermée de \( \aL(E)\) est le groupe orthogonal \( O(n,\eR)\).
\end{theorem}
\index{densité!points extrémaux dans \( \aL\)}

\begin{proof}
    Nous notons \( \mB\) la boule unité fermée de \( \aL(E)\). Montrons pour commencer que les éléments de \( O(n)\) sont extrémaux dans \( \mB\). D'abord si \( A\in O(E)\) alors \( \| A \|=1\) parce que \( \| Ax \|=\| x \|\). Supposons maintenant que \( A\) n'est pas extrémal, c'est-à-dire qu'il est le milieu d'un segment joignant deux points (distincts) de la boule unité de \( \aL(E)\). Soient donc \( T,U\in\mB\) tels que \( A=\frac{ 1 }{2}(T+U)\). Pour tout \( x\in E\) tel que \( \| x \|=1\) nous avons
    \begin{equation}    \label{EqKTuAIIE}
        1=\| x \|=\| Ax \|=\frac{ 1 }{2}\| Tx+Ux \|\leq \frac{ 1 }{2}\big( \| Tx \|+\| Ux \| \big)\leq\frac{ 1 }{2}\big( \| T \|+| U | \big)\leq 1
    \end{equation}
    Toutes les inégalités sont en réalité des égalités. En particulier nous avons
    \begin{equation}
        \| Tx+Ux \|=\| Tx \|+\| Ux \|,
    \end{equation}
    mais alors nous sommes dans un cas d'égalité dans l'inégalité de Cauchy-Schwarz (théorème~\ref{ThoAYfEHG}) et donc il existe \( \lambda\geq 0\) tel que \( Tx=\lambda Ux\). Mais de plus les \sout{inégalité} égalités \eqref{EqKTuAIIE} nous donnent
    \begin{equation}
        \frac{ 1 }{2}\big( \| Tx \|+\| Ux \| \big)=1
    \end{equation}
    alors que nous savons que \( \| Tx \|,\| Ux \|\leq 1\), donc \( \| Tx \|=\| Ux \|=1\). La seule possibilité est d'avoir \( \lambda=1\) et donc que \( U=T\) parce que nous avons \( Tx=Ux\) pour tout \( x\) de norme \( 1\). Au final \( A\) n'est pas le milieu d'un segment dans \( \mB\).

    Nous passons donc à l'inclusion inverse : nous prouvons que les points extrémaux de \( \mB\) sont dans \( O(E)\). Pour cela nous prenons \( U\in\mB\setminus O(E)\) et nous allons montrer que \( U\) n'est pas un point extrémal : nous allons l'écrire comme milieu d'un segment dans \( \mB\).

    Par la seconde partie du théorème de décomposition polaire~\ref{ThoLHebUAU}, il existe \( Q\in O(n,\eR)\) et \( S\in S^+(n,\eR)\) tels que \( U=QS\). Nous diagonalisons \( S\) à l'aide de la matrice orthogonale \( P\) :
    \begin{equation}
        S=PDP^{-1}
    \end{equation}
    avec \( D=\diag(\lambda_i)\). En termes de normes, nous avons
    \begin{equation}
        \| U \|=\| S \|=\| S \|.
    \end{equation}
    En effet vu que \( Q\) est orthogonale, \( \| Ux \|=\| QSx \|=\| Sx \|\) pour tout \( x\), donc \( \| U \|=\| S \|\). De plus pour tout \( x\) nous avons
    \begin{equation}
        \| Sx \|=\| PDP^{-1} x \|=\| DP^{-1}x \|.
    \end{equation}
    Étant donné que \( P^{-1}\) est une bijection, le supremum des \( \| Sx \|\) sera le même que celui des \( \| Dx \|\) et donc \( \| S \|=\| D \|\). Étant donné que par définition \( \| U \|\leq 1\), nous avons aussi \( \| D \|\leq 1\) et donc \( 0\leq\lambda_i\leq 1\) (pour rappel, les valeurs propres de \( D\) sont positives ou nulles parce que \( S\) est ainsi).

    Comme \( U\notin O(E)\), au moins une des valeurs propres n'est pas \( 1\), supposons que ce soit \( \lambda_1\). Alors nous avons \( \alpha,\beta\in\mathopen[ -1 , 1 \mathclose]\) avec \( -1\leq \alpha<\beta\leq 1\) et \( \lambda_1=\frac{ 1 }{2}(\alpha+\beta)\). Nous posons alors
    \begin{subequations}
        \begin{align}
            D_1=\diag(\alpha,\lambda_2,\ldots, \lambda_n)\\
            D_2=\diag(\beta,\lambda_2,\ldots, \lambda_n).
        \end{align}
    \end{subequations}
    Nous avons bien \( D_1\neq D2\) et \( D_1+ D_2=D\). Par conséquent
    \begin{equation}
        U=\frac{ 1 }{2}\big( QPD_1P^{-1}+QPD_2P^{-1} \big)
    \end{equation}
    avec \( QPD_1P^{-1}\neq QPD_2^{-1}\). La matrice \( U \) est donc le milieu d'un segment. Reste à montrer que ce segment est dans \( \mB\). Pour ce faire, prenons \( x\in E\) et calculons :
    \begin{equation}
        \| QPD_iP^{-1}x \|=\| D_iP^{-1}x \|\leq\| P^{-1}x \|=\| x \|
    \end{equation}
    parce que \( \| D_i \|\leq 1\) et \( P^{-1}\) est orthogonale. Au final la norme de \( QPD_iP\) est plus petite que \( 1\) et donc \( U\) est bien le milieu d'un segment dans \( \mB\), et donc non extrémal.
\end{proof}

\begin{theorem}[\cite{NHXUsTa}] \label{ThoVBzqUpy}
    L'enveloppe convexe de \( O(n)\) dans \( \eM_n(\eR)\) est la boule unité pour la norme induite de \( \| . \|_2\) sur \( \eR^n\).
\end{theorem}
\index{convexité!enveloppe de $O(n)$}
\index{groupe!linéaire!enveloppe convexe de $\Omega(n)$}

\begin{proof}
    Nous notons \( \mB\) la boule unité fermée de \( \eM(n,\eR)\) et \( \Conv\big( O(n,\eR) \big)\) l'enveloppe convexe de \( O(n,\eR)\). Vu que \( \mB\) est convexe nous avons \( \Conv\big( O(n) \big)\subset\mB\).


    Maintenant nous devons prouver l'inclusion inverse. Pour ce faire nous supposons avoir un élément \( A\in \mB\setminus\Conv\big( O(n) \big)\) et nous allons dériver une contradiction.

    Remarquons que \( O(n)\) est compact par le lemme~\ref{LemTLlTAAf} et que par conséquent \( \Conv(O(n))\) est compacte par le corolaire~\ref{CorOFrXzIf} et donc fermée. Nous considérons un produit scalaire \( (X,Y)\mapsto X\cdot Y\) sur \( \eM\). Vu que \( \Conv\big( O(n) \big)\) est un fermé convexe nous pouvons considérer la projection\footnote{Le théorème de projection : théorème~\ref{ThoWKwosrH}.} sur \( \Conv(A)\) relativement au produit scalaire choisi.

    Nous notons \( P=\pr_{\Conv\big( O(n) \big)}(A)\). En vertu du théorème de projection, nous avons
    \begin{equation}    \label{EqYSisLTL}
        (A-P)\cdot (M-P)\leq 0
    \end{equation}
    pour tout \( M\in\Conv O(n)\). Notons \( B=A-P\) pour alléger les notations. L'équation \eqref{EqYSisLTL} s'écrit
    \begin{equation}    \label{EqQDLZqXQ}
        B\cdot M\leq B\cdot P.
    \end{equation}
    D'autre par vu que \( B \neq 0\) nous avons \( B\cdot B> 0\), c'est-à-dire \( B\cdot (A-P)>0\) et donc
    \begin{equation}
        B\cdot A>B\cdot P.
    \end{equation}
    En combinant avec \eqref{EqQDLZqXQ},
    \begin{equation}        \label{EqIQNlwql}
        B\cdot M\leq B\cdot P<B\cdot A.
    \end{equation}
    Nous utilisons maintenant la décomposition polaire, théorème~\ref{ThoLHebUAU}, pour écrire \( B=QS\) avec \( Q\in O(n)\) et \( S\in S^+(n,\eR)\). Vu que l'inégalité \eqref{EqIQNlwql} tient pour tout \( M\in\Conv(O(n))\), elle tient en particulier pour \( Q\in O(n)\). Donc
    \begin{equation}
        B\cdot Q=B\cdot A.
    \end{equation}
    Nous nous particularisons à présent au produit scalaire \( (X,Y)\mapsto\tr(X^tY)\) de la proposition~\ref{PropMAQoKAg}. D'abord
    \begin{equation}    \label{EaHVxWdau}
        B\cdot Q=\tr(B^tQ)=\tr(S^tQ^tQ)=\tr(S^t)=\tr(S),
    \end{equation}
    et ensuite l'inégalité \eqref{EaHVxWdau} devient
    \begin{equation}
        \tr(S)<B\cdot A=\tr(S^tQ^tA).
    \end{equation}
    Nous choisissons une basse \( \{ e_i \}\) diagonalisant \( S\) : \( Se_i=\lambda_ie_i\) vérifiant automatiquement \( \lambda_i\geq 0\) parce que \( S\) est semi-définie positive\footnote{Définition~\ref{DefAWAooCMPuVM}.}. Alors
    \begin{subequations}
        \begin{align}
            \tr(S)&<\tr(S^tQ^tA)\\
            &=\sum_i\langle S^tQ^tAe_i, e_i\rangle \\
            &=\sum_i\langle Ae_i, QSe_i\rangle \\
            &\leq \sum_i \| Ae_i \| | \lambda_i | \underbrace{\| Qe_i \|}_{=1} \\
            &\leq \sum_i\lambda_i   & A\in\mB\Rightarrow\| Ae_i \|\leq 1\\
            &=\tr(S).
        \end{align}
    \end{subequations}
    Il faut noter que la première inégalité est stricte, et donc nous avons une contradiction.
\end{proof}

%---------------------------------------------------------------------------------------------------------------------------
\subsection{Décomposition de Bruhat}
%---------------------------------------------------------------------------------------------------------------------------

\begin{theorem}[Décomposition de Bruhat]\index{Bruhat (décomposition)}\index{décomposition!Bruhat}    \label{ThoizlYJO}
    Soit \( \eK\) un corps; un élément \( M\in\GL(n,\eR)\) s'écrit sous la forme
    \begin{equation}
        M=T_1P_{\sigma}T_2
    \end{equation}
    où \( T_1\) et \( T_2\) sont des matrices triangulaires supérieures inversibles et où \( P_{\sigma}\) est une matrice de permutations \( \sigma\in S_n\). De plus il y a unicité de \( \sigma\).
\end{theorem}
\index{groupe!permutation}
\index{groupe!linéaire}
\index{matrice}

\begin{proof}
    Afin de rendre les choses plus visuelles, nous nous permettons de donner des exemples au fur et à mesure de la preuve. Nous prenons l'exemple de la matrice
    \begin{equation}
        \begin{pmatrix}
            1    &   3    &   4    \\
            2    &   5    &   6    \\
            0    &   7    &   8
        \end{pmatrix}.
    \end{equation}
    \begin{subproof}
    \item[Existence]
        Soit \( M\in \GL(n,\eR)\); vu qu'elle est inversible, on a un indice \( i_1\) maximum tel que \( M_{i_1,1}\neq 0\). Nous changeons toutes les lignes jusque là, c'est-à-dire que nous faisons, pour \( 1\leq i< i_1\),
        \begin{equation}        \label{EqGHUbwR}
            L_i\to L_i-\frac{ M_{i1} }{ M_{i_11} }L_{i_1}.
        \end{equation}
        Voir le lemme~\ref{LemooTQJXooGoIxsI}\ref{ITEMooXUGFooKcbrxs}.

        Nous avons donc obtenu une matrice dont la première colonne est nulle sauf la case numéro \( i_1\). L'opération \eqref{EqGHUbwR} revient à considérer la multiplication par la matrice de transvection
        \begin{equation}
            T_1^{(i)}=T_{ii_1}\left( -\frac{ M_{i1} }{ M_{i_11} } \right)
        \end{equation}
        pour tout \( i<i_1\). Pour rappel nous ne changeons que les lignes \emph{au-dessus} de la \( i_1\). Du coup les matrices \( T^{(i)}_1\) sont triangulaires supérieures. Nous avons donc la nouvelle matrice \( M_1=\left( \prod_{i<i_1}T_1^{(i)} \right)M\) pour laquelle toute la première colonne est nulle sauf un élément.

        Dans le cas de l'exemple, le «pivot» sera la ligne \( (2,5,6)\) et la matrice se transforme à l'aide de la matrice \( T_1=T_{12}(-1/2)\) :
        \begin{equation}    \label{EqyjXIYf}
            \begin{pmatrix}
                1    &   -1/2    &   0    \\
                0    &   1    &   0    \\
                0    &   0    &   1
            \end{pmatrix}
            \begin{pmatrix}
                1    &   3    &   4    \\
                2    &   5    &   6    \\
                0    &   7    &   8
            \end{pmatrix}=
            \begin{pmatrix}
                0    &   1/2    &   1    \\
                2    &   5    &   6    \\
                0    &   7    &   8
            \end{pmatrix}.
        \end{equation}


    Maintenant nous faisons de même avec les colonnes (en renommant \( M\) la matrice obtenue à l'étape précédente) :
    \begin{equation}
        C_j\to C_j-\frac{ M_{i_1j} }{ M_{i_11} }C_1,
    \end{equation}
    qui revient à multiplier à droite par les matrices \( T_{1j}(\frac{ M_{i_1i} }{ M_{i_11} })\) avec \( j>1\). Encore une fois ce sont des matrices triangulaires supérieures.

    Dans l'exemple, pour traiter la seconde colonne, nous multiplions \eqref{EqyjXIYf} à droite par la matrice \( T_{12}(-5/2)\) :
    \begin{equation}
            \begin{pmatrix}
                0    &   1/2    &   1    \\
                2    &   5    &   6    \\
                0    &   7    &   8
            \end{pmatrix}
            \begin{pmatrix}
                1    &   -5/2    &   0    \\
                0    &   1    &   0    \\
                0    &   0    &   1
            \end{pmatrix}=
            \begin{pmatrix}
                0    &   1/2    &   1    \\
                2    &   0    &   6    \\
                0    &   7    &   8
            \end{pmatrix}.
    \end{equation}
    Appliquer encore la matrice \( T_{13}(-6/2)\) apporte la matrice
    \begin{equation}
        \begin{pmatrix}
            0    &   1/2    &   1    \\
            2    &   0    &   0    \\
            0    &   7    &   8
        \end{pmatrix}.
    \end{equation}
    Enfin nous multiplions la matrice obtenue par \( \frac{1}{ M_{i_11} }\mtu\) pour normaliser à \( 1\) l'élément «pivot» que nous avions choisit. Dans notre exemple nous multiplions par \( 1/2\) pour trouver
    \begin{equation}        \label{Eqduglwu}
        \begin{pmatrix}
            0    &   1/4    &   1/2    \\
            1    &   0    &   0    \\
            0    &   7/2    &   4
        \end{pmatrix}.
    \end{equation}

    La matrice obtenue jusqu'ici possède une ligne et une colonne de zéros avec un \( 1\) à leur intersection, et elle est de la forme
    \begin{equation}
        M'=T_1MT_2
    \end{equation}
    où \( T_1\) et \( T_2\) sont triangulaires supérieures et inversibles, produits de matrices de transvection (et d'une matrice scalaire pour la normalisation).

    Il reste à recommencer l'opération avec la seconde colonne (qui n'est pas toute nulle parce que le déterminant est encore non nul) puis la suivante etc. Dans notre exemple de l'équation \eqref{Eqduglwu}, nous éliminerions le \( 1/4\) et le \( 4\) en utilisant le \( 7/2\).

    Encore une fois tout cela se fait à l'aide de matrice supérieures parce qu'à chaque étape, les colonnes précédent le pivot sont déjà nulles (saut un \( 1\)) et ne doivent donc pas être touchées.

    À la fin de ce processus, ce qui reste est une matrice \( TMT'\) qui ne contient plus que un seul \( 1\) sur chaque ligne et chaque colonne, c'est-à-dire une matrice de permutations : \( P_{\sigma}=TMT'\) et donc
    \begin{equation}
        M=T^{-1}_{\sigma}(T')^{-1}.
    \end{equation}

        \item[Unicité]

            Soient \( \sigma,\sigma\in S_n'\) tels que \( T_1P_{\sigma}T_2=S_1P_{\tau}S_2\) avec \( T_i\) et \( S_i\) triangulaires supérieures et inversibles. En posant \( T=T_2S_2^{-1}\) et \( S=T_1^{-1}S_1\), nous avons
            \begin{equation}
                P_{\sigma}T=SP_{\tau}
            \end{equation}
            où \( S\) et \( T\) sont des matrices triangulaires supérieures et inversibles. Par les calculs de la preuve du lemme~\ref{LemyrAXQs},
            \begin{subequations}
                \begin{numcases}{}
                    (P_{\sigma}T)_{kl}=T_{\sigma^{-1}(k)l}\\
                    (SP_{\tau})_{kl}=S_{k\tau(l)},
                \end{numcases}
            \end{subequations}
            et donc
            \begin{equation}    \label{EqKlmgOT}
                T_{\sigma^{-1}(k)l}=S_{k\tau(l)}.
            \end{equation}
            En écrivant cette équation avec \( k=\sigma(i)\) (nous rappelons que \( \sigma\) est bijective),
            \begin{equation}
                T_{il}=S_{\sigma(i)\tau(l)}.
            \end{equation}
            Nous savons que les termes diagonaux de \( T\) sont non nuls parce que \( T\) est triangulaire supérieure et inversible (donc pas de colonnes entières nulles). Nous avons donc, en prenant \( i=l=k\),
            \begin{equation}
                0\neq T_{kk}=S_{\sigma(k)\tau(k)}.
            \end{equation}
            La matrice étant triangulaire supérieure, cela implique
            \begin{equation}    \label{EqEmiBTX}
                \sigma(k)\leq\tau(k).
            \end{equation}
            De la même manière en écrivant \eqref{EqKlmgOT} avec \( l=\tau^{-1}(i)\),
            \begin{equation}
                S_{ki}=T_{\sigma^{-1}(k)\tau^{-1}(i)}
            \end{equation}
            et donc
            \begin{equation}
                \sigma^{-1}(k)\leq \tau^{-1}(k).
            \end{equation}
            En écrivant cela avec \( k=\sigma(j)\), nous avons \( j\leq \tau^{-1}\sigma(j)\) et en appliquant enfin \( \tau\),
            \begin{equation}
                \tau(j)\leq \sigma(j).
            \end{equation}
            En comparant avec \eqref{EqEmiBTX}, nous avons \( \sigma=\tau\).
    \end{subproof}
\end{proof}

%+++++++++++++++++++++++++++++++++++++++++++++++++++++++++++++++++++++++++++++++++++++++++++++++++++++++++++++++++++++++++++
\section{Sous-groupes du groupe linéaire}
%+++++++++++++++++++++++++++++++++++++++++++++++++++++++++++++++++++++++++++++++++++++++++++++++++++++++++++++++++++++++++++

\begin{lemma}[\cite{KXjFWKA}]       \label{LemOCtdiaE}
    Soit \( V\) un espace vectoriel de dimension finie muni d'une norme euclidienne \( \| . \|\). Soit \( K\) un compact convexe de \( V\) et \( G\), un sous-groupe compact de \( \GL(V)\) tel que
    \begin{equation}
        u(K)\subset K
    \end{equation}
    pour tout \( u\in G\). Alors il existe \( a\in K\) tel que \( u(a)=a\) pour tout \( u\in G\).
\end{lemma}
\index{groupe!linéaire!sous-groupes compacts}
\index{compacité!sous-groupes du groupe linéaire}

\begin{proof}
    Avant de nous lancer dans la preuve, nous avons besoin d'un petit résultat.
    \begin{subproof}
        \item[Un pré-résultat]

        Nous commençons par prouver que si \( v\in \aL(V)\) vérifie \( v(K)\subset K\), alors \( v\) a un point fixe dans \( K\). Pour cela nous considérons \( x_0\in K\) et la suite
        \begin{equation}
            x_k=\frac{1}{ k+1 }\sum_{i=0}^kv^i(x_0).
        \end{equation}
        Étant donné que \( K\) est convexe et stable par \( v\), la suite \( (x_k)\) est contenue dans \( K\) et accepte une sous-suite convergente\footnote{C'est Bolzano-Weierstrass, théorème~\ref{ThoBWFTXAZNH}.} que nous allons noter \( x_{\varphi(n)}\) avec \( \varphi\colon \eN\to \eN\) strictement croissante. Soit \( a\in K\) la limite :
        \begin{equation}
            \lim_{n\to \infty} x_{\varphi(n)}=a.
        \end{equation}
        Tant que nous y sommes nous pouvons aussi calculer \( v(x_k)\) :
        \begin{subequations}
            \begin{align}
                v(x_k)&=v\left( \frac{1}{ k+1 }\sum_{i=1}^kv^i(x_0) \right)\\
                &=\frac{1}{ k+1 }\sum_{i=0}^kv^{i+1}(x_0)\\
                &=x_k+\frac{1}{ k+1 }\Big( v^{k+1}(x_0)-x_0 \Big).      \label{EqUAfcaKG}
            \end{align}
        \end{subequations}
        La norme \( \| v^{k+1}(x_0)-x_0 \|\) est bornée par le diamètre de \( K\), donc en prenant la limite \( k\to \infty\) le second terme de \eqref{EqUAfcaKG} tend vers zéro. En prenant ces égalités en \( k=\varphi(n)\) et en prenant \( n\to\infty\), nous trouvons
        \begin{equation}
            v(a)=a,
        \end{equation}
        c'est-à-dire le résultat que nous voulions dans un premier temps.

    \item[Une norme sur \( V\)]

        Nous passons maintenant à la preuve du lemme. D'abord nous remarquons que le groupe \( G\) agit sur \( V\) par \( u\cdot x=u(x)\) et de plus, considérant la fonction continue
        \begin{equation}
            \begin{aligned}
                \alpha\colon G&\to V \\
                u&\mapsto u(x),
            \end{aligned}
        \end{equation}
        nous voyons que les orbites de cette action sont compactes en tant qu'image par \( \alpha\) du compact \( G\) (théorème~\ref{ThoImCompCotComp}). Nous posons
        \begin{equation}
            \begin{aligned}
                \nu\colon V&\to \eR^+ \\
                x&\mapsto \max_{u\in G}\| u(x) \|.
            \end{aligned}
        \end{equation}
        Cette définition a un sens parce que l'orbite \( \{ u(x)\tq u\in G \}\) est compacte dans \( V\) et donc l'ensemble des normes est compact dans \( \eR\) et admet un maximum. De plus cela donne une norme sur \( V\) parce que nous vérifions les conditions de la définition~\ref{DefNorme} :
        \begin{enumerate}
            \item
                Pour tout \( x,y\in V\) nous avons :
                \begin{equation}
                    \nu(x+y)=\max_{u\in G}\| u(x)+u(y) \|\leq \max_{u\in G}\left( \| u(x) \|+\| u(y) \| \right)\leq \nu(x)+\nu(y).
                \end{equation}
            \item
                Si \( \nu(x)=0\), alors l'égalité \( \max_{u\in G}\| u(x) \|=0\) nous enseigne que \( \| u(x) \|=0\) pour tout \( u\in G\) et donc en particulier avec \( u=\id\) nous trouvons \( x=0\).
            \item
                Pour tout \( \lambda\in \eR\) et \( x\in V\),
                \begin{equation}
                    \nu(\lambda x)=\max_{u\in G}\| u(\lambda x) \|=\max\| \lambda u(x) \|=\max| \lambda |\| u(x) \|=| \lambda |\nu(x).
                \end{equation}
        \end{enumerate}
        De plus la fonction \( \nu\) est constante sur les orbites de \( G\).

    \item[Un point fixe]

        Pour tout \( u\in G\) nous posons
        \begin{equation}
            F_u=\{ x\in K\tq u(x)=x \};
        \end{equation}
        par le pré-résultat, aucun de ces ensembles n'est vide. Ils sont de plus tous fermés par continuité de \( u\) (le complémentaire est ouvert). Nous devons prouver que \( \bigcap_{u\in G}F_u\neq \emptyset\) parce qu'une intersection serait un point fixe de tous les éléments de \( G\). Supposons donc que \( \bigcap_{u\in G}F_u=\emptyset\). Alors les complémentaires des \( F_u\) forment un recouvrement ouvert de \( K\) et nous pouvons en extraire un sous-recouvrement fini par compacité. Soient \( \{ u_i \}_{i=1,\ldots, p}\) les éléments qui réalisent ce recouvrement. Alors
        \begin{equation}
            \bigcap_{i=1}^pF_{u_i}=\emptyset.
        \end{equation}
        Nous considérons l'opérateur
        \begin{equation}
            v=\frac{1}{ p }\sum_{i=1}^pu_i\in\aL(V).
        \end{equation}
        Vu que \( K\) est convexe et stable sous chacun des \( u_i\), nous avons aussi \( v(K)\subset K\) et donc il existe \( a\in K\) tel que \( v(a)=a\). Pour ce \( a\), nous avons
        \begin{subequations}
            \begin{align}
                \nu\big( v(a) \big)&=\nu\left( \frac{1}{ p }\sum_{i=1}^pu_i(a) \right)      \label{EqDXSnwPb}\\
                &\leq \frac{1}{ p }\sum_{i=1}^p\nu\left( u_i(a) \right)\\
                &=\frac{1}{ p }\sum_{i=1}^p\nu(a)\\
                &=\nu(a)
            \end{align}
        \end{subequations}
        où nous avons utilisé la constance de \( \nu\) sur les orbites de \( G\). Par ailleurs nous savons que \( v(a)=a\), donc en réalité à gauche dans \eqref{EqDXSnwPb} nous avons \( \nu(a)\) et toutes les inégalités sont des égalités. Nous avons en particulier
        \begin{equation}        \label{EqBMjypoV}
                \nu\left( \sum_{i=1}^pu_i(a) \right) =\sum_{i=1}^p\nu\left( u_i(a) \right).
        \end{equation}
        Notons \( u_0\in G\) l'élément qui réalise le maximum de la définition de \( \nu\) pour le vecteur \( \sum_iu_i(a)\) :
        \begin{equation}
            \nu\left( \sum_i u_i(a) \right)=\| u_0\left( \sum_iu_i(a) \right) \|\leq\sum_i\| u_0u_i(a) \|\leq \sum_i\nu\big( u_i(a) \big).
        \end{equation}
        Mais nous venons de voir (équation \eqref{EqBMjypoV}) que l'expression de gauche est égale à celle de droite. Donc les inégalités sont des égalités et en particulier la première inégalité devient l'égalité
        \begin{equation}
            \| \sum_iu_0u_i(a)  \|=\sum_i\| u_0u_i(a) \|.
        \end{equation}
        En vertu du lemme~\ref{LemLPOHUme}, il existe des nombres positifs \( \lambda_i\) tels que
        \begin{equation}
            u_0u_1(a)=\lambda_2u_0u_2(a)=\ldots =\lambda_pu_0u_p(a).
        \end{equation}
        Du fait que \( u_0\) est inversible nous avons aussi
        \begin{equation}       \label{EqSTQwfIl}
            u_1(a)=\lambda_2u_2(a)=\ldots =\lambda_pu_p(a).
        \end{equation}
        Mais par constance de \( \nu\) sur les orbites nous avons \( \nu(u_i(a))=\nu(u_j(a))\) pour tout \( i\) et \( j\); en appliquant \( \nu\) à la série d'égalités \eqref{EqSTQwfIl}, nous trouvons que tous les \( \lambda_i\) doivent être égaux à \( 1\). En particulier
        \begin{equation}
            u_1(a)=u_2(a)=\ldots =u_p(a).
        \end{equation}

        Nous récrivons maintenant l'équation \( v(a)=a\) avec la définition de \( v\) :
        \begin{equation}
            a=v(a)=\frac{1}{ p }\sum_{i=1}^pu_i(a)=u_j(a)
        \end{equation}
        pour n'importe quel \( j\). Donc
        \begin{equation}
            a\in\bigcap_{i=1}^pF_{u_i},
        \end{equation}
        ce qui contredit notre hypothèse de départ.
        \end{subproof}
\end{proof}

\begin{proposition}[\cite{NHXUsTa,KXjFWKA,RXvMqkd}]     \label{PropQZkeHeG}
    Soit \( G\) un sous-groupe compact de \( \GL(n,\eR)\). Alors
    \begin{enumerate}
        \item
            Il existe une forme quadratique définie positive \( q\) sur \( \eR^n\) telle que \( G\subset \gO(q)\).
        \item
            Le groupe \( G\) est conjugué à un sous-groupe de \( \gO(n,\eR)\).
    \end{enumerate}
\end{proposition}
\index{groupe!action!utilisation}
\index{matrice!équivalence!dans le groupe linéaire}
\index{forme!quadratique!groupe orthogonal}
\index{groupe!orthogonal!d'une forme quadratique}
\index{endomorphisme!préservant une forme quadratique}

\begin{proof}
    Nous considérons le (pas tout à fait) morphisme de groupe
    \begin{equation}
        \begin{aligned}
            \rho\colon G&\to \GL\big( \gS(n,\eR) \big) \\
            u&\mapsto \rho_u\colon s\to  u^tsu,
        \end{aligned}
    \end{equation}
    et tant que nous y sommes à considérer, nous considérons l'ensemble
    \begin{equation}
        H=\{ M^tM\tq M\in G \}\subset \gS(n,\eR).
    \end{equation}
    Cet ensemble est constitué de matrices définies positives parce que si \( \langle M^tMx, x\rangle =0\), alors \(0= \langle Mx, Mx\rangle =\| Mx \|\), mais \( M\) étant inversible, cela implique que \( x=0\). Qui plus est cet ensemble est compact dans \( \GL(n,\eR)\) en tant qu'image du compact \( G\) par l'application continue \( M\mapsto M^tM\). L'enveloppe convexe \( K=\Conv(H)\) est alors également compacte par le théorème~\ref{CorOFrXzIf}. Enfin nous considérons \( L=\rho(G)\), qui est un sous-groupe compact de \( \GL\big( \gS(n,\eR) \big)\) parce que \( \rho_u\rho_v=\rho_{vu}\in\rho(G)\). Nous remarquons que \( \rho_u\) étant linéaire, elle préserve les combinaisons convexes et donc pour tout \( u\in G\), \( \rho_u(K)\subset K\).

    Bref, \( L\) est un sous-groupe compact de \( \GL(n,\eR)\) préservant le compact \( K\) de \( \gS(n,\eR)\). Par le lemme~\ref{LemOCtdiaE}, il existe \( s\in K\) tel que \( \rho_u(s)=s\) pour tout \( u\in G\). Ou encore :
    \begin{equation}
        u^tsu=s
    \end{equation}
    pour tout \( u\in G\). Fort de ce \( s\) bien particulier, nous considérons la forme quadratique associée : \( q(x)=x^tsx\). Cette forme est définie positive parce que \( s\) l'est. Nous avons \( G\subset \gO(q)\) parce que si \( u\in G\) alors
    \begin{equation}
        q\big( ux \big)=(ux)^tsux=x^t\underbrace{u^tsu}_{=s}x=q(x).
    \end{equation}
    Le premier point est prouvé.

    La matrice \( s\) est symétrique et définie positive. Le théorème \ref{ThoeTMXla} nous permet donc de la diagonaliser en \( \diag(\lambda_1,\ldots, \lambda_n)\) avec \( \lambda_i>0\), et ensuite transformée en la matrice \( \mtu_n\) par la matrice \( \diag(1/\sqrt{\lambda_i})\). Nous avons donc une matrice \( a\in\GL(n,\eR)\) telle que \( a^tsa=\mtu_n\). Avec ça, si \( u\in G\), nous avons
    \begin{equation}
        (a^{-1}ua)^t(a^{-1} ua)=(a^{-1}ua)^t\mtu_n(a^{-1} ua)=a^tu^t(a^t)^{-1}a^tsaa^{-1}ua=a^tu^tsua=a^tsa=\mtu,
    \end{equation}
    ce qui prouve que \( a^{-1} ua\) est dans \( \gO(n,\eR)\), et donc que \( a^{-1} G a\subset \gO(n,\eR)\).
\end{proof}



\chapter{Tribus, théorie de la mesure, intégration}
\input{40_mesure}
% This is part of Mes notes de mathématique
% Copyright (c) 2011-2018
%   Laurent Claessens, Carlotta Donadello
% See the file fdl-1.3.txt for copying conditions.

%+++++++++++++++++++++++++++++++++++++++++++++++++++++++++++++++++++++++++++++++++++++++++++++++++++++++++++++++++++++++++++
\section{Applications mesurables}
%+++++++++++++++++++++++++++++++++++++++++++++++++++++++++++++++++++++++++++++++++++++++++++++++++++++++++++++++++++++++++++

%---------------------------------------------------------------------------------------------------------------------------
\subsection{Propriétés}
%---------------------------------------------------------------------------------------------------------------------------

\begin{definition}[Fonction mesurable] \label{DefQKjDSeC}
    Soient \( (E,\tribA)\) et \( (F,\tribF)\) deux espaces mesurés. Une fonction \( f\colon E\to F\) est \defe{mesurable}{mesurable!fonction} si pour tout \( \mO\in \tribF\), l'ensemble \( f^{-1}(\mO)\) est dans \( \tribA\).
\end{definition}

\begin{definition}[Fonction borélienne]     \label{DefHHIBooNrpQjs}
    Une application \( f\colon (\Omega,\tribA)\to (\eR^d,\Borelien(\eR^d))\) est \defe{borélienne}{borélienne!fonction}\index{fonction!borélienne} si elle est mesurable, c'est-à-dire si pour tout \( B\in\Borelien(\eR^d)\) nous avons \( f^{-1}(B)\in\tribA\).

    Si rien n'est précisé, une application entre deux espaces topologiques est borélienne lorsqu'elle est mesurable en considérant la tribu borélienne sur \emph{les deux} espace.
\end{definition}
Si \( \tribA\) est une tribu sur un ensemble \( E\), nous notons \( m(\tribA)\)\nomenclature[P]{\( m(\tribA)\)}{Ensemble des fonctions \( \tribA\)-mesurables} l'ensemble des fonctions qui sont \( \tribA\)-mesurables.

Le plus souvent lorsque nous parlerons de fonctions \( f\colon X\to Y\) où \( Y\) est un espace topologique, nous considérons la tribu borélienne sur \( Y\). Ce sera en particulier le cas dans la théorie de l'intégration.

\begin{proposition}     \label{PROPooEFHKooARJBwW}
    Soient \( (S_i,\tribF_i)\) (\( i=1,2,3\)) des espaces mesurables et des fonctions mesurables \( f\colon S_1\to S_2\) et \( g\colon S_2\to S_3\). Alors la fonction \( g\circ f\colon S_1\to S_3\) est mesurable.
\end{proposition}

\begin{proof}
    Soit \( B\in\tribF_3\). Alors
    \begin{equation}
        (g\circ f)^{-1}(B)=f^{-1}\big( g^{-1}(B) \big)\in f^{-1}(\tribF_2)\subset\tribF_1.
    \end{equation}
\end{proof}

%---------------------------------------------------------------------------------------------------------------------------
\subsection{D'une tribu à l'autre}
%---------------------------------------------------------------------------------------------------------------------------

\begin{lemma}[\cite{TribuLi}]       \label{LemooVDXJooZNYelH}
    Soit une application \( f\colon S_1\to S_2\) et une tribu \( \tribF_2\) sur \( S_2\). Alors \( f^{-1}(\tribF_2)\) est une tribu sur \( S_1\)
\end{lemma}

\begin{proof}
    Il faut prouver les trois propriétés de la définition~\ref{DefjRsGSy} d'une tribu.
    \begin{enumerate}
        \item
            D'abord \( f\) est définit sur tout \( S_1\), donc \( f^{-1}(S_2)=S_1\) alors que \( S_2\in \tribF_2\).
        \item
            Soit \( A\in f^{-1}(\tribF_2)\), c'est-à-dire \( A=f^{-1}(B)\) pour un certain \( B\in \tribF_2\). En ce qui concerne le complémentaire :
            \begin{equation}
                A^c=f^{-1}(B)^c=S_1\setminus f^{-1}(B)=f^{-1}(S_2\setminus B)=f^{-1}(B^c).
            \end{equation}
        \item
            Si \( (A_i)_{i\in \eN}\) sont des éléments de \( f^{-1}(\tribF_2)\) avec \( A_i=f^{-1}(B_i)\) alors
            \begin{equation}
                \bigcup_iA_i=\bigcup_if^{-1}(B_i)=f^{-1}\big( \bigcup_iB_i \big).
            \end{equation}
            Ce qui est dans la dernière parenthèse est dans \( \tribF_2\) parce que cette dernière est une tribu.
    \end{enumerate}
\end{proof}

\begin{lemma}[\cite{TribuLi}]       \label{LemJYKBooBSXBXJ}
    Soit une application \( f\colon S_1\to S_2\) et \( \tribF\) une tribu de \( S_1\). Alors
    \begin{enumerate}
        \item
            L'ensemble
            \begin{equation}
                \tribF_f=\{  B\subset S_2\tq f^{-1}(B)\in \tribF  \}
            \end{equation}
            est une tribu sur \( S_2\).
        \item
            C'est la plus grande tribu de \( S_2\) pour laquelle \( f\) est mesurable.
    \end{enumerate}
\end{lemma}

\begin{proof}
    Encore les trois propriétés à vérifier.
    \begin{enumerate}
        \item
            \( S_2\in\tribF\), sont \( S_1=f^{-1}(S_2)\in \tribF_f\).
        \item
            Si \( A\in \tribF_f\) alors \( A=f^{-1}(B)\) pour un certain \( B\in \tribF\). Nous avons alors aussi \( B^c\in \tribF\) et donc
            \begin{equation}
                f^{-1}(B^c)=f^{-1}(B)^c=A^c.
            \end{equation}
            Par conséquent \( A^c\) est dans \( \tribF_f\).
        \item
            Si \( (A_i)\) sont des éléments de \( \tribF_f\) avec \( A_i=f^{-1}(B_i)\) pour \( B_i\in \tribF\) alors \( \bigcup_iB_i\in\tribF\) et
            \begin{equation}
                f^{-1}\big( \bigcup_iB_i \big)=\bigcup_if^{-1}(B_i)\in\tribF_f.
            \end{equation}
    \end{enumerate}
    En ce qui concerne la maximalité, si \( R\subset S_2\) n'est pas dans \( \tribF_f\) alors \( f^{-1}(R)\) n'est pas dans \( \tribF\) et donc \( f\) ne serait pas mesurable.
\end{proof}

\begin{definition}[Tribu engendrée] \label{DefNOJWooLGKhmJ}
    Soit une application \( f\colon S_1\to S_2\) et \( \tribF\) une tribu de \( S_1\). Alors conformément au lemme~\ref{LemJYKBooBSXBXJ} l'ensemble
            \begin{equation}
                \tribF_f=\{  B\subset S_2\tq f^{-1}(B)\in \tribF  \}
            \end{equation}
            est la \defe{tribu engendrée}{tribu!engendrée!par une application}.
\end{definition}

Le lemme suivant est également nommé «lemme de transfert».
\begin{lemma}[Lemme de transport]       \label{LemOQTBooWGYuDU}
    Soit \( f\colon S_1\to S_2\) une application et une classe \( \tribC\) de parties de \( S_2\). Alors
    \begin{equation}
        \sigma\big( f^{-1}(\tribC) \big)=f^{-1}\big( \sigma(\tribC) \big).
    \end{equation}
\end{lemma}
\index{lemme!de transport}

\begin{proof}
    Vu que \( \sigma(\tribC)\) es tune tribu, dans \( S_2\) alors le lemme~\ref{LemJYKBooBSXBXJ} dit que \( f^{-1}\big( \sigma(\tribC) \big)\) est une tribu qui contient en particulier \(  f^{-1}(\tribC) \). Nous en déduisons que \( \sigma\big( f^{-1}(\tribC) \big)\subset f^{-1}\big( \sigma(\tribC) \big)\).

    Réciproquement. Dans \( S_1\) nous avons la tribu \( \sigma\big( f^{-1}(\tribC) \big)\). Nous pouvons alors considérer la tribu
    \begin{equation}
        \tribF_f=\{ B\subset S_2\tq f^{-1}(B)\in\sigma\big( f^{-1}(\tribC) \big) \}.
    \end{equation}
    Montrons que \( \tribC\subset \tribF_f\). Lorsque \( B\in \tribC\) nous avons \( f^{-1}(B)\in f^{-1}(\tribC)\subset\sigma\big( f^{-1}(\tribC) \big)\). Du coup \( B\in \tribF_f\). Nous avons alors, en passant aux tribus engendrées :
    \begin{equation}
        \sigma(\tribC)\subset\sigma(\tribF_f)=\tribF_f.
    \end{equation}
    Si maintenant \( B\in\sigma(\tribC)\), nous avons \( f^{-1}(B)\in \sigma\big( f^{-1}(\tribC) \big)\), ce qui signifie que
    \begin{equation}
        f^{-1}\big( \sigma(\tribC) \big)\subset\sigma\big( f^{-1}(\tribC) \big).
    \end{equation}
\end{proof}

Le théorème suivant est important pour prouver qu'une application est mesurable. En effet, il permet de ne tester si une application est mesurable uniquement sur une partie génératrice de la tribu d'arrivé\footnote{Typiquement les ouverts pour les boréliens.}.
\begin{theorem}     \label{ThoECVAooDUxZrE}
    Soient des espaces mesurables \( ( S_1,\tribF_1 )\) et \( (S_2,\tribF_2)\) ainsi qu'une application \( f\colon S_1\to S_2\). S'il existe un ensemble de parties \( \tribC\) de \( S_2\) tel que
    \begin{itemize}
        \item \( \sigma(\tribC)=\tribF_2\)
        \item \( f^{-1}(B) \in \tribF_1 \) pour tout \( B\in \tribC\)
    \end{itemize}
    alors \( f\) est mesurable.
\end{theorem}

\begin{proof}
    Par hypothèse, \( \sigma(\tribC)=\tribF_2\) et \( f^{-1}(\tribC)\subset \tribF_1\) et nous pouvons utiliser le lemme de transfert~\ref{LemOQTBooWGYuDU} :
    \begin{equation}
        \sigma\big( f^{-1}(\tribC) \big)=f^{-1}\big( \sigma(\tribC) \big)
    \end{equation}
    qui s'écrit ici
    \begin{equation}
        \sigma\big( f^{-1}(\tribC) \big)=f^{-1}(\tribF_2).
    \end{equation}
    Mais vu que \( f^{-1}(\tribC)\subset \tribF_1\), nous avons aussi \( \sigma\big( f^{-1}(\tribC) \big)\subset \tribF_1\), ce qui signifie que
    \begin{equation}
        f^{-1}(\tribF_2)\subset \tribF_1.
    \end{equation}
    Cela est exactement le fait que \( f\) soit mesurable.
\end{proof}

Le théorème suivant est très important parce qu'en pratique c'est souvent lui, en conjonction avec la proposition~\ref{PropooLNBHooBHAWiD} qui permet de déduire qu'une fonction est borélienne.
\begin{theorem}[\cite{TribuLi}]     \label{ThoJDOKooKaaiJh}
    Soient \( X\) et \( Y\) deux espaces topologiques. Alors toute application continue \( f\colon X\to Y\) est borélienne\footnote{Définition~\ref{DefHHIBooNrpQjs}.}.
\end{theorem}

\begin{proof}
    Pour vérifier que \( f\) est borélienne, nous devons prouver que \( f^{-1}(B)\) est borélien pour tout borélien \( B\) de \( Y\). Heureusement, le théorème~\ref{ThoECVAooDUxZrE} nous permet de limiter la vérification aux \( B\) appartenant à une classe engendrant les boréliens de \( Y\).

    La classe en question est toute trouvée : ce sont les ouverts. Si \( \mO\) est un ouvert de \( Y\) alors \( f^{-1}(\mO)\) est un ouvert de \( X\) et donc un borélien de \( X\).
\end{proof}

Le théorème suivant donne une importante compatibilité entre l'induction de tribu et l'induction de topologie : la tribu induite à partir des boréliens sur un sous-espace topologique est la tribu des boréliens pour la topologie induite.
\begin{theorem}[\cite{TribuLi}]     \label{ThoSVTHooChgvYa}
    Soit \( X\), un espace topologique et \( Y\subset X\) une partie munie de la topologie induite. Alors
    \begin{equation}
        \Borelien(Y)=\Borelien(X)_Y
    \end{equation}
    où \( \Borelien(X)_Y\) est la tribu sur \( Y\) induite de \( \Borelien(X)\) par la définition~\ref{DefDHTTooWNoKDP}.
\end{theorem}

\begin{proof}
    Nous notons \( \tau_X\) et \( \tau_Y\) les topologies de \( X\) et \( Y\).
    \begin{subproof}
        \item[\( \Borelien(Y)\subset\Borelien(X)_Y\)]
            Si \( A\in \tau_Y\) alors \( A=Y\cap \Omega\) pour un \( \Omega\in \tau_X\). Mais vu que \(\Omega\) est un ouvert de \( X\), il est un borélien de \( X\), ce qui donne que \( Y\cap\Omega\) est un élément de \( \Borelien(X)_Y\). Cela prouve que \( \tau_Y\subset\Borelien(X)_Y\), c'est-à-dire que \( \Borelien(X)_Y\) est une tribu sur \( Y\) contenant les ouverts de \( Y\). Nous avons donc
            \begin{equation}
                \Borelien(X)\subset\Borelien(X)_Y.
            \end{equation}
        \item[Réciproquement]
            L'application \( \id\colon (Y,\tau_Y)\to (X,\tau_X)\) est continue parce que si \( \Omega\) est ouvert de \( X\) alors \( \id^{-1}(\Omega)=\Omega\cap Y\in \tau_Y\). Par conséquent l'identité est une application borélienne (théorème~\ref{ThoJDOKooKaaiJh}), ce qui signifie que \( \id^{-1}\big( \Borelien(X) \big)\subset\Borelien(Y)\), ou encore que si \( B\in\Borelien(X)\), alors \( \id^{-1}(B)=B\cap Y\in\Borelien(Y)\). Cela signifie que
            \begin{equation}
                \Borelien(X)_Y\subset \Borelien(Y).
            \end{equation}
    \end{subproof}
\end{proof}

\begin{corollary}       \label{CorooMJQYooFfwoTd}
    Si \( U\) est un borélien de l'espace topologique \( X\), alors les boréliens de \( U\) sont les boréliens de \( X\) inclus dans \( U\) :
    \begin{equation}
        \Borelien(U)=\{ B\in\Borelien(X)\tq B\subset U \}.
    \end{equation}
\end{corollary}

\begin{proof}
    Si \( B'\in\Borelien(U)\), le théorème~\ref{ThoSVTHooChgvYa} donne un borélien \( B\in\Borelien(X)\) tel que \( B'=B\cap U\). Mais \( U\) étant borélien de \( X\), l'intersection \( B\cap U\) est encore un borélien de \( X\).
\end{proof}
Ce corolaire s'applique en particulier lorsque \( U\) est un ouvert.

La proposition suivante montre comment il est possible de construire un espace mesuré à partir d'une bijection avec un espace mesuré déjà connu. Attention cependant : la mesure construite dans cette proposition n'est pas celle qui est le plus adapté. Voir la proposition \ref{PROPooILOEooBiumKD} et l'exemple \ref{PROPooILOEooBiumKD}.
\begin{proposition}     \label{PROPooXQHTooUxJoyq}
    Soient un espace mesuré \( (\Omega,\tribA,\mu)\), un ensemble \( \Omega'\) et une bijection \( \varphi\colon \Omega\to \Omega'\). Nous posons
    \begin{enumerate}
        \item
            \( \tribA'=\varphi(\tribA)\),
        \item
            \( \mu'(B)=\mu\big( \varphi^{-1}(B) \big)\) pour tout \( B\in\tribA'\).
    \end{enumerate}
    Alors \( (\Omega',\tribA',\mu')\) est un espace mesuré.
\end{proposition}

\begin{proof}
    En plusieurs points.
    \begin{subproof}
        \item[\( \tribA'\) est une tribu]
            Il faut vérifier les différents points de la définition \ref{DefjRsGSy}. D'abord, vu que \( \Omega\in\tribA\), nous avons \( \Omega'=\varphi(\Omega)\in \tribA'\). Pour le complémentaire, si \( B\in\tribA'\) alors \( B=\varphi(A)\) pour un certain \( A\in \tribA\). Vu que \( \tribA\) est une tribu nous avons alors \( \Omega\setminus A\in\tribA\) et donc \( \varphi(\Omega\setminus A)\in \tribA'\). Mais comme \( \varphi\) est bijective,
            \begin{equation}
                \varphi(\Omega\setminus A)=\Omega'\setminus\varphi(A)=\Omega'\setminus B.
            \end{equation}
            Le complémentaire de \( B\) est donc bien dans \( \tribA'\). Pour la troisième condition soient \( B_i\in\tribA'\). Pour chauqe \( i\), il existe \( A_i\in \tribA\) tel que \( B_i=\varphi(A_i)\). Nous avons \( \bigcup_iA_i\in \tribA\), donc
            \begin{equation}
                \bigcup_iB_i=\bigcup_i\varphi(A_i)=\varphi\big( \bigcup_iA_i \big)\in \tribA'.
            \end{equation}
            Nous avons fini de prouver que \( (\Omega',\tribA')\) était un espace mesurable.
        \item[\( \mu'\) est une mesure positive]
            D'abord \( \mu'(\emptyset)=\mu\big( \varphi^{-1}(\emptyset) \big)=\mu(\emptyset)=0\). Ensuite si les \( A_i\) sont disjoints dans \( \tribA'\) nous avons
            \begin{equation}
                \mu\big( \bigcup_{i=0}^{\infty}A_i \big)=\mu\left( \varphi^{-1}\big( \bigcup_{i=0}^{\infty}A_i \big) \right)=\mu\left( \bigcup_i\varphi^{-1}(A_i) \right)=\sum_{i=0}^{\infty}\mu\big( \varphi^{-1}(A_i) \big)=\sum_i\mu'(A_i).
            \end{equation}
    \end{subproof}
\end{proof}

\begin{proposition}     
    Soit une bijection continue d'inverse continue \( \varphi\colon \Omega\to \Omega'\). Alors
    \begin{equation}
        \varphi\big( \Borelien(\Omega) \big)=\Borelien(\Omega').
    \end{equation}
\end{proposition}

\begin{proof}
    Si \( A\in \Borelien(\Omega')\), alors \( A=\varphi\big( \varphi(A) \big)\in\varphi\big( \Borelien(\Omega) \big)\) parce que \( \varphi\) est continue et donc borélienne (proposition \ref{ThoJDOKooKaaiJh}). Le même raisonnement fonctionne dans l'autre sens parce que nous avons supposé que \( \varphi\) est continue et d'inverse continu.
\end{proof}

%+++++++++++++++++++++++++++++++++++++++++++++++++++++++++++++++++++++++++++++++++++++++++++++++++++++++++++++++++++++++++++ 
\section{Espace mesuré complet}
%+++++++++++++++++++++++++++++++++++++++++++++++++++++++++++++++++++++++++++++++++++++++++++++++++++++++++++++++++++++++++++

%--------------------------------------------------------------------------------------------------------------------------- 
\subsection{Partie négligeable}
%---------------------------------------------------------------------------------------------------------------------------

\begin{definition}  \label{DefAVDoomkuXi}
    Soit un espace mesuré \( (X,\tribA,\mu)\). Une partie \( N\) de \( X\) est \defe{négligeable}{négligeable!partie d'un espace mesuré} pour \( \mu\) s'il existe \( Y\in\tribA\) tel que \( N\subset Y\) et \( \mu(Y)=0\).
\end{definition}

\begin{lemma}   \label{LemVKNooOCOQw}
    L'ensemble des parties négligeables est stable par union dénombrable.
\end{lemma}

\begin{proof}
    Si les ensembles \( N_i\) sont négligeables, alors pour chaque \( i\) nous avons \( Y_i\in\tribA\) tel que \( N_i\subset Y_i\) et \( \mu(Y_i)=0\). Alors bien entendu \( \bigcup_iN_i\subset \bigcup_iY_i\) et en utilisant \eqref{EqWWFooYPCTt},
    \begin{equation}
        \mu\big( \bigcup_iY_i \big)\leq \sum_i\mu(Y_i)=0.
    \end{equation}
\end{proof}

\begin{definition}  \label{DefBWAoomQZcI}
    L'espace mesuré \( (X,\tribF,\mu)\) est \defe{complet}{complet!espace mesuré} si tout ensemble \( \mu\)-négligeable est dans \( \tribF\).
\end{definition}

Notons que la proposition~\ref{PropHYLooLgOCy} s'applique si \( (X,\tribF,\mu)\) est un espace mesuré et \( \tribN\) est l'ensemble des parties \( \mu\)-négligeables. C'est ce qui permet de donner le théorème suivant, que nous redémontrons de façon indépendante de la proposition~\ref{PropHYLooLgOCy}.
\begin{theorem}[Complétion d'espace mesuré\cite{MesureLebesgueLi,DXTooFCLru,BOQoojbFpP}]   \label{thoCRMootPojn}
    Soit un espace mesuré \( (X,\tribF,\mu)\) et \( \tribN\) l'ensemble des parties \( \mu\)-négligeables de \( X\).
    \begin{enumerate}
        \item
            Les ensembles suivants sont égaux :
            \begin{subequations}
                \begin{align}
                    \tribA&=\{ A\subset X\tq\exists B,C\in\tribF\tq B\subset A\subset C,\mu(C\setminus B)=0 \}\\
                    \tribB&=\{ B\cup N\tq  B\in\tribF,N\in\tribN \}    \label{EqFJIoorxZNU}\\
                    \tribC&=\{ A\subset X\tq \exists B\in\tribF\tq A\Delta B\in \tribN \}.
                \end{align}
            \end{subequations}
            Ici \( A\Delta B\) est la différence symétrique de \( A\) et \( B\), définition~\ref{DefBMLooVjlSG}.
        \item
            L'ensemble \( \hat\tribF=\tribA=\tribB=\tribC\) est une tribu.
        \item
            La définition
            \begin{equation}
                \begin{aligned}
                    \mu'\colon \tribB&\to \mathopen[ 0 , \infty \mathclose] \\
                    A\cup N&\mapsto \mu(A)
                \end{aligned}
            \end{equation}
            est cohérente.
        \item
            L'application \( \mu'\) ainsi définie est une mesure sur \( (X,\tribA)\).
        \item
            L'espace \( (X,\tribA,\mu')\) est complet.
        \item
            La mesure \( \mu'\) prolonge \( \mu\).
        \item   \label{thoCRMootPojnvii}
            La mesure \( \mu'\) est minimale au sens où toute mesure complète prolongeant \( \mu\) prolonge \( \mu'\).
    \end{enumerate}
\end{theorem}

\begin{proof}
    Commençons par prouver que les trois ensembles \( \tribA\), \( \tribB\) et \( \tribC\) sont égaux.
    \begin{subproof}
    \item[\( \tribA\subset\tribB\).]
        Soit \( A\in\tribA\). Alors nous avons des ensembles \( B,C\in\tribF \) tels que \( B\subset A\subset V\) avec \( \mu(C\setminus B)=0\). Alors nous avons aussi \( A=B\cup(C\setminus B)\), ce qui prouve que \( A\in\tribB\).
    \item[\( \tribB\subset\tribC\).]
        Soit \( A\in\tribB\), c'est-à-dire que \( A=B\cup N\) avec \( B\in\tribF\) et \( N\in\tribN\). Nous avons évidemment \( A\cup B=A\) et donc
        \begin{equation}
            A\Delta B=(A\cup B)\setminus(A\cap B)=A\setminus(A\cap B)=(B\cup N)\setminus(A\cap B)\subset N.
        \end{equation}
        Pour comprendre la dernière inclusion, si \( x\) appartient à \( A=B\cup N\) sans être dans \( N\) alors \( x\in B\) et donc \( x\in A\cap B\). Par conséquent nous avons \( A\Delta B\subset N\) et donc \( A\Delta B\in\tribN\).
    \item[\( \tribC\subset\tribA\)]
        Soit donc \( A\in\tribC\); il existe \( B\in\tribF\) tel que \( A\Delta B\in\tribN\) ou encore, il existe \( D\in\tribF\) tel que \( A\Delta B\subset D\) avec \( \mu(D)=0\). Si nous posons \( B'=B\cap D^c\) et \( C'=B\cup D\) alors nous prétendons avoir
        \begin{equation}
            B'\subset A\subset C'.
        \end{equation}
        Et nous le prouvons. En effet si \( x\in B\cap D^c\) alors en remarquant que \( B\) se divise en
        \begin{equation}
            B=(B\cap A)\cup\big(B\cap (A\Delta B)\big),
        \end{equation}
        et en nous souvenant que \( B\cap (A\Delta B)\subset D\), il vient que \( B\cap D^c\subset B\cap A\). Et en particulier \( x\in A\). D'autre part
        \begin{equation}
            A\subset B\cup(A\Delta B)\subset B\cup D.
        \end{equation}
        Nous avons donc bien \( B'\subset A\subset C'\). Par stabilité de la tribu \( \tribF\) sous les intersections et complémentaires nous avons aussi \( B',C'\in\tribF\). De plus
        \begin{equation}
            C'\setminus B'=(B\cup D)\setminus(B\cap D^c)\subset D,
        \end{equation}
         et donc
         \begin{equation}
             \mu(C'\setminus B')\leq \mu(D)=0.
         \end{equation}
    \end{subproof}

    Nous avons donc prouvé que \( \tribA\subset\tribB\subset\tribC\subset \tribA\), et donc que \( \tribA=\tribB=\tribC\). Nous allons donc maintenant noter \( \tribA\) indifféremment les trois ensembles. Nous prouvons à présent que c'est une tribu.

    \begin{subproof}

        \item[Tribu : le vide]

            Pas de problèmes à \( \emptyset\in\tribA\)

        \item[Tribu : complémentaire]

            Soit \( A\in\tribA\). Alors il existe \( B,C\in\tribF\) tels que \( B\subset A\subset C\) avec \( \mu(C\setminus B)=0\). En passant au complémentaire,
            \begin{equation}
                C^c\subset A^c\subset B^c.
            \end{equation}
            Mais \( B^c\setminus C^c=C\setminus B\), donc \( \mu(B^c\setminus C^c)=0\).

        \item[Tribu : union dénombrable]

            Soit \( (A_n)\) des éléments de \( \tribA\). Pour chaque \( n\) nous avons des ensembles \( B_n,C_n\in\tribF\) tels que\( B_n\subset A_n\subset C_n\) avec \( \mu(C_n\setminus B_n)=0\). En ce qui concerne les unions nous avons
            \begin{equation}
                \bigcup_nB_n\subset \bigcup_nA_n\subset \bigcup_nC_n,
            \end{equation}
            et
            \begin{equation}
                \big( \bigcup_nC_n\big)\setminus\big( \bigcup_nB_n\big)\subset \bigcup_n(C_n\setminus B_n).
            \end{equation}
            Par conséquent, en utilisant \eqref{EqWWFooYPCTt},
            \begin{equation}
                \mu\left( \big( \bigcup_nC_n\big)\setminus\big( \bigcup_nB_n\big)\right)\leq\mu\left(  \bigcup_n(C_n\setminus B_n)\right)\leq\sum_n\mu(C_n\setminus B_n)=0.
            \end{equation}
            Cela prouve que \( \bigcup_nA_n\in\tribA\), et donc que \( \tribA\) est une tribu.

        \item[Définition cohérente]

            Soient \( A,A'\in\tribF\) et \( N,N'\in\tribN\) tels que \( A\cup N=A'\cup N'\). Nous considérons \( Y,Y'\in\tribF\) tel que \( N\subset Y\), \( N'\subset Y'\) et \( \mu(Y)=\mu(Y')=0\). En vertu de \eqref{EqWWFooYPCTt} nous avons
            \begin{equation}
                \mu(A)\leq \mu(A\cup Y)\leq \mu(A'\cup Y\cup Y')\leq\mu(A')+\mu(Y)+\mu(Y')=\mu(A').
            \end{equation}
            En écrivant la même chose en échangeant les primes nous prouvons également \( \mu(A')\leq \mu(A)\). Au final \( \mu(A)=\mu(A')\), c'est-à-dire
            \begin{equation}
                \mu'(A\cup N)=\mu'(A'\cup N').
            \end{equation}
            La définition de \( \mu'\) est donc cohérente.
        \item[\( \mu'\) est une mesure]

            Le fait que \( \mu'\) soit positive et que \( \mu'(\emptyset)\) soit nul ne pose pas de problèmes. Il faut voir l'union dénombrable disjointe. Si les ensembles \( A_i=B_i\cup N_i\) sont disjoints, alors les \( B_i\) et le \( N_i\) sont tous disjoints deux à deux. De plus l'ensemble \( \bigcup_iN_i\) est négligeable parce que nous avons déjà vu que \( \tribN\) était stable par union dénombrable (\ref{EqWWFooYPCTt}). Donc
            \begin{equation}
                \mu'\left( \bigcup_i B_i\cup N_i \right)=\mu'\Big( \big( \bigcup_iB_i \big)\cup\underbrace{\big( \bigcup_iN_i \big)}_{\in\tribN} \Big)=\mu\big( \bigcup_iB_i \big)=\sum_u\mu(B_i)=\sum_i\mu'(B_i\cup N_i).
            \end{equation}
        \item[Espace complet]
            Un ensemble \( \mu'\)-négligeable est automatiquement \( \mu\)-négligeable. En effet si \( H\) est \( \mu'\)-négligeable, il existe \( B\in\tribF\) et \( N\in\tribN\) tels que \( H\subset B\cup N\) avec \( \mu(B)=0\). Vu que \( N\) est \( \mu\)-négligeable, il existe \( Y\in\tribF\) tel que \( N\subset Y\) et \( \mu(Y)=0\). Donc \( H\subset B\cup N\subset B\cup Y\) avec \( \mu(B\cup Y)=0\).

            Tous les ensembles \( \mu\)-négligeables faisant partie de \( \tribB\), tous les ensembles \( \mu'\)-négligeables font partie de \( \tribA\).
        \item[Prolongement]
            La mesure \( \mu'\) prolonge \( \mu\). En effet si \( A\in\tribF\) alors \( A=A\cup\emptyset\in\tribB\) et \( A\) est \( m'\)-mesurable. De plus \( \mu'(A)=\mu'(A\cup\emptyset)=\mu(A)\).
        \item[Minimalité]

            Soit un espace mesuré complet \( (X,\tribM,\nu)\) prolongeant \( (X,\tribF,\mu)\). Pour \( A\in\tribA\) nous devons prouver que \( A\in\tribM\) et que \( \mu'(A)=\nu(A)\). Il existe \( B\in\tribF\) et \( N\in\tribN\) tels que \( A=B\cup N\). Vu que \( N\) est \( \mu\)-négligeable, il est également \( \nu\)-négligeable et donc \( \nu\)-mesurable parce que \(\nu\) est complète : \( A\in\tribM\). En ce qui concerne l'égalité \( \mu'(A)=\nu(A)\) nous avons
            \begin{equation}
                \nu(B)\leq\nu(B\cup N)\leq \nu(B)+\nu(N)=\nu(B),
            \end{equation}
            donc \( \nu(A)=\nu(B\cup N)=\nu(B)=\mu(B)\). La dernière égalité est le fait que \( \nu\) prolonge \( \mu\). Mais par définition de \( \mu'\) nous avons aussi \( \mu'(A)=\mu'(B\cup N)=\mu(B)\). Au final \( \mu'(A)=\nu(A)=\mu(B)\).
    \end{subproof}
\end{proof}

\begin{definition}
    L'espace mesuré complet \( (X,\tribA,\mu')\) défini par le théorème~\ref{thoCRMootPojn} est l'\defe{espace mesuré complétée}{espace!mesuré!complété} de \( (X,\tribF,\mu)\).

    Nous noterons le complété de \( (S,\tribF,\mu)\) par \( (S,\hat\tribF,\hat \mu)\)\nomenclature[Y]{\( (S,\hat\tribF,\hat\mu)\)}{complété de l'espace mesuré \( (S,\hat\tribF,\hat\mu)\)}
\end{definition}

\begin{theorem}[Carathéodory\cite{MesureLebesgueLi}] \label{ThoUUIooaNljH}
    Soit \( S\) un ensemble et \( m^*\) une mesure extérieure sur \( S\). Alors
    \begin{enumerate}
        \item   \label{RPPooHSWWsi}
            l'ensemble \( \tribM\) des parties \( m^*\)-mesurables est une tribu,
        \item
            la restriction de \( m^*\) est une mesure sur \( (S,\tribM)\),
        \item
            l'espace mesuré \( (S,\tribM,m^*)\) est complet\footnote{Définition~\ref{DefBWAoomQZcI}.}.
    \end{enumerate}
\end{theorem}

\begin{proof}
    Une grosse partie de la preuve sera de prouver la stabilité de \( \tribM\) par union dénombrable quelconque; cela sera divisé en plusieurs parties.
    \begin{subproof}
    \item[Tribu : le vide]
        L'ensemble vide est \( m^*\)-mesurable.
    \item[Tribu : complémentaire]
        Soit \( A\in\tribM\) et \( X\in S\). La condition qui dirait \( A^c\in\tribM\) est :
        \begin{equation}
            m^*(X)=m^*(X\cap A^c)+m^*(X\cap A),
        \end{equation}
        qui est la même que celle qui dit que \( A\) est dans \( \tribM\).
    \item[Tribu : union finie]
        Soient \( A,B\in\tribM\) et \( X\subset S\). Alors, vu que \( m^*\) est une mesure extérieure,
        \begin{subequations}
            \begin{align}
                m^*(X)&\leq m^*\big( X\cap(A\cup B) \big)+m^*\big( X\cap (A\cup B)^x \big)\\
                &=m^*\big( (X\cap A)\cup(X\cap B) \big)+m^*\big( X\cap A^c\cap B^c \big).
            \end{align}
        \end{subequations}
        Mais nous pouvons écrire la première union sous forme d'une union disjointe de la façon suivante :
        \begin{equation}
            (X\cap A)\cup(X\cap B)=(X\cap A)\cup(X\cap B\cap A^c),
        \end{equation}
        ce qui donne
        \begin{subequations}
            \begin{align}
                m^*(X)&\leq m^*(X\cap A)+m^*(X\cap B\cap A^c)+m^*(X\cap A^c\cap B^c)        \label{subeqLYNooRdrgCi}\\
                &=m^*(X\cap A)+m^*(X\cap A^c)\\
                &=m^*(X)
            \end{align}
        \end{subequations}
        parce que les deux derniers termes de \eqref{subeqLYNooRdrgCi} se somment à \( m^*(X\cap A^c)\) parce que \( B\in \tribM\). La dernière ligne est le fait que \( A\) soit \( m^*\)-mesurable.
    \item[Union finie disjointe]
        Soient \( \{ A_1,\ldots, A_n \}\) des éléments deux à deux disjoints de \( \tribM\). Nous allons maintenant prouver par récurrence que
        \begin{equation}    \label{EqBRIooAnPCd}
            m^*\Big( X\cap\big( \bigcup_{k=1}^nA_k \big) \Big)=\sum_{k=1}^nm^*(X\cap A_k).
        \end{equation}
        Si \( n=1\) le résultat est évident. Sinon, le fait que \( A_{n+1}\) soit \( m^*\)-mesurable donne
        \begin{equation}
            m^*\Big( X\cap\big( \bigcup_{k=1}^{n+1}A_k \big) \Big)=m^*\Big( X\cap\big( \bigcup_{k=1}^{n+1}A_k \big)\cap A_{n+1} \Big)+m^*\Big( X\cap\big( \bigcup_{k=1}^{n+1}A_k \big)\cap A_{n+1}^c \Big).
        \end{equation}
        Le fait que les \( A_k\) soient disjoints implique aussi que
        \begin{equation}
            X\cap\big( \bigcup_{k=1}^{n+1}A_k \big)\cap A_{n+1}=X\cap A_{n+1}
        \end{equation}
        et
        \begin{equation}
            X\cap\big( \bigcup_{k=1}^{n+1}A_k \big)\cap A_{n+1}^c=X\cap\big( \bigcup_{k=1}^nA_k \big)
        \end{equation}
        et donc
        \begin{subequations}
            \begin{align}
                m^*\Big( X\cap\big( \bigcup_{k=1}^{n+1}A_k \big) \Big)&=m^*(X\cap A_{n+1})+m^*\Big( X\cap\big( \bigcup_{k=1}^nA_k \big) \Big)\\
                &\stackrel{rec.}{=}m^*(X\cap A_{n+1})+\sum_{k=1}^nm^*(X\cap A_k)\\
                &=\sum_{k=1}^{n+1}m^*(X\cap A_k).
            \end{align}
        \end{subequations}
        La relation \eqref{EqBRIooAnPCd} est prouvée.

        Notons qu'en particularisant à \( X=S\) nous avons
        \begin{equation}
            m^*\big( \bigcup_{k=1}^nA_k \big)=\sum_{k=1}^nm^*(A_k)
        \end{equation}
        dès que les \( A_k\) sont des éléments deux à deux disjoints de \( \tribM\).

    \item[Union dénombrable disjointe]

        Soit \( (A_n)_{n\in \eN}\) une suite d'éléments deux à deux disjoints dans \( \tribM\). Nous allons prouver les choses suivantes :
        \begin{itemize}
            \item \( \bigcup_nA_n\in\tribM\)
            \item \( m^*\big( \bigcup_nA_n \big)=\sum_nm^*(A_n)\)
        \end{itemize}
        où toutes les sommes et union sur \( n\) sont entre \( 1\) et \( \infty\).

        Nous posons \( A=\bigcup_kA_k\) et \( B_n=\bigcup_{k=1}^nA_k\). Nous savons que \( B_n\in\tribM\) pour tout \( n\) par le point précédent. Donc si \( X\in S\) nous avons
        \begin{subequations}
            \begin{align}   \label{EqGXLooRxqqg}
                m^*(X)&=m^*(X\cap B_n)+m^*(X\cap B_n^c)\\
                &=\sum_{k=1}^nm^*(X\cap A_k)+m^*(X\cap B_n^x)\\
                &\geq\sum_{k=1}^nm^*(X\cap A_k)+m^*(X\cap A^c)
            \end{align}
        \end{subequations}
        où nous avons utilisé la relation \eqref{EqBRIooAnPCd} sur les \( B_n\) ainsi que le fait que \( A^c\subset B_n^c\) (parce que \( B_n\subset A\)). L'inégalité \eqref{EqGXLooRxqqg} étant vraie pour tout \( n\), elle est vraie à la limite :
        \begin{subequations}
            \begin{align}
                m^*(X)&\geq \sum_{k=1}^{\infty}m^*(A\cap A_k)+m^*(X\cap A^c)\\
                &\geq m^*\Big( \bigcup_k(X\cap A_k) \Big)+m^*(X\cap A^c)\\
                &=m^*\Big( X\cap \big( \bigcup_kA_k \big) \Big)+m^*(X\cap A^c)\\
                &=m^*(X\cap A)+m^*(X\cap A^c),
            \end{align}
        \end{subequations}
        ce qui signifie que \( A\in\tribM\). La première des deux choses que nous voulions montrer est faite. En la particularisant à \( X=A\) et en tenant compte des faits que \( A\cap A_k=A_k\) et \( A\cap A^c=\emptyset\),
        \begin{equation}
            m^*(A)\geq \sum_{k=1}^{\infty}m^*(A\cap A_k)+m^*(A\cap A^c),
        \end{equation}
        c'est-à-dire que pour tout \( n\) nous avons
        \begin{equation}
            m^*\big( \bigcup_{k\in \eN}A_k \big)\geq \sum_{k=1}^nm^*(A_k).
        \end{equation}
        L'inégalité est encore vraie à la limite, et l'inégalité inverse étant toujours vraie pour une mesure extérieure,
        \begin{equation}
            m^*\big( \bigcup_{k\in \eN}A_k \big)=\sum_{k=1}^{\infty}m^*(A_k).
        \end{equation}

    \item[Union dénombrable quelconque]

        Soit maintenant une suite \( (A_n)_{n\in\eN}\) d'éléments de \( \tribM\) que nous ne supposons plus être disjoints. Nous nous ramenons au cas disjoint en posant
        \begin{subequations}
            \begin{numcases}{}
                B_1=A_1\\
                B_n=A_n\cap\big( \bigcup_{k=1}^{n-1}A_k \big)^c,
            \end{numcases}
        \end{subequations}
        c'est-à-dire que nous mettons dans \( B_n\) les éléments de \( A_n\) qui ne sont dans aucun des \( A_k\) précédents. Autrement dit, nous posons \( B_0=\emptyset\) et \( B_n=A_n\setminus B_{n-1}\). L'ensemble \( \tribM\) étant stable par réunion finie, par complément et par intersection finie nous avons \( B_n\in\tribM\). De plus les \( B_n\) sont disjoints, donc
        \begin{equation}
            \bigcup_{k=1}^{\infty}A_k=\bigcup_{k=1}^{\infty}B_k\in\tribM.
        \end{equation}
        La première égalité se justifie de la façon suivante : si \( x\in\bigcup_{k=1}^{\infty}A_k\) alors nous notons \( n_0\) le plus petit \( n\) tel que \( x\in A_n\) et alors \( x\in B_{n_0}\).
    \item[Espace complet]
        Nous prouvons à présent que \( (S,\tribM,m^*)\) est un espace mesuré complet. Soit \( N\) une partie \( m^*\)-négligeable de \( S\) et \( Y\in\tribM\) tel que \( m^*(Y)=0\) et \( N\subset Y\). D'abord \( m^*(N)=0\) parce que
        \begin{equation}
            m^*(N)\leq m^*(Y)=0.
        \end{equation}
        Si \( X\subset S\) nous avons
        \begin{subequations}
            \begin{align}
                X\cap N\subset   N&\Rightarrow m^*(X\cap N)=0\\
                X\cap N^c\subset X&\Rightarrow m^*(X\cap N^c)\leq m^*(X).
            \end{align}
        \end{subequations}
        Donc
        \begin{equation}
            m^*(X\cap N)+m^*(X\cap N^c)\leq m^*(X),
        \end{equation}
        ce qui montre que \( N\) est est \( m^*\)-mesurable.
    \end{subproof}
\end{proof}

\begin{normaltext}

Ce théorème nous pousse à adopter de la notation. Lorsqu'un espace mesuré \( (S,\tribF,\mu)\) est donné, nous noterons
\begin{equation}
    (S,\tribM,\mu^*)
\end{equation}
l'espace mesuré construit de la façon suivante. D'abord \( \mu^*\) est la mesure extérieure associée à \( \mu\) par la proposition~\ref{PropFDUooVxJaJ}. Ensuite \( \tribM\) est la tribu des parties \( \mu^*\)-mesurables, qui est bien une tribu parce que \( \mu^*\) est une mesure extérieure (\ref{ThoUUIooaNljH}). La proposition \eqref{PropOJFoozSKAE} dit alors que \( \tribF\subset\tribM\). De plus~\ref{ThoUUIooaNljH} nous explique que si \( A\in\tribF\) alors \( \mu(A)=\mu^*(A)\). Tout cela pour dire que
\begin{equation}    \label{EqXDPooKwWAF}
    (S,\tribF,\mu)\subset (S,\tribM,\mu^*).
\end{equation}
Et enfin,~\ref{ThoUUIooaNljH} nous dit que l'espace mesuré \( (S,\tribM,\mu^*)\) est complet.
\end{normaltext}

\begin{example} \label{ExOIXoosScTC}
    Montrons un cas dans lequel \( (S,\tribM,\mu^*)\) n'est pas \( \sigma\)-fini. Soit \( S\) un ensemble non dénombrable et \( \tribF\) la tribu des parties de \( S\) qui sont soit fini ou dénombrables soit de complémentaire fini ou dénombrable. Nous y mettons la mesure
    \begin{equation}
        \mu(A)=\begin{cases}
            0    &   \text{si } A\text{ est au plus dénombrable}\\
            \infty    &    \text{sinon}.
        \end{cases}
    \end{equation}
    Cette mesure n'est pas \( \sigma\)-finie parce qu'aucune union de dénombrables est non dénombrable. De plus \( (S,\tribF,\mu)\) est complet parce que toute partie contenue dans un ensemble fini ou dénombrable est fini ou dénombrable (\ref{PropQEPoozLqOQ}).

    \begin{subproof}
     \item[\( \tribF\) n'est pas \( \partP(S)\)]

        La tribu \( \tribF\) est différente de \( \partP(S)\). En effet \( S\) étant infini, il existe par~\ref{PropVCSooMzmIX} une bijection \( \varphi\colon \{ 1,2 \}\times S\to S\). Alors l'ensemble \( \varphi\big( \{ 1 \}\times S \big)\) est non dénombrable et son complémentaire
        \begin{equation}
            \varphi\big( \{ 1 \}\times S \big)^c=\varphi\big( \{ 2 \}\times S \big)
        \end{equation}
        n'est as dénombrable non plus. Cet ensemble n'est donc pas de \( \tribF\).

    \item[\( \tribM\) est \( \partP(S)\)]

        En effet, soit \( A\subset S\); il faut prouver que pour tout \( X\subset S\) nous avons
        \begin{equation}
            \mu^*(X)=\mu^*(X\cap A)+\mu^*(X\cap A^c).
        \end{equation}
        Nous prouvons cela en séparant les cas suivant que \( X\) est dénombrable ou non.

        Si \( X\) est fini ou dénombrable, alors \( X\cap A\) et \( X\cap A^c\) le sont également et nous avons \( \mu^*(X)=\mu(X)=0\) ainsi que \( \mu^*(X\cap A)=\mu^*(X\cap A^c)=0\).

        Si au contraire \( X\) n'est pas dénombrable,
        \begin{equation}
            \mu^*(X)=\inf_{\substack{A\in\tribF\\X\subset A}}\mu(A)=\infty,
        \end{equation}
         parce que \( X\) n'étant pas dénombrable, l'ensemble \( A\) ne l'est pas non plus et \( \mu(A)=\infty\). Mais comme \( X\) n'est pas dénombrable, soit \( X\cap A\) soit \( X\cap A^c\) (soit les deux) n'est pas dénombrable non plus; par conséquent
         \begin{equation}
             \mu^*(X\cap A)+\mu^*(X\cap A^c)=\infty.
         \end{equation}
    \end{subproof}

    Par conséquent \( (S,\tribF,\mu) \neq (S,\tribM,\mu^*)\). Mais vu que \( (S,\tribF,\mu)\) est complété nous devons avoir \( (S,\tribF,\mu)=(S,\hat\tribF,\hat\mu)\). Tout cela pour dire que nous avons un exemple avec
    \begin{equation}
        (S,\tribM,\mu^*)\neq (S,\hat\tribF,\hat \mu).
    \end{equation}
\end{example}

Nous avons deux façons de créer un espace complet à partir de \( (S,\tribF,\mu)\).
\begin{enumerate}
    \item
        Partir de la mesure extérieure \( \mu^*\) et construire \( (S,\tribM,\mu^*)\).
    \item
        Partir des ensembles \( \mu\)-négligeables, construire \( \hat\tribF\) et ensuite \( (S,\hat\tribF,\hat\mu)\).
\end{enumerate}
Ces deux façons ne sont pas équivalentes en général comme le montre l'exemple~\ref{ExOIXoosScTC}. Mais il sera montré par la proposition~\ref{PropIIHooAIbfj} que si \( (S,\tribF,\mu)\) est \( \sigma\)-fini alors les deux sont équivalent.

\begin{lemma}   \label{LemAESoofkMpi}
    Soit \( (S,\tribF,\mu)\) un espace mesuré. Alors pour tout \( X\subset S\) tel que \( \mu^*(X)<\infty\) il existe \( A\in\tribF\) tel que \( X\subset A\) et \( \mu^*(X)=\mu(A)\).
\end{lemma}
C'est-à-dire que \( \mu^*\) a beau être défini sur toutes les parties de \( S\), ce qu'il faut rajouter pour être \( \mu\)-mesurable, c'est pas grand chose.

\begin{proof}
    Par définition de la mesure extérieure associée à \( \mu\) en tant qu'infimum, pour tout \( n\geq 1\), il existe \( A_n\in\tribF\) tel que \( X\subset A_n\) et \( \mu(A_n)\leq \mu^*(X)+\frac{1}{ 2^n }\). Nous posons \( A=\bigcap_{n\geq 1}A_n\) et nous vérifions que ce \( A\) fait l'affaire.

    D'abord \( A\in\tribF\) parce qu'une tribu est stable par union dénombrable. Ensuite pour tout \( n\geq 1\) nous avons
    \begin{equation}
        \mu(A)\leq \mu(A_n)\leq \mu^*(X)+\frac{1}{ 2^n },
    \end{equation}
    et à la limite \( \mu(A)\leq \mu^*(X)\). Mais \( X\subset A\) implique \( \mu^*(X)\leq \mu(A)\) parce que \( \mu^*(X)\) l'infimum d'un ensemble contenant \( \mu(A)\).
\end{proof}

\begin{corollary}\label{LemXOUNooUbtpxm}
    Soit une mesure \( \mu\) et la mesure extérieure \( \mu^*\) associée\footnote{Par la proposition~\ref{PropFDUooVxJaJ}.}. Une partie \( N\) de \( X\) est négligeable si et seulement si \( \mu^*(N)=0\).
\end{corollary}

\begin{proof}
Si \( \mu^*\) est la mesure extérieure associée à \( \mu\) et si \( N\) est \( \mu\)-négligeable alors \( \mu^*(N)=0\) parce que
\begin{equation}
    \mu^*(N)\leq \mu^*(Y)=\mu(Y)=0
\end{equation}
pour un certain \( Y\) mesurable de mesure nulle contenant \( N\).

D'autre part si \( \mu^*(N)=0\) alors le lemme~\ref{LemAESoofkMpi} donne une partie mesurable \( A\) telle que \( N\subset A\) et \( \mu(A)=0\), c'est-à-dire que \( N\) est négligeable.
\end{proof}

\begin{lemma}       \label{LemOAEoocBDaO}
    Si l'espace mesuré \( (S,\tribF,\mu)\) est \( \sigma\)-fini alors l'espace mesuré \( (S,\tribM,\mu^*)\) est également \( \sigma\)-fini.
\end{lemma}

\begin{proof}
    Vu que \( (S,\tribF,\mu)\) est \( \sigma\)-fini, nous avons une suite croissante \( A_n\) d'éléments de \( \tribF\) tels que \( \bigcup_nA_n=S\) et telle que \( \mu(A_n)<\infty\) pour tout \( n\). Étant donné que \( \tribF\subset\tribM\), cette suite convient également pour montrer que \( (S,\tribM,\mu^*)\) est \( \sigma\)-fini parce que \( \mu^*(A_n)=\mu(A_n)<\infty\).
\end{proof}

La proposition suivante montre que si \( (S,\tribF,\mu)\) est \( \sigma\)-finie alors nous avons l'égalité.
\begin{proposition} \label{PropIIHooAIbfj}
    Soit \( (S,\tribF,\mu)\) un espace mesuré \( \sigma\)-fini, \( \mu^*\) la mesure extérieure associée et \( \tribM\) la tribu des ensembles \( \mu^*\)-mesurables\footnote{C'est bien une tribu par~\ref{ThoUUIooaNljH}\ref{RPPooHSWWsi}.}. Alors
    \begin{equation}
    (S,\tribM,\mu^*) = (S,\hat\tribF,\hat\mu).
    \end{equation}
\end{proposition}

\begin{proof}
    La proposition~\ref{PropOJFoozSKAE} indique que tous les éléments de \( \tribF\) sont \( \mu^*\)-mesurables, c'est-à-dire que \( \tribF\subset \tribM\). Mais l'espace \( (S,\tribM,\mu^*)\) est complet par le théorème de Carathéodory~\ref{ThoUUIooaNljH}, donc par minimalité du complété (\ref{thoCRMootPojn}\ref{thoCRMootPojnvii}),
    \begin{equation}
        (S,\hat\tribF,\hat\mu)\subset(S,\tribM,\mu^*)
    \end{equation}
    au sens où \( \hat\tribF\subset\tribM\) et si \( A\in\hat\tribF\) alors \( \hat\mu(A)=\mu^*(A)\). Notons que cette inclusion est vraie même si la mesure n'est pas \( \sigma\)-finie.

    Nous passons à l'inclusion inverse. Soit \( A\in\tribM\), c'est-à-dire que pour tout \( Y\subset S\) nous avons
    \begin{equation}    \label{EqTZAooTCdGg}
        \mu^*(Y)=\mu^*(Y\cap A)+\mu^*(Y\cap A^c).
    \end{equation}
    Nous allons montrer que \( A\in\hat\tribF\) en séparant les cas suivant que \( \mu^*(A)=\infty\) ou non.

    \begin{subproof}
        \item[Si \( \mu^*(A)<\infty\)]

        Par le lemme~\ref{LemAESoofkMpi}, il existe \( X\in\tribF\) tel que \( A\subset X\) et \( \mu^*(A)=\mu(X)\). Vu que \( (S,\tribF,\mu)\subset (S,\tribM,\mu^*)\) nous avons alors
        \begin{equation}    \label{EqKFQooQaont}
            \mu^*(A)=\mu(X)=\mu^*(X).
        \end{equation}
        Nous écrivons la relation \eqref{EqTZAooTCdGg} avec ce \( X\) en guise de \( Y\), et en nous souvenant que \( X\cap A=A\) et \( X\cap A^c=X\setminus A\) :
        \begin{equation}
            \mu^*(X)=\mu^*(A)+\mu^*(X\setminus A).
        \end{equation}
        En tenant compte de \eqref{EqKFQooQaont} et du fait que \( \mu^*(A)<\infty\), nous pouvons simplifier et trouver \( \mu^*(X\setminus A)=0\). Le lemme~\ref{LemAESoofkMpi} nous donne alors \( B\in\tribF\) tel que \( X\setminus A\subset B\) et \( \mu(B)=\mu^*(X\setminus A)=0\), c'est-à-dire que \( X\setminus A\) est \( \mu\)-négligeable. Par conséquent \( X\setminus A\in\hat\tribF\). En écrivant
        \begin{equation}
            A=X\setminus(X\setminus A),
        \end{equation}
        nous avons écrit \( A\) comme différence de deux éléments de \( \hat\tribF\) et nous concluons que \( A\in\hat\tribF\).

        \item[Si \( \mu^*(A)<\infty\)]

            Le lemme~\ref{LemOAEoocBDaO} nous indique que \( (S,\tribM,\mu^*)\) est \( \sigma\)-fini et il existe donc une suite \( (S_n)_{n\geq 1}\) dans \( \tribM\) telle que \( \bigcup_nS_n=S\) et \( \mu^*(S_n)<\infty\). L'ensemble \( A\cap S_n\) est un élément de \( \tribM\) vérifiant
            \begin{equation}
                \mu^*(A\cap S_n)\leq \mu^*(A)<\infty,
            \end{equation}
            ce qui implique que \( A\cap S_n\in\hat\tribF\) par la première partie. Maintenant \( A=\bigcup_n(A\cap S_n)\in\hat\tribF\) par union dénombrable d'éléments de la tribu \( \hat\tribF\).
    \end{subproof}
\end{proof}


\begin{proposition}[\cite{MonCerveau}] \label{PROPooAMIEooRomnMG}
    Soit un espace mesuré \( (\Omega,\tribF,\mu)\). Nous considérons un mesurable \( M\in \tribF\) ainsi que
    \begin{itemize}
        \item
            la tribu induite \( \tribF_M=\{ A\cap M\tq A\in \tribF \}\),
        \item
            la tribu complétée \( \hat\tribF\) de \( \tribF\) dans \( \Omega\),
        \item
            la tribu complétée \( \widehat{\tribF_M}\) de \( \tribF_M\) dans \( M\) (où nous avons considéré la mesure restreinte\footnote{Ce n'est pas ce qu'il se passe dans le cas de \( S^1\) par rapport à \( \eC\), voir la proposition \ref{PROPooDLBCooUfQZOa}\ref{ITEMooXDBTooYnauyi} malgré que \( S^1\) est un borélien de \( \eC\).} de \( \mu\)).
        \item
            la tribu induite \( (\hat\tribF)_M=\{ A\cap M\tq A\in \hat\tribF \}\) de \( \hat\tribF\) sur \( M\).
    \end{itemize}
    Alors
    \begin{equation}
        (\hat\tribF)_M=\widehat{\tribF_M}.
    \end{equation}
\end{proposition}

\begin{proof}
    L'utilisation de la proposition \ref{PROPooUNNSooMUQKfp} nous donne déjà les expressions alternatives
    \begin{equation}
        (\hat\tribF)_M=\{ A\cap M\tq A\in\hat\tribF \}=\{ A\subset M\tq A\in\hat\tribF \}
    \end{equation}
    et
    \begin{equation}
        \tribF_M=\{ A\cap M\tq A\in\tribF \}=\{ A\subset M\tq A\in\tribF \}.
    \end{equation}

    Pour prouver \( (\hat\tribF)_M=\widehat{\tribF_M}\) il faudra faire deux inclusions, et nous avons l'embarras du choix.
    \begin{subproof}
    \item[Première : \( \widehat{\tribF_M}\subset\{ M\cap A\tq A\in\hat\tribF \}\)]
        Un élément de \( \widehat{\tribF_M}\) est de la forme \( B\cup N\) où \( B\in \tribF_M\) et où \( N\) est négligeable\footnote{Pour rappel, une partie est négligeable quand elle est inclue à une partie de mesure nulle.} dans \( M\). Vu que \( B\in\tribF_M\), il existe \( A\in\tribF\) tel que \( B=A\cap M\). Vu que \( B\) et \( N\) sont dans \( M\) nous pouvons «factoriser» l'intersection :
            \begin{equation}
                B\cup N=M\cap (A\cup N)
            \end{equation}
            avec \( N\) négligeable dans \( M\) et donc également négligeable dans \( \Omega\). Donc \( A\cup N\in \hat\tribF\).

        \item[Deuxième : \( \{ M\cap A\tq A\in\hat\tribF \}\subset \widehat{\tribF_M}\)]

            Soit \( A\in\hat\tribF\). Nous avons une partie négligeable \( N\) de \( \Omega\) et un élément \( B\in \tribF\) tels que \( A=B\cup N\). Nous avons la décomposition
            \begin{equation}        \label{EQooWWQEooDRlnLN}
                M\cap(B\cup N)=(M\cap B)\cap(M\cap N).
            \end{equation}
            Il s'agit maintenant de nous assurer que cette décomposition implique que \( M\cap(B\cup N)\in \widehat{\tribF_M}\).
            
            Soit \( N_1\in \tribF\) tel que \( \mu(N_1)=0\) et \( N\subset N_1\). Vu que \( M\cap N_1\in \tribF\) (intersections dans une tribu), nous pouvons écrire
            \begin{equation}
                M\cap N\subset M\cap N_1
            \end{equation}
            avec \( \mu(M\cap N_1)=0\). Cela pour dire que \( M\cap N\) est négligeable dans \( M\). La décomposition \eqref{EQooWWQEooDRlnLN} est donc bien une union d'un élément de \( \tribF_M\) avec un négligeable de \( M\), et donc bien un élément de \( \widehat{\tribF_M}\).
    \end{subproof}
\end{proof}

\begin{normaltext}
    La principale application de la proposition \ref{PROPooAMIEooRomnMG} est le cas où \( \tribF=\Borelien(\eR^n)\) et \( M\) est un borélien \( B\) de \( \eR^n\). Dans ce cas, la proposition explique que la tribu de Lebesgue sur \( B\) (complétée depuis les boréliens de la topologie induite) est donnée directement par l'intersection entre \( B\) et la tribu de Lebesgue de \( \eR^n\). Donc sans devoir passer par la topologie induite, les boréliens et la completion :
    \begin{equation}
        \Lebesgue(\eR^n)_M=\widehat{\Borelien(\eR^n)_M}.
    \end{equation}
    Exemple dans la proposition \ref{PROPooHMSCooRIjcJq} qui donne une structure d'espace mesuré dans \( S^1\) à partir de la mesure de Lebesgue sur \( \eC\).
\end{normaltext}

%---------------------------------------------------------------------------------------------------------------------------
\subsection{Prolongement}
%---------------------------------------------------------------------------------------------------------------------------

Le théorème suivant est parfois nommé théorème d'extension de Carathéodory, par exemple sur Wikipédia. Le théorème de Carathéodory en étant un des ingrédients principaux, on comprend.
\begin{theorem}[Prolongement de Hahn\cite{MesureLebesgueLi}]    \label{ThoLCQoojiFfZ}
    Soit \( \tribA\) une algèbre de parties d'un ensemble \( S\) et \( \mu\) une mesure sur \( (S,\tribA)\). Soit \( \tribF=\sigma(\tribA)\) la tribu engendrée par \( \tribA\). Alors
    \begin{enumerate}
        \item
            La mesure \( \mu\) se prolonge en une mesure \( m\) sur \( \tribF\).
        \item
            Si \( \mu\) est \( \sigma\)-finie alors le prolongement est unique et \( m\) est \( \sigma\)-finie.
        \item
            Si \( \mu\) est finie, alors \( m\) l'est aussi.
    \end{enumerate}
\end{theorem}
\index{théorème!prolongement de Hahn}
\index{prolongement!théorème de Hahn}

\begin{proof}
    La proposition~\ref{PropIUOoobjfIB} nous donne une mesure extérieure \( \mu^*\) sur \( S\) dont la restriction à \( \tribA\) est \( \mu\). Si \( \tribM\) est la tribu des parties \( \mu^*\)-mesurables de \( S\) alors le théorème de Carathéodory~\ref{ThoUUIooaNljH} nous dit que \( (S,\tribM,\mu^*)\) est un espace mesuré.
    \begin{subproof}
        \item[\( \tribA\subset\tribM\)]
            Cette partie est une adaptation de ce qui a déjà été fait dans la preuve de la proposition~\ref{PropOJFoozSKAE}. Soit \( A\in\tribA\) et \( X\in S\); nous devons prouver la relation de la définition~\ref{DefTRBoorvnUY}. Vu que \( \mu^*\) est une mesure extérieure nous avons automatiquement
            \begin{equation}
                \mu^*(A)\leq \mu^*(X\cap A)+\mu^*(X\cap A^c).
            \end{equation}
            Il reste à prouver l'inégalité inverse. Soit une suite \( B_k\) d'éléments de \( \tribA\) telle que \( X\subset\bigcup_kB_k\); nous avons alors
            \begin{equation}
                \mu^*(X\cap A)\leq \mu^*\big( \bigcup_{k=1}^{\infty}B_k\cap A \big)\leq \sum_{k=1}^{\infty}\mu^*(B_k\cap A)=\sum_k\mu(B_k\cap A)
            \end{equation}
            où nous avons utilisé la définition~\ref{DefUMWoolmMaf}\ref{ItemARKooppZfDaiii} ainsi que le lemme~\ref{LemBFKootqXKl}. De la même façon,
            \begin{equation}
                \mu^*(X\cap A^c)\leq \sum_k\mu(B_k\cap A^c).
            \end{equation}
            Mettant les deux bouts ensemble, en remarquant que \( B_k\cap A\in\tribA\) et donc que \( \mu^*(B_k\cap A)=\mu(B_k\cap A)\),
            \begin{equation}
                \mu^*(X\cap A)+\mu^*(X\cap A^c)\leq \sum_k\mu(B_k\cap A)+\mu(B_k\cap A^c)=\sum_k\mu(B_k).
            \end{equation}
            La somme \( \mu^*(X\cap A)+\mu^*(X\cap A^c)\) est donc inférieure à chacun des éléments de l'ensemble sur lequel on prend l'infimum pour définir\footnote{Définition~\ref{EqRNJooQrcoa}.} \( \mu^*(X)\), donc
            \begin{equation}
                 \mu^*(X\cap A)+\mu^*(X\cap A^c)\leq \mu^*(X).
            \end{equation}
    \end{subproof}

    A fortiori nous avons \( \sigma(\tribA)\subset\tribM\) et donc \( (S,\sigma(\tribA),\mu^*)\) est un espace mesuré. Cela prouve l'existence d'une mesure prolongeant \( \mu\) à \( \sigma(\tribA)\).

    \begin{subproof}
        \item[Unicité]

            Nous supposons à présent que \( \mu\) est \( \sigma\)-finie. Soient \( m_1\) et \( m_2\) deux mesures prolongeant \( \mu\) et définies sur une tribu contenant \( \tribA\). Nous posons
            \begin{equation}
                \tribC=\{ A\in\tribA\tq\mu(A)<\infty \}.
            \end{equation}
            Dans l'optique d'utiliser le théorème d'unicité des mesures~\ref{ThoJDYlsXu}, nous prouvons que \( \sigma(\tribA)=\sigma(\tribC)\). Vu que \( \mu\) est \( \sigma\)-finie, il existe une suite croissante \( (S_n)\) d'éléments de \( \tribA\) telle que \( S=\bigcup_nS_n\) et \( \mu(S_n)\in\tribC\). Alors si \( A\in\tribA\) nous avons \( A=\bigcup_n(A\cap S_n)\), et donc \( A\in\sigma(\tribC)\). Donc \( \tribA\subset\sigma(\tribC)\). Mais étant donné que \( \tribC\subset\tribA\) nous avons aussi \( \sigma(\tribC)\subset\sigma(\tribA)\). Au final \( \sigma(\tribA)=\sigma(\tribC)\).

            Les mesures \( m_1\) et \( m_2\) sont des mesures sur \( \sigma(\tribC)\) coïncidant sur \( \tribC\) (parce que \( \tribC\subset\tribA\)). De plus la classe \( \tribC\) est stable par intersection finie et contient une suite croissante dont l'union est \( S\) (parce que \( \mu\) est \( \sigma\)-finie).

            Le théorème~\ref{ThoJDYlsXu} nous dit alors que \( m_1\) et \( m_2\) coïncident sur \( \sigma(\tribC)=\sigma(\tribA)\).

        \item[Extension finie et \( \sigma\)-finie]

            Enfin si \( \mu\) est \( \sigma\)-finie il existe \( S_n\in\tribA\) avec \( \mu(S_n)<\infty\) et \( \bigcup_nS_n=S\). Ces ensembles vérifient tout autant \( m(S_n)=\mu(S_n)<\infty\) pour tout prolongement \( m\) de \( \mu\).

            Idem si \( \mu\) est finie, tout prolongement est fini.
    \end{subproof}
\end{proof}

% TODO : parler de la compactification en un point de R ainsi que de la topologie sur R U {+infini,-infini}.
\begin{example}[\cite{MesureLebesgueLi}] \label{ExKCEoolsZrL}
Soit \( \tribA\), l'algèbre de parties de \( \eR\) formée par les réunions finies d'intervalles de la forme \( \mathopen] -\infty , a \mathclose[\), \( \mathopen[ a , b [\) et \( \mathopen[ b , +\infty [\) avec \( -\infty<a\leq b<+\infty\). Notons que les singletons ne font pas partie de \( \tribA\) parce que \( \mathopen[ a ,a[=\emptyset\). Nous posons
    \begin{equation}
        \mu(A)=\begin{cases}
            0    &   \text{si } A=\emptyset\\
            \infty    &    \text{sinon}.
        \end{cases}
    \end{equation}
    Cela donne une mesure (non \( \sigma\)-finie) sur \( (\eR,\tribA)\).

    Nous allons prouver que la tribu engendrée par \( \tribA\) est la tribu des boréliens et que \( \mu\) accepte (au moins) deux prolongements distincts à \( \sigma(\tribA)\).

    D'abord nous avons
    \begin{equation}
        \mathopen] a , b \mathclose[=\big( \mathopen -\infty , a \mathclose[\cup \mathopen[ b , +\infty [ \big)\cap\mathopen[ a , b [,
    \end{equation}
    donc toutes les boules ouvertes appartiennent à \( \sigma(\tribA)\). Ces dernières comprenant une base dénombrable de la topologie de \( \eR\) (par la proposition~\ref{PropNBSooraAFr}), tous les ouverts de \( \eR\) sont dans \( \sigma(\tribA)\). Par conséquent \( \Borelien(\eR)\subset(\eR^d)\). Mais en même temps tous les éléments de \( \tribA\) sont des boréliens, donc \( \Borelien(\eR)=\sigma(\tribA)\) parce que la fermeture en tant qu'algèbre de parties est plus petite que la fermeture en tant que tribu.

    La mesure de comptage prolonge \( \mu\) parce qu'à part l'ensemble vide, tous les éléments de \( \tribA\) sont infinis. Notons que les singletons sont dans \( \sigma(\tribA)\), donc la mesure de comptage prend d'autres valeurs que \( 0\) et \( +\infty\).

    Par ailleurs la mesure
    \begin{equation}
        \mu'(A)=\begin{cases}
            0    &   \text{si } A=\emptyset\\
            +\infty    &    \text{sinon}
        \end{cases}
    \end{equation}
    est également une mesure prolongeant \( \mu\) à \( \sigma(\tribA)=\Borelien(\eR)\).

    La mesure de comptage et \( \mu'\) sont deux prolongements distincts de \( \mu\).
\end{example}

%TODO : quelle partie de R n'est pas borélienne ?

\begin{example}[\cite{MesureLebesgueLi}]
    Nous montrons maintenant une mesure non \( \sigma\)-finie qui se prolonge en deux mesures distinctes, toutes deux \( \sigma\)-finies.

    Nous considérons la même algèbre \( \tribA\) de parties que celle donnée dans l'exemple~\ref{ExKCEoolsZrL}, mais cette fois vue sur \( \eQ\) uniquement. La mesure de comptage \( m\) sur \( (\eQ,\tribA)\) n'est pas \( \sigma\)-finie.

    Vu que les singletons sont des boréliens nous avons \( \sigma(\tribA)=\partP(\eQ)\), ce qui fait que \( (\eQ,\sigma(\tribA),m)\) est un prolongement \( \sigma\)-fini de \( m\). L'espace mesuré \( (\eQ,\sigma(\tribA),2m)\) est également \( \sigma\)-fini et est un prolongement distinct de \( (\eQ,\tribA,m)\).
\end{example}

\begin{proposition}     \label{PROPooORDCooJEsjzR}
    Soient des espaces mesurés \( (S_1,\tribF_1,\mu_1)\) et \( (S_2,\tribF_2, \mu_2)\) ainsi qu'une application \( \varphi\colon S_1\to S_2\) avec les hypothèses suivantes :
    \begin{enumerate}
        \item
            \( \varphi\) est une bijection,
        \item
            \( \varphi\) est mesurable d'inverse mesurable,
        \item
            si \( \mu_1(A)=0\) alors \( \mu_2\big( \varphi(A) \big)=0\),
        \item
            si \( \mu_2(A)=0\) alors \( \mu_1\big( \varphi^{-1}(A) \big)=0\).
    \end{enumerate}
    Alors
    \begin{equation}
        \hat\tribF_2=\varphi(\hat\tribF_1).
    \end{equation}
\end{proposition}

\begin{proof}
    Nous prouvons que \( \hat\tribF_1\subset \varphi(\hat\tribF_1)\). Vu la symétrie des hypothèses, l'inclusion inverse se fera de même.

    Soit \( A\in\hat\tribF_2\). Nous avons \( A=B\cup N\) avec \( B\in \tribF_2\) et \( N\), une partie \( \mu_2\)-négligeable. Nous considérons \( N_1\in\tribF_2\) tel que \( \mu_2(N_1)=0\) et \( N\subset N_1\). Notre but est maintenant de prouver que \( \varphi^{-1}(B\cup N)\in \hat\tribF_1\). 

    Vu que \( \varphi\) est une bijection, nous avons
    \begin{equation}
        \varphi^{-1}(B\cup N)=\varphi^{-1}(B)\cup \varphi^{-1}(N).
    \end{equation}
    Là-dedans, \( \varphi^{-1}(B)\in \tribF_1\) parce que \( \varphi\) est borélienne. Il nous reste à voir que \( \varphi^{-1}(N)\) est \( \mu_1\)-négligeable. Vu que \( N\subset N_1\), nous avons \( \varphi^{-1}(N)\subset\varphi^{-1}(N_1)\) où \( \varphi^{-1}(N_1)\in\tribF_1\).

    Par construction, \( \mu_2(N_1)=0\) et par hypothèse, \( \mu_1\big( \varphi^{-1}(N_1) \big)=0\).

    Au total,
    \begin{equation}
        \varphi^{-1}(B\cup N)=\underbrace{\varphi^{-1}(B)}_{\in\tribF_1}\cup\underbrace{\varphi^{-1}(N)}_{\mu_1\text{-négligeable}}\in\hat\tribF_1.
    \end{equation}
\end{proof}

%+++++++++++++++++++++++++++++++++++++++++++++++++++++++++++++++++++++++++++++++++++++++++++++++++++++++++++++++++++++++++++ 
\section{Tribu borélienne}
%+++++++++++++++++++++++++++++++++++++++++++++++++++++++++++++++++++++++++++++++++++++++++++++++++++++++++++++++++++++++++++

%///////////////////////////////////////////////////////////////////////////////////////////////////////////////////////////
\subsubsection{Définition}
%///////////////////////////////////////////////////////////////////////////////////////////////////////////////////////////

\begin{definition}[Tribu borélienne]        \label{DEFooQBQGooTqGdtY}
    La tribu des \defe{boréliens}{boréliens}\index{tribu!borélienne}, notée \( \Borelien(\eR^d)\) est la tribu engendrée par les ouverts de \( \eR^d\). Plus généralement si \( Y\) est un espace topologique, la tribu des boréliens est la tribu engendrée par les ouverts de \( Y\).
\end{definition}
\index{borélienne!tribu}

\begin{proposition} \label{PROPooYEkvbWBz}
    La tribu engendrée par une base dénombrable de la topologie est celle des boréliens.
\end{proposition}

\begin{proof}
    Si une base de topologie est donnée, tout ouvert peut être écrit comme union d'élément de la base, proposition~\ref{PropMMKBjgY}. Dans le cas d'une base dénombrable, cette union sera forcément dénombrable. Une tribu étant stable par union dénombrable, tout ouvert est dans la tribu engendrée par la base de topologie. Les autres boréliens suivent automatiquement.

    Dit avec plus de lettres et moins de phrases, si \( \tribD\) est une base dénombrable de la topologie de \( X\), et si \( \mO\) est un ouvert de \( X\), nous avons \( \mO=\bigcup_{i=1}^{\infty}A_i\) avec \( A_i\in\tribD\). Vu qu'une tribu est stable par union dénombrable\footnote{Définition~\ref{DefjRsGSy}\ref{ItemooPEQNooYiYNtN}}, nous avons \( \mO\in\sigma(\tribD)\). En conséquence de quoi \( \Borelien(X)\subset\sigma(\tribD)\).

    Mais comme \( \tribD\subset\Borelien(X)\) l'inclusion inverse est automatique. D'où l'égalité \( \Borelien(X)=\sigma(\tribD)\).
\end{proof}

%///////////////////////////////////////////////////////////////////////////////////////////////////////////////////////////
\subsubsection{Les boréliens de \texorpdfstring{$ \eR$}{R}}
%///////////////////////////////////////////////////////////////////////////////////////////////////////////////////////////

Nous rappelons que la topologie de \( \eR\) est celle des boules donnée par le théorème~\ref{ThoORdLYUu}. Nous rappelons (voir la proposition~\ref{PropNBSooraAFr} et sa preuve) que les boules ouvertes de la forme \( B(q,r)\) avec \( q,r\in \eQ\) forment une base dénombrable de la topologique de \( \eR\).

\begin{lemma}   \label{LemZXnAbtl}
    Soit \( \{ q_i \}\) une énumération des rationnels. La tribu engendrée par les ouverts \( \sigma_i=\mathopen] q_i , \infty \mathclose[\) est la tribu des boréliens.
\end{lemma}

\begin{proof}
    Si \( a<b\) dans \( \eQ\) alors \( \sigma_a\setminus\sigma_b=\mathopen] a , b \mathclose]\). Ensuite
    \begin{equation}
        \bigcup_{n\in \eN^*}\sigma_a\setminus\sigma_{b-\frac{1}{ n }}=\bigcup_{n\in \eN^*}\mathopen] a , b-\frac{1}{ n } \mathclose]=\mathopen] a , b \mathclose[.
    \end{equation}
    Par union dénombrable, tous les intervalles \( \mathopen] a , b \mathclose[\) avec \( a,b\in \eQ\) sont dans la tribu engendrée par les \( \sigma_i\).

        Ces boules ouvertes forment une base de la topologie de \( \eR\) par la proposition~\ref{PropNBSooraAFr} et la proposition~\ref{PROPooYEkvbWBz} conclu.
\end{proof}

\begin{example}
    Les singletons sont des boréliens de \( \eR\) parce que
    \begin{equation}
    \{ x \}=\Big( \mathopen] -\infty , x \mathclose[\cup\mathopen] x , +\infty \mathclose[ \Big)^c.
    \end{equation}

    Vu qu'une tribu est stable par union dénombrable, l'ensemble \( \eQ\) est un borélien de \( \eR\). Et comme les tribus sont stables par différence ensembliste (\ref{LemBWNlKfA}\ref{ItemXQVLooFGBQNj}), l'ensemble des irrationnels est un borélien de \( \eR\).
\end{example}

%///////////////////////////////////////////////////////////////////////////////////////////////////////////////////////////
\subsubsection{Diverses expressions}
%///////////////////////////////////////////////////////////////////////////////////////////////////////////////////////////

\begin{lemma}   \label{LEMooUPYDooPVjscA}
    Soient un espace topologique \( X\) et un borélien \( B\) de \( X\). Nous considérons sur \( B\) la topologie induite\footnote{Définition \ref{DefVLrgWDB}.} de \( X\) et les boréliens \( \Borelien(B)\) correspondants. Nous avons :
    \begin{equation}
        \Borelien(B)=\{ A\in \Borelien(X)\tq A\subset B \}=\{ B\cap A\tq A\in \Borelien(X) \}.
    \end{equation}
    En particulier,
    \begin{equation}    \label{EQooEUWVooCBUims}
        \Borelien(B)=\Borelien(X)_B.
    \end{equation}
\end{lemma}

\begin{proof}
    L'égalité
    \begin{equation}
        \{ A\in \Borelien(X)\tq A\subset B \}=\{ B\cap A\tq A\in \Borelien(X) \}
    \end{equation}
    est déjà dans la proposition \ref{PROPooUNNSooMUQKfp}.

    Nous démontrons maintenant que
    \begin{equation}
        \Borelien(B)=\{ A\cap B\tq A\in\Borelien(X) \}.
    \end{equation}
    Pour ce faire, nous nous rappelons du lemme de transport \ref{LemOQTBooWGYuDU}. Soit l'injection canonique \( f\colon B\to X\); pour tout \( A\subset X\) nous avons \( f^{-1}(A)=A\cap B\). 
    
    Nous considérons la classe \( \tribT\) des ouverts de \( X\). Par définition de la topologie induite, les ouverts de \( B\) sont les éléments de \( f^{-1}(\tribT)\). Donc
    \begin{equation}
        \sigma\big( f^{-1}(\tribT) \big)=\Borelien(B).
    \end{equation}
    Mais d'autre part,
    \begin{equation}
        f^{-1}\big( \sigma(\tribT) \big)=\{ A\cap B\tq A\in \Borelien(X) \}.
    \end{equation}
    Donc le lemme de transport \ref{LemOQTBooWGYuDU} nous dit que
    \begin{equation}
        \Borelien(B)=\sigma\big( f^{-1}(\tribT) \big)= f^{-1}\big( \sigma(\tribT) \big)=\{ A\cap B\tq A\in \Borelien(X) \}.
    \end{equation}

    Pour finir, l'égalité \eqref{EQooEUWVooCBUims} n'est autre que le fait que
    \begin{equation}
        \Borelien(X)_B=\{ B\cap A\tq A\in \Borelien(X) \},
    \end{equation}
    alors que nous venons de prouver que le membre de droite est \( \Borelien(B)\).
\end{proof}

%---------------------------------------------------------------------------------------------------------------------------
\subsection{Mesure image}
%---------------------------------------------------------------------------------------------------------------------------

Le produit d'une mesure par une fonction est définit par la propriété~\ref{PropooVXPMooGSkyBo}.

\begin{propositionDef}[Mesure image\cite{TribuLi}]     \label{PropJCJQooAdqrGA}
    Soient \( (S_1,\tribF_1)\) et \( (S_2,\tribF_2)\) des espaces mesurables. Soit \( \varphi\colon S_1\to S_2\) une application mesurable. Si \( m_1\) est une mesure positive sur \( S_1\) alors l'application définie par
    \begin{equation}
        m_2(A_2)=m_1\big( \varphi^{-1}(A_2) \big)
    \end{equation}
    est une mesure positive sur \( (S_2,\tribF_2)\).

    La mesure \( m_2\) ainsi définie est la \defe{mesure image}{mesure!image} de \( m_1\) par l'application \( \varphi\). Elle est notée \( \varphi(m_1)\).
\end{propositionDef}

\begin{proof}
    Il y a deux choses à vérifier pour avoir une mesure positive\footnote{Définition~\ref{DefBTsgznn}}. D'abord pour l'ensemble vide :
    \begin{equation}
        m_2(\emptyset)=m_1\big( \varphi^{-1}(\emptyset) \big)=m_1(\emptyset)=0.
    \end{equation}
    Ensuite pour l'additivité. Soient \( A_n\) dans \( \tribF_2\) des parties deux à deux disjointes et telles que \( \bigcup_nA_n\in\tribF_2\). Alors nous avons
    \begin{subequations}
        \begin{align}
            m_2\big( \bigcup_nA_n \big)&=m_1\Big( \varphi^{-1}(\bigcup_nA_n) \Big)\\
            =&m_1\big( \bigcup_n\varphi^{-1}(A_n) \big)\\
            &=\sum_nm_1\big( \varphi(A_n) \big)\\
            &=\sum_nm_2(A_n).
        \end{align}
    \end{subequations}
\end{proof}

\begin{lemma}
    Soient deux espaces mesurables \( (S_1,\tribF_1)\) et \( (S_2,\tribF_2)\) ainsi que deux mesures \( \mu\) et \( \nu\) sur \( (S_1,\tribF_1)\). Si \( \varphi\colon S_1\to S_2\) est mesurable et si \( \mu\leq \nu\) alors \( \varphi(\mu)\leq \varphi(\nu)\).
\end{lemma}

\begin{proof}
    Soit \( B\) mesurable dans \( (S_2,\tribF_2)\) (c'est-à-dire \( B\in \tribF_2\)). Alors
    \begin{equation}
        \varphi(\mu)(B)=\mu\big( \varphi^{-1}(B) \big)\leq\nu\big( \varphi^{-1}(B) \big)=\varphi(\nu)(B).
    \end{equation}
\end{proof}

Il est naturel de se demander comment il faut intégrer par rapport à une mesure image. La réponse sera dans le théorème~\ref{THOooVADUooLiRfGK}.

%---------------------------------------------------------------------------------------------------------------------------
\subsection{Régularité d'une mesure}
%---------------------------------------------------------------------------------------------------------------------------

Certaines mesures ont de la compatibilité avec la topologie. Nous allons étudier ça.

\begin{theorem}[\cite{TribuLi}]     \label{ThoPKGEooVrpsGU}
    Soit \( X\) un espace métrique et \( m\) une mesure positive bornée sur \( \big(X,\Borelien(X)\big)\). Alors si \( B\) est un borélien,
    \begin{enumerate}
        \item
            Régularité extérieure : \( m(B)=\inf\{ m(\Omega)\text{où } \Omega\text{ est un ouvert contenant } B \}\)
        \item
            Régularité intérieure : \( m(B)=\sup\{ m(F) \text{où } F\text{ est un fermé, } F\subset B \}\).
    \end{enumerate}
\end{theorem}

\begin{proof}
    Soit \( \tribF\) l'ensemble des \( B\in\Borelien(X)\) tels que pour tout \( \epsilon>0\), il existe \( \Omega_{\epsilon}\) ouvert et \( F_{\epsilon}\) fermé tels que \( F_{\epsilon}\subset B\subset \Omega_{\epsilon}\) et \( m(\Omega_{\epsilon}\setminus F_{\epsilon})\leq \epsilon\). Nous allons montrer que cela est une tribu contenant les ouverts. Comme cela est inclus dans la tribu borélienne, nous en déduirons que \( \tribF=\Borelien(X)\).
    \begin{subproof}
        \item[\( \tribF\) contient les ouverts]
            Soit \( \Omega\) un ouvert de \( X\). Alors \( \Omega^c\) est fermé et \( d(x,\Omega^c)=0\) si et seulement si \( x\in \Omega^c\) par la proposition~\ref{PropGULUooNzqZKj}. Nous pouvons donc écrire
            \begin{equation}
                 \Omega^c=\bigcap_{n\geq 1}\{ x\in X\tq d(x,\Omega^c)<\frac{1}{ n } \}.
            \end{equation}
            En passant au complémentaire et en posant \( F_n=\{ x\in X\tq d(x,\Omega^c)\geq \frac{1}{ n } \}\) nous avons
            \begin{equation}
                \Omega=\bigcup_{n\geq 1}F_n.
            \end{equation}
            Chacun des \( F_n\) est fermé parce que \( F_n\) est l'image réciproque du fermé \( \mathopen[ \frac{1}{ n } , \infty \mathclose[\) par l'application \( x\mapsto d(x,\Omega^c)\) qui est continue. De plus les \( F_n\) forment une suite croissante, donc le lemme~\ref{LemAZGByEs} nous assure que \( m(\Omega)=\lim_{n\to \infty}m(F_n)\). Et le lemme~\ref{LemPMprYuC} que \( m(\Omega\setminus F_n)=m(\Omega)-m(F_n)\).

                Soit \( \epsilon>0\). Il existe alors \( n_{\epsilon}\geq 1\) tel que
                \begin{equation}
                    m(\Omega\setminus F_n)=m(\Omega)-m(F_n)\leq \epsilon.
                \end{equation}
                Bref si \( \Omega\) est ouvert nous considérons \( \Omega_{\epsilon}=\Omega\) et \( F_{\epsilon}=F_{n_{\epsilon}}\) et nous avons
                \begin{equation}
                    F_{\epsilon}\subset \Omega\subset \Omega_{\epsilon}
                \end{equation}
                avec \( m(\Omega_{\epsilon}\setminus F_{\epsilon})\leq \epsilon\).

                L'ensemble \( \tribF\) contient les ouverts.

            \item[\( \tribF\) est une tribu]
                Il y a à vérifier les trois conditions de la définition~\ref{DefjRsGSy}.
                \begin{subproof}
                \item[Les ensembles faciles]
                    Les ensembles \( X\) et \( \emptyset\) sont dans \( \tribF\) parce qu'ils sont ouverts et fermés.
                \item[Complémentaire]
                    Soit \( B\in \tribF\), soit \( \epsilon>0\) et les ensembles \( F_{\epsilon} \) et \( \Omega_{\epsilon}\) qui vont avec. Alors en passant au complémentaire nous avons
                    \begin{equation}
                        \Omega_{\epsilon}^c\subset B^c\subset F_{\epsilon}^c
                    \end{equation}
                    De plus
                    \begin{equation}
                        F_{\epsilon}^c\setminus \Omega_{\epsilon}^c=F_{\epsilon}^c\cap(\Omega_{\epsilon}^c)^c=F_{\epsilon}^c\cap \Omega_{\epsilon}=\Omega_{\epsilon}\setminus F_{\epsilon}.
                    \end{equation}
                    Par conséquent
                    \begin{equation}
                        m(F_{\epsilon}^c\setminus \Omega_{\epsilon}^c)=m(\Omega_{\epsilon}\setminus F_{\epsilon})\leq \epsilon.
                    \end{equation}
                    Cela montre que \( B^c\in \tribF\).
                \item[Union dénombrable]
                    Soient \( (B_n)\) une suite d'éléments de \( \tribF\) et \( \epsilon>0\). Pour chaque \( n\) nous choisissons un ouvert \( \Omega_n\) et un fermé \( F_n\) tels que \( F_n\subset  B_n\subset \Omega_n\) et
                    \begin{equation}
                        m(\Omega_n\setminus F_n)\leq \frac{ \epsilon }{ 2^{n+2} }.
                    \end{equation}
                    Vu que \( \Omega_n\setminus B_n\subset \Omega_n\setminus F_n\) nous avons aussi
                    \begin{equation}
                        m(\Omega_n\setminus B_n)\leq m(\Omega_n\setminus F_n)\leq \frac{ \epsilon }{ 2^{n+2} }.
                    \end{equation}
                    Nous posons \( \Omega=\bigcup_{n\geq 1}\Omega_n\) (un ouvert) et \( B=\bigcup_{n\geq 1}B_n\) ainsi que \( A=\bigcup_{n\geq 1}F_n\) (qui n'est pas spécialement fermé).

                    Le but est de majorer \( m(\Omega\setminus F)\) où \( F\) est un fermé qui est encore à déterminer. Calculons déjà ceci :
                    \begin{subequations}
                        \begin{align}
                            \Omega\setminus B&=\bigcup_n\Omega_n\cap\big( \bigcup_kB_k \big)^c\\
                            &=\bigcup_n\Big( \Omega_n\cap\big( \bigcap_kB_k^c \big) \Big)\\
                            &\subset\bigcup_n\big( \Omega_n\cap B_n^c \big)\\
                            &=\bigcup_n(\Omega_n\setminus B_n)
                        \end{align}
                    \end{subequations}
                    où l'union n'est pas spécialement disjointe. Par conséquent,
                    \begin{equation}
                        m(\Omega\setminus B)\leq \sum_{n=1}^{\infty}m(\Omega_n\setminus B_n)\leq \sum_{n=1}^{\infty}\frac{ \epsilon }{ 2^{n+2} }=\frac{ \epsilon }{ 4 }.
                    \end{equation}
                    De la même façon nous avons
                    \begin{equation}
                        B\setminus A=\big( \bigcup_{n=1}^{\infty}B_n \big)\cap\big( \bigcup_{k=1}^{\infty}F_n \big)^c\subset \bigcup_{n=1}^{\infty}B_n\setminus F_n.
                    \end{equation}
                    Nous avons alors le inégalités de mesures
                    \begin{subequations}
                        \begin{align}
                            m(B\setminus A)&\leq \sum_{n=1}^{\infty}m(B_n\setminus F_n)\\
                            &\leq\sum_{n=1}^{\infty}m(\Omega_n\setminus F_n)\\
                            &\leq \frac{ \epsilon }{ 4 }.
                        \end{align}
                    \end{subequations}
                    C'est vraiment dommage que \( A\) ne soit pas en générale un fermé, sinon il répondrait à la question. Nous posons \( F'_1=F_1\) et \( F'_n=\bigcup_{k=1}^nF_k\). En tant qu'unions finies de fermés, les \( F'_n\) sont des fermés (lemme~\ref{LemQYUJwPC}\ref{ItemKJYVooMBmMbG}). De plus la suite \( (F'_n)\)  est croissante et l'union est \( A\). Par le lemme~\ref{LemAZGByEs}\ref{ItemJWUooRXNPci} nous avons
                    \begin{equation}
                        m(A)=m\big( \bigcup_nF'_n \big)=\lim_{n\to \infty} m(F'_n).
                    \end{equation}
                    Il existe donc \( n_{\epsilon}\) tel que
                    \begin{equation}
                        m(A)-m(F'_{n})\leq \epsilon
                    \end{equation}
                    Nous posons \( F=F'_{n_{\epsilon}}\). Vu que \( F\subset A\) nous avons aussi \( m(A\setminus F)=m(A)-m(F)\leq \epsilon\). Et en plus \( F\subset A\subset B\subset \Omega\), ce qui donne bien la propriété voulue \( F\subset B\subset \Omega\). Il reste à nous assurer de \( m(\Omega\setminus F)\). Nous avons d'abord
                    \begin{equation}
                        m(B\setminus F)=m\big( (B\setminus A)\cup (A\setminus F) \big)=m(B\setminus A)+m(A\setminus F)\leq \frac{ 5\epsilon }{ 4 }.
                    \end{equation}
                    Et enfin :
                    \begin{equation}
                        m(\Omega\setminus F)=m\big( (\Omega\setminus B)\cup (B\setminus F) \big)=m(\Omega\setminus B)+m(B\setminus F)\leq \frac{ 6\epsilon }{ 4 }.
                    \end{equation}
                    Et donc à redéfinition près de \( \epsilon\) c'est d'accord.

                \end{subproof}

                Il est donc établi que \( \tribF\) est une tribu. Qui plus est, l'ensemble \( \tribF\) est une tribu incluse aux boréliens et contenant les ouverts. Ergo \( \tribF=\Borelien(X)\).

            \item[Régularité extérieure]

                Soit \( B\) un borélien et \( \epsilon>0\). Alors il existe \( F_{\epsilon}\) fermé et \( \Omega_{\epsilon} \) ouvert tels que \( F_{\epsilon}\subset B\subset \Omega_{\epsilon}\) et \( m(\Omega_{\epsilon}\setminus F_{\epsilon})\leq \epsilon\). Vu que \( B\subset \Omega_{\epsilon}\) pour tout \( \epsilon\), nous avons aussi
                \begin{equation}
                    m(B)\leq \inf_{\epsilon}m(\Omega_{\epsilon}).
                \end{equation}
                Mais comme \( \mu(\Omega_{\epsilon})\geq m(B)\) pour tout \( \epsilon\), nous avons en réalité \( m(B)=\inf_{\epsilon}m(\Omega_{\epsilon})\).

                Soit maintenant un ouvert \( \Omega\) tel que \( B\subset \Omega\). Nous devons prouver l'existence d'un \( \epsilon>0\) tel que \( m(\Omega_{\epsilon})\leq m(\Omega)\). Cela permettra de conclure que l'infimum sur tous les ouverts contenant \( B\) est égal à l'infimum sur les ouverts de la forme \( \Omega_{\epsilon}\).

                Nous posons \( m(\Omega)=m(B)+\delta\) et avec \( \epsilon\leq \delta\) nous avons
                \begin{equation}
                    m(\Omega_{\epsilon}\setminus B)\leq m(\Omega_{\epsilon}\setminus F_{\epsilon})\leq \epsilon
                \end{equation}
                et donc aussi
                \begin{equation}
                    m(\Omega_{\epsilon})\leq m(B)+\epsilon\leq m(B)+\delta=m(\Omega).
                \end{equation}
            \item[Régularité intérieure]

                Elle se fait de même.
    \end{subproof}
\end{proof}

\begin{definition}      \label{DefFMTEooMjbWKK}
    Soit \( X\) un espace topologique et \( m\) une mesure positive sur \( \big( X,\Borelien(X) \big)\).
    \begin{enumerate}
        \item       \label{ItemTTPTooStDcpw}
            \( m\) est une \defe{mesure de Borel}{mesure!de Borel} si elle est finie sur tout compact.
        \item
            \( m\) est \defe{régulière extérieurement}{mesure!régulière!extérieure} si \( \forall B\in\Borelien(X)\),
            \begin{equation}
                m(B)=\inf\{ m(\Omega)\tq \Omega\text{ est ouvert et } B\subset \Omega \}
            \end{equation}
        \item
            \( m\) est \defe{régulière intérieurement}{mesure!régulière!intérieure} si \( \forall B\in\Borelien(X)\),
            \begin{equation}
                m(B)=\sup\{ m(K)\tq K\text{ est compact et } K\subset B  \}
            \end{equation}
        \item
            \( m\) est une mesure \defe{régulière}{mesure!régulière} si elle est régulière dans les deux sens.
        \item
            \( m\) est une \defe{mesure de Radon}{mesure!de Radon} si elle est de Borel et régulière.
    \end{enumerate}
\end{definition}
\index{régularité!d'une mesure}

\begin{proposition}     \label{PropNCASooBnbFrc}
    Soit \( X\) un espace localement compact et dénombrable à l'infini\footnote{Définitions~\ref{DefEIBYooAWoESf} et~\ref{DefFCGBooLpnSAK}.} Alors toute mesure de Borel sur \( \big( X,\Borelien(X) \big)\) est de Radon.
\end{proposition}

\begin{proof}
    Nous avons une suite exhaustive\footnote{Définition~\ref{LemGDeZlOo}.} de compacts \( X_k\) tels que
    \begin{equation}
        X=\bigcup_{k\geq 1}X_k=\bigcup_{k\geq 1}\Int(X_k).
    \end{equation}
    \begin{subproof}
    \item[Régularité intérieure]
    Soit \( B\), un borélien de \( X\); nous avons \( B=\bigcup_{k\geq 1}(B\cap X_k)\) et comme cette union est croissante,
    \begin{equation}
        m(B)=\lim_{k\to \infty} m(B\cap X_k)
    \end{equation}
    par le lemme~\ref{LemAZGByEs}\ref{ItemJWUooRXNPci}. Dans la suite, il va y avoir beaucoup de considérations sur les topologies induites. Nous nommons \( \tau_k\) la topologie de \( X_k\) induite depuis celle de \( X\). Il ne faudra pas confondre les expressions «un compact \emph{de} $X_k$»  et «un compact \emph{dans} \( X_k\)». La première parle d'un compact pour la topologie \( \tau_k\). La seconde parle d'un compact pour la topologie de \( X\), inclus dans \( X_k\).


    Si \( a<m(B)\) alors il existe \( k\geq 1\) tel que \( a<m(B\cap X_k)\), c'est-à-dire
    \begin{equation}
        a<m(B\cap X_k)\leq m(B).
    \end{equation}
    Mais \( (X_k,m)\) est un espace mesuré borné parce que \( m\) est de Borel et \( X_k\) est compact. Par conséquent la (restriction de la) mesure \( m\) est régulière sur l'espace mesuré \( \big( X_k,\Borelien(X_k) \big)\) par le théorème~\ref{ThoPKGEooVrpsGU}. De plus l'ensemble \( B\cap X_k\) est un borélien de \( (X_k,\tau_k)\) parce que
    \begin{equation}
        B\cap X_k\in\Borelien(X)_{X_k}=\Borelien(X_k)
    \end{equation}
    où nous avons utilisé la propriété de compatibilité entre topologie induite et tribu des borélien du théorème~\ref{ThoSVTHooChgvYa}. Il existe donc un fermé \( F_{\epsilon}\) de \( (X_k,\tau_k)\) tel que
    \begin{subequations}
        \begin{numcases}{}
            F_{\epsilon}\subset B\cap X_k\\
            m(B\cap X_k)\leq m(F_{\epsilon})+\epsilon.
        \end{numcases}
    \end{subequations}
    En mettant bout à bout les inégalités nous avons trouvé
    \begin{equation}
        a<m(B\cap X_k)\leq m(F_{\epsilon})+\epsilon<m(F_{\epsilon}),
    \end{equation}
    et donc en particulier \( a<m(F_{\epsilon})\). L'ensemble \( F_{\epsilon}\) est en plus un compact de \( (X,\tau_X)\). En effet \( X_k\) étant fermé de \( (X,\tau_X)\), le lemme~\ref{LemBWSUooCCGvax} nous dit que \( F_{\epsilon}\) est un fermé de \( (X,\tau_X)\). Mais \( X_k\) étant compact, \( F_{\epsilon}\) est un fermé inclus dans un compact, il est donc compact (lemme~\ref{LemnAeACf}).

    Pour tout \( a<m(B)\) nous avons trouvé un compact \( F_{\epsilon}\) inclus dans \( B\) dont la mesure est plus grande que \( a\). Cela prouve la régularité intérieure de la mesure \( m\).

\item[Régularité extérieure]

    Soit un borélien \( B\) de \( X\). Si \( m(B)=\infty\) alors tous les ouverts contenant \( B\) ont mesure infinie et \( m(B)\) en est évidemment le supremum. Nous supposons donc que \( m(B)<\infty\).

    Nous notons \( \tau_k\) la topologie induite de \( X\) sur \( \Int(X_k)\). Nous posons \( B_k=B\cap\Int(X_k)\). L'espace \( \big( \Int(X_k),m \big)\) est un espace mesuré borné et \( B_k\in \Borelien\Big( \Int(X_k) \Big)\). Il existe donc un ouvert \( \Omega_k\) de \( \big( \Int(X_k),\tau_k \big)\) tel que \( B_k\subset \Omega_k\) et
    \begin{equation}
        m(\Omega_k\setminus B_k)\leq \frac{ \epsilon }{ 2^k }.
    \end{equation}
    De plus \( \Int(X_k)\) est un ouvert de \( (X,\tau_X)\), donc en réalité \( \Omega_k\) est un ouvert de \( X\). Nous posons
    \begin{equation}
        \Omega=\bigcup_{k=1}^{\infty}\Omega_k
    \end{equation}
    qui est encore un ouvert de \( (X,\tau_X)\).

    Il est temps de voir que \( \Omega\) vérifie \( m(\Omega\setminus B)\leq \epsilon\). Pour cela,
    \begin{subequations}
        \begin{align}
            \Omega\setminus B=\big( \bigcup_k\Omega_k \big)\cap\big( \bigcup_lB_l \big)^c\\
            &=\big( \bigcup_k\Omega_k \big)\cap\big( \bigcap B_l^c \big)\\
            &\subset\bigcup_k(\Omega_k\cap B_k^c)\\
            &=\bigcup_k(\Omega_k\setminus B_k),
        \end{align}
    \end{subequations}
    ce qui donne au niveau des mesures :
    \begin{equation}
        m(\Omega\setminus B)\leq\sum_{k=1}^{\infty}m(\Omega_k\setminus B_k)\leq\sum_{k=1}^{\infty}\frac{ \epsilon }{ 2^k }=\epsilon.
    \end{equation}
    \end{subproof}
\end{proof}

\begin{remark}      \label{RemooOAGCooRHpjxd}
    Exprimé sur \( \eR^N\), la proposition~\ref{PropNCASooBnbFrc} s'exprime en disant que toute mesure de Borel sur \( \eR^N\) est régulière. Typiquement, l'espace \( X\) dont il est question est un ouvert de \( \eR^N\).
\end{remark}

%---------------------------------------------------------------------------------------------------------------------------
\subsection{Théorème de récurrence}
%---------------------------------------------------------------------------------------------------------------------------

Soit \( X\) un espace mesurable, \( \mu\) une mesure finie sur \( X\) et \( \phi\colon X\to X\) une application mesurable\footnote{Définition \ref{}.} préservant la mesure, c'est-à-dire que pour tout ensemble mesurable \( A\subset X\),
\begin{equation}
    \mu\big( \phi^{-1}(A) \big)=\mu(A).
\end{equation}
Si \( A\subset X\) est un ensemble mesurable, un point \( x\in A\) est dit \defe{récurrent}{récurrent!point d'un système dynamique} par rapport à \( A\) si et seulement si pour tout \( p\in \eN\), il existe \( k\geq p\) tel que \( \phi^k(x)\in A\).

\begin{theorem}[\wikipedia{fr}{Théorème_de_récurrence_de_Poincaré}{Théorème de récurrence de Poincaré}.]     \label{ThoYnLNEL}
    Si \( A\) est mesurable dans \( X\), alors presque tous les points de \( A\) sont récurrents par rapport à \( A\).
\end{theorem}

\begin{proof}
    Soit \( p\in \eN\) et l'ensemble
    \begin{equation}
        U_p=\bigcup_{k=p}^{\infty}\phi^{-k}(A)
    \end{equation}
    des points qui repasseront encore dans \( A\) après \( p\) itérations  de \( \phi\). C'est un ensemble mesurable en tant que union d'ensembles mesurables (pour rappel, les tribus sont stables par union dénombrable, comme demandé à la définition~\ref{DefjRsGSy}), et nous avons donc
    \begin{equation}
        \mu(U_p)\leq \mu(X)<\infty.
    \end{equation}
    De plus \( U_p=\phi^{-p}(U_0)\), donc \( \mu(U_p)=\mu(U_0)\). Vu que \( U_p\subset U_p\), nous avons
    \begin{equation}
        \mu(U_0\setminus U_p)=0.
    \end{equation}
    Étant donné que \( A\subset U_0\) nous avons a fortiori que
    \begin{equation}
        \{ x\in A\tq x\notin U_p \}\subset U_0\setminus U_p,
    \end{equation}
    et donc
    \begin{equation}
        \mu\{ x\in A\tq x\notin U_p \}=0.
    \end{equation}
    Cela signifie exactement que l'ensemble des points \( x\) de \( A\) tels que aucun des \( \phi^k(x)\) avec \( k\geq p\) n'est dans \( A\) est de mesure nulle.
\end{proof}


%+++++++++++++++++++++++++++++++++++++++++++++++++++++++++++++++++++++++++++++++++++++++++++++++++++++++++++++++++++++++++++
\section{Mesurabilité des fonctions à valeurs réelles}
%+++++++++++++++++++++++++++++++++++++++++++++++++++++++++++++++++++++++++++++++++++++++++++++++++++++++++++++++++++++++++++

Nous allons parler de la mesurabilité de fonctions
\begin{equation}
    f\colon (S,\tribF)\to \big( \bar \eR,\Borelien(\bar \eR) \big)
\end{equation}
où \( \bar \eR=\eR\cup\{ \pm\infty \}\).

\begin{normaltext}      \label{normooGAAJooUPCbzG}
Nous convenons que \( 0\times\pm\infty=0\) parce que nous voulons qu'une droite (qui est un rectangle dont une mesure est \( 0\) et l'autre \( \infty\)) soit de mesure nulle dans \( \eR^2\).

Les produits et sommes \( \pm\infty\pm\pm\infty\) et \( \pm\infty\times \pm\infty\) sont ceux que l'on croit. Sauf bien entendu \( +\infty-\infty\) et \( 1/0\) qui ne sont toujours pas définis.
\end{normaltext}

\begin{lemma}       \label{LEMooBLOLooAdNViv}
    L'ensemble \( B\) est un borélien de \( \bar \eR\) si et seulement s'il existe un borélien \( B_0\) de \( \eR\) tel que \( B\) soit \( B_0\) ou \( B_0\cup\{ -\infty \}\) ou \( B_0\cup\{ -\infty \}\) ou \( B_0\cup\{ +\infty,-\infty \}\).
\end{lemma}

\begin{proof}
    Vu que la topologie usuelle sur \( \eR\) est la topologie induite de celle sur \( \bar \eR\), la tribu induite l'est aussi par le théorème~\ref{ThoJDOKooKaaiJh}. Donc si \( B\) est un borélien de \( \bar \eR\), l'ensemble \( B\cap \eR\) est un borélien de \( \eR\).
\end{proof}

\begin{lemma}[\cite{TribuLi}]       \label{LemooCRVJooQosHPq}
    Si \( \mS_0\) est l'ensemble des intervalles du type
    \begin{equation}
        \begin{aligned}[]
        \mathopen] \alpha , \beta \mathclose[,&&\mathopen[ -\infty , \beta \mathclose[,&&\mathopen] \alpha , +\infty \mathclose]
        \end{aligned}
    \end{equation}
    avec \( -\infty<\alpha<\beta<+\infty\) alors \( \sigma(\mS_0)=\Borelien(\bar\eR)\).
\end{lemma}

\begin{proof}
Les intervalles \( \mathopen] \alpha , \beta \mathclose[\) engendrent la topologie de \( \eR\)\footnote{Parce toutes les boules sont des intervalles de ce type et que les boules forment une base de topologie, proposition~\ref{PropNBSooraAFr}.}, donc \( \Borelien(\eR)\subset\sigma(\mS_0)\). De plus le lemme~\ref{LemBWNlKfA} nous autorise à dire que
    \begin{equation}
        \bigcap_{n\geq 1}\mathopen[ n , +\infty \mathclose]=\{ +\infty \}\in\sigma(\mS_0).
    \end{equation}
    Par conséquent tous les ensembles énumérés dans le lemme~\ref{LEMooBLOLooAdNViv} font partie de \( \sigma(\mS_0)\). Cela implique que \( \Borelien(\bar\eR)\subset\sigma(\mS_0)\).

    Pour l'inclusion inverse, \( \sigma(\mS_0)\) est engendré par des parties qui font parie de \( \Borelien(\bar \eR)\), donc \( \sigma(\mS_0)\subset\Borelien(\bar \eR)\).
\end{proof}

%---------------------------------------------------------------------------------------------------------------------------
\subsection{Fonctions à valeurs réelles sur un espace mesurable}
%---------------------------------------------------------------------------------------------------------------------------

\begin{theorem}     \label{THOooWHFLooKYGsOm}
    Soit un espace mesurable \( (S,\tribF)\) et une fonction \( f\colon S\to \bar \eR\). Les faits suivants sont équivalents.
    \begin{enumerate}
        \item\label{ITEMooHAMHooYLqUhVi}
            La fonction \( f\) est mesurable.
        \item\label{ITEMooHAMHooYLqUhVii}
            L'ensemble \( \{ f<a \}\) est dans \( \tribF\) pour tout \( a\in \eR\)
        \item\label{ITEMooHAMHooYLqUhViii}
            L'ensemble \( \{ f\leq a \}\) est dans \( \tribF\) pour tout \( a\in \eR\)
    \end{enumerate}
\end{theorem}

\begin{proof}
    Plusieurs implications à prouver.
    \begin{subproof}
        \item[\ref{ITEMooHAMHooYLqUhVi}\( \Rightarrow\)\ref{ITEMooHAMHooYLqUhVii}]
            Vu que \( f\) est mesurable et que \( \mathopen[ -\infty , a \mathclose[\in\Borelien(\bar\eR)\), nous avons \( f^{-1}\big( \mathopen[ -\infty , a \mathclose[ \big)\in\tribF\).
        \item[\ref{ITEMooHAMHooYLqUhVii}\( \Rightarrow\)\ref{ITEMooHAMHooYLqUhVi}]
            Nous posons \( \tribA=\{ \mathopen[ -\infty , a \mathclose[\tq a\in \eR \}\).

                Nous avons \( \tribA\subset\mS_0\) (le \( \mS_0\) du lemme~\ref{LemooCRVJooQosHPq}). Et de plus,
            \begin{equation}
            \mathopen] \alpha , \beta \mathclose[=\mathopen[ -\infty , \beta \mathclose[\setminus\mathopen[ -\infty , \alpha \mathclose]=\mathopen[ -\infty , \beta \mathclose[\setminus\bigcap_{n\geq 1}\mathopen[ -\infty , \alpha+\frac{1}{ n } \mathclose[.
            \end{equation}
        Donc \( \mathopen] \alpha , \beta \mathclose[\in\sigma(\tribA)\).

            Et aussi :
            \begin{equation}
                \mathopen] \alpha , +\infty \mathclose]=\bar\eR\setminus\mathopen[ -\infty , \alpha+\frac{1}{ n } \mathclose[,
            \end{equation}
        ce qui donne \( \mathopen] \alpha , +\infty \mathclose]\in \sigma(\tribA)\).

        Au final, \( \mS_0\subset\sigma(\tribA)\) et donc \( \sigma(\mS_0)\subset\sigma(\tribA)\). Le lemme~\ref{LemooCRVJooQosHPq} nous dit que \( \sigma(\mS_0)=\Borelien(\bar \eR)\). Nous avons donc bien \( \sigma(\mS_0)=\sigma(\tribA)=\Borelien(\bar\eR)\).

        par ailleurs, nous savons que \( f^{-1}(\tribA)\subset\tribF\) parce que les éléments de \( \tribA\) sont de la forme \( \{ f<a \}\). Cela donne \( \sigma\big( f^{-1}(\tribA) \big)=\tribF\). Mais \( \sigma\big( f^{-1}(\tribA) \big)\) peut aussi s'exprimer par le lemme de transport \ref{LemOQTBooWGYuDU} : \( \sigma\big( f^{-1}(\tribA) \big)=f^{-1}\big( \sigma(\tribA) \big)\). En combinant les deux,
        \begin{equation}
            f^{-1}\big( \sigma(\tribA) \big)=\tribF,
        \end{equation}
        et en remplaçant \( \sigma(\tribA)\) par \( \Borelien(\bar \eR)\) nous avons ce que nous voulions :
        \begin{equation}
            f^{-1}\big( \Borelien(\bar\eR) \big)\in\tribF,
        \end{equation}
        ce qui signifie que \( f\) est mesurable.
        \item[\ref{ITEMooHAMHooYLqUhViii}\( \Rightarrow\)\ref{ITEMooHAMHooYLqUhVii}]
            Nous avons
            \begin{equation}
                \{ f<a \}=\bigcup_{n\geq 1}\{ f\leq a-\frac{1}{ n } \}.
            \end{equation}
            donc cela est une union dénombrable d'éléments de \( \tribF\). Donc \( \{ f<a \}\) est dans \( \tribF\).
        \item[\ref{ITEMooHAMHooYLqUhVi}\( \Rightarrow\)\ref{ITEMooHAMHooYLqUhViii}]
            Nous avons
            \begin{equation}
                \{ f\leq a \}=\{ f<a \}\cup f^{-1}\big( \mathopen[ -\infty , a \mathclose] \big).
            \end{equation}
            Le premier ensemble est dans \( \tribF\) par~\ref{ITEMooHAMHooYLqUhVii}. Ensuite \( \mathopen[ -\infty , a \mathclose]\) est un fermé de \( \bar \eR\) et donc un borélien de \( \bar \eR\). Son image réciproque est donc un élément de \( \tribF\) parce que \( f\) est mesurable. Au final nous avons bien \( \{ f\leq a \}\in\tribF\).
    \end{subproof}
\end{proof}

\begin{lemma}[\cite{NBoIEXO}]   \label{LemFOlheqw}
    Une fonction \( f\colon X\to \eR\) est mesurable si et seulement si \( f^{-1}(I)\) est mesurable pour tout \( I\) de la forme \( \mathopen] a , \infty \mathclose[\).
\end{lemma}

\begin{proof}
    Nous devons prouver que \( f^{-1}(A)\) est mesurable dans \( X\) pour tout borélien \( A\) de \( \eR\). Nous posons
    \begin{equation}
        S=\{ A\subset \eR\tq f^{-1}(A)\text{ est mesurable dans } X \}
    \end{equation}
    et nous prouvons que cela est une tribu. D'abord \( f^{-1}(\eR)=X\), et \( X\) est mesurable, donc \( \eR\in S\). Ensuite si \( A\in S\) alors \( f^{-1}(A^c)=f^{-1}(A)^c\). En tant que complémentaire d'un mesurable de \( X\), l'ensemble \( f^{-1}(A)^c\) est mesurable dans \( X\). Et enfin si \( A_n\in S \) alors \( f^{-1}(\bigcup_nA_n)=\bigcup_nf^{-1}(A_n)\) qui est encore mesurable dans \( X\) en tant qu'union de mesurables.

    Donc \( S\) est une tribu qui contient tous les ensembles de la forme \( \mathopen] a , \infty \mathclose]\). Le lemme~\ref{LemZXnAbtl} conclu que \( S\) contient tous les boréliens de \( \eR\).
\end{proof}

\begin{lemma}[\cite{NBoIEXO}]   \label{LemIGKvbNR}
    Soit \( f_n\colon X\to \eR\) une suite de fonctions mesurables\footnote{Ici \( X\) est un espace mesuré et \( \eR\) est muni des boréliens.}. Alors \( \sup_n f_n\) est mesurable.
\end{lemma}

\begin{proof}
    Nous avons
    \begin{subequations}
        \begin{align}
            (\sup f_n)^{-1}\big( \mathopen] a , \infty \mathclose] \big)&=\{ x\in X\tq (\sup f_n)(x)>a \}\\
            &=\bigcup_n\{ x\in X\tq f_n(x)>a \}\\
            &=\bigcup_nf_n^{-1}\big( \mathopen] a , \infty \mathclose] \big).
        \end{align}
    \end{subequations}
    Étant donné que \( f_n\) est mesurable et que \( \mathopen] a , \infty \mathclose]\) est mesurable, chacun des \( f_n^{-1}\big( \mathopen] a , \infty \mathclose] \big) \) est mesurable dans \( X\). L'ensemble \( (\sup f_n)^{-1}\big( \mathopen] a , \infty \mathclose] \big)\) donc une union dénombrable de parties mesurables. Il est donc mesurable.

    Le lemme~\ref{LemFOlheqw} conclu que \( \sup f_n\) est mesurable.
\end{proof}

\begin{proposition}\label{PropFYPEOIJ}
    Si \( f_n\colon X\to \eR\) est une suite de fonctions mesurables et positives, alors la fonction\footnote{Définition \ref{DEFooYEIUooCAgrxI} pour la série de fonctions.} \( \sum_nf_n\) est mesurable.
\end{proposition}

\begin{proof}
    Nous considérons les fonctions \( s_k(x)=\sum_{n=0}^kf_n(x)\) qui vaut éventuellement \( \infty\) en certains points. Nous avons
    \begin{equation}
        \sum_nf_n(x)=\sup_ks_k(x),
    \end{equation}
    donc le lemme~\ref{LemIGKvbNR} nous donne la mesurabilité de la somme de \( f_n\).
\end{proof}

\begin{definition}      \label{ooUDHFooJjKscR}
    Soit \( (S,\tribF)\) un espace mesurable.
    Une \defe{partition mesurable dénombrable}{partition!dénombrable mesurable} de e \( S\) est une suite  \( (S_n)_{n\geq 1}\) de parties de \( S\) telles que
    \begin{enumerate}
        \item
            \( S_n\in\tribF\) pour tout \( n\),
        \item
            \( S_N\cap S_k=\emptyset\) si \( n\neq k\),
        \item
            \( S=\bigcup_{n\geq 1}S_n\).
    \end{enumerate}
\end{definition}

\begin{lemma}[Lemme de recollement]     \label{LEMooXAPQooPpZUmP}
    Soit \( (S_n)\) une partition mesurable dénombrable de l'espace mesurable $(S,\tribF)$. Soit \( (S',\tribF')\) un autre espace mesurable et des fonctions mesurables
    \begin{equation}
        f_n\colon (S_n,\tribF_{S_n})\to (S',\tribF')
    \end{equation}
    où \( \tribF_{S_n}\) est la tribu induite\footnote{Définition~\ref{DefDHTTooWNoKDP}.}. Alors la fonction
    \begin{equation}
        \begin{aligned}
            f\colon (S,\tribF)&\to (S',\tribF') \\
            x&\mapsto  f_n(x)\text{ si } x\in S_n
        \end{aligned}
    \end{equation}
    est mesurable.
\end{lemma}

\begin{proof}
    Soit \( A'\in\tribF'\); nous devons prouver que \( f^{-1}(A')\in \tribF\). Nous savons que
    \begin{equation}        \label{EqooGKFFooEwTdtg}
        f^{-1}(A')=\bigcup_{n\geq 1}f_n^{-1}(A'),
    \end{equation}
    qui est une union dénombrable d'éléments \( f_n^{-1}(A')\in\tribF_{S_n}\).

    Vu que \( S_n\in \tribF\) nous avons \( \tribF_{S_n}\subset\tribF\) parce qu'un élément de \( \tribF_{S_n}\) est de la forme \( S_n\cap B\) avec \( B\in\tribF\). Du coup pour chaque \( n\) nous avons
    \begin{equation}
        f_n^{-1}(A')\in\tribF_{S_n}\subset \tribF.
    \end{equation}
    Au final l'égalité \eqref{EqooGKFFooEwTdtg} écrit \( f^{-1}(A')\) comme une union d'éléments de \( \tribF\) et est donc un élément de \( \tribF\).
\end{proof}

\begin{proposition}     \label{PROPooODDVooEEmmTX}
    Soit \( (S,\tribF)\) un espace mesurable et des applications mesurables \( f,g\colon S\to \bar \eR\). Alors les fonctions suivantes sont mesurables :
    \begin{enumerate}
        \item
            \( \lambda f\) pour tout \( \lambda\in \eR\)
        \item
            \( f+g\) si elle existe.
        \item
            \( 1/f\) si elle existe.
        \item
            \( fg\).
    \end{enumerate}
\end{proposition}

\begin{proof}
    Commençons par clarifier « si elle existe». La fonction \( f+g\) n'existe pas au point \( x\in S\) si \( f(x)=+\infty\) et \( g(x)=-\infty\). La fonction \( 1/f\) n'existe pas au point \( x\in S\) si \( f(x)=0\). Voir le point~\ref{normooGAAJooUPCbzG}.
    \begin{subproof}
    \item[La partie où \( f+g\) existe est mesurable]
        La partie de \( S\) sur laquelle \( f+g\) existe est
        \begin{equation}
            \{ x\in S\tq \big( f(x),g(x) \big)\neq (+\infty,-\infty),\big( f(x),g(x) \big)\neq (-\infty,+\infty) \}.
        \end{equation}
        Nous avons
        \begin{equation}
            \{  (f,g)=(+\infty,-\infty) \}=\{ f=\infty \}\cap\{ g=-\infty \}
        \end{equation}
        qui est un ensemble mesurable parce que, par exemple,
        \begin{equation}
            \{ +\infty \}=\bigcap_{n\geq 1}\mathopen[ n , +\infty \mathclose].
        \end{equation}
        La cas \( (-\infty,+\infty)\) est identique, et au final la partie de \( S\) sur laquelle \( f+g\) n'existe pas est mesurable. Par complémentarité la partie sur laquelle \( f+g\) existe est également mesurable\footnote{Parfois on a envie de dire que l'affirmation «\( A\) est mesurable» ne passe pas le test de Popper.}.
    \item[Idem pour la partie sur laquelle \( 1/f\) existe]
        Idem.
    \item[Mesurabilité de \( \lambda f\)]
        Si \( \lambda=0\), nous avons une fonction constante dont la mesurabilité est évidente\footnote{Prenez quand même le temps d'y penser.}. Nous supposons \( \lambda>0\). Alors
        \begin{equation}
            \{ \lambda f<a \}=\{ f<a/\lambda \}\in \tribF.
        \end{equation}
        .Pour \( \lambda<0\) nous avons de la même manière
        \begin{equation}
            \{ \lambda f<a \}=\{ f>a/\lambda \}\in \tribF.
        \end{equation}
        Ce dernier point est suffisant pour que \( \lambda f\) soit mesurable par la théorème~\ref{THOooWHFLooKYGsOm}\ref{ITEMooHAMHooYLqUhViii} et par complémentarité.
    \item[Mesurabilité de \( f+g\)]
        Soit \( a\in \eR\); le théorème~\ref{THOooWHFLooKYGsOm} nous demande d'avoir envie de prouver que \(  \{ f+g <a\} \in \tribF \). Nous avons
        \begin{equation}
            f(x)+g(x)<a
        \end{equation}
        si et seulement si
        \begin{equation}
            f(x)<a-g(x)
        \end{equation}
        si et seulement si
        \begin{equation}
            \exists q\in \eQ\tq f(x)<q<a-g(x).
        \end{equation}
        Donc
        \begin{equation}
            \{ f+g<a \}=\bigcup_{q\in \eQ}\Big( \{ f<q \}\cap\{ g<a-r \} \Big),
        \end{equation}
        qui est une union dénombrable d'éléments de \( \tribF\). Donc \( \{ f+g<a \}\in \tribF\) et \( f+g\) est mesurable.

        Note qu'en toute rigueur il faudrait  «\( \cap\text{là où } f+g\text{ est définie}\)» un peu partout, mais cela ne change rien parce que l'intersection de deux parties mesurables est mesurable.

    \item[Mesurabilité de \( 1/f\)]
        Soit \( a\in \eR\). Si \( a>0\) alors
        \begin{equation}
            \{ 1/f<a \}=\{ f<0 \}\cup\{ f>\frac{1}{ a } \}\in\tribF.
        \end{equation}
        et si \( a<0\) alors
        \begin{equation}
            \{ 1/f<a \}=\{ f<0 \}\cap\{ f>\frac{1}{ a } \}\in\tribF.
        \end{equation}
    \item[Mesurabilité de \( fg\)]
        Nous allons la prouver en plusieurs fois.
        \begin{subproof}
        \item[Si \( f\) est mesurable alors \( f^2\) est mesurable]
            Si \( a\leq 0\) alors \( \{ f^2<a \}=\emptyset\). Si \( a>0\) nous avons
            \begin{equation}
                 \{ f^2<a \}=\{ -\sqrt{a}<f<\sqrt{a} \}\in\tribF.
            \end{equation}
        \item[\( f\mtu_A\) est mesurable]
            Soit \( A\in \tribF\), et prouvons que \( f\mtu_A\) est mesurable. Par définition,
            \begin{equation}
                (f\mtu_A)(x)=\begin{cases}
                    f(x)    &   \text{si } x\in A\\
                    0    &    \text{si } x\notin A.
                \end{cases}
            \end{equation}
            Nous posons \begin{equation}
                \begin{aligned}
                    f_1\colon A^c&\to \bar \eR \\
                    x&\mapsto 0
                \end{aligned}
            \end{equation}
            et
            \begin{equation}
                \begin{aligned}
                    f_2\colon A&\to \bar \eR \\
                    x&\mapsto f(x).
                \end{aligned}
            \end{equation}
            Alors nous avons
            \begin{equation}
                (\mtu_Af)(x)=\begin{cases}
                    f_1(x)    &   \text{si } x\in A^c\\
                    f_2(x)    &    \text{si } x\in  A.
                \end{cases}
            \end{equation}
            Les ensembles \( A\) et $A^c$ forment une partition mesurable dénombrable de \( S\). La fonction \( f_1\) est mesurable; pour prouver que \( f_2\) est mesurable, nous l'écrivons \( f_2=f\circ j_A\) où \( j_A\colon A\to S\) est l'injection canonique. L'application
            \begin{equation}
                j_A\colon (A,\tribF_A)\to (S,\tribF)
            \end{equation}
            est mesurables parce que si \( B\in\tribF\) alors \( j_A^{-1}(B)=A\cap B\in\tribF_A\). D'autre part l'application
            \begin{equation}
                f\colon (S,\tribF)\to \big( \bar \eR,\Borelien(\bar \eR) \big)
            \end{equation}
            est mesurable par hypothèse. La composée \( f_2=f\circ j_A\) est alors mesurable par la proposition~\ref{PROPooEFHKooARJBwW}. Le lemme de recollement~\ref{LEMooXAPQooPpZUmP} nous donne alors la mesurabilité de \( f\mtu_A\).

        \item[Le produit \( fg\) est mesurable]
            Nous posons
            \begin{equation}
                F=\{ x\in S\tq | f(x) |<+\infty,| g(x) |<\infty \}.
            \end{equation}
            En tant qu'intersection de deux ensembles mesurables, \( F\) est mesurable. Par la partie précédente, les applications \( f_1=g\mtu_F\) et \( g_1=g\mtu_F\) sont mesurables. L'application \( f_1+g_1\colon S\to \eR\) est encore mesurable. Par conséquent l'application
            \begin{equation}
                f_1g_1=\frac{ 1 }{2}\big( (f_1+g_1)^2-f_1^2-g_1^2 \big)
            \end{equation}
            est mesurable.

            Voyons maintenant ce qui se passe en dehors de \( F\). Nous allons utiliser le lemme de recollement sur la fonction
            \begin{equation}
                (fg)(x)=\begin{cases}
                    (f_1f_2)(x)    &   \text{si } x\in F\\
                    -\infty    &    \text{si } x\in\mU\\
                    0    &    \text{si } x\in \mV\\
                    +\infty    &    \text{si } x\in\mW
                \end{cases}
            \end{equation}
            où \( F,\mU,\mV,\mW\) forment une partition mesurable dénombrable\footnote{Définition~\ref{ooUDHFooJjKscR}.} de \( S\). Pour le sport nous montrons que \( \mU\) est mesurable :
            \begin{subequations}
                \begin{align}
                    \mU&=\big( \{ f=-\infty \}\cap\{ g>0 \} \big)\\
                    &\cup\big( \{ f=+\infty \}\cap\{ g<0 \} \big)\\
                    &\cup\big( \{ g=-\infty \}\cap\{ f>0 \} \big)\\
                    &\cup\big( \{ g=+\infty \}\cap\{ f<0 \}\big).
                \end{align}
            \end{subequations}
        \end{subproof}
    \end{subproof}
\end{proof}

\begin{proposition}     \label{ooABKWooPbfSOZ}
    Si \( f_n\colon S\to \bar \eR\) est une suite de fonctions mesurables, alors les fonctions \( \inf_n f_n\) et \( \sup_nf_n\) sont mesurables.
\end{proposition}

\begin{proof}
    Nous avons les découpages
    \begin{equation}
        \{ \inf_nf_n<a \}=\bigcup_n\{ f_n<a \}\in\tribF
    \end{equation}
    et
    \begin{equation}        \label{EQooNYKVooDOjOXM}
        \{ \sup_nf_n\leq a \}=\bigcap_n\{ f_n\leq a \}\in\tribF.
    \end{equation}
    Le théorème~\ref{THOooWHFLooKYGsOm} permet de conclure.
\end{proof}
Note que pour \eqref{EQooNYKVooDOjOXM} nous ne pouvions pas utiliser les inégalités strictes parce que \( \{ \sup_nf_n<a \}\) n'est pas spécialement égal à \( \bigcap_n\{ f_n<a \}\).

\begin{normaltext}
    La proposition~\ref{ooABKWooPbfSOZ} nous permet de définir les parties positives et négatives de \( f\) par \( f^+=\sup(f,0)\) et \( f^-=\sup(-f,0)\). Ce sont des applications mesurables. Nous avons les décompositions
    \begin{subequations}
        \begin{align}
            f=f^+-f^-\\
            | f |=f^++f^-.
        \end{align}
    \end{subequations}
\end{normaltext}

\begin{corollary}       \label{CORooNXYUooEcvDlP}
    Si \( f\colon S\to \bar \eR\) est mesurable alors les applications \( f^+\), \( f^-\) et \( | f |\) sont mesurables en tant qu'applications \( S\to\bar \eR^+\).
\end{corollary}

\begin{proof}
    Nous faisons la preuve pour \( f^+\). Nous savons que \( f^+\colon S\to \bar \eR\) est mesurable par la proposition~\ref{ooABKWooPbfSOZ}. Nous considérons l'injection canonique \( f\colon \bar \eR^+\to \bar \eR\) et
    \begin{equation}
        \begin{aligned}
            f_1^+\colon S&\to \bar \eR^+ \\
            x&\mapsto f^+(x).
        \end{aligned}
    \end{equation}
    Alors \( f_1^+=j\circ f^+\) est mesurable. Et c'est bien cela que nous voulions.

\end{proof}
Note : \( f^+\) et \( f_1^+\) sont exactement les mêmes fonctions. Elles ne diffèrent que par la tribu que nous considérons sur l'espace d'arrivée. Nous allons à partir de maintenant les noter toutes deux \( f^+\).

\begin{remark}
    L'application \( | f |\) peut être mesurable sans que \( f\) le soit. Soit en effet une partie \( A\notin \tribF\), et posons
    \begin{equation}
        f(x)=\begin{cases}
            1    &   \text{si } x\in A\\
            -1    &    \text{si } x\in A^c.
        \end{cases}
    \end{equation}
    Alors \( f^{-1}(\{ 1 \})=A\) n'est pas mesurable alors que \( | f |(x)=1\) pour tout \( x\).
\end{remark}

Il est temps d'aller relire les définitions~\ref{ooMVZAooVVCOnP}.

\begin{proposition}     \label{PropooMFIBooJzaleK}
    Si les fonctions \( f_n\colon S\to \bar \eR\) sont mesurables alors les fonctions \( \limsup f_n\) et \( \liminf f_n\) sont mesurables.
\end{proposition}

\begin{proof}
    Par le lemme~\ref{ooAQTEooYDBovS} nous écrivons \( \limsup_nf_n(x)=\inf_{n\geq 1}\sup_{k\geq n} f_k(x)\). Pour chaque \( k\) nous considérons la fonction \( g_k=\sup_{n\geq k}f_n\). Par la proposition~\ref{ooABKWooPbfSOZ}, les fonctions \( g_k\) sont mesurables. En utilisant encore la même proposition, \( \inf_{n\geq 1}g_k\) est encore mesurable.
\end{proof}

\begin{proposition}[\cite{ooKKLCooZRxJnn}]      \label{PropooDXBGooSFqrai}
    Si \( f_n\colon S\to \bar \eR\) est une suite de fonctions mesurables telle dont la limite ponctuelle existe, alors la limite est mesurable.
\end{proposition}

\begin{proof}
    Si la limite existe, elle est égale à la limite supérieure par le lemme~\ref{ooIQIKooXWwAmM}. Or la limite supérieure est mesurable par la proposition~\ref{PropooMFIBooJzaleK}.
\end{proof}

%---------------------------------------------------------------------------------------------------------------------------
\subsection{Fonction étagée}
%---------------------------------------------------------------------------------------------------------------------------

\begin{definition}[\cite{ooARRSooBLWdam}]\label{DefBPCxdel}
    Soit \( (S,\tribF)\) un espace mesurable et une fonction \( f\colon S\to \big( \bar\eR,\Borelien(\bar\eR) \big)\). Il serait dommage de confondre les trois concepts suivants.
    \begin{itemize}
        \item
    Une \defe{fonction simple}{simple!fonction} est une fonction dont l'image est constituée d'un nombre fini de valeurs.
\item
    Une \defe{fonction étagée}{étagée!fonction} est une fonction simple qui est elle-même une fonction mesurable.
\item
    Une \defe{fonction en escalier}{escalier} est une fonction étagée dont les valeurs sont constantes sur des intervalles : ce sont donc des fonctions constantes par morceaux.
    \end{itemize}
\end{definition}

Dans les trois cas, la fonction \( f\) peut être écrite comme somme de fonctions caractéristiques :
\begin{equation}
    f(x)=\sum_{j=1}^p\alpha_j\mtu_{A_j}(x)
\end{equation}
où \( A_j=f^{-1}(\alpha_j)\). Ce qui change est la nature des \( A_j\).

\begin{itemize}
    \item Si \( f\) est  simple, les \( A_j\) sont quelconques.
    \item Si \( f\) est étagée, les \( A_i\) peuvent être choisis mesurables parce que \( \{\alpha_i \}\) est un borélien, ce qui fait que \( A_i=f^{-1}(\alpha_i)\) un choix mesurable.
    \item Si \( f\) est en escalier, les \( A_i\) sont des intervalles.
\end{itemize}

\begin{definition}
    La \defe{forme canonique}{forme canonique!fonction simple} d'une fonction simple \( f\) est la suivante. Soit \( \{ \alpha_i \}_{i=1,\ldots, l}\) les valeurs distinctes prises par \( f\) et \( A_i=f^{-1}(\alpha_i)\). La forme canonique de \( f\) est alors
    \begin{equation}
        f=\sum_{i=1}^l\alpha_i\mtu_{A_i}.
    \end{equation}
\end{definition}

\begin{lemma}   \label{LEMooNWLTooCDuRQI}
    Si \( f\) est une fonction simple dont la représentation canonique est
    \begin{equation}
        f=\sum_{i=1}^l\alpha_i\mtu_{A_i},
    \end{equation}
    alors
    \begin{enumerate}
        \item
            les \( A_i\) sont disjoints,
        \item
            l'union fait \( S\) : \( S=\bigcup_iA_i\).
    \end{enumerate}
\end{lemma}

\begin{probleme}
    Le lemme~\ref{LemYFoWqmS} et le théorème~\ref{THOooXHIVooKUddLi} disent la même chose alors que la preuve du théorème~\ref{THOooXHIVooKUddLi} est beaucoup plus compliquée.La démonstration du lemme serait fausse ?

    M'est avis que ce que le théorème donne en plus est la convergence uniforme en cas de fonction bornée. La suite \eqref{EqooXQYIooSSJwtM} ne va pas converger uniformément.
\end{probleme}

\begin{lemma}[Limite croissante de fonctions étagées\cite{MonCerveau}]    \label{LemYFoWqmS}
    Soit \( f\colon (S,\tribF)\to \bar\eR\) une fonction positive mesurable. Il existe une suite \( f_n\colon S\to \eR\) de fonctions étagées positives telles que \( f_n\to f\) ponctuellement et \( f_n \leq f\).
\end{lemma}

\begin{proof}
    Nous considérons \( (q_n)\) une suite parcourant tous les rationnels positifs\footnote{Nous rappelons que \( \eQ\) est dénombrable et dense dans \( \eR\) par la proposition~\ref{PropooUHNZooOUYIkn}.} avec \( q_0=0\) pour être sûr.
    Pour \( n\in \eN\) nous définissons la fonction
    \begin{equation}        \label{EqooXQYIooSSJwtM}
        f_n(x)= \max\{ q_i\tq i\leq n,\, q_i\leq f(x) \}.
    \end{equation}
    L'ensemble sur lequel le maximum est pris n'est pas vide parce que \( q_0=0\). La fonction \( f_n\) est simple parce qu'elle ne prend que \( n\) valeurs différentes. Nous avons aussi par construction que \(  f_n(x)\leq f(x) \). Nous avons aussi pour tout \( x\in S\) que \( f_n(x)\to f(x)\) parce que \( \eQ\) est dense dans \( \eR\).

    En ce qui concerne le fait que \( f_n\) est mesurable, nous notons \( \{ r_0,\ldots, r_{n} \}\) l'ensemble des \( \{ q_0,\ldots, q_n \}\) classés dans l'ordre croissant. Nous posons en plus \( r_{n+1}=+\infty\). Nous avons alors
    \begin{equation}
        f_n^{-1}(r_k)=\{ x\in S\tq f(x)\geq r_k,f(x)<r_{k+1} \}=\{ f\geq r_k \}\cap\{ f<r_{k+1} \}.
    \end{equation}
    En tant qu'intersection de deux ensembles mesurables, le théorème \ref{THOooWHFLooKYGsOm} dit que \( f_n^{-1}(r_k)\) est mesurable.
\end{proof}

\begin{remark}
    Pour avoir \(  f_n <| f |\) nous pouvons poser
    \begin{equation}
        f_n(x)=\begin{cases}
            \max\{ q_i\tq i\leq n,\, q_i\leq f(x) \}    &   \text{si } f(x)\geq 0\\
            \min\{ q_i\tq i\leq n,\, q_i\geq f(x) \}    &    \text{si } f(x)< 0\text{.}
        \end{cases}
    \end{equation}
\end{remark}

\begin{theorem}[Théorème fondamental d'approximation, thème \ref{THEMEooKLVRooEqecQk}\cite{TribuLi,YHRSDGc,ooRCYWooNAeaTA}]\label{THOooXHIVooKUddLi}
    Soit un espace mesuré \( (S,\tribA,\mu)\).
    \begin{enumerate}
        \item
    Soit une fonction mesurable \( f\colon S\to \mathopen[ 0 , +\infty \mathclose]\). Alors il existe une suite croissante de fonctions \( \varphi_n\colon S\to \mathopen[ 0 , +\infty \mathclose[\) étagées positives dont la limite ponctuelle est \( f\).
    \item
        Si de plus \( f\) est bornée, la convergence est uniforme.
    \item
        Idem pour \( f\) à valeurs dans \( \bar \eR\) ou \( \eC\).
    \end{enumerate}
\end{theorem}

\begin{proof}
    Nous découpons l'intervalle \( \mathopen[ 0 , n \mathclose]\) en plusieurs morceaux.
    \begin{equation}
        I_{n,k}=\begin{cases}
            \mathopen[ \frac{ k }{ 2^n } , \frac{ k+1 }{ 2^n } \mathclose[    &   \text{si } 0\leq k\leq n2^n-1\\
                \mathopen[ n , \infty \mathclose]    &    \text{si } k=n2^n.
        \end{cases}
    \end{equation}
    Nous posons \( S_{n,k}=f^{-1}(I_{n,k})\). Ce sont des ensembles mesurables parce que \( f\) est mesurable. Et de plus pour chaque \( n\), la suite \( (S_{n,k})_{k\geq 0}\) est une partition mesurable finie de \( S\). Nous posons
    \begin{equation}
        \varphi_n=\sum_{k=0}^{n2^n}\frac{ k }{ 2^n }\mtu_{S_{n,k}}.
    \end{equation}
    C'est-à-dire que sur chaque \( S_{n,k}\) nous approximons \( f\) par le bas. La fonction \( \varphi_n\) est étagée et positive : \( 0\leq \varphi_n(x)\leq f(x)\) par construction.
    \begin{subproof}
    \item[Croissance]
        Nous allons voir que \( \varphi_n\leq \varphi_{n+1}\). Soit \( k\neq n2^n\). Si \( x\in S_{n,k}\) alors \( \varphi_n(x)=\frac{ k }{ 2^n }\) et nous avons aussi la décomposition
        \begin{equation}
            S_{n,k}=S_{n+1,2k}\cup S_{n+,2k+1}.
        \end{equation}
        Si \( x\in S_{n+1,2k}\) alors \( \varphi_{n+1}(x)=\frac{ 2k }{ 2^{n+1} }=\frac{ k }{ 2^n }=\varphi_n(x)\). Et si \( x\in S_{n+1,2k+1}\) alors
        \begin{equation}
            \varphi_{n+1}(x)=\frac{ 2k+1 }{ 2^{n+1} }=\frac{ k+\frac{ 1 }{2} }{ 2^n }>\varphi_n(x).
        \end{equation}

        Il reste à traiter le cas \( x\in\{ f\geq n \}\). Dans ce cas nous avons \( \varphi_n(x)=n\). Il y a encore deux cas à traiter :
        \begin{equation}
            \{ f\geq n \}=\{ f\in\mathopen[ n , n+1 \mathclose[ \}\cup\{ f\in\mathopen[ n+1 , \infty \mathclose] \}.
        \end{equation}
        Pour plus de simplicité dans les notations, nous notons \( \bar n=n2^n\), c'est-à-dire que \( I_{n,\bar n}\) est le \( I_{n,k}\) avec le \( k\) le plus grand possible. Nous avons
        \begin{equation}
            I_{n,\bar n}=\mathopen[ n , n+1 \mathclose[\cup\mathopen[ n+1 , \infty \mathclose].
        \end{equation}
        Le premier élément se décompose en \( I_{n+1,k}\) avec \( k<n+1\) (nous préciserons plus tard exactement les valeurs de \( k\)) tandis que le second est \( \mathopen[ n+1 , \infty \mathclose]=I_{n+1,\overline{ n+1 }}\).

        Pour \( x\in S_{n+1,\overline{ n+1 }}\) nous avons
        \begin{equation}
            \varphi_{n+1}(x)=\frac{ (n+1)2^{n+1} }{ 2^{n+1} }=n+1>\varphi_n(x).
        \end{equation}
        Si au contraire \( f(x)\in\mathopen[ n , n+1 \mathclose[ \) nous devons précisément voir quels sont les \( k\) qui font en sorte que \( I_{n+1,k}\) recouvre \( \mathopen[ n , n+1 \mathclose[\). Le plus petit \( k\) est donné par \( \frac{ k }{ 2^{n+1} }=n\), c'est-à-dire \( k=n2^{n+1}\) et le plus grand \( k\) est donné par \( \frac{ k }{ 2^{n+1} }<n+1\), c'est-à-dire \( k=2^{n+1}(n+1)-1\). Donc si \( f(x)\in\mathopen[ n , n+1 \mathclose[\) alors \( x\in S_{n+1,k}\) avec
            \begin{equation}
                n2^{n+1}\leq k\leq (n+1)2^{n+1}-1
            \end{equation}
            Dans ce cas
            \begin{equation}
                \varphi_{n+1}(x)=\frac{ k }{ 2^{n+1} }\geq \frac{ n2^{n+1} }{ 2^{n+1} }=n=\varphi_n(x).
            \end{equation}
    \item[Convergence ponctuelle]
        Si \( f(x)<\infty\) alors il existe\footnote{Le vrai snob citera ici le lemme~\ref{LemooMWOUooVWgaEi}.} \( n_0\in\eN\) tel que \( f(x)<n_0\). Pour \( bn\geq n_0\) nous avons \( f(x)<n\) et donc \( \varphi_n(x)\) se calcule à partir d'un des intervalles de taille \( 1/2^n\) :
        \begin{equation}
            \varphi_n(x)=\frac{ k }{ 2^n }\leq f(x)<\frac{ k+1 }{ 2^n }.
        \end{equation}
        Donc
        \begin{equation}
            | \varphi_n(x)-f(x) |\leq \frac{1}{ 2^n },
        \end{equation}
        ce qui signifie que \( \lim_{n\to \infty} \varphi_n(x)=f(x)\).

        Si \( f(x)=+\infty\) alors \( f(x)>n\) pour tout \( n\). Et alors \( \varphi_n(x)=n\) pour tout \( n\), ce qui donne bien \( \varphi_n(x)\to \infty\).
    \item[Convergence uniforme]
        Soit \( f\) bornée : \( 0\leq f(x)<M\) pour tout \( x\in S\). Soit aussi \( \epsilon>0\). Nous prenons \( n_0>M\) tel que \( \frac{1}{ 2^{n_0}<\epsilon }\). Alors pour tout \( n\geq n_0\) nous avons
        \begin{equation}
            0\leq f(x)-\varphi_n(x)\leq \frac{1}{ 2^n }\leq \frac{1}{ 2^{n_0} }\leq \epsilon.
        \end{equation}
        Note qu'il n'y a pas de valeurs absolues parce que nous savons déjà que la limite est croissante.
    \end{subproof}
\end{proof}

%---------------------------------------------------------------------------------------------------------------------------
\subsection{Fonctions réelles à variables réelles}
%---------------------------------------------------------------------------------------------------------------------------

Nous nous particularisons à présent au cas de fonctions
\begin{equation}
    f\colon \big( \eR,\Borelien(\eR) \big)\to \big( \bar\eR,\Borelien(\bar \eR) \big).
\end{equation}

\begin{normaltext}[\cite{MonCerveau}]      \label{NORMooNFOMooYnaflN}
    Anticipons un peu pour expliquer pourquoi ce que nous allons faire maintenant est suffisant pour ce que nous avons en tête\footnote{Pour rappel, nous avons en tête de définir une théorie de la mesure afin d'y définir des intégrales. En particulier nous allons étudier l'inttégrale de Lebesgue et en ce qui concerne \( \eR^n\), nous aurons la tribu de Lebesgue.}. Toutes les fonctions mesurables 
    \begin{equation}
        f\colon \big( \eR,\Borelien(\eR) \big)\to \big( \bar\eR,\Borelien(\bar \eR) \big)
    \end{equation}
    seront a fortiori mesurables au sens de 
    \begin{equation}
        f\colon \big( \eR,\Lebesgue(\eR) \big)\to \big( \bar\eR,\Borelien(\bar \eR) \big)
    \end{equation}
    où \( \Lebesgue(\eR)\) est la tribu de Lebesgue sur \( \eR\), c'est-à-dire la tribu complétée de celle des boréliens (définition~\ref{DefooYZSQooSOcyYN}).
\end{normaltext}

\begin{normaltext}      
    Nous allons maintenant donner quelques conditions pour que des fonctions soient mesurables au sens de la tribu des boréliens sur l'espace d'arrivée et de départ. Ces résultats seront donc immédiatement applicables à la théorie de l'intégration où nous considérons la tribu de Lebesgue sur l'espace de départ.

    Autrement dit, les résultats présentés ici sont un peu plus forts que ce dont nous avons réellement besoin \ldots ou alors ce sont les hypothèses que nous allons nous mettre en théorie de l'intégration qui seront un peu plus fortes que nécessaires. C'est une question de point de vue.
\end{normaltext}

\begin{corollary}       \label{CorooJYDVooCrXVun}
    Si \( I\) est un intervalle de \( \eR\), alors toute application monotone \( f\colon I\to \eR\) est borélienne.
\end{corollary}

\begin{proof}
    Vu que \( f\) est monotone, l'ensemble \( \{ f<a \}\) est un intervalle. Or tous les intervalles sont boréliens, donc \( f\) est mesurable par le théorème~\ref{THOooWHFLooKYGsOm}.
\end{proof}

\begin{definition}
Si \( I\) est un intervalle de \( \eR\), une fonction \( f\colon I\to \eR\) a une propriété (monotone, mesurable, continue, etc.) \defe{par morceaux}{morceau!fonction continue ou monotone}\index{fonction!monotone!par morceaux}\index{fonction!continue!par morceaux} s'il existe une suite strictement croissante de points \( (x_I)_{i\in \eZ}\) dans \( I\) telle que \( f\) ait la propriété sur chacun des ouverts \( \mathopen] x_j ,x_{j+1} \mathclose[.\).
\end{definition}
Dans cette définition, les points sont numérotés par \( \eZ\) et non par \( \eN\) parce que nous nous laissons la liberté d'avoir une infinité de points de chacun des deux côtés.

\begin{proposition}     \label{PropooLNBHooBHAWiD}
    Soit \( I\) un intervalle de \( \eR\) et une fonction \( f\colon I\to \eR\). Si \( f\) est continue ou monotone par morceaux sur \( I\) alors elle y est borélienne.
\end{proposition}

\begin{proof}
L'ensemble \( \{  \mathopen] x_j , x_{j+1} \mathclose[  \}_{j\in \eZ}\cup\{ x_i \}_{i\in \eZ}\) forme une partition mesurable dénombrable de \( I\) (les singletons sont des boréliens). À une belle redéfinition près de la numérotation (deux fois \( \eZ\) va dans \( \eN\)), nous les appelons \( (I_n)_{n\in \eN}\), et nous définissons les fonctions \( f_n\) comme étant les restrictions de \( f\) aux intervalles \( I_k\).

    Toute fonction sur un singleton est mesurable. Toute fonction continue sur un ouvert est mesurable (théorème~\ref{ThoJDOKooKaaiJh}). Toute fonction monotone sur un ouvert est mesurable (corolaire~\ref{CorooJYDVooCrXVun}).

    Le lemme de recollement~\ref{LEMooXAPQooPpZUmP} donne alors la mesurabilité de \( f\).
\end{proof}

\begin{normaltext}
    Toutes les fonctions que nous pouvons écrire explicitement sont mesurables \ldots en tout cas toutes celles que l'on trouve en pratique. En effet nous avons déjà toutes les fonctions continues par morceaux via la proposition~\ref{PropooLNBHooBHAWiD} et ensuite toutes les limites par la proposition~\ref{PropooDXBGooSFqrai}. Cela donne les séries, les dérivées, les primitives, etc.
\end{normaltext}

\input{135_mesure}
% This is part of Mes notes de mathématique
% Copyright (c) 2011-2020
%   Laurent Claessens, Carlotta Donadello
% See the file fdl-1.3.txt for copying conditions.

%+++++++++++++++++++++++++++++++++++++++++++++++++++++++++++++++++++++++++++++++++++++++++++++++++++++++++++++++++++++++++++
\section{Mesure de Lebesgue sur \texorpdfstring{$ \eR$}{R}}
%+++++++++++++++++++++++++++++++++++++++++++++++++++++++++++++++++++++++++++++++++++++++++++++++++++++++++++++++++++++++++++
\label{SecZTFooXlkwk}

Nous notons \( \mS\) l'ensemble des intervalles\footnote{Définition~\ref{DefEYAooMYYTz}.} de \( \eR\).

\begin{proposition}
    L'ensemble réunions finies d'éléments de \( \mS\) est une algèbre de parties de \( \eR\) que nous allons noter \( \tribA_{\mS}\).
\end{proposition}

\begin{proof}

    Nous devons vérifier la définition~\ref{DefTCUoogGDud}. Les ensembles \( \eR\) et \( \emptyset\) sont des intervalles et font donc partie de \( \tribA_{\mS}\).

    Si \( A\in\tribA_{\mS}\) se décompose en union d'intervalles de la forme \( (a_k,b_k)\) avec \( k=1,\ldots, n\) (ici nous mettons des parenthèses au lieu de crochets parce qu'à priori nous ne savons pas). Alors
    \begin{equation}
        A^c=\bigcup_{k=0}^{k}(b_k,a_{k+1})
    \end{equation}
    où nous avons posé \( b_0=-\infty\) et \( a_{n+1}=+\infty\). Ici encore les parenthèses sont soit fermées soit ouvertes en fonction de ce qu'étaient celles dans la décomposition de \( A\). Quoi qu'il en soit, cette décomposition de \( A^c\) montre que \( A^c\in\tribA_{\mS}\).

    Enfin si \( A,B\in\tribA_{\mS}\) alors \( A\cup B\in\tribA_{\mS}\).
\end{proof}

\begin{lemma}
    Tout élément de \( \tribA_{\mS}\) admet une décomposition minimale unique en réunion finie d'intervalles. Cette décomposition est formée d'intervalles deux à deux disjoints.
\end{lemma}

\begin{proof}
    Nous allons montrer que si \( A\in\tribA_{\mS}\), alors la décomposition minimale consiste en les composantes connexes\footnote{Définition \ref{DEFooFHXNooJGUPPn}.} de \( A\). Pour cela nous rappelons que la proposition~\ref{PropInterssiConn} dit qu'une partie de \( \eR\) est connexe si et seulement si elle est un intervalle. D'abord cela nous dit immédiatement que les composantes connexes de \( A\) forment une décomposition de \( A\) en intervalles. Nous devons prouver qu'elle est minimale.

    Soit \( \{ C_k \}_{k=1,\ldots, n}\) les composantes connexes de \( A\). Aucun connexe de \( \eR\) contenu dans \( A\) ne peut intersecter plus d'un des \( C_k\), et par conséquent nous ne pouvons pas décomposer \( A\) en moins de \( n\) intervalles.

    Pour l'unicité, soit \( \{ I_k \}_{k=1,\ldots, n}\) un ensemble de \( n\) intervalles tels que \( \bigcup_{k=1}^nI_k=A\). Chacun des \( I_k\) intersecte un et un seul des \( C_k\). En effet si \( x\in I_k\cap C_i\) et \( y\in I_k\cap C_j\), alors \( \mathopen[ x , y \mathclose]\subset I_k\) parce que \( I_k\) est un intervalle. Mais \( C_i\) étant le plus grand connexe contenant \( x\), \( \mathopen[ x , y \mathclose]\subset C_i\) et de la même façon, \( \mathopen[ x , y \mathclose]\subset C_j\). Par conséquent \( C_i\) et \( C_j\) sont tous deux la composante connexe de \( x\) et \( y\). Nous en déduisons que \( C_i=C_j\), c'est-à-dire \( i=j\).

    Par ailleurs nous avons \( I_k\cap I_l=\emptyset\) dès que \( k\neq l\) parce que sinon l'ensemble \( I_k\cap I_l\) serait connexe et la décomposition des \( \{ I_k \}_{k=1,\ldots, n} \) ne serait pas minimale : en remplaçant \( I_k\) et \( I_l\) par \( I_k\cup I_l\) on aurait eu une décomposition contenant moins d'éléments. Donc à renumérotation près nous pouvons supposer que \( I_k\) intersecte \( C_l\) si et seulement si \( k=l\).

    Dans ce cas nous devons avoir \( I_k=C_k\), sinon les éléments de \( C_k\setminus I_k\) ne seraient pas dans \( \bigcup_{i=1}^nI_i\).
\end{proof}

\begin{definition}[longueur d'intervalle\cite{MesureLebesgueLi}]
    Si \( I\) est un intervalle d'extrémités \( a\) et \( b\) avec \( -\infty\leq a\leq b\leq +\infty\) alors nous définissons la \defe{longueur}{longueur!d'un intervalle}\index{intervalle!longueur} de \( I\) par
    \begin{equation}
        \ell(I)=\begin{cases}
            b-a    &   \text{si } -\infty<a\leq b< +\infty\\
            \infty    &    \text{si } a\text{ ou } b\text{ est infini}
        \end{cases}
    \end{equation}
    Si \( A\in\tribA_{\mS}\) et si sa décomposition minimale est \( A=\bigcup_{k=1}^nI_k\), alors on définit
    \begin{equation}
        \ell(A)=\sum_{k=1}^n\ell(I_k).
    \end{equation}
\end{definition}

Le lemme suivant nous indique que nous pouvons calculer la longueur d'un élément de \( \tribA_{\mS}\) sans savoir la décomposition minimale, pourvu que l'on connaisse une décomposition disjointe.
\begin{lemma}[\cite{MesureLebesgueLi}]\label{LemIUQooEzHun}
    Si
    \begin{equation}
        B=\bigcup_{r=1}^pJ_r
    \end{equation}
    est une décomposition de \( B\in\tribA_{\mS}\) en intervalles deux à deux disjoints alors
    \begin{equation}
        \ell(B)=\sum_{r=1}^p\ell(J_r).
    \end{equation}
\end{lemma}

\begin{proof}
    Nous prouvons dans un premier temps le résultat dans le cas où \( B=I\) est un intervalle. Soit \( I\) un intervalle et une décomposition en intervalles disjoints \( I=\bigcup_{r=1}^pJ_r\). Nous montrons qu'alors \( \ell(I)=\sum_{r=1}^p\ell(J_r)\). Nous verrons ensuite comment passer au cas où \( B\) est un élément générique de \( \tribA_{\mS}\).
    \begin{subproof}
    \item[Si \( B=I\) est un intervalle infini]

        Si \( I\) est infini alors un des \( J_r\) soit l'être et donc \( \sum_{r=1}^p\ell(J_r)=\infty=\ell(I)\).
    \item[Si \( B=I\) est un intervalle ininfini]

    Pour chaque \( r=1,\ldots, p\) nous notons \( a_r\) et \( b_r\) les extrémités de \( J_r\). Vu que les \( J_r\) sont connexes et disjoints, si \( a_k\leq a_l\) alors \( b_k\leq a_l\), sinon l'ensemble (non vide) \( \mathopen] a_l , b_k \mathclose[ \) serait dans l'intersection \( I_k\cap I_l\) qui, elle, est vide. Plus généralement, si \( x\in J_k\) et \( y\in J_l\) avec \( x<y\) alors pour tout \( x'\in J_k\) et tout \( y'\in J_l\) nous avons \( x'<y'\). Vu qu'il y a un nombre fini d'ensembles \( J_r\), nous pouvons les classer dans l'ordre croissant :
        \begin{equation}
            a_1\leq b_1\leq a_2\leq b_2\leq \ldots\leq b_{p-1}\leq a_p\leq b_p.
        \end{equation}
        Vu que les \( J_r\) sont disjoints et que leur union est connexe nous avons en réalité
        \begin{equation}
            a=a_1\leq b_1=a_2\leq b_2=a_3\leq\ldots\leq b_{p-1}= a_p\leq b_p,
        \end{equation}
        donc une somme télescopique donne
        \begin{equation}
            \ell(I)=b-a=\sum_{r=1}^p(b_r-a_r)=\sum_{r=1}^p\ell(J_r).
        \end{equation}

    \item[Si \( B\) n'est pas un intervalle]
        Soit \( \{ I_k \}_{k=1,\ldots, n}\) la décomposition minimale de \( B\). Alors
        \begin{equation}
            \spadesuit=\ell(B)=\sum_{k=1}^n\ell(I_k)=\sum_{k=1}^n\ell\big( \bigcup_{r=1}^p(I_k\cap J_r) \big).
        \end{equation}
        Mais \( I_k\) est un intervalle et s'écrit comme union disjointe \( I_k=\bigcup_{r=1}^p(I_k\cap J_r)\), donc par la première partie
        \begin{equation}
            \spadesuit=\sum_{k=1}^n\sum_{r=1}^p\ell(I_k\cap J_r)=\sum_{r=1}^p\sum_{k=1}^n\ell(I_k\cap J_r).
        \end{equation}
        Ici \( J_r\) est un intervalle qui se décompose en \( J_r=\bigcup_{k=1}^n(I_k\cap J_r)\), donc nous pouvons encore utiliser la première partie :
        \begin{equation}
            \spadesuit=\sum_{r=1}^p\ell(J_r),
        \end{equation}
        ce qu'il fallait.
    \end{subproof}
\end{proof}

\begin{lemma}   \label{LemPIOooRLkbo}
    Si \( A,B\in\tribA_{\mS}\) avec \( A\subset B\) alors \( \ell(A)\leq \ell(B)\).
\end{lemma}

\begin{proof}
    Nous avons évidemment \( B=A\cup B\setminus A\). Notons que \( B\setminus A\in\tribA_{\mS}\) par le lemme~\ref{LemBFKootqXKl}. Si \( \{ I_k \}\) est une décomposition disjointe de \( A\) et \( \{ J_i \}\) une de \( B\setminus A\) alors \( \{ I_k \}\cup\{ J_i \}\) est une décomposition disjointe de \( A\cup B\setminus A\) et le lemme~\ref{LemIUQooEzHun} nous dit que
    \begin{equation}
        \ell(B)=\ell(A\cup B\setminus A)=\ell(A)+\ell(B\setminus A).
    \end{equation}
    Par conséquent \( \ell(B)\geq \ell(A)\).
\end{proof}

\begin{lemma}   \label{LemUMVooZJgMu}
    Si \( I\) est un intervalle et s'il se décompose en
    \begin{equation}
        I=\bigcup_{n\in \eN}I_n
    \end{equation}
    où les \( I_n\) sont des intervalles disjoints, alors
    \begin{equation}
        \ell(I)=\sum_{n=1}^{\infty}\ell(I_n).
    \end{equation}
\end{lemma}

\begin{proof}
    Nous allons encore diviser la preuve en deux parties suivant que \( I\) soit de longueur finie ou pas.
    \begin{subproof}

        \item[Si \( I\) est de longueur finie]

            Soient \( a\) et \( b\) les extrémités de \( I\) : \( -\infty<a\leq b< +\infty\). Pour tout \( N\geq 1\) nous avons
            \begin{equation}
                \sum_{n=1}^N\ell(I_n)=\ell\big( \bigcup_{n=1}^nI_n \big)\leq \ell(I).
            \end{equation}
            La première égalité est le lemme dans le cas d'une union finie~\ref{LemIUQooEzHun}. L'inégalité est le lemme~\ref{LemPIOooRLkbo}. Cela étant vrai pour tout $N$, à la limite \( N\to\infty\) nous conservons l'inégalité :
            \begin{equation}
                \sum_{n=1}^{\infty}\ell(I_n)\leq \ell(I).
            \end{equation}
            Nous devons encore voir l'inégalité inverse. Pour cela nous supposons que \( a<b\). Sinon \( \ell(I)=0\) et tous les \( I_n\) doivent être vide sauf un qui contiendra seulement \( \{ a \}\) (si \( I\) le contient).

            Soit \( \epsilon>0\) avec \( \epsilon<b-a\) et l'intervalle
            \begin{equation}
                \mathopen[ a+\frac{ \epsilon }{ 4 } , b-\frac{ \epsilon }{ 4 } \mathclose]=\mathopen[ a' , b' \mathclose]\subset I.
            \end{equation}
            Si les \( a_n\) et le \( b_n\) sont le extrémités des \( I_n\) alors
            \begin{equation}
                \mathopen[ a' , b' \mathclose]\subset I=\bigcup_{n\geq 1}I_n\subset\bigcup_{n\geq 1}\mathopen] a_n-\frac{ \epsilon }{ 2^{n+2} } , b_n+\frac{ \epsilon }{ 2^{n+2} } \mathclose[=\bigcup_{n\geq 1}\mathopen] a'_n , b'_n \mathclose[
            \end{equation}
            où nous avons posé \( a'_n=a_n-\epsilon/2^{n+2}\) et \( b'_n=b_n+\epsilon/2^{n+2}\). Nous avons donc recouvert le compact\footnote{Lemme~\ref{LemOACGWxV}.} \( \mathopen[ a' , b' \mathclose]\) par des ouverts. Nous pouvons donc en extraire un sous-recouvrement fini (c'est la définition de la compacité), c'est-à-dire une partie finie \( F\) de \( \eN\) telle que
            \begin{equation}
                \mathopen[ a' , b' \mathclose]\subset \bigcup_{n\in F}\mathopen] a'_n , b'_n \mathclose[.
            \end{equation}
            Le lemme~\ref{LemPIOooRLkbo} nous dit alors que
            \begin{equation}
                \heartsuit=b'-a'\leq \ell\big( \bigcup_{n\in F}\mathopen] a'_n , b'_n \mathclose[ \big)\leq \sum_{n\in F}(b'_n-a'_n).
            \end{equation}
            La seconde inégalité se prouve en recopiant\footnote{Nous ne pouvons pas invoquer directement le lemme~\ref{LemZQUooMdCpq} parce que nous n'avons pas encore prouvé que $\ell$ était une mesure sur $ (\eR,\tribA_{\mS})$.} la preuve de~\ref{LemZQUooMdCpq}. Nous continuons le calcul :
            \begin{equation}
                \heartsuit\leq\sum_{n\in F}(b_n-a_n)+\sum_{n\in F}\frac{ \epsilon }{ 2^{n+1} }\leq \sum_{n\in F}(b_n-a_n)+\frac{ \epsilon }{2}.
            \end{equation}
            Mais \( b'-a'=(b-a)-\frac{ \epsilon }{2}\), donc
            \begin{equation}
                b-a-\frac{ \epsilon }{2}\leq \sum_{n\in F}(b_n-a_n)+\frac{ \epsilon }{2}.
            \end{equation}
            D'où nous déduisons que
            \begin{equation}
                \ell(I)=b-a\leq \sum_{n\in F}(b_n-a_n)+\epsilon\leq \sum_{n\in \eN}(b_n-a_n)+\epsilon=\sum_{n\in \eN}\ell(I_n)+\epsilon.
            \end{equation}
            Cela étant valable pour tout \( \epsilon\) nous déduisons que
            \begin{equation}
                \ell(I)\leq\sum_{n\in \eN}\ell(I_n).
            \end{equation}

        \item[Si \( I\) est de longueur infinie]

        Étant donné que \( I\) est un intervalle de longueur infinie, il doit au moins contenir un ensemble du type \( \mathopen] -\infty , a \mathclose]\) ou \( \mathopen[ a , +\infty [\); donc  pour tout \( M>0\), il existe \( N\geq 1\) tel que
            \begin{equation}
                \ell\big( I\cap\mathopen[ -N , N \mathclose] \big)\geq M.
            \end{equation}
            Mais \( I\cap\mathopen[ -N , N \mathclose]\) est un intervalle et
            \begin{equation}
                I\cap\mathopen[ -N , N \mathclose]=\bigcup_{n\in \eN}I_n\cap\mathopen[ -N , N \mathclose]
            \end{equation}
            qui est une union disjointe. Par conséquent,
            \begin{equation}
                M\leq \ell\big( I\cap\mathopen[ -N , N \mathclose] \big)=\sum_n\ell\big( I_n\cap\mathopen[ -N , N \mathclose] \big)\leq\sum_n\ell(I_n).
            \end{equation}
            Cela étant vrai pour tout \( M>0\), nous concluons que
            \begin{equation}
                \sum_{n\in \eN}\ell(I_n)=\infty.
            \end{equation}
    \end{subproof}
\end{proof}

\begin{remark}
    Pour la preuve de~\ref{LemUMVooZJgMu} nous ne pouvons pas classer les \( I_n\) en ordre croissant comme nous l'avons fait dans la preuve de~\ref{LemIUQooEzHun}. En effet si \( I=\mathopen[ 0 , 1 \mathclose]\) et que nous recouvrons \( \mathopen[ 0 , \frac{ 1 }{2} [\) et \( \mathopen] \frac{ 1 }{2} , 1 \mathclose]\) par une infinité d'intervalles chacun, nous ne pouvons plus les classer par ordre croissant.
\end{remark}

\begin{proposition}[\cite{MesureLebesgueLi}]     \label{PropULFoodgXrR}
    La fonction \( \ell\) ainsi définie est une mesure \( \sigma\)-finie sur l'algèbre de parties \( \tribA_{\mS}\).
\end{proposition}

\begin{proof}
    Le fait que \( \ell\) soit \( \sigma\)-finie provient par exemple du fait que \( \ell\big( \mathopen] -n , n \mathclose[ \big)=2n\) tandis que \( \bigcup_n\mathopen] -n , n \mathclose[=\eR\).

        Nous devons à présent prouver que \( \ell\) est additive. Soient \( (A_i)_{i\in \eN}\) des éléments disjoints de \( \tribA_{\mS}\), avec leurs décomposition minimales
            \begin{equation}
                A_i=\bigcup_{k=1}^nI^{(i)}_k.
            \end{equation}
            Pour chaque \( i\in \eN\), le lemme~\ref{LemUMVooZJgMu} nous indique que
            \begin{equation}
                \ell(A_i)=\sum_{k\in \eN}\ell(I^{(i)}_k).
            \end{equation}
            L'ensemble \( \eN\times \eN\) est dénombrable et nous pouvons considérer la décomposition
            \begin{equation}
                \bigcup_{i\in \eN}A_i=\bigcup_{(i,k)\in \eN\times \eN}I^{(i)}_k.
            \end{equation}
            Cette décomposition n'est pas spécialement minimale\footnote{$ A_1$ pourrait contenir $ \mathopen[ 0 , 1 \mathclose]$ et $ A_2$ contenir $ \mathopen] 1 , 2 \mathclose]$.} mais elle est disjointe.
            Le lemme~\ref{LemUMVooZJgMu} donne
            \begin{equation}
                \ell(\bigcup_i A_i)=\sum_{(i,k)\in \eN\times \eN}\ell(I_k^{(i)})=\sum_{i\in \eN}\left( \sum_{k\in \eN}\ell(I^{(i)}_k)\right)=\sum_{i\in \eN}\ell(A_i).
            \end{equation}
            La décomposition de la somme sur \( \eN^2\) en deux sommes sur \( \eN\) est faite en vertu de la proposition~\ref{PropVQCooYiWTs}.
\end{proof}

%---------------------------------------------------------------------------------------------------------------------------
\subsection{Mesure et tribu de Lebesgue}
%---------------------------------------------------------------------------------------------------------------------------

\begin{theorem} \label{ThoDESooEyDOe}
    Il existe une unique mesure \( \lambda\) sur \( \big( \eR,\Borelien(\eR) \big)\) telle que
    \begin{equation}
        \lambda\big( \mathopen] a , b \mathclose[ \big)=b-a
    \end{equation}
    pour tout \( a\leq b\) dans \( \eR\).
\end{theorem}

\begin{proof}

    L'existence provient du théorème de prolongement de Hahn~\ref{ThoLCQoojiFfZ} : la mesure \( \ell\) sur \( (\tribA_{\mS})\) se prolonge à \( \sigma(\tribA_{\mS})=\Borelien(\eR)\).

    Nous ne pouvons pas prouver l'unicité en invoquant la partie unicité de Hahn (c'est tentant parce que \( \ell\) est \( \sigma\)-finie) parce que dans ce théorème nous ne fixons la valeur de \( \lambda\) que sur une toute petite partie de \( \tribA_{\mS}\). Nous allons cependant voir que cette petite partie suffit à garantir l'unicité.

    La classe
    \begin{equation}
        \tribD=\{ \mathopen] a , b \mathclose[\tq -\infty<a\leq b< +\infty \}
    \end{equation}
    est stable par intersection finie et engendre la tribu borélienne. En effet \( \tribD\) contient toutes les boules et donc une base dénombrable de la topologie de \( \eR\) (proposition~\ref{PropNBSooraAFr}). Donc tous les ouverts de \( \eR\) sont dans \( \sigma(\tribD)\) et \( \sigma(\tribD)=\Borelien(\eR)\). Nous pouvons donc dire grâce au théorème~\ref{ThoJDYlsXu} qu'il y a unicité de la mesure sur \( \Borelien(\eR)\) lorsque les valeurs sur \( \tribD\) sont fixées.
\end{proof}

\begin{definition}      \label{DefooYZSQooSOcyYN}
    La mesure de l'espace mesuré \( \big( \eR,\Borelien(\eR),\lambda \big)\) donné par le théorème~\ref{ThoDESooEyDOe} est la \defe{mesure de Lebesgue}{mesure!de Lebesgue} sur \( \big( \eR,\Borelien(\eR) \big)\).

    Nous définissons aussi la \defe{tribu de Lebesgue}{tribu!de Lebesgue} par la proposition~\ref{PropIIHooAIbfj} : \( \big( \eR,\Lebesgue(\eR),\lambda \big)\) est l'espace mesuré complété de \( \big( \eR,\Borelien(\eR), \lambda \big)\).
\end{definition}


\begin{remark}
    Il n'est pas évident que la tribu de Lebesgue soit plus grande que celle des boréliens, ni que la tribu des parties soit plus grande que celle de Lebesgue. Nous mentionnons cependant les faits suivants.
    %TODO : donner des exemples
    \begin{enumerate}
        \item
            Il existe des ensembles mesurables non-boréliens, et cela ne nécessite pas l'axiome du choix. Un argument classique de cardinalité est donné dans \cite{SFYoobgQUp}. La construction la plus explicite que j'aie trouvée est dans \cite{XSHoosgoQa}, mais ça a l'air de demander des connaissances précises sur les ordinaux.
        \item
            Vu que l'ensemble de Cantor \( C\) est mesurable de mesure nulle (proposition~\ref{PropBEWooXZdKN}), tout sous-ensemble de Cantor est mesurable de mesure nulle parce que la tribu de Lebesgue est complète par définition. Le cardinal de \( \partP(C)\) est strictement supérieur à la puissance du continu, alors que le cardinal de l'ensemble des boréliens est au plus égal à la puissance du continu. Donc il existe des non boréliens contenus dans Cantor; de tels non boréliens sont alors mesurables au sens de Lebesgue.

        \item
            Si nous admettons l'axiome du choix alors il existe des ensembles non mesurables au sens de Lebesgue. Nous en verrons un dans l'exemple~\ref{EXooCZCFooRPgKjj}.
    \end{enumerate}
\end{remark}

\begin{example}[Un ouvert contenant tous les rationnels et de mesure arbitrairement petite]
    Il est possible de construire un ouvert de $\eR$ contenant \( \eQ\) et de mesure de Lebesgue plus petite que \( \epsilon\). Pour cela si \( (q_i)\) est une énumération des rationnels, il suffit de prendre
    \begin{equation}
        \mO=\bigcup_{n=1}^{\infty}B(q_n,\frac{ \epsilon }{ 2^{n+1} }).
    \end{equation}
    Cela est un ouvert comme union d'ouverts, ça contient tous les rationnels, et sa mesure se majore. En effet le théorème~\ref{ThoDESooEyDOe} donne \( \lambda\big( B(q_n,\frac{\epsilon }{ 2^n }) \big)=\frac{ \epsilon }{ 2^n }\). Vu que ces boules ne sont à priori pas disjointes, le lemme~\ref{LemPMprYuC} donne
    \begin{equation}
        \lambda(\mO) \leq \sum_{n=1}^{\infty}\frac{ \epsilon }{ 2^n }=\epsilon
    \end{equation}
    par \eqref{EqPZOWooMdSRvY} avec \( q=\frac{ 1 }{2}\).

    Par complémentarité, nous pouvons construire un ensemble fermé de mesure non nulle et ne contenant aucun rationnel. Et même un fermé dans \( \mathopen[ 0 , 1 \mathclose]\), de mesure \( 1-\epsilon\) ne contenant aucun rationnel.

    Cela peut surprendre parce qu'il existe des tonnes de suites d'irrationnels qui convergent vers des rationnels\footnote{Si \( q\in \eQ\) et \( r\in \eR\setminus \eQ\) alors la suite \( (q+r/10^k)_k\) est une suite d'irrationnels convergente vers le rationnel \( q\).}, et il semble difficile de créer un ensemble contenant beaucoup d'irrationnels tout en préservant la propriété de fermeture vis à vis des suites convergentes.
\end{example}

\begin{example}[Mesure finie, non borné]
    Il existe des parties de \( \eR\) qui sont de mesure finie sans être bornés. Par exemple en posant 
    \begin{equation}
        A=\bigcup_{n=1}^{\infty}B(n,\frac{1}{ 2^n }).
    \end{equation}
    La partie \( A\) n'est pas bornée parce que que \( \eN\subset A\). Mais en termes de mesure,
    \begin{equation}
        \lambda(A)\leq \sum_{n=1}^{\infty}\frac{1}{ 2^n }<\infty
    \end{equation}
    en vertu de la somme de la série géométrique, exemple \ref{ExZMhWtJS}.
\end{example}

%---------------------------------------------------------------------------------------------------------------------------
\subsection{Propriétés de la mesure de Lebesgue}
%---------------------------------------------------------------------------------------------------------------------------

\begin{proposition}
    Tout ensemble dénombrable de \( \eR\) est mesurable de mesure nulle.
\end{proposition}

\begin{proof}
    Un point de \( \eR\) est un intervalle de mesure nulle. Si \( D\) est dénombrable, il est union disjointes et dénombrable de points. Le lemme~\ref{LemUMVooZJgMu} nous dit alors que sa mesure est \( \lambda(D)=\sum_{i=1}^{\infty}\lambda(\{ a_i \})=0\).
\end{proof}

\begin{remark}
    Il existe cependant des ensembles non dénombrables et tout de même de mesure nulle. Par exemple l'ensemble de Cantor (voir la proposition~\ref{PropBEWooXZdKN}).
\end{remark}


\begin{proposition}     \label{PropooOACLooLMIUuY}
    La mesure de Lebesgue est invariante par translation, c'est-à-dire que si \( A\) est mesurable alors \( \lambda(A)=\lambda(A+\alpha)\) pour tout réel \( \alpha\).
\end{proposition}

\begin{proof}
    Nous commençons par les intervalles ouverts :
    \begin{equation}
    \lambda\big( \mathopen] a , b \mathclose[+\alpha \big)=\lambda\big( \mathopen] a+\alpha , b+\alpha \mathclose[ \big)=(b+\alpha)-(a+\alpha)=b-a=\lambda\big( \mathopen] a , b \mathclose[ \big).
    \end{equation}
    D'après ce qui est dit dans l'exemple~\ref{ExDMPoohtNAj}, la mesure de Lebesgue sur les boréliens est invariante par translation.

    Si \( A\) est mesurable alors il existe un borélien \( B\) et un ensemble négligeable \( N\) tels que \( A=B\cup N\) par la caractérisation~\ref{EqFJIoorxZNU} de la complétion. Alors \( A+\alpha=B+\alpha\cup N+\alpha\) et \( N+\alpha\) est encore un ensemble négligeable. Donc \( \lambda(A+\alpha)=\alpha(B+\alpha)=\lambda(B)\).
\end{proof}

Le mesure \( \ell\) définie sur l'algèbre de parties \( \tribA_{\mS}\) (voir proposition~\ref{PropULFoodgXrR}). La proposition~\ref{PropIUOoobjfIB} nous donne donc une mesure extérieure par
\begin{equation}    \label{EqJGXoogdKqb}
    \lambda^*(X)=\inf\{ \sum_n\ell(A_n);A_n\in\tribA_{\mS},X\subset\bigcup_nA_n \}.
\end{equation}

La proposition suivante montre que cette mesure extérieure peut être exprimée seulement avec des intervalles ouverts.
\begin{proposition} \label{PropTNOooDcfwn}
    Nous avons
    \begin{equation}
        \lambda^*(X)=\inf\{ \sum_{n\geq 1}\ell(I_n); I_n\text{ sont des intervalles ouverts et }X\subset\bigcup_nI_n \}.
    \end{equation}
\end{proposition}

\begin{proof}
    Nous savons que dans la définition \eqref{EqJGXoogdKqb}, chacun des \( A_n\) est une réunion disjointe d'intervalles (pas spécialement ouverts) deux à deux disjoints; donc
    \begin{equation}
        \lambda^*(X)=\inf\{ \sum_n\ell(I_n);I_n\in\mS,X\subset\bigcup_nI_n \}.
    \end{equation}
    Soit \( \epsilon>0\). Si \( A\subset\bigcup_nI_n\), pour chaque \( n\geq 1\) nous considérons un intervalle ouvert \( J_n\) tel que \( I_n\subset J_n\) et \( \ell(I_n)+\frac{ \epsilon }{ 2^n }\leq \ell(J_n)\). Faisant cela pour chacun des découpages de \( X\) en intervalles nous trouvons
    \begin{equation}
        \lambda^*(X)\leq \inf\{ \sum_n\ell(J_n)\text{ } J_n\text{ est ouvert et }X\subset\bigcup_nJ_n \}+\epsilon.
    \end{equation}
    Étant donné que \( \epsilon\) est arbitraire nous avons l'égalité.
\end{proof}

\begin{proposition}[\cite{MesureLebesgueLi}]    \label{PropMXIoojpKvd}
    Si \( X\subset \eR\) est tel que \( \lambda^*(X)<\infty\) alors
    \begin{enumerate}
        \item   \label{ItemGJUoozrDILi}
            Pour tout \( \epsilon>0\) il existe un ouvert \( \Omega_{\epsilon}\) tel que
            \begin{subequations}
                \begin{numcases}{}
                    X\subset\Omega_{\epsilon}\\
                    \lambda(\Omega_{\epsilon})\leq \lambda^*(X)+\epsilon.
                \end{numcases}
            \end{subequations}
        \item   \label{ItemGJUoozrDILii}
            Il existe une intersection dénombrable d'ouverts \( G\) telle que
            \begin{subequations}
                \begin{numcases}{}
                    X\subset G\\
                    \lambda(G)=\lambda^*(X).
                \end{numcases}
            \end{subequations}
    \end{enumerate}
\end{proposition}

\begin{proof}
    Pour~\ref{ItemGJUoozrDILi}, la proposition~\ref{PropTNOooDcfwn} nous a déjà dit que
    \begin{equation}
        \lambda^*(X)=\inf\{ \sum_n\ell(I_n)\text{ } I_n\text{ est un intervalle ouvert}, X\subset\bigcup_nI_n \},
    \end{equation}
    donc si \( \epsilon>0\), il existe des intervalles ouverts \( I_n\) tels que
    \begin{subequations}
        \begin{numcases}{}
            X\subset\bigcup_nI_n\\
            \sum_n\ell(I_n)\leq \lambda^*(X)+\epsilon.
        \end{numcases}
    \end{subequations}
    Si nous posons \( \Omega_{\epsilon}=\bigcup_nI_n\), alors nous avons bien
    \begin{subequations}
        \begin{numcases}{}
            X\subset\Omega_{\epsilon}\\
            \lambda(\Omega_{\epsilon})\leq\sum_n\ell(I_n)\leq \lambda^*(X)+\epsilon.
        \end{numcases}
    \end{subequations}

    En ce qui concerne~\ref{ItemGJUoozrDILii}, pour chaque \( k\geq 1\) nous considérons l'ensemble \( \Omega_{1/k}\) obtenu comme précédemment avec \( \epsilon=1/k\) et nous posons \( G=\bigcap_{k\geq 1}\Omega_{1/k}\). Cela est une intersection dénombrable d'ouverts vérifiant \( X\subset G\) (parce que \( X\subset \Omega_{1/k}\) pour tout \( k\)) et donc \( \lambda^*(X)\leq\lambda^*(G)=\lambda(G)\). De plus pour tout \( k\) nous avons
    \begin{equation}
        \lambda(G)\leq(\Omega_{1/k})\leq \lambda^*(X)+\frac{1}{ k }
    \end{equation}
    pour tout \( k\). En faisant \( k\to \infty\) nous avons
    \begin{equation}
        \lambda(G)\leq \lambda^*(X).
    \end{equation}
    Au final
    \begin{equation}
        \lambda(G)\leq \lambda^*(X)\leq \lambda(G),
    \end{equation}
    d'où l'égalité.
\end{proof}

\begin{corollary}
    Une partir \( N\subset \eR\) est négligeable\footnote{Définition~\ref{DefAVDoomkuXi}.} si et seulement si \( \lambda^*(N)=0\).
\end{corollary}

\begin{proof}
    Nous savons que si \( N\) est négligeable il existe un borélien \( Y\) tel que \( N\subset Y\) avec \( \lambda(Y)=0\). Par conséquent\footnote{Au péril d'être lourd nous rappelons que \( \lambda^*\) est défini sur toutes les parties de \( \eR\).}
    \begin{equation}
        \lambda^*(N)\leq \lambda^*(Y)=\lambda(Y)=0.
    \end{equation}

    Pour l'implication inverse nous supposons que \( \lambda^*(N)=0\) et nous prenons l'ensemble \( G\) définit par la proposition~\ref{PropMXIoojpKvd}\ref{ItemGJUoozrDILii} : c'est un borélien contenant \( N\) et tel que \( \lambda(G)=\lambda^*(N)=0\). L'ensemble \( N\) est donc négligeable.
\end{proof}

\begin{theorem}[Régularité extérieure de la mesure de Lebesgue] \label{ThoHFXooONFRN}
    Pour tout mesurable \( A\subset \eR\) nous avons
    \begin{equation}
        \lambda(A)=\inf\{ \lambda(\Omega); \Omega\text{ ouvert contenant } A \}.
    \end{equation}
\end{theorem}
\index{régularité!extérieure de la mesure de Lebesgue}

\begin{proof}
    Nous commençons par le cas où \( B\) est un borélien.
    \begin{subproof}

        \item[Si \( B\) borélien, \( \lambda(B)<\infty\)]

        Soit \( \epsilon>0\); par la proposition~\ref{PropMXIoojpKvd}\ref{ItemGJUoozrDILi} il existe un ouvert \( \Omega_{\epsilon}\) contenant \( B\) tel que \( \lambda(\Omega_{\epsilon})\leq \lambda^*(B)+\epsilon\). Vu qu'ici \( B\) est borélien, \( \lambda^*(B)=\lambda(B)\) et nous concluons que pour tout \( \epsilon\) il existe un ouvert \( \Omega_{\epsilon}\) tel que
        \begin{subequations}
            \begin{numcases}{}
                B\subset\Omega_{\epsilon}\\
                \lambda(\Omega_{\epsilon})\leq \lambda(B)+\epsilon,
            \end{numcases}
        \end{subequations}
        et donc
        \begin{equation}
            \lambda(B)=\inf\{ \lambda(\Omega);\text{ } \Omega\text{ ouvert contenant } B\text{ } \}.
        \end{equation}

        \item[Si \( B\) borélien, \( \lambda(B)=+\infty\)]

            Dans ce cas l'infimum est pris uniquement sur des ouverts \( \Omega\) tels que \( \lambda(\Omega)=\infty\).

        \item[Si \( A\) est mesurable non borélien]

    Nous passons maintenant au cas où \( A \) est mesurable sans être borélien. Il s'écrit donc \( A=B\cup N\) avec \( B\) borélien et \( N\) négligeable par la proposition~\ref{thoCRMootPojn}, et par définition \( \lambda(A)=\lambda(B)\). Si \( Y\) est un borélien tel que \( N\subset Y\) et \( \lambda(Y)=0\) alors
    \begin{subequations}
        \begin{align}
            \lambda(A)=\lambda(B)&=\inf\{ \lambda(\Omega)\tq \text{ } \Omega\text{ ouvert}, B\subset\Omega \}\label{subeqMTHoopkSKOi}\\
            &\leq\inf\{ \lambda(\Omega)\tq \text{ } \Omega\text{ ouvert}, B\cup N\subset\Omega \}  \label{subeqMTHoopkSKOii}\\
            &\leq\inf_{\Omega',Y'}\{ \lambda(\Omega'\cup Y')\tq \text{ } \Omega'\text{, } Y'\text{ ouverts}, B\subset\Omega', Y\subset Y' \}\label{subeqMTHoopkSKOiii}\\
            &\leq\inf_{\Omega',Y'}\{ \lambda(\Omega')+\lambda(Y')\tq \text{ } \Omega'\text{, } Y'\text{ ouverts},  B\subset\Omega',Y\subset Y' \}\label{subeqMTHoopkSKOiv}\\
            &\leq\inf_{\Omega'}\{ \lambda(\Omega')\tq \text{ } \Omega'\text{ ouvert},  B\subset\Omega \}\label{subeqMTHoopkSKOv}\\
            &=\lambda(B).
        \end{align}
    \end{subequations}
    Justifications :
    \begin{itemize}
        \item \eqref{subeqMTHoopkSKOi} Le cas borélien déjà fait.
        \item \eqref{subeqMTHoopkSKOii} Les ouverts \( \Omega\) tels que \( B\cup N\subset \Omega\) vérifient a fortiori \( B\subset \Omega\); nous avons donc agrandit l'ensemble sur lequel l'infimum est pris.
        \item \eqref{subeqMTHoopkSKOiii} Parmi les ouverts \( \Omega\) qui recouvrent \( B\cup N\), il y a ceux de la forme \( \Omega'\cup Y'\) où \( \Omega'\) recouvre \( B\) et \( Y'\) est un ouvert contenant \( Y\). Donc nous avons rétréci l'ensemble sur lequel l'infimum est pris et par conséquent agrandit l'infimum.
        \item \eqref{subeqMTHoopkSKOiv} Mesure d'une union majorée par la somme des mesures.
        \item \eqref{subeqMTHoopkSKOv} Vu que \( Y\) est borélien, \( \lambda(Y)=\inf_{ Y'\text{ ouvert}}\{ \lambda(Y')\tq Y\subset Y' \}=0\). Donc pour tout \( \Omega'\) et tout \( \epsilon>0\), nous pouvons trouver un \( Y'\) vérifiant les conditions tel que \( \lambda(\Omega')+\lambda(Y')\leq \lambda(\Omega')+\epsilon\).
    \end{itemize}
    Toutes les inégalités sont des égalités en en particulier \eqref{subeqMTHoopkSKOii} donne
    \begin{equation}
        \lambda(A)=\inf\{ \lambda(\Omega)\tq \text{ } \Omega\text{ ouvert}, B\cup N\subset\Omega \},
    \end{equation}
    ce qu'il fallait.
    \end{subproof}

\end{proof}

\begin{proposition}[\cite{MesureLebesgueLi}]    \label{PropEZNoofLkVb}
    Si \( A\) est mesurable dans \( \eR\) et si \( \epsilon>0\) alors il existe un ouvert \( \Omega_{\epsilon}\) et un fermé \( F_{\epsilon}\) tels que
    \begin{subequations}    \label{subeqHNEooaNqDu}
        \begin{numcases}{}
            F_{\epsilon}\subset A\subset \Omega_{\epsilon}\\
            \lambda(\Omega_{\epsilon}\setminus F_{\epsilon})\leq \epsilon.
        \end{numcases}
    \end{subequations}
\end{proposition}

\begin{proof}
    Nous commençons par le cas où \( A\) est un borélien, que nous noterons \( B\).
    \begin{subproof}
        \item[Première étape]

            Montrons qu'il existe un ouvert \( U_{\epsilon}\) tel que
            \begin{subequations}
                \begin{numcases}{}
                    B\subset U_{\epsilon}\\
                    \lambda(U_{\epsilon}\setminus B)\leq \frac{ \epsilon }{2}.
                \end{numcases}
            \end{subequations}
            Si \( \lambda(B)<\infty\) alors le théorème~\ref{ThoHFXooONFRN} nous donne un ouvert \( U_{\epsilon}\) tel que \( B\subset U_{\epsilon}\) et \( \lambda(U_{\epsilon})\leq \lambda(B)+\frac{ \epsilon }{2}\). Nous avons alors
            \begin{equation}
                \lambda(\Omega_{\epsilon}\setminus B)=\lambda(\Omega_{\epsilon})-\lambda(B)\leq \frac{ \epsilon }{2}.
            \end{equation}
            Si par contre \( \lambda(B)=\infty\), nous posons \( B_n=B\cap\mathopen[ -n , n \mathclose]\) et \( \epsilon_n=\epsilon/2^{n+1}\). Pour chaque \( n\) nous avons un ouvert \( \Omega_n\) tel que
            \begin{subequations}
                \begin{numcases}{}
                    B_n\subset \Omega_n\\
                    \lambda(\Omega_n\setminus B_n)\leq \frac{ \epsilon }{ 2^{n+1} }
                \end{numcases}
            \end{subequations}
            Par conséquent en posant \( \Omega=\bigcup_{n\geq 1}\Omega_n\) nous avons\footnote{Nous utilisons la petite relation ensembliste \( \big( \bigcup_nA_n \big)\setminus\big( \bigcup_nB_n \big)\subset \bigcup_n(A_n\setminus B_n)\).}
            \begin{subequations}
                \begin{numcases}{}
                    B\subset \Omega\\
                    \lambda(\Omega\setminus B)\leq \lambda\big( \bigcup_n(\Omega_n\setminus B_n) \big)\leq \sum_{n\geq 1}\lambda(\Omega_n\setminus B_n)=\frac{ \epsilon }{2}.
                \end{numcases}
            \end{subequations}
            La première étape est terminée.

        \item[Deuxième étape]

            Nous prouvons à présent qu'il existe un ouvert \( \Omega_{\epsilon}\) et un fermé \( F_{\epsilon}\) tels que
            \begin{subequations}
                \begin{numcases}{}
                    F_{\epsilon}\subset B\subset \Omega_{\epsilon}\\
                    \lambda(\Omega_{\epsilon}\setminus B)\leq \frac{ \epsilon }{2}\\
                    \lambda(B\setminus F_{\epsilon})\leq \frac{ \epsilon }{2}.
                \end{numcases}
            \end{subequations}
            L'ouvert \( \Omega_{\epsilon}\), nous l'avons déjà de l'étape précédente. Pour le fermé, nous appliquons la première étape au borélien \( B^c\); ce qui nous trouvons est un ouvert \( G_{\epsilon}\) tel que
            \begin{subequations}
                \begin{numcases}{}
                    B^c\subset G_{\epsilon}\\
                    \lambda(G_{\epsilon}\setminus B^c)\leq \frac{ \epsilon }{2}.
                \end{numcases}
            \end{subequations}
            En posant \( F_{\epsilon}=G_{\epsilon}^c\) nous avons un fermé tel que \( F_{\epsilon}\subset B\) et
            \begin{equation}
                \lambda(B\setminus F_{\epsilon})=\lambda(F_{\epsilon}^c\setminus B^c)=\lambda(G_{\epsilon}\setminus B^c)\leq \frac{ \epsilon }{2}.
            \end{equation}

        \item[Dernière étape]

            Les ensembles \( F_{\epsilon}\) et \( \Omega_{\epsilon}\) trouvés à la deuxième étape donnent bien les relations \eqref{subeqHNEooaNqDu}. En effet \( \Omega_{\epsilon}\setminus F_{\epsilon}=(\Omega_{\epsilon}\setminus B)\cup(B\setminus F_{\epsilon})\), donc
            \begin{equation}
                \lambda(\Omega_{\epsilon}\setminus F_{\epsilon})\leq \lambda(\Omega_{\epsilon}\setminus B)+\lambda(B\setminus F_{\epsilon})=\epsilon.
            \end{equation}
    \end{subproof}
    Nous passons au cas où \( A=B\cup N\) est mesurable. Nous commençons par prendre les \( \Omega_{\epsilon}\) et \( F_{\epsilon}\) qui correspondent à \( B\) :
    \begin{subequations}
        \begin{numcases}{}
            F_{\epsilon}\subset B\subset \Omega_{\epsilon}\\
            \lambda(\Omega_{\epsilon}\setminus F_{\epsilon})\leq \epsilon.
        \end{numcases}
    \end{subequations}
    Soit \( Y\) un borélien tel que \( N\subset Y\) et \( \lambda(Y)\) puis un ouvert \( Y'\) tel que \( \lambda(Y')\leq \epsilon\) et \( Y\subset Y'\). L'existence d'un tel \( Y'\) est assurée par la proposition~\ref{ThoHFXooONFRN} appliquée à \( Y\). Nous vérifions que les ensembles \( F_{\epsilon}\) et \( \Omega_{\epsilon}\cup Y'\) fonctionnent. En effet \( \Omega_{\epsilon}\cup Y'\setminus F_{\epsilon}\subset (\Omega_{\epsilon}\setminus F_{\epsilon})\cup Y'\), donc
    \begin{subequations}
        \begin{numcases}{}
            F_{\epsilon}\subset B\cup N\subset \Omega_{\epsilon}\cup Y'\\
            \lambda\big( (\Omega_{\epsilon}\setminus F_{\epsilon}) \big)\leq \lambda(\Omega_{\epsilon}\setminus F_{\epsilon})+\lambda(Y')\leq 2\epsilon.
        \end{numcases}
    \end{subequations}
    Donc en réalité il faut choisir \( \Omega_{\epsilon/2}\), \( F_{\epsilon/2}\) et \( \lambda(Y')\leq \epsilon/2\).
\end{proof}

\begin{theorem}[Régularité intérieure de la mesure de Lebesgue]     \label{THOooJNMCooPMvCDq}
    Si \( A\) est mesurable dans \( \eR\) alors
    \begin{equation}
        \lambda(A)=\sup\{ \lambda(K);  K\text{ compact contenu dans } A \}.
    \end{equation}
\end{theorem}
\index{régularité!intérieure de la mesure de Lebesgue}

\begin{proof}
    Par la proposition~\ref{PropEZNoofLkVb} nous avons
    \begin{equation}    \label{EqTPEooUHTbH}
        \lambda(A)=\sup_{ F\text{ fermé dans } A}\lambda(F).
    \end{equation}
    Pour un tel \( F\) nous posons \( K_n=F\cap\mathopen[ -n , n \mathclose]\) qui est compact\footnote{parce que fermé et borné, théorème de Borel-Lebesgue~\ref{ThoXTEooxFmdI}.} et contenu dans \( B\). De plus le lemme~\ref{LemAZGByEs}\ref{ItemJWUooRXNPcii} nous dit que
    \begin{equation}
        \lambda(F)=\lim_{n\to \infty} \lambda(K_n)
    \end{equation}
    Donc tous les \( \lambda(F)\) peuvent être arbitrairement approchés par un \( \lambda(K)\) avec \( K\) compact dans \( A\), et le supremum \eqref{EqTPEooUHTbH} n'est pas affecté en nous restreignant à prendre des compacts contenus dans \( B\) :
    \begin{equation}
        \lambda(A)=\sup_{ F\text{ fermé dans } A}\lambda(F)=\sup_{ K\text{ compact dans } A}\lambda(K).
    \end{equation}
\end{proof}

%--------------------------------------------------------------------------------------------------------------------------- 
\subsection{Fonctions mesurables}
%---------------------------------------------------------------------------------------------------------------------------

\begin{lemma}
    Soit une fonction \( f\colon \eR\to \eR\) mesurable telle que \( \lambda(f\neq 0)>0\). Alors il existe une partie mesurable \( M\) et \( m>0\) tels que \( \lambda(M)>0\) et \( f(x)>m\) pour tout \( x\in M\).
\end{lemma}

\begin{proof}
    Nous notons
    \begin{equation}
        D=\{ x\in \eR\tq f(x)>0 \},
    \end{equation}
    et nous supposons que \( \lambda(D)>0\) pour fixer les idées (si ce n'est pas le cas, nous prenons pour \( D\) la partie où \( f\) est strictement négative).

    Nous posons
    \begin{subequations}
        \begin{align}
            A_1&=\mathopen[ 1 , \infty \mathclose[\\
            A_n&=\mathopen[ \frac{1}{ n } , \frac{1}{ n-1 } \mathclose[.
        \end{align}
    \end{subequations}
    Ces parties \( A_n\) sont disjointes; donc les parties
    \begin{equation}
        D_n=\{ x\in \eR\tq f(x)\in A_n \}
    \end{equation}
sont également disjointes. Vu que \( \bigcup_nD_n=\mathopen] 0 , \infty \mathclose[\), nous avons \( D=\bigcup_{n\in \eN}D_n\). Vu que
    \begin{equation}
        \lambda(D)=\sum_{n=1}^{\infty}\lambda(D_n)>0,
    \end{equation}
    il existe au moins un \( N\) tel que \( \lambda(D_N)>0\). Pour \( x\in D_N\) nous avons
    \begin{equation}
        f(x)\in A_N=\mathopen[ \frac{1}{ N } , \frac{1}{ N-1 } \mathclose[.
    \end{equation}
    Donc pour \( x\in D_N\) nous avons \( f(x)>\frac{1}{ N }\).
\end{proof}

%---------------------------------------------------------------------------------------------------------------------------
\subsection{Ensemble de Vitali (non mesurable)}
%---------------------------------------------------------------------------------------------------------------------------

\begin{example}[Un ensemble non mesurable au sens de Lebesgue\cite{ooIARBooPdOgAQ}]      \label{EXooCZCFooRPgKjj}
    Nous considérons l'ensemble quotient \( \eR/\eQ\); chaque classe intersecte l'intervalle \( \mathopen[ 0 , 1 \mathclose]\). Grâce à l'axiome du choix (voir~\ref{NORooLMBYooYjUoju}) nous pouvons construire un ensemble \( V\) contenant un représentant dans \( \mathopen[ 0 , 1 \mathclose]\) de chaque classe. Un tel ensemble est un \defe{ensemble de Vitali}{Vitali (ensemble)}. Nous allons prouver que \( V\) n'est pas mesurable.

    Supposons que \( V\) soit mesurable. Alors tous les ensembles de la forme \( V+q\) (\( q\in \eQ\)) sont mesurables et ont même mesure par la proposition~\ref{PropooOACLooLMIUuY}. Nous posons
    \begin{equation}
        A=\bigcup_{\substack{q\in\eQ\\-1\leq q\leq 1}}(V+q)\subset\mathopen[ -1 , 2 \mathclose].
    \end{equation}
    Cela est une union disjointe d'ensembles mesurables. Donc
    \begin{equation}
        \lambda(A)=\sum_{\substack{q\in\eQ\\-1\leq q\leq 1}}\lambda(V+q).
    \end{equation}
    Vu que \( A\subset\mathopen[ -1 , 2 \mathclose]\) nous avons \( \lambda(A)\leq 3\) et donc tous les termes de la somme doivent être nuls. Nous avons donc \( \lambda(A)=0\).

    Prouvons toutefois que \( \mathopen[ 0 , 1 \mathclose]\subset A\), ce qui serait une contradiction. Soit \( x\in\mathopen[ 0 , 1 \mathclose]\); il est dans une des classes de \( \eR/\eQ\) et donc il existe \( v\in V\) tel que \( x-v\in \eQ\). De plus \( x,v\in \mathopen[ 0 , 1 \mathclose]\), donc
    \begin{equation}
        -1\leq x-v\leq 1.
    \end{equation}
    Cela fait que \( x\in V+(x-v)\subset A\). Nous avons donc \( x\in A\) et donc \( \mathopen[ 0 , 1 \mathclose]\subset A\). En conséquence de quoi nous aurions \( \lambda(A)\geq 1\).
\end{example}



%---------------------------------------------------------------------------------------------------------------------------
\subsection{Ensemble de Cantor}
%---------------------------------------------------------------------------------------------------------------------------

Nous considérons la fonction donnant l'écriture décimale des nombres définie en \eqref{EqXXXooOTsCK}.

\begin{definition}[Ensemble de Cantor]  \label{DefIYDooVIDJs}
    Soit \( K_0=\mathopen[ 0 , 1 [\) et les ensembles \( K_n\) définis par la récurrence
        \begin{equation}
            K_{n+1}=\big( \frac{1}{ 3 }K_n \big)\cup\big( \frac{1}{ 3 }(K_n+2) \big).
        \end{equation}
        L'ensemble
        \begin{equation}
            K=\bigcup_{n\geq 0}K_n
        \end{equation}
        est l'\defe{ensemble triadique de Cantor}{Cantor!ensemble}\index{ensemble!de Cantor}.
\end{definition}
Les principales propriétés de l'ensemble de Cantor sont qu'il est non dénombrable (proposition~\ref{PropTPPooDySbm}) et borélien de mesure nulle (proposition~\ref{PropBEWooXZdKN}).

\begin{normaltext}
    L'idée de base pour prouver que l'ensemble \( K\) est non dénombrable est que ses éléments sont les nombres qui s'écrivent en base \( 3\) sans utiliser le chiffre \( 1\). En prenant un nombre sans \( 1\) écrit en base \( 3\), en changeant tous les \( 2\) en \( 1\) et en lisant le résultat en base \( 2\), nous obtenons tous les nombres possibles en base \( 2\) et donc une quantité non dénombrable. L'idée est donc simple et astucieuse. La mise en musique est un peu plus délicate parce qu'il faut faire attention aux queues de suites; c'est pour cela que nous avons construit l'ensemble de Cantor en partant de \( \mathopen[ 0 , 1 [\) et non de \( \mathopen[ 0 , 1 \mathclose]\).
\end{normaltext}

Le lemme suivant dit précisément ce que nous entendons en disant que les éléments de l'ensemble de Cantor sont les nombres qui s'écrivent en base \( 3\) sans utiliser le chiffre \( 1\). Nous rappelons que \( \eD_3\) est l'ensemble des suites constituées de \( 0\), \( 1\) et \( 2\), et qui ne se terminent pas par une suite infinie de \( 2\), voir~\ref{NORMALooTZWYooPMgOIm} pour une définition précise.
\begin{lemma}[\cite{MonCerveau}]   \label{LemAZGoosKzEm}
    Soit \( n\in \eN\) et \( x\in \eD_3\) (définition \ref{NORMALooTZWYooPMgOIm}); nous avons \( \varphi_3(x)\in K_n\in\) si et seulement si \( x_1,\ldots, x_n\in\{ 0,2 \}\).
\end{lemma}

\begin{proof}
    Nous procédons par récurrence en commençant avec \( n=1\). Si \( x_1=1\) alors
    \begin{equation}
        \varphi_3(x)=\frac{1}{ 3 }+\sum_{k=2}^{\infty}\frac{ x_k }{ 3^k }\in\mathopen[ \frac{1}{ 3 } , \frac{ 2 }{ 3 } [.
    \end{equation}
    Notons que \( \varphi_3(x)=\frac{ 2 }{ 3 }\) est impossible parce que ça demanderait une queue de suite de \( 2\). Par conséquent \( \varphi_3(x)=\mathopen[ 0 , 1 [\setminus\mathopen[ \frac{1}{ 3 } , \frac{ 2 }{ 3 } [=K_1\).

        Nous passons à la récurrence.

        \begin{subproof}
        \item[Sens direct]

        Nous supposons que \( x_1,\ldots, x_{n+1}\in\{ 0,2 \}\) et nous montrons que \( \varphi_3(x)\in K_{n+1}\). La chose surprenante est que nous n'allons pas considérer deux cas suivant que \( x_{n+1}\) vaut \( 0\) ou \( 1\); nous allons considérer deux cas suivant\footnote{Pour comprendre pourquoi, faire un dessin de comment \( K_n\) se transforme en \( K_{n+1}\) et remarquer dans \( K_2\), les deux premiers segments ne sont pas une division du premier segment de \( K_1\), mais bien une copie des \emph{deux} segments de \( K_1\).} que \( x_1\) vaut \( 0\) ou \( 1\). Écrivons encore \( \varphi_3(x)\) :
    \begin{equation}
        \varphi_3(x)=\sum_{k=1}^{n+1}\frac{ x_k }{ 3^k }+\sum_{k=n+2}^{\infty}\frac{ x_k }{ 3^k }.
    \end{equation}
    \begin{subproof}
        \item[Si \( x_1=0\)]
            Alors nous avons
            \begin{equation}
                3\varphi_3(x)=\sum_{k=2}^{\infty}\frac{ x_k }{ 3^{k-1} }=\sum_{k=1}^{\infty}\frac{ x_{k+1} }{ 3^k }=\varphi_3(x_2,\ldots, x_n,x_{n+1},\ldots)
            \end{equation}
            Vu que par hypothèse \( x_2,\ldots, x_{n+1}\) sont dans \( \{ 0,2 \}\) nous avons \( 3\varphi_3(x)\in K_n\) par hypothèse de récurrence. Cela implique que \( \varphi_3(x)\in K_{n+1}\).
        \item[Si \( x_1=2\)]
            Alors
            \begin{equation}
                \varphi_3(x)=\frac{ 2 }{ 3 }+\sum_{k=2}^{\infty}\frac{ x_k }{ 3^k },
            \end{equation}
            et
            \begin{equation}
                3\varphi_3(x)-2=\sum_{k=1}^{\infty}\frac{ x_{k+1} }{ 3^k }=\varphi(x_2,\ldots, x_{n+1},\ldots),
            \end{equation}
            et donc là nous avons \( 3\varphi_3(x)-2\in K_n\), ce qui implique encore \( \varphi_3(x)\in K_{n+1}\).
    \end{subproof}

        \item[Sens réciproque]

            Nous devons maintenant prouver que \( \varphi_3(x)\in K_{n+1}\) implique \( x_1,\ldots, x_{n+1}\in\{ 0,2 \}\). Par le même calcul que précédemment nous avons soit
            \begin{equation}
                3\varphi_3(x)=\varphi_3(x_2,\ldots, x_{n+1},\ldots),
            \end{equation}
            si \( x_1=0\), soit
            \begin{equation}
                3\varphi_3(x)-2=\varphi_3(x_2,\ldots, x_{n+1},\ldots),
            \end{equation}
            si \( x_1=2\). Dans les deux cas, si \( x_l=1\) pour un certain \( 2\leq l\leq n+1\), alors l'hypothèse de récurrence donne que ces éléments ne sont pas dans \( K_n\) et donc \( \varphi_3(x)\) pas dans \( K_{n+1}\).

        \end{subproof}
\end{proof}

\begin{corollary}[\cite{MonCerveau}]   \label{CorSEDooJmeXt}
    En posant \( \eE=\{ x\in\eD_3\tq x_i\neq 1\forall i \}\) nous avons \( K=\varphi_3(\eE)\). Et plus précisément, \( \varphi_3\colon \eE\to K\) est une bijection.
\end{corollary}

\begin{proof}
    Nous divisons la preuve en trois étapes.
    \begin{subproof}
    \item[Image contenue dans \( K\)]
        Si \( x\in \eE\) et \( n\in \eN\) nous avons \( x_1,\ldots, x_n\in\{ 0,2 \}\) et donc \( \varphi_3(x)\in K_n\) par la proposition~\ref{LemAZGoosKzEm}. Donc
        \begin{equation}
            \varphi_3(x)\in\bigcup_{n\geq 1}K_n=K.
        \end{equation}
    \item[Injective]
        L'application \( \varphi_3\colon \eE\to K\) est injective parce qu'elle est déjà injective depuis \( \eD_3\).
    \item[Surjective]
        Soit \( p\in K\subset\mathopen[ 0 , 1 [\). Vu que \( \varphi_3\colon \eD_3\to \mathopen[ 0 , 1 [\) est surjective (théorème~\ref{ThoRXBootpUpd}), il existe \( x\in \eD_3\) tel que \( \varphi_3(x)=p\). Pour tout \( n\) nous avons \( \varphi_3(x)\in K_n\) et donc \( x_1,\ldots, x_n\in\{ 0,2 \}\) et donc au final \( x\in \eE\).
    \end{subproof}
\end{proof}

\begin{proposition}[\cite{MonCerveau}]    \label{PropTPPooDySbm}
    L'ensemble de Cantor est non dénombrable.
\end{proposition}

\begin{proof}

    Nous avons prouvé à la proposition~\ref{PropNNHooYTVFw} que l'ensemble \( \eD_2\) n'était pas dénombrable. Nous allons à présent prouver que l'application
    \begin{equation}
        \begin{aligned}
            \psi\colon \eD_2&\to K \\
            c&\mapsto \varphi_3(   c\text{ en remplaçant les } 1\text{ par des } 2  )
        \end{aligned}
    \end{equation}
    est une bijection. Le fait que \( \psi\) soit injective est une conséquence du fait que ce soit la composition de deux applications injectives (le remplacement et \( \varphi_3\)). Il faut par contre montrer que l'image est égale à \( K\), en notant qu'il n'est pas évident à priori que l'image soit contenue dans \( K\).

    L'opération qui consiste à remplacer les \( 1\) par des \( 2\) est une bijection \( \eD_2\to \eE\). Le corolaire~\ref{CorSEDooJmeXt} nous dit aussi que \( \varphi_3\colon \eE\to K\) est une bijection. En tant que composée de bijections, \( \psi\) est une bijection.

    Étant en bijection avec \( \eD_2\) qui n'est pas dénombrable par la proposition~\ref{PropNNHooYTVFw}, l'ensemble de Cantor n'est pas dénombrable.
\end{proof}

\begin{proposition}[Ensemble de Cantor]    \label{PropBEWooXZdKN}
    L'ensemble de Cantor\footnote{Définition~\ref{DefIYDooVIDJs}} est borélien, non dénombrable et de mesure nulle.
\end{proposition}

\begin{proof}
    Nous reprenons les notations de la définition~\ref{DefIYDooVIDJs}. Le fait que l'ensemble de Cantor soit non dénombrable a été prouvé dans la proposition~\ref{PropTPPooDySbm}.

    L'ensemble de Cantor étant une intersection dénombrable de boréliens, il est borélien par le lemme~\ref{LemBWNlKfA}. Vu que \( K_n\subset\mathopen[ 0 , 1 [\) nous avons \( \frac{1}{ 3 }K_n\leq \frac{1}{ 3 }\) et \( \frac{1}{ 3 }(K_n+2)\geq \frac{ 2 }{ 3 }\), donc \( K_n\) est une union disjointe de \( 2^n\) intervalles de mesure \( 2/3^n\). Nous avons donc
        \begin{equation}
            \lambda(K_n)=\left( \frac{ 2 }{ 3 } \right)^n.
        \end{equation}
        L'ensemble de Cantor étant contenu dans chacun des \( K_n\), sa mesure est plus petite que la mesure de chacun des \( K_n\) (lemme~\ref{LemPMprYuC}) et donc \( \lambda(K)\leq \left( \frac{ 2 }{ 3 } \right)^n\) pour tout \( n\); ergo \( \lambda(K)=0\).
\end{proof}

%--------------------------------------------------------------------------------------------------------------------------- 
\subsection{Mesure positive sans intervalle}
%---------------------------------------------------------------------------------------------------------------------------

Vu que la mesure de Lebesgue est basée sur la mesure des intervalles et quelques extensions, nous sommes en droit de croire qu'une partie de mesure strictement positive de \( \eR\) doit toujours contenir un intervalle, éventuellement à partie de mesure nulle près. Eh bien non.

\begin{example}[\cite{BIBooHPTSooFQrjLy}]       \label{EXooVZVIooXZvDaE}
Soient une énumération \( (q_i)\) de \( \eQ\cap\mathopen] 0 , 1 \mathclose[\) et une suite \( (r_i)\) telle que \( \sum_{i=0}^{\infty}r_i<\frac{ 1 }{2}\). Quitte à prendre \( r_i\) plus petit, supposons de plus que \( B(q_i,r_i)\subset \mathopen[ 0 , 1 \mathclose]\).
    
    Nous posons $J_n=B(q_i,r_i)$, \( J=\bigcup_{i=0}^{\infty}J_n\) et
    \begin{equation}
        B=\mathopen[ 0 , 1 \mathclose]\setminus J.
    \end{equation}
    Les parties \( J_i\) ne sont pas disjointes, donc, en notant \( \lambda\) la mesure de Lebesgue,
    \begin{equation}
        0<\lambda(J)\leq \sum_{i=0}^{\infty}\lambda(J_i)\leq \frac{ 1 }{2}.
    \end{equation}
    Mais, par définition, l'union \( \mathopen[ 0 , 1 \mathclose]=B\cup J\) est disjointe, donc
    \begin{equation}
        1=\lambda\big( \mathopen[ 0 , 1 \mathclose] \big)=\lambda(J)+\lambda(B).
    \end{equation}
    Nous en déduisons que
    \begin{equation}
        \frac{ 1 }{2}\leq \lambda(B)\leq 1.
    \end{equation}
    
    Je plaide que cette partie \( B\) ne contient non seulement aucun intervalle, mais qu'il est impossible de le compléter par une partie de mesure nulle pour obtenir un intervalle.

    Soit un intervalle \( I\) dans \( \mathopen[ 0 , 1 \mathclose]\). Il existe \( q_i\in I\) et donc\footnote{C'est ici que nous utilisons le fait que \( r_i\) est choisi pour que \( B(q_i,r_i)\) ne déborde pas de \( \mathopen[ 0 , 1 \mathclose]\). Sinon il aurait fallu chipoter et prendre seulement une partie de la boule.}
    \begin{equation}
        J_i\subset B\setminus I.
    \end{equation}
    Donc il n'existe pas de parties de mesure nulle qui, ajoutée à \( B\), contiendrait \( I\).
\end{example}

Vous voulez un truc dingue à propos de la partie \( J\) de l'exemple \ref{EXooVZVIooXZvDaE} ? Le théorème \ref{THOooJNMCooPMvCDq} nous dit qu'il existe dans \( J\) des compacts de mesure arbitrairement proches de \( \lambda(J)\). Il existe donc des compacts non seulement de mesure strictement positive mais même de mesure assez grande, tout en étant infiniment découpés.

%+++++++++++++++++++++++++++++++++++++++++++++++++++++++++++++++++++++++++++++++++++++++++++++++++++++++++++++++++++++++++++
\section{Intégrale par rapport à une mesure}
%+++++++++++++++++++++++++++++++++++++++++++++++++++++++++++++++++++++++++++++++++++++++++++++++++++++++++++++++++++++++++++

\begin{normaltext}
    Nous n'en avons pas encore terminé avec la théorie de la mesure, mais nous devons quand même définir les intégrales et voir quelques propriétés avant de continuer avec la mesure parce que la définition de la mesure sur un espace mesurable produit\footnote{Théorème~\ref{ThoWWAjXzi}.} passe par une intégrale.
\end{normaltext}

\begin{normaltext}      \label{NORMooFZEDooIxSgLe}
    En théorie de l'intégration, la convention est la suivante : pour une fonction \( f\colon X\to \eR\), nous considérons sur \( X\) la tribu des ensembles mesurables au sens de Lebesgue sur \( X\), \emph{tout en gardant celle des boréliens sur l'ensemble d'arrivée}. C'est-à-dire qu'en théorie de l'intégration, c'est
    \begin{equation}
        f\colon \big( X,\Lebesgue(X) \big)\to \big( \eR,\Borelien(\eR) \big).
    \end{equation}
    En particulier, \( f\colon \eR^n\to \eR^m\) sera mesurable si pour tout borélien \( A\) de \( \eR^m\) l'ensemble \( f^{-1}(A)\) est Lebesgue-mesurable dans \( \eR^n\).

    Étant donné qu'il est franchement difficile de créer des ensembles non mesurables au sens de Lebesgue, il est franchement difficile de créer des fonctions non mesurables à valeurs réelles. L'hypothèse de mesurabilité est donc toujours satisfaite dans les cas pratiques.

    Voir aussi le point \ref{NORMooNFOMooYnaflN}, et les résultats qui suivent.
\end{normaltext}

%---------------------------------------------------------------------------------------------------------------------------
\subsection{Définition pour les fonctions à valeurs positives}
%---------------------------------------------------------------------------------------------------------------------------

Voir le thème \ref{THEMEooHINHooJaSYQW}.

Une mesure \( \mu\) sur un espace mesurable \( (\Omega,\tribA)\) permet de définir une fonctionnelle linéaire sur l'ensemble des fonctions mesurables \( \Omega\to \eR\). Cette fonctionnelle linéaire est l'intégrale que nous allons définir à présent.

\begin{definition}  \label{DefTVOooleEst}
    Soient \( (\Omega,\tribA,\mu)\) un espace mesuré ainsi que \( Y\in\tribA\). Notre but est de définir
    \begin{equation}
        \int_Yfd\mu
    \end{equation}
    que nous nommons \defe{intégrales de \( f\)}{intégrale d'une fonction} de \( f\) sur \( Y\) pour la mesure \( \mu\).
    \begin{subproof}
    \item[Fonction étagée]
        Si \( f\) est une fonction étagée\footnote{Définition~\ref{DefBPCxdel}.}, et si sa forme canonique est \( f=\sum_{i=1}^n\alpha_i\mtu_{A_i}\) alors nous définissons
    \begin{equation}        \label{EqooGAFMooZLzjPs}
        \int_Yfd\mu=\sum_i\alpha_i\mu(Y\cap A_i).
    \end{equation}

\item[Fonction mesurable à valeurs positives]
    Pour une fonction \( \tribA\)-mesurable \( f\colon \Omega\to \mathopen[ 0 , \infty \mathclose]\) nous définissons l'intégrale de \( f\) sur \( Y\) par
    \begin{equation}        \label{EqDefintYfdmu}
        \int_Yfd\mu=\sup\Big\{ \int_Y\psi d\mu\,\text{où } \psi\text{ est une fonction étagée telle que } 0\leq \psi\leq f \Big\}.
    \end{equation}

    \end{subproof}
\end{definition}

\begin{remark}
    Toute fonction mesurable à valeurs dans \(  \mathopen[ 0 , +\infty \mathclose]   \) est intégrable (l'intégrale vaut éventuellement \( +\infty\)). Au moment où une fonction commence à prendre des valeurs positives et négatives, nous demandons à pouvoir intégrer séparément les parties positive et négative. C'est pour cela que nous disons qu'une fonction \( f\) à valeurs dans \( \eR\) est intégrable si \( | f |\) l'est.
\end{remark}

\begin{normaltext}
    Le nombre \( \int_0^{\infty}f\) est défini directement par \eqref{EqDefintYfdmu} complètement indépendamment d'une éventuelle limite \( \lim_{x\to 0} \int_0^xf\).
\end{normaltext}

\begin{normaltext}      \label{NORMooXTGBooKDnAhZ}
    Si la fonction n'est pas mesurable ? Alors nous n'avons pas défini son intégrale. Supposons la plus simple des fonctions non mesurables sur \( \Omega\) : la fonction indicatrice d'une partie non mesurable :
    \begin{equation}
        f(x)=\begin{cases}
            1    &   \text{si } x\in A\\
            0    &    \text{sinon. }
        \end{cases}
    \end{equation}
    où \( A\subset \Omega\) n'est pas mesurable\footnote{Ça existe, par exemple \ref{EXooCZCFooRPgKjj}.}.

    Nous supposons que l'espace mesuré \( (\Omega,\tribF,\mu)\) est complet (définition~\ref{DefBWAoomQZcI}). Vu que \( A\) n'est pas mesurable, il n'est pas contenu dans une partie négligeable (parce que l'espace est complet), et nous voulons que l'intégrale ne soit pas nulle; sinon on se demande bien à quoi sert une intégrale.

    Toute fonction étagée minorant \( f\) est forcément nulle en dehors de \( A\). Dès que \( B\) est une partie mesurable de mesure non nulle dans \( A\), le complémentaire de \( B\) dans \( A\) est encore non mesurable, et nous voulons encore que l'intégrale de \( f\) sur ce complémentaire soit non nul.

    Mais comme \( A\) n'est pas mesurable et que \( \mtu_A\) n'est le supremum d'aucune suite de fonctions mesurables (lemme~\ref{LemIGKvbNR}), bien que le supremum qui définirait l'intégrale de \( f\) existe (toute partie de \( \eR\) a un supremum), il est sans espoir que ce supremum ait un sens que l'on puisse interpréter en tant que mesure de \( f\).
\end{normaltext}

%--------------------------------------------------------------------------------------------------------------------------- 
\subsection{Premières propriétés}
%---------------------------------------------------------------------------------------------------------------------------

\begin{normaltext}
    Si \( (\Omega^n(\cA)\tribA,\mu)\) est un espace mesurable, et si \( Y\) est un élément de \( \tribA\), nous avons l'espace mesurable \( (Y,\tribA_Y,\mu_Y)\) donné par
    \begin{itemize}
        \item \( \tribA_y=\{ B\cap Y\tq B\in \tribA \}\),
        \item \( \mu_Y=\mu\).
    \end{itemize}
    Et là, nous arrivons à un problème de notations parce que \( \int_Yfd\mu\) peut désigner l'intégrale de \( f\) sur \( Y\) dans \( (\Omega,\tribA,\mu)\) ou l'intégrale de \( f\) sur \( Y\) dans \( (Y,\tribA_Y,\mu_Y)\).

    Heureusement, nous allons tout de suite montrer que ces deux choses sont identiques.
\end{normaltext}


\begin{lemma}
    Soit un espace mesuré \( (\Omega,\tribA,\mu)\) ainsi que \( Y\in\tribA\). Nous considérons une fonction \( f\colon \Omega\to \eR^+\) qui est \( \tribA\)-mesurable et intégrable sur \( Y\).
    
    Alors, avec des notations que j'espère être claires,
    \begin{enumerate}
        \item
            \( f\) est \( \tribA_Y\)-mesurable,
        \item
            \( f\) est \( (Y,\tribA_Y,\mu_Y)\)-intégrable,
        \item
            nous avons l'égalité
    \begin{equation}
        \int_{(Y,\tribA_Y,\mu_Y)}f|_Y=\int_{(Y\subset \Omega,\tribA,\mu)}f.
    \end{equation}
    \end{enumerate}
\end{lemma}

\begin{proof}
    Nous considérons les deux ensembles suivants :
    \begin{subequations}
        \begin{align}
            S_1&=\{ \psi\text{ étagées sur } Y\text{ et majorées par } f|_Y \}\\
            S_2&=\{ \psi\text{ étagées sur } \Omega\text{ et majorées par } f \}.
        \end{align}
    \end{subequations}
    Nous considérons l'application suivante :
    \begin{equation}
        \begin{aligned}
            s\colon S_1&\to S_2 \\
            s(\psi)(x)&=\begin{cases}
                \psi(x)    &   \text{si } x\in Y\\
                0    &    \text{sinon.}
            \end{cases}
        \end{aligned}
    \end{equation}
    L'application \( s\) est une bijection.

    Pour \( \psi\in S_1\) nous avons 
    \begin{equation}
        \psi=\sum_{k=1}^n\psi_k\mtu_{A_k}|_Y
    \end{equation}
    avec \( A_k\in \tribA_k\subset \tribA\) et \( \mtu_{A_k}|_Y\colon Y\to \{ 0,1 \}\). Nous avons aussi
    \begin{equation}
        s(\psi)=\sum_{\psi_k}\mtu_{A_k}
    \end{equation}
    avec \( \mtu_{A_k}\colon \Omega\to \{ 0,1 \}\).

    En ce qui concerne les intégrales de ces fonctions étagées, nous avons
    \begin{subequations}
        \begin{align}
            \int_{(Y,\tribA_Y,\mu_Y)}\psi&=\sum_{k=1}^n\psi_k\mu_Y(A_k\cap Y)\\
            &=\sum_{k=1}^n\psi_k\mu(A_k) \label{SUBEQooGWYGooOuucEo}\\
            &=\int_{(Y\subset \Omega,\tribA,\mu)}s(\psi).
        \end{align}
    \end{subequations}
    Justifications. Pour passer à \eqref{SUBEQooGWYGooOuucEo} nous avons utilisé d'abord que \( A_k\subset Y\) et ensuite que \( \mu_Y(A_k)=\mu(A_k)\).

    Nous sommes maintenant prêts à prouver l'égalité du lemme. Nous avons ceci :
    \begin{subequations}
        \begin{align}
            \int_{(Y,\tribA_Y,\mu_Y)}f|_Y&=\sup\{ \int_{(Y,\tribA_Y,\mu_Y)}\psi\tq \psi\in S_1 \}\\
            &=\sup\{ \int_{(Y\subset \Omega,\tribA,\mu)}s(\psi)\tq \psi\in S_1 \}\\
            &=\sup\{ \int_{(Y\subset \Omega,\tribA,\mu)}\varphi\tq \varphi\in S_2 \}\\
            &=\int_{(Y\subset\Omega,\tribA,\mu)}f.
        \end{align}
    \end{subequations}
\end{proof}

\begin{lemma}       \label{LemooPJLNooVKrBhN}
    Si \( (\Omega,\tribA,\mu)\) est un espace mesuré et si \( B\in \tribA\) alors
    \begin{equation}
        \mu(B)=\int_B1d\mu=\int_{\Omega}\mtu_B.
    \end{equation}
\end{lemma}

\begin{proof}
    La fonction caractéristique d'une partie mesurable est une fonction étagée dont la forme canonique est \( \mtu_B=1\cdot \mtu_B+0\times \mtu_{B^c}\). Son intégrale est donc
    \begin{equation}
        \int\mtu_Bd\mu=1\times \mu(B)+0\times \mu(B^c)=\mu(B)
    \end{equation}
    parce que \( 0\times \mu(B^c)=0\), même si \( \mu(B^c)=\infty\), comme nous l'avons convenu en~\ref{normooGAAJooUPCbzG}.
\end{proof}

\begin{proposition}[\cite{MonCerveau}]      \label{PROPooGTMVooPHcrRl}
    Soient une fonction \( f\colon (\Omega,\tribA,\mu)\to \eR^+\) et une fonction \( g\) intégrable sur \( \Omega\) telle que \( f\leq g\). Alors \( f\) est intégrable.
\end{proposition}

\begin{proof}
    Une fonction étagée qui minore \( f\) minore également \( g\). Donc l'ensemble sur lequel il faut faire le supremum pour définir \( \int_{\Omega}f\) est inclus dans celui pour \( \int_{\Omega}g\). Le second supremum étant fini, le premier l'est également.
\end{proof}

\begin{lemma}       \label{LEMooSPOFooBxDEAV}
    Soient un espace mesuré \( (\Omega,\tribA,\mu)\), une fonction \( f\colon \Omega\to \eR^+\) et \( Y\in\tribA\). Nous avons :
    \begin{equation}        \label{EQooSBDKooPTDEcr}
        \int_Yfd\mu=\int_{\Omega}f\mtu_Yd\mu.
    \end{equation}
\end{lemma}

\begin{proof}
    En plusieurs parties, selon la généralité.
    \begin{subproof}
        \item[Si \( f\) est étagée]
            Nous posons \( f=\sum_{k=1}^nf_k\mtu_{A_k}\) avec \( f_k\in \eR^+\). Dans ce cas,
            \begin{equation}
                f\mtu_Y=\sum_kf_k\mtu_{A_k\cap Y}
            \end{equation}
            est encore une fonction étagée. Donc nous avons d'une part
            \begin{equation}
                \int_{\Omega}f\mtu_Y=\int_{\Omega}\sum_kf_k\mtu_{A_k\cap Y}=\sum_kf_k\mu(A_k\cap Y),
            \end{equation}
            et d'autre part,
            \begin{equation}
                \int_Yfd\mu=\sum_kf_k\mu(Y\cap A_k),
            \end{equation}
        \item[Si \( f\) est à valeurs positives]
            Nous posons
            \begin{equation}
                S_1=\{ \psi\text{ étagées sur } \Omega\tq 0\leq \psi\leq  f\mtu_{Y} \}
            \end{equation}
            et
            \begin{equation}
                S_2=\{ \psi\mtu_Y\tq \psi\text{ étagée avec } 0\leq\psi\leq f \}.
            \end{equation}
            Nous prouvons que \( S_1=S_2\). 
            
            Si \( \psi\in S_1\), alors
            \begin{equation}
                0\leq \psi\leq f\mtu_Y\leq f.
            \end{equation}
            De plus comme \( \psi=0\) hors de \( Y\) nous avons \( \psi=\psi\mtu_Y\).

            Pour l'autre inclusion, soit \( 0\leq \psi\leq f\) pour une fonction étagée \( \psi\) et montrons que \( \psi\mtu_Y\in S_1\). L'application \( \psi\mtu_Y\) est étagée sur \( \Omega\) et vérifie
            \begin{equation}
                0\leq \psi\mtu_Y\leq f\mtu_Y
            \end{equation}
            parce que \( \psi\leq f\).
        \item[L'égalité à prouver]
            Dans l'égalité \ref{EQooSBDKooPTDEcr} à prouver, le membre de droite est, d'après la définition \ref{EqDefintYfdmu},
            \begin{equation}
                \int_{\Omega}f\mtu_Y=\sup\{ \int\psi\tq \psi\in S_1 \}.
            \end{equation}
            Il nous reste donc à prouver que \(  \int_Yf\) se calcule de la même façon avec les éléments de \( S_2\). D'abord nous copions la définition :
            \begin{equation}
                \int_Yf=\sup\{ \int_Y\psi\tq 0\leq \psi\leq f \}.
            \end{equation}
            Ensuite nous réfléchissons un peu. Si \( 0\leq \psi\leq f\) avec \( \psi=\sum_k\psi_k\mtu_{A_k}\), alors
            \begin{equation}
                \int_Y\psi=\sum_k\mu(A_k\cap Y)=\int_Y\psi\mtu_{Y}=\int_{\Omega}\psi\mtu_Y.
            \end{equation}
            La dernière égalité est la partie déjà faite, à propos des fonctions étagées. Nous avons donc bien
            \begin{equation}
                \int_Yf=\sup\{ \int_Ys\tq s\in S_2 \}.
            \end{equation}
    \end{subproof}
\end{proof}

%--------------------------------------------------------------------------------------------------------------------------- 
\subsection{Propriétés plus avancées}
%---------------------------------------------------------------------------------------------------------------------------

%---------------------------------------------------------------------------------------------------------------------------
\subsubsection{Convergence monotone}
%---------------------------------------------------------------------------------------------------------------------------

Le théorème suivant est très utile parce que le théorème fondamental d'approximation~\ref{THOooXHIVooKUddLi} donne les fonctions étagées qu'il faut.

\begin{theorem}[Théorème de la convergence monotone ou de Beppo-Levi\cite{mathmecaChoi}] \label{ThoRRDooFUvEAN}
    Soit un espace mesuré \( (\Omega,\tribA,\mu)\) et \( (f_n)\) une suite croissante de fonctions mesurables à valeurs dans \( \mathopen[ 0 , \infty \mathclose]\). Alors la limite ponctuelle \( \lim_{n\to \infty} f_n\) existe, est mesurable et
    \begin{equation}    \label{EqFHqCmLV}
        \lim_{n\to \infty} \int_{\Omega}f_nd\mu= \int_{\Omega}\lim_{n\to \infty} f_nd\mu,
    \end{equation}
    cette intégrable valant éventuellement \( \infty\).
\end{theorem}
\index{théorème!convergence!monotone}
\index{théorème!Beppo-Levi}
\index{permuter!limite et intégrale!convergence monotone}

\begin{proof}
    La limite ponctuelle de la suite est la fonction à valeurs dans \( \mathopen[ 0 , \infty \mathclose]\) donnée par
    \begin{equation}
        f(x)=\lim_{n\to \infty} f_n(x).
    \end{equation}
    Ces limites existent parce que pour chaque \( x\) la suite \( f_n(x)\) est une suite numérique croissante. Nous notons
    \begin{equation}
        I_0=\int_{\Omega}fd\mu.
    \end{equation}
    Nous posons par ailleurs
    \begin{equation}
        I_n=\int_{\Omega}f_n.
    \end{equation}
    Cela est une suite numérique croissante qui a par conséquent une limite que nous notons \( I=\lim_{n\to \infty} I_n\). Notre objectif est de montrer que \( I=I_0\). D'abord par croissance de la suite, pour tous $n$ nous avons \( I_n\leq I_0\), par conséquent \( I\leq I_0\).

    Nous prouvons maintenant l'inégalité dans l'autre sens en nous servant de la définition \eqref{EqDefintYfdmu}. Soit une fonction simple \( h\) telle que \( h\leq f\), et une constante \( 0<C<1\). Nous considérons les ensembles
    \begin{equation}
        E_n=\{ x\in\Omega\tq f_n(x)\geq Ch(x) \}.
    \end{equation}
    Ces ensembles vérifient les propriétés \( E_n\subset E_{n+1}\) et \( \bigcup_{n=1}^{\infty}E_n=\Omega\). Pour chaque \( n\) nous avons les inégalités
    \begin{equation}
        \int_{\Omega}f_n\geq\int_{E_n}f_n\geq C\int_{E_n}h.
    \end{equation}
    Si nous prenons la limite \( n\to\infty\) dans ces inégalités,
    \begin{equation}
        \lim_{n\to \infty} \int_{\Omega}f_n\geq C\lim_{n\to \infty} \int_{E_n}h=C\int_{\Omega}h.
    \end{equation}
    Par conséquent \( \lim_{n\to \infty} \int f_n\geq C\int_{\Omega}h\). Mais étant donné que cette inégalité est valable pour tout \( C\) entre \( 0\) et \( 1\), nous pouvons l'écrire sans le \( C\) :
    \begin{equation}        \label{EqzAKEaU}
        \lim_{n\to \infty} \int_{\Omega}f_n\geq \int_{\Omega}h.
    \end{equation}
    Par définition, l'intégrale de \( f\) est donné par le supremum des intégrales de \( h\) où \( h\) est une fonction simple dominée par \( f\). En prenant le supremum sur \( h\) dans l'équation \eqref{EqzAKEaU} nous avons
    \begin{equation}
        \lim_{n\to \infty} \int_{\Omega}f_n\geq\int_{\Omega}f,
    \end{equation}
    ce qu'il nous fallait.
\end{proof}

\begin{remark}
    La proposition~\ref{THOooXHIVooKUddLi} ainsi que le lemme~\ref{LemYFoWqmS} montrent qu'une fonction mesurable peut-être écrite comme limite croissante de fonctions simples. Cela permet de démontrer des théorèmes en commençant par prouver sur les fonctions simples et en utilisant Beppo-Levi pour généraliser.
\end{remark}

\begin{remark}
    Une des raisons de demander la positivité des fonctions \( f_n\) est de n'avoir pas d'ambiguïté à parler d'intégrales qui valent \( \infty\). Si par exemple nous prenons \( \Omega=\mathopen[ 0 , 1 \mathclose]\) et que nous considérons
    \begin{equation}
        f_n(x)=\begin{cases}
            0    &   \text{si } x\leq \frac{1}{ n }\\
            \frac{1}{ x }    &    \text{sinon}.
        \end{cases}
    \end{equation}
    Ce sont des fonctions intégrables, mais la limite étant la fonction \( 1/x\), l'égalité \eqref{EqFHqCmLV} est une égalité entre deux intégrales valant \( \infty\).
\end{remark}

\begin{corollary}[Inversion de somme et intégrales] \label{CorNKXwhdz}
    Si \( (u_n)\) est une suite de fonctions mesurables positives ou nulles, alors
    \begin{equation}
        \sum_{i=0}^{\infty}\int u_i=\int\sum_{i=0}^{\infty}u_i.
    \end{equation}
\end{corollary}
\index{permuter!somme et intégrale}

\begin{proof}
    Nous considérons la suite des sommes partielles de \( (u_n)\) : \( f_n(x)=\sum_{i=0}^nu_n(x)\). Le théorème de la convergence monotone (théorème~\ref{ThoRRDooFUvEAN}) implique que
    \begin{equation}
        \lim_{n\to \infty} \int f_n=\int\lim_{n\to \infty} f_n.
    \end{equation}
    Nous remplaçons maintenant \( f_n\) par sa valeur en termes des \( u_i\) et dans le membre de gauche nous permutons l'intégrale avec la somme finie :
    \begin{equation}
        \lim_{n\to \infty} \sum_{i=0}^{\infty}\int u_n=\int\sum_{i=0}^{\infty}u_n,
    \end{equation}
    ce qu'il fallait démontrer.
\end{proof}

%///////////////////////////////////////////////////////////////////////////////////////////////////////////////////////////
\subsubsection{Lemme de Fatou}
%///////////////////////////////////////////////////////////////////////////////////////////////////////////////////////////

\begin{lemma}[Lemme de Fatou]\index{lemme!Fatou}\index{Fatou}   \label{LemFatouUOQqyk}
    Soit \( (\Omega,\tribA,\mu)\) un espace mesuré et \( f_n\colon \Omega\to \mathopen[ 0 , \infty \mathclose]  \) une suite de fonctions mesurables. Alors la fonction \( f(x)=\liminf f_n(x)\) est mesurable et
    \begin{equation}
        \int_{\Omega}\liminf f_nd\mu\leq\liminf\int_{\Omega}fd\mu.
    \end{equation}
\end{lemma}
%TODO : pour la mesurabilité, il faudra citer un théorème du genre de celui fait avec le sup.

\begin{proof}
    Nous posons
    \begin{equation}
        g_n(x)=\inf_{i\geq n}f_i(x).
    \end{equation}
    Cela est une suite croissance de fonctions positives mesurables telles que, par définition,
    \begin{equation}
        \lim_{n\to \infty}g_n(x)=\liminf f_n(x).
    \end{equation}
    Nous pouvons y appliquer le théorème de la convergence monotone,
    \begin{equation}
        \lim_{n\to \infty} \int g_n(x)=\int\liminf f_n(x).
    \end{equation}
    Par ailleurs, pour chaque \( i\geq n\) nous avons
    \begin{equation}
        \int g_n\leq \int f_i,
    \end{equation}
    en passant à l'infimum nous avons
    \begin{equation}
        \int g_n\leq \inf_{i\geq n}\int f_i,
    \end{equation}
    et en passant à la limite nous avons
    \begin{equation}
        \int\liminf f_n=\lim_{n\to \infty} \int g_n\leq \lim_{n\to \infty} \inf_{i\geq n}\int f_i=\liminf_{i\to\infty}\inf f_i.
    \end{equation}
\end{proof}

L'inégalité donnée dans ce lemme n'est en général pas une égalité, comme le montre l'exemple suivant :
\begin{equation}
    f_i=\begin{cases}
        \mtu_{\mathopen[ 0 , 1 \mathclose]}    &   \text{si } i\text{ est pair}\\
        \mtu_{\mathopen[ 1 , 2 \mathclose]}    &    \text{si } i\text{ est impair}.
    \end{cases}
\end{equation}
Nous avons évidemment \( g_n(x)=0\) tandis que \( \int_{\mathopen[ 0 , 2 \mathclose]}f_i=1\) pour tout \( i\).

\begin{theorem}[\cite{MesureLebesgueLi}]        \label{ThoooCZCXooVvNcFD}
    Soient \( f,g\) des fonctions étagées positives sur \( (\Omega,\tribA,\mu)\). Alors si \( \alpha\in\mathopen[ 0 , \infty \mathclose]\) nous avons
    \begin{enumerate}
        \item
            \begin{equation}
                \int_{\Omega}(\alpha f)d\mu=\alpha\int_{\Omega}fd\mu.
            \end{equation}
        \item       \label{ITEMooBLEVooDznQTY}
            \begin{equation}
                \int_{\Omega}(f+g)d\mu=\int_{\Omega}fd\mu+\int_{\Omega}gd\mu.
            \end{equation}
        \item\label{ITEMooOJRAooQkoQyD}
    Si \( a_k\in \eR^+\) et si les \( f_k\) sont étagées positives,
    \begin{equation}
        \int_{\Omega}\left( \sum_{k=1}^na_kf_k \right)=\sum_{k=1}^na_k\left( \int_{\Omega} f_kd\mu \right).
    \end{equation}
    \end{enumerate}
\end{theorem}

\begin{proof}
    En ce qui concerne le produit par un nombre, tout repose sur le fait que
    \begin{equation}
        (\alpha f)^{-1}(\alpha a_i)=f^{-1}(a_i),
    \end{equation}
    ce qui fait que si la représentation canonique de \( f\) est \( f=\sum_ia_i\mtu_{A_i}\) alors la représentation canonique de \( \alpha f\) est \( \alpha f=\sum_i(\alpha a_i)\mtu_{A_i}\). Donc
    \begin{equation}
        \int_{\Omega}\alpha fd\mu=\sum_i\alpha a_i\mu(A_i)=\alpha \sum_ia_i\mu(A_i)=\alpha\int_{\Omega}fd\mu.
    \end{equation}

    Pour la somme c'est plus lourd. Soient les formes canoniques
    \begin{subequations}
        \begin{align}
            f&=\sum_ia_i\mtu_{A_i}\\
            g&=\sum_jb_j\mtu_{B_i}.
        \end{align}
    \end{subequations}
    Vu que l'union des \( B_j\) est \( \Omega\) nous avons l'union disjointe \( A_i=\bigcup_jA_i\cap B_j\) et donc \( \mu(A_i)=\sum_j\mu(A_i\cap B_j)\). Nous avons donc pour les intégrales :
    \begin{subequations}
        \begin{align}
            \int_{\Omega}fd\mu&=\sum_ia_i\sum_j\mu(A_i\cap B_j)\\
            \int_{\Omega}gd\mu&=\sum_ib_k\sum_l\mu(B_k\cap A_l).
        \end{align}
    \end{subequations}
    Pour la somme :
    \begin{equation}
        \int_{\Omega}fd\mu+\int_{\Omega}gd\mu=\sum_{k,l}(a_k+b_l)\mu(A_k\cap B_l).
    \end{equation}

    Nous devons maintenant évaluer \( \int_{\Omega}(f+g)d\mu\). Pour cela nous remarquons que si \( c\in (f+g)(\Omega)\) (l'ensemble des valeurs atteintes pas \( f+g\)), alors nous notons
    \begin{equation}
        I_c=\{ (k,l)\tq a_k+b_l=c \}
    \end{equation}
    et nous avons
    \begin{equation}
        \{ f+g=c \}=\bigcup_{(k,l)\in I_c}(A_k\cap B_l),
    \end{equation}
    et comme cette union est disjointe, nous pouvons faire la somme des mesures :
    \begin{equation}
        \mu(f+g=c)=\sum_{(k,l)\in I_c}\mu(A_k\cap B_l).
    \end{equation}
    Cela nous permet de faire le calcul suivant :
    \begin{subequations}
        \begin{align}
            \int_{\Omega}(f+g)d\mu&=\sum_{c\in (f+g)(\Omega)}c\mu(f+g=c)\\
            &=\sum_{c\in(f+g)(\Omega)}c\sum_{(k,l)\in I_c}\mu(A_k\cap B_l)\\
            &=\sum_{c\in(f+g)(\Omega)}\sum_{(k,l)\in I_c} (a_k+b_l) \mu(A_k\cap B_l)
        \end{align}
    \end{subequations}
    Dans cette double somme, tous les couples \( (k,l)\) sont tirés une et une seule fois parce qu'ils sont tous dans un et un seul des \( I_c\), donc
    \begin{subequations}
        \begin{align}
            \int_{\Omega}(f+g)d\mu&= \sum_{c\in(f+g)(\Omega)}\sum_{(k,l)\in I_c} (a_k+b_l) \mu(A_k\cap B_l)\\
            &=\sum_{(k,l)}(a_k+b_l)\mu(A_k\cap B_l)\\
            &=\int_{\Omega}fd\mu+\int_{\Omega}gd\mu.
        \end{align}
    \end{subequations}
\end{proof}

\begin{remark}
    Si \( f=\sum_ka_k\mtu_{A_k}\) n'est pas une décomposition canonique, il n'en reste pas moins que chacun des \( \mtu_{A_k}\) est la forme canonique de lui-même. Donc le théorème~\ref{ThoooCZCXooVvNcFD} s'applique et nous avons quand même
    \begin{equation}
        \int_{\Omega}fd\mu=\sum_ka_k\mu(A_k).
    \end{equation}
\end{remark}

\begin{proposition} \label{PROPooOVDEooDJvOau}
    Soient deux fonctions mesurables \( f,g\colon \Omega\to \mathopen[ 0 , +\infty \mathclose]\). Alors
    \begin{equation}
        \int_{\Omega}(f+g)=\int_{\Omega}f+\int_{\Omega}g.
    \end{equation}
\end{proposition}

\begin{proof}
    Soient des suites \( f_n\to f\) et \( g_n\to g\) fournies par le théorème fondamental d'approximation~\ref{THOooXHIVooKUddLi}. Par le théorème de la convergence monotone~\ref{ThoRRDooFUvEAN} nous avons d'une part
    \begin{equation}
        \lim_{n\to \infty} \int_{\Omega}(f_n+g_n)=\int_{\Omega}\int(f+g),
    \end{equation}
    et par le théorème~\ref{ThoooCZCXooVvNcFD} nous avons d'autre part
    \begin{equation}
        \lim_{n\to \infty} \int_{\Omega}(f_n+g_n)=\lim_{n\to \infty} \big( \int f_n+\int g_n \big)=\int f+\int g
    \end{equation}
    où nous avons encore utilisé la convergence monotone.

    En égalant les deux, nous avons notre résultat.
\end{proof}

%---------------------------------------------------------------------------------------------------------------------------
\subsection{Fonctions à valeurs réelles}
%---------------------------------------------------------------------------------------------------------------------------

L'intégrale d'une fonction à valeurs dans \( \mathopen[ 0 , +\infty \mathclose]\) étant faite, nous passons aux fonctions à valeurs dans \( \mathopen[ -\infty, +\infty \mathclose]\).

\begin{propositionDef}[\cite{MonCerveau}]  \label{DefTCXooAstMYl}
    Soit une fonction mesurable \( f\colon \Omega\to  \bar \eR \). Nous considérons les deux fonction suivantes à valeurs dans \( \mathopen[ 0 , +\infty \mathclose]\) :
    \begin{subequations}
        \begin{align}
            f^+(x)&=\begin{cases}
                0    &   \text{si } f(x)<0\\
                f(x)    &    \text{si } f(x)\geq 0.
            \end{cases}\\
            f^-(x)&=\begin{cases}
                0    &   \text{si } f(x)>0\\
                -f(x)    &    \text{si } f(x)\leq 0.
            \end{cases}
        \end{align}
    \end{subequations}
    Nous avons \( \int_{\Omega}| f |<\infty\) si et seulement si \( \int_{\Omega} f^+<\infty  \) et \( \int_{\Omega}f^-<\infty\).

    Dans ce cas nous disons que \( f\) est \defe{intégrable}{intégrable} au sens de Lebesgue et nous posons
    \begin{equation}    \label{EqUHSooWfgUty}
        \int_{\Omega}f=\int_{\Omega}f^+-\int_{\Omega}f^-
    \end{equation}
\end{propositionDef}

\begin{proof}
    Vu que \( f\) est mesurable, les fonctions \( f^+\) et \( f^-\) sont également mesurables et nous avons l'égalité
    \begin{equation}
        | f |=f^++f^-.
    \end{equation}
    La proposition~\ref{PROPooOVDEooDJvOau} nous dit alors que
    \begin{equation}
        \int_{\Omega} | f |=\int_{\Omega}f^++\int_{\Omega}f^-.
    \end{equation}
    Dans cette égalité, tous les nombres sont dans \( \mathopen[ 0 , \infty \mathclose]\). Le membre de gauche vaut \( +\infty\) si et seulement si au moins un des deux de droite vaut \( +\infty\).
\end{proof}

Nous verrons comment donner un sens à \( \int_{\Omega}f\) dans certains cas où \( f\) n'est pas intégrable sur \( \Omega\) dans la section~\ref{SecGAVooBOQddU} sur les intégrales impropres.

Nous définissons aussi
\begin{equation}
    \mu(f)=\int_{\Omega}f
\end{equation}
si \( f\) est une fonction mesurable sur \( \Omega\).

\begin{lemma}       \label{LEMooMWKTooIKomSw}
    Pour \( f\colon \Omega\to \eR\) nous avons \( \int_{\Omega}| f |<\infty\) si et seulement si \( \int_{\Omega}f\) existe et est finie.
\end{lemma}

\begin{proof}
    Deux sens.
    \begin{subproof}
        \item[\( \Rightarrow\)]
            La proposition \ref{DefTCXooAstMYl} nous indique que \( \int_{\Omega}f^+\) et \( \int_{\Omega}f^-\) sont finies. Dans ce cas, la partie «définition» de \ref{DefTCXooAstMYl} donne \( \int_{\Omega}f=\int_{\Omega}f^+-\int_{\Omega}f^-<\infty\).
        \item[\( \Leftarrow\)]
            Nous n'avons défini \( \int_{\Omega}f\) que dans le cas où les intégrales de \( f^+\) et \( f^-\) sont finies.
    \end{subproof}
\end{proof}
Ce lemme justifie pourquoi nous appelons l'espace \( L^1\) l'espace des «fonctions intégrables».

\begin{remark}
    Dans \( \eR^d\), quasiment toutes les fonctions et ensembles sont mesurables. En effet la construction d'ensembles non mesurables demande obligatoirement l'utilisation de l'axiome du choix; de tels ensembles doivent être construits «exprès pour». Il y a très peu de chances pour que vous tombiez sur un ensemble non mesurable de \( \eR^d\) sans que vous ne vous en rendiez compte.

    Il y en a un en l'exemple \ref{EXooCZCFooRPgKjj}.
\end{remark}

\begin{remark}
    «Mesurable» ne signifie pas «intégrable». Par exemple la fonction
    \begin{equation}
        \begin{aligned}
            f\colon \eR&\to \bar\eR \\
            \omega&\mapsto\begin{cases}
            \frac{1}{ \omega }    &   \text{si } \omega\neq 0\\
            \infty    &    \text{si }\omega=0.
            \end{cases}
        \end{aligned}
    \end{equation}
    est mesurable, mais non intégrable.
\end{remark}

%--------------------------------------------------------------------------------------------------------------------------- 
\subsection{Additivité de l'intégrale}
%---------------------------------------------------------------------------------------------------------------------------

\begin{lemma}   \label{LemPfHgal}
    Soit une fonction \( f\colon \Omega\to \eR\) telle que \( | f(x)|\leq g(x) \) pour tout \( x\in\Omega\). Si \( g\) est intégrable, alors \( f\) est intégrable.
\end{lemma}

\begin{proof}
    La fonction \( g\) est manifestement à valeurs réelles positives. La proposition~\ref{PROPooGTMVooPHcrRl} nous dit alors que \( | f |\) est intégrable. Ensuite c'est au tour de la proposition~\ref{DefTCXooAstMYl} de conclure à l'intégrabilité de \( f\).
\end{proof}

\begin{proposition}     \label{PROPooFIYEooCpdmwZ}
    Soient deux fonctions intégrables sur \( (S,\tribF,\mu)\) et à valeurs dans \( \eC\). Alors \( f+g\) est intégrable et
    \begin{equation}
        \int_S(f+g)d\mu=\int_Sfd\mu+\int_Sgd\mu.
    \end{equation}
\end{proposition}

\begin{proof}
    En plusieurs étapes suivant la généralité de \( f\) et \( g\).
    \begin{subproof}
        \item[Si \( f\) et \( g\) sont étagées et positives]
            C'est le théorème~\ref{ThoooCZCXooVvNcFD}\ref{ITEMooBLEVooDznQTY} déjà prouvé.
        \item[Si \(f\) et \( g\) sont à valeurs positives]
            Le théorème fondamental d'approximation~\ref{THOooXHIVooKUddLi} nous permet de considérer des suites croissantes de fonctions étagées positives \( (f_k)\) et \( (g_k)\) qui vérifient \( f_k\to f\) et \( g_k\to g\).

            Pour chaque \( k\) nous avons
            \begin{equation}        \label{EQooXXYOooUhkOJL}
                \int_S(f_k+g_k)d\mu=\int_Sf_kd\mu+\int_Sg_kd\mu.
            \end{equation}
            De plus, la suite \( k\mapsto f_k+g_k\) est une suite croissante de fonctions étagées positives convergeant vers \( f+g\). Le théorème de la convergence monotone~\ref{ThoRRDooFUvEAN} nous permet donc de passer à la limité dans \eqref{EQooXXYOooUhkOJL} et de permuter toutes les limites avec toutes les intégrales, des deux côtés.
        \item[\( f\) et \( g\) à valeurs réelles]
            Il faut diviser le domaine en de nombreuses régions suivant les signes de \( f\), \( g\) et \( f+g\).
    \end{subproof}
\end{proof}

Nous prouvons à présent l'additivité de l'intégrale pour des unions finie. Une version pour les unions dénombrables sera donnée dans les propositions \ref{PROPooTFOAooJBwmCV} et \ref{PROPooDWYNooWKJmEV}.
\begin{proposition}[\( \sigma\)-additivité finie]     \label{PropOPSCooVpzaBt}
    Si \( A,B\subset \Omega\) sont des parties disjointes de \( (\Omega,\tribA,\mu)\) et si \( f\colon \Omega\to \eR\) est intégrable sur \( A\cup B\) alors les intégrales \( \int_Af\) et \( \int_Bf\) existent et
    \begin{equation}
        \int_{A\cup B}f=\int_Af+\int_Bf.
    \end{equation}
\end{proposition}

\begin{proof}
    Vu que \( A\) et \( B\) sont disjoints, \( \mtu_{A\cup B}=\mtu_A+\mtu_B\). En utilisant alors le lemme \ref{LEMooSPOFooBxDEAV} et la proposition \ref{PROPooFIYEooCpdmwZ} nous avons le calcul
    \begin{equation}
        \int_{A\cup B}f=\int_{\Omega}f\mtu_{A\cup B}=\int_{\Omega} f\mtu_A+\int_{\Omega}f\mtu_B=\int_Af+\int_Bf.
    \end{equation}
\end{proof}


%---------------------------------------------------------------------------------------------------------------------------
\subsection{Fonctions à valeurs vectorielles (dimension finie)}
%---------------------------------------------------------------------------------------------------------------------------

Nous voulons intégrer des fonctions du type
\begin{equation}
    f \colon \Omega\to V
\end{equation}
où \( \Omega\) et \( V\) sont des espaces vectoriels. Nous expliquons à présent plus précisément le cadre.

\begin{normaltext}      \label{NORMooTQBIooBaScjt}
    Nous considérons à présent un espace vectoriel normé \( (V,\| . \|)\) de dimension finie, et un espace mesuré \( (\Omega,\tribA,\mu)\).

    Attention à ne pas confondre espace de départ et espace d'arrivée. Vu que \( V\) est un espace topologique, nous avons bien entendu les boréliens de \( V\), et pour peut que nous ayons une mesure sur \( V\) (qui qui n'est pas compliqué à créer à partir de celle canonique de \( \eR^n\) et un isomorphisme), nous avons déjà une définition de \( \int_Vfd\mu\) lorsque \( f\colon V\to \eR\).

    Ici nous nous proposons non d'intégrer \( f\colon V\to \eR\) mais bien \( f\colon (\Omega,\tribA,\mu)\to V\) où \( V\) est un espace vectoriel normé.

    Le lemme suivant est la point de départ pour définir les intégrales de fonctions à valeurs dans un espace vectoriel de dimension finie. Pour les fonctions à valeurs dans un espace de dimension infine (par exemple de Banach), il existe des choses, mais c'est un peu plus compliqué.
\end{normaltext}

\begin{lemma}[\cite{MonCerveau}]        \label{LEMooCVHDooLJASAs}
    Soit un espace vectoriel \( V\) réel de dimension finie, muni de la norme \( N\). Soient une base \( \{ e_i \}\) de \( V\), et une fonction \( f\colon (\Omega,\tribA,\mu)\to V\) telle que la norme \( N(f)\colon \Omega\to \eR^+\) soit intégrable. Nous notons \( f_i\) les composantes de \( f\) : \( f(x)=\sum_if_i(x)e_i\).

    Alors pour chaque \( i\),
    \begin{enumerate}
        \item
            la fonction \( | f_i |\colon \Omega\to \eR^+\) est intégrable,
        \item
            la fonction \( f_i\colon \Omega\to \eR\) est intégrable.
    \end{enumerate}
\end{lemma}

\begin{proof}
    Si \( V\) était un espace muni d'un produit scalaire, et si la base \( \{ e_i \}\) était orthonormée, ce serait facile parce que la norme majore toutes les composantes. Hélas, ce n'est pas spécialement le cas. La base \( \{ e_i \}\) n'est pas spécialement orthonormée et même la norme \( N\) ne dérive pas spécialement d'un produit scalaire.

    Nous allons utiliser l'équivalence de toutes les normes en dimension finie (théorème~\ref{ThoNormesEquiv}) pour nous ramener au cas d'une norme euclidienne.

    Nous considérons sur \( V\) la norme «euclidienne» construite sur la base \( \{ e_i \}\) : \( \| \sum_iv_ie_i \|=\sum_i| v_i |^2\). Par équivalence des normes nous avons des nombres non nuls \( \lambda_1\) et \( \lambda_2\) tels que
    \begin{equation}
        N(v)\leq \lambda_1\| v \|,
    \end{equation}
    et
    \begin{equation}
        \| v \|\leq \lambda_2 N(v)
    \end{equation}
    pour tout \( v\in V\). Pour un \( i\) fixé nous avons alors les majorations
    \begin{equation}
        N\big( f_i(x)e_i \big)\leq \lambda_1\| f_i(x)e_i \|\leq \lambda_1\| f(x) \|\leq \lambda_1\lambda_2N\big( f(x) \big).
    \end{equation}
    En posant \( N_i=N(e_i)\) nous avons la majoration\footnote{Vous notez l'utilisation de la condition~\ref{ItemDefNormeii} de la définition~\ref{DefNorme} de la norme pour «convertir» la norme \( N\) en valeur absolue.}
    \begin{equation}
        | f_i(x) |\leq \frac{ \lambda_1\lambda_2 }{ N(e_i) }N\big( f(x) \big).
    \end{equation}
    L'application
    \begin{equation}
        \begin{aligned}
            | f_i |\colon \Omega&\to \eR^+ \\
            x&\mapsto | f_i(x) |
        \end{aligned}
    \end{equation}
    est donc une fonction à valeurs réelles positives, majorée par une fonction intégrable (la fonction \( x\mapsto N\big( f(x) \big)\)). Elle est donc intégrable par le lemme~\ref{LemPfHgal}.

    La fonction \( f_i\) elle-même est alors intégrable par la proposition~\ref{DefTCXooAstMYl}.
\end{proof}

Notons que ce lemme est en réalité très simple si \( V\) est un espace vectoriel normé dont la norme découle d'un produit scalaire, comme c'est le cas pour \( \eC\). D'ailleurs, il ne faut pas se voiler la face : le cas d'intégrales de fonctions à valeurs dans \( \eC\) sera dans le Frido le cas de loin le plus courant. À ce propos, nous n'avons pas encore défini ce que nous voulons noter \( \int_{\Omega}fd\mu\) lorsque \( f\) est une fonction à valeurs vectorielles. Comblons vite ce manque \ldots

\begin{propositionDef}[\cite{MonCerveau}]       \label{PROPooOFSMooLhqOsc}
    Soit une fonction \( f\colon \Omega\to V\) où \( V\) est un espace vectoriel normé de dimension finie. Soit une base \( \{ e_i \}\) de \( V\).  Si la fonction \( \| f \|\colon \Omega\to \eR^+\) est intégrable, alors
    \begin{enumerate}
        \item
            toutes les composantes \( f_i\colon \Omega\to \eR\) sont intégrables,
        \item
            le vecteur
            \begin{equation}        \label{EQooQCKMooZCbybq}
                \sum_i(\int_{\Omega}f_i) e_i
            \end{equation}
            ne dépend pas de la base choisie.
    \end{enumerate}
    Dans ce cas, la fonction \( f\) est dite \defe{intégrable}{intégrable!fonction à valeurs vectorielles} et nous définissons
    \begin{equation}
        \int_{\Omega}fd\mu=\sum_i(\int_{\Omega}f_i) e_i.
    \end{equation}
\end{propositionDef}

\begin{proof}
    Le fait que les composantes soient intégrables est le lemme~\ref{LEMooCVHDooLJASAs}. Soient deux bases de \( V\), \( \{ e_i \}\) et \( \{ s_{\alpha} \}\), liées conformément à \eqref{EQooFRQRooSMsQQB} par la relation \( s_{\alpha}=\sum_iQ_{i\alpha}e_i\) pour une certaine matrice inversible \( Q\). Nous avons pour tout \( x\in \Omega\) :
    \begin{equation}
        f(x)=\sum_if_i(x)e_i=\sum_{\alpha}f_{\alpha}(x)s_{\alpha}
    \end{equation}
    avec \( f_{\alpha}(x)=\sum_if_i(x)Q_{\alpha i}^{-1}\) par la formule \eqref{EQooFXYLooCRmRdA}.

    Notons pour être pointilleux que les ensembles \( \{ e_i \}\) et \( \{ s_{\alpha} \}\) ne sont pas indexés par le même ensemble, de telle sorte que \( f_i\) ne peut pas être confondu avec \( f_{\alpha}\), même lorsqu'on attribue des valeurs à \( i\) et à \( \alpha\).

    Comme combinaisons linéaires des fonctions \( f_i\) qui sont intégrables, les fonctions \( f_{\alpha}\) sont intégrables (proposition~\ref{PROPooFIYEooCpdmwZ}). En écrivant \( \int_{\Omega}f\) par rapport à la base \( \{ s_{\alpha} \}\) nous trouvons :
    \begin{subequations}
        \begin{align}
            \sum_{\alpha}(\int f_{\alpha})s_{\alpha}&=\sum_{\alpha}\big( \int \sum_if_i(x)Q_{\alpha i}^{-1}dx \big)\sum_jQ_{j\alpha}e_j\\
            &=\sum_j\int\sum_{\alpha i}f_i(x)Q_{\alpha i}^{-1}Q_{j\alpha}dxe_j\\
            &=\sum_j\int f_j(x)dxe_j\\
            &=\sum_j(\int f_i)e_i
        \end{align}
    \end{subequations}
    où nous avons permuté des sommes finies et des intégrales des fonctions \( f_i\), à valeurs dans \( \eR\) en vertu de la proposition~\ref{PROPooFIYEooCpdmwZ}
\end{proof}

La proposition suivante est, pour les intégrales à valeurs vectorielles, analogue à la proposition \ref{DefTCXooAstMYl}.

\begin{proposition}     \label{PROPooNSCPooCMkrZl}
    Soit une fonction mesurable \( f\colon \Omega\to (V,\| . \|)\). Soit une base \( \{ e_i \}\) de \( V\) et la décomposition \( f=\sum_if_ie_i\).

    Nous avons équivalence entre
    \begin{enumerate}
        \item       \label{ITEMooYLADooCXKEds}
            \( \int_{\Omega}\| f \|<\infty\)
        \item       \label{ITEMooLEYEooQTGwmt}
            \( \int_{\Omega}| f_i |<\infty\)
        \item       \label{ITEMooYDDAooMKwDIR}
            \( \int_{\Omega}f_i^+<\infty\) et \( \int_{\Omega}f_i^-<\infty\).
    \end{enumerate}
\end{proposition}

\begin{proof}
    L'équivalence entre les points~\ref{ITEMooLEYEooQTGwmt} et~\ref{ITEMooYDDAooMKwDIR} est la proposition~\ref{DefTCXooAstMYl}. Nous démontrons l'équivalence entre~\ref{ITEMooYLADooCXKEds} et~\ref{ITEMooLEYEooQTGwmt}.

    Vu que toutes les normes sont équivalentes sur \( V\), nous considérons en particulier la norme associée à la base \( \{ e_i \}\) donnée par
    \begin{equation}
        N(x)=\sum_i| x_i |.
    \end{equation}
    Il existe des constantes \( \lambda_1\) et \( \lambda_2\) telles que
    \begin{equation}
        \lambda_1\big( \sum_i| f_i(x) | \big)\leq \| f(x) \|\leq \lambda_2\big( \sum_i| f_i(x) | \big)
    \end{equation}
    pour tout \( x\in \Omega\).

    La première inégalité dit que si \( \int_{\Omega}\| f \|<\infty\), alors \( \lambda_1\big( \sum_i\int_{\Omega}| f-i | \big)<\infty\). Et vu que chacun des termes est positif, ils sont tous finis.

    La seconde inégalité donne l'implication dans réciproque.
\end{proof}

%---------------------------------------------------------------------------------------------------------------------------
\subsection{Quelques propriétés}
%---------------------------------------------------------------------------------------------------------------------------

Le lemme suivant nous aide à détecter des fonctions presque partout nulles.
\begin{lemma}   \label{Lemfobnwt}
    Soit \( f\) une fonction mesurable positive ou nulle telle que
    \begin{equation}
        \int_{\Omega}fd\mu=0.
    \end{equation}
    Alors \( f=0\) \( \mu\)-presque partout.
\end{lemma}

\begin{proof}
    L'ensemble des points \( x\in\Omega\) tels que \( f(x)\neq 0\) peut s'écrire comme une union dénombrable disjointe :
    \begin{equation}
        \{ x\in\Omega\tq f(x)\neq 0 \}=\bigcup_{i=0}^{\infty}E_i
    \end{equation}
    avec
    \begin{subequations}
        \begin{align}
            E_0&=\{ x\in\Omega\tq f(x)>1 \}\\
            E_i&=\{ x\in\Omega\tq \frac{1}{ i+1 }\leq f(x)<\frac{1}{ i } \}.
        \end{align}
    \end{subequations}
    Si un des ensembles \( E_i\) est de mesure non nulle, alors nous pouvons considérer la fonction simple \( h(x)=\frac{1}{ i+1 }\mtu_{E_i}\) dont l'intégrale sur \( \Omega\) est strictement positive. Par conséquent le supremum de la définition \eqref{EqDefintYfdmu} est strictement positif.

    Nous savons donc que \( \mu(E_i)=0\) pour tout \( i\). Étant donné que la mesure d'une union disjointe dénombrable est égale à la somme des mesures, nous avons
    \begin{equation}
        \mu\{ x\in\Omega\tq f(x)\neq 0 \}=0,
    \end{equation}
    ce qui signifie que \( f\) est nulle \( \mu\)-presque partout.
\end{proof}

\begin{corollary}   \label{CorjLYiSm}
    Soit \( f\) une fonction mesurable sur l'espace mesuré \( (\Omega,\tribA,\mu)\) telle que
    \begin{equation}
        \int_{\Omega}f\mtu_{f>0}d\mu=0.
    \end{equation}
    Alors \( f\leq 0\) presque partout.
\end{corollary}

\begin{proof}
    Nous avons l'égalité d'ensembles
    \begin{equation}
        \{ f\mtu_{f>0}\neq 0 \}=\{ \mtu_{f>0}\neq 0 \}.
    \end{equation}
    Mais lemme~\ref{Lemfobnwt} implique que \( f\mtu_{f>0}\) est nulle presque partout, c'est-à-dire que la mesure de l'ensemble du membre de gauche est nulle par conséquent
    \begin{equation}
        \mu\{ \mtu_{f>0}\neq 0 \}=0.
    \end{equation}
    Cela signifie que la fonction \( f\) est presque partout négative ou nulle.
\end{proof}

%---------------------------------------------------------------------------------------------------------------------------
\subsection{Permuter limite et intégrale}
%---------------------------------------------------------------------------------------------------------------------------

%---------------------------------------------------------------------------------------------------------------------------
\subsubsection{Convergence uniforme}
%---------------------------------------------------------------------------------------------------------------------------

\begin{proposition}[Permuter limite et intégrale]       \label{PropbhKnth}
    Soit \( f_n\to f\) uniformément sur un ensemble mesuré \( A\) de mesure finie. Alors si les fonctions \( f_n\) et \( f\) sont intégrables sur \( A\), nous avons
    \begin{equation}
        \lim_{n\to \infty} \int_A f_n=\int_A \lim_{n\to \infty} f_n.
    \end{equation}
\end{proposition}

\begin{proof}
    Notons \( f\) la limite de la suite \( (f_n)\). Pour tout \( n\) nous avons les majorations
    \begin{subequations}
        \begin{align}
            \left| \int_A f_n d\mu-\int_A fd\mu \right| &\leq \int_A| f_n-f |d\mu\\
            &\leq \int_A \| f_n-f \|_{\infty}d\mu\\
            &=\mu(A)\| f_n-f \|_{\infty}
        \end{align}
    \end{subequations}
    où \( \mu(A)\) est la mesure de \( A\). Le résultat découle maintenant du fait que \( \| f_n-f \|_{\infty}\to 0\).
\end{proof}
Il existe un résultat considérablement plus intéressant que cette proposition. En effet, l'intégrabilité de \( f\) n'est pas nécessaire. Cette hypothèse peut être remplacée soit par l'uniforme convergence de la suite (théorème~\ref{ThoUnifCvIntRiem}), soit par le fait que les normes des \( f_n\) sont uniformément bornées (théorème de la convergence dominée de Lebesgue~\ref{ThoConvDomLebVdhsTf}).

\begin{theorem}[\cite{BJblWiS}]			\label{ThoUnifCvIntRiem}
    La limite uniforme d'une suite de fonctions intégrables sur un borné est intégrable, et on peut permuter la limite et l'intégrale.

    Plus précisément, soit \( A\) un ensemble de \( \mu\)-mesure finie et \( f_n\colon A\to \eR\) des fonctions intégrables sur \( A\). Si la limite \( f_n\to f\) est uniforme, alors \( f\) est intégrable sur \( A\) et nous pouvons inverser la limite et l'intégrale :
    \begin{equation}
        \lim_{n\to \infty} \int_A f_n=\int_A\lim_{n\to \infty} f_n.
    \end{equation}
\end{theorem}

\begin{proof}
    Soit \( \epsilon>0\) et \( n\) tel que \( \| f_n-f \|_{\infty}\leq \epsilon\) (ici la norme uniforme est prise sur \( A\)). Étant donné que \( f_n\) est intégrable sur \( A\), il existe une fonction simple \( \varphi_n\) qui minore \( f_n\) telle que
    \begin{equation}
        \left| \int_{A}\varphi_n-\int_A f_n \right| <\epsilon.
    \end{equation}
    La fonction \( \varphi_n+\epsilon\) est une fonction simple qui majore la fonction \( f\). Si \( \psi\) est une fonction simple qui minore \( f\), alors
    \begin{equation}
        \int_A\psi\leq\int_A\varphi_n+\epsilon\leq\int_A f_n+\epsilon\mu(A).
    \end{equation}
    Par conséquent le supremum qui définit \( \int_A f\) existe, ce qui montre que \( f\) est intégrable. Le fait qu'on puisse inverser la limite et l'intégrale est maintenant une conséquence de la proposition~\ref{PropbhKnth}.
\end{proof}

\begin{remark}
    L'hypothèse sur le fait que \( A\) soit de mesure finie est importante. Il n'est pas vrai qu'une suite uniformément convergente de fonctions intégrables est intégrables. En effet nous avons par exemple la suite
    \begin{equation}
        f_n(x)=\begin{cases}
            1/x    &   \text{si } x<n\\
            0    &    \text{sinon}
        \end{cases}
    \end{equation}
    qui converge uniformément vers \( f(x)=1/x\) sur \( A=\mathopen[ 1 , \infty [\). Le limite n'est cependant guerre intégrable sur \( A\).
\end{remark}

%---------------------------------------------------------------------------------------------------------------------------
\subsubsection{Convergence dominée de Lebesgue}
%---------------------------------------------------------------------------------------------------------------------------

\begin{theorem}[Convergence dominée de Lebesgue]        \label{ThoConvDomLebVdhsTf}
    Soit une suite de fonctions \( (f_n)_{n\in \eN}\) sur \( (\Omega,\tribA,\mu)\) à valeurs dans \( \eC\) ou \( \eR\). Nous supposons que
    \begin{enumerate}
        \item
            Pour chaque \( n\) nous avons \( f_n\in L^1(\Omega,\tribA,\mu)\),
        \item
            \( f_n\to f\) simplement presque partout sur \( \Omega\),
        \item 
            Il existe une fonction \( g\in L^1(\Omega)\) telle que
            \begin{equation}
                | f_n(x) | \leq g(x)
            \end{equation}
            pour presque\footnote{S'il n'y avait pas le «presque» ici, ce théorème serait à peu près inutilisable en probabilité ou en théorie des espaces \( L^p\), comme dans la démonstration du théorème de Fischer-Riesz~\ref{ThoGVmqOro} par exemple.} tout \( x\in\Omega\) et pour tout \( n\in \eN\). 
    \end{enumerate}
    Alors
    \begin{enumerate}
        \item
            \( f\) est intégrable,
        \item
           $\lim_{n\to \infty} \int_{\Omega}f_n=\int_\Omega f$,
        \item
            $\lim_{n\to \infty} \int_{\Omega}| f_n-f |=0$.
    \end{enumerate}
\end{theorem}
\index{théorème!convergence!dominée de Lebesgue}
\index{dominée!convergence (Lebesgue)}
\index{permuter!limite et intégrale!convergence dominée}

\begin{proof}

    La fonction limite \( f\) est intégrable parce que \( | f |\leq g\) et \( g\) est intégrable\footnote{Par le lemme \ref{LemPfHgal}}. Par hypothèse nous avons
    \begin{equation}
        -g(x)\leq f_n(x)\leq g(x).
    \end{equation}
    En particulier la fonction \( g_n=f_n+g\) est positive et mesurable si bien que le lemme de Fatou~\ref{LemFatouUOQqyk} implique
    \begin{equation}
        \int_{\Omega}\liminf g_n\leq\liminf\int_{\Omega}g_n.
    \end{equation}
    Évidemment nous avons \( \liminf g_n=f+g\), de telle sorte que
    \begin{equation}
        \int f+\int g\leq \liminf\int g_n=\liminf\int f_n+\int g,
    \end{equation}
    et le nombre \( \int g\) étant fini, nous pouvons le retrancher des deux côtés de l'inégalité :
    \begin{equation}
        \int f\leq\liminf\int f_n.
    \end{equation}
    Afin d'obtenir une minoration de \( \int f\) nous refaisons exactement le même raisonnement en utilisant la suite de fonctions \( k_n=-f_n\to k=-f\). Nous obtenons que
    \begin{equation}
        \int k\geq\liminf\int k_n=-\limsup\int f_n,
    \end{equation}
    et par conséquent
    \begin{equation}    \label{IneqsndMYTO}
        \liminf\int f_n\leq\int f\leq\limsup\int f_n.
    \end{equation}
    La limite supérieure étant plus grande ou égale à la limite inférieure, les trois quantités dans les inégalités \eqref{IneqsndMYTO} sont égales.

    Nous prouvons maintenant le troisième point. Soit la suite de fonctions
    \begin{equation}
        h_n(x)=| f_n(x)-f(x) |
    \end{equation}
    qui tend ponctuellement vers zéro. De plus
    \begin{equation}
    h_n(x)\leq | f_n(x) |+| f(x) |\leq 2g(x),
    \end{equation}
    ce qui prouve que les \( h_n\) majorés par une fonction intégrable. Donc
    \begin{equation}
        \lim_{n\to \infty} \int_{\Omega}| f_n-f |= \lim_{n\to \infty} \int_{\Omega}h_n(x)dx=\int_{\Omega}\lim_{n\to \infty} | f_n(x)-f(x) |=0
    \end{equation}
\end{proof}

\begin{remark}
    Lorsque nous travaillons sur des problèmes de probabilités, la fonction \( g\) peut être une constante parce que les constantes sont intégrables sur un espace de probabilité.
\end{remark}

\begin{corollary}       \label{CorCvAbsNormwEZdRc}
    Soit \( (a_i)_{i\in \eN}\) une suite numérique absolument convergente. Alors elle est convergente. Il en est de même pour les séries de fonctions si on considère la convergence ponctuelle.
\end{corollary}

\begin{proof}
    L'hypothèse est la convergence de l'intégrale \( \int_{\eN}| a_i |dm(i)\) où \( dm\) est la mesure de comptage. Étant donné que \( | a_i |\leq | a_i |\), la fonction \( a_i\) (fonction de \( i\)) peut jouer le rôle de \( g\) dans le théorème de la convergence dominée de Lebesgue (théorème~\ref{ThoConvDomLebVdhsTf}).
\end{proof}

%--------------------------------------------------------------------------------------------------------------------------- 
\subsection{Additivité de l'intégrale de Lebesgque}
%---------------------------------------------------------------------------------------------------------------------------

Les propositions \ref{PROPooTFOAooJBwmCV} et \ref{PROPooDWYNooWKJmEV} démontrent la même chose. La différence est la méthode utilisée pour permuter une somme et une intégrale. Dans le premier cas, nous utilisons la convergence monotone (et sommes obligés de séparer le cas où \( f\) est positive), alors que dans le second cas, nous utilisons la convergence dominée de Lebesgue, et nous ne devons pas faire de séparation d'après la positivité de \( f\).

\begin{proposition}[\( \sigma\)-additivité dénombrable\cite{MonCerveau}]      \label{PROPooTFOAooJBwmCV}
    Si \( (A_i)_{i\in \eN} \) sont des parties mesurables disjointes de \( (\Omega,\tribA,\mu)\) et si \( f\colon \Omega\to \eR\) est intégrable sur \( \bigcup_{i=0}^{\infty}A_i\)  alors les intégrales \( \int_{A_i}fd\mu\) existent et
    \begin{equation}
        \int_{\bigcup_iA_i}fd\mu=\sum_{i=0}^{\infty}\int_{A_i}fd\mu.
    \end{equation}
\end{proposition}

\begin{proof}
    En deux cas d'après la positivité de \( f\).
    \begin{subproof}
        \item[Si \( f\) est positive]
            Nous posons \( f_N=f\mtu_{\bigcup_{i=0}^NA_i}\). Cette suite de fonctions vérifie la limite
            \begin{equation}
                \lim_{N\to \infty} f_N=f\mtu_{\bigcup_{i=0}^{\infty}}.
            \end{equation}
            De plus, pour chaque \( N\) nous avons
            \begin{equation}
                \int_{\Omega}f_N=\int_{\Omega}f\mtu_{\bigcup_iA_i}=\int_{\bigcup_iA_i}f=\sum_{i=0}^N\int_{A_i}f
            \end{equation}
            Justifications:
            \begin{itemize}
                \item La proposition \ref{LEMooSPOFooBxDEAV} pour l'introduction de la fonction caractéristique de \( \bigcup_iA_i\)
                \item La proposition \ref{PropOPSCooVpzaBt} qui traite le cas de la sous-additivité finie pour la dernière égalité.
            \end{itemize}
            La suite \( (f_N)_{n\in \eN}\) est une suite croissante de fonctions mesurables\footnote{La fonction \( f\) elle-même est mesurable; c'est inclus à la définition de «intégrable».} et positives. Donc le théorème de la convergence monotone \ref{ThoRRDooFUvEAN} s'applique et
            \begin{subequations}
                \begin{align}
                \sum_{i=0}^{\infty}\int_{A_i}f&=\lim_{N\to \infty} \sum_{i=0}^N\int_{A_i}f
                =\lim_{N\to \infty} \int_{\bigcup_{i=0}^{N}A_i}f
                =\lim_{N\to \infty} \int_{\Omega}f_N\\
                &=\int_{\Omega}\lim_{N\to \infty} f_N
                =\int_{\Omega}f\mtu_{\bigcup_{i=0}^{\infty}}
                =\int_{\bigcup_iA_i}fd\mu.
                \end{align}
            \end{subequations}
        \item[Si \( f\) est à valeurs réelles]
            Si \( f\) est à valeurs dans \( \eR\), alors \( f=f_{+}-f_{-}\) où \( f_{+}\) et \( f_{-}\) sont intégrables. Nous avons alors
            \begin{subequations}
                \begin{align}
                    \int_{\bigcup_{i=0}^{\infty}A_i}fd\mu&=\int_{\bigcup_iA_i}f_{+}-\int_{\bigcup_iA_i}f_{-}\\
                    &=\sum_{k=0}^{\infty}\int_{A_k}f_+-\int_{k=0}^{\infty}f_-\\
                    &=\sum_{k=0}^{\infty}\big( \int_{A_k}f_+-\int_{A_k}f_- \big)        \label{SUBEQooMTZPooLqMHKP}\\
                    &=\sum_{k=0}^{\infty}\int_{A_k}f.       \label{SUBEQooVZNMooRmFoLq}
                \end{align}
            \end{subequations}
            Justifications :
            \begin{itemize}
                \item Pour \eqref{SUBEQooMTZPooLqMHKP}, c'est l'associativité de la somme, proposition \ref{PROPooUEBWooUQBQvP}.
                \item Pour \eqref{SUBEQooVZNMooRmFoLq}, c'est la proposition \ref{PROPooFIYEooCpdmwZ}.
            \end{itemize}
        \end{subproof}
\end{proof}

\begin{proposition}[\( \sigma\)-additivité\cite{BIBooAWGNooUzLMUB}]     \label{PROPooDWYNooWKJmEV}
    Soit un espace mesuré \( (\Omega,\tribA,\mu)\). Nous considérons des parties disjointes \( \{ A_i \}_{i\in \eN}\) de \( \Omega\) telles que \( \bigcup_{k=0}^{\infty}A_k=\Omega\). Si \( f\in L^1(\Omega)\), alors
    \begin{equation}
        \int_{\Omega}fd\mu=\sum_{k=0}^{\infty}\int_{A_k}fd\mu.
    \end{equation}
\end{proposition}

\begin{proof}
    Nous posons \( \Omega_n=\bigcup_{k=0}^nA_k\) ainsi que \( f_n=f\mtu_{\Omega_n}\). Pour chaque \( N\in \eN\) nous avons
    \begin{subequations}        \label{EQSooBREOooWzviSK}
        \begin{align}
            \sum_{k=0}^N\int_{A_k}fd\mu&=\int_{\bigcup_{k=0}^NA_k}f     \label{EQooCVVVooTIINmz}\\
            &=\int_{\Omega_N}f\\
            &=\int_{\Omega}f_N.     \label{SUBEQooJZLQooKlOoes}
        \end{align}
    \end{subequations}
    Justifications :
    \begin{itemize}
        \item Pour \eqref{EQooCVVVooTIINmz}, c'est la proposition \ref{PropOPSCooVpzaBt} qui traite du cas de sommes finies.
        \item Pour \eqref{SUBEQooJZLQooKlOoes}  c'est la proposition \ref{PROPooTFOAooJBwmCV}.
    \end{itemize}
    L'idée est maintenant de passer à la limite des deux côtés de \eqref{EQSooBREOooWzviSK}. Voici le raisonnement :
    \begin{itemize}
        \item Nous montrons qu'à droite, la limite existe et vaut \( \int_{\Omega}fd\mu\).
        \item Le fait que la limite du membre de droite existe implique l'existence de la limite du membre de gauche.
        \item La limite du membre de gauche vaut \( \sum_{k=0}^{\infty}\int_{A_k}fd\mu\).
    \end{itemize}
    La limite du membre de droite s'établi avec le théorème de la convergence dominée de Lebesgue \ref{ThoUnifCvIntRiem}.
    \begin{itemize}
        \item Nous avons convergence simple \( f_n\to f\) parce que \( \bigcup_{n=0}^{\infty}A_i=\Omega\).
        \item La fonction \( g=| f |\) est intégrable sur \( \Omega\) parce que \( f\in L^1(\Omega)\) par hypothèse.
        \item Pour tout \( n\in \eN\) et pour tout \( x\in \Omega\) nous avons \( | f_n(x) |\leq g(x)\) parce que \( | f_n(x) |\) est soit égal à \( g(x)\) soit égal à zéro suivant que \( x\in \Omega_n\) ou non.
    \end{itemize}
    Donc le théorème de la convergence dominée est applicable. La limite du membre de droite de \eqref{EQSooBREOooWzviSK} existe et vaut :
    \begin{equation}
        \lim_{N\to \infty} \int_{\Omega}f_N=\int_{\Omega}f.
    \end{equation}
    Nous pouvons alors prendre aussi la limite du membre de gauche dans \eqref{EQSooBREOooWzviSK} et obtenir le résultat attendu.
\end{proof}

%---------------------------------------------------------------------------------------------------------------------------
\subsection{Produit d'une mesure par une fonction (mesure à densité)}
%---------------------------------------------------------------------------------------------------------------------------

\begin{propositionDef}[Produit d'une mesure par une fonction\cite{MonCerveau,ooGMNAooSLnIio}]\label{PropooVXPMooGSkyBo}
    Soit un espace mesuré \( (\Omega,\tribF,\mu)\) et une fonction mesurable positive \( w\colon \Omega\to \bar\eR^+\). Alors la formule
    \begin{equation}
        (w\cdot \mu)(A)=\int_Awd\mu
    \end{equation}
    pour tout \( A\in \tribF\) définit une mesure positive sur \( (\Omega,\tribF)\) appelée \defe{produit}{produit!d'une mesure par une fonction} de la mesure \( \mu\) par la fonction \( w\). La fonction \( w\) est la \defe{densité}{densité!mesure} de la mesure \( w\cdot \mu\) par rapport à la mesure \( \mu\).
\end{propositionDef}

\begin{proof}
    D'abord \( (w\cdot \mu)(\emptyset)=0\) parce que le lemme~\ref{LemooPJLNooVKrBhN} donne
    \begin{equation}
        (w\cdot \mu)(\emptyset)=\int_{\Omega}w\mtu_{\emptyset}d\mu=\int_{\Omega}0d\mu=0\times \mu(\Omega)=0
    \end{equation}
    où nous avons (éventuellement) utilisé deux fois la convention \( 0\times \infty=0\).


    Ensuite si les ensembles \( A_i\) sont des éléments deux à deux disjoints de \( \tribF\) alors nous avons \( \mtu_{\bigcup_{i=1}^{\infty}A_i}=\sum_{i=1}^{\infty}\mtu_{A_i}\), et donc
    \begin{equation}
        (w\cdot \mu)(\bigcup_{i=0}^{\infty}A_i) = \int_{\bigcup_{i=0}^{\infty}A_i}wd\mu=\sum_{i=0}^{\infty}\int_{A_i}wd\mu=\sum_{i=0}^{\infty}(w\cdot\mu)(A_i).  \end{equation}
    où nous avons utilisé la \( \sigma\)-additivité dénombrable de l'intégrale de la proposition \ref{PROPooTFOAooJBwmCV}.
\end{proof}

En particulier nous parlons souvent de mesure à densité par rapport à la mesure de Lebesgue. C'est alors la construction suivante. 

\begin{definition}
    Si \( \mu\) est une mesure sur \( \eR^d\), une fonction \( f\colon \eR^d\to \eR\) est une \defe{densité}{densité d'une mesure} pour \( \mu\) si pour tout \( A\subset\eR^d\) nous avons
    \begin{equation}
        \mu(A)=\int_Af(x)dx
    \end{equation}
    où \( dx\) est la mesure de Lebesgue.

    Si la mesure \( \mu\) admet une densité, nous disons que c'est une \defe{mesure à densité}{mesure à densité} par rapport à la mesure de Lebesgue.
\end{definition}

\begin{example}
    Toutes les mesures n'admettent pas de densité. Par exemple la mesure de Dirac donnée par
    \begin{equation}        \label{EQooDMFCooVEManF}
        \nu(A)=\begin{cases}
            1    &   \text{si } 0\in A\\
            0    &    \text{sinon. }
        \end{cases}
    \end{equation}
    n'a pas de densité par rapport à la mesure de Lebesgue.
\end{example}

La meure \( \nu\) de l'exemple \ref{EQooDMFCooVEManF} admet, au sens des distributions, la mesure de Dirac \( \delta\) comme densité, mais c'est une autre histoire qui vous sera contée une autre fois.

\begin{proposition}[\cite{ooGMNAooSLnIio}]  \label{PropooJMWAooDzfpmB}
    Soit une fonction mesurable \( w\colon (S,\tribF,\mu)\to \bar \eR^+\).
    \begin{enumerate}
        \item
            Si $f\colon S\to \bar\eR^+$ est mesurable, alors \( f\cdot(w\cdot \mu)=(fg)\cdot \mu\).
        \item
            Si \( f\colon S\to \bar \eR\) ou \( \eC\) est mesurable, elle est \( w\cdot\mu\)-intégrable si et seulement si \( fw\) est \( \mu\)-intégrable. Dans ce cas, nous avons encore \( f\cdot(w\cdot \mu)=(fg)\cdot\mu\).
    \end{enumerate}
    Attention : dans le cas où \( f\) est à valeurs dans \( \eC\), alors il faut que \( w\) soit à valeurs finies dans \( \eR\) parce que nous n'avons pas définit \( \infty\times z\) lorsque \( z\in \eC\).
\end{proposition}

\begin{proof}
    Nous commençons par prouver le résultat pour la fonction caractéristique de l'ensemble mesurable \( A\). Nous avons : $\mtu_A\cdot(w\cdot \mu)(B)=\int_B\mtu_Ad(w\cdot \mu)$. Mais par définition, l'intégrale d'une fonction indicatrice est la mesure de l'ensemble indiqué. En passant sur le fait que \( \mtu_A\mtu_B=\mtu_{A\cap B}\),
    \begin{equation}
        \int_B\mtu_Ad(w\cdot \mu)=   (w\cdot\mu)(A\cap B)=\int_S\mtu_{A\cap B}wd\mu=\int_S\mtu_A\mtu_Bwd\mu=\int_B\mtu_Awd\mu=(\mtu_Aw)\cdot\mu(B).
    \end{equation}

    Supposons maintenant que \( f\) soit une fonction étagées qui s'écrit \( f=\sum_ka_k\mtu_{A_k}\) où les \( A_k\) sont des ensembles mesurables disjoints. Alors le calcul est le suivant, en utilisant le fait que sur \( A_k\), on a \( a_k=f(x)\) :
    \begin{subequations}
        \begin{align}
            f\cdot(g\cdot \mu)B&=\int_Bfd(g\cdot \mu)\\
            &=\sum_ka_k(g\cdot\mu)(A_k\cap B)\\
            &=\sum_ka_k\int_{A_k\cap B}gf\mu\\
            &=\int_{A_k\cap}f(x)g(x)d\mu(x)\\
            &=\sum_k(fg\cdot\mu)(A_k\cap B)\\
            &=(fg\cdot\mu)(B)
        \end{align}
    \end{subequations}
    parce que les \( A_k\cap B\) forment une partition de l'ensemble \( B\) (voir le point~\ref{ItemQFjtOjXiii} de la définition~\ref{DefBTsgznn}).

    Si \( f\colon S\to \bar\eR^+\) est mesurable, le théorème~\ref{THOooXHIVooKUddLi} donne une suite croissante \( f_n\) de fonctions étagées positives convergeant (ponctuellement) vers \( f\). Vu que la fonction \( w\) est positive, nous avons aussi la limite positive et croissante \( wf_n\to wf\). Ainsi l'utilisation du théorème de la convergence monotone est justifié dans le calcul suivant :
    \begin{equation}
        \int_Sfd(w\cdot \mu)=\lim_{n\to \infty} \int_Sf_nd(w\cdot\mu)=\lim_{n\to \infty} \int_S(wf_n)d\mu=\int_Swfd\mu.
    \end{equation}

    Nous passons maintenant au cas général où \( f\) est une fonction à valeurs dans \( \bar\eR\) ou \( \eC\) (avec \( w\) finie dans ce dernier cas). Nous avons la chaine d'équivalences
    %\begin{itemize}{$\Leftrightarrow$}
    \begin{itemize}
            \renewcommand{\labelitemi}{$\Leftrightarrow$}
        \item \( f\) est \( (w\cdot\mu)\) intégrable
        \item \( | f |\) est \( (w\cdot\mu)\)-intégrable
        \item \( | f |w\) est \( \mu\)-intégrable
        \item \( | fw |\) est \( \mu\)-intégrable.
    \end{itemize}

    Si cela est le cas, la formule se démontre en se ramenant au cas déjà prouvé des fonctions positives en utilisant les \( (fw)^+=f^+w\), \( (fw)^-=f^-w\) etc.
\end{proof}

%---------------------------------------------------------------------------------------------------------------------------
\subsection{Mesure et topologie}
%---------------------------------------------------------------------------------------------------------------------------

\begin{example}[Un compact n'est pas toujours de mesure finie]      \label{EXooKQDRooVMWaEC}
    Soit l'espace mesurable \( (\eR,\Borelien(\eR))\) réel avec ses boréliens et la fonction
    \begin{equation}
        \begin{aligned}
            w\colon \big( \eR,\Borelien(\eR) \big)&\to\big( \bar \eR,\Borelien(\bar \eR) \big) \\
            x&\mapsto \begin{cases}
                \frac{1}{ | x | }    &   \text{si } x\neq 0\\
                +\infty    &    \text{si }x=0.
            \end{cases}
        \end{aligned}
    \end{equation}
    Essayons d'étudier la mesure de densité \( w\) par rapport à la mesure de Lebesgue.
    \begin{subproof}
    \item[\( w\) est mesurable]
    Soit un borélien \( B\) de \( \bar \eR\). Si \( B\) ne contient pas \( \infty\) alors \( w^{-1}(B)\) est un borélien de \( \eR\) par continuité de l'application restreinte \( w\colon \eR\setminus\{ 0 \}\to \eR \). Ici nous avons par exemple appliqué la proposition~\ref{PropooLNBHooBHAWiD} à chacun des deux intervalles \( \mathopen] -\infty , 0 \mathclose[\) et \( \mathopen] 0 , \infty \mathclose[\). Si \( +\infty\in B\) alors
        \begin{equation}
            w^{-1}(B)=w^{-1}\big( B\setminus\{ 0 \} \big)\cup w^{-1}(\{ \infty \})=  w^{-1}\big( B\setminus\{ 0 \} \big)\cup \{ 0 \},
        \end{equation}
        qui est borélien par union de boréliens.
    \item[Mesure produit]
    La proposition~\ref{PropooVXPMooGSkyBo} nous assure alors qu'en posant\footnote{Avec un mini abus de notation : si \( 0\in B\), cette notation n'est pas tout à fait correcte.}
    \begin{equation}
        \mu(B)=\int_B\frac{1}{ | x | }d\lambda(x)
    \end{equation}
    où \(  \lambda \) est la mesure de Lebesgue, nous avons une mesure.

\item[Mesure du singleton]

    Pour avoir les idées claires, nous pouvons nous demander la mesure \( \mu\big( \{ 0 \} \big)\). Nous cela nous devons calculer
    \begin{equation}
        \int_{\{ 0 \}}\frac{1}{ | x | }d\lambda(x)=\int_{\{ 0 \}}w(x)d\lambda(x)
    \end{equation}
    où là, l'abus de notation n'est plus possible. Mais quelle que soit la fonction étagée \( h=\sum_i\alpha_i\caract_{A_i}\) considérée,
    \begin{equation}
        \int_{\{ 0 \}}h(x)d\lambda(x)=\sum_i\alpha_i\lambda\big( A_i\cap\{ 0 \} \big)=0.
    \end{equation}

    Attention : ceci n'a rien de particulier à la fonction \( x\mapsto 1/| x |\). Lorsqu'une mesure a une densité par rapport à Lebesgue, la mesure d'un singleton sera toujours nulle.

\item[Mesure de la boule compacte]

    Il n'en reste pas moins que \( \mu\big( \mathopen[ -1 , 1 \mathclose] \big)=\infty\).

    \end{subproof}
\end{example}

\begin{normaltext}
     En réalité, il n'y a pas de liens forts entre mesure et topologie. Un espace topologique est une chose, et y mettre une mesure en est une autre. Bien entendu, une topologie étant donnée, nous pouvons considérer la tribu des boréliens et y mettre une mesure un peu quelconque. Il n'y a pas de choix canonique.

     Notons que même dans l'exemple de compact de mesure infinie~\ref{EXooKQDRooVMWaEC}, la mesure introduite n'est pas sans lien avec la topologie de \( \eR\). En effet pour avoir une mesure à densité par rapport à Lebesgue, nous avons dû prendre une application mesurable par rapport à la tribu des boréliens, laquelle est éminemment liée à la topologie. Il y a donc parfaitement moyen de construire des espaces mesurés tenant compte de la topologie, et ayant des propriétés qui ne sont pas celle attendues.
\end{normaltext}

Quand les choses sont faciles, ça se passe bien. La proposition suivante dit qu'une fonction continue sur un compact y est intégrable; sauf que pour dire cela de façon précise, il faut un peu bosser parce qu'il y a de écueils à éviter, tels que l'exemple \ref{EXooKQDRooVMWaEC}.
\begin{proposition}[\cite{MonCerveau}]      \label{PROPooKFRSooANzglT}
    Soit un espace mesuré \( (K,\tribA, \mu)\) et une fonction \( f\colon K\to \eR\). Nous supposons pas mal de trucs techniques :
    \begin{enumerate}
        \item
            La mesure est finie : \( \mu(K)<\infty\). 
        \item
            L'ensemble \( K\) est par ailleurs un espace topologique compact\footnote{Nous ne prétendons pas que la tribu \( \tribA\) soit liée à la topologie de \( K\).}. 
        \item   \label{ITEMooBKYHooWnxUGL}
            La fonction \( f\) est continue pour les topologies de \( K\) et de \( \eR\).
        \item   \label{ITEMooJCNUooJzIlKI}
            La fonction \( f\) est mesurable pour la tribu \( \tribA\) de \( K\) et la tribu des boréliens de \( \eR\).
    \end{enumerate}
    Alors \( f\) est intégrable sur \( K\) et \( \int_K| f |<\infty\).
\end{proposition}

L'hypothèse \ref{ITEMooJCNUooJzIlKI} ne se déduit pas nécessairement de l'hypothèse \ref{ITEMooBKYHooWnxUGL}. Dans les cas usuels, nous avons bien «continue implique mesurable», mais si \( \tribA\) n'a aucun rapport avec la topologie \ldots hum \ldots

\begin{proof}
    Si nous écrivons \( f(x)=f^+(x)-f^-(x)\) avec \( f^+\) et \( f^-\) prenant des valeurs positives ou nulles\cite{ooFSBCooVpuWaV}, en vertu de la proposition \ref{DefTCXooAstMYl}, si nous devons prouver séparément \( \int_Kf^+<\infty\) et \( \int_Kf^-<\infty\). Nous allons donc prouver cette proposition en plusieurs étapes.
    \begin{subproof}
        \item[Si \( f\) est positive]
            La fonction \( f\) est continue sur \( K\) qui est compact (même en tant qu'espace topologique en soi; il n'est pas nécessaire d'être compact \emph{dans} quelque chose), donc elle a un maximum par le théorème \ref{ThoWeirstrassRn} nommons \( M\) ce maximum. Donc \( f\colon K\to \mathopen[ 0 , M \mathclose]\). De plus la mesure \( \mu\) sur \( K\) est finie et vérifie disons \( \mu(K)=m\). 

            Soit une fonction étagée \( h\colon K\to \eR^+\) majorée par \( f\). Nous notons
            \begin{equation}
                h(x)=\sum_{i=1}^n\alpha_i\mtu_{A_i}(x)
            \end{equation}
            où les \( A_i\) sont des éléments de \( \tribA\). Vu que \( 0\leq h(x)\leq f(x)\leq M\), nous avons\footnote{Définition \eqref{EqooGAFMooZLzjPs}.}
            \begin{equation}
                \int_Kh=\sum_{i=1}^n\alpha_i\mu(K\cap A_i)\leq \sum_{i=1}^nM\mu(K\cap A_i)\leq M= \mu(K)=Mm
            \end{equation}
            parce que les \( A_i\) sont disjoints et vérifient \( \bigcup_iA_i=K\) (lemme \ref{LEMooNWLTooCDuRQI}).

            Donc tous les éléments de l'ensemble sur lequel nous prenons le supremum dans la définition \eqref{EqDefintYfdmu} sont contenus dans \( \mathopen[ 0 , Mm \mathclose]\). Le supremum est donc dans \( \mathopen[ 0 , Mm \mathclose]\) et est alors strictement plus petit que l'infini.

        \item[Si \( f\) est positive ou négative]
            Nous appliquons la première partie séparément à \( f^+\) et \( f^-\). Et nous avons alors que \( f\) est intégrable et
            \begin{equation}
                \int_K| f |=\int_Kf^++\int_Kf^-<\infty.
            \end{equation}
    \end{subproof}
\end{proof}

%+++++++++++++++++++++++++++++++++++++++++++++++++++++++++++++++++++++++++++++++++++++++++++++++++++++++++++++++++++++++++++
\section{Propriétés des intégrales}
%+++++++++++++++++++++++++++++++++++++++++++++++++++++++++++++++++++++++++++++++++++++++++++++++++++++++++++++++++++++++++++

\begin{theorem}[\cite{ooGMNAooSLnIio}]      \label{THOooVADUooLiRfGK}
    Soient deux espaces mesurables \( (S_1,\tribF_1)\) et \( (S_2,\tribF_2)\) ainsi qu'une application mesurable \( \varphi\colon S_1\to S_2\). Soit encore \( \mu\), une mesure positive sur \( (S_1,\tribF_1)\).

    Si \( f\colon S_2\to\bar \eR\) ou \( \eC\) est mesurable alors,
    \begin{enumerate}
        \item      \label{ItemooKMBIooZpHJSS}
            \( f\) est \( \varphi(\mu)\)-intégrale si et seulement si \( f\circ\varphi\) est \( \mu\)-intégrable.
        \item       \label{ItemooLAPYooUreDEl}
            dans le cas où \( f\) est \( \varphi(\mu)\)-intégrable, nous avons
            \begin{equation}        \label{EqooSOHXooXSbdoy}
                \int_{S_2}fd\big( \varphi(\mu) \big)=\int_{S_1}(f\circ\varphi)d\mu.
            \end{equation}
    \end{enumerate}
\end{theorem}

\begin{proof}
    L'intégrabilité est la définition~\ref{DefTCXooAstMYl}, et demande que \( | f |\) soit intégrable. L'égalité \eqref{EqooSOHXooXSbdoy} a un sens si les deux membres sont infinis. Tant que les fonctions considérées sont positives, le point~\ref{ItemooKMBIooZpHJSS} est immédiat. Ce n'est qu'au moment où les fonctions considérées deviennent à valeurs dans \( \eC\) ou \( \eR\) que l'intégrabilité de \( | f |\) commence à jouer parce qu'il faut que \(  f^+  \) et \( f^-\) soient séparément intégrables.

    Nous allons prouver la formule \eqref{EqooSOHXooXSbdoy} pour des fonctions de plus en plis générales. Pour la suite nous notons \( \mu'=\varphi(\mu)\).

    \begin{subproof}
        \item[Pour \( f=\mtu_B\), \( B \) mesurable]
            Soit \( B\in\tribF_2 \). Nous avons \( \mtu_B\circ\varphi=\mtu_{\varphi^{-1}(B)}\). Donc en utilisant le lemme~\ref{LemooPJLNooVKrBhN} nous avons
            \begin{equation}
                \int_{S_2}\mtu_{B}d\mu'=\mu'(B)=\mu\big( \varphi^{-1}(B) \big)=\int_{S_1}\mtu_{\varphi^{-1}(B)}d\mu=\int_{S_1}(\mtu_B\circ \varphi)d\mu.
            \end{equation}
        \item[\( f\) est étagée positive]

            La fonction \( f\) peut être écrite sous la forme
            \begin{equation}
                f=\sum_{k=1}^na_k\mtu_{B_k}
            \end{equation}
            avec \( B_k\in\tribF_2\) et \( a_k\in \eR^+\). Nous avons alors, en utilisant la sous-additivité de l'intégrale du théorème~\ref{ThoooCZCXooVvNcFD}\ref{ITEMooOJRAooQkoQyD},
            \begin{subequations}
                \begin{align}
                    \int_{S_2}fd\mu'&=\sum_ka_k\int_{S_2}\mtu_{B_k}d\mu'\\
                    &=\sum_ka_k\int_{S_1}(\mtu_{B_k}\circ\varphi)d\mu\\
                    &=\int_{S_1}\Big( \sum_ka_k\mtu_{B_k} \Big)\circ \varphi d\mu\\
                    &=\int_{S_1}(f\circ\varphi)d\mu.
                \end{align}
            \end{subequations}
        \item[\( f\) à valeurs dans \( \bar \eR^+\)]

            Vu que \( f\) est mesurable, par le théorème~\ref{THOooXHIVooKUddLi} il existe une suite croissante de fonctions étagées positives convergeant vers \( f\). Soit donc cette suite, \( f_n\colon S_2\to \eR^+\). Les fonctions \( f_n\circ\varphi\) sont étagées et positives et nous avons aussi la limite ponctuelle et croissante \( f_n\circ\varphi\to f\circ\varphi\) parce que \( \varphi\) est continue. Le théorème de la convergence monotone (théorème~\ref{ThoRRDooFUvEAN}) permet d'écrire ceci :
            \begin{equation}
                \int_{S_2}fd\mu'=\lim\int_{S_2}f_nd\mu'= \lim\int_{S_1}(f_n\circ\varphi)d\mu=\int_{S_1}(f\circ\varphi)d\mu.
            \end{equation}
        \item[Pour \( f\colon S_2\to \bar \eR\) ou \( \eC\) ]

            C'est maintenant que l'intégrabilité va jouer. Nous avons \( | f |\circ\varphi=| f\circ\varphi |\), donc
            \begin{equation}
                \int_{S_2}| f |d\mu'=\int_{S_1}| f |\circ\varphi d\mu=\int_{S_1}| f\circ \varphi |d\mu,
            \end{equation}
            ce qui montre que \( f\) est \( \mu'\)-intégrable si et seulement si \( f\circ\varphi\) est \( \mu\)-intégrable.

            De plus si \(f=f^+-f^- \) alors \( f^+\circ\varphi=(f\circ\varphi)^+\), \( f^-\circ\varphi=(f\circ\varphi)^-\), et de façon similaire pour les parties imaginaires et réelles.
    \end{subproof}
\end{proof}

%+++++++++++++++++++++++++++++++++++++++++++++++++++++++++++++++++++++++++++++++++++++++++++++++++++++++++++++++++++++++++++
\section{Mesure à densité}
%+++++++++++++++++++++++++++++++++++++++++++++++++++++++++++++++++++++++++++++++++++++++++++++++++++++++++++++++++++++++++++

%---------------------------------------------------------------------------------------------------------------------------
\subsection{Théorème de Radon-Nikodym}
%---------------------------------------------------------------------------------------------------------------------------

\begin{definition}[\cite{PersoFeng}]
    Soient \( \mu\) et \( \nu\) deux mesures sur l'espace mesurable \( (\Omega,\tribA)\). Nous disons que la mesure \( \mu\) est \defe{dominée}{dominée!mesure} par \( \nu\) si pour tout ensemble mesurable \( A\), \( \nu(A)=0\) implique \( \mu(A)=0\).

    Si \( \nu\) est une mesure positive et \( \mu\) une mesure, nous disons que \( \mu\) est \defe{absolument continue}{mesure!absolument continue} par rapport à \( \nu\) si \( \nu(A)=0\) implique \( \mu(A)=0\). On note aussi \( \mu\ll\nu\)\nomenclature[Y]{$\mu\ll\nu$}{La mesure \( \mu\) est absolument continue par rapport à la mesure \( \nu\).}.
\end{definition}

La mesure \( \mu\) est \defe{portée}{portée!mesure} par l'ensemble \( E\in\tribA\) si pour tout \( A\in\tribA\),
\begin{equation}
    \mu(A)=\mu(A\cap E).
\end{equation}

Nous écrivons que \( \mu\perp\nu\)\nomenclature[Y]{\( \mu\perp\nu\)}{mesures perpendiculaires} s'il existe un ensemble \( E\in\tribA\) tel que \( \mu\) soit porté par \( E\) et \( \nu\) soit porté par \( \complement E\).

\begin{theorem}[Radon-Nikodym\cite{NikoLi}]     \label{THOooEFVUooGKApaV}
    Soient \( \mu\) et \( \nu\) deux mesures \( \sigma\)-finies sur un espace métrisable \( (\Omega,\tribA)\).
    \begin{enumerate}
        \item
            Il existe un unique couple de mesures \( \mu_1\) et \( \mu_2\) telles que
            \begin{enumerate}
                \item
                    \( \mu=\mu_1+\mu_2\)
                \item
                    \( \mu_1\) est dominé par \( \nu\)
                \item
                    \( \mu_2\perp \nu\).
            \end{enumerate}
            Dans ce cas, les mesures \( \mu_1\) et \( \mu_2\) sont positives et \( \sigma\)-finies.
        \item
            À égalité \(  \nu\)-presque partout près, il existe une unique fonction mesurable positive \( f\) telle que pour tout mesurable \( A\),
            \begin{equation}
                \mu_1(A)=\int_Ad\mu_1=\int_{\Omega}\mtu_Afd \nu.
            \end{equation}
        \item
            À égalité \( \nu\)-presque partout près, il existe une unique fonction positive mesurable \( h\) telle que \( \mu_1=h\nu\).
    \end{enumerate}
\end{theorem}
\index{théorème!Radon-Nikodym}

\begin{corollary}   \label{CorZDkhwS}
    Si \( \mu\) es une mesure \( \sigma\)-finie dominée par la mesure \( \sigma\)-finie \( m\), alors \( \mu\) possède une unique fonction de densité.
\end{corollary}

\begin{corollary}       \label{CorDomDens}
    Soient \( \mu\) et \( m\), deux mesures positives \( \sigma\)-finies sur \( (\Omega,\tribA)\). Alors \( m\) domine \( \mu\) si et seulement si \( \mu\) possède une densité par rapport à \( m\).
\end{corollary}

\begin{proof}
    Si \( \mu\) est dominée par \( m\), alors la décomposition \( \mu=\mu+0\) satisfait le théorème de Radon-Nikodym. Par conséquent il existe une fonction \( f\) telle que
    \begin{equation}
        \mu(A)=\int_Afdm.
    \end{equation}
    Cette fonction est alors une densité pour \( \mu\) par rapport à \( m\).

    Pour la réciproque, nous supposons que \( \mu\) a une densité \( f\) par rapport à \( m\), et que \( A\) est une ensemble de \( m\)-mesure nulle :
    \begin{equation}
        m(A)=\int_{\Omega}\mtu_Adm=0.
    \end{equation}
    Cela signifie que la fonction \( \mtu_A\) est \( m\)-presque partout nulle. La fonction produit \( \mtu_Af\) est également nulle \( m\)-presque partout, et par conséquent
    \begin{equation}
        \mu(A)=\int_{\Omega}\mtu_Afdm=0.
    \end{equation}
\end{proof}

\begin{probleme}
    Est-ce que la démonstration de cela ne demande pas la convergence monotone d'une façon ou d'une autre ?
\end{probleme}

%---------------------------------------------------------------------------------------------------------------------------
\subsection{Mesure complexe}
%---------------------------------------------------------------------------------------------------------------------------

\begin{definition}[Mesure complexe\cite{TLRRooOjxpTp}] \label{DefGKHLooYjocEt}
    Si \( (\Omega,\tribA)\) est un espace mesurable, une \defe{mesure complexe}{mesure!complexe} est une application \( \mu\colon \tribA\to \eC\) telle que
    \begin{enumerate}
        \item
            $\mu(\emptyset)=0$,
        \item
            \( \nu\) est sous-additive : si les ensembles \( A_i\in\tribA\), alors \( \sum_i\mu(A_i)=\mu(\bigcup_iA_i)\).
    \end{enumerate}
\end{definition}
Notons que la série $\sum_i\mu(A_i)$ est alors nécessairement absolument convergente. En effet changer l'ordre de la somme ne change pas l'union, et donc ne change pas la valeur de la somme. Si \( \sigma\colon \eN\to \eN\) est une permutation,
\begin{equation}
    \sum_i\mu(A_{\sigma(i)})=\mu\big( \bigcup_iA_{\sigma(i)} \big)=\mu\big( \bigcup_iA_i \big)=\sum_i\mu(A_i).
\end{equation}
Le théorème~\ref{PopriXWvIY} dit alors que la somme doit être absolument convergente.


\begin{theorem}[Radon-NikoDym complexe\footnote{L'histoire du nom de ce théorème est intéressante. Lorsque monsieur et madame Rèmederdonnukodym apprirent que leurs amis, les Rèmedelaboulechevelue avaient appelé leur fils Théo, ils décidèrent d'en faire autant. C'est en souvenir de ces circonstances que monsieur Nikodym (prénommé Radon) décida de faire des math.}]\label{ThoZZMGooKhRYaO}
    Soit \( \mu\) une mesure positive sur \( (\Omega,\tribA)\) et \( \nu\) une mesure complexe. Alors
    \begin{enumerate}
        \item
            Il existe un unique couple de mesures complexes \( \nu_a\), \( \nu_s\) sur \( (\Omega,\tribA)\) tel que
            \begin{enumerate}
                \item
                    \( \nu=\nu_a+\nu_s\)
                \item
                    \( \nu_a\ll\mu\)
                \item
                    \( \nu_s\perp \mu\).
            \end{enumerate}
        \item
            Ces mesures satisfont alors \( \nu_a\perp\nu_s\).
        \item
            Il existe une fonction intégrable \( h\colon \Omega\to \eC\) telle que \( \nu_a=h\mu\).
        \item
            La fonction \( h\) est unique à \( \mu\)-équivalence près.
        \item   \label{ItemDIXOooFqOkgGv}
            Si de plus \( \nu\ll \mu\) alors \( \nu=h\mu\).
    \end{enumerate}
\end{theorem}
\index{théorème!Radon-Nikodym!complexe}
\begin{proof}
    No proof.
\end{proof}

\begin{remark}  \label{RemSYRMooZPBhbQ}
    Le point~\ref{ItemDIXOooFqOkgGv} est souvent utilisé sous la forme
    \begin{equation}
        \nu(A)=\int_{\Omega}\mtu_A(\omega)h(\omega)d\mu(\omega)=\int_{A}h(\omega)d\mu(\omega).
    \end{equation}
\end{remark}

%---------------------------------------------------------------------------------------------------------------------------
\subsection{Théorème d'approximation}
%---------------------------------------------------------------------------------------------------------------------------

\begin{lemma}[\cite{YHRSDGc}]       \label{LEMooCGKXooYWjRwk}
    Soit un espace topologique métrique \( (\Omega,d)\). Nous considérons sa tribu des boréliens\footnote{Définition \ref{DEFooQBQGooTqGdtY}.} \( \Borelien(\Omega)\) ainsi qu'une mesure finie \( \mu\) sur \( \big(\Omega,\Borelien(\Omega)\big)\). 

    Soit un borélien \( A\) de \( \Omega\) et \( \epsilon>0\). 

    Il existe un fermé \( F\) et un ouvert \( V\) de \( \Omega\) tels que
    \begin{enumerate}
        \item
            \( F\subset A\subset V\)
        \item
            \( \mu(V\setminus F)<\epsilon\).
    \end{enumerate}
\end{lemma}

\begin{proof}
    Soit la famille \( \tribD\) des parties \( D\) de \( \Omega\) qui vérifient la propriété suivante : pour tout \( \epsilon>0\), il existe un fermé \( F\) et un ouvert \( V\) de \( \Omega\) tels que \( F\subset D\subset V\) et \( \mu(V\setminus F)<\epsilon\).

    Nous allons prouver que \( \tribD\) est une tribu qui contient tous les ouverts.

    \begin{subproof}
        \item[\( \tribD\) contient les ouverts]
            Soit un ouvert \( D\). Nous posons
            \begin{equation}
                F_n=\{ x\in \Omega\tq d(x,D^c)\geq 2^{-n} \}.
            \end{equation}
            \begin{subproof}
                \item[\( F_n\) est fermé]
                
                    Le lemme \ref{LEMooJNRTooZyKiFC} montre que le complémentaire \( F_n^c\) est ouvert. Donc \( F_n\) est fermé.
                \item[\( D\subset \bigcup_{n\in \eN}F_n\)]
                    Si \( x\in D\), alors il existe \( \delta>0\) tel que \( B(x,\delta)\subset D\) (parce que \( D\) est ouvert). Donc \( d(x,V^c)\geq \delta\). Donc \( x\in F_n\) pour \( 2^{-n}<\delta\).
                \item[\( \bigcup_{n\in \eN}F_n\subset D\)]
                    Si \( x\in F_n\), nous avons \( d(x,D^c)>0\), c'est-à-dire que \( x\) n'est pas dans \( D^c\). Autrement dit, \( x\in D\).
                \item[\( \bigcup_{n\in \eN}F_n = D\)]
                    Nous avons donc l'égalité
                    \begin{equation}
                        D=\bigcup_{n\in \eN}F_n.
                    \end{equation}
            \end{subproof}
            Vu que \( F_n\subset F_{n+1}\), le lemme \ref{LemAZGByEs}\ref{ItemJWUooRXNPci} nous indique que
            \begin{equation}
                \lim_{n\to \infty} \mu(F_n)=\mu\big( \bigcup_{k\in \eN}F_k \big)=\mu(D).
            \end{equation}
            Étant donné que la mesure est finie, nous pouvons écrire cela sous la forme
            \begin{equation}
                \mu(D)-\mu(F_n)\to 0.
            \end{equation}
            Pour chaque \( n\) nous avons l'encadrement
            \begin{equation}
                F_n\subset D\subset D
            \end{equation}
            où \( F_n\) et \( D\) sont ouverts. Lorsque \( \epsilon\) est donné, il suffit de prendre \( n\) assez grand pour avoir \( \mu(D\setminus F_n)<\epsilon\) pour avoir un encadrement de \( D\) par un fermé et un ouvert (\( D\) lui-même) dont la différence des mesures est plus petite que \( e\).

            Tout cela pour dire que \( D\in\tribD\).
        \item[\( \tribD\) est une tribu]
            Il faut vérifier les trois points de la définition \ref{DefjRsGSy}.
            \begin{subproof}
                \item[\( \Omega\in\tribD\)]
                    Nous venons de voir que les ouverts sont dans \( \tribD\). Or \( \Omega\) est un ouvert.
                \item[\( D\in \tribD\) implique \( D^c\in \tribD\)]
                    Soit \( F\) fermé et \( V\) ouvert tels que \( F\subset D\subset V\). Nous avons aussi
                    \begin{equation}
                        V^c\subset D^c\subset F^c
                    \end{equation}
                    où \( V^c\) est fermé et \( F^c\) est ouvert. De plus \( F^c\setminus = V\setminus F\) et donc
                    \begin{equation}
                        \mu(F^c\setminus V^c)=\mu(V\setminus F).
                    \end{equation}
                    Nous pouvons donc choisir \( F\) et \( V\) pour avoir \( \mu(F^c\setminus V^c)<\epsilon\).
                \item[\( \bigcup_{i\in \eN}D_i\in\tribD\)]
                    Soient \( D_i\in t\tribD\). Pour chaque \( n\) nous posons
                    \begin{equation}
                        F_n\subset D_n\subset V_n
                    \end{equation}
                    en choisissant \( V_n\) et \( F_n\) de telle sorte que \( \mu(V_n\setminus F_n)<2^{-n}\epsilon\).

                    Nous posons 
                    \begin{equation}
                        Y_N=\bigcup_{n=0}^NF_n,
                    \end{equation}
                    et
                    \begin{equation}
                        Y=\bigcup_{n=0}^{\infty}F_n.
                    \end{equation}
                    Chacun des \( Y_N\) est fermé en tant qu'union finie de fermés (lemme \ref{LemQYUJwPC}\ref{ItemKJYVooMBmMbG}). Mais \( Y\) ne l'est pas spécialement\footnote{Par exemple \( A_n=\mathopen[ 1/n , 2 \mathclose]\) sont des fermés dont l'union est \( \mathopen] 0 , 2 \mathclose]\) qui n'est pas fermé.}. Le lemme \ref{LemAZGByEs} nous dit cependant que \( \mu(Y)=\lim_{N\to \infty} \mu(Y_N)\).

                    Nous posons
                    \begin{equation}
                        D=\bigcup_{n=0}^{\infty}D_n
                    \end{equation}
                    ainsi que
                    \begin{equation}
                        V=\bigcup_{n\in \eN} V_n.
                    \end{equation}
                    La partie \( V\) est ouverte dans \( \Omega\) comme union d'ouverts (c'est dans le définition d'une topologie). Nous avons, pour tout \( N\), l'encadrement
                    \begin{equation}        \label{EQooOALEooLAHpVi}
                        Y_N=\bigcup_{n=0}^NF_n\subset Y\subset D\subset V.
                    \end{equation}
                    Nous prouvons à présent que \( \lim_{N\to \infty} \mu(V\setminus Y_N)=0\), de telle sorte que l'encadrement \eqref{EQooOALEooLAHpVi} dise que \( D\in\tribD\).

                    D'abord nous avons
                    \begin{equation}        \label{EQooYVVBooCNvSnx}
                        V\setminus Y\subset \bigcup_n(V_n\setminus F_n)
                    \end{equation}
                    parce que si \( x\in V\setminus Y\), alors \( x\in V_i\) pour un certain \( i\), mais vu que \( x\) n'est pas dans \( Y\), il n'est dans aucun des \( F_n\) donc en particulier pas dans \( F_i\) et \( x\in V_n\setminus F_i\).

                    Un peu de calcul :
                    \begin{subequations}
                        \begin{align}
                            \mu(V)-\mu(Y)&= \mu(V\setminus Y)   \label{SUBEQooCSQYooYXBhYy}\\
                            &\leq\mu\big( \bigcup_n(V_n\setminus F_n) \big)     \label{SUBEQooVUCJooHjObZw}\\
                            &\leq \sum_{n=0}^{\infty}\mu(V_n\setminus F_n)      \label{SUBEQooTAGKooTtYtzw}\\
                            &=\sum_{n=0}^{\infty}2^{-n}\epsilon\\
                            &=2\epsilon.        \label{SUBEQooMDAAooXKEajJyi}
                        \end{align}
                    \end{subequations}
                    Justifications:
                    \begin{itemize}
                        \item Pour \eqref{SUBEQooCSQYooYXBhYy}, c'est le lemme \ref{LemPMprYuC}.
                        \item Pour \eqref{SUBEQooVUCJooHjObZw}, c'est \eqref{EQooYVVBooCNvSnx}.
                        \item Pour \eqref{SUBEQooTAGKooTtYtzw}, c'est le lemme \ref{LemPMprYuC}\ref{ITEMooABPYooFQEzqE}.
                        \item Pour \eqref{SUBEQooMDAAooXKEajJyi}, c'est la série géométrique \eqref{EqPZOWooMdSRvY}.
                    \end{itemize}
                    
                    Nous choisissons maintenant \( N\) assez grand pour que \( \mu(Y)-\mu(Y_N)<\epsilon\). Nous avons alors l'encadrement
                    \begin{equation}
                        Y_N\subset Y\subset D\subset V
                    \end{equation}
                    avec 
                    \begin{equation}
                        \mu(V\setminus Y_N)=\mu(V)-\mu(Y_N)=\underbrace{\mu(V)-\mu(y)}_{\leq 2\epsilon}+\mu(Y)-\mu(Y_N)\leq 2\epsilon+\epsilon=3\epsilon.
                    \end{equation}
            \end{subproof}
    \end{subproof}
    Nous avons donc montré que \( \tribD\) était une tribu contenant les ouverts. Donc \( \tribD\) contient tous les boréliens.
\end{proof}

\begin{lemma}[\cite{YHRSDGc}]       \label{LEMooZDFVooFUgFGZ}
    Soit un espace topologique métrique \( (\Omega,d)\). Nous considérons sa tribu des boréliens \( \Borelien(\Omega)\) ainsi qu'une mesure \( \mu\) sur \( \big(\Omega,\Borelien(\Omega)\big)\). 

    Soient un ouvert \( W\subset \Omega\) tel que \( \mu(W)<\infty\) et un borélien \( A\) tel que \( A\subset W\). Soit aussi \( \epsilon>0\).

    Il existe un fermé \( F\) et un ouvert \( V\) tels que 
    \begin{enumerate}
        \item \( \mu(V)<\infty\),
        \item
     \( \mu(V\setminus F)<\epsilon\),
 \item et $F\subset A\subset V$.
    \end{enumerate}
\end{lemma}

\begin{proof}
    Vu que la mesure de \( W\) est finie, nous considérons la mesure finie
    \begin{equation}
        \begin{aligned}
            \nu\colon \Borelien(\Omega)&\to \mathopen[ 0 , \mu(W) \mathclose] \\
            B&\mapsto \mu(B\cap W). 
        \end{aligned}
    \end{equation}
    La partie \( A\) étant borélienne; par le lemme \ref{LEMooCGKXooYWjRwk}, nous avons un fermé \( F\) et un ouvert \( V_1\) ouvert tels que
    \begin{equation}
        F\subset A\subset V_1
    \end{equation}
    et \( \nu(V_1\setminus F)<\epsilon\). Nous posons \( V=V_1\cap W\); vu que \( A\subset W\) et \( A\subset V_1\) nous avons aussi \( A\subset V_1\cap  W\) et donc l'encadrement
    \begin{equation}
        F\subset A\subset V\subset W.
    \end{equation}
    En ce qui concerne la mesure :
    \begin{equation}
        \mu(V\setminus F)=\mu(V)-\mu(F)=\mu(V\cap W)-\mu(F\cap W)=\nu(B)-\nu(F)<\epsilon.
    \end{equation}
\end{proof}

\begin{theorem}[Théorème d'approximation, thème \ref{THEMEooKLVRooEqecQk}\cite{YHRSDGc}]     \label{ThoAFXXcVa}
    Soit un espace topologique métrique \( (\Omega,d)\). Nous considérons sa tribu des boréliens \( \Borelien(\Omega)\) ainsi qu'une mesure \( \mu\) sur \( \big(\Omega,\Borelien(\Omega)\big)\). 

    Soient un ouvert \( W\subset \Omega\) tel que \( \mu(W)<\infty\) et un un borélien \( A\) tel que \( A\subset W\). Soit aussi \( \epsilon>0\).

    Il existe un fermé \( F\subset W\) et une fonction  \( f\in C^0(\Omega,\eR)\) vérifiant
    \begin{enumerate}
        \item
            \( F\subset A\subset W\),
        \item       \label{ITEMooOZVJooSViuds}
            \( f|_F=1\),
        \item       \label{ITEMooIEFSooHXYZrK}
            \( f|_{W^c}=0\)
        \item       \label{ITEMooSOQVooBbvfgy}
            \( \| f-\mtu_A \|_{L^1}<\epsilon\)
    \end{enumerate}
\end{theorem}

\begin{proof}
    Par le lemme \ref{LEMooZDFVooFUgFGZ}, il existe un fermé \( F\) et un ouvert \( V\) tels que
    \begin{equation}
        F\subset A\subset V\subset W
    \end{equation}
    et \( \mu(V\setminus F)<\epsilon\). Nous posons alors
    \begin{equation}
        f(x)=\frac{ d(x,V^c) }{ d(x,V^c)+d(x,F) }.
    \end{equation}
    Le dénominateur de cette expression ne s'annule jamais parce que si \( d(x,V^c)=0\), c'est que \( x\in V^c\). Mais alors \( x\) n'est pas dans \( V\) et donc pas dans \( F\) non plus. La partie \( F\) étant fermée, \( d(x,F)>0\) par lemme \ref{LEMooEQIZooLpsbOe}. De plus la fonction \( f\) est continue par le lemme \ref{LEMooCFGTooIfdcfk}.

    \begin{subproof}
    \item[Pour \ref{ITEMooOZVJooSViuds}]
        Si \( x\in F\), alors \( d(x,F)=0\), et \( f\) devient
        \begin{equation}
            f(x)=\frac{ d(x,V^c) }{ d(x,V^c) }=1
        \end{equation}
    \item[Pour \ref{ITEMooIEFSooHXYZrK}]
        Si \( x\in W^c\), alors \( x\in V^c\) et \( d(x,V^c)=0\) si bien que \( f(x)=0\). 
    \item[Pour \ref{ITEMooSOQVooBbvfgy}]
        Les premiers points montrent que
        \begin{equation}
            \mtu_F\leq f\leq \mtu_V.
        \end{equation}
        Mais nous avons aussi, par ailleurs,
        \begin{equation}
            \mtu_{F}\leq \mtu_Aleq \mtu_V.
        \end{equation}
        Ces deux encadrement, par le lemme \ref{LEMooEGYLooCGrDrl} donnent l'encadrement
        \begin{equation}
            | f-\mtu_A |\leq \mtu_V-\mtu_F.
        \end{equation}
        En ce qui concernent les intégrales nous avons alors
        \begin{subequations}
            \begin{align}
                \int_{\Omega}| \mtu_A-f |&\leq \int_{\Omega}(\mtu_V-\mtu_F)d\mu\\
                &=\mu(V)-\mu(F) \label{SUBEQooVJDXooFtCelQ}\\
                &<\epsilon.
            \end{align}
        \end{subequations}
        Pour \eqref{SUBEQooVJDXooFtCelQ}, c'est le lemme \ref{LemooPJLNooVKrBhN}.
    \end{subproof}
\end{proof}

%+++++++++++++++++++++++++++++++++++++++++++++++++++++++++++++++++++++++++++++++++++++++++++++++++++++++++++++++++++++++++++ 
\section{Produit de mesures}
%+++++++++++++++++++++++++++++++++++++++++++++++++++++++++++++++++++++++++++++++++++++++++++++++++++++++++++++++++++++++++++

\begin{lemma}[Propriété des sections\cite{NBoIEXO}] \label{LemAQmWEmN}
    Soient \( \tribA_1\) et \( \tribA_2\) des tribus sur les ensembles \( \Omega_1\) et \( \Omega_2\). Si \( A\in\tribA_1\otimes\tribA_2\) alors pour tout \( x\in \Omega_1\) et \( y\in\Omega_2\), les ensembles
    \begin{subequations}    \label{subEqCTtPccK}
        \begin{align}
            A_1(y)=\{ x\in\Omega_1\tq (x,y)\in A \}\\
            A_2(x)=\{ y\in\Omega_2\tq (x,y)\in A \}
        \end{align}
    \end{subequations}
    sont mesurables.
\end{lemma}
\index{section!propriété des}

\begin{proof}
    Soit \( y\in\Omega_2\); nous allons prouver le résultat pour \( A_1(y)\). Pour cela nous notons
    \begin{equation}
        S=\{ A\in \tribA_1\otimes\tribA_2\tq \forall y\in\Omega_2, A_1(y)\in\tribA_1 \},
    \end{equation}
    et nous allons noter que \( S\) est une tribu contenant les rectangles. Par conséquent, \( S\) sera égal à \( \tribA_1\otimes \tribA_2\).

    \begin{subproof}
        \item[Les rectangles]

            Considérons le rectangle \( A=X\times Y\) et si \( y\in \Omega_2\) alors
            \begin{equation}
                A_1(y)=\{ x\in \Omega_1\tq (x,y)\in X\times Y \}.
            \end{equation}
            Donc soit \( y\in Y\) alors \( A_1(y)=X\in\tribA_1\), soit \( y\notin Y\) et alors \( A_1(y)=\emptyset\in\tribA_1\).

        \item[Tribu : ensemble complet]

            Nous avons \( \Omega_1\times \Omega_2\in S\) parce que c'est un rectangle.

        \item[Tribu : complémentaire] Soit \( A\in S\). Montrons que \( A^c\in S\). Nous avons d'abord
            \begin{equation}
                (A^c)_1(y)=\{ x\in \Omega_1\tq (x,y)\in A^c \}.
            \end{equation}
            D'autre part
            \begin{equation}
                A_1(y)^c=\{ x\in\Omega_1\tq (x,y)\notin A \}=\{ x\in \Omega_1\tq (x,y)\in A^c \}=(A^c)_1(y).
            \end{equation}
            Vu que \( \tribA_1\) est une tribu et que par hypothèse \( A_1(y)\in\tribA_1\), nous avons aussi \( A_1(y)^c\in S\), et donc \( (A^c)_1(y)\in \tribA_1\), ce qui prouve que \( A^c\in S\).

        \item[Tribu : union dénombrable] Soit une suite \( A_n\in S\). Nous avons
            \begin{subequations}
                \begin{align}
                (\bigcup_nA_n)_1(y)&=\{ x\in\Omega_1\tq (x,y)\in \bigcup_nA_n \}\\
                &=\bigcup_n\{ x\in\Omega_1\tq (x,y)\in A_n \}\\
                &=\bigcup_n(A_n)_1(y),
                \end{align}
            \end{subequations}
            et ce dernier ensemble est dans \( \tribA_1\) parce que c'est une union dénombrable d'éléments de \( \tribA_1\).

    \end{subproof}
    Nous avons donc prouvé que \( S\) est une tribu contenant les rectangles, donc \( S\) contient au moins \( \tribA_1\otimes \tribA_2\).
\end{proof}

\begin{corollary}
    Si \( f\colon \Omega_1\times \Omega_2\to \eR\) est une fonction mesurable\footnote{Définition~\ref{DefQKjDSeC}.} sur \( X\times Y\) alors pour chaque \( y\) dans \( \Omega_2\), la fonction
    \begin{equation}
        \begin{aligned}
            f_y\colon X&\to \eR \\
            x&\mapsto f(x,y)
        \end{aligned}
    \end{equation}
    est mesurable.
\end{corollary}

\begin{proof}
    Soit \( \mO\) un ensemble mesurable de \( \eR\) (i.e. un borélien), et \( y\in \Omega_2\). Nous avons
    \begin{equation}
        f_y^{-1}(\mO)=\{ x\in X\tq f(x,y)\in \mO \}=A_1(y)
    \end{equation}
    où
    \begin{equation}
        A=\{ (x,y)\in \Omega_1\times \Omega_2\tq f(x,y)\in \mO \}=f^{-1}(\mO).
    \end{equation}
    Ce dernier est mesurable parce que \( f\) l'est.
\end{proof}

\begin{theorem}[\cite{NBoIEXO}\footnote{Modèle non contractuel : des notations et la définition de \( \lambda\)-système peuvent varier entre la référence et le présent texte.}]    \label{ThoCCIsLhO}
    Soient \( (\Omega_i,\tribA_i,\mu_i)\) (\( i=1,2\)) deux espaces mesurés \( \sigma\)-finie. Soit \( A\in\tribA_1\otimes \tribA_2\). Alors les fonctions\footnote{Voir la notation du lemme~\ref{subEqCTtPccK}.}
    \begin{subequations}
        \begin{align}
            x\mapsto\mu_2\big( A_2(x) \big)\\
            y\mapsto\mu_1\big( A_1(y) \big)
        \end{align}
    \end{subequations}
    sont mesurables et
    \begin{equation}    \label{EqRKXwsQJ}
        \int_{\Omega_1}\mu_2\big( A_2(x) \big)d\mu_1(x)=\int_{\Omega_2}\mu_2\big( A_1(y) \big)d\mu_2(y).
    \end{equation}
\end{theorem}

\begin{proof}
    Nous supposons d'abord que \( \mu_1\) et \( \mu_2\) sont finies et nous notons \( \tribD\) le sous-ensemble de \( \tribA_1\otimes \tribA_2\) sur lequel le théorème est correct. Nous allons commencer par prouver que \( \tribD\) est un \( \lambda\)-système.

    \begin{subproof}
        \item[\( \lambda\)-système : différence ensembliste]
            Soient \( A,B\in\tribD\) avec \( A\subset B\). Nous avons
            \begin{subequations}
                \begin{align}
                    (B\setminus A)_1(y)&=\{ x\in \Omega_1\tq(x,y)\in B\setminus A \}\\
                    &=\{ x\in \Omega_1\tq(x,y)\in B\}\setminus\{ x\in \Omega_1\tq(x,y)\in  A \}\\
                    &=B_1(y)\setminus A_1(y).
                \end{align}
            \end{subequations}
            Vu que \( A_1(y)\subset B_1(y)\) et que les mesures sont finies le lemme~\ref{LemPMprYuC} nous donne
            \begin{equation}
                \mu_1\big( (B\setminus A)_1(y) \big)=\mu_1\big( B_1(y) \big)-\mu_1\big( A_1(y) \big),
            \end{equation}
            et similairement pour \( 1\leftrightarrow 2\). Les deux fonctions (de \( y\)) à droite étant mesurables, nous avons la mesurabilité de la fonction \( y\mapsto \mu_1\big( (B\setminus A)_1(y) \big)\).

            Prouvons la formule intégrale en nous rappelant que la formule \eqref{EqRKXwsQJ} est supposée correcte pour \( A\) et \( B\) séparément :
            \begin{subequations}
                \begin{align}
                    \int_{\Omega_2}\mu_1\big( (B\setminus A)_1(y) \big)d\mu_2(y)&=\int_{\Omega_2}\mu_1\big( B_1(y) \big)d\mu_2(y)-\int_{\Omega_2}\mu_1\big( A_1(y) \big)d\mu_2(y)\\
                    &=\int_{\Omega_1}\mu_2\big( B_2(x) \big)d\mu_1(x)-\int_{\Omega_1}\mu_2\big( A_2(x) \big)d\mu_1(x)\\
                    &=\int_{\Omega_1}\mu_2\big( (B\setminus A)_2(x) \big)d\mu_1(x).
                \end{align}
            \end{subequations}


        \item[\( \lambda\)-système : limite de suite croissante]

            Soit \( (A_n)\) une suite croissante dans \( \tribD\); nous posons \( B_n=A_n\setminus A_{n-1}\) et \( A_0=\emptyset\) de telle sorte à travailler avec une suite d'ensembles disjoints qui satisfait \( \bigcup_nA_n=\bigcup_nB_n\). Vu que la suite est croissante nous avons \( A_{n-1}\subset A_n\) et donc \( B_n\in\tribD\) par le point déjà fait sur la différence ensembliste. Nous avons :
            \begin{subequations}
                \begin{align}
                    \mu_1\big( (\bigcup_nB_n)_1(y) \big)&=\{ x\in \Omega_1\tq (x,y)\in\bigcup_nB_n \}\\
                    &=\bigcup_n\{ x\in\Omega_1\tq (x,y)\in B_n \}\\
                    &=\bigcup_n (B_n)_1(y).
                \end{align}
            \end{subequations}
            Par conséquent, par la propriété~\ref{ItemQFjtOjXiii} d'une mesure nous avons
            \begin{equation}
                \mu_1\big( (\bigcup_nB_n)_1(y) \big)=\sum_n\mu_1\big( (B_n)_1(y) \big).
            \end{equation}
            En tant que somme de fonctions positives et mesurables, la fonction
            \begin{equation}
                y\mapsto\sum_n\mu_1\big( (B_n)_1(y) \big)
            \end{equation}
            est mesurable par la proposition~\ref{PropFYPEOIJ}. Il faut encore vérifier la formule intégrale. Le gros du boulot est de permuter une somme et une intégrale par le corolaire~\ref{CorNKXwhdz} :
            \begin{subequations}
                \begin{align}
                    \int_{\Omega_2}\sum_n\mu_1\big( (B_n)_1(y) \big)d\mu_2(y)&=\sum_n\int_{\Omega_2}\mu_1\big( (B_n)_1(y) \big)d\mu_2(y)\\
                    &=\sum_n\int_{\Omega_1}\mu_2\big( (B_n)_2(x) \big)d\mu_1(x)\\
                    &=\int_{\Omega_1}\sum_n\mu_2\big( (B_n)_2(x) \big)d\mu_1(x)\\
                    &=\int_{\Omega_1}\mu_2\big( (\bigcup_nB_n)_1(y) \big)d\mu_1(x).
                \end{align}
            \end{subequations}
    \end{subproof}
    Maintenant que \( \tribD\) est un $\lambda$-système contenant les rectangles, le lemme~\ref{LemLUmopaZ} dit que la tribu engendrée par \( \tribD\) (c'est-à-dire \( \tribA_1\otimes \tribA_2\)) est le $\lambda$-système \( \tribD\) lui-même.

    La preuve est finie dans le cas de mesures finies. Nous commençons maintenant à prouver dans le cas où les mesures \( \mu_1\) et \( \mu_2\) sont seulement \( \sigma\)-finies. Nous considérons des suites croissantes \( \Omega_{i,n}\to\Omega_i\) d'ensembles mesurables et de mesure finie : \( \mu_i(\Omega_{i,n})<\infty\). D'abord remarquons que
    \begin{equation}\label{EqNFuBzBF}
        \mu_2\Big( (A\cap \Omega_{1,j}\times E_{2,j})_2(x) \Big)=\mu_2\Big( A_2(x)\cap \Omega_{2,j} \Big)\mtu_{\Omega_{1,j}}.
    \end{equation}
    En effet,
    \begin{subequations}
        \begin{align}
            \heartsuit&=(A\cap\Omega_{1,j}\times E_{2,j})_2(x)\\
            &=\{ y\in\Omega_2\tq (x,y)\in A\cap \Omega_{1,j}\times E_{2,j} \}\\
            &=\{ y\in \Omega_2\tq (x,y)\in A\times \Omega_{2,j} \}\cap\{ y\in\Omega_2\tq (x,y)\in \Omega_{1,j}\times \Omega_{2,j} \}.
        \end{align}
    \end{subequations}
    Si \( y\in \Omega_{1,j}\) alors \( \{ y\in \Omega_2\tq (x,y)\in \Omega_{1,j}\times \Omega_{2,j} \}=\Omega_{2,j}\) et dans ce cas
    \begin{equation}
        \heartsuit=\{ y\in \Omega_2\tq (x,y)\in A\times \Omega_{2,j} \}\cap \Omega_{2,j}=A_2(x)\cap E_{2,j}.
    \end{equation}
    Et inversement, si \( x\notin \Omega_{1,j}\) alors \( \heartsuit=\emptyset\). Dans les deux cas nous avons \eqref{EqNFuBzBF}.

    Les ensembles \( A\cap \Omega_{1,j}\times \Omega_{2,j}\) étant de mesure finie, nous pouvons leur appliquer la première partie :
    \begin{equation}
        \int_{\Omega_1}\mu_2\Big( (A\cap\Omega_{1,j}\times \Omega_{2,j})_2(x) \Big)d\mu_1(x)=\int_{\Omega_2}\mu_1\Big( (A\cap\Omega_{1,j}\times \Omega_{2,j})_1(y) \Big)d\mu_2(u),
    \end{equation}
    ou encore
    \begin{equation}
        \int_{\Omega_1}\mu_2\Big( A_2(x)\cap \Omega_{2,j} \Big)\mtu_{\Omega_{1,j}}(x)d\mu_1(x)=\int_{\Omega_2}\mu_1\Big( A_1(y)\cap \Omega_{1,j} \Big)\mtu_{\Omega_{2,j}}(y)d\mu_2(y).
    \end{equation}
    Ce que nous avons dans ces intégrales sont (par rapport à \( j\)) des suites croissantes de fonction positives; nous pouvons donc permuter une limite et une intégrale. En sachant que si \( k\to \infty\), alors
    \begin{subequations}
        \begin{align}
            \mtu_{1,j}(x)\to 1\\
            \mu_2\big( A_2(x)\cap \Omega_2,j \big)\to\mu_2\big( A_2(x) \big),
        \end{align}
    \end{subequations}
    nous trouvons le résultat demandé.
\end{proof}

\begin{theoremDef}[\cite{FubiniBMauray,MesIntProbb}]   \label{ThoWWAjXzi}
    Soient \( \mu_i\) des mesures $\sigma$-finies sur \( (\Omega_i,\tribA_i)\) (\( i=1,2\)).
    \begin{enumerate}
        \item

    Il existe une et une seule mesure, notée \( \mu_1\otimes \mu_2\), sur \( (\Omega_1\times\Omega_2,\tribA_1\otimes\tribA_2)\) telle que
    \begin{equation}    \label{EqOIuWLQU}
        (\mu_1\otimes\mu_2)(A_1\times A_2)=\mu_1(A_1)\mu_2(A_2)
    \end{equation}
    pour tout \( A_1\in \tribA_1\) et \( A_2\in\tribA_2\).
\item
    Cette mesure est donnée par la formule\footnote{Voir les notations du lemme~\ref{LemAQmWEmN}.}
    \begin{equation}   \label{EqDFxuGtH}
        (\mu_1\otimes \mu_2)(A)=\int_{\Omega_1}\mu_2\big( A_2(x) \big)d\mu_1(x)=\int_{\Omega_2}\mu_1\big( A_1(y) \big)d\mu_2(y).
    \end{equation}
    Cette mesure est la \defe{mesure produit}{mesure!produit} de \( \mu_1\) par \( \mu_2\).
\item
    La mesure \( \mu_1\otimes \mu_2\) ainsi définie est \( \sigma\)-finie.
    \end{enumerate}
\end{theoremDef}
\index{mesure!produit}

\begin{proof}
    La partie «existence» sera divisée en deux parties : l'une pour prouver que les formules \eqref{EqDFxuGtH} donnent une mesure et une pour montrer que cette mesure vérifie la condition \eqref{EqOIuWLQU}.
    \begin{subproof}
    \item[Unicité]

    L'ensemble des rectangles de \( \Omega_1\times \Omega_2\) engendre la tribu \( \tribA_1\otimes\tribA_2\), est fermé par intersection et contient une suite croissante d'ensembles \( P_n\times R_n\) de mesure finie (\( \mu(P_n\times R_n)<\infty\)) telle que \( P_n\times R_n\to \Omega_1\times \Omega_2\). Cette suite est donné par le fait que \( \mu_1\) et \( \mu_2\) sont \( \sigma\)-finies. En effet si \( (X_n)\) et \( (Y_n)\) sont des recouvrements dénombrables de \( \Omega_1\) et \( \Omega_2\) par des ensembles de mesure finie, en posant \( P_n=\bigcup_{k=1}^nX_n\) et \( R_n=\bigcup_{k=1}^nY_n\) nous avons bien une suite croissante de rectangles qui tendent vers \( \Omega_1\times \Omega_2\). Avec ces rectangles en main, le théorème~\ref{ThoJDYlsXu} donne l'unicité.

\item[Les formules définissent une mesure]
    Le théorème~\ref{ThoCCIsLhO} dit que ces formules ont un sens et que l'égalité entre les deux intégrales est correcte. Nous prouvons à présent qu'elles déterminent effectivement une mesure sur \( (\Omega_1\times\Omega_2,\tribA_1\otimes \tribA_2)\).

    Pour tout \( A\in \tribA_1\otimes \tribA_2\), \( \mu(A)\geq 0\) parce que \( \mu\) est donnée par l'intégrale d'une fonction positive.

    En ce qui concerne la condition d'unions dénombrable disjointe, soient \( A^{(i)}\) des éléments disjoints de \( \tribA_1\otimes \tribA_2\); nous commençons par remarquer que
    \begin{subequations}
        \begin{align}
            \left( \bigcup_{i=1}^{\infty}A^{(i)} \right)_2(x)&=\{ y\in\Omega_2\tq (x,y)\in\bigcup_{i=1}^{\infty}A^{(i)} \}\\
            &=\bigcup_{i=1}^{\infty}\{ y\in\Omega_2\tq (x,y)\in A^{(i)} \}\\
            &=\bigcup_{i=1}^{\infty}A^{(i)}_2(x).
        \end{align}
    \end{subequations}
    Par conséquent,
    \begin{subequations}
        \begin{align}
            \mu\left( \bigcup_{i=1}^{\infty}A^{(i)} \right)&=\int_{\Omega_1}\mu_2\left(    \Big( \bigcup_{i=1}^{\infty}A^{(i)} \Big)_2(x)     \right)d\mu_1(x)\\
            &=\int_{\Omega_1}\sum_{i=1}^{\infty}\mu_2\big( A^{(i)}_2(x) \big)d\mu_1(x)\\
            &=\int_{\Omega_1}\lim_{n\to \infty} \sum_{i=1}^{n}\mu_2\big( A^{(i)}_2(x) \big)d\mu_1(x).
        \end{align}
    \end{subequations}
    où nous avons utilisé l'additivité de la mesure \( \mu_2\). À ce niveau, il serait commode de permuter la somme et l'intégrale. Pour ce faire nous considérons la suite (croissante) de fonctions
    \begin{equation}
        f_n(x)=\sum_{i=1}^n\mu_2\big( A_2^{(i)}(x) \big).
    \end{equation}
    Nous pouvons permuter la limite et l'intégrale grâce au théorème de la convergence monotone~\ref{ThoRRDooFUvEAN}; ensuite la somme se permute avec l'intégrale en tant que somme finie :
    \begin{subequations}
        \begin{align}
            \mu\left( \bigcup_{i=1}^{\infty}A^{(i)} \right)&=\lim_{n\to \infty} \sum_{i=1}^n\int_{\Omega_1}\big( A_2^{(i)}(x) \big)d\mu_1(x)\\
            &=\lim_{n\to \infty} \sum_{i=1}^n\mu(A^{(i)})\\
            &=\sum_{i=1}^{\infty}\mu( A^{(i)} ).
        \end{align}
    \end{subequations}

\item[Elles vérifient la condition]
    Prouvons que les formules \eqref{EqDFxuGtH} se réduisent à \eqref{EqOIuWLQU} dans le cas des rectangles. Soit donc \( A=X_1\times X_2\) avec \( X_i\in\tribA_i\). Alors
    \begin{equation}
        A_1(y)=\{ x\in\Omega_1\tq (x,y)\in X_1\times X_2 \}
    \end{equation}
    et
    \begin{equation}
        \mu_1\big( A_1(y) \big)=\mtu_{X_2}(y)\mu_1(X_1),
    \end{equation}
    donc
    \begin{subequations}
        \begin{align}
            (\mu_1\otimes\mu_2)(A)&=\int_{\Omega_2}\mu_1\big( A_1(y) \big)d\mu_2(y)\\
            &=\int_{\Omega_2}\mu_1(X_1)\mtu_{X_2}(y)d\mu_2(y)\\
            &=\mu_1(X_1)\int_{\Omega_2}\mtu_{X_2}(y)d\mu_2(y)\\
            &=\mu_1(X_1)\mu_2(X_2).
        \end{align}
    \end{subequations}
    Pour cela nous avons utilisé le fait que l'intégrale de la fonction caractéristique d'un ensemble mesurable est la mesure de cet ensemble.
    \end{subproof}
\end{proof}

\begin{definition}[Produit d'espaces mesurés]  \label{DefUMlBCAO}
    Si \( (\Omega_i,\tribA_i,\mu_i)\) sont deux espaces mesurés, l'\defe{espace produit}{produit!espaces mesurés} est l'ensemble \( \Omega_1\times \Omega_2\) muni de la tribu produit \( \tribA_1\otimes \tribA_2\) de la définition~\ref{DefTribProfGfYTuR} et de la mesure produit \( \mu_1\otimes \mu_2\) définie par le théorème~\ref{ThoWWAjXzi}.
\end{definition}

\begin{remark}
    Il n'est pas garantit que la tribu \( \tribA_1\otimes\tribA_2\) soit la tribu la plus adaptée à l'ensemble \( S_1\times S_2\). Dans le cas de \( \eR^N\), il se fait que c'est le cas : en prenant des produits des boréliens sur \( \eR\) on obtient bien les boréliens sur \( \eR^N\), voir proposition~\ref{CorWOOOooHcoEEF}.
\end{remark}

%+++++++++++++++++++++++++++++++++++++++++++++++++++++++++++++++++++++++++++++++++++++++++++++++++++++++++++++++++++++++++++
\section{Tribu et mesure de Lebesgue sur \texorpdfstring{$ \eR^d$}{Rd}}
%+++++++++++++++++++++++++++++++++++++++++++++++++++++++++++++++++++++++++++++++++++++++++++++++++++++++++++++++++++++++++++

\begin{definition}[Mesure de Lebesgue]      \label{DEFooSWJNooCSFeTF}
    En plusieurs étapes.
    \begin{enumerate}
        \item
            D'abord nous avons la mesure \( \lambda_N\) sur \( \eR^n\) définie sur
            \begin{equation}
                \big( \eR^d,\Borelien(\eR)\otimes\ldots\otimes\Borelien(\eR) \big)
            \end{equation}
            comme le produit \( \lambda\otimes\ldots\otimes \lambda\) via la définition~\ref{DefUMlBCAO}.
        \item
            Ensuite nous nous souvenons du corolaire~\ref{CorWOOOooHcoEEF} qui donne \( \lambda_N\) comme une mesure sur
            \begin{equation}
                \big( \eR^N,\Borelien(\eR^N) \big).
            \end{equation}
        \item
            Et enfin nous considérons la completion de la mesure \( \lambda_N\) (théorème~\ref{thoCRMootPojn}), que nous notons encore \( \lambda_N\).
    \end{enumerate}
\end{definition}

\begin{proposition}[\cite{ooRCYWooNAeaTA}]     \label{PropSKXGooRFHQst}
    Tout ouvert de \( \eR^n\) est une union dénombrable et disjointe de cubes semi-ouverts.
\end{proposition}

\begin{proof}
    Nous allons même montrer que ces cubes peuvent être choisis sur un quadrillage.

    Soit \( G\) un ouvert de \( \eR^n\). Soit \( \{ Q_i^{1} \}_{i\in \eN}\) un découpage de \( \eR^n\) en cubes semi-ouverts de côté \( 1\) et dont les sommets sont en les coordonnées entières. Ils sont de la forme
    \begin{equation}
        \prod_{i=1}^n\mathopen[ n_i , n_i+1 \mathclose[
    \end{equation}
    où les \( n_i\) sont des entiers. Ce sont des cubes disjoints. Nous considérons ensuite pour chaque \( k>1\) le découpage \( \{ Q_i^{(k)} \}_{i\in\eN}\) de \( \eR^n\) en cubes de côtés \( 2^{-k}\) qui consiste à découper en \( 2\) les côtés des cubes du découpage \( Q^{(k-1)}\). Ces cubes forment encore un découpage dénombrable de \( \eR^n\) en des cubes disjoints. Ils sont de la forme
    \begin{equation}
        \prod_{i=1}^n\mathopen[ \frac{ n_i }{ 2^k } , \frac{ n_i+1 }{ 2^k } \mathclose[
    \end{equation}
    où les \( n_i\) sont encore entiers. Ensuite nous considérons \( \mE\) l'union de tous les \( Q_i^{(k)}\) contenus dans \( G\).

    Montrons que \( \mE=G\). D'abord \( \mE\subset G\) parce que \( \mE\) est une union d'ensembles contenus dans \( G\). Ensuite si \( x\in G\), il existe une boule de rayon \( r\) autour de \( x\) contenue dans \( G\); alors un des ensembles \( Q_i^{(k)}\) avec \( 2^{-j}<\frac{ r }{2}\) est contenue dans \( B(x,r)\) et donc dans \( \mE\).

    Bien entendu l'union qui donne \( \mE\) n'est pas satisfaisante par ce que les \( Q_i^{(k+1)}\) sont contenus dans les \( Q_i^{(k)}\); les intersections sont donc loin d'être vides.

    Nous faisons ceci :
    \begin{subequations}
        \begin{align}
            R^{(0)}&=\{ Q_i^{(1)} \text{contenu dans } G \}\\
            R^{(k+1)}&=\{ Q_i^{(k+1)}\text{contenus dans } G\text{ et pas dans } R^{(k)} \}.
        \end{align}
    \end{subequations}
    En fin de compte l'union de tous les ensembles contenus dans les \( R^{(k)}\) forment encore \( \eR^n\), mais sont d'intersection vide.
\end{proof}

Les cubes dont il est question dans cette preuve, de côtés \( 2^{-k}\) sont souvent appelés des cubes \defe{dyadiques}{dyadique}.

\begin{corollary}[\cite{ooRCYWooNAeaTA}]     \label{CorTHDQooWMSbJe}
    Tout ouvert de \( \eR^n\) est une union dénombrable de cubes presque disjoints\footnote{«presque» au sens où les intersections éventuelles sont de mesure de Lebesgue nulle.}.
\end{corollary}

\begin{proof}
    Il suffit de prendre les cubes de la proposition~\ref{PropSKXGooRFHQst} et de les fermer. Ce que l'on ajoute est de mesure nulle.
\end{proof}

\begin{remark}
    La proposition~\ref{PropSKXGooRFHQst} est une propriété seulement de la topologie de \( \eR^n\) alors que le corolaire fait intervenir la mesure de Lebesgue parce qu'il faut bien dire que les intersections sont de mesure (de Lebesgue) nulle.
\end{remark}

%---------------------------------------------------------------------------------------------------------------------------
\subsection{Ensembles négligeables}
%---------------------------------------------------------------------------------------------------------------------------

\begin{lemma}[\cite{VSMEooLwNLHd}]      \label{LemWHKJooGPuxEN}
    L'image d'une partie négligeables de \( \eR^N\) par une application Lipschitz est négligeable.
\end{lemma}

\begin{proof}
    Soit \( N\) une partie négligeable de \( \eR^N\) et une application Lipschitz \( f\colon N\to \eR^N\). Soit \( Q\subset \eR^N\) un cube borné de côté \( r\). Pour tout \( x,x'\in N\cap Q\) nous avons
    \begin{equation}
        \| f(x)-f(x') \|\leq C\| x-x' \|\leq Cr.
    \end{equation}
    Donc \( f(N\cap Q)\) est dans une boule de rayon \( Cr\). Mais comme toutes les normes sont équivalentes\footnote{Proposition~\ref{PropLJEJooMOWPNi}} sur \( \eR^N\) nous pouvons tout aussi bien prendre la norme \( \| . \|_1\) au lieu de la norme \( \| . \|_2\) (qui est toujours la norme prise implicitement lorsqu'on parle de \( \eR^n\)), de telle sorte que les boules soient des cubes. Quoi qu'il en soit, \( f(N\cap Q)\) est contenu dans un cube de côté \( 2Cr\) et au niveau de la mesure extérieure,
    \begin{equation}
        m^*\big( f(N\cap Q) \big)\leq (2Cr)^N=(2C)^Nr^N,
    \end{equation}
    ou encore
    \begin{equation}
        m\big(f(N\cap Q)\big)\leq (2C)^Nm(Q)
    \end{equation}
    parce que \( r^N\) est la mesure du cube \( Q\).

    Soit maintenant \( \epsilon>0\); vu que \( N\) est négligeable, il existe un ouvert \( U\) contenant \( N\) et tel que \( m(U)<\epsilon\). Ce \( U\) est une union presque disjointe de cubes dyadiques \( (Q_n)\) par le corolaire~\ref{CorTHDQooWMSbJe}. Nous avons alors
    \begin{subequations}
        \begin{align}
            m^*\big( f(N) \big)&=m^*\big( f(\bigcup_nN\cap Q_n) \big)\\
            &=m^*\big( \bigcup_nf(N\cap Q_n) \big)\\
            &\leq \sum_nm^*(f(N\cap Q_n))\\
            &\leq \sum_n(2C)^Nm(Q_n)\\
            &=(2C)^Nm(U)\\
            &<(2C)^d\epsilon.
        \end{align}
    \end{subequations}
    Au final, \( m^*\big( f(N) \big)\leq (2C)^N\epsilon\).  L'ensemble \( N\) est donc négligeable parce que le lemme~\ref{LemXOUNooUbtpxm} le dit : \( m^*(N)=0\).
\end{proof}

\begin{corollary}
    Un sous-espace vectoriel strict de \( \eR^N\) est négligeable.
\end{corollary}

\begin{proof}
    Un sous-espace vectoriel strict de \( \eR^N\) de dimension \( k<N\) est l'image de
    \begin{equation}
        A=\{ t_1e_1+\cdots +t_ke_k\tq t_i\in \eR \}
    \end{equation}
    par une application linéaire. Ce \( A\) est un pavé de mesure de Lebesgue nulle. Donc l'image est négligeable par le lemme~\ref{LemWHKJooGPuxEN}.
\end{proof}

%---------------------------------------------------------------------------------------------------------------------------
\subsection{Parties et fonctions mesurables}
%---------------------------------------------------------------------------------------------------------------------------

Pour rappel, la notion d'application de classe \( C^1\) est donnée par la définition~\ref{DefJYBZooPTsfZx}.

\begin{proposition}     \label{PropRDRNooFnZSKt}
    Soient \( U\) et \( V\) des ouverts de \( \eR^N\) et \( \phi\colon U\to V\) un \( C^1\)-difféomorphisme. Si \( E\subset U\) est mesurable, alors \( \phi(E)\) est mesurable\footnote{Ici «mesurable» parle de mesurabilité au sens de la tribu de Lebesgue, c'est-à-dire pas seulement les boréliens.}.
\end{proposition}

\begin{proof}
    Si \( E\) est mesurable, il existe un borélien \( B\) et un ensemble négligeable \( N\) tels que \( E=B\cup N\). Vu que \( \phi\) est un homéomorphisme, l'application \( \phi^{-1}\) est borélienne parce que continue (théorème~\ref{ThoJDOKooKaaiJh}). Nous avons
    \begin{equation}
        \phi(B)=(\phi^{-1})^{-1}(B),
    \end{equation}
    c'est-à-dire que \( \phi(B)\) est l'image inverse de \( B\) par \( \phi^{-1}\). L'ensemble \( \phi(B)\) est donc borélien.

    Il reste à voir que \( \phi(N)\) est négligeable. Soit \( Q\subset U\) une cube compact. L'application \( d\phi\colon Q\to \aL(\eR^N,\eR^N)\) est continue et donc bornée (par la remarque~\ref{RemATQVooDnZBbs}) sur le compact \( Q\). Par les accroissements finis (théorème~\ref{ThoNAKKght}), l'application \( \phi\) est donc Lipschitz sur \( Q\). La partie \( \phi(N\cap Q)\) est alors négligeable par le lemme~\ref{LemWHKJooGPuxEN}. Pour conclure,
    \begin{equation}
        \phi(N)=\bigcup_i\phi(N\cap Q_i)
    \end{equation}
    où les \( Q_i\) sont tous des cubes compacts. Donc \( \phi(N)\) est une union dénombrable d'ensembles négligeables; ergo négligeable lui-même par le lemme~\ref{LemVKNooOCOQw}.
\end{proof}

\begin{proposition}
    Soient \( U\) et \( V\) des ouverts de \( \eR^N\) et \( \phi\colon U\to V\) un \( C^1\)-difféomorphisme. Si \( f\colon V\to \eC\) est mesurable, alors \(f\circ \phi\colon U\to \eC\) l'est.
\end{proposition}

\begin{proof}
    Soit \( A\) une partie mesurable de \( \eC\). Il nous faut prouver que
    \begin{equation}
        (f\circ\phi)^{-1}(A)=\phi^{-1}\big( f^{-1}(A) \big)
    \end{equation}
    soit mesurable. Par hypothèse , \( f^{-1}(A)\) est mesurable. Vu que \( \phi\) est un \( C^1\)-difféomorphisme, elle et son inverse sont mesurables par la proposition~\ref{PropRDRNooFnZSKt}. Donc l'image du mesurable \( f^{-1}(A)\) par \( \phi^{-1}\) est encore mesurable.
\end{proof}

%---------------------------------------------------------------------------------------------------------------------------
\subsection{Propriétés d'unicité}
%---------------------------------------------------------------------------------------------------------------------------

\begin{corollary}       \label{CorMPDAooDJRrom}
    La mesure \( \lambda_N\) est l'unique mesure sur \(   (\eR^N,  \Borelien(\eR^N) )   \) à satisfaire
    \begin{equation}
        \mu\big( \prod_{i=1}^N\mathopen[ a_i , b_i \mathclose] \big)=\prod_{i=1}^n| a_i-b_i |
    \end{equation}
\end{corollary}

\begin{proof}
    Par définition de la mesure produit, \( \lambda_N\) est l'unique mesure sur \(   (\eR^N,  \Borelien(\eR)\otimes\ldots\otimes\Borelien(\eR) )   \) à satisfaire la condition. La proposition~\ref{CorWOOOooHcoEEF} conclut.
\end{proof}

Vu que les compacts de \( \eR^n\) sont les fermés bornés (théorème~\ref{ThoXTEooxFmdI}), et que tout borné est dans un tel produit d'intervalle, la mesure de Lebesgue est une mesure de Borel (définition~\ref{DefFMTEooMjbWKK}\ref{ItemTTPTooStDcpw}).

\begin{theorem}[\cite{PMTIooJjAmWR}]        \label{THOooTMWHooThsDHj}
    La mesure de Lebesgue est invariante par translation. Autrement dit si \( A\) est mesurable dans \( \eR^n\) et si \( a\in \eR^n\) alors \( A+a\) est mesurable et
    \begin{equation}
        \lambda_N(A+a)=\lambda_N(A).
    \end{equation}
\end{theorem}

\begin{proof}
    Nous supposons que \( A\) est borélien; sinon il l'est à ensemble négligeable près. Nous notons \( t_a\) la translation et nous nommons \( \mu\) la mesure donnée par
    \begin{equation}
        \mu(A)=\lambda_N(A+a).
    \end{equation}
    Vu que
    \begin{equation}
        \mu\big( \prod_{n=1}^N\mathopen[ r_n , s_n \mathclose[ \big)=\lambda_N\big( \prod_i\mathopen[ r_n+a_n , s_n+a_n [ \big)=\prod_i| s_n-r_n |.
    \end{equation}
    Vu qu'il y a unicité de la mesure vérifiant cette propriété (corolaire~\ref{CorMPDAooDJRrom}), nous avons \( \mu=\lambda_N\).
\end{proof}

Pour la suite nous notons \( Q_0\) le cube unité de \( \eR^N\) : \( Q_0=\big( \mathopen[ 0 , 1 \mathclose[ \big)^N\).

\begin{theorem}[\cite{PMTIooJjAmWR}]        \label{ThoCABFooHbUzWc}
    Soit \( \mu\) une mesure positive sur \( \eR^N\) telle que
    \begin{enumerate}
        \item
            \( \mu\) soit invariante par translation (des boréliens),
        \item
            \( \mu(Q_0)=1\).
    \end{enumerate}
    Alors \( \mu=\lambda_N\).
\end{theorem}

\begin{proof}
    Pour simplifier l'écriture nous faisons \( N=2\). Notre but est de prouver que \( \mu(  \mathopen[ 0 , r \mathclose[\times \mathopen[ 0 , r' \mathclose[ )=rr'\) pour tout \( r,r'\in \eR\).

    \begin{subproof}
    \item[Longueur =\( 1/J\)]
        Soient \( J,K\) des entiers. Nous pouvons diviser le cube \( Q_0\) en rectangles de côtés \( 1/J\) et \( A/K\) :
        \begin{equation}
            Q_0=\bigcup_{\substack{1\leq j\leq J\\1\leq k\leq K}}\mathopen[ \frac{ j-1 }{ J } , \frac{ j }{ J } \mathclose[\times \mathopen[ \frac{ k-1 }{ K } , \frac{ k }{ K } \mathclose[
        \end{equation}
        où l'union est disjointe. En ce qui concerne la mesure nous commençons par utiliser la sous-additivité :
        \begin{equation}
            \mu(Q_0)=\sum_{\substack{1\leq j\leq J\\1\leq k\leq K}}\mu\left(  \mathopen[ \frac{ j-1 }{ J } , \frac{ j }{ J } \mathclose[\times \mathopen[ \frac{ k-1 }{ K } , \frac{ k }{ K } \mathclose[      \right).
        \end{equation}
        Nous utilisons ensuite, sur chacun des termes séparément l'invariance par translation selon les vecteurs \( (\frac{ j-1 }{ J },0)\) et \( ( 0,\frac{ k-1 }{ K } )\) :
        \begin{equation}
            1=\mu(Q_0)=\sum_{\substack{1\leq j\leq J\\1\leq k\leq K}}\mu\left(  \mathopen[ 0,\frac{1}{ J } \mathclose[\times \mathopen[0,\frac{1}{ K }\mathclose[      \right)=JK\mu\mu\left(  \mathopen[ 0,\frac{1}{ J } \mathclose[\times \mathopen[0,\frac{1}{ K }\mathclose[      \right),
        \end{equation}
        et donc
        \begin{equation}
            \mu\left(  \mathopen[ 0,\frac{1}{ J } \mathclose[\times \mathopen[0,\frac{1}{ K }\mathclose[      \right)=\frac{1}{ J }\times \frac{1}{ K }.
        \end{equation}
    \item[Longueur \( L/K\)]

        Soient \( L,M\) des entiers et calculons :
        \begin{subequations}
            \begin{align}
                \mu\left( \mathopen[ \frac{ 0 }{ J } , \frac{ L }{ J } \mathclose[\times \mathopen[ \frac{ 0 }{ K } , \frac{ M }{ K } \mathclose[ \right)&=\sum_{\substack{0\leq l\leq L-1\\0\leq m\leq M-1}}\mu\left(   \mathopen[    \frac{ l }{ J },\frac{ l+1 }{ J }  \mathclose[\times \mathopen[ \frac{ m }{ K } , \frac{ m+1 }{ K } \mathclose[      \right)\\
                    &=LM\mu\left(  \mathopen[ \frac{ 0 }{ J } , \frac{ 1 }{ J } \mathclose[\times \mathopen[ \frac{ 0 }{ K } , \frac{ 1 }{ K } \mathclose[  \right)\\
                        &=LM\times \frac{1}{ J }\times \frac{1}{ K }.
            \end{align}
        \end{subequations}
        Nous avons donc, pour tout \( J,K,L,M\) :
        \begin{equation}
            \mu\left( \mathopen[ 0 , \frac{ L }{ J } \mathclose[\times \mathopen[ 0, \frac{ M }{ K } \mathclose[ \right)=\frac{ L }{ J }\times \frac{ M }{ K },
        \end{equation}
        c'est-à-dire que pour tout \( r,s\in \eQ^+\) nous avons
        \begin{equation}
            \mu\big(   \mathopen[ 0 , r \mathclose[\times \mathopen[ 0 , s \mathclose[ \big)=rs.
        \end{equation}
    \item[Longueur réelle]
        Nous passons au cas de longueur réelle. Soit \( a>0\) et une suite croissante de rationnels \( r_n\to a\). Une telle suite existe par la proposition~\ref{PropSLCUooUFgiSR}. L'intervalle \( \mathopen[ 0 , a \mathclose[\) s'écrit sous la forme d'une union croissante \( \mathopen[ 0 , a \mathclose[=\bigcup_{n\geq 1}\mathopen[ 0 , r_n \mathclose[\); le lemme~\ref{LemAZGByEs}\ref{ItemJWUooRXNPci} peut être utilisé et nous avons
        \begin{equation}
            \mu\big( \mathopen[ 0 , a \mathclose[ \big)=\mu\left( \bigcup_{n\geq 1}\mathopen[ 0 , r_n \mathclose[ \right)=\lim_{n\to \infty} \mu\big( \mathopen[ 0 , r_n \mathclose[ \big)=\lim_{n\to \infty} r_n=a.
        \end{equation}
    \end{subproof}

    Enfin, si \( a,a'\in \eR\), l'invariance par translation donne
    \begin{equation}
        \mu\big( \mathopen[ a , a' \mathclose[ \big)=\mu\big( \mathopen[ 0 , a'-a \mathclose[ \big)=a'-a.
    \end{equation}
    Par unicité de la mesure ayant cette propriété, nous avons \( \mu=\lambda_N\).
\end{proof}

\begin{corollary}       \label{CorKGMRooHWOQGP}
    Si \( \mu\) est une mesure positive sur \( \eR^N\) invariante par translation et telle que \( \mu(Q_0)=C<\infty\) alors \( \mu=C\lambda_N\).
\end{corollary}

\begin{proof}
    Si \( C>0\) nous considérons la mesure \( \frac{1}{ C }\mu\) qui vérifie \( (\frac{1}{ C }\mu)(Q_0)=1\). En conséquence du théorème~\ref{ThoCABFooHbUzWc}, \( \frac{1}{ C }\mu=\lambda_N\) et \( \mu=C\lambda_N\).

    Si au contraire \( C=0\) alors nous pouvons paver \( \eR^N\) avec des cubes \( Q_i\) de côté \( 1\) qui ont tous mesure \( 0\). Par conséquent, \( \eR^N=\bigcup_{i=1}^{\infty}Q_i\), donc \( \mu(\eR^N)=\sum_i\mu(Q_i)=0\). Par conséquent \( \mu=0\) parce que toute partie de \( \eR^N\) a une mesure au maximum égale à celle de \( \eR^N\).
\end{proof}

%---------------------------------------------------------------------------------------------------------------------------
\subsection{Régularité}
%---------------------------------------------------------------------------------------------------------------------------

Les différentes notions de régularité pour une mesure sont données dans la définition~\ref{DefFMTEooMjbWKK}. Ce sont essentiellement des questions de compatibilité entre la mesure et la topologie.
\begin{proposition}
    La mesure de Lebesgue est une mesure de Radon sur tout ouvert de \( \eR^N\).
\end{proposition}

\begin{proof}
    Soit \( V\) un ouvert de \( \eR^N\). C'est localement compact et dénombrable à l'infini. Il suffit de prouver que \( \lambda_N\) est de Borel sur \( V\) pour que le théorème~\ref{PropNCASooBnbFrc} conclue à la régularité de la mesure de Lebesgue.

    Soit \( K\) un compact de \( V\). Par la proposition~\ref{PropGBZUooRKaOxy} c'est également un compact de \( \eR^N\). Par conséquent \( K\) est dans un pavé fermé de \( \eR^N\) du type
    \begin{equation}
        K\subset \prod_{n=1}^N\mathopen[ a_n , b_n \mathclose]
    \end{equation}
    et donc en passant par le corolaire~\ref{CorMPDAooDJRrom},
    \begin{equation}
        \lambda_N(K)\leq \prod_{i=1}^N(b_n-a_n)<\infty.
    \end{equation}
    Nous avons démontré que \( \lambda_N\) reste fini sur tout compact de \( V\).
\end{proof}

% This is part of Mes notes de mathématique
% Copyright (c) 2011-2020
%   Laurent Claessens, Carlotta Donadello
% See the file fdl-1.3.txt for copying conditions.

%+++++++++++++++++++++++++++++++++++++++++++++++++++++++++++++++++++++++++++++++++++++++++++++++++++++++++++++++++++++++++++
\section{Propriétés de l'intégrale de Lebesgue}
%+++++++++++++++++++++++++++++++++++++++++++++++++++++++++++++++++++++++++++++++++++++++++++++++++++++++++++++++++++++++++++

Un lemme qui a l'air de rien, mais qui au final est souvent utilisé; tellement qu'on l'oublie un peu.
\begin{lemma}[\cite{MonCerveau}]       \label{LEMooWKSWooPptdEm}
    Soit un compact \( K\) de \( \eR\) et une fonction continue \( f\colon \eR\to \eR\). Alors l'intégrale
    \begin{equation}
        \int_Kf
    \end{equation}
    existe et est finie.
\end{lemma}

\begin{proof}
    Vu que \( f\) est continue sur le compact \( K\), elle y atteint une borne supérieure\footnote{Nous ne nous lasserons jamais de citer le théorème de Weierstrass \ref{ThoWeirstrassRn}.} que nous nommons \( M\).

    Soit \( R\) tel que \( B(0,R)\) contienne \( K\). La fonction \( (M+1)\mtu_{B(0,R)}\) majore strictement \( f\) sur le mesurable \( B(0,R)\). L'ensemble sur lequel nous prenons le supremum dans la définition \eqref{EqDefintYfdmu} de l'intégrale de \( f\) contient donc au moins le nombre fini \( (M+1)\mu(K)\). Le supremum existe et est fini (proposition \ref{DefSupeA}).
\end{proof}

%--------------------------------------------------------------------------------------------------------------------------- 
\subsection{Quelques limites dans les bornes}
%---------------------------------------------------------------------------------------------------------------------------

Dans le cas de l'intégrale de Lebesgue définie par \ref{DefTVOooleEst}, si \( f\) est une fonction sur \( \eR\) et si \( \lambda\) est la mesure de Lebesgue, nous avons une définition directe de
\begin{equation}
    \int_{0}^{\infty}f\lambda.
\end{equation}
Nous sommes cependant en droit de nous demander si nous n'aurions pas également ceci :
\begin{equation}        \label{EQooDVKKooCiFzmA}
    \lim_{x\to \infty} \int_0^xf\lambda=\int_0^{\infty}fd\lambda.
\end{equation}
Lorsque l'intégrale considérée est celle de Riemann, l'égalité \eqref{EQooDVKKooCiFzmA} est une définition. Ici, ça va être une propriété.

\begin{normaltext}
    Tant que nous sommes à parler de limites dans les bornes, nous aurions pu vouloir, pour les séries, suivre le chemin suivant :
    \begin{itemize}
        \item Définir l'intégrale de Lebesgue sur un espace mesuré.
        \item Prendre au passage le cas particulier \( \sum_{k=0}^{\infty}a_k=\int_\eN a\) où \( a\colon \eN\to \eR\) est une fonction mesurable pour la mesure de comptage.
        \item Démontrer qu'avec ces définitions, \( \sum_{k=0}^{\infty}=\lim_{N\to \infty} \sum_{k=0}^Na_k\).
    \end{itemize}
    Or le dernier point est pris comme définition et son égalité avec l'intégrale pour la mesure de comptage est une propriété\footnote{Proposition \ref{PROPooPNQAooDRLcCm}.}. Pourquoi ? Parce que la définition \ref{DefBTsgznn} de mesure positive demande déjà d'avoir défini les sommes sur \( \eN\).
\end{normaltext}

\begin{lemma}
    Soit une partie mesurable \( A\subset \eR^+\) de mesure finie. Alors
    \begin{equation}
        \lim_{M\to 0} \lambda\big( A\cap\mathopen[ M , \infty \mathclose[ \big)=0.
    \end{equation}
\end{lemma}

\begin{proof}
    La fonction
    \begin{equation}
        \begin{aligned}
            f\colon \eR^+&\to \eR^+ \\
            x&\mapsto \lambda\big( A\cap\mathopen[ x , \infty \mathclose[ \big)
        \end{aligned}
    \end{equation}
    est décroissante et bornée vers le bas par \( 0\). Elle possède dont une limite \( \ell\geq 0\) (corolaire \ref{CorFHbMqGGyi}). Nous allons prouver que \( \ell=0\) en calculant la limite sur les entiers.

    Nous posons \( J_k=\mathopen[ k , k+1 \mathclose[\). Pour \( n\in \eN\) nous avons :
    \begin{subequations}
        \begin{align}
            f(n)&=\lambda\big( A\cap\mathopen[ n , \infty \mathclose[ \big)\\
                &=\lambda\big( \bigcup_{k=n}^{\infty}(A\cap J_k) \big)\\
                &=\sum_{k=n}^{\infty}\lambda(A\cap J_k).
        \end{align}
    \end{subequations}
    
    Mais nous savons par hypothèse sur la mesure de \( A\) que
    \begin{equation}
        \lambda(A)=\sum_{k=0}^{\infty}\lambda(A\cap J_k)<\infty.
    \end{equation}
    Donc \( f(n)\) est une queue de série convergente. Elle tend donc vers zéro par le lemme \ref{LEMooFUCOooCOqLRj}. C'est à dire que
    \begin{equation}
        \lim_{n\to \infty} f(n)=0.
    \end{equation}
    Et comme \( \lim_{x\to \infty}f(x) \) existe et vaut \( \ell\), la seule possibilité est \( \ell=0\).
\end{proof}

\begin{lemma}       \label{LEMooMUHWooZPbMDb}
    Soit une fonction mesurable \( f\colon \eR\to \eR^+\). Alors
    \begin{equation}
        \lim_{x\to \infty} \int_x^{\infty}fd\lambda=0.
    \end{equation}
\end{lemma}

\begin{proof}
    Nous posons
    \begin{equation}
        F(x)= \int_x^{\infty}fd\lambda.
    \end{equation}
    Nous commençons par prouver que c'est une fonction décroissante. En effet,
    \begin{equation}
        F(x)-F(x+a)=\int_{x}^{\infty}f-\int_{x+a}^{\infty}f=\int_{x}^{x+a}f+\int_{x+a}^{\infty}f-\int_{x+a}^{\infty}=\int_x^{a+a}f\geq 0.
    \end{equation}
    Nous avons utilisé \ref{PropOPSCooVpzaBt}.

    Vu que \( f\) prend ses valeurs dans \( \eR^+\), nous avons \( F(x)\geq 0\) pour tout \( x\). La fonction \( F\) est décroissante et bornée vers le bas. Donc elle a une limite :
    \begin{equation}
        \lim_{x\to \infty} F(x)=\ell\geq 0.
    \end{equation}
    
    Supposons \( \ell >0\) et posons \( 0<\epsilon<\ell\). Soit \( M\) tel que pour tout \( x>M\) nous ayons
    \begin{equation}        \label{EQooZVKQooQDxHXp}
        \int_x^{\infty}f>m.
    \end{equation}
    Soit \( a\in \eR^+\) tel que
    \begin{equation}
        | \int_a^{\infty}f(t)-\ell\,dt |<\epsilon.
    \end{equation}
    En vertu de \eqref{EQooZVKQooQDxHXp} nous considérons \( a_0>a\) tel que 
    \begin{equation}
        \int_{a_0}^{\infty}f=I_0>m.
    \end{equation}
    Nous construisons la suite strictement croissante \( (a_k)\) de la façon suivante :
    \begin{equation}
        \int_{a_k}^{\infty}f=I_k>m
    \end{equation}
    et
    \begin{equation}
        | \int_{a_k}^{a_{k+1}}f-I_k |<\epsilon.
    \end{equation}
    Donc pour \( k\) nous avons
    \begin{equation}
        \int_{a_k}^{a_{k+1}}f\geq I_k-\epsilon\geq m-\epsilon.
    \end{equation}
    Mais
    \begin{equation}
        \int_{a_0}^{\infty}=\sum_{k=0}^{\infty}\int_{a_k}^{a_{k+1}}f\geq \sum_k(m-\epsilon)=\infty.
    \end{equation}
    Nous avons une contradiction.
\end{proof}

%--------------------------------------------------------------------------------------------------------------------------- 
\subsection{Mesure de comptage et série}
%---------------------------------------------------------------------------------------------------------------------------

\begin{definition}[mesure de comptage]      \label{DEFooILJRooByDzhs}
    Soit \( (S,\tribF)\) un ensemble mesurable. La \defe{mesure de comptage}{mesure!de comptage} sur \( (S,\tribF)\) est la mesure définie par
    \begin{equation}
        m(A)=\begin{cases}
            \Card(A)    &   \text{si } A\text{ est fini}\\
            +\infty    &    \text{sinon}.
        \end{cases}
    \end{equation}
\end{definition}
Cette mesure est utilisée pour voir des séries comme des intégrales sur \( (\eN,\partP(\eN),m)\).

\begin{proposition}     \label{PROPooPNQAooDRLcCm}
    Soit l'espace mesuré \( \big( \eN,\partP(\eN), m \big)\)\footnote{La mesure de comptage sur \( \eN\) est donnée en la définition \ref{DEFooILJRooByDzhs}.}. Nous considérons une suite \( a\colon \eN\to \eR\).

    \begin{enumerate}
        \item
            L'intégrale \( \int_{\eN}a\,dm\) existe si et seulement si la série \( \sum_{n=0}^{\infty}a_n\) existe.
        \item
            Si \( \int_{\eN}a\,dm\) existe, alors
            \begin{equation}
                \int_{\eN}a\,dm=\sum_{n=0}^{\infty}a_n.
            \end{equation}
    \end{enumerate}
\end{proposition}

\begin{example}
    La mesure de comptage \( m\) sur \( \eN\) muni de la tribu de ses parties est \( \sigma\)-finie parce que \( E_n=\{ 0,\ldots, n \}\) est de mesure finie et \( \bigcup_{n\in \eN}E_n=\eN\).
\end{example}

\begin{example}
    L'intervalle \( I=\mathopen[ 0 , 1 \mathclose]\) muni de la tribu de toutes ses parties et de la mesure de comptage est un espace mesuré non \( \sigma\)-fini.
\end{example}

%---------------------------------------------------------------------------------------------------------------------------
\subsection{Théorème de la moyenne}
%---------------------------------------------------------------------------------------------------------------------------

\begin{theorem}[\cite{MonCerveau}]      \label{ThoooEZLGooMChwLT}
    Soit \( Q\) un compact connexe par arcs et une fonction continue \( f\colon Q\to \eR\). Si \( \lambda\) est la mesure de Lebesgue, alors il existe \( a\in Q\) tel que
    \begin{equation}
        f(a)=\frac{1}{ \lambda(Q) }\int_Qfd\lambda
    \end{equation}
\end{theorem}

\begin{proof}
    En posant \( I=\int_Qfd\lambda\) nous avons immédiatement
    \begin{equation}        \label{EqooTYQCooVxdazW}
        \min(f)\lambda(Q)\leq I\leq \max(f)\lambda(Q)
    \end{equation}
    où le minimum et le maximum existent parce que \( f\) est continue sur un compact. Si une des deux inégalités est une égalité alors la fonction est constante. En effet supposons que la première inégalité soit une égalité; si la fonction n'était pas constante, il existerait une boule sur laquelle \( f\) serait strictement supérieure à \( \min(f)\). En intégrant d'abord sur cette boule et ensuite sur le complémentaire nous obtenons une intégrale plus grande que \( \min(f)\lambda(Q)\).

    Soit \( \epsilon>0\). Il existe \( \alpha,\beta\in Q\) tels que \( f(\alpha)\leq\min(f)+\epsilon\) et \( f(\beta)\geq\max(f)-\epsilon\). Soit \( \gamma\colon \mathopen[ 0 , 1 \mathclose]\to Q\) un chemin continu tel que \( \gamma(0)=\alpha\) et \( \gamma(1)=\beta\). La fonction \( f\circ \gamma\colon \mathopen[ 0 , 1 \mathclose]\to \eR\) est alors continue et vérifie \( (f\circ\gamma)(0)\leq \min(f)+\epsilon\) et \( (f\circ\gamma)(1)\geq \max(f)-\epsilon\).

    Si \( \epsilon\) est assez petit et vu que les inégalités \eqref{EqooTYQCooVxdazW} sont strictes,
    \begin{equation}
        \lambda(Q)(f\circ\gamma)(0)\leq \min(f)\lambda(Q)+\epsilon\lambda(Q)<I<\max(f)\lambda(Q)-\epsilon\lambda(Q)\leq\lambda(Q)(f\circ \gamma)(1).
    \end{equation}
    Par le théorème des valeurs intermédiaires~\ref{ThoValInter}, il existe \( t_0\in\mathopen[ 0 , 1 \mathclose]\) tel que \( \lambda(Q)(f\circ\gamma)(t_0)=I\). Le point \( a=\gamma(t_0)\) vérifie
    \begin{equation}
        f(a)=\frac{1}{ \lambda(Q) }\int_Qfd\lambda.
    \end{equation}
\end{proof}

%---------------------------------------------------------------------------------------------------------------------------
\subsection{Primitives et intégrales}
%---------------------------------------------------------------------------------------------------------------------------

En termes de notations, si \( a<b\) nous posons
\begin{equation}
    \int_a^bf(t)dt=\int_{\mathopen[ a , b \mathclose]}f.
\end{equation}
Si par contre \( a>b\) nous posons \( \int_a^bf=-\int_b^af\).

\begin{proposition}[Primitive et intégrale\cite{TrenchRealAnalisys}] \label{PropEZFRsMj}
    Soit \( f\) une fonction intégrable sur \( \mathopen[ a , b \mathclose]\) et continue sur \( \mathopen] a , b \mathclose[\). Alors la fonction
    \begin{equation}
        \begin{aligned}
            F\colon \mathopen[ a , b \mathclose]&\to \eR \\
            x&\mapsto \int_{\mathopen[ a , x \mathclose]}f(t)dt.
        \end{aligned}
    \end{equation}
est l'unique primitive de \( f\) sur \( \mathopen] a , b \mathclose[\) s'annulant en \( x=a\).
\end{proposition}
\index{primitive et intégrale}

\begin{proof}
Nous devons prouver que \( F\) est dérivable et que pour tout \( x_0\in\mathopen] a , b \mathclose[\) nous avons \( F'(x_0)=f(x_0)\). Soit \( \epsilon>0\). Par continuité de \( f\) en \( x_0\), il existe une fonction \( \alpha\colon \eR\to \eR\) telle que
    \begin{equation}
        f(x_0+h)=f(x_0)+\alpha(h)
    \end{equation}
    avec \( \lim_{h\to 0} \alpha(h)=0\). Cette dernière limite signifie qu'il existe un \( \delta>0\) tel que \( |\alpha(h)|<\epsilon\) pour tout \( h\) tel que \( | h |<\delta\), c'est-à-dire pour tout \( h\in B(0,\delta)\). À partir de maintenant nous ne considérons plus que de tels \( h\).

    Notre travail maintenant est de prouver que \( F\) est dérivable en \( x_0\), et de montrer que la dérivée est \( f(x_0)\). Pour cela,
    \begin{subequations}
        \begin{align}
            F(x_0+h)-F(x_0)&=\int_{x_0}^{x_0+h}f(t)dt\\
        &=\int_0^hf(x_0+t)dt\\
        &=\int_0^h\big[ f(x_0)+\alpha(t) \big]dt\\
        &=hf(x_0)+\int_0^{h}\alpha(t)dt.
        \end{align}
    \end{subequations}

    Nous avons donc montré que pour tout \( \epsilon>0\), il existe un \( \delta\) (défini via la fonction \( \alpha\)) tel que \( | h |<\delta\) implique
    \begin{equation}
        \left| \frac{ F(x_0+h)-F(x_0) }{ h }-f(x_0) \right| <\epsilon.
    \end{equation}
    Cela signifie que 
    \begin{equation}
        \lim_{h\to 0} \frac{ F(x_0+h)-F(x_0) }{ h }=f(x_0),
    \end{equation}
    qui n'est rien d'autre que le fait que \( F\) est dérivable en \( x_0\) et que sa dérivée est \( f(x_0)\).

    Le fait que \( F\) s'annule en \( x=a\) est par sa définition. L'unicité provient du corolaire~\ref{CorZeroCst}.
\end{proof}

\begin{theorem}[Théorème fondamental du calcul intégral]    \label{ThoRWXooTqHGbC}
    Soit \( f\) une fonction continue sur un intervalle ouvert \( I\) contenant strictement l'intervalle \( \mathopen[ a , b \mathclose]\subset \eR\) et \( F\) une primitive de \( f\) sur \( I\). Alors
    \begin{equation}
        \int_a^bf(t)dt=F(b)-F(a).
    \end{equation}
\end{theorem}
\index{théorème fondamental du calcul intégral}

\begin{proof}
    Nous avons vu par la proposition~\ref{PropEZFRsMj} que la fonction
    \begin{equation}
        \begin{aligned}
            G\colon \mathopen[ a , b \mathclose]&\to \eR \\
            x&\mapsto  \int_a^xf(t)dt
        \end{aligned}
    \end{equation}
    était l'unique primitive de \( f\) sur \( \mathopen] a , b \mathclose[\) à s'annuler pour \( x=a\). Nous avons évidemment
    \begin{equation}
        \int_a^bf(t)dt=G(b).
    \end{equation}
    Si \( F\) est une primitive quelconque, il suffit de soustraire sa valeur en \( x=a\) : \( G(x)=F(x)-F(a)\) et donc
    \begin{equation}
        \int_a^bf(t)dt=G(b)=F(b)-F(a),
    \end{equation}
    comme il fallait le prouver.
\end{proof}

Le théorème fondamental s'écrit souvent sous la forme\footnote{Par exemple dans les théorèmes du reste des polynômes de Taylor \ref{THOooEUVEooXZJTRL} et de Cauchy-Lipschitz \ref{ThokUUlgU}.}
\begin{equation}        \label{EqooBBCYooNweVrF}
    f(x)=f(a)+\int_a^xf'(t)dt.
\end{equation}
Sous cette forme, il faut penser que nous calculons \( f(x)\) en un point pas trop éloigné de \( a\), en sachant \( f(a)\) et en intégrant la dérivée entre les deux.

\begin{remark}
    Le lien entre primitive et intégrale est fondamentalement lié à l'invariance par translation de la mesure de Lebesgue, et non à la construction précise de cette mesure. Mais en même temps, la mesure de Lebesgue est l'unique à être invariante par translation.
\end{remark}

Quelque remarques.
\begin{enumerate}
    \item
        Le théorème fondamental du calcul intégral est à utiliser pour calculer des intégrales des fonctions réelles lorsqu'on a des primitives sur un domaine strictement plus large que le domaine sur lequel nous voulons intégrer.
    \item
        Une version pour les intégrales impropres sera donnée au corolaire~\ref{CorMUIooXREleR}.
    \item
    Une primitive est forcément une fonction continue parce qu'une primitive est dérivable.
\item
    Le théorème fondamental du calcul intégral ne sert pas qu'à calculer des intégrales à partir de primitives. Il sert aussi à démontrer des résultats plus théoriques, comme le théorème \ref{THOooXZQCooSRteSr}.
    \item
        En vertu du corolaire~\ref{CorZeroCst}, une fonction ne possède qu'une seule primitive à constante près.
\end{enumerate}


%---------------------------------------------------------------------------------------------------------------------------
\subsection{Exemples et applications}
%---------------------------------------------------------------------------------------------------------------------------

Si \( f\) est une fonction définie sur un intervalle \( I\) et y admettant des primitives, nous notons
\begin{equation}
    \int f(x)dx
\end{equation}
l'ensemble des primitives de \( f\) sur \( I\) :
\begin{equation}
    \int f(x)dx=\left\{    F(x)+C\tq C\in \eR   \right\}
\end{equation}
où \( F\) est une quelconque primitive de \( f\).

\begin{example}
    Une primitive bien connue de \(  f\colon x\mapsto x^2 \) est la fonction \( F\colon x\to \frac{ x^3 }{ 3 }\). Nous écrivons donc
    \begin{equation}
        \int x^2dx=\frac{ x^3 }{ 3 }+C.
    \end{equation}
    Cela est un abus de notations terrible pour dire en réalité
    \begin{equation}
        \{ x\mapsto \frac{ x^3 }{ 3 }+C\tq C\in \eR \}.
    \end{equation}
\end{example}

En termes de notations, nous posons
\begin{equation}\label{Thfondcalc}
    \int_a^bf(t)dt=\Big[ F(t) \Big]_{t=a}^{t=b}=F(b)-F(a).
\end{equation}

\begin{remark}
  La valeur de l'intégrale ne dépend pas de la primitive qu'on choisi pour le calculer, car si $F_1$ et $F_2$ sont deux primitives de $f$ alors $F_1 = F_2 + C$ et $F_1(b)-F_1(a) = (F_2(b) + C)-(F_2(a)+C) = F_2(b)-F_2(a)$.
\end{remark}

\begin{remark}
  Si l'intervalle d'intégration est réduit à un seul point alors la valeur de l'intégrale est zéro. Nous le savions déjà, et cela est cohérent avec le théorème fondamental car $ \int_a^af(t)dt=F(a)-F(a) =0$.
\end{remark}

\begin{remark}
  Toute intégrale d'une fonction impaire sur un intervalle symétrique par rapport à l'origine est nulle.
\end{remark}

\begin{proposition}[\cite{MonCerveau}]      \label{PROPooJYIAooXLkbMx}
    Soient des espaces vectoriels normés \( E\) et \( F\) où \( F\) est de dimension finie\footnote{Sinon l'intégrale dont nous allons parler n'est pas définie au sens où nous n'en avons pas donné de définition. Voir \ref{NORMooTQBIooBaScjt}.}. Nous considérons une fonction \( f\colon E\to F\) de classe \( C^1\) ainsi qu'un chemin \( \gamma\colon \mathopen[ 0 , 1 \mathclose]\to E\) de classe \( C^1\) également.

    Alors nous avons l'égalité
    \begin{equation}
        \int_0^1(df)_{\gamma(t)}\big( \gamma'(t) \big)=f\big( \gamma(1) \big)-f\big( \gamma(0) \big).
    \end{equation}
\end{proposition}

\begin{proof}
    Nous posons
    \begin{equation}
        \begin{aligned}
            f\colon \mathopen[ 0 , 1 \mathclose]&\to F \\
            t&\mapsto (f\circ\gamma)(t). 
        \end{aligned}
    \end{equation}
    Cette fonction vérifie \( g'(t)=(df)_{\gamma(t)}\big( \gamma'(t) \big)\) par le lemme \ref{LEMooKNBVooQSowos}. Le théorème fondamental du calcul intégral\footnote{Théorème \ref{ThoRWXooTqHGbC}.} nous permet donc d'écrire
    \begin{equation}
        \int_0^1(df)_{\gamma(t)}\big( \gamma'(t) \big)dt=\int_0^1g'(t)df=g(1)-g(0).
    \end{equation}
    Notons que \( g\) est continue grâce aux hypothèses de classe \( C^1\) pour \( \gamma\) et \( f\).
\end{proof}

%---------------------------------------------------------------------------------------------------------------------------
\subsection{Permuter limite et dérivée}
%---------------------------------------------------------------------------------------------------------------------------

\begin{normaltext}      \label{NORMALooGYUEooKrYjyz}
    Voici une preuve alternative du théorème \ref{THOooXZQCooSRteSr}. Elle utilise des intégrales; elle demande donc plus de dependences.

\begin{theorem}[\cite{TrenchRealAnalisys}]    
    Soient une suite de fonctions \( f_i\colon \eR\to \eR\), une fonction \( f\colon \eR\to \eR\) et une fonction \( g\colon \eR\to \eR\) telles que
    \begin{enumerate}
        \item
            \( f_i\) est de classe \( C^1\) pour tout \( i\),
        \item
            \( f_i\to f\) simplement,
        \item
            \( f_i'\to g\) uniformément sur tout compact.
    \end{enumerate}
    Alors
    \begin{enumerate}
        \item
            \( f\) est de classe \( C^1\),
        \item
            \( f'=g\),
        \item
            \( f_i\to f\) uniformément sur tout compact.
    \end{enumerate}
\end{theorem}

    \begin{proof}
        Nous commençons par considérer \( x_0\in \eR\) et un intervalle compact \( K\) contenant \( x_0\). Nous montrons que \( f'(x_0)=g(x_0)\) en plusieurs étapes.
        \begin{subproof}
        \item[Une formule intégrale]
        Par hypothèse, les fonctions \( f_i\) sont continues (en particulier sur un ouvert contenant \( K\)), et le théorème fondamental de l'analyse \ref{ThoRWXooTqHGbC} donne
        \begin{equation}        \label{EQooFUBZooOVUhep}
            f_i(x)=f_i(x_0)+\int_{x_0}^xf_i'(t)dt
        \end{equation}
        pour tout \( x\in K\). Nous avons envie de prendre la limite \( i\to \infty\) en permutant la limite avec l'intégrale. Pour cela nous allons utiliser la convergence dominée de Lebesgue.

    \item[Convergence dominée]
        La convergence uniforme sur tout compact des fonctions continues \( f'_i\) vers \( g\) donne la continuité de \( g\), théorème \ref{ThoUnigCvCont}. En particulier \( g\) est bornée et donc intégrable sur le compact \( \mathopen[ x_0 , x \mathclose]\). Mais il en faut plus pour le théorème de la convergence dominée de Lebesgue (théorème \ref{ThoConvDomLebVdhsTf}). Soit \( a>0\); il existe \( N\) tel que pour tout \( i>n\) nous ayons \( \| f'_i-g \|<a\). Avec cela nous avons
        \begin{equation}
            | f'_i(x) |<| g(x) |+a
        \end{equation}
        pour tout \( x\in K\). En particulier, la fonction \( x\mapsto g(x)+a\) fonctionne pour la convergence dominée et nous pouvons permuter la limite et l'intégrale dans \eqref{EQooFUBZooOVUhep}.

    \item[Passage à la limite]

        En passant à la limite \( i\to \infty\) dans \eqref{EQooFUBZooOVUhep} nous trouvons
        \begin{equation}        \label{EQooAECSooZpoJhd}
            f(x)=f(x_0)+\int_{x_0}^xg(t)dt.
        \end{equation}
    \item[Premières conclusions]

        Il suffit maintenant de prendre la dérivée de \eqref{EQooAECSooZpoJhd} au point \( x=x_0\) grâce à la proposition \ref{PropEZFRsMj} :
        \begin{equation}
            f'(x_0)=g(x_0).
        \end{equation}
        Cela nous donne l'égalité \( f=g\) parce que \( x_0\) était arbitraire.

        De plus \( g\) est continue comme limite uniforme des fonctions continues \( f'_i\). Plus précisément, pour voir la continuité de \( g\) en \( x_0\), prendre un ouvert borné \( B(x_0,r)\) autour de \( x_0\), et ensuite un compact \( K\) contenant cet ouvert. La convergence uniforme \( f'_i\to g\) sur \( K\) implique la convergence uniforme sur \( B(x_0,r)\) et donc la continuité sur \( B(x_0,r)\) (théorème \ref{ThoUnigCvCont}).

    \item[\( f_i\to f\) uniforme sur tout compact]

        Un compact n'étant pas spécialement connexe, nous ne pouvons pas reprendre le travail fait jusqu'ici sans prendre une petite précaution. Soit un compact \( L\). Cette partie de \( \eR\) étant bornée\footnote{Par le théorème de Borel-Lebesgue \ref{ThoXTEooxFmdI}}, nous pouvons prendre \( r\) assez grand pour que \( L\subset \overline{ B(0,r) }\). Nous posons \( K=\overline{ B(0,r) }\) et nous prouvons la convergence uniforme \( f_i\to f\) sur \( K\). A fortiori, cela donnera la convergence uniforme sur \( L\).

        Prenons la différence entre \eqref{EQooAECSooZpoJhd} et \eqref{EQooFUBZooOVUhep} :
        \begin{subequations}
            \begin{align}
                | f(x)-f_i(x) |&=\big| f(x_0)-f_i(x_0)+\int_{x_0}^x g(t)-f'_i(t)dt \big|\\
                &\leq | f(x_0)-f_i(x_0) |+\Big| \int_{x_0}^x| g(t)-f_i'(t) |dt  \Big|       \label{SUBEQooIWSJooGckNmj}\\
                &\leq | (f-f_i)(x_0) |+| x-x_0 |\| g-f'_i \|_K.
            \end{align}
        \end{subequations}
        Notez les valeurs absolues autour de l'intégrale dans \eqref{SUBEQooIWSJooGckNmj}. Elles sont nécessaires parce que \( x\) est dans un voisinage de \( x_0\), sans que nous sachions si \( x\geq x_0\) ou \( x\leq x_0\) (ça change le signe de l'intégrale).

        Nous avons donc
        \begin{equation}
            \| f-f_i \|\leq | (f-f_i)(x_0) |+\diam(K)\| g-f'_i \|
        \end{equation}
        où \( \diam(K)\) est le diamètre de \( K\), c'est-à-dire la plus grande distance entre deux éléments de \( K\) c'est un nombre fini parce que \( K\) est borné. Il majore évidemment \( | x-x_0 |\). Le membre de droite tend vers zéro si \( i\to \infty\) parce que nous avons convergence simple \( f_i\to f\) et donc \( (f-f_i)(x_0)\to 0\), et parce que nous avons convergence uniforme sur tout compact, donc \( \| g-f_i' \|\to 0\).

        Nous avons donc bien \( \lim_{i\to \infty}\| f-f_i \|=0\), c'est-à-dire convergence uniforme de \( (f_i)\) vers \( f\) sur \( K\).

        \end{subproof}
    \end{proof}
\end{normaltext}

La proposition suivante est la généralisation à \( \eR\) de la proposition \ref{PROPooSGLGooIgzque}.
\begin{proposition}     \label{PROPooKIASooGngEDh}
    Pour tout \( \alpha\in \eR\), si \( f_{\alpha}(x)=x^{\alpha}\) alors
    \begin{equation}
        f'_{\alpha}(x)=\alpha x^{\alpha-1}.
    \end{equation}
    Au niveau du domaine, c'est \( \eR\) auquel il faut enlever \( \{ 0 \}\) si \( \alpha-1<0\).
\end{proposition}

\begin{proof}
    Soient \( \alpha\in \eR\) et une suite de rationnels \( \alpha_i\) qui converge vers \( \alpha\). Le plus amateurs d'abstraction diront \( (\alpha_i)\in \alpha\) en référence à la proposition \ref{PropooEPFCooMtDOfP}.

    Nous notons \( f_{\alpha}(x)=x^{\alpha}\) et \( f_i(x)=x^{\alpha_i}\). Par définition nous avon
    \begin{equation}
        f_i\to f_{\alpha}
    \end{equation}
    ponctuellement. De plus en utilisant la proposition \ref{PROPooSGLGooIgzque} nous savons que \( f'_i(x)=\alpha_i x^{\alpha_i-1}\). En posant \( g(x)=\alpha x^{\alpha-1}\) nous avons donc
    \begin{equation}
        f'_i\to g.
    \end{equation}
    ponctuellement. Mais \( f'_i\) est continue pour tout \( i\) et \( g\) également. Donc la convergence \( f_i\to f_{\alpha}\) est uniforme sur tout compact\footnote{Proposition \ref{PROPooFWVIooCzXojO}.}. Le théorème \ref{THOooXZQCooSRteSr} nous permet de permuter limite et dérivée pour avoir \( g=f'_{\alpha}\).
\end{proof}

%---------------------------------------------------------------------------------------------------------------------------
\subsection{Intégrales impropres}
%---------------------------------------------------------------------------------------------------------------------------
\label{SecGAVooBOQddU}

% TODO : l'exemple avec arcsin(1/x)-1/x de la page
%  http://fr.wikipedia.org/wiki/Intégrale_impropre

\begin{definition}[\cite{TrenchRealAnalisys}]
    Une fonction \( f\colon D\subset\eR\to \eR\) est \defe{localement intégrable}{localement!intégrable} sur un intervalle \( I\) si \( f\) est intégrable sur tout intervalle compact contenu dans \( I\).
\end{definition}
\index{intégrale!impropre}

%Dans \cite{TrenchRealAnalisys}, la proposition~\ref{PropCJAooQhNYkp} est prise comme une définition de \( \int_a^bf\) lorsque \( f\) est localement intégrable sur \( \mathopen[ a , b [\). Le point est que lui, il ne passe pas par Lebesgue et la construction abstraite d'intégrale par rapport à une mesure. Nous par contre nous avons déjà une définition de
%\begin{equation}
%    \int_a^bf=\int_{\mathopen[ a , b \mathclose]}f
%\end{equation}
%pour tout choix de \( a\), \( b\) et \( f\), que ce soit borné ou non.

\begin{proposition}     \label{PropCJAooQhNYkp}
    Soit \( f\colon \mathopen[ a , b \mathclose]\to \eR\) une fonction intégrable. Alors
    \begin{equation}    \label{EqPPMooBQDTYl}
        \int_{\mathopen[ a , b \mathclose]}f=\lim_{x\to b^-} \int_a^xf.
    \end{equation}
\end{proposition}

\begin{proof}
    Notons que la valeur de \( f\) en \( b\) n'a strictement aucune importance parce que l'intégrale de Lebesgue ne dépend pas du choix de la valeur de la fonction en un ensemble de mesure nulle; et en même temps la limite à gauche de \eqref{EqPPMooBQDTYl} ne dépend pas non plus de la valeur de \( f\) en \( b\). Bref si \( f\) n'est pas définie en \( b\), nous pouvons poser \( f(b)=42\).

    Notons de plus que du point de vue de l'intégrale de Lebesgue, \( \int_{\mathopen[ a , b \mathclose]}\) et \( \int_{\mathopen[ a , b [}\) sont identiques et valent toutes les deux \( \int_a^b\) (lorsque ça existe).

    Supposons d'abord que \( f\) est positive. Alors nous posons \( f_n=f\mtu_{\mathopen[ a , b-\frac{1}{ n } \mathclose]}\). Ponctuellement nous avons la limite croissante \( f_n\to f\) et de plus
    \begin{equation}
        \lim_{x\to b^-} \int_{\mathopen[ a , x \mathclose]}f=\lim_{n\to \infty} \int_{\mathopen[ a , b \mathclose]}f_n.
    \end{equation}
    Chacun des \( f_n\) est intégrable sur \( \mathopen[ a , b \mathclose]\). Le théorème de Beppo-Levi~\ref{ThoRRDooFUvEAN} implique que \( f\) est intégrable sur \( \mathopen[ a , b \mathclose]\) et que
    \begin{equation}
        \lim_{n\to \infty} \int_a^bf_n=\int_a^bf.
    \end{equation}
    Cela montre que dans le cas d'une fonction \( f\) positive nous avons bien \eqref{EqPPMooBQDTYl}.

    Si \( f\) n'est pas positif, alors nous la décomposons en partie positive et négative \( f=f^+-f^{-}\) et par définition de l'intégrale d'une fonction non positive,
    \begin{equation}
        \lim_{x\to b^-} \int_{\mathopen[ a , x [}f=\lim\int f^{+}-\lim\int f^-.
    \end{equation}
\end{proof}

Il peut cependant arriver que la limite \( \lim_{x\to b} \int_a^bf\) existe alors que \( f\) n'est pas intégrable sur \( \mathopen[ a , b \mathclose]\). C'est l'ennui des fonctions non positives. Un exemple classique est
\begin{equation}\label{EqMMVooDSpgfz}
    \int_0^{\infty}\frac{ \sin(t) }{ t }dt
\end{equation}

\begin{definition}[\cite{DWNooWUZxRP}]      \label{DEFooINPOooWWObEz}
    Si
    \begin{equation}
        \lim_{x\to b} \int_a^bf
    \end{equation}
    existe alors nous disons que l'intégrale est \defe{convergente}{intégrale!convergente} en \( b\). Ce procédé de limite est l'intégrale \defe{impropre}{intégrale!impropre} de \( f\) sur \( \mathopen[ a , b \mathclose]\).
\end{definition}

\begin{example}[Intégale impropre]
    Nous considérons la fonction \( f\colon \mathopen[ 0 , \infty [\to \eR\) définie par
    \begin{equation}
        f(x)=\begin{cases}
            \frac{1}{ n }    &   \text{si } x\in\mathopen[ 2n-2 , 2n-1 [\\
                -\frac{1}{ n }    &    \text{si } x\in\mathopen[ 2n-1 , 2n [\text{.}
        \end{cases}
    \end{equation}
    Par la divergence de la série harmonique, \( \int_{0}^{\infty}| f |\) n'existe pas. La fonction \( f\) n'est donc pas intégrable au sens de Lebesgue (définition~\ref{DefTCXooAstMYl}).

    Cependant pour tout \( n\) pair nous avons
    \begin{equation}
        \int_0^nf=0.
    \end{equation}
    Du coup pour tout \( x\geq 0\) nous avons
    \begin{equation}
        \int_0^xf=\int_{2n}^xf
    \end{equation}
    où \( 2n\) est le plus grand nombre pair inférieur à \( x\). Nous avons \( | x-2n |\leq 2\) et \( | f(x) |\leq \frac{1}{ n }\) pour \( x\in\mathopen[ 2n , x \mathclose]\). Donc
    \begin{equation}
        \int_{2n}^xf\leq \frac{ 2 }{ n }.
    \end{equation}
    Nous avons par conséquent
    \begin{equation}
        \lim_{x\to \infty} \int_0^xf=0,
    \end{equation}
    ce qui signifie que l'intégrale de \( f\) sur \( \mathopen[ 0 , \infty [\) converge au sens des intégrales impropres.
\end{example}


L'intégrale \eqref{EqMMVooDSpgfz} est une intégrale convergente mais la fonction n'est pas intégrable (parce que pour être intégrale il faut que \( | f |\) soit intégrable). Nous pouvons ainsi dire que cette intégrale converge mais n'existe pas.

Le corolaire suivant nous autorise à utiliser le théorème fondamental du calcul intégral~\ref{ThoRWXooTqHGbC} même dans les cas limites.
\begin{corollary}   \label{CorMUIooXREleR}
    Si \( f\) est localement intégrable sur \( \mathopen[ a , b \mathclose]\) et si \( F\) est une primitive de \( f\) sur tout ouvert de \( \mathopen[ a , b \mathclose]\) alors
    \begin{equation}
        \int_a^bf=\lim_{x\to b^-} F(x)-F(a).
    \end{equation}
\end{corollary}
\index{primitive!et intégrale}

\begin{proof}
    Pour chaque \( x\) dans \( \mathopen[ a , b [\) nous avons
    \begin{equation}
        \int_a^xf=F(x)-F(b).
    \end{equation}
    La proposition~\ref{PropCJAooQhNYkp} nous explique que la limite \( x\to b^-\) du membre de gauche existe et vaut \( \int_a^bf\). Donc également le membre de droite :
    \begin{equation}
        \int_a^bf=\lim_{x\to b^-} \int_a^xf=\lim_{x\to b^-} F(x)-F(b).
    \end{equation}
\end{proof}

La convergence des intégrales de fonctions \( \frac{1}{ x^{\alpha} }\) en \( 0\) et \( \infty\) est une question classique de l'intégration. De plus ces fonctions servent souvent à utiliser une théorème de comparaison (type intégrale dominée de Lebesgue).
\begin{proposition} \label{PropBKNooPDIPUc}
    Deux intégrales remarquables.
    \begin{enumerate}
        \item

            Nous avons
    \begin{equation}
        \int_0^1\frac{1}{ x^\alpha }=\infty
    \end{equation}
    si et seulement si \( \alpha\geq 1\).

\item   \label{ITEMooJFSXooHmgmEj}

    Nous avons
    \begin{equation}
        \int_1^{\infty}\frac{1}{ x^{\alpha} }=\infty
    \end{equation}
    si et seulement si \( \alpha\leq1\).

    \end{enumerate}

\end{proposition}

\begin{proof}
La fonction \( \frac{1}{ x^{\alpha} }\) admet la primitive \( F(x)=\frac{1}{ 1-\alpha }\frac{1}{ x^{\alpha-1} }\) sur tout compact de \( \mathopen] 0 , \infty \mathclose[\). Le corolaire~\ref{CorMUIooXREleR} nous permet\footnote{Tout ce que nous avons fait avec la borne \( b\) de l'intégrale \( \int_a^b\) reste valable avec la borne \( a\).} de dire que \( \int_0^1\frac{1}{ x^{\alpha} }\) vaudra
    \begin{equation}
        \lim_{x\to 0-^+} \frac{1}{ 1-\alpha }\frac{1}{ x^{\alpha-1} }.
    \end{equation}
    Cela est strictement plus petit que \( \infty\) si et seulement si \( \alpha<1\).
\end{proof}

%++++++++++++++++++++++++++++++++++++++++++++++++++++++++++++++++++++++++++++++++++++++++++++++++++
\section{Changement de variables dans une intégrale multiple}
%++++++++++++++++++++++++++++++++++++++++++++++++++++++++++++++++++++++++++++++++++++++++++++++++++

Dans ce qui suit, \( U\) et \( V\) sont des ouverts de \( \eR^N\) et \( \phi\colon U\to V\) est un \( C^1\)-difféomorphisme. Nous notons \( \mQ\) l'ensemble des cubes fermés dans \( U\) dont les côtés sont parallèles aux axes.

%---------------------------------------------------------------------------------------------------------------------------
\subsection{Des lemmes}
%---------------------------------------------------------------------------------------------------------------------------

\begin{lemma}[\cite{PMTIooJjAmWR}]      \label{LemooJYCGooIkkDVn}
    Soient \( \mu\) et \( \nu\) deux mesures de Borel sur l'ouvert \( U\) de \( \eR^N\). Si \( \mu(Q)\leq \nu(Q)\) pour tout \( Q\in \mQ\) alors \( \mu(B)\leq \nu(B)\) pour tout borélien \( B\).
\end{lemma}

\begin{proof}
    Si \( Q\) est un cube semi-ouvert, c'est-à-dire de la forme
    \begin{equation}
        Q=\prod_{i=1}N\mathopen[ a_n , a_n+h \mathclose[\subset U
    \end{equation}
    alors \( Q\) est une réunion croissante de cubes fermés du type \( \mathopen[ a_n+\epsilon , a_n+h-\epsilon \mathclose]\), et donc \( \mu(Q)\leq \nu(Q)\) par le lemme~\ref{LemAZGByEs}\ref{ItemJWUooRXNPci}. La propriété est donc vraie pour les cubes semi-ouverts.

    Si \( \Omega\) est un ouvert, alors il est réunion disjointe dénombrable de cubes semi-ouverts par la proposition~\ref{PropSKXGooRFHQst}. Donc pour tout ouvert \( \Omega\subset U\) nous avons \( \mu(\Omega)\leq\nu(\Omega)\). En vertu de la proposition~\ref{PropNCASooBnbFrc} et de la remarque~\ref{RemooOAGCooRHpjxd}, les mesures \( \mu\) et \( \nu\) sont régulières, et l'inégalité au niveau des ouverts se répercute en inégalité pour tout boréliens de \( U\) :
    \begin{equation}
        \mu(B)\leq \nu(B)
    \end{equation}
    pour tout \( B\in\Borelien(U)\). Notons que \( U\) étant ouvert dans \( \eR^N\), les boréliens de \( U\) sont exactement les boréliens de \( \eR^N\) inclus dans \( U\) par le corolaire~\ref{CorooMJQYooFfwoTd}.
\end{proof}

\begin{lemma}[\cite{PMTIooJjAmWR}]      \label{LemooJCEDooBRyjRg}
    Soit une application \( \theta\colon U\to \eR^N\) de classe \( C^1\) où \( U\) est ouvert dans \( \eR^N\). Pour tout \( Q\in\mQ\) nous avons
    \begin{equation}
        \lambda_N\big( \theta(Q) \big)\leq\sup_{s\in Q}\| d\theta_s \|^N\lambda_N(Q).
    \end{equation}
\end{lemma}

\begin{proof}
    Nous notons \( h\) la longueur du côté du cube. Le théorème des accroissements finis~\ref{val_medio_2}, pour la composante \( \theta_i\) donne, pour \( u,v\in Q\) :
    \begin{equation}        \label{EqooFZMAooKWdzxJ}
        \big|  \theta_i(u)-\theta_i(v) \big|\leq\sup_{s\in Q}\| (d\theta_i)_s \|\| u-v \|\leq \sum_{s\in Q}\| (d\theta_i)_s \|h.
    \end{equation}
    D'autre part nous avons (nous écrivons pour \( N=2\) pour être plus court) :
    \begin{equation}
        d\theta_s(u)=\Dsdd{ \theta_1(s+tu)e_1+\theta_2(s+tu)e_2 }{t}{0}=(d\theta_1)_s(u)e_1+(d\theta_2)_s(u)e_2.
    \end{equation}
    Donc pour chaque \( i\) : \( \| d\theta_s \|\geq \| (d\theta_i)_s \|\), et nous continuons la majoration \eqref{EqooFZMAooKWdzxJ} :
    \begin{equation}
        \big|  \theta_i(u)-\theta_i(v) \big|\leq\leq \sum_{s\in Q}\| (d\theta_i)_s \|h\leq \sup_{s\in Q}\| d\theta_s \|h.
    \end{equation}

    Les points \( \theta(u)\) et \( \theta(v)\) sont donc dans un cube de côté \( \sup_{s\in Q}\| d\theta_s \|h\), ce qui permet de majorer \( \lambda_N\big( \theta(Q) \big)\) par
    \begin{equation}
        \lambda_N\big( \theta(Q) \big)\leq \left( \sup_{s\in Q}\| d\theta_s \|h \right)^N=\left( \sup_{s\in Q}\| d\theta_s \| \right)^N\lambda_N(Q)
    \end{equation}
    où le dernier facteur provient de l'égalité \( h^N=\lambda_N(Q)\).
\end{proof}

%---------------------------------------------------------------------------------------------------------------------------
\subsection{Déterminant et mesure de Lebesgue}
%---------------------------------------------------------------------------------------------------------------------------

Dans la suite, \( Q_0\) désigne le cube unité : \( Q_0=\big( \mathopen[ 0 , 1 \mathclose[ \big)^N\).

\begin{theorem}[Interprétation géométrique du déterminant\cite{PMTIooJjAmWR}]    \label{ThoBVIJooMkifod}
    Soit une application linéaire \( T\colon \eR^N\to \eR^N\). Alors pour tout borélien \( B\) de \( \eR^N\),
    \begin{equation}
        \lambda_N\big( T(B) \big)=| \det(T) |\lambda_N(B).
    \end{equation}
\end{theorem}
\index{déterminant!interprétation géométrique}

\begin{proof}
    Nous considérons la mesure positive \( \mu\) donnée par \( \mu(B)=\lambda_N\big( T(B) \big)\), qui est bien une mesure par la proposition~\ref{PropJCJQooAdqrGA}. Cette mesure est invariante par translation parce que \( \lambda_N\) l'est :
    \begin{equation}
        \mu(B+a)=\lambda_N\big( T(B)+a \big)=\lambda_N\big( T(B) \big)=\mu(B).
    \end{equation}
    De plus, \( T(Q_0)\) est borné et nous notons \( \mu(Q_0)=C\). Nous avons \( \mu=C\lambda_N\) par le corolaire~\ref{CorKGMRooHWOQGP}.

    \begin{subproof}
        \item[\( C(T_1T_2)=C(T_1)C(T_2)\)]
            Par définition,
            \begin{subequations}
                \begin{align}
                    C(T_1T_2)\lambda_N(B)&=\lambda_N\big( (T_1T_2)(B) \big)\\
                    &=\lambda_N\big( T_1(T_2B) \big)=C(T_1)\lambda_N\big( T_2(B) \big)=C(T_1)C(T_2)\lambda_N(B).
                \end{align}
            \end{subequations}
            Par conséquent la fonction \( C\) est multiplicative :
            \begin{equation}
                C(T_1T_2)=C(T_1)C(T_2).
            \end{equation}
            Et en plus, \( C(\id)=1\).
        \item[Matrice diagonale]
            En guise de \( T\), nous considérons l'application linéaire diagonale donnée par \( De_i=d_ie_i\), ou, sous forme matricielle, \( D=\diag(d_1,\ldots, d_N)\) qui fait
            \begin{equation}
                T(Q_0)=\mathopen[ 0 , d_1 \mathclose[\times \ldots\times \mathopen[  0, d_N \mathclose[
            \end{equation}
            La mesure de cela est \( |d_1\cdots d_N|\), ce qui nous donne
            \begin{equation}
                C(D)=| d_1\ldots d_N |=| \det(D) |.
            \end{equation}
        \item[Matrice orthogonale]
            Nous considérons maintenant \( T=U\) où \( U\) est une matrice orthogonale (\( UU^t=1\)). Une matrice orthogonale est une isométrie\footnote{Proposition~\ref{PropKBCXooOuEZcS}.} qui conserve donc la boule unité : \( UB(0,1)=B(0,1)\). Nous avons
            \begin{equation}
                \lambda_N\big( B(0,1) \big)=\lambda_N\big( UB(0,1) \big)=C(U)\lambda_N\big( B(0,1) \big)
            \end{equation}
            par conséquent \( C(U)=1\), et \( 1\) est justement le déterminant de \( U\).
        \item[Matrice quelconque]
            Nous savons par le corolaire~\ref{CorAWYBooNCCQSf} de la décomposition polaire que toute matrice peut être écrite sous la forme \( T=U_1DU_2\) où \( U_i\) sont orthogonales et \( D\) est diagonale. Donc \( C(T)=C(U_1)C(D)C(U_2)=\det(U_1)\det(D)\det(U_2)=\det(U_2DU_2)=\det(T)\) parce que le déterminant est multiplicatif (proposition~\ref{PropYQNMooZjlYlA}\ref{ItemUPLNooYZMRJy}).
    \end{subproof}
\end{proof}

Ce théorème donne une interprétation géométrique du déterminant en tant que facteur de dilatation des volumes lors de l'utilisation d'une application linéaire. Si \( T\) est une application linéaire quelconque,
\begin{equation}
    \lambda_N\big( T(Q_0) \big)=| \det(T) |\lambda_N(Q_0)=| \det(T) |.
\end{equation}
Le déterminant de \( T\) est le volume de l'image du cube unité par l'application \( T\).

De la même façon, en utilisant l'application linéaire \( T(x)=ax\) nous avons pour tout borélien \( B\) :
\begin{equation}
    \lambda_N(aB)=a^N\lambda_N(B).
\end{equation}
Une dilatation d'un facteur \( a\) des longueurs provoque une multiplication par \( a^N\) des volumes.

%---------------------------------------------------------------------------------------------------------------------------
\subsection{Le théorème et sa démonstration}
%---------------------------------------------------------------------------------------------------------------------------

\begin{theorem}[Changement de variable\cite{VSMEooLwNLHd,PMTIooJjAmWR}]         \label{THOooUMIWooZUtUSg}
    Soient \( U\) et \( V\) des ouverts de \( \eR^N\) ainsi qu'un \( C^1\)-difféomorphisme \(\phi\colon U\to V\). Nous notons \( J_{\phi}\) la fonction
    \begin{equation}
        \begin{aligned}
            J_{\phi}\colon \eR^N&\to \eR \\
            a&\mapsto \det(d\phi_a). 
        \end{aligned}
    \end{equation}
    Alors :
    \begin{enumerate}
        \item   \label{ItemVWYDooOzwnyfi}
            Si \( E\subset U\) est borélien, alors \( \phi(E)\) est borélien et
            \begin{equation}
                \lambda_N\big( \phi(E) \big)=\int_E| J_{\phi} |d\lambda_N,
            \end{equation}
            c'est-à-dire \( \phi^{-1}(\lambda_N)=| J_{\phi} |\cdot \lambda_N\).
        \item       \label{ITEMooEZUBooGBuDOS}
            Si \( f\colon V\to \mathopen[ 0 , +\infty \mathclose]\) est mesurable alors la fonction
            \begin{equation}
                (f\circ\phi)\times | J_{\phi} |\colon U\to \mathopen[ 0 , \infty \mathclose]
            \end{equation}
            l'est également et\footnote{L'intégrabilité d'une fonction est la définition~\ref{DefTCXooAstMYl} qui stipule que l'intégrale de \( | f(x) |\) est finie. L'égalité proposée a un sens si les deux membres sont infinis. Il n'y a donc pas d'hypothèses d'intégrabilité obligatoire pour écrire une intégrale lorsque la fonction a des valeurs positives.}
            \begin{equation}        \label{EqRANEooQsFhbC}
                \int_Vfd\lambda_N=\int_U(f\circ\phi)(x)| J_{\phi}(x) |d\lambda_N(x).
            \end{equation}
        \item       \label{ITEMooAJGDooGHKnvj}
            Si \( f\colon V\to \eC\) est mesurable alors elle est intégrable si et seulement si \( (f\circ \phi)\times | J_{\phi} |\colon U\to \eC\) est intégrable. Si c'est le cas, alors nous avons encore la formule de changement de variables :
            \begin{equation}        \label{EQooLYAWooTArAZR}
                \int_Vfd\lambda_N=\int_{\phi^{-1}(V)} (f\circ \phi)| J_{\phi} |d\lambda_N.
            \end{equation}
    \end{enumerate}
\end{theorem}


\begin{proof}
    Attention : la preuve va être longue.
    \begin{enumerate}
        \item
            Le fait que \( \phi(E)\) soit borélien lorsque \( E\) l'est est la proposition~\ref{PropRDRNooFnZSKt}. En ce qui concerne la formule annoncée, il faut travailler.
            \begin{subproof}
            \item[Inégalité dans un sens (cubes)]
                Nous commençons par prouver l'inégalité
                \begin{equation}        \label{EqooQCXXooSjGzks}
                    \lambda_N\big( \phi(Q) \big)\leq \int_Q| J_{\phi}(x) |dx
                \end{equation}
                pour tout \( Q\in \mQ\). On peut diviser le côté du cube \( Q\) en \( k\) éléments de longueurs égales. Le cube est alors divisé en \( k^N\) petits cubes d'intérieurs disjoints. Nous les nommons \( Q_i\) (\( i=1,\ldots, k^N\)) Nous avons alors
                \begin{equation}
                    \sum_i\lambda_N(Q_i)=\sum_i\lambda_N\big( \Int(Q_i) \big)=\lambda_N\big( \bigcup_i\Int(Q_i) \big)\leq \lambda_N(Q)\leq \sum_i\lambda_N(Q_i).
                \end{equation}
                La dernière inégalité est le fait que les intersections ne sont pas disjointes. Toutes ces inégalités sont en réalité des égalités et en particulier : \( \lambda_N(Q)=\sum_i\lambda_N(Q_i)\).

                Soit \( a\in Q_i\). Posons
                \begin{equation}
                    \begin{aligned}
                        \theta&\colon U&\to U \\
                        \theta&=(d\phi_{a})^{-1}\circ\phi
                    \end{aligned}
                \end{equation}
                Cela appelle deux commentaires. D'abord l'application \( d\phi_{a}\colon U\to V\) est inversible parce que \( \phi\) est un difféomorphisme (lemme~\ref{LemooTJSZooWkuSzv}). Ensuite, l'application \( \theta\) est la composée de \( (d\phi_{a})\) (qui est linéaire) et de \( \phi\) qui est de classe \( C^1\); donc \( \theta\) est de classe \( C^1\). Donc le lemme~\ref{LemooJCEDooBRyjRg} s'applique. La différentielle de \( \theta\) n'est pas trop compliquée à écrire parce que nous avons la formule de différentielle d'une composée (théorème~\ref{ThoAGXGuEt}) et le fait que \( (d\phi_{a})^{-1}\) qui est linéaire et donc sa propre différentielle (lemme~\ref{LemooXXUGooUqCjmp}). Nous avons donc \( d\theta=(d\phi_a)^{-1}\circ d\phi\), et le lemme donne
                \begin{equation}
                    \lambda_N\left( (d\phi_a)^{-1}\phi(a) \right)\leq \sup_{s\in Q_i}\|    (d\phi_a)^{-1}\circ d\phi_s  \|^N\lambda_N(Q_i)
                \end{equation}
                Étant donné que \( (d\phi_a)^{-1}\) est une application linéaire, la proposition~\ref{ThoBVIJooMkifod} s'applique, et donc
                \begin{equation}
                    \lambda_N\left( (d\phi_a)^{-1}\phi(a) \right)=| \det(d\phi_a)^{-1} |\lambda_N\big( \phi(a) \big).
                \end{equation}
                Le déterminant d'une application réciproque est donné par la proposition~\ref{PropYQNMooZjlYlA}\ref{ItemooPJVYooYSwqaE} :
                \begin{equation}
                    \det\big( (d\phi_a)^{-1} \big)=\frac{1}{ \det\big( d\phi_a \big) }=\frac{1}{ J_{\phi}(a) }.
                \end{equation}
                Recollant les morceaux,
                \begin{equation}
                    \lambda_N\big( \phi(Q_i) \big)\frac{1}{ J_{\phi}(a) }\leq \sup_{s\in Q_i}\| (d\phi_a)^{-1}\circ d\phi_s \|^N\lambda_N(Q_),
                \end{equation}
                ou encore :
                \begin{equation}
                    \lambda_N\big( \phi(Q_i) \big)\leq | J_{\phi}(a) |\sup_{s\in Q_i}\| (d\phi_a)^{-1}\circ d\phi_s \|^N\lambda_N(Q_i).
                \end{equation}
                Vu que \( a\) et \( s\) sont proches l'un de l'autre (on peut choisir encore la taille du cube), nous pouvons espérer que \( (d\phi_a)^{-1}\) ne soit pas loin d'être l'inverse de \( d\phi_s\). Et c'est en effet le cas. Pour s'en assurer, remarquons que l'application
                \begin{equation}
                    d\phi\colon Q_i\to \aL(\eR^N,\eR^N)
                \end{equation}
                est continue et même uniformément continue parce que \( Q_i\) est compact. De plus la composition de différentielles étant un produit de matrices nous pouvons permuter la limite dans le calcul suivant :
                \begin{equation}
                    \lim_{s\to a}(d\phi_a)^{-1}\circ d\phi_s=(d\phi_a)^{-1}\circ\lim_{s\to a}d\phi_s=\mtu.
                \end{equation}
                Donc si \( \epsilon>0\) est donné, il existe \( \delta\) tel que pour tout \( s\in B(a,\delta)\), \( \| (d\phi_a)^{-1}\circ d\phi_s-\mtu \|\leq \epsilon\). En ce qui concerne les  normes, si \( \| A-\mtu \|\leq \epsilon\) alors \( \| A \|\leq \| A-\mtu \|+\| \mtu \|\leq \epsilon+1\).

                Cela étant dit, nous nous souvenons que nous avions découpé \( U\) en un nombre fini de cubes \( Q_i\) d'égales dimensions; il suffit de prendre \( k\) suffisamment grand pour que la diagonale des cubes sot plus petite que le minimum des \( \delta_i\). Avec un tel découpage,
                \begin{equation}
                    \sup_{s\in Q_i}\| (d\phi_a)^{-1}\circ d\phi_s \|\leq 1+\epsilon
                \end{equation}
                et par conséquent
                \begin{equation}        \label{EqooQRMNooZduAkX}
                    \lambda_N\big( \phi(Q_i) \big)\leq (1+\epsilon)^N| J_{\phi}(a_i) |\lambda_N(Q_i)
                \end{equation}
                où nous avons ajouté un indice \( i\) au point \( a\) pour nous rappeler que nous avons choisi \( a\in Q_i\).

                Le théorème de la moyenne~\ref{ThoooEZLGooMChwLT} appliqué à l'intégrale \( \int_{Q_i}| J_{\phi}(t) |d\lambda_N(t)\) donne l'existence d'un \( a_i\in Q_i\) tel que
                \begin{equation}
                    | J_{\phi}(a_i) |=\frac{1}{ \lambda_N(Q_i) }\int_{Q_i}| J_{\phi} |d\lambda_N.
                \end{equation}
                Ce point \( a_i\) vérifie l'inégalité \eqref{EqooQRMNooZduAkX} comme tout point de \( Q_i\). Nous sommons ces inégalités sur tous les \( i\) :
                \begin{subequations}
                    \begin{align}
                        \lambda_N\big( \phi(Q) \big)&\leq\sum_i\lambda_N\big( \phi(Q_i) \big)\\
                        &\leq (1+\epsilon^N\sum_i\left( \frac{1}{ \lambda_N(Q_i)\int_{Q_i}| J_{\phi} |d\lambda_N } \right)\lambda_N(Q_i)\\
                        &=(1+\epsilon)^N\sum_i\int_{Q_i}| J_{\phi} |d\lambda_N\\
                        &=(1+\epsilon)^N\int_Q| J_{\phi} |d\lambda_N
                    \end{align}
                \end{subequations}
                où nous avons utilisé le fait que \( \mtu_Q=\sum_i\mtu_{Q_i}\) presque partout. En prenant le limite \( \epsilon\to 0\) nous trouvons
                \begin{equation}
                    \lambda_N\big( \phi(Q) \big)\leq \int_Q| J_{\phi} |d\lambda_N.
                \end{equation}
                L'inégalité \eqref{EqooQCXXooSjGzks} est prouvée.
            \item[Inégalité pour les boréliens]

                Soit \( B\) un borélien de \( U\). Vu que \( U\) et \( V\) sont des ouverts de \( \eR^N\), les mesures de Lebesgue sur \( U\) et sur \( V\) sont les mêmes que celles sur \( \eR^n\)  par le corolaire~\ref{CorooMJQYooFfwoTd}.

                Par les définitions~\ref{PropooVXPMooGSkyBo} et~\ref{PropJCJQooAdqrGA}, les applications \( \mu\) et \( n\) définies par \( \mu=\phi^{-1}(\lambda_N)\) et \( \nu=| J_{\phi} |\lambda_N\) sont des mesures positives sur \( U\) (de Borel, qui plus est). L'inégalité \eqref{EqooQCXXooSjGzks} à peine prouvée s'écrit \( \mu(Q)\leq \nu(Q)\) pour tout cube \( Q\). Le lemme~\ref{LemooJYCGooIkkDVn} nous dit alors que l'inégalité tient pour tout borélien.

            \item[Inégalité dans l'autre sens]

                En utilisant la notation de la mesure image et du produit d'une mesure par une fonction\footnote{Définition~\ref{PropJCJQooAdqrGA} et~\ref{PropooVXPMooGSkyBo}}, nous pouvons écrire l'inégalité prouvée sous la forme \( \phi^{-1}(\lambda_N)\leq | J_{\phi} |\lambda_N\). En inversant les rôles de \( U\) et \( V\) (et donc de \( \phi\) et \( \phi^{-1}\)) nous avons aussi
                \begin{equation}
                    \phi(\lambda_N)\leq| J_{\phi^{-1}} |\lambda_N.
                \end{equation}
                En y appliquant \( \phi^{-1}\) et le lemme~\ref{PropJCJQooAdqrGA},
                \begin{equation}        \label{EqooHJCHooVIaheI}
                    \lambda_N\leq \phi^{-1}\big( | J_{\phi^{-1}} |\lambda_N \big).
                \end{equation}
                Nous prouvons à présent que \( \phi^{-1}\big( | J_{\phi^{-1}} |\cdot \lambda_N \big)=\Big( | J_{\phi^{-1}} |\circ\phi \Big)\cdot \phi^{-1}(\lambda_N)\) en appliquant à un borélien \( B\) de \(U\).
                D'une part
                \begin{subequations}
                    \begin{align}
                        \phi^{-1}\big( | J_{\phi^{-1}} |\cdot\lambda_N \big)(B)&=\big( | J_{\phi^{-1}} |\cdot\lambda_N \big)\phi(B)\\
                        &=\int_{\phi(B)}| J_{\phi^{-1}} |d\lambda_N,
                    \end{align}
                \end{subequations}
                et d'autre part,
                \begin{subequations}
                    \begin{align}
                        \big( | J_{\phi^{-1}} |\circ\phi \big)\cdot\phi^{-1}(\lambda_N)B&=\int_{\eR^N}\mtu_B(x)\big( | J_{\phi^{-1}} |\circ\phi \big)(x)d\big( \phi^{-1}(\lambda_N) \big)(x)\\
                        &=   \int_{\eR^N}\mtu_B\big( \phi^{-1}(x) \big)\big( | J_{\phi^{-1}} |\circ\phi \big)\big( \phi^{-1}(x) \big)d\lambda_N(x)       \label{ooDKSWooXwQwgO}\\
                        &=\int_{\eR^N}\mtu_{\phi(B)}| J_{\phi^{-1}} |\\
                        &=\int_B| J_{\phi^{-1}} |d\lambda_N.
                    \end{align}
                \end{subequations}
                Justification :
                \begin{itemize}
                    \item Pour \eqref{ooDKSWooXwQwgO}, le théorème~\ref{THOooVADUooLiRfGK}\ref{ItemooLAPYooUreDEl}.
                \end{itemize}

                L'équation \eqref{EqooHJCHooVIaheI} devient alors
                \begin{equation}
                    \lambda_N\leq \big( | J_{\phi^{-1}} |\circ\phi \big)\cdot \phi^{-1}(\lambda_N).
                \end{equation}
                Nous allons faire le produit de cette mesure par \( | J_{\phi} |\) en nous souvenant que \( J_{\phi}(x)=\det\big( d\phi_x \big)\). Par le lemme~\ref{LemooTJSZooWkuSzv} nous avons aussi \(   (d\phi_x)^{-1}=d\phi^{-1}_{\phi(x)} \) et donc, par la propriété~\ref{PropYQNMooZjlYlA}\ref{ITEMooZMVXooLGjvCy} du déterminant,
                \begin{equation}
                    J_{\phi}(x)=\frac{1}{ \det\big( d\phi^{-1}_{\phi(x)} \big) }=\frac{1}{ J_{\phi^{-1}}\big( \phi(x) \big) }.
                \end{equation}
                Nous avons
                \begin{equation}
                    | J_{\phi} |\cdot\lambda_N\leq | J_{\phi} |\cdot\big( | J_{\phi^{-1}} |\circ\phi \big)\cdot\phi^{-1}(\lambda_N).
                \end{equation}
                En utilisant la proposition~\ref{PropooJMWAooDzfpmB}, il s'agit de multiplier la mesure \( \phi^{-1}(\lambda_N)\) par la fonction
                \begin{equation}
                    x\mapsto | J_{\phi}(x)J_{\phi^{-1}}\big( \phi(x) \big) |=1.
                \end{equation}
                Nous avons donc bien
                \begin{equation}
                    | J_{\phi} |\cdot \lambda_N\leq \phi^{-1}(\lambda_N),
                \end{equation}
                et donc l'égalité
                \begin{equation}
                    | J_{\phi} |\cdot\lambda_N=\phi^{-1}(\lambda_N),
                \end{equation}
                c'est-à-dire le point~\ref{ItemVWYDooOzwnyfi}.
            \end{subproof}
        \item
            Le fait que la fonction proposée soit mesurable est le fait que la mesurabilité n'est pas affectée par produit et composition (propositions~\ref{PROPooODDVooEEmmTX} et~\ref{PROPooEFHKooARJBwW}), et le fait que pour les mêmes raisons, l'application \( J_{\phi}\colon U\to \eR\) est également mesurable. En ce qui concerne la formule nous allons la démontrer dans le cas de fonctions de plus en plus générales.
            \begin{subproof}
            \item[Pour les fonctions indicatrices]
                Soit \( B\) un borélien de \( U\). Considérons la fonction \( f=\mtu_{\phi(B)}\). Alors
                \begin{equation}    \label{EqYXRFooJEqVBH}
                        \int_V fd\lambda_N=\int_{\eR^N}\mtu_{\phi(B)}(y)\mtu_V(y)d\lambda_N(y)
                        =\int_{\eR^N}\mtu_{\phi(B)}d\lambda_N
                        =\lambda_N\big( \phi(B) \big).
                \end{equation}
                parce que \( V=\phi(U)\) et \( B\subset U\), donc \( \mtu_{\phi(B)}\mtu_{\phi(U)}=\mtu_{\phi(B)}\). D'autre part, pour calculer l'autre membre de \eqref{EqRANEooQsFhbC} nous remarquons que \( f=\mtu_{\phi(B)}=\mtu_B\circ\phi^{-1}\), ce qui donne
                \begin{equation}        \label{EqHWRQooKIfPTu}
                    \int_Uf\big( \phi(x) \big)| J_{\phi}(x) |d\lambda_N(x)=\int_U\mtu_B| J_{\phi} |d\lambda_N=\int_B| J_{\phi} |d\lambda_N.
                \end{equation}
                L'ensemble \( B\) étant borélien, il est extrêmement mesurable, ce qui fait que le point~\ref{ItemVWYDooOzwnyfi} s'applique : les expressions \eqref{EqYXRFooJEqVBH} et \eqref{EqHWRQooKIfPTu} sont égales.

            \item[Pour les fonctions étagées]

                   Soit \( f\colon V\to \eR^+\) une fonction étagée :
                   \begin{equation}
                       f(x)=\sum_{i=1}^na_i\mtu_{A_i}(x)
                   \end{equation}
                   Nous pouvons faire le calcul suivant :
                   \begin{subequations}
                       \begin{align}
                           \int_Vfd\lambda_N&=\int_V\sum_ia_i\mtu_{A_i}d\lambda_N\\
                           &=\sum_ia_i\int_{V}\mtu_{A_i}d\lambda_N      \label{ooNESRooDuNUYF}\\
                           &=\sum_i\int_U(\mtu_{a_i}\circ\phi)(x)| J_{\phi}(x) |d\lambda_N(x)   \label{ooYXHSooKMPrIT}\\
                           &=\sum_ia_i\int_U\mtu_{\phi^{-1}(A_i)}| J_{\phi}(x) |d\lambda_N(x)\\
                           &=\int_V\underbrace{\sum_ia_i\mtu_{\phi^{-1}(A_i)}(x)}_{=(f\circ\phi)(x)}| J_{\phi}(x) |d\lambda_N(x)\\
                           &=\int_V(f\circ\phi)| J_{\phi} |d\lambda_N.
                       \end{align}
                   \end{subequations}
                   Justifications :
                   \begin{itemize}
                       \item Pour \eqref{ooNESRooDuNUYF} : linéarité de l'intégrale, théorème~\ref{ThoooCZCXooVvNcFD}\ref{ITEMooBLEVooDznQTY}\footnote{Il est remarquable que nous n'utilisons cette linéarité que pour les fonctions étagées.}
                       \item Pour \eqref{ooYXHSooKMPrIT} : le cas des fonctions indicatrices est utilisé pour chaque \( i\) entre \( 1\) et \( n\).
                   \end{itemize}

               \item[Fonction mesurable positive]
                   Soit \( f\colon V\to \mathopen[ 0 , \infty \mathclose]\). Par le théorème fondamental d'approximation~\ref{THOooXHIVooKUddLi}, il existe une suite croissante de fonctions étagées et mesurables \( \varphi_n\colon V\to \mathopen[ 0 , \infty \mathclose[\) dont la limite ponctuelle est \( f\).  Nous avons alors le calcul suivant :
                       \begin{subequations}
                           \begin{align}
                               \int_Vfd\lambda_N&=\lim_{n\to \infty} \int_V\varphi_nd\lambda_N  \label{ooGMMFooXLHijj}\\
                               &=\lim_{n\to \infty} \int_U(\varphi_n\circ\phi)| J_{\phi} |d\lambda_N \label{ooWIFWooXELNUs}\\
                               &=\int_U\lim_{n\to \infty} (\varphi_n\circ\phi)| J_{\phi} |d\lambda_N \label{ooNKXNooUYeWKo}\\
                               &=\int_U(f\circ\phi)| J_{\phi} |d\lambda_N       \label{ooOAIDooAILHIB}.
                           \end{align}
                       \end{subequations}
                       Justifications :
                       \begin{itemize}
                           \item Pour \eqref{ooGMMFooXLHijj}, c'est le théorème de la convergence monotone~\ref{ThoRRDooFUvEAN}.
                           \item Pour \eqref{ooWIFWooXELNUs}, c'est le présent théorème pour la fonction étagée \( \varphi_n\).
                           \item Pour \eqref{ooNKXNooUYeWKo}, c'est encore la convergence dominée, justifiée par le fait que \(  \varphi_n\circ\phi    \) est également une suite croissante : si \( x\in U\) alors \( \varphi_{n+1}\big( \phi(x) \big)\geq \varphi_n\big( \phi(x) \big)   \).\
                           \item Pour \eqref{ooOAIDooAILHIB}, c'est la limite ponctuelle \( \varphi_n\big( \phi(x) \big)\to f\big( \phi(x) \big)\).
                       \end{itemize}
            \end{subproof}
        \item
            La partie sur l'intégrabilité repose sur le fait que  \( | f |\circ\phi=| f\circ\phi |\). Ici \( | . |\) est le module et non une valeur absolue. Les faits suivants sont équivalents :
            \begin{itemize}
                \item la fonction \( f\colon V\to \eC\) est intégrable
                \item la fonction \( | f |\colon V\to \eR\) est intégtrable
                \item la fonction \( (| f |\circ\phi)| J_{\phi} |\colon U\to \eR\) est intégrable (par le point~\ref{ITEMooEZUBooGBuDOS}).
                \item la fonction \( (f\circ\phi)| J_{\phi} |\colon U\to \eR\) est intégrable.
            \end{itemize}
            En ce qui concerne la formule, il s'agit seulement d'appliquer le point~\ref{ITEMooEZUBooGBuDOS} aux parties positives, négatives, imaginaires et réelles de \( f\).
    \end{enumerate}
\end{proof}

Notons que la formule peut être écrite sous la forme
\begin{equation}        \label{EQooQKARooELPCFO}
    \langle f, g\rangle_V=\langle f\circ\phi, (g\circ\phi)| J |\rangle_U,
\end{equation}
qui est plus pratique lorsqu'on parle de produits scalaires. Pour rappel, \( \phi\colon U\to C\) est un \( C^1\)-difféomorphisme.

\begin{normaltext}
La formule de changement de variables peut être comprise de la façon suivante. Si $\phi$ est linéaire  alors le facteur $|J_{\phi}|$ est la mesure de l'image par $\phi$ d'une portion de $\eR^p$ de mesure $1$, sinon  $|J_{\phi}|$ est le rapport entre la mesure de l'image d'un élément infinitésimale de volume de $\eR^p$ et sa mesure originale.

Soit $\phi(u,v)=g(u,v)e_1+h(u,v)e_2$ un difféomorphisme dans $\eR^2$. Soit $(x_0, y_0)$ l'image par $\phi$ de $(u_0,v_0)$. On considère le petit rectangle $R$ de sommets $(u_0,v_0)$, $(u_0+\Delta u,v_0)$, $(u_0+\Delta u,v_0+\Delta v)$ et $(u_0,v_0+\Delta v)$. L'image de $R$ n'est pas un rectangle en général, mais peut être bien approximée par le rectangle de sommets $(x_0,y_0)$, $(x_0 ,y_0)+ \phi_{u}\Delta u$, $(x_0 ,y_0)+\phi_{u}\Delta u +\phi_{v}\Delta v$ et  $(x_0 ,y_0)+ \phi_{v}\Delta v$ et son aire est $\| \phi_{u}\times \phi_{v}\| \Delta u\Delta v$. La valeur $|\phi_{u}\times \phi_{v}|$ est exactement $|J_{\phi}|$
\end{normaltext}

%---------------------------------------------------------------------------------------------------------------------------
\subsection{Exemples}
%---------------------------------------------------------------------------------------------------------------------------

Un exemple avec une exponentielle est donnée dans l'exemple \ref{EXooNIOZooWxciAC}.

Énormément d'exemples sont disponibles avec les coordonnées polaires et toutes leurs variations. Cependant les fonctions trigonométriques ne seront vues que plus tard; les coordonnées polaires, cylindrique et sphériques seront vues en section \ref{SECooWTPRooZbOSzO} et les exemples d'utilisation pour les intégrales seront dans la section \ref{SECooOOPPooZLbaEH}.

%+++++++++++++++++++++++++++++++++++++++++++++++++++++++++++++++++++++++++++++++++++++++++++++++++++++++++++++++++++++++++++ 
\section{Changement d'espace mesuré}
%+++++++++++++++++++++++++++++++++++++++++++++++++++++++++++++++++++++++++++++++++++++++++++++++++++++++++++++++++++++++++++

\begin{proposition}[\cite{MonCerveau}]      \label{PROPooILOEooBiumKD}
    Soit un espace mesuré \( (\Omega,\tribA,\mu)\). Soit un ensemble \( \Omega'\) et une bijection \( \varphi\colon \Omega\to \Omega'\). Nous posons
    \begin{enumerate}
        \item
            \( \tribA'=\varphi(\tribA)\),
        \item
            \( \mu'(B)=\mu\big( \varphi^{-1}(B) \big)\) pour tout \( B\in\tribA'\).
    \end{enumerate}
    Soit enfin une fonction mesurable \( f\colon \Omega\to X\).

    Alors
    \begin{enumerate}
        \item
            Le triple \( (\Omega',\tribA',\mu')\) est un espace mesuré.
        \item
            L'application \( f\circ\varphi^{-1}\colon \Omega'\to X\) est mesurable.
        \item
            Nous avons l'égalité
            \begin{equation}
                \int_{\Omega}fd\mu=\int_{\Omega'}(f\circ\varphi^{-1})d\mu'.
            \end{equation}
    \end{enumerate}
\end{proposition}

\begin{proof}
    La proposition \ref{PROPooXQHTooUxJoyq} montre déjà que \( (\Omega',\tribA',\mu')\) est un espace mesuré.

    Soit une partie \( S\) mesurable dans \( X\). Alors \( f^{-1}(S)\) est mesurable dans \( \Omega\) par hypothèse sur \( f\), c'est-à-dire que \( f^{-1}(S)\in\tribA\). Ensuite \( (\varphi\circ f^{-1})(S)\) est mesurable dans \( \Omega'\) par hypothèse sur \( \varphi\). Cela prouve que \(  f\circ\varphi^{-1} \) est une application mesurable.

    Nous avons encore à prouver l'égalité d'intégrale. Par la définition \ref{DefTVOooleEst} nous avons
    \begin{equation}
        \int_{\Omega}fd\mu=\sup\{ \sum_{i=1}^na_i\mu(A_i) \}
    \end{equation}
    où le supremum est sur tous les \( n\) et tous les choix de \( A_i\in\tribA\), \( a_i\in\eR^+\) tels que \( f|_{A_i}>a_i\). Vu que \( \tribA'=\varphi(\tribA)\), si \( A_i\in \tribA\) et \( a_i\) sont choisis, nous avons aussi 
    \begin{equation}
        f\circ^{-1}|_{\varphi(A_i)}\geq a_i
    \end{equation}
    avec \( \varphi(A_i)\in\tribA'\). Donc pour un choix de \( \{ (A_i,a_i) \}\) donné,
    \begin{equation}
        \sum_{i=1}^na_i\mu(A_i)=\sum_{i=1}^na_i\mu'\big( \varphi(A_i) \big).
    \end{equation}
    Au final,
    \begin{equation}
        \int_{\Omega}fd\mu=\sup\{ \sum_{i=1}^na_i\mu(A_i) \}=\sup\{ \sum_ia_i\mu'\big( \varphi(A_i) \big) \}=\int_{\varphi(\Omega)}f\circ\varphi^{-1}d\mu'.
    \end{equation}
\end{proof}

\begin{remark}[Ce n'est pas la mesure que nous voulons]     \label{REMooOMYYooNFiKOs}
    La mesure donnée par la proposition \ref{PROPooILOEooBiumKD} n'est pas celle que nous voulons d'habitude sur \( \Omega'\). Anticipons un peu pour comprendre. Prenons l'exemple de la partie \( C\) de \( \eR\) donnée par
    \begin{equation}
        C=\{ (x,y)\in \eR^2\tq y=x^2,x\in \mathopen] 0 , 3 \mathclose[ \}.
    \end{equation}
    \begin{enumerate}
        \item
            La façon correcte de définir la longueur de \( C\) est de prendre une limite d'approximations par des morceaux de droites, comme fait à la définition \ref{DEFooDNZWooXmxhsU}.
        \item
            Cette définition de la longueur peut être exprimée sous forme intégrale par le théorème \ref{ThoLongueurIntegrale} qui nous assure que
            \begin{equation}
                l(C)=\int_0^3\| \varphi'(t) \|dt=\int_0^3\sqrt{ 1+4t^2 }dt\neq \mu'(C).
            \end{equation}
        En effet, \( \mu'(C)=\mu\big( \varphi^{-1}(C) \big)=\mu\big( \mathopen] 0 , 3 \mathclose[ \big)=3\), alors que pour tout \( t\) nous avons \( \sqrt{ 1+4t^2 }>1\) et donc \( l(C)>3\).
        \item
            Donc \( \mu'\) n'est pas exactement ce que nous aurions pu vouloir appeler la «mesure» de \( C\).
        \item
            La mesure à considérer sur \( C\) doit donc plutôt être quelque chose comme le produit de la mesure \( \mu'\) par la fonction \( \| \varphi' \|\). Mais cela est une autre histoire qui vous sera contée une autre fois.
        \item       \label{ITEMooJTKCooYQknqo}
            Dans le cas de \( S^1\), nous avons \( \varphi(x)= e^{ix}\), et \( \| \varphi'(x) \|=1\). Donc la mesure donné ici est probablement bien celle que nous voulons. Peut-être à coefficient \( \frac{1}{ 2\pi }\) près pour avoir une normalisation \( \mu'(S^1)=1\). Cela est également une autre histoire qui vous sera contée une autre fois; par exemple dans la proposition \ref{PROPooHMSCooRIjcJq}.
    \end{enumerate}
\end{remark}

%+++++++++++++++++++++++++++++++++++++++++++++++++++++++++++++++++++++++++++++++++++++++++++++++++++++++++++++++++++++++++++
\section{Théorème de Fubini-Tonelli et de Fubini}
%+++++++++++++++++++++++++++++++++++++++++++++++++++++++++++++++++++++++++++++++++++++++++++++++++++++++++++++++++++++++++++

Nous rappelons que \( \eR^n\) muni de la mesure de Lebesgue est un espace mesuré \( \sigma\)-fini, conformément à la définition~\ref{DefBTsgznn}.

Le théorème de Fubini-Tonelli parle de fonctions réelles et non complexes, et même positives. Le truc est que ce théorème va servir de base pour construire les autres. Si nous avons une fonction à valeurs complexes, elle se décompose en parties réelles et imaginaires qui elles-mêmes se décomposent en parties positives et négatives. Au final, les preuves pour \( f\colon \Omega\to \eC\) se ramènent à appliquer quatre fois le théorème pour \( f\colon \Omega\to \bar \eR^+\).
\begin{theorem}[Fubini-Tonelli\cite{NBoIEXO}]\label{ThoWTMSthY}
    Soient \( (\Omega_i,\tribA_i,\mu_i)\) deux espaces mesurés \( \sigma\)-finis, et \( (\Omega,\tribA,\mu)\) l'espace produit. Soit une fonction \( f\colon \Omega_1\times \Omega_2\to \eR\) une fonction mesurable et positive (valant éventuellement \( \infty\) à certains endroits)
    Alors :
    \begin{enumerate}
        \item       \label{ITEMooUTMNooVIBdpP}
            Les fonction
            \begin{equation}        \label{EQooWLADooQwNhEy}
                F_1\colon x\mapsto \int_{\Omega_2}f(x,y)d\mu_2(y)
            \end{equation}
            et
            \begin{equation}
                F_2\colon y\mapsto \int_{\Omega_1}f(x,y)d\mu_1(x)
            \end{equation}
            sont mesurables.
        \item   \label{ITEMooFKQUooCoCOLV}
            Toutes les intégrales imaginables existent et sont égales :
            \begin{subequations}    \label{EqJRVtOGx}
                \begin{align}
                    \int_{\Omega_1\times \Omega_2}f(x,y)d(\mu_1\otimes \mu_2)(x,y)&=\int_{\Omega_1}\left[ \int_{\Omega_2}f(x,y)d\mu_2(y) \right]d\mu_1(x)\\
                &=\int_{\Omega_2}\left[ \int_{\Omega_1}f(x,y)d\mu_1(x) \right]d\mu_2(y)
                \end{align}
            \end{subequations}
            où tous les membres de l'égalité valent éventuellement \( +\infty\).
    \end{enumerate}
\end{theorem}
\index{théorème!Fubini-Tonelli}

\begin{proof}
    Commençons par prouver le théorème dans le cas d'une fonction caractéristique d'un ensemble mesurable : \( f(x,y)=\mtu_{A}(x,y)\) pour un certain ensemble \( A\subset \Omega_1\times \Omega_2\). Dans ce cas,
    \begin{equation}
        F_1(x)=\int_{\Omega_2}\mtu_A(x,y)d\mu_2(y)=\int_{\Omega_2}\mtu_{A_1(y)}(x)d\mu_2(y)=\mu_2\big( A_1(x) \big),
    \end{equation}
    et nous avons déjà vu au théorème~\ref{ThoCCIsLhO} que cette fonction \( F_1\) était alors mesurable. En utilisant maintenant les égalités \eqref{EqDFxuGtH} ainsi que le fait que \( \mtu_A(x,y)=\mtu_{A_2(x)}(y)\) nous avons
    \begin{subequations}
        \begin{align}
            \int_{\Omega_1\times \Omega_2}\mtu_A(x,y)d(\mu_1\otimes \mu_2)(x,y)&=(\mu_1\otimes \mu_2)(A)\\
            &=\int_{\Omega_1}\mu_2\big( A_2(x) \big)d\mu_1(x)\\
            &=\int_{\Omega_1}\left[   \int_{\Omega_2}\mtu_{A_2(x)}(y)d\mu_2(y)  \right]d\mu_1(x)\\
            &=\int_{\Omega_1}\left[ \int_{\Omega_2}\mtu_A(x,y)d\mu_2(y) \right]d\mu_1(x).
        \end{align}
    \end{subequations}
    Le théorème étant valable pour les fonctions caractéristiques, il est valable pour les fonctions simples (définition~\ref{DefBPCxdel}) par linéarité de l'intégrale.

    Si \( f\) n'est pas une fonction simple, alors la proposition~\ref{THOooXHIVooKUddLi} nous donne une suite croissante de fonctions simples et positives convergeant ponctuellement vers \( f\). La partie du théorème sur les fonctions simples dit que pour chaque \( n\) l'intégrale
    \begin{equation}
        \int_{\Omega_1\times \Omega_2}f_n(x,y)d(\mu_1\otimes\mu_2)(x,y)
    \end{equation}
    peut être décomposée comme il faut en suivant la formule \eqref{EqJRVtOGx}. Il faut pouvoir permuter la limite et l'intégrale dans chacun de cas. D'abord le théorème de la convergence monotone~\ref{ThoRRDooFUvEAN} appliqué à l'espace \( \Omega_1\times \Omega_2\) dit que
    \begin{equation}
        \lim_{n\to \infty} \int_{\Omega_1\times \Omega_2}f_n(x,y)d(\mu_1\otimes \mu_2)(x,y)= \int_{\Omega_1\times \Omega_2}f(x,y)d(\mu_1\otimes \mu_2)(x,y).
    \end{equation}
    Ensuite, pour chaque \( x\in\Omega_1\), les fonctions
    \begin{equation}
        \sigma_n(y)=\int_{\Omega_1}f_n(x,y)d\mu_1(x)
    \end{equation}
    forment une suite croissante de fonctions mesurables; nous leur appliquons encore le théorème de la convergence monotone :
    \begin{subequations}
        \begin{align}
            \lim_{n\to \infty} \int_{\Omega_2}\left[ \int_{\Omega_1}f_n(x,y)d\mu_1(x) \right]d\mu_2(y)&=\lim_{n\to \infty} \int_{\Omega_2}\sigma_n(y)d\mu_2(y)\\
            &=\int_{\Omega_2}\left[\lim_{n\to \infty} \int_{\Omega_1}f_n(x,y)d\mu_1(x)\right]d\mu_2(y)\\
            &=\int_{\Omega_2}\left[ \int_{\Omega_1}f(x,y)d\mu_1(x) \right]d\mu_2(y)
        \end{align}
    \end{subequations}
    où nous avons utilisé une seconde fois Beppo-Levi.
\end{proof}

\begin{remark}
    Les formules \eqref{EqJRVtOGx} sont bien, mais ne garantissent en aucun cas que \( f\in L^1(\Omega_1\times \Omega_2)\) : il faut encore que ces intégrales soient finies.
\end{remark}

\begin{corollary}[\cite{MesIntProbb}]           \label{CorTKZKwP}
    Soient \( (\Omega_i,\tribA_i,\mu_i)\) deux espaces mesurés \( \sigma\)-finis, et \( (\Omega,\tribA,\mu)\) l'espace produit\footnote{Définition~\ref{DefUMlBCAO}.}. Soit une fonction mesurable \( f\colon \Omega\to \eR\text{ ou }\eC\). Alors les conditions suivantes sont équivalentes
    \begin{enumerate}
        \item   \label{ITEMooZRAXooTRDIlZ}
            \( f\in L^1(\Omega_1\times \Omega_2)\),
        \item       \label{ITEMooJMPLooZKwxQC}
            \begin{equation}
                \int_{\Omega_1}\left[ \int_{\Omega_2}| f |d\mu_2 \right]d\mu_1 <\infty,
            \end{equation}
        \item   \label{ITEMooLLBCooTRycwG}
            \begin{equation}
                \int_{\Omega_2}\left[ \int_{\Omega_1}| f |d\mu_1 \right]d\mu_2 <\infty.
            \end{equation}
    \end{enumerate}
\end{corollary}

\begin{proof}

    Nous commençons par supposer que \( f\) est à valeurs dans \( \eR\). La notation \( | f |\), pour l'instant,  dénote donc bien la valeur absolue et non le module.

    La fonction \( | f |\) est mesurable et positive par hypothèse et par le fait que si \( f\) est mesurable, alors \( | f |\) l'est également par le corolaire~\ref{CORooNXYUooEcvDlP}. Le théorème~\ref{ThoWTMSthY}\ref{ITEMooFKQUooCoCOLV} nous dit alors que les intégrales suivantes existent et sont égales :
    \begin{equation}        \label{EQooAIQGooNtBOuC}
            \int_{\Omega_1\times \Omega_2}| f |d(\mu_1\otimes \mu_2)=\int_{\Omega_1}\left[ \int_{\Omega_2}|f(x,y)|d\mu_2(y) \right]d\mu_1(x)
            =\int_{\Omega_2}\left[ \int_{\Omega_1}|f(x,y)|d\mu_1(x) \right]d\mu_2(y).
    \end{equation}
    Attention : rien ne dit encore que ces intégrales sont finies.

    \begin{subproof}
        \item[\ref{ITEMooZRAXooTRDIlZ} implique~\ref{ITEMooJMPLooZKwxQC} et~\ref{ITEMooLLBCooTRycwG}]
            Si \( f\in L^1(\Omega_1\times \Omega_2)\) alors \( | f |\) y est également. Cela implique que le membre de droite de \eqref{EQooAIQGooNtBOuC} est fini. Les deux autres sont alors également finis.
        \item[\ref{ITEMooJMPLooZKwxQC} ou~\ref{ITEMooLLBCooTRycwG} implique~\ref{ITEMooZRAXooTRDIlZ}]
            Les expressions à droite de \eqref{EQooAIQGooNtBOuC} sont finies. Donc celle de gauche également. Cele signifie que \( | f |\in L^1(\Omega_1\times \Omega_2)\). Par conséquent \( f\) est également dans \(L^1(\Omega_2\times \Omega_2) \).
    \end{subproof}

    Nous passons maintenant au cas où \( f\) est à valeurs dans \( \eC\). Nous décomposons
    \begin{equation}
        f=f_R+if_I
    \end{equation}
    où \( f_R\) et \( f_I\) sont des fonctions réelles. Nous avons
    \begin{equation}        \label{EQooZEOAooIMwKwk}
        \int_{\Omega}| f |\leq \int_{\Omega}| f_R |+\int_{\Omega}| f_I |.
    \end{equation}
    Donc si \( f_R\) et \( f_I\) sont dans \( L^1(\Omega)\), la fonction \( f\) le sera aussi. De même,
    \begin{equation}
        \int_{\Omega}| f_R |\leq \int_{\Omega}| f |,
    \end{equation}
    qui donne l'inverse : si \( f\in L^1(\Omega)\) alors \( f_R,f_I\in L^1(\Omega)\). Bref, \( f\) est intégrable sur \( \Omega\) si et seulement si \( f_R\) et \( f_I\) le sont.

    Supposons que \( f\in L^1(\Omega_1\times \Omega_2)\). Alors
    \begin{subequations}
        \begin{align}
            \int_{\Omega_1}\left[ \int_{\Omega_2}| f |d\mu_2 \right]d\mu_1&\leq \int_{\Omega_1}\left[ \int_{\Omega_2}| f_R | \right]+\int_{\Omega_1}\left[ \int_{\Omega_2}| f_I | \right]<\infty
        \end{align}
    \end{subequations}
    où nous avons appliqué~\ref{ITEMooZRAXooTRDIlZ} implique~\ref{ITEMooJMPLooZKwxQC} aux fonctions \( f_R\) et \( f_I\) qui sont dans \( L^1(\Omega_1\times \Omega_2)\) parce que \( f\) y est.

    Dans l'autre sens, si
    \begin{equation}
        \int_{\Omega_1}\left[ \int_{\Omega_2}| f | \right]<\infty,
    \end{equation}
    alors en remplaçant \( | f |\) par \( | f_R |\) ou par \( | f_I |\) nous restons fini. En appliquant alors «\ref{ITEMooJMPLooZKwxQC} implique~\ref{ITEMooZRAXooTRDIlZ}» nous trouvons que \( f_R\) et \( f_I\) sont dans \( L^1(\Omega_1\times \Omega_2)\). Et cela implique que \( f\in L^1(\Omega_1\times \Omega_2)\).
\end{proof}

\begin{theorem}[Fubini\cite{MesIntProbb}]\label{ThoFubinioYLtPI}
    Soient \( (\Omega_i,\tribA_i,\mu_i)\) deux espaces mesurés \( \sigma\)-finis, et \( (\Omega,\tribA,\mu)\) l'espace produit. Soit
    \begin{equation}
        f\in L^1\big( (\Omega,\tribA),\eC \big),
    \end{equation}
    c'est-à-dire une fonction à valeurs mesurable et intégrable sur \( \Omega\). Alors :
    \begin{enumerate}
        \item       \label{ITEMooVFGWooZTePQS}
            Pour presque tout \( x\in \Omega_1\), la fonction \( y\mapsto f(x,y)\) est \( L^1(\Omega_2)\).
        \item       \label{ITEMooCYMKooUdizni}
            Si nous posons
            \begin{equation}
                \varphi_f(x)=\int_{\Omega_2}f(x,y)d\mu_2(y);
            \end{equation}
            alors \( \varphi_f\in L^1(\Omega_1)\).
        \item   \label{ItemQMWiolgiii}
            Nous avons la formule d'inversion d'intégrale
            \begin{subequations}
                \begin{align}
                \int_{\Omega}fd(\mu_1\otimes \mu_2)&=\int_{\Omega_1}\varphi_fd\mu_1\\
                &=\int_{\Omega_1}\left[ \int_{\Omega_2}f(x,y)d\mu_2(y) \right]d\mu_1(x)\\
                &=\int_{\Omega_2}\left[ \int_{\Omega_1}f(x,y)d\mu_1(x) \right]d\mu_2(y).
                \end{align}
            \end{subequations}
    \end{enumerate}

\end{theorem}
\index{théorème!Fubini!espace mesuré}

\begin{proof}
    Nous commençons par supposer que \( f\) est à valeurs réelles : \( f\in L^1\big( (\Omega_1\times \Omega_2,\tribA_1\otimes\tribA_2 ),\eR\big)\). Nous décomposons la fonction \( f\) en parties positives et négatives : \( f=f^+-f^-\) avec \( f^+\) et \( f^-\) positives ou nulles. Nous avons évidemment
    \begin{equation}
        \int_{\Omega_1\times \Omega_2}| f^+ |\leq \int_{\Omega_1\times \Omega_2}| f |<\infty.
    \end{equation}
    Donc \( f^+\) et \( f^-\) sont des éléments de \( L^1(\Omega_1\times \Omega_2)\).

    \begin{subproof}
    \item[Pour~\ref{ITEMooVFGWooZTePQS}]

    Nous posons
    \begin{equation}
        \varphi_{f^+}(x)=\int_{\Omega_2}f^{+}(x,y)d\mu_2(y)
    \end{equation}
    pour tous les \( x\in \Omega_1\) pour lesquels cette intégrale est bien définie. Vu que \( f^+\) est positive et mesurable, le théorème de Fubini-Tonelli~\ref{ThoWTMSthY}\ref{ITEMooUTMNooVIBdpP} s'applique donc pour nous dire que \( \varphi_{f^+}\) est mesurable.

    De plus le résultat \eqref{EqJRVtOGx} appliqué à \( f^+\) donne
    \begin{equation}        \label{EQooSETWooRwkCuW}
        \int_{\Omega_1}\varphi_{f^+}d\mu_1=\int_{\Omega_1\times \Omega_2}f^+d(\mu_1\otimes \mu_2)<\infty.
    \end{equation}
    Le fait que le tout soit fini est une conséquence du fait déjà mentionné que \( f^+\in L^1(\Omega_1\times \Omega_2\). Vu que \( \varphi_{f^+}\) est une fonction positive, l'inégalité \eqref{EQooSETWooRwkCuW} signifie que \( \varphi_{f^+}\in L^1(\Omega_1,\mu_1)\).

    En particulier, \( \varphi_{f^+}(x)<\infty\) pour presque tout \( x\in\Omega_1\). C'est-à-dire pour presque tout \( x\in \Omega_1\) :
    \begin{equation}
        \int_{\Omega_2}f^+(x,y)d\mu_2(y)<\infty,
    \end{equation}
    et sachant que \( f^+\geq 0\) nous avons \( f^+(x,\cdot)\in L^1(\Omega_2)\) pour presque tout \( x\).

        \item[Pour~\ref{ITEMooCYMKooUdizni}]

            Partout où \( \varphi_{f^+}\) et \( \varphi_{f^-}\) sont finies nous avons
            \begin{equation}
                \varphi_f=\varphi_{f^+}-\varphi_{f^-},
            \end{equation}
            et comme cela a lieu presque partout, nous pouvons considérer une partie mesurable \( A\subset \Omega_1\) telle que \( \mu_1(A)=0\) et \( \varphi_f(x)=\varphi_{f^+}(x)-\varphi_{f^-}(x)\) pour tout \( x\) hors de \( A\). Bref, nous posons
            \begin{equation}
                g(x)=\begin{cases}
                    \varphi_{f^+}-\varphi_{f^-}(x)    &   \text{si } x\in A^c\\
                    0    &    \text{si } x\in A.
                \end{cases}
            \end{equation}
            Cette fonction \( g\) est mesurable et \( g=\varphi_f\) presque partout. De plus
            \begin{equation}
                \int_{\Omega_1}| g |d\mu_1   = \int_{A^c}| g |\leq \int_{A^c}\varphi_{f^+}+\int_{A^c}\varphi_{f^-}<\infty.
            \end{equation}
            La dernière inégalité est le fait que \( \varphi_{f^{\pm}}\) sont dans \( L^1(\Omega_1)\). Et notons au passage que nous aurions pu laisser toutes les intégrales sur \( \Omega_1\) sans faire de précisions sur la distinction entre \( \Omega_1\) et \( A^c\) parce que la partie de \( \Omega_1\) sur laquelle \( \varphi_{f^{\pm}}\) sont infinies est trop petite pour changer la valeur de l'intégrale.

            Nous avons donc \( g\in L^1(\Omega_1)\), et par conséquent également \( \varphi_f\in L^1(\Omega_1)\) parce que ces deux fonctions sont égales presque partout (les classes sont égales).

        \item[Pour~\ref{ItemQMWiolgiii}]

            En utilisant l'équation \eqref{EQooSETWooRwkCuW} nous avons
            \begin{subequations}
                \begin{align}
                    \int_{\Omega_1}\varphi_fd\mu_1&=\int gd\mu_1=\int_{\Omega_1}\varphi_{f^+}-\int_{\Omega_1}\varphi_{f^-}\\
                    &=\int_{\Omega_1\times }f^+d\mu-\int_{\Omega_1\times \Omega_2}f^-d\mu\\
                    &=\int_{\Omega_1\times \Omega2}fd\mu.
                \end{align}
            \end{subequations}
            Et toutes ces intégrales sont finies.
    \end{subproof}

    Et c'est maintenant que nous considérons le cas complexe. Nous décomposons \( f=f_R+if_I\) avec des fonctions réelles \( f_R\) et \( f_I\). Comme déjà mentionné autour de \eqref{EQooZEOAooIMwKwk}, les fonctions \( f_R\) et \( f_I\) sont intégrables. Nous leur appliquons le théorème.

    Les valeurs de \( x\) pour lesquelles \( f_R(x,\cdot)\) et \( f_I(x,\cdot)\) ne sont pas dans \( L^1(\Omega_2)\) forment un ensemble de mesure nulle, nommons le \( A\). En posant
    \begin{equation}
        g(x,y)=\begin{cases}
            f_R(x,y)+if_I(x,y)    &   \text{si } x\in A^c\\
            0    &    \text{si } x\in A,
        \end{cases}
    \end{equation}
    nous avons que \( g(x,\cdot)\) est intégrable pour tout \( x\in A^c\). Vu que pour ces valeurs de \( x\) nous avons \( g(x,y)=f(x,y)\) nous en déduisons que pour \( x\in A^c\) nous avons aussi \( f(x,\cdot)\in L^1(\Omega_2)\).

    Les autres points se traitent de la même façon\quext{Attention : je n'ai pas vérifié explicitement. C'est juste une intuition. Vérifiez et écrivez-moi pour dire si c'est bon ou non.}.
\end{proof}

\begin{normaltext}      \label{NORMooKIRJooPvyPWQ}
    En pratique, il n'est pas toujours évident qu'une fonction soit intégrable sur \( \Omega_1\times \Omega_2\). Pour permuter des intégrales sur une fonction à deux paramètres nous faisons comme suit.
    \begin{enumerate}
        \item
            Nous testons l'intégrabilité en chaine de \( | f |\), et si c'est bon, le corolaire~\ref{CorTKZKwP} nous donne \( f\in L^1(\Omega_1\times \Omega_2)\).
        \item
            Nous utilisons le théorème de Fubini~\ref{ThoFubinioYLtPI} pour séparer et permuter les intégrales comme des ingénieurs.
    \end{enumerate}

    Si la fonction \( (x,y)\mapsto f(x)g(y)\) satisfait aux hypothèse du théorème de Fubini alors
    \begin{equation}    \label{EqTJEEsJW}
        \int_{\Omega_1\times \Omega_2} f(x)g(y)dx\otimes dy=\left( \int_{\Omega_1}f(x)dx \right)\left( \int_{\Omega_2}g(y)dy \right).
    \end{equation}
    Le théorème de Fubini est souvent utilisé sous cette forme.

\end{normaltext}

\begin{example}[Nécessité d'avoir des mesures \( \sigma\)-finies]
    Nous montrons que le théorème ne tient pas si une des deux mesures n'est pas \( \sigma\)-finie. Soit \( I=\mathopen[ 0 , 1 \mathclose]\). Nous considérons l'espace mesuré
    \begin{equation}
        (I,\Borelien(I),\lambda)
    \end{equation}
    où \( \Borelien(I)\) est la tribu des boréliens sur \( I\) et \( \lambda\) est la mesure de Lebesgue (qui est $\sigma$-finie). D'autre part nous considérons l'espace mesuré
    \begin{equation}
        (I,\partP(I),m)
    \end{equation}
    où \( \partP(I)\) est l'ensemble des parties de \( I\) et \( m\) est la mesure de comptage. Cette dernière n'est pas $\sigma$-finie parce que les seuls ensembles de mesure finie pour la mesure de comptage sont des ensembles finis, or une union dénombrable d'ensemble finis ne peut pas recouvrir l'intervalle \( I\).

    Nous allons montrer que dans ce cadre, l'intégrale de la fonction indicatrice de la diagonale sur \( I^2\) ne vérifie pas le théorème de Fubini. Étant donné que \( \Borelien(I)\subset\partP(I)\) nous avons
    \begin{equation}
        \Borelien(I^2)\subset\Borelien(I)\otimes\partP(I).
    \end{equation}
    Soit \( \Delta=\{ (x,x)\tq x\in I \}\). La fonction
    \begin{equation}
        \begin{aligned}
            g\colon I^2&\to \eR \\
            (x,y)&\mapsto x-y
        \end{aligned}
    \end{equation}
    est continue et \( \Delta=g^{-1}(\{ 0 \})\) est donc fermé dans \( I^2\). L'ensemble \( \Delta\) est donc un borélien de \( I^2\) et par conséquent un élément de la tribu \( \Borelien(I)\otimes\partP(I)\). La fonction indicatrice \( \mtu_{\Delta}\) est alors mesurable pour l'espace mesuré
    \begin{equation}
        (I\times I,\Borelien(I)\otimes\partP(I),\lambda\otimes m).
    \end{equation}
    Pour \( x\) fixé nous avons
    \begin{equation}
        \mtu_{\Delta}(x,y)=\begin{cases}
            1    &   \text{si } y= x\\
            1    &    \text{si } y\neq x
        \end{cases}=\mtu_{\{ x \}}(y),
    \end{equation}
    et donc
    \begin{subequations}
        \begin{align}
            A_1&=\int_I\left( \int_I\mtu_{\Delta}(x,y)dm(y) \right)d\lambda(x)\\
            &=\int_I\left( \int_I\mtu_{\{ x \}}(y)dm(y) \right)d\lambda(x)\\
            &=\int_I\Big( m(\{ x \}) \Big)d\lambda(x)\\
            &=\int_I 1d\lambda(x)\\
            &=1.
        \end{align}
    \end{subequations}
    Par contre le support de \( \mtu_{\Delta}\) étant de mesure nulle pour la mesure de Lebesgue, nous avons
    \begin{equation}
        \int_I\mtu_{\Delta}(x,y)d\lambda(x)=0
    \end{equation}
    et par conséquent
    \begin{equation}
        A_2=\int_I\left( \int_I\mtu_{\Delta}(x,y)d\lambda(x) \right)dm(y)=0.
    \end{equation}
    Nous voyons donc que le théorème de Fubini ne s'applique pas.
\end{example}

\begin{example}  \label{EXooLUFAooGcxFUW}
    Nous nous proposons de calculer l'intégrale suivante en utilisant le théorème de Fubini :
    \begin{equation}
        G=\int_{\eR} e^{-x^2}dx=\sqrt{ \pi }
    \end{equation}
    alors que la fonction \( x\mapsto  e^{-x^2}\) n'a pas de primitives parmi les fonctions élémentaires.

    Nous allons le faire de deux façons. Une première directe en utilisant le théorème de Fubini sur un domaine non borné, et une seconde en utilisant Fubini sur un domaine borné, et en passant à la limite ensuite.

    \begin{subproof}
        \item[Fubini, domaine non borné]

    Par symétrie nous pouvons nous contenter de calculer
    \begin{equation}
        G_+=\int_0^{\infty} e^{-x^2}dx.
    \end{equation}
    L'astuce est de passer par l'intermédiaire
    \begin{subequations}
        \begin{align}
            H&=\int_{\eR^+\times\eR^+} e^{-(x^2+y^2)}dxdy       \label{EqIntFausasub}\\
            &=\int_{\eR^+}\left( \int_{\eR^+} e^{-x^2} e^{-y^2}dx \right)dy\\
            &=\left( \int_{\eR^+} e^{-x^2} dx\right)^2\\
            &=G_+^2
        \end{align}
    \end{subequations}
    L'intégrale \eqref{EqIntFausasub} se calcule en passant aux coordonnées polaires et le résultat est \( H=\frac{ \pi }{ 4 }\). Nous avons alors \( G=\frac{ \sqrt{\pi} }{ 2 }\) et
    \begin{equation}
        \int_{\eR} e^{-x^2}=\sqrt{\pi}.
    \end{equation}

        \item[Fubini, domaine borné, puis limite]
    Une variante, qui n'applique pas Fubini sur un domaine non borné. Nous commençons par écrire
\begin{equation}
	I=\int_{-\infty}^{+\infty} e^{-x^2} dx := \lim_{R \to +\infty} \int_{-R}^{+R} e^{-x^2} dx
\end{equation}
et puis nous faisons le calcul
\begin{equation}		\label{EqCalculInteeemoisxcar}
	\begin{aligned}[]
		I^2 &= \lim_{R \to +\infty} \left( (\int_{-R}^{+R} e^{-x^2} dx)( \int_{-R}^{+R} e^{-y^2} dy) \right) \\
		&= \lim_{R \to +\infty} \left( \iint_{K_R}e^{-(x^2+y^2)} dx dy \right) \\
		&= \lim_{R \to +\infty} \left( \iint_{C_R}e^{-(x^2+y^2)} dx dy \right)
	\end{aligned}
\end{equation}
où $K$ est le carré de demi-côté $R$ centré à l'origine et de côtés parallèles aux axes et $C_R$ est le cercle de rayon $R$ centré à l'origine.

	La première étape à justifier est simplement l'application de Fubini. Pour le passage de l'intégrale du carré vers le cercle, définissons
	\begin{equation}
		\begin{aligned}[]
			I_K(r)&=\int_{K_r}f,&I_C(r)&=\int_{C_r}f
		\end{aligned}
	\end{equation}
	où $K_r$ est la carré de demi-côté $r$ et $C_r$ est le cercle de rayon $r$. Le demi-côté du carré inscrit à $C_r$ est $\sqrt{2}$, donc pour tout $r$ nous avons
	\begin{equation}
		I_K(\sqrt{2}r)\leq I_C(r)<I_K(r),
	\end{equation}
	et en prenant la limite, nous avons évidement
	\begin{equation}
		\lim_{r\to \infty}I_K(\sqrt{2}r)=\lim_{r\to\infty}I_K(r),
	\end{equation}
	et donc cette limite est également égale à $\lim_{r\to\infty}I_C(t)$.

    Il ne reste qu'à calculer la dernière intégrale sur le cercle en passant aux coordonnées polaires :
	\begin{equation}
        \int_{C_R} e^{-(x^2+y^2)}dxdy=\int_0^{2\pi}d\theta\int_0^Rr e^{-r^2}dr=\pi(1- e^{-R^2}).
	\end{equation}
	La limite donne $\pi$, nous en déduisons que
    \begin{equation}    \label{EqFDvHTg}
		\int_{-\infty}^{\infty} e^{-x^2}dx=\sqrt{\pi}.
	\end{equation}
    \end{subproof}

\end{example}

Le théorème de Fubini-Tonelli nous permet également d'inverser des sommes et des séries. En effet une somme n'est rien d'autre qu'une intégrale pour la mesure de comptage :
\begin{equation}
    \sum_{n=0}^{\infty}a_n=\int_{\eN}a_ndm(n).
\end{equation}
La proposition suivante montre comment il faut faire.

\begin{proposition}\label{PropInversSumIntFub}
    Soient les espaces mesurés \( (\eN,\partP(\eN),m)\), \( (\eR^n,\Borelien(\eR^n),\lambda)\) où \( \lambda\) est la mesure de Lebesgue ainsi qu'une suite de fonctions positives \( f_n\colon \eR^d\to \eR\). Nous supposons de plus que la fonction \( f_n\) soit intégrable pour tout \( n\) et que les résultats forment une suite sommable. Alors
    \begin{equation}
        \sum_{n=0}^{\infty}\int_{\eR^n}f_n(x)dx=\int_{\eR^d}\sum_{n\in \eN}f_n(x)dx.
    \end{equation}
\end{proposition}
\index{mesure!de comptage}
\index{permuter!intégrale!et série}

\begin{proof}
    Nous pouvons la récrire le membre de gauche sous la forme
    \begin{equation}
        \int_{\eN}\left( \int_{\eR^n}f(n,x)dx \right)dm(n)
    \end{equation}
    avec la notation évidente \( f(n,x)=f_n(x)\). Prouvons que la fonction \( f\colon \eN\times\eR^d\to \eR\) ainsi définie est une fonction mesurable pour l'espace mesuré
    \begin{equation}
        \big( \eN\times\eR^d,\partP(\eN)\otimes\Borelien(\eR^d),m\otimes\lambda \big).
    \end{equation}
    Si \( A\subset\eR\), nous avons
    \begin{equation}
        f^{-1}(A)=\bigcup_{n\in\eN}\{ n \}\times f_n^{-1}(A).
    \end{equation}
    Chacun des ensembles dans l'union appartient à la tribu \( \partP(\eN)\times\Borelien(\eR^d)\) tandis que les tribus sont stables sous les unions dénombrables. La fonction \( f\) est donc mesurable. Comme nous avons supposé que \( f\) était positive, le théorème de Fubini-Tonelli s'applique et nous avons
    \begin{equation}
        \int_{\eR^d}\left( \int_{\eN}f(n,x)dm(n) \right)dx=\int_{\eR^d}\sum_{n\in \eN}f_n(x)dx.
    \end{equation}
\end{proof}

\begin{theorem}[Fubini]\label{ThoFubini}
Soit $(x,t)\mapsto f(x,y)\in\bar \eR$ une fonction intégrable sur $B_n\times B_m\subset\eR^{n+m}$ où $B_n$ et $B_m$ sont des ensembles mesurables de $\eR^n$ et $\eR^m$. Alors :
\begin{enumerate}
\item pour tout $x\in B_n$, sauf éventuellement en les points d'un ensemble $G\subset B_n$ de mesure nulle, la fonction $y\in B_m\mapsto f(x,y)\in\bar\eR$ est intégrable sur $B_m$
\item
la fonction
\begin{equation}
    \begin{aligned}
        B_n\setminus G&\to \eR \\
        x&\mapsto \int_{B_n}f(x,y)dy
    \end{aligned}
\end{equation}
est intégrable sur $B_n\setminus G$.
\item
On a
\begin{equation}
	\int_{B_n\times B_m}f(x,y)dxdy=\int_{B_n}\left( \int_{B_m}f(x,y)dy \right)dx.
\end{equation}

\end{enumerate}
\end{theorem}
\index{théorème!Fubini!dans $ \eR^n$}
\index{Fubini!théorème!dans $ \eR^n$}

Notons en particulier que si $f(x,y)=\varphi(x)\phi(y)$, alors $\int_{B_m}\varphi(y)dy$ est une constante qui peut sortir de l'intégrale sur $B_n$, et donc
\begin{equation}		\label{EqFubiniFactori}
	\int_{B_n\times B_m}\varphi(x)\phi(y)dxdy=\int_{B_n}\varphi(x)dx\int_{B_m}\phi(y)dy.
\end{equation}



\chapter{Suites et séries de fonctions}
% This is part of Mes notes de mathématique
% Copyright (c) 2011-2020
%   Laurent Claessens
% See the file fdl-1.3.txt for copying conditions.

Les généralités sur les suites et séries de fonctions, c'est dans la section \ref{SECooTDZNooJvjPks}.

%+++++++++++++++++++++++++++++++++++++++++++++++++++++++++++++++++++++++++++++++++++++++++++++++++++++++++++++++++++++++++++ 
\section{Séries de fonctions}
%+++++++++++++++++++++++++++++++++++++++++++++++++++++++++++++++++++++++++++++++++++++++++++++++++++++++++++++++++++++++++++

%---------------------------------------------------------------------------------------------------------------------------
\subsection{Intégration de séries de fonctions}
%---------------------------------------------------------------------------------------------------------------------------

\begin{theorem}      \label{ThoCciOlZ}
    La somme uniforme de fonctions intégrables sur un ensemble de mesure fini est intégrable et on peut permuter la somme et l'intégrale.

    En d'autres termes, supposons que \( \sum_{n=0}^{\infty}f_n\) converge uniformément vers \( F\) sur \( A\) avec \( \mu(A)<\infty\). Si \( F\) et \( f_n\) sont des fonctions intégrables sur \( A\) alors
    \begin{equation}
        \int_AF(x)d\mu(x)=\sum_{n=0}^{\infty}\int_Af_n(x)d\mu(x).
    \end{equation}
\end{theorem}
\index{permuter!somme et intégrale}

\begin{proof}
    Ce théorème est une conséquence du théorème~\ref{ThoUnifCvIntRiem}. En effet nous définissons la suite des sommes partielles
    \begin{equation}
        F_N=\sum_{n=0}^Nf_n.
    \end{equation}
    La limite \( \lim_{N\to \infty} F_N=F\) est uniforme. Par conséquent la fonction \( F\) est intégrable et
    \begin{equation}
        \int_A F=\lim_{N\to \infty} \int_AF_N=\lim_{N\to \infty} \int_A\sum_{n=0}^Nf_n=\lim_{N\to \infty} \sum_{n=0}^N\int_Af_n=\sum_{n=0}^{\infty}\int_Af_n.
    \end{equation}
    La première égalité est le théorème~\ref{ThoUnifCvIntRiem}, les autres sont de simples manipulations rhétoriques.
\end{proof}

Le théorème suivant est une paraphrase du théorème de la convergence dominée de Lebesgue (\ref{ThoConvDomLebVdhsTf}).
\begin{theorem}     \label{ThoockMHn}
    Soient des fonctions \( (f_n)_{n\in \eN}\) telles que \( \sum_{n=0}^Nf_n\) soit intégrable sur \( (\Omega,\tribA,\mu)\) pour chaque \( N\). Nous supposons que la somme converge simplement vers
    \begin{equation}
        f(x)=\sum_{n=0}^{\infty}f_n(x)
    \end{equation}
    et qu'il existe une fonction \( g\) telle que
    \begin{equation}
        \left| \sum_{n=0}^Nf_n \right| <g
    \end{equation}
    pour tout \( N\in \eN\). Alors
    \begin{enumerate}
        \item
            \( \sum_{n=0}^{\infty}f_n\) est intégrable,
        \item
            on peut permuter somme et intégrale :
            \begin{equation}
                \lim_{N\to \infty} \int_{\Omega}\sum_{n=0}^Nf_nd\mu=\int_{\Omega}\sum_{n=0}^{\infty}f_n,
            \end{equation}
        \item
            \begin{equation}
                \lim_{N\to \infty} \int_{\Omega}\left| \sum_{n=0}^Nf_n-\sum_{n=0}^{\infty}f_n \right| =\lim_{N\to \infty} \int_{\Omega}\left| \sum_{n=N}^{\infty}f_n \right| =0.
            \end{equation}
    \end{enumerate}
\end{theorem}


\begin{theorem} \label{ThoCSGaPY}
    Soit \( f_n\) des fonctions \( C^1\mathopen[ a , b \mathclose]\) telles que
    \begin{enumerate}
        \item
            la série \( \sum_n f_n(x_0)\) converge pour un certain \( x_0\in\mathopen[ a , b \mathclose]\),
        \item
            la série des dérivées \( \sum_n f'_n\) converge uniformément sur \( \mathopen[ a , b \mathclose]\).
    \end{enumerate}
    Alors la série \( \sum_n f_n\) converge vers une fonction \( F\) et
    \begin{enumerate}
        \item
            La convergence est uniforme sur \( \mathopen[ a , b \mathclose]\).
        \item
            La fonction \( F\) est dérivable
        \item
            \( F'(x)=\sum_nf'_n(x)\).
    \end{enumerate}
\end{theorem}

%---------------------------------------------------------------------------------------------------------------------------
\subsection{Différentiabilité}
%---------------------------------------------------------------------------------------------------------------------------

\begin{lemma}
    Soient \( E\) et \( F\) deux espaces vectoriels normés. Si la suite \( (T_n)\) converge vers \( T\) dans \( \aL(E,F)\), alors pour tout \( v\in E\) nous avons
    \begin{equation}
        \left( \sum_{n=0}^{\infty}T_n \right)(v)=\sum_{n=0}^{\infty}T_n(v).
    \end{equation}
\end{lemma}

\begin{probleme}
    À mon avis si on a un ouvert connexe par arcs dans un espace vectoriel normé, alors il est connexe par arcs de classe \( C^1\), c'est-à-dire que deux points peuvent être liés par un chemin de classe \( C^1\).

    Je n'en suis pas certain.

    Si vous êtes sûr de vous, vous pouvez affaiblir les hypothèses du théorème \ref{ThoLDpRmXQ} et supprimer la définition \ref{DEFooHOXOooKUqTQU} qui ne sert à rien d'autre.
\end{probleme}

\begin{definition}      \label{DEFooHOXOooKUqTQU}
    Soit un espace vectoriel normé \( E\). Un ouvert \( \Omega\) est dit connexe par arcs de classe \( C^1\) si pour tout choix de \( a,b\in \Omega\), il existe une application \( \gamma\colon \mathopen[ 0 , 1 \mathclose]\to \Omega\) de classe \( C^1\) telle que \( \gamma(0)=a\) et \( \gamma(1)=b\).
\end{definition}

\begin{theorem}[\cite{DHdwZRZ}] \label{ThoLDpRmXQ}
    Soient \( E\) et \( F\) deux espaces vectoriels normés, \( \Omega\) un ouvert connexe par arcs de classe \( C^1\) de \( E\). Soit \( (u_n)\) une suite de fonctions \( u_n\colon \Omega\to F\) telle que
    \begin{enumerate}
        \item
            pour tout \( n\), la fonction \( u_n\) est de classe \( C^1\) sur \( \Omega\),
        \item
            la série \( \sum_nu_n\) converge simplement sur \( \Omega\),
        \item
            la série des différentielles \( \sum_n(du_n)\) converge normalement sur tout compact de \( \Omega\).
    \end{enumerate}
    Alors la somme \( u=\sum_nu_n\) est de classe \( C^1\) sur \( \Omega\) et sa différentielle est donnée par
    \begin{equation}
        du=\sum_{n=0}^{\infty}du_n.
    \end{equation}
\end{theorem}

\begin{proof}
    Pour chaque \( n\), la fonction \( du_n\colon \Omega\to \aL(E,F)\) est une fonction continue parce que \( u_n\) est de classe \( C^1\). La série convergeant normalement, la fonction \( \sum_{n=0}^{\infty}du_n\) est également continue par la proposition~\ref{PropOMBbwst}. La difficulté de ce théorème est donc de prouver que cela est bien la différentielle de la fonction \( \sum_nu_n\), c'est-à-dire que
    \begin{equation}
        d\left( \sum_{n=0}^{\infty}u_n \right)=\sum_{n=0}^{\infty}du_n.
    \end{equation}

    Soient \( a,x\in \Omega\). Nous considérerions bien le segment \( \mathopen[ a , x \mathclose]\), mais vu que \( \Omega\) n'est supposé que connexe par arcs de classe \( C^1\) (définition \ref{DEFooHOXOooKUqTQU}), nous ne pouvons pas faire mieux pour joindre \( a\) à \( x\) que choisir un chemin de classe \( C^1\)
    \begin{equation}
        \gamma\colon \mathopen[ 0 , 1 \mathclose]\to \Omega
    \end{equation}
    tel que \( \gamma(0)=a\) et \( \gamma(1)=b\).

    L'astuce est de poser
    \begin{equation}
        \begin{aligned}
            f_n\colon \mathopen[ 0 , 1 \mathclose]&\to \eR \\
            t&\mapsto (du_n)_{\gamma(t)}\big( \gamma'(t) \big),
        \end{aligned}
    \end{equation}
    et d'en étudier l'intégrale\footnote{Cela revient à étudier l'intégrale de la forme différentielle \( du_n\) sur le chemin \( \gamma\). Voir la définition \ref{DEFooRMHGooFtMEPB} et tout ce qui s'en suit.}.

    \begin{subproof}
        \item[Permuter somme et intégrale]
            Nous voudrions permuter la somme et l'intégrale dans l'expression \( \int_0^1\sum_if_i(t)dt\). Pour cela nous commençons par regarder quelques majorations de normes.

            D'abord \( \gamma\) est de classe \( C^1\), ce qui fait que \( \gamma'\) est continue. Vu que la norme est une application continue, la fonction \( t\mapsto \| \gamma'(t) \|\) est également continue sur le compact \( \mathopen[ 0 , 1 \mathclose]\). Elle est donc majorée par une constante que nous nommons \( M\). C'est le théorème de Weierstrass \ref{ThoWeirstrassRn}.

            Ensuite nous avons le calcul
            \begin{equation}
                    \| f_i(t) \|=\| (du_i)_{\gamma(t)}\big( \gamma'(t) \big) \|
                    \leq\| (du_i)_{\gamma(t)} \|\| \gamma'(t) \|  
                    \leq M\| du_i \|_{\infty}<\infty.
            \end{equation}
            Justifications :
            \begin{itemize}
                \item 
                    Pour la première inégalité. C'est le lemme \ref{LEMooIBLEooLJczmu}. 
                \item Pour la seconde inégalité. Il s'agit de l'inégalité évidente
                    \begin{equation}
                        \| du_i \|_{\infty}=\sup_{x\in \gamma\big( \mathopen[ 0 , 1 \mathclose] \big)}\| (du_i)_x \|
                    \end{equation}
                    Notons que la norme \( \| . \|_{\infty}\) ne réfère pas à un supremum sur \( E\), mais seulement sur l'image de \( \gamma\). Nous aurions pu faire preuve d'un peu de créativité dans les notations.
                \item 
                    L'application \( du_i\) est continue sur le compact \( \gamma\big( \mathopen[ 0 , 1 \mathclose] \big)\). Donc le supremum est fini et atteint.
            \end{itemize}

            Maintenant nous posons
            \begin{equation}
                g_n(t)=\sum_{i=0}^nf_i(t).
            \end{equation}
            Nous avons la majoration
            \begin{equation}
                \| g_n(t) \|\leq \sum_{i=0}^n\| f_i(t) \|\leq M\sum_{i=0}^n\| du_i \|_{\infty}<\infty.
            \end{equation}
            Le fait que le tout soit fini est l'hypothèse de convergence normale sur tout compact. Le compact en question est \( \gamma\big( \mathopen[ 0 , 1 \mathclose] \big)\).

            C'est le moment d'utiliser le théorème de la convergence dominée de Lebesgue \ref{ThoConvDomLebVdhsTf}. Attention aux notations un peu décalées. Nous avons \( g_n\to \sum_{i=1}^{\infty}f_i\) (convergence simple) et \( \| g_n(t) \|\leq A\) où \( A\) est une constante que nous voyons comme une fonction constante intégrable sur le compact \( \mathopen[ 0 , 1 \mathclose]\). Nous permutons la limite et l'intégrale :
            \begin{subequations}
                \begin{align}
                \int_0^1\sum_{i=0}^{\infty}f_i(t)dt&=\int_0^1(\lim_{n\to \infty} g_n)(t)dt\\
                &=\lim_{n\to \infty} \int_0^1g_n(t)dt\\
                &=\lim_{n\to \infty} \int_{0}^1\sum_{i=0}^nf_i(t)dt\\
                &=\lim_{n\to \infty} \sum_{i=0}^n\int_0^1f_i(t)dt\\
                &=\sum_{i=0}^{\infty}\int_0^1f_i(t)dt.
                \end{align}
            \end{subequations}
        \item[Accroissements]

            Nous pouvons maintenant faire le petit calcul suivant :
            \begin{equation}
                \sum_n\int_0^1(du_n)_{\gamma(t)}\big( \gamma'(t) \big)dt=\sum_n\Big( u_n\big( \gamma(1) \big)-u_n\big( \gamma(0) \big) \Big)=\sum_n\big( u_n(x)-u_n(a) \big)=u(x)-u(a)
            \end{equation}
            où nous avons utilisé le théorème fondamental du calcul intégral sous la forme de la proposition \ref{PROPooJYIAooXLkbMx}.

            Nous retenons l'égalité
            \begin{equation}        \label{EQooTXYWooIDxVri}
                u(x)=u(a)+\int_0^1\sum_n(du_n)_{\gamma(t)}\big( \gamma'(t) \big)dt.
            \end{equation}
        \item[Remarque]
            La formule \eqref{EQooTXYWooIDxVri} n'est pas une forme de formule des accroissements finis qui parlerait d'évaluer une fonction \( u\) en \( x\) en partant de \( a\) et en intégrant \( du\) le long d'un chemin joignant \( a\) et \( x\).

            Ce serait le cas si nous pouvions permuter la somme et la différentielle qui se trouvent dans l'intégrale. Or permuter somme et différentielle est précisément l'objet du théorème que nous sommes en train de prouver.

        \item[Différentielle]
        
            Forts de la formule \eqref{EQooTXYWooIDxVri}, nous calculons \( du_a(v)\), c'est-à-dire la différentielle de \( u\) au point \( a\) appliquée au vecteur \( v\in F\). Pour cela, nous savons que \( \Omega\) est ouvert, donc \( \Omega\) contient une boule de rayon \( r\) autour de \( a\), ce qui nous permet de dire que pour un \( \) donné, le point \( a+sv\) est dans \( \Omega\) pour tout \( s\in B(0,\epsilon)\) lorsque \( \epsilon\) n'est pas trop grand. Pour chacun de ces \( s\), nous considérons un chemin de classe \( C^1\) joignant \( a\) à \( a+sv\). Ce chemin sera noté
            \begin{equation}
                \gamma_s\colon \mathopen[ 0 , 1 \mathclose]\to \Omega
            \end{equation}
            et \( \gamma(0)=a\), \( \gamma(1)=a+sv\). Nous avons le calcul
            \begin{subequations}
                \begin{align}
                    du_a(v)&=\Dsdd{ u(a+sv) }{s}{0}\\
                    &=\Dsdd{ \int_0^1\sum_{n=0}^{\infty}(du_n)_{\gamma_s(t)}\big( \gamma_s'(t) \big)dt }{s}{0}\\
                    &=\Dsdd{ \sum_{n=0}^{\infty} \int_0^1(du_n)_{\gamma_s(t)}\big( \gamma'_s(t) \big)dt  }{s}{0}    \label{SUBEQooFSPSooCPErXj}\\
                    &=\Dsdd{ \sum_n\big( u_n(a+sv)-u_n(a) \big) }{s}{0}\\
                    &=\Dsdd{ (\sum_nu_n)(a+sv) }{s}{0}\\
                    &=\sum_n\Dsdd{ u_n(a+sv) }{s}{0}    \label{SUBEQooDQODooIPMfDo}\\
                    &=\sum_n(du_n)_a(v)\\
                    &=\Big( \sum_n(du_n)_a \Big)(v)\\
                    &=\Big( \sum_ndu_n \Big)_a(v)\label{SUBEQooQGOQooLvXuaX}.
                \end{align}
            \end{subequations}
            Justifications :
            \begin{itemize}
                \item Pour \ref{SUBEQooFSPSooCPErXj}. Permuter la somme et l'intégrale comme plus haut.
                \item Pour \ref{SUBEQooDQODooIPMfDo}. Permuter une somme et une dérivée classique des fonctions \( \eR\to F\) données par \( s\mapsto u_n(a+sv)\). Il s'agit d'utiliser le théorème \ref{THOooXZQCooSRteSr} sur chaque composantes dans \( F\).
                \item Pour \ref{SUBEQooQGOQooLvXuaX}. Chaque \( du_n\) est une application \( du_n\colon E\to \aL(E,F)\). Au fait près que la notation est plus lourde, il s'agit simplement d'une définition de la somme pontuelle d'une suite de fonctions : \( \sum_nf_n(a)=(\sum_nf_n)(a)\). Dans ce cas-ci, le tout est encore un élément de \( \aL(E,F)\) que nous appliquons à \( v\).
            \end{itemize}
    \end{subproof}
\end{proof}

%+++++++++++++++++++++++++++++++++++++++++++++++++++++++++++++++++++++++++++++++++++++++++++++++++++++++++++++++++++++++++++
\section{Séries entières}
%+++++++++++++++++++++++++++++++++++++++++++++++++++++++++++++++++++++++++++++++++++++++++++++++++++++++++++++++++++++++++++

Dans cette section nous allons parler de séries complexes autant que de séries réelles. L'étude des propriétés à proprement parler complexes des séries entières (holomorphie) sera effectuée dans le chapitre dédié, voir le théorème~\ref{ThomcPOdd} et ses conséquences.

%---------------------------------------------------------------------------------------------------------------------------
\subsection{Disque de convergence}
%---------------------------------------------------------------------------------------------------------------------------

Une \defe{série de puissance}{série!de puissance} est une série de la forme
\begin{equation}		\label{eqseriepuissance}
	\sum_{k=0}^{\infty}c_k(z-z_0)^k
\end{equation}
où $z_0\in \eC$ est fixé, $(c_k)$ est une suite complexe fixée, et $z$ est un paramètre complexe. Nous disons que cette série est \emph{centrée} en $z_0$.

\begin{definition}
    Une \defe{série entière}{série!entière} est une somme de la forme
    \begin{equation}
        \sum_{n=0}^{\infty}a_nz^n
    \end{equation}
    avec \( a_n,z\in\eC\).
\end{definition}
Une série entière peut définir une fonction
\begin{equation}
    f(z)=\sum_na_nz^n.
\end{equation}
Le but de cette section est d'étudier des conditions sur la suite \( (a_n)\) qui assurent la continuité de \( f\) ou la possibilité de dériver ou intégrer la série terme à terme.

\begin{definition}  \label{DefZWKOZOl}
    Soit \( \sum_{n\in \eN}a_nz^n\) une série entière. Le \defe{rayon de convergence}{rayon!de convergence} de cette série est le nombre
    \begin{equation}
        R=\sup\{ r\in \eR^+\tq \text{la suite }(a_nr^n)\text{ est bornée} \}\in\mathopen[ 0 , \infty \mathclose].
    \end{equation}
    La boule \( B(0,R)\) est le \defe{disque de convergence}{disque de convergence} de la série.
\end{definition}
Le rayon de convergence d'une série ne dépend que des réels \( | a_n |\), même si à la base \( a_n\in \eC\).

\begin{remark}      \label{REMooYOTEooKvxHSf}
    Si pour tout \( n\) nous avons \( | b_n |\geq | a_n |\) alors le rayon de convergence de la série \( \sum_na_nz^n\) est au moins aussi grand que celui de la série \( \sum_nb_nz^n\). Cela y compris lorsque l'un ou l'autre des rayons de convergences est infini.
\end{remark}

\begin{lemma}[Critère d'Abel]\index{critère!Abel}   \label{LemmbWnFI}
    Soit \( R>0\) le rayon de convergence de la somme \( \sum_na_nz^n\) et \( z\in \eC\).
    \begin{enumerate}
        \item
            Si \( | z |<R\) alors la série converge absolument.
        \item
            Si \( R<\infty\) et si \( | z |>R\) alors la série diverge.
    \end{enumerate}
\end{lemma}

\begin{proof}
    Démonstration en deux parties.
    \begin{enumerate}
        \item

            Si \( | z |<R\) alors la suite \( (a_nz^n)\) est bornée et il existe un nombre \( M\in \eR\) tel que \( | a_n |r^n\leq M\) pour tout \( n\). Nous considérons alors un \( r\) tel que \( | z |<r<R\) et nous pouvons calculer :
            \begin{equation}
                | a_nz^n |=| a_n |r^n\big( \frac{ | z | }{ r } \big)^n\leq M\left( \frac{ | z | }{ r } \right)^n
            \end{equation}
            Vu que \( | z |<r\) nous tombons sur la série géométrique \eqref{EqZQTGooIWEFxL} qui converge. Par le critère de comparaison\footnote{Lemme~\ref{LemgHWyfG}.} la série \( \sum_{n=0}^{\infty}| a_nz^n |\) converge.

        \item
            Par définition du rayon de convergence, la suite \( (a_nz^n)\) n'est donc pas bornée et la série ne peut pas converger à cause de la proposition~\ref{PROPooYDFUooTGnYQg}.
    \end{enumerate}
\end{proof}

Le critère d'Abel parle bien de convergence absolue, et non de convergence normale. Pour chaque \( t\), la série \( \sum_k | a_nt^k |\) converge. Si par contre nous posons \( u_k(t)=a_kt^k\), nous n'avons à priori pas la convergence normale \( \sum_k\| u_k \|_{\infty}\), même pas si la norme est la norme supremum sur \( B(0,R)\)\quext{Il y aurait par contre bien convergence sur tout compact ? Cher lecteur, dites moi ce que vous en pensez}. Prenons comme exemple simplement \( a_k=1\) pour tout \( k\). Pour tout \( | t |<1\), la série \( \sum_k t^k\) converge absolument (série géométrique), mais nous aurions \( \| u_k \|_{\infty}=1\) et donc divergence évidente de \( \sum_k\| u_k \|_{\infty}\).

La proposition suivante sera surtout utile lorsqu'on parlera de dérivée.
\begin{proposition}[\cite{KOWMooXhcOoy}]        \label{PropHDIUooKTbVSX}
    Quel que soit le nombre \( \alpha\in \eR\), les séries \( \sum_na_nz^n\) et \( \sum_nn^{\alpha}a_nz^n\) ont même rayon de convergence.
\end{proposition}

\begin{proof}
    Nous posons
    \begin{subequations}
        \begin{align}
            E=\{ r\in \eR^+\tq \text{  } (a_nr^n)\text{ est borné } \}
            E'=\{ r\in \eR^+\tq \text{  } (n^{\alpha}a_nr^n)\text{ est borné } \}
        \end{align}
    \end{subequations}
    Et aussi \( R=\sup(E)\), \( R'=\sup(E')\). Le fait que \( E'\geq E\) est facile. Nous supposons \( R>0\) et nous considérons \( r<R\) (c'est-à-dire \( r\in E\)).  Nous allons montrer que \( r\in E'\). Pour cela nous prenons un nombre \( s\) tel que \( r<s<R\). Nous avons
    \begin{equation}
        n^{\alpha}a_nr^n=n^{\alpha}a_n\left( \frac{ r }{ s } \right)^ns^n=n^{\alpha}\left( \frac{ r }{ s } \right)^na_ns^n.
    \end{equation}
    Mais \( r/s<1\), donc le lemme~\ref{LemLJOSooEiNtTs} dit que \( n^{\alpha}(r/s)^n\to 0\). Cela est donc borné par une constante \( M\). Donc
    \begin{equation}
        n^{\alpha}a_nr^n\leq Ma_ns^n.
    \end{equation}
    Mais la suite \( (a_ns^n)\) est bornée. Donc la suite \( n^{\alpha}a_nr^n\) est également bornée, ce qui prouve que \( r\in E'\).
\end{proof}

\begin{remark}
    Au fond, cette proposition n'est rien d'autre que dire que dans \( n^\alpha r^n\), l'effet «convergent» est \( r^n\) qui est une décroissance exponentielle tandis que l'effet «divergent» est \( n^{\alpha}\) qui a une croissance seulement polynomiale.
\end{remark}

\begin{theorem}[Formule de Hadamard] \label{ThoSerPuissRap}
    Le rayon de convergence\footnote{Définition \ref{DefZWKOZOl}.} de la série entière \( \sum_n c_n z^n\) est donné par une des deux formules
    \begin{equation}		\label{EqRayCOnvSer}
        \frac{1}{ R } =\limsup\sqrt[k]{| a_k |}
    \end{equation}
    ou
    \begin{equation}		\label{EqAlphaSerPuissAtern}
        \frac{1}{ R }=\limite k \infty \abs{\frac{a_{k+1}}{a_k}}
    \end{equation}
    lorsque $a_k$ est non nul à partir d'un certain $k$.

    Si une de ces formules donne \( 1/R=0\), alors le rayon de convergence est infini.
\end{theorem}
\index{formule!Hadamard}\index{Hadamard!formule}		

Le disque $| z-z_0 |\leq R$ est le \defe{disque de convergence}{disque de convergence} de la série \( \sum_n a_n(z-z_0)^n\). Notons que le critère d'Abel ne dit rien pour les points tels que $| z-z_0 |=R$. Il faut traiter ces points au cas par cas. Et le pire, c'est qu'une série donnée peut converger pour certain des points sur le bord du disque, et diverger en d'autres. Le théorème d'Abel radial (théorème~\ref{ThoLUXVjs}) nous donnera quelques informations sur le sujet.

Il y a un dessin à la figure~\ref{LabelFigDisqueConv}.
\newcommand{\CaptionFigDisqueConv}{À l'intérieur du disque de convergence, la convergence est absolue. En dehors, la série diverge. Sur le cercle proprement dit, tout peut arriver.}
\input{auto/pictures_tex/Fig_DisqueConv.pstricks}

Si les suites \( a_n\) et \( b_n\) sont équivalentes, alors les séries correspondantes auront le même rayon de convergence. Cela ne signifie pas que sur le bord du disque de convergence, elles aient même comportement. Par exemple nous avons
\begin{equation}
    \frac{1}{ \sqrt{n} }\sim \frac{1}{ \sqrt{n} }+\frac{ (-1)^n }{ n }.
\end{equation}
En même temps, en \( z=-1\) la série
\begin{equation}
    \sum_{n\geq 1}\frac{ z^n }{ \sqrt{n} }
\end{equation}
converge par le critère des séries alternées\footnote{Théorème \ref{THOooOHANooHYfkII}.}. Par contre la série
\begin{equation}
    \sum_{n\geq 1}\left( \frac{1}{ \sqrt{n} }+\frac{ (-1)^n }{ n } \right)z^n
\end{equation}
ne converge pas pour \( z=-1\).

\begin{example}
    Soit \( \alpha\in \eR\) et considérons la série \( \sum_{n\geq 1}a_nz^n\) où \( a_n\) est la \( n\)-ième décimale de \( \alpha\). Si \( \alpha\) est un nombre décimal limité, la suite \( (a_n)\) est finie et le rayon de convergence est infini. Sinon, pour tout \( N\) il existe un \( n>N\) tel que \( a_n\neq 0\) et la suite \( (a_n)\) ne tend pas vers zéro. Par conséquent la série
    \begin{equation}
        \sum_{n}a_nz^n
    \end{equation}
    diverge pour \( z=1\) et le rayon de convergence satisfait \( R\leq 1\). Nous avons aussi \( | a_n |\leq 9\), de telle manière à ce que la série soit bornée et par conséquent majorée en module par \( 9z^n\), ce qui signifie que \( R\geq 1\).

    Nous déduisons alors \( R=1\).
\end{example}

%---------------------------------------------------------------------------------------------------------------------------
\subsection{Propriétés de la somme}
%---------------------------------------------------------------------------------------------------------------------------

Le théorème suivant donne une formule (dit «produit de Cauchy») pour le produit de deux séries entières. Nous en donnons une adaptation dans le cas de séries de puissances dans une algèbre normée dans la proposition \ref{PROPooFMEXooCNjdhS}.
\begin{theorem}     \label{ThokPTXYC}
    Soient \( \sum_na_nz^n\) et \( \sum b_nz^n\) deux séries de rayon de convergences respectivement \( R_a\) et \( R_b\).
    \begin{enumerate}
        \item   \label{IteWlajij}
            Si \( R_s\) est le rayon de convergence de \( \sum_n(a_n+b_n)z^n\), nous avons
            \begin{equation}
                R_s\geq \min\{ R_a,R_b \}
            \end{equation}
            et nous avons l'égalité si pour tout \( |z |\leq\min\{ R_a,R_b \}\), \( \sum (a_n+b_n)z^n=\sum_n a_nz^n+\sum_nb_nz^n\).
        \item
            Si \( \lambda\neq 0\) la série \( \sum_n(\lambda a_n)z^n\) a le même rayon de convergence que la série \( \sum_na_nz^n\) et si \( | z |<R_a\) nous avons
            \begin{equation}
                \sum_{n=0}^{\infty}(\lambda a_n)z^n=\lambda\sum_{n=0}^{\infty}a_nz^n.
            \end{equation}
        \item
            Le \defe{produit de Cauchy}{Cauchy!produit}\index{produit!de Cauchy} des deux séries est donné par
            \begin{equation}        \label{EqFPGGooDQlXGe}
                \left( \sum_na_nz^n \right)\left( \sum_k b_kz^k \right)=       \sum_{n=0}^{\infty}\left( \sum_{i+j=n}a_ib_j \right)z^n=\sum_{n=0}^{\infty}\left( \sum_{k=0}^{n}a_kb_{n-k} \right)z^n.
            \end{equation}
            Si \( R_p\) est le rayon de convergence de ce produit nous avons
            \begin{equation}
                R_p\geq \min\{ R_a,R_b \}
            \end{equation}
            et si \( | z |<\min\{ R_a,R_b \}\) alors
            \begin{equation}
                \sum_{n=0}^{\infty}\left( \sum_{i+j=n}a_ib_j \right)z^n=\left( \sum_{n=0}^{\infty}a_nz^n \right)\left( \sum_{n=0}^{\infty}b_nz^n \right).
            \end{equation}

    \end{enumerate}

\end{theorem}

\begin{proof}
    Nous prouvons la partie sur le produit de Cauchy. En utilisant la propriété du produit de la somme par un scalaire nous avons
    \begin{subequations}
        \begin{align}
            \left( \sum_{n=0}^{\infty}a_nz^n \right)\left( \sum_{m=0}^{\infty}b_mz^m \right)&=\sum_{n=0}^{\infty}\left( \sum_{m=0}^{\infty}b_ma_nz^{m+n} \right)\\
            &=\lim_{N\to \infty} \lim_{M\to \infty} \sum_{n=0}^N\sum_{m=0}^Mb_ma_nz^{m+n}\\
            &=\lim_{N\to \infty} \lim_{M\to \infty} \sum_{k=0}^{N+M}\sum_{i+k=k}b_ia_jz^k\\
            &=\lim_{N\to \infty} \sum_{k=0}^{\infty}\sum_{i+k=k}b_ia_jz^k\\
            &=\sum_{k=0}^{\infty}\sum_{i+j=k}b_ia_jz^k.
        \end{align}
    \end{subequations}
\end{proof}

\begin{example}
    Montrons un produit de Cauchy dont le rayon de convergence est strictement plus grand que le minimum. D'abord nous considérons
    \begin{equation}
        A=1-z,
    \end{equation}
    c'est-à-dire \( a_0=1\), \( a_1=-1\), \( a_{n\geq 2}=0\) avec \( R_a=\infty\). Ensuite nous considérons
    \begin{equation}
        B=\sum_nz^n,
    \end{equation}
    c'est-à-dire \( B=(1-z)^{-1}\) et \( R_b=1\). Le produit de Cauchy de ces deux séries valant \( 1\), le rayon de convergence est infini.
\end{example}

\begin{example}
    Nous montrons que
    \begin{equation}
        \sum_{n=0}^{\infty}(n+1)x^n=\frac{1}{ (1-x)^2 }
    \end{equation}
    pour \( x\in\mathopen] -1 , 1 \mathclose[\).

    Étant donné que pour tout \( r\) dans \( \mathopen] -1 , 1 \mathclose[\) la suite \( (n+1)r^n\) est bornée, le rayon de convergence est correct. Pour les \( x\) dans ce domaine nous avons
    \begin{equation}        \label{EqIwbuTk}
        \frac{1}{ (1-x)^2 }=\frac{1}{ (1-x) }\frac{1}{ (1-x) }=\left( \sum_{n=0}^{\infty}x^n \right)\left( \sum_{m=0}^{\infty}z^m \right).
    \end{equation}
    Nous devons expliciter ce produit de Cauchy en utilisant le théorème~\ref{ThokPTXYC}. Pour tout \( i\) nous avons \( a_i=b_i=1\). Par conséquent le produit \eqref{EqIwbuTk} devient
    \begin{equation}
        \sum_{n=0}^{\infty}\sum_{i+j=n}x^n=\sum_{n=0}^{\infty}(n+1)x^n.
    \end{equation}
\end{example}

\begin{theorem}
    Une série entière converge normalement sur tout disque fermé inclus au disque de convergence.
\end{theorem}

\begin{proof}
    Toute boule fermée inclue à \( B(0,R)\) est inclue à la boule \( \overline{ B(0,r) }\) pour un certain \( r<R\). Nous nous concentrons donc sur une telle boule fermée.

    Pour chaque \( n\) nous posons \( u_n(z)=a_nz^n\) que nous voyons comme une fonction sur \( \overline{ B(0,r) }\). Pour tout \( n\in \eN\) et tout \( z\in\overline{ B(0,r) }\) nous avons
    \begin{equation}
        \| u_n \|_{\infty}\leq| a_nz^n |\leq | a_n |r^n.
    \end{equation}
    Étant donné que \( r<R\) la série \( \sum_n | a_n |r^n\) converge et la série \( \sum_n\| u_n \|\) est convergente. La série \( \sum_na_nz^n\) est alors normalement convergente.
\end{proof}

\begin{example}
    Encore une fois nous n'avons pas d'informations sur le comportement au bord. Par exemple la série \( \sum_nz^n\) a pour rayon de convergence \( R=1\), mais \( \sup_{z\in B(0,1)}| z^n |=1\) et nous n'avons pas de convergence normale sur la boule fermée.
\end{example}

La convergence normale n'est donc pas de mise sur tout l'intérieur du disque de convergence. La continuité, par contre est effective sur la boule. En effet si \( z_0\in B(0,R)\) alors il existe un rayon \( 0<r<R\) tel que \( B(z_0,r)\subset B(0,R)\). Sur \( B(z_0,r)\) nous avons convergence normale et donc continuité en \( z_0\).

La différence est que la continuité est une propriété locale tandis que la convergence normale est une propriété globale.

\begin{proposition}
    Soit \( f(z)=\sum_na_nz^n\) avec un rayon de convergence \( R\). Si \( \sum | a_n |R^n\) converge alors
    \begin{enumerate}
        \item
            la série \( \sum_na_nz^n\) converge normalement sur \( \overline{ B(0,R) }\),
        \item
            \( f\) est continue sur \( \overline{ B(0,R) }\).
    \end{enumerate}
\end{proposition}

\begin{proof}
    La conclusion est claire dans l'intérieur du disque de convergence. En ce qui concerne le bord, chacune des sommes partielles est une fonction continue. De plus nous avons \( \| u_n \|\leq | a_n |R^n\), dont la série converge. Par conséquent nous avons convergence normale sur le disque fermé.
\end{proof}

Le théorème suivant permet de donner, dans le cas de fonctions réelle, des informations sur la convergence en une des deux extrémités de l'intervalle de convergence.
\begin{theorem}[Convergence radiale de Abel]\index{Abel!convergence radiale} \label{ThoLUXVjs}
    Soit \( f(x)=\sum_na_nx^n\) une série réelle de rayon de convergence \( 0<R<\infty\).
    \begin{enumerate}
        \item
            Si \( \sum a_nR^n\) converge, alors \( f\) est continue sur \( \mathopen[ 0 , R \mathclose]\).
        \item
            Si \( \sum_na_n(-R)^n\) converge, alors \( f\) est continue sur \( \mathopen[ -R , 0 \mathclose]\).
    \end{enumerate}
\end{theorem}

La proposition \ref{PROPooKPBIooJdNsqX} donnera un exemple d'utilisation pour la série de \( \ln(1-x)\) (qui n'est pas encore définie à ce moment).


Le résultat suivant permet d'identifier deux séries complexes lorsque leurs valeurs sur \( \eR\) sont identiques.
\begin{proposition}
    Soient les séries \( f(z)=\sum a_nz^n\) et \( g(z)=\sum b_n z^n\) convergentes dans \( B(0,R)\). Si \( f(x)=g(x)\) pour \( x\in \mathopen[ 0 , R [\) alors \( a_n=b_n\).
\end{proposition}

\begin{proof}
    Soit \( n_0\) le plus petit entier tel que \( a_{n_0}\neq b_{n_0}\). Pour tout \( z\in B(0,R)\) nous avons
    \begin{equation}
        f(z)-g(z)=\sum_{n=n_0}^{\infty}(a_n-b_n)z^n=z^{n_0}\varphi(z)
    \end{equation}
    où
    \begin{equation}
        \varphi(z)=\sum_{n\geq 0}(a_{n+n_0}-b_{n+n_0})z^n.
    \end{equation}
    Par le théorème~\ref{ThokPTXYC}\ref{IteWlajij} le rayon de convergence de \( \varphi\) est plus grand que \( R\) et la fonction \( \varphi\) est continue en \( 0\). Étant donné que \( \varphi(0)=a_{n_0}-b_{n_0}\neq 0\) et que \( \varphi\) est continue nous avons un \( \rho\) tel que \( \varphi\neq 0\) sur \( B(0,\rho)\). Or cela n'est pas possible parce que au moins sur la partie réelle de cette dernière boule, \( \varphi\) doit être nulle.
\end{proof}

\begin{proposition}[\cite{GYDXooJJusGH,MonCerveau}]     \label{PropSNMEooVgNqBP}
    Si la série entière \( \sum_{n\geq 0}a_nz^n\) a un rayon de convergence \( R\) alors
    \begin{enumerate}
        \item
            La somme est une fonction holomorphe dans le disque de convergence.
        \item       \label{ItemUULDooEGRNiA}
            La somme est différentiable et
            \begin{equation}
                du_{z_0}(z)=\sum_{n=1}^{\infty}na_nz_0^{n-1}z.
            \end{equation}
        \item
    De plus pour tout \( z_0\in B(0,R)\), on pose\footnote{Pour rappel, dans tout ce texte, \( B(a,r)\) est une boule \emph{ouverte}.}
    \begin{subequations}
        \begin{align}
            S(z)&=\sum_{n\geq 0}a_nz^n\\
            T(z)&=\sum_{n\geq 1}na_nz^{n-1}=\sum_{n=0}^{\infty}(n+1)a_{n+1}z^n.
        \end{align}
    \end{subequations}
    Alors  nous avons
    \begin{equation}    \label{EqVQDPooOPICwN}
        \lim_{z\to z_0}\frac{ S(z)-S(z_0) }{ z-z_0 }=T(z_0).
    \end{equation}
    \end{enumerate}
\end{proposition}

\begin{proof}
    Nous allons prouver, en utilisant le théorème~\ref{ThoLDpRmXQ}, que la somme est une fonction différentiable et que la différentielle est \( \eC\)-linéaire. La proposition~\ref{PropKJUDooJfqgYS} nous dira alors que la somme est \( \eC\)-dérivable.

    Nous posons \( u_n(z)=a_nz^n\), qui est une fonction de classe \( C^1\). En ce qui concerne sa différentielle nous considérons \( z_0\in B(0,R)\)  et nous avons    (si \( n=0\) alors la différentielle est nulle)
    \begin{subequations}
        \begin{align}
            (du_n)_{z_0}(z)&=\Dsdd{ u_n(z_0+tz) }{t}{0}\\
            &=\Dsdd{ a_n(z_0+tz)^n }{t}{0}\\
            &=\Dsdd{ na_n(z_0^{n-1}tz) }{t}{0}\\
            &=na_nz_0^{n-1}z.
        \end{align}
    \end{subequations}
    En cours de calcul nous avons développé \( (z_0+tz)^n\) et gardé seulement les termes de degré \( 1\) en \( t\). Il y en a \( n\) et ils sont tous égaux à \( z_0^{n-1}tz\).

    La convergence simple \( \sum_nu_n\) est dans les hypothèses. Il reste à prouver que la somme des différentielles converge uniformément sur tout compact autour de \( z_0\) ne débordant pas du disque ouvert de convergence. Soit \( K\) un compact autour de \( z_0\). Dans le calcul suivant nous utilisons une première fois la norme uniforme de \( du_n\) vu comme fonction de \( K\) vers \( \aL(\eC,\eC)\) et une fois la norme opérateur\footnote{Définition~\ref{DefNFYUooBZCPTr}.} de \( (du_n)_{z_0}\) comme application linéaire \( \eC\to \eC\) :
    \begin{subequations}
        \begin{align}
            \| du_n \|_k&=\sup_{z_0\in K}\| (du_n)_{z_0} \|\\
            &=\sup_{z_0\in K}\sup_{| z |=1}| (du_n)_{z_0}(z) |\\
            &=\sup_{z_0\in K}\sup_{| z |=1}| na_nz_0^{n-1}z |\\
            &=\sup_{z_0\in K}n| a_n | |z_0 |^{n-1}.
        \end{align}
    \end{subequations}
    Vu que \( z\mapsto| z |^{n-1}\) est une application continue sur le compact \( K\), elle atteint son maximum (théorème~\ref{ThoWeirstrassRn}). Nous considérons \( z_K\), un point qui réalise le supremum. Ce nombre est dans le disque de convergence parce que \( K\) est un compact autour de \( z_0\).

    Nous devons prouver que \( \sum_nn| a_n | |z_K |^{n-1}\) converge. Vu que \( | z_K |\) est une constante (par rapport à \( n\)) nous pouvons étudier la convergence en écrivant \( | z_K |^n\) au lieu de \( | z_K |^{n-1}\).

    La suite \( (a_n| z_K |^n)\) est une suite bornée. Soit \( M\) tel que \( | a_n | |z_K |^n<M\) pour tout \( n\). Nous considérons de plus \( r\) de telle sorte que \( K\subset B(0,r)\subset B(0,R)\). En particulier \( | z_K |<r\) et nous avons
    \begin{equation}
        n| a_n | |z_K |^n\leq n| a_n |r^n\left( \frac{ | z_K | }{ r } \right)^n\leq nM\left( \frac{ | z_K | }{ r } \right)^n.
    \end{equation}
    Nous savons que ce qui est dans la parenthèse est plus petit que \( 1\), mais que \( \sum_nnx^n\) converge dès que \( | x |<1\). Par conséquent
    \begin{equation}
        \sum_n\| du_n \|_K
    \end{equation}
    converge et le théorème~\ref{ThoLDpRmXQ} fonctionne : \( du=\sum_{n=1}^{\infty}du_n\) et la somme \( \sum_nu_n\) est de classe \( C^1\).

    La différentielle de \( \sum_nu_n\) s'exprime explicitement par
    \begin{equation}        \label{EqJBFMooMjSABz}
        du_{z_0}(z)=\sum_{n=1}^{\infty}na_nz_0^{n-1}z.
    \end{equation}
    Cette forme montre que \( du_{z_0}\) est une application \( \eC\)-linéaire et donc la somme est \( \eC\)-dérivable par la proposition~\ref{PropKJUDooJfqgYS}. Ergo holomorphe sur le disque de convergence par définition~\ref{DefMMpjJZ}.

    En ce qui concerne la formule \eqref{EqVQDPooOPICwN}, elle provient de la formule \eqref{EqPAEFooYNhYpz} : \( f'(z_0)\) est donné par la facteur multiplicatif de \( du_{z_0}\). En l'occurrence la formule \eqref{EqJBFMooMjSABz} nous donne
    \begin{equation}
        f'(z_0)=\sum_{n\geq 1}na_nz_0^{n-1}.
    \end{equation}
\end{proof}

%---------------------------------------------------------------------------------------------------------------------------
\subsection{Dérivation}
%---------------------------------------------------------------------------------------------------------------------------

\begin{lemma}       \label{LemFVMaSD}
    Soit une série entière \( \sum a_nz^n\) de rayon de convergence \( R\). Les séries
    \begin{equation}
        \sum \frac{ a_n }{ n+1 }z^{n+1}
    \end{equation}
    et
    \begin{equation}
        \sum_{n\geq 1}na_nz^{n-1}
    \end{equation}
    ont même rayon de convergence \( R\).
\end{lemma}

Notons toutefois que nonobstant ce lemme, les séries dont il est question peuvent se comporter différemment sur le bord du disque de convergence. En effet la série
\begin{equation}
    \sum \frac{1}{ n }z^n
\end{equation}
diverge pour \( z=1\) alors que
\begin{equation}
    \sum\frac{1}{ n(n+1) }z^{n+1}
\end{equation}
converge pour \( z=1\).


Les théorèmes de dérivation et d'intégration de séries de fonctions (théorèmes~\ref{ThoCciOlZ} et~\ref{ThoCSGaPY}) fonctionnent bien dans le cas des séries entières. Ils donnent la proposition~\ref{ProptzOIuG} pour la dérivation et~\ref{PropfeFQWr} pour l'intégration.

\begin{proposition}     \label{ProptzOIuG}
    Soit la série entière
    \begin{equation}
        f(x)=\sum_{n=0}^{\infty}a_n x^n
    \end{equation}
    de rayon de convergence \( R\). Alors la fonction \( f\) est \( C^1\) sur \( \mathopen] -R , R \mathclose[\) et se dérive terme à terme :
    \begin{equation}
        f'(x)=\sum_{n=1}^{\infty}na_nx^{n-1}
    \end{equation}
    pour tout \( x\in\mathopen] -R , R \mathclose[\).
\end{proposition}
\index{permuter!série entière et dérivation}

\begin{proof}
    Nous savons que la série \( \sum_{n=1}^{\infty}na_nx^{n-1}\) a le même rayon de convergente que celui de la série \( f\). En particulier cette série des dérivées converge normalement sur tout compact dans \( \mathopen] -R , R \mathclose[\) et la somme est continue. Le théorème~\ref{ThoCSGaPY} conclu.
\end{proof}

\begin{remark}
    À part lorsqu'on parle de fonction \( \eR\to \eR\), la notion de classe \( C^k\) s'entend au sens de la différentielle, et non de la dérivée, voir les définitions~\ref{DefPNjMGqy}. C'est cela qui explique la structure de la démonstration de la proposition~\ref{PropSNMEooVgNqBP}.
\end{remark}

\begin{corollary}[\cite{GYDXooJJusGH,MonCerveau}]       \label{CorCBYHooQhgara}
    La somme d'une série entière est de classe \( C^{\infty}\) sur le disque ouvert de convergence.
\end{corollary}

\begin{proof}
    La proposition~\ref{PropSNMEooVgNqBP} a démontré en réalité nettement plus : sur le disque ouvert de convergence, la somme est une fonction holomorphe. Il n'est cependant pas possible de conclure ainsi parce que le fait qu'une fonction holomorphe est \( C^{\infty}\) ne sera démontré qu'au coût de nombreux efforts dans le théorème~\ref{ThomcPOdd}\ref{ItemMRRTooMChmuZ}.

    \begin{subproof}
    \item[Cas réel]
        Nous considérons la série entière \( \sum_na_nx^n\) pour \( x\in \eR\) de rayon de convergence \( R\). Une simple récurrence sur la proposition~\ref{ProptzOIuG} donne le résultat.
    \item[Cas complexe]
        Attention : le fait d'être de classe \( C^k\) est le fait d'être \( k\) fois \emph{différentiable}. Rien à voir avec la \( \eC\)-dérivabilité.

        En ce qui concerne la différentiabilité nous avons la proposition~\ref{PropSNMEooVgNqBP} qui dit que dans le disque de convergence, la fonction \( u(z)=\sum_na_nz^n\) a pour différentielle l'application \( du\colon \eC\to \aL_{\eC}(\eC,\eC)\),
        \begin{equation}
            \begin{aligned}
                du\colon \eC&\to \aL_{\eC}(\eC,\eC) \\
                du_{z_0}(z)&=\big( \sum_{n=0}^{\infty}(n+1)a_{n+1}z_0^n \big)z.
            \end{aligned}
        \end{equation}
        Nous allons éviter de considérer la différentielle seconde comme une application
        \begin{equation}
            d^2u\colon \eC\to \aL\big( \eC,\aL(\eC,\eC) \big)
        \end{equation}
        parce que ça nous mènerait trop loin pour parler de la différentielle \( k\)\ieme. Au lieu de cela nous allons considérer l'isomorphisme d'espace vectoriel
        \begin{equation}
            \begin{aligned}
                \psi\colon \eC&\to \aL_{\eC}(\eC,\eC) \\
                z_0&\mapsto \psi(z_0) z=z_0z.
            \end{aligned}
        \end{equation}
        Dans cette optique nous écrivons :
        \begin{equation}
            du_{z_0}=\psi\big( \sum_{n=0}^{\infty}(n+1)a_{n+1} z_0^n\big)
        \end{equation}
        ou encore :
        \begin{equation}
            (\psi^{-1}\circ d)u(z_0)=\sum_{n\geq 0}(n+1)a_{n+1}z_{0}^n.
        \end{equation}
        Nous allons prouver par récurrence que l'égalité suivante est vraie (y compris le fait que la somme converge) :
        \begin{equation}
            (\psi^{-1}\circ d)^ku(z_0)=\sum_{n=0}^{\infty}\frac{ (n+k)! }{ n! }a_{n+k}z_0^n.
        \end{equation}
        Prouvons d'abord que cette somme converge pour tout \( k\). Nous avons \( (n+k)!/n!<(n+k)^k\) et donc il suffit de prouver que la série de coefficients \( n^ka_n\) converge. C'est le cas par la proposition~\ref{PropHDIUooKTbVSX}.

        Nous pouvons calculer la différentielle de \( (\psi^{-1}\circ d)^ku\) en dérivant terme à terme en utilisant (encore) la proposition~\ref{PropSNMEooVgNqBP}\ref{ItemUULDooEGRNiA} :
        \begin{subequations}
            \begin{align}
                d\big( (\psi^{-1}\circ d)^k u\big)_{z_0}(z)&=\sum_{n=1}^{\infty}\frac{ (n+k)! }{ n! }a_{n+k}na_{0}^{n-1}z\\
                &=\sum_{n=0}^{\infty}\frac{ (n+k+1)! }{ n! }a_{n+k+1}z_{0}^nz.
            \end{align}
        \end{subequations}
        Nous appliquons \( \psi^{-1}\) à cela :
        \begin{equation}
            (\psi^{-1}\circ d)^{k+1}u(z_0)=\sum_{k=0}^{\infty}\frac{ (n+k+1)! }{ n! }a_{n+k+1}z_0^n.
        \end{equation}

    \item[Dérouler à l'envers]

        Nous allons maintenant utiliser la proposition~\ref{PropEKLTooSvZjdW} pour montrer que \( u\) est de classe \( C^k\) pour tout \( k\). Nous avons démontré que \( (\psi^{-1}\circ d)^ku\) était différentiable. Par conséquent, \( d\big( (\psi^{-1}\circ d)^{k-1}u \big)\) est différentiable et donc \( (\psi^{-1}\circ d)^{k-1}\) est de classe \( C^1\). En continuant ainsi, \( (\psi^{-1}\circ d)^{k-l}u\) est de classe \( C^l\) et \( u\) est de classe \( C^k\).
    \end{subproof}
\end{proof}

Le lemme suivant est encore essentiellement valable dans un espace de Banach (proposition~\ref{PropQAjqUNp}).
\begin{lemma}       \label{LemPQFDooGUPBvF}
    La série entière \( \sum_{n\geq 0}z^{nk}\) a un rayon de convergence \( 1\) et converge vers la fonction
    \begin{equation}
        \sum_{n\geq 0}z^{nk}=\frac{1}{ 1-z^k }.
    \end{equation}

    Lorsque \( | \omega |=1\) nous avons aussi un rayon de convergence \( 1\) pour la série
    \begin{equation}        \label{EqSSHZooLwCBAZ}
        \frac{1}{ \omega-z }=\sum_{k\geq 0}\omega^{-k-1}z^k.
    \end{equation}

    Sous les mêmes hypothèses sur \( \omega\) nous avons encore la série
    \begin{equation}
        \frac{1}{ (\omega-z)^k }=\frac{1}{ (k-1)! }\sum_{s=0}^{\infty}\omega^{-s-1-k}\frac{ (s+k-1)! }{ s! }z^s
    \end{equation}

\end{lemma}

\begin{proof}
    Les coefficients de la série sont \( a_n=1\) lorsque \( n\) est multiple de \( k\) et \( a_n=0\) autrement. Donc pour \( r=1\) la suite \( r^na_n\) reste bornée\footnote{Utilisation directe de la définition~\ref{DefZWKOZOl}.}. Cela prouve que le rayon de convergence est au moins \( 1\). Par ailleurs si \( r>1\) alors clairement la suite \( (a_nr^n)\) n'est pas bornée. Cela prouve le rayon de convergence égal à \( 1\).

    Soit donc \( z\in B(0,1)\). Nous avons
    \begin{equation}
        \left( \sum_{n\geq 0}z^{nk} \right)(1-z^k)=\sum_{n\geq 0}z^{nk}-\sum_{n\geq 0}z^{(n+1)k}.
    \end{equation}
    Le premier terme de la première somme vaut \( 1\) tandis que tous les autres termes s'annulent deux à deux.

    En ce qui concerne la série \eqref{EqSSHZooLwCBAZ}, elle s'obtient facilement :
    \begin{equation}
        \frac{1}{ \omega-z }=\frac{1}{  \omega }\frac{1}{ 1-\frac{ z }{ \omega } }=\frac{1}{ \omega }\sum_{s=0}^{\infty}\left( \frac{ z }{ \omega } \right)^s=\sum_s\omega^{-s-1}z^s.
    \end{equation}

    La troisième série s'obtient en dérivant la seconde, ce qui est permis dans le disque de convergence par la proposition~\ref{ProptzOIuG}.
\end{proof}

\begin{remark}
    Sur le bord du disque de convergence, la série \( \sum_nz^{nk}\) ne converge pas. En effet le rayon étant \( 1\), sur le bord nous avons la série \( \sum_n e^{ink\theta}\) dont la norme du terme général ne tend pas vers zéro.
\end{remark}

%---------------------------------------------------------------------------------------------------------------------------
\subsection{Intégration}
%---------------------------------------------------------------------------------------------------------------------------

\begin{proposition} \label{PropfeFQWr}
    Soit la série entière $\sum a_nx^n$ de rayon de convergence \( R\).
    \begin{enumerate}
        \item
            Pour tout segment \( \mathopen[ a , b \mathclose]\subset\mathopen] -R , R \mathclose[\) nous pouvons intégrer terme à terme :
            \begin{equation}
                \int_a^b\left( \sum_{n=0}^{\infty}a_nx^n\right)dx=\sum_{n=0}^{\infty}a_n\int_a^bx^ndx.
            \end{equation}
        \item
            La série entière obtenue en intégrant terme à terme a le même rayon de convergence que celui de la série de départ.
    \end{enumerate}
\end{proposition}
\index{permuter!série entière et intégration}

\begin{proof}
    La première assertion est un cas particulier du théorème général~\ref{ThoCciOlZ}. Pour le rayon de convergence, le lemme~\ref{LemFVMaSD} fait le travail.
\end{proof}

Vu que le rayon de convergence ne varie pas par la dérivation ou par l'intégration et qu'une série entière est de classe \(  C^{\infty}\) sur son disque de convergence, nous pouvons dériver terme à terme autant de fois que nous le voulons sans faire de fautes dans le disque de convergence.

%+++++++++++++++++++++++++++++++++++++++++++++++++++++++++++++++++++++++++++++++++++++++++++++++++++++++++++++++++++++++++++ 
\section{Séries de Taylor}
%+++++++++++++++++++++++++++++++++++++++++++++++++++++++++++++++++++++++++++++++++++++++++++++++++++++++++++++++++++++++++++
\label{SECooDWRMooUKSuPh}

\begin{normaltext}
    Avant de commencer, une petite formule de dérivation toute simple que nous allons utiliser souvent :
    \begin{equation}        \label{EqSOFdwhw}
        (z^k)^{(l)}=\begin{cases}
            0   &   \text{si } l>k\\
            \frac{ k! }{ (k-l)! }z^{k-l}    &    \text{sinon.}
        \end{cases}
    \end{equation}

    Dans les cas où il est permis de dériver terme à terme, nous avons la formule
    \begin{equation}        \label{EQooTNOMooJZClvE}
        f^{(p)}(x)=\sum_ka_k(x^k)^{(p)}=\sum_{k=p}^{\infty}a_k\frac{ k! }{ (k-p)! }x^{k-p}
    \end{equation}
\end{normaltext}

%--------------------------------------------------------------------------------------------------------------------------- 
\subsection{Polynôme de Taylor d'une série entière}
%---------------------------------------------------------------------------------------------------------------------------

Le polynôme de Taylor d'une fonction définie par une série entière s'obtient en tronquant la série. Cela est une assez bonne nouvelle que nous allons démontrer maintenant.

\begin{proposition}[\cite{MonCerveau}]      \label{PROPooQLHNooRsBYbe}
    Soit une série entière
    \begin{equation}
        f(x)=\sum_ka_kx^k
    \end{equation}
    de rayon de convergence \( R>0\).

    Pour tout \( n\in \eN\), il existe une fonction \( \alpha\) telle que
    \begin{equation}    \label{EQooSXUJooFjsVek}
        f(x)=\sum_{k=0}^na_kx^k+\alpha(x)x^n
    \end{equation}
    et 
    \begin{equation}
        \lim_{t\to 0} \alpha(t)=0.
    \end{equation}
    Tout ceci étant convenu que
    \begin{itemize}
        \item 
            l'égalité \eqref{EQooSXUJooFjsVek} est uniquement valable sur le disque de convergence,
        \item La fonction \( \alpha\) dépend de \( n\).
    \end{itemize}
\end{proposition}

\begin{proof}
    Le corolaire \ref{CorCBYHooQhgara} nous indique que \( f\) est de classe \(  C^{\infty}\) sur \( \mathopen] -R , R \mathclose[\) et que nous pouvons dériver terme à terme.

        En utilisant la formule \eqref{EQooTNOMooJZClvE} et en l'évaluant en \( x=x_0\), tous les termes s'annulent sauf \( k=p\):
        \begin{equation}
            f^{(p)}(0)=p!a_p.
        \end{equation}
        Le théorème de Taylor \ref{ThoTaylor} nous indique alors qu'il existe \( \alpha\colon \eR\to \eR\) telle que \( \lim_{t\to 0} \alpha(t)=0\) et
        \begin{equation}
            f(x)=\sum_{k=0}^{n}a_kx^k+\alpha(x)x^n.
        \end{equation}
\end{proof}

%--------------------------------------------------------------------------------------------------------------------------- 
\subsection{Une majoration pour le reste}
%---------------------------------------------------------------------------------------------------------------------------

\begin{lemma}       \label{LEMooOVPIooAPWFOm}
    Soit une fonction \( f\colon \eR\to \eR\) dérivable \( n+1\) fois sur \( B(a,R)\). Alors pour tout \( x\in B(a,r)\),
    \begin{equation}
        f(x)=f(a)+\sum_{k=1}^{n-1}\frac{ (x-a)^{k} }{ k! }f^{(k)}(a)+\int_a^{x}\ldots\int_a^{u_{n-1}}f^{(n)}(u_n)du_n\ldots du_1.
    \end{equation}
\end{lemma}

\begin{proof}
    Nous allons intensivement utiliser le théorème fondamental du calcul intégral \ref{ThoRWXooTqHGbC} sous la forme de la formule \eqref{EqooBBCYooNweVrF}. Nous avons d'abord
    \begin{equation}
        f(x)=f(a)+\int_a^x f'(u_1)du_1=\int_a^x\big[ f'(a)+\int_a^{u_1}f''(u_2)du_2 \big]du_1.
    \end{equation}
    Toute l'astuce de ce théorème est de continuer à substituer \( f^{(k)}(t)\) par \( f^{(k)}(a)\) plus une intégrale de \( a\) à \( t\) de \( f^{(k+1)}(u)\). Nous démontrons ainsi par récurrence que
    \begin{equation}        \label{EQooOWJMooHATpMV}
        f(x)=f(a)+\sum_{k=1}^{n-1}\frac{ (x-a)^k }{ k! }f^{(k)}(a)+\int_a^x\cdots\int_a^{u_{n-1}}f^{(n)}(u_n)du_n\ldots du_1.
    \end{equation}
    La preuve de cela se fait en substituant
    \begin{equation}
        f^{(n)}(u_n)=f^{(n)}(a)+\int_{a}^{u_n}f^{(n+1)}(u_{n+1})du_{n+1}
    \end{equation}
    et en remarquant (encore par récurrence par exemple) que
    \begin{equation}
        \int_a^x\ldots \int_a^{u_{n-1}}du_n\ldots du_1=\frac{ (x-a)^n }{ n! }.   
    \end{equation}
\end{proof}

Le théorème suivant donne majoration du reste du polynôme de Taylor. Il est un premier pas dans la démonstration de formules comme
\begin{equation}
    \lim_{n\to \infty} P_n(x)=f(x)
\end{equation}
lorsque \( P_n\) est un polynôme de Taylor autour d'un point \( a\neq x\). Nous ne saurions trop insister sur le fait que de telles formules ne seraient valables que pour une classe relativement restreintes de fonctions.
\begin{theorem}[Inégalité de Taylor\cite{ooNVJGooGKwDWG}]       \label{THOooEUVEooXZJTRL}
    Soit une fonction \( f\colon \eR\to \eR\) dérivable \( n+1\) fois et telle que \( | f^{(n+1)}(x) |\leq M_N\) sur \( B(a,d)\). Alors
    \begin{equation}
        | R_n(x) |\leq \frac{ M_n }{ (n+1)! }| x-a |^{n+1}
    \end{equation}
    où \( R(x)=f(x)-P_n(x)\) et où \( P_n\) sont les polynômes de Taylor autour de \( a\in \eR\).
\end{theorem}

\begin{proof}
    Nous pouvons écrire la formule du lemme \ref{LEMooOVPIooAPWFOm} pour \( n+1\) au lieu de \( n\); cela donne
    \begin{equation}
        f(x)=P_n(x)+\int\cdots,
    \end{equation}
    et donc
    \begin{equation}
        | R_n(x) |=| P_n(x)-f(x) |=\int_a^x\ldots\int_a^{u_n}f^{(n+1)}(x)du_n\cdots du_{1}
    \end{equation}
    En effectuant toutes les intégrales nous trouvons\quext{Je me demande si je n'ai pas une faute entre \( n\) et \( n+1\) quelque part. Relisez attentivement et écrivez-moi si vous trouvez une faute.}
    \begin{equation}
        | R_n(x) |\leq \frac{ M_n }{ (n+1)! }| x-a |^{n+1}.
    \end{equation}
\end{proof}
Cette formule pour le reste est très bien, mais pour l'exploiter au maximum de ses possibilités, il faudra la notion de convergence de suite de fonctions, et en particulier la notion de série de fonctions, pour pouvoir écrire 
\begin{equation}
    f(x)=\sum_{k=0}^{\infty}\frac{ f^{(k)}(a) }{ k! }x^k
\end{equation}
lorsque cela est possible. Nous renvoyons donc aux séries de Taylor, section \ref{SECooDWRMooUKSuPh}, et en particulier aux fonctions analytiques de la sous-section \ref{SUBSECooXKHWooEzqGRJ}.

%--------------------------------------------------------------------------------------------------------------------------- 
\subsection{Fonctions analytiques}
%---------------------------------------------------------------------------------------------------------------------------
\label{SUBSECooXKHWooEzqGRJ}

Nous avons vu les polynômes de Taylor et déjà noté qu'il n'est pas en général vrai que \( \lim_{n\to \infty} P_n(x)=f(x)\) pour des \( x\) même proches du point autour duquel les polynômes de Taylor \( P_n\) sont calculés.

Nous allons maintenant étudier la classe des fonctions pour lesquelles la série de Taylor est égale à la fonction de départ. D'abord une proposition montrant que les coefficients de Taylor sont les seuls pour lesquels il est possible d'espérer avoir une telle propriété.
\begin{proposition}[\cite{ooSBUJooIuujhF}]      \label{PROPooTRWVooETTtbP}
    Soit une fonction donnée par la série entière
    \begin{equation}
        f(x)=\sum_{k=0}^{\infty}c_n(x-a)^n
    \end{equation}
    sur la boule de convergence \( B(a,R)\) avec \( R>0\) (hypothèse : le rayon de convergence est strictement positif). Alors
    \begin{equation}
        c_n=\frac{ f^{(n)}(a) }{ n! }.
    \end{equation}
\end{proposition}

\begin{proof}
    Par hypothèse, nous avons un rayon de convergence \( R>0\), et le corolaire \ref{CorCBYHooQhgara} nous indique que \( f\) y est de classe \(  C^{\infty}\). Et nous pouvons dériver terme à terme par la proposition \ref{ProptzOIuG}. Cela pour dire qu'il nous est autorisé d'utiliser la formule \eqref{EQooTNOMooJZClvE} pour calculer les dérivées de \( f\) au point \( a\). Nous avons d'abord
    \begin{equation}
        f^{(p)}(x)=\sum_{n=p}^{\infty}c_n\frac{ n! }{ (n-p)! }(x-a)^{n-p},
    \end{equation}
    et donc
    \begin{equation}
        f^{(p)}(a)=c_pp!
    \end{equation}
    qui donne immédiatement le résultat.
\end{proof}

\begin{proposition}
    Soit l'intervalle \( I=B(a,r)\). Si il existe \( M\) tel que 
    \begin{equation}
        | f^{(n)}(x) |\leq \frac{ M }{ r^n }n!
    \end{equation}
    pour tout \( x\in B(a,r)\). Alors nous avons la convergence simple
    \begin{equation}
        P_n\to f
    \end{equation}
    sur \( B(a,r)\). Ici, \( P_n\) est le polynôme de Taylor d'ordre \( n\) pour la fonction \( f\) autour du point \( a\)\footnote{Pour être complet, il faut préciser que \( P_n\) est calculé dans ZFC. C'est pour cela que nous n'écrivons pas des lourdeurs comme \( P_{n,a}(f)(x)\); si il fallait donner tout le contexte dans la notation, on n'en sortirait pas.

Ah, et tant que j'y suis si vous ne savez pas ce qu'est ZFC, je vous déconseille fortement de répéter cela à un jury d'agrég, entre autres parce que vous allez attirer la question «vraiment ? Vous utilisez C ? Où ? Pourquoi ?». Et là, bonne chance.}.
\end{proposition}

\begin{proof}
    Vu que nous avons \( | f^{(n)}(x) |\leq \frac{ M }{ r^n }n!\) pour tout \( x\), nous pouvons poser 
    \begin{equation}
        M_n=\frac{ M }{ r^n }n!
    \end{equation}
    dans le théorème \ref{THOooEUVEooXZJTRL} pour le faire fonctionner. Nous avons alors
    \begin{equation}
        | R_n(x) |\leq \frac{ M }{ r^n }n!\frac{1}{ (n+1)! }| x-a |^{n+1}=\frac{ M }{ n+1 }| x-a |\left| \frac{ x-a }{ r } \right|^n.
    \end{equation}
    Vu que \( x\in B(a,r)\) nous avons \( | x-a |<r\) et donc \( |(x-a)/r |^n<1\). Nous pouvons aussi majorer \( | x-a |\) par \( r\) et écrire
    \begin{equation}
        | R_n(x) |\leq \frac{ rM }{ n+1 }.
    \end{equation}
    Nous avons donc bien \( \lim_{n\to \infty} R_n(x)\to 0\).
\end{proof}

% This is part of Mes notes de mathématique
% Copyright (c) 2011-2020
%   Laurent Claessens
% See the file fdl-1.3.txt for copying conditions.

%+++++++++++++++++++++++++++++++++++++++++++++++++++++++++++++++++++++++++++++++++++++++++++++++++++++++++++++++++++++++++++
\section{Exponentielle sur une algèbre normée}
%+++++++++++++++++++++++++++++++++++++++++++++++++++++++++++++++++++++++++++++++++++++++++++++++++++++++++++++++++++++++++++

%--------------------------------------------------------------------------------------------------------------------------- 
\subsection{Définition}
%---------------------------------------------------------------------------------------------------------------------------

Dans ce qui suit, nous considérons une algèbre commutative.
\begin{propositionDef}[Exponentielle\cite{MonCerveau}]       \label{DEFooSFDUooMNsgZY}
    Soit \( (A,\| . \|)\) une algèbre\footnote{Définition~\ref{DefAEbnJqI}.} commutative de dimension finie sur \( \eC\) munie d'une norme d'algèbre. Pour \( x\in A\) nous définissons
    \begin{equation}        \label{EQooCUVTooGNOrFj}
        \exp(x)=\sum_{k=0}^{\infty}\frac{ x^k }{ k! }.
    \end{equation}
    Cette définition a les propriétés suivantes :
    \begin{enumerate}
        \item
            C'est bien défini pour tout \( x\in A\). C'est-à-dire que pour chaque \( x\), la série \eqref{EQooCUVTooGNOrFj} converge.
        \item
            Cela donne une application continue \( \exp\colon A\to A\).
        \item       \label{ITEMooGGVAooVfhGuu}
            La fonction \( \exp\) est différentiable et
            \begin{equation}        \label{EQooKWBUooLUdBAw}
                (d\exp)_x(y)=\exp(x)y,
            \end{equation}
            le dernier produit étant la structure d'algèbre sur \( A\).
    \end{enumerate}
\end{propositionDef}

\begin{proof}
    Pour la différentiabilité de \( \exp\), nous voulons utiliser le théorème~\ref{ThoLDpRmXQ}. Pour cela nous posons
    \begin{equation}
        u_k(x)=\frac{ x^k }{ k! }
    \end{equation}

    \begin{subproof}
        \item[Convergence simple]
            Nous prouvons la convergence simple, c'est-à-dire pour chaque \( x\) séparément, de la série \eqref{EQooCUVTooGNOrFj} dans deux buts. D'abord de nous assurer que la définition posée de \( \exp\) a un sens, et ensuite pour commencer à vérifier les hypothèses du théorème~\ref{ThoLDpRmXQ}.

            Nous montrons que les sommes partielles forment une suite de Cauchy. Nous fixons \( x\in A\) et nous posons
            \begin{equation}
                s_n=\sum_{k=0}^{\infty}\frac{ x^k }{ k! }.
            \end{equation}
            Soient \( p>q\), deux entiers. Nous avons :
            \begin{equation}        \label{EQooYNZNooDaiPhU}
                \| s_p-s_q \|=\| \sum_{k=q+1}^p\frac{ x^k }{ k! } \|\leq \sum_{k=q+1}^p\frac{ \| x^k \| }{ k! }\leq \sum_{k=q+1}^p\frac{ \| x \|^k }{ k! }
            \end{equation}
            où nous avons utilisé le fait que la norme sur \( A\) soit une norme d'algèbre.

            C'est le moment d'utiliser la série exponentielle donnée dans l'exemple~\ref{ExIJMHooOEUKfj} que nous appliquons avec \( t=\| x \|\). La série donnée par les coefficients \( a_k=\| x \|^k/k!\) converge et ses sommes partielles forment en particulier une suite de Cauchy. Donc ce que nous avons à droite dans \eqref{EQooYNZNooDaiPhU} peut être rendu arbitrairement petit lorsque \( p\) et \( q\) sont grands.

        \item[\( u_k\) est continue]
            Il s'agit de remarquer que \( (x+h)^k=x^k+hC(x,h)\) où \( C\) est une fonction bornée de \( h\) (lorsque \( h\) est dans un voisinage de \( 0\in A\)). Donc
            \begin{equation}
                \| (x+h)^k-x^k \|\leq \| h \|\| C(x,h) \|\to 0.
            \end{equation}
        \item[Candidat différentielle de \( u_k\)]
            Nous trouvons à présent un candidat à être différentielle de \( u_k\). Pour cela nous faisons le calcul suivant, sans trop nous soucier de la rigueur :
            \begin{equation}
                (du_k)_x(y)=\Dsdd{ u_k(x+ty) }{t}{0}=k\frac{1}{ k! }x^{k-1}y=u_{k-1}(x)y.
            \end{equation}
        \item[\( u_k\) est différentiable]
            Nous fixons \( x\in A\) et nous posons \( T(y)=u_{k-1}(x)y\). Ensuite nous vérifions que cela vérifie la définition de la différentielle : nous devons calculer
            \begin{equation}        \label{EQooNPKGooVmEYAV}
                \lim_{h\to 0} \frac{ u_k(x+h)-u_k(x)-T(h) }{ \| h \| }=\lim_{h\to 0} \frac{ (x+h)^k-x^k-kx^{k-1}h }{ k! \| h \| }=\clubsuit.
            \end{equation}
            Vous vous souvenez de la formule pour \( (x+h)^k\) ? Essayez de vous en souvenir. Le premier terme est \( x^k\), et le second est \( kx^{k-1}h\). Pour le reste c'est un polynôme dont tous les termes contiennent au moins \( h^2\). Nous avons donc
            \begin{equation}
                \clubsuit=\lim_{h\to 0} \frac{ h^2P(x,h) }{ k!\| h \| }=0.
            \end{equation}
            Nous en concluons que \( u_k\) est différentiable et que
            \begin{equation}
                (du_k)_x(y)=u_{k-1}(x)y.
            \end{equation}
        \item[\( u_k\) est de classe \( C^1\)]
            Nous devons démontrer que la différentielle est continue; cela est la continuité de l'application
            \begin{equation}
                \begin{aligned}
                    du_k\colon A&\to \aL(A,A) \\
                    x&\mapsto (du_k)_x.
                \end{aligned}
            \end{equation}
            La topologie sur \( A\) est celle de la norme, et celle sur \( \aL(A,A)\) est celle de la norme opérateur associée à la norme sur $A$. Nous avons\footnote{N'oubliez pas de faire à part le cas \( k=0\) parce que ce qui suit n'est correct que pour \( k\geq 1\).} :
            \begin{subequations}
                \begin{align}
                    \lim_{h\to 0} \| (du_k)_{x+h}-(du_k)_x \|&=\lim_{h\to 0} \sup_{\| y \|=1}\| u_{k-1}(x+h)y-u_{k-1}(x)y \|\\
                    &\leq\lim_{h\to 0} \sup_{\| y \|=1}\| u_{k+1}(x+h)-u_{k-1}(x) \|\| y \|\\
                    &=\lim_{h\to 0} \| u_{k+1}(x+h)-u_{k-1}(x) \|.
                \end{align}
            \end{subequations}
            Le fait que cette limite valle zéro est maintenant la continuité de \( u_{k-1}\).

        \item[Convergence normale sur tout compact]

            Soit un compact \( K\) de \( A\). Par le théorème de Borel-Lebesgue~\ref{ThoXTEooxFmdI}, \( K\) est fermé et borné. C'est pour ceci que nous avons supposé que \( A\) était de dimension finie sur \( \eR\). Soit donc \( R>0\) tel que \( \| y \|<R\) pour tout \( y\in K\). Nous avons
            \begin{equation}
                \| du_k \|_K=\sup_{x\in K}\| (du_k)_x \|=\sup_{x\in K}\frac{ \| x^{k-1} \| }{ (k-1)! }\leq \sup_{x\in K}\frac{ \| x \|^{k-1} }{ (k-1)! }\leq \frac{ R^{k-1} }{ (k-1)! }.
            \end{equation}
            Mais la série \( \sum_{k=0}^{\infty}\frac{ R^k }{k!}\) converge. Nous avons donc la convergence normale demandée.

        \item[Conclusion]

            Le théorème~\ref{ThoLDpRmXQ} conclu que l'exponentielle est de classe \( C^1\) et que sa différentielle est donnée par la formule
            \begin{equation}
                (d\exp)_x(y)=\sum_{k=0}^{\infty}(du_k)_x(y)=\sum_{k=1}^{\infty}(du_k)_x(y)=\sum_{k=0}^{\infty}u_k(x)y=\exp(x)y.
            \end{equation}
            Notez le jeu d'indices : \( du_k=0\) lorsque \( k=0\) (ce qui permet de faire commencer la somme à \( 1\)) et ensuite \( du_k\) fait intervenir \( u_{k-1}\) (ce qui fait revenir le départ de la somme à \( k=0\)).

    \end{subproof}
\end{proof}

\begin{normaltext}
    Lorsque nous disons que la différentielle de l'exponentielle est l'exponentielle elle-même, nous référons au point~\ref{DEFooSFDUooMNsgZY}\ref{ITEMooGGVAooVfhGuu} : la différentielle de \( \exp\) en \( x\) est l'opérateur de multiplication par \( \exp(x)\).

    Nous pouvons comprendre maintenant que \( \exp\) est même de classe \(  C^{\infty}\) parce qu'à chaque différentiation nous tombons sur la même fonction, laquelle est de classe au moins \( C^1\).

    Cependant, pour formaliser ça, il faut un peut travailler. Le cauchemar des différentielles successives d'une application \( A\to A\) est que les espaces en jeu sont des emboîtements terribles de \( \aL(A,\aL(A,\aL(A,A)))\).

    Ce qui nous sauve est que l'espace \( \aL(A,V)\) est un \( A\)-module, quel que soit \( V\). En particulier lorsque \( V\) est lui-même déjà un emboîtement. Faisons un lemme pour voir comment ça fonctionne.
\end{normaltext}

%--------------------------------------------------------------------------------------------------------------------------- 
\subsection{Différentielles}
%---------------------------------------------------------------------------------------------------------------------------

\begin{lemma}[\cite{MonCerveau}]
    Soient deux espaces vectoriels normés \( E\) et \( V\) tels que \( V\) soit un \( E\)-module\footnote{Définition~\ref{DEFooHXITooBFvzrR}.}. Nous supposons les normes soient telles que \( \| xv \|_{V}\leq \| x \|_E\| v \|_V\).

    Soit une fonction différentiable \( f\colon E\to V\) telle que la différentielle \( df\colon E\to \aL(E,V)\) soit de la forme
    \begin{equation}
        df_x(y)=yg(x)
    \end{equation}
    pour une certains fonction différentiable \( g\colon E\to V\).

    Alors \( f\) est \( C^1\), et deux fois différentiable telle que
    \begin{equation}
        \begin{aligned}
            d^2f\colon E&\to \aL\big( E,\aL(E,V) \big) \\
            (d^2f)_x(y)z&=z(dg_x)(y)
        \end{aligned}
    \end{equation}
    pour tout \( x,y,z\in E\).
\end{lemma}

\begin{proof}
    En plusieurs étapes.
    \begin{subproof}
        \item[\( f\) est \( C^1\)]
            Nous savons, par hypothèse, que \( f\) est différentiable. Il faut montrer que sa différentielle est continue, en remarquant déjà que \( g\) est continue parce que différentiable.

            Soit \( x_k\stackrel{E}{\longrightarrow}x\), et calculons \( \| df_{x_k}-df_x \|\) :
            \begin{equation}
                \begin{aligned}[]
                    \| df_{x_k}-df_x \|&=\sup_{\| y \|=1}\| df_{x_k}(y)-df_x(y) \|\\
                    &=\sup_{\| y \|=1}\| \big(g(x_k)-g(x)\big)y \|\\
                    &\leq\sup_{\| y \|=1}\| g(x_k)-g(x) \|\| y \|\\
                    &=\| g(x_k)-g(x) \|.
                \end{aligned}
            \end{equation}
            Donc nous avons bien \(df_{x_k}\stackrel{\aL(E,V)}{\longrightarrow}df_x\), ce qui signifie la continuité de \( df\). Donc \( f\) est de classe \( C^1\).

        \item[\( f\) est deux fois différentiable]

            Pour montrer que \( df\) est différentiable, nous mettons directement dans la définition \eqref{DefDifferentiellePta} le candidat
            \begin{equation}
                \begin{aligned}
                    T_x(h)\colon R&\to V \\
                    T_x(h)z&=zdg_x(y).
                \end{aligned}
            \end{equation}
            Nous devons vérifier la limite suivante :
            \begin{equation}        \label{EQooTBCKooRxBCum}
                \lim_{h\stackrel{E}{\longrightarrow} 0} \frac{ df_{x+h}-df_x-T_x(h) }{ \| h \| }=0.
            \end{equation}
            Étudions la norme du numérateur :
            \begin{subequations}
                \begin{align}
                    \| df_{x+h}-df_x-T_x(h) \|&=\sup_{\| y \|=1}\| df_{x+h}(y)-df_x(y)-T_x(h)y \|\\
                    &=\sup_{\| y \|=1}\| yg(x+h)-yg(x)-ydg_x(h) \|\\
                    &\leq \sup_{\| y \|=1}\| y \| \| g(x+h)-g(x)-dg_x(h) \|.
                \end{align}
            \end{subequations}
            La limite \eqref{EQooTBCKooRxBCum} se déduit donc de la différentiabilité de \( g\).
    \end{subproof}
    Note : la partie démontrant que \( f\) est \( C^1\) n'est pas strictement obligatoire parce qu'en vérifiant que \( f\) est deux fois différentiable, nous vérifions de facto que \( df\) est en particulier continue.
\end{proof}

\begin{lemma}[\cite{MonCerveau}]   \label{LEMooTUWQooMCCDcm}
    Soient des algèbres normées \( A\) et \( V\) telles que \( V\) soit un \( A\)-module vérifiant \( \| xv \|\leq \| x \|\| v \|\) pour tout \( x\in A\) et \( v\in V\). Alors \( \aL(A,V)\) est un \( A\)-module vérifiant \( \| x\alpha \|\leq \| x \|\|\alpha  \|\) pour tout \( x\in A\) et \( \alpha\in \aL(A,V)\).
\end{lemma}

\begin{proof}
    C'est un simple calcul utilisant la norme opérateur :
    \begin{equation}
            \| x\alpha \|=\sup_{\| y \|=1}\| (x\alpha)y \|
            =\sup_{\| y \|=1}\| x\alpha(y) \|
            \leq \sup_{\| y \|=1}\| x \|\| \alpha(y) \|
            =\| x \|\sup_{\| y \|=1}\| \alpha(y) \|
            =\| x \|\| \alpha \|.
    \end{equation}
\end{proof}

\begin{proposition}[\cite{MonCerveau}]      \label{PROPooTBDAooQouzSk}
    La fonction \( \exp\colon A\to A\) est de classe \(  C^{\infty}\) et vérifie, pour tout \( k\geq 1\) la récurrence
    \begin{equation}
        (d^k\exp)_x(y)=y(d^{k-1}\exp)_x.
    \end{equation}
\end{proposition}

\begin{proof}
    La formule proposée fonctionne avec \( k=1\) :
    \begin{equation}
        (d\exp)_x(y)=y\exp(x).
    \end{equation}
    C'est la relation~\ref{EQooKWBUooLUdBAw}.

    Nous considérons \( k>1\), nous supposons que \( \exp\) est de classe \( C^{k-1}\) et \( k\) fois différentiable. Nous allons prouver que \( \exp\) est alors de classe \( C^k\) et \( k+1\) fois différentiable, et que la différentielle de \( d^k\exp\) est donné par la formule
    \begin{equation}
        (d^{k+1}\exp)_x(y)=y(d^{k}\exp)_x.
    \end{equation}

    Pour nous mettre au clair avec les espaces en présence, nous supposons que
    \begin{subequations}
        \begin{align}
            d^{k-1}\exp&\colon A\to \aL(A,V)\\
            d^{k}\exp&\colon A\to \aL\big( A,\aL(A,V) \big)
        \end{align}
    \end{subequations}
    pour un certain espace vectoriel normé \( V\), lequel est un de ces terrifiants emboîtement de type \( \aL\Big( A,\aL\big( A,\aL(A,A) \big) \Big)\). Il est bien un espace vectoriel normé, et également un \( A\)-module parce qu'on peut toujours définir la multiplication d'un élément \( v\in V\) par un élément \(x\in A\) comme étant la multiplication par \( x\) du résultat final de l'évaluation emboîtée, laquelle se termine par un élément de \( A\). Donc tout se met bien.

    Quoi qu'il en soit, nous posons
    \begin{equation}
        T_x(y)=y(d^{k}\exp)_x
    \end{equation}
    et nous vérifions ce que cela donne dans la définition de la différentielle. Si nous avons
    \begin{equation}
        \lim_{h\to 0} \frac{ (d^k\exp)_{x+h}-(d^k\exp)_x-T_x(h) }{ \| h \| }=0
    \end{equation}
    alors nous aurons prouvé tout ce qu'il nous faut.

    Le numérateur est une application \( A\to \aL(A,V)\); nous en écrivons la norme comme il se doit :
    \begin{subequations}
        \begin{align}
            \|   (d^k\exp)_{x+h}-(d^k\exp)_x-T_x(h) \|&=\sup_{\| y \|=1}\| (d^{k}\exp)_{x+h}(y)-(d^k\exp)_x(y)-h(d^{k}\exp)_xy \|\\
            &=\sup_{\| y \|=1}\| y(d^{k-1}\exp)_{x+h}-y(d^{k-1}\exp)_x-h(d^k\exp)_xy \|\\
            &=\sup_{\| y \|=1}\| y(d^{k-1}\exp)_{x+h}-y(d^{k-1}\exp)_x-hy(d^{k-1}\exp)_x \|\\
            &\leq \| (d^{k-1}\exp)_{x+h}-(d^{k-1}\exp)_x-h(d^{k-1}\exp)_x \|\\
            &=\| (d^{k-1}\exp)_{x+h}-(d^{k-1}\exp)_x-(d^{k}\exp)_x(h) \|.
        \end{align}
    \end{subequations}
    Dans ce calcul nous avons utilisé le lemme~\ref{LEMooTUWQooMCCDcm} et \( T_x(h)y=h(d^{k}\exp)_xy\).
    Maintenant, la limite
    \begin{equation}
        \lim_{h\to 0} \frac{  \| (d^{k-1}\exp)_{x+h}-(d^{k-1}\exp)_x-(d^{k}\exp)_x(h) \|.}{ \| h \| }
    \end{equation}
    n'est rien d'autre que la limite arrivant dans la définition du fait que \( d^k\exp\) est la différentielle de \( d^{k-1}\exp\). Cette limite est donc zéro comme nous voulions le prouver.
\end{proof}


Le théorème suivant est très important parce qu'il permet de définir l'exponentielle d'une matrice. Et les exponentielles de matrices sont utiles, entre très nombreuses autres choses pour résoudre certaines équations différentielles.
\begin{theoremDef}[\cite{MonCerveau}]      \label{THOooFGTQooZPiVLO}
    Soit une algèbre normée \( A\) (pas spécialement commutative). La formule
    \begin{equation}
        \exp(x)=\sum_{k=0}^{\infty}\frac{ x^k }{ k! }
    \end{equation}
    définit une fonction différentiable dont la différentielle est donnée par\quext{La fonction exponentielle est, j'en suis quasiment certain, de classe \(  C^{\infty}\). Si vous connaissez un moyen pas trop douloureux de prouver cela, faites-le moi savoir.}
    \begin{equation}        \label{EQooFGPPooZKHeXU}
        (d\exp)_x(y)=\sum_{j,j\in \eN}\frac{ x^iyx^j }{ (i+j+1)! }
    \end{equation}
\end{theoremDef}

\begin{normaltext}
    Nous ne démontrons pas cela ici.

    Il s'agit d'une adaptation de la proposition~\ref{DEFooSFDUooMNsgZY}. Là où il faut faire attention, c'est dans l'équation \eqref{EQooNPKGooVmEYAV} : il n'y a pas \( k\) termes \( x^{k-1}h\) dans \( (x+h)^k\), mais \( k\) termes de la forme \( x^ihx\). C'est pour cela que la différentielle n'est pas donnée par \( T(y)=u_{k-1}(x)y\), mais bien par la somme \eqref{EQooFGPPooZKHeXU}.

    M'est avis en réalité que toute la démonstration du théorème~\ref{PropXFfOiOb} passe facilement au cas présent.
\end{normaltext}

%--------------------------------------------------------------------------------------------------------------------------- 
\subsection{Séries dans une algèbre normée}
%---------------------------------------------------------------------------------------------------------------------------

Nous allons parler d'exponentielle de matrice. Avant cela, quelques propriétés qui sont valables sur des algèbres normées. Le principal exemple que nous avons en tête est \( \eA=\eM(n,\eC)\).

\begin{proposition}[\cite{MonCerveau}]      \label{PROPooMZZQooEhQsgQ}
    Soit une algèbre normée \( \eA\). Soient une suite d'éléments \( A_k\in \eA\) et un élément \( B\). Nous supposons que la somme \( \sum_{k=0}^{\infty}A_k\) converge. Alors
    \begin{equation}
        B\sum_kA_k=\sum_k(BA_k).
    \end{equation}
\end{proposition}

\begin{proof}
    Soit \( N\in \eN\). Nous avons:
    \begin{subequations}
        \begin{align}
            \| \sum_{k=0}^NBA_k-B\sum_{k=0}^{\infty}A_k \|&=\| B\sum_{k=N+1}^{\infty}A_k \| \label{SUBEQooDTNAooWpXOKP}\\
            &\leq \| B \|\| \sum_{k=N+1}^{\infty}A_k \|     \label{SUBEQooJPSJooAqXtOJ}
        \end{align}
    \end{subequations}
    Justifications:
    \begin{itemize}
        \item Pour \eqref{SUBEQooDTNAooWpXOKP}, c'est la linéarité du produit matriciel.
        \item Pour \eqref{SUBEQooJPSJooAqXtOJ}, c'est que la norme est une norme d'algèbre\footnote{Définition \ref{DefJWRWQue}. Pour rappel, la norme opérateur en est une par le lemme \ref{LEMooFITMooBBBWGI}.}.
    \end{itemize}
    À droite, la limite \( N\to \infty\) donne zéro parce que \( \| B \|\) est un simple nombre, et \( \| \sum_{k=N+1}^{\infty}A_k \|\) est une queue de suite convergente par hypothèse.

    Nous avons donc bien convergence
    \begin{equation}
        \lim_{N\to \infty}\sum_{k=0}^{N}BA_k=B\sum_{k=0}^{\infty}A_k.
    \end{equation}
\end{proof}

Nous adaptons le produit de Cauchy du théorème \ref{ThokPTXYC} au cas d'une algèbre normée.
\begin{proposition}[\cite{MonCerveau}]      \label{PROPooFMEXooCNjdhS}
    Soient une algèbre normée \( \eA\), un élément \( A\in \eA\), ainsi que des séries convergentes \( \sum_{k=0}^{\infty}a_kA^k\) et \( \sum_{l=0}^{\infty}b_lA^l\). Alors
    \begin{equation}
        \left( \sum_ka_kA^k \right)\left( \sum_lb_lA^l \right)=\sum_{n=0}^{\infty}\big( \sum_{m=0}^na_mb_{n-m} \big)A^n.
    \end{equation}
\end{proposition}

\begin{proof}
    Un calcul :
    \begin{subequations}
        \begin{align}
            \left( \sum_ka_kA^k \right)\left( \sum_lb_lA^l \right) &=\sum_k\big( \sum_lb_lA^l \big)a_kA^k       \label{SUBEQooFAECooWFCaNW}\\
            & = \sum_k\big( \sum_lb_la_kA^{l+k} \big)   \label{SUBEQooDZTHooMwmKxJ}\\
            &=\lim_{K\to\infty} \sum_{k=0}^K\big( \lim_{L\to \infty} \sum_{l=0}^Lb_la_kA^{k+l} \big)\\
            &=\lim_{K\to \infty} \lim_{L\to \infty} \sum_{k=0}^K\sum_{l=0}^Lb_la_kA^{k+l}       \label{SUBEQooISSHooJsyMTv}\\
            &=\lim_{K\to \infty} \lim_{L\to \infty} \sum_{n=0}^{K+L}\sum_{m=0}^na_mb_{n-m}A^n       \label{SUBEQooAWUQooZCHIXH}\\
            &=\lim_{K\to \infty} \sum_{n=0}^{\infty}\sum_{m=0}^na_mb_{n-m}A^m       \label{SUBEQooUVOBooSPGjrA}\\
            &=\sum_{n=0}^{\infty}\sum_{m=0}^na_mb_{n-m}A^m      \label{SUBEQooCGRGooGIDCYv}
        \end{align}
    \end{subequations}
    Justifications :
    \begin{itemize}
        \item Pour \eqref{SUBEQooFAECooWFCaNW}, la proposition \ref{PROPooMZZQooEhQsgQ} nous permet d'entrer l'élément \( \sum_lb_lA^l\in \eA\) dans la somme sur \( k\).
        \item 
            Pour \eqref{SUBEQooDZTHooMwmKxJ}, c'est la même chose.
        \item
            Pour \eqref{SUBEQooISSHooJsyMTv}, la somme sur \( k\) étant finie (pour chaque \( K\)), elle commute avec la limite sur \( L\).
        \item
            Pour \eqref{SUBEQooAWUQooZCHIXH}, c'est une manipulation de sommes finies. On regroupe les termes selon les puissances de \( A\).
        \item
            Pour \eqref{SUBEQooUVOBooSPGjrA}, c'est effectuer la limite sur \( L\) pour \( K\) fixé.
        \item
            Pour \eqref{SUBEQooCGRGooGIDCYv}, l'expression dans la limite sur \( K\) ne dépend pas de \( K\). Donc nous pouvons simplement supprimer la limite.
    \end{itemize}
\end{proof}

%--------------------------------------------------------------------------------------------------------------------------- 
\subsection{Exponentielle de matrice}
%---------------------------------------------------------------------------------------------------------------------------
\label{SECooBYQBooZifJsg}

\begin{proposition}[\cite{BIBooROFBooCcFjms}]
    Soient des matrices \( A,P\in \eM(n,\eC)\) telle que \( P\) soit inversible. Alors\footnote{La définition de l'exponentielle de matrice est \ref{DEFooSFDUooMNsgZY} où la convergence de la somme est celle de la norme opérateur \ref{DefNFYUooBZCPTr}.}
    \begin{equation}
        e^{P^{-1}AP}=P^{-1} e^{A}P.
    \end{equation}
\end{proposition}

\begin{proof}
    Pour chaque \( m\in \eN\) nous avons \( (P^{-1}AP)^m=P^{-1} A^mP\). Ensuite,
    \begin{equation}
        e^{P^{-1}AP}=\sum_k\frac{(P^{-1}AP)^k}{ k! }=\sum_k\frac{ P^{-1}A^kP }{ k! }=P^{-1}\sum_k\frac{ A^k }{ k! }P.
    \end{equation}
    Nous avons utilisé la proposition \ref{PROPooMZZQooEhQsgQ} pour sortir \( P^{-1}\) à gauche et \( P\) à droite de la somme.
\end{proof}

\begin{proposition}[\cite{BIBooROFBooCcFjms}]       \label{PROPooFLHPooRhLiZE}
    L'exponentielle de matrice vérifie
    \begin{enumerate}
        \item       \label{ITEMooCVALooEfLQCyI}
            \( e^0=\id\)
        \item       \label{ITEMooNGPWooIyPEQt}
            \( A^m e^{A}= e^{A}A^m\)
        \item       \label{ITEMooEOSMooQWjcjA}
            \( ( e^{A})^t= e^{(A^t)}\)
        \item       \label{ITEMooROPJooMarenu}
            Si \( AB=BA\) alors \( A e^{B}= e^{B}A\) et \(  e^{A} e^{B}= e^{B} e^{A}\).
    \end{enumerate}
\end{proposition}

\begin{proof}
    Point par point.
    \begin{subproof}
        \item[Pour \ref{ITEMooCVALooEfLQCyI}]
            Juste substituer \( A=0\) dans la définition. Tous les termes tombent sauf le premier. Il faut utiliser le fait que \( A^0=\id\).
        \item[Pour \ref{ITEMooNGPWooIyPEQt}]
            Il faut utiliser la proposition \ref{PROPooMZZQooEhQsgQ} pour écrire
            \begin{equation}        \label{EQooLUUVooCtUtIC}
                A^m\sum_k\frac{ A^k }{ k! }=\sum_k\frac{ A^mA^k }{ k! }=\sum_{k}\frac{ A^kA^m }{ k! }=\sum_k\frac{ A^k }{ k! }A^m.
            \end{equation}
        \item[Pour \ref{ITEMooEOSMooQWjcjA}]
            Pour chaque \( k\) nous avons l'égalité \( (A^k)^t=(A^t)^k\). En utilisant encore le coup de la queue de suite qui converge vers zéro,
            \begin{equation}
                \| \sum_{k=0}^N\frac{ (A^t)^k }{ k! }-( e^{A})^t \|=\| \sum_{k=N+1}^{\infty}\frac{ (A^t)^k }{ k! } \|\to 0.
            \end{equation}
        \item[Pour \ref{ITEMooROPJooMarenu}]
            Pour prouver \( A e^{B}= e^{B}A\), c'est le même genre de manipulations que \eqref{EQooLUUVooCtUtIC}.

            Maintenant, vu que \( A\) et \( e^B\) commutent, l'égalité à peine prouvée montre que \(  e^{A}\) et \(  e^{B}\) commutent.
    \end{subproof}
\end{proof}

\begin{proposition}[\cite{BIBooROFBooCcFjms}]       \label{PROPooKDKDooCUpGzE}
    Soient \( A\in \eM(n,\eC)\) ainsi que \( s,t\in \eC\). Alors
    \begin{equation}
         e^{sA} e^{tA} = e^{(s+t)A}.
    \end{equation}
\end{proposition}

\begin{proof}
    Nous calculons le produit \( e^{sA} e^{tA}\) par le produit de Cauchy de la proposition \ref{PROPooFMEXooCNjdhS} :
    \begin{equation}
        \clubsuit = \left( \sum_k\frac{ t^k }{ k! }A^k \right)\left( \sum_l\frac{ s^l }{ l! }A^l \right)=\sum_{n=0}^{\infty}\sum_{m=0}^n\frac{ t^m }{ m! }\frac{ s^{n-m} }{ (n-m)! }A^n.
    \end{equation}
    À ce point, nous multiplions et divisons par \( n!\) et nous réarrangons la somme de la façon suivante :
    \begin{equation}
        \clubsuit = \sum_{n=0}^{\infty}\frac{ A^n }{ n! }\sum_{m=0}^n\frac{ n! }{ m!(n-m)! }t^ms^{n-m}.
    \end{equation}
    Nous reconnaissons la somme sur \( m\) comme étant un binôme de Newton\footnote{Proposition \ref{PropBinomFExOiL}.} pour \( (t+s)^n\). Nous avons donc finalement
    \begin{equation}
        \clubsuit = \sum_{n=0}^{\infty}\frac{ \big( (t+s)A \big)^n }{ n! }= e^{(t+s)A}.
    \end{equation}
\end{proof}

La proposition suivante dit que les exponentielles de matrices sont inversibles. Elle ne dit pas que toutes les matrices inversibles sont des exponentielles. Ce sera la proposition \ref{PropKKdmnkD}.
\begin{proposition}[\cite{BIBooROFBooCcFjms}]       \label{PROPooRERRooMutKcg}
    Si \( A\in \eM(n,\eC)\), alors \(  e^{A}\) est inversible et 
    \begin{equation}
        ( e^{A})^{-1}= e^{-A}.
    \end{equation}
\end{proposition}

\begin{proof}
    Il suffit de prendre \( s=1\) et \( t=-1\) dans la proposition \ref{PROPooKDKDooCUpGzE} et nous avons
    \begin{equation}
        e^{A} e^{-A}= e^{0}=\mtu.
    \end{equation}
    Cela prouve que \(  e^{A}\) est inversible et que son inverse est \(  e^{-A}\).
\end{proof}

\begin{proposition}[\cite{BIBooROFBooCcFjms}]       \label{PROPooSDNNooQtHkhA}
    Soit \( A\in \eM(n,\eC)\). Nous considérons l'application
    \begin{equation}
        \begin{aligned}
            f\colon \eR&\to \GL(n,\eC) \\
            t&\mapsto  e^{tA}. 
        \end{aligned}
    \end{equation}
    Nous avons la formule de dérivation
    \begin{equation}
        f'(t)=A e^{tA}.
    \end{equation}
\end{proposition}

\begin{proof}
    Il s'agit de calculer la limite en utilisant la proposition \ref{PROPooKDKDooCUpGzE} :
    \begin{subequations}
        \begin{align}
            f'(t)&=\lim_{\epsilon\to 0}\frac{  e^{(t+\epsilon)A}- e^{tA} }{ \epsilon }\\
            &= e^{tA}\lim_{\epsilon\to 0}\frac{  e^{\epsilon A}-\mtu }{ \epsilon }\\
            &= e^{tA}\lim_{\epsilon\to 0}\frac{1}{ \epsilon }\big( \epsilon A+\sum_{k=2}^{\infty}\frac{ (\epsilon A)^k }{ k! } \big)\\
            &= e^{tA}\lim_{\epsilon\to 0}\big( A+\sum_{k=2}^{\infty}\frac{ \epsilon^{k-1}A }{ k! } \big)\\
            &=  e^{tA}A\\
            &= A e^{tA}
        \end{align}
    \end{subequations}
    où à la toute fin nous avons aussi utilisé la commutation de la proposition \ref{PROPooFLHPooRhLiZE}\ref{ITEMooNGPWooIyPEQt}.
\end{proof}

Le théorème suivant montre que le produit d'exponentielle de matrices suit la règle usuelle tant que les matrices commutent. Cela est cependant plutôt l'exception que la règle. À priori nous avons \(  e^{A} e^{B}\neq  e^{A+B}\).
\begin{theorem}[\cite{BIBooROFBooCcFjms}]       \label{THOooXCPEooYGyLOp}
    Soient \( A,B\in \eM(n,\eC)\) telles que \( AB=BA\). Alors
    \begin{equation}
        e^{A+B}= e^{A} e^{B}.
    \end{equation}
\end{theorem}

\begin{proof}
    Vu que \( A\) et \( B\) commutent nous avons \( A e^{tB}= e^{tB}A\) (proposition \ref{PROPooFLHPooRhLiZE}\ref{ITEMooROPJooMarenu}). Ensuite nous posons
    \begin{equation}
        g(t)= e^{t(A+B)} e^{-tB} e^{-tA}.
    \end{equation}
    Nous calculons la dérivée de \( g\) en utilisant la règle de Leibnitz et la proposition \ref{PROPooSDNNooQtHkhA} :
    \begin{equation}
        \begin{aligned}[]
            g'(t)&=(A+B) e^{t(A+B)} e^{-tB} e^{-tA}\\
            &\quad + e^{t(A+B)}(-B) e^{-tB} e^{-tA}\\
            &\quad +  e^{t(A+B)} e^{-tB}(-A) e^{-tA}.
        \end{aligned}
    \end{equation}
    Vu que \( A\), \( B\) et \( A+B \) commutent, nous pouvons réarranger les facteurs en
    \begin{equation}
        \begin{aligned}[]
            g'(t)&=(A+B) e^{t(A+B)} e^{-tB} e^{-tA}\\
            &\quad -B e^{t(A+B)} e^{-tB} e^{-tA}\\
            &\quad -A  e^{t(A+B)} e^{-tB} e^{-tA}.
        \end{aligned}
    \end{equation}
    Enfin, cela fait
    \begin{equation}
        g'(t)=(A+B-B-A) e^{t(A+B)} e^{-tB} e^{-tA}=0.
    \end{equation}
    Donc \( g\) est constante et nous avons
    \begin{equation}
        e^{t(A+B)} e^{-tB} e^{-tA}=g(0)=\mtu.
    \end{equation}
    En multipliant à droite par \(  e^{tA} e^{tB}\) nous trouvons
    \begin{equation}
        e^{t(A+B)}= e^{tA} e^{tB}
    \end{equation}
    comme annoncé.
\end{proof}

%+++++++++++++++++++++++++++++++++++++++++++++++++++++++++++++++++++++++++++++++++++++++++++++++++++++++++++++++++++++++++++
\section{Exponentielle et logarithme dans les réels}
%+++++++++++++++++++++++++++++++++++++++++++++++++++++++++++++++++++++++++++++++++++++++++++++++++++++++++++++++++++++++++++

Pour avoir une vue synthétique du plan, voir le thème \ref{THEMEooKXSGooCsQNoY}.

%--------------------------------------------------------------------------------------------------------------------------- 
\subsection{L'équation différentielle}
%---------------------------------------------------------------------------------------------------------------------------

Pour la suite nous notons \( y\) une solution de l'équation \( y'=y\), \( y(0)=1\), et nous allons en donner des propriétés indépendamment de l'existence, donnée par le théorème~\ref{ThoKRYAooAcnTut}.

\begin{proposition} \label{PropTLECooEiLbPP}
    Quelques propriétés de \( y\) (si elle existe) :
    \begin{enumerate}
        \item
            Pour tout \( x\in \eR\) nous avons \( y(x)y(-x)=1\).
        \item
            \( y(x)>0\) pour tout \( x\).
        \item
            \( y\) est strictement croissante.
    \end{enumerate}
\end{proposition}

\begin{proof}
    Nous posons \( \varphi(x)=y(x)y(-x)\) et nous dérivons :
    \begin{equation}
        \varphi'(x)=y'(x)y(-x)-y(x)y'(-x)=0.
    \end{equation}
    Donc \( \varphi\) est constante\footnote{Proposition~\ref{PropGFkZMwD}.}. Vu que \( \varphi(0)=1\) nous avons automatiquement \( y(x)y(-x)=1\) pour tout \( x\).

Les deux autres allégations sont simples : si \( y(x_0)<0\) alors il existe \( t\in\mathopen] x_0 , 1 \mathclose[\) tel que \( y(t)=0\), ce qui est impossible parce que \( y(t)y(-t)=1\). La stricte croissance de \( y\) s'ensuit.
\end{proof}

\begin{proposition}     \label{PROPooGGUIooExVHPM}
    Quelques formules pour tout \( a,b\in \eR\) et \( n\in \eZ\) :
    \begin{enumerate}
        \item       \label{ITEMooMPSUooWQpVQJ}
            \( y(a+b)=y(a)y(b)\)
        \item
            \( y(na)=y(a)^n\)
        \item
            \( y\left( \frac{ a }{ n } \right)=\sqrt[n]{y(a)}\).
    \end{enumerate}
\end{proposition}

\begin{proof}
    Nous posons \( h(x)=y(a+b-x)y(x)\) et nous avons encore \( h'(x)=0\) dont nous déduisons que $h$ est constante. De plus
    \begin{equation}
        h(0)=y(a+b)y(0)=y(a+b)
    \end{equation}
    et
    \begin{equation}
        h(b)=y(a)y(b).
    \end{equation}
    Vu que \( h\) est constante, ces deux expressions sont égales : \( y(a+b)=y(a)y(b)\).

    Forts de cette relation, une récurrence donne \( y(na)=y(a)^n\) pour tout \( n\in \eN\). De plus
    \begin{equation}
        y(a)=y\left( \frac{ a }{ n }\times n \right)=y\left( \frac{ a }{ n } \right)^n,
    \end{equation}
    ce qui donne \( y(a)=y(a/n)^n\) ou encore \( y(a/n)=\sqrt[n]{y(a)}\).

    Enfin pour les négatifs, si \( n\in \eN\),
    \begin{equation}
        y(-na)=\frac{1}{ y(na) }=\frac{1}{ y(a)^n }=y(a)^{-n}.
    \end{equation}
    Et de la même façon,
    \begin{equation}
        y\left( -\frac{ a }{ n } \right)=\frac{1}{ y\left( \frac{ a }{ n } \right) }=\sqrt[n]{\frac{1}{ y(a) }}=\sqrt[-n]{y(a)}.
    \end{equation}
\end{proof}

%--------------------------------------------------------------------------------------------------------------------------- 
\subsection{Existence}
%---------------------------------------------------------------------------------------------------------------------------

Jusqu'ici nous avons donné des propriétés d'une éventuelle fonction \( y\) qui vérifierait l'équation différentielle. Il est temps de montrer qu'une telle fonction existe.

\begin{theorem} \label{ThoKRYAooAcnTut}
    La série entière
    \begin{equation}    \label{EqEIGZooKWSvPS}
        \exp(x)=\sum_{k=0}^{\infty}\frac{ x^k }{ k! }
    \end{equation}
    définit une fonction dérivable solution de
    \begin{subequations}
        \begin{numcases}{}
            y'=y        \label{EQooSEIHooNmQKiC}\\
            y(0)=1.
        \end{numcases}
    \end{subequations}
\end{theorem}
\index{exponentielle!existence}

\begin{proof}
    La formule de Hadamard (théorème~\ref{ThoSerPuissRap}) donne le rayon de convergence de la série \eqref{EqEIGZooKWSvPS} par
    \begin{equation}
        \frac{1}{ R }=\lim_{k\to \infty} \frac{ \frac{1}{ (k+1)! } }{ \frac{1}{ k! } }=\lim_{k\to \infty} \frac{1}{ k+1 }=0.
    \end{equation}
    Donc nous avons un rayon de convergence infini. La fonction \( y\) est définie sur \( \eR\) et la proposition~\ref{ProptzOIuG} nous dit que \( y\) est dérivable. Nous pouvons aussi dériver terme à terme :
    \begin{equation}
            y'(x)=\sum_{k=0}^{\infty}\frac{ kx^{k-1} }{ k! }=\sum_{k=1}^{\infty}\frac{ kx^{k-1} }{ k! }=\sum_{k=1}^{\infty}\frac{ x^{k-1} }{ (k-1)! }=\sum_{k=0}^{\infty}\frac{ x^k }{ k! }=y(x).
    \end{equation}
    Notez le petit jeu d'indice de départ de \( k\). Dans un premier temps, nous remarquons que \( k=0\) donne un terme nul et nous le supprimons, et dans un second temps nous effectuons la simplification des factorielles (qui ne fonctionne pas avec \( k=0\)).
\end{proof}

\begin{normaltext}
    Nous savons que la fonction \( y\) existe parce qu'une solution de l'équation différentielle \( y'=y\), \( y(0)=1\) est donnée par la fameuse série (théorème \ref{ThoKRYAooAcnTut}). À part cela, ce qui a été fait avec cette équation différentielle ne permet pas de prouver l'existence de \( y\). Donc, du point de vue de «définir l'exponentielle par son équation différentielle», c'est pas encore gagné. Notons au passage que le nombre \( e\) n'est pas encore bien défini via l'équation différentielle.
\end{normaltext}

%--------------------------------------------------------------------------------------------------------------------------- 
\subsection{Le nombre de Neper \texorpdfstring{$ e$}{e}}
%---------------------------------------------------------------------------------------------------------------------------

Nous savons par le théorème \ref{ThoKRYAooAcnTut} que \( x\mapsto \exp(x)\) est une solution de l'équation différentielle exponentielle (avec la bonne condition initiale). Or une telle solution est unique par la proposition \ref{PropDJQSooYIwwhy}.

\begin{definition}[Le nombre de Neper]
    Nous notons \( e\) le nombre \( \exp(1)\).
\end{definition}

\begin{proposition}     \label{PropCELWooLBSYmS}
    Pour tout \( x\in \eR\), nous avons
    \begin{equation}        \label{EQooBFIHooKopcmf}
        \exp(x)=e^x.
    \end{equation}
\end{proposition}

\begin{proof}
    Soit \( y\) vérifiant la fameuse équation différentielle. Nous savons que \( y=\exp\) parce que c'est l'unique solution (proposition \ref{PropDJQSooYIwwhy}). Nous avons :
    \begin{equation}
        y(x)=y(1)^x.
    \end{equation}
    Si \( q\in \eQ\) alors \( q=a/b\) et
    \begin{equation}
        y(q)=y\left( \frac{ a }{ b } \right)=y\left( a\times \frac{1}{ b } \right)=y\left( \frac{1}{ b } \right)^a=\big( \sqrt[b]{y(1)} \big)^a=y(1)^{a/b}=y(1)^{q}.
    \end{equation}
    Le résultat est prouvé pour les rationnels.

    En ce qui concerne un élément général \( x\in \eR\), la fonction \( x\mapsto y(x)\) est continue sur \( \eR\), et la fonction \( x\mapsto e^x\) également (proposition \ref{DEFooOJMKooJgcCtq}). Ces deux fonctions étant égales sur \( \eQ\), elles sont égales sur \( \eR\) par la proposition  \ref{PropCJGIooZNpnGF}).
\end{proof}

Une conséquence des propositions \ref{PropCELWooLBSYmS} et \ref{PROPooGCBZooTcyGtO} est que 

\begin{subequations}    \label{EqLOIUooHxnEDn}
    \begin{align}
        \lim_{x\to -\infty}  e^{x}=0\\
        \lim_{x\to +\infty}  e^{x}=+\infty,
    \end{align}
\end{subequations}
et en particulier, 
\begin{equation}
    \begin{aligned}
    \exp\colon \eR&\to \mathopen] 0 , \infty \mathclose[ \\
        x&\mapsto  e^{x} 
    \end{aligned}
\end{equation}
est une bijection.

%--------------------------------------------------------------------------------------------------------------------------- 
\subsection{Application réciproque : logarithme}
%---------------------------------------------------------------------------------------------------------------------------

\begin{propositionDef}    \label{DEFooELGOooGiZQjt}
    L'application \(\exp\colon \eR\to \mathopen] 0 , \infty \mathclose[\) est une bijection. L'application réciproque
    \begin{equation}
        \ln\colon \mathopen] 0 , \infty \mathclose[\to \eR
    \end{equation}
    est le \defe{logarithme}{logarithme!sur les réels positifs}.
\end{propositionDef}

\begin{proof}
Le fonction exponentielle est dérivable, toujours strictement positive, donc strictement croissante. Les limites en \( \pm \infty\) sont \( 0\) et \( +\infty\). Le théorème des valeurs intermédiaires~\ref{ThoValInter} nous dit que c'est une bijection. En effet, l'injectivité est la stricte croissance. En ce qui concerne la surjection, soit \( y\in \mathopen] 0 , \infty \mathclose[\). Vu que la limite en \( -\infty\) est zéro, il existe \( A\in \eR\) tel que \( \exp(x)<y\) pour tout \( x<A\), et de la même façon, il existe \( B\in \eR\) tel que \( \exp(x)>y\) pour tout \( x>B\). Si \( a<A\) et \( b>B\) alors \( \exp(a)<y\) et \( \exp(b)>y\), donc \( y\) est dans l'image de \( \mathopen[ a , b \mathclose]\) par l'exponentielle.
\end{proof}

\begin{lemma}[\cite{MonCerveau}]        \label{LEMooCYGTooEjXEUu}
    Le logarithme est une fonction continue.
\end{lemma}

\begin{proof}
    C'est une conséquence du théorème de la bijection \ref{ThoKBRooQKXThd}\ref{ItemEJZooKuFoeFiv}, et de la continuité de l'exponentielle sur \( \eR\), qui est une partie du théorème \ref{ThoKRYAooAcnTut}.
\end{proof}

\begin{proposition}[\cite{MonCerveau}]      \label{PROPooLAOWooEYvXmI}
    Pour tout \( x,y\in \eR\) et pour \( a>0\) nous avons
    \begin{equation}
        \ln(\frac{1}{ x })=-\ln(x),
    \end{equation}
    et
    \begin{equation}        \label{EQooJVMUooVpUKyo}
        \ln(xy)=\ln(x)+\ln(y),
    \end{equation}
    et
    \begin{equation}        \label{EQooEJQSooWCczXy}
        \ln(a^x)=x\ln(a)
    \end{equation}
    et
    \begin{equation}
        a^x= e^{x\ln(a)}.
    \end{equation}
\end{proposition}

\begin{proof}
    Nous avons, par la proposition \ref{PROPooVADRooLCLOzP},
    \begin{equation}
        e^{-\ln(x)}=\frac{1}{  e^{\ln(x)} }=\frac{1}{ x }.
    \end{equation}
    En prenant le logarithme des deux côtés nous trouvons
    \begin{equation}
        -\ln(x)=\ln\left( \frac{1}{ x } \right).
    \end{equation}

    Nous pouvons continuer avec la suivante.

    Par définition, \( \ln(xy)\) est donné par \( \exp\big( \ln(xy) \big)=xy\). Mais nous avons aussi, par la proposition \ref{PROPooVADRooLCLOzP} :
    \begin{equation}
        e^{\ln(x)+\ln(y)}=e^{\ln(x)}e^{\ln(y)}=xy.
    \end{equation}
    Nous avons donc démontré \eqref{EQooJVMUooVpUKyo}.

    La relation \eqref{EQooEJQSooWCczXy} de démontre d'abord pour \( x\in \eN\), puis pour \( x\in \eQ\) et enfin pour \( x\in\eR\). Si \( n\in \eN\) alors la relation \eqref{EQooJVMUooVpUKyo} donne immédiatement
    \begin{equation}
        \ln(a^n)=n\ln(a).
    \end{equation}
    pour tout \( x\in \eR\).

    Si \( m,n\in \eN\), le nombre \( a^{n/m}\) est par définition le \( x>0\) tel que
    \begin{equation}
        x^m=a^n.
    \end{equation}
    En prenant le logarithme des deux côtés : \( \ln(x^m)=\ln(a^n)\) et en utilisant la relation déjà démontrée pour \( \eN\) nous trouvons \( m\ln(x)=n\ln(a)\) et donc
    \begin{equation}
        \ln(a^{m/n})=\ln(x)=\frac{ m }{ n }\ln(a).
    \end{equation}
    La relation est donc démontré pour \( \ln(a^q)\) avec \( q\in \eQ^+\).

    Nous passons à \( q=-m/n\in \eQ^-\), c'est-à-dire toujours \( m,n\in \eN\). Nous avons, en utilisant la proposition \ref{PROPooLAOWooEYvXmI},
    \begin{equation}
        \ln(a^{-q})=\ln(\frac{1}{ a^q })=-\ln(a^q)=-q\ln(a).
    \end{equation}

    Enfin si \( x\in \eR\) nous considérons une suite de rationnels \( x_k\to x\). Pour chaque \( k\) nous avons
    \begin{equation}
        \ln(a^{x_k})=x_k\ln(a).
    \end{equation}
    Nous prenons la limite deux deux côtés. À droite nous avons tout de suite \( x\ln(a)\), et à gauche, par continuité de la fonction \( \ln\) (lemme \ref{LEMooCYGTooEjXEUu}) et de la fonction puissance (définition \ref{DEFooOJMKooJgcCtq}) nous trouvons \( \ln(a^x)\).
\end{proof}

La formule \eqref{EQooEJQSooWCczXy} en particulier est pratique pour réexprimer des fonctions puissances compliquées en écrivant
\begin{equation}        \label{EQooYEWCooKyravP}
    a^x= e^{\ln(a^x)}= e^{x\ln(a)}.
\end{equation}
Cela aide à calculer la dérivée de \( x\mapsto a^x\). 

Notons que certains prennent \eqref{EQooYEWCooKyravP} comme définition de la fonction puissance.

%--------------------------------------------------------------------------------------------------------------------------- 
\subsection{Approximations numériques de \texorpdfstring{$ e$}{e}}
%---------------------------------------------------------------------------------------------------------------------------

Nous donnons maintenant quelques approximations numériques de \( e\), particulièrement inefficaces.

\begin{lemma}
    Nous avons
    \begin{equation}
        2<e<3.
    \end{equation}
\end{lemma}

\begin{proof}
    Nous savons que \( y(0)=1\) et \( y'(0)=1\). La fonction \( y\) est strictement croissante (et donc sa dérivée aussi). Nous avons donc \( y'(x)>1\) pour tout \( x\in\mathopen] 0 , 1 \mathclose]\), et donc
    \begin{equation}
        y(1)>1+1\times 1=2.
    \end{equation}
    Sachant que \( 2>y'(x)\) pour tout \( x\in \mathopen] 0 , 1 \mathclose[\) nous pouvons refaire le coup de l'approximation affine, cette fois en majorant :
        \begin{equation}
            y(1)<1+2\times 1=3.
        \end{equation}
\end{proof}

De la même façon nous savons que
\begin{equation}
    y(\frac{1}{ n })>1+\frac{1}{ n }
\end{equation}
parce que \( y'\) est minoré par \( 1\) sur \( \mathopen] 0 , \frac{1}{ n } \mathclose[\). Avec cela nous avons aussi la majoration
\begin{equation}
    y(\frac{1}{ n })<1+\frac{1}{ n }\times \left( 1+\frac{1}{ n } \right)=1+\frac{1}{ n }+\frac{1}{ n^2 }.
\end{equation}
Et enfin nous pouvons donner l'encadrement, valable pour tout \( n\) :
\begin{equation}
    \left( 1+\frac{1}{ n } \right)^n<y(1)<\left( 1+\frac{1}{ n }+\frac{1}{ n^2 } \right)^n.
\end{equation}
Pour \( n=10\) nous trouvons
\begin{equation}
    2.50<e<2.83.
\end{equation}

Bien que ce soit à mon avis humainement pas possible à faire à la main nous avons, pour \( n=100\) :
\begin{equation}
    2.70<e<2.7317
\end{equation}
Cela reste un encadrement très modeste.

Une méthode plus efficace consiste à calculer directement le développement de définition
\begin{equation}
    e=\exp(1)=\sum_{k=0}^{\infty}\frac{1}{ n! }.
\end{equation}
\lstinputlisting{tex/sage/sageSnip013.sage}

\begin{probleme}
    Comment trouver, avec cette méthode, un \emph{encadrement pour \( e\) ?}
\end{probleme}

Ce petit programme, avec \( 5\) termes donne \( e\simeq 65/24\simeq 2.708\). Avouez que c'est déjà bien mieux.

%--------------------------------------------------------------------------------------------------------------------------- 
\subsection{Résumé des propriétés de l'exponentielle}
%---------------------------------------------------------------------------------------------------------------------------

\begin{theorem}  \label{ThoRWOZooYJOGgR}
    Les choses que nous savons sur l'exponentielle :
    \begin{enumerate}
        \item       \label{ITEMooEIKKooLNoaRD}
            Il y a unicité de la solution à l'équation différentielle
            \begin{subequations}    \label{subeqBKJNooJQtbBD}
        \begin{numcases}{}
            y'=y\\
            y(0)=1.
        \end{numcases}
    \end{subequations}
    \item
        L'équation différentielle \eqref{subeqBKJNooJQtbBD} possède une solution donnée par la série entière\nomenclature[Y]{\( \exp\)}{exponentielle}
        \begin{equation}    \label{EqUARSooKXnQxu}
        \exp(x)=\sum_{k=0}^{\infty}\frac{ x^k }{ k! }
    \end{equation}
\item
    Cette solution est une bijection \( y\colon \eR\to \mathopen] 0 , \infty \mathclose[\).
    \item   \label{ItemYTLTooSnfhOu}
        La fonction \( y\) ainsi définie est de classe \(  C^{\infty}\).
\item
    Elle est également donnée par la formule
    \begin{equation}
        \exp(x)=e^x
    \end{equation}
    où \( e\) est définit par \( e=\exp(1)\).
\item
    Elle vérifie
    \begin{equation}        \label{EQooVFXUooBfwjJY}
        e^{a+b}= e^{a} e^{b}
    \end{equation}
    \end{enumerate}
\end{theorem}
Nous nommons \defe{exponentielle}{exponentielle} cette fonction.

\begin{proof}
    Point par point.
    \begin{enumerate}
        \item
            C'est la proposition~\ref{PropDJQSooYIwwhy}.
        \item
            C'est le théorème~\ref{ThoKRYAooAcnTut}.
        \item
            Le rayon de convergence de la série \eqref{EqUARSooKXnQxu} est infini (théorème~\ref{ThoKRYAooAcnTut}); elle est donc définie sur \( \eR\). Le fait que ce soit une bijection est dû au fait qu'elle est strictement croissante (proposition~\ref{PropTLECooEiLbPP}) ainsi qu'aux limites \eqref{EqLOIUooHxnEDn}.
        \item
            Vu que \( y=y'\), \( y\) est dérivable. Mais comme \( y'\) est alors égale à une fonction dérivable, \( y'\) est dérivable. En dérivant l'égalité \( y'=y\) nous obtenons \( y''=y'\) et le jeu continue.
        \item
            C'est la proposition~\ref{PropCELWooLBSYmS}.
        \item
            C'est la proposition~\ref{PROPooGGUIooExVHPM}\ref{ITEMooMPSUooWQpVQJ}.
    \end{enumerate}
\end{proof}

\begin{example}[Un endomorphisme sans polynôme annulateur\cite{RombaldiO}]     \label{ExooLRHCooMYLQTU}
    l'exponentielle permet de donner un exemple d'un endomorphisme n'ayant pas de polynôme annulateur\footnote{Voir la définition~\ref{DefooOHUXooNkPWaB} et ce qui suit.} : l'endomorphisme de dérivation
    \begin{equation}
        \begin{aligned}
            D\colon C^{\infty}(\eR,\eR)&\to  C^{\infty}(\eR,\eR) \\
            f&\mapsto f'
        \end{aligned}
    \end{equation}
    n'a pas de polynôme annulateur. En effet supposons que \( P=\sum_{k=0}^{p}a_kX^k\) en soit un, et considérons les fonctions \( f_{\lambda}\colon t\mapsto  e^{\lambda t}\). Nous avons
    \begin{equation}
            0=P(D)f_{\lambda}
            =\sum_ka_kD^k(f_{\lambda})
            =\sum_ka_k\lambda^kf_{\lambda}
            =P(\lambda)f_{\lambda}.
    \end{equation}
    Par conséquent \( \lambda\) est une racine de \( P\) pour tout \( \lambda\in \eR\). Cela implique que \( P=0\).

    D'ailleurs si on y pense bien, cet exemple n'est qu'un habillage de l'exemple~\ref{ExooDTUJooIMqSKn}.
\end{example}

\begin{proposition}\label{ExZLMooMzYqfK}
    Quelques propriétés du logarithme.
    \begin{enumerate}
        \item
            Le logarithme est une application dérivable et strictement croissante.
        \item
            Le logarithme est la primitive de \( x\mapsto\frac{1}{ x }\) qui s'annule en \( x=1\).
    \end{enumerate}
\end{proposition}

\begin{proof}
    Elle est donc bijective, d'inverse continue et dérivable par le théorème~\ref{ThoKBRooQKXThd} et la proposition~\ref{PropMRBooXnnDLq}.

    La dérivée de la fonction logarithme peut être calculée en utilisant la formule \eqref{EqWWAooBRFNsv}, mais aussi de façon plus piettone en écrivant l'expression suivante, valable pour tout \( x\in \eR\) :
    \begin{equation}
        \ln\big( \exp(x) \big)=x,
    \end{equation}
    que nous pouvons dériver en utilisant le théorème de dérivation des fonctions composées :
    \begin{equation}
        \ln'\big( \exp(x) \big)\exp'(x)=1.
    \end{equation}
    Mais \( \exp'(x)=\exp(x)\), donc
    \begin{equation}
        \ln'(y)=\frac{1}{ y }
    \end{equation}
    pour tout \( y\) dans l'image de \( \exp\), c'est-à-dire pour tout \( y\) dans l'ensemble de définition de \( \ln\).

    Par ailleurs, \( \exp(0)=1\) donc
    \begin{equation}
        \ln(1)=\ln\big( \exp(0) \big)=0.
    \end{equation}

    En ce qui concerne l'unicité d'une primitive s'annulant en \( x=1\), c'est le corolaire~\ref{CorZeroCst}.
\end{proof}

%--------------------------------------------------------------------------------------------------------------------------- 
\subsection{Dérivée de la fonction puissance}
%---------------------------------------------------------------------------------------------------------------------------

\begin{example}     \label{EXooGMRIooUucRez}
    Soit la fonction \( f(x,y)=x^y\), définie en \ref{DEFooOJMKooJgcCtq}. Nous allons en calculer les dérivées partielles au point \( (1,2)\). Notons que \( f\) n'est pas définie pour \( x<0\), mais que cela n'a pas d'importance parce que nous pouvons nous restreindre à un voisinage du point \( (1,2)\). La première dérivée partielle est facile :
    \[
        \partial_x f(1,2)=(yx^{y-1})_{(x,y)=(1,2)}=2.
    \]
    Pour la seconde, il faut utiliser les propriétés de l'exponentielle et du logarithme. D'abord le logarithme est par définition l'application réciproque de l'exponentielle (définition \ref{DEFooELGOooGiZQjt}), donc 
    \begin{equation}
        x^y=\exp\big( \ln(x^y) \big).
    \end{equation}
    Ensuite nous calculons en utilisant la proposition \ref{PROPooLAOWooEYvXmI} :
    \[
        \partial_y f(1,2)=\partial_y\left(e^{y\ln x}\right)_{(x,y)=(1,2)}=\left(\ln x e^{y\ln x}\right)_{(x,y)=(1,2)}=\ln\big( 1- e^{2\ln(1)} \big)=0.
    \]
\end{example}


Cet exemple est facilement généralisable aux fonctions de la forme \( x\mapsto u(x)^{v(x)}\). Voici une proposition qui dit comment faire.
\begin{proposition}[\cite{MonCerveau}]     \label{PROPooKUULooKSEULJ}
    Soit une fonction dérivable \( u\colon \eR\to \eR\) et \( a>0\). Nous avons
    \begin{equation}
        \left( a^u\right)'=u'\ln(a)a^u.
    \end{equation}

    Si de plus \( u(x)>0\) pour tout \( x\), nous avons
    \begin{equation}
        \left( u^a \right)'=au'u^{a-1}.
    \end{equation}
\end{proposition}

\begin{proof}
    Nous considérons la fonction \( f(x)= a^{u(x)}\). Vu que \( f(x)>0\) pour tout \( x\), nous pouvons en prendre le logarithme et écrire l'égalité, valable pour tout \( x\) :
    \begin{equation}
        f(x)= e^{\ln(a^{u(x)})}=\exp\big( u(x)\ln(a) \big).
    \end{equation}
    Sachant la dérivée de l'exponentielle, cela n'est rien d'autre que la dérivée d'une fonction composée :
    \begin{equation}
        f'(x)=\ln(a) u'(x) e^{u(x)\ln(a)}.
    \end{equation}
    
    Pour l'autre, nous posons 
    \begin{equation}
        g(x)=u(x)^a,
    \end{equation}
    qui peut encore s'écrire sous la forme
    \begin{equation}
        g(x)= e^{a\ln\big( u(x) \big)}.
    \end{equation}
    Ici encore, c'est la dérivée de fonctions composées qui donne le résultat.
\end{proof}

%--------------------------------------------------------------------------------------------------------------------------- 
\subsection{Dérivée du logarithme}
%---------------------------------------------------------------------------------------------------------------------------

\begin{lemma}       \label{LEMooTGCBooJdkLpg}
Si \( u\colon \eR\to \mathopen] 0 , \infty \mathclose[\) est dérivable alors \( \ln(u)'=\dfrac{ u' }{ u }\).
\end{lemma}

\begin{proof}
    Cela est une conséquence du théorème de dérivation des fonctions composées : si \( g(x)=\ln(u(x))\) alors
    \begin{equation}
        g'(x)=\ln'\big( u(x) \big)u'(x)=\frac{1}{ u(x) }u'(x).
    \end{equation}
\end{proof}

%--------------------------------------------------------------------------------------------------------------------------- 
\subsection{Taylor pour l'exponentielle}
%---------------------------------------------------------------------------------------------------------------------------

\begin{proposition}[Développement de l'exponentielle]       \label{PROPooQBRGooAhGrvP}
    Pour tout entier \( n\), il existe une fonction \( \alpha\colon \eR\to \eR\) telle que \( \lim_{t\to 0} \alpha(t)=0\) et
    \begin{equation}
        e^x=\sum_{k=0}^n\frac{ x^k }{ k! }+x^n\alpha(x).
    \end{equation}
\end{proposition}

\begin{proof}
    Il s'agit de la proposition \ref{PROPooQLHNooRsBYbe} appliquée à la série entière \eqref{DEFooSFDUooMNsgZY}.
\end{proof}

%--------------------------------------------------------------------------------------------------------------------------- 
\subsection{Analycité}
%---------------------------------------------------------------------------------------------------------------------------

Vu que \( \exp(x)\) est défini par une série entière (définition \ref{DEFooSFDUooMNsgZY}) et vu la proposition \ref{PROPooTRWVooETTtbP}, il n'est pas étonnant que \( \exp\) soit analytique. Traitons ce cas.

\begin{example}[Analycité de l'exponentielle]
   Soient \( a\in \eR\) et \( R>0\). Nous démontrons que \( \exp\) est analytique sur \( B(a,R)\). Si \( f(x)= e^{x}\), alors \( f^{(n)}(x)= e^{x}\) pour tout \( n\) (équation \eqref{EQooSEIHooNmQKiC}). Nous avons donc
   \begin{equation}
       | f^{(n)}(x) |< e^{a+R}
   \end{equation}
   pour tout \( x\in B(a,R)\). Nous partons de l'expression \eqref{THOooEUVEooXZJTRL} du reste :
   \begin{equation}
       | R_n(x) |\leq \frac{ M_n }{ (n+1)! }| x-a |^{n+1}\leq \frac{  e^{a+R} }{ (n+1)! }R^{n+1}.
   \end{equation}
   Mais nous avons la limite 
   \begin{equation}
       \lim_{n\to \infty} \frac{ R^{n+1} }{ (n+1)! }=0
   \end{equation}
   pour tout \( R\). 

   Donc avec les polynômes de Taylor \( P_n\) calculés en \( a\), nous avons \( P_n\to \exp\) simplement sur \( \eR\).

   Nous pouvons donc développer la fonction exponentielle autour de n'importe quel point, et avoir convergence des polynômes vers l'exponentielle sur tout \( \eR\). Vous accepterez cependant que si \( a\) et \( x\) sont éloignés, la convergence \( P_n(x)\to \exp(x)\) peut être extrêmement lente.
\end{example}

%--------------------------------------------------------------------------------------------------------------------------- 
\subsection{Autres propriétés et petits calculs}
%---------------------------------------------------------------------------------------------------------------------------

\begin{lemma}   \label{LemPEYJooEZlueU}
Si \( a,b\in\mathopen] 0 , \infty \mathclose[\) alors
    \begin{equation}
        \ln(ab)=\ln(a)+\ln(b)
    \end{equation}
    et
    \begin{equation}    \label{EqOOZGooOWkGlA}
        \ln\left( \frac{1}{ b } \right)=-\ln(b).
    \end{equation}
\end{lemma}

\begin{proof}
    Nous posons \( f(x)=\ln(ax)\) qui est une fonction dérivable. Alors \( f'(x)=\frac{ a }{ ax }=\frac{1}{ x }\). Cette fonction \( f\) est donc une primitive de \( \frac{1}{ x }\) et il existe une constante \( K\) telle que
    \begin{equation}
        f(x)=\ln(x)+K.
    \end{equation}
    Vu que \( \ln(1)=0\) nous avons \( K=f(1)= \ln(a)\). Donc
    \begin{equation}
        \ln(ax)=\ln(x)+\ln(a).
    \end{equation}

    En ce qui concerne la seconde formule à démontrer, nous avons
    \begin{equation}
        \ln(1)=\ln\left( \frac{1}{ b }b \right)=\ln\left( \frac{1}{ b } \right)+\ln(b).
    \end{equation}
    Étant donné que $\ln(1)=0$ nous en déduisons la formule \eqref{EqOOZGooOWkGlA}.
\end{proof}

\begin{lemma}
    Si les suites \( (u_n)\) et \( (v_n)\) sont équivalentes\footnote{Définition \ref{DEFooEWRTooKgShmT}.} et si \( (v_n)\) admet une limite \( l\) différente de \( 1\), alors les suites \( (\ln u_n)\) et \( (\ln v_n)\) sont équivalentes.
\end{lemma}

\begin{proof}
    En effet si \( u_n=v_n\alpha(n)\) alors en utilisant la formule du lemme \ref{LemPEYJooEZlueU},
    \begin{equation}
        \ln(u_n)=\ln(v_n)+\ln\big( \alpha(n) \big)=\ln(v_n)\left( 1+\frac{ \ln\big( \alpha(n) \big) }{ \ln(v_n) } \right),
    \end{equation}
    et comme \( \alpha(n)\to 1\), la parenthèse tend vers \( 1\).
\end{proof}

%--------------------------------------------------------------------------------------------------------------------------- 
\subsection{Taylor pour le logarithme}
%---------------------------------------------------------------------------------------------------------------------------

Vu que \( \ln(0)\) n'existe pas, il n'est pas question de développer \( \ln\) autour de \( x=0\). À la place, nous allons le développer autour de \( x=1\) et plus précisément nous allons étudier Taylor pour la fonction \( f(x)=\ln(1+x)\). Les résultats seront résumés dans la proposition \ref{PROPooKPBIooJdNsqX}.

\begin{proposition}[\cite{MonCerveau}]     \label{PROPooWCUEooJudkCV}
    Soit la fonction
    \begin{equation}
        \begin{aligned}
        f\colon \mathopen] -1 , \infty \mathclose[&\to \eR \\
            x&\mapsto \ln(1+x). 
        \end{aligned}
    \end{equation}
    Pour tout \( n\), il existe une fonction \( \alpha\colon \eR\to \eR\) telle que \( \lim_{t\to 0} \alpha(t)=0\) et
    \begin{equation}
        f(x)=\sum_{k=1}^n\frac{ (-1)^{k+1} }{ k }x^k+\alpha(x)x^n
    \end{equation}
    pour tout \( x\) dans le domaine de \( f\).

    Notez la somme qui part de \( k=1\) et non \( k=0\).
\end{proposition}

\begin{proof}
    Nous utilisons la formule de Taylor-Young (proposition \ref{PropVDGooCexFwy}). La première dérivée de \( f\) se calcule en utilisant le lemme \ref{LEMooTGCBooJdkLpg} :
    \begin{equation}
        f'(x)=\frac{1}{ 1+x }.
    \end{equation}
    Pour les dérivées suivantes, c'est juste du calcul et nous pouvons prouver par récurrence que
    \begin{equation}        \label{EQooKEAOooGmTLJF}
        f^{(k)}(x)=\frac{ (k-1)!(-1)^{k+1} }{ (1+x)^k }.
    \end{equation}
    En ce qui concerne l'évaluation en zéro :
    \begin{equation}
        f^{(k)}(0)=\begin{cases}
            0    &   \text{si } k=0\\
            (k-1)!(-1)^{k+1}    &    \text{sinon.}
        \end{cases}
    \end{equation}
    Du fait que \( f^{(0)}(0)=\ln(1)=0\), la somme commence à \( k=1\) et non \( k=0\). Nous avons
    \begin{equation}
        f(x)=\sum_{k=1}^{n}\frac{ f^{(k)}(0) }{ k! }x^k+\alpha(x)x^n=\sum_{k=1}^n\frac{ (-1)^{k+1} }{ k }x^k+\alpha(x)x^n.
    \end{equation}
\end{proof}

Nous étudions les polynômes de la série de Taylor pour
\begin{equation}
    \begin{aligned}
    f\colon \mathopen] -1 , \infty \mathclose[&\to \eR \\
        x&\mapsto \ln(1+x). 
    \end{aligned}
\end{equation}

Les dérivées successives de \( f\) ont déjà été calculées en \eqref{EQooKEAOooGmTLJF}. Nous développons autour de \( x=0\). Donc \( f(0)=\ln(1)=0\) et pour les autres,
\begin{equation}
    f^{(k)}(0)=(-1)^{k+1}(k-1)!.
\end{equation}
Pour les polynômes de Taylor, nous avons
\begin{equation}
    P_n(x)=\sum_{k=1}^n\frac{ (-1)^{k+1} }{ k }x^k
\end{equation}
où vous noterez la somme qui part de \( k=1\) et non de \( k=0\). Nous avons aussi la série de Taylor de \( f\) donnée par
\begin{equation}        \label{EQooTAREooKfpTPo}
    T(x)=\sum_{k=1}^{\infty}\frac{ (-1)^{k+1} }{ k }x^k.
\end{equation}
La somme est une limite ponctuelle, là où elle existe.

Jusqu'à présent, la seule certitude à props de \( T\) est que \( T(0)=f(0)=0\). Pour le reste :
\begin{itemize}
    \item Rien ne dit que \( T(x)\) existe pour d'autres \( x\) que \( x=1\).
    \item Et même si \( T(x)\) existait pour d'autres \( x\) (c'est-à-dire si le rayon de convergence de \eqref{EQooTAREooKfpTPo} était strictement plus grand que zéro), rien n'assurerait que la valeur serait celle de \( f\).
    \item Et même si \( T(x)\) convergeait vers \( f\) sur son disque de convergence, ce ne serait pas encore assez pour dire que \( f\) est analytique, parce que l'analycité demande que les séries de Taylor autour de \emph{chaque} point converge vers \( f\). Or ici nous ne parlons encore que de \( T\) qui est la série autour de \( x=0\).
\end{itemize}

\begin{lemma}       \label{LEMooWMGGooRpAxBa}
    La série de Taylor de \( x\mapsto \ln(1+x)\) autour de \( x=0\) converge sur \( \mathopen] -1 , 1 \mathclose]\). Elle ne converge pas pour \( x=-1\).
\end{lemma}

\begin{proof}
        
    En ce qui concerne le rayon de convergence de \( T\), nous utilisons la formule de Hadamard\footnote{Théorème \ref{ThoSerPuissRap}.} avec
    \begin{equation}
        a_k=\frac{ (-1)^{k+1} }{ k }.
    \end{equation}
    Ce que nous trouvons est
    \begin{equation}
        \frac{1}{ R }=\lim_{k\to \infty} | \frac{ a_{k+1} }{ a_k } |=\lim_{k\to \infty} \frac{ k }{ k+1 }=1.
    \end{equation}
    Le rayon de convergence de \( T\) est donc \( 1\). Nous avons donc que \( P_n\to T\) sur \( \mathopen] -1 , 1 \mathclose[\), et peut-être que \( P_n\to T\) en \( x=\pm 1\).

    Pour \( x=-1\). L'intuition nous dit que ce serait \( \ln(0)\) qui n'est pas défini. C'est le cas parce que 
    \begin{equation}
        P_n(-1)=\sum_{k=1}^n\frac{ (-1)^{k+1}(-1)^k }{ k }=-\sum_{k=1}^n\frac{1}{ k }.
    \end{equation}
    La limite \( n\to \infty\) diverge. Donc \( T\) n'est pas définie en \( x=-1\).

    Pour \( x=1\) par contre,
    \begin{equation}
        P_n(1)=\sum_{k=1}^n\frac{ (-1)^{k+1} }{ k }.
    \end{equation}
    Le critère des séries alternées\footnote{Théorème \ref{THOooOHANooHYfkII}.} nous donne la convergence de cette série.
\end{proof}

Nous savons maintenant que la série de Taylor \( T\) converge sur \( \mathopen] -1 , 1 \mathclose]\), et que \( T(0)=f(0)=\ln(1)=0\). Le premier gros morceau intéressant vient maintenant : nous allons prouver que \( T(x)\) converge vers ce que nous croyons, c'est-à-dire \( \ln(1+x)\) en personne.

\begin{proposition}     \label{PROPooKPBIooJdNsqX}
Pour tout \( x\in\mathopen] -1 , 1 \mathclose]\) nous avons
    \begin{equation}        \label{EqweEZnV}
        \ln(1+x)=\sum_{k=1}^{\infty}\frac{ (-1)^{k+1} }{ k }x^k
    \end{equation}
    De plus nous avons
    \begin{equation}    \label{EqKUQmOZ}
        \sum_{k=1}^{\infty}\frac{ (-1)^{k+1} }{ k }=\ln(2).
    \end{equation}
\end{proposition}

\begin{proof}
Il s'agit d'utiliser l'expression du reste fourni par le théorème \ref{THOooSIGRooJTLvlV}. Pour tout \( x\in \mathopen] -1 , \infty \mathclose[\), il existe un \( c\in\mathopen] 0 , x \mathclose[\) (le \( c\) dépend de \( x\)) tel que
    \begin{equation}
        P_n(x)-f(x)=\frac{ f^{(n+1)}(c) }{ (n+1)! }x^{n+1}.
    \end{equation}
    Cela est parce que \( f\) est de classe \(  C^{\infty}\). Calculons un peu : 
    \begin{subequations}
        \begin{align}
            P_n(x)-f(x)&=\frac{ f^{(n+1)}(c) }{ (n+1)! }x^{n+1}\\
            &=\frac{ (-1)^nn! }{ (1+c)^{n+1} }\frac{1}{ (n+1)! }x^{n+1}\\
            &=\frac{ (-1)^n }{ n+1 }\left( \frac{ x }{ 1+c } \right)^{n+1}.
        \end{align}
    \end{subequations}
Lorsque \( x>1\), il n'y a aucune garantie sur la convergence de cela pour \( n\to \infty\). Pour rappel, \( c\in\mathopen] 0 , x \mathclose[\). Si par contre \( x\in\mathopen] -1 , 1 \mathclose[\), alors nous savons que
    \begin{equation}
        \left| \frac{ x }{ 1+c } \right| <1,
    \end{equation}
    et donc convergence \( P_n(x)-f(x)\to 0\).

    Jusqu'ici nous avons prouvé que pour la série de Taylor converge vers \( \ln(1+x)\) pour \( x\in \mathopen] -1 , 1 \mathclose[\). Nous avons également vu que la série converge pour \( x=1\). Donc la fonction 
        \begin{equation}
            g(x)=\sum_{k=1}^{\infty}\frac{ (-1)^{k+1} }{ k }x^k
        \end{equation}
    est de continue sur \( \mathopen] -1 , 1 \mathclose]\) et égale à \( \ln(x+1)\) sur \( \mathopen] -1 , 1 \mathclose[\). Vu que \( f\colon x\mapsto \ln(x+1)\) est continue sur \( \mathopen] -1 , \infty \mathclose[\), nous avons également \( g(1)=f(1)=\ln(2)\).

    Ceci nous mène au dernier point de notre proposition : \( g(1)=\ln(2)\) s'écrit précisément
    \begin{equation}
        \sum_{k=1}^{\infty}\frac{ (-1)^{k+1} }{ k }=\ln(2).
    \end{equation}
\end{proof}

\begin{normaltext}
    La formule \eqref{EqKUQmOZ} peut sembler très chouette pour trouver des approximations de \( \ln(2)\). Le problème est qu'elle ne donne aucune idée de l'erreur commise en tronquant la série.

    Vous pouvez, certre écrite
    \begin{equation}
        1-\frac{ 1 }{2}+\frac{1}{ 3 }-\frac{1}{ 4 }+\frac{1}{ 5 }=\frac{ 47 }{ 60 }\simeq 0.78.
    \end{equation}
    Ce calcul n'a aucune valeur pour affirmer que \( \ln(2)\) doit être proche de \( 0.78\). Ni même pour affirmer que \( \ln(2)<1\).

    Avoir des valeurs numériques de \( \ln(2)\) (c'est-à-dire que «chiffres corrects devant ou derrière la virgule») demande d'avoir un encadrement. Cela doit donc se faire avec des formules de séries avec reste; les formules exactes qui somment jusqu'à l'inifini sont inutiles pour avoir des approximations numériques.

    Dans le cas de \( \ln(2)\), une approximation numérique sera donnée à l'aide de Taylor avec reste intégrale dans la proposition \ref{PROPooHOMYooFclkCU}.
\end{normaltext}

\begin{lemma}
    Soit la fonction\footnote{Pour la définition du logarithme, c'est la définition~\ref{DEFooELGOooGiZQjt}.}
    \begin{equation}
        f(x)=\frac{ \ln(1+x) }{ x }
    \end{equation}
    \begin{enumerate}
        \item
        Elle admet un prolongement de classe \(  C^{\infty}\) sur \( \mathopen] -1 , \infty \mathclose[\).
        \item
            \( f(0)=1\).
    \end{enumerate}
    La seconde condition étant évidemment avec un abus de notation entre \( f\) et son prolongement, parce que \( f\) n'est pas définie en zéro.
\end{lemma}

\begin{proof}
    La difficulté étant de voir que \( f\) a un prolongement en zéro et qu'elle y est de classe \(  C^{\infty}\).

    La \ref{PROPooKPBIooJdNsqX} nous donne l'égalité
    \begin{equation}
        \ln(1+x)=\sum_{k=1}^{\infty}\frac{ (-1)^{k+1} }{ k }x^k
    \end{equation}
    pour tout \( x\in \mathopen] -1 , 0 \mathclose]\); en particulier pour \( x=0\). Nous faisons le petit calcul suivant :
    \begin{subequations}        \label{SUBEQooRLQOooEzNFDp}
        \begin{align}
            \frac{1}{ x }\ln(1+x)&= \frac{1}{ x }\sum_{n=1}^{\infty}\frac{ (-1)^{n+1} }{ n }x^n\\
            &=\sum_{n=1}^{\infty}\frac{ (-1)^{n+1} }{ n }x^{n-1}\\
            &=\sum_{n=0}^{\infty}\frac{ (-1)^k }{ k+1 }x^k.
        \end{align}
    \end{subequations}
    Ce calcul n'est pas valable pour \( x=0\), mais ça ne nous empèche pas de poser
    \begin{equation}
        T(x)=\sum_{n=0}^{\infty}\frac{ (-1)^k }{ k+1 }x^k,
    \end{equation}
    qui, lui, est bien définie en zéro. Le rayon de convergence de la série \( T\) est égal à \( 1\), de telle sorte que
    \begin{equation}
        T\colon \mathopen] -1 , 1 \mathclose[\to \eR \\
    \end{equation}
    de classe \(  C^{\infty}\), et est égale à \( f\) sur \( \mathopen] -1 , 1 \mathclose[\setminus\{ 0 \}\).

    La série \( T\) est donc le prolongement demandé. En ce qui concerne \( f(0)\), c'est un abus pour écrire \( T(0)\) qui vaut immédiatement \( 1\).
\end{proof}

Notons qu'un calcul de limite
\begin{equation}
    \lim_{x\to 0} \frac{ \ln(1+x) }{ x }
\end{equation}
donnait la valeur \( f(0)=1\). Donc prolonger avec \( f(0)=1\) était la seule possibilité pour avoir une fonction continue. De là à dire que le prolongement ainsi créé est de classe \(  C^{\infty}\), c'est une autre histoire, qui est résoue par les séries entières.

%--------------------------------------------------------------------------------------------------------------------------- 
\subsection{Développements et calcul de limites}
%---------------------------------------------------------------------------------------------------------------------------

Lors d'un calcul de limite, développer une partie d'une expression peut être utile.

\begin{example}
    À calculer :
    \begin{equation}
        \lim_{x\to 0} \frac{ \ln(1+x) }{ x }.
    \end{equation}
    Cela est une indétermination de type \( \frac{ 0 }{ 0 }\). Le développement limité du numérateur\footnote{Proposition \ref{PROPooWCUEooJudkCV}.} nous donne une fonction \( \alpha(x)\) telle que \( \lim_{x\to 0} \alpha(x)=0\) et
    \begin{equation}
        \frac{ \ln(1+x) }{ x }=\frac{ x-\frac{ x^2 }{2}+x^2\alpha(x) }{ x }=1-\frac{ x }{ 2 }+x\alpha(x).
    \end{equation}
    Sur le membre de droite la limite est facile à calculer :
    \begin{equation}
        \lim_{x\to 0} \frac{ \ln(1+x) }{ x }=\lim_{x\to 0} \Big( 1-\frac{ x }{ 2 }+x\alpha(x) \Big) =1.
    \end{equation}
\end{example}

%+++++++++++++++++++++++++++++++++++++++++++++++++++++++++++++++++++++++++++++++++++++++++++++++++++++++++++++++++++++++++++
\section{Vitesses des puissances, de l'exponentielle et du logarithme}
%+++++++++++++++++++++++++++++++++++++++++++++++++++++++++++++++++++++++++++++++++++++++++++++++++++++++++++++++++++++++++++

%---------------------------------------------------------------------------------------------------------------------------
\subsection{Un peu de théorie}
%---------------------------------------------------------------------------------------------------------------------------

Voici une série de résultats qui lient les vitesses des polynômes, du logarithme et de l'exponentielle.

\begin{proposition}     \label{PROPooKVIFooGdKpfP}
    Nous avons :
    \begin{enumerate}
        \item   \label{ITEMooCDSQooSIctbz}
            Pour tout \( \alpha>0\), il existe \( N\) tel que \( \ln(n)\leq n^{\alpha}\) pour tout \( n\geq N\).
        \item       \label{ITEMooZMAWooTbDNAd}
            Pour tout \( p>0\) et tout \( \alpha>0\), il existe \( N\) tel que 
            \begin{equation}
                \ln(n)^p<n^{\alpha}
            \end{equation}
            pour tout \( n\geq N\).
        \item       \label{ITEMooBLNOooZQNTfd}
            Pour tout \( n\geq 1\) nous avons la limite
            \begin{equation}
                \lim_{x\to 0^+} x^n\ln(x)=0.
            \end{equation}
        \item       \label{ITEMooMLNMooAyJTox}
            Nous avons
            \begin{equation}
                \lim_{x\to 0^+} \frac{ \ln(1-x) }{ x }=-1.
            \end{equation}
        \item       \label{ITEMooIQEKooBionsK}
            L'exponentielle croit plus vite que tout polynôme, et plus vite que que logarithme :
    \begin{equation}        \label{EqExpDecrtPlusVite}
        \lim_{t\to\infty} e^{-t}(\ln t)^{n}t^{\alpha}=0
    \end{equation}
    pour tout $n$ et pour tout $\alpha$.
\item       \label{ITEMooDUQWooNvAvmR}
    Pour tout \( n>0\), nous avons la limite
    \begin{equation}
        \lim_{x\to 0^+} x^n e^{1/x}=\infty.
    \end{equation}
    \end{enumerate}
\end{proposition}

Le point \ref{ITEMooCDSQooSIctbz} et sa généralisation \ref{ITEMooZMAWooTbDNAd} nous font dire que le logarithme croît moins vite que n'importe quel polynôme.

\begin{proof}
    En plusieurs parties.
    \begin{subproof}
        \item[Pour \ref{ITEMooCDSQooSIctbz}]
            En effet, nous avons, par la règle de l'Hospital (proposition~\ref{PROPooBZHTooHmyGsy}),
            \begin{equation}
                \lim_{x\to\infty} \frac{ x^{\alpha} }{ \ln(x) }=\lim_{x\to\infty} \frac{ \alpha x^{\alpha-1} }{ 1/x }=\lim_{x\to\infty} \alpha x^{\alpha}=\infty
            \end{equation}
            quand $\alpha>0$. La dérivée du logarithme est dans la proposition \ref{ExZLMooMzYqfK}.
        \item[Pour \ref{ITEMooZMAWooTbDNAd}]
        \item[Pour \ref{ITEMooBLNOooZQNTfd}]
            Lorsque \( x\neq 0\) nous avons
            \begin{equation}
                x^n\ln(x)=\frac{ \ln(x) }{ 1/x^n },
            \end{equation}
            qui est un cas \( \frac{ \infty }{ \infty }\). Nous nous en remettons à la règle de l'Hospital \ref{PROPooTJVCooMeUhIy}. D'abord nous nous assurons de la limite des dérivées :
            \begin{equation}
                \lim_{x\to 0^+} \frac{ 1/x }{ -nx^{-n-1} }=\lim_{x\to 0^+} -\frac{1}{ n }\frac{ x^{n+1} }{ x }=0.
            \end{equation}
            La règle de l'Hospital conclu à l'existence de la limite demandée et à son égalité à \( 0\).
        \item[Pour \ref{ITEMooMLNMooAyJTox}]
            En effet, par la règle de l'Hospital \ref{PROPooBZHTooHmyGsy},
            \begin{equation}    \label{EqGICpOX}
                \lim_{x\to 0} -\frac{ \ln(1-x) }{ x }=\lim_{x\to 0} -\frac{ \frac{ -1 }{ 1-x } }{ 1 }=\lim_{x\to 0} \frac{1}{ 1-x }=1
            \end{equation}
        \item[Pour \ref{ITEMooIQEKooBionsK}]
        \item[Pour \ref{ITEMooDUQWooNvAvmR}]
            Nous passons au logarithme :
            \begin{equation}
                \ln(x^n e^{1/x})=\ln(x^n)+\ln( e^{1/x})=n\ln(x)+\frac{1}{ x }=\frac{ n x\ln(x)+1 }{ x }.
            \end{equation}
            Grâce à la limite déjà prouvée en \ref{ITEMooBLNOooZQNTfd}, le numérateur tend vers \( 1\) lorsque \( x\to 0^+\). Donc le tout tend vers \( +\infty\). Au final,
            \begin{equation}
                \lim_{x\to 0^+} x^n e^{1/x}=\lim_{x\to 0^+}  e^{\ln(x^n e^{1/x})}=\infty.
            \end{equation}
    \end{subproof}
\end{proof}

\begin{lemma}       \label{LEMooNYFVooXjFShk}
    Si \( P\) est un polynôme et si \( a>0\), alors
    \begin{equation}
        \lim_{x\to \infty}  e^{-ax}P(x)=0
    \end{equation}
\end{lemma}

\begin{proof}
    Nous prouvons par récurrence que pour tout \( n\), nous avons \(  e^{-ax}x^n\to 0\). D'abord nous écrivons\footnote{En utilisant \ref{PROPooVADRooLCLOzP}\ref{ITEMooSCJBooNVJZah}.}
    \begin{equation}
        f(x)= e^{-ax}x=\frac{ x }{  e^{ax} },
    \end{equation}
    et ensuite la règle de l'Hospital \ref{PROPooTJVCooMeUhIy} nous donne
    \begin{equation}
        \lim_{x\to \infty} \frac{ x }{  e^{ax} }=\lim_{x\to \infty} \frac{1}{ a e^{ax} }=0.
    \end{equation}

    En ce qui concerne la récurrence, c'est encore la règle de l'Hospital :
    \begin{equation}
        \lim_{x\to \infty} \frac{ x^n }{  e^{ax} }=\frac{ n }{ a }\lim_{x\to \infty} \frac{ x^{n-1} }{  e^{ax} }=0.
    \end{equation}
\end{proof}

\begin{example}     \label{EXooQNCJooFpnvnf}
   Le lemme \ref{LemLJOSooEiNtTs} a déjà prouvé la limite 
    \begin{equation}
        \lim_{n\to \infty} n^{\alpha}a^n
    \end{equation}
    pour tout \( \alpha>0\) et \( a<1\).

    L'utilisation de propriétés de l'exponentielle nous permet de donner une nouvelle preuve, plus courte\footnote{C'est toujours facile de prétendre qu'une preuve est plus courte qu'une autre lorsqu'on utilise en une ligne des très gros théorèmes qui ont mis dix pages à être démontrés.}.

    Le théorème \ref{ThoRWOZooYJOGgR} et la proposition \ref{PROPooLAOWooEYvXmI} nous permettent de passer à l'exponentielle. Pour chaque \( n\) nous avons :
    \begin{equation}        \label{EqLKLQooLIlWgm}
        n^{\alpha}a^n= e^{\alpha\ln(n)+n\ln(a)}.
    \end{equation}
    Ce qui est dans l'exponentielle est
    \begin{equation}
        \alpha\ln(n)+n\ln(a)=n\big(\alpha \frac{ \ln(n) }{ n }+\ln(a) \big).
    \end{equation}
    Dans la parenthèse, \( \ln(a)<0\) et \( \frac{ \ln(n) }{ n }\to 0\). Donc ce qui est dans l'exponentielle \eqref{EqLKLQooLIlWgm} tend vers \( -\infty\) et au final l'expression demandée tend vers zéro.
\end{example}

\begin{remark}
    Vous ne pouvez pas à priori considérer l'exemple \ref{EXooQNCJooFpnvnf} comme une preuve alternative au lemme \ref{LemLJOSooEiNtTs}, parce que vous n'êtes pas sûr que dans toute la théorie permettant de définir l'exponentielle (en particulier la convergence de \( \sum_kx^k/k!\)), le lemme n'est pas utilisé\quext{Faites la vérification et dites moi si c'est bon.}.
\end{remark}

\begin{proposition} \label{PropBQGBooHxNrrf}
    Pour tout polynôme \( P\) et pour tout \( a>0\) la fonction \( f(x)=P(x) e^{-ax}\) est intégrable\footnote{Définition~\ref{DefTCXooAstMYl}.} sur \( \mathopen[ 0 , \infty [\).
\end{proposition}

\begin{proof}
    Nous avons \( f(x)=P(x) e^{-ax/2} e^{-ax/2}\), et par la vitesse comparée des exponentielles et polynômes, pour un certain \( M>0\) nous pouvons affirmer que \( P(x) e^{-ax/2}<1\) sur \( \mathopen[ M , 0 [\). Dès lors
        \begin{equation}
            | f(x) |< e^{-ax/2},
        \end{equation}
        qui est intégrable.
\end{proof}

\begin{example}     \label{EXooAGEOooQdQkrS}
    La fonction logarithme (définition~\ref{DEFooELGOooGiZQjt}) n'est pas définie pour \( x\leq 0\). Par conséquent la fonction \( f(x)=x\ln(|x|)\) n'est pas définie en \( x=0\). Elle est bien définie pour \( x<0\) et vérifie
    \begin{equation}
        \lim_{x\to 0} x\ln(|x|)=0.
    \end{equation}
    Nous pouvons donc définir la fonction
    \begin{equation}
        \begin{aligned}
            \tilde f\colon \eR&\to \eR \\
            x&\mapsto \begin{cases}
                x\ln(| x |)    &   \text{si } x\neq 0\\
                0    &    \text{si } x=0.
            \end{cases}
        \end{aligned}
    \end{equation}
    Contrairement à la fonction initiale \( f\), cette fonction \( \tilde f\) est définie et continue en \( 0\).

    Notez que sur le graphe de la fonction \( \tilde f\), la courbe est bien régulière en \( x=0\).
    \begin{center}
       \input{auto/pictures_tex/Fig_XJMooCQTlNL.pstricks}
    \end{center}
\end{example}


\begin{example}     \label{EXooGESBooQYOCpk}
    
Prenons deux suites $\{a_n\}$ et  $\{b_n\}$ qui tendent toutes les deux vers l'infini (resp. 0). On dira que la suite $\{a_n\}$ converge plus vite (resp. plus lentement) que la suite $\{b_n\}$ si $\lim_{n\rightarrow \infty}\f{a_n}{b_n} = \infty$, aussi vite si $\lim_{n\rightarrow \infty}\f{a_n}{b_n} $ existe et est finie, et plus lentement (resp. plus vite)  si $\lim_{n\rightarrow \infty}\f{a_n}{b_n} = 0$.
\begin{enumerate}
	\item Montrer qu'il existe deux suites qui tendent vers $\infty$ (ou 0) mais qui n'ont pas la même  vitesse d'approche.
	\item Montrer que pour toute suite qui tend vers  $\infty$ (ou 0), il existe une suite qui tend vers  $\infty$ (ou 0) plus vite.
	\item Donner une suite non exponentielle qui tend vers l'infini plus vite que la suite $x_k = e^k$.
\end{enumerate}

Voici quelque éléments de réponse.

\begin{enumerate}
\item $x_n=n$ et $y_n=n^2$, et les inverses pour des suites qui tendent vers zéro.
\item Si $x_n\to\infty$, la suite $x_n^2$ tend plus vite.
\item La suite $x_n=n!$ tend vers \( \infty\) vite que l'exponentielle. En effet, le nombre $e^k$ n'est rien d'autre que le produit itéré $e\cdot e\cdot\ldots\cdot e$. Comparez
\begin{equation}
	e\cdot e\cdot e\cdot\ldots\cdot e
\end{equation}
avec
\begin{equation}
	1\cdot 2\cdot 3\cdot\ldots\cdot 10.
\end{equation}
Étant donné que $e<3$, nous avons
\begin{equation}
	\frac{ e^k }{ k! }<\frac{ e^2 }{ 2 }\cdot\left( \frac{ e }{ 3 } \right)^{k-2}\to 0.
\end{equation}

\end{enumerate}
\end{example}
<++>

%---------------------------------------------------------------------------------------------------------------------------
\subsection{Nombres premiers}
%---------------------------------------------------------------------------------------------------------------------------

\begin{theorem} \label{ThonfVruT}
    Soit \( P\), l'ensemble des nombres premiers. Alors la somme \( \sum_{p\in P}\frac{1}{ p }\) diverge et plus précisément,
    \begin{equation}
        \sum_{\substack{p\leq x\\p\in P}}\frac{1}{ p }\geq \ln(\ln(x))-\ln(2).
    \end{equation}
\end{theorem}
\index{nombre!premier}
\index{convergence!rapidité}
\index{série!numérique}

\begin{proof}
    Nous posons
    \begin{equation}
        S_x=\{  q\leq x\text{ avec } q\text{ sans facteurs carrés} \}
    \end{equation}
    et
    \begin{equation}
        P_x=\{ p\in P\tq p\leq x \}.
    \end{equation}
    Si
    \begin{equation}
        K_x=\{  (q,m)\text{ tels que } q\text{ n'a pas de facteurs carrés et } qm^2\leq x \},
    \end{equation}
    alors nous avons
    \begin{equation}
        K_x=\bigcup_{q\in S_x}\bigcup_{m\leq \sqrt{x/q}}(q,m).
    \end{equation}
    Par définition et par le lemme~\ref{LemheKdsa} nous avons aussi
    \begin{equation}
        \{ n\leq x \}=\{ qm^2\tq (q,m)\in K_x \}.
    \end{equation}
    Tout cela pour décomposer la somme
    \begin{equation}        \label{EqpoJpuC}
        \sum_{n\leq x}\frac{1}{ n }=\sum_{q\in S_x}\sum_{m\leq\sqrt{x/q}}\frac{1}{ m^2 }\leq \sum_{q\in S_x}\frac{1}{ q }\underbrace{\sum_{m\geq 1}\frac{1}{ m^2 }}_{=C}.
    \end{equation}
    Nous avons aussi
    \begin{subequations}
        \begin{align}
            \prod_{p\in P_x}\left( 1+\frac{1}{ p } \right)&=1+\sum_{p\in P_x}\frac{1}{ p }+\sum_{\substack{p,q\in P_x\\p<q}}\frac{1}{ pq }+\sum_{\substack{p,q,r\in P_x\\p<q<r}}\frac{1}{ pqr }+\ldots\\
            &\geq 1+\sum_{p\in P_x}\frac{1}{ p }+\sum_{\substack{p,q\in P_x\\pq\leq x}}\frac{1}{ pq }+\sum_{\substack{p,q,r\in P_x\\pqr\leq x}}\frac{1}{ pqr }+\ldots
        \end{align}
    \end{subequations}
    Les sommes sont finies. Les sommes s'étendent sur toutes les façons de prendre des produits de nombres premiers distincts de telle sorte de conserver un produit plus petit que \( x\); c'est-à-dire que les sommes se résument en une somme sur les éléments de \( S_x\) :
    \begin{equation}        \label{EqooilOz}
        \exp\left( \sum_{p\in P_x}\frac{1}{ p } \right)\geq\prod_{p\in P_x}\left( 1+\frac{1}{ p } \right)\geq \sum_{q\in S_x}\frac{1}{ q }.
    \end{equation}
    La première inégalité est simplement le fait que \( 1+u\leq e^u\) si \( u\geq 0\) (directe de la définition~\ref{ThoRWOZooYJOGgR}). Les inégalités suivantes proviennent du fait que le logarithme est une primitive de la fonction inverse (proposition~\ref{ExZLMooMzYqfK}) :
    \begin{equation}
        \ln(x)\leq \sum_{n\geq x}\int_{n}^{n+1}\frac{dt}{ t }\leq \sum_{n\geq x}\frac{1}{ n }.
    \end{equation}
    Nous prolongeons ces inégalités avec les inégalités \eqref{EqpoJpuC} et \eqref{EqooilOz} :
    \begin{equation}
        \ln(x)\leq \sum_{n\geq x}\frac{1}{ n }\leq C\sum_{q\in S_x}\frac{1}{ q }\leq C\leq \exp\left( \sum_{p\in P_x}\frac{1}{ p } \right).
    \end{equation}
    En passant au logarithme,
    \begin{equation}
        \ln\big( \ln(x) \big)\leq\ln(C)+\sum_{p\in P_x}\frac{1}{ p }.
    \end{equation}
    Ceci montre la divergence de la série de droite. Nous cherchons maintenant une borne pour \( C\). Pour cela nous écrivons
    \begin{subequations}
        \begin{align}
            \sum_{n=1}^N\frac{1}{ n^2 }&\leq 1+\sum_{n=2}\frac{1}{ n(n-1) }\\
            &=1+\sum_{n=2}^N\left( \frac{1}{ n-1 }-\frac{1}{ n } \right)\\
            &=1+1-\frac{1}{ N }\\
            &\leq 2.
        \end{align}
    \end{subequations}
    Donc \( C\leq 2\).
\end{proof}
Ce théorème prend une nouvelle force en considérant le théorème de Müntz~\ref{ThoAEYDdHp} qui dit qu'alors l'ensemble \( \Span\{ x^p\tq  p\text{ est premier} \}\) est dense dans les fonctions continues sur \( \mathopen[ 0 , 1 \mathclose]\) muni de la norme uniforme ou \( \| . \|_2\).

%---------------------------------------------------------------------------------------------------------------------------
\subsection{Quelques limites}
%---------------------------------------------------------------------------------------------------------------------------

Nous voyons à présent quelques calculs de limite et de développements mettant en scène des logarithmes et exponentielles.

\begin{example}\label{compose1}
    Pour trouver le développement de la fonction \( f(x)= e^{-2x}\), il suffit d'écrire celui de \( e^t\) et de remplacer ensuite $t$ par \( -2x\). Le développement à l'ordre \( 3\) de la fonction exponentielle est :
    \begin{equation}
        e^t=1+t+\frac{ t^2 }{2}+\frac{ t^3 }{ 6 }+t^3\alpha(t).
    \end{equation}
    Le développement de \( f(x)= e^{-2x}\) sera donc
    \begin{equation}
        f(x)=1-2x+\frac{ 4x^2 }{ 2 }-\frac{ 8x^3 }{ 6 }-8x^3\alpha(-2x).
    \end{equation}
    Donc le polynôme de degré \( 3\) partie régulière de \( g\) est :
    \begin{equation}
        1-2x+2x^2-\frac{ 4 }{ 3 }x^3,
    \end{equation}
    et la fonction reste correspondante est :
    \begin{equation}
        \alpha_g(x)=-8\alpha(-2x).
    \end{equation}
\end{example}

\begin{example}
    Nous savons les développements
    \begin{equation}
        f(x)=\ln(1+x)\sim x-\frac{ x^2 }{ 2 }+\frac{ x^3 }{ 3 }
    \end{equation}
    et
    \begin{equation}
        \sin(x)\sim x-\frac{ x^3 }{ 6 }.
    \end{equation}
    Nous obtenons le développement d'ordre \( 3\) de la fonction \( x\mapsto \ln\big( 1+\sin(x) \big)\) en écrivant
    \begin{equation}    \label{EqGXMooWKQkIL}
        \ln\big( 1+\sin(x) \big)\sim \big( x-\frac{ x^3 }{ 6 } \big)-\frac{ 1 }{2}\left( x-\frac{ x^3 }{ 6 } \right)^2+\frac{1}{ 3 }\left( x-\frac{ x^3 }{ 6 } \right)^3.
    \end{equation}
    Il s'agit maintenant de trouver les termes qui sont de degré inférieur ou égale à \( 3\).

    D'abord
    \begin{equation}
        \left( x-\frac{ x^3 }{ 6 } \right)^2=x^2-\frac{ x^4 }{ 3 }+\frac{ x^6 }{ 36 }\sim x^2
    \end{equation}
    Nous avons alors aussi
    \begin{equation}
        \left( x-\frac{ x^3 }{ 6 } \right)^6\sim x^2\left( x-\frac{ x^3 }{ 6 } \right)\sim x^3.
    \end{equation}
    En replaçant tout ça dans \eqref{EqGXMooWKQkIL} nous trouvons
    \begin{equation}
        \ln\big( 1+\sin(x) \big)\sim x-\frac{ x^2 }{2}+\frac{ x^3 }{ 6 }.
    \end{equation}
\end{example}

\begin{example}	\label{ExBCDookjljhjk}
    Calculer
    \begin{equation}\label{EqABCoolkjh}
        \lim_{x\to \infty}  e^{1/x}\sqrt{1+4x^2}-2x.
    \end{equation}
    Nous allons effectuer un développement asymptotique de la partie «difficile» de l'expression posant d'abord $x=1/h$. Si $f(x)=e^{1/x}\sqrt{1-4x^2}$ alors
    \begin{equation}
	g(h)=\frac{1}{|h|}e^h\sqrt{h^2+4}=\frac{1}{h}\big(  1+h+h\alpha(h) \big)\big( 2+h\beta(h) \big).
    \end{equation}
    La première parenthèse est le développement de $e^h$ et la seconde celui de $\sqrt{h^2+4}$. Nous nous apprêtons à faire la limite $x\to\infty$ qui correspond à $h\to 0^+$, nous pouvons donc supposer que $h>0$ et omettre la valeur absolue. En effectuant le produit et en regroupant tous les termes contenant $h^2$, $\alpha(h)$ ou $\beta(h)$ dans un seul terme $h\gamma(h)$,
    \begin{equation}
	f(h)=\frac{1}{h}\big(  2+2h+h\gamma(h) \big)=\frac{2}{h}+2+\gamma(h)=2x+2+\gamma(1/x)
    \end{equation}
    où $\gamma$ est une fonction vérifiant $\lim_{t\to 0}\gamma(t)=0$.

    Nous sommes maintenant en mesure de calculer la limite \eqref{EqABCoolkjh} :
    \begin{equation}
	\lim_{x\to\infty}e^{1/x}\sqrt{1+x^2}-2x= \lim_{x\to \infty}\big(  2x+2+\gamma(1/x)-2x \big)=2.
    \end{equation}
\end{example}

%+++++++++++++++++++++++++++++++++++++++++++++++++++++++++++++++++++++++++++++++++++++++++++++++++++++++++++++++++++++++++++
\section{Trigonométrie hyperbolique}
%+++++++++++++++++++++++++++++++++++++++++++++++++++++++++++++++++++++++++++++++++++++++++++++++++++++++++++++++++++++++++++

\begin{definition}
    Les fonctions \defe{cosinus hyperbolique}{cosinus!hyperbolique} et \defe{sinus hyperbolique}{sinus!hyperbolique} sont les fonctions définies sur $\eR$ par les formules suivantes :
    \begin{subequations}
        \begin{align}
            \cosh(x)&=\frac{  e^{x}+ e^{-x} }{2}\\
            \sinh(x)&=\frac{  e^{x}- e^{-x} }{2}.
        \end{align}
    \end{subequations}
    Si vous ne vous rappelez plus la définition de \( e^x\), c'est \ref{DEFooSFDUooMNsgZY}.
\end{definition}

\begin{proposition}     \label{PROPooUNHHooIksdoJ}
    Quelques propriétés algébriques des fonctions trigonométriques hyperboliques.
    \begin{enumerate}
        \item
            \( \cosh(-x)=\cosh(x)\)
        \item
            \( \sinh(-x)=-\sinh(x)\)
        \item
            \( \cosh^2(x)-\sinh^2(x)=1\)
        \item
            \( \cosh(x)\cosh(y)+\sinh(x)\sinh(y)=\cosh(x+y)\)
        \item       \label{ITEMooOJRFooUCUaDl}
            \( \cosh(x)\cosh(y)-\sinh(x)\sinh(y)=\cosh(x-y)\)
        \item
            \( \cosh(x)\sinh(y)+\sinh(x)\cosh(y)=\sinh(x+y)\)
        \item
            \( \cosh(x)\sinh(y)-\sinh(x)\cosh(y)=-\sinh(x-y)\).
    \end{enumerate}
\end{proposition}

\begin{proof}
    Si s'agit simplement de remplacer les définitions et d'utiliser les formules concernant les puissances, dont la formule \eqref{EQooEWIHooDRAQGR}.
\end{proof}

\begin{proposition}     \label{PROPooAOOHooXvLfrZ}
    Quelques propriétés analytiques des fonctions trigonométriques hyperboliques.
    \begin{enumerate}
        \item
            \( \cosh'(x)=\sinh(x)\)
        \item
            \( \sinh'(x)=\cosh(x)\).
        \item       \label{ITEMooZNZLooNMQFWr}
            \( \cosh(x)\geq 1\).
    \end{enumerate}
\end{proposition}

\begin{proof}
    Pour les dérivées, il s'agit d'utiliser la dérivation de l'exponentielle, laquelle est facile par le théorème \ref{ThoRWOZooYJOGgR}\ref{ITEMooEIKKooLNoaRD}.

    Pour \ref{ITEMooZNZLooNMQFWr}, nous commençons par les \( x\geq 0\). D'abord $\cosh(0)=1$. Ensuite \( \cosh'(x)=\sinh(x)=\frac{  e^{x}- e^{-x} }{ 2 }\). Vu que \( x> 0\) nous avons \(  e^{x}> e^{-x}>0\). Donc la dérivée de \( \cosh\) est strictement positive sur \( \mathopen] 0 , \infty \mathclose[\). La fonction y est donc partout plus grande que \( \cosh(0)=1\).

    Pour les \( x<0\), nous avons la fait que \( \cosh\) est paire.
\end{proof}

\begin{proposition}     \label{PROPooQLNYooIIOdvm}
    La fonction \( \sinh\colon \eR\to \eR\) est bijective.
\end{proposition}

\begin{proof}
    En deux parties.
    \begin{subproof}
        \item[Injective]
            Si \( \sinh(a)=\sinh(b)\), alors le théorème de Rolle \ref{ThoRolle} affirme qu'il existe \( c\in \mathopen] a , b \mathclose[\) tel que \( \sinh'(c)=0\). Mais la proposition \ref{PROPooAOOHooXvLfrZ} nous dit que \( \sinh'(x)=\cosh(c)\geq 1\). Donc impossible.

            \item[Surjective]
                Nous avons
                \begin{equation}
                    \lim_{x\to -\infty} \sinh(x)=-\infty
                \end{equation}
                et
                \begin{equation}
                    \lim_{x\to\infty } \sinh(x)=\infty.
                \end{equation}
                Soit \( y\in \eR\). Il existe \( m<0\) tel que \( \sinh(m)<y\) et \( M>0\) tel que \( \sinh(M)>y\). Le théorème des valeurs intermédiaires \ref{ThoValInter} nous enseigne qu'il existe \( x\in \mathopen[ m , M \mathclose]\) tel que \( \sinh(x)=y\).
    \end{subproof}
\end{proof}

\begin{proposition}[\cite{MonCerveau}]      \label{PROPooWEHGooOBqSHY}
    Soient \( a,b\in \eR\) tels que \( a-b^2=1\). Il existe un unique \( (x,\sigma)\in \eR\times \{ \pm1 \}\) tel que
    \begin{subequations}        \label{SUBEQSooBIYDooIBuduV}
        \begin{numcases}{}
            a=\sigma\cosh(x)\\
            b=\sinh(x).
        \end{numcases}
    \end{subequations}
\end{proposition}

\begin{proof}
    Vu que le sinus hyperbolique est une bijection\footnote{Proposition \ref{PROPooQLNYooIIOdvm}.}, il existe un unique \( x\in \eR\) tel que \( \sinh(x)=b\). Maintenant un petit calcul :
    \begin{equation}
        a^2=1+\sinh(x)^2=1+\frac{  e^{2x}+ e^{-2x}-2 }{ 4 }=\frac{  e^{2x}+ e^{-2x}+2 }{ 4 }=\cosh(x)^2.
    \end{equation}
    Vu que \( \cosh(x)^2=a^2\), il existe un unique \( \sigma\in\{ \pm1 \}\) tel que \( \sigma\cosh(x)=a\).
\end{proof}

Les représentations graphiques sont ceci :
\begin{center}
   \input{auto/pictures_tex/Fig_UNVooMsXxHa.pstricks}
\end{center}

La \defe{tangente hyperbolique}{tangente hyperbolique} est donnée par le quotient
\begin{equation}
    \tanh(x)=\frac{ \sinh(x) }{ \cosh(x) }.
\end{equation}

%+++++++++++++++++++++++++++++++++++++++++++++++++++++++++++++++++++++++++++++++++++++++++++++++++++++++++++++++++++++++++++
\section{Séries entières de matrices}
%+++++++++++++++++++++++++++++++++++++++++++++++++++++++++++++++++++++++++++++++++++++++++++++++++++++++++++++++++++++++++++

%---------------------------------------------------------------------------------------------------------------------------
\subsection{Différentiabilité}
%---------------------------------------------------------------------------------------------------------------------------

\begin{proposition} \label{PropAMBXKgV}
    Soit \( (a_n)\) une suite dans \( \eC\) de rayon de convergence \( R\) et la fonction
    \begin{equation}
        \begin{aligned}
            f\colon \eM(n,\eR)&\to \eM(n,\eR) \\
            A&\mapsto \sum_{k=0}^{\infty}a_kA^k
        \end{aligned}
    \end{equation}
    Alors
    \begin{enumerate}
        \item
            La différentielle de \( f\) sur \( B(0,R)\) est
            \begin{equation}    \label{EqRDVodDa}
                df_A(U)=\sum_{k=0}^{\infty}a_k\sum_{l=0}^{k-1}A^lUA^{k-1-l},
            \end{equation}
            c'est-à-dire que l'on peut différentier terme à terme. (Ici c'est \( A\) qui est dans \( B(0,R)\))
        \item
            La convergence de la somme~\ref{EqRDVodDa} est absolue.
        \item
            La convergence de la somme~\ref{EqRDVodDa} est normale sur tout compact.
        \item
            La fonction \( f\) est de classe \( C^1\) sur \( B(0,R)\), c'est-à-dire que la fonction \( A\mapsto df_A\) est continue.
    \end{enumerate}
\end{proposition}
Notons que \( df_A\) n'est pas tout à fait une série entière. Cependant, en ce qui concerne les normes, c'est tout comme si ça l'était.

\begin{proof}
    Nous posons \( u_k(A)=a_kA^k\), qui est une fonction de classe \(  C^{\infty}\) et dont la différentielle est donnée par
    \begin{equation}
        (du_k)_A(U)=\Dsdd{ u_k(A+tU) }{t}{0}=a_k\Dsdd{ (A+tU)^k }{t}{0};
    \end{equation}
    en distribuant le produit nous trouvons tout un tas de termes dont seuls ceux contenant exactement une fois \( tU\) ne vont pas s'annuler. Étant donné que \( U\) et \( A\) ne commutent pas nous avons l'expression un peu moche
    \begin{equation}
        (du_k)_A(U)=\sum_{l=0}^{k-1}a_kA^lUA^{k-1-l}.
    \end{equation}
    En ce qui concerne la norme, nous regardons celle de \( (du_k)_A\) pour un \( A\) fixé; c'est-à-dire que nous en regardons la norme opérateur :
    \begin{equation}
        \| (du_k)_A \|=\sup_{\| U \|=1}\| \sum_{l=0}^{k-1}a_kA^lUA^{k-1-l} \|\leq \sum_{l=0}^{k-1}| a_k |\| A \|^{l}\| A \|^{k-1-l}\leq k| a_k |\| A \|^{k-1}.
    \end{equation}
    Pour donner la convergence nous considérons un nombre \( r\) tel que \( \| A \|<r<R\), de telle sorte que la suite \( (a_nr^n)\) soit bornée par un nombre \( M\) et que nous puissions écrire
    \begin{equation}    \label{EqTGEwhnL}
        \| (du_k)_A \|\leq k| a_k |\| A \|^{k-1}=\frac{ k| a_k |\| A \|^k }{ \| A \| }=\frac{ k| a_k | }{ \| A \| }r^k\left( \frac{ \| A \| }{ r } \right)^k\leq \frac{ M }{ \| A \| }k\left( \frac{ \| A \| }{ r } \right)^k,
    \end{equation}
    dont la série converge. Nous avons donc convergence absolue de la série
    \begin{equation}
        \sum_{k=0}^{\infty}(du_k)_A.
    \end{equation}
    Passons à la convergence normale sur tout compact. Nous nous fixons \( r<R\) et nous nous intéressons à la norme de \( du_k\) sur \( \overline{ B(0,r) }\), c'est-à-dire
    \begin{equation}
        \| du_k \|_{\infty}=\sum_{x\in\overline{ B(0,r) }}\| (du_k)_A \|.
    \end{equation}
    Vu que \( \overline{ B(0,r) }\) est compact, ce supremum est un maximum et nous pouvons noter \( A_k\) la matrice qui le réalise. Nous réalisons alors les mêmes manipulations que pour \eqref{EqTGEwhnL} :
    \begin{equation}
        \| du_k \|_{\infty}=\| (du_k)_{A_k} \|\leq k| a_k |\| A_k \|^{k-1}\leq  k| a_k |r^{k-1}=\frac{1}{ r }k| a_k |r^k.
    \end{equation}
    Nous prenons maintenant \( r<r_0<R\) et \( M\), un majorant de \( (a_nr_0^n)\), de telle sorte qu'en multipliant et divisant par \( r_0^k\),
    \begin{equation}
        \| du_k \|_{\infty}\leq \frac{ k| a_k |r_0^k }{ r }\frac{ r^k }{ r_0^k }\leq \frac{ kM }{ r }\left( \frac{ r }{ r_0 } \right)^k,
    \end{equation}
    dont la série converge. Nous avons donc convergence normale sur tout compact. Par voie de \sout{fait} conséquences nous avons continuité de la série
    \begin{equation}
        \sum_{k=0}^{\infty}(du_k)_A
    \end{equation}
    et convergence vers \( df_A\) par le théorème~\ref{ThoLDpRmXQ}.
\end{proof}

\begin{proposition} \label{PropQIIURAh}
    Si le rayon de convergence de la série \( u(A)=\sum_{k=0}^{\infty}a_kA^k\) est \( R\), alors
    \begin{enumerate}
        \item
            elle converge normalement sur tout compact de \( B(0,R)\);
        \item
            la fonction \( u\) y est de classe \(  C^{\infty}\).
    \end{enumerate}
\end{proposition}

\begin{proof}
    Nous posons
    \begin{equation}
        \begin{aligned}
            u_k\colon \eM(n,\eR)&\to \eM(n,\eR) \\
            A&\mapsto a_kA^k
        \end{aligned}
    \end{equation}
    qui est évidemment une fonction de classe \(  C^{\infty}\). Nous étudions la \( j\)\ieme\ différentielle en \( m\), pour \( k>j\) (dans une série, nous ne nous intéressons pas aux premiers termes). La \( j\)\ieme\ différentielle appliquée à \( v_1\) appliquée à \( v_2\), etc s'exprime de la façon suivante :
    \begin{equation}
        (d^ju_k)_m(v_1,\ldots, v_j)=\frac{ d  }{ d t_1 }\ldots\frac{ d  }{ d t_j }\Big( u_k(m+t_1v_1+\cdots +t_jv_j)    \Big)_{t_i=0}.
    \end{equation}
    Dans le produit \( (m+t_1v_1+\cdots +t_jv_j)^k\), seuls les termes contenant exactement une fois chacun des \( t_i\) ne s'annulera pas après avoir fait la dérivée et évalué en \( t_i=0\). Combien de termes cela fait ? Parmi les \( k\) facteurs, il faut en placer \( j\) qui ne sont pas \( m\) (cela fait \( \binom{ k }{ j }\) possibilités), et puis il faut ordonner ces \( j\) termes, cela fait encore \( j!\) possibilités. Au final,
    \begin{equation}
        \| (d^ju_k)_m \|\leq | a_k | \binom{ k }{ j }j!\| m \|^{k-j}=| a_k |P(k)\| m \|^{k-j}
    \end{equation}
    où \( P(k)=\frac{ k! }{ (k-j)! }\) est un polynôme de degré \( j\).

    Afin d'étudier la convergence normale sur tout compact de la série des \( d^ju_k\), nous considérons \( r<r_0<R\) et nous allons prouver la convergence normale sur \( \overline{ B(0,r) }\). Vu que c'est un compact, il existe une matrice \( m_k\in\overline{ B(0,r) }\) telle que
    \begin{subequations}
        \begin{align}
            \| d^ju_k \|_{\infty}&=\| (d^ju_k)_{m_k} \|\\
            &\leq | a_k |P(k)\| m_k \|^{k-j}\\
            &\leq | a_k |P(k)r^{k-j}\\
            &=\frac{ | a_k |P(k) }{ r^j }r^k\\
            &=\frac{ | a_k |r_0^kP(k) }{ r^j }\left( \frac{ r }{ r_0 } \right)^k\\
            &\leq \frac{ M }{ r^j }P(k)\left( \frac{ r }{ r_0 } \right)^k
        \end{align}
    \end{subequations}
    où \( M\) est un majorant de \( a_nr^n\). Vu que \( r_0/r<1\), la somme sur \( k\) converge et nous avons convergence normale sur tout compact de
    \begin{equation}
        d^j\sum_{k=0}^{\infty}a_kA^k=\sum_{k=0}^{\infty}d^j(a_kA^k)
    \end{equation}
    avec un peu d'abus de notation.
\end{proof}

%+++++++++++++++++++++++++++++++++++++++++++++++++++++++++++++++++++++++++++++++++++++++++++++++++++++++++++++++++++++++++++ 
\section{Exponentielle de matrices}
%+++++++++++++++++++++++++++++++++++++++++++++++++++++++++++++++++++++++++++++++++++++++++++++++++++++++++++++++++++++++++++

\begin{proposition} \label{PropKKdmnkD}
    Une matrice complexe est inversible si et seulement si elle est une exponentielle.

    Autrement dit :
    \begin{equation}
        \GL(n,\eC)= e^{\eM(n,\eC)}.
    \end{equation}
\end{proposition}
\index{exponentielle!de matrice}
\index{décomposition!Jordan!et exponentielle de matrice}

\begin{proof}
    Nous avons déjà prouvé dans la proposition \ref{PROPooRERRooMutKcg} que toutes les exponentielles étaient inversibles. Ici nous nous concentrons sur la réciproque.

    Soit \( A\in \GL(n,\eC)\); nous allons donner une matrice \( B\in \eM(n,\eC)\) telle que \( A=\exp(B)\). D'abord remarquons qu'il suffit de prouver le résultat pour une matrice par classe de similitude. En effet si \( A=\exp(B)\) et si \( M\) est inversible alors
    \begin{subequations}    \label{EqqACuGK}
        \begin{align}
            \exp(MBM^{-1})&=\sum_k\frac{1}{ k! }(MBM^{-1})^k\\
            &=\sum_k\frac{1}{ k! }MB^kM^{-1}\\
            &=M\exp(B)M^{-1}.
        \end{align}
    \end{subequations}
    Donc \( MAM^{-1}=\exp(MBM^{-1})\). Nous pouvons donc nous contenter de trouver un logarithme pour les blocs de Jordan. Nous supposons donc que \( A=(\mtu+N)\) avec \( N^m=0\).
    En nous inspirant de \eqref{EqweEZnV}, nous posons\footnote{Le logarithme d'un nombre n'est pas encore définit à ce moment, mais cela ne nous empêche pas de poser une définition ici pour une application des réels vers les matrices.}
    \begin{equation}
        D(t)=tN-\frac{ t^2 }{ 2 }N^2+\cdots +(-1)^m\frac{ t^{m-1} }{ m-1 }N^{m-1}
    \end{equation}
    et nous allons prouver que \(  e^{D(1)}=\mtu+N\). Notons que \( N\) étant nilpotente, cette somme ainsi que toutes celles qui viennent sont finies. Il n'y a donc pas de problèmes de convergences dans cette preuve (si ce n'est les passages des équations \eqref{EqqACuGK}).

    Nous posons \( S(t)= e^{D(t)}\) (la somme est finie), et nous avons
    \begin{equation}
        S'(t)=D'(t) e^{D(t)}
    \end{equation}
    Afin d'obtenir une expression qui donne \( S'\) en termes de \( S\), nous multiplions par \( (\mtu+tN)\) en remarquant que \( (\mtu+tN)D'(t)=N\) nous avons
    \begin{equation}
        (\mtu+tN)S'(t)=NS(t).
    \end{equation}
    En dérivant à nouveau,
    \begin{equation}    \label{EqKjccqP}
        (\mtu+tN)S''(t)=0.
    \end{equation}
    La matrice \( (\mtu+tN)\) est inversible parce que son noyau est réduit à \( \{ 0 \}\). En effet si \( (\mtu+tN)x=0\), alors \( Nx=-\frac{1}{ t }x\), ce qui est impossible parce que \( N\) est nilpotente. Ce que dit l'équation \eqref{EqKjccqP} est alors que \( S''(t)=0\). Si nous développons \( S(t)\) en puissances de \( t\) nous nous arrêtons au terme d'ordre \( 1\) et nous avons
    \begin{equation}
        S(t)=S(0)+tS'(0)=\mtu+tD'(0)=1+tN.
    \end{equation}
    En \( t=1\) nous trouvons \( S(1)=\mtu+N\). La matrice \( D(1)\) donnée est donc bien un logarithme de $\mtu+N$.
\end{proof}

%---------------------------------------------------------------------------------------------------------------------------
\subsection{Diagonalisabilité d'exponentielle}
%---------------------------------------------------------------------------------------------------------------------------

\begin{proposition}[\cite{fJhCTE}]      \label{PropCOMNooIErskN}
    Si \( A\in \eM(n,\eR)\) a un polynôme caractéristique scindé, alors \( A\) est diagonalisable si et seulement si \( e^A\) est diagonalisable.
\end{proposition}
\index{décomposition!Dunford!application}
\index{exponentielle!de matrice}
\index{diagonalisable!exponentielle}

\begin{proof}
    Si \( A\) est diagonalisable, alors il existe une matrice inversible \( M\) telle que \( D=M^{-1}AM\) soit diagonale (c'est la définition~\ref{DefCNJqsmo}). Dans ce cas nous avons aussi \( (M^{-1}AM)^k=M^{-1}A^kM\) et donc \( M^{-1}e^AM=e^{M^{-1}AM}=e^D\) qui est diagonale.

    La partie difficile est donc le contraire.

    \begin{subproof}
        \item[Qui est diagonalisable et comment ?]
            Nous supposons que \( e^A\) est diagonalisable et nous écrivons la décomposition de Dunford (théorème~\ref{ThoRURcpW}) :
            \begin{equation}
                A=S+N
            \end{equation}
            où \( S\) est diagonalisable, \( N\) est nilpotente, \( [S,N]=0\). Nous avons besoin de prouver que \( N=0\).

            Les matrices \( A\) est \( S\) commutent; en passant au développement nous en déduisons que \( A\) et \( e^S\) commutent, puis encore en passant au développement que \( e^A\) et \( e^S\) commutent. Vu que \( S\) est diagonalisable, \( e^S\) l'est et par hypothèse \( e^A\) est également diagonalisable. Donc \( e^A\) et \( e^{-S}\) sont simultanément diagonalisables par la proposition~\ref{PropGqhAMei}.

            Étant donné que \( A\) et \( S\) commutent, nous avons \( e^N=e^{A-S}=e^Ae^{-S}\), et nous en déduisons que \( e^N\) est diagonalisable vu que les deux facteurs \( e^A\) et \( e^{-S}\) sont simultanément diagonalisables.

        \item[Unipotence]

            Si \( r\) est le degré de nilpotence de \( N\), nous avons
            \begin{equation}    \label{EqQHjvLZQ}
                e^N-\mtu=N+\frac{ N^2 }{2}+\cdots +\frac{ N^{r-1} }{ (r-1)! }.
            \end{equation}
            Donc
            \begin{equation}
                (e^N-\mtu)^k=\left( N+\frac{ N^2 }{2}+\cdots +\frac{ N^{r-1} }{ (r-1)! } \right)^k
            \end{equation}
            où le membre de droite est un polynôme en \( N\) dont le terme de plus bas degré est de degré \( k\). Donc \( (e^N-\mtu)\) est nilpotente et \( e^N\) est unipotente.

            Si \( M\) est la matrice qui diagonalise \( e^N\), alors la matrice diagonale \( M^{-1}e^NM\) est tout autant unipotente que \( e^N\) elle-même. En effet,
            \begin{subequations}
                \begin{align}
                    (M^{-1}e^NM-\mtu)^r&=\sum_{k=0}^r\binom{ r }{ k }(-1)^{r-k}M^{-1}(e^N)^kM\\
                    &=M^{-1}\left( \sum_{k=0}^r\binom{ r }{ k }(-1)^{r-k}(e^N)^k \right)M\\
                    &=M^{-1}(e^N-\mtu)^rM\\
                    &=0.
                \end{align}
            \end{subequations}

            La matrice \( M^{-1}e^NM\) est donc une matrice diagonale et unipotente; donc \( M^{-1}e^NM=\mtu\), ce qui donne immédiatement que \( e^N=\mtu\).

        \item[Polynômes annulateurs]

            En reprenant le développement \eqref{EqQHjvLZQ} sachant que \( e^N=\mtu\), nous savons que
            \begin{equation}
                N+\frac{ N^2 }{2}+\cdots +\frac{ N^{r-1} }{ (r-1)! }=0.
            \end{equation}
            Dit en termes pompeux (mais non moins porteurs de sens), le polynôme
            \begin{equation}
                Q(X)=X+\frac{ X^2 }{2}+\cdots +\frac{ X^{r-1} }{ (r-1)! }
            \end{equation}
            est un polynôme annulateur de \( N\).

            La proposition~\ref{PropAnnncEcCxj} stipule que le polynôme minimal d'un endomorphisme divise tous les polynômes annulateurs. Dans notre cas, \( X^r\) est un polynôme annulateur et donc le polynôme minimal de \( N\) est de la forme \( X^k\). Donc il est \( X^r\) lui-même.

            Nous avons donc \( X^r\divides Q\). Mais \( Q\) est un polynôme contenant le monôme \( X\) donc \( X^r\) ne peut diviser \( Q\) que si \( r=1\). Nous en concluons que \( X\) est un polynôme annulateur de \( N\). C'est-à-dire que \( N=0\).

        \item[Conclusion]

            Vu que Dunford\footnote{Théorème~\ref{ThoRURcpW}.} dit que \( A=S+N\) et que nous venons de prouver que \( N=0\), nous concluons que \( A=S\) avec \( S\) diagonalisable.

    \end{subproof}
\end{proof}


\input{76_series_fonctions}


\chapter{Représentations et caractères}
\input{63_representations}

\chapter{Encore de l'analyse (et c'est pas fini)}
% This is part of Mes notes de mathématique
% Copyright (c) 2011-2015,2017-2019
%   Laurent Claessens
% See the file fdl-1.3.txt for copying conditions.

%+++++++++++++++++++++++++++++++++++++++++++++++++++++++++++++++++++++++++++++++++++++++++++++++++++++++++++++++++++++++++++
\section{Densité des polynômes}
%+++++++++++++++++++++++++++++++++++++++++++++++++++++++++++++++++++++++++++++++++++++++++++++++++++++++++++++++++++++++++++

\begin{corollary}   \label{CorRSczQD}
    Si \( X\subset \eR\) est compact et de mesure finie\footnote{Dans \( \eR\) cette hypothèse est évidemment superflue par rapport à l'hypothèse de compacité; mais ça suggère des généralisations \ldots}, alors l'ensemble des polynômes est denses dans \( \big( C(X,\eR),\| . \|_2 \big)\).
\end{corollary}

\begin{proof}
    Si \( f\) est une fonction dans \( C(X,\eR)\) et si \( \epsilon\geq 0\) est donné alors nous pouvons considérer un polynôme \( P\) tel que \( \| f-P \|_{\infty}\leq \epsilon\). Dans ce cas nous avons
    \begin{equation}
        \| f-P \|_2^2=\int_X| f(x)-P(x) |^2dx\leq \int_X\epsilon^2dx=\epsilon^2\mu(X)
    \end{equation}
    où \( \mu(X)\) est la mesure de \( X\) (finie par hypothèse).
\end{proof}

%+++++++++++++++++++++++++++++++++++++++++++++++++++++++++++++++++++++++++++++++++++++++++++++++++++++++++++++++++++++++++++ 
\section{Primitive et intégrale}
%+++++++++++++++++++++++++++++++++++++++++++++++++++++++++++++++++++++++++++++++++++++++++++++++++++++++++++++++++++++++++++

Nous avons déjà parlé de primitive de fonction continue en la proposition \ref{ThoEXXyooCLwgQg}.

\begin{proposition} \label{PropHFWNpRb}
    Soit \( I \) un intervalle borné ouvert de \( \eR\). Une fonction \( h\in C^{\infty}_c(I)\) admet une primitive dans \(  C^{\infty}_c(I)\) si et seulement si \( \int_Ih=0\).
\end{proposition}

\begin{proof}
    Si une primitive \( H\) de \( h\) est à support compact, alors
    \begin{equation}
        \int_Ih=H(b)-H(a)=0-0=0.
    \end{equation}
    Pas de problèmes dans ce sens.

    Supposons maintenant que \( \int_Ih=0\). Le fait que \( h\) admette une primitive dans \(  C^{\infty}(I)\) est évident : toute fonction continue admet une primitive\footnote{Théorème~\ref{ThoEXXyooCLwgQg}.}. Soit \( H\) une telle primitive et \( \tilde H=H-H(b)\). Alors \( \tilde H(b)=0\) et
    \begin{equation}
        \tilde H(a)=H(a)-H(b)=-\int_Ih=0.
    \end{equation}
    Nous rappelons que le support d'une fonction est \emph{la fermeture} de l'ensemble des points de non-annulation.

    Supposons que le support de \( h\) soit inclus dans \( \mathopen[ m , M \mathclose]\subset\mathopen] a , b \mathclose[\). En prenant des nombres \( m'\) et \( M'\) tels que \( a<m'<m\) et \( M<M'<b\) (nous insistons sur le caractère strict de ces inégalités), la fonction \( h\) est nulle sur \( \mathopen[ a , m' \mathclose]\) et sur \( \mathopen[ M' , b \mathclose]\); la fonction \( \tilde H\) doit donc y être constante. Mais nous avons déjà vu que \( \tilde H(a)=\tilde H(b)=0\). Donc l'ensemble des points sur lesquels \( \tilde H\) n'est pas nul est inclus dans \( \mathopen] m' , M' \mathclose[\) et donc est strictement (des deux côtés) inclus dans \( I\).
\end{proof}

%---------------------------------------------------------------------------------------------------------------------------
\subsection{Théorème taubérien de Hardy-Littlewood}
%---------------------------------------------------------------------------------------------------------------------------

Un théorème \defe{taubérien}{taubérien}\index{théorème!taubérien} est un théorème qui compare les modes de convergence d'une série.

\begin{lemma}
    Si \( f\) et \( g\) sont des fonctions continues, alors \( s(x)=\max\{ f(x),g(x) \}\) est également une fonction continue.
\end{lemma}

\begin{proof}
    Soit \( x_0\) et prouvons que \( s\) est continue en \( x_0\). Si \( f(x_0)\neq g(x_0)\) (supposons \( f(x_0)>g(x_0)\) pour fixer les idées), alors nous avons un voisinage de \( x_0\) sur lequel \( f>g\) et alors \( s=f\) sur ce voisinage et la continuité provient de celle de \( f\).

    Si au contraire \( f(x_0)=g(x_0)=s(x_0)\) alors si \( (a_n)\) est une suite tendant vers \( x_0\), nous prenons \( N\) tel que \( \big| f(a_n)-f(x_0) \big|\leq \epsilon\) pour tout \( n>N\) et \( M\) tel que \( \big| g(a_n)-g(x_0) \big|\leq \epsilon\) pour tout \( n> M\). Alors pour tout \( n>\max\{ N,M \}\) nous avons
    \begin{equation}
        \big| s(a_n)-s(x_0) \big|\leq \epsilon,
    \end{equation}
    d'où la continuité de \( s\) en \( x_0\).
\end{proof}

La proposition suivante dit que si une fonction connaît un saut, alors on peut le lisser par une fonction continue.
\begin{proposition} \label{PropTIeYVw}
    Soit \( f\) continue sur \( \mathopen[ a , x_0 [\) et sur \( \mathopen[ x_0 , b \mathclose]\) avec \( f(x_0^-)<f(x_0)\). En particulier nous supposons que \( f(x^-)\) existe et est finie. Alors pour tout \( \epsilon>0\), il existe une fonction continue \( s\) telle que sur \( \mathopen[ a , b \mathclose]\) on ait \( s\leq f\) et
    \begin{equation}
        \int_a^bs(x)-f(x)\,dx\leq \epsilon.
    \end{equation}
\end{proposition}

\begin{proof}
    Nous notons \( A\) la taille du saut :
    \begin{equation}
        A=f(x_0)-f(x_0^-).
    \end{equation}
    Quitte à changer \( a\) et \( b\), nous pouvons supposer que
    \begin{equation}
        f(x)<f(x_0)+\frac{ A }{ 3 }
    \end{equation}
    pour \( x\in \mathopen[ a , x_0 [\) et
    \begin{equation}
        f>f(x_0)+\frac{ 2A }{ 3 }
    \end{equation}
    pour \( x\in \mathopen[ x_0 , b \mathclose]\). C'est le théorème des valeurs intermédiaires qui nous permet de faire ce choix.

    Soit \( m(x)\) la droite qui joint le point \( \big( x_0-\epsilon, f(x_0-\epsilon) \big)\) au point \( \big( x_0,f(x_0^+) \big)\). Nous posons
    \begin{equation}
        s(x)=\begin{cases}
            f(x)    &   \text{si } x<x_0-\epsilon\\
            \max\{ m(x),f(x) \}    &   \text{si } x_0-\epsilon\leq x\leq x_0\\
            f(x)    &    \text{si }x>x_0.
        \end{cases}
    \end{equation}
    En vertu des différents choix effectués, c'est une fonction continue. En effet
    \begin{equation}
        s(x_0-\epsilon)=\max\{ f(x_0-\epsilon),f(x_0,\epsilon) \}=f(x_0-\epsilon)
    \end{equation}
    et
    \begin{equation}
        s(x_0)=\max\{ m(x_0),f(x_0^+) \}=f(x_0^+)
    \end{equation}
    parce que \( m(x_0)=f(x_0^+)\). En ce qui concerne l'intégrale, si nous posons
    \begin{equation}
        M=\sup_{x,y\in \mathopen[ a , b \mathclose]}| f(x)-f(y) |,
    \end{equation}
    nous avons
    \begin{equation}
        \int_a^bs-f=\int_{x_0-\epsilon}^{x_0}s-f\leq \epsilon M.
    \end{equation}
\end{proof}

\begin{lemma}\label{LemauxrKN}
    Pour tout polynôme \( P\), nous avons la formule
    \begin{equation}
        \lim_{x\to 1^-} (1-x)\sum_{n=0}^{\infty}x^nP(x^n)=\int_0^1P(x)dx.
    \end{equation}
\end{lemma}

\begin{proof}
    D'abord pour \( P=1\), la formule se réduit à la série harmonique connue. Ensuite nous prouvons la formule pour le polynôme \( P=X^k\) et la linéarité fera le reste pour les autres polynômes. Nous avons
    \begin{equation}
        (1-x)\sum_nx^nx^{kn}=(1-x)\sum_n(x^{1+k})^n=\frac{ 1-x }{ 1-x^{1+k} }=\frac{1}{ 1+x+\cdots+x^k }.
    \end{equation}
    Donc
    \begin{equation}
        \lim_{x\to 1^-} (1-x)\sum_nx^nP(x^n)=\frac{1}{ 1+k }.
    \end{equation}
    Par ailleurs, c'est vite vu que
    \begin{equation}
        \int_0^1 x^kdx=\frac{1}{ k+1 }.
    \end{equation}
\end{proof}

\begin{theorem}[Hardy-Littlewood\cite{ytMOpe}]\index{théorème!Hardy-Littlewood}\index{Hardy-Littlewood (théorème)}      \label{ThoPdDxgP}
    Soit \( (a_n)\) une suite réelle telle que
    \begin{enumerate}
        \item
            \( \frac{ a_n }{ n }\) tend vers une constante,
        \item
            \( F(x)=\sum_{n=0}^{\infty}a_nx^n\) a un rayon de convergence \( \geq 1\),
        \item
            \( \lim_{x\to 1^-} F(x)=l\).
    \end{enumerate}
    Alors \( \sum_{n=0}^{\infty}a_n=l\).
\end{theorem}
\index{convergence!suite numérique}
\index{série!nombres}
\index{série!fonctions}
\index{limite!inversion}
\index{approximation!par polynômes}

\begin{proof}
    Quitte à prendre la suite \( b_0=a_0-l\) et \( b_n=a_n\), on peut supposer \( l=0\).

    Soit \( \Gamma\) l'ensemble des fonctions
    \begin{equation}
         \gamma\colon \mathopen[ 0 , 1 \mathclose]\to \eR
    \end{equation}
    telles que
    \begin{enumerate}
        \item
            $\sum_{n=0}^{\infty}a_n\gamma(x^n)$ converge pour \( 0\leq x<1\),
        \item
            \( \lim_{x\to 1^-} \sum_{n\geq 0}a_n\gamma(^n)=0\).
    \end{enumerate}
    Ce \( \Gamma\) est un espace vectoriel.
    \begin{subproof}
    \item[Les polynômes sont dans \( \Gamma\)]
        Soit \( \gamma(t)=t^s\). Pour \( 0\leq x<1\) nous avons
        \begin{equation}
            \sum_{n=0}^{\infty}a_n\gamma(x^n)=\sum_{n=0}^{\infty}a_nx^{ns}<\sum_{n=0}^{\infty}a_nx^n.
        \end{equation}
        Donc la condition de convergence est vérifiée. En ce qui concerne la limite,
        \begin{equation}
            \lim_{x\to 1^-} \sum_{n=0}^{\infty}a_nx^{ns}=\lim_{x\to 1^-} F(x^s)=0
        \end{equation}
        parce que par hypothèse, \( \lim_{x\to 1^-} F(x)=0\).

    \item[Définition de la fonction qui va donner la réponse]
        Nous considérons la fonction 
        \begin{equation}
            g(t)=\begin{cases}
                0    &   \text{si } 0\leq t<1/2\\
                1    &    \text{si } 1/2\leq t\leq 1,
            \end{cases}
        \end{equation}
        c'est-à-dire \( g=\mtu_{\mathopen[ \frac{ 1 }{2} , 1 \mathclose]}\). Nous montrons que si \( g\in \gamma\), alors le théorème est terminé. Si \( 0\leq x\leq 1\), on a \( 0\leq x^n<1/2\) dès que
        \begin{equation}
            n>-\frac{ \ln(2) }{ \ln(x) }
        \end{equation}
        avec une note comme quoi \( \ln(x)<0\), donc la fraction est positive. Nous désignons par \( N_x\) la partie entière de ce \( n\) adapté à \( x\). L'idée est que la fonction  \( g(x^n)\) est la fonction indicatrice de \(0 \leq n\leq N_x\), et donc
        \begin{equation}
            \sum_{n\geq 0}a_ng(x^n)=\sum_{n=0}^{N_x}a_n.
        \end{equation}
        Mais si \( x\to 1^-\), alors \( N_x\to \infty\), donc
        \begin{equation}
            \lim_{N\to \infty} \sum_{n=0}^Na_n=\lim_{x\to 1^-} \sum_{n=0}^{N_x}a_n=\lim_{x\to 1^-} \sum_{n\in \eN}a_ng(x^n),
        \end{equation}
        et cela fait zéro si \( g\in \Gamma\).

    \item[Approximation de \( g\) par des polynômes]

        Nous considérons la fonction
        \begin{equation}
            h(t)=\frac{ g(t)-t }{ t(1-1) }=\begin{cases}
                \frac{1}{ t-1 }    &   \text{si } t\in \mathopen[ 0 , 1/2 [\\
                \frac{1}{ t }    &    \text{si } t\in \mathopen[ 1/2 , 1 \mathclose].
            \end{cases}
        \end{equation}
        La seconde égalité est au sens du prolongement par continuité. La fonction \( h\) est une fonction non continue qui fait un saut de \( -2\) à \( 2\) en \( x=1/2\). En vertu de la proposition~\ref{PropTIeYVw} (un peu adaptée), nous pouvons considérer deux fonctions continues \( s_1\) et \( s_2\) telles que
        \begin{equation}
            s_1\leq h\leq s_2
        \end{equation}
        et
        \begin{equation}
            \int_{0}^1s_2-s_1\leq \epsilon.
        \end{equation}
        Notons que l'inégalité \( s_1\leq s_2\) doit être stricte sur au moins un petit intervalle autour de \( x=1/2\). Soient \( P_1\) et \( P_2\), deux polynômes tels que \( \| P_1-s_1 \|_{\infty}\leq \epsilon\) et \( \| P_2-s_2 \|_{\infty}\leq \epsilon\) (ici la norme supremum est prise sur \( \mathopen[ 0 , 1 \mathclose]\)). C'est le théorème de Stone-Weierstrass (\ref{ThoGddfas}) qui nous permet de le faire.

        Nous posons aussi\footnote{À ce niveau, je crois qu'il y a une faute de frappe dans \cite{ytMOpe}.}
        \begin{subequations}
            \begin{align}
                Q_1=P_1+\epsilon\\
                Q_2=P_2-\epsilon.
            \end{align}
        \end{subequations}
        Nous avons
        \begin{equation}
            \int_0^1Q_1-Q_2\leq\int_0^1 Q_1-P_1+P_1-P_2+P_2-Q_2.
        \end{equation}
        Pour majorer cela, d'abord \( Q_1-P_1=P_2-Q2=\epsilon\), ensuite,
        \begin{equation}
            P_1-P_2=P_1-s_1+s_1-s_2+s_2-P_2
        \end{equation}
        dans lequel nous avons \( P_1-s_1\leq \epsilon\), \( s_2-P_2\leq \epsilon\) et \( \int_0^1s_1-s_2\leq\epsilon\). Au final, nous posons \( q=Q_2-Q_1\) et nous avons
        \begin{equation}
            \int_0^1q\leq 5\epsilon.
        \end{equation}
        Enfin nous posons aussi
        \begin{equation}
            R_i(x)=x+x(1-x)Q_i.
        \end{equation}
        Ces polynômes vérifient \( R_i(0)=0\), \( R_i(1)=1\) et
        \begin{equation}
            R_1\leq g\leq R_2
        \end{equation}
        parce que
        \begin{equation}
            Q_1\leq P_1\leq h\leq  P_2\leq Q_2
        \end{equation}
        et
        \begin{equation}
            t+t(1-t)Q_1\leq \underbrace{t+t(1-t)h(t)}_{g(t)}\leq t+t(1-t)Q_2.
        \end{equation}

    \item[Preuve que \( g\) est dans \( \Gamma\)]

        D'abord si \( 0\leq x<1\), \( x^N<\frac{ 1 }{2}\) pour un certain \( N\), et alors \( g(x^N)=0\). Du coup la série
        \begin{equation}
            \sum_{n=0}^{\infty}a_ng(x^n)=\sum_{n=0}^{N}a_n
        \end{equation}
        est une somme finie qui converge donc.

        D'autre part nous prenons \( M\) tel que \( | a_n |<\frac{ M }{ n }\) pour tout \( n\). Nous majorons \( \sum_{n \in \eN}a_ng(x^n)\) en utilisant \( R_1\). Mais vu que \( R_1\) est un polynôme, nous pouvons dire que \( | \sum_{n=0}^{\infty}a_nR_1(x^n) |\leq \epsilon\) en prenant \( x\in\mathopen[ \lambda , 1 [\) et \( \lambda\) assez grand. Nous avons :
        \begin{subequations}
            \begin{align}
                \left| \sum_{n=0}^{\infty}a_ng(x^n) \right| &\leq\left| \sum_{n=0}^{\infty}a_ng(x^n)-\sum_{n=0}^{\infty}a_nR_1(x^n) \right| +\underbrace{\left| \sum_{n=0}^{\infty}a_nR_1(x^n) \right|}_{\leq \epsilon} \\
                &\leq \epsilon+\sum_{n=0}^{\infty}| a_n |(g-R_1)(x^n)\\
                &\leq \epsilon+\sum_{n=0}^{\infty}| a_n |(R_2-R_1)(x^n)\\
                &\leq \epsilon+M\sum_{n=0}^{\infty}\frac{ x^n(1-x^n) }{ n }(Q_2-Q_1)(x^n) \label{SUBEQooAIQWooJADvKs} \\
                &=\epsilon+M\sum_{n=0}^{\infty}\frac{ x^n(1-x^n) }{ n }q(x^n)\\
                &\leq \epsilon+M(1-x)\sum_nx^nq(x^n).   \label{subeqtZXDvu}
            \end{align}
        \end{subequations}  
        Justifications :
        \begin{itemize}
            \item La ligne \eqref{SUBEQooAIQWooJADvKs} est par le fait que $R_2-R_1=x(1-x)(Q_2-Q_1)$.
            \item La ligne \eqref{subeqtZXDvu} provient d'une majoration sauvage de \( 1/n\) par \( 1\) et de \( 1-x^n\) par \( 1-x\). 
        \end{itemize}
        Par le lemme \ref{LemauxrKN}, nous avons alors
        \begin{equation}
            \lim_{x\to 1^-} | \sum_na_ng(x^n) |\leq \epsilon+M\int_0^1q\leq 6\epsilon.
        \end{equation}
    \end{subproof}
\end{proof}

%---------------------------------------------------------------------------------------------------------------------------
\subsection{Théorème de Müntz}
%---------------------------------------------------------------------------------------------------------------------------

\begin{theorem}[Théorème de Müntz\cite{jqZSyG,oYGash,ooRIPFooALoEWM}]  \label{ThoAEYDdHp}
    Soit \( C_0\big( \mathopen[ 0 , 1 \mathclose] \big)\), l'espace des fonctions continues sur \( \mathopen[ 0 , 1 \mathclose]\) muni de la norme \( \| . \|_{\infty}\) ou \( \| . \|_2\) et une suite \( (\alpha_n)\) strictement croissante de nombres positifs. Nous notons \( \phi_{\lambda}\) la fonction \( x\mapsto x^{\lambda}\).

    Alors
    \begin{equation}
        \overline{  \Span\{1, \phi_{\alpha_n} \} }
    \end{equation}
    est dense dans \( C_0\big( \mathopen[ 0 , 1 \mathclose] \big)\)  si et seulement si
    \begin{equation}
        \sum_{n=2}^{\infty}\frac{1}{ \alpha_n }=+\infty.
    \end{equation}
\end{theorem}

Nous prouvons le théorème pour la norme \( \| . \|_2\).
\begin{proof}
    Soit \( m\in \eR^+\); nous notons \( \Delta_N(m)\) la distance entre \( \phi_m\) et \( \Span\{ \phi_{\alpha_1},\ldots, \phi_{\alpha_N} \}\). Cette distance peut être évaluée avec le déterminant de Gram\index{déterminant!Gram} (proposition~\ref{PropMsZhIK})
    \begin{equation}
        \Delta_N(m)^2=\frac{ G(\phi_m,\phi_{\alpha_1},\ldots, \phi_{\alpha_N}) }{ G(\phi_{\alpha_1},\ldots, \phi_{\alpha_N}) }.
    \end{equation}
    Pour calculer cela nous avons besoin des produits scalaires\footnote{C'est ici qu'on se particularise à la norme \( \| . \|_2\).}
    \begin{equation}
        \langle \phi_a, \phi_b\rangle =\int_0^1 x^{a+b}dx=\frac{1}{ a+b+1 }.
    \end{equation}
    Pour avoir des notations plus compactes, nous notons \( \alpha_0=m\). Donc nous avons à calculer le déterminant
    \begin{equation}
        G(\phi_m,\phi_{\alpha_1},\ldots, \phi_{\alpha_N})=\det\begin{pmatrix}
            \frac{1}{ \alpha_i+\alpha_j+1 }
         \end{pmatrix}
    \end{equation}
    où \( i,j=0,\ldots, N\). Nous reconnaissons un déterminant de Cauchy (proposition~\ref{ProptoDYKA})\index{déterminant!Cauchy} en posant, dans \( \frac{1}{ \alpha_i+\alpha_j+1 }\), \( a_i=\alpha_i\) et \( b_j=\alpha_j+1\). Étant donné que \( b_j-b_i=a_j-a_i\), nous avons
    \begin{equation}
        G(\phi_m,\phi_{\alpha_1},\ldots, \phi_{\alpha_N})=\frac{ \prod_{0\leq i<j\leq N}  (\alpha_j-\alpha_i)^2 }{ \prod_{i=0}^N\prod_{j=0}^N (\alpha_i+\alpha_j+1).}
    \end{equation}
    Nous séparons maintenant les termes où \( i\) ou \( j\) sont nuls. En ce qui concerne le dénominateur, il faut prendre tous les couples \( (i,j)\) avec \( i\) et \( j\) éventuellement égaux à zéro. Nous décomposant cela en trois paquets. Le premier est \( (0,0)\); le second est \( (0,i)\) (chaque couple arrive en fait deux fois parce qu'il y a aussi \( (i,0)\)); et le troisième sont les \( i,j\) tous deux différents de zéro :
    \begin{equation}
        (2m+1)\prod_{ij}(\alpha_i+\alpha_j+1)\prod_i(\alpha_i+m+1)^2.
    \end{equation}
    Notons que dans le produit central, le carré est contenu dans le fait qu'on écrit \( \prod_{ij}\) et non \( \prod_{i<j}\). Nous avons donc
    \begin{equation}
        G(\phi_m,\phi_{\alpha_1},\ldots, \phi_{\alpha_N})=\frac{ \prod_{i<j}(\alpha_i-\alpha_j)^2\prod_i(\alpha_i-m)^2 }{ (2m+1)\prod_{ij}(\alpha_i+\alpha_j+1)\prod_i(\alpha_i+m+1)^2 }.
    \end{equation}

    Le calcul de \( G(\phi_{\alpha_1},\ldots, \phi_{\alpha_N})\) est plus simple\footnote{Je crois qu'il y a une faute de frappe dans le dénominateur de \cite{jqZSyG}.} :
    \begin{equation}
        G(\phi_{\alpha_1},\ldots, \phi_{\alpha_N})=\frac{ \prod_{i<j}(\alpha_i-\alpha_j)^2 }{ \prod_{ij}(\alpha_i+\alpha_j+1) }.
    \end{equation}
    En divisant l'un par l'autre il ne reste que les facteurs comprenant \( m\) et en prenant la racine carrée,
    \begin{equation}    \label{EqANiuNB}
        \Delta_N(m)=\frac{1}{ \sqrt{2m+1} }\prod_{i=1}^N\left| \frac{ \alpha_i-m }{ \alpha_i+m+1 } \right| .
    \end{equation}

    Nous passons maintenant à la preuve proprement dite. Supposons que \( V=\Span\{ \phi_{\alpha_i},i\in \eN \}\) est dense. Si \( m\) est un des \( \alpha_i\), il peut évidemment être approché par les \( \phi_{\alpha_i}\). Mais vue la densité de \( V\), un \( \phi_m\) avec \( m\neq \alpha_i\) (pour tout \( i\)) alors \( \phi_m\) peut également être arbitrairement approché par les \( \phi_{\alpha_i}\), c'est-à-dire que
    \begin{equation}
        \lim_{N\to \infty} \Delta_N(m)=0.
    \end{equation}
    Nous posons
    \begin{equation}
        u_n=\ln\left( \frac{ \alpha_n-m }{ \alpha_n+m+1 } \right)
    \end{equation}
    et nous prouvons que la série \( \sum_nu_n\) diverge. En effet nous nous souvenons de la formule \( \ln(ab)=\ln(a)+\ln(b)\), de telle sorte que la \( N\)\ieme somme partielle de \( \sum_nu_n\) est
    \begin{equation}
        \ln\left( \frac{ \alpha_1-m }{ \alpha_1+m+1 }\cdot\ldots\cdot \frac{ \alpha_N-m }{ \alpha_N+m+1 } \right)=\ln\left( \sqrt{2m+1}\Delta_N(m) \right),
    \end{equation}
    qui tend vers \( -\infty\) lorsque \( N\to \infty\).

    Si la suite \( (\alpha_n)\) est majorée et plus généralement si nous n'avons pas \( \alpha_n\to \infty\), alors évidemment la série \( \sum_n\frac{1}{ \alpha_n }\) diverge. Nous supposons donc que \( \lim_{n\to \infty} \alpha_n=\infty\). Nous avons aussi\quext{Je crois qu'il y a une faute de signe dans la dernière expression de \cite{oYGash}.}
    \begin{equation}
        u_n=\ln\left( \frac{ \alpha_n-m }{ \alpha_n+m+1 } \right)=\ln\left( 1-\frac{ 2m+1 }{ \alpha_n+m+1 } \right)\sim-\frac{ 2m+1 }{ \alpha_n }.
    \end{equation}
    Une justification est donné à l'équation \eqref{EqGICpOX}. Ce que nous avons surtout est
    \begin{equation}
        \sum_n u_n\sim -(2m+1)\sum_n\frac{1}{ \alpha_n }.
    \end{equation}
    Étant donné que la série de gauche diverge, celle de droite diverge\footnote{Nous utilisons le fait que si \( u_n=\sum v_n\) en tant que suites et si \( \sum_nu_n\) diverge, alors \( \sum_nv_n\) diverge.}.

    Nous faisons maintenant le sens opposé : nous supposons que la série \( \sum_n1/\alpha_n\) diverge et nous nous posons
    \begin{equation}
        V=\Span\{ \phi_{\alpha_n}\tq n\in \eN \}.
    \end{equation}
    Il suffit de prouver que \( \phi_m\in \bar V\) pour tout \( m\) parce qu'un corolaire du théorème de Stone-Weierstrass~\ref{CorRSczQD} montre que \( \Span\{ \phi_k\tq k\in \eN \}\) est dense dans \( C\) pour la norme \( \| . \|_2\).

    Si \( \alpha_n\to \infty\), nous avons :
    \begin{equation}
        u_n\sim\frac{ 2m+1 }{ \alpha_n }\to 0
    \end{equation}
    et alors \( \Delta_N(m)\to 0\). Dans ce cas nous avons immédiatement \( \phi_m\in \bar V\).

    Si par contre \( \alpha_n\) ne tend pas vers l'infini, nous repartons de l'expression \eqref{EqANiuNB}, nous posons \( 0<\alpha=\sup_i\alpha_i\) et nous calculons :
    \begin{subequations}
        \begin{align}
            \sqrt{2m+1}\Delta_N(m)&=\prod_{i=1}^N\frac{ | \alpha_i-m | }{ \alpha_i+m+1 }\\
            &\leq \prod_{i=1}^N\frac{ \alpha_i+m }{ \alpha_i+m+1 }\\
            &=\prod_{i=1}^N\left( 1-\frac{ 1 }{ \alpha_i+m+1 } \right)\\
            &\leq \prod_{i=1}^N\left( 1-\frac{1}{ \alpha+m+1 } \right)\\
            &=\left( 1-\frac{1}{ \alpha+m+1 } \right)^N.
        \end{align}
    \end{subequations}
    Cette dernière expression tend vers \( 0\) lorsque \( N\to \infty\).
\end{proof}

\begin{remark}      \label{REMooGPYYooCQJwFa}
    Certaines sources\footnote{Dont le rapport du jury 2014} citent le théorème de Müntz comme ceci (avec un implicite que \( \alpha_i\neq 0\)):
    \begin{equation}        \label{EQooPCSZooUDSzwQ}
        \overline{ \Span\{1, \phi_{\alpha_i} \} }=C\big( \mathopen[ 0 , 1 \mathclose] \big) \Leftrightarrow \sum_{i\geq 1}\frac{1}{ \alpha_i }=+\infty.
    \end{equation}
    Que penser de la présence explicite du \( 1\) (c'est-à-dire de \( \phi_0\)) ou non dans l'ensemble ?

    Première chose : la présence éventuelle de \( \phi_0\) est la raison pour laquelle nous faisons commencer la somme à \( i=2\) et non \( i=1\). Dans le même ordre d'idée, si $\Span\{ \phi_{\alpha_i} \}$  est dense, alors en prenant n'importe quelle queue de suite, ça reste dense.

    Prouvons donc l'énoncé \eqref{EQooPCSZooUDSzwQ}. Si \( \Span\{ 1,\phi_{\alpha_i} \}\) est dense, alors en posant \( \beta_1=0\), \( \beta_i=\alpha_{i-1}\) notre théorème prouve que \( \sum_{\beta=2}^{\infty}\frac{1}{ \beta_i }=+\infty\), cela est exactement que \( \sum_{i=1}^{\infty}\frac{1}{ \alpha_i }=+\infty\). Dans l'autre sens, si \( \sum_{i\geq 1}\frac{1}{ \alpha_i }=+\infty\), alors nous avons aussi \( \sum_{i\geq 2}\frac{1}{ \alpha_i }=+\infty\) et notre théorème dit que \( \Span \{ \phi_{\alpha_i} \}\) est dense. A fortiori, \( \Span\{ 1,\phi_{\alpha_i} \}\) est dense.
\end{remark}

\begin{example}
    Nous savons depuis le théorème~\ref{ThonfVruT} que la somme des inverses des nombres premiers diverge.
\end{example}

%+++++++++++++++++++++++++++++++++++++++++++++++++++++++++++++++++++++++++++++++++++++++++++++++++++++++++++++++++++++++++++
\section{Intégrales convergeant uniformément}
%+++++++++++++++++++++++++++++++++++++++++++++++++++++++++++++++++++++++++++++++++++++++++++++++++++++++++++++++++++++++++++

%---------------------------------------------------------------------------------------------------------------------------
\subsection{Définition et propriété}
%---------------------------------------------------------------------------------------------------------------------------

\begin{definition}      \label{DEFooSHWAooWtswtp}
    Soit \( (\Omega,\mu)\) un espace mesuré. Nous disons que l'intégrale
    \begin{equation}
        \int_{\Omega}f(x,\omega)d\mu(\omega)
    \end{equation}
    \defe{converge uniformément}{convergence!uniforme!intégrale} en \( x\) si pour tout \( \epsilon>0\), il existe un compact \( K_{\epsilon}\) tel que pour tout compact \( K\) tel que \( K_{\epsilon}\subset K\) nous avons
    \begin{equation}
        \left| \int_{\Omega\setminus K}f(x,\omega)d\mu(\omega) \right| \leq \epsilon.
    \end{equation}
    Le point important est que le choix de \( K_{\epsilon}\) ne dépend pas de \( x\).
\end{definition}

\begin{lemma}       \label{LemOgQdpJ}
    Soit
    \begin{equation}
        F(x)=\int_{\Omega}f(x,\omega)d\mu(\omega),
    \end{equation}
    une intégrale uniformément convergente. Pour chaque \( k\in \eN\) nous considérons un compact \( K_k\) tel que
    \begin{equation}
        \left| \int_{\Omega\setminus K_k}f(x,\omega)d\mu(\omega) \right| \leq\frac{1}{ k }.
    \end{equation}
    Alors la suite de fonctions \( F_k\) définie par
    \begin{equation}
        F_k(x)=\int_{K_k}f(x,\omega)d\mu(\omega)
    \end{equation}
    converge uniformément vers \( F\).
\end{lemma}

\begin{proof}
    Nous avons
    \begin{subequations}
        \begin{align}
            \big| F_k(x)-F(x) \big|&=\left| \int_{K_k}f(x,\omega)d\mu(\omega)-\int_{\Omega}f(x,\omega)d\mu(\omega) \right| \\
            &=| \int_{\Omega\setminus K_k}f(x,\omega)d\mu(\omega) |\\
            &\leq \frac{1}{ k }.
        \end{align}
    \end{subequations}
\end{proof}

%------------------------------------------------------------------------------------------------------------------------
\subsection{Critères de convergence uniforme}
%---------------------------------------------------------------------------------------------------------------------------

Afin de tester l'uniforme convergence d'une intégrale, nous avons le \defe{critère de Weierstrass}{critère!Weierstrass}:
\begin{theorem}		\label{ThoCritWeiIntUnifCv}
Soit $f(x,t)\colon [\alpha,\beta]\times[a,\infty[ \to \eR$, une fonction dont la restriction à toute demi-droite $x=cst$ est mesurable. Si $| f(x,t) |< \varphi(t)$ et $\int_a^{\infty}\varphi(t)dt$ existe, alors l'intégrale
\begin{equation}
	\int_0^{\infty}f(x,t)dt
\end{equation}
est uniformément convergente.
\end{theorem}

Le théorème suivant est le \defe{critère d'Abel}{critère!Abel pour intégrales} :
\begin{theorem}		\label{ThoAbelIntUnif}
	Supposons que $f(x,t)=\varphi(x,t)\psi(x,t)$ où $\varphi$ et $\psi$ sont bornée et intégrables en $t$ au sens de Riemann sur tout compact $[a,b]$, $b\geq a$. Supposons que :
	\begin{enumerate}
		\item $| \int_a^{T}\varphi(x,t)dt |\leq M$ où $M$ est indépendant de $T$ et de $x$,
		\item $\psi(x,t)\geq 0$,
		\item pour tout $x\in[\alpha,\beta]$, $\psi(x,t)$ est une fonction décroissante de $t$,
		\item les fonctions $x\mapsto \psi(x,t)$ convergent uniformément vers $0$ lorsque $t\to\infty$.
	\end{enumerate}
	Alors l'intégrale
	\begin{equation}
		\int_a^{\infty}f(x,t)dt
	\end{equation}
	est uniformément convergente.
\end{theorem}

\begin{remark}
    Étant donné que la fonction sinus est bornée, il est tentant de l'utiliser comme $\varphi$ dans le critère d'Abel (théorème~\ref{ThoAbelIntUnif}). Hélas,
    \begin{equation}
        \int_0^T\sin(xt)=-\frac{ 1 }{ x }\big( \cos(xT)-\cos(x) \big),
    \end{equation}
    qui n'est pas bornée pour tout $x$ ! Poser $\varphi(x,t)=\sin(xt)$ \emph{ne fonctionne pas} pour assurer la convergence uniforme sur un intervalle qui contient des $x$ arbitrairement proches de $0$. Le critère d'Abel avec $\varphi(x,t)=\sin(xt)$ ne permet que de conclure à l'uniforme convergence \emph{sur tout compact} ne contenant pas $0$. Cela est toutefois souvent suffisant pour étudier la continuité ou la dérivabilité en se servant du fameux coup du compact.
\end{remark}

%+++++++++++++++++++++++++++++++++++++++++++++++++++++++++++++++++++++++++++++++++++++++++++++++++++++++++++++++++++++++++++
\section{Fonctions définies par une intégrale}
%+++++++++++++++++++++++++++++++++++++++++++++++++++++++++++++++++++++++++++++++++++++++++++++++++++++++++++++++++++++++++++
\label{SecCHwnBDj}
\index{suite!de fonctions intégrables}
\index{fonction!définie par une intégrale}
\index{permuter limite et intégrale}

Soit \( (\Omega,\mu)\) un espace mesuré. Nous nous demandons dans quel cas l'intégrale
\begin{equation}
    F(x)=\int_{\Omega}f(x,\omega)d\omega
\end{equation}
définit une fonction \( F\) continue, dérivable ou autre.

Dans la suite nous allons considérer des fonctions \( f\) à valeurs réelles. Quitte à passer aux composantes, nous pouvons considérer des fonctions à valeurs vectorielles. Par contre le fait que \( x\) soit dans \( \eR\) ou dans \( \eR^n\) n'est pas spécialement une chose facile à traiter.

%---------------------------------------------------------------------------------------------------------------------------
\subsection{Continuité sous l'intégrale}
%---------------------------------------------------------------------------------------------------------------------------
\index{continuité!fonction définie par une intégrale}

Nous allons présenter deux théorèmes donnant la continuité de \( F\).
\begin{enumerate}
    \item
        Si \( f\) est majorée par une fonction ne dépendant pas de \( x\), nous avons le théorème~\ref{ThoKnuSNd},
    \item
        si l'intégrale est uniformément convergente, nous avons le théorème~\ref{ThotexmgE}.
\end{enumerate}

\begin{theorem} \label{ThoKnuSNd}
    Soit \( (\Omega,\mu)\) est un espace mesuré, soit \( x_0\in \eR^m\) et \( f\colon U\times \Omega\to \eR\) où \( U\) est ouvert dans \( \eR^m\). Nous supposons que
    \begin{enumerate}
        \item
            Pour chaque \( x\in \eR^m\), la fonction \( \omega\mapsto f(x,\omega)\) est dans \( L^1(\Omega,\mu)\).
        \item
            Pour chaque \( \omega\in \Omega\), la fonction \( x\mapsto f(x,\omega)\) est continue en \( x_0\).
            %TODO : peut-être qu'on peut dire seulement pour presque tout omega dans Omega, voir la proposition~\ref{prop:fdefint}.
        \item       \label{ItemNAuYNG}
            Il existe une fonction \( G\in L^1(\Omega)\) telle que
            \begin{equation}
                | f(x,\omega) |\leq G(\omega)
            \end{equation}
            pour tout \( x\in U\).
    \end{enumerate}
    Alors la fonction
    \begin{equation}
        \begin{aligned}
            F\colon U&\to \eR \\
            x&\mapsto \int_{\Omega}f(x,\omega)d\mu(\omega)
        \end{aligned}
    \end{equation}
    est continue en \( x_0\).
\end{theorem}
\index{permuter!limite et intégrale!espace mesuré}

\begin{proof}
    Soit \( (x_n)\) une suite convergente vers \( x_0\). Nous considérons la suite de fonctions \( f_n\colon \Omega\to \eR\) définies par
    \begin{equation}
        f_n(\phi)=f(x_n,\omega).
    \end{equation}
    sur qui nous pouvons utiliser le théorème de la convergence dominée (théorème~\ref{ThoConvDomLebVdhsTf}) pour obtenir
    \begin{subequations}
        \begin{align}
            \lim_{n\to \infty} F(x_n)&=\lim_{n\to \infty} \int_{\Omega}f(x_n,\omega)d\mu(\omega)\\
            &=\int_{\Omega}\lim_{n\to \infty} f(x_n,\omega)d\mu(\omega)\\
            &=\int_{\Omega}f(x,\omega)d\mu(\omega)\\
            &=F(x).
        \end{align}
    \end{subequations}
    Nous avons utilisé la continuité de \( f(.,\omega)\).
\end{proof}


Si nous avons un peu de compatibilité entre la topologie et la mesure, alors nous pouvons utiliser l'uniforme convergence d'une intégrale pour obtenir la continuité d'une fonction définie par une intégrale.

\begin{theorem} \label{ThotexmgE}
    Soit \( (\Omega,\mu)\) un espace topologique mesuré tel que tout compact est de mesure finie. Soit une fonction \( f\colon \eR\times \Omega\to \eR\) telle que
    \begin{enumerate}
        \item
            Pour chaque \( x\in \eR\), la fonction \( f(x,.)\) est \( L^1(\Omega,\mu)\).
        \item
            Pour chaque \( \omega\in \Omega\), la fonction \( f(.,\omega)\) est continue en \( x_0\).
        \item
            L'intégrale
            \begin{equation}
                F(x)=\int_{\Omega}f(x,\omega)d\mu(\omega)
            \end{equation}
            est uniformément convergente\footnote{Définition~\ref{DEFooSHWAooWtswtp}.}.
    \end{enumerate}
    Alors la fonction \( F\) est continue en \( x_0\).
\end{theorem}
\index{permuter!limite et intégrale!espace mesuré}

\begin{proof}
    Nous reprenons les notations du lemme~\ref{LemOgQdpJ}. Les fonctions
    \begin{equation}
        F_k(x)=\int_{K_k}f(x,\omega)d\mu(\omega)
    \end{equation}
    existent parce que les fonctions \( f(x,.)\) sont dans \( L^1(\Omega)\). Montrons que les fonctions \( F_k\) sont continues. Soit une suite \( x_k\to x_0\) nous avons
    \begin{equation}
        \lim_{n\to \infty} F_k(x_n)=\lim_{n\to \infty} \int_{K_k}f(x_n,\omega)d\mu(\omega).
    \end{equation}
    Nous pouvons inverser la limite et l'intégrale en utilisant le théorème de la convergence dominée. Pour cela, la fonction \( f(x_n,\omega)\) étant continue sur le compact \( K_k\), elle y est majorée par une constante. Le fait que les compacts soient de mesure finie (hypothèse) implique que les constantes soient intégrales sur \( K_k\). Le théorème de la convergence dominée implique alors que
    \begin{equation}
        \lim_{n\to \infty} F_k(x_n)=\int_{K_k}\lim_{n\to \infty} f(x_n,\omega)d\mu(\omega)=\int_{K_k}f(x_0,\omega)d\mu(\omega)=F_k(x_0).
    \end{equation}
    Nous avons utilisé le fait que \( f(.,\omega)\) était continue en \( x_0\).

    Le lemme~\ref{LemOgQdpJ} nous indique alors que la convergence \( F_k\to F\) est uniforme. Les fonctions \( F_k\) étant continues, la fonction \( F\) est continue.
\end{proof}

Pour finir, citons ce résultat concernant les fonctions réelles.
\begin{theorem}		\label{ThoInDerrtCvUnifFContinue}
    Nous considérons \( F(x)=\int_a^{\infty}f(x,t)dt\). Si \( f\) est continue sur $[\alpha,\beta]\times[a,\alpha[$ et l'intégrale converge uniformément, alors $F(x)$ est continue.
\end{theorem}

%---------------------------------------------------------------------------------------------------------------------------
\subsection{Le coup du compact}
%---------------------------------------------------------------------------------------------------------------------------

Nous avons vu des fonctions définies par toute une série de processus de limite (suites, séries, intégrales). Une des questions centrales est de savoir si la fonction limite est continue, dérivable, intégrale, etc. étant donné que les fonctions sont continues.

Pour cela, nous inventons le concept de \emph{convergence uniforme}. Si la limite (série, intégrale) est uniforme, alors la fonction limite sera continue. Il arrive qu'une limite ne soit pas uniforme sur un intervalle ouvert $]0,1]$, et que nous voulions quand même prouver la continuité sur cet intervalle. C'est à cela que sert la notion de convergence uniforme \emph{sur tout compact}. En effet, la notion de continuité est une notion locale : savoir ce qu'il se passe dans un petit voisinage autour de $x$ est suffisant pour savoir la continuité en $x$ (idem pour sa dérivée).

Si nous avons uniforme convergence sur tout compact de $]0,1]$, mais pas uniforme convergence sur cet intervalle, la limite sera quand même continue sur $\mathopen] 0 , 1 \mathclose]$. En effet, si $x\in]0,1]$, il existe un ouvert autour de $x$ contenu dans un compact contenu dans $]0,1]$. L'uniforme convergence sur ce compact suffit à prouver la continuité en $x$.

Déduire la continuité sur un ouvert à partir de l'uniforme convergence sur tout compact de l'ouvert est appelé faire le \defe{coup du compact}{compact!le coup du}.


%---------------------------------------------------------------------------------------------------------------------------
\subsection{Dérivabilité sous l'intégrale}
%---------------------------------------------------------------------------------------------------------------------------
\index{dérivabilité!fonction définie par une intégrale}

Nous traitons à présent de la dérivabilité de la fonction \( F\) définie comme intégrale de \( f\).

\begin{theorem}[Dérivation sous le signe intégral, formule de Leibnitz\cite{MesIntProbb,BIBooSNRXooSXmdxt}]    \label{ThoMWpRKYp}
    Soit \( (\Omega,\mu)\) un espace mesuré et une fonction \( f\colon \eR\times \Omega\to \eR\) dont nous voulons étudier la dérivabilité en \(a\in \eR\). Nous supposons qu'il existe \( \delta>0\), \( A\) mesurable de mesure nulle dans \( \Omega\) tels que
    \begin{enumerate}
        \item   \label{ITEMooAFVMooAeCEco}
           Pour tout \( x\in \eR\), la fonction \( \omega\mapsto f(x,\omega)\) soit dans \( L^1(\Omega)\).
       \item       \label{ITEMooXIZXooGPYFyT}
            L'application \( x\mapsto f(x,\omega)\) est dérivable pour tout \( x\in B(a,\delta)\) et pour tout \( \omega\in \complement A\).
        \item   \label{ITEMooDTTIooWkldfB}
            Il existe une fonction \( G\) intégrable sur \( \Omega\) telle que
            \begin{equation}
                \left| \frac{ \partial f }{ \partial x }(x,\omega) \right| \leq G(\omega)
            \end{equation}
            pour tout \( x\in B(a,\delta)\) et pour tout \( \omega\in\complement A\).
    \end{enumerate}
    Alors la fonction
    \begin{equation}
        F(x)=\int_{\Omega}f(x,\omega)d\mu(\omega)
    \end{equation}
    est dérivable en \( a\) et nous pouvons permuter la dérivée et l'intégrale :
    \begin{equation}
        F'(a)=\int_{\Omega}\frac{ \partial f }{ \partial x }(a,\omega)d\mu(\omega).
    \end{equation}
\end{theorem}
\index{permuter!dérivée et intégrale!dans \( \eR\)}

\begin{proof}
    Soit une suite \( (x_n)\) dans \( B(a,\delta)\) telle que \( x_n\neq a\) et \( x_n\to a\). Si la limite
    \begin{equation}
        \lim_{n\to \infty} \frac{ F(a)-F(x_n) }{ a-x_n }
    \end{equation}
    existe et ne dépend pas de la suite choisie, alors la fonction \( F\) est dérivable en \( a\) et sa dérivée vaut cette limite. Autrement dit, nous nous mettons en devoir d'étudier la limite
    \begin{equation}    \label{EqLIiralx}
        \lim_{n\to \infty} \int_{\Omega}\frac{ f(a,\omega)-f(x_n,\omega) }{ a-x_n }d\omega.
    \end{equation}
    montrer qu'elle existe, ne dépend pas de la suite choisie et vaut \( \int_{\Omega}\partial_xf(a,\omega)d\omega\). On y va.

    \begin{subproof}
        \item[La bonne suite de fonctions]
            D'abord nous posons
            \begin{equation}    \label{EqAFOUbQB}
                g_n(\omega)=\frac{ f(x_n,\omega)-f(a,\omega) }{ x_n-a }.
            \end{equation}
            Nous montrons à présent que cette suite vérifie les hypothèses du théorème de la convergence dominée \ref{ThoConvDomLebVdhsTf}.
            \begin{itemize}
                \item
                    Chacune des fonctions \( g_n\) est dans \( L^1(\Omega)\) parce que, \( a\) étant fixé, l'élément \( x_n\) est dans \( B(a,\delta)\setminus\{ a \}\); le dénominateur n'a donc aucun rôle. L'hypothèse \ref{ITEMooAFVMooAeCEco} montre que \( \omega\mapsto f(x_n\omega)\) et \( \omega\mapsto f(a,\omega)\) sont dans \( L^1\). La somme est donc dans \( L^1\) (proposition \ref{PROPooFIYEooCpdmwZ}).
                \item
                    Par l'hypothèse \ref{ITEMooXIZXooGPYFyT}, pour chaque \( \omega\) nous avons une fonction dérivable. Nous pouvons donc passer à la limite :
                    \begin{equation}
                        \lim_{n\to \infty} g_n(x)=\frac{ \partial f }{ \partial x }(a,\omega).
                    \end{equation}

                \item
            
                    En ce qui concerne la majoration de \( g_n\), nous utilisons le théorème des accroissements finis \ref{ThoAccFinis}\ref{ITEMooXRQKooDBFpdQ}. Pour chaque \( \omega\), ce théorèmes peut être utilisé sur la fonction \( x\mapsto f(x,\omega)\) qui est dérivable. Nous avons :
                    \begin{equation}
                        \big| \frac{  f(x_n,\omega)-f(a,\omega)   }{ x_n-a  } \big|\leq \sup_{x\in \mathopen[ a , x_n \mathclose]}| \frac{ \partial f }{ \partial x }(x,\omega) |\leq G(\omega). 
                    \end{equation}
                    Nous avons utilisé l'hypothèse \ref{ITEMooDTTIooWkldfB}.
            \end{itemize}
            Les hypothèses de la convergence dominée sont satisfaites.
                
            \item[Convergence dominée]
                Le théorème de la convergence dominée de Lebesgue (théorème~\ref{ThoConvDomLebVdhsTf}) nous permet alors de calculer la limite \eqref{EqLIiralx} :
                \begin{equation}
                    \lim_{n\to \infty} \int_{\Omega}g_n(\omega)d\omega=\int_{\Omega}\lim_{n\to \infty} g_n(\omega)d\omega=\int_{\Omega}\frac{ \partial f }{ \partial x }(a,\omega)d\omega.
                \end{equation}
                Notons que l'existence de la dernière intégrale fait partie du théorème de la convergence dominée.

                Nous avons donc prouvé que la limite de gauche existait et ne dépendant pas de la suite choisie. Donc \( F\) est dérivable en \( a\) et la dérivée vaut cette limite :
                \begin{equation}
                    F'(a)=\int_{\Omega}\frac{ \partial f }{ \partial x }(a,\omega)d\mu(\omega).
                \end{equation}
    \end{subproof}
\end{proof}

\begin{theorem}
		Supposons $f$ continue et sa dérivée partielle $\frac{ \partial f }{ \partial x }$ continue sur $[\alpha,\beta]\times[a,\alpha[$. Supposons que $F(x)=\int_a^{\infty}f(x,t)dt$ converge et que $\int_a^{\infty}\frac{ \partial f }{ \partial x }dt$ converge uniformément. Alors $F$ est $C^1$ sur $[\alpha,\beta]$ et
		\begin{equation}
			\frac{ dF }{ dx }=\int_a^{\infty}\frac{ \partial f }{ \partial x }dt.
		\end{equation}
\end{theorem}

En ce qui concerne les fonctions dans \( \eR^n\), il y a les  propositions~\ref{PropDerrSSIntegraleDSD} et~\ref{PropAOZkDsh} qui parlent de différentiabilité sous l'intégrale.

%---------------------------------------------------------------------------------------------------------------------------
\subsection{Absolue continuité}
%---------------------------------------------------------------------------------------------------------------------------

\begin{definition}      \label{DefAbsoluCont}
    Une fonction \( F\colon \eR\to \eR\) est \defe{absolument continue}{absolument continue} sur \( \mathopen[ a , b \mathclose]\) s'il existe une fonction \( f\) sur \( \mathopen[ a , b \mathclose]\) telle que
    \begin{equation}
        F(x)=\int_a^xf(t)dt
    \end{equation}
    pour tout \( x\in\mathopen[ a , b \mathclose]\).
\end{definition}

\begin{theorem}     \label{ThoDerSousIntegrale}
    Soient \( A\) un ouvert de \( \eR\) et \( \Omega\) un espace mesuré. Soient une fonction \( f\colon A\times \Omega\to \eR\) et
    \begin{equation}
        F(x)=\int_{\Omega}f(x,\omega)d\omega.
    \end{equation}
    Nous supposons les points suivants.
    \begin{enumerate}
        \item
            La fonction \( f\) est mesurable en tant que fonction \( A\times\Omega\to \eR\). Pour chaque \( x\in A\), la fonction \( f(x,\cdot)\) est intégrable sur \( \Omega\).
        \item
            Pour presque tout \( \omega\in\Omega\), la fonction \( f(x,\omega)\) est une fonction absolument continue de \( x\).
        \item
            La fonction \( \frac{ \partial f }{ \partial x }\) est localement intégrable, c'est-à-dire que pour tout \( \mathopen[ a , b \mathclose]\subset A\),
            \begin{equation}
                \int_a^b\int_{\Omega}\left| \frac{ \partial f }{ \partial x }(x,\omega) \right| d\omega\,dx<\infty.
            \end{equation}
    \end{enumerate}
    Alors la fonction \( F\) est absolument continue et pour presque tout \( x\in A\), la dérivée est donné par
    \begin{equation}
        \frac{ d }{ dx }\int_{\Omega}f(x,\omega)d\omega=\int_{\Omega}\frac{ \partial f }{ \partial x }(x,\omega)d\omega.
    \end{equation}
\end{theorem}

La proposition suivante sera utilisée entre autres pour montrer que sous l'hypothèse d'une densité continue, la loi exponentielle est sans mémoire, proposition~\ref{PropREXaIBg}.
\begin{proposition}		\label{PropDerrFnAvecBornesFonctions}
Soit $f(x,t)$ une fonction continue sur $[\alpha,\beta]\times[a,b]$, telle que $\frac{ \partial f }{ \partial x }$ existe et soit continue sur $]\alpha,\beta[\times[a,b]$. Soient $\varphi(x)$ et $\psi(x)$, des fonctions continues de $[\alpha,\beta]$ dans $\eR$ et admettant une dérivée continue sur $]\alpha,\beta [$. Alors la fonction
\begin{equation}
	F(x)=\int_{\varphi(x)}^{\psi(x)}f(x,t)dt
\end{equation}
admet une dérivée continue sur $]\alpha,\beta[$ et
\begin{equation}	\label{EqFormDerrFnAvecBorneNInt}
	\frac{ dF }{ dx }=\int_{\varphi(x)}^{\psi(x)}\frac{ \partial f }{ \partial x }(x,t)dt+f\big( x,\psi(x) \big)\cdot\frac{ d\psi }{ dx }- f\big( x,\varphi(x) \big)\cdot\frac{ d\varphi }{ dx }.
\end{equation}
\end{proposition}
\index{permuter!dérivée et intégrale!dans \( \eR\) avec les bornes}
%TODO : une preuve de ce théorème ? allons allons ...

L'exemple qui suit devrait pouvoir être rendu rigoureux en utilisant des distributions correctement.

\begin{example} \label{ExfYXeQg}
    Si \( g\) est une fonction continue, la fonction suivante est une primitive de \( g\) :
    \begin{equation}
        \int_0^xf(t)dt=\int_0^{\infty}f(t)\mtu_{t<x}(t)dt.
    \end{equation}
    Nous nous proposons de justifier \emph{de façon un peu heuristique} le fait que ce soit bien une primitive de \( g\) en considérant la fonction
    \begin{equation}
        f(t,x)=g(t)\mtu_{t<x}(t).
    \end{equation}
    Nous posons
    \begin{equation}
        F(x)=\int_0^{\infty}f(x,t)dt,
    \end{equation}
    et nous calculons \( F'\) en permutant la dérivée et l'intégrale\footnote{Ceci n'est pas rigoureux : il faudrait avoir un théorème à propos de distributions qui permet de le faire.}. D'abord,
    \begin{equation}
        f(t,x)=\begin{cases}
            g(t)    &   \text{si } t\in \mathopen[ 0 , x \mathclose]\\
            0    &    \text{sinon.}
        \end{cases}
    \end{equation}
    La dérivée de \( f\) par rapport à \( x\) est donnée par la distribution
    \begin{equation}
        \frac{ \partial f }{ \partial x }(t_0,x_0)=g(t_0)\delta(t_0-x_0).
    \end{equation}
    Donc
    \begin{equation}
        F'(x_0)=\int_0^{\infty}\frac{ \partial f }{ \partial x }(t,x_0)dt=\int_0^{\infty}g(t)\delta(t-x_0)=g(x_0),
    \end{equation}
    comme attendu.
\end{example}

Cet exemple est rendu rigoureux par la proposition suivante.
\begin{proposition} \label{PropJLnPpaw}
    Si \( f\in L^1(\eR)\), alors la fonction
    \begin{equation}
        F(x)=\int_{-\infty}^xf(t)dt
    \end{equation}
    est presque partout dérivable et pour les points où elle l'est nous avons \( F'(x)=f(x)\).
\end{proposition}
\index{fonction!définie par une intégrale}
%TODO : une preuve.

%---------------------------------------------------------------------------------------------------------------------------
\subsection{Différentiabilité sous l'intégrale}
%---------------------------------------------------------------------------------------------------------------------------

Le théorème suivant est restrictif sur l'ensemble d'intégration (qui doit être compact), mais accepte des fonctions de plusieurs variables, ce qui est un premier pas vers la différentiabilité.
\begin{proposition}[Dérivation sous l'intégrale]		\label{PropDerrSSIntegraleDSD}
    Supposons $A\subset\eR^m$ ouvert et $B\subset\eR^n$ compact. Nous considérons une fonction \( f\colon A\times B\to \eR\). Si pour un $i\in\{ i,\ldots,n \}$, la dérivée partielle $\frac{ \partial f }{ \partial x_i }$ existe dans $A\times B$ et est continue, alors la fonction
    \begin{equation}
        F(x)=\int_Bf(x,t)dt
    \end{equation}
    admet une dérivée partielle dans la direction \( x_i\) sur \( A\). Cette dérivée partielle y est continue et
    \begin{equation}
        \frac{ \partial F }{ \partial x_i }(a)=\int_B\frac{ \partial f }{ \partial x_i }(a,t)dt,
    \end{equation}
    pour tout \( a\) dans l'ouvert \( A\).
\end{proposition}
\index{fonction!définie par une intégrale}
\index{permuter!dérivée et intégrale!\( \eR^n\)}

\begin{proof}
    Nous procédons en plusieurs étapes.
    \begin{subproof}
    \item[\( F\) est dérivable]

        Nous voulons prouver que \( \frac{ \partial F }{ \partial x_i }(a,t)\) existe. Pour cela nous posons
        \begin{equation}
            g_l(t)=\frac{ f(a_1,\ldots, a_i+\epsilon_l,\ldots, a_n,t)-f(a_1,\ldots, a_i,\ldots, a_n,t) }{ \epsilon_l }
        \end{equation}
        où \( \epsilon_l\) est une suite de nombres tendant vers zéro. La fonction \( f\) est dérivable dans la direction \( x_i\) si et seulement si \( \lim_{l\to \infty}g_l(t) \) existe et ne dépend pas du choix de la suite. À ce moment, la valeur de la dérivée partielle sera cette limite. Dans notre cas, nous savons que \( f\) admet une dérivée partielle dans la direction \( x_i\) et donc nous avons
        \begin{equation}
            \frac{ \partial f }{ \partial x_i }(a,t)=\lim_{l\to \infty} g_l(t).
        \end{equation}

        De la même façon pour \( F\) nous avons
        \begin{equation}
            \frac{ \partial F }{ \partial x_i }=\lim_{l\to \infty} \int_{B}g_l(t)dt.
        \end{equation}
        Sous-entendu : si la limite de droite ne dépend pas de la suite choisie, alors \( \frac{ \partial F }{ \partial x_i }\) existe et vaut cette limite.

        Vu la continuité de \( f\), le seul point à vérifier pour le théorème de la convergence dominée de Lebesgue est l'existence d'une fonction intégrable de \( t\) majorant \( g_l\). Pour cela le théorème de accroissements finis (théorème~\ref{ThoAccFinis}) appliqué à la fonction \( \epsilon\mapsto f(a_n,\ldots, a_i+\epsilon,\ldots, a_n)\) nous dit que
        \begin{equation}
            f(a_1,\ldots, a_i+\epsilon_l,\ldots, a_n,t)-f(a_1,\ldots, a_i,\ldots, a_n,t)=\epsilon_l\frac{ \partial f }{ \partial x_i }(a_1,\ldots, \theta,\ldots, a_n,t)
        \end{equation}
        pour un certain \( \theta\in B(a_i,\epsilon_l)\). Notons que ce \( \theta\) dépend de \( t\) mais pas de \( l\). Vu que \( \partial_if\) est continue par rapport à ses deux variables, si \( K\) est un voisinage compact autour de \( a\), il existe \( M>0\) tel que
        \begin{equation}    \label{EqMXqviPC}
            \left| \frac{ \partial f }{ \partial x_i }(x,t) \right| < M
        \end{equation}
        pour tout \( x\in K\) et tout \( t\in B\). La valeur de \( \frac{ \partial f }{ \partial x_i }(a_1,\ldots, \theta,\ldots, a_n,t)\) est donc bien majorée par rapport à \( \theta\) et par rapport à \( t\) en même temps par une constante qui n'a pas de mal à être intégrée sur le compact \( B\).

        Le théorème de la convergence dominée (théorème~\ref{ThoConvDomLebVdhsTf}) s'applique donc bien et nous avons
        \begin{equation}
            \lim_{l\to \infty} \int_Bg_l(t)dt=\int_B\lim_{l\to \infty} g_l(t)=\int_B\frac{ \partial f }{ \partial x_i }(a,t)dt.
        \end{equation}
        Le membre de droite ne dépendant pas de la suite \( \epsilon_l\) choisie, le membre de gauche est bien la dérivée de \( F\) par rapport à \( x_i\) et nous avons
        \begin{equation}
            \frac{ \partial F }{ \partial x_i }(a)=\int_B\frac{ \partial f }{ \partial x_i }(a,t)dt.
        \end{equation}
        Cela prouve la première partie de la proposition.

    \item[La dérivée est continue]

        Soit \( K\) un voisinage compact autour de \( a\) et \( U'\) un ouvert tel que \( a\in U'\subset K\). Nous avons encore la majoration \eqref{EqMXqviPC} sur \( U'\) et donc le théorème de continuité sous l'intégrale~\ref{ThoKnuSNd} nous indique que la fonction
        \begin{equation}
            \begin{aligned}
                U'&\to \eR \\
                x&\mapsto \int_{B}\frac{ \partial f }{ \partial x_i }(x,t)dt
            \end{aligned}
        \end{equation}
        est continue en \( a\).

    \end{subproof}
\end{proof}

Une conséquence de la proposition~\ref{PropDerrSSIntegraleDSD} est que si elle fonctionne pour tous les \( i\), alors \( F\) est différentiable et même de classe \( C^1\), et la différentielle de \( F\) s'obtient comme intégrale de la différentielle de \( f\).

\begin{proposition}\label{PropAOZkDsh}
    Supposons $A\subset\eR^m$ ouvert et $B\subset\eR^n$ compact. Si pour tout $i\in\{ i,\ldots,n \}$, la dérivée partielle $\frac{ \partial f }{ \partial x_i }$ existe dans $A\times B$ et est continue, alors \( F\) est de classe \( C^1\) et
    \begin{equation}
        (dF)_a=\int_B(df_t)_adt
    \end{equation}
    où \( f_t(x)=f(x,t)\).
\end{proposition}
\index{permuter!différentielle et intégrale!\( \eR^n\)}

\begin{proof}
    En vertu de la proposition~\ref{PropDerrSSIntegraleDSD}, toutes les dérivées partielles de \( F\) sont continues. Cela implique que \( F\) est de classe \( C^1\) par le théorème \ref{THOooBEAOooBdvOdr} et que la différentielle s'écrive en termes des dérivées partielles avec la formule usuelle. Nous avons alors
    \begin{subequations}
        \begin{align}
            (dF)_a(u)&=\sum_k\frac{ \partial F }{ \partial x_k }(a)u_k\\
            &=\int_B\sum_k\frac{ \partial f }{ \partial x_k }(a,t)dt\\
            &=\int_B\sum_k\frac{ \partial f_t }{ \partial x_k }(a)u_kdt\\
            &=\int_B (df_t)_a(u)dt.
        \end{align}
    \end{subequations}
    Cela est la formule annoncée.
\end{proof}

Un autre théorème tourne autour du pot, et me semble inutile.
\begin{theorem} \label{ThoOLAQyRL}
    Soit \( (\Omega,\mu)\) un espace mesuré, une fonction \( f\colon \eR^n\times \Omega\to \eR\) et \( a\in \eR^n\). Nous considérons la fonction
    \begin{equation}
        F(x)=\int_{\Omega}f(x,\omega)d\mu(\omega).
    \end{equation}
    Pour chaque \( k=1,\ldots, n\) nous supposons avoir
    \begin{equation}
        \frac{ \partial F }{ \partial x_k }(a)=F_{|_k}'(a)=\int_{\Omega}\frac{ \partial f_{|_k} }{ \partial t }(a_k,\omega)d\mu(\omega)
    \end{equation}
    où \( F_{|_k}(t)=F(a_1,\ldots, t,\ldots, a_n)\) et \( f_{|_k}\) est définie de façon similaire.

    Nous supposons de plus que les fonctions \( \partial_{x_k}F\) sont continues.

    Alors \( F\) est de classe \( C^1\) et sa différentielle est donnée par
    \begin{equation}
        df_a=\int_{\Omega}(df_{\omega})_ad\omega
    \end{equation}
    où \( f_{\omega}\) est définie par \( f_{\omega}(x)=f(x,\omega)\).
\end{theorem}

\begin{proof}
    Étant donné que les dérivées partielles de \( F\) en \( a\) existent et sont continues, le théorème \ref{THOooBEAOooBdvOdr} dit que \( F\) est différentiable et que
    \begin{equation}
        dF_a(u)=\sum_{k=1}^n\frac{ \partial F }{ \partial x_k }(a)u_k.
    \end{equation}
    La linéarité de l'intégrale et les hypothèses nous donnent alors
    \begin{subequations}
        \begin{align}
            df_a(u)&=\sum_{k=1}^n\frac{ \partial F }{ \partial x_k }(a)u_k\\
            &=\int_{\Omega}\sum_k\frac{ \partial f_{|_k} }{ \partial t }(a_k;\omega)u_kd\mu(\omega)\\
            &=\int_{\Omega}\sum_k\frac{ \partial f }{ \partial x_k }(a;\omega)u_kd\mu(\omega)\\
            &=\int_{\Omega}(df_{\omega})_a(u)d\mu(\omega),
        \end{align}
    \end{subequations}
    et donc \( df_a=\int_{\Omega}(df_{\omega})_ad\mu(\omega)\).
\end{proof}
Notons qu'en passant aux composantes, ce théorème fonctionne tout aussi bien pour des fonctions à valeurs dans un espace vectoriel normé de dimension finie plutôt que dans \( \eR\).

\begin{lemma}[Hadamard\cite{MVIooKLsjpa}]   \label{LemWNBooGPlIwT}
    Soit une fonction \( f\colon \eR^n\to \eR\) de classe \( C^p\) avec \( p\geq 1\). Pour tout \( a\in \eR^n\) il existe des fonctions \( g_1\),\ldots, \( g_n\) de classe \( C^{p-1}\) telles que
    \begin{equation}
        f(x)=f(a)+\sum_{i=1}^n(x_i-a_i)g_i(x).
    \end{equation}
\end{lemma}
\index{lemme!Hadamard}

\begin{proof}
    Vu que \( f\) est de classe \( C^1\), le théorème fondamental de l'analyse~\ref{ThoRWXooTqHGbC} fonctionne et
    \begin{equation}    \label{EqZLTooVKmGln}
        f(x)-f(a)=\int_0^1\frac{ d }{ dt }\Big[ f\big( a+t(x-a) \big) \Big]dt=\int_0^1\sum_{i=1}^n\frac{ \partial f }{ \partial x_i }\big( a+t(x-a) \big)(x_i-a_i).
    \end{equation}
    Plus de détails : la fonction \( t\mapsto \frac{ d }{ dt }\Big[ f\big( a+t(x-a) \big) \Big]\) possède comme primitive la fonction \( F(t)=f\big( a+t(x-a) \big)\).

    Nous posons
    \begin{equation}
        g_i(x)=\int_0^1\frac{ \partial f }{ \partial x_i }\big( a+t(x-a) \big)dt
    \end{equation}
    Le fait que l'intégrale existe est simplement le fait qu'il s'agit d'une fonction continue sur un compact et donc majorée par une constante. Pour voir que \( g_i\) est de classe \( C^{p-1}\) nous pouvons calculer \( \frac{ \partial g_i }{ \partial x_k }\) en permutant dérivée et intégrale par la proposition~\ref{PropDerrSSIntegraleDSD} :
    \begin{equation}
        \frac{ \partial g_i }{ \partial x_k }(x)=\int_0^1\frac{ \partial  }{ \partial x_k }\left( \frac{ \partial f }{ \partial x_i }\big( a+t(x-a) \big) \right)dt=\int_0^1 t\frac{ \partial^2f }{ \partial x_k\partial x_i }\big( a+t(x-a) \big).
    \end{equation}
    Nous pouvons ainsi permuter \( p-1\) dérivées tout en gardant une fonction continue dans l'intégrale. Le théorème~\ref{ThoKnuSNd} nous donne alors une fonction continue. Ainsi toutes les fonctions
    \begin{equation}
        \frac{ \partial^{p-1}g_i }{ \partial x_{i_1}\ldots\partial x_{i_{p-1}} }
    \end{equation}
    sont continues et \( g_i\) est de classe \( C^{p-1}\) par le théorème \ref{THOooPZTAooTASBhZ}.

    En repartant de \eqref{EqZLTooVKmGln} nous avons alors bien ce qui était annoncé :
    \begin{equation}
        f(x)=f(a)+\sum_{i=1}^ng_i(x)(x_i-a_i).
    \end{equation}
\end{proof}

\begin{corollary}       \label{CorQBXHooZVKeNG}
    Soit \( \phi\in\swD(\eR)\) tel que \( \phi^{(k)}(x_0)=0\) pour tout \( k\leq n\). Alors il existe une fonction \( \psi\in\swD(\eR)\) telle que
    \begin{equation}
        \phi(x)=(x-x_0)^{n+1}\psi(x)
    \end{equation}
    pour tout \( x\in \eR\).
\end{corollary}

\begin{proof}
    En utilisant le lemme de Hadamard~\ref{LemWNBooGPlIwT} avec \( a=x_0\), \( n=1\) et \( f(x_0)=0\), nous avons une fonction \( g_1\) à support compact telle que
    \begin{equation}        \label{EqTOJGooWZBBRJ}
        \phi(x)=\phi(x_0)+(x-x_0)g_1(x).
    \end{equation}
    Alors \( \phi'(x)=g_1(x)+(x-x_0)g'_1(x)\), ce qui donne immédiatement \( g_1(x_0)=0\) et donc une fonction \( g_2\) telle que \( g_1(x)=(x-x_0)g_2(x)\). En injectant dans \eqref{EqTOJGooWZBBRJ} nous avons
    \begin{equation}
        \phi(x)=(x-x_0)^2g_2(x).
    \end{equation}
    Il suffit de continuer ainsi tant que les dérivées de \( \phi\) s'annulent.
\end{proof}

%+++++++++++++++++++++++++++++++++++++++++++++++++++++++++++++++++++++++++++++++++++++++++++++++++++++++++++++++++++++++++++
\section{Deux théorèmes de point fixe}
%+++++++++++++++++++++++++++++++++++++++++++++++++++++++++++++++++++++++++++++++++++++++++++++++++++++++++++++++++++++++++++

Nous allons voir Picard. Les autres théorème de point fixe que sont Brouwer, Schauder et Markov-Kakutani sont plus bas\footnote{Dans la section \ref{SECooDWMPooWZgzRZ}.} parce qu'ils utilisent de l'intégration. Voir le thème \ref{THEMEooWAYJooUSnmMh} pour les retrouver.

%---------------------------------------------------------------------------------------------------------------------------
\subsection{Points fixes attractifs et répulsifs}
%---------------------------------------------------------------------------------------------------------------------------

\begin{definition}      \label{DEFooTMZUooMoBDGC}
    Soit \( I\) un intervalle fermé de \( \eR\) et \( \varphi\colon I\to I\) une application \( C^1\). Soit \( a\) un point fixe de \( \varphi\). Nous disons que \( a\) est \defe{attractif}{point fixe!attractif}\index{attractif!point fixe} s'il existe un voisinage \( V\) de \( a\) tel que pour tout \( x_0\in V\) la suite \( x_{n+1}=\varphi(x_n)\) converge vers \( a\). Le point \( a\) sera dit \defe{répulsif}{répulsif!point fixe} s'il existe un voisinage \( V\) de \( a\) tel que pour tout \( x_0\in V\) la suite \( x_{n+1}=\varphi(x_n)\) diverge.
\end{definition}

\begin{lemma}[\cite{DemaillyNum}]
    Soit \( a\) un point fixe de \( \varphi\).
    \begin{enumerate}
        \item
    Si \( | \varphi'(a) |<1\) alors \( a\) est attractif et la convergence est au moins exponentielle.
\item
    Si \( | \varphi'(a) |>1\) alors \( a\) est répulsif et la divergence est au moins exponentielle.
    \end{enumerate}
\end{lemma}

\begin{proof}
    Si \( | \varphi'(a)<1 |\) alors il existe \( k\) tel que \( | \varphi'(a) |<k<1\) et par continuité il existe un voisinage \( V\) de \( a\) dans lequel \( | \varphi'(x) |<k\) pour tout \( x\in V\). En utilisant le théorème des accroissements finis nous avons
    \begin{equation}
        | x_n-a |=\big| f(x_{n-1}-a) \big|\leq k| x_{n-1}-a |
    \end{equation}
    et par récurrence
    \begin{equation}
        | x_n-a |\leq k^n| x_0-a |.
    \end{equation}

    Le cas \( | \varphi'(a)>1 |\) se traite de façon similaire.
\end{proof}

\begin{remark}
    Dans le cas \(| \varphi'(a) |=1\), nous ne pouvons rien conclure. Si \( \varphi(x)=\sin(x)\) nous avons \( \sin(x)<x\) et le point \( a=0\) est attractif. A contrario, si \( \varphi(x)=\sinh(x)\) nous avons \( |\sinh(x)|>|x|\) et le point \( a=0\) est répulsif.
\end{remark}

%---------------------------------------------------------------------------------------------------------------------------
\subsection{Picard}
%---------------------------------------------------------------------------------------------------------------------------

\begin{definition}      \label{DEFooRSLCooAsWisu}
    Une application \( f\colon (X,\| . \|_X)\to (Y,\| . \|_Y)\) entre deux espaces métriques est une \defe{contraction}{contraction} si elle est \( k\)-\defe{Lipschitz}{Lipschitz} pour un certain \( 0\leq k<1\), c'est-à-dire si pour tout \( x,y\in X\) nous avons
    \begin{equation}
        \| f(x)-f(y) \|_Y\leq k\| x-y \|_{X}.
    \end{equation}
\end{definition}

\begin{theorem}[Picard \cite{ClemKetl,NourdinAnal}\footnote{Il me semble qu'à la page 100 de \cite{NourdinAnal}, l'hypothèse H1 qui est prouvée ne prouve pas Hn dans le cas \( n=1\). Merci de m'écrire si vous pouvez confirmer ou infirmer. La preuve donnée ici ne contient pas cette «erreur».}.]     \label{ThoEPVkCL}
    Soit \( X\) un espace métrique complet et \( f\colon X\to X\) une application contractante, de constante de Lipschitz \( k\). Alors \( f\) admet un unique point fixe, nommé \( \xi\). Ce dernier est donné par la limite de la suite définie par récurrence
    \begin{subequations}
        \begin{numcases}{}
            x_0\in X\\
            x_{n+1}=f(x_n).
        \end{numcases}
    \end{subequations}
    De plus nous pouvons majorer l'erreur par
    \begin{equation}    \label{EqKErdim}
        \| x_n-x \|\leq \frac{ k^n }{ 1-k }\| x_n-x_{n-1} \|\leq \frac{ k^n }{ 1-k }\| x_1-x_0 \|.
    \end{equation}

    Soit \( r>0\), \( a\in X\) tels que la fonction \( f\) laisse la boule \( K=\overline{ B(a,r) }\) invariante (c'est-à-dire que \( f\) se restreint à \( f\colon K\to K\)). Nous considérons les suites \( (u_n)\) et \( (v_n)\) définies par
    \begin{subequations}
        \begin{numcases}{}
            u_0=v_0\in K\\
            u_{n+1}=f(v_n), v_{n+1}\in B(u_n,\epsilon).
        \end{numcases}
    \end{subequations}
    Alors le point fixe \( \xi\) de \( f\) est dans \( K\) et la suite \( (v_n)\) satisfait l'estimation
    \begin{equation}
        \| v_n-\xi \|\leq \frac{ k^n }{ 1-k }\| u_1-u_0 \|+\frac{ \epsilon }{ 1-k }.
    \end{equation}
\end{theorem}
\index{théorème!Picard}
\index{point fixe!Picard}

La première inégalité \eqref{EqKErdim} donne une estimation de l'erreur calculable en cours de processus; la seconde donne une estimation de l'erreur calculable avant de commencer.

\begin{proof}

    Nous commençons par l'unicité du point fixe. Si \( a\) et \( b\) sont des points fixes, alors \( f(a)=a\) et \( f(b)=b\). Par conséquent
    \begin{equation}
        \| f(a)-f(b) \|=\| a-b \|,
    \end{equation}
    ce qui contredit le fait que \( f\) soit une contraction.

    En ce qui concerne l'existence, notons que si la suite des \( x_n\) converge dans \( X\), alors la limite est un point fixe. En effet en prenant la limite des deux côtés de l'équation \( x_{n+1}=f(x_n)\), nous obtenons \( \xi=f(\xi)\), c'est-à-dire que \( \xi\) est un point fixe de \( f\). Notons que nous avons utilisé ici la continuité de \( f\), laquelle est une conséquence du fait qu'elle soit Lipschitz. Nous allons donc porter nos efforts à prouver que la suite est de Cauchy (et donc convergente parce que \( X\) est complet). Nous commençons par prouver que \( \| x_{n+1}-x_n \|\leq k^n\| x_0-x_1 \|\). En effet pour tout \( n\) nous avons
    \begin{equation}
        \| x_{n+1}-x_n \|=\| f(x_n)-f(x_{n-1}) \|\leq k\| x_n-x_{n-1} \|.
    \end{equation}
    La relation cherchée s'obtient alors par récurrence. Soient \( q>p\). En utilisant une somme télescopique,
    \begin{subequations}
        \begin{align}
            \| x_q-x_p \|&\leq \sum_{l=p}^{q-1}\| x_{l+1}-x_l \|\\
            &\leq\left( \sum_{l=p}^{q-1}k^l \right)\| x_1-x_0 \|\\
            &\leq\left(\sum_{l=p}^{\infty}k^l\right)\| x_1-x_0 \|.
        \end{align}
    \end{subequations}
    Étant donné que \( k<1\), la parenthèse est la queue d'une série qui converge, et donc tend vers zéro lorsque \( p\) tend vers l'infini.

    En ce qui concerne les inégalités \eqref{EqKErdim}, nous refaisons une somme télescopique :
    \begin{subequations}
        \begin{align}
            \| x_{n+p}-x_n \|&\leq \| x_{n+p}-x_{n+p-1} \|+\cdots +\| x_{n+1}-x_n \|\\
            &\leq k^p\| x_n-x_{n-1} \|+k^{p-1}\| x_n-x_{n-1} \|+\cdots +k\| x_n-x_{n-1} \|\\
            &=k(1+\cdots +k^{p-1})\| x_n-x_{n-1}\|  \\
            &\leq \frac{ k }{ 1-k }\| x_n-x_{n-1} \|.
        \end{align}
    \end{subequations}
    En prenant la limite \( p\to \infty\) nous trouvons
    \begin{equation}        \label{EqlUMVGW}
        \| \xi-x_n \|\leq \frac{ k }{ 1-k }\| x_n-x_{n-1} \|\leq \frac{ k }{ 1-k }\| x_1-x_0 \|.
    \end{equation}

    Nous passons maintenant à la seconde partie du théorème en supposant que \( f\) se restreigne en une fonction \( f\colon K\to K\). D'abord \( K\) est encore un espace métrique complet, donc la première partie du théorème s'y applique et \( f\) y a un unique point fixe.

    Nous allons montrer la relation par récurrence. Tout d'abord pour \( n=1\) nous avons
    \begin{equation}
        \| v_1-\xi \|\leq\| v_1-u_1 \|+\| u_1-\xi \|\leq \epsilon+\frac{ k }{ 1-k }\| u_1-u_0 \|
    \end{equation}
    où nous avons utilisé l'estimation \eqref{EqlUMVGW}, qui reste valable en remplaçant \( x_1\) par \( u_1\)\footnote{Elle n'est cependant pas spécialement valable si on remplace \( x_n\) par \( u_n\).}. Nous pouvons maintenant faire la récurrence :
    \begin{subequations}
        \begin{align}
            \| v_{n+1}-\xi \|&\leq \| v_{n+1}-u_{n+1} \|+\| u_{n+1}-\xi \|\\
            &\leq \epsilon+k\| v_n-\xi \|\\
            &\leq \epsilon+k\left( \frac{ k^n }{ 1-k }\| u_1-u_0 \|+\frac{ \epsilon }{ 1-k } \right)\\
            &=\frac{ \epsilon }{ 1-k }+\frac{ k^{n+1} }{ 1-k }\| u_1-u_0 \|.
        \end{align}
    \end{subequations}
\end{proof}

\begin{remark}
    Ce théorème comporte deux parties d'intérêts différents. La première partie est un théorème de point fixe usuel, qui sera utilisé pour prouver l'existence de certaines équations différentielles.

    La seconde partie est intéressante d'un point de vie numérique. En effet, ce qu'elle nous enseigne est que si à chaque pas de calcul de la récurrence \( x_{n+1}=f(x_n)\) nous commettons une erreur d'ordre de grandeur \( \epsilon\), alors le procédé (la suite \( (v_n)\)) ne converge plus spécialement vers le point fixe, mais tend vers le point fixe avec une erreur majorée par \( \epsilon/(k-1)\).
\end{remark}

\begin{remark}
Au final l'erreur minimale qu'on peut atteindre est de l'ordre de \( \epsilon\). Évidemment si on commet une faute de calcul de l'ordre de \( \epsilon\) à chaque pas, on ne peut pas espérer mieux.
\end{remark}

\begin{remark}  \label{remIOHUJm}
    Si \( f\) elle-même n'est pas contractante, mais si \( f^p\) est contractante pour un certain \( p\in \eN\) alors la conclusion du théorème de Picard reste valide et \( f\) a le même unique point fixe que \( f^p\). En effet nommons \( x\) le point fixe de \( f\) : \( f^p(x)=x\). Nous avons alors
    \begin{equation}
        f^p\big( f(x) \big)=f\big( f^p(x) \big)=f(x),
    \end{equation}
    ce qui prouve que \( f(x)\) est un point fixe de \( f^p\). Par unicité nous avons alors \( f(x)=x\), c'est-à-dire que \( x\) est également un point fixe de \( f\).
\end{remark}

Si la fonction n'est pas Lipschitz mais presque, nous avons une variante.
\begin{proposition}
    Soit \( E\) un ensemble compact\footnote{Notez cette hypothèse plus forte} et si \( f\colon E\to E\) est une fonction telle que
    \begin{equation}        \label{EqLJRVvN}
        \| f(x)-f(y) \|< \| x-y \|
    \end{equation}
    pour tout \( x\neq y\) dans \( E\) alors \( f\) possède un unique point fixe.
\end{proposition}

\begin{proof}
    La suite \( x_{n+1}=f(x_n)\) possède une sous-suite convergente. La limite de cette sous-suite est un point fixe de \( f\) parce que \( f\) est continue. L'unicité est due à l'aspect strict de l'inégalité \eqref{EqLJRVvN}.
\end{proof}

\begin{theorem}[Équation de Fredholm]\index{Fredholm!équation}\index{équation!Fredholm}     \label{ThoagJPZJ}
    Soit \( K\colon \mathopen[ a , b \mathclose]\times \mathopen[ a , b \mathclose]\to \eR\) et \( \varphi\colon \mathopen[ a , b \mathclose]\to \eR\), deux fonctions continues. Alors si \( \lambda\) est suffisamment petit, l'équation
    \begin{equation}
        f(x)=\lambda\int_a^bK(x,y)f(y)dy+\varphi(x)
    \end{equation}
    admet une unique solution qui sera de plus continue sur \( \mathopen[ a , b \mathclose]\).
\end{theorem}

\begin{proof}
    Nous considérons l'ensemble \( \mF\) des fonctions continues \( \mathopen[ a , b \mathclose]\to\mathopen[ a , b \mathclose]\) muni de la norme uniforme. Le lemme~\ref{LemdLKKnd} implique que \( \mF\) est complet. Nous considérons l'application \( \Phi\colon \mF\to \mF\) donnée par
    \begin{equation}
        \Phi(f)(x)=\lambda\int_a^bK(x,y)f(y)dy+\varphi(x).
    \end{equation}
    Nous montrons que \( \Phi^p\) est une application contractante pour un certain \( p\). Pour tout \( x\in \mathopen[ a , b \mathclose]\) nous avons
    \begin{subequations}
        \begin{align}
            \| \Phi(f)-\Phi(g) \|_{\infty}&\leq \| \Phi(f)(x)-\Phi(g)(x) \|\\
            &=| \lambda |\Big\| \int_a^bK(x,y)\big( f(y)-g(y) \big)dy  \Big\|\\
            &\leq | \lambda |\| K \|_{\infty}| b-a |\| f-g \|_{\infty}
        \end{align}
    \end{subequations}
    Si \( \lambda\) est assez petit, et si \( p\) est assez grand, l'application \( \Phi^p\) est donc une contraction. Elle possède donc un unique point fixe par le théorème de Picard~\ref{ThoEPVkCL}.
\end{proof}

%+++++++++++++++++++++++++++++++++++++++++++++++++++++++++++++++++++++++++++++++++++++++++++++++++++++++++++++++++++++++++++
\section{Théorèmes de point fixes et équations différentielles}
%+++++++++++++++++++++++++++++++++++++++++++++++++++++++++++++++++++++++++++++++++++++++++++++++++++++++++++++++++++++++++++

%---------------------------------------------------------------------------------------------------------------------------
\subsection{Théorème de Cauchy-Lipschitz}
%---------------------------------------------------------------------------------------------------------------------------

Nous démontrons ici deux théorèmes de Cauchy-Lipschitz. De nombreuses propriétés annexes seront démontrées dans le chapitre sur les équations différentielles, section~\ref{SECooNKICooDnOFTD}.

Le théorème de Cauchy-Arzella \ref{ThoHNBooUipgPX} sera pour plus tard parce qu'il utilise Schauder \ref{ThovHJXIU}. 

\begin{theorem}[Cauchy-Lipschitz\cite{SandrineCL,ZPNooLNyWjX}] \label{ThokUUlgU}
    Nous considérons l'équation différentielle
    \begin{subequations}        \label{XtiXON}
        \begin{numcases}{}
            y'(t)=f\big( t,y(t) \big)\\
            y(t_0)=y_0
        \end{numcases}
    \end{subequations}
    avec \( f\colon U=I\times \Omega\to \eR^n\) où \( I\) est ouvert dans \( \eR\) et \( \Omega\) ouvert dans \( \eR^n\). Nous supposons que \( f\) est continue sur \( U\) et localement Lipschitz\footnote{Définition~\ref{DefJSFFooEOCogV}. Notons que nous ne supposons pas que \( f\) soit une contraction.} par rapport à \( y\).

    Alors il existe un intervalle \( J\subset I\) sur lequel la solution au problème est unique. De plus toute solution du problème est une restriction de cette solution à une partie de \( J\). La solution sur \( J\) (dite «solution maximale») est de classe \( C^1\).
\end{theorem}
\index{théorème!Cauchy-Lipschitz}

% Il serait tentant de mettre ce théorème dans la partie sur les équations différentielles, mais ce n'est pas aussi simple :
% Il est utilisé pour calculer la transformée de Fourier de la Gaussienne (lemme LEMooPAAJooCsoyAJ) dans le chapitre sur la transformée de Fourier.

\begin{proof}
    Nous divisions la preuve en plusieurs étapes (même pas toutes simples).
    \begin{subproof}
    \item[Cylindre de sécurité]

    Précisons l'espace fonctionnel \( \mF\) adéquat. Soient \( V\) et \( W\) les voisinages de \( t_0\) et \( y_0\) sur lesquels \( f\) est localement Lipschitz. Nous considérons les quantités suivantes :
    \begin{enumerate}
        \item
            \( M=\sup_{V\times W}f\) ;
        \item
            \( r>0\) tel que \( \overline{ B(y_0,r) }\subset V\)
        \item
            \( T>0\) tel que \( \overline{ B(t_0,T) }\subset W\) et \( T<r/M\).
    \end{enumerate}
    Nous considérons alors l'ensemble
    \begin{equation}
        \mF=C^0\big( \overline{ B(t_0,T) },\overline{ B(y_0,r) } \big)
    \end{equation}
    que nous munissons de la norme uniforme. Par le lemme~\ref{LemdLKKnd} l'espace \( \big( \mF,\| . \|_{\infty} \big)\) est complet.

    \item[Une application \( \Phi\colon \mF\to \mF\)]


        Si \( y\) est une solution de l'équation différentielle considérée, elle vérifie\footnote{C'est le théorème fondamental du calcul intégral \ref{ThoRWXooTqHGbC}.}
    \begin{equation}        \label{EqPGLwcL}
        y(t)=y_0+\int_{t_0}^tf\big( u,y(u) \big)du.
    \end{equation}
    Ceci nous incite à considérer l'opérateur \( \Phi\colon \mF\to \mF\) défini par
    \begin{equation}
        \Phi(y)(t)=y_0+\int_{t_0}^tf\big( u,y(u) \big)du.
    \end{equation}

    Pour que l'application \( \Phi\) soit utile nous devons montrer que pour tout \( y\in \mF\),
    \begin{itemize}
        \item l'application \( \Phi(y)\) est bien définie,
        \item pour tout \( t\in\overline{ B(y_0,r) }\) nous avons \( \Phi(y)(t)\in\overline{ B(t_0,T) }\),
        \item l'application $\Phi(y)\colon    \overline{ B(t_0,T) }\to \overline{ B(y_0,r)} $ est continue.
    \end{itemize}
    Attention : nous ne prétendons pas que \( \Phi\) elle-même soit continue. C'est parti.
    \begin{subproof}
    \item[\( \Phi(y)\) est bien définie]

        Il faut montrer que l'intégrale converge. Le calcul de \( \Phi(y)(t)\) ne se fait qu'avec \( t\in \overline{ B(t_0,T) }\). Vu que \( u\) prend ses valeurs dans \( \mathopen[ t_0 , t \mathclose]\) et que \( y\in\mF\), le nombre \( y(u)\) est toujours dans \( \overline{ B(y_0,r) }\). Ceci pour dire que dans l'intégrale, la fonction \( f\) n'est considérée que sur \( \mathopen[ t_0 , t \mathclose]\times \overline{ B(y_0,r) }\subset V\times W\). La fonction \( f\) est donc uniformément majorable, et l'intégrale ne pose pas de problèmes.

    \item[\( \Phi(y)(t)\in \overline{ B(t_0,T) }\)]

    Prouvons que \( \Phi(y)(t)\in\overline{ B(y_0,r) }\). Pour cela, notons que
    \begin{equation}
        | \Phi(y)(t)-y_0 |\leq \int_{t_0}^t |f\big( u,y(u) \big)|du\leq | t-t_0 |\| f \|_{\infty}.
    \end{equation}
    Étant donné que \( t\in\overline{ B(t_0,T) }\) nous avons \( | t-t_0 |\leq r/M\) et donc \( | \Phi(y)(t)-y_0 |\leq r\).

    \item[\( \Phi(y)\) est continue]

        Nous pourrions invoquer le théorème~\ref{ThoKnuSNd}, mais nous allons le faire à la main. Soit \( s_0\in B(t_0,T)\) et prouvons que \( \Phi(y)\) est continue en \( s_0\). Pour cela nous prenons \( s\in B(s_0,\delta)\) et nous calculons :
        \begin{equation}
            | \Phi(y)(s)-\Phi(y)(s_0) |\leq \int_{s_0}^s|f\big( u,y(u) \big)|du\leq | s_0-s |\| f \|_{\infty}.
        \end{equation}
        C'est le fait que \( f\) soit bornée dans le cylindre de sécurité qui fait en sorte que cela tende vers zéro lorsque \( s\to s_0\).
    \end{subproof}

    L'équation \eqref{EqPGLwcL} signifie que \( y\) est un point fixe de \( \Phi\). L'espace \( \mF\) étant complet le théorème de point fixe de Picard (théorème~\ref{ThoEPVkCL}) s'applique. Nous allons montrer qu'il existe un \( p\in\eN\) tel que \( \Phi^p\) soit contractante. Par conséquent \( \Phi^p\) aura un unique point fixe qui sera également unique point fixe de \( \Phi\) par la remarque~\ref{remIOHUJm}.

\item[Contractante]

    Prouvons donc que \( \Phi^p\) est contractante pour un certain \( p\). Pour cela nous commençons par montrer la formule suivante par récurrence :
    \begin{equation}        \label{EqRAdKxT}
        \big\| \Phi^p(x)(t)-\Phi^p(y)(t) \big\|\leq \frac{ k^p| t-t_0 |^p }{ p! }\| x-y \|_{\infty}
    \end{equation}
    pour tout \( x,y\in\mF\), et pour tout \( t\in\overline{ B(t_0,T) }\). Pour \( p=0\) la formule \eqref{EqRAdKxT} est vérifiée parce que \( \| x-y \|_{\infty}\) est le supremum de \( \| x(t)-y(t) \|\) pour \( t\in\overline{ B(t_0,T) }\). Supposons que la formule soit vraie pour \( p\) et calculons pour \( p+1\). Pour tout \( t\in\overline{ B(t_0,T) }\) nous avons
    \begin{subequations}
        \begin{align}
            \big\| \Phi^{p+1}(x)(t)-\Phi^{p+1}(y)(t) \big\|&\leq \left| \int_{t_0}^t\big\| f\big( u,\Phi^p(x)(u) \big)-f\big( u,\Phi^p(y)(u) \big) \big\|du \right| \\
            &\leq \left| \int_{t_0}^tk\| \Phi^p(x)(u)-\Phi^p(y)(u) \|du \right|    \label{subIKYixF}\\
            &\leq \left| \int_{t_0}^tk\frac{ k^p| t-t_0 | }{ p! }\| x-y \|_{\infty} \right| \label{subxkNjiV} \\
            &=\frac{ k^{p+1}| t-t_0 |^{p+1} }{ (p+1)! }\| x-y \|_{\infty}.
        \end{align}
    \end{subequations}
    Justifications :
    \begin{itemize}
        \item \eqref{subIKYixF} parce que \( f\) est Lipschitz.
        \item \eqref{subxkNjiV} par hypothèse de récurrence.
    \end{itemize}
    La formule \eqref{EqRAdKxT} est maintenant établie. Nous pouvons maintenant montrer que \( \Phi^p\) est une contraction pour un certain \( p\). Pour tout \( t\in \overline{ B(t_0,T) }\) nous avons
    \begin{equation}
         \| \Phi^p(x)(t)-\Phi^p(y)(t) \|\leq \frac{ k^p }{ t! }| t-t_0 |^p\| x-y \|_{\infty}     \leq \frac{ k^pT^p }{ p! }\| x-y \|_{\infty}
    \end{equation}
    où nous avons utilisé le fait que \( | t-t_0 |^p<T^p\). En prenant le supremum sur \( t\) des deux côtés il vient
    \begin{equation}
        \| \Phi^p(x)-\Phi^p(y) \|_{\infty}\leq\frac{ k^pT^p }{ p! }\| x-y \|_{\infty}.
    \end{equation}
    Le membre de droite tend vers zéro lorsque \( p\to\infty\) parce que \( k<1\) et \( T^p/p!\to 0\)\footnote{C'est le terme général du développement de \(  e^{T}\) qui est une série convergente.}. Nous concluons donc que \( \Phi^p\) est une contraction pour un certain \( p\).

\item[Conclusion]

    L'unique point fixe de \( \Phi\) est alors l'unique solution continue de l'équation différentielle \eqref{XtiXON}. Par ailleurs l'équation elle-même \( y'=f(t,y)\) demande implicitement que \( y\) soit dérivable et donc continue. Nous concluons que l'unique point fixe de \( \Phi\) est l'unique solution de l'équation différentielle donnée. Cette dernière est automatiquement \( C^1\) parce que si \( y\) est continue alors \( u\mapsto f(u,y(u))\) est continue, c'est-à-dire que \( y'\) est continue.

\item[Unicité]

    Nous passons maintenant à la partie «prolongement maximum» du théorème. Soient \( x_1\) et \( x_2\) deux solutions maximales du problème \eqref{XtiXON} sur des intervalles \( I_1\) et \( I_2\) respectivement. Les intervalles \( I_1\) et \( I_2\) contiennent \( \overline{ B(t_0,r) }\) sur lequel \( x_1=x_2\) par unicité.


    Nous allons maintenant montrer que pour tout \( t\geq t_0\) pour lequel \( x_1\) ou \( x_2\) est défini, \( x_1(t)\) et \( x_2(t)\) sont définis et sont égaux. Le raisonnement sur \( t\leq t_0\) est similaire.

    Supposons que l'ensemble des \( t\geq t_0\) tels que \( x_1=x_2\) soit ouvert à droite, c'est-à-dire soit de la forme \( \mathopen[ t_0 ,b [\). Dans ce cas, soit \( x_1\) soit \( x_2\) (soit les deux) cesse d'exister en \( b\). En effet si nous avions les fonctions \( x_i\) sur \(\mathopen[ t_0 , b+\epsilon [\) alors l'équation \( x_1=x_2\) définirait un fermé dans \( \mathopen[ t_0 , b+\epsilon [\). Supposons pour fixer les idées que \( x_1\) cesse d'exister : le domaine de \( x_1\) (parmi les \( t\geq 0\)) est \( \mathopen[ t_0 , b [\) et sur ce domaine nous avons \( x_1=x_2\). Dans ce cas \( x_1\) pourrait être prolongé en \( x_2\) au-delà de \( b\). Si \( x_1\) et \( x_2\) s'arrêtent d'exister en même temps en \( b\), alors nous avons bien \( x_1=x_2\).

    Nous devons donc traiter le cas où \( x_1=x_2\) sur \( \mathopen[ t_0 , b \mathclose]\) alors que \( x_1\) et \( x_2\) existent sur \( \mathopen[ t_0 , b+\epsilon [\) pour un certain \( \epsilon\).

    Nous pouvons appliquer le théorème d'existence locale au problème
    \begin{subequations}
        \begin{numcases}{}
            y'=f(t,y)\\
            y(b)=x_1(b).
        \end{numcases}
    \end{subequations}
    Il existe un voisinage de \( b\) sur lequel la solution est unique. Sur ce voisinage nous devons donc avoir \( x_1=x_2\), ce qui contredit le fait que \( x_1\neq x_2\) en dehors de \( \mathopen[ t_0 , b \mathclose]\).

    Donc \( x_1\) et \( x_2\) existent et sont égaux sur au moins \( I_1\cup I_2\).
    \end{subproof}
\end{proof}

Le théorème de Cauchy-Lipschitz donne existence et unicité d'une solution maximale. Cependant cette solution peut ne pas exister partout où les hypothèses sur \( f\) sont remplies. En d'autres termes, il peut arriver que \( f\) soit Lipschitz jusqu'à \( t_1\), mais que la solution maximale ne soit définie que jusqu'en \( t_2<t_1\). Ce cas fait l'objet du théorème d'explosion en temps fini~\ref{CorGDJQooNEIvpp}.

Sous quelques hypothèses nous pouvons nous assurer de l'existence d'une solution unique sur tout \( \eR\).


\ifbool{isGiulietta}{Ce théorème de Cauchy-Lipschitz global est utilisé pour faire le lien entre les représentations des algèbres de Lie et celles du groupe, voir la proposition \ref{PROPooXCGMooKlJlwp}.}{}

\begin{theorem}[Cauchy-Lipschitz global\cite{ooJZJPooAygxpk,KXjFWKA}]       \label{THOooZIVRooPSWMxg}
    Soit un intervalle \( I\) de \( \eR\), \( y_0\in \eR^n\), \( t_0\in I\) et une fonction continue \( f\colon I\times \eR^n\to \eR^n\) telle que pour tout compact \( K\) dans \( I\), il existe \( k>0\) tel que
    \begin{equation}
        \| f(t,y_1)-f(t,y_2) \|\leq k\| y_1-y_2 \|
    \end{equation}
    pour tout \( t\in K\) et \( y_1,y_2\in \eR^n\).

    Alors le problème
    \begin{subequations}        \label{EQSooBNREooUTfbMH}
        \begin{numcases}{}
            y'(t)=f\big( t,y(t) \big)\\
            y(t_0)=y_0
        \end{numcases}
    \end{subequations}
    possède une unique solution \( y\colon I\to \eR^n\) sur \( I\).
\end{theorem}

\begin{proof}
    Soit un intervalle compact \( K\) dans \( I\) et contenant \( t_0\). Nous notons \( \ell\) le diamètre de \( K\). Sur l'espace \( E=C^0(K,\eR^n)\) nous considérons la topologie uniforme : \( (E,\| . \|_{\infty})\). C'est un espace complet par le lemme~\ref{LemdLKKnd} (nous utilisons le fait que \( \eR^n\) soit complet, proposition~\ref{PROPooTFVOooFoSHPg}). Nous allons utiliser l'application suivante :
    \begin{equation}        \label{EQooJUTBooILBKoE}
        \begin{aligned}
            \Phi\colon E&\to E \\
            \Phi(y)(t)&=y_0+\int_{t_0}^tf\big( s,y(s) \big)ds
        \end{aligned}
    \end{equation}
    Démontrons quelques faits à propos de \( \Phi\).
    \begin{subproof}
        \item[La définition fonctionne bien]
            Nous devons commencer par prouver que cette application est bien définie. Si \( y\in E\) alors \( f\) et \( y\) sont continues; l'application \( s\mapsto f\big(s,y(s)\big)\) est donc également continue. L'intégrale de cette fonction sur le compact \( \mathopen[ t_0 , t \mathclose]\) ne pose alors pas de problèmes. En ce qui concerne la continuité de \( \phi(y)\) sous l'hypothèse que \( y\) soit continue,
    \begin{equation}
        \| \Phi(y)(t)-\Phi(y)(t') \|\leq \int_t^{t'}\| f(s,y(s)) \|ds\leq M| t-t' |
    \end{equation}
    où \( M\) est une majoration de \( \| s\mapsto f\big( s,y(s) \big) \|_{\infty,K}\).

        \item[Si \( y\) est solution alors \( \Phi(y)=y\)]

            Supposons que \( y\) soit une solution de l'équation différentielle \eqref{EQSooBNREooUTfbMH}. Alors, vu que \( y'(t)=f\big( t,y(t) \big)\) nous avons :
            \begin{equation}
                y(t)=y_0+\int_{t_0}^ty'(s)ds=y_0+\int_{t_0}^tf\big( s,y(s) \big)ds=\Phi(y)(t).
            \end{equation}

        \item[Si \( \Phi(y)=y\) alors \( y\) est solution]

            Nous avons, pour tout \( t\) :
            \begin{equation}
                y(t)=y_0+\int_{t_0}^tf\big( s,y(s) \big)ds.
            \end{equation}
            Le membre de droite est dérivable par rapport à \( t\), et la dérivée fait \(  f\big( t,y(t) \big)   \). Donc le membre de gauche est également dérivable et nous avons bien
            \begin{equation}
                y'(t)=f\big( t,y(t) \big).
            \end{equation}
            De plus \( y(t_0)=y_0+\int_{t_0}^{t_0}\ldots=y_0\).
    \end{subproof}

    Nous sommes encore avec \( K\) compact et \( E=C^0(K,\eR^n)\) muni de la norme uniforme. Nous allons montrer que \( \Phi\) est une contraction de \( E\) pour une norme bien choisie.

    \begin{subproof}
        \item[Une norme sur \( E\)]
            Pour \( y\in E\) nous posons
            \begin{equation}
                \| y \|_k=\max_{t\in K}\big(  e^{-k| t-t_0 |}\| y(t) \| \big).
            \end{equation}
            Ce maximum est bien définit et fini parce que la fonction de \( t\) dedans est une fonction continue sur le compact \( K\). Cela est également une norme parce que si \( \| y \|_k=0\) alors \(  e^{-k| t-t_0 |}\| y(t) \|=0\) pour tout \( t\). Étant donné que l'exponentielle ne s'annule pas, \( \| y(t) \|=0\) pour tout \( t\).
        \item[Équivalence de norme]

            Nous montrons que les normes \( \| . \|_k\) et \( \| . \|_{\infty}\) sont équivalentes\footnote{Définition~\ref{DefEquivNorm}} :
            \begin{equation}        \label{EQooSQYWooBTXvDL}
                \| y \|_{\infty} e^{-k\ell}\leq \| y \|_k\leq \| y \|_{\infty}
            \end{equation}
            pour tout \( y\in E\). Pour la première inégalité, \( \ell\geq | t-t_0 |\) pour tout \( t\in K\), et \( k>0\), donc
            \begin{equation}
                \| y(t) \| e^{-k\ell}\leq  e^{-k| t-t_0 |}\| y(t) \|.
            \end{equation}
            En prenant le maximum des deux côtés, \( \| y \|_{\infty} e^{-k\ell}\leq \| y \|_k\).

            En ce qui concerne la seconde inégalité dans \eqref{EQooSQYWooBTXvDL}, \( k| t-t_0 |\geq 0\) et donc \(  e^{-k| t-t_0 |}<1\).

    \end{subproof}
    Vu que les normes \( \| . \|_{\infty}\) et \( \| . \|_k\) sont équivalentes, l'espace \( (E,\| . \|_k)\) est tout autant complet que \( (E,\| . \|_{\infty})\). Nous démontrons à présent que \( \Phi\) est une contraction dans \( (E,\|  \|_k)\).

    Soient \( y,z\in E\). Si \( t\geq t_0\) nous avons
    \begin{subequations}        \label{SUBEQSooEXVYooDkyTuB}
        \begin{align}
            \| \Phi(y)(t)-\Phi(z)(t) \|&\leq \int_{t_0}^t\| f\big( s,y(s) \big)-f\big( s,z(s) \big) \|ds\\
            &\leq k\int_{t_0}^t\| y(s)-z(s) \|ds.
        \end{align}
    \end{subequations}
    Il convient maintenant de remarquer que
    \begin{equation}
        \| y(t) \|= e^{-k| t-t_0 |} e^{k| t-t_0 |}\| y(t) \|\leq \| y \|_k e^{k| t-t_0 |}.
    \end{equation}
    Nous pouvons avec ça prolonger les inégalités \eqref{SUBEQSooEXVYooDkyTuB} par
    \begin{equation}
        \| \Phi(y)(t)-\Phi(z)(t) \|\leq k\| y-z \|_k\int_{t_0}^t e^{k| s-t_0 |}ds=k\| y-z \|_k\int_{t_0}^t e^{k(s-t_0)}ds
    \end{equation}
    où nous avons utilisé notre supposition \( t\geq t_0\) pour éliminer les valeurs absolues. L'intégrale peut être faite explicitement, mais nous en sommes arrivés à un niveau de fainéantise tellement inconcevable que

\lstinputlisting{tex/sage/sageSnip014.sage}

Au final, si \( t\geq t_0\),
    \begin{equation}
        \| \Phi(y)(t)-\Phi(z)(t) \|\leq \| y-z \|_k\big(  e^{k(t-t_0)}-1 \big).
    \end{equation}
    Si \( t\leq t_0\), il faut retourner les bornes de l'intégrale avant d'y faire rentrer la norme parce que \( \| \int_0^1f \|\leq \int_0^1\| f \|\), mais ça ne marche pas avec \( \| \int_1^0f \|\). Pour \( t\leq t_0\) tout le calcul donne
    \begin{equation}
        \| \Phi(y)(t)-\Phi(z)(t) \|\leq \| y-z \|_k\big(  e^{k(t_0-t)}-1 \big).
    \end{equation}
    Les deux inéquations sont valables a fortiori en mettant des valeurs absolues dans l'exponentielle, de telle sorte que pour tout \( t\in K\) nous avons
    \begin{equation}
        e^{-k| t_0-t |}\| \phi(y)(t)-\Phi(z)(t) \|\leq \| y-z \|_k\big( 1- e^{-k| t_0-t |} \big).
    \end{equation}
    En prenant le supremum sur \( t\),
    \begin{equation}
        \| \Phi(y)-\Phi(z) \|_k\leq \| y-z \|_k(1- e^{-k\ell}),
    \end{equation}
    mais \( 0<(1- e^{e-k\ell})<1\), donc \( \Phi\) est contractante pour la norme \( \| . \|_k\). Vu que \( (E,\| . \|_k)\) est complet, l'application \( \Phi\) y a un unique point fixe par le théorème de Picard~\ref{ThoEPVkCL}.

    Ce point fixe est donc l'unique solution de l'équation différentielle de départ.

    \begin{subproof}
        \item[Existence et unicité sur \( I\)]
            Il nous reste à prouver que la solution que nous avons trouvée existe sur \( I\) : jusqu'à présent nous avons démontré l'existence et l'unicité sur n'importe quel compact dans \( I\).

            Soit une suite croissante de compacts \( K_n\) contenant \( t_0\) (par exemple une suite exhaustive comme celle du lemme~\ref{LemGDeZlOo}). Nous avons en particulier
            \begin{equation}
                I=\bigcup_{n=0}^{\infty}K_n.
            \end{equation}
        \item[Existence sur \( I\)]

            Soit \( y_n\) l'unique solution sur \( K_n\). Il suffit de poser
            \begin{equation}
                y(t)=y_n(t)
            \end{equation}
            pour \( n\) tel que \( t\in K_n\). Cette définition fonctionne parce que si \( t\in K_n\cap K_m\), il y a forcément un des deux qui est inclus dans l'autre et le résultat d'unicité sur le plus grand des deux donne \( y_n(t)=y_m(t)\).

        \item[Unicité sur \( I\)]

            Soient \( y\) et \(z \) des solutions sur \( I\); vu que \( I\) n'est pas spécialement compact, le travail fait plus haut ne permet pas de conclure que \( y=z\).

            Soit \( t\in I\). Alors \( t\in K_n\) pour un certain \( n\) et \( y\) et \( z\) sont des solutions sur \( K_n\) qui est compact. L'unicité sur \( K_n\) donne \( y(t)=z(t)\).
    \end{subproof}
\end{proof}

\begin{normaltext}
    Il y a d'autres moyens de prouver qu'une solution existe globalement sur \( \eR\). Si \( f\) est globalement bornée, le théorème d'explosion en temps fini donne quelques garanties, voir~\ref{NORMooZROGooZfsdnZ}.
\end{normaltext}

Le théorème suivant donne une version du théorème de Cauchy-Lipschitz lorsque la fonction \( f\) dépend d'un paramètre. Ce théorème n'utilise rien de fondamentalement nouveau. Nous le donnons seulement pour montrer que l'on peut choisir l'espace \( \mF\) de façon un peu maligne pour élargir le résultat. Si vous voulez un théorème de Cauchy-Lipschitz avec paramètre vraiment intéressant, allez voir le théorème~\ref{PROPooPYHWooIZhQST}.

\begin{theorem}[Cauchy-Lipschitz avec paramètre\cite{MonCerveau,ooXVPAooTQUIRw}]           \label{THOooDTCWooSPKeYu}
    Soit un intervalle ouvert \( I\) de \( \eR\), un connexe ouvert \( \Omega\) de \( \eR^n\) et un intervalle ouvert \( \Lambda\) de \( \eR^d\). Soit une fonction \( f\colon I\times \Omega\times \Lambda\to \eR^n\) continue et localement Lipschitz en \( \Omega\). Soient \( t_0\in I\), \( y_0\in \Omega\) et \( \lambda_0\in \Lambda\). Il existe un voisinage compact de \( (t_0,y_0,\lambda_0)\) sur lequel le problème
    \begin{subequations}
        \begin{numcases}{}
            y'_{\lambda}(t)=f\big( t,y_{\lambda}(t),\lambda \big)\\
            y_{\lambda}(t_0)=y_0
        \end{numcases}
    \end{subequations}
    possède une unique solution. De plus \( (t,\lambda)\mapsto y_{\lambda}(t)\) est continue\footnote{Ici, la surprise est que ce soit continu par rapport à \( \lambda\). Le fait qu'elle le soit par rapport à \( t\) est clair depuis le départ parce que c'est finalement rien d'autre que le Cauchy-Lipschitz vieux et connu.}.
\end{theorem}

\begin{proof}

    \begin{probleme}
        Ceci est une idée de la preuve. Je n'ai pas vérifié toutes les étapes. Soyez prudent.

    \end{probleme}

    D'abord nous avons un voisinage compact \( V\times \overline{ B(y_0,r) }\times \Lambda_0\) de \( (t_0,y_0,\lambda_0)\) sur lequel $f$ est bornée. Ensuite nous récrivons l'équation différentielle sous la forme
    \begin{subequations}
        \begin{numcases}{}
            \frac{ \partial y }{ \partial t }(t,\lambda)=f\big( t,y(t,\lambda),\lambda \big)\\
            y(t_0,\lambda)=y_0.
        \end{numcases}
    \end{subequations}
    pour une fonction \( y\colon V\times \Lambda_0\to \eR^n\).

    Nous posons \( \mF=C^0\big( V\times\Lambda_0 ,\eR^n\big)\) et nous y définissons l'application
    \begin{equation}
        \begin{aligned}
            \Phi\colon \mF&\to \mF \\
            \Phi(y)(t,\lambda)&=y_0+\int_{t_0}^tf\big( s,y(s,\lambda),\lambda \big)ds.
        \end{aligned}
    \end{equation}
    Il y a plein de vérifications à faire\cite{ooXVPAooTQUIRw}, mais je parie que \( \Phi\) est bien définie, et que une de ses puissances est une contraction de \( (\mF,\| . \|_{\infty})\). L'unique point fixe est une solution de notre problème et est dans \( C^0\), donc \( (t,\lambda)\mapsto y(t,\lambda)=y_{\lambda}(t)\) est de classe \( C^0\), c'est-à-dire continue.
\end{proof}

\begin{normaltext}
    Ce théorème marque un peu la limite de ce que l'on peut faire avec la méthode des points fixes dans le cadre de Cauchy-Lipschitz : nous sommes limités à la continuité de la solution parce que les espaces \( C^p\) ne sont pas complets\footnote{Par exemple, le théorème de Stone-Weierstrass~\ref{ThoGddfas} nous dit que la limite uniforme de polynômes (de classe \(  C^{\infty}\)) peut n'être que continue. Voir aussi le thème~\ref{THMooOCXTooWenIJE}.}. Il n'y a donc pas d'espoir d'adapter la méthode pour prouver que si \( f\) est de classe \( C^p\) alors \( (t,\lambda)\mapsto y_{\lambda}(t)\) est de classe \( C^p\). On peut, à \( \lambda\) fixé prouver que \( t\mapsto y_{\lambda}(t)\) est de classe \( C^p\) (utiliser une récurrence), mais pas plus.

    La régularité \( C^1\) de \( y\) par rapport à la condition initiale sera l'objet du théorème~\ref{THOooSTHXooXqLBoT}. Ce résultat n'est vraiment pas facile et utilise des ingrédients bien autres qu'un point fixe. Ensuite la régularité \( C^p\) par rapport à la condition initiale et par rapport à un paramètre seront presque des cadeaux (proposition~\ref{PROPooINLNooDVWaMn} et~\ref{PROPooPYHWooIZhQST}).
\end{normaltext}

\begin{example}[\cite{ooSBHXooOMnaTC}]          \label{EXooJXIGooQtotMc}
    Nous savons que le théorème de Picard permet de trouver le point fixe par itération de la contraction à partir d'un point quelconque. Tentons donc de résoudre
    \begin{subequations}
        \begin{numcases}{}
            y'(t)=y(t)\\
            y(0)=1
        \end{numcases}
    \end{subequations}
    dont nous savons depuis l'enfance que la solution est l'exponentielle\footnote{Voir par exemple le théorème \ref{ThoKRYAooAcnTut}.}. Partons donc de la fonction constante \( y_0=1\), et appliquons la contraction \eqref{EQooJUTBooILBKoE} :
    \begin{equation}
        u_1=1+\int_0^1u_0(s)ds=1+t.
    \end{equation}
    Ensuite
    \begin{equation}
        u_2=1+\int_0^t(1+s)ds=1+t+\frac{ t^2 }{2}.
    \end{equation}
    Et on voit que les itérations suivantes vont donner l'exponentielle.

    Nous sommes évidemment en droit de se dire que nous avons choisi un bon point de départ. Tentons le coup avec une fonction qui n'a rien à voir avec l'exponentielle : \( u_0(x)=\sin(x)\).

    Le programme suivant permet de faire de belles investigations numériques en partant d'à peu près n'importe quelle fonction :

\lstinputlisting{tex/sage/picard_exp.py}

    Ce programme fait \( 30\) itérations depuis la fonction \( \sin(x)\) pour tenter d'approximer \( \exp(x)\). Pour donner une idée, après \( 7\) itérations nous avons la fonction suivante :
    \begin{equation}
        \frac{1}{ 60 }x^5+\frac{1}{ 24 }x^4+\frac{ 1 }{2}x^2+2x-\sin(x)+1.
    \end{equation}
    Nous voyons que les coefficients sont des factorielles, mais pas toujours celles correspondantes à la puissance, et qu'il manque certains termes par rapport au développement de l'exponentielle que nous connaissons. Bref, le polynôme qui se met en face de \( \sin(x)\) s'adapte tout seul pour compenser.

    Et après \( 30\) itérations, ça donne quoi ? Voici un graphe de l'erreur entre \( u_{30}(x)\) et \( \exp(30)\) :


\begin{center}
   \input{auto/pictures_tex/Fig_XOLBooGcrjiwoU.pstricks}
\end{center}

    Pour donner une idée, \( \exp(10)\simeq 22000\). Donc il y a une faute de \( 0.01\) sur \( 22000\). Pas mal.

\end{example}

% This is part of Mes notes de mathématique
% Copyright (c) 2011-2015,2018-2020
%   Laurent Claessens
% See the file fdl-1.3.txt for copying conditions.


%+++++++++++++++++++++++++++++++++++++++++++++++++++++++++++++++++++++++++++++++++++++++++++++++++++++++++++++++++++++++++++
                    \section{Théorèmes d'inversion locale et de la fonction implicite}
%+++++++++++++++++++++++++++++++++++++++++++++++++++++++++++++++++++++++++++++++++++++++++++++++++++++++++++++++++++++++++++

%---------------------------------------------------------------------------------------------------------------------------
\subsection{Mise en situation}
%---------------------------------------------------------------------------------------------------------------------------

Dans un certain nombre de situation, il n'est pas possible de trouver des solutions explicites aux équations qui apparaissent. Néanmoins, l'existence «théorique» d'une telle solution est souvent déjà suffisante. C'est l'objet du théorème de la fonction implicite.

Prenons par exemple la fonction sur $\eR^2$ donnée par
\begin{equation}
    F(x,y)=x^2+y^2-1.
\end{equation}
Nous pouvons bien entendu regarder l'ensemble des points donnés par $F(x,y)=0$. C'est le cercle dessiné à la figure~\ref{LabelFigCercleImplicite}.
\newcommand{\CaptionFigCercleImplicite}{Un cercle pour montrer l'intérêt de la fonction implicite. Si on donne \( x\), nous ne pouvons pas savoir si nous parlons de \( P\) ou de \( P'\).}
\input{auto/pictures_tex/Fig_CercleImplicite.pstricks}

%\ref{LabelFigCercleImplicite}.
%\newcommand{\CaptionFigCercleImplicite}{Un cercle pour montrer l'intérêt de la fonction implicite.}
%\input{auto/pictures_tex/Fig_CercleImplicite.pstricks}

Nous ne pouvons pas donner le cercle sous la forme $y=y(x)$ à cause du $\pm$ qui arrive quand on prend la racine carrée. Mais si on se donne le point $P$, nous pouvons dire que \emph{autour de $P$}, le cercle est la fonction
\begin{equation}
    y(x)=\sqrt{1-x^2}.
\end{equation}
Tandis que autour du point $P'$, le cercle est la fonction
\begin{equation}
    y(x)=-\sqrt{1-x^2}.
\end{equation}
Autour de ces deux points, donc, le cercle est donné par une fonction. Il n'est par contre pas possible de donner le cercle autour du point $Q$ sous la forme d'une fonction.

Ce que nous voulons faire, en général, est de voir si l'ensemble des points tels que
\begin{equation}
    F(x_1,\ldots,x_n,y)=0
\end{equation}
peut être donné par une fonction $y=y(x_1,\ldots,x_n)$. En d'autre termes, est-ce qu'il existe une fonction $y(x_1,\ldots,x_n)$ telle que
\begin{equation}
    F\big( x_1,\ldots,x_n,y(x_1,\ldots,x_n)\big)=0.
\end{equation}

Plus généralement, soit une fonction
\begin{equation}
    \begin{aligned}
        F\colon D\subset \eR^n\times \eR^m&\to \eR^m \\
        (x,y)&\mapsto \big( F_1(x,y),\ldots, F_m(x,y) \big)
    \end{aligned}
\end{equation}
avec $x = (x_1,\ldots, x_n)$ et $y = (y_1,\ldots,y_m)$. Pour chaque $x$ fixé, on s'intéresse aux solutions du système de $m$ équations $F(x,y) = 0$ pour les inconnues $y$ ; en particulier, on voudrait pouvoir écrire $y = \varphi(x)$ vérifiant $F(x,\varphi(x)) = 0$.

%---------------------------------------------------------------------------------------------------------------------------
\subsection{Théorème d'inversion locale}
%---------------------------------------------------------------------------------------------------------------------------

\begin{lemma}[\cite{ZCKMFRg}] \label{LemGZoqknC}
    Soit \( E\) un espace de Banach (métrique complet) et \( \mO\) un ouvert de \( E\). Nous considérons une \( \lambda\)-contraction \( \varphi\colon \mO\to E\). Alors l'application
    \begin{equation}
    f\colon x\mapsto x+\varphi(x)
    \end{equation}
    est un homéomorphisme entre \( \mO\) et un ouvert de \( E\). De plus \( f^{-1}\) est Lipschitz de constante plus petite ou égale à \( (1-\lambda)^{-1}\).
\end{lemma}
Cette proposition utilise le théorème de point fixe de Picard~\ref{ThoEPVkCL},
et sera utilisée pour démontrer le théorème d'inversion locale~\ref{ThoXWpzqCn}.
% note que garder deux lignes ici est important pour vérifier les références vers le futur : la seconde ligne peut être ignorée, pas la seconde.

\begin{proof}
        Soient \( x_1,x_2\in\mO\). Nous posons \( y_1=f(x_1)\) et \( y_2=f(x_2)\). En vertu de l'inégalité de la proposition~\ref{PropNmNNm} nous avons
        \begin{subequations}    \label{subEqEBJsBfz}
            \begin{align}
            \big\| f(x_2)-f(x_1) \big\|&=\big\| x_2+\varphi(x_2)-x_1-\varphi(x_1) \big\|\\
        &\geq \Big|        \| x_2-x_1 \|-\big\| \varphi(x_2)-\varphi(x_1) \big\|  \Big|\\
    &\geq   (1-\lambda)\| x_2-x_1 \|.
            \end{align}
        \end{subequations}
        À la dernière ligne les valeurs absolues sont enlevées parce que nous savons que ce qui est à l'intérieur est positif. Cela nous dit d'abord que \( f\) est injective parce que \( f(x_2)=f(x_1)\) implique \( x_2=x_1\). Donc \( f\) est inversible sur son image. Nous posons \( A=f(\mO)\) et nous devons prouver que que \( f^{-1}\colon A\to \mO\) est continue, Lipschitz de constante majorée par \( (1-\lambda)^{-1}\) et que \( A\) est ouvert.

    Les inéquations \eqref{subEqEBJsBfz} nous disent que
    \begin{equation}
    \big\| f^{-1}(y_1)-f^{-1}(y_2) \big\|\leq \frac{ \| y_1-y_2 \| }{ 1-\lambda },
    \end{equation}
    c'est-à-dire que
    \begin{equation}
        f^{-1}\big( B(y,r) \big)\subset B\big( f^{-1}(y),\frac{ r }{ 1-\lambda } \big),
    \end{equation}
    ce qui signifie que \( f^{-1}\) est Lipschitz de constante souhaitée et donc continue.

    Il reste à prouver que \( f(\mO)\) est ouvert. Pour cela nous prenons \( y_0=f(x_0)\) dans \( f(\mO)\) est nous prouvons qu'il existe \( \epsilon\) tel que \( B(y_0,\epsilon)\) soit dans \( f(\mO)\). Il faut donc que pour tout \( y\in B(y_0,\epsilon)\), l'équation \( f(x)=y\) ait une solution. Nous considérons l'application
    \begin{equation}
        L_y\colon x\mapsto y-\varphi(x).
    \end{equation}
    Ce que nous cherchons est un point fixe de \( L_y\) parce que si \( L_y(x)=x\) alors \( y=x+\varphi(x)=f(x)\). Vu que
    \begin{equation}
        \big\| L_y(x)-L_y(x') \big\|=\big\| \varphi(x)-\varphi(x') \big\|\leq\lambda\| x-x' \|,
    \end{equation}
    l'application \( L_y\) est une contraction de constante \( \lambda\). Par ailleurs \( x_0\) est un point fixe de \( L_{y_0}\), donc en vertu de la caractérisation \eqref{EqDZvtUbn} des fonctions Lipschitziennes,
    \begin{equation}
        L_{y_0}\big( \overline{ B(x_0,\delta) } \big)\subset \overline{ B\big( L_{y_0}(x_0),\lambda\delta \big) }=\overline{ B(x_0,\lambda\delta) }.
    \end{equation}
    Vu que pour tout \( y\) et \( x\) nous avons \( L_y(x)=L_{y_0}(x)+y-y_0\),
    \begin{equation}
    L_y\big( \overline{ B(x_0,\delta) } \big)=L_{y_0}\big( \overline{ B(x_0,\delta) } \big)+(y-y_0)\subset \overline{ B(x_0,\lambda\delta) }+(y-y_0)\subset \overline{ B(x_0),\lambda\delta+\| y-y_0 \| }.
    \end{equation}
    Si \( \epsilon<(1-\lambda)\delta\) alors \( \lambda\delta+\| y-y_0 \|<\delta\). Un tel choix de \( \epsilon>0\) est possible parce que \( \lambda<1\). Pour une telle valeur de \( \epsilon\) nous avons
    \begin{equation}
        L_y\big( \overline{ B(x_0,\delta) } \big)\subset \overline{ B(x_0,\delta) }.
    \end{equation}
    Par conséquent \( L_y\) est une contraction sur l'espace métrique complet \( \overline{ B(x_0,\delta) }\), ce qui signifie que \( L_y\) y possède un point fixe par le théorème de Picard~\ref{ThoEPVkCL}.
\end{proof}

Le théorème d'inversion locale s'énonce de la façon suivante dans \( \eR^n\) :
\begin{theorem}[Inversion locale dans \( \eR^n\)]    \label{THOooQGGWooPBRNEX}      % Ne pas mettre de label ici parce qu'il faut référencer l'autre, celui dans Banach.
    Soit \( f\in C^k(\eR^n,\eR^n)\) et \( x_0\in \eR^n\). Si \( df_{x_0}\) est inversible, alors il existe un voisinage ouvert \( U\) de \( x_0\) et \( V\) de \( f(x_0)\) tels que \( f\colon U\to V\) soit un \( C^k\)-difféomorphisme. (c'est-à-dire que \( f^{-1}\) est également de classe \( C^k\))
\end{theorem}

Nous allons le démontrer dans le cas un peu plus général (mais pas plus cher\footnote{Sauf la justification de la régularité de l'application \( A\mapsto A^{-1}\)}) des espaces de Banach en tant que conséquence du théorème de point fixe de Picard~\ref{ThoEPVkCL}.

\begin{theorem}[Inversion locale dans un espace de Banach\cite{OWTzoEK,ZCKMFRg}] \label{ThoXWpzqCn}
    Soit une fonction \( f\in C^p(E,F)\) avec \( p\geq 1\) entre deux espaces de Banach. Soit \( x_0\in E\) tel que \( df_{x_0}\) soit une bijection bicontinue\footnote{En dimension finie, une application linéaire est toujours continue et d'inverse continu.}. Alors il existe un voisinage ouvert \( V\) de \( x_0\) et \( W\) de \( f(x_0)\) tels que
    \begin{enumerate}
        \item
        \( f\colon V\to W\) soit une bijection,
    \item
        \( f^{-1}\colon W\to V\) soit de classe \( C^p\).
    \end{enumerate}
\end{theorem}
\index{application!différentiable}
\index{théorème!inversion locale}

\begin{proof}
    Nous commençons par simplifier un peu le problème. Pour cela, nous considérons la translation \( T\colon x\mapsto x+x_0 \) et l'application linéaire
    \begin{equation}
        \begin{aligned}
            L\colon \eR^n&\to \eR^n \\
            x&\mapsto (df_{x_0})^{-1}x
        \end{aligned}
    \end{equation}
    qui sont tout deux des difféomorphismes (\( L\) en est un par hypothèse d'inversibilité). Quitte à travailler avec la fonction \( k=L\circ f\circ T\), nous pouvons supposer que \( x_0=0\) et que \( df_{x_0}=\mtu\). Pour comprendre cela il faut utiliser deux fois la formule de différentielle de fonction composée de la proposition~\ref{EqDiffCompose} :
    \begin{equation}
        dk_0(u)=dL_{(f\circ T)(0)}\Big( df_{T(0)}dT_0(u) \Big).
    \end{equation}
    Vu que \( L\) est linéaire, sa différentielle est elle-même, c'est-à-dire \( dL_{(f\circ T)(0)}=(df_{x_0})^{-1}\), et par ailleurs \( dT_0=\mtu\), donc
    \begin{equation}
        dk_0(u)=(df_{x_0})^{-1}\Big( df_{x_0}(u) \Big)=u,
    \end{equation}
    ce qui signifie bien que \( dk_0=\mtu\). Pour tout cela nous avons utilisé en plein le fait que \( df_{x_0}\) était inversible.

Nous posons \( g=f-\mtu\), c'est-à-dire \( g(x)=f(x)-x\), qui a la propriété \( dg_0=0\). Étant donné que \( g\) est de classe \( C^1\), l'application\footnote{Ici \( \GL(F)\) est l'ensemble des applications linéaires, inversibles et continues de \( F\) dans lui-même. Ce ne sont pas spécialement des matrices parce que nous n'avons pas d'hypothèses sur la dimension de \( F\), finie ou non.}
    \begin{equation}
        \begin{aligned}
            dg\colon E&\to \GL(F) \\
            x&\mapsto dg_x
        \end{aligned}
    \end{equation}
    est continue. En conséquence de quoi nous avons un voisinage \( U'\) de \( 0 \) pour lequel
    \begin{equation}    \label{EqSGTOfvx}
        \sup_{x\in U'}\| dg_x \|<\frac{ 1 }{2}.
    \end{equation}
    Maintenant le théorème des accroissements finis~\ref{ThoNAKKght} (\ref{val_medio_2} pour la dimension finie) nous indique que pour tout \( x,x'\in U'\) nous avons\footnote{Ici nous supposons avoir choisi \( U'\) convexe afin que tous les \( a\in \mathopen[ x , x' \mathclose]\) soient bien dans \( U'\) et donc soumis à l'inéquation \eqref{EqSGTOfvx}, ce qui est toujours possible, il suffit de prendre une boule.}
    \begin{equation}
        \| g(x')-g(x) \|\leq \sup_{a\in\mathopen[ x , x' \mathclose]}\| dg_a \| \cdot \| x-x' \|\leq \frac{ 1 }{2}\| x-x' \|,
    \end{equation}
    ce qui prouve que \( g\) est une contraction au moins sur l'ouvert \( U'\). Nous allons aussi donner une idée de la façon dont \( f\) fonctionne : si \( x_1,x_2\in U'\) alors
    \begin{subequations}
        \begin{align}
            \| x_1-x_2 \|&=\| g(x_1)-f(x_1)-g(x_2)+f(x_2) \| \\
            &\leq \| g(x_1)-g(x_2) \|+\| f(x_1)-f(x_2) \|\\
            &\leq \frac{ 1 }{2}\| x_1-x_2 \|+\| f(x_1)-f(x_2) \|,
        \end{align}
    \end{subequations}
    ce qui montre que
    \begin{equation}
        \| x_1-x_2 \|\leq 2\| f(x_1)-f(x_2) \|.
    \end{equation}
    Maintenant que nous savons que \( g\) est contractante de constante \( \frac{ 1 }{2}\) et que \( f=g+\mtu\) nous pouvons utiliser la proposition~\ref{LemGZoqknC} pour conclure que \( f\) est un homéomorphisme sur un ouvert \( U\) (partie de \( U'\)) de \( E\) et \( f^{-1}\) a une constante de Lipschitz plus petite ou égale à \( (1-\frac{ 1 }{2})^{-1}=2\).

    Nous allons maintenant prouver que \( f^{-1}\) est différentiable et que sa différentielle est donnée par \( (df^{-1})_{f(x)}=(df_x)^{-1}\).

    Soient \( a,b\in U\) et \( u=b-a\). Étant donné que \( f\) est différentiable en \( a\), il existe une fonction \( \alpha\in o(\| u \|)\) telle que
    \begin{equation}
        f(b)-f(a)-df_a(u)=\alpha(u).
    \end{equation}
    En notant \( y_a=f(a)\) et \( y_b=f(b)\) et en appliquant \( (df_a)^{-1}\) à cette dernière équation,
    \begin{equation}
        (df_a)^{-1}(y_b-y_a)-u=(df_a)^{-1} \big( \alpha(u) \big).
    \end{equation}
    Vu que \( df_a\) est bornée (et son inverse aussi), le membre de droite est encore une fonction \( \beta\) ayant la propriété \( \lim_{u\to 0}\beta(u)/\| u \|=0\); en réordonnant les termes,
    \begin{equation}
        b-a=(df_a)^{-1}(y_b-y_a)+\beta(u)
    \end{equation}
    et donc
    \begin{equation}
        f^{-1}(y_b)-f^{-1}(y_a)-(df_a)^{-1}(y_b-y_a)=\beta(u),
    \end{equation}
    ce qui prouve que \( f^{-1}\) est différentiable et que \( (df^{-1})_{y_a}=(df_a)^{-1}\).

    La différentielle \( df^{-1}\) est donc obtenue par la chaine
    \begin{equation}
    \xymatrix{%
        df^{-1}\colon f(U) \ar[r]^-{f^{-1}}     &   U'\ar[r]^-{df}&\GL(F)\ar[r]^-{\Inv}&\GL(F)
       }
    \end{equation}
    où l'application \( \Inv\colon \GL(F)\to \GL(F)\) est l'application \( X\mapsto X^{-1}\) qui est de classe \(  C^{\infty}\) par le théorème~\ref{ThoCINVBTJ}. D'autre part, par hypothèse \( df\) est une application de classe \( C^{k-1}\) et donc au minimum \( C^0\) parce que \( k\geq 1\). Enfin, l'application \( f^{-1}\colon f(U)\to U\) est continue (parce que la proposition~\ref{LemGZoqknC} précise que \( f\) est un homéomorphisme). Donc toute la chaine est continue et \( df^{-1}\) est continue. Cela entraine immédiatement que \( f^{-1}\) est \( C^1\) et donc que toute la chaine est \( C^1\).

    Par récurrence nous obtenons la chaine
    \begin{equation}
    \xymatrix{%
        df^{-1}\colon f(U) \ar[r]^-{f^{-1}}_-{C^{k-1}}     &   U'\ar[r]^-{df}_-{C^{k-1}}&\GL(F)\ar[r]^-{\Inv}_-{ C^{\infty}}&\GL(F)
       }
    \end{equation}
    qui prouve que \( df^{-1}\) est \( C^{k-1} \) et donc que \( f^{-1}\) est \( C^k\). La récurrence s'arrête ici parce que \( df\) n'est pas mieux que \( C^{k-1}\).
\end{proof}

%---------------------------------------------------------------------------------------------------------------------------
\subsection{Théorème de la fonction implicite}
%---------------------------------------------------------------------------------------------------------------------------

Nous énonçons et le démontrons le théorème de la fonction implicite dans le cas d'espaces de Banach.
\begin{theorem}[Théorème de la fonction implicite dans Banach\cite{SNPdukn}] \label{ThoAcaWho}
    Soient \( E\), \( F\) et \( G\) des espaces de Banach et des ouverts \( U\subset E\), \( V\subset F\). Nous considérons une fonction \( f\colon U\times V\to G\) de classe \( C^r\) telle que\footnote{La notation \( d_y\) est la différentielle partielle de la définition~\ref{VJM_CtSKT}.}
    \begin{equation}
        d_yf_{(x_0,y_0)}\colon F\to G
    \end{equation}
    soit un isomorphisme pour un certain \( (x_0,y_0)\in U\times V\).

    Alors nous avons des voisinages \( U_0\) de \( x_0\) dans \( E\) et \( W_0\) de \( f(x_0,y_0)\) dans \( G\) et une fonction de classe \( C^r\)
    \begin{equation}
        g\colon U_0\times W_0\to V
    \end{equation}
    telle que
    \begin{equation}
        f\big( x,g(x,w) \big)=w
    \end{equation}
    pour tout \( (x,w)\in U_0\times W_0\).

    Cette fonction \( g\) est unique au sens suivant : il existe un voisinage \( V_0 \) de \( y_0\) tel que si \( (x,y)\in U_0\times V_0\) et \( w\in W_0\) satisfont à \( f(x,y)=w\) alors \( y=g(x,w)\). Autrement dit, la fonction \( g\colon U_0\times W_0\to V_0\) est unique.
\end{theorem}
\index{théorème!fonction implicite dans Banach}

\begin{proof}
    Nous commençons par considérer la fonction
    \begin{equation}
        \begin{aligned}
            \Phi\colon U\times V&\to E\times G \\
            (x,y)&\mapsto \big( x,f(x,y) \big)
        \end{aligned}
    \end{equation}
    et sa différentielle
    \begin{subequations}
        \begin{align}
            d\Phi_{(x_0,y_0)}(u,v)&=\Dsdd{ \big( x_0+tu,f(x_0+tu,y_0+tv) \big) }{t}{0}\\
            &=\left( \Dsdd{ x_0+tu }{t}{0},\Dsdd{ f(x_0+tu,y_0+tv) }{t}{0} \right)\\
            &=\left( u,df_{(x_0,y_0)}(u,v) \right).
        \end{align}
    \end{subequations}
    Nous utilisons alors la proposition~\ref{PropLDN_nHWDF} pour conclure que
    \begin{equation}
        d\Phi_{(x_0,y_0)}(u,v)=\big( u,(d_1f)_{(x_0,y_0)}(u)+(d_2f)_{(x_0,y_0)}(v) \big),
    \end{equation}
    mais comme par hypothèse \( (d_2f)_{(x_0,y_0)}\colon F\to G\) est un isomorphisme, l'application \( d\Phi_{(x_0,y_0)}\colon E\times F\to E\times G\) est également un isomorphisme. Par conséquent le théorème d'inversion locale~\ref{ThoXWpzqCn} nous indique qu'il existe un voisinage \( \mO\) de \( (x_0,y_0)\) et \( \mP\) de \( \Phi(x_0,y_0)\) tels que \( \Phi\colon \mO\to \mP\) soit une bijection et \( \Phi^{-1}\colon \mP\to \mO\) soit de classe \( C^r\). Vu que \( \mP\) est un voisinage de
    \begin{equation}
        \Phi(x_0,y_0)=\big( x_0,f(x_0,y_0) \big),
    \end{equation}
    nous pouvons par~\ref{PropDXR_KbaLC} le choisir un peu plus petit de telle sorte à avoir \( \mP=U_0\times W_0\) où \( U_0\) est un voisinage de \( x_0\) et \( W_0\) un voisinage de \( f(x_0,y_0)\). Dans ce cas nous devons obligatoirement aussi restreindre \( \mO\) à \( U_0\times V_0\) pour un certain voisinage \( V_0\) de \( y_0\). L'application \( \Phi^{-1}\) a obligatoirement la forme
    \begin{equation}    \label{EqMHT_QrHRn}
        \begin{aligned}
            \Phi^{-1}\colon U_0\times W_0&\to U_0\times V_0 \\
            (x,w)&\mapsto \big( x,g(x,w) \big)
        \end{aligned}
    \end{equation}
    pour une certaine fonction \( g\colon U_0\times W_0\to V\). Cette fonction \( g\) est la fonction cherchée parce qu'en appliquant \( \Phi\) à \eqref{EqMHT_QrHRn},
    \begin{equation}
        (x,w)=\Phi\big( x,g(x,w) \big)=\Big( x,f\big( x,g(x,w) \big) \Big),
    \end{equation}
    qui nous dit que pour tout \( x\in U_0\) et tout \( w\in W_0\) nous avons
    \begin{equation}
        f\big( x,g(x,w) \big)=w.
    \end{equation}

    Si vous avez bien suivi le sens de l'équation \eqref{EqMHT_QrHRn} alors vous avez compris l'unicité. Sinon, considérez \( (x,y)\in U_0\times V_0\) et \( w\in W_0\) tels que \( f(x,y)=w\). Alors \( \big( x,f(x,y) \big)=(x,w)\) et
    \begin{equation}
        \Phi(x,y)=(x,w).
    \end{equation}
    Mais vu que \( \Phi\colon U_0\times V_0\to U_0\times W_0\) est une bijection, cette relation définit de façon univoque l'élément \( (x,y)\) de \( U_0\times V_0\), qui ne sera autre que \( g(x,w)\).
\end{proof}

Le théorème de la fonction implicite s'énonce de la façon suivante pour des espaces de dimension finie.
% Attention : avant de citer ce théorème, voir s'il est suffisant. Ici \varphi a une variable; dans l'autre énoncé il en a deux.
\begin{theorem}[Théorème de la fonction implicite en dimension finie]   \label{ThoRYN_jvZrZ}
    Soit une fonction \( F\colon \eR^n\times \eR^m\to \eR^m\) de classe \( C^k\) et \( (\alpha,\beta)\in \eR^n\times \eR^m\) tels que
    \begin{enumerate}
        \item
            \( F(\alpha,\beta)=0\),
        \item
            \( \frac{ \partial (F_1,\ldots, F_m) }{ \partial (y_1,\ldots, y_m) }\neq 0\), c'est-à-dire que \( (d_yF)_{(\alpha,\beta)} \) est inversible.
    \end{enumerate}
    Alors il existe un voisinage ouvert \( V\) de \( \alpha\) dans \( \eR^n\), un voisinage ouvert \( W\) de \( \beta\) dans \( \eR^m\) et une application \( \varphi\colon V\to W\) de classe \( C^k\)  telle que pour tout \( x\in V\) on ait
    \begin{equation}
        F\big( x,\varphi(x) \big)=0.
    \end{equation}
    De plus si \( (x,y)\in V\times W\) satisfait à \( F(x,y)=0\), alors \( y=\varphi(x)\).
\end{theorem}
\index{théorème!fonction implicite dans \( \eR^n\)}

\begin{remark}\label{RemPYA_pkTEx}
    Notons que cet énoncé est tourné un peu différemment en ce qui concerne le nombre de variables dont dépend la fonction implicite : comparez
    \begin{subequations}
        \begin{align}
            f\big( x,g(x,w) \big)=w\\
            F\big( x,\varphi(x) \big)=0.
        \end{align}
    \end{subequations}
    Le deuxième est un cas particulier du premier en posant
    \begin{equation}
        F(x,y)=f(x,y)-f(x_0,y_0)
    \end{equation}
    et donc en considérant \( w\) comme valant la constante \( f(x_0,y_0)\); dans ce cas la fonction \( g\) ne dépend plus que de la variable \( x\).

\end{remark}

\begin{example}
    La remarque~\ref{RemPYA_pkTEx} signifie entre autres que le théorème~\ref{ThoAcaWho} est plus fort que~\ref{ThoRYN_jvZrZ} parce que le premier permet de choisir la valeur d'arrivée. Parlons de l'exemple classique du cercle et de la fonction \( f(x,y)=x^2+y^2\). Nous savons que
    \begin{equation}
        f(\alpha,\beta)=1.
    \end{equation}
    Alors le théorème~\ref{ThoAcaWho} nous donne une fonction \( g\) telle que
    \begin{equation}
        f(x,g(x,r))=r
    \end{equation}
    tant que \( x\) est proche de \( \alpha\), que \( r\) est proche de \( 1\) et que \( g\) donne des valeurs proches de \( \beta\).

    L'énoncé~\ref{ThoRYN_jvZrZ} nous oblige à travailler avec la fonction \( F(x,y)=x^2+y^2-1\), de telle sorte que
    \begin{equation}
        F(\alpha,\beta)=0,
    \end{equation}
    et que nous ayons une fonction \( \varphi\) telle que
    \begin{equation}
        F(x,\varphi(x))=0.
    \end{equation}
    La fonction \( \varphi\) ne permet donc que de trouver des points sur le cercle de rayon \( 1\).
\end{example}

%---------------------------------------------------------------------------------------------------------------------------
\subsection{Exemple}
%---------------------------------------------------------------------------------------------------------------------------

Le théorème de la fonction implicite a pour objet de donner l'existence de la fonction $\varphi$. Maintenant nous pouvons dire beaucoup de choses sur les dérivées de $\varphi$ en considérant la fonction
\begin{equation}
    x\mapsto F\big( x,\varphi(x) \big).
\end{equation}
Par définition de $\varphi$, cette fonction est toujours nulle. En particulier, nous pouvons dériver l'équation
\begin{equation}
    F\big( x,\varphi(x) \big)=0,
\end{equation}
et nous trouvons plein de choses.

\begin{example} \label{EXooTLNAooCJHPnq}
    Prenons par exemple la fonction
    \begin{equation}
        F\big( (x,y),z \big)=ze^z-x-y,
    \end{equation}
    et demandons nous ce que nous pouvons dire sur la fonction $z(x,y)$ telle que
    \begin{equation}
        F\big( x,y,z(x,y) \big)=0,
    \end{equation}
    c'est-à-dire telle que
    \begin{equation}        \label{EqDefZImplExemple}
        z(x,y) e^{z(x,y)}-x-y=0.
    \end{equation}
    pour tout $x$ et $y\in\eR$. Nous pouvons facilement trouver $z(0,0)$ parce que
    \begin{equation}
        z(0,0) e^{z(0,0)}=0,
    \end{equation}
    donc $z(0,0)=0$.

    Nous pouvons dire des choses sur les dérivées de $z(x,y)$. Voyons par exemple $(\partial_xz)(x,y)$. Pour trouver cette dérivée, nous dérivons la relation \eqref{EqDefZImplExemple} par rapport à $x$. Ce que nous trouvons est
    \begin{equation}
        (\partial_xz)e^z+ze^z(\partial_xz)-1=0.
    \end{equation}
    Cette équation peut être résolue par rapport à $\partial_xz$~:
    \begin{equation}
        \frac{ \partial z }{ \partial x }(x,y)=\frac{1}{ e^z(1+z) }.
    \end{equation}
    Remarquez que cette équation ne donne pas tout à fait la dérivée de $z$ en fonction de $x$ et $y$, parce que $z$ apparaît dans l'expression, alors que $z$ est justement la fonction inconnue. En général, c'est la vie, nous ne pouvons pas faire mieux.

    Dans certains cas, on peut aller plus loin. Par exemple, nous pouvons calculer cette dérivée au point $(x,y)=(0,0)$ parce que $z(0,0)$ est connu :
    \begin{equation}
        \frac{ \partial z }{ \partial x }(0,0)=1.
    \end{equation}
    Cela est pratique pour calculer, par exemple, le développement en Taylor de $z$ autour de $(0,0)$.
\end{example}

\begin{example}
    Est-ce que l'équation \( e^{y}+xy=0\) définit au moins localement une fonction \( y(x)\) ? Nous considérons la fonction
    \begin{equation}
        f(x,y)=\begin{pmatrix}
            x    \\
            e^{y}+xy
        \end{pmatrix}
    \end{equation}
    La différentielle de cette application est
    \begin{equation}
            df_{(0,0)}(u)=\frac{ d }{ dt }\Big[ f(tu_1,tu_2) \Big]_{t=0}
            =\frac{ d }{ dt }\begin{pmatrix}
                tu_1    \\
                e^{tu_2}+t^2u_1u_2
            \end{pmatrix}_{t=0}
            =\begin{pmatrix}
                u_1    \\
                u_2
            \end{pmatrix}.
    \end{equation}
    L'application \( f\) définit donc un difféomorphisme local autour des points \( (x_0,y_0)\) et \( f(x_0,y_0)\). Soit \( (u,0)\) un point dans le voisinage de \( f(x_0,y_0)\). Alors il existe un unique \( (x,y)\) tel que
    \begin{equation}
        f(x,y)=\begin{pmatrix}
               x \\
            e^y+xy
        \end{pmatrix}=
        \begin{pmatrix}
            u    \\
                0
        \end{pmatrix}.
    \end{equation}
    Nous avons automatiquement \( x=u\) et \( e^y+xy=0\). Notons toutefois que pour que ce procédé donne effectivement une fonction implicite \( y(x)\) nous devons avoir des points de la forme \( (u,0)\) dans le voisinage de \( f(x_0,y_0)\).
\end{example}


%+++++++++++++++++++++++++++++++++++++++++++++++++++++++++++++++++++++++++++++++++++++++++++++++++++++++++++++++++++++++++++
\section{Décomposition polaire (régularité)}
%+++++++++++++++++++++++++++++++++++++++++++++++++++++++++++++++++++++++++++++++++++++++++++++++++++++++++++++++++++++++++++

\begin{normaltext}      \label{NomDJMUooTRUVkS}
    Nous allons montrer que l'application
    \begin{equation}
        \begin{aligned}
            f\colon S^{++}(n,\eR)&\to S^{++}(n,\eR) \\
            A&\mapsto \sqrt{A}
        \end{aligned}
    \end{equation}
    est une difféomorphisme.

    Cependant \( S^{++}(n,\eR)\) n'est pas un ouvert de \( \eM(n,\eR)\) et nous ne savons pas ce qu'est la différentielle d'une application non définie sur un ouvert. Nous allons donc en réalité montrer que l'application racine carrée existe sur un voisinage de chacun des points de \( S^{++}(n,\eR)\). Et comme une union quelconque d'ouverts est un ouvert, la fonction \( f\) sera bien définie sur un ouvert de \( \eM(n,\eR)\).
\end{normaltext}

\begin{lemma}       \label{LemLBFOooDdNcgy}
    L'application
    \begin{equation}
        \begin{aligned}
            f\colon S^{++}(n,\eR)&\to S^{++}(n,\eR) \\
            A&\mapsto A^2
        \end{aligned}
    \end{equation}
    est un \(  C^{\infty}\)-difféomorphisme.
\end{lemma}

\begin{proof}
    Prouvons d'abord que \( f\) prend ses valeurs dans \( S^{++}(n,\eR)\). Si \( A\in S^{++}(n,\eR)\) alors par la diagonalisation~\ref{ThoeTMXla} elle s'écrit \( A=QDQ^{-1}\) où \( D\) est diagonale avec des nombres strictement positifs sur la diagonale. Avec cela, \( A^2=QD^2Q^{-1}\) où \( D^2\) contient encore des nombres strictement positifs sur la diagonale.

    L'application \( f\) étant essentiellement des polynômes en les entrées de \( A\), elle est de classe \( C^{\infty}\).

    Passons à l'étude de la différentielle. Comme mentionné en~\ref{NomDJMUooTRUVkS} nous allons en réalité voir \( f\) sur un ouvert de \( \eM(n,\eR)\) autour de \( A\in S^{++}(n,\eR)\). Par conséquent si \( A\in S^{++}(n,\eR)\),
    \begin{subequations}
        \begin{align}
            df\colon S^{++}(n,\eR)&\to \aL\big( \eM(n,\eR),\eM(n,\eR) \big)\\
            df_A\colon \eM(n,\eR)&\to \eM(n,\eR).
        \end{align}
    \end{subequations}
    Le calcul de \( df_A\) est facile. Soit \( u\in \eM(n,\eR)\) et faisons le calcul en utilisant la formule du lemme \eqref{LemdfaSurLesPartielles} :
    \begin{subequations}
        \begin{align}
            df_A(u)&=\Dsdd{ f(A+tu) }{t}{0}\\
            &=\Dsdd{ A^2+tAu+tuA+t^2u^2 }{t}{0}\\
            &=Au+uA.
        \end{align}
    \end{subequations}
    Nous allons utiliser le théorème d'inversion locale~\ref{ThoXWpzqCn} à la fonction \( f\). Dans la suite, \( A\) est une matrice de \( S^{++}(n,\eR)\).

    \begin{subproof}
        \item[\( df_A\) est injective]
            Soit \( M\in \eM(n,\eR)\) dans le noyau de \( df_A\). En posant \( M'=A^{-1}MQ\) nous avons \( M=QM'Q^{-1}\) et on applique \( df_A\) à \( QM'Q^{-1}\) :
            \begin{equation}
                df_A(QM'Q^{-1})=Q\big( DM+MD \big)Q^{-1}.
            \end{equation}
            où \( D=\begin{pmatrix}
                \lambda_1    &       &       \\
                    &   \ddots    &       \\
                    &       &   \lambda_n
                \end{pmatrix}\) avec \( \lambda_i>0\). La matrice \( D\) est inversible. Nous avons \( M'=-DM'D^{-1}\), et en coordonnées,
                \begin{subequations}
                    \begin{align}
                        M'_{ij}&=-\sum_{kl}D_{ikM'_{kl}}D^{-1}_{lj}\\
                        &=-\sum_{kl}\lambda_i\delta_{ik}M'_{kl}\frac{1}{ \lambda_j }\delta_{lj}\\
                        &=-\frac{ \lambda_i }{ \lambda_i }M'_{ij}.
                    \end{align}
                \end{subequations}
                C'est-à-dire que \( M'_{ij}=-\frac{ \lambda_i }{ \lambda_j }M'_{ij}\) avec \( -\frac{ \lambda_i }{ \lambda_j }<0\). Cela implique \( M'=0\) et par conséquent \( M=0\).
            \item[\( df_A\) est surjective]
                Soit \( N\in \eM(n,\eR)\); nous cherchons \( M\in \eM(n,\eR)\) tel que \( df_A(M)=N\). Nous posons \( N'=Q^{-1} NQ\) et \( M=QM'Q^{-1}\), ce qui nous donne à résoudre \( df_D(M')=N'\). Passons en coordonnées :
                \begin{equation}
                        (DM'+M'D)_{ij}=\sum_k(\delta_{ik}\lambda_iM'_{kj}+M'_{ik}\delta_{kj}\lambda_j)=M'_{ij}(\lambda_i+\lambda_j)
                \end{equation}
                où \( \lambda_i+\lambda_j\neq 0\). Il suffit donc de prendre la matrice \( M'\) donnée par
                \begin{equation}
                    M'_{ij}=\frac{1}{ \lambda_i+\lambda_j }N'_{ij}
                \end{equation}
                pour que \( df_A(M')=N'\).
    \end{subproof}

    Le théorème d'inversion locale donne un voisinage \( V\) de $A$ dans \( \eM(n,\eR)\) et un voisinage \( W\) de \( A^2\) dans \( \eM(n,\eR)\) tels que \( f\colon V\to W\) soit une bijection  et \( f^{-1}\colon W\to V\) soit de même régularité, en l'occurrence \( C^{\infty}\).
\end{proof}

\begin{remark}
    Oui, il y a des matrices non symétriques qui ont une unique racine carrée.
\end{remark}

La proposition suivante, qui dépend du le théorème d'inversion locale par le lemme~\ref{LemLBFOooDdNcgy}, donne plus de régularité à la décomposition polaire donnée dans le théorème~\ref{ThoLHebUAU}.
\begin{proposition}[Décomposition polaire : cas réel (suite)]       \label{PropWCXAooDuFMjn}
    L'application
    \begin{equation}
        \begin{aligned}
            f\colon \gO(n,\eR)\times S^{++}(n,\eR)&\to \GL(n,\eR) \\
            (Q,S)&\mapsto SQ
        \end{aligned}
    \end{equation}
    est un difféomorphisme de classe \( C^{\infty}\).
\end{proposition}

\begin{proof}
    Si \( M\) est donnée dans \( \GL(n,\eR)\) alors la décomposition polaire\footnote{Proposition~\ref{ThoLHebUAU}.} \( M=QS\) est donnée par \( S=\sqrt{MM^t}\) et \( Q=MS^{-1}\). Autrement dit, si nous considérons la fonction de décomposition polaire
    \begin{equation}
        f\colon \gO(n,\eR)\times S^{++}(n,\eR)\to \GL(n,\eR)
    \end{equation}
    alors
    \begin{equation}
        f^{-1}(M)=\big(  M(\sqrt{MM^t})^{-1},\sqrt{MM^t}  \big).
    \end{equation}
    Nous avons vu dans le lemme~\ref{LemLBFOooDdNcgy} que la racine carrée était un \( C^{\infty}\)-difféomorphisme. Le reste n'étant que des produits de matrices, la régularité est de mise.
\end{proof}

%+++++++++++++++++++++++++++++++++++++++++++++++++++++++++++++++++++++++++++++++++++++++++++++++++++++++++++++++++++++++++++
\section{Théorème de Von Neumann}
%+++++++++++++++++++++++++++++++++++++++++++++++++++++++++++++++++++++++++++++++++++++++++++++++++++++++++++++++++++++++++++

\begin{lemma}[\cite{KXjFWKA}]
    Soit \( G\), un sous-groupe fermé de \( \GL(n,\eR)\) et
    \begin{equation}
        \mL_G=\{ m\in \eM(n,\eR)\tq  e^{tm}\in G\,\forall t\in\eR \}.
    \end{equation}
    Alors \( \mL_G\) est un sous-espace vectoriel de \( \eM(n,\eR)\).
\end{lemma}

\begin{proof}
    Si \( m\in\mL_G\), alors \( \lambda m\in\mL_G\) par construction. Le point délicat à prouver est le fait que si \( a,b\in \mL_G\), alors \( a+b\in\mL_G\). Soit \( a\in \eM(n,\eR)\); nous savons qu'il existe une fonction \( \alpha_a\colon \eR\to \eM\) telle que
    \begin{equation}
        e^{ta}=\mtu+ta+\alpha_a(t)
    \end{equation}
    et
    \begin{equation}
        \lim_{t\to 0} \frac{ \alpha(t) }{ t }=0.
    \end{equation}
    Si \( a\) et \( b\) sont dans \( \mL_G\), alors \(  e^{ta} e^{tb}\in G\), mais il n'est pas vrai en général que cela soit égal à \(  e^{t(a+b)}\). Pour tout \( k\in \eN\) nous avons
    \begin{equation}
        e^{a/k} e^{b/k}=\left( \mtu+\frac{ a }{ k }+\alpha_a(\frac{1}{ k }) \right)\left( \mtu+\frac{ b }{ k }+\alpha_b(\frac{1}{ k }) \right)=\mtu+\frac{ a+b }{2}+\beta\left( \frac{1}{ k } \right)
    \end{equation}
   où \( \beta\colon \eR\to \eM\) est encore une fonction vérifiant \( \beta(t)/t\to 0\). Si \( k\) est assez grand, nous avons
   \begin{equation}
       \left\| \frac{ a+b }{ k }+\beta(\frac{1}{ k })  \right\|<1,
   \end{equation}
   et nous pouvons profiter du lemme~\ref{LemQZIQxaB} pour écrire alors
   \begin{equation}
       \left(  e^{a/k} e^{b/k} \right)^k= e^{k\ln\big(\mtu+\frac{ a+b }{ k }+\beta(\frac{1}{ k })\big)}.
   \end{equation}
   Ce qui se trouve dans l'exponentielle est
   \begin{equation}
       k\left[ \frac{ a+b }{ k }+\alpha( \frac{1}{ k })+\sigma\left( \frac{ a+b }{ k }+\alpha(\frac{1}{ k }) \right) \right].
   \end{equation}
   Les diverses propriétés vues montrent que le tout tend vers \( a+b\) lorsque \( k\to \infty\). Par conséquent
   \begin{equation}
       \lim_{k\to \infty} \left(  e^{a/k} e^{b/k} \right)^k= e^{a+b}.
   \end{equation}
   Ce que nous avons prouvé est que pour tout \( t\), \(  e^{t(a+b)}\) est une limite d'éléments dans \( G\) et est donc dans \( G\) parce que ce dernier est fermé.
\end{proof}

Vu que \( \mL_G\) est un sous-espace vectoriel de \( \eM(n,\eR)\), nous pouvons considérer un supplémentaire \( M\).

\begin{lemma}   \label{LemHOsbREC}
    Il n'existe pas se suites \( (m_k)\) dans \( M\setminus\{ 0 \}\) convergeant vers zéro et telle que \(  e^{m_k}\in G\) pour tout \( k\).
\end{lemma}

\begin{proof}
    Supposons que nous ayons \( m_k\to 0\) dans \( M\setminus\{ 0 \}\) avec \(  e^{m_k}\in G\). Nous considérons les éléments \( \epsilon_k=\frac{ m_k }{ \| m_k \| }\) qui sont sur la sphère unité de \(\GL(n,\eR)\). Quitte à prendre une sous-suite, nous pouvons supposer que cette suite converge, et vu que \( M\) est fermé, ce sera vers \( \epsilon\in M\) avec \( \| \epsilon \|=1\). Pour tout \( t\in \eR\) nous avons
    \begin{equation}
        e^{t\epsilon}=\lim_{k\to \infty}  e^{t\epsilon_k}.
    \end{equation}
    En vertu de la décomposition d'un réel en partie entière et décimale, pour tout \( k\) nous avons \( \lambda_k\in \eZ\) et \( | \mu_k |\leq \frac{ 1 }{2}\) tel que \( t/\| m_k \|=\lambda_k+\mu_k\). Avec ça,
    \begin{equation}
        e^{t\epsilon}=\lim_{k\to \infty}\exp\Big( \frac{ t }{ m_k }m_k \Big)=\lim_{k\to \infty}  e^{\lambda_km_k} e^{\mu_km_k}.
    \end{equation}
    Pour tout \( k\) nous avons \(  e^{\lambda_km_k}\in G\). De plus \( | \mu_k |\) étant borné et \( m_k\) tendant vers zéro nous avons \(  e^{\mu_km_k}\to 1\). Au final
    \begin{equation}
        e^{t\epsilon}=\lim_{k\to \infty}  e^{t\epsilon_k}\in G
    \end{equation}
    Cela signifie que \( \epsilon\in\mL_G\), ce qui est impossible parce que nous avions déjà dit que \( \epsilon\in M\setminus\{ 0 \}\).
\end{proof}

\begin{lemma}   \label{LemGGTtxdF}
    L'application
    \begin{equation}
        \begin{aligned}
            f\colon \mL_G\times M&\to \GL(n,\eR) \\
            l,m&\mapsto  e^{l} e^{m}
        \end{aligned}
    \end{equation}
    est un difféomorphisme local entre un voisinage de \( (0,0)\) dans \( \eM(n,\eR)\) et un voisinage de \( \mtu\) dans \( \exp\big( \eM(n,\eR) \big)\).
\end{lemma}
Notons que nous ne disons rien de \(  e^{\eM(n,\eR)}\). Nous n'allons pas nous embarquer à discuter si ce serait tout \( \GL(n,\eR)\)\footnote{Vu les dimensions y'a tout de même peu de chance.} ou bien si ça contiendrait ne fut-ce que \( G\).

\begin{proof}
    Le fait que \( f\) prenne ses valeurs dans \( \GL(n,\eR)\) est simplement dû au fait que les exponentielles sont toujours inversibles. Nous considérons ensuite la différentielle : si \( u\in \mL_G\) et \( v\in M\) nous avons
    \begin{equation}
        df_{(0,0)}(u,v)=\Dsdd{ f\big( t(u,v) \big) }{t}{0}=\Dsdd{  e^{tu} e^{tv} }{t}{0}=u+v.
    \end{equation}
    L'application \( df_0\) est donc une bijection entre \( \mL_G\times M\) et \( \eM(n,\eR)\). Le théorème d'inversion locale~\ref{ThoXWpzqCn} nous assure alors que \( f\) est une bijection entre un voisinage de \( (0,0)\) dans \( \mL_G\times M\) et son image. Mais vu que \( df_0\) est une bijection avec \( \eM(n,\eR)\), l'image en question contient un ouvert autour de \( \mtu\) dans \( \exp\big( \eM(n,\eR) \big)\).
\end{proof}

\begin{theorem}[Von Neumann\cite{KXjFWKA,ISpsBzT,Lie_groups}]       \label{ThoOBriEoe}
    Tout sous-groupe fermé de \( \GL(n,\eR)\) est une sous-variété de \( \GL(n,\eR)\).
\end{theorem}
\index{théorème!Von Neumann}
\index{exponentielle!de matrice!utilisation}

\begin{proof}
    Soit \( G\) un tel groupe; nous devons prouver que c'est localement difféomorphe à un ouvert de \( \eR^n\). Et si on est pervers, on ne va pas faire localement difféomorphe à un ouvert de \( \eR^n\), mais à un ouvert d'un espace vectoriel de dimension finie. Nous allons être pervers.

    Étant donné que pour tout \( g\in G\), l'application
    \begin{equation}
        \begin{aligned}
            L_g\colon G&\to G \\
            h&\mapsto gh
        \end{aligned}
    \end{equation}
    est de classe \(  C^{\infty}\) et d'inverse \(  C^{\infty}\), il suffit de prouver le résultat pour un voisinage de \( \mtu\).

    Supposons d'abord que \( \mL_G=\{ 0 \}\). Alors \( 0\) est un point isolé de \( \ln(G)\); en effet si ce n'était pas le cas nous aurions un élément \( m_k\) de \( \ln(G)\) dans chaque boule \( B(0,r_k)\). Nous aurions alors \( m_k=\ln(a_k)\) avec \( a_k\in G\) et donc
    \begin{equation}
        e^{m_k}=a_k\in G.
    \end{equation}
    De plus \( m_k\) appartient forcément à \( M\) parce que \( \mL_G\) est réduit à zéro. Cela nous donnerait une suite \( m_k\to 0\) dans \( M\) dont l'exponentielle reste dans \( G\). Or cela est interdit par le lemme~\ref{LemHOsbREC}. Donc \( 0\) est un point isolé de \( \ln(G)\). L'application \(\ln\) étant continue\footnote{Par le lemme~\ref{LemQZIQxaB}.}, nous en déduisons que \( \mtu\) est isolé dans \( G\). Par le difféomorphisme \( L_g\), tous les points de \( G\) sont isolés; ce groupe est donc discret et par voie de conséquence une variété.

    Nous supposons maintenant que \( \mL_G\neq\{ 0 \}\). Nous savons par la proposition~\ref{PropXFfOiOb} que
    \begin{equation}
        \exp\colon \eM(n,\eR)\to \eM(n,\eR)
    \end{equation}
    est une application \(  C^{\infty}\) vérifiant \( d\exp_0=\id\). Nous pouvons donc utiliser le théorème d'inversion locale~\ref{ThoXWpzqCn} qui nous offre donc l'existence d'un voisinage \( U\) de \( 0\) dans \( \eM(n,\eR)\) tel que \( W=\exp(U)\) soit un ouvert de \( \GL(n,\eR)\) et que \( \exp\colon U\to W\) soit un difféomorphisme de classe \(  C^{\infty}\).

    Montrons que quitte à restreindre \( U\) (et donc \( W\) qui reste par définition l'image de \( U\) par \( \exp\)), nous pouvons avoir \( \exp\big( U\cap\mL_G \big)=W\cap G\). D'abord \( \exp(\mL_G)\subset G\) par construction. Nous avons donc \( \exp\big( U\cap\mL_G \big)\subset W\cap G\). Pour trouver une restriction de \( U\) pour laquelle nous avons l'égalité, nous supposons que pour tout ouvert \( \mO\) dans \( U\),
    \begin{equation}
        \exp\colon \mO\cap\mL_G\to \exp(\mO)\cap G
    \end{equation}
    ne soit pas surjective. Cela donnerait un élément de \( \mO\cap\complement\mL_G\) dont l'image par \( \exp\) n'est pas dans \( G\). Nous construisons ainsi une suite en considérant une boule \( B(0,\frac{1}{ k })\) inclue à \( U\) et \( x_k\in B(0,\frac{1}{ k })\cap\complement\mL_G\) vérifiant \(  e^{x_k}\in G\). Vu le choix des boules nous avons évidemment \( x_k\to 0\).

    L'élément \(  e^{x_k}\) est dans \(  e^{\eM(n,\eR)}\) et le difféomorphisme du lemme~\ref{LemGGTtxdF}\quext{Il me semble que l'utilisation de ce lemme manque à l'avant-dernière ligne de la preuve chez \cite{KXjFWKA}.} nous donne \( (l_k,m_k)\in \mL_G\times M\) tel que \(  e^{l_k} e^{m_k}= e^{x_k}\). À ce point nous considérons \( k\) suffisamment grand pour que \(  e^{x_k}\) soit dans la partie de l'image de \( f\) sur lequel nous avons le difféomorphisme. Plus prosaïquement, nous posons
    \begin{equation}
        (l_k,m_k)=f^{-1}( e^{x_k})
    \end{equation}
    et nous profitons de la continuité pour permuter la limite avec \( f^{-1}\) :
    \begin{equation}
        \lim_{k\to \infty} (l_k,m_k)=f^{-1}\big( \lim_{k\to \infty}  e^{x_k} \big)=f^{-1}(\mtu)=(0,0).
    \end{equation}
    En particulier \( m_k\to 0\) alors que \(  e^{m_k}= e^{x_k} e^{-l_k}\in G\). La suite \( m_k\) viole le lemme~\ref{LemHOsbREC}. Nous pouvons donc restreindre \( U\) de telle façon à avoir
    \begin{equation}
        \exp\big( U\cap\mL_G \big)=W\cap G.
    \end{equation}
    Nous avons donc un ouvert de \( \mL_G\) (l'ouvert \( U\cap\mL_G\)) qui est difféomorphe avec l'ouvert \( W\cap G\) de \( G\). Donc \( G\) est une variété et accepte \( \mL_G\) comme carte locale.

\end{proof}

\begin{remark}
    En termes savants, nous avons surtout montré que si \( G\) est un groupe de Lie d'algèbre de Lie \( \lG\), alors l'exponentielle donne un difféomorphisme local entre \( \lG\) et \( G\).
\end{remark}

%+++++++++++++++++++++++++++++++++++++++++++++++++++++++++++++++++++++++++++++++++++++++++++++++++++++++++++++++++++++++++++
\section{Recherche d'extrémums}
%+++++++++++++++++++++++++++++++++++++++++++++++++++++++++++++++++++++++++++++++++++++++++++++++++++++++++++++++++++++++++++

%---------------------------------------------------------------------------------------------------------------------------
\subsection{Extrema à une variable}
%---------------------------------------------------------------------------------------------------------------------------

\begin{definition}
Soit $f\colon A\subset \eR\to \eR$ et $a\in A$. Le point $a$ est un \defe{maximum local}{maximum!local} de $f$ s'il existe un voisinage $\mU$ de $a$ tel que $f(a)\geq f(x)$ pour tout $x\in\mU\cap A$. Le point $a$ est un \defe{maximum global}{maximum!global} si $f(a)\geq g(x)$ pour tout $x\in A$.
\end{definition}

La proposition basique à utiliser lors de la recherche d'extrémums est la suivante :
\begin{proposition}     \label{PROPooNVKXooXtKkuz}
Soit $f\colon A\subset\eR\to \eR$ et $a\in\Int(A)$. Supposons que $f$ est dérivable en $a$. Si $a$ est un \href{http://fr.wikipedia.org/wiki/Extremum}{extrémum} local, alors $f'(a)=0$.
\end{proposition}

La réciproque n'est pas vraie, comme le montre l'exemple de la fonction $x\mapsto x^3$ en $x=0$ : sa dérivée est nulle et pourtant $x=0$ n'est ni un maximum ni un minimum local.

Cette proposition ne sert donc qu'à sélectionner des \emph{candidats} extrémum. Afin de savoir si ces candidats sont des extrémums, il y a la proposition suivante.
\begin{proposition}
Soit $f\colon I\subset \eR\to \eR$, une fonction de classe $C^k$ au voisinage d'un point $a\in\Int I$. Supposons que
\begin{equation}
    f'(a)=f''(a)=\ldots=f^{(k-1)}(a)=0,
\end{equation}
et que
\begin{equation}
    f^{(k)}(a)\neq 0.
\end{equation}
Dans ce cas,
\begin{enumerate}

\item
Si $k$ est pair, alors $a$ est un point d'extrémum local de $f$, c'est un minimum si $f^{(k)}(a)>0$, et un maximum si $f^{(k)}(a)<0$,
\item
Si $k$ est impair, alors $a$ n'est pas un extrémum local de $f$.

\end{enumerate}
\end{proposition}

Note : jusqu'à présent nous n'avons rien dit des extrémums \emph{globaux} de $f$. Il n'y a pas grand chose à en dire. Si un point d'extrémum global est situé dans l'intérieur du domaine de $f$, alors il sera extrémum local (a fortiori). Ou alors, le maximum global peut être sur le bord du domaine. C'est ce qui arrive à des fonctions strictement croissantes sur un domaine compact.

Une seule certitude : si une fonction est continue sur un compact, elle possède une minimum et un maximum global par le théorème~\ref{ThoMKKooAbHaro}.

Soit une fonction $f\colon I\to \eR$, et soit $a\in I$. Si $f'(a)>0$, alors la tangente au graphe de $f$ au point $\big( a,f(a) \big)$ sera une droite croissante (coefficient directeur positif). Cela ne veut pas spécialement dire que la fonction elle-même sera croissante, mais en tout cas cela est un bon indice.

\begin{example}
	Si $f(x)=x^2$, il est connu que $f'(x)=2x$. Nous avons donc que $f'$ est positive si $x\geq 0$ et $f'>$ est négative si $x<0$. Cela correspond bien au fait que $x^2$ est décroissante sur $\mathopen] -\infty , 0 \mathclose[$ et croissante sur $\mathopen] 0 , \infty \mathclose[$.
\end{example}

Sur la figure~\ref{LabelFigWIRAooTCcpOV}, nous avons dessiné la fonction $f(x)=x\cos(x)$ et sa dérivée. Nous voyons que partout où la dérivée est négative, la fonction est décroissante tandis que, inversement, partout où la dérivée est positive, la fonction est croissante.
\newcommand{\CaptionFigWIRAooTCcpOV}{La fonction $f(x)=x\cos(x)$ en bleu et sa dérivée en rouge.}
\input{auto/pictures_tex/Fig_WIRAooTCcpOV.pstricks}

Les extrémums de la fonction $f$ sont donc placés là où $f'$ change de signe. En effet si $f'(x)<0$ pour $x<a$ et $f'(x)>0$ pour $x>a$, la fonction est décroissante jusqu'à $a$ et est ensuite croissante. Cela signifie que la fonction connait un creux en $a$. Le point $a$ est donc un minimum de la fonction.

Attention cependant. Le fait que $f'(a)=0$ ne signifie pas automatiquement que $f$ a un maximum ou un minimum en $a$. Nous avons par exemple tracé sur la figure~\ref{LabelFigVBOIooRHhKOH} les fonctions $x^3$ et sa dérivée. Il est à noter que, conformément à ce que l'on pense, certes la dérivée s'annule en $x=0$, mais elle ne change pas de signe.

\newcommand{\CaptionFigVBOIooRHhKOH}{La dérivée de $x^3$ s'annule en $x=0$, mais ce n'est ni un minimum ni un maximum.}
\input{auto/pictures_tex/Fig_VBOIooRHhKOH.pstricks}

%---------------------------------------------------------------------------------------------------------------------------
\subsection{Extrema libre}
%---------------------------------------------------------------------------------------------------------------------------

\begin{definition}      \label{DEFooYJLZooLkEAYf}
Un point $a$ à l'intérieur du domaine d'une fonction $f\colon A\subset\eR^n\to \eR$ est un \defe{point critique}{critique!point} de $f$ lorsque $df(a)=0$.
\end{definition}

Ces points sont analogues aux points où la dérivée d'une fonction sur $\eR$ s'annule. Les points critiques de $f$ sont dons les candidats à être des points d'extrémum.

Dans le cas d'une fonction de deux variables,l la proposition~\ref{PROPooFWZYooUQwzjW} nous permet de voir \( (d^2f)_a\) comme étant la matrice
\begin{equation}
    d^2f(a)=\begin{pmatrix}
    \frac{ d^2f  }{ dx^2 }(a)   &   \frac{ d^2f  }{ dx\,dy }(a) \\
    \frac{ d^2f  }{ dy\,dx }(a)     &   \frac{ d^2f  }{ dy^2 }(a)
\end{pmatrix}.
\end{equation}
Dans le cas d'une fonction $C^2$, cette matrice est symétrique.

\begin{proposition}[\cite{ooZSEQooEGRdCK}] \label{PropUQRooPgJsuz}
    Soit un ouvert \( \Omega\) de \( \eR^n\) et \( a\in \Omega\). Soit une fonction \( f\colon \Omega\to \eR\) différentiable en \( a\). Si \( a\) est un extrémum local de \( f\), alors \( a\) est un point critique de \( f\).
\end{proposition}

\begin{proof}
    Nous supposons que \( a\) est un maximum local (ce sera la même chose si \( a\) est un minimum). Soit \( r>0\) tel que \( f(x)\leq f(a)\) pour tout \( x\in B(a,r)\) (et tel que cette boule reste dans \( \Omega\)). Soit \( u\in \eR^n\) assez petit pour que \( a\pm u\in B(a,r)\) de sorte que la définition suivante ait un sens :
    \begin{equation}
        \begin{aligned}
            g\colon \mathopen[ -1 , 1 \mathclose]&\to \eR \\
            t&\mapsto f(a+tu)
        \end{aligned}
    \end{equation}
    Cette fonction est différentiable en \( t=0\) (composée de fonctions différentiables, proposition~\ref{PROPooBWZFooTxKavX}) et a un maximum local en \( t=0\). Donc \( g'(0)=0\) par la proposition~\ref{PROPooNVKXooXtKkuz}. Donc
    \begin{equation}
        0=\Dsdd{ f(a+tu) }{t}{0}=df_a(u).
    \end{equation}
\end{proof}

%---------------------------------------------------------------------------------------------------------------------------
\subsection{Extrema et Hessienne}
%---------------------------------------------------------------------------------------------------------------------------

\begin{proposition}[\cite{MonCerveau,ooOQEZooBaRMjY,ooLJMHooMSBWki}]     \label{PropoExtreRn}
    Soit un ouvert \( \Omega\) de \( \eR^n\) et une fonction \( f\colon \Omega\to \eR\) deux fois différentiable ainsi que \( a\in\Omega\).
    \begin{enumerate}
        \item   \label{ITEMooCBMYooQQMqQL}
            Si \( a\) est un point critique de \( f\) et si il existe \( r\) tel que \( (d^2f_x)\) est semi-définie positive pour tout \( x\in B(a,r)\) alors \( f\) possède un minimum local en \( a\).
        \item   \label{ITEMooCVFVooWltGqI}
            Si $a$ est un point critique\footnote{Définition~\ref{DEFooYJLZooLkEAYf}.} de $f$, et si $d^2f_a$ est strictement définie positive\footnote{La fonction \( f\) est de classe \( C^2\), donc les dérivées croisées sont égales et \( d^2f\) est symétrique. La définition~\ref{DefAWAooCMPuVM} s'applique donc.}, alors $a$ est un minimum local strict de $f$,
        \item\label{ItemPropoExtreRn}
            Si $a$ est un minimum local, alors $(d^2f)_a$ est semi-définie positive.
    \end{enumerate}
\end{proposition}
\index{extrémums}

\begin{proof}
    Nous subdivisons la preuve.

    \begin{subproof}

    \item[\ref{ITEMooCBMYooQQMqQL}]

    % position 382218354
    Soit \( h\) tel que \( a+h\in B(a,r)\). Nous allons montrer que \( f(a)\leq f(a+h)\); cela montrera que \( x=a\) est un minimum local. Pour cela nous utilisons un développement de Taylor\footnote{Proposition~\ref{PropDevTaylorPol}.} : il existe \( c\in \mathopen] a , a+h \mathclose[\) tel que
        \begin{equation}
            f(a+h)=f(a)+df_a(h)+\frac{ 1 }{2}(d^2f)_c(h,h)\geq f(a)
        \end{equation}
        parce que par hypothèse \( (d^2f)_c\) est définie positive et parce que \( df_a=0\).

    \item[\ref{ITEMooCVFVooWltGqI}]

        La forme bilinéaire \( d^2f_a\) est strictement définie positive, donc il existe \( \alpha>0\) tel que
        \begin{equation}
            d^2f_a(h,h)>\alpha\| h \|^2
        \end{equation}
        pour tout \( h\). Nous écrivons encore Taylor : il existe une fonction \( \epsilon\) telle que \( \lim_{h\to 0} \epsilon(h)=0\) et
        \begin{equation}
            f(a+h)=g(a)+df_a(h)+\frac{ 1 }{2}(d^2f)_a(h,h)+\| h \|^2\epsilon(h).
        \end{equation}
        En tenant compte du fait que \( df_a=0\),
        \begin{equation}
            f(a+h)>f(a)+\| h \|^2\big( \frac{ 1 }{2}\alpha+ \epsilon(h) \big).
        \end{equation}
        La limite de \( \epsilon\) nous dit qu'il existe \( r>0\) tel que \( \| \epsilon(h) \|<\frac{ 1 }{2}\alpha\) pour tout \( h\in B(0,r)\). Pour ces valeurs de \( h\) nous avons
        \begin{equation}
            f(a+h)>f(a).
        \end{equation}
        Donc \( a\) est un minimum local strict de \( f\).

    \item[\ref{ItemPropoExtreRn}]
    Si \( a\) est un minimum local, nous savons déjà que \( df_a=0\) par la proposition~\ref{PropUQRooPgJsuz}. Nous écrivons le développement de Taylor de \( f\) à l'ordre \( 2\) de la proposition~\ref{PROPooTOXIooMMlghF} :
    \begin{equation}
        f(a+h)=f(a)+df_a(h)+\frac{ 1 }{2}(d^2f)_a(h,h)+\| h \|^2\alpha(\| h \|).
    \end{equation}
    En prenant \( h\) assez petit pour que \( a+h\) ne sorte pas de la boule dans laquelle \( a\) est un minimum, nous avons \( f(a+h)-f(a)>0\). Donc
    \begin{equation}
        \frac{ 1 }{2}(d^2f)_a(h,h)+\| h \|^2\alpha(\| h \|)>0
    \end{equation}
    Nous divisons cela par \( \| h \|^2\) et notons \( e_h=h/\| h \|\) :
    \begin{equation}
        \frac{ 1 }{2}(d^2f)_a(e_h,e_h)+\alpha(\| h \|)>0.
    \end{equation}
    À la limite \( h\to 0\), le premier terme est constant tandis que le deuxième tend vers zéro. À la limite,
    \begin{equation}
        (d^2f)_a(e_h,e_h)\geq 0.
    \end{equation}
    La caractérisation du lemme~\ref{LemWZFSooYvksjw}\ref{ITEMooMOZYooWcrewZ} nous dit alors que \( (d^2f)_a\) est semi-définie positive.
    \end{subproof}
\end{proof}

La partie~\ref{ItemPropoExtreRn} est tout à fait comparable au fait bien connu que, pour une fonction $f\colon \eR\to \eR$, si le point $a$ est minimum local, alors $f'(a)=0$ et $f''(a)>0$.

Notons que le point~\ref{ItemPropoExtreRn} ne parle pas de minimum strict, et donc pas de matrice \emph{strictement} définie positive.

\begin{example}[Proposition~\ref{PropoExtreRn}\ref{ITEMooCVFVooWltGqI} sans point critique]
    L'hypothèse de point critique pour l'utilisation de la stricte définition positive de \( d^2f_a\) est nécessaire. Soit en effet la fonction
    \begin{equation}
        f(x)=x^2+x.
    \end{equation}
    Elle vérifie \( f''(0)=2\), de telle sorte que sa différentielle seconde en zéro soit strictement définie positive. Le point \( x=0\) n'est cependant même pas un minimum local. Entre autres parce que \( f'(0)=1\neq 0\).
\end{example}

La méthode pour chercher les extrémums de $f$ est donc de suivre le points suivants :
\begin{enumerate}
    \item
        Trouver les candidats extrémums en résolvant $\nabla f=(0,0)$,
    \item
        écrire $d^2f(a)$ pour chacun des candidats
    \item
        calculer les valeurs propres de $d^2f(a)$, déterminer si la matrice est définie positive ou négative,
    \item
        conclure.
\end{enumerate}

Une conséquence de la proposition~\ref{PropcnJyXZ}\ref{ItemluuFPN}\footnote{La matrice $d^2f(a)$ est toujours symétrique quand $f$ est de classe $C^2$.} est que si \( \det M<0\), alors le point \( a\) n'est pas  un extrémum dans le cas où $M=d^2f(a)$ par le point~\ref{ItemPropoExtreRn} de la proposition~\ref{PropoExtreRn}.

\begin{example}
    Soit la fonction \( f(x,y)=x^4+y^4-4xy\). C'est une fonction différentiable sans problèmes. D'abord sa différentielle est
    \begin{equation}
        df=\big(4x^3-4y;4y^3-4x),
    \end{equation}
    et la matrice des dérivées secondes est
    \begin{equation}
        M=d^2f(x,y)=\begin{pmatrix}
            12x^2    &   -4    \\
            -4    &   12y^2
        \end{pmatrix}.
    \end{equation}
    Nous avons \( fd=0\) pour les trois points \( (0,0)\), \( (1,1)\) et \( -1,-1\).

    Pour le point \( (0,0)\) nous avons
    \begin{equation}
        M=\begin{pmatrix}
            0    &   -4    \\
            -4    &   0
        \end{pmatrix},
    \end{equation}
    dont les valeurs propres sont \( 4\) et \( -4\). Elle n'est donc semi-définie ou définie rien du tout. Donc \( (0,0)\) n'est pas un extrémum local.

    Au contraire pour les points \( (1,1)\) et \( (-1,-1)\) nous avons
    \begin{equation}
        M=\begin{pmatrix}
            12    &   -4    \\
            -4    &   12
        \end{pmatrix},
    \end{equation}
    dont les valeurs propres sont \( 16\) et \( 8\). La matrice \( d^2f\) y est donc définie positive. Ces deux points sont donc extrémums locaux.
\end{example}

%---------------------------------------------------------------------------------------------------------------------------
\subsection{Un peu de recettes de cuisine}
%---------------------------------------------------------------------------------------------------------------------------

\begin{enumerate}
\item Rechercher les points critiques, càd les $(x,y)$ tels que
\[\begin{cases} \frac{\partial f}{\partial x}(x,y) = 0 \\ \frac{\partial f}{\partial y}(x,y) = 0 \end{cases} \]
En effet, si $(x_0,y_0)$ est un extrémum local de $f$, alors $\frac{\partial f}{\partial x}(x_0,y_0) = 0 = \frac{\partial f}{\partial y}(x_0,y_0)$.
\item Déterminer la nature des points critiques: «test» des dérivées secondes:
\[\text{On pose }H(x_0,y_0) = \frac{\partial^2 f}{\partial x^2}(x_0,y_0)\frac{\partial f^2}{\partial y^2}(x_0,y_0) - \left(\frac{\partial^2 f}{\partial x\partial y}(x_0,y_0)\right)^2\]
\begin{enumerate}
\item Si $H(x_0,y_0) > 0$ et $\frac{\partial^2 f}{\partial x^2}(x_0,y_0) > 0 \Longrightarrow (x_0,y_0)$ est un minimum local de $f$.
\item Si $H(x_0,y_0) > 0$ et $\frac{\partial^2 f}{\partial x^2}(x_0,y_0) < 0 \Longrightarrow (x_0,y_0)$ est un maximum local de $f$.
\item Si $H(x_0,y_0) < 0 \Longrightarrow f$ a un point de selle en $(x_0,y_0)$.
\item Si $H(x_0,y_0) = 0 \Longrightarrow$ on ne peut rien conclure.
\end{enumerate}
\end{enumerate}

%---------------------------------------------------------------------------------------------------------------------------
\subsection{Extrema liés}
%---------------------------------------------------------------------------------------------------------------------------

Soit $f$, une fonction sur $\eR^n$, et $M\subset \eR^n$ une variété de dimension $m$. Nous voulons savoir quelles sont les plus grandes et plus petites valeurs atteintes par $f$ sur $M$.

Pour ce faire, nous avons un théorème qui permet de trouver des extrémums \emph{locaux} de $f$ sur la variété. Pour rappel, $a\in M$ est une \defe{extrémum local de $f$ relativement}{extrémum!local!relatif} à l'ensemble $M$ s'il existe une boule $B(a,\epsilon)$ telle que $f(a)\leq f(x)$ pour tout $x\in B(a,\epsilon)\cap M$.

\begin{theorem}[Extrema lié \cite{ytMOpe}] \label{ThoRGJosS}
    Soit \( A\), un ouvert de \( \eR^n\) et
    \begin{enumerate}
        \item
            une fonction (celle à minimiser) $f\in C^1(A,\eR)$,
        \item
            des fonctions (les contraintes) $G_1,\ldots,G_r\in C^1(A,\eR)$,
        \item
            $M=\{ x\in A\tq G_i(x)=0\,\forall i\}$,
        \item
            un extrémum local $a\in M$ de $f$ relativement à $M$.
    \end{enumerate}
    Supposons que les gradients $\nabla G_1(a)$, \ldots,$\nabla G_r(a)$ soient linéairement indépendants. Alors $a=(x_1,\ldots,x_n)$ est une solution de \( \nabla L(a)=0\) où
    \begin{equation}
        L(x_1,\ldots,x_n,\lambda_1,\ldots,\lambda_r)=f(x_1,\ldots,x_n)+\sum_{i=1}^r\lambda_iG_i(x_1,\ldots,x_n).
    \end{equation}
    Autrement dit, si \( a\) est un extrémum lié, alors \( \nabla f(a)\) est une combinaisons des \( \nabla G_i(a)\), ou encore il existe des \( \lambda_i\) tels que
    \begin{equation}    \label{EqRDsSXyZ}
        df(a)=\sum_i\lambda_idG_i(a).
    \end{equation}
\end{theorem}
\index{théorème!inversion locale!utilisation}
\index{extrémum!lié}
\index{théorème des extrémums liés}
\index{application!différentiable!extrémums lié}
\index{variété}
\index{rang!différentielle}
\index{forme!linéaire!différentielle}
La fonction $L$ est le \defe{lagrangien}{lagrangien} du problème et les variables \( \lambda_i\) sont les \defe{multiplicateurs de Lagrange}{multiplicateur!de Lagrange}\index{Lagrange!multiplicateur}.

\begin{proof}
    Si \( r=n\) alors les vecteurs linéairement indépendantes \( \nabla G_i(a) \) forment une base de \( \eR^n\) et donc évidemment les \( \lambda_i\) existent. Nous supposons donc maintenant que \( r<n\). Nous notons \( (z_i)_{i=1\ldots n}\) les coordonnées sur \( \eR^n\).

    La matrice
    \begin{equation}
        \begin{pmatrix}
            \frac{ \partial G_1 }{ \partial z_1 }(a)    &   \cdots    &   \frac{ \partial G_1 }{ \partial z_n }(a)    \\
            \vdots    &   \ddots    &   \vdots    \\
            \frac{ \partial G_r }{ \partial z_1 }(a)    &   \cdots    &   \frac{ \partial G_r }{ \partial z_n }(a)
        \end{pmatrix}
    \end{equation}
    est de rang \( r\) parce que les lignes sont par hypothèses linéairement indépendantes. Nous nommons \( (y_i)_{i=1,\ldots, r}\) un choix de \( r\) parmi les \( (z_i)\) tels que
    \begin{equation}
        \det\begin{pmatrix}
            \frac{ \partial G_1 }{ \partial y_1 }    &   \ldots    &   \frac{ \partial G_1 }{ \partial y_r }    \\
            \vdots    &   \ddots    &   \vdots    \\
            \frac{ \partial G_r }{ \partial y_1 }    &   \ldots    &   \frac{ \partial G_r }{ \partial y_r }
        \end{pmatrix}\neq 0.
    \end{equation}
    Nous identifions \( \eR^n\) à \( \eR^s\times \eR^r\) dans lequel \( \eR^r\) est la partie générée par les \( (y_i)_{i=1,\ldots, r}\). Les coordonnées sur \( \eR^s\) seront nommées \( (x_j)_{j=1,\ldots, s}\), de telle sorte que les coordonnées sur \( \eR^n\) setont \( x_1,\ldots, x_s,y_1,\ldots, y_r\). Dans ces coordonnées, nous nommons \( a=(\alpha,\beta)\) avec \( \alpha\in \eR^s\) et \( \beta\in \eR^r\).

    Si nous notons \( G=(G_1,\ldots, G_r)\), le théorème de la fonction implicite (théorème~\ref{ThoAcaWho})  nous dit qu'il existe un voisinage \( U'\) de \( \alpha\in \eR^n\), un voisinage \( V'\) de \( \beta\in \eR^r\) et une fonction \( \varphi\colon U'\to V'\) de classe \( C^1\) telle que si \( (x,y)\in U'\times V'\), alors
    \begin{equation}
        g(x,y)=0
    \end{equation}
    si et seulement si \( y=\varphi(x)\). Nous posons maintenant
    \begin{subequations}
        \begin{align}
            \psi(x)&=(x,\varphi(x))\\
            h(x)&=f\big( \psi(x) \big).
        \end{align}
    \end{subequations}
    Nous avons \( \psi(\alpha)=a\) et \( \psi(x)\in M\) pour tout \( x\in U'\). La fonction \( h\) a donc un extrémum local en \( \alpha\) et donc les dérivées partielles de \( h\) y sont nulles. Cela signifie que
    \begin{equation}
        0=\frac{ \partial h }{ \partial x_i }(\alpha)=\sum_{j=1}^n\frac{ \partial f }{ \partial x_j }\frac{ \partial x_j }{ \partial x_i }+\sum_{k=1}^r\frac{ \partial f }{ \partial y_k }\frac{ \partial \varphi_k }{ \partial x_i },
    \end{equation}
    c'est-à-dire
    \begin{equation}
        \frac{ \partial f }{ \partial x_i }(\alpha)+\sum_{k=1}^r\frac{ \partial f }{ \partial y_k }(a)\frac{ \partial \varphi_k }{ \partial x_i }(\alpha)=0
    \end{equation}
    pour tout \( i=1,\ldots, s\). D'autre part pour tout $k$, la fonction \( l_k(x)=G_k\big( x,\varphi(x) \big)\) est constante et vaut zéro; ses dérivées partielles sont donc nulles :
    \begin{equation}
        \frac{ \partial l }{ \partial x_i }(\alpha)=\frac{ \partial G_k }{ \partial x_i }(\alpha)+\sum_{k=1}^r\frac{ \partial G_k }{ \partial y_k }(a)\frac{ \partial \varphi_k }{ \partial x_i }(\alpha)=0
    \end{equation}
    pour tout \( i=1,\ldots, s\) et \( k=1,\ldots, r\).

    Les \( s\) premières colonnes de la matrice
    \begin{equation}
        \begin{pmatrix}
            \frac{ \partial f }{ \partial x_1 }   &   \cdots    &   \frac{ \partial f }{ \partial x_s }    &   \frac{ \partial f }{ \partial y_1 }    &   \cdots    &   \frac{ \partial f }{ \partial y_r }\\
            \frac{ \partial G_1 }{ \partial x_1 }    &   \cdots    &   \frac{ \partial G_1 }{ \partial x_s }    &   \frac{ \partial G_1 }{ \partial y_1 }    &   \cdots    &   \frac{ \partial G_1 }{ \partial y_r }\\
            \vdots    &   \vdots    &   \vdots    &   \vdots    &   \vdots    &   \vdots\\
            \frac{ \partial G_r }{ \partial x_1 }    &   \cdots    &   \frac{ \partial G_r }{ \partial x_s }    &   \frac{ \partial G_r }{ \partial y_1 }    &  \cdots   & \frac{ \partial G_r }{ \partial y_r }
        \end{pmatrix}
    \end{equation}
    s'expriment en termes des \( r\) dernières. La matrice est donc au maximum de rang \( r\). Notons que la première ligne est \( \nabla f\) et les \( r\) suivantes sont les \( \nabla G_i\). Vu que ces lignes sont des vecteurs liés, il existe \( \mu_0,\ldots, \mu_r\) tels que
    \begin{equation}
        \mu_0\nabla f+\sum_{i=1}^r\mu_i\nabla G_i=0.
    \end{equation}
    Par hypothèse les \( \nabla G_i\) sont linéairement indépendants, ce qui nous dit que \( \mu_0\neq 0\). Donc nous avons ce qu'il nous faut :
    \begin{equation}
        \nabla f(a)=\sum_i\frac{ \mu_i }{ \mu_0 } \nabla G_i(a).
    \end{equation}

    Notons qu'au vu de l'expression \eqref{EqRDsSXyZ}, le fait que les formes \( \{ dG_i(a) \}_{1\leq i\leq r}\) forment une partie libre dans \( (\eR^n)^*\) implique que les \( \lambda_i\) sont uniques.
\end{proof}

La proposition suivante est la même que~\ref{ThoRGJosS}.
\begin{proposition} \label{PropfPPUxh}
    Soit \( U\), un ouvert de \( \eR^n\) et des fonctions de classe \( C^1\) \( f,g_1,\ldots, g_r\colon U\to \eR\). Nous considérons
    \begin{equation}
        \Gamma=\{ x\in U\tq g_1(x)=\ldots=g_r(x)=0 \}.
    \end{equation}
    Soit \( a\) un extrémum de \( f|_{\Gamma}\). Supposons que les formes \( dg_1,\ldots, dg_r\) soient linéairement indépendantes en \( a\). Alors il existe \( \lambda_1,\ldots, \lambda_r\) dans \( \eR\) tel que
    \begin{equation}
        df_a=\sum_{i=1}^r\lambda_i(dg_i)_a.
    \end{equation}
\end{proposition}

En pratique les candidats extrémums locaux sont tous les points où les gradients ne sont pas linéairement indépendants, plus tous les points donnés par l'équation $\nabla L=0$. Parmi ces candidats, il faut trouver lesquels sont maximums ou minima, locaux ou globaux.

L'existence d'extrémums locaux se prouve généralement en invoquant de la compacité, et en invoquant le lemme suivant qui permet de réduire le problème à un compact.

\begin{lemma}       \label{LemmeMinSCimpliqueS}
    Soit $S$, une partie de $\eR^n$ et $C$, un ouvert de $\eR^n$. Si $a\in\Int S$ est un minimum local relatif à $S\cap C$, alors il est un minimum local par rapport à $S$.
\end{lemma}

\begin{proof}
    Nous avons que $\forall x\in B(a,\epsilon_1)\cap S\cap C$, $f(x)\geq f(x)$. Mais étant donné que $C$ est ouvert, et que $a\in C$, il existe un $\epsilon_2$ tel que $B(a,\epsilon_2)\subset C$. En prenant $\epsilon=\min\{ \epsilon_1,\epsilon_2 \}$, nous trouvons que $f(x)\geq f(a)$ pour tout $x\in B(a,\epsilon)\cap(S\cap C)=B(a,\epsilon)\cap S$.
\end{proof}

%+++++++++++++++++++++++++++++++++++++++++++++++++++++++++++++++++++++++++++++++++++++++++++++++++++++++++++++++++++++++++++
\section{Fonctions convexes}
%+++++++++++++++++++++++++++++++++++++++++++++++++++++++++++++++++++++++++++++++++++++++++++++++++++++++++++++++++++++++++++
\label{SECooVZWWooUjxXYi}

\begin{definition}[\cite{BIBooRZGXooTAYeTG}]  \label{DefVQXRJQz}
    Une fonction $f$ d’un intervalle $I$ de \( \eR\) vers \( \eR\) est dite \defe{convexe}{fonction!convexe}\index{convexité!fonction} lorsque, pour tous \( x_1\) et \( x_2\) de $I$ et tout $\lambda$ dans $[0, 1]$ nous avons
    \begin{equation}        \label{EQooYNAPooFePQZy}
        f\big(\lambda\, x_1+(1-\lambda)\, x_2\big) \leq \lambda\, f(x_1)+(1-\lambda)\, f(x_2)
    \end{equation}

    Si pour tout \( \theta\in \mathopen] 0 , 1 \mathclose[\) et pour tout \( x\neq y\) dans \( I\) nous avons
    \begin{equation}     
        f\big(\lambda\, x_1+(1-\lambda)\, x_2\big) < \lambda\, f(x_1)+(1-\lambda)\, f(x_2)
    \end{equation}
    alors nous disons que la fonction \( f\) est \defe{strictement convexe}{strictement convexe} sur \( I\).

    Une fonction est \defe{concave}{concave} si son opposée est convexe.
\end{definition}

\begin{normaltext}[\cite{BIBooRZGXooTAYeTG}]
    Les différents résultats pour les fonctions convexes s'adaptent généralement sans mal aux fonctions strictement convexes. Une nuance cependant : de même que les fonctions dérivables convexes sont celles qui ont une dérivée croissante, les fonctions dérivables strictement convexes sont celles qui ont une dérivée strictement croissante (proposition~\ref{PropYKwTDPX}). En revanche, il ne faudrait pas croire que la dérivée seconde d'une fonction dérivable strictement convexe est nécessairement une fonction à valeurs strictement positives (voir théorème~\ref{ThoGXjKeYb}) : la dérivée d'une fonction strictement croissante peut s'annuler occasionnellement, ou plus exactement peut s'annuler sur un ensemble de points d'intérieur vide. Penser à \( x\mapsto x^4\) pour un exemple de fonction strictement convexe dont la dérivée seconde s'annule.
\end{normaltext}

%---------------------------------------------------------------------------------------------------------------------------
\subsection{Inégalité des pentes}
%---------------------------------------------------------------------------------------------------------------------------

Dans l'étude des fonctions convexes nous allons souvent utiliser la fonction \defe{taux d'accroissement}{taux d'accroissement} qui est, pour \( \alpha\) dans le domaine de convexité de \( f\) définie par
\begin{equation}    \label{EqRYBazWd}
    \begin{aligned}
        \tau_{\alpha}\colon I\setminus\{ \alpha \}&\to \eR \\
        x&\mapsto \frac{ f(x)-f(\alpha) }{ x-\alpha }.
    \end{aligned}
\end{equation}

\begin{proposition}[Inégalité des pentes\cite{OJIMBtv}] \label{PropMDMGjGO}
    Soit \( f\) une fonction convexe sur un intervalle \( I\subset \eR\). Alors pour tout \( a<b<c\) dans \( I\) nous avons\footnote{Les inégalités sont strictes si la fonction \( f\) est strictement convexe.}
    \begin{equation}
        \frac{ f(b)-f(a)  }{ b-a }\leq\frac{ f(c)-f(a) }{ c-a }\leq \frac{ f(c)-f(b) }{ c-b }.
    \end{equation}
    En d'autres termes,
    \begin{equation}
        \tau_a(b)\leq\tau_a(c)\leq \tau_b(c),
    \end{equation}
    c'est-à-dire que \( \tau\) est croissante en ses deux arguments.
\end{proposition}
\index{inégalité!des pentes}

\begin{proof}
    D'abord les inégalités \( a<b<c\) impliquent \( 0<b-a<c-a\) et donc
    \begin{equation}
        \lambda=\frac{ b-a }{ c-a }<1.
    \end{equation}
    L'astuce est de remarquer que \( (1-\lambda)a+\lambda c=b\). Donc \( \lambda\) a toutes les bonnes propriétés pour être utilisé dans la définition de la convexité :
    \begin{equation}
        f\big( (1-\lambda)a+\lambda c \big)\leq \lambda f(c)+(1-\lambda)f(a),
    \end{equation}
    c'est-à-dire
    \begin{equation}
        f(b)-f(a)\leq \lambda\big( f(c)-f(a) \big)
    \end{equation}
    ou encore, en remplaçant \( \lambda\) par sa valeur :
    \begin{equation}
        \frac{ f(b)-f(a) }{ b-a }\leq \frac{ f(c)-f(a) }{ c-a }.
    \end{equation}
    Cela fait déjà une des inégalités à savoir.

    D'autre part en partant de \( -a<-b<-c\) nous posons
    \begin{equation}
        0<\lambda=\frac{ c-b }{ c-a }.
    \end{equation}
    Nous avons à nouveau \( b=(1-\lambda)c+\lambda a\) et nous pouvons obtenir la seconde inégalité
    \begin{equation}
        \frac{ f(c)-f(a) }{ c-a }\leq \frac{ f(c)-f(b) }{ c-b }.
    \end{equation}
\end{proof}

Géométriquement, l'inégalité des pentes se comprend facilement : le coefficient angulaire de la corde du graphe augmente. Donc si \( x<y<z\), le coefficient moyen entre \( x\) et \( y\) est plus petit que celui entre \( x\) et \( z\) qui est plus petit que celui entre \( y\) et \( z\).

Donc si le coefficient angulaire moyen entre \( a\) et \( b+u\) vaut celui entre \( a\) et \( b\), ce coefficient ne peut qu'être constant entra \( a\) et \( b\) : sinon il serait plus grand entre \( b\) et \( b+u\) et la moyenne sur \( a\to b+u\) serait plus grande que sa moyenne sur \( a\to b\). Mais avoir un coefficient angulaire constant signifie être une droite.

En résumé, si une fonction est convexe et non strictement convexe, alors son graphe est une droite. C'est en gros cela que la proposition~\ref{PROPooOCOEooEGybmS} clarifiera.

%---------------------------------------------------------------------------------------------------------------------------
\subsection{Convexité et régularité}
%---------------------------------------------------------------------------------------------------------------------------

\begin{lemma}[\cite{BIBooRZGXooTAYeTG}]   \label{LemKLTsHIQ}
    Une fonction convexe sur un ouvert
    \begin{enumerate}
        \item
            y admet des dérivées à gauche et à droite en chaque point,
        \item
            y est continue.
    \end{enumerate}
\end{lemma}

\begin{proof}
    Soit \( I=\mathopen] a , b \mathclose[\) un intervalle sur lequel \( f\) est convexe et \( \alpha\in I\). Nous allons prouver que \( f\) est continue en \( \alpha\). Nous considérons \( \tau_{\alpha}\) le taux d'accroissement définit par \eqref{EqRYBazWd}; c'est une fonction croissante comme précisé dans l'inégalité des trois pentes~\ref{PropMDMGjGO} et de plus \( \tau_{\alpha}(x)\) est bornée supérieurement par \( \tau_{\alpha}(b)\) pour \( x<\alpha\) et inférieurement par \( \tau_{\alpha}(a)\) pour \( x>\alpha\). Les limites existent donc et sont finies par la proposition~\ref{PropMTmBYeU}. Autrement dit les limites
        \begin{subequations}
            \begin{align}
                \lim_{x\to \alpha+} \frac{ f(x)-f(\alpha) }{ x-\alpha }&=\lim_{x\to \alpha^+} \tau_{\alpha}(x)=\inf_{t>\alpha}\tau_{\alpha}(t)\\
                \lim_{x\to \alpha^-} \frac{ f(x)-f(\alpha) }{ x-\alpha }&=\lim_{x\to \alpha^-} \tau_{\alpha}(x)=\sup_{t<\alpha}\tau_{\alpha}(t).
            \end{align}
        \end{subequations}
        existent et sont finies, c'est-à-dire que la fonction \( f\) admet une dérivée à gauche et à droite.

        Pour tout \( x\) nous avons les inégalités
        \begin{equation}
            \tau_{\alpha}(a)\leq \frac{ f(x)-f(\alpha) }{ x-\alpha }\leq \tau_{\alpha}(b).
        \end{equation}
        En posant \( k=\max\{ \tau_{\alpha}(a),\tau_{\alpha}(b) \}\) nous avons
        \begin{equation}
            \big| f(x)-f(\alpha) \big|\leq k| x-\alpha |.
        \end{equation}
        La fonction est donc Lipschitzienne et par conséquent continue par la proposition~\ref{PropFZgFTEW}.
\end{proof}

\begin{remark}
    Les dérivées à gauche et à droite ne sont à priori pas égales. Penser par exemple à une fonction affine par morceaux dont les pentes augmentent à chaque morceau.
\end{remark}

%---------------------------------------------------------------------------------------------------------------------------
\subsection{Dérivées d'une fonction convexe}
%---------------------------------------------------------------------------------------------------------------------------

\begin{proposition}[\cite{RIKpeIH,ooGCESooQzZtVC,MonCerveau}] \label{PropYKwTDPX}
    Une fonction dérivable sur un intervalle \( I\) de \( \eR\)
    \begin{enumerate}
        \item       \label{ITEMooUTSAooJvhZNm}
            est convexe si et seulement si sa dérivée est croissante sur \( I\).
        \item       \label{ITEMooLLSIooFwkxtV}
            est strictement convexe si et seulement si sa dérivée est strictement croissante sur \( I\)
    \end{enumerate}
\end{proposition}

\begin{proof}
    Pour la preuve de~\ref{ITEMooUTSAooJvhZNm} et~\ref{ITEMooLLSIooFwkxtV}, nous allons démontrer les énoncés «non stricts»  et indiquer ce qu'il faut changer pour obtenir les énoncés «stricts».
    \begin{subproof}
    \item[Sens direct]
    Nous supposons que \( f\) est convexe. Soient \( a<b\) dans \( I\) et \( x\in\mathopen] a , b \mathclose[\). D'après l'inégalité des pentes~\ref{PropMDMGjGO},
        \begin{equation}        \label{EqATDLooIcqdDI}
            \frac{ f(x)-f(a) }{ x-a }\leq\frac{ f(b)-f(a) }{ b-a }\leq \frac{ f(b)-f(x) }{ b-x }.
        \end{equation}
        En faisant la limite \( x\to a\) nous avons
        \begin{equation}
            f'(a)\leq \frac{ f(b)-f(a) }{ b-a }
        \end{equation}
        et la limite \( x\to b\) donne
        \begin{equation}
            \frac{ f(b)-f(a) }{ b-a }\leq f'(b).
        \end{equation}
        Ici les inégalités sont non à priori strictes, même si \( f\) est strictement convexe : même avec des inégalités strictes dans \eqref{EqATDLooIcqdDI}, le passage à la limite rend l'inégalité non stricte. Quoi qu'il en soit nous avons
        \begin{equation}        \label{EqQGVMooBpuvNr}
            f'(a)\leq f'(b).
        \end{equation}
    \item[Sens direct : strict]
         Nous savons déjà que \( f'\) est croissante. Si \eqref{EqQGVMooBpuvNr} était une égalité, alors \( f'\) serait constante sur \( \mathopen] a , b \mathclose[\) parce qu'en prenant \( c\) entre \( a\) et \( b\) nous aurions \( f'(a)\leq f'(c)\leq f'(b)\) avec \( f'(a)=f'(b)\). Donc \( f'(a)=f'(c)\). Avoir \( f'\) constante sur un intervalle est contraire à la stricte convexité.

         \item[Sens réciproque]

             Nous supposons que \( f'\) est croissante et nous considérons \( a<b\) dans \( I\) ainsi que \( \lambda\in \mathopen[ 0 , 1 \mathclose]\). Nous posons \( x=\lambda a+(1-\lambda)b\), et nous savons que \( a\leq x\leq b\). Le théorème des accroissements finis~\ref{ThoAccFinis} donne \( c_1\in\mathopen] a , x \mathclose[\) et \( c_2\in \mathopen] x , b \mathclose[\) tels que
                 \begin{equation}
                     f'(c_1)=\frac{ f(x)-f(a) }{ x-a }
                 \end{equation}
                 et
                 \begin{equation}
                     f'(c_2)=\frac{ f(b)-f(x) }{ b-x }.
                 \end{equation}
                 Et en plus \( c_1<c_2\). Vu que \( f'\) est croissante nous avons \( f'(c_1)\leq f'(c_2)\) et donc
                 \begin{equation}       \label{EqSAOCooWAwClQ}
                     \frac{ f(x)-f(a) }{ x-a }\leq\frac{ f(b)-f(x) }{ b-x }.
                 \end{equation}
                 En remplaçant \( x\) par sa valeur en termes de \( \lambda\), \( a\) et \( b\) nous avons \( x-a=(1-\lambda)(b-a)\) et \( b-x=\lambda(b-a)\), et l'inégalité \eqref{EqSAOCooWAwClQ} nous donne
                 \begin{equation}
                     f(x)\leq \lambda f(a)+(1-\lambda)f(b).
                 \end{equation}
             \item[Sens réciproque : strict]
                 Si \( f'\) est strictement croissante, nous avons \( f'(c_2)<f'(c_2)\) et les inégalité suivantes sont strictes, ce qui donne
                 \begin{equation}
                     f(x)< \lambda f(a)+(1-\lambda)f(b).
                 \end{equation}
    \end{subproof}
\end{proof}

\begin{theorem}[\cite{RIKpeIH}] \label{ThoGXjKeYb}
    Soit une fonction \( f\) de classe \( C^2\).
    \begin{enumerate}
        \item       \label{ITEMooIUTQooTkRMoyBP}
            Est convexe si et seulement si \( f''\) est positive.
        \item       \label{ITEMooXUOMooYIoOtv}
            Si \( f''\) est strictement positive, elle est strictement convexe.
    \end{enumerate}
\end{theorem}

\begin{proof}
    En deux parties.
    \begin{subproof}
    \item[Pour \ref{ITEMooIUTQooTkRMoyBP}]
            La fonction est \( C^2\), donc \( f''\) est positive si et seulement si \( f'\) est croissante (proposition~\ref{PropGFkZMwD}) alors que la proposition~\ref{PropYKwTDPX} nous jure que \( f\) sera convexe si et seulement si \( f'\) est croissante.
        \item[Pour \ref{ITEMooXUOMooYIoOtv}]
            Si \( f''\) est strictement positive, \( f'\) sera strictement croissante et donc \( f\) strictement convexe (proposition \ref{PropYKwTDPX}).
    \end{subproof}
\end{proof}

\begin{remark}      \label{REMooVRPQooIybxmp}
    Une fonction peut être strictement convexe sans que sa dérivée seconde ne soit toujours strictement positive. En exemple : \( x\mapsto x^4\) est strictement convexe alors que sa dérivée seconde s'annule en zéro.
\end{remark}

\begin{example} \label{ExPDRooZCtkOz}
    Quelques exemples utilisant le théorème~\ref{ThoGXjKeYb}
    \begin{enumerate}
        \item
    La fonction \( x\mapsto x^2\) est convexe parce que sa dérivée seconde est la constante (positive) \( 2\).
\item La fonction \( x\mapsto\frac{1}{ x }\) est convexe sur \( \eR^+\setminus\{ 0 \}\) (sa dérivée seconde est \( 2x^{-3}\)).
\item       \label{ITEMooRXSBooDBerbx}
    La fonction exponentielle est strictement convexe par le théorème \ref{ThoGXjKeYb}.
\item
    La fonction \( \ln\) est concave parce que la dérivée seconde de \( -\ln\) est \( \frac{1}{ x^2 }\) qui est strictement positif.
    \end{enumerate}
\end{example}

Nous en faisons une en détail; elle sera utile en analyse fonctionnelle, lors de l'étude des espaces \( L^p\). Voir par exemple le théorème de la projection \ref{THOooRJFUooQivDKm}.
\begin{lemma}       \label{LEMooSXTXooZOmtKq}
    Soient \( p>1\) et la fonction
    \begin{equation}
        \begin{aligned}
            f\colon \mathopen] 0 , \infty \mathclose[&\to \eR \\
            x&\mapsto x^p 
        \end{aligned}
    \end{equation}
    est strictement convexe.
\end{lemma}

\begin{proof}
    La proposition \ref{PROPooKIASooGngEDh} nous permet de dire que la fonction \( f\) est de classe \(  C^{\infty}\) et que la dérivée seconde est donnée par
    \begin{equation}
        f''(x)=p(p-1)x^{p-2}.
    \end{equation}
    Cela est strictement positif pour tous les \( x\) considérée, le théorème \ref{ThoGXjKeYb} conclu.
\end{proof}

%---------------------------------------------------------------------------------------------------------------------------
\subsection{Graphe d'une fonction convexe}
%---------------------------------------------------------------------------------------------------------------------------

L'idée principale du graphe d'une fonction convexe est qu'il est toujours au dessus du graphe de ses tangentes (lorsqu'elles existent). Lorsqu'elles n'existent pas, le lemme~\ref{LemKLTsHIQ} donne des coefficients directeurs de droites qui vont rester en dessous du graphe de la fonction.

\begin{proposition}[\cite{ooKCFNooVrqYhc}]      \label{PROPooOCOEooEGybmS}
    Une fonction convexe est strictement convexe si et seulement s'il n'existe aucun intervalle de longueur non nulle sur lequel elle coïncide avec une fonction affine.
\end{proposition}

\begin{proof}
    Si sur l'intervalle (non réduit à un point) \( \mathopen[ x , y \mathclose]\), la fonction convexe \( f\) coïncide avec une fonction affine, alors \( f(t)=at+b\) et pour \( \lambda\in\mathopen] 0 , 1 \mathclose[\) nous avons
        \begin{equation}
                f\big( \lambda x+(1-\lambda)y \big)=a\lambda x+a(1-\lambda)y+b=\lambda f(x)+(1-\lambda)f(y)
        \end{equation}
        où nous avons remplacé \( b\) par \( \lambda b+(1-\lambda)b\). Par conséquent la fonction n'est pas strictement convexe.

    Nous supposons maintenant que la fonction convexe \( f\) n'est pas strictement convexe sur l'intervalle \( I\). Il existe \( x\neq y\in I\) et \( \lambda\in \mathopen] 0 , 1 \mathclose[\) tels que
        \begin{equation}
            f\big( \lambda x+(1-\lambda)y \big)=\lambda f(x)+(1-\lambda)f(y).
        \end{equation}
    Nous posons \( z=\lambda x+(1-\lambda)y\) et \( u\in\mathopen] x , z \mathclose[\) pour écrire des inégalités des pentes entre \( x<u<z<y\). Plus précisément si nous notons \( a\to b\) la pente de \( a\) à \( b\), c'est-à-dire \( a\to b=\frac{ f(b)-f(a) }{ b-a }\), alors les inégalités des pentes pour \( x<u<z\) puis \( u<z<y\) donnent
        \begin{equation}        \label{EqooBMEFooMpoEzd}
            x\to z\leq u\to z\leq z\to y.
        \end{equation}
        Voyons maintenant qu'en réalité \( z\to y=x\to z\). En effet en replaçant
        \begin{equation}
            f(y)=\frac{ f(z)-\lambda f(x) }{ 1-\lambda }
        \end{equation}
        et
        \begin{equation}
            y=\frac{ \lambda x }{ 1-\lambda }
        \end{equation}
        dans l'expression \( z\to y=\frac{ f(y)-f(z) }{ y-z }\) nous obtenons
        \begin{equation}
            z\to y=\frac{ f(y)-f(z) }{ y-z }=\frac{ f(z)-f(x) }{ z-x }=x\to z.
        \end{equation}
        Les inégalités \eqref{EqooBMEFooMpoEzd} sont donc des égalités :
        \begin{equation}
            \frac{ f(z)-f(x) }{ z-x }=\frac{ f(z)-f(u) }{ z-u }=\frac{ f(y)-f(z) }{ y-z }.
        \end{equation}
        Nous avons donc montré que le nombre \( a=\frac{ f(z)-f(u) }{ z-u }\) ne dépend pas de \( u\). Nous avons alors
        \begin{equation}
            f(z)-f(u)=a(z-u)
        \end{equation}
        ou encore :
        \begin{equation}
            f(u)=f(z)-a(z-u),
        \end{equation}
    ce qui signifie que sur \( \mathopen] x , z \mathclose[\), la fonction \( f\) est affine.
\end{proof}

\begin{proposition} \label{PROPooQPOSooDZlUAJ}
    Une fonction dérivable sur un intervalle \( I\) de \( \eR\) est convexe si et seulement si son graphe est au dessus de chacune de ses tangentes.
\end{proposition}

\begin{proof}
    En deux parties.
    \begin{subproof}
        \item[Sens direct]
            Soient \( x,y\in I\). Nous voulons :
            \begin{equation}
                f(y)\geq f(x)+f'(x)(y-x).
            \end{equation}
            Étant donné que nous aurons besoin, dans le quotient différentiel de quelque chose comme \( f(x+t)-f(x)\) nous écrivons la définition \eqref{EQooYNAPooFePQZy} de la convexité en inversant les rôles de \( x\) et \( y\) et en manipulant un peu :
            \begin{subequations}
                \begin{align}
                    f\big( ty+(1-t)x \big)\leq tf(y)+(1-t)f(x)\\
                    f\big( x+t(y-x) \big)\leq tf(y)+(1-t)f(x)\\
                    f\big(  x+t(y-x)  \big)=f(x)\leq tf(y)-tf(x)
                \end{align}
            \end{subequations}
            Nous divisons par \( t\) :
            \begin{equation}
                \frac{ f\big( x+t(y-x) \big)-f(x) }{ t }\leq f(y)-f(x).
            \end{equation}
            Le passage à la limite \( t\to 0\) donne
            \begin{equation}
                (y-x)f'(x)\leq f(y)-f(x),
            \end{equation}
            ce qu'il fallait.
        \item[Sens inverse]
            Pour tout \( x,y\in I\) nous supposons avoir
            \begin{equation}        \label{EQooEXXIooHXJnER}
                f(y)\geq f(x)+f'(x)(y-x).
            \end{equation}
            Si nous supposons \( x\neq y\) et si nous posons \( z=\lambda x+(1-\lambda)y\) nous voulons prouver que
            \begin{equation}
                f(z)\leq \lambda f(x)+(1-\lambda)f(y).
            \end{equation}
            Pour cela nous écrivons l'inégalité \eqref{EQooEXXIooHXJnER} avec les couples \( (x,z)\) et \( (y,z)\) :
            \begin{subequations}
                \begin{align}
                    f(x)\geq f(z)+f'(z)'(x-z)\\
                    f(y)\geq f(z)+f'(z)'(y-z)
                \end{align}
            \end{subequations}
            En multipliant la première par \( \lambda\) et la seconde par \( (1-\lambda)\) et en sommant,
            \begin{subequations}
                \begin{align}
                    \lambda f(x)+(1-\lambda)f(y)&\geq \lambda f(z)+\lambda f'(z)(x-z)+(1-\lambda)f(z)+(1-\lambda)f'(z)(y-z)\\
                    &=f(z)+f'(z)\big( \lambda(x-z)+(1-\lambda)(y-z) \big)\\
                    &=f(z).
                \end{align}
            \end{subequations}
    \end{subproof}
\end{proof}

\begin{proposition}[\cite{MonCerveau}] \label{PropNIBooSbXIKO}
    Soit \( f\colon \eR\to \eR \) une fonction convexe et \( a\in \eR\). Il existe une constante \( c_a\in \eR\) telle que pour tout \( x\) nous ayons
    \begin{equation}    \label{EqSKIooSeAekM}
        f(x)-f(a)\geq c_a(x-a).
    \end{equation}
    Autrement dit, le graphe de la fonction \( f\) est toujours au dessus de la droite d'équation
    \begin{equation}
        y=f(a)+c_a(x-a).
    \end{equation}
\end{proposition}

\begin{proof}
    Les dérivées à gauche et à droite de \( f\) données par le lemme~\ref{LemKLTsHIQ} sont les candidats tout cuits pour être coefficient directeur de la droite que l'on cherche. Nous allons prouver qu'en posant
    \begin{equation}
        c_a=\inf_{t>a}\tau_a(t),
    \end{equation}
    la droite \( y=f(a)+c_a(x-a)\) répond à la question\footnote{En prenant l'autre, $c_a'=\sup_{t<a}\tau_a(t)$, ça fonctionne aussi. En pensant à une fonction affine par morceaux, on remarque qu'en choisissant un nombre entre les deux, nous avons plus facilement une inégalité stricte dans \eqref{EqSKIooSeAekM}.}.

    Nous devons prouver que le nombre \( \Delta_x=f(x)-\big( f(a)+c_a(x-a) \big)\) est positif pour tout \( x\).
    \begin{subproof}

    \item[Si \( x>a\)]

        Nous divisons par \( x-a\) et nous devons prouver que \( \frac{ \Delta_x }{ x-a }\) est positif :
        \begin{subequations}
            \begin{align}
                \frac{ \Delta_x }{ x-a }&=\frac{ f(x)-f(a) }{ x-a }-c_a\\
                &=\tau_a(x)-\inf_{t>a}\tau_a(t)\\
                &\geq 0
            \end{align}
        \end{subequations}
        parce que \( t\to\tau_a(t)\) est croissante et que \( x>a\).

    \item[Si \( x<a\)]

        Nous divisons par \( x-a\) et nous devons prouver que \( \frac{ \Delta_x }{ x-a }\) est négatif :
        \begin{subequations}
            \begin{align}
                \frac{ \Delta_x }{ x-a }&=\frac{ f(x)-f(a) }{ x-a }-c_a\\
                &=\tau_a(x)-\inf_{t>a}\tau_a(t)\\
                &\leq 0
            \end{align}
        \end{subequations}
        parce que \( t\to\tau_a(t)\) est croissante et que \( x<a\).
    \end{subproof}
\end{proof}

\begin{proposition}[\cite{MonCerveau}] \label{PropPEJCgCH}
    Si \( g\) est une fonction convexe, il existe deux suites réelles \( (a_n)\) et \( (b_n)\) telles que
    \begin{equation}
        g(x)=\sup_{n\in \eN}(a_nx+b_n).
    \end{equation}
\end{proposition}
\index{fonction!convexe}
\index{densité!de \( \eQ\) dans \( \eR\)!utilisation}

\begin{proof}
    Pour \( u\in \eR\) nous considérons \( a(u)\) et \( b(u)\) tels que la droite \( y(x)=a(u)x+b(u)\) vérifie \( y(u)=g(u)\) et \( y(x)\leq g(x)\) pour tout \( x\). Cela est possible par la proposition~\ref{PropNIBooSbXIKO}. Il s'agit d'une droite coupant le graphe de \( g\) en \( x=u\) et restant en dessous. Nous considérons alors \( (u_n)\) une suite quelconque dense dans \( \eR\) (disons les rationnels pour fixer les idées) et nous posons
    \begin{subequations}
        \begin{numcases}{}
            a_n=a(u_n)\\
            b_n=b(u_n).
        \end{numcases}
    \end{subequations}
    Si \( q\in \eQ\) alors \( a_nx+b_n\leq g(x)\) pour tout \( n\) et \( g(q)\) est le supremum qui est atteint pour le \( n\) tel que \( u_n=q\). Si maintenant \( x\) n'est pas dans \( \eQ\) il faut travailler plus.

    Nous prenons \( (\tilde q_n)\), une sous-suite de \( (q_n)\) convergeant vers \( x\) et \( N\) suffisamment grand pour que pour tout \( n\geq N\) on ait \( | \tilde q_n-x |\leq \epsilon\) et \( | g(\tilde q_n)-g(x) |\leq \epsilon\); cela est possible grâce à la continuité de \( g\) (lemme~\ref{LemKLTsHIQ}). Ensuite les sous-suites \( (\tilde a_n)\) et \( (\tilde b_n)\) sont celles qui correspondent :
    \begin{equation}
        \tilde a_n\tilde q_n+\tilde b_n=g(\tilde q_n).
    \end{equation}
    Nous considérons la majoration
    \begin{subequations}
        \begin{align}
            | \tilde a_nx+\tilde b_n-g(x) |&\leq| \tilde a_nx+\tilde b_n-(\tilde a_n\tilde q_n+\tilde b_n) |+\underbrace{| \tilde a_n\tilde q_n+\tilde b_n-g(\tilde q_n) |}_{=0}+\underbrace{| g(\tilde q_n)-g(x) |}_{\leq \epsilon}\\
            &\leq | \tilde a_n | |x-\tilde q_n |+\epsilon\\
            &=\epsilon\big( | \tilde a_n |+1 \big).
        \end{align}
    \end{subequations}
    Il nous reste à montrer que \( | \tilde a_n |\) est borné par un nombre ne dépendant pas de \( n\) (pour les \( n>N\)).

    Vu que la droite de coefficient directeur \( \tilde a_n\) et passant par le point \( \big( \tilde q_n,g(\tilde q_n) \big)\) reste en dessous du graphe de \( g\), nous avons pour tout \( n\) et tout \( y\in \eR\) l'inégalité
    \begin{equation}
        g(y)\geq \tilde a_n(y-\tilde q_n)+g(\tilde q_n)\in \tilde a_nB(y-x,\epsilon)+B\big( g(x),\epsilon \big).
    \end{equation}
    Si \( \tilde a_n\) n'est pas borné vers le haut, nous prenons \( y\) tel que \( B(y-x,\epsilon)\) soit minoré par un nombre \( k\) strictement positif et nous obtenons
    \begin{equation}
        g(y)\geq k\tilde a_n+l
    \end{equation}
    avec \( k\) et \( l\) indépendants de \( n\). Cela donne \( g(y)=\infty\). Si au contraire \( \tilde a_n\) n'est pas borné vers le bas, nous prenons $y$ tel que \( B(y-x,\epsilon)\) est majoré par un nombre \( k\) strictement négatif. Nous obtenons encore \( g(y)=\infty\).

    Nous concluons que \( | \tilde a_n |\) est bornée.
\end{proof}

\begin{lemma}[\cite{KXjFWKA}]   \label{LemXOUooQsigHs}
    L'application
    \begin{equation}
        \begin{aligned}
            \phi\colon S^{++}(n,\eR)&\to \eR \\
            A&\mapsto \det(A)
        \end{aligned}
    \end{equation}
    est \defe{log-convave}{concave!log-concave}\index{log-concave}, c'est-à-dire que l'application \( \ln\circ\phi\) est concave\footnote{La définition~\ref{DEFooELGOooGiZQjt} du logarithme ne fonctionne que pour les réels strictement positifs. C'est le cas du déterminant d'une matrice réelle symétrique strictement définie positive.}. De façon équivalente, si \( A,B\in S^{++}\) et si \( \alpha+\beta=1\), alors
    \begin{equation}    \label{EqSPKooHFZvmB}
        \det(\alpha A+\beta B)\geq \det(A)^{\alpha}\det(B)^{\beta}.
    \end{equation}
\end{lemma}
Ici \( S^{++}\) est l'ensemble des matrices symétriques strictement définies positives, définition~\ref{DefAWAooCMPuVM}.

\begin{proof}
    En plusieurs étapes.
    \begin{subproof}
        \item[Pseudo-réduction]
            Le théorème de pseudo-réduction simultanée, corolaire~\ref{CorNHKnLVA}, appliqué aux matrices \( A\) et \( B\) nous donne une matrice inversible \( Q\) telle que
            \begin{subequations}
                \begin{numcases}{}
                    B=Q^tDQ\\
                    A=Q^tQ
                \end{numcases}
            \end{subequations}
            avec
            \begin{equation}
                D=\begin{pmatrix}
                    \lambda_1    &       &       \\
                        &   \ddots    &       \\
                        &       &   \lambda_n
                \end{pmatrix},
            \end{equation}
            \( \lambda_i>0\). Nous avons alors
            \begin{equation}
                \det(A)^{\alpha}\det(B)^{\beta}=\det(Q)^{2\alpha}\det(Q)^{2\beta}\det(D)^{\beta}=\det(Q)^2\det(D)^{\beta}
            \end{equation}
            (parce que \( \alpha+\beta=1\)) et
            \begin{subequations}
                \begin{align}
                    \det(\alpha A+\beta B)&=\det(\alpha Q^tQ+\beta Q^tDQ)\\
                    &=\det\big( Q^t(\alpha\mtu+\beta D)Q \big)\\
                    &=\det(Q)^2\det(\alpha\mtu+\beta D).
                \end{align}
            \end{subequations}
        \item[Ré-expression]
            L'inégalité \eqref{EqSPKooHFZvmB} qu'il nous faut prouver se réduit donc  à
            \begin{equation}
                \det(\alpha \mtu+\beta D)\geq \det(D)^{\beta}.
            \end{equation}
            Vue la forme de \( D\) nous avons
            \begin{equation}
                \det(\alpha\mtu+\beta D)=\prod_{i=1}^n(\alpha+\beta\lambda_i)
            \end{equation}
            et
            \begin{equation}
                \det(D)^{\beta}=\big( \prod_{i=1}^{n}\lambda_i \big)^{\beta}.
            \end{equation}
            Il faut donc prouver que
            \begin{equation}\label{EqGFLooOElciS}
                \prod_{i=1}^n(\alpha+\beta\lambda_i)\geq \big( \prod_{i=1}^n\lambda_i \big)^{\beta}.
            \end{equation}
            Cette dernière égalité de produit sera prouvée en passant au logarithme. 
        \item[Logarithme]
            Vu que le logarithme est concave par l'exemple~\ref{ExPDRooZCtkOz}, nous avons pour chaque \( i\) que
            \begin{equation}
                \ln(\alpha+\beta\lambda_i)\geq \alpha\ln(1)+\beta\ln(\lambda_i)=\beta\ln(\lambda_i).
            \end{equation}
            En sommant cela sur \( i\) et en utilisant les propriétés de croissance et de multiplicativité du logarithme nous obtenons successivement
            \begin{subequations}
                \begin{align}
                    \sum_{i=1}^n\ln(\alpha+\beta\lambda_i)\geq \beta\sum_i\ln(\lambda_i)\\
                    \ln\big( \prod_i(\alpha+\beta\lambda_i) \big)\geq\ln\Big( \big( \prod_i\lambda_i \big)^{\beta} \Big)\\
                    \prod_i(\alpha+\beta\lambda_i)\geq\big( \prod_i\lambda_i \big)^{\beta},
                \end{align}
            \end{subequations}
            ce qui est bien \eqref{EqGFLooOElciS}.
    \end{subproof}
\end{proof}

Rappel de notations : \( \eR^+=\mathopen[ 0 , \infty \mathclose[\). Voir la remarque~\ref{REMooOCXLooKQrDoq}.
\begin{lemma}[\cite{MonCerveau}]        \label{LEMooNUDOooVfVPkw}
    Soit une fonction strictement convexe \( g\colon \eR^+\to \eR^+\). Soit une fonction \( f\colon \eR^+\times \eR^+\to \eR^+\) vérifiant
    \begin{enumerate}
        \item
            \( f(0,0)=0\),
        \item
            \( f(tx,ty)=tf(x,y)\) pour tout \( t\) (tant que ça ne déborde pas du domaine)
        \item
            \( f(1,y)=g(y)\) pour tout \( y\)
        \item
            \( f(0,y)=f(y,0)\).
    \end{enumerate}
    Alors \( f\) est convexe.
\end{lemma}

\begin{proof}
    Nous devons prouver que pour toute paire de points \( A,B\) sur le graphe de \( f\), le segment \( \mathopen[ A , B \mathclose]\) est au-dessus du graphe de \( f\). Ledit graphe étant d'ailleurs constitué de droites joignant \( (0,0,0)\) et les points du graphe de \( g\) (situé en \( x=1\)).

    Nous notons \( \mC\) le graphe de \( f\).

    \begin{subproof}
        \item[Une corde alignée à \( O\)]
            Soient deux point \( A\) et \( B\) alignés à l'origine \( O\). Un point quelconque de \( \mathopen[ A , B \mathclose]\) (et même de toute la droite) s'écrit
            \begin{equation}
                \big( tA_x,tA_y,tf(A_x,A_y) \big)=\big( tA_x,tA_y,f(tA_x,tA_y) \big),
            \end{equation}
            et donc est sur le graphe de \( f\).

        \item[Autre corde]
            Nous prouvons que \( \mathopen[ A , B \mathclose]\cap \mC=\{ A,B \}\).

            Si \( A\) et \( B\) sont dans le plan \( x=0\) alors c'est d'accord parce que le graphe de \( f\) dans le plan est le même que celui de la fonction strictement convexe \( g\).

            Si \( A\) et \( B\) ne sont pas alignés à \( O\) et si ils ne sont pas dans le plan \( x=0\) alors le plan \( AOB\) coupe le plan \( x=1\) en une droite.

            Nous supposons l'existence d'un point \( C\in \mathopen] A , B \mathclose[\cap\mC\).

            Nous considérons la droite \( (OA)\) qui est contenue dans ce plan et dans \( \mC\) (au moins la partie positive) et nous notons \( A'\) son intersection avec le plan \( x=1\). Même chose pour \( B\) et \( C\) qui donnent \( B'\) et \( C'\).

            Cela nous donne des points \( A'\), \( B'\) et \( C'\) qui sont alignés dans le graphe de \( f\) en \( x=1\). Or le graphe de \( f\) en \( x=1\) est le graphe de la fonction \( g\) qui est strictement convexe et qui ne contient donc pas de points alignés.

            Nous en concluons que si \( A,B\in\mC\) alors \( \mathopen] A , B \mathclose[\) est soit complètement strictement au-dessus de \( \mC\) soit complètement strictement en-dessous de \( \mC\).

    \end{subproof}

    Nous prouvons à présent que toutes les cordes sont au-dessus de \( \mC\). Pour cela, soient \( A,B\in \mathopen] 0 , \infty \mathclose[\times \mathopen] 0 , \infty \mathclose[\), deux points non alignés à \( O=(0,0)\). Nous considérons les points \( A',B'\) qui sont les intersections entre les droites \( (AO)\) et \( (BO)\) et la droite \( x=1\) ainsi que le chemin \( \sigma\) qui parcours le segment \( \mathopen[ A , A' \mathclose]\) et le chemin \( \gamma\) qui parcours le segment \( \mathopen[ B , B' \mathclose]\) :
    \begin{equation}
        \begin{aligned}[]
            \sigma(0)&=A,&\gamma(0)&=B,\\
            \sigma(1)&=A',&\gamma(1)&=B'.
        \end{aligned}
    \end{equation}
    Pour tout \( u\), la seule droite passant par \( O\) et par \( \sigma(u)\) passe également par \( A\), et pas par \( B\). En conséquence de quoi, pour tout \( u_1,u_2\in \mathopen[ 0 , 1 \mathclose]\), la droite \( \big( \sigma(u_1)\sigma(u_2) \big)\) ne passe pas par \( (0,0)\).

    Nous considérons à présent non seulement la corde joignant \( \big( A,f(A) \big)\) à \( \big( B,f(B) \big)\) et la corde joignant \( \big( A',f(A') \big)\) à \( \big( B',f(B') \big)\) mais également toutes les cordes intermédiaires (si vous aimez les gros mots, vous pouvez parler d'homotopie) :
    \begin{equation}
        c(u,t)=t\Big( \sigma(t), f\big(\sigma(u)\big) \Big)+(1-t)\Big( \gamma(t),f\big( \gamma(t) \big) \Big)
    \end{equation}
    Pour chaque \( u\in\mathopen[ 0 , 1 \mathclose]\), cela représente une corde entre deux points non alignés à \( (0,0,0)\) et donc une corde qui est soit strictement au-dessus de \( \mC\) soit strictement en-dessous (à par les points correspondant à \( t=0\) et \( t=1\) qui, eux, sont sur \( \mC\)).

    Soit \( t_0\in \mathopen] 0 , 1 \mathclose[\). La courbe \( c(u,t_0)\) avec \( u\in\mathopen[ 0 , 1 \mathclose]\) ne touche jamais \( \mC\). Or le point \( c(1,t_0)\) est au-dessus de \( \mC\), donc le point \( c(0,t_0)\) est également au-dessus de \( \mC\).

    Nous en concluons que toutes les cordes entre \( (A,f(A)) \) et \( (B,f(B))\) est située au-dessus de \( \mC\) et non en-dessous de \( \mC\).

\end{proof}

%---------------------------------------------------------------------------------------------------------------------------
\subsection{Convexité et hessienne}
%---------------------------------------------------------------------------------------------------------------------------

\begin{definition}      \label{DEFooKCFPooLwKAsS}
    Soit une partie convexe \( U\) de \( \eR^n\) et une fonction \( f\colon U\to \eR\).
    \begin{enumerate}
        \item
        La fonction \( f\) est \defe{convexe}{convexe!fonction sur \( \eR^n\)} si pour tout \( x,y\in U\) avec \( x\neq y\) et pour tout \( \theta\in\mathopen] 0 , 1 \mathclose[\) nous avons
            \begin{equation}
                f\big( \theta x+(1-\theta)y \big)\leq \theta f(x)+(1-\theta)f(y).
            \end{equation}
        \item
            Elle est \defe{strictement convexe}{strictement!convexe!sur \( \eR^n\)} si nous avons l'inégalité stricte.
    \end{enumerate}
\end{definition}

\begin{proposition}[\cite{ooLJMHooMSBWki}]      \label{PROPooYNNHooSHLvHp}
    Soit \( \Omega\) ouvert dans \( \eR^n\) et \( U\) convexe dans \( \Omega\), et une fonction différentiable \( f\colon U\to \eR\).
    \begin{enumerate}
        \item       \label{ITEMooRVIVooIayuPS}
            La fonction \( f\) est convexe sur \( U\) si et seulement si pour tout \( x,y\in U\),
            \begin{equation}
                f(y)\geq f(x)+df_x(y-x).
            \end{equation}
        \item       \label{ITEMooCWEWooFtNnKl}
            La fonction \( f\) est strictement convexe sur \( U\) si et seulement si pour tout \( x,y\in U\) avec \( x\neq y\),
            \begin{equation}
                f(y)>f(x)+df_x(y-x).
            \end{equation}
    \end{enumerate}
\end{proposition}

\begin{proof}
    Nous avons quatre petites choses à démontrer.
    \begin{subproof}
    \item[\ref{ITEMooRVIVooIayuPS} sens direct]
        Soit une fonction convexe \( f\). Nous avons :
        \begin{equation}
            f\big( (1-\theta)x+\theta y \big)\leq (1-\theta)f(x)+\theta f(y),
        \end{equation}
        donc
        \begin{equation}
            f\big( x+\theta(y-x) \big)-f(x)\leq \theta\big( f(y)-f(x) \big)
        \end{equation}
        Vu que \( \theta>0\) nous pouvons diviser par \( \theta\) sans changer le sens de l'inégalité :
        \begin{equation}        \label{EQooAXXFooHWtiJh}
            \frac{ f\big( x+\theta(y-x) \big)-f(x) }{ \theta }\leq f(y)-f(x).
        \end{equation}
        Nous prenons la limite \( \theta\to 0^+\). Cette limite est égale à a limite simple \( \theta\to 0\) et vaut (parce que \( f\) est différentiable) :
        \begin{equation}
            \frac{ \partial f }{ \partial (y-x) }(x)\leq f(y)-f(x),
        \end{equation}
        et aussi
        \begin{equation}
            df_x(y-x)\leq f(y)-f(x)
        \end{equation}
        par le lemme~\ref{LemdfaSurLesPartielles}.
    \item[\ref{ITEMooRVIVooIayuPS} sens inverse]
        Pour tout \( a\neq b\) dans \( U\) nous avons
        \begin{equation}        \label{EQooEALSooJOszWr}
            f(b)\geq f(a)+df_a(b-a).
        \end{equation}
    Pour \( x\neq y\) dans \( U\) et pour \( \theta\in\mathopen] 0 , 1 \mathclose[\) nous écrivons \eqref{EQooEALSooJOszWr} pour les couples \( \big( \theta x+(1-\theta)y,y \big)\) et \( \big( \theta x+(1-\theta)y,x \big)\). Ça donne :
        \begin{equation}
            f(y)\geq f\big( \theta x+(1-\theta)y \big)+df_{\theta x+(1-\theta)y}\big( \theta(y-x) \big),
        \end{equation}
        et
        \begin{equation}
            f(x)\geq f\big( \theta x+(1-\theta)y \big)+df_{\theta x+(1-\theta)y}\big( (1-\theta)(x-y) \big).
        \end{equation}
        La différentielle est linéaire; en multipliant la première par \( (1-\theta)\) et la seconde par \( \theta\) et en la somme, les termes en \( df\) se simplifient et nous trouvons
        \begin{equation}
            \theta f(x)+(1-\theta)f(y)\geq f\big( \theta x+(1-\theta)y \big).
        \end{equation}
    \item[\ref{ITEMooCWEWooFtNnKl} sens direct]
        Nous avons encore l'équation \eqref{EQooAXXFooHWtiJh}, avec une inégalité stricte. Par contre, ça ne va pas être suffisant parce que le passage à la limite ne conserve pas les inégalités strictes. Nous devons donc être plus malins.

        Soient \( 0<\theta<\omega<1\). Nous avons \( (1-\theta)x+\theta y\in \mathopen[ x , (1-\omega)x+\omega y \mathclose]\), donc nous pouvons écrire \( (1-\theta)x+\theta y\) sous la forme \( (1-s)x+s\big( (1-\omega)x+\omega y \big)\). Il se fait que c'est bon pour \( s=\theta/\omega\) (et aussi que nous avons \( \theta/\omega<1\)). Donc nous avons
        \begin{subequations}
            \begin{align}
            f\big( (1-\theta)x+\theta y \big)&=f\Big( (1-\frac{ \theta }{ \omega })x+\frac{ \theta }{ \omega }\big( (1-\omega)x+\omega y \big) \Big)\\
            &<(1-\frac{ \theta }{ \omega })f(x)+\frac{ \theta }{ \omega }f\big( (1-\omega)x+\omega y \big).
            \end{align}
        \end{subequations}
        Cela nous permet d'écrire
        \begin{equation}
            \frac{ f\big( (1-\theta)x+\theta y \big)-f(x) }{ \theta }<\frac{ f\big( (1-\omega)x+\omega y \big) }{ \omega }<f(y)-f(x).
        \end{equation}
        Le seconde inégalité est le pendant de \eqref{EQooAXXFooHWtiJh}. Maintenant en passant à la limite pour \( \theta\) nous conservons une inégalité stricte par rapport à \( f(y)-f(x)\) :
        \begin{equation}
            df_x(y-x)<f(y)-f(x).
        \end{equation}
    \end{subproof}
\end{proof}

% Il faut laisser les sauts de lignes suivants, pour rechercher efficacement les références vers le futur.
Avant de lire la proposition suivante, il faut relire la proposition~\ref{PROPooFWZYooUQwzjW} et ce qui s'y rapporte.
Lire aussi la remarque~\ref{REMooVRPQooIybxmp} qui indique
qu'il n'y a pas de réciproque dans l'énoncé~\ref{ITEMooHAGQooYZyhQk}.
\begin{proposition}[\cite{ooLJMHooMSBWki}]      \label{PROPooBMIRooFkQSAb}
    Soit une fonction \( f\colon \Omega\to \eR\) deux fois différentiable sur l'ouvert \( \Omega\) de \( \eR^n\) et un convexe \( U\subset \Omega\).
    \begin{enumerate}
        \item       \label{ITEMooZQCAooIFjHOn}
            La fonction \( f\) est convexe sur \( U\) si et seulement si
            \begin{equation}        \label{EQooIBDCooJYdiBb}
                (d^2f)_x(y-x,y-x)\geq 0
            \end{equation}
            pour tout \( x,y\in U\).
        \item       \label{ITEMooHAGQooYZyhQk}
            Si pour tout \( x\neq y\) dans \( U\) nous avons
            \begin{equation}
                (d^2f)_x(y-x,y-x)>0
            \end{equation}
            alors la fonction \( f\) est strictement convexe sur \( U\).
    \end{enumerate}
\end{proposition}

\begin{remark}      \label{REMooYCRKooEQNIkC}
    Notons que la condition \eqref{EQooIBDCooJYdiBb} n'est pas équivalente à demander \( (d^2f)_x(h,h)\geq 0\) pour tout \( h\). En effet nous ne demandons la positivité que dans les directions atteignables comme différence de deux éléments de \( U\). La partie \( U\) n'est pas spécialement ouverte; elle pourrait n'être qu'une droite dans \( \eR^3\). Dans ce cas, demander que \( f\) (qui est \( C^2\) sur l'ouvert \( \Omega\)) soit convexe sur \( U\) ne demande que la positivité de \( (d^2f)_x\) appliqué à des vecteurs situés sur la droite \( U\).
\end{remark}

\begin{proof}
    Il y a trois parties à démontrer.
    \begin{subproof}
    \item[\ref{ITEMooZQCAooIFjHOn} sens direct]

        Soit une fonction convexe \( f\) sur \( U\). Soient aussi \( x,y\in U\) et \( h=y-x\). Nous utilisons ma version préférée de Taylor\footnote{Si vous présentez ceci au jury d'un concours, vous devriez être capable de raconter ce que signifie \( d^2f\), et pourquoi nous l'utilisons comme une \( 2\)-forme.} : celui de la proposition~\ref{PROPooTOXIooMMlghF} :
        \begin{equation}
            f(x+th)=f(x)+tdf_x(h)+\frac{ t^2 }{2}(d^2_x)(h,h)+t^2\| h \|^2\alpha(th)
        \end{equation}
        avec \( \lim_{s\to 0}\alpha(s)=0\). Le fait que \( f\) soit convexe donne
        \begin{equation}
            0\leq f(x+th)-f(x)-tdf_x(h),
        \end{equation}
        et donc
        \begin{equation}
            0\leq \frac{ t^2 }{2}(d^2f)_x(h,h)+f^2\| h \|^2\alpha(th).
        \end{equation}
        En multipliant par \( 2\) et en divisant par \( t^2\),
        \begin{equation}
            0\leq (d^2f)_x(h,h)+2\| h \|^2\alpha(th).
        \end{equation}
        En prenant \( t\to 0\) nous avons bien  \( (d^2f)_x(y-x,y-x)\geq 0\).

    \item[\ref{ITEMooZQCAooIFjHOn} sens inverse]

        Soient \( x,y\in U\). Nous écrivons Taylor en version de la proposition~\ref{PROPooWWMYooPOmSds} :
        \begin{equation}
            f(y)=f(x)+df_x(y-x)+\frac{ 1 }{2}(d^2f)_z(y-x,y-x)
        \end{equation}
    pour un certain \( z\in\mathopen] x , y \mathclose[\). En vertu de ce qui a été dit dans la remarque~\ref{REMooYCRKooEQNIkC} nous ne pouvons pas évoquer l'hypothèse \eqref{EQooIBDCooJYdiBb} pour conclure que \( (d^2f)_z(y-x,y-x)\geq 0\). Il y a deux manières de nous sortir du problème :
        \begin{itemize}
            \item Trouver \( s\in U\) tel que \( y-x=s-z\).
            \item Trouver un multiple de \( y-x\) qui soit de la forme \( y-x\).
        \end{itemize}
        La première approche ne fonctionne pas parce que \( s=y-x+z\) n'est pas garanti d'être dans \( U\); par exemple avec \( x=1\), \( z=2\), \( y=3\) et \( U=\mathopen[ 0 , 3 \mathclose]\). Dans ce cas \( s=4\notin U\).

        Heureusement nous avons \( z=\theta x+(1-\theta)y\), donc \( z-x=(1-\theta)(y-x)\). Dans ce cas la bilinéarité de \( (d^2f)_z\) donne\footnote{Si vous avez bien suivi, la bilinéarité est contenue dans la proposition~\ref{PROPooFWZYooUQwzjW}.}
        \begin{equation}
            f(y)=f(x)+df_x(y-x)+\underbrace{\frac{ 1 }{2}\frac{1}{ (1-\theta)^2 }(d^2f)_z(z-x,z-x)}_{\geq 0}.
        \end{equation}
        Nous en déduisons que \( f\) est convexe par la proposition~\ref{PROPooYNNHooSHLvHp}\ref{ITEMooRVIVooIayuPS}.
    \item[\ref{ITEMooHAGQooYZyhQk}]

        Le raisonnement que nous venons de faire pour le sens inverse de~\ref{ITEMooZQCAooIFjHOn} tient encore, et nous avons
        \begin{equation}
            f(y)=f(x)+df_x(y-x)+\underbrace{\frac{ 1 }{2}\frac{1}{ (1-\theta)^2 }(d^2f)_z(z-x,z-x)}_{> 0}
        \end{equation}
        d'où nous déduisons la stricte convexité de \( f\) par la proposition~\ref{PROPooYNNHooSHLvHp}\ref{ITEMooCWEWooFtNnKl}.
    \end{subproof}
\end{proof}

\begin{corollary}       \label{CORooMBQMooWBAIIH}
    Soit un ouvert \( \Omega\) de \( \eR^n\) et une fonction deux fois différentiable \( f\) sur \( \Omega\).
    \begin{enumerate}
        \item   \label{ITEMooUAFTooXfCviI}
            La fonction \( f\) est convexe si et seulement si pour tout \( x\), la matrice hessienne \( d^2f_x\) est semi-définie positive.
        \item   \label{ITEMooDGISooPlRLOd}
            Si pour tout \( x\) de \( \Omega\), la matrice hessienne \( d^2f_x\) est strictement définie positive, alors \( f \) est strictement convexe.
    \end{enumerate}
\end{corollary}

\begin{proof}
    Nous pouvons voir ce résultat comme une conséquence directe de la proposition~\ref{PROPooBMIRooFkQSAb} en posant \( U=\Omega\). Nous allons cependant en donner une démonstration directe.

    Soit \( a\in \Omega\) et posons la fonction
    \begin{equation}
        \begin{aligned}
            g\colon \Omega&\to \eR \\
            x&\mapsto f(x)-f(a)-(df)_a(x-a).
        \end{aligned}
    \end{equation}
    Nous allons calculer des différentielles de \( f\), et une chose importante à comprendre est que la différentielle de la fonction \( x\mapsto df_a(x-a)\) ne fait pas intervenir la différentielle seconde de \( f\); c'est la différentielle de \( a\mapsto df_a(x)\) qui demanderait la différentielle seconde de \( f\). Ici la point \( a\) étant donné, \( df_a\) est une application linéaire sans histoires. En particulier, \( df_a(x-a)=df_a(x)-df_a(a)\).

    La fonction \( g\) vérifie :
    \begin{enumerate}
        \item
            \( g(a)=0\),
        \item
            \( dg_x=df_x-df_a\), parce que la différentielle de \( x\mapsto df_a(x)\) est \( x\mapsto df_a(x)\) en vertu du lemme~\ref{LemooXXUGooUqCjmp}.
        \item
            \( dg_a=0\). Le point \( a\) est un point critique de \( g\).
        \item
            \( d^2g_x=d^2f_x\) parce que la différentielle de \( x\mapsto df_a\) est nulle.
    \end{enumerate}
    Ceci étant dit, nous pouvons commencer avec la preuve.
    \begin{subproof}
        \item[\ref{ITEMooUAFTooXfCviI} sens direct]

            Nous supposons que \( f\) est convexe. Alors \( g(x)\geq 0\) pour tout \( x\) par la caractérisation~\ref{PROPooYNNHooSHLvHp}\ref{ITEMooRVIVooIayuPS}. Cela signifie que \( x=0\) est un minimum global de \( g\). Par conséquent la proposition~\ref{PropoExtreRn}\ref{ItemPropoExtreRn} nous dit que la Hessienne \( d^2f_a\) est semi-définie positive.


        \item[\ref{ITEMooUAFTooXfCviI} sens inverse]

            Nous sommes dans le cas de la proposition~\ref{PropoExtreRn}\ref{ITEMooCBMYooQQMqQL}. Le point \( x=a\) est un minimum local de \( g\), ce qui signifie que \( g(x)\geq 0\) pour tout \( x\) de \( \Omega\). Encore une fois la caractérisation~\ref{PROPooYNNHooSHLvHp}\ref{ITEMooRVIVooIayuPS} nous permet de conclure.

        \item[\ref{ITEMooDGISooPlRLOd}]

            La fonction \( g\) vérifie les conditions de~\ref{PropoExtreRn}\ref{ITEMooCVFVooWltGqI}, donc \( x=0\) est un minimum local strict de \( g\). La caractérisation~\ref{PROPooYNNHooSHLvHp}\ref{ITEMooCWEWooFtNnKl} nous fait conclure que \( f\) est strictement convexe.

    \end{subproof}
\end{proof}

\begin{normaltext}
    Nous rappelons que, avec \( p>1\), la fonction
    \begin{equation}
        \begin{aligned}
            f\colon \mathopen[ 0 , \infty \mathclose[&\to \eR \\
            x&\mapsto x^p 
        \end{aligned}
    \end{equation}
    est strictement croissante. La proposition \ref{PROPooRXLNooWkPGsO} est formelle sur ce point.
\end{normaltext}

\begin{lemma}
    Soient un réel \( a\), et \( p>1\). L'équation
    \begin{equation}
        | x |^p=ax
    \end{equation}
    possède au plus deux solutions réelles.
\end{lemma}

\begin{proof}
    En deux parties suivant le signe de \( a\).
    \begin{subproof}
        \item[Si \( a=0\)]
            Alors l'équation est \( | x |^p=0\), et l'unique solution est \( x=0\)
        \item[Si \( a>0\)]
            Si \( x<0\) alors \( ax<0\) et \( | x |^p-ax>| x |^p>0\). Cela prouve que notre équation n'a pas de solutions \( x<0\) lorsque \( a>0\). 

            Cherchons donc des solutions avec \( x\geq 0\). D'abord \( | x |=x\) et ensuite, en posant \( p=1+\delta\) (\( \delta>0\)), nous avons la factorisation
            \begin{equation}
                x^p-ax=x(x^{\delta}-a)=0.
            \end{equation}
            Poser \( x=0\) est clairement une solution. Si \( x\neq 0\), nous avons le raisonnement suivant. Les réels formant un corps, c'est un anneau intègre qui vérifie alors la règle du produit nul\footnote{Voir le lemme \ref{LemAnnCorpsnonInterdivzer} et la définition \ref{DEFooTAOPooWDPYmd}.}. Cela pour dire que si \( x\neq 0\), alors
            \begin{equation}
                x^{\delta}=a.
            \end{equation}
            La proposition \ref{PROPooEXGKooCqzLor} nous dit que cela a une unique solution dans les réels positifs.
        \item[Si \( a<0\)] Ce cas se traite de façons similaire\quext{Vérifiez par vous-même et écrivez-moi si il y a un problème (personnellement, je n'ai pas vérifié.).}.
    \end{subproof}
\end{proof}

\begin{proposition}     \label{PROPooLIGIooPrHYlb}
    Soit \( 1<p<\infty\). La fonction
    \begin{equation}
        \begin{aligned}
            f\colon \eR^n&\to \eR \\
            x&\mapsto \| x \|^p 
        \end{aligned}
    \end{equation}
    est strictement convexe.
\end{proposition}

\begin{proof}
    La preuve va se diviser en deux parties. D'abord nous allons utiliser la matrice Hessienne pour démontrer le résultat sur l'ouvert \( \eR^n\setminus\{ 0 \}\), et ensuite nous allons un peu bricoler pour ajouter \( 0\) au domaine de stricte convexité\quext{Si quelqu'un sait comment éviter ce bricolage en deux parties, je suis preneur.}.

    \begin{subproof}
    \item[La Hessienne]
        Nous notons
        \begin{equation}
            f_p(x)=\| x \|^p=\big( \sum_ix_i^2 \big)^{p/2},
        \end{equation}
        et nous dérivons :
        \begin{equation}        \label{EQooNZZWooGeAlyj}
            \frac{ \partial f_p }{ \partial x_i }(x)=\frac{ p }{2}2x_i\big( \sum_ix_i^2 \big)^{(p-2)/2}=px_if_{p-2}(x).
        \end{equation}
        Cela n'est déjà pas bien défini en \( x=0\) lorsque \( p<2\), mais qu'importe ? Nous dérivons encore en utilisant entre autres la formule \eqref{EQooNZZWooGeAlyj} elle-même avec \( p\to p-2\) :
        \begin{equation}
            \frac{ \partial^2f_p  }{ \partial x_j\partial x_i }(x)=p_{\delta_{ij}}\| x \|^{p-2}+p(p-2)x_ix_j\| p-4 \|.
        \end{equation}
        Nous avons donc la matrice Hessienne
        \begin{equation}
            H_{ij}(x)=p\delta_{ij}\| x \|^{p-2}+p(p-2)x_ix_j\| x \|^{p-4}.
        \end{equation}
        Pour prouver que cette matrice est strictement définie positive, nous avons le choix entre la proposition \ref{PropcnJyXZ}\ref{ITEMooTJVQooYmRkas} ou le lemme \ref{LemWZFSooYvksjw}\ref{ITEMooSKRAooOgHbGA}. Nous utilisons le second. Nous avons\footnote{Si vous ne savez pas où placer les indices, voyez la proposition \ref{PROPooZKWXooWmEzoA}.} :
        \begin{subequations}
            \begin{align}
                y\cdot H(x)y&=\sum_{kl}y_kH(x)_{kl}y_l\\
                &=p\sum_ly_l^2\| x \|^{p-2}+p(p-2)\| x \|^{p-4}\sum_{kl}x_ky_kx_ly_l\\
                &=p\| y \|^2\| x \|^{p-2}+p(p-2)\| x \|^{p-4}(x\cdot y)^2\\
                &=p\| x \|^{p-4}\big( \| y \|^2\| x \|^2+(p-2)(x\cdot y)^2 \big)\\
                &>p\| x \|^{p-4}\big( \| y \|^2\| x \|^2-(x\cdot y)^2 \big) \label{SUBEQooUSZOooCqgWPE}\\   
                &\geq 0     \label{SUBEQooBXQKooZcarVv}.
            \end{align}
        \end{subequations}
        Justifications :
        \begin{itemize}
            \item Pour \ref{SUBEQooUSZOooCqgWPE}.
                Vu que \( p>1\) nous avons \( p-2>-1\). Là, l'inégalité est stricte et c'est important.
            \item Pour \ref{SUBEQooBXQKooZcarVv}. C'est l'inégalité de Cauchy-Schwarz du théorème \ref{ThoAYfEHG}.
        \end{itemize}
    Voila. La matrice Hessienne est strictement définie positive par le lemme \ref{LemWZFSooYvksjw}\ref{ITEMooSKRAooOgHbGA} sur \( \eR^n\setminus\{ 0 \}\). Le corolaire \ref{CORooMBQMooWBAIIH}\ref{ITEMooUAFTooXfCviI} nous indique que pour tout \( x,y\in \eR^n\setminus\{ 0 \}\) et pour tout \( \theta\in \mathopen] 0 , 1 \mathclose[\),
        \begin{equation}        \label{EQooFXXZooBZhJYY}
            f_p\big( \theta x+(1-\theta)y \big)<\theta f_p(x)+(1-\theta)f_p(y)
        \end{equation}
        pourvu que \( \theta x+ (1-\theta)y\neq 0\) pour tout \( \theta\).

    \item[La suite]

        Nous devons prouver que l'inéquation \eqref{EQooFXXZooBZhJYY} tient également lorsque \( \theta x+(1-\theta)y=0\) pour une certaine valeur \( \theta=\theta_0\in \mathopen] 0 , 1 \mathclose[\).

    \item[Les cordes passant par zéro]

        Pour ce faire, nous allons montrer que le segment de droite joignant \( \big( a,f_p(a) \big)\) à \( (b,f_p(b))\) est toujours au-dessus de la courbe \( \big( x,f_p(x) \big)\). Nous commençons par \( b=0\). Vu que \( p>1\) et que \( \theta\in \mathopen] 0 , 1 \mathclose[\) nous avons
        \begin{equation}
            \| \theta a \|^p=| \theta |^p\| a \|^p<\theta\| a \|^p=\theta f_p(a).
        \end{equation}

    \item[Les cordes passant au-dessus de zéro]
        
        Dans le cas \( b\neq 0\) nous considérons
        \begin{equation}
            \begin{aligned}
                l_1\colon \mathopen[ a , b \mathclose]&\to \eR \\
                x&\mapsto l_1(x) 
            \end{aligned}
        \end{equation}
        tel que \( \big( x,l_1(x) \big)\) soit (le segment) la droite joignant \( \big( a,\| a \|^p \big)\) à \( \big( b,\| b \|^p \big)\).

        Nous considérons aussi, pour \( x\in\mathopen[ 0 , a \mathclose]\) la fonction \( l_2\) telle que \( \big( x,l_2(x) \big)\) soit la droite joignant \( (0,0)\) à \( \big( a,\| a \|^p \big)\). Nous avons déjà vu que pour \( x\in \mathopen] 0 , a \mathclose[\) nous avons \( l_1(x)>\| x \|^p\).
        
        Nous avons \( l_1(a)=l_2(a)=\| a \|^p\). Donc \( l_2(x)-l_1(x)\) ne change pas de signe sur \( \mathopen[ 0 , a \mathclose]\). Mais comme \( l_1(0)>0=l_2(0)\) nous avons
        \begin{equation}
            l_2(x)>l_1(x)
        \end{equation}
        pour tout \( x\in \mathopen[ 0 , a \mathclose]\).

        Au final, \( l_2(x)>l_1(x)>f_p(x)\).

        Pour la partie \( \mathopen[ b , 0 \mathclose]\) nous faisons de même en considérant \( l_3\) de telle sorte que \( \big( x,l_3(x) \big)\) soit le segment joignant \( (0,0)\) à \( \big( b,\| b \|^p \big)\).
    \end{subproof}
\end{proof}

%---------------------------------------------------------------------------------------------------------------------------
\subsection{Quelques inégalités}
%---------------------------------------------------------------------------------------------------------------------------

%///////////////////////////////////////////////////////////////////////////////////////////////////////////////////////////
\subsubsection{Inégalité de Jensen}
%///////////////////////////////////////////////////////////////////////////////////////////////////////////////////////////
\index{inégalité!Jensen}
\index{convexité!inégalité de Jensen}

\begin{proposition}[Inégalité de Jensen]    \label{PropXIBooLxTkhU}
    Soit \( f\colon \eR\to \eR\) une fonction convexe et des réels \( x_1\),\ldots,  \( x_n\). Soient des nombres positifs \( \lambda_1\),\ldots,  \( \lambda_n\) formant une combinaison convexe\footnote{Définition~\ref{DefIMZooLFdIUB}.}. Alors
    \begin{equation}
        f\big( \sum_i\lambda_ix_i \big)\leq \sum_i\lambda_if(x_i).
    \end{equation}
\end{proposition}
\index{inégalité!Jensen!pour une somme}

\begin{proof}
    Nous procédons par récurrence sur \( n\), en sachant que \( n=2\) est la définition de la convexité de \( f\). Vu que
    \begin{equation}
        \sum_{k=1}^n\lambda_kx_k=\lambda_nx_n+(1-\lambda_n)\sum_{k=1}^{n-1}\frac{ \lambda_kx_k }{ 1-\lambda_n },
    \end{equation}
    nous avons
    \begin{equation}
        f\big( \sum_{k=1}^n\lambda_kx_k \big)\leq \lambda_nf(x_n)+(1-\lambda_n)f\big( \sum_{k=1}^{n-1}\frac{ \lambda_kx_k }{ 1-\lambda_n } \big).
    \end{equation}
    La chose à remarquer est que les nombres \( \frac{ \lambda_k }{ 1-\lambda_n }\) avec \( k\) allant de \( 1\) à \( n-1\) forment eux-mêmes une combinaison convexe. L'hypothèse de récurrence peut donc s'appliquer au second terme du membre de droite :
    \begin{equation}
        f\big( \sum_{k=1}^n\lambda_kx_k \big)\leq \lambda_nf(x_n)+(1-\lambda_n)\sum_{k=1}^{n-1}\frac{ \lambda_k }{ 1-\lambda_n }f(x_k)=\lambda_nf(x_n)+\sum_{k=1}^{n-1}\lambda_kf(x_k).
    \end{equation}
\end{proof}

%///////////////////////////////////////////////////////////////////////////////////////////////////////////////////////////
\subsubsection{Inégalité arithmético-géométrique}
%///////////////////////////////////////////////////////////////////////////////////////////////////////////////////////////

La proposition suivante dit que la moyenne arithmétique de nombres strictement positifs est supérieure ou égale à la moyenne géométrique.
\begin{proposition}[Inégalité arithmético-géométrique\cite{CENooZKvihz}]    \label{PropWDPooBtHIAR}
    Soient \( x_1\),\ldots, \( x_n\) des nombres strictement positifs. Nous posons
    \begin{equation}
        m_a=\frac{1}{ n }(x_1+\cdots +x_n)
    \end{equation}
    et
    \begin{equation}
        m_g=\sqrt[n]{x_1\ldots x_n}
    \end{equation}
    Alors \( m_g\leq m_a\) et \( m_g=m_a\) si et seulement si \( x_i=x_j\) pour tout \( i,j\).
\end{proposition}
\index{inégalité!arithmético-géométrique}

\begin{proof}
    Par hypothèse les nombres \( m_a\) et \( m_g\) sont tout deux strictement positifs, de telle sorte qu'il est équivalent de prouver \( \ln(m_g)\leq \ln(m_a)\) ou encore
    \begin{equation}
        \frac{1}{ n }\big( \ln(x_1)+\cdots +\ln(x_n) \big)\leq \ln\left( \frac{ x_1+\cdots +x_n }{ n } \right).
    \end{equation}
    Cela n'est rien d'autre que l'inégalité de Jensen de la proposition~\ref{PropXIBooLxTkhU} appliquée à la fonction \( \ln\) et aux coefficients \( \lambda_i=\frac{1}{ n }\).
\end{proof}

%///////////////////////////////////////////////////////////////////////////////////////////////////////////////////////////
\subsubsection{Inégalité de Kantorovitch}
%///////////////////////////////////////////////////////////////////////////////////////////////////////////////////////////

\begin{proposition}[Inégalité de Kantorovitch\cite{EYGooOoQDnt}]    \label{PropMNUooFbYkug}
    Soit \( A\) une matrice symétrique strictement définie positive dont les plus grandes et plus petites valeurs propres sont \( \lambda_{min}\) et \( \lambda_{max}\). Alors pour tout \( x\in \eR^n\) nous avons
    \begin{equation}
        \langle Ax, x\rangle \langle A^{-1}x, x\rangle \leq \frac{1}{ 4 }\left( \frac{ \lambda_{min} }{ \lambda_{max} }+\frac{ \lambda_{max} }{ \lambda_{min} } \right)^2\| x^4 \|.
    \end{equation}
\end{proposition}
\index{inégalité!Kantorovitch}

\begin{proof}
    Sans perte de généralité nous pouvons supposer que \( \| x \|=1\). Nous diagonalisons\footnote{Théorème spectral~\ref{ThoeTMXla}.} la matrice \( A\) par la matrice orthogonale  \( P\in\gO(n,\eR)\) : \( A=PDP^{-1}\) et \( A^{-1}=PD^{-1}P^{-1}\) où \( D\) est  une matrice diagonale formée des valeurs propres de \( A\).

    Nous posons \( \alpha=\sqrt{\lambda_{min}\lambda_{max}}\) et nous regardons la matrice
    \begin{equation}
        \frac{1}{ \alpha }A+\alpha A^{-1}
    \end{equation}
    dont les valeurs propres sont
    \begin{equation}
        \frac{ \lambda_i }{ \alpha }+\frac{ \alpha }{ \lambda_i }
    \end{equation}
    parce que les vecteurs propres de \( A\) et de \( A^{-1}\) sont les mêmes (ce sont les valeurs de la diagonale de \( D\)). Nous allons quelque peu étudier la fonction
    \begin{equation}
        \theta(x)=\frac{ x }{ \alpha }+\frac{ \alpha }{ x }.
    \end{equation}
    Elle est convexe en tant que somme de deux fonctions convexes. Elle a son minimum en \( x=\alpha\) et ce minimum vaut \( \theta(\alpha)=2\). De plus
    \begin{equation}
        \theta(\lambda_{max})=\theta(\lambda_{min})=\sqrt{\frac{ \lambda_{min} }{ \lambda_{max} }}+\sqrt{\frac{ \lambda_{max} }{ \lambda_{min} }}.
    \end{equation}
    Une fonction convexe passant deux fois par la même valeur doit forcément être plus petite que cette valeur entre les deux\footnote{Je ne suis pas certain que cette phrase soit claire, non ?} : pour tout \( x\in\mathopen[ \lambda_{min} , \lambda_{max} \mathclose]\),
    \begin{equation}
        \theta(x)\leq  \sqrt{\frac{ \lambda_{min} }{ \lambda_{max} }}+\sqrt{\frac{ \lambda_{max} }{ \lambda_{min} }}.
    \end{equation}

    Nous sommes maintenant en mesure de nous lancer dans l'inégalité de Kantorovitch.
    \begin{subequations}
        \begin{align}
            \sqrt{\langle Ax, x\rangle \langle A^{-1}x, x\rangle }&\leq\frac{ 1 }{2}\left( \frac{ \langle Ax, x\rangle  }{ \alpha }+\alpha\langle A^{-1}x, x\rangle  \right)\label{subEqUKIooCWFSkwi}\\
            &=\frac{ 1 }{2}\langle   \big( \frac{ A }{ \alpha }+\alpha A^{-1} \big)x , x\rangle \\
            &\leq\frac{ 1 }{2}\Big\| \big( \frac{ A }{ \alpha }+\alpha A^{-1} \big)x \|\| x \| \label{subEqUKIooCWFSkwiii}\\
            &\leq \frac{ 1 }{2}\| \frac{ A }{ \alpha }+\alpha A^{-1} \| \label{subEqUKIooCWFSkwiv}
        \end{align}
    \end{subequations}
    Justifications :
    \begin{itemize}
        \item~\ref{subEqUKIooCWFSkwi} par l'inégalité arithmético-géométrique, proposition~\ref{PropWDPooBtHIAR}. Nous avons aussi inséré \( \alpha\frac{1}{ \alpha }\) dans le produit sous la racine.
        \item~\ref{subEqUKIooCWFSkwiii} par l'inégalité de Cauchy-Schwarz, théorème~\ref{ThoAYfEHG}.
        \item~\ref{subEqUKIooCWFSkwiv} par la définition de la norme opérateur de la proposition~\ref{DefNFYUooBZCPTr}
    \end{itemize}
    La norme opérateur est la plus grande des valeurs propres. Mais les valeurs propres de \( A/\alpha+\alpha A^{-1}\) sont de la forme \( \theta(\lambda_i)\), et tous les \( \lambda_i\) sont entre \( \lambda_{min} \) et \( \lambda_{max}\). Donc la plus grande valeur propre de \( A/\alpha+\alpha A^{-1}\) est \( \theta(x)\) pour un certain \( x\in\mathopen[ \lambda_{min} , \lambda_{max} \mathclose]\). Par conséquent
    \begin{equation}
            \sqrt{\langle Ax, x\rangle \langle A^{-1}x, x\rangle }\leq \frac{ 1 }{2}\| \frac{ A }{ \alpha }+\alpha A^{-1} \| \leq \sqrt{\frac{ \lambda_{min} }{ \lambda_{max} }}+\sqrt{\frac{ \lambda_{max} }{ \lambda_{min} }}.
    \end{equation}
\end{proof}

%///////////////////////////////////////////////////////////////////////////////////////////////////////////////////////////
\subsubsection{Inégalité de Hölder}
%///////////////////////////////////////////////////////////////////////////////////////////////////////////////////////////

Si vous cherchiez l'inégalité de Hölder dans \( L^p\), c'est la proposition \ref{ProptYqspT}. Les normes \(  \ell^p\) sont définies dans \ref{PROPooCLZRooIRxCnZ}.

\begin{lemma}[\cite{BIBooVEXYooXCkRQV}]     \label{LEMooLGGDooGLGFHj}
    Soient \( x,y>0\) ainsi que \( \alpha,\beta>0\) tels que \( \alpha+\beta=1\). Alors
    \begin{equation}
        xy\leq \alpha x^{1/\alpha}+\beta e^{1/\beta}.
    \end{equation}
    Nous avons
    \begin{equation}
        xy= \alpha x^{1/\alpha}+\beta e^{1/\beta}
    \end{equation}
    si et seulement si \( x^{1/\alpha}=y^{1/\beta}\).
\end{lemma}

\begin{proof}
    Nous utilisons le logarithme\footnote{Définition \ref{DEFooELGOooGiZQjt}} et ses propriétés (surtout la proposition \ref{PROPooLAOWooEYvXmI}). D'abord
    \begin{equation}
        xy= e^{\ln(xy)}= e^{\ln(x)+\ln(y)}= e^{\alpha\frac{ \ln(x) }{ \alpha }+\frac{ \ln(y) }{ \beta }}
    \end{equation}
    Vu que l'exponentielle est strictement convexe (exemple \ref{ExPDRooZCtkOz}\ref{ITEMooRXSBooDBerbx}) et vu que \( \alpha+\beta=1\), nous avons
    \begin{equation}        \label{EQooNLQIooAYiEAO}
        xy e^{\alpha\frac{ \ln(x) }{ \alpha }+\frac{ \ln(y) }{ \beta }}\leq \alpha e^{\ln(x)/\alpha}+\beta e^{\ln(y)/\beta}=\alpha x^{1/\alpha}+\beta y^{1/\beta}.
    \end{equation}
    Vu que \( \alpha\) et \( \beta\) ne sont pas nuls, l'inégalité \eqref{EQooNLQIooAYiEAO} est une égalité si et seulement si 
    \begin{equation}
        \frac{ \ln(x) }{ \alpha }=\frac{ \ln(y) }{ \beta }.
    \end{equation}
    Cela signifie \( \ln(x^{1/\alpha})=\ln(y^{1/\beta})\), qui implique \( x^{1/\alpha}=y^{1/\beta}\) parce que le logarithme est une bijection.
\end{proof}

\begin{corollary}       \label{CORooTCBZooAcZxaC}
    Si \( p,q>0\) vérifient \( \frac{1}{ p }+\frac{1}{ q }=1\) et si \( p>1\) alors nous avons
    \begin{equation}        \label{EQooWKTSooQwRsLz}
        xy\leq \frac{1}{ p }x^{p}+\frac{1}{ q }y^q
    \end{equation}
    pour tout \( x,y\geq 0\), avec une égalité si et seulement si \( x^p=y^q\).
\end{corollary}

\begin{proof}
    Il suffit de poser \( \alpha=1/p\) et \( \beta=1/q\) et appliquer le lemme \ref{LEMooLGGDooGLGFHj}.
\end{proof}

\begin{theorem}[Inégalité de Hölder\cite{BIBooVEXYooXCkRQV}]        \label{THOooYHMJooBlXfpl}
    Soient \( p,q>0\) tels que \( \frac{1}{ p }+\frac{1}{ q }=1\). Pour tout \( x,y\in \eR^n\) nous avons
    \begin{equation}
        | x\cdot y |=\sum_{i=1}^n| x_iy_i |\leq \| x \|_p\| x \|_q.
    \end{equation}
    Il y a égalité si et seulement si \( x_iy_i\) est de signe constant\quext{Je n'utilie pas cette hypothèse de signe constant. Il doit y avoir une subtilité qui m'a échappée. Soyez prudente en lisant et écrivez-moi si vous trouvez une erreur.} et les vecteurs \( \sum_i| x_i |^pe_i\) et \( \sum_i| y_i |^qe_i\) sont proportionnels.
\end{theorem}

\begin{proof}
    En plusieurs parties.
    \begin{subproof}
        \item[Le cas des vecteurs nuls]
            Si \( x\) ou \( y\) est nul, les inégalités sont évidentes. Donc nous supposons que non.
        \item[Première inégalité]

            En ce qui concerne la première inégalité,
            \begin{equation}
                | x\cdot y |=| \sum_ix_iy_i |\leq \sum_i| x_i | |y_i |\leq \sum_i\left( \frac{1}{ p }| x_i |^p+\frac{1}{ q }| y_i |^q \right)
            \end{equation}
            où nous avons utilisé le corolaire \ref{CORooTCBZooAcZxaC} dans chaque terme de la somme en tenant compte du fait que \( | x_i |\) et \( | y_i |\) sont positifs.

        \item[Seconde inégalité]
            Pour la seconde inégalité, nous commençons avec \( \| x \|_p=\| y \|_q=1\). Utilisant encore le corolaire \ref{CORooTCBZooAcZxaC} pour chaque terme, nous avons
            \begin{equation}
                \sum_i| x_iy_i |\leq\frac{1}{ p }\underbrace{\sum_i| x_i |^p}_{=1}+\frac{1}{ q }\underbrace{\sum_i| y_i |^q}_{=1}=\frac{1}{ p }+\frac{1}{ q }=1=\| x \|_p\| y \|_q.
            \end{equation}
            
            Si maintenant \( x\) et \( y\) sont arbitraires non nuls dans \( \eR^n\), nous posons \( x'=x/\| x \|_p\) et \( y'=y/\| y \|_q\); nous savons déjà que
            \begin{equation}        \label{EQooRRECooNpopuo}
                \sum_i| x'_iy'_i |\leq 1.
            \end{equation}
            En remplaçant \( x'_i\) par \( x_i/\| x \|_p\) et \( y'_i\) par \( y_i/\| y \|_q\), l'inégalité \eqref{EQooRRECooNpopuo} devient
            \begin{equation}
                \sum_i\frac{ | x_i |y_i }{ \| x \|_p\| y \|_q  }\leq 1,
            \end{equation}
            ce qui signifie
            \begin{equation}
                \sum_i| x_iy_i |\leq \| x \|_p\| y \|_q.
            \end{equation}
        \item[Cas d'égalité, dans un sens]
            Nous notons
            \begin{equation}
                \mD=\{ (x,y)\in \eR^n\times \eR^n\tq \sum_i| x_iy_i |=\| x \|_p\| y \|_q \}.
            \end{equation}
            \begin{subproof}
            \item[Multiplications]
            D'abord si \( (x,y)\in\mD\), alors \( (\mu x, \lambda y)\in \mD\) pour tout \( \mu,\lambda\in \eR\). En effet,
            \begin{equation}
                \sum_i| (\mu x)_i(\lambda y)_i |=| \mu | |\lambda |\sum_i| x_iy_i |=| \mu | |\lambda |\| x \|_p\| y \|_q=\|\mu x \|_p\| \lambda y \|_q.
            \end{equation}
        \item[Avec normes égales à \( 1\)]
                Soit \( (x,y)\in \mD\) tels que \( \| x \|_p=\| y \|_q=1\). Nous avons en particulier,
                \begin{equation}    \label{EQooDFNWooOSTygU}
                    1=\sum_i| x_iy_i |\leq \frac{1}{ p }\sum_i| x_i |^p+\frac{1}{ q }\sum_i| y_i |^q
                \end{equation}
                grâce à l'inégalité \eqref{EQooWKTSooQwRsLz} appliquée à chaque terme. Vu que \( \| x \|_p=1\), nous avons \( \sum_i| x_i |^p=1\), de telle sorte que le membre de droite de \eqref{EQooDFNWooOSTygU} se réduise à \( \frac{1}{ p }+\frac{1}{ q }=1\).

                Nous pouvons donc écrire
                \begin{equation}
                    1=\sum_i| x_iy_i |\leq \frac{1}{ p }\sum_i| x_i |^p+\frac{1}{ q }\sum_i| y_i |^q=1.
                \end{equation}
                L'inégalité est donc une égalité :
                \begin{equation}
                    \sum_i| x_iy_i |=\sum_i\left( \frac{1}{ p }| x_i |^p+\frac{1}{ q }| y_i |^q \right).
                \end{equation}
                Mais chaque terme à gauche est en inégalité avec le terme correspondant à droite :
                \begin{equation}        \label{EQooAGFKooUsYrWT}
                    | x_iy_i |\leq \frac{1}{ p }| x_i |^p+\frac{1}{ q }| y_i |^q.
                \end{equation}
                Pour que le tout soit une égalité, il faut que chaque inégalité \eqref{EQooAGFKooUsYrWT} soit une égalité. Pour chaque \( i\), nous avons
                \begin{equation}
                    | x_iy_i |=\frac{1}{ p }| x_i |^p+\frac{1}{ q }| y_i |^q.
                \end{equation}
                La condition d'égalité du corolaire \ref{CORooTCBZooAcZxaC} nous dit alors que \( | x_i |^p=| y_i |^q\).
        \item[Avec normes arbitraires]
                Soit donc \( (x,y)\in \mD\). Nous savons qu'en posant \( x'=x/\| x \|_p\) et \( y'=y/\| y \|_q\) nous avons \( (x',y')\in \mD\) et donc
                \begin{equation}
                    \left( \frac{ | x_i | }{ \| x \|_p } \right)^p=\left( \frac{ | y_i | }{ \| y \|_q } \right)^q.
                \end{equation}
                Cela donne tout de suite
                \begin{equation}
                    | x_i |^p=\frac{ \| x \|_p^p }{ \| y \|_q^q }| y_i |^q,
                \end{equation}
                ce qui est bien ce que nous voulions : le vecteur \( \sum_i| x_i |^pe_i\) est proportionnel au vecteur \( \sum_i| y_i |^qe_i\).
            \end{subproof}
        \item[Cas d'égalité dans l'autre sens\cite{MonCerveau}]
            Nous supposons que les vecteurs \( \sum_i| x_i |^pe_i\) et \( \sum_i| y_i |^qe_i\) sont proportionnels. Nous nommons \( c^q\) le facteur de proportionnalité, c'est-à-dire que nous posons
            \begin{equation}
                | x_i |^p=c^q| y_i |^q.
            \end{equation}
            Dans ce cas, pour chaque \( i\), les nombres \( c| y_i |\) et \( | x_i |\) sont dans le cas d'égalité du corolaire \ref{CORooTCBZooAcZxaC}. Nous avons alors
            \begin{subequations}        \label{SUBEQSooVULLooPGWUIP}
                \begin{align}
                    \sum_i| x_iy_i |&=\frac{1}{ c }\sum_ic| x_i | |y_i |\\
                    &=\frac{1}{ c }\left( \frac{1}{ p }| x_i |^p+\frac{1}{ q }\big( c| y_i | \big)^q \right)\\
                    &=\frac{1}{ c }\sum_i\left( \frac{1}{ p }| x_i |^p+\frac{1}{ q }| x_i |^p \right)\\
                    &=\frac{1}{ c }\sum_i| x_i |^p.
                \end{align}
            \end{subequations}
            Et c'est maintenant que nous subdivisons.
            \begin{subproof}
                \item[Si \( \| x \|_p=\| y \|_q=1\)]
                    Dans ce cas, l'égalité \eqref{SUBEQSooVULLooPGWUIP} se réduisent à
                    \begin{equation}
                        \sum_i| x_iy_i |=\frac{1}{ c }.
                    \end{equation}
                    Mais l'hypothèse sur les normes donne
                    \begin{equation}
                        1=\sum_i| x_i |^p=\sum_ic^q| y_i |^q=c^q\sum_i| y_i |^q=c^q.
                    \end{equation}
                    Donc \( c=1\) et nous avons bien
                    \begin{equation}
                        \sum_i| x_iy_i |=\frac{1}{ c }=1=\| x \|_p\| y \|_q.
                    \end{equation}
                \item[Pour des normes arbitraires]
                    Soit \( (x,y)\in \eR^n\times \eR^n\) tels que \( | x_i |^p=c^q| y_i |^q\). Nous posons comme d'habitude \( x'=x/\| x \|_p\) et \( y'=y/\| y \|_q\). En utilisant le cas «de norme \( 1\)» nous avons
                    \begin{equation}
                        1=\sum_i| x'_iy'_i |=\frac{1}{ \| x \|_p\| y_q \| }\sum_i| x_iy_i |.
                    \end{equation}
                    Donc \( \sum_i| x_iy_i |=\| x \|_p\| y \|_q\) comme nous le voulions.
            \end{subproof}
    \end{subproof}
\end{proof}

La majoration de la proposition suivante sera utile pour les inégalités de Clarkson du lemme \ref{LEMooLTROooVusGte}. Pour d'autres inégalités (plus simples) autour des normes \( \| . \|_p\), voir le thème \ref{THEMEooUJVXooZdlmHj}.
\begin{proposition}[\cite{BIBooVEXYooXCkRQV}]       \label{PROPooQZTNooGACMlQ}
    Si \( x\in \eR^n\) et si \( 0<q<p\), alors
    \begin{equation}
        \| x \|_q\leq n^{\frac{1}{ q }-\frac{1}{ p }}   \| x \|_p.
    \end{equation}
    En particulier, si \( 0 < q < p\), alors
    \begin{equation}
        \| x \|_p\leq \| x \|_p.
    \end{equation}
\end{proposition}

\begin{proof}
    Nous posons \( P=p/q\) et \( Q=P/(P-1)\). Les nombres \( P\) et \( Q\) sont des exposants conjugués, parce que
    \begin{equation}
        \frac{1}{ P }+\frac{1}{ Q }=\frac{ q }{ p }+\frac{ p-q }{ p }=1.
    \end{equation}
    Nous posons \( y=(1,\ldots, 1)\in \eR^n\) ainsi que
    \begin{equation}
        v=\sum_i| x_i |^qe_i,
    \end{equation}
    et nous écrivons l'inégalité de Hölder de la proposition \ref{THOooYHMJooBlXfpl} sur les vecteurs \( v\) et \( y\) :
    \begin{equation}
        \sum_i| v_iy_i |\leq \| v \|_P\| y \|_Q.
    \end{equation}
    En déballant,
    \begin{subequations}
        \begin{align}
            \sum_i| x_i |^q&\leq \left( \sum_i(| x_i |^q)^P \right)^{1/P}\big( \underbrace{\sum_i1^Q}_{=n} \big)^{1/Q}\\
            &=\left( \sum_i | x_i |^p \right)^{q/p}n^{1/Q}\\
            &=n^{1-q/p}\| x \|_p^q.
        \end{align}
    \end{subequations}
    Cela donne
    \begin{equation}
        \| x \|_q^q\leq n^{1-q/p}\| x \|_p^q.
    \end{equation}
    En prenant la puissance \( 1/q\) des deux côtés,
    \begin{equation}
        \| x \|_q\leq   n^{\frac{1}{ q }-\frac{1}{ p }}   \| x \|_p
    \end{equation}
\end{proof}

% This is part of Mes notes de mathématique
% Copyright (c) 2006-2019
%   Laurent Claessens
% See the file fdl-1.3.txt for copying conditions.

%+++++++++++++++++++++++++++++++++++++++++++++++++++++++++++++++++++++++++++++++++++++++++++++++++++++++++++++++++++++++++++
\section{Trucs et astuces de calcul d'intégrales}
%+++++++++++++++++++++++++++++++++++++++++++++++++++++++++++++++++++++++++++++++++++++++++++++++++++++++++++++++++++++++++++
\label{SECooKSOFooEVKDLh}

Afin d'alléger le texte de calculs parfois un peu longs, nous regroupons ici les intégrales à une variable que nous devons utiliser dans les autres parties du cours.

%---------------------------------------------------------------------------------------------------------------------------
\subsection{Quelques intégrales «usuelles»}
%---------------------------------------------------------------------------------------------------------------------------

\begin{enumerate}
	\item	\label{ItemIntegrali}
		L'intégrale
		\begin{equation}
			\boxed{I=\int x\ln(x)dx=\frac{ x^2 }{2}\big( \ln(x)-\frac{ 1 }{2} \big)}
		\end{equation}
		se fait par partie en posant
		\begin{equation}
			\begin{aligned}[]
				u&=\ln(x),		& dv&=x\,dx\\
				du&=\frac{1}{ x }\,dx,	& v&=\frac{ x^2 }{2},
			\end{aligned}
		\end{equation}
		et ensuite
		\begin{equation}
			I=\ln(x)\frac{ x^2 }{2}-\int\frac{ x }{2}=\frac{ x^2 }{2}\big( \ln(x)-\frac{ 1 }{2} \big).
		\end{equation}

	\item
		L'intégrale
		\begin{equation}
			\boxed{I=\int x\ln(x^2)dx=x^2\ln(x)-\frac{ x^2 }{2}.}
		\end{equation}
		En utilisant le fait que $\ln(u^2)=2\ln(u)$, nous retombons sur une intégrale du type~\ref{ItemIntegrali} :
		\begin{equation}
			I=x^2\ln(x)-\frac{ x^2 }{2}.
		\end{equation}
	\item
		L'intégrale
		\begin{equation}		\label{EqTrucIntxlnxsqpun}
			\boxed{I=\int x\ln(1+x^2)dx=\frac{ 1 }{2}\ln(x^2+1)(x^2+1)-x^2-\frac{ 1 }{2}}
		\end{equation}
		se traite en posant $v=1+x^2$ de telle sorte à avoir $dx=\frac{ dv }{ 2x }$ et donc
		\begin{equation}
			I=\frac{ 1 }{2}\ln(x^2+1)(x^2+1)-x^2-\frac{ 1 }{2}.
		\end{equation}

	\item
		L'intégrale
		\begin{equation}
			I=\int \cos(\theta)\sin(\theta)\ln\left( 1+\frac{1}{ \cos^2(\theta) } \right)\,d\theta
		\end{equation}
		demande le changement de variable $u=\cos(\theta)$, $d\theta=-\frac{ du }{ \sin(\theta) }$. Nous tombons sur l'intégrale
		\begin{equation}
			I=-\int u\ln\left( \frac{ 1+u^2 }{ u^2 } \right)=-\int u\ln(1+u^2)+\int u\ln(u^2),
		\end{equation}
		qui sont deux intégrales déjà faites. Nous trouvons
		\begin{equation}
			I=-\frac{ 1 }{2}\ln\left( \frac{ \sin^2(\theta)-1 }{ \sin^2(\theta)-2 } \right)\sin^2(\theta)-\ln\big( \sin^2(\theta)-2 \big)+\frac{ 1 }{2}\ln\big( \sin^2(\theta)-1 \big)
		\end{equation}

	\item
		L'intégrale
		\begin{equation}
			\boxed{\int \frac{ r^3 }{ 1+r^2 }dr=\frac{ r^2 }{2}-\frac{ 1 }{2}\ln(r^2+1).}
		\end{equation}
		commence par faire la division euclidienne de $r^3$ par $r^2+1$; ce que nous trouvons est $r^3=(r^2+1)r-r$. Il reste à intégrer
		\begin{equation}
			\int \frac{ r^3 }{ 1+r^2 }dr=\int r\,dr-\int\frac{ r }{ 1+r^2 }dr.
		\end{equation}
		La fonction dans la seconde intégrale est $\frac{ r }{ 1+r^2 }=\frac{ 1 }{2}\frac{ f'(r) }{ f(r) }$ où $f(r)=1+r^2$, et donc $\int \frac{ r }{ 1+r^2 }=\frac{ 1 }{2}\ln(1+r^2)$. Au final,
		\begin{equation}
			I=\frac{ 1 }{2}r^2-\frac{ 1 }{2}\ln(r^2+1).
		\end{equation}


	\item
		L'intégrale
		\begin{equation}	\label{EqTrucIntsxcxdx}
			\boxed{I=\int \cos(\theta)\sin(\theta)d\theta=\frac{ \sin^2(\theta) }{ 2 }}
		\end{equation}
		se traite par le changement de variable $u=\sin(\theta)$, $du=\cos(\theta)d\theta$, et donc
		\begin{equation}
			\int\cos(\theta)\sin(\theta)d\theta=\int udu=\frac{ u^2 }{2}=\frac{ \sin^2(\theta) }{ 2 }.
		\end{equation}
	\item
		L'intégrale
		\begin{equation}	\label{EqTrucsIntsqrtAplusu}
			\boxed{\int\sqrt{1+x^2}dx=\frac{ x }{2}\sqrt{1+x^2}+\frac{ 1 }{2}\arcsinh(x)}
		\end{equation}
		s'obtient en effectuant le changement de variable $u=\sinh(\xi)$.

    \item
        L'intégrale
        \begin{equation}        \label{EqTrucIntcossqsinsq}
            \boxed{ \int\cos^2(x)\sin^2(x)dx=\frac{ x }{ 8 }-\frac{ \sin(4x) }{ 32 } }
        \end{equation}
        s'obtient à coups de formules de trigonométrie. D'abord, $\sin(t)\cos(t)=\frac{ 1 }{2}\sin^2(2t)$ fait en sorte que la fonction à intégrer devient
        \begin{equation}
            f(x)=\frac{1}{ 4 }\sin^2(x).
        \end{equation}
        Ensuite nous utilisons le fait que $\sin^2(t)=(1-\cos(2t))/2$ pour transformer la formule à intégrer en
        \begin{equation}
            f(x)=\frac{ 1-\cos(4x) }{ 8 }.
        \end{equation}
        Cela s'intègre facilement en posant $u=4x$, et le résultat est
        \begin{equation}
            \int f(x)dx=\frac{ x }{ 8 }-\frac{ \sin(4x) }{ 32 }.
        \end{equation}

    \item

        La fonction
        \begin{equation}
            \sinc(x)=\frac{ \sin(x) }{ x }
        \end{equation}
        est le \defe{sinus cardinal}{sinus cardinal} de \( x\). Nous allons montrer que
        \begin{equation}    \label{EqKNOmLEd}
            \boxed{  \int_0^{\infty}\big| \sinc(x) \big|dx=\infty  }.
        \end{equation}
        D'abord nous avons
        \begin{equation}
            \int_{(n-1)\pi}^{n\pi}\frac{ \big| \sin(t) \big| }{ t }dt\geq \int_{(n-1)\pi}^{n\pi}\frac{ \big| \sin(t) \big| }{ n\pi }dt,
        \end{equation}
        mais par périodicité,
        \begin{equation}
            \int_{(n-1)\pi}^{n\pi}\big| \sin(t) \big|dt=\int_0^{\pi}\sin(t)dt=2.
        \end{equation}
        Par conséquent
        \begin{equation}
            \int_0^{n\pi}\big| \sinc(t) \big|dt\geq \frac{ 2 }{ \pi }\sum_{k=1}^n\frac{1}{ k },
        \end{equation}
        ce qui diverge lorsque \( n\to \infty\).

    \item
        Les intégrales, pour \( \epsilon>0\),
        \begin{equation}        \label{EQooNCVIooWqbbrH}
            \boxed{ \int_0^{\infty}\cos(kx) e^{-\epsilon x}dx=\frac{ \epsilon }{ k^2+\epsilon^2 } }
        \end{equation}
        et
        \begin{equation}        \label{EQooSAYUooSatbGc}
            \boxed{  \int_0^{\infty}\sin(kx) e^{-\epsilon x}dx=\frac{ k }{ k^2+\epsilon^2 }     }
        \end{equation}
        se calculent deux fois par partie. Nous posons
        \begin{subequations}
            \begin{align}
                I&=\int_0^{\infty}\cos(kx) e^{-\epsilon x}dx\\
                J&=\int_0^{\infty}\sin(kx) e^{-\epsilon x}dx.
            \end{align}
        \end{subequations}
        L'intégrale \( I\) s'effectue par partie en posant \( u=\cos(kx)\) et \( v'= e^{-\epsilon x}\). Un peu de calcul montre que
        \begin{equation}
            I=\frac{1}{ \epsilon }-\frac{ k }{ \epsilon }J.
        \end{equation}
        Par ailleurs l'intégrale \( J\) se fait également par partie pour obtenir
        \begin{equation}
            J=\frac{ k }{ \epsilon }I.
        \end{equation}
        En résolvant pour \( I\) et \( J\) les deux équations déduites, nous trouvons
        \begin{subequations}
            \begin{align}
                I&=\frac{ \epsilon }{ k^2+\epsilon^2 }\\
                J&=\frac{ k }{ k^2+\epsilon^2 }.
            \end{align}
        \end{subequations}
\end{enumerate}

%---------------------------------------------------------------------------------------------------------------------------
\subsection{Reformer un carré au dénominateur}
%---------------------------------------------------------------------------------------------------------------------------
\label{subsecCarreDenoPar}

Lorsqu'on a un second degré au dénominateur, le bon plan est de reformer un carré parfait. Par exemple :
\begin{equation}
	x^2+2x+2=(x+1)^2+1.
\end{equation}
Ensuite, le changement de variable $t=x+1$ est pratique parce que cela donne $t^2+1$ au dénominateur.

Cherchons
\begin{equation}
	I=\int \frac{ 1-x }{ x^2+2x+2 }dx=\int\frac{ 1-x }{ (x+1)^2+1 }dx=\int\frac{ 1-(t-1) }{ t^2+1 }
\end{equation}
où nous avons fait le changement de variable $t=x+1$, $dt=dx$. L'intégrale se coupe maintenant en deux parties :
\begin{equation}
	I=\int\frac{ -t }{ t^2+1 }+\int \frac{ 2 }{ t^2+1 }.
\end{equation}
La seconde est dans les formulaires et vaut
\begin{equation}
	2\arctan(t)=2\arctan(x+1),
\end{equation}
tandis que la première est presque de la forme $f'/f$ :
\begin{equation}
	\int\frac{ t }{ t^2+1 }=\frac{ 1 }{2}\int \frac{ 2t }{ t^2+1 }=\frac{ 1 }{2}\ln(t^1+1)=\frac{ 1 }{2}\ln(u^2+2u+2).
\end{equation}

%---------------------------------------------------------------------------------------------------------------------------
\subsection{Décomposition en fractions simples}
%---------------------------------------------------------------------------------------------------------------------------

La décomposition en fractions simples décrite en~\ref{SUBSECooSIYXooDDHUdD} permet d'intégrer des fractions rationnelles. Elle peut parfois être évitée par la méthode de Rothstein-Trager que nous expliquerons dans \ref{subSecBCRYooRVjFpS}.

%+++++++++++++++++++++++++++++++++++++++++++++++++++++++++++++++++++++++++++++++++++++++++++++++++++++++++++++++++++++++++++
\section{Algorithme du gradient à pas optimal}
%+++++++++++++++++++++++++++++++++++++++++++++++++++++++++++++++++++++++++++++++++++++++++++++++++++++++++++++++++++++++++++

Une idée pour trouver un minimum à une fonction est de prendre un point \( p\) au hasard, calculer le gradient \(\nabla f(p) \) et suivre la direction \(-\nabla f(p)\) tant que ça descend. Une fois qu'on est «dans le creux», recalculer le gradient et continuer ainsi.

Nous allons détailler cet algorithme dans un cas très particulier d'une matrice \( A\) symétrique et strictement définie positive.
\begin{itemize}
    \item Dans la proposition~\ref{PROPooYRLDooTwzfWU} nous montrons que résoudre le système linéaire \( Ax=-b\) est équivalent à minimiser une certaine fonction.
    \item La proposition~\ref{PropSOOooGoMOxG} donnera une méthode itérative pour trouver ce minimum.
\end{itemize}

\begin{definition}  \label{DefQXPooYSygGP}
    Si \( X\) est un espace vectoriel normé et \( f\colon X\to \eR\cup\{ \pm\infty \}\) nous disons que \( f\) est \defe{coercive}{coercive} sur le domaine non borné \( P\) de \( X\) si pour tout \( M\in \eR\), l'ensemble
    \begin{equation}
        \{ x\in P\tq f(x)\leq M \}
    \end{equation}
    est borné.
\end{definition}
En langage imagé la coercivité de \( f\) s'exprime par la limite
\begin{equation}
    \lim_{\substack{\| x \|\to \infty\\x\in P}}f(x)=+\infty.
\end{equation}


Nous rappelons que \( S^{++}(n,\eR)\) est l'ensemble des matrices symétriques strictement définies positives définies en~\ref{NORMooAJLHooQhwpvr}.
\begin{proposition}     \label{PROPooYRLDooTwzfWU}
    Soit \( A\in S^{++}(n,\eR)\) et \( b\in \eR^n\). Nous considérons l'application
    \begin{equation}
        \begin{aligned}
            f\colon \eR^n&\to \eR \\
            x&\mapsto \frac{ 1 }{2}\langle Ax, x\rangle +\langle b, x\rangle .
        \end{aligned}
    \end{equation}
    Alors :
    \begin{enumerate}
        \item
            Il existe un unique \( \bar x\in \eR^n\) tel que \( A\bar x=-b\).
        \item
            Il existe un unique \( x^*\in \eR^n\) minimisant \( f\).
        \item
            Ils sont égaux : \( \bar x=x^*\).
    \end{enumerate}
\end{proposition}

\begin{proof}

    Une matrice symétrique strictement définie positive est inversible, entre autres parce qu'elle se diagonalise par des matrices orthogonales (qui sont inversibles) et que la matrice diagonalisée est de déterminant non nul : tous les éléments diagonaux sont strictement positifs. Voir le théorème spectral symétrique~\ref{ThoeTMXla}.

    D'où l'unicité du \( \bar x\) résolvant le système \( Ax=-b\) pour n'importe quel \( b\).

    \begin{subproof}
    \item[\( f\) est strictement convexe]

        La fonction \( f\) s'écrit
    \begin{equation}
        f(x)=\frac{ 1 }{2}\sum_{kl}A_{kl}x_lx_k+\sum_kb_kx_k.
    \end{equation}
    Elle est de classe \( C^2\) sans problèmes, et il est vite vu que \( \frac{ \partial^2f }{ \partial x_i\partial x_j }=A_{ij}\), c'est-à-dire que \( A\) est la matrice hessienne de \( f\). Cette matrice étant strictement définie positive par hypothèse, la fonction \( f\) est strictement convexe par le corolaire~\ref{CORooMBQMooWBAIIH}\ref{ITEMooDGISooPlRLOd}.

\item[\( f\) est coercive]
    Montrons à présent que \( f\) est coercive. Nous avons :
    \begin{subequations}
        \begin{align}
            | f(x) |&=\big| \frac{ 1 }{2}\langle Ax, x\rangle +\langle b, x\rangle  \big|\\
            &\geq\frac{ 1 }{2}| \langle Ax, x\rangle  |-| \langle b, x\rangle  |\\
            &\geq\frac{ 1 }{2}\lambda_{max}\| x \|^2-\| b \|\| x \|
        \end{align}
    \end{subequations}
    Pour la dernière ligne nous avons nommé \( \lambda_{max}\) la plus grande valeur propre de \( A\) et utilisé Cauchy-Schwarz pour le second terme. Nous avons donc bien \( | f(x) |\to \infty\) lorsque \( \| x \|\to\infty\) et la fonction \( f\) est coercive.
    \end{subproof}

    Soit \( M\) une valeur atteinte par \( f\). L'ensemble
    \begin{equation}
        \{ x\in \eR^n\tq f(x)\leq M \}
    \end{equation}
    est fermé (parce que \( f\) est continue) et borné parce que \( f\) est coercive. Cela est donc compact\footnote{Théorème~\ref{ThoXTEooxFmdI}} et \( f\) atteint un minimum qui sera forcément dedans. Cela est pour l'existence d'un minimum.

    Pour l'unicité du minimum nous invoquons la convexité : si \( \bar x_1\) et \( \bar x_2\) sont deux points réalisant le minimum de \( f\), alors
    \begin{equation}
        f\left( \frac{ \bar x_1+\bar x_2 }{2} \right)<\frac{ 1 }{2}f(\bar x_1)+\frac{ 1 }{2}f(\bar x_2)=f(\bar x_1),
    \end{equation}
    ce qui contredit la minimalité de \( f(\bar x_1)\).

    Nous devons maintenant prouver que \( \bar x\) vérifie l'équation \( A\bar x=-b\). Vu que \( \bar x\) est minimum local de \( f\) qui est une fonction de classe \( C^2\), le théorème des minimums locaux~\ref{PropUQRooPgJsuz} nous indique que \( \bar x\) est solution de \( \nabla f(x)=0\). Calculons un peu cela avec la formule
    \begin{equation}
        df_x(u)=\Dsdd{ f(x+tu) }{t}{0}=\frac{ 1 }{2}\big( \langle Ax, u\rangle +\langle Au, x\rangle  \big)+\langle b, u\rangle =\langle Ax, u\rangle +\langle b, u\rangle =\langle Ax+b, u\rangle .
    \end{equation}
    Donc demander \( df_x(u)=0\) pour tout \( u\) demande \( Ax+b=0\).
\end{proof}

\begin{proposition}[Gradient à pas optimal] \label{PropSOOooGoMOxG}
    Soit \( A\in S^{++}(n,\eR)\) (\( A\) est une matrice symétrique strictement définie positive) et \( b\in \eR^n\). Nous considérons l'application
    \begin{equation}
        \begin{aligned}
            f\colon \eR^n&\to \eR \\
            x&\mapsto \frac{ 1 }{2}\langle Ax, x\rangle +\langle b, x\rangle .
        \end{aligned}
    \end{equation}
    Soit \( x_0\in \eR^n\). Nous définissons la suite \( (x_k)\) par
    \begin{equation}
        x_{k+1}=x_k+t_kd_k
    \end{equation}
    où
    \begin{itemize}
        \item
    \( d_k=-(\nabla f)(x_k)\)
\item
    \( t_k\) est la valeur minimisant la fonction \( t\mapsto f(x_k+td_k)\) sur \( \eR\).
    \end{itemize}

    Alors pour tout \( k\geq 0\) nous avons
    \begin{equation}
        \| x_k-\bar x \|\leq K \left( \frac{ c_2(A)-1 }{ c_2(A)+1 } \right)^k
    \end{equation}
    où \( c_2(A)=\frac{ \lambda_{max} }{ \lambda_{min} }\) est le rapport ente la plus grande et la plus petite valeur propre\quext{Cela est certainement très lié au conditionnement de la matrice \( A\), voir la proposition~\ref{PROPooNUAUooIbVgcN}.} de la matrice \( A\) et \( \bar x\) est l'unique élément de \( \eR^n\) à minimiser \( f\).
\end{proposition}

\begin{proof}
    Décomposition en plusieurs points.
    \begin{subproof}
    \item[Existence de \( \bar x\)]
        Le fait que \( \bar x\) existe et soit unique est la proposition~\ref{PROPooYRLDooTwzfWU}.
    \item[Si \( (\nabla f)(x_k)=0\)]
    D'abord si \( \nabla f(x_k)=0\), c'est que \( x_{k+1}=x_k\) et l'algorithme est terminé : la suite est stationnaire. Pour dire que c'est gagné, nous devons prouver que \( x_k=\bar x\). Pour cela nous écrivons (à partir de maintenant «\( x_k\)» est la \( k\)\ieme composante de \( x\) qui est une variable, et non le \( x_k\) de la suite)
    \begin{equation}
        f(x)=\frac{ 1 }{2}\sum_{kl}A_{kl}x_lx_k+\sum_{k}b_kx_k
    \end{equation}
    et nous calculons \( \frac{ \partial f }{ \partial x_i }(a)\) en tenant compte du fait que \( \frac{ \partial x_k }{ \partial x_i }=\delta_{ki}\). Le résultat est que \( (\partial_if)(a)=(Ax+b)_i\) et donc que
    \begin{equation}
        (\nabla f)(a)=Aa+b.
    \end{equation}
    Vu que \( A\) est inversible (symétrique définie positive), il existe un unique \( a\in \eR^n\) qui vérifie cette relation. Par la proposition~\ref{PROPooYRLDooTwzfWU}, cet élément est le minimum \( \bar x\).

    Cela pour dire que si \( a\in \eR^n\) vérifie \( (\nabla f)(a)=0\) alors \( a=\bar x\). Nous supposons donc à partir de maintenant que \( \nabla f(x_k)\neq 0\) pour tout \( k\).
        \item[\( t_k\) est bien défini]

            Pour \( t\in \eR\) nous avons
            \begin{equation}    \label{EqKEHooYaazQi}
                f(x_k+td_k)=f(x_k)+\frac{ 1 }{2}t^2\langle Ad_k, d_k\rangle +t\langle \underbrace{Ax_k+b}_{=-d_k}, d_k\rangle=\frac{ 1 }{2}t^2\langle Ad_k, d_k\rangle -t_k\| d_k \|^2 +f(x_k).
            \end{equation}
            qui est un polynôme du second degré en \( t\). Le coefficient de \( t^2\) est \( \frac{ 1 }{2}\langle Ad_k, d_k\rangle >0\) parce que \( d_k\neq 0\) et \( A\) est strictement définie positive. Par conséquent la fonction \( t\mapsto f(x_k+td_k)\) admet bien un unique minimum. Nous pouvons même calculer \( t_k\) parce que l'on connaît pas cœur le sommet d'une parabole :
            \begin{equation}    \label{EqVWJooWmDSER}
                t_k=-\frac{ \langle Ax_k+b, d_k\rangle  }{ \langle Ad_k, d_k\rangle  }=\frac{ \| d_k \|^2 }{ \langle Ad_k, d_k\rangle  }
            \end{equation}
            parce que \( d_k=-\nabla f(x_k)=-(Ax_k+b)\).

        \item[La valeur de \( d_{k+1}\)]

            Par définition, \( d_{k+1}=-\nabla f(x_{k+1})=-(Ax_{k+1}+b)\). Mais \( x_{k+1}=x_k+t_kd_k\), donc
            \begin{equation}
                d_{k+1}=-Ax_k-t_kAd_k-b=d_k-t_kAd_k
            \end{equation}
            parce que \( -Ax_k-b=d_k\).

            Par ailleurs, \( \langle d_{k+1}, d_k\rangle =0\) parce que
            \begin{equation}
                \langle d_{k+1}, d_k\rangle =\langle d_k, d_k\rangle -t_k\langle d_k, Ad_k\rangle =\| d_k \|^2-\frac{ \| d_k \|^2 }{ \langle Ad_k, d_k\rangle  }\langle d_k, Ad_k\rangle =0
            \end{equation}
            où nous avons utilisé la valeur \eqref{EqVWJooWmDSER} de \( t_k\).

        \item[Calcul de \( f(x_{k+1})\)]

            Nous repartons de \eqref{EqKEHooYaazQi} où nous substituons la valeur \eqref{EqVWJooWmDSER} de \( t_k\) :
            \begin{equation}
                f(x_{k+1})=f(x_k)+\frac{ 1 }{2}\frac{ \| d_k \|^4 }{ \langle Ad_k, d_k\rangle  }-\frac{ \| d_k \|^4 }{ \langle Ad_k, d_k\rangle  }=f(x_k)-\frac{ 1 }{2}\frac{ \| d_k \|^4 }{ \langle Ad_k, d_k\rangle  }.
            \end{equation}

        \item[Encore du calcul \ldots]

            Vu que le produit \( \langle Ad_k, d_k\rangle \) arrive tout le temps, nous allons étudier \( \langle A^{-1}d_k, d_k\rangle \). Le truc malin est d'essayer d'exprimer ça en termes de \( \bar x\) et \( \bar f=f(\bar x)\). Pour cela nous calculons \( f(\bar x)\) :
            \begin{equation}
                \bar f=f(\bar x)=f(-A^{-1} b)=-\frac{ 1 }{2}\langle b, A^{-1}b\rangle .
            \end{equation}
            Ayant cela en tête nous pouvons calculer :
            \begin{subequations}
                \begin{align}
                    \langle A^{-1}d_k, d_k\rangle &=\langle A^{-1}(Ax_k+b), Ax_k+b\rangle \\
                    &=\langle x_k, Ax_k\rangle +\langle A^{-1}b, Ax_k\rangle +\langle b, x_k\rangle+\underbrace{\langle A^{-1}b, b\rangle}_{-2\bar f} \\
                    &=\langle x_k, Ax_k\rangle +2\langle x_k, b\rangle  -2\bar f \label{subeqVIIooVzZlRc}\\
                    &=2\big( f(x_k)-\bar f \big)
                \end{align}
            \end{subequations}
            où nous avons utilisé le fait que \( \langle x, Ay\rangle =\langle Ax, y\rangle \) parce que \( A\) est symétrique.

        \item[Erreur sur la valeur du minimum]

            Nous voulons à présent estimer la différence \( f(x_{k+1})-\bar f\). Pour cela nous mettons en facteur \( f(x_k)-\bar f\) dans \( f(x_{k+1}-\bar f)\); et d'ailleurs c'est pour cela que nous avons calculé \( \langle A^{-1}d_k, d_k\rangle \) : parce que ça fait intervenir \( f(x_k)-\bar f\).
            \begin{subequations}
                \begin{align}
                    f(x_{k+1})-\bar f&=f(x_k)-\frac{ 1 }{2}\frac{ \| d_k \|^4 }{ \langle Ad_k, d_k\rangle  }-\bar f\\
                    &=\big( f(x_k)-\bar f \big)\left( 1-\frac{ 1 }{2}\frac{ \| d_k \|^{4} }{ \langle Ad_k, d_k\rangle \big( f(x_k)-\bar f \big) } \right)\\
                    &=\big( f(x_k)-\bar f \big)\left( 1-\frac{ \| d_k \|^{4} }{ \langle Ad_k, d_k\rangle \langle A^{-1}d_k, d_k\rangle  } \right).\label{subeqGFDooRAwAJk}
                \end{align}
            \end{subequations}
            Nous traitons le dénominateur à l'aide de l'inégalité de Kantorovitch~\ref{PropMNUooFbYkug}. Nous avons
            \begin{equation}
                \frac{ \| d_k \|^4 }{ \langle Ad_k, d_k\rangle \langle A^{-1}d_k, d_k\rangle  }\geq \frac{ \| d_k \|^4 }{ \frac{1}{ 4 }\left( \sqrt{c_2(A)}+\frac{1}{ \sqrt{c_2(A)} } \right)^2\| d_k \|^4 }=\frac{ 4c_2(A) }{ (c_2(A)+1)^2 }.
            \end{equation}
            Mettre cela dans \eqref{subeqGFDooRAwAJk} est un calcul d'addition de fractions :
            \begin{equation}
                f(x_{k+1})-\bar f\leq \big( f(x_k)-\bar f \big)\left( \frac{ c_2(A)-1 }{ c_2(A)+1 } \right)^2.
            \end{equation}
            Par récurrence nous avons alors
            \begin{equation}    \label{eqANKooNPfCFj}
                f(x_k)-\bar f\leq \big( f(x_0)-\bar f \big)\left( \frac{ c_2(A)-1 }{ c_2(A)+1 } \right)^{2k}.
            \end{equation}
            Notons qu'il n'y a pas de valeurs absolues parce que \( \bar f\) étant le minimum de \( f\), les deux côtés de l'inégalité sont automatiquement positifs.

        \item[Erreur sur la position du minimum]

            Nous voulons à présent étudier la norme de \( x_k-\bar x\). Pour cela nous l'écrivons directement avec la définition de \( f\) en nous souvenant que \( b=-A\bar x\) :
            \begin{subequations}
                \begin{align}
                    f(x_k)-\bar f&=\frac{ 1 }{2}\langle Ax_k, x_k\rangle +\langle A\bar x, x_k\rangle +\frac{ 1 }{2}\langle A\bar x, \bar x\rangle +\langle A\bar x, \bar x\rangle \\
                    &=\frac{ 1 }{2}\langle Ax_k, x_k\rangle -\langle A\bar x, x_k\rangle +\frac{ 1 }{2}\langle A\bar x, \bar x\rangle \\
                    &=\frac{ 1 }{2}\langle Ax_k, x_k\rangle -\frac{ 1 }{2}\langle A\bar x, x_k\rangle-\frac{ 1 }{2}\langle A\bar x, x_k\rangle +\frac{ 1 }{2}\langle A\bar x, \bar x\rangle \\
                    &=\frac{ 1 }{2}\Big( \langle A(x_k-\bar x), x_k\rangle +\langle A\bar x, \bar x-x_k\rangle  \Big)\\
                    &=\frac{ 1 }{2}\Big( \langle A(x_k-\bar x), (x_k-\bar x)\rangle  \Big)
                \end{align}
            \end{subequations}
            où à la dernière ligne nous avons fait \( \langle A\bar x, \bar x-x_k\rangle =\langle \bar x, A(\bar x-x_k)\rangle \) en vertu de la symétrie de \( A\).

            Les produits de la forme \( \langle Ay, y\rangle \) sont majorés par \( \lambda_{min}\| y \|^2\) parce que \( \lambda_{min}\) est la plus grande valeur propre de \( A\). Dans notre cas,
            \begin{equation}    \label{EqVMRooUMXjig}
                f(x_k)-\bar f\geq \frac{ 1 }{2}\lambda_{min}\| x_k-\bar x \|^2
            \end{equation}

        \item[Conclusion]

            En combinant les inéquations \eqref{EqVMRooUMXjig} et \eqref{eqANKooNPfCFj} nous trouvons
            \begin{equation}
                \frac{ 1 }{2}\lambda_{min}\| x_k-\bar x \|^2\leq f(x_k)-\bar f\leq \big( f(x_0)-\bar f \big)\left( \frac{ c_2(A)-1 }{ c_2(A)+1 } \right)^{2k},
            \end{equation}
            c'est-à-dire
            \begin{equation}
                \| x_k-\bar x \|\leq \sqrt{\frac{ 2\big( f(x_0)-\bar f \big) }{ \lambda_{min} +1}}^{2k}.
            \end{equation}
    \end{subproof}
\end{proof}

Notons que lorsque \( c_2(A)\) est proche de \( 1\) la méthode converge rapidement. Par contre si \( c_2(A)\) est proche de zéro, la méthode converge lentement.

%+++++++++++++++++++++++++++++++++++++++++++++++++++++++++++++++++++++++++++++++++++++++++++++++++++++++++++++++++++++++++++
\section{Formes quadratiques, signature, et lemme de Morse}
%+++++++++++++++++++++++++++++++++++++++++++++++++++++++++++++++++++++++++++++++++++++++++++++++++++++++++++++++++++++++++++

\begin{normaltext}      \label{NORMooHSWKooLtUbRl}
    Soit \( (E,\| . \|_E)\) un espace vectoriel réel normé de dimension finie \( n\). L'ensemble des formes quadratiques réelles\footnote{Définition~\ref{DefBSIoouvuKR}.} sur \( E\) est vu comme l'ensemble des matrices symétriques \( S_n(\eR)\); il sera noté \( Q(E)\) et le sous-ensemble des formes quadratiques non dégénérées est \( S_n(\eR)\cap\GL(n,\eR)\) qui sera noté \( \Omega(E)\)\nomenclature[B]{\( \Omega(E)\)}{formes quadratiques non dégénérées}. Nous rappelons que la correspondance est donnée de la façon suivante. 
    
    Si \( A\in S_n(\eR)\), la forme quadratique associée est \( q_A\) donnée par le produit scalaire \( q_A(x)=x\cdot Ax\).

    Pour information, le lemme de Morse est le lemme \ref{LemNQAmCLo}.
\end{normaltext}

\begin{normaltext}      \label{NORMooQZFLooYnILtn}
Nous noterons encore \( Q^+(E)\)\nomenclature[B]{\( Q^+(E)\)}{formes quadratiques positives} les formes quadratiques positives sur \( E\) et \( Q^{++}(E)\)\nomenclature[B]{\( Q^{++}(E)\)}{formes quadratiques strictement définies positives} les formes quadratiques strictement définies positives sur \( E\).
\end{normaltext}

Sur \( Q(E)\) nous mettons la norme
\begin{equation}
    N(q)=\sup_{\| x \|_E=1}| q(x) |,
\end{equation}
qui du point de vue de \( S_n(\eR)\) est
\begin{equation}    \label{EqDOgBNAg}
    N(A)=\sup_{\| x \|_E=1}| x^tAx |.
\end{equation}
Notons que à droite, c'est la valeur absolue usuelle sur \( \eR\).

Nous savons par le théorème de Sylvester (théorème~\ref{ThoQFVsBCk}) que dans \( \eM(n,\eR)\), toute matrice symétrique de signature \( (p,q)\) est semblable à la matrice
\begin{equation}
    \mtu_{p,q}=\begin{pmatrix}
        \mtu_p    &       &       \\
        &   \mtu_{p}    &       \\
        &       &   0_{n-p-q}
    \end{pmatrix}.
\end{equation}
Donc deux matrices de \( S_n\) sont semblables si et seulement si elles ont la même signature (même si elles ne sont pas de rang maximum, cela soit dit au passage). Si nous notons \( S_n^{p,q}(\eR)\)\nomenclature[B]{\( S_n^{p,q}(\eR)\)}{matrices symétriques réelles de signature \( (p,q)\)} l'ensemble des matrices réelles symétriques de signature \( (p,q)\), alors
\begin{equation}
    S_n^{p,q}(\eR)=\{ P^tAP\tq P\in \GL(n,\eR) \}
\end{equation}
où \( A\) est une quelconque ce ces matrices.

Nous voudrions en savoir plus sur ces ensembles. En particulier nous aimerions savoir si la signature est une notion «stable» au sens où ces ensembles seraient ouverts dans \( S_n\). Pour cela nous considérons l'action de \( \GL(n,\eR)\) sur \( S_n\) définie par
\begin{equation}
    \begin{aligned}
        \alpha\colon \GL(n,\eR)\times S_n(\eR)&\to S_n(\eR) \\
        (P,A)&\mapsto P^tAP
    \end{aligned}
\end{equation}
faite exprès pour que les orbites de cette action soient les ensembles \( S_n^{p,q}(\eR)\).

La proposition suivante montre que lorsque \( p+q=n\), c'est-à-dire lorsqu'on parle de matrices de rang maximum, les ensembles \( S_n^{p,q}(\eR)\) sont ouverts, c'est-à-dire que la signature d'une forme quadratique est une propriété «stable» par petite variations des éléments de matrice. Notons tout de suite que si le rang n'est pas maximum, le théorème de Sylvester dit qu'elle est semblable à une matrice diagonale avec des zéros sur la diagonale; en modifiant un peu ces zéros, on peut modifier évidemment la signature.
\begin{proposition}[\cite{KXjFWKA}] \label{PropNPbnsMd}
    Soit \( (E,\| . \|_{E})\) un espace vectoriel normé de dimension finie. Alors
    \begin{enumerate}
        \item
            les formes quadratiques non dégénérées forment un ouvert dans l'ensemble des formes quadratiques,
        \item
            les ensembles \( S_n^{p,q}(\eR)\) avec \( p+q=n\) sont ouverts dans \( S_n(\eR)\),
        \item   \label{ItemGOhRIiViii}
            les composantes connexes de \( \Omega(E)\) sont les \( S_n^{p,q}(\eR)\) avec \( p+q=n\),
        \item   \label{ItemGOhRIiViv}
            les \( S_n^{p,q}(\eR)\) non dégénérés sont connexes par arc.
    \end{enumerate}
\end{proposition}
\index{connexité!signature d'une forme quadratique}
\index{matrice!symétrique!réelle}
\index{forme!quadratique}

\begin{proof}
    Cette preuve est donnée du point de vue des matrices. La différence entre le point~\ref{ItemGOhRIiViii} et~\ref{ItemGOhRIiViv} est que dans le premier nous prouvons la connexité de \( S_n^{p,q}(\eR)\) à partir de la connexité de \( \GL^+(n,\eR)\), tandis que dans le second nous prouvons la connexité par arc de \( S_n^{p,q}(\eR)\) à partir de la connexité par arc de \( \GL^+(n,\eR)\). Bien entendu le second implique le premier.
    \begin{enumerate}
        \item
            Il s'agit simplement de remarquer que \( Q(E)=S_n(\eR)\), que \( \Omega(E)=S_n(\eR)\cap\GL(n,\eR)\) et que le déterminant est une fonction continue sur \( \eM(n,\eR)\).
        \item
            Soit \( A_0\in S_n^{p,q}(\eR)\). Le théorème de Sylvester~\ref{ThoQFVsBCk} nous donne une matrice inversible \( P\) telle que \( P^tA_0P=\mtu_{p,q}\). Nous allons montrer qu'il existe un voisinage \( \mU\) de \( \mtu_{p,q}\) contenu dans \( S_n^{p,q}(\eR)\). À partir de là, l'ensemble \( (P^{-1})^t\mU P^{-1}\) sera un voisinage de \( A_0\) contenu dans \( S_n^{p,q}(\eR)\).

            Nous considérons les espaces vectoriels
            \begin{subequations}
                \begin{align}
                    F&=\Span\{ e_1,\ldots, e_p \}\\
                    G&=\Span\{ e_{p+1},\ldots, e_n \}
                \end{align}
            \end{subequations}
            La norme euclidienne \( \| . \|_p\) sur \( F\) est équivalente à la norme \( | . |_E\) par le théorème~\ref{ThoNormesEquiv}. Donc il existe une constante \( k_1>0\) telle que pour tout \( x\in F\),
            \begin{equation}    \label{EqMViCjJJ}
                \| x \|_p\geq k_1\| x \|_E.
            \end{equation}
            De la même façon sur \( G\), il existe une constante \( k_2>0\) telle que
            \begin{equation}    \label{EqSFwOcDw}
                \| x \|_q\geq k_2\| x \|_E.
            \end{equation}
            Si nous posons \( k=\min\{ k_1^2,k_2^2 \}\), alors nous avons
            \begin{subequations}
                \begin{align}
                    \forall x\in F,\quad &\| x \|_p^2\geq k_1^2\| x \|_E^2\geq k\| x \|_E^2\\
                    \forall x\in G,\quad &\| x \|_q^2\geq k_2^2\| x \|_E^2\geq k\| x \|_E^2.
                \end{align}
            \end{subequations}

            Soit une matrice \( A\in S_n(\eR)\) telle que \( N(A-\mtu_{p,q})<k\), c'est-à-dire que \( A\) est dans un voisinage de \( \mtu_{p,q}\) pour la norme sur \( S_n(\eR)\) donné par \eqref{EqDOgBNAg}. Si \( x\) est non nul dans \( E\), nous avons
            \begin{equation}
                \big| x^t(A-\mtu_{p,q})x \big|\leq N(\mtu_{p,q}-A)\| x \|^2\leq k\| x \|^2.
            \end{equation}
            En déballant la valeur absolue, cela signifie que
            \begin{equation}
                -k\| x \|_E^2\leq x^t(A-\mtu_{p,q})x\leq k\| x \|^2.
            \end{equation}
            Si \( x\in F\), alors la première inéquation et \eqref{EqMViCjJJ} donnent
            \begin{equation}
                x^tAx\geq \| x \|_p^2-k\| x \|_E^2>0
            \end{equation}
            Si \( x\in G\), alors la seconde inéquation et \eqref{EqSFwOcDw} donnent
            \begin{equation}
                x^tAx\leq  k\| x \|_E^2-\| x \|_q^2<0.
            \end{equation}

            Nous avons donc montré que \( x\mapsto x^tAx\) est positive sur \( F\) et négative sur \( G\), ce qui prouve que \( A\) est bien de signature \( (p,q)\) et appartient donc à \( S_n^{p,q}(\eR)\). Autrement dit nous avons
            \begin{equation}
                B(\mtu_{p,q},k)\subset S_n^{p,q}(\eR).
            \end{equation}

        \item
            Cette partie de la preuve provient essentiellement de \cite{VKqpMYL}, et fonctionne pour tous les \( S_n^{p,q}(\eR)\), même pour ceux qui ne sont pas de rang maximum.

            Soit \( A\in S_n^{p,q}(\eR)\). Nous savons que \( \GL(n,\eR)\) a deux composantes connexes (proposition~\ref{PROPooBIYQooWLndSW}). Vu que l'application
            \begin{equation}
                \begin{aligned}
                    \alpha\colon \GL(n,\eR)&\to S_n \\
                    P&\mapsto P^tAP
                \end{aligned}
            \end{equation}
            est continue, l'image d'un connexe de \( \GL(n,\eR)\) par \( \alpha\) est connexe (proposition~\ref{PropGWMVzqb}). En particulier, \( \alpha\big( \GL^{\pm}(n,\eR) \big)\) sont deux connexes et nous savons que \( S_n^{p,q}(\eR)\) a au plus ces deux composantes connexes.

            Notre but est maintenant de trouver une intersection entre les parties \( \alpha\big( \GL^+(n,\eR) \big)\) et \( \alpha\big( \GL^-(n,\eR) \big)\)\quext{À ce point, il me semble que \cite{VKqpMYL} fait erreur parce que la matrice \( -\mtu_n\) est de déterminant \( 1\) lorsque \( n\) est pair. L'argument donné ici provient de \cite{KXjFWKA}}. Soit par le théorème de Sylvester, soit par le théorème de diagonalisation des matrices symétriques réelles~\ref{ThoeTMXla}, il existe une matrice \( P\in \GL(n,\eR)\) diagonalisant \( A\). En suivant la remarque~\ref{RemGKDZfxu}, et en notant \( Q\) la matrice obtenue à partir de \( P\) en changeant le signe de sa première ligne, nous avons
            \begin{equation}
                \alpha(Q)=Q^tAQ=P^tAP=\alpha(P).
            \end{equation}
            Or si \( P\in \GL^+(n,\eR)\), alors \( Q\in \GL^-(n,\eR)\) et inversement. Donc nous avons trouvé une intersection entre \( \alpha\big( \GL^+(n,\eR) \big)\) et \( \alpha\big( \GL^-(n,\eR) \big)\).

        \item

            Soient \( A\) et \( B\) dans \( S_n^{p,q}(\eR)\cap\GL(n,\eR)\). Par le théorème de Sylvester, il existe \( P\) et \( Q\) dans \( \GL(n,\eR)\) telles que \( A=P^t\mtu_{p,q}P\) et \( B=Q^t\mtu_{p,q}Q\). Par la remarque~\ref{RemGKDZfxu} nous pouvons choisir \( P\) et \( Q\) dans \( \GL^+(n,\eR)\). Ce dernier groupe étant connexe par arc, il existe un chemin
            \begin{equation}
                    \gamma\colon \mathopen[ 0 , 1 \mathclose]\to \GL^+(n,\eR)
            \end{equation}
            tel que \( \gamma(0)=P\) et \( \gamma(1)=Q\). Alors le chemin
            \begin{equation}
                s\mapsto \gamma(s)^t\mtu_{p,q}\gamma(s)
            \end{equation}
            est un chemin continu dans \( S_n^{p,q}(\eR)\) joignant \( A\) à \( B\).
    \end{enumerate}
\end{proof}
% TODO : prouver la connexité par arc de GL^+(n,\eR) et mettre une référence ici.

Nous savons déjà de la proposition~\ref{PropNPbnsMd} que les ensembles \( S_n^{p,q}(\eR)\) (pas spécialement de rang maximum) sont ouverts dans \( S_n(\eR)\). Le lemme suivant nous donne une précision à ce sujet, dans le cas des matrices de rang maximum, en disant que la matrice qui donne la similitude entre \( A_0\) et \( A\) est localement un \( C^1\)-difféomorphisme de \( A\).
\begin{lemma}   \label{LemWLCvLXe}
    Soit \( A_0\in \Omega(\eR^n)= S_n\cap\GL(n,\eR)\), une matrice symétrique inversible. Alors il existe un voisinage \( V\) de \( A_0\) dans \( S_n\) et une application \( \phi\colon V\to \GL(n,\eR)\) qui
    \begin{enumerate}
        \item
            est de classe \( C^1\),
        \item
            est telle que pour tout \( A\in V\), \( \varphi(A)^t A_0\phi(A)=A\).
    \end{enumerate}
\end{lemma}
\index{groupe!\( \GL(n,\eR)\)}
\index{forme!quadratique}
\index{matrice!symétrique}
\index{matrice!semblables}

\begin{proof}
    Nous considérons l'application
    \begin{equation}
        \begin{aligned}
            \varphi\colon \eM(n,\eR)&\to S_n \\
            M&\mapsto M^tA_0M.
        \end{aligned}
    \end{equation}
    Étant donné que les composantes de \( \varphi(M)\) sont des polynômes en les entrées de \( M\), cette application est de classe \( C^1\) -- et même plus. Soit maintenant \( H\in \eM(n,\eR)\) et calculons \( d\varphi_{\mtu}(H)\) par la formule \eqref{EqOWQSoMA} :
    \begin{subequations}
        \begin{align}
            d\varphi_{\mtu}(H)&=\Dsdd{ \varphi(\mtu+tH) }{t}{0}\\
            &=\Dsdd{ (\mtu+tH^t)A_0(\mtu+tH) }{t}{0}\\
            &=\Dsdd{ A_0+tA_0H+tH^tA_0+t^2H^tA_0H }{t}{0}\\
            &=A_0H+H^tA_0.
        \end{align}
    \end{subequations}
    Donc
    \begin{equation}
        d\varphi_{\mtu}(H)=(A_0H)+(A_0H)^t.
    \end{equation}
    Par conséquent
    \begin{equation}
        \ker(d\varphi_{\mtu})=\{ H\in \eM(n,\eR)\tq A_0H\text{ est antisymétrique} \},
    \end{equation}
    et si nous posons
    \begin{equation}
        F=\{ H\in \eM(n,\eR)\tq A_0H\text{ est symétrique} \}
    \end{equation}
    nous avons
    \begin{equation}
        \eM(n,\eR)=F\oplus\ker(d\varphi_{\mtu})
    \end{equation}
    parce que toute matrice peur être décomposée de façon unique en partir symétrique et antisymétrique. De plus l'application
    \begin{equation}    \label{EqGTBusDm}
        \begin{aligned}
            f\colon F&\to S_n \\
            H&\mapsto A_0H
        \end{aligned}
    \end{equation}
    est une bijection linéaire. D'abord \( A_0H=0\) implique \( H=0\) parce que \( A_0\) est inversible, et ensuite si \( X\in S_n\), alors \( X=A_0A_0^{-1}X\), ce qui prouve que \( X\) est l'image par \( f\) de \( A_0^{-1}X\) et donc que \( f\) est surjective.

    Maintenant nous considérons la restriction \( \psi=\varphi_{|_F}\), \( \psi\colon F\to S_n\). Remarquons que \( \mtu\in F\) parce que \( A_0\in S_n\). L'application \( d\psi_{\mtu}\) est une bijection. En effet d'abord
    \begin{equation}
        d(\varphi_{|_F})_{\mtu}=(d\varphi_{\mtu})_{|_F},
    \end{equation}
    ce qui prouve que
    \begin{equation}
        \ker(d\psi_{\mtu})=\ker(d\varphi_{\mtu})\cap F=\{ 0 \},
    \end{equation}
    ce qui prouve que \( d\psi_{\mtu}\) est injective. Pour montrer que \( d\psi_{\mtu}\) est surjective, il suffit de mentionner le fait que \( \dim F=\dim S_n\) du fait que l'application \eqref{EqGTBusDm} est une bijection linéaire.

    Nous pouvons utiliser le théorème d'inversion locale (théorème~\ref{ThoXWpzqCn}) et conclure qu'il existe un voisinage ouvert \( U\) de \( \mtu\) dans \( F\) tel que \( \psi\) soit un difféomorphisme \( C^1\) entre \( U\) et \( V=\psi(U)\). Vu que \( \GL(n,\eR)\) est ouvert dans \( \eM(n,\eR)\), nous pouvons prendre \( U\cap \GL(n,\eR)\) et donc supposer que \( U\subset \GL(n,\eR)\).

    Pour tout \( A\in V\), il existe une unique \( M\in U\) telle que \( \psi(M)=A\), c'est-à-dire telle que \( A=M^tA_0M\). Cette matrice \( M\) est \( \psi^{-1}(A)\) et est une matrice inversible. Bref, nous posons
    \begin{equation}
        \begin{aligned}
            \phi\colon V&\to \GL(n,\eR) \\
            A&\mapsto \psi^{-1}(A),
        \end{aligned}
    \end{equation}
    et ce \( \phi\) est de classe \( C^1\) sur \( V\) parce que c'est ce que dit le théorème d'inversion locale. Cette application répond à la question parce que \( V\) est un voisinage de \( \varphi(\mtu)=A_0\) et pour tout \( A\in V\) nous avons
    \begin{equation}
        \phi(A)^tA_0\phi(A)=\varphi^{-1}(A)^tA_0\varphi^{-1}(A)=A.
    \end{equation}
\end{proof}

%+++++++++++++++++++++++++++++++++++++++++++++++++++++++++++++++++++++++++++++++++++++++++++++++++++++++++++++++++++++++++++
\section{Ellipsoïde de John-Loewner}
%+++++++++++++++++++++++++++++++++++++++++++++++++++++++++++++++++++++++++++++++++++++++++++++++++++++++++++++++++++++++++++

C'est le moment de relire les conventions de notations données en \ref{NORMooHSWKooLtUbRl} et \ref{NORMooQZFLooYnILtn} ainsi que la définition \ref{DefBSIoouvuKR} d'une forme quadratique.

Soit \( q\) une forme quadratique sur \( \eR^n\) ainsi que \( \mB\) une base orthonormée de \( \eR^n\) dans laquelle la matrice de  \( q\) est diagonale. Dans cette base, la forme \( q\) est donnée par la proposition~\ref{PropFWYooQXfcVY} :
\begin{equation}
    q(x)=\sum_i\lambda_ix_i
\end{equation}
où les \( \lambda_i\) sont les valeurs propres de \( q\).

Plus généralement nous notons \( mat_{\mB}(q)\)\nomenclature[A]{\( mat_{\mB}(q)\)}{matrice de \( q\) dans la base \( \mB\)} la matrice de \( q\) dans la base \( \mB\) de \( \eR^n\).

\begin{proposition} \label{PropOXWooYrDKpw}
    Soit \( \mB\) une base orthonormée de \( \eR^n\) et l'application\footnote{L'ensemble \( Q(E)\) est l'ensemble des formes quadratiques sur \( E\).}
    \begin{equation}
        \begin{aligned}
            D\colon Q(\eR^n)&\to \eR \\
            q&\mapsto \det\big( mat_{\mB}(q) \big) .
        \end{aligned}
    \end{equation}
    Alors :
    \begin{enumerate}
        \item
            La valeur et \( D\) ne dépend pas du choix de la base orthonormée \( \mB\).
        \item
            La fonction \( D\) est donnée par la formule \( D(q)=\prod_i\lambda_i\) où les \( \lambda_i\) sont les valeurs propres de \( q\).
        \item
            La fonction \( D\) est continue.
    \end{enumerate}
\end{proposition}

\begin{proof}
    Soit \( q\) une forme quadratique sur \( \eR^n\). Nous considérons \( \mB\) une base de diagonalisation de \( q\) :
    \begin{equation}
        q(x)=\sum_i\lambda_ix_i
    \end{equation}
    où les \( x_i\) sont les composantes de \( x\) dans la base \( \mB\). Par définition, la matrice \( mat_{\mB}(q)\) est la matrice diagonale contenant les valeurs propres de \( q\).

    Nous considérons aussi \( \mB_1\), une autre base orthonormées de \( \eR^n\). Nous notons \( S=mat_{\mB_1}(q)\); étant symétrique, cette matrice se diagonalise par une matrice orthogonale : il existe \( P\in\gO(n,\eR)\) telle que
    \begin{equation}
        S=P mat_{\mB}(q)P^t;
    \end{equation}
    donc \( \det(S)=\det(PP^t)\det\big( \diag(\lambda_1,\ldots, \lambda_n) \big)=\lambda_1\ldots\lambda_n\). Ceci prouve en même temps que \( D\) ne dépend pas du choix de la base et que sa valeur est le produit des valeurs propres.

    Passons à la continuité. L'application déterminant \( \det\colon S_n(\eR^n)\to \eR\) est continue car polynôme en les composantes. D'autre par l'application \( mat_{\mB}\colon Q(\eR^n)\to S_n(\eR)\) est continue par la proposition~\ref{PropFSXooRUMzdb}. L'application  \( D\) étant la composée de deux applications continues, elle est continue.
\end{proof}

\begin{lemma}[Volume d'un ellipsoïde\cite{KXjFWKA, MonCerveau}]       \label{LEMooLSTOooZiEOdx}
    Soit une forme quadratique strictement définie positive \( q\) sur \( \eR^n\). Nous considérons l'ellipsoïde
    \begin{equation}
        \ellE=\{ x\in \eR^n\tq q(x)\leq r^2 \}
    \end{equation}
    pour un certain \( r>0\).

    Alors le volume de \( \ellE\) est donné par
    \begin{equation}
        V_q=\frac{ V_0 r^n }{ \sqrt{ D(q) } }
    \end{equation}
    où \( V_0\) est une constante\footnote{C'est le volume de la boule unité dans \( \eR^n\) et ce n'est pas tout à fait évident à calculer \cite{ooVLVAooXWmUVB}.} et \( D\) est l'application de la proposition \ref{PropOXWooYrDKpw}.
\end{lemma}

\begin{proof}
    Vu que la matrice de \( q\) est symétrique, elle est diagonalisable par une matrice orthogonale (théorème \ref{ThoeTMXla}). Autrement dit, il existe une base orthonormée \( \mB=\{ e_1,\ldots, e_n \}\) de \( \eR^n\) telle que
    \begin{equation}    \label{EqELBooQLPQUj}
        q(x)=\sum_{i=1}^na_ix_i^2
    \end{equation}
    où \( x_i=\langle e_i, x\rangle \) et les \( a_i\) sont tous strictement positifs. 

    À priori nous devrions calculer
    \begin{equation}
        \int_{\ellE}dx,
    \end{equation}
    mais nous effectuons le changement de variable\footnote{Théorème \ref{THOooUMIWooZUtUSg}.} associé à la matrice qui diagonalise \( q\) et nous devons simplement calculer
    \begin{equation}
        V_q=\int_{\sum_ia_ix_i^2<r^2}dx
    \end{equation}
    parce que le jacobien de ce changement de variable est \( 1\) (déterminant d'une matrice orthogonale).

    Tout cela pour dire que nous nommons \( \ellE_q\) l'ellipsoïde associée à la forme quadratique \( q\) et \( V_q\) son volume que nous allons maintenant calculer\footnote{Le volume ne change pas si nous écrivons l'inégalité stricte au lieu de large dans le domaine d'intégration; nous le faisons pour avoir un domaine ouvert.} :
    \begin{equation}
        V_q=\int_{\sum_ia_ix_i^2<r^2}dx
    \end{equation}
    Cette intégrale est écrite de façon plus simple en utilisant le \( C^1\)-difféomorphisme
    \begin{equation}
        \begin{aligned}
            \varphi\colon \ellE_q&\to B(0,1) \\
            x&\mapsto \frac{1}{  r }\Big( x_1\sqrt{a_1},\ldots, x_n\sqrt{a_n} \Big).
        \end{aligned}
    \end{equation}
    Le fait que \( \varphi\) prenne bien ses valeurs dans \( B(0,1)\) est un simple calcul : si \( x\in\ellE_q\), alors
    \begin{equation}
        \sum_i\varphi(x)_i^2=\frac{1}{ r^2 }\sum_ia_ix_i^2=\frac{1}{ r^2 }q(x)<1.
    \end{equation}
    Cela nous permet d'utiliser le théorème de changement de variables~\ref{THOooUMIWooZUtUSg} :
    \begin{equation}
        V_q=\int_{\sum_ia_ix_i^2<r^2}dx=\frac{r^n}{ \sqrt{a_1\ldots a_n} }\int_{B(0,1)}dx.
    \end{equation}
    La dernière intégrale est le volume de la boule unité dans \( \eR^n\) et nous la notons \( V_0\). La proposition~\ref{PropOXWooYrDKpw} nous permet d'écrire \(V_q\) sous la forme
    \begin{equation}
        V_q=\frac{ V_0 r^n}{ \sqrt{D(q)} }.
    \end{equation}
\end{proof}

\begin{normaltext}
    Le théorème suivant dit en substance que si \( K\) est compact, alors il existe un unique ellipsoïde de volume minimal centré en l'origine et contenant \( K\). Il faut se rendre compte que l'ellipsoïde n'est pas celui que l'on croirait intuitivement parce que la contrainte \emph{centrée en l'origine} est forte. Si \( K=\overline{ B\big( (4,0), 1 \big) }\), alors l'ellipsoïde donnée n'est pas du tout \( K\) lui-même comme on pourrait s'y attendre. Ce serait probablement quelque chose comme la boule centrée en \( (0,0)\) et de rayon \( 5\)\quext{Je ne sais pas très bien si il y a moyen de faire mieux. Ce serait sans doute un bon exercice; faites-moi savoir si vous avez la réponse.}.

    Ce que l'on voudrait est un ellipsoïde qui soit centrée où il faut pour que le volume soit minimal. Nous verrons que c'est possible en la proposition \ref{PROPooVIDPooOGrRJh}, mais qu'alors l'unicité est moins évidente (voir la remarque dans \cite{ooJWHFooGQQhUW}). Si vous voulez en entendre parler, vous pouvez lire \cite{ooASOAooNwZFKS,ooWLGRooFScSaM}.
\end{normaltext}

\begin{proposition}[Ellipsoïde de John-Loewner\cite{KXjFWKA}]   \label{PropJYVooRMaPok}
    Soit \( K\) compact dans \( \eR^n\) et d'intérieur non vide. Il existe une unique ellipsoïde\footnote{Définition~\ref{DefOEPooqfXsE}.} (pleine) centrée en l'origine de volume minimal contenant \( K\).
\end{proposition}
\index{déterminant!utilisation}
\index{extrémums!volume d'un ellipsoïde}
\index{convexité!utilisation}
\index{compacité!utilisation}

\begin{proof}
    Nous subdivisons la preuve en plusieurs parties.
    \begin{subproof}
        \item[Volume de l'ellipsoïde]
            Soit \( \ellE\) un ellipsoïde centré en l'origine. La proposition~\ref{PropWDRooQdJiIr} et son corolaire~\ref{CorKGJooOmcBzh} nous indiquent que
            \begin{equation}
                \ellE=\{ x\in \eR^n\tq q(x)\leq 1 \}
            \end{equation}
            pour une certaine forme quadratique strictement définie positive \( q\). Le lemme \ref{LEMooLSTOooZiEOdx} nous donne alors le volume de \( \ellE\) par
            \begin{equation}
                V_q=\frac{ V_0 }{ \sqrt{ D(q) } }
            \end{equation}
            où \( V_0\) est une constante.
        \item[Existence de l'ellipsoïde]

            Nous voulons trouver un ellipsoïde contenant \( K\) de volume minimal, c'est-à-dire une forme quadratique \( q\in Q^{++}(\eR^n)\) telle que
            \begin{itemize}
                \item \( D(q)\) soit maximal
                \item \( q(x)\leq 1\) pour tout \( x\in K\).
            \end{itemize}
            Nous considérons l'ensemble des candidats semi-définis positifs.
            \begin{equation}
                A=\{ q\in Q^+\tq q(x)\leq 1\forall x\in K \}.
            \end{equation}
            Nous allons montrer que \( A\) est convexe, compact et non vide dans \( Q(\eR^n)\); il aura ainsi un maximum de la fonction continue \( D\) définie sur \( Q(\eR^n)\). Nous montrerons ensuite que le maximum est dans \( Q^{++}\). L'unicité sera prouvée à part.

            \begin{subproof}
            \item[Non vide]
                L'ensemble \( K\) est compact et donc borné par \( M>0\). La forme quadratique \( q\colon x\mapsto \| x \|^2/M^2\) est dans \( A\) parce que si \( x\in K\) alors
                \begin{equation}
                    q(x)=\frac{ \| x \|^2 }{ M^2 }\leq 1.
                \end{equation}
            \item[Convexe]
                Soient \( q,q'\in A\) et \( \lambda\in\mathopen[ 0 , 1 \mathclose]\). Nous avons encore \( \lambda q+(1-\lambda)q'\in Q^+\) parce que
                \begin{equation}
                    \lambda q(x)+(1-\lambda)q'(x)\geq 0
                \end{equation}
                dès que \( q(x)\geq 0\) et \( q'(x)\geq 0\).
            D'autre part si \( x\in K\) nous avons
            \begin{equation}
                \lambda q(x)+(1-\lambda)q'(x)\leq \lambda+(1-\lambda)=1.
            \end{equation}
            Donc \( \lambda q+(1-\lambda)q'\in A\).

        \item[Fermé]

            Pour rappel, la topologie de \( Q(\eR^n)\) est celle de la norme \eqref{EqZYBooZysmVh}. Nous considérons une suite \( (q_n)\) dans \( A\) convergeant vers \( q\in Q(\eR^n)\) et nous allons prouver que \( q\in A\), de sorte que la caractérisation séquentielle de la fermeture (proposition~\ref{PropLFBXIjt}) conclue que \( A\) est fermé. En nommant \( e_x\) le vecteur unitaire dans la direction \( x\) nous avons
            \begin{equation}
                \big| q(x) \big|=\big| \| x \|^2q(e_x) \big|\leq \| x \|^2N(q),
            \end{equation}
            de sorte que notre histoire de suite convergente  donne pour tout \( x\) :
            \begin{equation}
                \big| q_n(x)-q(x) \big|\leq \| x \|^2N(q_n-q)\to 0.
            \end{equation}
            Vu que \( q_n(x)\geq 0\) pour tout \( n\), nous devons aussi avoir \( q(x)\geq 0\) et donc \( q\in Q^+\) (semi-définie positive). De la même manière si \( x\in K\) alors \( q_n(x)\leq 1\) pour tout \( n\) et donc \( q(x)\leq 1\). Par conséquent \( q\in A\) et \( A\) est fermé.

        \item[Borné]

            La partie \( K\) de \( \eR^n\) est borné et d'intérieur non vide, donc il existe \( a\in K\) et \( r>0\) tel que \( \overline{ B(a,r) }\subset K\). Si par ailleurs \( q\in A\) et \( x\in\overline{ B(0,r) }\) nous avons \( a+x\in K\) et donc \( q(a+x)\leq 1\). De plus \( q(-a)=q(a)\leq 1\), donc
            \begin{equation}
                \sqrt{q(x)}=\sqrt{q\big( x+a-a \big)}\leq \sqrt{q(x+a)}+\sqrt{q(-a)}\leq 2
            \end{equation}
            par l'inégalité de Minkowski~\ref{PropACHooLtsMUL}. Cela prouve que si \( x\in\overline{ B(0,r) }\) alors \( q(x)\leq 4\). Si par contre \( x\in\overline{ B(0,1) }\) alors \( rx\in\overline{ B(0,r) } \) et
            \begin{equation}
                0\leq q(x)=\frac{1}{ r^2 }q(rx)\leq \frac{ 4 }{ r^2 },
            \end{equation}
            ce qui prouve que \( N(q)\leq \frac{ 4 }{ r^2 }\) et que \( A\) est borné.


            \end{subproof}

            L'ensemble \( A\) est compact parce que fermé et borné, théorème de Borel-Lebesgue~\ref{ThoXTEooxFmdI}. L'application continue \( D\colon Q(\eR^n)\to \eR\) de la proposition~\ref{PropOXWooYrDKpw} admet donc un maximum sur le compact \( A\). Soit \( q_0\) ce maximum.

            Nous montrons que \( q_0\in Q^{++}(\eR^d)\). Nous savons que l'application \( f\colon x\mapsto \frac{ \| x \|^2 }{ M^2 }\) est dans \( A\) et que \( D(f)>0\). Vu que \( q_0\) est maximale pour \( D\), nous avons
            \begin{equation}
                D(q_0)\geq D(f)>0.
            \end{equation}
            Donc \( q_0\in Q^{++}\).

        \item[Unicité]

            S'il existe une autre ellipsoïde de même volume que celle associée à la forme quadratique \( q_0\), nous avons une forme quadratique \( q\in Q^{++}\) telle que \( q(x)\leq 1\) pour tout \( x\in K\). C'est-à-dire que nous avons \( q_0,q\in A\) tels que \( D(q_0)=D(q)\).

            Nous considérons la base canonique \( \mB_c\) de \( \eR^n\) et nous posons \( S=mat_{\mB_c}(q)\), \( S_0=mat_{\mB_c}(q_0)\). Étant donné que \( A\) est convexe, \( (q_0+q)/2\in A\) et nous allons prouver que cet élément de \( A\) contredit la maximalité de \( q_0\). En effet
            \begin{equation}
                D\left( \frac{ q+q_0 }{ 2 }\right)=\det\left( \frac{ S+S_0 }{2} \right)
            \end{equation}
            Nous allons utiliser le lemme~\ref{LemXOUooQsigHs} qui dit que le déterminant est log-concave sous la forme de l'équation \eqref{EqSPKooHFZvmB} avec \( \alpha=\beta=\frac{ 1 }{2}\) :
            \begin{equation}    \label{eqBHJooYEUDPC}
                D\left( \frac{ q+q_0 }{ 2 }\right)=\det\left( \frac{ S+S_0 }{2} \right)>\sqrt{\det(S)}\sqrt{\det(S_0)}=\det(S_0)=D(q_0).
            \end{equation}
            Nous avons utilisé le fait que \( D(q_0)=D(q)\) qui signifie que \( \det(S_0)=\det(S)\). L'inéquation \eqref{eqBHJooYEUDPC} contredit la maximalité de \( D(q_0)\) et donne donc l'unicité.
    \end{subproof}
\end{proof}

Dans la proposition suivante nous oublions l'unicité, mais nous démontrons qu'il existe un ellipsoïde de volume minimal parmi les ellipsoïdes centrées où l'on veut et non seulement en zéro. La source de cette proposition est \cite{MonCerveau}, et comme toujours avec cette source, vous devez regarder à la fois l'énoncé et la preuve avec un oeil encore plus prudent que d'habitude.
\begin{proposition}[\cite{MonCerveau}]      \label{PROPooVIDPooOGrRJh}
    Soit un compact d'intérieur non vide \( K\) dans \( \eR^n\). Il existe un ellipsoïde de volume minimal contenant \( K\).
\end{proposition}

\begin{proof}
    Au lieu de considérer le compact \( K\) et de chercher où centrer l'ellipsoïde afin qu'elle puisse contenir \( K\) en un volume minimal, nous allons chercher comment translater \( K\) pour qu'un ellipsoïde centré en zéro puisse contenir l'image translétée de \( K \) en un volume minimal.

    Pour \( a\in \eR^n\), nous notons \( \ellE_a\) la plus petite ellipsoïde centrée en \( 0\) contenant \( K+a\) (translation de \( K\) par le vecteur \( a\)). Elle est bien définie par la proposition \ref{PropJYVooRMaPok}. Notre jeu est maintenant d'étudier la fonction\footnote{ Notons que cette fonction est bien définie, même sans la partie unicité de \ref{PropJYVooRMaPok} parce que même si \( \ellE_a\) n'était pas bien définie, son volume, lui, est bien défini.}
    \begin{equation}
        \begin{aligned}
            f\colon \eR^n&\to \eR^+ \\
            a&\mapsto \volume(\ellE_a). 
        \end{aligned}
    \end{equation}
    Nous allons montrer que \( f\) est continue et qu'elle peut être restreinte à un compact sans risque de rater un minimum.

    \begin{subproof}
        \item[À propos de forme quadratique]
            Nous notons \( q\) la forme quadratique de l'ellipsoïde \( \ellE_0\) et \( A\) sa matrice symétrique associée. Si \( x\in K+h\) nous avons, pour un certain \( k\in K\) :
            \begin{subequations}
                \begin{align}
                    q(x)&=q(k+h)\\
                    &=\langle k+h, A(k+h)\rangle \\
                    &=\langle k,Ak, \rangle +\langle k,Ahx, \rangle +\langle h, Ak\rangle +\langle h, Ah\rangle.
                \end{align}
            \end{subequations}
            En tenant compte du fait que \( A=A^t\) nous avons aussi \( \langle k, Ah\rangle =\langle h, Ak\rangle \) et donc
            \begin{equation}
                q(x)=2\langle k, Ah\rangle +q(h).
            \end{equation}
            En ce qui concerne la norme, pour tout \( x\in K+h\) nous avons donc la majoration
            \begin{subequations}
                \begin{align}
                | q(x) |&\leq 1+2\| k \|\| Ah \|+q(h)\\
                &\leq 1+2| K |\| Ah \|+q(h)
                \end{align}
            \end{subequations}
            où nous avons noté \( | K |=\sup_{k\in K}\| k \|\). Vu que \( q(x)\) est en fait toujours positif nous pouvons oublier la valeur absolue à gauche et conclure que \( K+h\) est contenu dans l'ellipsoïde
            \begin{equation}
                \ellF_1=\{ x\in \eR^n\tq q(x)\leq 1+2| K |\| Ah \|+q(h) \}.
            \end{equation}

        \item[Volumes]

            Nous avons donc \( \volume(\ellE_h)\leq \volume(\ellF_1)\). En reprenant la formule du lemme \ref{LEMooLSTOooZiEOdx} pour le volume de l'ellipsoïde nous avons la majoration
            \begin{equation}
                \volume(\ellE_h)\leq \volume(\ellE_0)\big( 1+2| K |\| Ah \|+q(h) \big)^n.
            \end{equation}
            Autrement dit, en ce qui concerne notre fonction \( f\):
            \begin{equation}        \label{EQooLVZCooJVxVNx}
                f(h)\leq f(0)\big( 1+2| K |\| Ah \|+q(h) \big)^n.
            \end{equation}
            
            Nous pouvons faire le même raisonnement en partant de \( K+h\) et en voyant comment modifier le rayon de \( \ellE_h\) pour que \( \ellE_0\) y rentre. Autrement dit, nous refaisons le raisonnement en posant \( K'=K+h\) et en étudiant \( K'-h\). Si \( q_1\) est la forme quadratique de \( \ellE_h\) et si \( A_1\) est sa matrice symétrique,
            \begin{equation}
                \volume(\ellE_0)\leq \volume(\ellE_h)\big( 1+2| K+h |\| A_1h \|+q_1(h) \big)^n
            \end{equation}
            En termes de \( f\):
            \begin{equation}        \label{EQooRHWNooDisTvn}
                f(0)\leq f(h) \big( 1+2| K+h |\| A_1h \|+q_1(h) \big)^n
            \end{equation}
        \item[Encadrement]

            Les majoration \eqref{EQooLVZCooJVxVNx} et \eqref{EQooRHWNooDisTvn} nous permettent de créer un encadrement pour \( f(h)\). En effet, en écrivant \eqref{EQooRHWNooDisTvn} et en continuant la majoration en remplaçant \( f(h)\) par \eqref{EQooLVZCooJVxVNx}, nous obtenons
            \begin{equation}
                \begin{aligned}[]
                    f(0)\leq f(h) \big( 1&+2| K+h |\| A_1h \|+q_1(h) \big)^n\\
                    &\leq f(0)\big( 1+2| K |\| Ah \|+q(h) \big)^n \big( 1+2| K+h |\| A_1h \|+q_1(h) \big)^n.
                \end{aligned}
            \end{equation}
            Vu que nous avons dans l'idée de prendre des \( h\) petits (en norme) et que les parenthèses tendent vers \( 1\) lorsque \( h\) est petit, nous pouvons supposer qu'elles sont non nulles et allègrement les passer au dénominateur. Nous divisons donc tout par le coefficient de \( f(h)\):
            \begin{equation}
                \frac{ f(0) }{  \big( 1+2| K+h |\| A_1h \|+q_1(h) \big)^n }\leq f(h)\leq f(0)  \big( 1+2| K |\| Ah \|+q(h) \big)^n
            \end{equation}
            Nous utilisons la règle de l'étau\footnote{Théorème \ref{ThoRegleEtau}.} pour conclure que
            \begin{equation}        \label{EQooCMAQooMkkEVR}
                \lim_{h\to 0} \volume(\ellE_h)=\volume(\ellE_0).
            \end{equation}
        \item[Continuité]
            Si \( a\in \eR^n\) nous pouvons appliquer la limite \eqref{EQooCMAQooMkkEVR} pour l'ellipsoïde \( \ellE_a\) au lieu de \( \ellE_0\), c'est-à-dire en partant du compact \( K+a\) au lieu de \( K\). Cela donne
            \begin{equation}
                \lim_{h\to 0} \volume(\ellE_{a+h})=\volume(\ellE_a).
            \end{equation}
            Donc \( f\) est continue sur \( \eR^n\).

        \item[Compact]

            Nous avons en hypothèse que \( K\) est d'intérieur non vide. Il existe donc \( a\in K\) et \( r>0\) tel que \( B(a,r)\subset K\). Soit l'hyperplan affine \( H\) passant par \( a\) et perpendiculaire à \( a\). Nous notons \( S=H\cap B(a,r)\). Nous avons évidemment \( S\subset K\) et donc \( S\subset \ellE_0\). Vu que l'ellipsoïde \( \ellE_0\) est convexe et contient \( K\), tout le cône de base \( S\) et de hauteur \( a\) est dans \( \ellE_0\).

            Sans rentrer dans le détails du calcul du volume de ce cône, son volume tend vers l'infini si \( a\) tend vers l'infini\footnote{Topologie du complété en un point, j'imagine que vous voyez de quoi il en retourne. Sinon c'est la définition \ref{PROPooHNOZooPSzKIN}, et il faut sans doute adapter le lemme \ref{LEMooERABooQjLBzW} au cas de \( \eR^n\) pour tout faire dans les règles.}

            Nous avons donc une majoration
            \begin{equation}
                \volume(\ellE_h)\geq \text{volume du cône de base } S\text{ et de hauteur } a+h.
            \end{equation}
            Dès que ce volume est plus grand que \( \volume(\ellE_0)\), il est impossible que \( \volume(\ellE_h)\leq \volume(\ellE_0)\). Soit \( R>0\) tel que pour tout \( \| h \|\geq R\), le volume du cône soit plus grand que \( \volume(\ellE_0)\). Le minimum de \( f\) est certainement dans \( B(0,R)\) parce que au moins
            \begin{equation}
                \volume(\ellE_0)\in f\big( B(0,R) \big).
            \end{equation}

        \item[Minimum]

            Nous considérons la fonction
            \begin{equation}
                \begin{aligned}
                    f\colon \overline{ B(0,R) }&\to \eR^+ \\
                    h&\mapsto \volume(\ellE_h). 
                \end{aligned}
            \end{equation}
            Elle est continue sur un compact, et le théorème de Weierstrass \ref{ThoWeirstrassRn} nous dit alors qu'elle atteint son minimum. Il n'y a cependant pas unicité de la valeurs de \( h\) pour laquelle \( f(h)\) est minimum.

        \item[Envoi]

            Soit \( s\in \eR^n\) tel que \( f(s)\) soit minimal. Cela signifie que parmi tous les ellipsoïdes centrés en zéro et contenant des ensembles de la forme \( K+h\), le plus petit est celui qui contient \( K+s\). Soit \( \ellE_s\) cet ellipsoïde.

            Je prétend que \( \ellE_s-s\) est le plus petit ellipsoïde contenant \( K\), tout centres confondus. En effet, si \( \ellF\) est un ellipsoïde centré en \( a\) et contenant \( K\), alors \( \ellF-a\) est un ellipsoïde centré en zéro et contenant \( K-a\). Nous avons alors
            \begin{equation}
                \volume(\ellF-a)\geq \volume(\ellE_s)=\volume(\ellE_s-s).
            \end{equation}
            L'invariance de la mesure par translation est le théorème \ref{THOooTMWHooThsDHj}.
    \end{subproof}
\end{proof}

\input{80_Newton}

\chapter{Trigonométrie, isométries}
% This is part of (everything) I know in mathematics
% Copyright (c) 2011-2013,2016-2020
%   Laurent Claessens
% See the file fdl-1.3.txt for copying conditions.

%+++++++++++++++++++++++++++++++++++++++++++++++++++++++++++++++++++++++++++++++++++++++++++++++++++++++++++++++++++++++++++
\section{Trigonométrie}
%+++++++++++++++++++++++++++++++++++++++++++++++++++++++++++++++++++++++++++++++++++++++++++++++++++++++++++++++++++++++++++

%---------------------------------------------------------------------------------------------------------------------------
\subsection{Définitions, périodicité et quelques valeurs remarquables}
%---------------------------------------------------------------------------------------------------------------------------

\begin{propositionDef}[Défintion du cosinus et du sinus]        \label{PROPooZXPVooBjONka}
    La série
    \begin{equation}
        \cos(x)=\sum_{n=0}^{\infty}\frac{ (-1)^n }{ (2n)! }x^{2n}
    \end{equation}
    définit une fonction \( \cos\colon \eR\to \eR\) de classe \(  C^{\infty}\). Nous l'appelons \defe{cosinus}{cosinus}.

    La série
    \begin{equation}        \label{EQooCMRFooCTtpge}
        \sin(x)=\sum_{n=0}^{\infty}\frac{ (-1)^n }{ (2n+1)! }x^{2n+1}
    \end{equation}
    définit une fonction \( \sin\colon \eR\to \eR\) de classe \(  C^{\infty}\). Nous l'appelons \defe{sinus}{sinus}.
\end{propositionDef}

\begin{proof}
    La série entière définissant \( \cos(x)\) a pour coefficients
    \begin{equation}
        a_n=\begin{cases}
            0    &   \text{si } n\text{ est impair}\\
            \frac{ (-1)^{n/2} }{ n! }    &   \text{si } n\text{ est pair}.
        \end{cases}
    \end{equation}
    Nous pouvons la majorer par la série entière donnée par les coefficients
    \begin{equation}
        b_n=\begin{cases}
            1/n!    &   \text{si } n\text{ est impair}\\
            \frac{ (-1)^{n/2} }{ n! }    &   \text{si } n\text{ est pair}.
        \end{cases}
    \end{equation}
    Quelle que soit la parité de \( k\) nous avons toujours
    \begin{equation}
        | \frac{ b_{k+1} }{ b_k } |=\frac{1}{ k+1 },
    \end{equation}
    de telle sorte que la formule d'Hadamard \eqref{EqAlphaSerPuissAtern} nous donne \( R=\infty\) pour la série \( \sum_{k=0}^{\infty}b_kx^k\). A fortiori\footnote{Remarque~\ref{REMooYOTEooKvxHSf}.} le rayon de convergence pour la série du cosinus est infini.

    L'assertion concernant le sinus se démontre de même.

    En ce qui concerne le fait que les fonctions \( \sin\) et \( \cos\) sont de classe \(  C^{\infty}\) sur \( \eR\), il faut invoquer le corolaire~\ref{CorCBYHooQhgara}.
\end{proof}

Par substitution directe dans les séries, nous avons immédiatement
\begin{subequations}        \label{SUBEQooTTNNooXzApSM}
    \begin{align}
        \cos(0)&=1\\
        \sin(0)&=1.
    \end{align}
\end{subequations}

\begin{lemma}       \label{LEMooBBCAooHLWmno}
    En ce qui concerne la dérivation, nous avons
    \begin{subequations}
        \begin{align}
            \sin'&=\cos\\
            \cos'&=-\sin.
        \end{align}
    \end{subequations}
\end{lemma}

\begin{proof}
    Il s'agit de se permettre de dériver terme à terme (proposition~\ref{ProptzOIuG}) les séries qui définissent le sinus et le cosinus.
\end{proof}

\begin{lemma}       \label{LEMooAEFPooGSgOkF}
    Les fonctions sinus et cosinus vérifient
    \begin{equation}        \label{EQooNYCZooApyyRd}
        \cos^2(x)+\sin^2(x)=1
    \end{equation}
    pour tout \( x\in \eR\).
\end{lemma}

\begin{proof}
    Posons \( f(x)=\sin^2(x)+\cos^2(x)\) et dérivons :
    \begin{equation}
        f'(x)=2\sin(x)\cos(x)+2\cos(x)(-)\sin(x)=0.
    \end{equation}
    La fonction \( f\) est donc constante par le corolaire~\ref{CORooEOERooYprteX}. Nous avons donc pour tout \( x\) :
    \begin{equation}
        f(x)=f(0)=\sin^2(0)+\cos^2(0)=1.
    \end{equation}
    Le dernier calcul s'obtient en substituant directement \( x\) par zéro dans les séries : \( \sin(0)=0\) et \( \cos(0)=1\).
\end{proof}

%--------------------------------------------------------------------------------------------------------------------------- 
\subsection{Fonction puissance (pour les complexes)}
%---------------------------------------------------------------------------------------------------------------------------

La fonction puissance a déjà fait l'objet de nombreuses définitions et extensions. Voir le thème \ref{THEMEooBSBLooWcaQnR}. Nous allons maintenant définir \( a^z\) pour \( a>0\) et \( z\in \eC\). 

\begin{definition}      \label{DEFooRBTDooNLcWGj}
    Pour le nombre \( e\in \eR\) et le nombre imaginaire pur \( iy\) (\( y\in \eR\)), nous définissons
    \begin{equation}
        e^{iy}=\exp(iy)
    \end{equation}
    où \( \exp\) est la série usuelle de la définition \ref{DEFooSFDUooMNsgZY}. Pour un nombre complexe général \( x+yi\) nous définissons
    \begin{equation}
        e^{x+iy}= e^{x} e^{iy}.
    \end{equation}
    Et enfin, si \( a>0\) et si \( z\in \eC\) nous définissons
    \begin{equation}
        a^z= e^{z\ln(a)},
    \end{equation}
    la fonction logarithme ici étant celle \( \ln\colon \mathopen] 0 , \infty \mathclose[\to \eR\) définie par la proposition \ref{DEFooELGOooGiZQjt}.
\end{definition}

Si \( z\in \eC\) et si \( n\in \eZ\), la définition de \( z^n\) ne pose pas de problèmes, c'est la définition \ref{DEFooGVSFooFVLtNo}.

\begin{normaltext}  \label{DefJilXoM}
    Soit \( z=x+iy\in \eC\). L'exponentielle \( \exp(x+yi)\) est déjà définie en \ref{DEFooSFDUooMNsgZY}; elle est la fonction donnée par
    \begin{equation}
        \begin{aligned}
            \exp\colon \eC&\to \eC \\
            z&\mapsto \sum_{n=0}^{\infty}\frac{ z^n }{ n! }.
        \end{aligned}
    \end{equation}
\end{normaltext}

\begin{proposition}     \label{PROPooXEYFooIEaPvU}
Le rayon de convergence\footnote{Définition \ref{DefZWKOZOl}.} de la série exponentielle est infini.
\end{proposition}

\begin{proof}
    La formule de Hadamard de la proposition \ref{PROPooMXCDooBffXbl} est à utiliser avec \( a_k=j!\). Nous avons
    \begin{equation}
        \frac{1}{ R }=\lim_{k\to \infty} \left| \frac{ (n+1)! }{ n! } \right| =\lim_{k\to \infty} (n+1)=\infty.
    \end{equation}
    Donc \( R=\infty\).
\end{proof}

\begin{proposition}
    Pour tout \( z\in \eC\) nous avons
    \begin{equation}
        \exp(z)= e^{z}.
    \end{equation}
\end{proposition}

\begin{proposition}[\cite{RomainBoilEnt}]     \label{PropdDjisy}
    Quelques propriétés de l'exponentielle.
    \begin{enumerate}
        \item
            Le fonction \( \exp\) est continue.
        \item       \label{ITEMooRLHCooJTuYKV}
            Nous avons la formule \(  e^{z+w}= e^{z}e^w\) pour tout \( z,w\in \eC\).
        \item
            \( (e^z)^{-1}= e^{-z}\)
        \item
            \( (\exp(z))^n=\exp(nz)\).
    \end{enumerate}
\end{proposition}

\begin{proof}
    La proposition \ref{PROPooXEYFooIEaPvU} nous enseigne que le rayon de convergence est infini. La fonction ainsi définie est alors continue par la proposition \ref{PropUEMoNF}.

    Les séries \( \exp(z)\) et \( \exp(w)\) ayant un rayon de convergence infini nous pouvons utiliser le produit de Cauchy (théorème~\ref{ThokPTXYC}) :
    \begin{subequations}
        \begin{align}
            e^{z} e^{w}&=\sum_{n=0}^{\infty}\left( \sum_{i+j=n}\frac{ z^iw^j }{ i!j! } \right)\\
            &=\sum_{n=0}^{\infty}\left( \sum_{i=0}^n\frac{ z^iw^{n-i} }{ i!(n-i)! } \right)\\
            &=\sum_{n=0}^{\infty}\frac{1}{ n! }\sum_{i=0}^{n}{n\choose i}z^iw^{n-i}\\
            &=\sum_{n=0}^{\infty}\frac{1}{ n! }(z+w)^{n}\\
            &=\exp(z+w).
        \end{align}
    \end{subequations}
    Nous avons utilisé la formule du binôme (proposition~\ref{PropBinomFExOiL}).

    Les autres propriétés énoncées sont des corolaires :
    \begin{equation}
        e^{z} e^{-z}= e^{0}=1.
    \end{equation}
\end{proof}

D'autres propriétés de l'exponentielle sur \( \eC\), entre autres l'holomorphie, sont données dans le théorème \ref{THOooNGOIooEECfAv}.


\begin{lemma}[\cite{MonCerveau}]        \label{LEMooTDGKooWdpUTD}
    Soient \( a>0\), \( z\in \eC\) et \( n\in \eZ\). Alors
    \begin{equation}
        (a^z)^n=a^{nz}.
    \end{equation}
\end{lemma}

%--------------------------------------------------------------------------------------------------------------------------- 
\subsection{Formules de trigonométrie}
%---------------------------------------------------------------------------------------------------------------------------

Le lemme suivant est un premier pas pour le paramétrage du cercle dont nous parlerons dans la proposition \ref{PROPooZEFEooEKMOPT}.
\begin{lemma}       \label{LEMooHOYZooKQTsXW}
    Nous avons la formule
    \begin{equation}        \label{EQooRVPJooTMwNTU}
        e^{ix}=\cos(x)+i\sin(x)
    \end{equation}
    pour tout \( x\in \eR\).

    En particulier pour tout \( x\), nous avons \( |  e^{ix} |=1\).
\end{lemma}

\begin{proof}
    La définition de l'exponentielle sur \( \eC\) est la définition~\ref{DEFooSFDUooMNsgZY}. Cette définition fonctionne parce que \( \eC\) est une algèbre normée, et que \( \eC\) est un \( \eC\)-module vérifiant l'inégalité \(  | zz' |\leq | z | |z' | \) (en l'occurrence, une égalité).

    Nous remarquons que que \( i^k\) vaut \( 1\), \( i\), \( -1\), \( -i\). Donc un terme sur deux est imaginaire pur et parmi ceux-là, un sur deux est positif. À bien y regarder, les termes imaginaires purs forment la série du sinus et ceux réels la série du cosinus.

    Si vous aimez les formules,
    \begin{equation}
            e^{iy}=\sum_{n=0}^{\infty}\frac{ (iy)^n }{ n! }
            =\sum_{n=0}^{\infty}(-1)^n\frac{ y^{2n} }{ (2n)! }+i\sum_{n=0}^{\infty}(-1)^n\frac{ y^{2n+1} }{ (2n+1)! }
            =\cos(y)+i\sin(y).
    \end{equation}
    Nous avons utilisé le fait que \( i^{2n}=(-1)^n\) et \( i^{2n+1}=i(-1)^n\).
\end{proof}

\begin{lemma}       \label{LEMooJAWBooJGfZIL}
    Nous avons les formules d'addition d'angles\footnote{Rien ne nous empêche de donner ce nom à ces formules, mais seriez-vous capable de définir précisément le mot «angle» ?}
    \begin{subequations}        \label{SUBEQSooFSSMooHcYwRc}
        \begin{align}
            \cos(a+b)=\cos(a)\cos(b)-\sin(a)\sin(b) \label{EQooJYEMooQaOMib}\\
            \sin(a+b)=\cos(a)\sin(b)+\sin(a)\cos(b) \label{EQooECAUooQzckDv}\\
            \cos(a-b)=\cos(a)\cos(b)+\sin(a)\sin(b) \label{EQooCVZAooQfocya}
        \end{align}
    \end{subequations}
    pour tout \( a\), \( b\) réels.
\end{lemma}

\begin{proof}
    Nous utilisons la formule d'addition dans l'exponentielle, proposition \eqref{EQooVFXUooBfwjJY} et la formule \eqref{EQooRVPJooTMwNTU} avant de séparer les parties réelles et imaginaires :
    \begin{equation}
        e^{i(a+b)}= e^{ia} e^{ib}=\cos(a)\cos(b)-\sin(a)\sin(b)+i\big( \cos(a)\sin(b)+\sin(a)\cos(b) \big).
    \end{equation}
    Cela est également égal à
    \begin{equation}
        \cos(a+b)+i\sin(a+b).
    \end{equation}
    En identifiant les parties réelle et imaginaires, nous obtenons les formules \eqref{EQooJYEMooQaOMib} et \eqref{EQooCVZAooQfocya} annoncées.

    Pour la formule \eqref{EQooCVZAooQfocya}, il suffit de se souvenir que \( \sin(-b)=-\sin(b)\) et \( \cos(-b)=\cos(b)\) (ces deux égalités sont immédiatement visibles sur les développements en série : l'un a uniquement des puissances paires et l'autre impaires) et d'écrire \eqref{EQooJYEMooQaOMib} avec \( -b\) au lieu de \( b\).
\end{proof}

\begin{corollary}       \label{CORooQZDQooWjMXTF}
    Les formules suivantes pour les duplications d'angles s'ensuivent :
    \begin{subequations}
        \begin{align}
            \cos(2a)&=\cos^2(a)-\sin^2(a)\\
            \sin(2a)&=2\cos(a)\sin(a).      \label{SUBEQooLRJDooQuFvux}
        \end{align}
    \end{subequations}
\end{corollary}

\begin{proof}
    Poser \( b=a\) dans les relations du lemme~\ref{LEMooJAWBooJGfZIL}.
\end{proof}

\begin{lemma}       \label{LEMooPQWWooMdPWUT}
    Un sous-groupe de \( (\eR,+)\) est soit dense dans \( \eR\) soit de la forme \( p\eZ\) pour un certain réel \( p\neq 0\).
\end{lemma}

\begin{proof}
    Soit \( A\), un sous-groupe de \( (\eR,+)\) qui ne soit pas dense. Soit un intervalle \( \mathopen] a , b \mathclose[\) qui n'intersecte pas \( A\) (si vous voulez frimer, vous noterez ici que nous utilisons le fait que les intervalles ouverts forment une base de la topologie de \( \eR\)). Si \( d=| b-a |\), l'ensemble \( A\) ne contient pas deux éléments séparés par strictement moins de \( d\). Soit \( p\), le plus petit élément strictement positif de \( A\); nous avons \( p\geq d\) (parce que \( 0\in A\) de toutes façons).

        Vu que \( A\) est un groupe nous avons \( p\eZ\subset A\).

        Pour l'inclusion inverse, si \( x\in A\) est hors de \( p\eZ\), il existe un \( y\in p\eZ\) avec \( | x-y |<p\). Et donc le nombre \( | x-y |\) est dans \( A\) tout en étant plus petit que \( p\). Contradiction.
\end{proof}

\begin{propositionDef}[Périodicité, le nombre \( \pi\)\cite{ooUMDHooHrJpfV}]      \label{PROPooFRVCooKSgYUM}
    Plusieurs choses à propos de la périodicité de la fonction \( \cos\).
    \begin{enumerate}
        \item
            La fonction \( \cos\) est périodique.
        \item
            Un nombre \( T>0\) est une période si et seulement si \( \cos(T)=1\) et \( \sin(T)=0\).
    \end{enumerate}
    
    Nous définissons le nombre \( \pi>0\) comme étant la moitié de la période de la fonction \( \cos\) :
    \begin{equation}
        2\pi=\min\{ T>0\tq \cos(x+T)=\cos(x)\,\forall x \}.
    \end{equation}

\end{propositionDef}

\begin{proof}
    Plusieurs étapes.
    \begin{subproof}
        \item[La fonction cosinus n'est pas toujours positive]
    Supposons d'abord que \( \cos(x)>0\) pour tout \( x\in \eR\). Dans ce cas, la fonction \( \sin\) est strictement croissante. Mais les deux fonctions sont bornées par \( 1\) du fait de la formule \( \cos^2(x)+\sin^2(x)=1\). La fonction \( \sin\) étant croissante et bornée, elle est convergente vers un réel par la proposition~\ref{PropMTmBYeU} :
    \begin{equation}
        \lim_{x\to \infty} \sin(x)=\ell
    \end{equation}
    pour un certain \( \ell>0\). Avec ça nous avons aussi (pour cause de dérivée) \( \lim_{x\to \infty} \sin'(x)=0\), c'est-à-dire \( \lim_{x\to \infty} \cos(x)=0\). Mais vu que \( \cos^2(x)+\sin^2(x)=1\) nous en déduisons que \( \lim_{x\to \infty} \sin(x)=1\). Mézalor \( \lim_{x\to \infty} \cos'(x)=-1\), ce qui donne que la fonction \( \cos\) n'est pas bornée. Cela est impossible. Nous en déduisons que \( \cos(x)\) n'est pas toujours positive.

\item[Il existe \( T>0\) tel que \( \cos(T)=1\) et \( \sin(T)=0\)]

    Par ce que nous venons de faire, il existe \( r>0\) tel que \( \cos(r)=0\). Pour cette valeur, nous avons aussi obligatoirement \( \sin(r)=\pm 1\). Nous avons aussi, en utilisant les formules \eqref{SUBEQSooFSSMooHcYwRc},
    \begin{subequations}
        \begin{align}
            \cos(2r)=\cos^2(r)-\sin^2(r)=-1\\
            \sin(2r)=2\cos(r)\sin(r)=0.
        \end{align}
    \end{subequations}
    et par conséquent
    \begin{subequations}
        \begin{align}
            \cos(4r)=\cos^2(2r)-\sin^2(2r)=1\\
            \sin(4r)=2\cos(2r)\sin(2r)=0.
        \end{align}
    \end{subequations}
    Donc \( T=4r\) fonctionne.

\item[Si \( T\) est une période]
    Nous entrons dans le vif de la preuve. Soit un \( T>0\) tel que \( \cos(x+T)=\cos(x)\) pour tout \( x\in \eR\). Avec la formule d'addition d'angle dans le cosinus nous cherchons un \( T\) tel que
    \begin{equation}
        \cos(x+T)=\cos(x)\cos(T)-\sin(x)\sin(T)=\cos(x)
    \end{equation}
    et donc tel que
    \begin{equation}        \label{EQooELSAooLNtBnm}
        \cos(x)\big( \cos(T)-1 \big)=\sin(x)\sin(T).
    \end{equation}
    Nous dérivons cette équation :
    \begin{equation}        \label{EQooCECFooLpxXaw}
        -\sin(x)\big( \cos(T)-1 \big)=\cos(x)\sin(T).
    \end{equation}
    Nous multiplions chacune des deux équations \eqref{EQooELSAooLNtBnm} et \eqref{EQooCECFooLpxXaw} par \( \sin(x)\) et \( \cos(x)\) pour obtenir les quatre relations suivantes :
    \begin{subequations}
        \begin{align}
            \cos^2(x)\big( \cos(T)-1 \big)-\sin(x)\cos(x)\sin(T)=0   \label{SUBEQooLGQXooIrLMLW}\\
            -\sin(x)\cos(x)\big( \cos(T)-1 \big)-\cos^2(x)\sin(T)=0     \label{SUBEQooCHTDooKwvyZF}\\
            \sin(x)\cos(x)\big( \cos(T)-1 \big)-\sin^2(x)\sin(T)=0 \label{SUBEQooEWPTooTLCUMf}\\
            \sin^2(x)\big( \cos(T)-1 \big)-\sin(x)\cos(x)\sin(T)=0  \label{SUBEQooGBXTooCFekGJ}
        \end{align}
    \end{subequations}
    En faisant \eqref{SUBEQooLGQXooIrLMLW} moins \eqref{SUBEQooGBXTooCFekGJ} nous trouvons \( \cos(T)=1\). Et en sommant \eqref{SUBEQooCHTDooKwvyZF} avec \eqref{SUBEQooEWPTooTLCUMf} nous avons \( -\sin(T)=0\).

\item[Si \( T>0\) est tel que \( \sin(T)=0\) et \( \cos(T)=1\)]

    Alors les formules d'addition d'angle du lemme \ref{LEMooJAWBooJGfZIL} donnent tout de suite
    \begin{equation}
        \cos(x+T)=\cos(x).
    \end{equation}

    \end{subproof}

    À de niveau nous croyons avoir prouvé que \( \cos\) était périodique et que la période est donnée par
    \begin{equation}
        \min\{ T>0\tq \sin(T)=0,\cos(T)=1 \}.
    \end{equation}
    Or rien n'est moins sûr parce qu'il pourrait arriver que ce minimum n'existe pas, c'est-à-dire que l'infimum soit zéro. Autrement dit, il peut arriver que l'ensemble des périodes soit dense. Plus précisément, soit \( P\subset \eR\) l'ensemble des périodes de \( \cos\). C'est un sous-groupe de \( (\eR,+)\) et le lemme~\ref{LEMooPQWWooMdPWUT} nous dit que \( P\) est soit dense dans \( \eR\) soit de la forme \( p\eZ\) pour un \( p>0\).

    Si \( P\) est dense, soit \( t\in \eR\) et une suite \( (t_n)\) dans \( P\) telle que \( t_n\to t\). Pour tout \( x\) et tout \( n\) nous avons
    \begin{equation}
        \cos(x+t_n)=\cos(x),
    \end{equation}
    Vu que la fonction cosinus est continue, nous pouvons passer à la limite et écrire \( \cos(x+t)=\cos(x)\). Cela étant valable pour tout \( x\) et pour tout \( t\), la fonction cosinus est constante. Or nous savons que ce n'est pas le cas, donc \( P\) n'est pas dense. Donc cosinus est périodique.
\end{proof}

\begin{proposition}     \label{PROPooKNLAooLwQHea}
    La fonction \( \sin\) est périodique de période \( \pi\) et
    \begin{equation}
        2\pi=\min\{ T>0\tq \sin(T)=0,\cos(T)=1 \}.
    \end{equation}
\end{proposition}

\begin{normaltext}
Notons que tout ceci ne nous donne pas la plus petite indication d'ordre de grandeur de la valeur de \( \pi\). Cela peut encore être \( 0.1\) autant que \( 500\).
\end{normaltext}

\begin{proposition}[\cite{ooUMDHooHrJpfV,MonCerveau}]      \label{PROPooMWMDooJYIlis}
    Des propriétés à la chaine à propos des sinus, cosinus et de leurs périodes.
    \begin{enumerate}
        \item       \label{ITEMooRJZHooCXcKmM}
            Le nombre \( 2\pi\) est le plus petit nombre strictement positif tel que
            \begin{subequations}
                \begin{numcases}{}
                    \cos(2\pi)=1\\
                    \sin(2\pi)=0.
                \end{numcases}
            \end{subequations}
        \item
            Les fonctions \( \sin\) et \( \cos\) sont périodiques de période \( 2\pi\).
        \item
            Nous avons \( \cos(\pi)=- 1\) et \( \sin(\pi)=0\).
        \item
            Pour tout \( a\in \eR\) nous avons
            \begin{subequations}
                \begin{align}
                    \cos(a+\pi)&=-\cos(\pi)\\
                    \sin(a+\pi)&=-\sin(\pi).
                \end{align}
            \end{subequations}
        \item       \label{ITEMooHDQNooYHVCkg}
            Le nombre \( \pi\) est le plus petit \( T>0\) tel que \( \cos(T)=-1\) et \( \sin(T)=0\).
        \item       \label{ITEMooWFNUooYAybDB}
            Nous avons
            \begin{subequations}        \label{SUBEQSooBTNPooSvCAHO}
                \begin{numcases}{}
                    \cos(\pi/2)=0\\
                    \sin(\pi/2)=1.
                \end{numcases}
            \end{subequations}
        \item
            Nous avons les formules
            \begin{subequations}        \label{EQSooRJZGooCFVqbZ}
                \begin{numcases}{}
                    \cos(x+\pi/2)=-\sin(x)\\
                    \sin(x+\pi/2)=\cos(x)
                \end{numcases}
            \end{subequations}
            pour tout \( x\in \eR\).
        \item       \label{ITEMooMQQPooGwOdbt}
            Le nombre \( \pi/2\) est le plus petit \( T>0\) vérifiant \( \sin(T)=1\), \( \cos(T)=0\).
        \item
            Nous avons les valeurs
            \begin{subequations}
                \begin{numcases}{}
                    \cos(\frac{ 3\pi }{ 2 })=0\\
                    \sin(\frac{ 3\pi }{ 2 })=-1.
                \end{numcases}
            \end{subequations}
        \item       \label{ITEMooQKPKooEPeHER}
            Le nombre \( \pi/2\) est le plus petit \( T>0\) tel que \( \cos(T)=0\).
        \item               \label{ITEMooMEXUooGfSInJ}
            Pour tout \( x\in \mathopen] 0 , \frac{ \pi }{ 2 } \mathclose[\), nous avons \( \cos(x)>0\) et \( \sin(x)>0\).
    \end{enumerate}
\end{proposition}

\begin{proof}
    C'est parti.
    \begin{enumerate}
        \item
            Le fond de la proposition~\ref{PROPooFRVCooKSgYUM} est que toutes les périodes \( T>0\) vérifient \( \cos(T)=1\) et \( \sin(T)=0\). La définition de \( \pi\) est que c'est la plus petite période.
        \item
            En utilisant le fait que l'une est la dérivée de l'autre, si \( T\) est une période de \( \cos\) nous avons
            \begin{subequations}
                \begin{align}
                    \sin(x+T)&=-\cos'(x+T)\\
                    &=-\lim_{\epsilon\to 0}\frac{ \cos(x+T+\epsilon)-\cos(x+T) }{\epsilon  }\\
                    &=-\lim_{\epsilon\to 0}\frac{ \cos(x+\epsilon)-\cos(x) }{ \epsilon }\\
                    &=-\cos'(x)\\
                    &=\sin(x).
                \end{align}
            \end{subequations}
            Nous déduisons que toute période de \( \cos\) est une période de \( \sin\). De la même façon, nous pouvons prouver le contraire : toute période de \( \sin\) est une période de \( \cos\).
        \item
            D'un côté nous avons
            \begin{equation}
                \cos(2\pi)=\cos^2(\pi)-\sin^2(\pi)=1
            \end{equation}
            parce que \( \cos(2\pi)=\cos(0)=1\). Vu que \( \cos(\pi)\) et \( \sin(\pi)\) sont bornés par \( -1\) et \( 1\), nous devons avoir \( \sin(\pi)=0\) et \( \cos(\pi)=\pm 1\).

            Mais d'un autre côté, le nombre \( 2\pi\) est le plus petit \( T\) vérifiant \( \cos(T)=1\), \( \sin(T)=0\). Donc avoir \( \cos(\pi)=1\) n'est pas possible. Nous concluons
            \begin{subequations}
                \begin{numcases}{}
                    \cos(\pi)=-1\\
                    \sin(\pi)=0.
                \end{numcases}
            \end{subequations}
        \item
            Il s'agit d'utiliser les formules d'addition d'angles du lemme~\ref{LEMooJAWBooJGfZIL} pour calculer \( \cos(a+\pi)\) et \( \sin(a+\pi)\) en tenant compte du fait que \( \cos(\pi)=-1\) et \( \sin(\pi)=0\).
        \item
        Soit \( a\in\mathopen] 0 , \pi \mathclose[\) tel que \( \cos(a)=-1\) et \( \sin(a)=0\). Alors nous avons
            \begin{subequations}
                \begin{align}
                    \cos(a+\pi)=-\cos(\pi)=1\\
                    \sin(a+\pi)=-\sin(\pi)=0,
                \end{align}
            \end{subequations}
        ce qui donnerait \( a+\pi\in\mathopen] \pi , 2\pi \mathclose[\) dont le cosinus est \( 1\) et le sinus est zéro. Mais nous savons déjà que \( 2\pi\) est le minimum pour cette propriété.
        \item
            Nous avons
            \begin{equation}
                -1=\cos(\pi)=\cos^2(\pi/2)-\sin^2(\pi/2),
            \end{equation}
            donc \( \cos(\pi/2)=0\) et \( \sin^2(\pi/2)=1\), ce qui donne \( \sin(\pi/2)=\pm 1\).

        Nous devons départager le \( \pm\). Pour cela nous savons que \( \sin'(0)=\cos(0)=1\), donc il existe \( \epsilon>\epsilon\) tel que pour tout \( x\in\mathopen] 0 , \epsilon \mathclose[\) nous avons \( 0<\cos(x)<1\) et \( 0\sin(x)<1\). Nous choisissons \( \epsilon\) plus petit que \( \pi/2\) .

        Supposons que \( \sin(\pi/2)=-1\). Le théorème des valeurs intermédiaires~\ref{ThoValInter} dit qu'il existe \( x_0\in\mathopen] \epsilon , \pi/2 \mathclose[\) tel que \( \sin(x_0)=0\). Pour cette valeur de \( x_0\) nous devons aussi avoir \( \cos(x_0)=\pm 1\). Mais vu que \( 2\pi\) est minium pour avoir \( \cos=1\) et \( \sin=0\) nous devons avoir \( \cos(x_0)=-1\). Alors nous avons aussi
            \begin{subequations}
                \begin{align}
                    \cos(x_0+\pi)=\cos(x_0)\cos(\pi)-\sin(x_0)\sin(\pi)=-\cos(x_0)=1\\
                    \sin(x_0+\pi)=\cos(x_0)\sin(\pi)+\sin(x_0)\cos(\pi)=\sin(x_0)=0.
                \end{align}
            \end{subequations}
            Encore une fois par minimalité de \( 2\pi\), cela ne va pas. Conclusion : \( \sin(\pi/2)=1\).
        \item
            Il s'agit encore d'utiliser les formules d'addition d'angle en tenant compte des valeurs remarquables \( \cos(\pi/2)=0\) et \( \sin(\pi/2)=1\).
        \item
        Supposons \( x_0\in\mathopen] 0 , \pi/2 \mathclose[\) tel que \( \sin(x_0)=1\) et \( \cos(x_0)=0\). En utilisant les formules \eqref{EQSooRJZGooCFVqbZ} nous avons
            \begin{subequations}
                \begin{align}
                    \cos(x_0+\pi/2)=-1\\
                    \sin(x_0+\pi/2)=0,
                \end{align}
            \end{subequations}
            avec \( x_0+\pi/2<\pi\). Cela contredirait la minimalité de \( \pi\).
        \item
            Il s'agit d'utiliser les formules \eqref{EQSooRJZGooCFVqbZ} :
            \begin{subequations}
                \begin{align}
                    \cos(\frac{ 3\pi }{ 2 })=\cos(\pi+\pi/2)=-\sin(\pi)=0\\
                    \sin(\frac{ 3\pi }{ 2 })=\sin(\pi+\pi/2)=\cos(\pi)=-1.
                \end{align}
            \end{subequations}
        \item
            Si \( \cos(x_0)=0\) alors \( \sin(x_0)=-1\) (parce que \( \sin(x_0)=1\) est déjà exclu). Alors \( \cos(x_0+\pi/2)=1\) et \( \sin(x_0+\pi/2)=0\), ce qui est également impossible.
        \item
        La fonction cosinus est continue (proposition \ref{PROPooZXPVooBjONka}) et \( \cos(0)=1\). Le théorème des valeurs intermédiaires implique que si \( \cos(x)\leq 0\), alors il existe \( t\in \mathopen] 0 , x \mathclose]\) avec \( \cos(x)=0\). Cela n'est pas possible pour \( x<\pi/2\) par par le point \ref{ITEMooMQQPooGwOdbt}.

        Le cosinus est positif sur l'intervalle considéré et \( \sin'(x)=\cos(x)\). Donc \( \sin(0)=0\) et la dérivée est positive. La proposition \ref{PropGFkZMwD} conclu que \( \sin\) est strictement croissante et donc strictement positive.
    \end{enumerate}
\end{proof}

\begin{corollary}   \label{CORooTFMAooHDRrqi}
    Des nombres \( x,y\in \eR\) vérifient \(  e^{ix}= e^{iy}\) si et seulement si il existe \( k\in\eZ\) tel que \( y=x+2k\pi\).
\end{corollary}

Tout cela nous permet de calculer quelques valeurs remarquables de cosinus et sinus ainsi que d'écrire le tableau de variations de sinus et cosinus.

\begin{lemma}       \label{LEMooIGNPooPEctJy}
    Nous avons les valeurs remarquables
    \begin{equation}
        \sin(\frac{ \pi }{ 4 })=\cos(\frac{ \pi }{ 4 })=\frac{ \sqrt{ 2 } }{2}.
    \end{equation}
\end{lemma}

\begin{proof}
    La relation \eqref{SUBEQooLRJDooQuFvux} donne
    \begin{equation}
        0=\cos(\pi/2)=\cos^2(\pi/4)-\sin^2(\pi/4).
    \end{equation}
    Donc \( \cos^2(\pi/4)=\sin^2(\pi/4)\). Mais vu que \( \sin(\pi/4)\) et \( \cos(\pi/4)\) sont positifs, ils sont égaux.

    Mais \( \sin^2(\pi/4)+\cos^2(\pi/4)=1\). Donc le nombre \( x=\cos(\pi/4)=\sin(\pi/4)\) vérifie l'équation \( 2x^2=1\), donc l'unique solution positive est \( x=\frac{1}{ \sqrt{ 2 } }=\frac{ \sqrt{ 2 } }{2}\).
\end{proof}

\begin{lemma}       \label{LEMooRMHAooDEAPMw}
    Nous avons la valeur remarquable
    \begin{equation}
        \cos(\frac{ \pi }{ 3 })=\frac{ 1 }{2}.
    \end{equation}
\end{lemma}

\begin{proof}
    Il faut utiliser la formule \eqref{EQooJYEMooQaOMib} avec \( \cos(\pi)=\cos(2\pi/3+\pi/3)\) en sachant que \( \cos(\pi)=-1\). Ensuite \( \cos(2\pi/3)=\cos(\pi/3+\pi/3)\). En décomposant ainsi, nous exprimons \( -1=\cos(\pi)\) en termes de \( \cos(\pi/3)\) et de \( \sin(\pi/3)\). En substituant \( \sin^2(\pi/3)=1-\cos^2(\pi/3)\) nous trouvons que le nombre \( \cos(\pi/3)\) vérifie l'équation
    \begin{equation}
        4x^3-3x+1=0.
    \end{equation}
    Croyez-le ou non, les solutions de cette équation sont \( x=-1\) et \( x=1/2\). Allez. Faisons comme si nous le savions pas. En tout cas, ces deux nombres sont des solutions, et nous avons la factorisation\footnote{Factorisation d'un polynôme en sachant des racines, proposition \ref{PropHSQooASRbeA}.}
    \begin{equation}
        4x^3-3x+1=(2x-1)^2(x+1).
    \end{equation}
    Donc \( 1/2\) est de multiplicité \( 2\) et \( -1\) de multiplicité \( 1\). Le théorème~\ref{ThoSVZooMpNANi} nous dit qu'il n'y a alors pas d'autres racines que ces deux-là\footnote{Nous attirons votre attention sur le fait que cela n'est en aucun cas une trivialité.}.

    Nous en déduisons que la valeur de \( \cos(\pi/3)\) est soit \( 1/2\) soit \( -1\). La proposition~\ref{PROPooMWMDooJYIlis}\ref{ITEMooHDQNooYHVCkg} nous dit qu'il est impossible que \( \cos(\pi/3)\) soit égal à \( -1\) parce que \( \pi/3<\pi\). Donc \( \cos(\pi/3)=1/2\) comme annoncé.
\end{proof}

\begin{remark}
    Vous avez déjà sans doute vu la démonstration de \( \cos(\unit{30}{\degree})=1/2\) à partir de la figure~\ref{LabelFigGVDJooYzMxLW}. Il n'est pas possible de l'utiliser parce que cela n'est en réalité pas loin d'être la définition de l'angle entre deux droites.

    Si vous voulez savoir la définition de l'angle entre deux droites, il faut passer par la définition~\ref{DEFooFLGNooCZUkHY}, laquelle se base sur le lemme~\ref{LEMooHRESooQTrpMz} qui, elle-même, se base sur la proposition~\ref{PROPooKSGXooOqGyZj}.

    Bref, à notre niveau, nous sommes encore loin de pouvoir faire des raisonnements trigonométriques sur base de géométrie dans les triangles.
\end{remark}

\begin{proposition}     \label{PROPooJFAGooYjRJcb}
    Pour tout \( x\in \mathopen[ 0 , \pi/4 \mathclose[\) nous avons \( \cos(x)>\sin(x)\).
\end{proposition}

\begin{proof}
    Nous posons \( f(x)=\cos(x)-\sin(x)\). Elle vérifie \( f(0)=1\). En utilisant les dérivées du lemme \ref{LEMooBBCAooHLWmno}, nous trouvons
    \begin{equation}
        f'(x)=-\big( \sin(x)+\cos(x) \big).
    \end{equation}
Mais sur \( \mathopen] 0 , \pi/2 \mathclose[\) nous avons \( \cos(x)>0\) et \( \sin(x)>0\) (proposition \ref{PROPooMWMDooJYIlis}\ref{ITEMooMEXUooGfSInJ}). Donc \( f\) est strictement décroissante. Elle ne peut donc passer qu'une seule fois par zéro. Le lemme \ref{LEMooIGNPooPEctJy} nous indique que \( f(\pi/4)=0\). Donc \( f(x)>0\) sur \( \mathopen[ 0 , \pi/4 \mathclose[\).
\end{proof}
<++>

\begin{proposition}
    Quelque valeurs trigonométriques.
    \begin{multicols}{2}
    \begin{enumerate}
        \item
            Pour le sinus :
            \begin{enumerate}
                \item
                    \( \sin(0)=0\)
                \item
                    \( \sin(\pi/6)=1/2\)
                \item
                    \( \sin(\pi/4)=\sqrt{ 2 }/2\)
                \item
                    \( \sin(\pi/3)=\sqrt{ 3 }/2\)
                \item
                    \( \sin(\pi/2)=1\)
            \end{enumerate}
            
        \item
            Pour le cosinus :
            \begin{enumerate}
                \item
                    \( \cos(0)=1\)
                \item
                    \( \cos(\pi/6)=\sqrt{ 3 }/2\)
                \item
                    \( \cos(\pi/4)=\sqrt{ 2 }/2\)
                \item
                    \( \cos(\pi/3)=1/2\)
                \item
                    \( \cos(\pi/2)=0\)
            \end{enumerate}
        \item
            Pour la tangente :
            \begin{enumerate}
                \item
                    \( \tan(0)=0\)
                \item
                    \( \tan(\pi/6)=\sqrt{ 3 }/3\)
                \item
                    \( \tan(\pi/4)=1\)
                \item
                    \( \tan(\pi/3)=\sqrt{ 3 }\)
                \item
                    \( \tan(\pi/2)\) est non défini.
            \end{enumerate}
    \end{enumerate}
\end{multicols}
\end{proposition}

\begin{proof}
    Plusieurs ont déjà été faites. Les autres ne seront pas démontrées dans l'ordre énoncé.
    \begin{subproof}
        \item[$\sin(0)=0$]
            Substitution dans la définition \eqref{EQooCMRFooCTtpge}.
        \item[$ \sin(\pi/4)=\sqrt{ 2 }/2$] 
            C'est le lemme \ref{LEMooIGNPooPEctJy}.
        \item[$ \sin(\pi/3)=1/\sqrt{ 2 }$] 
            Nous utilisons la formule \( \sin^2(x)+\cos^2(x)=1\) avec \( x=\pi/3\). Cela donne \( \sin^2(\pi/3)=1/2\). Nous en déduisons que \( \sin(\pi/3)\) vaut \( \pm\frac{1}{ \sqrt{ 2 } }\).

            La proposition \ref{PROPooMWMDooJYIlis}\ref{ITEMooHDQNooYHVCkg} nous dit que \( \sin\) est positive sur \(\mathopen[ 0 , \pi \mathclose]\). Donc c'est bien la possibilité \( 1/\sqrt{ 2 }\) qui est la bonne.
        \item[$\sin(\pi/6)=1/2$ et $\cos(\pi/6)=\sqrt{ 3 }/2 $]
            Nous partons de l'équation \eqref{SUBEQooLRJDooQuFvux} pour écrire
            \begin{equation}
                \sin(\pi/3)=2\cos(\pi/6)\sin(\pi/6).
            \end{equation}
            Nous avons déjà vu que \( \sin(\pi/3)=\sqrt{ 3 }/2\). En posant \( x=\sin(\pi/6)\) nous avons également \( \cos(\pi/6)=\sqrt{ 1-x^2 }\) parce que nous savons que la fonction cosinus est positive sur \( \mathopen[ 0 , \pi/2 \mathclose]\) (proposition \ref{PROPooMWMDooJYIlis}\ref{ITEMooMEXUooGfSInJ}). Nous avons donc l'équation
            \begin{equation}
                \frac{ \sqrt{ 3 } }{2}=2x\sqrt{ 1-x^2 }.
            \end{equation}
            Nous passons au carré et posons \( y=x^2\). Après quelque manipulations, 
            \begin{equation}
                16y^2-16y+3=0.
            \end{equation}
            Cela donne deux possibilités pour \( y\) : \( \frac{ 3 }{ 4 }\) et \( \frac{1}{ 4 }\). Vu que \( x>0\), nous pouvons simplement passer à la racine carré : \( x=\sqrt{ 3 }/2\) ou \( x=1/2\).

            Notez que si nous avion posé \( x=\cos(\pi/6)\) au lieu de \( x=\sin(\pi/6)\), nous aurions obtenu le même résultat. Donc \( \sin(\pi/6)\) et \( \cos(\pi/6)\) peuvent tout deux avoir les valeurs \( \sqrt{ 3 }/2\) ou \( 1/2\). Cela fait \( 4\) possibilités.

            Étant donné que \( \sin^2(\pi/6)+\cos^2(\pi/6)=1\), les deux possibilités avec \( \sin(\pi/6)=\cos(\pi/6)\) sont exclues.

            La proposition \ref{PROPooJFAGooYjRJcb} nous dit aussi que \( \cos(\pi/6)>\sin(\pi/6)\). Donc \( \cos(\pi/6)=\sqrt{ 3 }/2\) et \( \sin(\pi/6)=1/2\).
        \item[$\sin(\pi/2)=1 $] C'est dans \eqref{SUBEQSooBTNPooSvCAHO}.
        \item[$\cos(0)=1 $] Substitution dans la définition.
        \item[$\cos(\pi/6)=\sqrt{ 3 }/2 $] Déjà fait avec le sinus de \( \pi/6\).
        \item[$\cos(\pi/4)=\sqrt{ 2 }/2 $]  Lemme \ref{LEMooIGNPooPEctJy}.
        \item[$\cos(\pi/3)=1/2 $] Lemme \ref{LEMooRMHAooDEAPMw}.
        \item[$\cos(\pi/2)=0 $] Dans \eqref{SUBEQSooBTNPooSvCAHO}.
    \end{subproof}
    Toutes les valeurs pour la tangente s'obtiennent maintenant par la définition en calculant \( \tan(x)=\frac{ \sin(x) }{ \cos(x) }\).
\end{proof}

Voici un tableau qui rappelle les valeurs à retenir pour les fonctions sinus, cosinus et tangente.
\begin{equation}\label{PGooIMQFooTnBdIl}
    \begin{array}[]{|c|c|c|c|}
      \hline
      x&\sin(x)&\cos(x)&\tan(x)\\
      \hline
      0&0&1&0\\
      \hline
      \pi/6&1/2&\sqrt{3}/2&\sqrt{3}/3\\
      \hline
      \pi/6&1/2&\sqrt{3}/2&\sqrt{3}/3\\
      \hline
      \pi/4&\sqrt{2}/2&\sqrt{2}/2&1\\
      \hline
      \pi/3&\sqrt{3}/2&1/2&\sqrt{3}\\
      \hline
      \pi/2&1&0&\text{N.D.}\\
      \hline
    \end{array}
\end{equation}
où «N.D.»  signifie «non défini».

Rappelons le graphe de la fonction sinus :
\begin{center}
   \input{auto/pictures_tex/Fig_TWHooJjXEtS.pstricks}
\end{center}
celui de la fonction cosinus :
\begin{center}
   \input{auto/pictures_tex/Fig_JJAooWpimYW.pstricks}
\end{center}


\begin{lemma}
  Pour toute valeur de $x\in \eR$ on a $|\sin(x)|\leq |x|$.
\end{lemma}

\begin{proof}
        Nous séparons des cas en fonction des valeurs.
    \begin{itemize}
    \item Si $0\leq x\leq \pi/2$ alors le sinus de $x$ est la longueur du cathète verticale du triangle rectangle de sommets $O = (0,0)$, $A = (\cos(x), \sin(x))$ et $B = (\cos(x), 0)$. Le triangle de sommets $A$, $B$ et $C = (1, 0)$ est aussi rectangle et nous savons que chacun des cathètes ne peut pas être plus long que l'hypoténuse. Donc $\sin(x)$ est inférieur à la longueur du segment $AC$. À son tour le segment $AC$ ne peut pas être plus long que l'arc de cercle $\wideparen{A0C}$, car le chemin le plus court entre deux points du plan est toujours donné par un morceau de droite. La longueur de l'arc de cercle $\frown{AC}$ est \emph{par définition} la mesure en radiants de l'angle $\widehat{AOC}$, qui est $x$ et on a l'inégalité $\sin(x)\leq x$.
    \item Si $-\pi/2\leq x\leq 0$ le m\^eme raisonnement que au point précédent permet de conclure que $\sin(x)\leq |x|$.
    \item Nous savons par ailleurs que la fonction sinus prend ses valeurs dans l'intervalle $[-1,1]$ et donc pour tout $x$ tel que $|x|\geq \pi/2 \equiv 1,57\ldots$ on a forcement $|\sin(x)|\leq |x|$.
    \end{itemize}
\end{proof}

\begin{subequations}
    \begin{numcases}{}
        x=r\cos\theta\\
        y=r\sin\theta
    \end{numcases}
\end{subequations}
avec \( r\in\mathopen] 0 , \infty \mathclose[\) et \( \theta\in\mathopen[ 0 , 2\pi [\). Le jacobien vaut \( r\).

\begin{example}\label{developcosenpisur3}
    Développer la fonction \( \cos\) autour de \( x=\frac{ \pi }{ 3 }\). Utiliser la valeur remarquable du lemme \ref{LEMooRMHAooDEAPMw}. Nous développons autour de \( h=0\) la fonction \( \cos(\frac{ \pi }{ 3 }+h)\) :
    \begin{equation}
        \cos\big( \frac{ \pi }{ 3 }+h \big)\sim \cos\big( \frac{ \pi }{ 3 } \big)+h\cos'(\frac{ \pi }{ 3 })+\frac{ h^2 }{2}\cos''\big( \frac{ \pi }{ 3 } \big)=\frac{ 1 }{2}-\frac{ \sqrt{3} }{2}h-\frac{1}{ 4 }h^2.
    \end{equation}
    Il est aussi possible d'écrire cela en notant \( x=x_0+h\), c'est-à-dire en remplaçant \( h\) par \( x-\frac{ \pi }{ 3 }\) :
    \begin{equation}
        \cos(x)\sim\frac{ 1 }{2}-\frac{ \sqrt{3} }{ 2 }(x-\frac{ \pi }{ 3 })-\frac{1}{ 4 }(x-\frac{ \pi }{ 3 })^2.
    \end{equation}
\end{example}

Pour donner une idée nous avons dessiné sur le graphe suivant la fonction sinus et ses développements d'ordre \( 4\) autour de zéro et autour de \( 3\pi/4\).
\begin{center}
   \input{auto/pictures_tex/Fig_WJBooMTAhtl.pstricks}
\end{center}

% This is part of (everything) I know in mathematics
% Copyright (c) 2011-2019
%   Laurent Claessens
% See the file fdl-1.3.txt for copying conditions.

%+++++++++++++++++++++++++++++++++++++++++++++++++++++++++++++++++++++++++++++++++++++++++++++++++++++++++++++++++++++++++++ 
\section{Très modeste approximation de \texorpdfstring{$ \pi$}{pi}}
%+++++++++++++++++++++++++++++++++++++++++++++++++++++++++++++++++++++++++++++++++++++++++++++++++++++++++++++++++++++++++++

Nous sommes en droit de vouloir une valeur approchée de \( \pi\).
\begin{lemma}       \label{LEMooJWSGooExmtDA}
    Nous avons l'approximation numérique
    \begin{equation}
        2\sqrt{ 2 }<\pi<4.
    \end{equation}
\end{lemma}

\begin{proof}
    Grace au lemme~\ref{LEMooIGNPooPEctJy} nous savons que la fonction \( \sin\) passe de \( 0\) à \( \sqrt{ 2 }/2\) sur un intervalle de taille \( \pi/4\) avec une dérivé majorée par \( 1\). Par conséquent
    \begin{equation}
        \frac{ \pi }{ 4 }>\frac{ \sqrt{ 2 } }{2}
    \end{equation}
    et donc\footnote{Sérieusement, êtes vous capables de trouver une approximation de \( \sqrt{ 2 }\) en ne vous basant que sur des choses vues jusqu'ici ?}
    \begin{equation}
        \pi>2\sqrt{ 2 }\simeq 2.82
    \end{equation}
    De plus la fonction \( \sin\) passe de \( 0\) à \( \sqrt{ 2 }/2\) sur un intervalle de taille \( \pi/4\) avec une dérivée majorée par \( \sqrt{ 2 }/2\), donc
    \begin{equation}
        \frac{ \pi }{ 4 }<\frac{ \sqrt{ 2 }/2 }{ \sqrt{ 2 }/2 },
    \end{equation}
    ce qui donne
    \begin{equation}
        \pi<4.
    \end{equation}
\end{proof}

Pour avoir une meilleur approximation de \( \pi\), nous pouvons remarquer que \( \pi\in\mathopen] 2.82 , 4 \mathclose[\), et que cet intervalle est suffisamment petit pour ne pas recouvrir l'intervalle correspondant pour \( 2\pi\). L'équation \( \cos(x)=-1\) possède donc une unique solution dans cet intervalle (et cette solution est \( \pi\)). Nous pouvons donc faire une dichotomie pour trouver la valeur de \( \pi\), pourvu que nous ayons une façon d'évaluer des valeurs de \( \cos(x)\) de façon pas trop ridicule.

%+++++++++++++++++++++++++++++++++++++++++++++++++++++++++++++++++++++++++++++++++++++++++++++++++++++++++++++++++++++++++++
\section{Cercle trigonométriques}
%+++++++++++++++++++++++++++++++++++++++++++++++++++++++++++++++++++++++++++++++++++++++++++++++++++++++++++++++++++++++++++

\begin{proposition}[\cite{ooIEJXooIYpBbd}]      \label{PROPooWZFGooMVLtFz}
    Soient des fonctions \( f,g\colon I\to \eR\) de classe \(  C^{1}\) sur l'ouvert \( I\) de \( \eR\) telles que \( f^2+g^2=1\). Soit \( t_0\in I\) et \( \theta_0\) tel que \( f(t_0)=\cos(\theta_0)\) et \( g(t_0)=\sin(\theta_0)\).

    Alors il existe une unique fonction continue \( \theta\colon I\to \eR\) telle que
    \begin{subequations}
        \begin{numcases}{}
            \theta(t_0)=\theta_0\\
            f=\cos\circ \theta\\
            g=\sin\circ \theta.
        \end{numcases}
    \end{subequations}
\end{proposition}

\begin{proof}
    Nous commençons par l'existence, en passant par les nombres complexes. Soit \( h\colon I\to \eC\) définie par \( h=f+ig\). Nous avons \( h\bar h=1\) et nous définissons
    \begin{equation}
        \theta(t)=\theta_0-i\int_{t_0}^th'(s)\overline{ h(s) }ds.
    \end{equation}
    Cette intégrale existe pour tout \( t\) parce que les fonctions \( f\) et \( g\) étant de classe \(  C^{\infty}\), elles sont bornées sur le compact \( \mathopen[ t_0 , t  \mathclose]\). De plus \( \theta\) est une fonction continue parce que c'est une primitive (proposition~\ref{PropEZFRsMj})\footnote{En réalité nous appliquons la proposition~\ref{PropEQRooQXazLz} à chacune des parties réelles et imaginaires de la fonction $s\mapsto h'(s)\overline{ h(s) }$.}.

    La dérivée de \( \theta\) est la fonction \( s\mapsto -i h'(s)\overline{ h(s) }\).

    Utilisant la formule du lemme~\ref{LEMooHOYZooKQTsXW} sur la forme trigonométrique des nombres complexes, nous calculons :
    \begin{equation}
        \Dsdd{ h e^{-i\theta} }{t}{0}= e^{-i\theta}(h'-h\theta')= e^{-i\theta}(h'-ih(-i)h'\bar h)=0.
    \end{equation}
    Par conséquent il existe \( c\in \eC\) tel que \( h e^{-i\theta}=c\). Mais \( h(t_0)=f(t_0)+ig(t_0)=\cos(\theta_0)+i\sin(\theta_0)= e^{i\theta_0}\), du coup
    \begin{equation}
        h(t_0) e^{-i\theta(t_0)}=c
    \end{equation}
    donne immédiatement \( c=1\), ou encore \(  e^{i\theta(t)}=h(t)\), c'est-à-dire que
    \begin{equation}
        f+ig=\cos\circ\theta+i\sin\circ\theta,
    \end{equation}
    ce qu'il fallait pour l'existence.

    Pour l'unicité nous supposons avoir une autre fonction, \(\alpha\) qui satisfait aux exigences. Pour tout \( t\in I\) nous avons
    \begin{equation}
        e^{i\theta(t)}= e^{i\alpha(t)}.
    \end{equation}
    Il existe donc une fonction \( n\colon I\to \eN\) telle que \( \theta(t)=\alpha(t)+2n(t)\pi\). Par continuité de \( \theta\) et \( \alpha\), la fonction \( n\) doit être constante, mais vu que \( \theta(t_0)=\alpha(t_0)\) nous avons \( n=1\).
\end{proof}


%---------------------------------------------------------------------------------------------------------------------------
\subsection{Les fonctions tangente et arc tangente}
%---------------------------------------------------------------------------------------------------------------------------

\begin{definition}
    La fonction \defe{tangente}{tangente} est :
    \begin{equation}
        \tan(x)=\frac{ \sin(x) }{ \cos(x) }
    \end{equation}
    où \( \sin\) et \( \cos\) sont de la définition~\ref{PROPooZXPVooBjONka}.
\end{definition}
La fonction tangente n'est pas définie sur les points de la forme \( x=\frac{ \pi }{2}+k\pi\), \( k\in \eZ\). Une interprétation géométrique, qui justifie le nom, est donnée sur la figure~\ref{LabelFigTgCercleTrigono}.
\newcommand{\CaptionFigTgCercleTrigono}{Interprétation géométrique de la fonction tangente. La tangente de l'angle $\theta$ est positive (et un peu plus grande que $1$) tandis que celle de la tangente de l'angle $\varphi$ est négative.}
\input{auto/pictures_tex/Fig_TgCercleTrigono.pstricks}

\begin{proposition}
    La fonction
    \begin{equation}
        \begin{aligned}
        \tan\colon \mathopen] -\frac{ \pi }{ 2 } , \frac{ \pi }{2} \mathclose[&\to \eR \\
            x&\mapsto \tan(x)
        \end{aligned}
    \end{equation}
    est une bijection.
\end{proposition}

\begin{proof}
    Le cosinus ne s'annulant pas sur l'intervalle donné, la fonction est bien définie. Nous avons
    \begin{equation}
        \lim_{x\to \pi/2^-} \tan(x)=+\infty
    \end{equation}
    parce que la limite du sinus est \( 1\) est celle du cosinus est zéro par les valeurs positives. Le même raisonnement donne la limite en \( -\pi/2\) qui vaut \( -\infty\). Le théorème des valeurs intermédiaires\footnote{Théorème~\ref{ThoValInter}.} dit que la fonction tangente est alors surjective sur \( \eR\).

    Par ailleurs en utilisant les règles de calcul comme la dérivation du quotient~\ref{PROPooOUZOooEcYKxn}\ref{ITEMooMUNQooLiKffz} nous trouvons
    \begin{equation}
        \tan'(x)=\tan^2(x)+1,
    \end{equation}
    ce qui nous donne une dérivée partout strictement positive, et donc une fonction strictement croissante et donc injective.
\end{proof}

Le graphe de la fonction tangente est sur la figure~\ref{LabelFigPVJooJDyNAg}. % From file PVJooJDyNAg
\newcommand{\CaptionFigPVJooJDyNAg}{Le graphe de la fonction tangente.}
\input{auto/pictures_tex/Fig_PVJooJDyNAg.pstricks}

En ce qui concerne la bijection réciproque nous avons le théorème suivant.
\begin{theorem}     \label{THOooUSVGooOAnCvC}
    La fonction inverse de la tangente,
    \begin{equation}
        \begin{aligned}
        \arctan\colon \eR&\to \left] -\frac{ \pi }{2} , \frac{ \pi }{2} \right[ \\
            x&\mapsto \arctan(x)
        \end{aligned}
    \end{equation}
    nommée \defe{arc tangente}{arc tangente} est
    \begin{enumerate}
        \item
            impaire et strictement croissante sur \( \eR\).
        \item       \label{ITEMooMNHLooOVhIIb}
            dérivable sur \( \eR\) de dérivée
            \begin{equation}        \label{EQooGCHGooPlwYWt}
                \arctan'(x)=\frac{1}{ 1+x^2 }.
            \end{equation}
    \end{enumerate}
\end{theorem}

\begin{proof}
    Il est immédiatement visible sur son développement de définition \eqref{EQooCMRFooCTtpge} que la fonction sinus est impaire. Une vérification similaire montre que la fonction cosinus est paire. La fonction tangente est alors impaire et sa réciproque l'est tout autant.

    La fonction arc tangente est également dérivable (donc continue) par la proposition~\ref{PropMRBooXnnDLq} parce que la fonction tangente l'est. Notons qu'ici nous nous sommes restreint à \( \mathopen] -\pi/2 , \pi/2 \mathclose[\). Sinon, le résultat est faux.

    La formule proposée pour la dérivée provient également de la proposition~\ref{PropMRBooXnnDLq} et de la dérivée de la tangente :
\end{proof}

\begin{lemma}       \label{LEMooHRDCooGtnyeQ}
    Nous avons les limites
    \begin{enumerate}
        \item
            $\lim_{x\to \infty} \arctan(x)=\frac{ \pi }{2}$,
        \item
            \( \lim_{x\to -\infty} \arctan(x)=-\frac{ \pi }{2}\).
    \end{enumerate}
\end{lemma}

\begin{lemma}       \label{LEMooJKIUooEMMOrs}
    Nous avons la valeur remarquable
    \begin{equation}
        \arctan(1/\sqrt{ 3 })=\frac{ \pi }{ 6 }.
    \end{equation}
\end{lemma}

Le nombre \( \arctan(x_0)\) se calcule en cherchant l'angle \( \theta\in\mathopen[ -\frac{ \pi }{2} , \frac{ \pi }{2} \mathclose]\) dont la tangente vaut \( x_0\). Nous obtenons le tableau de valeurs suivant :

\begin{lemma}       \label{LEMooPQNCooDkEUyw}
    Quelques valeurs remarquables de l'arc tangente :
\begin{equation}
    \begin{array}[]{|c|c|c|c|c|}
        \hline
        x&0&\frac{1}{ \sqrt{3} }&1&\sqrt{3}\\
        \hline
        \arctan(x)&0&\frac{ \pi }{ 6 }&\frac{ \pi }{ 4 }&\frac{ \pi }{ 3 }\\
        \hline
    \end{array}
\end{equation}
\end{lemma}

En ce qui concerne la représentation graphique de la fonction \( x\mapsto\arctan(x)\), elle s'obtient «en retournant» la partie entre \( -\frac{ \pi }{2}\) et \( \frac{ \pi }{ 2 }\) du graphique de la fonction tangente :
\begin{center}
   \input{auto/pictures_tex/Fig_UQZooGFLNEq.pstricks}
\end{center}

%---------------------------------------------------------------------------------------------------------------------------
\subsection{La fonction arc sinus}
%---------------------------------------------------------------------------------------------------------------------------

Nous voulons étudier la fonction
\begin{equation}
    \begin{aligned}
        \sin\colon \eR&\to \mathopen[ -1 , 1 \mathclose] \\
        x&\mapsto \sin(x)
    \end{aligned}
\end{equation}
et sa réciproque éventuelle.

La fonction sinus est continue sur \( \eR\) mais n'est pas bijective : elle prend une infinité de fois chaque valeur de \( J=\mathopen[ -1 , 1 \mathclose]\). Pour définir une bijection réciproque de la fonction sinus en utilisant le théorème~\ref{ThoKBRooQKXThd}, nous devons donc choisir un intervalle à partir duquel la fonction sinus est monotone. Nous choisissons l'intervalle
\begin{equation}
    I=\mathopen[ -\frac{ \pi }{ 2 } , \frac{ \pi }{2} \mathclose].
\end{equation}
La fonction
\begin{equation}
    \begin{aligned}
        \sin\colon \mathopen[ -\frac{ \pi }{2} , \frac{ \pi }{2} \mathclose]&\to \mathopen[ -1 , 1 \mathclose] \\
        x&\mapsto \sin(x)
    \end{aligned}
\end{equation}
est une bijection croissante et continue. Nous avons donc le résultat suivant.
\begin{theorem}[Définition et propriétés de arc sinus]
    Nous nommons \defe{arc sinus}{arc sinus} la bijection inverse de la fonction \( \sin\colon I\to J\). La fonction
    \begin{equation}
        \begin{aligned}
            \arcsin\colon \mathopen[ -1 , 1 \mathclose]&\to \mathopen[ -\frac{ \pi }{2} , \frac{ \pi }{2} \mathclose] \\
            x&\mapsto \arcsin(x)
        \end{aligned}
    \end{equation}
    ainsi définie est
    \begin{enumerate}
        \item
            continue et strictement croissante;
        \item
            impaire : pour tout \( x\in\mathopen[ -1 , 1 \mathclose]\) nous avons \( \arcsin(-x)=-\arcsin(x)\).
    \end{enumerate}
\end{theorem}

\begin{proof}
    Nous prouvons le fait que \( \arcsin\) est impaire. Un élément de l'ensemble de définition de \( \arcsin\) est de la forme \( y=\sin(x)\) avec \( x\in\mathopen[ -\pi/2 , \pi/2 \mathclose]\). La relation \eqref{EqHQRooNmLYbF} s'écrit dans notre cas
    \begin{equation}    \label{EqVUWooUwVxVp}
        x=\arcsin\big( \sin(x) \big).
    \end{equation}
    Nous écrivons d'une part cette équation avec \( -x\) au lieu de \( x\) :
    \begin{equation}    \label{EqRLYooIwOvSz}
        -x=\arcsin\big( \sin(-x) \big)=\arcsin\big( -\sin(x) \big)=\arcsin(-y);
    \end{equation}
    et d'autre part nous multiplions \eqref{EqVUWooUwVxVp} par \( -1\) :
    \begin{equation}    \label{EqTGIooDeRYyT}
        -x=-\arcsin\big( \sin(x) \big)=-\arcsin(y).
    \end{equation}
    En égalisant les valeurs \eqref{EqRLYooIwOvSz} et \eqref{EqTGIooDeRYyT} nous trouvons
    \begin{equation}
        \arcsin(-y)=-\arcsin(y),
    \end{equation}
    ce qui signifie que \( \arcsin\) est une fonction impaire.
\end{proof}
Notons que cette preuve repose sur le fait que tout élément de l'ensemble de définition de la fonction arc sinus peut être écrit sous la forme \( \sin(x)\) pour un certain \( x\).

Si \( x_0\in\mathopen[ -1 , 1 \mathclose]\) est donné, calculer \( \arcsin(x_0)\) revient à trouver un angle \( \theta_0\) dans \( \mathopen[ -\frac{ \pi }{2} , \frac{ \pi }{2} \mathclose]\) pour lequel \( \sin(\theta_0)=x_0\). Un tel angle sera forcément unique.

\begin{remark}
  La définition de arc sinus découle du choix de l'intervalle $I$, qui est une convention. Il aurait été possible de faire un choix différent : pourriez-vous trouver la réciproque de la fonction sinus sur l'intervalle $[\pi/2, 3\pi/2]$ ? Le mieux est de l'écrire comme une translatée de arc sinus, en utilisant le fait que sinus est une fonction périodique.
\end{remark}

\begin{example}
    Pour calculer \( \arcsin(1)\), il faut chercher un angle entre \( -\frac{ \pi }{2}\) et \( \frac{ \pi }{ 2 }\) ayant \( 1\) pour sinus : résoudre \( \sin(\theta)=1\). La solution est \( \theta=\frac{ \pi }{2}\) et nous avons donc \( \arcsin(1)=\frac{ \pi }{2}\).
\end{example}

À l'aide des valeurs remarquables de la fonction sinus nous obtenons le tableau suivant de valeurs remarquables pour l'arc sinus.
\begin{equation*}
    \begin{array}[]{|c|c|c|c|c|c|}
        \hline
        x&0&\frac{ 1 }{2}&\frac{ \sqrt{2} }{2}&\frac{ \sqrt{3} }{2}&1\\
          \hline
          \arcsin(x)&0&\frac{ \pi }{ 6 }&\frac{ \pi }{ 4 }&\frac{ \pi }{ 3 }&\frac{ \pi }{ 2 }\\
          \hline
           \end{array}
\end{equation*}
Les autres valeurs remarquables peuvent être déduites du fait que l'arc sinus est une fonction impaire.

En ce qui concerne la dérivabilité de la fonction arc sinus, en application de la proposition~\ref{PropMRBooXnnDLq} elle est dérivable en tout \( y=\sin(x)\) tel que \( \sin'(x)\neq 0\), c'est-à-dire tel que \( \cos(x)\neq 0\). Or \( \cos(x)=0\) pour \( x=\pm\frac{ \pi }{2}\), ce qui correspond à \( y=\sin(\pm\frac{ \pi }{2})=\pm 1\). La fonction arc sinus est donc dérivable sur \( \mathopen] -1 , 1 \mathclose[\). Nous avons donc la propriété suivante pour la dérivabilité.

\begin{proposition}
    La fonction arc sinus est continue sur \( \mathopen[ -1 , 1 \mathclose]\) et dérivable sur \( \mathopen] -1 , 1 \mathclose[\). Pour tout \( y\in\mathopen] -1 , 1 \mathclose[\), la dérivée est donnée par la formule \eqref{EqWWAooBRFNsv}, qui dans ce cas s'écrit
        \begin{equation}
            \arcsin'(y)=\frac{1}{ \cos\big( \arcsin(y) \big) }=\frac{1}{ \sqrt{1-y^2} }.
        \end{equation}
\end{proposition}
La dernière égalité viens du fait que si $x=\arcsin(y)$ alors $y = \sin(x)$ et $\cos(x)= \sqrt{1-\sin^2(x)} = \sqrt{1-y^2}$.

Pour comprendre la dernière égalité, remarquer que dans le dessin suivant, \( \theta=\arcsin(y)\), donc $y = \sin(\theta)$, et \( x=\cos(\theta)\).
\begin{center}
    \input{auto/pictures_tex/Fig_BIFooDsvVHb.pstricks}
\end{center}

Notons enfin que le graphe de la fonction arc sinus est donné à la figure~\ref{LabelFigFGRooDhFkch}. % From file FGRooDhFkch
\newcommand{\CaptionFigFGRooDhFkch}{Le graphe de la fonction \( x\mapsto \arcsin(x)\)}
\input{auto/pictures_tex/Fig_FGRooDhFkch.pstricks}

%---------------------------------------------------------------------------------------------------------------------------
\subsection{La fonction arc cosinus}
%---------------------------------------------------------------------------------------------------------------------------

Nous voulons étudier la fonction
\begin{equation}
        \cos\colon \eR\to \mathopen[ -1 , 1 \mathclose]
\end{equation}
et son éventuelle réciproque. Encore une fois il n'est pas possible d'en prendre la réciproque globale parce que ce n'est pas une bijection; ne fut-ce que parce qu'elle est périodique (proposition~\ref{PROPooFRVCooKSgYUM}). Nous choisissons de considérer l'intervalle \( \mathopen[ 0 , \pi \mathclose]\) sur lequel la fonction cosinus est continue et strictement monotone décroissante.

Nous avons alors le résultat suivant :

\begin{propositionDef}     \label{PROPooZOZHooSMoYQD}
    Pour définir la fonction arcsinus.

    \begin{enumerate}
        \item
    La fonction
    \begin{equation}
            \cos\colon \mathopen[ 0 , \pi \mathclose]\to \mathopen[ -1 , 1 \mathclose]
    \end{equation}
    est une bijection continue strictement décroissante.
    \item
    Sa bijection réciproque est la fonction
    \begin{equation}
            \arccos\colon \mathopen[ -1 , 1 \mathclose]\to \mathopen[ 0 , \pi \mathclose] \\
    \end{equation}
    nommée \defe{arc cosinus}{arc cosinus}.
    \item
        La fonction arc cosinus est continue, strictement décroissante.
    \item
        Elle est dérivable et pour tout \( y\in\mathopen] -1 , 1 \mathclose[\), sa dérivée est donnée par
    \begin{equation}
        \arccos'(y)=\frac{1}{ -\sin\big( \arccos(y) \big) }=\frac{ -1 }{ \sqrt{1-y^2} }.
    \end{equation}
    \end{enumerate}
\end{propositionDef}

\begin{proof}
    La fonction cosinus est continue et même de classe \(  C^{\infty}\) par la proposition~\ref{PROPooZXPVooBjONka}. Elle est strictement décroissant parce que sa dérivée (\( -\sin\)) y est strictement positive (strictement à dans l'intérieur du domaine).

    Le fait que arc cosinus soit une bijection continue strictement monotone est dans le théorème de la bijection~\ref{ThoKBRooQKXThd}. La dérivabilité et la formule sont de la proposition~\ref{PropMRBooXnnDLq}.
\end{proof}

Pour \( y_0\in\mathopen[ -1 , 1 \mathclose]\), trouver la valeur de \( \arccos(y_0)\) revient à résoudre l'équation \( \cos(x_0)=y_0\). Cela nous permet de construire une tableau de valeurs :
\begin{equation*}
    \begin{array}[]{|c|c|c|c|c|c|c|c|c|c|}
        \hline
        x&-1&-\frac{ \sqrt{3} }{2}&-\frac{ \sqrt{2} }{2}&-\frac{ 1 }{2}&0&\frac{ 1 }{2}&\frac{ \sqrt{2} }{2}&\frac{ \sqrt{3} }{2}&1\\
          \hline
          \arccos(x)&\pi&\frac{ 5\pi }{ 6 }&\frac{ 3 }{ 4 }\pi&\frac{ 2 }{ 3 }\pi&\frac{ 1 }{2}\pi&\frac{ \pi }{ 3 }&\frac{1}{ 4 }\pi&\frac{1}{ 6 }\pi&0\\
          \hline
           \end{array}
\end{equation*}

\begin{remark}
    Certes la fonction cosinus est paire (vue sur \( \eR\)), mais la fonction arc cosinus ne l'est pas car elle est une bijection entre \(\mathopen[ -1 , 1 \mathclose]\) et \(\mathopen[ 0 , \pi \mathclose]\).
\end{remark}

\begin{example}
    Cherchons \( \arccos(\frac{ 1 }{2})\). Il faut trouver un angle \( \theta\in\mathopen[ 0 , \pi \mathclose]\) tel que \( \cos(\theta)=\frac{ 1 }{2}\). La solution est \( \theta=\frac{ \pi }{ 3 }\). Donc \( \arccos(\frac{ 1 }{2})=\frac{ \pi }{ 3 }\).

    Il n'est cependant pas immédiat d'en déduire la valeur de \( \arccos(-\frac{ 1 }{2})\). En effet \( \theta=\arccos(-\frac{ 1 }{2})\) si et seulement si \( \cos(\theta)=-\frac{ 1 }{2}\) avec \( \theta\in\mathopen[ 0 , \pi \mathclose]\). La solution est \( \theta=\frac{ 2\pi }{ 3 }\).
\end{example}

En ce qui concerne la représentation graphique, il suffit de tracer la fonction cosinus entre \( 0\) et \( \pi\) puis de prendre le symétrique par rapport à la droite \( y=x\).

\begin{center}
    \input{auto/pictures_tex/Fig_GMIooJvcCXg.pstricks}
\end{center}

%---------------------------------------------------------------------------------------------------------------------------
\subsection{Une meilleure approximation de \( \pi\)}
%---------------------------------------------------------------------------------------------------------------------------

Nous avions laissé le nombre \( \pi\) avec l'approximation assez minable de \( 2\sqrt{ 2 }<\pi<4\) en le lemme~\ref{LEMooJWSGooExmtDA}. Nous pouvons maintenant faire nettement mieux.

Le lemme~\ref{LEMooJKIUooEMMOrs} donne
\begin{equation}
    \arctan(1/\sqrt{ 3 })=\pi/6
\end{equation}
et l'idée est de donner un développement de \( \arctan\) autour de zéro, de l'évaluer en \( 1/\sqrt{ 3 }\) et d'égaliser le résultat à \( \pi/6\). Tout cela donne lieu à des calcules peut-être fastidieux, mais comme un gars l'a fait dès l'an 1424\cite{ooOMUNooGROVUu} pour trouver \( 16\) décimales correctes, nous faisons comme si c'était facile.

Pour trouver le développement en série de Taylor (théorème~\ref{ThoTaylor}) de arc tangente autour de \( x=0\), il faut partir de la formule \eqref{EQooGCHGooPlwYWt} et sans doute pas mal calculer et faire une récurrence\quext{Je n'ai pas fait le calcul, merci de me faire savoir si il y a une astuce.}. Le résultat est :
\begin{equation}
    \arctan(x)=\sum_{k=0}^{\infty}\frac{ (-1)^{k}x^{2k+1} }{ 2k+1 },
\end{equation}
valable pour \( x\in \mathopen] -1 , 1 \mathclose[\). Avec cela nous avons
\begin{equation}
    \arctan(\frac{1}{ \sqrt{ 3 } })=\sum_{k=0}^{\infty}\frac{ (-1)^k }{ (2k+1)3^k }\times \frac{1}{ \sqrt{ 3 } }=\frac{ \pi }{ 6 },
\end{equation}
et donc
\begin{equation}
    \pi=\frac{ 6 }{ \sqrt{ 3 } }\sum_{k=0}^{\infty}\frac{ (-1)k }{ (2k+1)3^k }.
\end{equation}

Pour donner une idée du fait que ça fonctionne pas mal, voici le calcul pour quelques termes :
\lstinputlisting{tex/sage/sageSnip012.sage}
Calculer \( 5\) termes donne déjà \( 3.15\). Et on est à \( 10^{-6}\) de la bonne réponse avec \( 20\) termes. Et avec $58$ termes, on n'est à \( 10^{-16}\).

\begin{probleme}
    Pour bien faire, il faudrait étudier le reste et donner un encadrement.
\end{probleme}

%---------------------------------------------------------------------------------------------------------------------------
\subsection[Forme trigonométrique des nombres complexes]{Forme polaire ou trigonométrique des nombres complexes}
%---------------------------------------------------------------------------------------------------------------------------

Un nombre complexe étant représenté par deux nombres, on peut le représenter dans un plan appelé « plan de Gauss ». La plupart des opérations sur les nombres complexes ont leur interprétation géométrique dans ce plan.

Dans le plan de Gauss, le module d'un complexe $z$ représente la distance entre $0$ et $z$. On appelle \Defn{argument} de $z$ (noté $\arg z$) l'angle (déterminé à $2\pi$ près) entre le demi-axe des réels positifs et la demi-droite qui part de $0$ et passe par $z$. Le module et l'argument d'un complexe permettent de déterminer univoquement ce complexe puisqu'on a la formule
\[z = a + bi = \module z \left( \cos(\arg(z)) + i \sin(\arg(z)) \right)\]

L'argument de $z$ se détermine via les formules
\[\frac a {\module z} = \cos(\arg(z)) \quad \frac b {\module z} = \sin(\arg(z))\]
ou encore par la formule
\[
\frac b a = \tan(\arg(z)) \quad \text{en vérifiant le quadrant.}
\]
La vérification du quadrant vient de ce que la tangente ne détermine l'angle qu'à $\pi$ près.

%---------------------------------------------------------------------------------------------------------------------------
\subsection{Angle entre deux vecteurs}
%---------------------------------------------------------------------------------------------------------------------------

\begin{propositionDef} \label{DEFooSVDZooPWHwFQ}
    Soient des vecteurs \( X,Y\in \eR^2\). Il existe un unique \( \theta\in \mathopen[ 0 , \pi \mathclose]\) tel que
    \begin{equation}		\label{eqDefAngleVect}
        \cos(\theta)=\frac{ X\cdot Y }{ \| X \|\| Y \| }.
    \end{equation}
    Ce réel est appelé \defe{angle}{angle entre deux vecteurs} entre \( X\) et \( Y\).
\end{propositionDef}

\begin{proof}
    Si $a$ et $b$ sont des réels, l'inégalité $| a |\leq b$ peut se développer en une double inégalité
    \begin{equation}
        -b\leq a\leq b.
    \end{equation}
    L'inégalité de Cauchy-Schwarz \eqref{EQooZDSHooWPcryG} devient alors
    \begin{equation}
        -\| X \|\| Y \|\leq X\cdot Y\leq\| X \|\| Y \|.
    \end{equation}
    Si $X\neq 0$ et $Y\neq 0$, nous en déduisons
    \begin{equation}
        -1\leq\frac{ X\cdot Y }{ \| X \|\| Y \| }\leq 1.
    \end{equation}
    Il existe donc par la proposition~\ref{PROPooZOZHooSMoYQD} un angle $\theta\in\mathopen[ 0 , \pi \mathclose]$ tel que
    \begin{equation}	
        \cos(\theta)=\frac{ X\cdot Y }{ \| X \|\| Y \| }.
    \end{equation}
\end{proof}

\begin{normaltext}
    Certains n'hésitent pas à écrire la formule
    \begin{equation}		\label{eqPropCosThet}
        X\cdot Y=\| X \|\| Y \|\cos(\theta).
    \end{equation}
    comme une définition du produit scalaire. C'est ce qui arrive lorsqu'on défini les fonctions trigonométriques à partir de relations dans les triangles rectangles.
\end{normaltext}

Notez que les angles entre deux vecteurs sont toujours plus petits ou égaux à \unit{180}{\degree}.

La longueur de la projection du point $P$ sur la droite horizontale va naturellement être égale à $\cos(\theta)$. En effet, si nous notons $X$ un vecteur horizontal de norme $1$, cette projection est donné par $P\cdot X$. Mais en reprenant l'équation \eqref{eqPropCosThet}, nous voyons que
\begin{equation}
	P\cdot X=\| P \|\| X \|\cos(\theta),
\end{equation}
tandis qu'ici nous avons $\| P \|=\| X \|=1$.

Nous appelons $\sin(\theta)$ la longueur de la projection sur l'axe vertical.

Quelques dessins nous convainquent que
\begin{equation}
	\begin{aligned}[]
		\sin(\theta+2\pi)&=\sin(\theta)&\cos(\theta+2\pi)&=\sin(\theta),\\
		\sin(\theta+\frac{ \pi }{2})&=\cos(\theta)&\cos(\theta+\frac{ \pi }{2})&=-\sin(\theta),\\
		\sin(\pi-\theta)&=\sin(\theta)&\cos(\pi-\theta)&=-\cos(\theta).
	\end{aligned}
\end{equation}
Le théorème de Pythagore nous montre aussi l'importante relation
\begin{equation}
	\sin^2(\theta)+\cos^2(\theta)=1.
\end{equation}

Quelques valeurs remarquables pour les sinus et cosinus :
\begin{equation}
	\begin{aligned}[]
		\sin 0&=0,&\sin\frac{ \pi }{ 6 }&=\frac{ 1 }{2},&\sin\frac{ \pi }{ 4 }&=\frac{ \sqrt{2} }{2},&\sin\frac{ \pi }{ 3 }&=\frac{ \sqrt{3} }{2},&\sin\frac{ \pi }{2}&=1,&\sin\pi&=0\\
		\cos 0&=1,&\cos\frac{ \pi }{ 6 }&=\frac{ \sqrt{3} }{2},&\cos\frac{ \pi }{ 4 }&=\frac{ \sqrt{2} }{2},&\cos\frac{ \pi }{ 3 }&=\frac{ 1 }{2},&\cos\frac{ \pi }{2}&=0,&\cos\pi&=-1
	\end{aligned}
\end{equation}

Nous pouvons prouver simplement que $\sin(\unit{30}{\degree})=\frac{ 1 }{2}$ et $\cos(\unit{30}{\degree})=\frac{ \sqrt{3} }{2}$ en s'inspirant de la figure~\ref{LabelFigGVDJooYzMxLW}. % From file GVDJooYzMxLW
\newcommand{\CaptionFigGVDJooYzMxLW}{Un triangle équilatéral de côté $1$.}
\input{auto/pictures_tex/Fig_GVDJooYzMxLW.pstricks}

%---------------------------------------------------------------------------------------------------------------------------
\subsection{Aire du parallélogramme}
%---------------------------------------------------------------------------------------------------------------------------

% TODO. Il faut revoir ce calcul d'aire à la lumière du fait que nous définissons le mot «aire» avec des intégrales.
% TODOooWBFOooORrGjZ

\begin{remark}      \label{RemaAireParalProdVect}
    Le nombre $\| a \|\| b \|\sin(\theta)$ est l'aire du parallélogramme formé par les vecteurs $a$ et $b$, comme cela se voit sur la figure~\ref{LabelFigBNHLooLDxdPA}. % From file BNHLooLDxdPA
\newcommand{\CaptionFigBNHLooLDxdPA}{Calculer l'aire d'un parallélogramme.}
\input{auto/pictures_tex/Fig_BNHLooLDxdPA.pstricks}
\end{remark}

\begin{proposition}     \label{PropNormeProdVectoabsint}
    Nous avons
    \begin{equation}
        \| a\times b \|=\| a \|\| b \|\sin(\theta)
    \end{equation}
    où $\theta\in\mathopen[ 0.\pi \mathclose]$ est l'angle formé par $a$ et $b$.
\end{proposition}

\begin{proof}
    En utilisant la décomposition du produit vectoriel, nous avons
    \begin{equation}
        \begin{aligned}[]
            \| a\times b \|^2&=\begin{vmatrix}
                a_2    &   a_3    \\
                b_2    &   b_3
            \end{vmatrix}^2+\begin{vmatrix}
                a_1    &   a_3    \\
                b_1    &   b_3
            \end{vmatrix}^2+\begin{vmatrix}
                a_1    &   a_2    \\
                b_1    &   b_2
            \end{vmatrix}^2\\
            &=(a_2b_3-b_2a_3)^2+(a_1b_3-a_3b_1)^2+(a_1b_2-a_2b_1)^2\\
            &=(a_1^2+a_2^2+a_3^2)(b_1^2+b_2^2+b_3^2)-(a_1b_1+a_2b_2+a_3b_3)^2\\
            &=\| a \|^2\| b \|^2-(a\cdot b)^2\\
            &=\| a \|^2\| b \|^2-\| a \|^2\| b \|^2\cos^2(\theta)\\
            &=\| a \|^2\| b \|^2\big( 1-\cos^2(\theta) \big)\\
            &=\| a \|^2\| b \|^2\sin^2(\theta).
        \end{aligned}
    \end{equation}
    D'où le résultat. Nous avons utilisé la formule de la définition \eqref{DEFooSVDZooPWHwFQ} donnant l'angle en fonction du produit scalaire.
\end{proof}

\begin{normaltext}      \label{NORMooWWOKooWzScnZ}
Si les vecteurs $a$, $b$ et $c$ ne sont pas coplanaires, alors la valeur absolue du produit mixte (voir équation \eqref{EqProduitMixteDet}) $a\cdot(b\times c)$ donne le volume du parallélépipède construit sur les vecteurs $a$, $b$ et $c$.

En effet si $\varphi$ est l'angle entre $b\times c$ et $a$, alors la hauteur du parallélépipède vaut $\| a \|\cos(\varphi)$ parce que la direction verticale est donnée par $b\times c$, et la hauteur est alors la «composante verticale» de $a$. Par conséquent, étant donné que $\| b\times c \|$ est l'aire de la base, le volume du parallélépipède vaut\footnote{Le calcul de ce volume mériterait une certaine réflexion, surtout à partir du moment où nous avons décidé de définir les fonctions trigonométriques à partir de son développement (définition~\ref{PROPooZXPVooBjONka}).}
\begin{equation}
    V=\| b\times c\|  \| a \|\cos(\varphi).
\end{equation}
Or cette formule est le produit scalaire de $a$ par $b \times c$; ce dernier étant donné par le déterminant de la matrice formée des composantes de $a$, $b$ et $c$ grâce à la formule \eqref{EqProduitMixteDet}.
\end{normaltext}

La valeur absolue du déterminant
\begin{equation}        \label{EqDeratb}
    \begin{vmatrix}
        a_1    &   a_2    \\
        b_1    &   b_2
    \end{vmatrix}
\end{equation}
est l'aire du parallélogramme déterminé par les vecteurs $\begin{pmatrix}
    a_1    \\
    a_2
\end{pmatrix}$ et $\begin{pmatrix}
    b_1    \\
    b_2
\end{pmatrix}$. En effet, d'après la remarque~\ref{RemaAireParalProdVect}, l'aire de ce parallélogramme est donnée par la norme du produit vectoriel
\begin{equation}
    \begin{pmatrix}
        a_1    \\
        a_2    \\
        0
    \end{pmatrix}\times
    \begin{pmatrix}
          b_1  \\
        b_2    \\
        0
    \end{pmatrix}=\begin{vmatrix}
        e_x    &   e_y    &   e_z    \\
        a_1    &   a_2    &   0    \\
        b_1    &   b_2    &   0
    \end{vmatrix}=
    \begin{vmatrix}
        a_1    &   a_2    \\
        b_1    &   b_2
    \end{vmatrix}e_z,
\end{equation}
donc la norme $\| a\times b \|$ est bien donnée par la valeur absolue du déterminant \eqref{EqDeratb}.

%+++++++++++++++++++++++++++++++++++++++++++++++++++++++++++++++++++++++++++++++++++++++++++++++++++++++++++++++++++++++++++ 
\section{Paramétrisation du cercle}
%+++++++++++++++++++++++++++++++++++++++++++++++++++++++++++++++++++++++++++++++++++++++++++++++++++++++++++++++++++++++++++

Nous allons parler de paramértisation du cercle. L'ensemble \( S^1\) sera vu tantôt comme le cercle dans \( \eR^2\), tantôt comme le cercle dans \( \eC\). Nous n'allons pas pousser le vice jusqu'à écrire explicitement les isomorphismes lorsque nous passons d'une représentation à l'autre. Parmi les identifications que nous allons faire sans ménagement, il y a l'identification entre les applications
\begin{equation}
    \begin{aligned}
        \gamma\colon \mathopen[ 0 , 2\pi \mathclose]&\to \eR^2 \\
        t&\mapsto \big( \cos(t),\sin(t) \big) 
    \end{aligned}
\end{equation}
et
\begin{equation}
    \begin{aligned}
        \varphi\colon \mathopen[ 0 , 2\pi \mathclose[&\to \eC \\
            t&\mapsto  e^{it}. 
    \end{aligned}
\end{equation}
C'est évidemment la formule \(  e^{ti}=\cos(t)+i\sin(t)\) (lemme \ref{LEMooHOYZooKQTsXW}) qui permet de transformer \( \gamma\) en \( \varphi\) et inversement. De plus \( \eR^2\) et \( \eC\) sont isomorphes en tant qu'espaces vectoriels normés (et aussi donc topologiques).

Nous allons voir deux choses à propos de cette application :
\begin{itemize}
\item 
    Elle est continue, mais son inverse n'est pas continue. En considérant seulement la restriction \( \varphi\colon \mathopen] 0 , 2\pi \mathclose[\to S^2\setminus\{ (1,0) \}\) nous avons un difféomorphisme, et donc une possibilité de changement de variables dans l'intégrale (théorème \ref{THOooUMIWooZUtUSg}).

    Le fait qu'il manque un point est sans importante parce que nous n'allons considérer que la mesure de Lebesgue ou des variations simples autour de la mesure de Lebesgue.        

\item
    La fonction \( \varphi\colon \mathopen[ 0 , 2\pi \mathclose[\to S^1\) est une bijection borélienne d'inverse borélien\footnote{Proposition \ref{PROPooQFYHooEajmbW}.}. Donc nous pouvons transposer toute la théorie de la mesure de \( S^1\) à \( \mathopen[ 0 , 2\pi \mathclose[\) sans «triche».
\end{itemize}

Tout cela pour dire que nous allons donner un tas de justifications pour écrire des égalités du type
\begin{equation}
    \int_{S^1}f=\int_{0}^{2\pi}f\circ\varphi.
\end{equation}

%--------------------------------------------------------------------------------------------------------------------------- 
\subsection{Bijection continue}
%---------------------------------------------------------------------------------------------------------------------------

\begin{proposition}     \label{PROPooKSGXooOqGyZj}
    L'application
    \begin{equation}
        \begin{aligned}
            \gamma\colon \mathopen[ 0 , 2\pi \mathclose[&\to S^1\subset \eR^2 \\
            t&\mapsto \big( \cos(t),\sin(t) \big)
        \end{aligned}
    \end{equation}
    est une bijection continue.
\end{proposition}

\begin{proof}
    La continuité découle de la continuité des composantes. Le fait que l'image de \( \gamma\) soit dans \( S^1\) découle immédiatement du fait que \( \sin^2+\cos^2=1\).

    Pour la bijection, il faut injectif et surjectif.
    \begin{subproof}
        \item[Injectif]
            Soient \( x_1<x_2\) tels que \( \sin(x_1)=\sin(x_2)\) et \( \cos(x_1)=\cos(x_2)\). Supposons pour fixer les idées que \( \sin(x_1)>0\) et \( \cos(x_1)>0\) : si ce n'est pas le cas, il faut traiter séparément les \( 4\) possibilités de combinaisons de signes.

            Nous avons obligatoirement \( x_1,x_2\in\mathopen[ 0 , \frac{ \pi }{ 2 } \mathclose[\). Vu que \( \sin(x_1)=\sin(x_2)\), il existe par le théorème de Rolle~\ref{ThoRolle} un élément \( c\in \mathopen] x_1 , x_2 \mathclose[\) tel que \( \sin'(c)=0\), c'est-à-dire \( \cos(c)=0\). Cela contredirait la proposition~\ref{PROPooMWMDooJYIlis}\ref{ITEMooQKPKooEPeHER} à moins que \( x_1=x_2\).

            \item[Surjectif]

                Soient \( x,y\) tels que \( x^2+y^2=1\). Supposons pour varier les plaisirs que \( x<0\) et \( y>0\). Vu que la fonction \( \cos\) va de \( 0\) à \( -1\) lorsque \( x\) va de \( \pi/2\) à \( \pi\), le théorème des valeurs intermédiaires donne \( t\in\mathopen[ \pi/2 , \pi \mathclose]\) tel que \( \cos(t)=x\). Pour cette valeur de \( x\) nous avons
                \begin{equation}
                    \cos^2(x)+\sin^2(x)=1,
                \end{equation}
                et donc \( \sin^2(x)=y^2\), ce qui donne \( \sin(x)=\pm y\). Mais pour \( x\in \mathopen[ \pi/2 , \pi \mathclose]\) nous avons \( \sin(t)>0\). Par conséquent \( \sin(t)=y\).
    \end{subproof}
\end{proof}

\begin{example} \label{EXooJFDPooBZADKs}
    L'application
    \begin{equation}
        \begin{aligned}
            \varphi\colon \mathopen] 0 , 2\pi \mathclose[&\to S^1 \\
                x&\mapsto \begin{pmatrix}
                    \cos(x)    \\
                    \sin(x)
                \end{pmatrix}
        \end{aligned}
    \end{equation}
est un continue par la proposition \ref{PROPooKSGXooOqGyZj}. Vu que \( \mathopen] 0 , 2\pi \mathclose[\) est connexe (proposition~\ref{PropInterssiConn}) la proposition~\ref{PropGWMVzqb} implique que le cercle privé d'un point est connexe.
\end{example}


Allez\ldots Dans l'intro nous avions dit que nous n'allions pas faire explicitement les isomorphismes. Faisons-le quand même une fois, mais c'est bien parce que c'est vous hein.
\begin{proposition}     \label{PROPooZEFEooEKMOPT}
    L'application
    \begin{equation}
        \begin{aligned}
            f\colon \mathopen[ 0 , 2\pi \mathclose[&\to S^1 \\
                x&\mapsto  e^{ix}
        \end{aligned}
    \end{equation}
    est une bijection. Ici, \( S^1\) est l'ensemble des nombres complexes de norme \( 1\).
\end{proposition}

\begin{proof}
    Nous savons que
    \begin{equation}
        \begin{aligned}
            \varphi\colon \eR^2&\to \eC \\
            (x,y)&\mapsto x+iy
        \end{aligned}
    \end{equation}
    est une bijection isométrique. C'est pour cela que nous allons nous permettre de noter \( S^1\) le cercle unité dans \( \eR^2\) aussi bien que l'ensemble des nombres complexes de norme \( 1\).

    Sur \( \eR^2\) nous avons l'application
    \begin{equation}
        \begin{aligned}
            \gamma\colon \mathopen[ 0 , 2\pi \mathclose[&\to S^1\subset \eR^2 \\
                t&\mapsto \begin{pmatrix}
                    \cos(t)    \\
                    \sin(t)
                \end{pmatrix}
        \end{aligned}
    \end{equation}
    qui est une bijection continue (c'est la proposition~\ref{PROPooKSGXooOqGyZj}). Et enfin le lemme~\ref{LEMooHOYZooKQTsXW} nous donne \(  e^{ix}=\cos(x)+i\sin(x)\).

    Avec tout ça, l'application \( \varphi^{-1}\circ f\colon \mathopen[ 0 , 2\pi \mathclose[\to S^1 \) est une bijection continue. Et comme \( \varphi\) l'est également, \( f\) est une bijection continue.
\end{proof}

La proposition suivante donne les coordonnées polaires sur \( \eC\). La régularité est l'objet du théorème \ref{THOooBETSooXSQhdX} (à part le fait que ce dernier parle de \( \eR^2\) et non de \( \eC\)).
\begin{proposition}     \label{PROPooRFMKooURhAQJ}
    Pour tout nombre complexe \( z\), il existe un unique \( \theta\in\mathopen[ 0 , 2\pi \mathclose[\) tel que
        \begin{equation}
            z=| z | e^{i\theta}.
        \end{equation}
\end{proposition}

\begin{proof}
    Soit \( z\in \eC\). Nous considérons \( z'=z/| z |\) qui est de norme \( 1\). Donc il existe un unique \( \theta\in\mathopen[ 0 , 2\pi \mathclose[\) tel que \( z'= e^{i\theta}\) (proposition \ref{PROPooZEFEooEKMOPT}).

    Pour ce \( \theta\) nous avons \( z=| z | e^{i\theta}\).
\end{proof}
Bien entendu, le \( \theta\) est unique dans \( \mathopen[ 0 , 2\pi \mathclose[\), mais il n'est pas du tout unique dans \( \eR\).

%--------------------------------------------------------------------------------------------------------------------------- 
\subsection{Inverse}
%---------------------------------------------------------------------------------------------------------------------------
\label{SUBSECooWFNMooOuZBRN}

Nous pouvons écrire un inverse de la fonction \( \varphi\) grâce à la fonction arc tangente introduite au théorème \ref{THOooUSVGooOAnCvC}. 
La fonction que nous écrivons à présent est la fonction \( \arg_{0^{-}} \) définie par \eqref{EQooNKKDooOuJxXe}. Elle n'est pas exactement la fonction argument définie par \eqref{EQooPJVFooSEKTny}.

Nous avons :
\begin{equation}        \label{EQooSAYFooRFVSPc}
    \begin{aligned}
        \varphi^{-1}\colon S^1&\to \mathopen[ 0 , 2\pi \mathclose[ \\
        x+iy&\mapsto  
    \begin{cases}
        \arctg(y/x)    &   \text{si } x>0,y\geq 0\\
        \frac{ \pi }{2}    &    \text{si }(x,y)=(0,1)\\
        \pi-\arctg(-y/x)    &    \text{si }x<0,y\geq 0\\
        \pi+\arctg(y/x)    &    \text{si }x<0,y<0\\
        2\pi-\arctg(-y/x)    &    \text{si }x>0,y<0
    \end{cases}
    \end{aligned}
\end{equation}
Chacune des branche est continue parce que la fonction arc tangente l'est. Trois des raccords sont également continus grâce aux limites du lemme \ref{LEMooHRDCooGtnyeQ}.

L'application \( \varphi^{-1}\) n'est cependant pas continue au point \( (1,0)\)\footnote{Vu que nous avons considéré \( S^1\subset \eC\), nous aurions dû noter «\( 1\)» ce point. Mais vous vous imaginez le clash de notation avec le \( 1\in \mathopen[ 0 , 2\pi \mathclose[\subset \eR\)?}. C'est l'objet du lemme suivant.

\begin{lemma}       \label{LEMooEQVRooMAffCw}
    L'application \( \varphi^{-1}\colon S^1\to \mathopen[ 0 , 2\pi \mathclose[\) n'est pas continue en \( (1,0)\). Mais elle est continue ailleurs. Autrement dit,
        \begin{equation}
        \varphi^{-1}\colon S^1\setminus\{ (1,0) \}\to \mathopen] 0 , 2\pi \mathclose[
        \end{equation}
        est continue.
\end{lemma}

\begin{proof}
    En effet, \( \varphi^{-1}\) serait continue si l'image de tout ouvert de \( \mathopen[ 0 , 2\pi \mathclose[\) par \( \varphi\) serait ouverte dans \( S^1\) (topologie induite de \( \eC\)). Prenons un petit ouvert \( \mathopen[ 0 , \epsilon \mathclose[\) (si vous êtes étonnés, c'est que vous n'avez pas bien la topologie induites en tête). Son image contient le point \( (1,0)\), mais aucun point \( (x,y)\) avec \( y<0\).
       
    Montrons que tout voisinage de \( (1,0)\) dans \( \eC\) contient des points \( x+iy\) de \( S^1\) avec \( y<0\). Un point de \( S^1\) est de la forme \( \cos(t)+i\sin(t)\). Nous avons :
    \begin{equation}
        | \cos(t)+i\sin(t)-1 |^2=\big( \cos(t)-1 \big)^2+\sin^2(t)=2\big( 1-\cos(t) \big).
    \end{equation}
    Soit \( \delta>0\), et montrons que \( B\big( (1,0),\delta \big)\cap S^1\) contient des points d'ordonnées négatives. D'abord il existe \( \epsilon>0\) tel que pour \( t=2\pi-\epsilon\),
    \begin{equation}
        2\big( 1-\cos(t) \big)<\delta.
    \end{equation}
    Ensuite pour de tels \( t\), nous avons \( \sin(t)<0\). Donc les points de \( S^1\) correspondant à \( 2\pi-\epsilon\) sont dans \( S^1\cap B\big( (1,0),\delta \big)\).

    Bref, l'image de \( \mathopen[ 0 , \epsilon \mathclose[\) n'est pas un ouvert de \( S^1\).
\end{proof}

%---------------------------------------------------------------------------------------------------------------------------
\subsection{Cercle trigonométrique}
%---------------------------------------------------------------------------------------------------------------------------

Le \href{http://fr.wikiversity.org/wiki/Trigonométrie/Cosinus_et_sinus_dans_le_cercle_trigonométrique}{cercle trigonométrique} est le cercle de rayon $1$ représenté à la figure~\ref{LabelFigCercleTrigono}. Sa longueur est $2\pi$.
\newcommand{\CaptionFigCercleTrigono}{Le cercle trigonométrique.}
\input{auto/pictures_tex/Fig_CercleTrigono.pstricks}

Nous verrons plus tard que la longueur de l'arc de cercle intercepté par un angle $\theta$ est égal à $\theta$. Les radians sont donc l'unité d'angle les plus adaptés au calcul de longueurs sur le cercle.

%TODOooLMZOooWNDjgq : remettre ce lien après le fork
%Voir exercice~\ref{exoGeomAnal-0034}.

%--------------------------------------------------------------------------------------------------------------------------- 
\subsection{Du point de vue de la tribu, mesure et co.}
%---------------------------------------------------------------------------------------------------------------------------

Nous avons considéré sur \( S^1\) la topologie induite de \( \eC\). Nous allons y mettre la tribu induite de celle de Lebesgue de \( \eC\). Mais nous n'allons pas y mettre la \emph{mesure} induite de \( \eC\); sinon tout serait toujours de mesure nulle.

\begin{proposition}[\cite{MonCerveau}]      \label{PROPooQFYHooEajmbW}
    L'application \( \varphi\) est borélienne d'inverse borélien, c'est-à-dire
    \begin{equation}
        \Borelien(S^1)=\varphi\big( \Borelien(\mathopen[ 0 , 2\pi \mathclose[) \big).
    \end{equation}
\end{proposition}

\begin{proof}
    L'inclusion  \(\Borelien(S^1)\subset\varphi\big( \Borelien(\mathopen[ 0 , 2\pi \mathclose[) \big) \) est la plus simple : si \( A\in\Borelien(S^1)\), alors \( \varphi^{-1}(A)\in\Borelien\big( \mathopen[ 0 , 2\pi \mathclose[ \big)\) parce que \( \varphi\colon \mathopen[ 0 , 2\pi \mathclose[\to S^1\) est continue et donc borélienne (théorème \ref{ThoJDOKooKaaiJh}).

    Pour l'autre inclusion, il faudra faire par étapes.
    \begin{subproof}
        \item[Ouvert ne contenant pas zéro]
            Si \( A\) est un ouvert de \( \mathopen[ 0 , 2\pi \mathclose[\) ne contenant pas \( 0\), il est un ouvert de \( \eR\) ou de \( \mathopen] 0 , 2\pi \mathclose[\). Le lemme \ref{LEMooEQVRooMAffCw} nous indique que son image par \( \varphi\) est ouverte dans \( S^1\). En particulier, \( \varphi(A)\in\Borelien(S^1)\).
            \item[Ouvert de la forme \( \mathopen[ 0 , \epsilon \mathclose[\)] 
                Nous supposons que \( \epsilon\) est petit. Disons pour fixer les idées, plus petit que \( \pi/2\). Nous avons :
                \begin{equation}
                    \varphi\big( \mathopen[ 0 , \epsilon \mathclose[ \big)=\varphi\big( \mathopen] 0 , \epsilon \mathclose[ \big)\cup\varphi\big( \{ 0 \} \big). 
                \end{equation}
                Le premier élément de l'union est un ouvert, et le second un unique point. L'union est un borélien.
            \item[Ouvert général]
                Si un ouvert de \( \mathopen[ 0 , 2\pi\mathclose[\) ne contient pas \( 0\), son image est ouverte. Nous nous penchons sur le cas d'un ouvert contenant \( 0\).

                Si un ouvert de \( \mathopen[ 0 , 2\pi \mathclose[\) contient \( 0\), alors il contient un ouvert de la forme \( \mathopen[ 0 , \epsilon \mathclose[\), parce qu'un ouvert contient une boule autour de chacun de ses points (théorème \ref{ThoPartieOUvpartouv} couplé au fait que nous sommes dans la topologie induite de \( \eR\)).

                Si \( A\) est un ouvert contenant zéro, alors
                \begin{equation}
                    A=\mathopen[ 0 , \epsilon \mathclose[\cup\big( A\setminus\mathopen[ 0 , \frac{ \epsilon }{2} \mathclose] \big).
                \end{equation}
                Nous avons déjà vu que l'image du premier élément de l'union est un borélien. Étant donné que \( A\setminus \mathopen[ 0 , \frac{ \epsilon }{2} \mathclose]\) est un ouvert ne contenant pas zéro, son image est un ouvert. Donc le l'image de \( A\) est un borélien.

            \item[Pause]
                Nous avons déjà vu que l'image par \( \varphi\) de tout ouvert de \( \mathopen[ 0 , 2\pi \mathclose[\) était un borélien de \( S^1\). Nous devons en déduire que l'image de tout borélien de \( \mathopen[ 0 , 2\pi \mathclose[\) est un borélien de \( S^1\).

                    C'est ce que nous faisons maintenant

                \item[Boréliens]
                
                    Nous utilisons le lemme de transport \ref{LemOQTBooWGYuDU} avec l'application \( \varphi^{-1}\) et l'ensemble des ouverts :
                    \begin{equation}
                        \varphi\big( \sigma(\tribC) \big)=\sigma\big( \varphi(\tribC) \big)
                    \end{equation}
                    où \( \tribC\) est la tribu des ouverts dans \( \mathopen[ 0 , 2\pi \mathclose[\). L'ensemble \( \sigma(\tribC)\) est par définition l'ensemble \( \Borelien\big( \mathopen[ 0 , 2\pi \mathclose[ \big)\). D'autre part nous avons vu que l'image d'un ouvert est un borélien : \( \varphi(\tribC)\subset\Borelien(S^1)\). Nous avons donc
                        \begin{equation}
                                \varphi\big( \Borelien(\mathopen[ 0 , 2\pi \mathclose[) \big)=\sigma\big( \varphi(\tribC) \big)\subset\sigma\big( \Borelien(S^1) \big)\subset\Borelien(S^1).
                        \end{equation}
    \end{subproof}
    La preuve est terminée. 
\end{proof}

\begin{proposition}[Boréliens sur \( S^1\)\cite{MonCerveau}]      \label{PROPooHMSCooRIjcJq}
    Soit la structure usuelle d'espace mesurable \( (\eC,\Borelien(\eC))\). Nous considérons
    \begin{itemize}
        \item la tribu \( \Borelien(\eC)_{S^1}\) induite de la tribu des boréliens  de \( \eC\) vers \( S^1\),
        \item la tribu \( \Borelien(S^1)\) des boréliens de \( S^1\) construite à partir de la topologie induite de \( \eC\) vers \( S^1\).
        \item la bijection \( \varphi\colon \mathopen[ 0 , 2\pi \mathclose[\to S^1\),
            \item la mesure de Lebesgue sur \( \mathopen[ 0 , 2\pi \mathclose[\) (induite de celle sur \( \eR\)) et sur \( \eC\), que nous noterons toutes deux \( \lambda\).
    \end{itemize}
    Alors 
    \begin{enumerate}
        \item       \label{ITEMooSUNEooRhAdep}
            Nous avons les expressions
            \begin{subequations}
                \begin{align}
                    \Borelien(\eC)_{S^1}&=\{A\in\Borelien(\eC)\tq A\subset S^1\} \\
                    &=\{A\cap S^1\tq A\in\Borelien(\eC)\}       \label{SUBEQooYZGCooDqXmft}
                \end{align}
            \end{subequations}
        \item       \label{ITEMooGYPNooRaZbNW}
            Nous avons
            \begin{equation}
                \Borelien(S^1) = \Borelien(\eC)_{S^1}=\varphi\Big( \Borelien\big( \mathopen[ 0 , 2\pi \mathclose[ \big) \Big).
            \end{equation}
       \item\label{ITEMooFUXKooFQdoaw}
           En définissant \( \mu\colon \Borelien(S^1)\to \eR\) par
           \begin{equation}         \label{EQooKHZRooSrFMdo}
               \mu(A)=\frac{ \lambda\big( \varphi^{-1}(A) \big) }{ 2\pi },
           \end{equation}
           le triple \( \big( S^1,\Borelien(S^1), \mu \big)\) est un espace mesuré.
       \item\label{ITEMooBQLRooOsqesg}
           L'espace mesuré \( \big( S^1,\Borelien(S^1), \mu \big)\) est fini et
            \begin{equation}
                \mu(S^1)=1.
            \end{equation}
    \end{enumerate}
\end{proposition}

\begin{proof}
    Point par point.
    \begin{subproof}
        \item[Pour \ref{ITEMooSUNEooRhAdep}]
            C'est la proposition \ref{PROPooUNNSooMUQKfp}.
        \item[Pour \ref{ITEMooGYPNooRaZbNW}]
            La première égalité est le lemme \ref{LEMooUPYDooPVjscA}. Le fait que \( \Borelien(S^1)=\varphi\Big( \Borelien\big( \mathopen[ 0 , 2\pi \mathclose[ \big) \Big)\) est déjà la proposition \ref{PROPooQFYHooEajmbW}.
            \item[Pour \ref{ITEMooFUXKooFQdoaw}]
                Nous devons d'abord nous assurer que la formule ait un sens. Cela est chose aisée; si \( A\in \Borelien(S^1)\), le point \ref{ITEMooGYPNooRaZbNW} nous indique que \( \varphi^{-1}(A)\in \Borelien\big( \mathopen[ 0 , 2\pi \mathclose[ \big)\). Ensuite, nous devons vérifier les deux conditions de la définition \ref{DefBTsgznn} pour avoir un espace mesuré.

                En premier lieu,
                \begin{equation}
                    \mu(\emptyset)=\frac{1}{ 2\pi }\lambda\big( \varphi^{-1}(\emptyset) \big)=\frac{1}{ 2\pi }(\emptyset)=0.
                \end{equation}
                En en second lieu, si les \( A_i\in \Borelien(S^1)\) sont disjoints, les \( \varphi^{-1}(A_i)\) sont également disjoints parce que \( \varphi^{-1}\) est une bijection. Donc
                \begin{subequations}
                    \begin{align}
                        \mu(\bigcup_iA_i)&=\frac{1}{ 2\pi }\lambda\big( \bigcup_i\varphi^{-1}(A_i) \big)\\
                        &=\frac{1}{ 2\pi }\sum_i\lambda\big( \varphi^{-1}(A_i) \big)\\
                        &=\sum_i\frac{ \lambda\big( \varphi^{-1}(A_i) \big) }{ 2\pi }\\
                        &=\sum_i\mu(A_i).
                    \end{align}
                \end{subequations}
                D'accord.
            \item[Pour \ref{ITEMooGYPNooRaZbNW}]
                En ce qui concerne la mesure de \( S^1\) pour \(\mu\) nous avons simplement
                \begin{equation}
                    \mu(S^1)=\frac{ \lambda\big( \mathopen[ 0 , 2\pi \mathclose[ \big) }{ 2\pi }=1.
                \end{equation}
    \end{subproof}
\end{proof}

Maintenant que \( (S^1,\Borelien(S^1), \mu)\) est un espace mesuré, nous pouvons compléter la tribu \( \Borelien(S^1)\) pour la mesure \( \mu\).

\begin{definition}
    La \defe{tribu de Lebesgue}{tribu de Lebesgue sur $ S^1$} sur \( S^1\) est la mesure complétée pour
    \begin{equation}
        \big( S^1,\Borelien(S^1),\mu \big)
    \end{equation}
    où \( \mu\) est la mesure définie par la proposition \ref{PROPooHMSCooRIjcJq}. Nous notons \( \Lebesgue(S^1)\) la tribu et encore \( \mu\) la mesure.
\end{definition}

\begin{proposition}[Lebesgue sur \( S^1\)\cite{MonCerveau}]     \label{PROPooDLBCooUfQZOa}
    Soit la structure d'espace mesuré complet \( \big( S^1,\Lebesgue(S^1), \mu \big)\). Nous considérons
    \begin{itemize}
        \item la tribu \( \Lebesgue(\eC)_{S^1}\) induite de la tribu des boréliens  de \( \eC\) vers \( S^1\),
        \item la bijection \( \varphi\colon \mathopen[ 0 , 2\pi \mathclose[\to S^1\),
    \end{itemize}
    Alors 
    \begin{enumerate}
        \item               \label{ITEMooQMHDooHEThPf}
            La tribu \( \Lebesgue(\eC)_{S^1}\) est la tribu de toutes les parties de \( S^1\).
        \item       \label{ITEMooNIRNooKSeyCa}
            La tribu \( \Lebesgue(S^1)\) est donnée par 
            \begin{equation}
                \Lebesgue(S^1)=\varphi\big( \Lebesgue(\eR)_{\mathopen[ 0 , 2\pi \mathclose[} \big)=\varphi\big( \Lebesgue(\mathopen[ 0 , 2\pi \mathclose[) \big)
            \end{equation}
            où \( \Lebesgue(\mathopen[ 0 , 2\pi \mathclose[)\) est la tribu sur \( \mathopen[ 0 , 2\pi \mathclose[\) obtenue par completion de la tribu des boréliens de la topologie induite.
        \item       \label{ITEMooXDBTooYnauyi}
            Nous avons l'inclusion stricte
            \begin{equation}
                \Lebesgue(S^1)\subsetneq\Lebesgue(\eC)_{S^1}.
            \end{equation}
    \end{enumerate}
\end{proposition}

\begin{proof}
    Point par point.
    \begin{subproof}
        \item[Pour \ref{ITEMooQMHDooHEThPf}]
            Si \( A\subset S^1\), alors \( A\) est une partie de \( S^1\) qui est mesurable et de mesure nulle pour \( \eC\). Donc \( A\) est \( \lambda\)-négligeable et par conséquent mesurable.
        \item[Pour \ref{ITEMooNIRNooKSeyCa}]
            Il s'agit de prouver que
            \begin{equation}
                \widehat{\Borelien(S^1)}=\varphi\big( \widehat{\Borelien\big( \mathopen[ 0 , 2\pi \mathclose[ \big)} \big).
            \end{equation}
            Ce n'est rien d'autre que la proposition \ref{PROPooORDCooJEsjzR}. La seconde partie de l'égalité est la proposition \ref{PROPooAMIEooRomnMG}
        \item[Pour \ref{ITEMooXDBTooYnauyi}]
            Comme indiqué au point \ref{ITEMooQMHDooHEThPf}, la tribu \( \Lebesgue(\eC)_{S^1}\) est la tribu de toutes les parties de \( S^1\); l'incusion est donc évidente. Le point pas tout à fait évident à prouver est l'existence de parties de \( S^1\) à n'être pas dans \( \Lebesgue(S^1)\).

            Soit \( V\) non mesurable dans \( \mathopen[ 0 , 2\pi \mathclose[\) (prenez quelque chose comme l'ensemble de Vitali de l'exemple \ref{EXooCZCFooRPgKjj}). Vu que, par le point \ref{ITEMooNIRNooKSeyCa},
                \begin{equation}
                    \Lebesgue(S^1)=\varphi\big( \Lebesgue(\eR)_{\mathopen[ 0 , 2\pi \mathclose[} \big),
                \end{equation}
                la partie \( \varphi^{-1}(V)\) ne peut pas être dans \( \Lebesgue(S^1)\).
    \end{subproof}
\end{proof}

Si vous en voulez plus à propos de \( S^1\) et la façon dont on passe la structure depuis \( \mathopen[ 0 , 2\pi \mathclose[\), vous pouvez lire la proposition \ref{PROPooDJERooYirMru} qui donne la structure de
\begin{equation}
    L^2\big( S^1,\Lebesgue(S^1), \mu \big)
\end{equation}
qui sera, sans surprises la même que celle de
\begin{equation}
    L^2\big( \mathopen[ 0 , 2\pi \mathclose[,\Lebesgue\big( \mathopen[ 0 , 2\pi \mathclose[ \big), \lambda \big).
\end{equation}

%+++++++++++++++++++++++++++++++++++++++++++++++++++++++++++++++++++++++++++++++++++++++++++++++++++++++++++++++++++++++++++
\section{Exemples trigonométriques}
%+++++++++++++++++++++++++++++++++++++++++++++++++++++++++++++++++++++++++++++++++++++++++++++++++++++++++++++++++++++++++++

Nous mettons ici quelques exemples concernant les fonctions trigonométriques, qui n'ont pas pu être mis dans les chapitres le plus adapté, parce que ces derniers sont plus haut dans la table des matière.

\begin{example}     \label{EXooSPFDooSluUGV}
    Prouvons que la fonction\footnote{La définition de la fonction sinus est \ref{PROPooZXPVooBjONka}.} $f(x)=x\sin(x)$ tend vers zéro lorsque $x$ tend vers $0$. D'abord, nous coinçons la fonction entre deux fonctions connues :
	\begin{equation}
		0\leq| x\sin(x) |=| x | |\sin(x) |\leq | x |.
	\end{equation}
	Donc $| x\sin(x) |$ est coincé entre $g(x)=0$ et $h(x)=| x |$. Ces deux fonctions tendent vers $0$ lorsque $x\to 0$, et donc $f(x)$ tend vers zéro.
\end{example}

%--------------------------------------------------------------------------------------------------------------------------- 
\subsection{Quelques équations trigonométriques}
%---------------------------------------------------------------------------------------------------------------------------

La proposition suivante se voit très facilement sur le cercle trigonométrique, mais il faut le démontrer.
\begin{proposition}[\cite{MonCerveau}]     \label{PROPooTUUUooVrAGQo}
    Si \( \theta_0\in \mathopen[ 0 , 2\pi \mathclose[\) vérifie \( \cos(\theta_0)=x_0\), alors l'ensemble de solutions de l'équation \( \cos(\theta)=x_0\) (d'inconnue \( \theta\)) est
        \begin{equation}
            \{ \theta_0,2\pi-\theta_0 \}.
        \end{equation}
        Cet ensemble est un singleton si et seulement si \( x_0=\pm1\).
\end{proposition}

\begin{proof}
    Commençons par prouver que \( \theta_0\) et \( 2\pi-\theta_0\) sont des solutions. Le nombre \( \theta_0\) est solution par hypothèse. En ce qui concerne \( 2\pi-\theta_0\), il est possible d'utiliser la formule d'addition d'angle \eqref{EQooCVZAooQfocya} :
    \begin{equation}        \label{EQooUCAOooTQsUUq}
        \cos(2\pi-\theta_0)=\cos(2\pi)\cos(\theta_0)+\sin(2\pi)\sin(\theta_0).
    \end{equation}
    La proposition \ref{PROPooMWMDooJYIlis}\ref{ITEMooRJZHooCXcKmM} nous indique que \( \cos(2\pi)=1\) et \( \sin(2\pi)=0\). Donc l'égalité \eqref{EQooUCAOooTQsUUq} se réduit à \( \cos(2\pi -x_0)=\cos(2\pi)\).

    Le lemme \ref{LEMooAEFPooGSgOkF} dit que si \( \cos(\theta)=x_0\), alors
    \begin{equation}
        \sin(\theta)=\pm\sqrt{ 1-x_0^2 }.
    \end{equation}
    Nous avons donc soit
    \begin{equation}
        \begin{pmatrix}
            \cos(\theta)    \\ 
            \sin(\theta)    
        \end{pmatrix}=\begin{pmatrix}
            x_0    \\ 
            \sqrt{ 1-x_0^2 }    
        \end{pmatrix},
    \end{equation}
    soit
    \begin{equation}
        \begin{pmatrix}
            \cos(\theta)    \\ 
            \sin(\theta)    
        \end{pmatrix}=\begin{pmatrix}
            x_0    \\ 
            -\sqrt{ 1-x_0^2 }    
        \end{pmatrix},
    \end{equation}
    Vu que \( \theta\mapsto\big( \cos(\theta),\sin(\theta) \big)\) est une bijection avec \( S^1\) (proposition \ref{PROPooKSGXooOqGyZj}), chacune de ces deux possibilités possède une unique solution. L'ensemble des solutions de \( \cos(\theta)=x_0\) possède donc au maximum deux éléments.

    L'ensemble des solutions possède exactement une solution lorsque les points \( \big( x_0,\sqrt{ 1-x_0^2 } \big)\) et \( \big( x_0,-\sqrt{ 1-x_0^2 } \big)\) sont identiques. Cela est le cas si et seulement si \( \sqrt{ 1-x_0^2 }=0\), c'est-à-dire si et seulement si \( x_0=\pm 1\).
\end{proof}

%--------------------------------------------------------------------------------------------------------------------------- 
\subsection{Développements en série}
%---------------------------------------------------------------------------------------------------------------------------

\begin{proposition}[Taylor pour cosinus]     \label{PROPooNPYXooTuwAHP}
    Le développement du cosinus est donné par
	\begin{equation}
		\cos(x)=1-\frac{ x^2 }{ 2 }+\frac{ x^4 }{ 4! }-\frac{ x^6 }{ 6! }\cdots
	\end{equation}
    C'est-à-dire que pout tout \( n\in  \eN\), il existe une fonction \( \alpha\colon \eR\to \eR\) telle que \( \lim_{t\to 0} \alpha(t)=0\) et
    \begin{equation}        \label{EQooGQOIooIkwbJV}
        \cos(x)=\sum_{k=0}^{n}\frac{ (-1)^{2k} }{ (2k)! }x^{2k}+\alpha(x)x^{2n+1}.
    \end{equation}
    En ce qui concerne le sinus, pour tout \( n\) nous avons une fonction \( \alpha\colon \eR\to \eR\) telle que \( \lim_{t\to 0} \alpha(t)=0\) et
    \begin{equation}        \label{EQooKYJAooRebHgc}
        \sin(x)=\sum_{k=0}^n\frac{ (-1)^kx^{2k+1} }{ (2k+1)! }+x^{2n+2}\alpha(x).
    \end{equation}
\end{proposition}

\begin{proof}
    Il s'agit d'utiliser la proposition \ref{PROPooQLHNooRsBYbe}, en faisant attention à l'ordre. Le fait est que dans \eqref{EQooGQOIooIkwbJV}, nous avons écrit le polynôme de degré \( 2n+1\) (et non seulement \( 2n\)), en sachant que le terme d'ordre \( 2n+1\) est nul.

    C'est pour cela que nous avons pu écrire \( \alpha(x)x^{2n+1}\) au lieu de \( \alpha(x)x^{2n}\) qui aurait été attendu.

    Même raisonnement pour le développement du sinus.
\end{proof}

\begin{remark}
    Quelques remarques concernant l'ordre du polynôme.
    \begin{enumerate}
        \item
            
Notons que nous aurions aussi pu écrire le reste sous la forme \( \alpha(x)x^{2n}\), mais ça aurait été avec une autre fonction \( \alpha\) : celle correspondant au développement à l'ordre \( 2n\) au lieu de \( 2n+1\).
\item

Les développements de sinus et de cosinus ont un terme sur deux qui est nul. C'est pour cela qu'en ayant une polynôme de degré \( 2p\), nous avons le développement d'ordre \( 2p+1\).

\item

   Nous aurions pu utiliser les dérivées données dans la proposition \ref{LEMooBBCAooHLWmno} et les valeurs spéciales \eqref{SUBEQooTTNNooXzApSM}.
    \end{enumerate}
\end{remark}

\begin{corollary}
    Il existe une fonction \( \alpha\colon \eR\to \eR\) telle que \( \lim_{t\to 0} \alpha(t)/t=0\) et 
    \begin{equation}        \label{EQooDLGIooXyfmtC}
        \sin(x)=x+\alpha(x).
    \end{equation}
    Nous avons la limite
    \begin{equation}
        \lim_{x\to 0} \frac{ \sin(x) }{ x }=1.
    \end{equation}
\end{corollary}

\begin{proof}
    Il s'agit de prendre la formule \eqref{EQooKYJAooRebHgc} avec \( n=0\). Cela donne tout de suite \eqref{EQooDLGIooXyfmtC}. Pour la limite, on divise par \( x\), ce qui donne (pour tout \( x\neq 0\)) 
    \begin{equation}
        \frac{ \sin(x) }{ x }=1+\frac{ \alpha(x) }{ x }.
    \end{equation}
    Et justement la fonction \( \alpha\) la la propriété que \( \lim_{x\to 0} \alpha(x)/x=0\).
\end{proof}

\begin{example}
    Cherchons le développement limité à l'ordre \( 5\) de \( \tan(x)=\frac{ \sin(x) }{ \cos(x) }\). Nous utilisons les développements de la proposition \ref{PROPooNPYXooTuwAHP} : 
    \begin{subequations}
        \begin{align}
            \sin(x)&=x-\frac{ x^3 }{ 6 }+\frac{ x^5 }{ 120 }+x^5\alpha_1(x)\\
            \cos(x)&=1-\frac{ x^2 }{ 2 }+\frac{ x^4 }{ 24 }+x^5\alpha_2(x).
        \end{align}
    \end{subequations}
    Nous calculons alors la division des deux polynômes, en classant les puissances dans l'ordre croissant (c'est le sens inverse de ce qui est fait pour la divisions euclidienne !) :
    \begin{equation*}
        \begin{array}[]{ccccccccccc|c}
            &x&-&\frac{1}{ 6 }x^3&+&\frac{1}{ 120 }x^5&&&&&&1-\frac{ 1 }{2}x^2+\frac{1}{ 24 }x^4\\
            \cline{12-12}
            -\Big( &x&-&\frac{ 1 }{2}x^3&+&\frac{1}{ 24 }x^5&\Big)& & && &x+\frac{1}{ 3 }x^3+\frac{ 2 }{ 15 }x^5\\
            \cline{2-6}
            & & &\frac{1}{ 3 }x^3&-&\frac{1}{ 30 }x^5& & & & && \\
            &&-\Big(  &\frac{1}{ 3 }x^3&-&\frac{1}{ 6 }x^5&+&\frac{1}{ 72 }x^7&\Big) & & \\
            \cline{4-8}
            & & & & &\frac{ 2 }{ 15 }x^5&-&\frac{1}{ 72 }x^7& & & \\
            & & &  &-\Big(  &\frac{ 2 }{ 15 }x^5&-&\frac{1}{ 15 }x^7&+&\frac{1}{ 180 }x^9&\Big)& \\
            \cline{6-10}
            & & & & & & &\frac{ 29 }{ 360 }x^7&-&\frac{1}{ 180 }x^9&& \\
        \end{array}
    \end{equation*}
    Nous avons continué la division jusqu'à obtenir un reste de degré plus grand que \( 5\). Le développement à l'ordre $5$ de la fonction tangente autour de zéro est alors (proposition \ref{PROPooMANAooXhuanS})
    \begin{equation}
        \tan(x)=x+\frac{1}{ 3 }x^3+\frac{ 2 }{ 15 }x^5+x^5\alpha(x).
    \end{equation}
    Notons que, vu que le reste ne nous intéresse pas vraiment, nous aurions pu ne pas calculer les coefficients des termes en \( x^7\) et \( x^8\). La dernière soustraction était également inutile.
\end{example}

%--------------------------------------------------------------------------------------------------------------------------- 
\subsection{Intégration}
%---------------------------------------------------------------------------------------------------------------------------

\begin{example}
    Comme nous le voyons sur le dessin suivant,
    \begin{equation}
        \int_{-3\pi/2}^{3\pi/2}\sin(x)\,dx=0
    \end{equation}
    parce que les deux parties bleues s'annulent avec les deux parties rouges (qui sont comptées comme des aires négatives).
    \begin{center}
       \input{auto/pictures_tex/Fig_JSLooFJWXtB.pstricks}
    \end{center}
\end{example}

%--------------------------------------------------------------------------------------------------------------------------- 
\subsection{Changement de variables dans une intégrale}
%---------------------------------------------------------------------------------------------------------------------------

\begin{example}     \label{EXooNIOZooWxciAC}
    Soit $V$ la région trapézoïdale de sommets $(0,-1)$, $(1,0)$, $(2,0)$, $(0,-2)$, comme à la figure~\ref{LabelFigZTTooXtHkcissLabelSubFigZTTooXtHkci0}. Calculons ensemble l'intégrale double
    \[
    \int_{V}e^{\frac{x+y}{x-y}}\,dV,
    \]
    avec le changement de variable $\psi(x,y)=(x+y,x-y)$. C'est-à-dire que nous considérons les nouvelles variables
    \begin{subequations}
        \begin{numcases}{}
            u=x+y\\
            v=x-y.
        \end{numcases}
    \end{subequations}
    Il faut remarquer d'abord que le changement de variable proposé est dans le mauvais sens. On écrit alors $\phi(u,v)=\psi^{-1}(u,v)=\big((u+v)/2, (u-v)/2\big)$, c'est-à-dire
    \begin{subequations}
        \begin{numcases}{}
            x=\frac{ u+v }{ 2 }\\
            y=\frac{ u-v }{2}.
        \end{numcases}
    \end{subequations}
    La région qui correspond à $V$ est $U$, le trapèze de sommets  $(-1,1)$, $(1,1)$, $(2,2)$ et $(-2,2)$, qu'on voit sur la figure~\ref{LabelFigZTTooXtHkcissLabelSubFigZTTooXtHkci1} et qu'on décrit par
    \[
    U=\{ (u,v)\in\eR^2\,\vert\, 1\leq v\leq 2, \, -v\leq u\leq v\}.
    \]

    % Celui-ci a été supprimée le 17 juillet 2014
    %\ref{LabelFigexamplechangementvariables}
    %\newcommand{\CaptionFigexamplechangementvariables}{Avant et après le changement de variables}
    %\input{auto/pictures_tex/Fig_examplechangementvariables.pstricks}

    %The result is on figure~\ref{LabelFigZTTooXtHkci}. % From file ZTTooXtHkci
    %See also the subfigure~\ref{LabelFigZTTooXtHkcissLabelSubFigZTTooXtHkci0}
    %See also the subfigure~\ref{LabelFigZTTooXtHkcissLabelSubFigZTTooXtHkci1}
    \newcommand{\CaptionFigZTTooXtHkci}{Avant et après le changement de variables}
    \input{auto/pictures_tex/Fig_ZTTooXtHkci.pstricks}

    On observe que $U$ est une région du premier type tandis que $V$ n'est pas du premier ou du deuxième type. Le déterminant de la  matrice  jacobienne de $\psi^{-1}$ est  $J_{\psi^{-1}}$,
    \begin{equation}
     J_{\psi^{-1}}(u,v)= \left\vert\begin{array}{cc}
    \frac{1}{2} & \frac{1}{2} \\
    \frac{1}{2}  & -\frac{1}{2}
    \end{array}\right\vert= -\frac{1}{2}.
    \end{equation}
    On a alors, en utilisant le fait que \( F(x)=a e^{x/a}\) est une primitive de \( f(x)= e^{x/a}\) (proposition \ref{ThoKRYAooAcnTut}) ainsi que le théorème fondamental de l'analyse (théorème \ref{ThoRWXooTqHGbC}),
    \[
    \int_{V}e^{\frac{x+y}{x-y}}\,dV=\int_{U}e^{\frac{u}{v}}\,\frac{1}{2}\,dV=\int_1^2\int_{-v}^{v}e^{\frac{u}{v}}\,\frac{1}{2}\, du\,dv= \frac{3}{4}(e-e^{-1}).
    \]
\end{example}



% This is part of (everything) I know in mathematics
% Copyright (c) 2011-2017, 2019
%   Laurent Claessens
% See the file fdl-1.3.txt for copying conditions.

%+++++++++++++++++++++++++++++++++++++++++++++++++++++++++++++++++++++++++++++++++++++++++++++++++++++++++++++++++++++++++++
\section{Isométries de l'espace euclidien}
%+++++++++++++++++++++++++++++++++++++++++++++++++++++++++++++++++++++++++++++++++++++++++++++++++++++++++++++++++++++++++++

Nous considérons l'espace affine euclidien \( A=\affE_n(\eR)\) modelé sur \( \eR^n\) avec sa métrique usuelle. Un premier grand résultat sera le théorème~\ref{ThoDsFErq} qui dira que les isométries de cet espace sont des applications linéaires.

%---------------------------------------------------------------------------------------------------------------------------
\subsection{Structure du groupe  \texorpdfstring{\( \Isom(\eR^n)\)}{Isom(Rn)} }
%---------------------------------------------------------------------------------------------------------------------------

Si vous ne voulez pas savoir ce qu'est un produit semi-direct de groupes, vous pouvez lire seulement le point~\ref{ITEMooLLUIooIGsknv} du théorème suivant, et passer directement à la remarque~\ref{REMooLUEZooIwvTqu}.
\begin{theorem}     \label{THOooQJSRooMrqQct}
    Un peu de structure sur \( \Isom(\eR^n)\).
    \begin{enumerate}
        \item       \label{ITEMooLLUIooIGsknv}
            L'application
            \begin{equation}
                \begin{aligned}
                    \psi\colon T(n)\times \gO(n)&\to \Isom(\eR^n) \\
                    (v,\Lambda)&\mapsto \tau_v\circ\Lambda
                \end{aligned}
            \end{equation}
            est une bijection. Ici,  \( T(n)\) est le groupe des translations de \( \eR^n\).
        \item
            Un couple \( (v,\Lambda)\in T(n)\times\SO(n)\) agit sur \( x\in \eR^n\) par
            \begin{equation}
                (v,\Lambda)x=\Lambda x+v
            \end{equation}
            au sens où \( \psi(v,\Lambda)x=\Lambda x+v\).
        \item       \label{ITEMooEWSIooNKzRxB}
            En tant que groupes,
            \begin{equation}
                \Isom(\eR^n)\simeq T(n)\times_{\rho}\gO(n)
            \end{equation}
            où \( \rho\) représente l'action adjointe de \( \gO(n)\) sur \( T(n)\) et \( \times_{\rho}\) dénotes le produit semi-direct de la définition~\ref{DEFooKWEHooISNQzi}.
    \end{enumerate}
\end{theorem}

\begin{proof}
    Point par point.
    \begin{enumerate}
        \item
            Prouvons que l'application proposée est injective et surjective. Notons aussi que ce point ne parle pas de structure de groupe, mais seulement d'une bijection en tant qu'ensembles.
            \begin{subproof}
                \item[Injection]
                    Si \( \psi(v,\Lambda)=\psi(w,\Lambda')\) alors en appliquant sur \( x=0\) nous avons tout de suite \( v=w\). Et ensuite \( \Lambda=\Lambda'\) est immédiat.
                \item[Surjection]
                    Une isométrie \( g\in\Isom(\eR^n)\) est une application \( g\colon \eR^n\to \eR^n\) telle que \( d(x,y)=d\big( g(x),g(y) \big)\). Dans le cas de \( \eR^n\) cela se traduit par
                    \begin{equation}
                        \| x-y \|=\big\| g(x)-g(y) \big\|,
                    \end{equation}
                    Vu que \( x\mapsto\| x \|\) est une forme quadratique, elle tombe sous le coup du théorème~\ref{ThoDsFErq}, ce qui nous permet de dire que \( g\) est affine. Or par définition une application est affine lorsqu'elle est la composée d'une translation et d'une application linéaire.

                    Donc \( g=\tau_v\circ \Lambda\) pour une certaine application linéaire isométrique \( \Lambda\colon \eR^n\to \eR^n\). L'application \( \Lambda\) est donc dans \( \gO(n)\) par la proposition \ref{PropKBCXooOuEZcS}\ref{ITEMooOWMBooHUatNb}.
            \end{subproof}
        \item
            C'est seulement le fait que \( (\tau_v\circ\Lambda)x=\tau_v\big( \Lambda x \big)=\Lambda(x)+v\).
        \item
            Nous allons étudier l'application
            \begin{equation}
                \psi\colon T(n)\times_{\rho}O(n)\to \Isom(\eR^n).
            \end{equation}
            \begin{subproof}
            \item[Le produit semi-direct est bien définit]
                Il faut montrer que
                \begin{equation}
                    \begin{aligned}
                        \rho\colon O(n)&\to \Aut\big( T(n) \big) \\
                        \Lambda&\mapsto \AD(\Lambda)
                    \end{aligned}
                \end{equation}
                est correcte.

                D'abord pour \( \Lambda\in O(n)\), nous avons bien \( \rho_{\Lambda}(\tau_v)\in T(n)\) parce qu'en appliquant à \( x\in \eR^n\),
                    \begin{equation}
                        (\Lambda\tau_v\Lambda^{-1})(x)=\Lambda\big( \tau_v(\Lambda^{-1} x) \big)=\Lambda\big( \Lambda^{-1}x+v \big)=x+\Lambda(v)=\tau_{\Lambda(v)}(x).
                    \end{equation}
                    Donc \( \rho_{\Lambda}(\tau_v)=\tau_{\Lambda(v)}\).

                    De plus, \( \rho_{\Lambda}\in\Aut\big( T(n) \big)\) parce que
                    \begin{equation}
                        \rho_{\Lambda}\big( \tau_v\circ \tau_w \big)=\rho_{\Lambda}(\tau_v)\circ\rho_{\Lambda}(\tau_v),
                    \end{equation}
                    comme on peut aisément vérifier que les deux membres sont égaux à \( \tau_{\Lambda(v+w)}\).
                \item[\( \psi\) est une bijection]
                    Cela est déjà vérifié.
                \item[\( \psi\) est un homomorphisme]
                    Nous avons d'une part
                    \begin{equation}
                        \psi\big( (v,g)(w,h) \big)=\psi\big( v\rho_g(w),gh \big)=\tau_v\circ g\circ\tau_w\circ g^{-1}\circ g\circ h=\tau_v\circ g\circ\tau_w\circ h.
                    \end{equation}
                    Et d'autre part,
                    \begin{equation}
                        \psi(v,g)\circ\psi(w,h)=\tau_v\circ g\circ \tau_w\circ h,
                    \end{equation}
                    ce qui est la même chose.
            \end{subproof}
    \end{enumerate}
\end{proof}

\begin{remark}      \label{REMooLUEZooIwvTqu}
    Notons au passage la loi de groupe sur les couples qui est donnée, pour tout \( v,v'\in \eR^n\), \( \Lambda,\Lambda'\in\SO(n)\), par
    \begin{equation}    \label{EqDiHcut}
            (v,\Lambda)\cdot(v',\Lambda')=(\Lambda v'+v,\Lambda\Lambda')
    \end{equation}
    comme le montre le calcul suivant :
    \begin{subequations}
        \begin{align}
            (v,\Lambda)\cdot(v',\Lambda')x&=(v,\Lambda)(\Lambda'x+v')\\
            &=\Lambda\Lambda'x+\Lambda v'+v\\
            &=(\Lambda v'+v,\Lambda\Lambda')x.
        \end{align}
    \end{subequations}
\end{remark}

\begin{proposition}[\cite{ooZYLAooXwWjLa}]      \label{PROPooDHYWooXxEXvl}
    Soient \( n\geq 1\) et \( R\) un élément de \( \gO(n)\) de déterminant \( -1\) tels que \( R^2=\id\). En posant \( C_2=\{ \id,R \}\) nous avons
    \begin{equation}
        \gO(n)=\SO(n)\times_{\rho} C_2
    \end{equation}
\end{proposition}

\begin{proof}
    Notons qu'un élément \( R\) comme décrit dans l'énoncé existe. Par exemple il y a l'application  \( (x_1,\ldots, x_n)\mapsto (-x_1,x_2,\ldots, x_n)\). 

    Cela étant dit, nous allons montrer que
    \begin{equation}
        \begin{aligned}
            \psi\colon \SO(n)\times C_2&\to \gO(n) \\
            (A,h)&\mapsto Ah.
        \end{aligned}
    \end{equation}
    est un isomorphisme.
    \begin{subproof}
        \item[Injectif]
            Soient \( A,B\in \SO(n)\) et \( h,k\in C_2\) tels que \( \psi(A,h)=\psi(B,k)\), c'est-à-dire tels que \( Ah=Bk\). Vu que \( \det(A)=\det(B)=1\) nous avons \( \det(h)=\det(k)\). Mais comme \( C_2\) contient un élément de déterminant \( 1\) et un élément de déterminant \( -1\), nous avons \( h=k\). De là \( A=B\).
        \item[Surjectif]
            Soit \( X\in\gO(n)\). Si \( \det(X)=1\) alors \( X\in \SO(n)\) et \( X=\psi(X,\mtu)\). Si par contre \( \det(X)=-1\) alors \( XR\in\SO(n)\) parce que \( \det(XR)=1\) et nous avons
            \begin{equation}
                \psi(XR,R)=XR^2=X.
            \end{equation}
        \item[Homomorphisme]
            Nous avons
            \begin{equation}
                \psi\Big( (A,h)(B,k) \Big)=\psi\big( A\rho_h(B),hk \big)=A(hBh^{-1})hk=AhBk,
            \end{equation}
            tandis que
            \begin{equation}
                \psi(A,h)\psi(B,k)=AhBk,
            \end{equation}
            qui est la même chose.
    \end{subproof}
\end{proof}

%+++++++++++++++++++++++++++++++++++++++++++++++++++++++++++++++++++++++++++++++++++++++++++++++++++++++++++++++++++++++++++
\section{Isométries dans $\eR^n$}
%+++++++++++++++++++++++++++++++++++++++++++++++++++++++++++++++++++++++++++++++++++++++++++++++++++++++++++++++++++++++++++

\begin{definition}
    Un \defe{hyperplan}{hyperplan} de \( \eR^n\) est un sous-espace affine de dimension \( n-1\).
\end{definition}

\begin{lemmaDef}
    Si un hyperplan \( H\) de \( \eR^n\) est donné, et si \( x\in \eR^n\), il existe un unique point \( y\in \eR^n\) tel que
    \begin{enumerate}
        \item
            \( x-y\perp H\),
        \item
            Le segment \( [x,y]\) coupe \( H\) en son milieu.
    \end{enumerate}
    La \defe{réflexion}{réflexion!par rapport à un hyperplan} \( \sigma_H\) est l'application $\sigma_H\colon \eR^n\to \eR^n $ qui à \( x\) fait correspondre ce \( y\).
\end{lemmaDef}

\begin{proof}
    Il faut vérifier que les conditions données définissent effectivement un unique point de \( \eR^n\). Soit \( H_0\) le sous-espace vectoriel parallèle à \( H\) et une base orthonormée \( \{ e_1,\ldots, e_{n-1} \}\) de \( H_0\). Nous complétons cela en une base orthonormée de \( \eR^n\) avec un vecteur \( e_n\). Si \( H=H_0+v\), quitte à décomposer \( v\) en une partie parallèle et une partie perpendiculaire à \( H\), nous avons
    \begin{equation}
        H=H_0+\lambda e_n
    \end{equation}
    pour un certain \( \lambda\).

    Une droite passant par \( x\) et perpendiculaire à \( H\) est de la forme \( t\mapsto x+te_n\). Si \( x=\sum_{i=1}^{n}x_ie_i\) alors l'unique point de cette droit à être dans \( H\) est le point tel que \(   x_ne_n+te_n=\lambda e_n   \), c'est-à-dire \( t=-x_n\). L'unique point \( y\) sur cette droite à être tel que \( [x,y ]\) coupe \( H\) en son milieu est celui qui correspond à \( t=-2x_n\).
\end{proof}

Notons au passage que cette preuve donne une formule pour \( \sigma_H\) :
\begin{equation}        \label{EQooRTWLooLPsUpY}
    \sigma_H(x)=\sum_{i=1}^{n-1}x_ie_i-x_ne_n.
\end{equation}
Il s'agit donc de changer le signe de la composante perpendiculaire à \( H\).

\begin{lemma}       \label{LEMooWYVRooQmWqvM}
    Dans cette même base si \( H_0\) est l'hyperplan parallèle à \( H\) et passant par l'origine, nous écrivons \( H=H_0+\lambda e_n\) pour un certain \( \lambda\). Alors
    \begin{equation}
        \sigma_H=\sigma_{H_0}+2\lambda e_n.
    \end{equation}
\end{lemma}

\begin{proof}
    Un élément \( x\in \eR^n\) peut être décomposé dans la base adéquate en \( x=x_H+x_ne_n\). Nous savons de la formule \eqref{EQooRTWLooLPsUpY} que
    \begin{equation}
        \sigma_H(x)=x_H-x_ne_n.
    \end{equation}
    Mais vu que \( \sigma_{H_0}(x_H)=x_H-2\lambda e_n\) nous avons
    \begin{equation}
            \sigma_{H_0}(x)+2\lambda e_n=\sigma_{H_0}(x_H+x_ne_n)+2\lambda e_N=x_H-2\lambda e_n-x_ne_n+2\lambda e_n=x_H-x_ne_n.
    \end{equation}
\end{proof}

Le lemme suivant est une généralisation du fait que tous les points de la médiatrice d'un segment sont à égale distance des deux extrémités du segment (très utile lorsqu'on étudie les triangles isocèles).
\begin{lemma}[\cite{ooZYLAooXwWjLa}]        \label{LEMooDPLYooJKZxiM}
    Soient deux points distincts \( x_0,y_0\in \eR^n\) l'ensemble \( H\subset \eR^n\) donné par
    \begin{equation}
        H=\{ x\in \eR^n\tq d(x,x_0)=d(x,y_0) \}.
    \end{equation}
    Alors \( H\) est l'hyperplan orthogonal au vecteur \( v=y_0-x_0\) et \( H\) passe par le milieu du segment \( [x_0,y_0] \).
\end{lemma}

\begin{proof}
    Nous savons que
    \begin{equation}
        d(x,x_0)^2=\langle x-x_0, x-x_0\rangle =\| x \|^2+\| x_0 \|^2-2\langle x, x_0\rangle,
    \end{equation}
    ou encore
    \begin{equation}
        \| x_0 \|^2-\| y_0 \|^2=2\langle x, x_0-y_0\rangle .
    \end{equation}
    En posant \( v=y_0-x_0\) et en considérant la forme linéaire
    \begin{equation}
        \begin{aligned}
            \beta\colon \eR^n&\to \eR \\
            x&\mapsto \langle x, v\rangle ,
        \end{aligned}
    \end{equation}
    Nous avons \( x\in H\) si et seulement si \( \beta(x)=\frac{ 1 }{2}\big( \| y_0 \|^2-\| x_0 \|^2 \big)=\lambda\). En d'autres termes, \( H=\beta^{-1}(\lambda)\). Par la proposition~\ref{PROPooAKJBooMkmsiV} la partie \( H\) est un sous-espace affine. C'est même un translaté de \( \ker(\beta)\), et comme \( \ker(\beta)\) est l'espace vectoriel des vecteurs perpendiculaires à \( v\), nous avons \( \dim(H)=\dim\big( \ker(\beta) \big)=n-1\).

    Le fait que \( H\) contienne le milieu du segment \( [x_0,y_0]\) est par définition.
\end{proof}

Pour le lemme suivant, et pour que la récurrence se passe bien nous disons que l'ensemble vide est un espace vectoriel de dimension \( -1\).
\begin{lemma}[\cite{ooYVHDooLeexeT}]       \label{LEMooJCDRooGAmlwp}
    Soit un espace euclidien \( E\) de dimension \( n\).
    \begin{enumerate}
        \item       \label{ITEMooFYEDooIJZBjP}
            Si \( f\) est une isométrie de \( E\) satisfaisant
            \begin{equation}
                \dim\big( \Fix(f) \big)=n-k
            \end{equation}
            alors \( f\) peut être écrit comme composition de \( k\) réflexions hyperplanes.
        \item       \label{ITEMooJTZVooWvyfDD}
            Une isométrie de \( E\) peut être écrite sous la forme de \( \rang(f-\id)\) réflexions, mais pas moins.
        \item       \label{ITEMooUCZWooSbyPwt}
            Toute isométrie de \( \eR^n\) peut être écrite comme composition de \( n+1\) réflexions.
    \end{enumerate}
\end{lemma}

\begin{proof}
    Les deux parties importantes à démontrer sont les points \ref{ITEMooFYEDooIJZBjP} et la partie «pas moins» de \ref{ITEMooJTZVooWvyfDD}. Le reste sont des reformulations.
        \begin{subproof}
        \item[Pour \ref{ITEMooFYEDooIJZBjP}]

    Nous faisons une récurrence sur \( k\geq 0\).

    Pour l'initialisation, si \( k=0\) alors \( \dim\big( \Fix(f) \big)=n\), c'est-à-dire que \( f\) fixe tout \( \eR^n\), autant dire que \( f\) est l'identité, une composition de zéro réflexions.

    Pour la récurrence, nous supposons que le lemme est démontré jusqu'à \( k\geq 0\). Soit donc \( f\in\Isom(\eR^n)\) tel que
    \begin{equation}
        \dim\big( \Fix(f) \big)=n-(k+1).
    \end{equation}
    Vu que \( k\geq 0\), la dimension de \( \Fix(f)\) est strictement plus petite que \( n\), donc il existe un \( x_0\in \eR^n\) tel que \( f(x_0)\neq x_0\). Nous posons
    \begin{equation}
        H=\{ x\in E\tq d(x,x_0)=d\big( x,f(x_0) \big)  \}.
    \end{equation}
    Par le lemme~\ref{LEMooDPLYooJKZxiM}, ce \( H\) est l'hyperplan orthogonal à \( v=f(x_0)-x_0\) et passant par le milieu du segment \( [x_0,f(x_0)]\).

    Nous posons \( g=\sigma_H\circ f\). Vu que \( g(x_0)=\sigma_H(f(x_0))=x_0\), ce \( x_0\) est un point fixe de \( g\). Le fait que \( \sigma_H\big( f(x_0) \big)=x_0\) est vraiment la définition de l'hyperplan \( H\).

    Nous avons donc
    \begin{equation}
        x_0\in\Fix(g)\setminus\Fix(f).
    \end{equation}
    Mais nous prouvons de plus que \( \Fix(f)\subset\Fix(g)\). En effet si \( y\in Fix(f)\) alors \( y\in H\) parce que
    \begin{equation}
        d(y,x_0)=d\big( f(y),f(x_0) \big)=d\big( y, f(x_0) \big).
    \end{equation}
    Vu que \( y\in H\) nous avons \( y\in \Fix(g)\) parce que
    \begin{equation}
        g(y)=\sigma_H\big( f(y) \big)=\sigma_H(y)=y.
    \end{equation}
    Tout cela pour dire que l'ensemble \( \Fix(g)\) est \emph{strictement} plus grand que \( \Fix(f)\). Et comme ce sont des espaces affines nous pouvons parler de dimension :
    \begin{equation}
        \dim\big( \Fix(g) \big)>\dim\big( \Fix(f) \big).
    \end{equation}
    Par hypothèse de récurrence, l'application \(  g\) peut être écrite comme composition de \( k\) réflexions. Donc l'application
    \begin{equation}
        f=\sigma_H\circ g
    \end{equation}
    est une composition de \( k+1\) réflexions.
        \item[Pour \ref{ITEMooJTZVooWvyfDD}, existence]

            Ce point est une reformulation du point \ref{ITEMooFYEDooIJZBjP}. Le fait est que \( \Fix(f)=\ker(f-\id)\) parce que \( x\in\Fix(f)\) si et seulement si \( f(x)=x\) si et seulement si \( (f-\id)x=0\). Nous utilisons le théorème du rang \ref{ThoGkkffA} à l'endomorphisme \( f-\id\) :
            \begin{equation}
                \dim\big( \Fix(f) \big)=\dim\big( \ker(f-\id) \big)=\dim(E)-\rang(f-\id).
            \end{equation}
            En remplaçant par les valeurs :
            \begin{equation}
                n-k=n-\rang(f-\id).
            \end{equation}
            Or le point \ref{ITEMooFYEDooIJZBjP} donnait \( f\) comme composée de \( n-k\) réflexions. Donc \( f\) est composée de \( \rang(f-\id)\) réflexions.
        \item[Pour \ref{ITEMooJTZVooWvyfDD}, «pas moins»]

            Supposons que \( f=\sigma_1\circ\ldots \circ \sigma_r\) où \( \sigma_i\) est la réflexion de l'hyperplan \( H_i\). Nous devons prouver que \( r\geq \rang(f-\id)\). Nous avons
            \begin{equation}
                \bigcap_{i=1}^rH_i\subset \ker(f-\id).
            \end{equation}
            D'autre part, la proposition \ref{PROPooRCLNooJpIMMl} nous donne \( \dim\bigcap_iH_i\geq n-r\). Donc
            \begin{equation}
                n-r\leq \dim\big( \bigcap_{i=1}^r \big)\leq\dim\big( \ker(f-\id) \big)=n-\rang(f-\id).
            \end{equation}
            Donc \( n-r\leq n-\rang(f-\id)\) ou encore
            \begin{equation}
                r\geq \rang(f-\id).
            \end{equation}

        \item[Pour \ref{ITEMooUCZWooSbyPwt}]
    La première partie de ce théorème n'est rien d'autre que le lemme~\ref{LEMooJCDRooGAmlwp} parce que le pire cas est celui où le fixateur de \( f\) est réduit à l'ensemble vide, et dans ce cas l'application \( f\) est une composition de \( n+1\) réflexions.
        \end{subproof}
\end{proof}

\begin{proposition}     \label{PROPooUSKEooUbNVfs}
    Un élément de \( \SO(3)\) qui fixe deux vecteurs linéairement indépendants est l'identité.
\end{proposition}

\begin{proof}
    Soit un élément \( A\in \SO(3)\) et deux vecteurs linéairement indépendants \( v_1,v_3\in \eR^3\) tels que \( Av_1=v_1\) et \( Av_2=v_2\). Vu que \( v_1\) et \( v_2\) sont linéairement indépendants, le théorème de la base incomplète \ref{ThonmnWKs} nous permet de considérer \( v_3\in \eR^3\) tel que \( \{ v_1,v_2,v_3 \}\) soit une base. Dans cette base, la matrice de \( A\) est de la forme
    \begin{equation}
        A=\begin{pmatrix}
            1    &   0    &   a    \\
            0    &   1    &   b    \\
            0    &   0    &   c
        \end{pmatrix}.
    \end{equation}
    Le déterminant de cette matrice est \( c\). Or \( \det(A)=1\) parce qu'elle est dans \( \SO(3)\). Donc \( c=1\). Le fait que \( A\) soit orthogonale implique que la troisième colonne doit être un vecteur de norme \( 1\). Donc \( a=b=0\).

    Donc \( A=\id\).
\end{proof}

\begin{corollary}       \label{CORooJCURooSRzSFb}
    Tout élément de \( \SO(3)\) peut être écrit comme composée de deux réflexions.
\end{corollary}

\begin{proof}
    Un élément de \( \SO(3)\) est une isométrie de \( \eR^3\) parce que si \( A\in\SO(3)\) alors\footnote{Opérateur orthogonal, définition \ref{DEFooYKCSooURQDoS}.}
    \begin{equation}
        \langle Ax, Ay\rangle =\langle A^*Ax, y\rangle =\langle x, y\rangle .
    \end{equation}
    Donc si le rang de \( A\) est \( k\), alors \( A\) est la composée de \( 3-k\) réflexions par le lemme \ref{LEMooJCDRooGAmlwp}.

    Si \( A=\id\), c'est bon parce que l'identité est la composée de deux réflexions égales. Nous supposons que \( A\) n'est pas l'identité.

    Comme discuté dans l'exemple \ref{EXooIPLOooSNfiWg}, l'opérateur \( A\) possède trois valeurs propres dans \( \eC\) dont une réelle, et deux complexes conjuguées. Nous les notons \( \lambda\in \eR\) et \( \alpha,\bar \alpha\in \eC\). Le déterminant de \( A\), qui vaut \( 1\), est le produit de ces trois valeurs propres, c'est-à-dire \( \lambda| \alpha |^2\). En particulier \( \lambda>0\).

    Si \( v\) est un vecteur propre correspondant à la valeur propre \( \lambda\), nous avons \(  \| v \|= \| Av \|=| \lambda |\| v \|\) parce que \( A\) est une isométrie. Donc \( \lambda=\pm 1\).

    Au final, \( \lambda=1\). Cela signifie que \( A\) laisse au moins un vecteur invariant. Vu que \( A\) n'est pas l'identité, la proposition \ref{PROPooUSKEooUbNVfs} nous indique qu'il n'y a pas d'autres vecteurs de \( \eR^3\) à être fixé par \( A\). Donc \( \dim\big( \Fix(A) \big)=1\) et le lemme \ref{LEMooJCDRooGAmlwp}\ref{ITEMooFYEDooIJZBjP} nous s'écrit avec \( n=3\), \( k=2\) et implique que \( A\) est la composée de deux réflexions.
\end{proof}

\begin{lemma}       \label{LEMooMCVKooKzmlAg}
    Soit un hyperplan \( H\) et un vecteur \( v\) de \( \eR^n\). Nous avons
    \begin{equation}
        \tau_v\circ \sigma_H\circ\tau_v^{-1}=\sigma_{\tau_v(H)}.
    \end{equation}
\end{lemma}

\begin{proof}
    Pour ce faire nous considérons une base adaptée. Les vecteurs \( \{ e_1,\ldots, e_{n-1} \}\) forment une base orthonormée de \( H_0\) et \( e_n\) complète en une base orthonormée de \( \eR^n\). Soit \( H_0\) l'hyperplan parallèle à \( H\) et passant par l'origine; nous avons, pour un certain \( \lambda\in \eR\),
    \begin{equation}
        H=H_0+\lambda e_n
    \end{equation}
    D'un autre côté, le vecteur \( v\) peut être décomposé en \( v=v_1+v_2\) où \( v_1\perp H\) et \( v_2\parallel H\). Alors
    \begin{equation}
        \tau_v(H)=H+v=H+v_2=H_0+\lambda e_n+v_2.
    \end{equation}
    Nous pouvons maintenant utiliser le lemme~\ref{LEMooWYVRooQmWqvM} pour exprimer la transformation \( \sigma_{\tau_v(H)}\) :
    \begin{equation}        \label{EQooNYKFooXprXav}
        \sigma_{\tau_v(H)}(x)=\sigma_{H_0}(x)+ 2\lambda e_n+2v_2
    \end{equation}

    Mais d'autre part,
    \begin{equation}
        (\tau_v\circ \sigma_H\circ\tau_{v}^{-1})(x)=v+\sigma_H(x-v)=v+\sigma_{H_0}(x-v)+2\lambda e_n.
    \end{equation}
    Vue la décomposition de \( v=v_1+v_2\) nous avons \( \sigma_{H_0}(v)=-v_1+v_2\) et donc
    \begin{equation}        \label{EQooGOHEooALPRFB}
        (\tau_v\circ \sigma_H\circ\tau_{v}^{-1})(x)= v+  \sigma_{H_0}(x)+v_1-v_2+2\lambda e_n=\sigma_{H_0}+2v_1+2\lambda e_n.
    \end{equation}
    Les expressions \eqref{EQooNYKFooXprXav} et \eqref{EQooGOHEooALPRFB} coïncident, d'où l'égalité recherchée.
\end{proof}

\begin{theorem}[\cite{ooZYLAooXwWjLa}]      \label{THOooWBIYooCtWoSq}
    Une isométrie de \( (\eR^n,d)\) préserve l'orientation si et seulement si est elle composition d'un nombre pair de réflexions.
\end{theorem}

\begin{proof}
    Nous définissons
    \begin{equation}
        \begin{aligned}
            \epsilon\colon \Isom(\eR^n)&\to \{ \pm 1 \} \\
            \tau_v\circ \alpha&\mapsto \det(\alpha)
        \end{aligned}
    \end{equation}
    où nous nous référons à la décomposition unique d'un élément de \( \Isom(\eR^n)\) sous la forme \( \tau_v\circ \alpha\) avec \( \alpha\in O(n)\) donnée par le théorème~\ref{THOooQJSRooMrqQct}\ref{ITEMooEWSIooNKzRxB}.

    Le noyau de \( \epsilon\) est alors la partie
    \begin{equation}
        \ker(\epsilon)=\eR^n\times_{\AD}\SO(n).
    \end{equation}
    Une isométrie \( f\) préserve l'orientation si et seulement si \( \epsilon(f)=1\). Vu que toutes les isométries sont des compositions de réflexions (première partie), il nous suffit de montrer que \( \epsilon(\epsilon_H)=-1\) pour qu'une isométrie préserve l'orientation si et seulement si elle est composition d'un nombre pair de réflexions.

    Nous commençons par prouver que pour tout vecteur \( v\), \( \epsilon\big( \sigma_H \big)=\epsilon\big( \sigma_{\tau_v(H)} \big)\). Pour cela nous utilisons le lemme~\ref{LEMooMCVKooKzmlAg} et le fait que \( \epsilon\) est un homomorphisme :
    \begin{equation}
        \epsilon(\sigma_{\tau_v(H)})=\epsilon(\tau_v)\epsilon(\sigma_H)\epsilon(\tau_v^{-1})=\epsilon(\sigma_H)
    \end{equation}
    parce que la partie linéaire d'une translation est l'identité (et donc \( \epsilon(\tau_v)=1\) pour tout \( v\)).

    Nous avons donc \( \epsilon(\sigma_H)=\epsilon(\sigma_{H_0})\). En ce qui concerne \( \sigma_{H_0}\), dans la base adaptée la matrice est
    \begin{equation}
        \sigma_{H_0}=\begin{pmatrix}
             1   &       &       &       \\
                &   \ddots    &       &       \\
                &       &   1    &       \\
                &       &       &   -1
         \end{pmatrix},
    \end{equation}
    dont le déterminant est \( -1\).
\end{proof}

Pour en savoir plus sur le groupe des isométries, il faut lire le théorème de Cartan-Dieudonné dans \cite{JGAdTA}.

%+++++++++++++++++++++++++++++++++++++++++++++++++++++++++++++++++++++++++++++++++++++++++++++++++++++++++++++++++++++++++++ 
\section{Groupes finis d'isométries}
%+++++++++++++++++++++++++++++++++++++++++++++++++++++++++++++++++++++++++++++++++++++++++++++++++++++++++++++++++++++++++++

\begin{definition}      \label{DEFooCUYLooAlbtzv}
    Si \( X\) est une partie finie de \( \eR^n\), le \defe{barycentre}{barycentre!cas vectoriel} de \( X\) est le point
    \begin{equation}
        B_X=\frac{1}{ | X | }\sum_{x\in X}x
    \end{equation}
    où \( | X |\) est le cardinal de \( X\).
\end{definition}
Cela est à mettre en relation avec la définition dans le cadre affine~\ref{LemtEwnSH}.

\begin{lemma}[\cite{ooZYLAooXwWjLa}]        \label{LEMooSEZYooYceLIb}
    Les applications affines de \( \eR^n\) préservent le barycentre\footnote{Définition \ref{DEFooCUYLooAlbtzv}.} des parties finies.
\end{lemma}

\begin{proof}
    Soit une partie finie \( X\) de \( \eR^n\) et une application affine \( f\in\Aff(\eR^n)\). Nous devons prouver que
    \begin{equation}
        f(B_X)=B_{f(X)}.
    \end{equation}

    Nous savons que toute application affine est une composée de translation et d'une application linéaire : \( f=\tau_v\circ g\) avec \( v\in \eR^n\) et \( g\in \GL(n,\eR)\). Nous vérifions le résultat séparément pour \( \tau_v\) et pour \( g\).

    D'une part,
    \begin{equation}
        B_{\tau_v(X)}=\frac{1}{ | \tau_v(X) | }\sum_{y\in \tau_v(X)}y=\frac{1}{ | X | }\sum_{x\in X}(x+v)=B_x+\frac{1}{ | X | }\sum_{x\in X}v=B_x+v=\tau_v(B_X).
    \end{equation}
    Nous avons utilisé le fait que \( X\) et \( \tau_v(X)\) possèdent le même nombre d'éléments, ainsi que le fait d'avoir une somme de \( | X |\) termes tous égaux à \( v\).

    D'autre part,
    \begin{equation}
        B_{g(X)}=\frac{1}{ | X | }\sum_{x\in X}g(x)=g\big( \frac{1}{ |X | }\sum_{x\in X}x \big)=g(B_X)
    \end{equation}
    où nous avons utilisé la linéarité de \( g\) dans tous ses retranchements.
\end{proof}

\begin{proposition}     \label{PROPooLAEBooWdcBoe}
    Points fixes d'un sous-groupe.
    \begin{enumerate}
        \item
            Soit \( H\) un sous-groupe finie des isométries de \( (\eR^n,d)\). Alors il existe \( v\in \eR^n\) tel que \( f(v)=v\) pour tout \( f\in H\).
        \item
            Si \( H\) est un sous-groupe de \( \Isom(\eR^n,d)\) n'acceptant pas de points fixes, alors il est infini.
    \end{enumerate}
\end{proposition}

\begin{proof}
    Le groupe \( H\) agit sur \( \eR^n\), et si \( x\in \eR^n\) nous pouvons considérer son orbite 
    \begin{equation}
        Hx=\{f(x)\tq f\in H\},
    \end{equation}
    qui est une partie finie de \( \eR^n\). Considérons son barycentre $v=B_{Hx}$. Soit \( f\in H\). Alors  
    \begin{subequations}
        \begin{align}
            f(v)&=f(B_{Hx})\\
            &=B_{f(Hx)}     \label{SUBEQooOQBZooYlIbgN}\\
            &=B_{Hx}        \label{SUBEQooXWEGooSoezYg}\\
            &=v,
        \end{align}
    \end{subequations}
    Justifications:
    \begin{itemize}
        \item Pour \eqref{SUBEQooOQBZooYlIbgN}, c'est le lemme \ref{LEMooSEZYooYceLIb}.
        \item Pour \eqref{SUBEQooXWEGooSoezYg}, c'est le fait que, \( f\in H\) étant donné, l'application \( g\mapsto fg\) est une bijection de \( H\), donc
            \begin{equation}
                f(Hx)=\{(fg)(x)\tq h\in H\}=\{g(x)\tq g\in H\}=Hx.
            \end{equation}
    \end{itemize}
    Bref, \( v\) est fixé par \( H\).

    La seconde affirmation n'est rien d'autre que la contraposée de la première.
\end{proof}

\begin{proposition}     \label{PROPooEUFIooDUIYzi}
    À propos de groupes finis d'isométries.
    \begin{enumerate}
        \item
            Tout sous-groupe fini de \( \Isom(\eR^n)\) est isomorphe à un sous-groupe fini de \( \gO(n)\).
        \item
            Tout sous-groupe fini de \( \Isom^+(\eR^n)\) est isomorphe à un sous-groupe fini de \( \SO(n)\).
    \end{enumerate}
\end{proposition}

\begin{proof}
    Soit \( H\) un sous-groupe fini de \( \Isom(\eR^n)\) et \( v\in \eR^n\) un élément fixé par \( H\) (comme garantit par la proposition~\ref{PROPooLAEBooWdcBoe}). Nous posons
    \begin{equation}
        \begin{aligned}
            \phi\colon H&\to \Isom(\eR^n) \\
            f&\mapsto \tau_v^{-1}\circ f\circ \tau_v.
        \end{aligned}
    \end{equation}

    \begin{subproof}
        \item[\( \phi\) est un homomorphisme]
            Les opération du type \( \phi=\AD(\tau_v)\) sont toujours des homomorphismes.
        \item[\( \phi\) consiste à extraire la partie linéaire]
            Si \( f=\tau_w\circ g\) alors
            \begin{subequations}
                \begin{align}
                    \phi(f)(x)&=(\tau_{-v}\circ\tau_w\circ g\circ\tau_v)(x)\\
                    &=\tau_{w-v}(   g(x)+g(v)  )\\
                    &=g(x)+g(v)-v+w
                \end{align}
            \end{subequations}
            Mais \( g(v)+w=f(v)\) et nous savons que \( f(v)=v\). Donc il ne reste que \( \phi(f)(x)=g(x)\).
        \item[\( \phi\) est injective]
            Si \( f=\tau_w\circ g\) vérifie \( \phi(f)=\id\), il faut en particulier que \( g=\id\). Mais \( H\) est fini et ne peut donc pas contenir de translations non triviales. Donc \( w=0\) et \( f=\id\).
    \end{subproof}
    Donc \( \phi\) est une injection à valeur dans les transformations linéaires de \( \Isom(\eR^n)\). Autrement dit, \( \phi\) est un isomorphisme entre \( H\) et son image, laquelle image est dans \( \gO(n)\).

    En ce qui concerne la seconde partie, si \( f\in\Isom^+(\eR^n)\), alors \( \phi(f)\) y est aussi, tout en étant linéaire. Donc \( \phi(f)\in \SO(n)\).
\end{proof}

L'extraction de la partie linéaire est injective ? Certe c'est prouvé, mais on peut se demander ce qu'il se passe si \( H\) contient deux éléments qui ont la même partie linéaire. Cela n'est pas possible parce si \( f_1=\tau_{w_1}\circ g\) et \( f_2=\tau_{w_2}\circ g\) sont dans \( H\) alors \( f_1f_2^{-1}=\tau_{w_1+w_2}\) est également dans \( H\), ce qui n'est pas possible si \( H\) est fini.
 
%---------------------------------------------------------------------------------------------------------------------------
\subsection{Groupe diédral}
%---------------------------------------------------------------------------------------------------------------------------
\label{subsecHibJId}

\begin{proposition}     \label{PROPooUPPTooZBFvPg}
    Les racines de l'unité dans \( \eC\), c'est-à-dire la partie
    \begin{equation}
        \{  e^{2ik\pi/n},k=0,\ldots, n-1 \},
    \end{equation}
    forment un polynôme régulier.
\end{proposition}
% TODOooVDNMooJFwymI : prouver que les racines de l'unité forment un polygone régulier, PROPooUPPTooZBFvPg

%///////////////////////////////////////////////////////////////////////////////////////////////////////////////////////////
\subsubsection{Définition et générateurs : vue géométrique}
%///////////////////////////////////////////////////////////////////////////////////////////////////////////////////////////

\begin{definition}  \label{DEFooIWZGooAinSOh}
    Le \defe{groupe diédral}{groupe!diédral} \( D_n\)\nomenclature[R]{\( D_n\)}{groupe diédral} (\( n\geq 3\)) est le groupe des isométries de \( (\eC,d)\) laissant invariant l'ensemble
    \begin{equation}
        \{  e^{2ik\pi/n},k=0,\ldots, n-1 \}
    \end{equation}
    des racines de l'unité.
\end{definition}
\index{groupe!agissant sur un ensemble!diédral}
\index{groupe!en géométrie}
\index{groupe!fini!diédral}
\index{groupe!permutation!diédral}

\begin{normaltext}
    La proposition \ref{PROPooUPPTooZBFvPg} nous permet de dire que le groupe diédral est le groupe des isométries de \( \eR^2\) laissant invariant un polygone régulier à \( n\) côtés.
    C'est un peu pour cela que nous n'avons défini \( D_n\) que pour \( n\geq 3\); et un peu aussi pour une raison technique qui arrivera dans la preuve du lemme \ref{LEMooCUVPooMZKnzo}.
\end{normaltext}

\begin{lemma}       \label{LEMooCUVPooMZKnzo}
    Nous avons
    \begin{equation}
        D_n\subset O(2,\eR).
    \end{equation}
\end{lemma}

\begin{proof}
    Si \( f\in D_n\), alors \( f( e^{2ik\pi/n}) \) doit être l'un des \(  e^{2ik'\pi/n}\), et vu que \( f\) conserve les longueurs dans \( \eC\), nous devons avoir
    \begin{equation}
        1=d(0, e^{2ik\pi/n})=d\big( f(0), e^{2ik'\pi/n} \big).
    \end{equation}
    Donc \( f(0)\) est à l'intersection de tous les cercles de rayon \( 1\) centrés en les \(  e^{2ik\pi/n}\), ce qui montre que \( f(0)=0\) (dès que \( n\geq 3\)). Par conséquent notre étude du groupe diédral ne doit prendre en compte que les isométries vectorielles de \( \eR^2\). En d'autres termes
    \begin{equation}
        D_n\subset O(2,\eR).
    \end{equation}
\end{proof}

\begin{proposition}[\cite{tzHydF}]
    Le groupe \( D_n\) contient un sous-groupe cyclique d'ordre \( 2\) et un sous-groupe cyclique d'ordre \( n\).
\end{proposition}

\begin{proof}
    Nous notons \( s\) la conjugaison complexe\footnote{C'est une réflexion; la réflexion d'axe \( \eR\) dans \( \eC\).}. C'est un élément d'ordre \( 2\) qui est dans \( D_n\) parce que
    \begin{equation}    \label{EqSUshknP}
        s\big(  e^{2ki\pi/n} \big)= e^{2(n-k)i\pi/n}.
    \end{equation}

    De la même façon, la rotations d'angle \(2\pi/n\), que l'on note \( r\), agit sur les racines de l'unité et engendre un le groupe d'ordre \( n\) des rotations d'angle \(2 k\pi/n\).
\end{proof}

\begin{proposition}[\cite{tzHydF}]
    Si \( s\) est la conjugaison complexe et \( r\) la rotation d'angle \( 2\pi/n\), alors \( (sr)^2=\id\).
\end{proposition}

\begin{proof}
    Si \( z^n=1\), alors
    \begin{equation}
        (srsr)z=srs e^{2 i\pi/n}z=sr\big( e^{-2\pi i/n\bar z}\big)=s\bar z=z.
    \end{equation}
\end{proof}

\begin{proposition}[\cite{tzHydF}] \label{PropLDIPoZ}
    Nous notons \( s\) la conjugaison complexe et \( r\) la rotation d'angle \( 2\pi/n\).
    \begin{enumerate}
        \item
            Le groupe diédral \( D_n\) est engendré par \( s\) et \( r\). 
        \item       \label{ITEMooOEBHooULRmZk}
            Tous les éléments de \( D_n\) s'écrivent sous la forme \( r^m\) ou \( s\circ r^m\).
    \end{enumerate}
\end{proposition}
\index{groupe!diédral!générateurs (preuve)}
\index{racine!de l'unité}
\index{géométrie!avec nombres complexes}
\index{géométrie!avec des groupes}
\index{isométrie!de l'espace euclidien \( \eR^2\)}

\begin{proof}
    Nous considérons les points \( A_0=1\) et \( A_k= e^{2ki\pi/n}\) avec \( k\in\{ 1,\ldots, n-1 \}\). Par convention, \( A_n=A_0\). L'action des éléments \( s\) et \( r\) sur ces points est
    \begin{subequations}
        \begin{align}
            r(A_k)&=A_{k+1}\\
            s(A_k)&=A_{n-k}.
        \end{align}
    \end{subequations}
    Cette dernière est l'équation \eqref{EqSUshknP}.

    Soit \( f\in D_n\). Étant donné que c'est une isométrie de \( \eR^2\) avec un point fixe (le point \( 0\)), \( f\) est soit une rotation soit une réflexion.
    %TODO : il faut démontrer ce point et mettre un lien vers ici.

    Supposons pour commencer que un des \( A_k\) est fixé par \( f\). Dans ce cas \( f\) a deux points fixes : \( O\) et \( A_k\) et est donc la réflexion d'axe \( (OA_k)\). Dans ce cas, nous avons \( f=s\circ r^{n-2k}\). En effet
    \begin{equation}
        s\circ r^{n-2k}(A_k)=s(A_{k+n-2k})=s(A_{n-k})=A_k.
    \end{equation}
    Donc \( O\) et \( A_k\) sont deux points fixes de l'isométrie \( f\); donc \( f\) est bien la réflexion sur le bon axe.

    Nous passons à présent au cas où \( f\) ne fixe aucun des \( A_k\).
    \begin{enumerate}
        \item
            Supposons que \( f\) soit une rotation. Si \( f(A_k)=A_m\), alors l'angle de la rotation est
            \begin{equation}
                \frac{ 2(m-k)\pi }{ n },
            \end{equation}
            et donc \( f=r^{m-k}\), qui est de la forme demandée.
        \item
            Supposons à présent que \( f\) soit une réflexion d'axe \( \Delta\). Cette fois, \( \Delta\) ne passe par aucun des points \( A_k\), par contre \( \Delta\) passe par \( 0\). Nous commençons par montrer que \( \Delta\) doit être la médiatrice d'un des côtés \( [A_p,A_{p+1}]\) du polygone. Vu que \( \Delta\) passe par \( O\) et n'est aucune des droites \( (OA_k)\), cette droite passe par l'intérieur d'un des triangles \( OA_pA_{p+1}\) et intersecte donc le côté correspondant.

            Notre tâche est de montrer que \( \Delta\) coupe \( [A_p,A_{p+1}]\) en son milieu. Dans ce cas, \( \Delta\) sera automatiquement perpendiculaire parce que le triangle \( OA_pA_{p+1}\) est isocèle en \( O\). Nommons \( l\) la longueur des côtés du polygone, \( P=\Delta\cap[A_p,A_{p+1}]\), \( x=d(A_p,P)\) et \( \delta=d(A_p,\Delta)\). Vu que \( f\) est la symétrie d'axe \( \Delta\), nous avons aussi \( d\big( f(A_p),\Delta \big)=\delta\) et \( d\big( A_p,f(A_p) \big)=2\delta\). D'autre part, par la définition de la distance, \( \delta<x\). Si \( x<\frac{ l }{2}\), alors \( \delta<\frac{ \delta }{2}\) et donc \( d\big( A_p,f(A_p) \big)<l\). Or cela est impossible parce que le polygone ne possède aucun sommet à distance plus courte que \( l\) de \( A_p\).

            De la même manière si \( x>\frac{ l }{2}\), nous raisonnons avec \( A_{p+1}\) pour obtenir une contradiction. Nous en concluons que la seule possibilité est \( x=\frac{ l }{2}\), et donc \( f(A_p)=A_{p+1}\). Montrons alors que \( f=s\circ r^{n-2p-1}\). Il faut montrer que c'est une réflexion qui envoie \( A_p\) sur \( A_{p+1}\). D'abord c'est une réflexion parce que
            \begin{equation}
                \det(sr^{n-2p-1})=\det(s)\det(r^{n-2p-1})=-1
            \end{equation}
            parce que \( \det(s)=-1\) alors que \( \det(r^k)=1\) parce que \( r\) est une rotation dans \( \SO(2)\). Ensuite nous avons
            \begin{equation}
                s\circ r^{n-2p-1}(A_p)=s(A_{p+n-2p-1})=s(A_{n-p-1})=A_{n-(n-p-1)}=A_{p+1}.
            \end{equation}

            Donc \( s\circ r^{n-2p-1}\) est bien une réflexion qui envoie \( A_p\) sur \( A_{p+1}\).

    \end{enumerate}
\end{proof}

\begin{corollary}   \label{CorWYITsWW}
La liste des éléments de \( D_n\) est
\begin{equation}
    D_n=\{ 1,r,\ldots, r^{n-1},s,sr,\ldots, sr^{n-1} \}
\end{equation}
et \( | D_n |=2n\).
\end{corollary}

\begin{proof}
    Nous savons par la proposition~\ref{PropLDIPoZ} que tous les élément de \( D_n\) s'écrivent sous la forme \( r^k\) ou \( sr^k\). Vu que \( r\) est d'ordre \( n\), il ne faut considérer que \( k\in\{ 1,\ldots, n-1 \}\). Les éléments \( 1\), \( r\),\ldots, \( r^{n-1}\) sont tous différents, et sont (pour des raisons de déterminant) tous différents des \( sr^k\). Les isométries \( sr^k\) sont toutes différentes entre elles pour essentiellement la même raison :
    \begin{equation}
        sr^k(A_p)=s(A_{p+k})=A_{n-p+k}
    \end{equation}
    donc si \( k\neq k'\), \( sr^k(A_p)\neq sr^{k'}(A_p)\). La liste des éléments de \( D_n\) est donc
    \begin{equation}
        D_n=\{ 1,r,\ldots, r^{n-1},s,sr,\ldots, sr^{n-1} \}
    \end{equation}
    et donc \( | D_n |=2n\).
\end{proof}

\begin{example}     \label{EXooHNYYooUDsKnm}
    Nous considérons le carré \( ABCD\) dans \( \eR^2\) et nous cherchons les isométries de \( \eR^2\) qui laissent le carré invariant. Nous nommons les points comme sur la figure~\ref{LabelFigIsomCarre}. La symétrie d'axe vertical est nommée \( s\) et la rotation de \( 90\) degrés est notée \( r\).
    \newcommand{\CaptionFigIsomCarre}{Le carré dont nous étudions le groupe diédral.}
    \input{auto/pictures_tex/Fig_IsomCarre.pstricks}

    Il est facile de vérifier que toutes les symétries axiales peuvent être écrites sous la forme \( r^is\). De plus le groupe engendré par \( s\) agit sur le groupe engendré par \( r\) parce que
    \begin{equation}
        (srs^{-1})(A,B,C,D)=sr(B,A,D,C)=s(A,D,C,B)=(B,C,D,A),
    \end{equation}
    c'est-à-dire \( srs^{-1}=r^{-1}\). Nous sommes alors dans le cadre du corolaire~\ref{CoroGohOZ} et nous pouvons écrire que
    \begin{equation}
        D_4=\gr(r)\times_{\sigma}\gr(s).
    \end{equation}
\end{example}

%///////////////////////////////////////////////////////////////////////////////////////////////////////////////////////////
\subsubsection{Table de multiplication}
%///////////////////////////////////////////////////////////////////////////////////////////////////////////////////////////

La proposition \ref{PropLDIPoZ} nous indique que tous les éléments de \( D_n\) s'écrivent sous la forme \( s^{\epsilon}\circ r^m\) avec \( \epsilon\in\{ 0,1 \}\). Nous allons maintenant écrire la table de multiplication pour de telles transformations de \( \eC\).

\begin{lemma}       \label{LEMooBNJFooAbhsUa}
    Si \( R\) est une rotation autour de \( 0\) (dans \( \eC\)), et si \( s\) est la conjugaison complexe, alors
    \begin{equation}
        rs=sr^{-1}
    \end{equation}
\end{lemma}

\begin{proof}
    Il s'agit seulement d'un calcul en écrivant \( R\) comme la multiplication par \(  e^{i\alpha}\). Nous avons
    \begin{equation}
        (Rs)z= e^{i\alpha}\bar z=s\big(  e^{-i\alpha}z \big)=sR^{-1}z.
    \end{equation}
\end{proof}

\begin{proposition}     \label{PROPooPYDLooLgiUjk}
    Si \( \epsilon_1,\epsilon_2\in\{ 0,1 \}\) et si \( k,l\in \eZ\) nous avons
    \begin{equation}
        (s^{\epsilon_1}r^k)(s^{\epsilon_2}r^l)=s^{\epsilon_1+\epsilon_2}r^{l+(-1)^{\epsilon_1}k}.
    \end{equation}
\end{proposition}

\begin{proof}
    Si \( \epsilon_2=1\) alors nous utilisons le lemme \ref{LEMooBNJFooAbhsUa} pour trouver
    \begin{equation}
        (s^{\epsilon_1}r^k)(s^{\epsilon_2}r^l)=s^{\epsilon_1}(r^ks^{\epsilon_2})r^l=s^{\epsilon_1}s^{\epsilon_2}r^{-k}r^l.
    \end{equation}
    La proposition est déjà prouvée dans ce cas.

    Passons à \( \epsilon_2=0\). Dans ce cas nous avons
    \begin{equation}
        (s^{\epsilon_1}r^k)(s^{\epsilon_2}r^l)=s^{\epsilon_1}r^{k+l},
    \end{equation}
    et c'est bon.
\end{proof}

%///////////////////////////////////////////////////////////////////////////////////////////////////////////////////////////
\subsubsection{Générateurs : vue abstraite}
%///////////////////////////////////////////////////////////////////////////////////////////////////////////////////////////

\begin{normaltext}      \label{NORMooCCUEooRRENed}
    Nous allons montrer que \( D_n\) peut être décrit de façon abstraite en ne parlant que de ses générateurs. Nous considérons un groupe \( G\) engendré par des éléments \( a\) et \( b\) tels que
    \begin{enumerate}
        \item
            \( a\) est d'ordre \( 2\),
        \item
            \( b\) est d'ordre \( n\) avec \( n\geq 3\),
        \item
            \( abab=e\).
    \end{enumerate}
    Nous allons prouver que ce groupe doit avoir la même liste d'éléments que celle du corolaire~\ref{CorWYITsWW}.
\end{normaltext}

\begin{proposition}[\cite{tzHydF}]
    Le groupe \( G\) n'est pas abélien.
\end{proposition}

\begin{proof}
    Nous savons que \( abab=e\), donc \( abab^{-1}=b^{-2}\), mais \( b^{-2}\neq e\) parce que \( b\) est d'ordre \( n>2\). Donc \( abab^{-1}\neq e\). En manipulant un peu :
    \begin{equation}
        e\neq abab^{-1}=(ab)(ba^{-1})^{-1}=(ab)(ba)^{-1}
    \end{equation}
    parce que \( a^{-1}=a\). Donc \( ab\neq ba\).
\end{proof}

\begin{lemma}[\cite{tzHydF}]        \label{LemKKXdqdL}
    Pour tout \( k\) entre \( 1\) et \( n-1\) nous avons
    \begin{equation}
        \AD(a)b^k=ab^ka^{-1}=ab^ka=b^{-k}.
    \end{equation}
\end{lemma}

\begin{proof}
    Nous faisons la démonstration par récurrence. D'abord pour \( k=1\), nous devons avoir \( aba=b^{-1}\), ce qui est correct parce que par construction de \( G\) nous avons \( abab=e\). Ensuite nous supposons que le lemme tient pour \( k\) et nous regardons ce qu'il se passe avec \( k+1\) :
    \begin{equation}
            ab^{k+1}ba=ab^kba=\underbrace{ab^ka}_{b^{-k}}\underbrace{aba}_{b^{-1}}=b^{-k}b^{-1}=b^{-(k+1)}.
    \end{equation}
\end{proof}

\begin{proposition}     \label{PROPooVQARooWuKHMZ}
    L'élément \( a\) n'est pas une puissance de \( b\).
\end{proposition}

\begin{proof}
    Supposons le contraire : \( a=b^k\). Dans ce cas nous aurions
    \begin{equation}
        e=(ab)(ab)=b^{k+1}b^{k+1}=b^{2k+2}=b^{2k}b^2=a^2b^2=b^2,
    \end{equation}
    ce qui signifierait que \( b\) est d'ordre \( 2\), ce qui est exclu par construction.
\end{proof}

\begin{proposition}[\cite{tzHydF}]      \label{PROPooEPVGooQjHRJp}
    La liste des éléments de \( G\) est donnée par
    \begin{equation}
        G=\{ 1,b,\cdots,b^{n-1},a,ab,\ldots, ab^{n-1} \}=\{ a^{\epsilon}b^k\}_{\substack{\epsilon=0,1\\k=0,\ldots, n-1}}
    \end{equation}
    Les éléments de ces listes sont distincts.
\end{proposition}

\begin{proof}
    Étant donné que \( a\) n'est pas une puissance de \( b\), les éléments \( 1\), \( a\), \( b\),\ldots, \( b^{n-1}\) sont distincts. De plus si \( k\) et \( m=k+p\) sont deux éléments distincts de \( \{ 1,\ldots, n-1 \}\), nous avons \( ab^k\neq ab^m\) parce que si \( ab^k=ab^{k+p}\), alors \( a=ab^p\) avec \( p<n\), ce qui est impossible. Pour la même raison, \( ab^k\neq e\), et \( ab^k\neq b^m\).

    Au final les éléments \( 1,a,b,\ldots, b^{n-1},ab,\ldots, ab^{n-1}\) sont tous différents. Nous devons encore voir qu'il n'y en a pas d'autres.

    Par définition le groupe \( G\) est engendré par \( a\) et \( b\), donc tout élément \( x\in G\) s'écrit $x=a^{m_1}b^{k_1}\ldots a^{m_r}b^{k_r}$ pour un certain \( r\) et avec pour tout \( i\), \( k_i\in\{ 1,\ldots, n-1 \}\) (sauf \( k_r\) qui peut être égal à zéro) et \( m_i=1\), sauf \( m_1\) qui peut être égal à zéro. Donc
    \begin{equation}
        x=a^mb^{k_1}ab^{k_2}a\ldots b^{k_{r-1}}ab^{k_r}
    \end{equation}
    où \( m\) et \( k_r\) peuvent éventuellement être zéro. En utilisant le lemme~\ref{LemKKXdqdL} sous la forme \( b^{k_i}a=ab^{-k_i}\), quitte à changer les valeurs des exposants, nous pouvons passer tous les \( a \) à gauche et tous les \( b\) à droite pour finir sous la forme \( x=a^kb^m\).

    Donc non, il n'existe pas d'autres éléments dans \( G\) que ceux déjà listés.
\end{proof}

\begin{lemma}[\cite{MonCerveau}]        \label{LemooNFRIooPWuikH}
    Tout élément de \( G\) s'écrit de façon unique sous la forme \( a^{\epsilon}b^k\) ou \( b^ka^{\epsilon}\) avec \( \epsilon=0,1\) et \( k=0,\ldots, n-1\).
\end{lemma}

\begin{proof}
    Nous commençons par la forme \( a^{\epsilon}b^k\). L'existence est la proposition~\ref{PROPooEPVGooQjHRJp}. Pour l'unicité nous supposons \( a^{\epsilon}b^k=a^{\sigma}b^l\) et nous décomposons en \( 4\).
    \begin{subproof}
        \item[\( \epsilon=0\), \( \sigma=0\)]
            Alors \( b^k=b^l\). Mais \( b\) étant d'ordre \( n\) et \( k,l\) étant égaux au maximum à \( n-1\), cette égalité implique \( k=l\).
        \item[\( \epsilon=0\), \( \sigma=1\)]
            Alors \( b^k=ab^l\), ce qui donne \( a=b^{k-l}\), ce qui est interdit par la proposition~\ref{PROPooVQARooWuKHMZ}.
        \item[\( \epsilon=1\), \( \sigma=0\)]
            Même problème.
        \item[\( \epsilon=1\), \( \sigma=1\)]
            Encore une fois \( b^k=b^l\) implique \( k=l\).
    \end{subproof}
    En ce qui concerne la forme \( b^ka^{\epsilon}\), l'existence est à montrer. Soit l'élément \( g=a^{\epsilon}b^k\) et cherchons à le mettre sous la forme \( b^la^{\sigma}\). Si \( \epsilon=0\) c'est évident. Sinon \( \epsilon=1\) et nous avons par le lemme~\ref{LemKKXdqdL}
    \begin{equation}
        ab^k=b^{-k}a^{-1}=b^{-k}b^na=b^{-k}a.
    \end{equation}
    En ce qui concerne l'unicité, nous refaisons \( 4\) cas pour \( b^ka^{\epsilon}=b^la^{\sigma}\) comme précédemment et ils se traitement exactement comme précédemment.
\end{proof}

\begin{theorem}     \label{THOooYITHooTNTBuG}
    Les groupes \( G\) et \( D_n\) sont isomorphes.
\end{theorem}

\begin{proof}
        Nous utilisons l'application
    \begin{equation}
        \begin{aligned}
            \psi\colon G&\to D_n \\
            a^kb^m&\mapsto s^kr^m.
        \end{aligned}
    \end{equation}
    C'est évidemment bien défini et bijectif, mais c'est également un homomorphisme parce que si nous calculons \( \psi\) sur un produit, nous devons comparer
    \begin{equation}        \label{EqBULPilp}
        \psi\big( a^{k_1}b^{m_1}a^{k_2}b^{m_2} \big)
    \end{equation}
    avec
    \begin{equation}        \label{EqIVEIphI}
        \psi\big( a^{k_1}b^{m_1}\big)\psi\big(a^{k_2}b^{m_2} \big)= s^{k_1}r^{m_1}s^{k_2}r^{m_2}.
    \end{equation}
    Vu que \( D_n\) et \( G\) ont les mêmes propriétés qui permettent de permuter \( a\) et \( b\) ou \( s\) et \( r\), l'expression à l'intérieur du \( \psi\) dans \eqref{EqBULPilp} se simplifie en \( a^kb^m\) avec les même \( k\) et \( n\) que l'expression à droite dans \eqref{EqIVEIphI} ne se simplifie en \( s^kr^m\).
\end{proof}

\begin{corollary}
    Toutes les propriétés démontrées pour \( G\) sont vraies pour \( D_n\). En particulier, avec quelques redites :
    \begin{enumerate}
        \item
            Le groupe \( D_n\) peut être défini comme étant le groupe engendré par un élément \( s\) d'ordre \( 2\) et un élément \( r\) d'ordre \( n-1\) assujettis à la relation \( srsr=e\).
        \item
            Le groupe \( D_n\) n'est pas abélien.
        \item
            Pour tout \( k\in\{ 1,\ldots, n-1 \}\) nous avons \( sr^ks=r^{-k}\).
        \item
            L'élément \( s\) ne peut pas être obtenu comme une puissance de \( r\).
        \item
            La liste des éléments de \( D_n\) est
            \begin{equation}
                D_n=\{ 1,r,\ldots, r^{n-1},s,sr,\ldots, sr^{n-1} \}
            \end{equation}
        \item
            Le groupe diédral \( D_n\) est d'ordre \( 2n\).
    \end{enumerate}
\end{corollary}

\begin{proposition}
    En posant \( C_n=\{ r^k \}_{k=0,\ldots, n-1}\) et \( C_2=\{ a^{\epsilon} \}_{\epsilon=0,1}\), nous pouvons exprimer \( D_n\) comme le produit semi-direct
    \begin{equation}
        D_n=C_n\times_{\rho}C_2
    \end{equation}
    où \( \rho\) désigne l'action adjointe.
\end{proposition}

\begin{proof}
    L'isomorphisme est :
    \begin{equation}
        \begin{aligned}
            \psi\colon C_n\times_{\rho}C_2&\to D_n \\
            (b^k,a^{\epsilon})&\mapsto b^ka^{\epsilon}.
        \end{aligned}
    \end{equation}
    \begin{subproof}
        \item[Action adjointe]
            L'application \( \rho_{a^{\epsilon}}=\AD(a^{\epsilon})\) est toujours un homomorphisme. Vu que \( a^{\epsilon}\) est soit \( e\) soit \( a\), nous allons nous restreindre à \( a\) et oublier l'exposant \( \epsilon\). Il faut montrer que\( \AD(a)\in\Aut(C_n)\). En utilisant le lemme~\ref{LemKKXdqdL},
            \begin{equation}
                \AD(a)b^k=ab^ka^{-1}=b^{-k}=b^{n-k}.
            \end{equation}
            L'application \( \AD(a)\colon C_n\to C_n\) est donc bijective et homomorphique. Ergo isomorphisme.
        \item[Injectif]
            Si \( \psi(b^k,a^{\epsilon})=\psi(b^l,a^{\sigma})\), alors par unicité du lemme~\ref{LemooNFRIooPWuikH} nous avons \( k=l\) et \( \epsilon=\sigma\).
        \item[Surjectif]
            Par la partie «existence»  du lemme~\ref{LemooNFRIooPWuikH}.
        \item[Homomorphisme]
            Lorqu'on prend deux sous-groupes d'un même groupe (ici le groupe des isométrique de \( \eR^2\)), et que l'on tente de faire un produit demi-direct en utilisant l'action adjointe, nous avons toujours un homomorphisme. Dans notre cas, le calcul est :
            \begin{equation}
                \psi\big( (b^k,a^{\epsilon})(b^l,a^{\sigma}) \big)=b^k\rho_{a^{\epsilon}}(b^l)a^{\epsilon+\sigma}=b^ka^{\epsilon}b^la^{-\epsilon}a^{\epsilon+\sigma}=b^ka^{\epsilon}b^la^{\sigma}=\psi(b^k,a^{\epsilon})\psi(b^l,a^{\sigma}).
            \end{equation}
    \end{subproof}
\end{proof}

%///////////////////////////////////////////////////////////////////////////////////////////////////////////////////////////
\subsubsection{Classes de conjugaison}
%///////////////////////////////////////////////////////////////////////////////////////////////////////////////////////////
\label{subsubsecZQnBcgo}

Pour les classes de conjugaison du groupe diédral nous suivons \cite{HRIMAJJ}.

D'abord pour des raisons de déterminants\footnote{Vous notez qu'ici nous utilisons un argument qui utilise la définition de \( D_n\) comme isométries de \( \eR^2\). Si nous avions voulu à tout prix nous limiter à la définition «abstraite» en termes de générateurs, il aurait fallu trouver autre chose.}, les classes des éléments de la forme \( r^k\) et de la forme \( sr^k\) ne se mélangent pas. Nous notons \( C(x)\) la classe de conjugaison de \( x\), et \( y\cdot x=yxy^{-1}\).

Les relations que nous allons utiliser sont
\begin{subequations}
    \begin{align}
        sr^ks=r^{-k}\\
        rs=sr^{-1}=sr^{n-1}.
    \end{align}
\end{subequations}

La classe de conjugaison qui ne rate jamais est bien entendu \( C(1)={1}\). Nous commençons les vraies festivités \( C(r^{m})\). D'abord \( r^k\cdot r^m=r^m\), ensuite
\begin{equation}
    (sr^k)\cdot r^m=sr^kr^mr^{-k}s^{-1}=sr^ms^{-1}=r^{-m}.
\end{equation}
Donc
\begin{equation}    \label{EqVFfFxgi}
    C(r^m)=\{ r^m,r^{-m} \}.
\end{equation}
À ce niveau il faut faire deux remarques. D'abord si \( m>\frac{ n }{2}\), alors \( C(r^m)\) est la classe de \( C^{n-m}\) avec \( n-m<\frac{ n }{2}\). Donc les classes que nous avons trouvées sont uniquement à lister avec \( m<\frac{ n }{2}\). Ensuite si \( m=\frac{ n }{2}\) alors \( r^m=r^{-m}\) et la classe est un singleton. Cela n'arrive que si \( n\) est pair.

Nous passons ensuite à \( C(s)\). Nous avons
\begin{equation}
    r^k\cdot s=r^ksr^{-k}=ssr^ksr^{-k}=sr^{-k}r^{-k}=sr^{n-2k},
\end{equation}
et
\begin{equation}
    (sr^k)\cdot s=\underbrace{sr^ks}_{r^{-k}}r^{-k}s^{-1}=r^{-2k}s=r^{n-2k}s=sr^{(n-1)(n-2k)}=sr^{n^2-2kn-n+2k}=sr^{2k}.
\end{equation}
donc
\begin{equation}
    C(s)=\{ sr^{n-2k},sr^{2k} \}_{k=0,\ldots, n-1}.
\end{equation}
Ici aussi l'écriture n'est pas optimale : peut-être que pour certains \( k\) il y a des doublons. Nous reportons l'écriture exacte à la discussion plus bas qui distinguera \( n\) pair de \( n\) impair. Notons juste que si \( n\) est pair, l'élément \( sr\) n'est pas dans la classe \( C(s)\).

Nous en faisons donc à présent le calcul en gardant en tête le fait qu'il n'a de sens que si \( n\) est pair. D'abord
\begin{equation}
    s\cdot (sr)=ssrs=rs=sr^{n-1}.
\end{equation}
Ensuite
\begin{equation}
    (sr^k)\cdot (sr)=sr^ksrr^{-k}s=r^{-2k+1}s=sr^{2k-1}.
\end{equation}
Avec \( k=\frac{ n }{2}\), cela rend \( s\cdot (sr)\), donc pas besoin de le recopier. Nous avons
\begin{equation}
    C(sr)=\{ sr^{2k-1} \}_{k=1,\ldots, n-1}.
\end{equation}

%///////////////////////////////////////////////////////////////////////////////////////////////////////////////////////////
\subsubsection{Le compte pour $ n$ pair}
%///////////////////////////////////////////////////////////////////////////////////////////////////////////////////////////
\label{SubsubsecROVmHuM}

Si \( n\) est pair, nous avons les classes
\begin{subequations}
    \begin{align}
        C(1)&=\{ 1 \}       &&& 1\text{ élément}\\
        C(r^m)&=\{ r^m,r^{m-1} \}&\text{ pour }&0<m<\frac{ n }{2}   & \frac{ n }{2}-1\text{ fois } 2\text{ éléments}\\
        C(r^{n/2})&=\{ r^{n/2} \}   &&&  1\text{ élément}\\
        C(s)&=\{ sr^{2k} \}_{k=0,\ldots, \frac{ n }{2}-1} &&&  \frac{ n }{2}\text{ éléments}\\
        C(sr)&=\{ sr^{2k+1} \}_{k=0,\ldots, \frac{ n }{2}-1} &&&  \frac{ n }{2}\text{ éléments}.
    \end{align}
\end{subequations}
Au total nous avons bien listé \( 2n\) éléments comme il se doit, dans \(  \frac{ n }{2}+3\) classes différentes.

%///////////////////////////////////////////////////////////////////////////////////////////////////////////////////////////
\subsubsection{Le compte pour $ n$ impair}
%///////////////////////////////////////////////////////////////////////////////////////////////////////////////////////////
\label{GJIzDEP}

Si \( n\) est impair, nous avons les classes
\begin{subequations}
    \begin{align}
        C(1)&=\{ 1 \}       &&& 1\text{ élément}\\
        C(r^m)&=\{ r^m,r^{m-1} \}&\text{ pour }&0<m<\frac{ n-1 }{2}   & \frac{ n-1 }{2}\text{ fois } 2\text{ éléments}\\
        C(s)&=\{ sr^k \}_{k=0,\ldots, n-1} &&&  n\text{ éléments}
    \end{align}
\end{subequations}
Au total nous avons bien listé \( 2n\) éléments comme il se doit, dans \(  \frac{ n+3 }{2}\) classes différentes.

%---------------------------------------------------------------------------------------------------------------------------
\subsection{Applications : du dénombrement}
%---------------------------------------------------------------------------------------------------------------------------

%///////////////////////////////////////////////////////////////////////////////////////////////////////////////////////////
\subsubsection{Le jeu de la roulette}
%///////////////////////////////////////////////////////////////////////////////////////////////////////////////////////////
\label{pTqJLY}
\index{groupe!fini}
\index{groupe!de permutations}
\index{groupe!et géométrie}
\index{combinatoire}
\index{dénombrement}

Soit une roulette à \( n\) secteurs que nous voulons colorier en \( q\) couleurs\cite{HEBOFl}. Nous voulons savoir le nombre de possibilités à rotations près. Soit d'abord \( E\) l'ensemble des coloriages possibles sans contraintes; il y a naturellement \( q^n\) possibilités. Sur l'ensemble \( E\), le groupe cyclique \( G\) des rotations d'angle \( 2\pi/n\) agit. Deux coloriages étant identiques si ils sont reliés par une rotation, la réponse à notre problème est donné par le nombre d'orbites de l'action de \( G\) sur \( E\) qui sera donnée par la formule du théorème de Burnside~\ref{THOooEFDMooDfosOw}.

Nous devons calculer \( \Card\big( \Fix(g) \big)\) pour tout \( g\in G\). Soit \( g\), un élément d'ordre \( d\) dans \( G\). Si \( g\) agit sur la roulette, chaque secteur a une orbite contenant \( d\) éléments. Autrement dit, \( g\) divise la roulette en \( n/d\) secteurs. Un élément de \( E\) appartenant à \( \Fix(g)\) doit colorier ces \( n/d\) secteurs de façon uniforme; il y a \( q^{n/d}\) possibilités.

Il reste à déterminer le nombre d'éléments d'ordre \( d\) dans \( G\). Un élément de \( G\) est donné par un nombre complexe de la forme \(  e^{2ik\pi/n}\). Les éléments d'ordre \( d\) sont les racines primitives\footnote{Une racine non primitive \( 8\)ième de l'unité est par exemple \( i\). Certes \( i^8=1\), mais \( i^4=1\) aussi. Le nombre \( i\) est d'ordre \( 4\).} \( d\)ièmes de l'unité. Nous savons que --par définition-- il y a \( \varphi(d)\) telles racines primitives de l'unité. Bref il y a \( \varphi(d)\) éléments d'ordre \( d\) dans \( G\).

La formule de Burnside nous donne maintenant le nombre d'orbites :
\begin{equation}
    \frac{1}{ n }\sum_{d|n}\varphi(d)q^{n/d}.
\end{equation}
Cela est le nombre de coloriage possibles de la roulette à \( n\) secteurs avec \( q\) couleurs.

%///////////////////////////////////////////////////////////////////////////////////////////////////////////////////////////
\subsubsection{L'affaire du collier}
%///////////////////////////////////////////////////////////////////////////////////////////////////////////////////////////
\label{siOQlG}

Nous avons maintenant des perles de \( q\) couleurs différentes et nous voulons en faire un collier à \( n\) perles. Cette fois non seulement les rotations donnent des colliers équivalents, mais en outre les symétries axiales (il est possible de retourner un collier, mais pas une roulette). Le groupe agissant sur \( E\) est maintenant le groupe diédral\footnote{Définition~\ref{DEFooIWZGooAinSOh}.}\index{diédral}\index{groupe!diédral} \( D_n\) conservant un polygone a \( n\) sommets.

Nous devons séparer le cas \( n\) impair du cas \( n\) pair.

Si \( n\) est impair, alors les axes de symétries passent par un sommet par le milieu du côté opposé. Le groupe \( D_n\) contient \( n\) symétries axiales. Nous avons donc maintenant
\begin{equation}
    | G |=2n.
\end{equation}
Nous écrivons la formule de Burnside
\begin{equation}
    \Card(\Omega)=\frac{1}{ 2n }\sum_{g\in G}\Card\big( \Fix(g) \big).
\end{equation}
Si \( g\) est une rotation, le travail est déjà fait. Si \( g\) est une symétrie, nous avons le choix de la couleur du sommet par lequel passe l'axe et le choix de la couleur des \( (n-1)/2\) paires de sommets. Cela fait
\begin{equation}
    qq^{(n-1)/2}=q^{\frac{ n+1 }{2}}
\end{equation}
possibilités. Nous avons donc
\begin{equation}
    \Card(\Omega)=\frac{1}{ 2n }\left( \sum_{d|n}q^{n/d}\varphi(d)+nq^{\frac{ n+1 }{2}} \right).
\end{equation}

Si \( n\) est pair, le choses se compliquent un tout petit peu. En plus de symétries axiales passant par un sommet et le milieu du côté opposé, il y a les axes passant par deux sommets opposés. Pour colorier un collier en tenant compte d'une telle symétrie, nous pouvons choisir la couleur des deux perles par lesquelles passe l'axe ainsi que la couleur des \( (n-2)/2\) paires de perles. Cela fait en tout
\begin{equation}
    q^2q^{\frac{ n-2 }{2}}=q^{\frac{ n+2 }{2}}.
\end{equation}
Le groupe \( G\) contient \( n/2\) tels axes.

Notons que cette fois \( G\) ne contient plus que \( n/2\) symétries passant par un sommet et un côté. L'ordre de $G$ est donc encore \( 2n\). La formule de Burnside donne
\begin{equation}
    \Card(\Omega)=\frac{1}{ 2n }\left( \sum_{d\divides n}\varphi(d)q^{n/d}+\frac{ n }{2}q^{(n+2)/2}+\frac{ n }{2}q^{n/2} \right).
\end{equation}

%--------------------------------------------------------------------------------------------------------------------------- 
\subsection{Points fixés par une affinité}
%---------------------------------------------------------------------------------------------------------------------------

\begin{lemma}[\cite{JGAdTA}]        \label{LEMooGUEGooTUXRsQ}
    Si \( n\geq 3\), alors toute droite est intersection de deux plans non isotropes.
\end{lemma}

\begin{proposition}[\cite{ooZYLAooXwWjLa}]      \label{PROPooVEEUooJQmmkN}
    Si une isométrie de \( \eR^n\) fixe un ensemble \( F\) de points, alors elle fixe l'espace affine engendrée par \( F\).
\end{proposition}

\begin{proof}
    Soit \( f\in \Isom(\eR^n)\) fixant \( F\). Par le théorème~\ref{ThoDsFErq}, c'est une application affine et l'ensemble \( \Fix(f)\) des points fixés par \( f\) est un sous-espace affine de \( \eR^n\), grâce à la proposition~\ref{PROPooYRCJooIcmUVI}.

    Donc \( \Fix(f)\) est un espace affine contenant \( F\). Vu que l'espace affine engendré par \( F\) est l'intersection de tous les espaces affines contenant \( F\), il est en particulier contenu dans \( \Fix(f)\).
\end{proof}

\begin{corollary}       \label{CORooZHZZooDgTzsW}
    Si \( f\) et \( g\) sont des isométries de \( \eR^n\) qui coïncident sur \( F\), alors elles coïncident sur l'espace affine engendré par \( F\).
\end{corollary}

\begin{proof}
    Nous considérons \( h=g^{-1}\circ f\) qui est une isométrie de \( \eR^n\) fixant \( F\). Elle fixe donc, par la proposition~\ref{PROPooVEEUooJQmmkN}, l'espace affine engendré par $F$. Or tout point fixé par \( h\) est un point sur lequel \( g\) et \( f\) coïncident.
\end{proof}

%+++++++++++++++++++++++++++++++++++++++++++++++++++++++++++++++++++++++++++++++++++++++++++++++++++++++++++++++++++++++++++
\section{Classification des isométries dans \( \eR^2\)}
%+++++++++++++++++++++++++++++++++++++++++++++++++++++++++++++++++++++++++++++++++++++++++++++++++++++++++++++++++++++++++++

%---------------------------------------------------------------------------------------------------------------------------
\subsection{Réflexions}
%---------------------------------------------------------------------------------------------------------------------------

Soit un espace vectoriel \( E\) de dimension \( 2\) muni d'un produit scalaire\footnote{Définition~\ref{DefVJIeTFj}.}. Cela pourrait très bien être \( \eR^2\), mais nous allons nous efforcer de l'appeler \( E\) pour rester un peu général.

\begin{lemmaDef}[Caractérisation des réflexions]        \label{DEFooLJKDooUaamen}
    Soit une droite \( \ell\) de \( \eR^2\). Il existe une unique application \( f\colon \eR^2\to \eR^2\) telle que
    \begin{enumerate}
        \item
            \( f(x)=x\) pour tout \( x\in \ell\).
        \item
            \( f\) échange les côtés de \( \ell\).
        \item
            \( f\) laisse invariants les droites perpendiculaires à \( \ell\) et les cercles dont le centre est sur \( \ell\).
    \end{enumerate}
    Cette application est la \defe{réflexion}{réflexion!dans \( \eR^2\)} d'axe \( \ell\).
\end{lemmaDef}

\begin{proof}
    Soit \( x\) hors de \( \ell\) et \( p\) la droite perpendiculaire à \( \ell\) et passant par \( x\). Nous avons \( f(x)\in p\). En nommant \( P\) l'intersection entre \( \ell\) et \( p\), nous considérons le cercle \( S(P,\| Px \|)\) qui est un cercle dont le centre est sur \( \ell\). Il contient \( x\) et donc \( f(x)\in S(P,\| Px \|)\).

    Donc \( f(x)\in p\cap S(P,\| Px \|)\). L'intersection entre un cercle et une droite contient de façon générique deux points. L'un est \( x\), mais \( f(x)=x\) n'est pas possible parce que \( x\) est hors de \( \ell\) et \( f\) doit inverser les côtés de \( \ell\). Donc \( f(x)\) est l'autre.

    Cela prouve l'unicité. En ce qui concerne l'existence, il suffit de noter que la réflexion \( \sigma_{\ell}\) satisfait les contraintes.
\end{proof}

\begin{lemma}       \label{LEMooZSDRooUkNYer}
    Soit une droite \( \ell\) est \( A\in E\). Alors
    \begin{equation}        \label{EQooVUQDooKuwszl}
        \sigma_{\ell}(A)=2\pr_{\ell}(A)-A
    \end{equation}
    où \( \pr_{\ell}\) est l'opération de projection orthogonale sur la droite \( \ell\).
\end{lemma}

\begin{proof}
    Nous posons \( f(X)=2\pr_{\ell}(X)-X\) et nous allons montrer que \( f=\sigma_{\ell}\) en vérifiant les conditions de la définition~\ref{DEFooLJKDooUaamen}.
    % Il faut laisser le saut de ligne ici.
    Nous nous gardons bien de faire un raisonnement du type «nous allons montrer que \( f\) et \( \sigma_{\ell}\) coïncident sur deux points, et sont donc égales par le corolaire~\ref{CORooZHZZooDgTzsW}» parce que nous ne savons pas encore que \( \sigma_{\ell}\) est une application affine, ni même que c'est une isométrie.

    Si \( X\in\ell\) alors \( \pr_{\ell}(X)=X\) et nous avons \( f(X)=2X-X=X\). Donc \( \ell\) est conservée.

    En ce qui concerne les deux côtés de \( \ell\), il existe une application linéaire \( s\colon E\to \eR\) et une constante \( c\in \eR\) telles qu'en posant \( \ell(X)=s(X)+c\), la droite \( \ell\) soit le lieux des points \( X\) tels que \( \ell(X)=0\). Un côté de la droite est \( \ell<0\) et l'autre côté est \( \ell>0\). Nous avons :
    \begin{subequations}
        \begin{align}
            \ell\big( f(A) \big)&=\ell\big( 2\pr_{\ell}(A)-A \big)\\
            &=s(2\pr_{\ell}(A)-A)+c\\
            &=2s\big( \pr_{\ell}(A) \big)-s(A)+c\\
            &=s\big( \pr_{\ell}(A) \big)-s(A)\\
            &=-c-s(A)\\
            &=-\ell(A)
        \end{align}
    \end{subequations}
    où nous avons utilisé le fait que, \( \pr_{\ell}(A)\) étant sur \( \ell\), \( s\big( \pr_{\ell}(A) \big)+c=0\). Nous avons donc \( \ell\big( f(A) \big)=-\ell(A)\), ce qui indique que \( A\) et \( f(A)\) sont de part et d'autre de \( \ell\).

    Si \( d\) est une droite perpendiculaire à \( \ell\) et si \( A\in d\) alors \( f(A)=2\pr_{\ell}(A)-A=  \big( \pr_{\ell}(A)-A \big)+A\in d  \) parce que \( \pr_{\ell}(A)\in d\) du fait que \( d\) soit précisément perpendiculaire à \( \ell\). Nous avons aussi utilisé le fait que si \( A,B,C\in d\) alors \( (B-A)+C\in d\); pensez que \( B-A\) est un vecteur directeur et que \( C\) est un point de \( d\).

    Enfin soit \( K\in\ell\) et un cercle \( S(K,r)\) centré en \( K\). Soit \( A\in S(K,r)\); nous devons vérifier que \( f(A)=S(K,r)\). Le segment \( [A,f(A)]\) est par définition perpendiculaire à \( \ell\). Soit \( M\), le milieu, qui est sur la droite \( \ell\). Les triangles \( AMK\) et \( f(A)MK\) sont rectangles en \( M\), et \( \| AM \|=\| Mf(A) \|\). Le théorème de Pythagore donne \( \| AK \|=\| f(A)K \|\). Donc le cercle centré en \( K\) est donc préservé par \( f\).

    Nous en déduisons que \( f=\sigma_{\ell}\).
\end{proof}

\begin{proposition}[\cite{ooIIMKooJpdFyk}]      \label{PROPooFSVEooWmJsnv}
    Une réflexion est une isométrie de \( (E,d)\) où \( d(A,B)=\| A-B \|\).
\end{proposition}

\begin{proof}
    Soient \( A,B\in E\); il faut vérifier que \( \| A-B \|=\| \sigma_{\ell}(A)-\sigma_{\ell}(B) \|\). Pour cela nous écrivons
    \begin{equation}
        B-A=\underbrace{B-\pr_{\ell}(B)}_{=a}+\underbrace{\pr_{\ell}(B)-\pr_{\ell}(A)}_{=b}+\underbrace{\pr_{\ell}(A)-A}_{=c}.
    \end{equation}
    Vu que \( b\perp a\) et \( b\perp c\) nous avons
    \begin{equation}
        \| B-A \|=\langle B-A, B-A\rangle =\| a \|^2+2\langle a, c\rangle +\| b \|^2+\| c \|^2.
    \end{equation}
    Nous pouvons faire le même jeu avec \( \sigma_{\ell}(B)-\sigma_{\ell}(A)\) en tenant compte du fait que \( \pr_{\ell}\big( \sigma_{\ell}(X) \big)=\pr_{\ell}(X)\) et que
    \begin{equation}
        \sigma_{\ell}(A)-\pr_{\ell}(A)=2\pr_{\ell}(A)-A-\pr_{\ell}(A)=-\big( A-\pr_{\ell}(A) \big).
    \end{equation}
    Là nous avons utilisé le lemme~\ref{LEMooZSDRooUkNYer}. Ce que nous trouvons est que
    \begin{equation}
        \sigma_{\ell}(B)-\sigma_{\ell}(A)=-a+b-c,
    \end{equation}
    et donc encore une fois
    \begin{equation}
        \| \sigma_{\ell}(B)-\sigma_{\ell}(A) \|=\| a \|^2-2\langle a, c\rangle +\| b \|^2+\| c \|^2.
    \end{equation}
\end{proof}

\begin{remark}
    Il faut bien comprendre que si l'axe de la réflexion ne passe par par \( 0\) (le zéro de l'espace vectoriel normé \( (E,\| . \|)\)), la réflexion n'est pas une isométrie de \( (E,\| . \|)\) au sens où nous n'avons pas \( \| \sigma_{\ell}(x) \|=\| x \|\).
\end{remark}

%---------------------------------------------------------------------------------------------------------------------------
\subsection{Segment, plan médiateur et équidistance}
%---------------------------------------------------------------------------------------------------------------------------

\begin{lemma}   \label{LEMooSZZWooPDHnGl}
    Un point \( M\) est sur la médiatrice du segment \( [A,B]\) si et seulement si \( \| M-A \|=\| M-B \|\).
\end{lemma}

\begin{lemma}       \label{LEMooVBVUooOTFFXT}
    Soient \( A\) et \( B\) de points de \( \eR^3\). Alors le plan médiateur du segment \( [A,B]\) est le lieu des points de \( \eR^3\) à être équidistants de \( A\) et \( B\).
\end{lemma}

\begin{proof}
    Nous nommons \( \sigma\) ce plan.

    Soit \( X\) un point équidistant de \( A\) et \( B\). Alors dans le plan \( (A,B,X)\), le triangle \( ABX\) est isocèle en \( X\), et la hauteur issue de \( X\) coupe perpendiculairement \( [A,B]\) en son milieu. Cela prouve que \( X\) est dans le plan médiateur du segment \( [A,B]\) (lemme~\ref{LEMooSZZWooPDHnGl}).

    Mettons au contraire que \( X\) est dans le plan médiateur de \( [A,B]\). Nous avons \( (X,M)\perp (A,B)\). Donc le triangle \( A,B,X\) est isocèle en \( X\) et donc \( X\) est équidistant de \( A\) et \( B\).
\end{proof}

\begin{lemma}       \label{LEMooTCIEooXdyuHu}
    Si \( A'\) est l'image de \( A\) par \( \sigma_{\ell}\) alors \( \ell\) est la médiatrice du segment \( [A,A']\).
\end{lemma}

\begin{proof}
    Soit \( M\in\ell\). Nous avons
    \begin{equation}
        \| A-M \|^2=\| \pr_{\ell}(A)-A \|^2+\| \pr_{\ell}(A)-M \|^2
    \end{equation}
    parce que \( A-\pr_{ell}(A)\perp M-\pr_{\ell}(A)\). Par ailleurs, vu que \( \sigma_{\ell}(A)=2\pr_{\ell}(A)-A\) et que \( \pr_{\ell}(A)=\pr_{\ell}(A')\),
    \begin{equation}
        \| \pr_{\ell}(A)-A \|=\| \pr_{\ell}(A')-A' \|.
    \end{equation}
    Nous avons donc
    \begin{equation}
        \| \sigma_{\ell}(A)-M \|^2=\| A-M \|^2,
    \end{equation}
    ce qui prouve que \( M\) est sur la médiatrice de \( [A',A]\) par le lemme~\ref{LEMooSZZWooPDHnGl}.
\end{proof}

\begin{normaltext}
    Si \( l\) est une droite dans \( \eR^2\), nous avons la réflexion \( \sigma_l\in\Isom(\eR^2)\) d'axe \( l\). Cela est une isométrie et donc une application affine par le théorème~\ref{ThoDsFErq}. Le lemme suivant détermine comment la réflexion \( \sigma_{\ell}\) se décompose en une translation et une application linéaire.
\end{normaltext}

\begin{lemma}   \label{LEMooVOJLooCFgdNG}
    Soit une droite \( \ell\). Alors
    \begin{equation}
        \sigma_{\ell}=\tau_{2w}\circ\sigma_{\ell_0}
    \end{equation}
    où \( \ell_0\) est la droite parallèle à \( \ell\) passant par l'origine, et \( w\) est le vecteur perpendiculaire à \( \ell\) tel que \( \ell_0=\ell+w\).
\end{lemma}

\begin{proof}
    Il faut trouver trois points non alignés sur lesquels les deux applications coïncident; cela suffira par le corolaire~\ref{CORooZHZZooDgTzsW}.

    Pour tous les points de \( \ell_0\), l'égalité fonctionne parce que si \( x\in\ell_0\),
    \begin{equation}
        \sigma_{\ell}(x)=x+2w,
    \end{equation}
    tandis que
    \begin{equation}
        \sigma_{\ell_0}(x)+2w=x+2w
    \end{equation}
    du fait que \( \sigma_{\ell_0}(x)=x\).

    Si \( x\in\ell\), alors
    \begin{equation}
        \sigma_{\ell}(x)=x
    \end{equation}
    tandis que
    \begin{equation}
        \sigma_{\ell_0}(x)+2w=x-2w+2w=x.
    \end{equation}
    Donc les applications affines \( \sigma_{\ell}\) et \( x\mapsto \sigma_{\ell_0}(x)+2w\) coïncident sur \( \ell\) et \( \ell_0\). Elles coïncident donc partout.
\end{proof}

%--------------------------------------------------------------------------------------------------------------------------- 
\subsection{Translations et réflexions}
%---------------------------------------------------------------------------------------------------------------------------

\begin{lemma}       \label{LEMooMKTXooYKZcdQ}
    Si \( A\in \eR^2\), si \( \ell\) est une droite de \( \eR^2\), alors nous avons
    \begin{equation}
        \pr_{\tau_A(\ell)}=\tau_A\circ \pr_{\ell}\circ\tau_A^{-1}.
    \end{equation}
\end{lemma}

\begin{proof}
    Soit \( x\in \eR^2\). Soit \( v\) unitaire dans la direction\footnote{Cela signifie qu'il existe \( a\in \eR^2\) tel que \( \ell=a+\eR v\).} de \( \ell\). La condition \( q=\pr_{\ell}(x)\) est le système
    \begin{subequations}
        \begin{numcases}{}
            q\in \ell\\
            (q-x)\cdot v=0.
        \end{numcases}
    \end{subequations}
    Nous voulons prouver que \( \tau_A(q)=\pr_{\tau_A(\ell)}\big( \tau_A(x) \big)\), c'est-à-dire que
    \begin{subequations}
        \begin{numcases}{}
            \tau_A(q)\in \tau_A(\ell)\\
            \big( \tau_A(q)-\tau_A(x) \big)\cdot v=0.
        \end{numcases}
    \end{subequations}
    Nous avons utilisé le fait que $v$ est un vecteur unitaire dans la direction de \( \tau_A(\ell)\) aussi bien que de \( \ell\).

    Vu que \( q\in \ell\), bien entendu que \( \tau_A(q)\in \tau_A(\ell)\). D'autre part, \( \tau_A(q)-\tau_A(x)=q-x\), donc
    \begin{equation}
            \big( \tau_A(q)-\tau_A(x) \big)\cdot v=(q-x)\cdot v=0.
    \end{equation}
\end{proof}

\begin{lemma}       \label{LEMooSMMMooAqsHWb}
    Si \( \ell\) est une droite de \( \eR^2\), si \( A\in \eR^2\), alors
    \begin{equation}
        \sigma_{\tau_A(\ell)}=\tau_A\sigma_{\ell}\tau_A^{-1}.
    \end{equation}
\end{lemma}

\begin{proof}
    Il s'agit d'un calcul mettant en scène les lemmes \ref{LEMooZSDRooUkNYer} et \ref{LEMooMKTXooYKZcdQ} :
    \begin{subequations}
        \begin{align}
            \big( \sigma_{\tau_A(\ell)}\tau_A \big)(x)&=2\pr_{\tau_A(\ell)}\big( \tau_A(x) \big)\\
            &=2\tau_A\big( \pr_{\ell}(x) \big)-\tau_A(x)\\
            &=2\pr_{\ell}(x)+2A-x-A\\
            &=2\pr_{\ell}(x)-x+A\\
            &=\tau_A\big( \sigma_{\ell}(x) \big).
        \end{align}
    \end{subequations}
\end{proof}

%---------------------------------------------------------------------------------------------------------------------------
\subsection{Rotations}
%---------------------------------------------------------------------------------------------------------------------------

\begin{definition}[Rotation en dimension \( 2\)]        \label{DEFooFUBYooHGXphm}
    Une \defe{rotation}{rotation!en dimension \( 2\)} d'un espace euclidien de dimension \( 2\) est une composée de deux réflexions d'axes non parallèles. L'identité est une rotation.
\end{definition}

\begin{normaltext}
    Quelques remarques à propos de cette définition.
    \begin{enumerate}
        \item
            Attention : nous ne parlons pas encore de rotations «vectorielles» : ici le centre de la rotation (que nous n'avons pas encore défini) peut ne pas être \( 0\).
        \item
            Dans la même veine : plus tard, lorsque nous saurons que les rotations sont des isométries de \( (E,d)\) où \( d(X,Y)=\| X-Y \|\), nous allons en réalité beaucoup plus souvent parler de rotations centrées en l'origine qu'en un point quelconque. C'est pourquoi à partir de~\ref{NORMooOUDJooRfbDEX} nous dirons le plus souvent «rotation»  pour «rotation centrée en \( 0\)». D'où les énoncés comme «les rotations sont les matrice orthogonales» (corolaire~\ref{CORooVYUJooDbkIFY}), qui \emph{stricto senus} de la définition~\ref{DEFooFUBYooHGXphm} sont faux.
        \item
            Une rotation est composée de deux réflexions d'axes non parallèles. Il est cependant trop tôt pour décréter que l'intersection de ces axes est le centre de la rotation. Rien ne dit en effet pour l'instant que deux décompositions différentes de la même rotation, avec des axes différents donnent le même point d'intersection.
        \item
            Pourquoi ajouter l'identité  ? Pour avoir un groupe. Dans le cas vectoriel, il est suffisant de demander d'être une composée de deux réflexions, parce que toutes les réflexions vectorielles ont des axes qui s'intersectent en \( 0\). Le cas des axes parallèles est seulement le cas des axes confondus et revient à l'identité.

            Si nous voulons avoir un groupe même pour les rotations centrées ailleurs qu'en zéro, nous devons ajouter «à la main» l'identité.
    \end{enumerate}

    Toutes ces remarques se résument par : «tout devient compliqué du fait que nous voulons considérer également les rotations centrées ailleurs qu'en zéro». En se contentent du cas vectoriel, de nombreuses choses sont plus simples.
\end{normaltext}

\begin{corollary}       \label{CORooNKKIooPGOUJl}
    Si \( A\neq B\) dans \( E\) alors il existe une unique réflexion envoyant \( A\) sur \( B\).
\end{corollary}

\begin{proof}
    En ce qui concerne l'existence, la réflexion dont l'axe est la médiatrice de \( [A,B]\) fait l'affaire. En ce qui concerne l'unicité, le lemme~\ref{LEMooTCIEooXdyuHu} nous dit que si \( A\) est envoyé sur \( B\), l'axe est forcément la médiatrice de \( [A,B]\).
\end{proof}

\begin{lemmaDef}[\cite{ooYPVPooYGSlNU}]        \label{LEMooIJELooLWqBfE}
    Soit une rotation \( r=\sigma_1\circ\sigma_2\) différente de l'identité. 
    \begin{enumerate}
        \item
            Elle admet un unique point fixe.
        \item
            Ce point fixe est l'intersection des axes \( \ell_1\cap\ell_2\).
    \end{enumerate}

    Le \defe{centre}{centre!d'une rotation} d'une rotation (autre que l'identité) est cet unique point fixe.
\end{lemmaDef}

\begin{proof}
    Nous nommons \( O=\ell_1\cap\ell_2\). Soit \( A\in E\), et supposons que \( r(A)=A\). Nous avons \( \sigma_1\circ r=s_2\) et donc
    \begin{equation}
        \sigma_1(A)=(\sigma_1\circ r)(A)=s_2(A).
    \end{equation}
    On pose \( B=\sigma_1(A)\). Alors \( \sigma_1\) et \( \sigma_2\) envoient tout deux \( A\) sur \( B\).

    Si \( A=B\) alors \( A\) est fixé par \( \sigma_1\) et donc appartient à \( \ell_1\). Même chose pour \( A\) est fixé par \( \sigma_2\) et donc \( A\in\ell_2\). Cela donne \( A=B=O\), et donc le point fixé par \( r\) est \( O\).

    Si \( A\neq B\) alors il existe une unique réflexion envoyant \( A\) sur \( B\) (corolaire~\ref{CORooNKKIooPGOUJl}). L'unicité signifie que \( \sigma_1=\sigma_2\). Dans ce cas, \( r=\sigma_1\circ\sigma_2=\id\).
\end{proof}

\begin{normaltext}      \label{NORMooDPBOooKkRuTn}
    La rotation \( \sigma_1\circ\sigma_2\) laisse évidemment fixé le point \( \ell_1\cap \ell_2\). Si \( \sigma_1\circ\sigma_2=\sigma_a\circ\sigma_b\) alors rien n'oblige les axes de \( \sigma_1\) et \( \sigma_2\) d'être identiques à ceux que \( \sigma_a\) et \( \sigma_b\). Mais l'intersection \( \ell_1\cap\ell_2\) doit être la même que l'intersection \( \ell_a\cap \ell_b\) parce que c'est l'unique point fixé par la composée. Cela nous permet de poser les définitions suivante.
\end{normaltext}


\begin{lemma}       \label{LEMooTZNWooTVOklu}
    Les rotation sont des isométries pour la distance : \( \| X-Y \|=\| r(X)-r(Y) \|\).
\end{lemma}

\begin{proof}
    Si \( r=\sigma_1\circ\sigma_2\), en utilisant le fait que \( \sigma_1\) et \( \sigma_2\) sont des isométries de \( (E,d)\) (\ref{PROPooFSVEooWmJsnv}) nous avons :
    \begin{equation}
        d(X,Y)=d\big( \sigma_2(X),\sigma_2(Y) \big)=d\big( \sigma_1\sigma_2(X),\sigma_1\sigma_2(Y) \big)=d\big( r(X),r(Y) \big).
    \end{equation}
\end{proof}

Ce lemme nous dit qu'une rotation de centre \( O\) vérifie \( \| OX \|=\| Or(X) \|\) pour tout \( X\).

\begin{proposition}[\cite{ooYPVPooYGSlNU}]      \label{PROPooNXJKooEDOczh}
    Soient \( A,B,O\in E\) tels que \( \| AO \|=\| BO \|\neq 0\). Alors il existe une unique rotation \( r\) centrée en \( O\) telle que \( r(A)=B\).
\end{proposition}

\begin{probleme}
    Attention : la preuve qui suit contient de nombreuses galipettes et improvisations personnelles. Relisez-la attentivement avant de la prendre pour argent comptant.

    La difficulté tient essentiellement à ce que cette preuve traite de façon vectorielle (tous les points sont des éléments de \( E\)) un énoncé qui est essentiellement affine : tous les points doivent être vus comme vecteurs partant de \( O\).

    Si vous comparez la preuve donnée ici avec celle de \cite{ooYPVPooYGSlNU}, vous remarquerez que dans ce dernier, seule la partie «\( A\) et \( O\) son alignés» est présente. C'est parce que lui, il se met directement dans le cas vectoriel et \( O=0\). Il a donc une preuve un tout petit peu moins générale, mais au moins ses isométries sont linéaires et non affines.
\end{probleme}

\begin{proof}
    Existence et unicité séparément.
    \begin{subproof}
        \item[Existence]
            Si \( A=B\), l'identité fait l'affaire. Sinon, \( \| A-O \|=\| B-O \|\) implique que la médiatrice de \( [A,B]\) contient \( O\). Soit \( \sigma_m\) la réflexion selon cette médiatrice. La rotation \( \sigma_m\circ\sigma_{(AO)}\) convient.

        \item[Unicité]

            Soit \( r\) une rotation de centre \( O\) et telle que \( r(A)=B\). Si \( A=B\) alors \( r=\id\) parce qu'une rotation autre que l'identité ne fixe que son centre par le lemme~\ref{LEMooIJELooLWqBfE}. Nous supposons que \( A\neq B\).

            Nous posons \( g=\sigma_m\circ r\). Alors \( g(A)=\sigma_m(B)=A\) parce que \( \sigma_m(B)=A\) et \( r(A)=B\). Cela signifie que \( g\) est une isométrie qui fixe \( A\).

            \begin{subproof}
                \item[Si \( A\) et \( O\) ne sont pas alignés]

                    Attention : ici \( O\) est un point de \( E\), pas le zéro de l'espace vectoriel \( E\). Lorsqu'on dit que \( A\) et \( O\) ne sont pas alignés, nous parlons bien d'alignement avec le zéro de \( E\).

                    Nous avons \( g(A)=A\) et \( g(O)=O\). Donc \( g\) coïncide avec \( \sigma_{(AO)}\) en deux points non alignés, c'est-à-dire en deux points pour lesquels l'espace engendré est tout \( E\). Nous en déduisons que \( g=\sigma_{(AO)}\).

                \item[Si \( A\) et \( O\) sont alignés]


            Soit maintenant un point \( C\) tel que \( A-O\perp C-O\) et
            \begin{equation}
                \| OC \|=\| OA \|=\| OB \|.
            \end{equation}
            Vu que \( g\) est une isométrie pour la distance sur \( E\), pas pour la norme, nous ne pouvons pas écrire \( g(C-O)\perp g(A-O)\) à partir de \( C-O\perp A-O\). Nous décomposons \( g(X)=s(X)+G\) où \( s\) est linéaire sur \( E\). Il est vite vu que \( s\) est une isométrie de \( (E,\| . \|)\) :
            \begin{equation}
                \| X-Y \|=\| g(X)-g(Y) \|=\| s(X)+G-s(y)-G \|=\| s(X)-s(Y) \|=\| s(X-Y) \|
            \end{equation}
            pour tout \( X,Y\in E\). Nous avons de plus \( g(A)=A\) et \( g(O)=O\), ce qui donne \( O=s(O)+G\) et \( A=s(A)+G\). En égalisant les valeurs de \( G\) nous avons
            \begin{equation}        \label{EQooPEWGooABHUvu}
                O-s(O)=A-s(A).
            \end{equation}
            Vu que \( s\) est une isométrie (une vraie) nous avons
            \begin{equation}
                s(A-O)\perp s(C-O),
            \end{equation}
            mais \( s(A-O)=s(A)-s(O)=A-O\) par \eqref{EQooPEWGooABHUvu}. Donc
            \begin{equation}
                A-O\perp s(C-O).
            \end{equation}
            Nous en concluons que \( s(C-O)=\pm (C-O)\). Parce que les vecteurs \( \pm(C-O)\) sont les deux seuls de norme \( \| AO \| =\| CO \|\) à être perpendiculaire à \( A-O\). Rappel : la définition de \( C\) et le fait que nous soyons en dimension \( 2\).

            Est-il possible d'avoir \( s(C-O)=C-O\) ? Cela donnerait
            \begin{subequations}
                \begin{align}
                    g(A-O)&=s(A)-s(O)+G=s(A)-O+O-s(O)+G=A-O+G\\
                    g(C-O)&=C-O+G,
                \end{align}
            \end{subequations}
            ce qui signifierait que \( g\) et \( \tau_G\) coïncideraient sur les points \( A-O\) et \( C-O\), et donc seraient égaux par le corolaire~\ref{CORooZHZZooDgTzsW}. Cela est cependant impossible parce que \( g\) fixe au moins les points \( A\) et \( O\) alors que la translation ne fixe aucun point. Nous en déduisons \( s(C-O)=-(C-O)\).

            Nous avons aussi, parce que \( (AO)\) est une droite passant par l'origine que
            \begin{equation}
                \sigma_{(AO)}(A-O)=A-O
            \end{equation}
            et parce que \( C-O\) est perpendiculaire à cette droite :
            \begin{equation}
                \sigma_{(AO)}(C-O)=-(C-O).
            \end{equation}
            Nous avons donc quand même que \( g\) et \( \sigma_{(AO)}\) coïncident sur deux points non alignés : \( A-O\) et \( C-O\).

            \end{subproof}

            Dans tous le cas, \( g=\sigma_{(AO)}\). Nous avons donc
            \begin{equation}
                \sigma_{(OA)}=\sigma_m\circ r,
            \end{equation}
            et donc \( r\) est fixé à
            \begin{equation}
                r=\sigma_m\circ\sigma_{(OA)}.
            \end{equation}
    \end{subproof}
\end{proof}

\begin{normaltext}
    Anticipons un peu et faisons semblant de déjà connaitre les matrices et les fonctions trigonométriques. La proposition~\ref{PROPooNXJKooEDOczh} nous dit qu'il existe une seule rotation amenant \( A\) sur \( B\). Vous pourriez objecter que le point \( (1,0)\) peut être amené sur \( (0,-1)\) soit par la rotation d'angle \( 3\pi/2\), soit par celle d'angle \( -\pi/2\). Il n'en est rien parce que ces deux rotations sont les mêmes ! Pensez-y. En tant qu'application \( \eR^2\to \eR^2\), la rotation \( R_{3\pi/2}\) est égale à \( R_{-\pi/2}\).
\end{normaltext}

Une rotation donnée peut être écrite de beaucoup de façons comme composée de deux réflexions. En fait d'autant de façons qu'il y a de réflexions.
\begin{proposition}[\cite{ooYPVPooYGSlNU}]      \label{PROPooKAZEooLTHWKe}
    Soit une rotation \( r\) de \( E\) centrée en \( O\). Pour toute réflexion \( \sigma_{\ell}\) telle que le centre de \( r\) soit sur \( \ell\), il existe une réflexion \( \sigma_1\) tells que \( r=\sigma_1\circ\sigma_{\ell}\). Il existe aussi une réflexion \( \sigma_2\) telle que \( r=\sigma_{\ell}\circ s_2\).
\end{proposition}

\begin{proof}
    Si \( r=\id\) c'est bon avec \( s_1=s_2=\sigma_{\ell}\). Sinon nous considérons \( A\neq O\) sur \( \ell\), et \( B=r(A)\). Nous savons que \( B\neq A\) parce que \( O\) est le seul point de \( E\) fixé par \( r\) (proposition~\ref{LEMooIJELooLWqBfE}). Il existe une réflexion (unique) \( \sigma_1\) faisant \( \sigma_1(A)=B\), et c'est le réflexion dont l'axe est la médiatrice de \( [A,B]\). Le point \( O\) est sur cette médiatrice parce que les rotations sont des isométries de \( (E,d)\) (lemme~\ref{LEMooTZNWooTVOklu}).

    La rotation \( \sigma_1\circ \sigma_{\ell}\) vérifie
    \begin{equation}
        (\sigma_1\circ\sigma_{\ell})(A)=\sigma_1(A)=B.
    \end{equation}
    Or \( \| OA \|=\| OB \|\), donc il y a unicité de la rotation centrée en \( O\) portant \( A\) sur \( B\) (proposition~\ref{PROPooNXJKooEDOczh}); nous avons donc \( r=\sigma_1\circ\sigma_{\ell}\).

    En ce qui concerne \( r=\sigma_{\ell}\circ\sigma_2\), il faut appliquer ce que nous venons de faire à la rotation \( r^{-1}\): il existe \( \sigma_2\) tel que \( r^{-1}=\sigma_2\circ\sigma_{\ell}\), ce qui donne
    \begin{equation}
        r=\sigma_{\ell}\circ\sigma_2.
    \end{equation}
\end{proof}

%--------------------------------------------------------------------------------------------------------------------------- 
\subsection{Rotation d'un angle donné}
%---------------------------------------------------------------------------------------------------------------------------

\begin{lemmaDef}        \label{DEFooADTDooKIZbrw}
    Soit \( \theta\in \eR\). Nous considérons l'application linéaire \( R_0(\theta)\colon \eR^2\to \eR^2\) dont la matrice dans la base canonique est
    \begin{equation}
        \begin{pmatrix}
            \cos(\theta)    &   -\sin(\theta)    \\ 
            \sin(\theta)    &   \cos(\theta)    
        \end{pmatrix}.
    \end{equation}
    \begin{enumerate}
        \item       \label{ITEMooIEKJooZfsAui}
    Nous avons
    \begin{equation}        \label{EQooEVCTooBpTDDq}
        R_0(\theta)=\sigma_{\ell}\circ s
    \end{equation}
    où \( s\) est la réflexion d'axe horizontal et \( \ell\) est la droite
    \begin{equation}
        \ell=\eR\begin{pmatrix}
            \cos(\theta/2)    \\ 
            \sin(\theta/2)    
        \end{pmatrix}.
    \end{equation}
\item       \label{ITEMooBEYOooMHRRYk}
    L'application \( R_0(\theta)\) est une rotation autour de \( (0,0)\).
\item     \label{ITEMooEQPAooQcsYfj}
          Si \( A\in \eR^2\), alors l'application
          \begin{equation}
              R_A(\theta)=\tau_A\circ R_0(\theta)\circ \tau_A^{-1}
          \end{equation}
            est une rotation autour de \( A\) nommée \defe{rotation d'angle \( \theta\)}{rotation d'angle \( \theta\)}.       
    \end{enumerate}
\end{lemmaDef}

\begin{proof}
    Nous allons prouver l'égalité \eqref{EQooEVCTooBpTDDq} en calculant les deux membres sur les vecteurs \( \begin{pmatrix}
        1    \\ 
        0    
    \end{pmatrix}\) et \( \begin{pmatrix}
        1    \\ 
        0    
    \end{pmatrix}\) qui forment une base.

    \begin{subproof}
        \item[Pour \( p=(1,0)\)]
            D'abord la chose facile\footnote{Ici comme partout dans le Frido nous ne faisons aucune différence entre \( (a,b)\) et \( \begin{pmatrix}
            a    \\ 
        b    
    \end{pmatrix}\); ce sont seulement deux façons différentes d'écrire le même élément de \( \eR^2\). Nous ne faisons pas du semblant de croire que l'un ou l'autre serait un «covecteur» suivant que l'on tourne notre page dans un sens ou un autre.} : \( s(1,0)=(1,0)\).

    Pour calculer \( \sigma_{\ell}(1,0)\), nous utilisons le lemme \ref{LEMooZSDRooUkNYer}; nous commençons par chercher la projection orthogonale \( q\) de \( p=(1,0)\) sur \( \ell\). Nous posons
    \begin{equation}        \label{EQooUFMWooWhwcHR}
            q=\lambda\begin{pmatrix}
                \cos(\theta/2)    \\ 
                \sin(\theta/2)    
            \end{pmatrix}
        \end{equation}
        et nous cherchons \( \lambda\) satisfaisant \( q\cdot(q-p)=0\). Un peu de calculs passant par \eqref{EQooNYCZooApyyRd} nous donne
        \begin{equation}
            q\cdot (q-p)=\lambda\big( \lambda-\cos(\theta/2) \big).
        \end{equation}
        Les deux solutions sont \( \lambda=0\) et \( \lambda=\cos(\theta/2)\). Mais la solution \( \lambda=0\) revient à dire que la droite \( \ell\) est verticale, c'est-à-dire \( \cos(\theta/2)=0\). Donc la solution est toujours donnée par
        \begin{equation}
            \lambda=\cos(\theta/2).
        \end{equation}
        Nous introduisons cette valeur dans \eqref{EQooUFMWooWhwcHR} pour fixer \( q\), et nous utilisons la formule du lemme \ref{LEMooZSDRooUkNYer} :
        \begin{equation}
            \sigma_{\ell}\begin{pmatrix}
                1    \\ 
                0    
            \end{pmatrix}=2q-p=\begin{pmatrix}
                2\cos^2(\theta/2)-1    \\ 
                2\cos(\theta/2)\sin(\theta/2)    
            \end{pmatrix}
            =\begin{pmatrix}
                \cos(\theta)    \\ 
                \sin(\theta)    
            \end{pmatrix}.
        \end{equation}
        Nous avons utilisé le formules trigonométrique de duplication d'angle (corolaire \ref{CORooQZDQooWjMXTF}).
    \item[Pour \( p=(0,1)\)]
        Cette fois \( s(p)=(0,-1)\) et l'équation pour déterminer \( \lambda\) est
        \begin{equation}
            0=q\cdot(q-p)=\begin{pmatrix}
                \lambda\cos(\theta/2)    \\ 
                \lambda\sin(\theta/2)    
            \end{pmatrix}\cdot\begin{pmatrix}
                \lambda\cos(\theta/2)    \\ 
                \lambda\sin(\theta/2)-1    
            \end{pmatrix}.
        \end{equation}
        Nous trouvons \( \lambda\big( \lambda+\sin(\theta/2) \big)=0\). Le cas \( \lambda=0\) signifie que la droite \( \ell\) est horizontale et donc que \( \sin(\theta/2)=0\). Donc la solution est dans tous les cas
        \begin{equation}
            \lambda=-\sin(\theta/2).
        \end{equation}
        Nous trouvons
        \begin{equation}
            \sigma_{\ell}\begin{pmatrix}
                0    \\ 
                -1    
            \end{pmatrix}=2q-p=\begin{pmatrix}
                -\sin(\theta)    \\ 
                \cos(\theta)    
            \end{pmatrix}.
        \end{equation}
    \end{subproof}
    Le calcul de \( \sigma_{\ell}\circ s\) étant fait sur une base, il est facile de reconstituer la matrice
    \begin{equation}
        \begin{pmatrix}
            \cos(\theta)    &   -\sin(\theta)    \\ 
            \sin(\theta)    &   \cos(\theta)    
        \end{pmatrix}.
    \end{equation}
    Cette matrice étant celle, par définition, de \( R_0(\theta)\), nous avons montré que \( R_0(\theta)\) était bien une rotation. Nous avons prouvé les points \ref{ITEMooIEKJooZfsAui} et \ref{ITEMooBEYOooMHRRYk}.

    Nous passons maintenant au point \ref{ITEMooEQPAooQcsYfj}. Il est facile de voir que \( A\) est un point fixe de \( R_{A}(\theta)\) parce que \( \tau_A^{-1}(A)=(0,0)\).

    Nommons \( \ell'\) la droite horizontale \( \eR(1,0)\). Nous avons, par le point précédent \( R_0(\theta)=\sigma_{\ell}\circ \sigma_{\ell'}\). En introduisant astucieusement \( \tau_A^{-1}\tau_A\) dans l'expression définissant \( R_A(\theta)\), nous avons
    \begin{equation}
        R_A(\theta)=\tau_A\sigma_{\ell}\sigma_{\ell'}\tau_A^{-1}=\underbrace{\tau_A\sigma_{\ell}\tau_A^{-1}}_{=\sigma_{\tau_A(\ell)}}\underbrace{\tau_A\sigma_{\ell'}\tau_A^{-1}}_{\sigma_{\tau_A(\ell')}}=\sigma_{\tau_A(\ell)}\sigma_{\tau_A(\ell')}.
    \end{equation}
    Nous avons utilisé le lemme \ref{LEMooSMMMooAqsHWb}.

    Nous voyons que \( R_A(\theta)\) est une composée de deux réflexions se coupant en \( A\). C'est donc une rotation centrée en \( A\).


\end{proof}

%---------------------------------------------------------------------------------------------------------------------------
\subsection{Rotations vectorielles}
%---------------------------------------------------------------------------------------------------------------------------

L'expression «rotation vectorielle» signifie rotation centrée en zéro. Elles sont «vectorielles» parce qu'elles sont linéaires comme nous le voyons à présent.

\begin{proposition}     \label{PROPooTFNSooFjiWHG}
    Quelques résultats à propos de rotations.
    \begin{enumerate}
        \item       \label{ITEMooONJOooRgycsQ}
    Toutes les rotations de \( \eR^2\) centrées en \( 0\) sont de la forme \( R_0(\theta)\) pour un \( \theta\in \eR\).
\item
    Les rotations de \( \eR^2\) centrées en \( 0\) sont des applications linéaires.
        \item       \label{ITEMooSIHZooBEJhdu}
            Si \( r\) est une rotation dans \( \eR^2\), il existe \( A\in \eR^2\) et \( \theta\in \eR\)
            \begin{equation}
                r=\tau_A\circ R_0(\theta)\circ \tau_A^{-1}.
            \end{equation}
    \end{enumerate}
\end{proposition}

\begin{proof}
    Soit \( r\) une rotation centrée en \( (0,0)\). La proposition \ref{PROPooKAZEooLTHWKe}, il existe une droite \( \ell\) passant par \( (0,0)\) telle que telle que \( r=\sigma_{\ell}\circ s\) où \( s\) est la réflexion d'axe horizontal : \( s(x,y)=(x,-y)\).

    Soit un vecteur unitaire \( v\in \ell\). Vu que \( v\in S^1\), la proposition \ref{PROPooKSGXooOqGyZj} nous donne \( t\in \mathopen[ 0 ,2\pi  \mathclose[\) tel que
        \begin{equation}
            v=\begin{pmatrix}
                \cos(t)    \\ 
                \sin(t)    
            \end{pmatrix}.
        \end{equation}
        En posant \( \theta=2 t\) nous avons
        \begin{equation}
            \ell=\eR\begin{pmatrix}
                \cos(\theta/2)    \\ 
                \sin(\theta/2)    
            \end{pmatrix},
        \end{equation}
        et donc \( r=R_0(\theta)\) qui est l'application linéaire définie en \ref{DEFooADTDooKIZbrw}.

    Et enfin nous voyons le point \ref{ITEMooSIHZooBEJhdu}. Soit \( A\) l'unique point fixe de la rotation \( r\). Cette dernière s'écrit alors \( r=\sigma_{\ell_1}\circ\sigma_{\ell_2}\) où \( \ell_1\) et \( \ell_2\) sont des droites telles que \( \ell_1\cap\ell_2=\{ A \}\).

    En utilisant le lemme \ref{LEMooSMMMooAqsHWb}, nous avons \( \sigma_{\ell_i}=\tau_A\circ \sigma_{\tau_A^{-1}(\ell_i)}\circ \tau_A^{-1}\). En substituant cela et en nous rendant compte que \( \tau_A^{-1}\tau_A=\id\) nous avons
    \begin{equation}
        r=\sigma_{\ell_1}\circ\sigma_{\ell_2}=\tau_A r_0\tau_1^{-1}
    \end{equation}
    où \( r_0\) est la rotation \( \sigma_{\tau_A^{-1}(\ell_1)}\circ \sigma_{\tau_A^{-1}}(\ell_2)\). Cette dernière est une rotation autour de \( (0,0)\) parce que \( \tau_A^{-1}(\ell_1)\cap \tau_A^{-1}(\ell_2)=\{ 0 \}\). Elle est donc, par le point \ref{ITEMooONJOooRgycsQ}, de la forme \( R_0(\theta)\) pour une certaine valeur de \( \theta\).

\end{proof}

\begin{proposition}[\cite{ooYPVPooYGSlNU}]      \label{PROPooWMESooNJMdxf}
    Les rotations basées en \( O\) forment un groupe abélien.
\end{proposition}

\begin{proof}
    L'identité est une rotation par définition. En ce qui concerne l'inverse, si \( r=\sigma_1\sigma_2\) alors \( r^{-1}=\sigma_2\sigma_1\). Nous commençons maintenant les choses pas tout à fait évidentes.
    \begin{subproof}
        \item[Composition]
            Soient des rotations \( r,r'\) centrées en \( O\). Soit également une réflexion \( \sigma\) dont l'axe contient \( O\). Alors la proposition~\ref{PROPooKAZEooLTHWKe} nous donne l'existence de \( \sigma_1\) et \( \sigma_2\) tels que \( r=\sigma_1\sigma\) et \( r'=\sigma\sigma_2\). Avec ça, la composition donne
            \begin{equation}
                rr'=\sigma_1\sigma\sigma\sigma_2=\sigma_1\sigma_2,
            \end{equation}
            qui est encore une rotation.
        \item[Commutativité]
            Soient deux rotations \( r\) et \( r'\) ainsi que des décompositions \( r=\sigma_1\sigma\), \( r'=\sigma\sigma_2\). Nous avons
            \begin{subequations}
                \begin{align}
                    rr'&=\sigma_1\sigma_2\\
                    r'r&=\sigma\sigma_2\sigma_2\sigma.
                \end{align}
            \end{subequations}
            Vu que \( t=\sigma_2\sigma_1\) est une rotation nous pouvons encore appliquer la proposition~\ref{PROPooKAZEooLTHWKe} pour avoir \( t=\sigma_2\sigma_1=\sigma\sigma_3\). Avec ça,
            \begin{equation}
                r'r=\sigma\sigma\sigma_3\sigma=\sigma_3\sigma.
            \end{equation}
            Mais aussi \( rr'=\sigma_1\sigma_2=t^{-1}=\sigma_3\sigma\). Nous avons donc bien \( rr'=r'r\), et le groupe est commutatif.
    \end{subproof}
\end{proof}

\begin{normaltext}      \label{NORMooOUDJooRfbDEX}
    Jusqu'à présent nous avons parlé de rotations «affines». Parmi elles, les rotations centrées en \( 0\) (zéro, l'origine de \( E\) comme espace vectoriel) sont de particulière importance. Ce sont des applications linéaires, et même des isométries. Dans la suite, nous allons souvent dire simplement «rotation» pour dire «rotation centrée en \( 0\)».

    Vu que nous allons maintenant prendre un point de vue plus vectoriel, nous allons noter les points de \( E\) avec des lettres comme \( x\), \( y\), \( u\), \( v\) et plus avec des majuscules, comme quand on avait un point de vue afin. En même temps, nous allons noter les applications \( E\to E \) par des lettres comme \( A\) et ne plus écrire les parenthèses. Bref, nous écrivons \( Au\) au lieu de \( r(A)\).
\end{normaltext}

\begin{lemma}       \label{LEMooSYZYooWDFScw}
    En dimension \( 2\), les réflexions vectorielles (c'est-à-dire dont l'axe passe par \( 0\)) ont un déterminant \( -1\).
\end{lemma}

\begin{proof}
    Soit une réflexion d'axe \( \ell\). Prenons une base orthonormale de \( E\) constituée de \( e_1\) sur \( \ell\) et de \( e_2\perp \ell\). Alors \( \sigma_{\ell}(e_1)=e_1\) et \( \sigma_{\ell}(e_2)=-e_2\). La formule du déterminant donne
    \begin{equation}
        \det(\sigma_{\ell})=e_1^*\big( \sigma_{\ell}(e_1) \big)e_1^*\big( \sigma_{\ell}(e_2) \big)-e_2^*\big( \sigma_{\ell}(e_1) \big)e_1^*\big( \sigma_{\ell}(e_2) \big)=1\times (-1)-0\times 0=-1.
    \end{equation}
    Nous utilisons de façon cruciale le fait que le calcul du déterminant ne dépende pas de la base choisie, lemme~\ref{LEMooQTRVooAKzucd}.
\end{proof}

\begin{proposition}     \label{PROPooTUJWooAjtEnQ}
    Les rotations\footnote{Centrées en \( 0\), nous ne le répéterons pas !} sont
    \begin{enumerate}
        \item
            des applications linéaires orthogonales au sens de la définition~\ref{DEFooYKCSooURQDoS},
        \item
             des applications de déterminant \( 1\),
    \end{enumerate}
\end{proposition}

\begin{proof}
    Le fait qu'elles soient linéaires est la proposition \ref{PROPooTFNSooFjiWHG}.

    Nous avons, pour tout \( u\in E\) l'égalité de la norme \( \| u \|\) et \( \| Au \|\) par le lemme~\ref{LEMooTZNWooTVOklu} appliqué à \( Y=0\). En termes de produits scalaires nous avons alors \( \langle Au, Au\rangle =\langle u, u\rangle \), et donc
    \begin{equation}
        \langle A^*Au, u\rangle =\| u \|^2.
    \end{equation}
    En particulier si \( \{ e_i \}_{i=1,\ldots, n}\) est une base orthonormée de \( E\) nous avons
    \begin{equation}
        (A^*Ae_i)_i=\| e_i \|^2=1,
    \end{equation}
    ce qui donne \( \| A^*Ae_i \|\geq 1\), avec égalité si et seulement si \( A^*Ae_i=e_i\). Ici nous avons utilisé le fait que \( \langle x, e_i\rangle =x_i\), et le fait que pour tout \( i\) nous ayons \( \| x \|\geq | x_i |\), avec égalité seulement si \( x\) est un multiple de \( e_i\).

    Par ailleurs l'inégalité de Cauchy-Schwarz~\ref{ThoAYfEHG} nous donne
    \begin{equation}
        \| u \|^2=| \langle A^*Au, u\rangle  | \leq \| A^*Au \|\| u \|
    \end{equation}
    et donc
    \begin{equation}
        \| u \|\leq \| A^*Au \|.
    \end{equation}
    Encore une fois, en appliquant cela à \( u=e_i\) nous trouvons \( 1\leq \| A^*Ae_i \|\). Vu que nous avions déjà l'inégalité dans l'autre sens, \( \| A^*Ae_i \|=1\). Et le cas d'égalité est uniquement possible avec \( A^*Ae_i=e_i\).

    Donc pour tout \( i\) de la base nous avons \( A^*Ae_i=e_i\). Nous avons donc \( A^*A=\id\) et l'application \( A\) est orthogonale.

    En ce qui concerne le déterminant, les réflexions sont de déterminant \( -1\) par le lemme~\ref{LEMooSYZYooWDFScw}, donc \( A=\sigma_1\circ\sigma_2\) est de déterminant \( 1\). Nous avons utilisé le fait que le déterminant était un morphisme : proposition~\ref{PropYQNMooZjlYlA}\ref{ItemUPLNooYZMRJy}.
\end{proof}

\begin{remark}
    Nous ne savons pas encore que les rotations forment tout le groupe \( \SO(2)\) des endomorphismes orthogonaux de déterminant \( 1\). Il faudra attendre le corolaire~\ref{CORooVYUJooDbkIFY} pour le savoir.
\end{remark}

\begin{lemma}       \label{LEMooMIJXooCjiQqP}
    L'application \( -\id\) est une rotation de \( \eR^2\).
\end{lemma}

\begin{proof}
    Soit une base orthonormée \( \{ e_1,e_2 \}\) de \( E\) et la rotation \( r=\sigma_1\sigma_2\) où \( \sigma_i\) est la réflexion le long de l'axe \( \ell_i=\{ te_i \}_{t\in \eR}\). Faut-il vous prouver que \( r=-\id\) ? La réflexion \( \sigma_2\) retourne la composante \( y\) d'un vecteur écrit dans la base \( \{ e_1,e_2 \}\) sans toucher à la composante \( x\). La réflexion \( \sigma_1\) fait le contraire.
\end{proof}

%--------------------------------------------------------------------------------------------------------------------------- 
\subsection{Matrice des transformations orthogonales}
%---------------------------------------------------------------------------------------------------------------------------

Nous donnons maintenant une forme générale (trigonométrique) pour les matrices de \( \SO(2)\). Nous ne pouvons cependant pas invoquer les lemmes \ref{LEMooAJMAooXPSKtS} ou \ref{LEMooHRESooQTrpMz} pour prétendre avoir une matrice des rotations, parce que nous n'avons pas encore prouvé que les rotations étaient des transformations orthogonales. Ce sera pour la proposition \ref{PROPooOTIVooZpvLnb}.

\begin{lemma}       \label{LEMooAJMAooXPSKtS}
    Si \( A\in \gO(2)\) alors il existe un unique \( \theta\in\mathopen[ 0 , 2\pi \mathclose[\) et un unique \( \epsilon=\pm 1\) tels que
    \begin{equation}
        A=\begin{pmatrix}
            \cos(\theta)    &   -\epsilon\sin(\theta)    \\
            \sin(\theta)    &   \epsilon\cos(\theta)
        \end{pmatrix}
    \end{equation}
\end{lemma}

\begin{proof}
    Soit une matrice \( A=\begin{pmatrix}
        a    &   b    \\
        c    &   d
    \end{pmatrix}\) et imposons qu'elle soit dans \( \gO(2)\). Le fait que \( A\) soit orthogonale impose
    \begin{equation}
        \begin{pmatrix}
            a    &   c    \\
            b    &   d
        \end{pmatrix}\begin{pmatrix}
            a    &   b    \\
            c    &   d
        \end{pmatrix}=\begin{pmatrix}
            a^2+c^2    &   ab+cd    \\
            ab+cd    &   b^2+d^2
        \end{pmatrix}=\begin{pmatrix}
            1    &   0    \\
            0    &   1
        \end{pmatrix}.
    \end{equation}
    Nous avons alors le système
    \begin{subequations}
        \begin{numcases}{}
            a^2+b^2=1\\
            b^2+d^2=1\\
            ab+cd=0
        \end{numcases}
    \end{subequations}
    La proposition~\ref{PROPooKSGXooOqGyZj} nous permet de déduire qu'il existe un unique \( \theta\in\mathopen[ 0 , 2\pi \mathclose[\) tel que \( a=\cos(\theta)\), \( c=\sin(\theta)\), ainsi que plusieurs \( \alpha\in \eR\) tel que \( b=\cos(\alpha)\), \( d=\sin(\alpha)\).

        Note : si nous voulons \( \alpha\in\mathopen[ 0 , 2\pi \mathclose[\), alors il y a unicité. Ici nous ne nous attachons pas à cette contrainte; nous savons qu'il en existe plusieurs, et nous allons en fixer un en fonction de \( \theta\). Le \( \alpha\) ainsi fixé ne sera peut-être pas dans \( \mathopen[ 0 , 2\pi \mathclose[\), mais ce ne sera pas grave.

        Les angles \( \theta\) et \( \alpha\) sont alors liés par la contrainte
        \begin{equation}
            \cos(\theta)\cos(\alpha)+\sin(\theta)\sin(\alpha)=0.
        \end{equation}
        Utilisant l'identité \eqref{EQooCVZAooQfocya} cela signifie que \( \cos(\theta-\alpha)=0\). Donc
        \begin{equation}
            \alpha\in\{ \theta+\frac{ \pi }{2}+k\pi \}_{k\in \eZ}.
        \end{equation}
        Si \( k\) est pair, ça donne
        \begin{subequations}
            \begin{align}
                \cos(\alpha)=-\sin(\theta)\\
                \sin(\alpha)=\cos(\theta)
            \end{align}
        \end{subequations}
        et alors
        \begin{equation}        \label{EQooNAMKooKACIfd}
            A=\begin{pmatrix}
                \cos(\theta)    &   -\sin(\theta)    \\
                \sin(\theta)    &   \cos(\theta)
            \end{pmatrix}.
        \end{equation}
        Si au contraire \( k\) est impair, alors
        \begin{subequations}
            \begin{align}
                \cos(\alpha)=\sin(\theta)\\
                \sin(\alpha)=-\cos(\theta),
            \end{align}
        \end{subequations}
        et
        \begin{equation}        \label{EQooJMYFooGgAiMJ}
            A=\begin{pmatrix}
                \cos(\theta)    &   \sin(\theta)    \\
                \sin(\theta)    &   -\cos(\theta)
            \end{pmatrix}.
        \end{equation}

        Nous avons démontré qu'une matrice de \( \gO(2)\) était forcément d'une des deux formes \eqref{EQooNAMKooKACIfd} ou \eqref{EQooJMYFooGgAiMJ}. Il est maintenant facile de vérifier que ces deux matrices sont effectivement dans \( \gO(2)\).
\end{proof}

\begin{lemma}       \label{LEMooHRESooQTrpMz}
    Tout élément de \( \SO(2)\) s'écrit (dans la base canonique) de façon unique sous la forme
    \begin{equation}
        \begin{pmatrix}
            \cos(\theta)    &   -\sin(\theta)    \\
            \sin(\theta)    &   \cos(\theta)
        \end{pmatrix}
    \end{equation}
    avec \( \theta\in\mathopen[ 0 , 2\pi \mathclose[\).
\end{lemma}

\begin{proof}
    Vu que \( \SO(2)\) est la partie de \( \gO(2)\) constitué des matrices de déterminant \( 1\), nous pouvons reprendre la forme donnée par le lemme~\ref{LEMooAJMAooXPSKtS} et fixer \( \epsilon\) par la contrainte sur le déterminant.

    Nous avons, en utilisant la relation du lemme~\ref{LEMooAEFPooGSgOkF},
    \begin{equation}
        \det\begin{pmatrix}
            \cos(\theta)    &   -\epsilon\sin(\theta)    \\
            \sin(\theta)    &   \epsilon\cos(\theta)
        \end{pmatrix}=\epsilon,
    \end{equation}
    et donc il faut et suffit de fixer \( \epsilon=1\).
\end{proof}



\begin{corollary}[\cite{MonCerveau}]        \label{CORooGGVUooLQYGET}
    Nous avons une bijection
    \begin{equation}
        \begin{aligned}
            \psi\colon \SO(2)&\to \mathopen[ 0 , 2\pi \mathclose[ \\
        \begin{pmatrix}
            \cos(\theta)    &   -\sin(\theta)    \\
            \sin(\theta)    &   \cos(\theta)
        \end{pmatrix}
              &\mapsto \theta,
        \end{aligned}
    \end{equation}
    et un isomorphisme de groupe
    \begin{equation}
        \begin{aligned}
            \varphi\colon \SO(2)&\to \gU(1)=\{  e^{i\theta} \}_{\theta\in \eR} \\
        \begin{pmatrix}
            \cos(\theta)    &   -\sin(\theta)    \\
            \sin(\theta)    &   \cos(\theta)
        \end{pmatrix}
        &\mapsto  e^{i\theta}.
        \end{aligned}
    \end{equation}
\end{corollary}

\begin{proof}
    La première assertion est une paraphrase du lemme~\ref{LEMooHRESooQTrpMz}. Pour la seconde, il faut vérifier que c'est bien un morphisme et une bijection.

    Pour le morphisme, ce sont les formules d'addition d'angle du lemme~\ref{LEMooJAWBooJGfZIL} qui jouent. En ce qui concerne la bijection\ldots

    \begin{subproof}
        \item[Surjection]
            Vu que \(  e^{i\theta+2ki\pi}= e^{i\theta}\), tout élément de \( \gU(1)\) est exponentielle de \( i\theta\) pour un \( \theta\in\mathopen[ 0 , 2\pi \mathclose[\).
        \item[Injection]
            Nous avons \( \varphi(A)=\varphi(B)\) lorsque les formes matricielles de \( A\) et \( B\) sous forme trigonométrique sont avec des angles différents d'un multiple de \( 2\pi\). Vu que les fonctions trigonométriques sont périodiques, nous avons \( A=B\) parce que leurs matrices sont égales.
    \end{subproof}
\end{proof}

%--------------------------------------------------------------------------------------------------------------------------- 
\subsection{Rotations, \( \SO(2)\) et matrice de rotation}
%---------------------------------------------------------------------------------------------------------------------------

\begin{corollary}[\cite{MonCerveau}] \label{CORooVYUJooDbkIFY}
    Le groupe des rotations centrées en \( (0,0)\) est le groupe \( \SO(2)\).
\end{corollary}

\begin{proof}

    Nous devons prouver deux choses : 
    \begin{itemize}
        \item Toutes les rotations sont des éléments de \( \SO(2)\).
        \item Tous les éléments de \( \SO(2)\) sont des rotations.
    \end{itemize}

    La proposition \ref{PROPooTFNSooFjiWHG} nous indique que toute rotation de \( \eR^2\) centrée en \( (0,0)\) est de la forme \( R_0(\theta)\), c'est-à-dire a une matrice de la forme
    \begin{equation}        \label{EQooSJNBooMPIRZS}
        \begin{pmatrix}
            \cos(\theta)    &   -\sin(\theta)    \\
            \sin(\theta)    &   \cos(\theta)
        \end{pmatrix}.
    \end{equation}
    Donc toute rotation est dans \( \SO(2)\).

    D'autre part, le lemme \ref{LEMooHRESooQTrpMz} indique que tout élément de \( \SO(2)\) a, dans la base canonique, une matrice de la forme \eqref{EQooSJNBooMPIRZS}. Le lemme \ref{DEFooADTDooKIZbrw} nous indique alors que c'est une rotation.
\end{proof}

\begin{proposition}         \label{PROPooOTIVooZpvLnb}
    Si \( r\) est une rotation de \( \eR^2\) centrée en \( (0,0)\), il existe un unique \( \theta\in\mathopen[ 0 , 2\pi \mathclose[\) tel que \( r=R_0(\theta)\).
\end{proposition}

\begin{proof}
    Soit une rotation \( r\) autour de \( (0,0)\). Le corolaire \ref{CORooVYUJooDbkIFY} nous dit qu'il existe \( \theta\in \eR\) tel que
    \begin{equation}
        r=R_0(\theta)=\begin{pmatrix}
            \cos(\theta)    &   -\sin(\theta)    \\ 
            \sin(\theta)    &   \cos(\theta)    
        \end{pmatrix}.
    \end{equation}
    Nous avons identifié l'application linéaire à sa matrice. L'élément \( \big( \cos(\theta), \sin(\theta) \big)\) est dans \( S^1\), et il existe donc, par la proposition \ref{PROPooKSGXooOqGyZj}, un unique \( t\in \mathopen[ 0 , 2\pi \mathclose[\) tel que \( \cos(t)=\sin(\theta)\) et \( \sin(t)=\sin(\theta)\). Pour ce \( t\) nous avons alors
        \begin{equation}
        r=R_0(\theta)=\begin{pmatrix}
            \cos(t)    &   -\sin(t)    \\ 
            \sin(t)    &   \cos(t)    
        \end{pmatrix}.
        \end{equation}
\end{proof}

\begin{proposition}     \label{PROPooISUCooRYJcwo}
    Nous avons la formule suivante pour la composition :
    \begin{equation}
        R_0(\alpha)\circ R_0(\beta)=R_0(\alpha+\beta).
    \end{equation}
\end{proposition}

\begin{proof}
    Par définition de \( R_0(\theta)\), dans la base canonique de \( \eR^2\), la composition se calcule avec le produit suivant, en utilisant les formules du lemme \ref{LEMooJAWBooJGfZIL} :
    \begin{subequations}
        \begin{align}
            \begin{pmatrix}
                \cos(\alpha)    &   -\sin(\alpha)    \\ 
                \sin(\alpha)    &   \cos(\alpha)    
            \end{pmatrix}&\begin{pmatrix}
                \cos(\beta)    &   -\sin(\beta)    \\ 
                \sin(\beta)    &   \cos(\beta)    
            \end{pmatrix}\\
            &=\begin{pmatrix}
                \cos(\alpha)\cos(\beta)  -\sin(\alpha)\sin(\beta)  &  -\cos(\alpha)\sin(\beta)-\sin(\alpha)\cos(\beta)     \\ 
                \sin(\alpha)\cos(\beta)+\cos(\alpha)\sin(\beta)    &   -\sin(\alpha)\sin(\beta)+\cos(\alpha)\cos(\beta)    
            \end{pmatrix}\\
            &=\begin{pmatrix}
                \cos(\alpha+\beta)    &   -\sin(\alpha+\beta)    \\ 
                \sin(\alpha+\beta)    &   \cos(\alpha+\beta)    
            \end{pmatrix}.
        \end{align}
    \end{subequations}
    Donc dans la base canonique, la matrice de \( R_0(\alpha)R_0(\beta)\) est celle de \( R_0(\alpha+\beta)\).
\end{proof}

%--------------------------------------------------------------------------------------------------------------------------- 
\subsection{Rotation et application affine}
%---------------------------------------------------------------------------------------------------------------------------

Nous considérons à nouveau la définition \ref{DEFooUAWZooXcMKve} d'une application affine ainsi que sa décomposition en application linéaire et translation donnée par le lemme \ref{LEMooYJCDooOGAHkF}. Nous voyons maintenant comment ces choses se mettent dans le cas d'une rotation non centrée en l'origine.

\begin{example}
    Soit \( A\in \eR^2\) ainsi qu'une rotation \( f\) autour de \( A\), c'est-à-dire une composition de deux symétries dont les axes se coupent en \( A\)\footnote{Voir la définition \ref{DEFooFUBYooHGXphm}.}. Nous allons extraire de \( f\) la partie linéaire définie en \ref{LEMooYJCDooOGAHkF}.

    Il existe des axes \( \ell_1\) et \( \ell_2\) tels que \( \ell_1\cap\ell_2=\{ A \}\) et tels que
    \begin{equation}
        f=s_{\ell_1}\circ s_{\ell_2}.
    \end{equation}
    En utilisant le lemme \ref{LEMooSMMMooAqsHWb},
    \begin{equation}
        f=t_A\circ s_{t^{-1}_A(\ell_1)}\circ t_A^{-1}\circ t_A\circ s_{t_A^{-1}(\ell_2)}\circ t_A^{-1}=t_A\circ s_{t^{-1}_A(\ell_1)}\circ s_{t_A^{-1}(\ell_2)}\circ t_A^{-1}.
    \end{equation}
    Vu que \( A\in\ell_i\), nous avons \( O=(0,0)\in t_A^{-1}(\ell_i)\). Donc les axes \( t_A^{-1}(\ell_1)\) et \( t_A^{-1}(\ell_2)\) se coupent en \( O\) et nous pouvons écrire
    \begin{equation}        %TODOooEZCRooAQsRkZ
        f=t_A\circ R\circ t_A^{-1}
    \end{equation}
    où \( R\) est une rotation centrée en \( O\); donc une application linéaire par la proposition \ref{PROPooTFNSooFjiWHG}.

    Nous avons
    \begin{subequations}
        \begin{align}
            f(x)&=(t_A\circ R\circ t_A^{-1})(x)\\
            &=(t_A\circ R)(x-A)\\
            &=t_A\big( R(x)-R(A) \big)\\
            &=R(x)+t_{R(A)+A}\\
            &=(t_{R(A)+A}\circ R)(x).
        \end{align}
    \end{subequations}
    Donc 
    \begin{equation}
        f=t_{R(A)+A}\circ R.
    \end{equation}
    Il est maintenant aisé de montrer que \( R\) est la partie linéaire de \( f\). Pour tout \( M,x\in \eR^2\) nous avons
    \begin{subequations}
        \begin{align}
            f(M+x)&=R(M+x)+R(A)+A\\
            &=R(M)+R(x)+R(A)+A\\
            &=R(x)+f(M).
        \end{align}
    \end{subequations}
    Donc ok pour la formule
    \begin{equation}
        f(M+x)=R(x)+f(M)
    \end{equation}
    et \( R\) est la partie linéaire de \( f\), voir la définition \ref{LEMooYJCDooOGAHkF}.
\end{example}

\begin{lemma}[\cite{MonCerveau}]        \label{LEMooUKEVooAEWvlM}
    Tout sous-groupe fini de \( \SO(2)\) est cyclique.
\end{lemma}

\begin{proof}
    Soit uns sous-groupe fini \(G\) de \( \SO(2)\).  Nous savons que \( \SO(2)\) est isomorphe à \( \gU(1)\) par le corolaire~\ref{CORooGGVUooLQYGET}, et en bijection avec \( \mathopen[ 0 , 2\pi \mathclose[\). Vu que \( G\) est fini, l'ensemble \( G\setminus\{ e \}\) il possède, dans \( \mathopen[ 0 , 2\pi \mathclose[\) un élément minimum non nul. Soit \( g_0\) ce minimum.

        Soit un élément \( g_1\) de \( G\) qui ne serait ni l'identité ni un multiple de \( g_0\). En particulier tous les nombres du type \( g_1-kg_0\) sont dans \( G\) (l'image de \( G\) dans \( \mathopen[ 0 , 2\pi \mathclose[\) en fait). Si \( g_1\) n'est pas un multiple de \( g_0\), il n'en reste pas moins que \( g_1=\lambda g_0\); alors en prenant pour \( k\) la partie entière de \( \lambda\), l'élément \( g_1-kg_0\) est plus petit que \( g_0\). Contradiction.
\end{proof}

%--------------------------------------------------------------------------------------------------------------------------- 
\subsection{Angle entre deux droites}
%---------------------------------------------------------------------------------------------------------------------------

Avant d'aborder la classification des isométries, nous devons parler de l'angle entre deux droites. Si \( \ell_1\) et \( \ell_2\) sont deux droites, alors il est bien clair deux angles peuvent prétendre être «l'angle entre \( \ell_1\) et \( \ell_2\)». De plus chacun de ces deux angles sont doubles parce que si \( \alpha\) peut prétendre être l'angle entre \( \ell_1\) et \( \ell_2\), alors \( -\alpha\) peut également prétendre.

\begin{remark}
    Nous ne parlons pas de l'angle entre \( \ell_1\) et \( \ell_2\) mais bien de l'angle \emph{de} \( \ell_1\) \emph{à} \( \ell_2\). L'ordre des droites est important.
\end{remark}

\begin{normaltext}
    Pour la suite, \( R_O(\alpha)\) est la rotation d'angle \( \alpha\) autour du point \( O\) tandis que \( R(\alpha)\) est la rotation d'angle \( \alpha\) autour de l'origine. 
\end{normaltext}

\begin{proposition}[\cite{ooGEXYooMTrOdH}]      \label{PROPooDWIMooQPkobw}
    Si \( u\) et \( v\) sont des vecteurs unitaires\footnote{De norme \( 1\).} de \( \eR^2\) alors il existe une unique rotation\footnote{Définition~\ref{DEFooFUBYooHGXphm}.} \( f\) telle que \( f(u)=v\).
\end{proposition}

\begin{proof}
    C'est la proposition~\ref{PROPooNXJKooEDOczh} appliquée à \( O=(0,0)\).
\end{proof}

\begin{remark}
    Notons l'unicité. Nous ne faisons pas de différences entre \( R_{\theta}\) et \( R_{\theta+2\pi}\) et les autres \( R_{\theta+2k\pi}\). En particulier si une rotation \( T\) est donnée, dire «\( T=R_{\theta}\)» ne définit pas un nombre \( \theta\) de façon univoque. Par contre ça définit une classe modulo \( 2\pi\), c'est-à-dire un élément \( \theta\in \eR/2\pi\).

    Nous avons déjà défini le groupe \( \SO(2)\) en la définition~\ref{DEFooJLNQooBKTYUy} et nous avons déterminé ses matrices dans \( \eR^2\) en le lemme~\ref{LEMooHRESooQTrpMz}.
\end{remark}



%--------------------------------------------------------------------------------------------------------------------------- 
\subsection{Angle orienté}
%---------------------------------------------------------------------------------------------------------------------------

La proposition~\ref{PROPooDWIMooQPkobw} donne une application
\begin{equation}
    T\colon S^1\times S^1\to \SO(2).
\end{equation}
Et nous avons une relation d'équivalence sur \( S^1\times S^1\) donnée par \( (u,v)\sim(u',v')\) si et seulement si il existe \( g\in\SO(2)\) telle que \( g(u)=u'\) et \( g(v)=v'\).

\begin{definition}[Angle orienté\cite{ooGEXYooMTrOdH}]      \label{DEFooVBKIooWlHvod}
    Les classes de \( S^1\times S^1\) pour cette relation d'équivalence sont les \defe{angles orientés de vecteurs}{angle!orienté de vecteurs}. Nous notons \( [u,v]\) la classe de \( (u,v)\).
\end{definition}

\begin{proposition}     \label{PROPooIWJQooGQJBWR}
    Nous avons \( T(u,v)=T(u',v')\) si et seulement si \( (u,v)\sim(u',v')\).
\end{proposition}

\begin{proof}
    En utilisant la commutativité du groupe \( \SO(2)\) nous avons équivalence entre les affirmations suivantes :
    \begin{itemize}
        \item \( (u,v)\sim (u',v')\)
        \item \( T(u,u')=T(v',v')\)
        \item \( T(u,u')\circ T(u',v)=T(v,v')\circ T(u',v)\)
        \item
            \( T(u,v)=T(u',v')\).
    \end{itemize}
\end{proof}

\begin{proposition}
    Nous avons une bijection
    \begin{equation}
        \begin{aligned}
            S\colon \frac{ S^1\times S^1 }{ \sim }&\to \SO(2) \\
            [u,v]&\mapsto T(u,v).
        \end{aligned}
    \end{equation}
\end{proposition}

\begin{proof}
    En plusieurs points.
    \begin{subproof}
    \item[\( S\) est bien définie]
        En effet si \( [u,v]=[z,t]\) alors \( T(u,v)=T(z,t)\).
    \item[Injectif]
        Si \( S[u,v]=S[z,t]\) alors \( T(u,v)=T(z,t)\), qui implique \( (u,v)\sim (z,t)\) par la proposition~\ref{PROPooIWJQooGQJBWR}.
    \item[Surjectif]
        Nous avons \( R_{\theta}=T(u,R_{\theta}u)\).
    \end{subproof}
\end{proof}

\begin{definition}[Somme d'angles orientés\cite{ooGEXYooMTrOdH}]
    Si \( [u,v]\) et \( [z,t]\) sont des angles orientés, nous définissons la somme par
    \begin{equation}
        [u,v]+[z,t]=S^{-1}\Big( S[u,v]\circ S[z,t] \Big).
    \end{equation}
\end{definition}

\begin{lemma}       \label{LEMooWISVooYsStJp}
    Quelques propriétés des angles plats liées à la somme.
    \begin{enumerate}
        \item
            \( (S^1\times S^1)/\sim\) est un groupe commutatif.
        \item       \label{ITEMooBKTFooWbEvIU}
            Relations de Chasles :
            \begin{equation}
                [u,v]+[v,w]=[u,w].
            \end{equation}
        \item
            \( -[u,v]=[v,u]\).
    \end{enumerate}
\end{lemma}

\begin{proof}
    Pour la relation de Chasles, ça se base sur la propriété correspondante sur \( T\) :
    \begin{subequations}
        \begin{align}
            [u,v]+[v,w]&=S^{-1}\Big( T(u,v)\circ T(v,w) \Big)\\
            &=S^{-1}\big( T(u,w) \big)\\
            &=[u,w].
        \end{align}
    \end{subequations}
    Pour l'inverse, la vérification est que
    \begin{equation}
        [u,v]+[v,u]=[u,u]=0.
    \end{equation}
\end{proof}

\begin{definition}      \label{DEFooFLGNooCZUkHY}
    La \defe{mesure}{mesure!angle entre vecteurs} de l'angle orienté \( [u,v]\) est \( [\theta]_{2\pi}\) si \( T[u,v]=R_{\theta}\).
\end{definition}
Notons dans cette définition qu'écrire \( T[u,v]=R_{\theta}\) dans \( \SO(2)\) ne définit pas \( \theta\), mais seulement sa classe modulo \( 2\pi\). C'est pour cela que la mesure de l'angle orienté n'est également définie que modulo \( 2\pi\).

Pour la suite nous allons nous intéresser à des vecteurs qui ont, dans l'idée, un point de départ et un point d'arrivée. Si \( A,B\in \eR^2\) nous notons
\begin{equation}
    \vect{ AB }=\frac{ B-A }{ \| B-A \| }.
\end{equation}
C'est le vecteur unitaire dans la direction «de \( B\) vers $A$».

\begin{theorem}[Théorème de l'angle inscrit\cite{ooRGSCooNgALYH}]       \label{THOooQDNKooTlVmmj}
    Soit un cercle \( \Gamma\) de centre \( O\) et trois points distincts \( A,B,M\in \Gamma\). Alors
    \begin{equation}
        2(\vect{ MA },\vect{ MB })\in (\vect{ OA },\vect{ OB })_{2\pi}
    \end{equation}
    où l'indice \( 2\pi\) indique la classe modulo \( 2\pi\).
\end{theorem}

\begin{proof}
    Le triangle \( MOA\) est isocèle en \( O\), donc les angles à la base sont égaux. Et de plus la somme des angles est dans \( [\pi]_{2\pi}\). Bon, entre nous, nous savons que la somme des angles est exactement \( \pi\), mais comme nous n'avons pas défini les angles autrement que modulo \( \pi\), nous ne pouvons pas dire mieux. Donc
    \begin{equation}
        2(\vect{ AB },\vect{ AO })+(\vect{ OB },\vect{ OA })\in [\pi]_{2\pi}.
    \end{equation}
    Il faut être sûr de l'orientation de tout cela. Le nombre \( (\vect{ AB },\vect{ AO })\) est l'angle qui sert à amener \( \vect{ AB }\) sur \( \vect{ AO }\). Vu que nous l'avons choisi dans le sens trigonométrique, il faut bien prendre les autres dans le sens trigonométrique et utiliser \( (\vect{ OA }, \vect{ OB })\) et non \( (\vect{ OB },\vect{ OA })\).

\begin{center}
   \input{auto/pictures_tex/Fig_YQIDooBqpAdbIM.pstricks}
\end{center}

De la même manière sur le triangle \( MOB\) nous écrivons
\begin{equation}
    2(\vect{ MB },\vect{ MO })+(\vect{ OM },\vect{ OB })\in[\pi]_{2\pi}.
\end{equation}
Nous faisons la différence entre les deux équations en remaquant que la différence de deux représentants de \( [\pi]_{2\pi}\) est un représentant de \( [0]_{2\pi}\) et en en nous souvenant que \( -(\vect{ MB },\vect{ MO })=(\vect{ MO },\vect{ MB })\) et les relations de Chasles du lemme~\ref{LEMooWISVooYsStJp}\ref{ITEMooBKTFooWbEvIU} nous avons :
\begin{equation}
    2(\vect{ MA },\vect{ MB })+(\vect{ OB },\vect{ OA })\in[0]_{2\pi}.
\end{equation}
\end{proof}

\begin{normaltext}
    Comment exprimer le fait qu'un angle orienté soit égal à \( \theta\) modulo \( \pi\) alors que les angles orientés sont des classes modulo \( 2\pi\) ? Nous ne pouvons certainement pas écrire
    \begin{equation}
        (u,v)=[\theta]_{\pi}
    \end{equation}
    parce que \( (u,v)\) est un élément de \( S^1\times S^1\) alors que \( [\theta]_{\pi}\) est un ensemble de nombres. Nous pouvons écrire
    \begin{equation}
        [u,v]\subset [\theta]_{\pi}.
    \end{equation}
    C'est cohérent parce que nous avons des deux côtés des ensembles de nombres. Les opérations permises sont l'égalité ou l'inclusion. L'égalité entre les deux ensembles n'est pas possible parce que la différence minimale ente deux éléments dans \( [u,v]\) est \( 2\pi\) alors que celle dans \( [\theta]_{\pi}\) est \( \pi\).

    Si \( u\) et \( v\) forment un angle droit, nous avons
    \begin{equation}
        [u,v]=\{ \frac{ \pi }{2}+2k\pi \}_{k\in \eZ}.
    \end{equation}
    Et cela est bien un sous-ensemble de \( [\pi/2]_{\pi}\).

    Pour exprimer que la différence entre deux angles orientés diffèrent de \( \pi\) nous devrions écrire :
    \begin{equation}
        [u,v]\subset[a,b]_{\pi}
    \end{equation}
    où le membre de droite signifie la classe modulo \( \pi\) d'un représentant de \( [a,b]\).

    Nous allons cependant nous permettre d'écrire
    \begin{equation}
        [u,v]=[a,b]_{\pi}
    \end{equation}
    voire carrément
    \begin{equation}
        (u,v)=(a,b)_{\pi}.
    \end{equation}
    Cette dernière égalité devant être comprise comme voulant dire que l'angle pour passer de \( u\) à \( v\) est soit le même que celui pour alle de \( a\) à \( b\) soit ce dernier plus \( \pi\).
\end{normaltext}

\begin{theorem}[\cite{ooRGSCooNgALYH}]      \label{THOooUDUGooTJKDpO}
    Soient \( 4\) points distincts du plan \( A,B,C,D\). Ils sont alignés ou cocycliques\footnote{C'est-à-dire sur un même cercle.} si et seulement si
    \begin{equation}
        (\vect{ CA },\vect{ CB })=(\vect{ DA },\vect{ DB })_{\pi}.
    \end{equation}
\end{theorem}

Nous allons seulement démontrer l'implication directe.
\begin{proof}
    Si les quatre points sont alignés nous avons \( [\vect{ CA },\vect{ CB }]=[0]_{2\pi}\) et \( [\vect{ DA },\vect{ DB }]=[0]_{2\pi}\). En particulier nous avons
    \begin{equation}
        [\vect{ CA },\vect{ CB }]=[\vect{ DA },\vect{ DB }]
    \end{equation}
    et a fortiori l'égalité modulo \( \pi\) au lieu de \( 2\pi\).

    Nous nous relâchons en termes de notations. Si les quatre points sont cocycliques, nous pouvons utiliser le théorème de l'angle inscrit~\ref{THOooQDNKooTlVmmj} dans les triangles \( ABC\) et \( ADB\) :
    \begin{subequations}
        \begin{align}
            2(\vect{ CA },\vect{ CB })=(\vect{ OA },\vect{ OB })_{2\pi}\\
            2(\vect{ DA },\vect{ DB })=(\vect{ OA },\vect{ OB })_{2\pi},
        \end{align}
    \end{subequations}
    ce qui donne \(  2(\vect{ CA },\vect{ CB })=2(\vect{ DA },\vect{ DB })_{2\pi}  \) et donc
    \begin{equation}
        (\vect{ CA },\vect{ CB })=(\vect{ DA },\vect{ DB })_{\pi}.
    \end{equation}

    Comme annoncé, nous ne faisons pas la preuve dans l'autre sens; elle peut être trouvée dans~\cite{ooRGSCooNgALYH}.
\end{proof}

\begin{example}     \label{EXooOXAAooZMdDfP}
    À propos de groupe engendré et de générateur\footnote{Définition \ref{DEFooWMFVooLDqVxR} et \ref{DefHFJWooFxkzCF}}. Soit \( G\) le groupe des rotations d'angle\footnote{Voir la définition \ref{DEFooFLGNooCZUkHY}.} \( k\pi/5\) (avec \( k \) entier). Ce groupe est constitué des \og{} dixièmes de tour \fg{}, puisque \( \frac{k\pi} 5 = \frac{2k\pi}{10}.\)
    
     La rotation d'angle \( 2 \pi/5\)  n'est pas génératrice parce qu'elle n'engendre que des \og{} cinquièmes de tour \fg{} : \( 4 \pi/5\), \( 6 \pi/ 5\),\( 8\pi/5\) et l'identité.
     
     Par contre, la rotation d'angle \( \pi/5\) est génératrice.
\end{example}

%---------------------------------------------------------------------------------------------------------------------------
\subsection{Angles et nombres complexes}
%---------------------------------------------------------------------------------------------------------------------------
\label{SUBSECooKNUVooUBKaWm}

Les nombres complexes peuvent être repérés par une norme et un angle, ce qui en fait un terrain propice à l'utilisation des angles orientés. Nous en ferons d'ailleurs usage dans \( \hat\eC=\eC\cup\{ \infty \}\) pour parler d'alignement, de cocyclicité et de birapport dans la proposition~\ref{PROPooSGCJooLnOLCx}.

Soient deux éléments \( z_1,z_2\in \eC\). Nous les écrivons sous la forme \( z_1=r_1 e^{i\theta_1}\) et \( z_2=r_2 e^{i\theta_2}\); remarquons que cela ne définit \( \theta_i\) qu'à \( 2\pi\) près. Nous avons
\begin{equation}
    [z_1,z_2]=[\theta_2-\theta_1]_{2\pi}.
\end{equation}

Soient maintenant \( a,b,c,d\in \eC\). Nous écrivons \( \vect{ ab }\) le vecteur unitaire dans le sens «de \( a\) vers \( b\)», c'est-à-dire un multiple positif bien choisi du nombre \( b-a\). Nous notons \( \theta_{ab}\) l'argument du nombre complexe \( b-a\), et nous avons encore
\begin{equation}
    [\vect{ ab },\vect{ cd }]=[\theta_{ab}-\theta_{cd}].
\end{equation}

Avec toutes ces notations, ce qui est bien est que les produits et quotients de nombres complexes se comportent très bien par rapport aux angles : l'argument de \( a/b\) est \( \theta_a-\theta_b\) et en particulier l'argument de
\begin{equation}
    \frac{ a-b }{ c-d }
\end{equation}
est dans la classe de l'angle orienté
\begin{equation}
    [\vect{ ba },\vect{ dc }].
\end{equation}


\begin{definition}      \label{DEFooUPUUooKAPFrh}
    Soient trois points \( A,O,B\in \eR^2\). Voici comment nous définissons l'angle \( \widehat{AOB}\); informelement c'est l'angle de la rotation qui fait aller de \( A\) vers \( B\).
    \begin{itemize}
        \item Nous nous mettons en l'origine : \( A'=A-O\) et \( B'=B-O\).
        \item Nous normalisons : \( A''=A'/\| A' \|\) et \( B''=B'/\| B' \|\).
        \item Soit \( f\), l'unique rotation telle que \( f(A'')=B''\) (proposition \ref{PROPooDWIMooQPkobw}).
        \item Soit \( \theta\) l'unique élément de \( \mathopen[ 0 , 2\pi \mathclose[\) tel que la matrice de \( f\) dans la base canonique soit
    \begin{equation}
        \begin{pmatrix}
            \cos(\theta)    &   -\sin(\theta)    \\
            \sin(\theta)    &   \cos(\theta)
        \end{pmatrix}
    \end{equation}
    par la proposition \ref{PROPooOTIVooZpvLnb}.
\item L'angle \( \widehat{AOB}\) est ce nombre.
    \end{itemize}
\end{definition}

Nous voyons que l'angle est toujours un nombre entre \( 0\) et \( 2\pi\). Par abus de notation, nous admettrons de temps en temps de parler d'angle en-dehors de cet intervalle.

\begin{lemmaDef}        \label{DEFooEGKOooRPGOAs}
    Si \( \ell_1\) et \( \ell_2\) sont deux droites de \( \eR^2\) sécantes au point \( O\) et si \( x\in\ell_1\) n'est pas \( O\), alors il existe un unique \( \alpha\in \mathopen[ 0 , \pi \mathclose[\) tel que \( R_O(\alpha)x\in \ell_2\). La valeur de \( \alpha\) ne dépend pas du choix du point \( x\in \ell_1\).

        Cet angle \( \alpha\) est l'\defe{angle}{angle!entre deux droites} de \( \ell_1\) à \( \ell_2\).
\end{lemmaDef}


\begin{proposition}[\cite{MonCerveau}]      \label{PROPooKVSHooRODGWE}
    Les angles sont invariants sous les translations. 

    Plus précisément, si \( A,B,S,v\in \eR^2\), alors
    \begin{equation}
        \reallywidehat{T_v(A)T_v(S)T_v(B)}=\reallywidehat{ASB}
    \end{equation}
    où \( T_v(X)=X+v\).
\end{proposition}

\begin{proof}
    Nous notons \( X_v=X+v\). Nous avons \( A'_v=A_v-S_v=(A+v)-(S+v)=A-S=A''\). Donc les vecteurs \( A''\) et \( B''\) à partir desquels est calculé \( \widehat{ASB}\) sont les mêmes que les vecteurs \(  A_v'' \) et \( B''_v\) qui servent à calculer \( \reallywidehat{T_v(A)T_v(S)T_v(B)}\).
\end{proof}

\begin{proposition}[\cite{MonCerveau}]      \label{PROPooYWKJooRjybUJ}
    Les angles sont invariants par rotations, c'est-à-dire que si \( A,B,S\in \eR^2\) et si \( R_{\theta}\) est une rotation, alors
    \begin{equation}
        \reallywidehat{ASB}=\reallywidehat{R_{\theta}(A)R_{\theta}(S)R_{\theta}(B)}.
    \end{equation}
\end{proposition}

\begin{proof}
    Pour être plus concis, nous écrivons \( A_{\theta}\) for \( R_{\theta}(A)\) et de même pour \( B\) et \( S\). Afin de calculer l'angle \reallywidehat{R_{\theta}(A)R_{\theta}(S)R_{\theta}(B)}, nous définissons
    \begin{subequations}
        \begin{numcases}{}
            A'_{\theta}=A_{\theta}-S_{\theta}\\
            S'_{\theta}=0\\
            B'_{\theta}=B_{\theta}-S_{\theta}.
        \end{numcases}
    \end{subequations}
    et
    \begin{subequations}
        \begin{numcases}{}
            A''_{\theta}=\frac{ A_{\theta}-S_{\theta} }{ \| A_{\theta}-S_{\theta} \| }\\
            B''_{\theta}=\frac{ B_{\theta}-S_{\theta} }{ \| B_{\theta}-S_{\theta} \| }.
        \end{numcases}
    \end{subequations}
    Par définition, l'angle est le \( \alpha\) tel que \( R_{\alpha}(A_{\theta}'')=B''_{\theta}\). Nous devons prouver que le même \( \alpha\) vérifie \( R_{\alpha}(A'')=B''\).

    Le fait que \( R_{\theta}\) soit une isométrie nous donne déjà
    \begin{equation}
        \| R_{\theta}(A)-R_{\theta}(B) \|=\| A-B \|.
    \end{equation}
    Ensuite, la relation de définition de \( \alpha\) s'écrit
    \begin{equation}
        \frac{ R_{\alpha}R_{\theta}(A)-R_{\alpha}R_{\theta}(S) }{ \| A_{\theta}-S_{\theta} \| }=\frac{ R_{\theta}(B)-R_{\theta}(S) }{ \| B_{\theta}-S_{\theta} \| }.
    \end{equation}
    Vu que $R_{\alpha}$ et \( R_{\theta}\) commutent, nous avons
    \begin{equation}
        R_{\theta}\frac{ R_{\alpha}(A)-R_{\alpha}(B) }{ \| A-S \| }=R_{\theta}\frac{ B-S }{ \| B-S \| },
    \end{equation}
    et comme \( R_{\theta}\) est inversible, cela donne \( R_{\alpha}(A'')=B''\).
\end{proof}

\begin{lemma}[\cite{MonCerveau}]        \label{LEMooJLHGooQIpKIE}
    Soit \( A\in \eR^2\) et une droite \( \ell_1\). Soit \( \ell_2\) une droite passant par \( A\) et intersectant \( \ell_1\) en \( O\). Alors
    \begin{equation}
        \sigma_{\ell_1}(A)=R_O(-2\alpha)A
    \end{equation}
    où \( \alpha\) est l'angle de \( \ell_1\) à \( \ell_2\).
\end{lemma}

\begin{proof}
    Nous allons utiliser des coordonnées autour de \( O\). Il existe un vecteur \( v\) tel que
    \begin{equation}
        A=O+v
    \end{equation}
    Par définition de l'angle \( \alpha\)\footnote{Définition \ref{DEFooEGKOooRPGOAs}.}, la droite \( \ell_2\) s'obtient par rotation d'angle \( \alpha\) depuis la droite \( \ell_1\). Donc le point
    \begin{equation}
        B=R_O(-\alpha)A
    \end{equation}
    est sur \( \ell_1\).

    Nous allons prouver que le point
    \begin{equation}
        D=R_O(-2\alpha)A
    \end{equation}
    est \( D=\sigma_{\ell_1}A\).

    Nous commençons par montrer que la droite \( (DA)\) est perpendiculaire à \( \ell_1\), c'est-à-dire que
    \begin{equation}
        (D-A)\cdot (B-O)=0.
    \end{equation}
    En utilisant le fait que
    \begin{equation}
        R_O(\alpha)(O+X)=O+R(\alpha)X,
    \end{equation}
    nous avons
    \begin{equation}
        D-A=R_O(-2\alpha)(O+v)-(O+v)=O+R(-2\alpha)v-O-v=R(-2\alpha)v-v
    \end{equation}
    et de la même façon,
    \begin{equation}
        B-O=R(-\alpha)v.
    \end{equation}
    Notons que tous les \( O\) se sont simplifiés et qu'il ne reste que des rotations usuelles. En utilisant le fait que \( R(\alpha)\) est une isométrie, nous pouvons alors calculer
    \begin{subequations}
        \begin{align}
            (D-A)\cdot (B-O)&=\langle R(-2\alpha)v-v, R(-\alpha)v\rangle \\
            &=\langle R(-\alpha)v-R(\alpha)v, v\rangle.
        \end{align}
    \end{subequations}
    En utilisant la matrice de rotation du lemme \ref{LEMooHRESooQTrpMz} nous trouvons
    \begin{equation}
        \big( R(-\alpha)-R(\alpha) \big)v=\begin{pmatrix}
            2\sin(\alpha)v_2    \\
            -2\sin(\alpha)v_1
        \end{pmatrix}
    \end{equation}
    et donc
    \begin{equation}
        \langle  \big( R(-\alpha)-R(\alpha) \big)v  , v\rangle =0.
    \end{equation}

    Le point \( D\) est bien sûr la droite perpendiculaire à \( \ell_1\) et passant par \( A\). Mais vu que \( D\) est obtenu à partir de \( A\) par une rotation, le point \( D\) est également sur le cercle de rayon \( \| OA \|\) et centré en \( O\). Ce cercle possède exactement deux intersections avec cette droite. Le premier est \( A\) et le second est \( \sigma_{\ell_1}(A)\). Vu que \( D\) n'est pas \( A\), nous avons \( D=\sigma_{\ell}(A)\).
\end{proof}

%---------------------------------------------------------------------------------------------------------------------------
\subsection{Classification}
%---------------------------------------------------------------------------------------------------------------------------

\begin{theorem}[\cite{ooZYLAooXwWjLa}]      \label{THOooRORQooTDWFdv}
    Toute isométrie du plan \( (\eR^2,d)\) est une composition d'au plus \( 3\) réflexions.
\end{theorem}

\begin{proof}
    Encore une fois nous décomposons la preuve en fonction du nombre de points fixes.
    \begin{subproof}
        \item[Si \( f\) n'a pas de points fixes]
            Soit \( x\in \eR^2\). Nous considérons le segment \( [f,f(x)]\) et nous nommons \( l\) sa médiatrice. Par construction, \( f(x)=\sigma_l(x)\). Nous posons \( g=\sigma_l\circ f\), et nous avons
            \begin{equation}
                g(x)=x.
            \end{equation}
            Donc nous avons \( f=\sigma_l\circ g\) avec \( x\in\Fix(g)\).
        \item[Si \( f\) a un unique point fixe]
            Soit \( x\) cet unique point fixe. Soit \( y\neq x\) et \( l\) la médiatrice de \( [y,f(y)]\). En posant \( g=\sigma_l\circ f\) nous avons
            \begin{equation}
                g(y)=y
            \end{equation}
            et \( g(x)=x\) parce que
            \begin{equation}
                d\big( x,f(y) \big)=d\big( f(x),f(y) \big)=d(x,y),
            \end{equation}
            ce qui donne que \( x\) est à égale distance de \( y\) et de \( f(y)\), c'est-à-dire que \( x\in l\) et par conséquent \( g(x)=(\sigma_l\circ f)(x)=\sigma_l(x)=x\).

            Donc \( g\) fixe \( x\) et \( y\) et donc toute la droite \( (xy)\).
        \item[Si \( f\) fixe une droite]
            Soit \( l\) une droite fixée par \( f\), et soient \( x,y\in l\) et \( z\notin l\) (avec \( x\neq y\)). Le fait que \( x\) et \( y\) soient des points fixes de \( f\) implique
            \begin{subequations}
                \begin{numcases}{}
                    d\big( x,f(z) \big)=d(x,z)\\
                    d\big( y,f(z) \big)=d(y,z)
                \end{numcases}
            \end{subequations}
            ce qui signifie que \( f(z)\) est sur l'intersection des deux cercles\footnote{L'intersection existe pare que \( d(x,z)+d(y,z)>d(x,y)\).} \( S\big( x,d(x,z) \big)\) et \( S\big( y, d(y,z) \big)\), et comme ce sont deux cercles centrés sur la droite \( l\), les intersections sont liées par \( \sigma_l\). Autrement dit, les intersections sont \( z\) et \( \sigma_l(z)\).

            Si \( f(z)=z\) alors \( f\) fixe trois points non alignés et fixe dont \( \eR^2\), c'est-à-dire \( f=\id\).

            Si par contre \( f(z)=\sigma_l(z)\) alors les isométries \( f\) et \( \sigma_l\) coïncident sur trois points et coïncident donc partout par le corolaire~\ref{CORooZHZZooDgTzsW} : \( f=\sigma_l\).
        \item[Conclusion]

            Nous avons montré que si \( \Fix(f)\) a dimension \( m\), alors il existe une droite pour laquelle \( f=\sigma_l\circ g\) avec \( \dim\big( \Fix(g) \big)>m\). Donc il faux au maximum trois pas pour avoir \( \dim\big( \Fix(g) \big)=2\) c'est-à-dire pour avoir \( g=\id\).
    \end{subproof}
\end{proof}

\begin{definition}      \label{DEFooJEOYooNwYtuQ}
    Une \defe{réflexion glissée}{réflexion!glissée} est une transformation du plan de la forme \( \tau_v\circ\sigma_{\ell}\) où le vecteur \( v\) est parallèle à la droite \( \ell\).
\end{definition}

\begin{theorem}[\cite{ooZYLAooXwWjLa}]      \label{THOooVRNOooAgaVRN}
    Les isométries du plan \( (\eR^2,d)\) sont exactement
    \begin{enumerate}
        \item
            l'identité (composée de \( 0\) réflexions),
        \item
            les réflexions,
        \item
            les translations (composées de \( 2\) translations d'axes parallèles),
        \item
            les rotations (composées de \( 2\) réflexions d'axes non parallèles),
        \item
            les réflexions glissées (composées de \( 3\) réflexions)
    \end{enumerate}
\end{theorem}

\begin{proof}
    Nous savons déjà que \( f\in \Isom(\eR^2)\) est une composée de \( 0\), \( 1\), \( 2\) ou \( 3\) réflexions.
    \begin{subproof}
        \item[Zéro réflexions]
            Alors c'est l'identité. Ce n'est pas très profond.
        \item[Une réflexion]
            Alors \( f\) est une réflexion. Toujours pas très profond.
        \item[Deux réflexions]
            Soit \( f=\sigma_{\ell_1}\circ\sigma_{\ell_2}\). Maintenant ça s'approfondit un bon coup.

            Nous supposons d'abord que \( \ell_1\parallel\ell_2\). Dans ce cas nous allons prouver que \( f=\tau_{2v}\) où \( v\) est le vecteur perpendiculaire à \(  \ell_1 \) tel que \( \ell_1+v=\ell_2\). Nous allons utiliser le lemme~\ref{LEMooVOJLooCFgdNG} pour montrer que \( \sigma_{\ell_1}\circ\sigma_{\ell_2}=\tau_{2v}\). Nous avons
            \begin{subequations}
                \begin{align}
                    \ell_1=\ell_0+w\\
                    \ell_2=\ell_0+w+v
                \end{align}
            \end{subequations}
            où \( w\) est un vecteur perpendiculaire à \( \ell_1\) et \( \ell_0\) est la droite passant par l'origine et parallèle à \( \ell_1\) et \( \ell_2\). Avec cela,
            \begin{subequations}
                \begin{align}
                    (\sigma_{\ell_1}\circ\sigma_{\ell_2})(x)&=\sigma_{\ell_1}\big( \sigma_{\ell_0}(x)+2w \big)\\
                    &=\sigma_{\ell_0}\big( \sigma_{\ell_0}(x)+2w \big)+2(v+w)\\
                    &=x+\underbrace{\sigma_{\ell_0}(2w)}_{-2w}+2v+2w\\
                    &=x+2v.
                \end{align}
            \end{subequations}
            Donc si \( f\) est composée de deux réflexions d'axes parallèles, alors \( f\) est une translation.

            Toujours dans le cas où \( f\) est composée de deux réflexions, nous supposons que \( f=\sigma_{\ell_2}\circ\sigma_{\ell_1}\) avec \( \ell_1\) et \( \ell_2\) non parallèles. Nous notons \( O\) le point d'intersection, et nous allons voir que \( f=R_O(2\alpha)\) où \( \alpha\) est l'angle de \( \ell_1\) à \( \ell_2\) donné par le lemme~\ref{DEFooEGKOooRPGOAs}.

            Soit \( x\in \ell_1\). Alors
            \begin{equation}
                f(x)=\sigma_{\ell_2}(x),
            \end{equation}
            et le lemme~\ref{LEMooJLHGooQIpKIE} nous donne un moyen de calculer \( \sigma_{\ell_2}(x)\) parce que \( \ell_1\) est une droite passant par \( x\) et coupant \( \ell_1\) au point \( O\). Le lemme dit que \( \sigma_{\ell_2}(x)=R_O(2\alpha)\). Remarque : c'est bien \( 2\alpha\) et non \( -2\alpha\) parce qu'il s'agit de l'angle de \( \ell_2\) à \( \ell_2\); il y a inversion des numéros entre ici et l'énoncé du lemme.

            Nous avons donc bien \( f(x)=R_O(2\alpha)x\) pour \( x\in \ell_1\).

            Si \( y\in\ell_2\) alors
            \begin{equation}
                f(y)=\sigma_{\ell_2}\big( R_O(-2\alpha)y \big)
            \end{equation}
            Nous posons \( z=\sigma_{\ell_1}(y)=R_O(-2\alpha)y\). Soit la droite \( \ell_3\) passant par \( O\) et \( z\). Vu que \( R_O(2\alpha)z=y\in \ell_2\), l'angle de \( \ell_3\) à \( \ell_2\) est \( 2\alpha\). Par conséquent
            \begin{equation}
                \sigma_{\ell_2}(z)=R_O\big( -2\times (-2\alpha) \big)z=R_O(4\alpha)z=R_O(4\alpha)R_O(-2\alpha)y=R_O(2\alpha)y.
            \end{equation}

            Donc les transformations \( f\) et \( R_O(2\alpha)\) coïncident pour tous les points des droites \( \ell_1\) et \( \ell_2\), qui ne sont pas parallèles. Cela prouve que \( f=R_{O}(2\alpha)\).

        \item[Trois réflexions]
            Nous écrivons \( f=\sigma_{\ell_3}\circ\sigma_{\ell_2}\circ\sigma_{\ell_1}\). Nous allons transformer cela progressivement en une symétrie glissée en passant par plusieurs étapes :
            \begin{enumerate}
                \item       \label{ITEMooHVYCooPhFMiv}
                    \( f=\sigma_{\ell}\circ\tau_v\),
                \item       \label{ITEMooUKGLooFlCcjt}
                    \( f=\tau_v\circ\sigma_{\ell}\),
                \item       \label{ITEMooWUCWooZSjofe}
                    \( f=\tau_v\circ\sigma_{\ell} \) avec \( v\parallel\ell\).
            \end{enumerate}
            À chacune de ces étapes, \( v\) et \( \ell\) vont changer. La dernière est une réflexion glissée.

            Nous commençons par supposer \( \ell_2\parallel\ell_3\). Dans ce cas, \( \sigma_{\ell_3}\circ\sigma_{\ell_2}\) est une translation, comme nous l'avons déjà vu. Alors \( f= \tau_v\circ\sigma_{\ell_1}\) et nous sommes déjà dans le cas~\ref{ITEMooUKGLooFlCcjt}.

            Nous supposons que \( \ell_2\) n'est pas parallèle à \( \ell_3\). Dans ce cas, si \( O=\ell_2\cap\ell_3\) nous avons
            \begin{equation}
                \sigma_{\ell_3}\circ\sigma_{\ell_2}=R_O(2\alpha)
            \end{equation}
            où \( \alpha\) est l'angle de \( \ell_2\) à \( \ell_3\). En réalité tant que l'angle de \( \ell'_3\) à \( \ell'_2\) est \( \alpha\) nous avons
            \begin{equation}
                \sigma_{\ell'_3}\circ\sigma_{\ell'_2}= \sigma_{\ell_3}\circ\sigma_{\ell_2}=R_O(2\alpha).
            \end{equation}
            Nous choisissons \( \ell'_2\) parallèle à \( \ell_1\), de telle sorte à ce que \( \sigma_{\ell'_2}\circ\sigma_{\ell_1}\) soit une translation. Alors nous avons
            \begin{equation}
                f=\sigma_{\ell_3}\circ\sigma_{\ell_2}\circ\sigma_{\ell_1}=\sigma_{\ell_3}\circ\sigma_{\ell'_2}\circ\sigma_{\ell'_1}=\sigma_{\ell_3}\circ\tau_v.
            \end{equation}
            où \( v\) est le vecteur de la translation en question.

            Nous avons donc prouvé que toute composition de trois réflexions peut être écrite soit sous la forme~\ref{ITEMooHVYCooPhFMiv} soit sous la forme~\ref{ITEMooUKGLooFlCcjt}.

            Nous prouvons à présent que toute transformation de la forme~\ref{ITEMooHVYCooPhFMiv} peut être écrite sous la forme~\ref{ITEMooUKGLooFlCcjt}. Plus précisément nous allons prouver que si \( \ell\) est une droite, \( v\) un vecteur et \( \ell_0\) la droite parallèle à \( \ell\) passant par l'origine, alors
            \begin{equation}
                \sigma_{\ell}\circ\tau_v=\tau_{\sigma_{\ell_0}(v)}\circ\sigma_l
            \end{equation}
            D'abord nous savons que \( \sigma_{\ell}(x)=\sigma_{\ell_0}(x)+2w\) où \( w\) est le vecteur tel que \( \ell=\ell_0+w\). Ensuite c'est un simple calcul utilisant le fait que \( \sigma_{\ell_0}\) est linéaire :
            \begin{equation}
                (\sigma_{\ell}\circ\tau_v)(x)=\sigma_l(x+v)=\sigma_{\ell_0}(x)+\sigma_{\ell_0}(v)+2w,
            \end{equation}
            et
            \begin{equation}
                (\tau_{\sigma_{\ell_0}(v)}\circ\sigma_{\ell})(x)=\sigma_{\ell_0}(v)+\sigma_{\ell}(x)=\sigma_{\ell_0}(v)+\sigma_{\ell_0}(x)+2w.
            \end{equation}
            L'égalité est faite.

            Nous montrons maintenant que toute transformation de la forme~\ref{ITEMooUKGLooFlCcjt} peut être mise sous la forme~\ref{ITEMooWUCWooZSjofe}. Soit donc \( f=\tau_v\circ\sigma_{\ell}\) où \( v\) et \( \ell\) ne sont pas spécialement parallèles.

            Pour cela nous décomposons \( v=v_1+v_2\) avec \( v_1\perp \ell\) et \( v_2\parallel\ell\) et nous posons \( \ell'=\ell+\frac{ 1 }{2}v_1\). Nous montrons que
            \begin{itemize}
                \item \( \tau_v\circ\sigma_{\ell}=\tau_{v_2}\circ\sigma_{\ell'}\)
                \item \( v_2\parallel \ell'\).
            \end{itemize}
            Pour le deuxième point, \( v_2\parallel\ell\) et bien entendu \( \ell'\parallel\ell\). Donc \( v_2\parallel\ell'\).

            Soit \( \ell_0\) la droite parallèle à \(  \ell\) et \( \ell'\) et passant par l'origine. Soit aussi le vecteur \( w\) tel que \( \ell=\ell_0+w\). Alors nous avons
            \begin{subequations}
                \begin{numcases}{}
                    \sigma_{\ell}=\sigma_{\ell_0}+2w\\
                    \sigma_{\ell'}=\sigma_{\ell_0}+2w+v_1
                \end{numcases}
            \end{subequations}
            Nous avons
            \begin{equation}
                (\tau_v\circ\sigma_{\ell})(x)=v+\sigma_{\ell_0}(x)+2w
            \end{equation}
            et
            \begin{subequations}
                \begin{align}
                    (\tau_{v_2}\circ\sigma_{\ell'})(x)&=v_2+\sigma_{\ell_0}(x)+2w+v_1\\
                    &=\sigma_{\ell_0}(x)+v+2w
                \end{align}
            \end{subequations}
            où dans la dernière ligne, nous avons regroupé \( v_1+v_2=v\). Et voilà.
    \end{subproof}
\end{proof}

%---------------------------------------------------------------------------------------------------------------------------
\subsection{Classification des isométries de \( \eR\)}
%---------------------------------------------------------------------------------------------------------------------------

\begin{definition}
    Soit \( x\in \eR\); nous notons \( \sigma_x\) la \defe{réflexion}{réflexion}\nomenclature[R]{\( \sigma_x\)}{réflexion par rapport à \( x\)} par rapport à \( x\), c'est-à-dire
    \begin{equation}
        \sigma_x(y)=2x-y.
    \end{equation}
\end{definition}

\begin{theorem}[\cite{ooZYLAooXwWjLa}]
    Toute isométrie de \( \eR\) est composée d'au plus \( 2\) réflexions. Plus précisément toute isométrie de \( \eR\) est dans une des trois catégories suivantes :
    \begin{itemize}
        \item l'identité (\( 0\) réflexions),
        \item les réflexions,
        \item les translations (\( 2\) réflexions)
    \end{itemize}
\end{theorem}

\begin{proof}
    Nous divisions la preuve en fonction du nombre de points fixés par l'isométrie \( f\in\Isom(\eR)\).
    \begin{subproof}
        \item[\( f\) fixe deux points distincts]
            Alors elle fixe l'espace affine engendrée par ces deux points par la proposition~\ref{PROPooVEEUooJQmmkN}. Donc \( f\) fixe tout \( \eR\) et est l'identité.
        \item[\( f\) fixe un unique point]
            Soit \( x\) l'unique point fixé par \( f\) et considérons \( y\neq x\). Vu que \( x=f(x)\) et que \( f\) est une isométrie,
            \begin{equation}
                d\big( x,f(y) \big)=d\big( f(x),f(y) \big)=d(x,y).
            \end{equation}
            Donc \( f(y)\) est à égale distance de \( x\) que \( y\). Autrement dit, \( f(y)\) est soit \( y\) soit \( \sigma_x(y)\). Mais comme \( x\) est unique point fixe, \( f(y)=\sigma_x(y)\). Ce raisonnement étant valable pour tout \( y\neq x  \) nous avons \( f=\sigma_x\).
        \item[\( f\) n'a pas de points fixes]
            Soient \( x\in \eR\) et \( y=\frac{ x+f(x) }{ 2 }\). Nous posons \( g=\sigma_y\circ f\). Alors \( x\) est un point fixe de \( g\) parce que
            \begin{equation}
                g(x)=\sigma_y\big( f(x) \big)=2y-f(x)=x.
            \end{equation}
            Donc soit \( g\) est l'identité soit \( g\) est une réflexion (par les points précédents). La possibilité \( g=\id\) est exclue parce que cela ferait \( f=\sigma_y\) alors que \( f\) n'a pas de points fixes. Donc \( g\) est une réflexion; et comme \( x\) est un point fixe de \( g\) nous avons \( g=\sigma_x\). Au final
            \begin{equation}
                f=\sigma_y\circ\sigma_x.
            \end{equation}
            Montrons que cela implique que \( f\) est une translation :
            \begin{equation}
                \sigma_y\sigma_x(z)=\sigma_y(2x-z)=2y-2x+z=z+2(y-x).
            \end{equation}
            Donc \( \sigma_y\circ\sigma_x\) est la translation de vecteur \( 2(y-x)\).
    \end{subproof}
\end{proof}


%---------------------------------------------------------------------------------------------------------------------------
\subsection{Isométries du tétraèdre régulier}
%---------------------------------------------------------------------------------------------------------------------------

\begin{definition}
    Un polyèdre \defe{régulier}{régulier!polyèdre} est un polyèdre dont les faces sont des polygones réguliers identiques dont tous les sommets joignent le même nombre d'arrêtes.
\end{definition}

\begin{definition}
    Le \defe{tétraèdre}{tétraèdre} est une pyramide à base triangulaire dont toutes les faces sont des triangles équilatéraux.
\end{definition}

\begin{proposition}[Isométries affines du tétraèdre régulier]       \label{PROPooVNLKooOjQzCj}
    Soient \( T\) un tétraèdre régulier et \( \Iso(T)\) son groupe d'isométries affines (définition~\ref{DEFooZGKBooGgjkgs}). Alors
    \begin{equation}
        \Iso(T)\simeq S_4
    \end{equation}
    où \( S_4\) est le groupe des permutations de quatre objets.
\end{proposition}

\begin{proof}
    Commençons par prouver qu'une isométrie préserve les sommets : l'image d'un sommet est un sommet. Pour cela nous considérons \( g\in \Iso(T)\) et nous supposons que l'image d'un sommet \( x\) soit à l'intérieur d'une arrête. Soient \( g(y)\) et \( g(z)\) deux points distincts de cette arrête situés à égale distance de \( g(x)\). Cela est possible parce que \( g\) est une bijection de \( \eR^3\). Aussi : \( y\neq z\). Mais une application affine préserve l'alignement (vous ne le croyez pas  ? regardez la forme donnée par le lemme \eqref{LEMooZZAIooOMiayy}), donc \( x\), \( y\) et \( z\) foment un triangle isocèle en \( x\) de points alignés et appartenant à \( T\). Cela est impossible si \( x\) est un sommet.

    Donc l'image d'un sommet est un sommet. Si nous numérotons les sommets \( x_1\),\ldots, \( x_4\), nous obtenons un morphisme de groupe \( \varphi\colon \Isom(T) \to S_4\) qui envoie \( g\) sur la permutation qui envoie \( 1\) sur le numéro du sommet \( g(x_1)\), \( 2\) sur le numéro du sommet \( g(x_2)\), etc.

    \begin{subproof}
    \item[Le morphisme \( \varphi\) est injectif]
        Supposons \( \varphi(g_1)=\varphi(g_2)\). Alors \( g_1^{-1}\circ g_2\) est une isométrie de \( (\eR^3,d)\) qui fixe les quatre sommets. Une application affine \( \eR^4\to\eR^3\) fixant \( 4\) point est l'identité par le lemme~\ref{LEMooDUMVooFtfFOe}. Donc \( g_1^{-1}g_2=\id\), ce qui prouve que \( g_1=g_2\). Vous noterez que nous utilisons l'unicité de l'inverse dans un groupe.

    \item[\( \varphi\) est surjectif]

        Nous savons que \( S_4\) est engendré par les transpositions (proposition~\ref{PropPWIJbu}). Or les transpositions sont dans l'image de \( \varphi\). En effet, notons les sommets de notre tétraèdre par \( A\), \( B\), \( D\) et \( D\) et considérons la transposition \( A\leftrightarrow B\). Elle est l'image par \( \varphi\) de la réflexion selon le plan \( \sigma\), médiateur du segment \( [A,B]\). Pour nous assurer de cela, nous devons nous assurer que \( C\) et \( D\) appartiennent à \( \sigma\). Cela est le contenu du lemme~\ref{LEMooVBVUooOTFFXT}.

    \item[Conclusion]

        L'application \( \varphi\) est un morphisme bijectif, c'est-à-dire un isomorphisme.

    \end{subproof}
\end{proof}

%---------------------------------------------------------------------------------------------------------------------------
\subsection{Représentation de \( S_4\) via les isométries du tétraèdre}
%---------------------------------------------------------------------------------------------------------------------------
\label{SUBSECooVEASooDUbsBh}


\begin{normaltext}
    Lorsque le tétraèdre a son barycentre en l'origine de \( \eR^3\), l'isomorphisme \( \varphi\colon \Iso(T)\to S_4\) donne une représentation de dimension \( 3\) de \( S_4\). Nous avons calculé les caractères de \( S_4\) en la section \ref{SecUMIgTmO} sans avoir besoin de savoir que l'une des représentations de dimension \( 3\) est cella que nous venons de trouver via le groupe des isométries du tétraèdre. Nous allons cependant également y calculer les caractères de la représentation \( \varphi\), pour le sport.
\end{normaltext}

Une des représentations trouvées (la représentation \( \rho_s\)) peut être vue comme le groupe \( \Iso(T)\) des isométries affine du tétraèdre grâce à la proposition \ref{PROPooVNLKooOjQzCj} qui donne un isomorphisme de groupe \( S_4\simeq \Iso(T)\) lorsque \( T\) est un tétraèdre régulier de \( \eR^3\).

Si le barycentre de \( T\) est situé à l'origine de \( \eR^3\), alors les éléments de \( \Iso(T)\) sont des applications linéaires parce que
\begin{itemize}
    \item les affinités laissent invariantes les barycentres (proposition~\ref{PROPooGSPZooRnVgiU}),
    \item les affinités qui laissent l'origine invariante sont linéaires (corolaire~\ref{CORooATCNooUwEPNI}).
\end{itemize}
Nous allons à présent calculer la trace de cette représentation, en utilisant le fait que nous la connaissions explicitement. Nous savons que les caractères sont constants sur les classes de conjugaison; nous allons donc écrire une matrice par classe de conjugaison (qui sont données dans l'exemple~\ref{EXooQAXRooBsPURs}).

Pour tout cela nous allons considérer un tétraèdre dont le centre (isobary) est en \( (0,0,0)\) et une base de \( \eR^3\) formée de trois sommets \( e_1\), \( e_2\) et \( e_3\). Vu que l'isobarycentre des quatre sommes est en \( (0,0,0)\), le quatrième somme est forcément le point de coordonnées \( e_4(-1,-1,-1)\), de telle sorte que \( e_1+e_2+e_3+e_4=0\).

\begin{description}
    \item[Les transpositions]

        Quelle isométrie de $\eR^3$ permute deux sommets du tétraèdre sans bouger les autres ? Pour permuter les sommets \( e_1\) et \( e_2\) en laissant \( e_3\) et \( e_4\), c'est le symétrie par rapport au plan médiateur de \( [e_1,e_2]\). Ce plan passe par les sommets \( e_3\) et \( e_4\), parce que le tétraèdre étant régulier, les points \( e_3\) \( e_4\) sont équidistants de \( e_1\) et \( e_2\). Le lemme~\ref{LEMooVBVUooOTFFXT} dit qu'alors ces points dont partie du plan médiateur.

        Dans notre base, la matrice de la transposition précédemment nommée \( (12)\) est
        \begin{equation}
            \begin{pmatrix}
                0    &   1    &   0    \\
                1    &   0    &   0    \\
                0    &   0    &   1
            \end{pmatrix},
        \end{equation}
        dont la trace est \( 1\). Donc \( \chi_s(12)=1\).

    \item[Les bitranspositions]

        La bitransposition \( (12)(34)\) est le produit des transpositions selon les plans médiateur de \( [e_1,e_2]\) et \( [e_3,e_4]\). Ces deux plans sont perpendiculaires, et l'intersection est la droite qui passe par les milieux. Cette droite est perpendiculaire aux deux segments en même temps. La matrice est :
        \begin{equation}
            \begin{pmatrix}
                0    &    1   &   -1    \\
                1    &   0    &   -1    \\
                0    &   0    &   -1
            \end{pmatrix}
        \end{equation}
        parce que \( e_1\mapsto e_2\), \( e_2\mapsto e_1\) et \( e_3\mapsto e_4\). Pour rappel, la matrice est formée des images des vecteurs de base. Cela donne
        \begin{equation}
            \chi_s\big( (12)(34) \big)=-1.
        \end{equation}

    \item[Les \( 3\)-cycles]

        La symétrie qui permute cycliquement les points \( e_1\), \( e_2\) et \( e_3\) est la rotation d'angle\footnote{Angle d'une rotation, définition \ref{DEFooADTDooKIZbrw}.} \( 2\pi/3\) dans le plan formé par les extrémités de ces trois vecteurs. Heureusement, la trace est invariante par changement de base; donc nous pouvons calculer la trace d'une rotation d'angle \( 2\pi/3\) dans n'importe quelle base. Par exemple :
        \begin{equation}
            \chi_s\big( (12)(34) \big)=\tr\begin{pmatrix}
                1    &   0    &   0    \\
                0    &   \cos(2\pi/3)    &   \sin(2\pi/3)    \\
                0    &   -\sin(2\pi/3)    &   \cos(2\pi/3)
            \end{pmatrix}=1+2\cos(2\pi/3)=0.
        \end{equation}

        Notons que, sans cette interprétation géométrique, nous y arrivons aussi facilement : dans notre base le \( 3\)-cycle est \( e_1\mapsto e_2\mapsto e_3\mapsto e_1\), donc la matrice est :
        \begin{equation}
            \begin{pmatrix}
                0    &   0    &   1    \\
                1    &   0    &   0    \\
                0    &   1    &   0
            \end{pmatrix},
        \end{equation}
        dont la trace est manifestement nulle : \( \chi_s\big( (123) \big)=0\).

    \item[Le \( 4\)-cycle]

        Il fait \( e_1\mapsto e_2\mapsto e_3\mapsto e_4\mapsto e_1\), dont la matrice est
        \begin{equation}        \label{EQooONDUooYlduup}
            \begin{pmatrix}
                0    &   0    &   -1    \\
                1    &   0    &   -1    \\
                0    &   1    &   -1
            \end{pmatrix},
        \end{equation}
        et la trace est \( \chi_s\big( (1,2,3,4) \big)=-1\).
\end{description}
Nous avons retrouvé les caractères de la représentation \( \rho_s\), et nous pouvons vérifier qu'elle est irréductible.

%+++++++++++++++++++++++++++++++++++++++++++++++++++++++++++++++++++++++++++++++++++++++++++++++++++++++++++++++++++++++++++ 
\section{Transformations de Lorentz}
%+++++++++++++++++++++++++++++++++++++++++++++++++++++++++++++++++++++++++++++++++++++++++++++++++++++++++++++++++++++++++++

Nous considérons dans cette section un nombre réel \( c>0\) ainsi que l'espace \( \eR^2\) muni du produit pseudo-scalaire\footnote{Définition \ref{DEFooLPBGooXLxubc}.} donné par la matrice
\begin{equation}
    \eta=\begin{pmatrix}
        c^2    &   0    \\ 
        0    &   -1    
    \end{pmatrix}.
\end{equation}
Et pour faire plus vrai, nous notons \( (x_0,x_1)\) les coordonnées sur \( \eR^2\). Ainsi
\begin{equation}
    x\cdot y=c^2x_0y_0-x_1y_1.
\end{equation}
Nous insistons sur le fait que cela n'est pas un produit scalaire.

\begin{lemma}[\cite{MonCerveau}]        \label{LEMooPZPZooVAdPVj}
    Soit \( c>0\). L'application
    \begin{equation}
        \begin{aligned}
        \varphi\colon \mathopen] -c , c \mathclose[&\to \eR \\
            v&\mapsto \frac{ -v/c }{ \sqrt{ 1-\frac{ v^2 }{ c^2 } } } 
        \end{aligned}
    \end{equation}
    est une bijection.
\end{lemma}

\begin{proof}
    Nous commençons par mentionner le fait que \( \varphi\) est continue du fait que le dénominateur ne s'annule pas. Une petite étude fonction montre que
    \begin{equation}
        \lim_{v\to -c} \varphi(v)=\infty,
    \end{equation}
    et
    \begin{equation}
        \lim_{v\to c} \varphi(v)=-\infty,
    \end{equation}
    et
    \begin{equation}
        \varphi'(v)=-\frac{1}{ c\sqrt{ 1-\frac{ v^2 }{ c^2 } } }-\frac{ v^2/c^3 }{ \left( 1-\frac{ v^2 }{ c^2 } \right)^{3/2} }<0.
    \end{equation}
    Tout cela fait que \( \varphi\) est bijective (entre autres par le théorème des valeurs intermédiaires \ref{ThoValInter} et la théorème dérivée et croissance \ref{PropGFkZMwD}).
\end{proof}

\begin{lemma}       \label{LEMooUZFKooSIjery}
    La forme bilinéaire
    \begin{equation}
        \begin{aligned}
            b\colon \eR^2\times \eR^2&\to \eR \\
            x,y&\mapsto x\cdot y
        \end{aligned}
    \end{equation}
    est non dégénérée\footnote{Définition \ref{DEFooNUBFooLfCqaK}.}.
\end{lemma}

\begin{proof}
    Soit \( (x_0,x_1)\in \eR^2\) tel que
    \begin{equation}
        b\big( (x_0,x_1), (y_0,y_1) \big)=0
    \end{equation}
    pour tout \( (y_0,y_1)\in \eR^2\). Nous avons
    \begin{equation}
        c^2x_0y_0-x_1y_1=0.
    \end{equation}
    En écrivant cela avec \( (y_0,y_1)=(1,0)\) puis \( (0,1)\) nous obtenons immédiatement que \( (x_0,x_1)=(0,0)\).
\end{proof}

\begin{theorem}     \label{THOooYHDWooWxVovH}
    Soit une bijection\quext{À mon avis, il y a moyen d'affaiblir cette hypothèse. Écrivez-moi si vous avez une idée.} \( f\colon \eR^2\to \eR^2\) telle que 
    \begin{equation}
        f(x)\cdot f(y)=x\cdot y
    \end{equation}
    pour tout \( x,y\in \eR^2\). Alors :
    \begin{enumerate}
        \item
            \( f\) est linéaire.
        \item
            Il existe un unique choix de \( (x,\sigma_1,\sigma_2)\in \eR\times \{ \pm1 \}\times \{ \pm1 \}\) tel que la matrice de \( f\) ait la forme
            \begin{equation}
                f=\begin{pmatrix}
                    \sigma_1\cosh(\xi)    &   \frac{ \sigma_1\sigma_2 }{ c }\sinh(\xi)    \\ 
                    c\sinh(\xi)    &   \sigma_2\cosh(\xi)    
                \end{pmatrix}.
            \end{equation}
        \item
        Il existe un unique \( v\in\mathopen] -c , c \mathclose[\) tel que la matrice de \( f\) ait la forme
            \begin{equation}
                f=\begin{pmatrix}
                    \frac{ \sigma_1 }{ \sqrt{ 1-\frac{ v^2 }{ c^2 } } }    &   -\frac{ \sigma_1\sigma_2 }{ c^2 }\frac{ v }{ \sqrt{ 1-\frac{ v^2 }{ c^2 } } }    \\ 
                    \frac{ -v }{ \sqrt{ 1-\frac{ v^2 }{ c^2 } } }    &   \frac{ \sigma_2 }{ \sqrt{ 1-\frac{ v^2 }{ c^2 } } }    
                \end{pmatrix}.
            \end{equation}
    \end{enumerate}
\end{theorem}

\begin{proof}
    Vu que notre produit pseudo-scalaire est non dégénéré (lemme \ref{LEMooUZFKooSIjery}), le fait que \( f\) soit linéaire est la proposition \ref{ThoDsFErq}. Nous posons 
    \begin{equation}
         A=\begin{pmatrix}
            \alpha    &   \beta    \\ 
            \gamma    &   \delta    
        \end{pmatrix}
    \end{equation}
    et, conformément à la proposition \ref{PROPooSYQMooEnZFdp} nous imposons \( A^t\eta A=\eta\). Après un petit produit matriciel nous obtenons :
    \begin{equation}
        \begin{pmatrix}
            c^2\alpha^2-\gamma^2    &   c^2\alpha\beta-\gamma\delta    \\ 
            c^2\alpha\beta-\gamma\delta    &   c^2\beta^2-\delta^2    
        \end{pmatrix}=\begin{pmatrix}
            c^2    &   0    \\ 
            0    &   -1    
        \end{pmatrix}.
    \end{equation}
    Voila quatre équations à résoudre pour les quatre inconnues \( \alpha, \beta,\gamma, \delta\). Déjà les équations des termes anti-diagonaux sont les mêmes. Nous recopions le reste :
    \begin{subequations}
        \begin{numcases}{}
            c^2\alpha^2-\gamma^2=c^2            \label{SUBEQooXZUGooITKZnH}\\
            c^2\alpha\beta-\gamma\delta=0       \label{SUBEQooDWQRooBeDaPw}\\
            c^2\beta^2-\delta^2=1.              \label{SUBEQooJAFLooGxmbaO}
        \end{numcases}
    \end{subequations}
    C'est le moment d'utiliser la proposition \ref{PROPooWEHGooOBqSHY}. La relation \eqref{SUBEQooXZUGooITKZnH} donne
    \begin{equation}
        \alpha^2-\left( \frac{ \gamma }{ c } \right)^2=1,
    \end{equation}
    ce qui implique l'existence (unique) de \( \xi_1\in \eR\) et \( \sigma_1\in \{ \pm 1 \}\) tels que
    \begin{subequations}        \label{SUBEQSooQUSIooRZRYSW}
        \begin{align}
            \gamma&=c\sinh(\xi_1)\\
            \alpha&=\sigma_1\cosh(\xi_1).
        \end{align}
    \end{subequations}
    La relation \eqref{SUBEQooJAFLooGxmbaO} implique quant à elle l'existence de \( \xi_2\in \eR\) et \( \sigma_2\in\{ \pm 1 \}\) tels que
    \begin{subequations}        \label{SUBEQSooLFHCooXVetmK}
        \begin{align}
            \delta&=\sigma_2\cosh(\xi_2)\\
            \beta&=\frac{1}{ c }\sinh(\xi_2).
        \end{align}
    \end{subequations}
    
    Nous substituons maintenant toutes les valeurs \eqref{SUBEQSooQUSIooRZRYSW} et \eqref{SUBEQSooLFHCooXVetmK} dans \eqref{SUBEQooDWQRooBeDaPw}. Cela donne
    \begin{equation}        \label{EQooHTMSooVYzJUS}
        \sigma_1\cosh(\xi_1)\sinh(\xi_2)=\sinh(\xi_1)\cosh(\xi_2).
    \end{equation}
    Nous mettons cette relation au carré et nous substituons \( \cosh(\xi_1)^2=1+\sinh^2(\xi_1)\). Ce que nous trouvons est
    \begin{equation}
        \sinh(\xi_1)^2=\sinh(\xi_2)^2,
    \end{equation}
    qui implique que \( \xi_1=\pm\xi_2\). Nous posons donc \( \xi_2=\sigma_3\xi_1\) pour un certain \( \sigma_3\in \{ \pm 1 \}\). Cela nous permet d'alléger la notation et d'écrire \( \xi\) au lieu de \( \xi_1\).
    
    Nous remettons la valeur \( \xi=\xi_1=\sigma_3\xi_2\) dans l'équation \eqref{EQooHTMSooVYzJUS} en tenant compte du fait que \( \sinh\) est impaire et \( \cosh\) est paire :
    \begin{equation}
        \sigma_1\sigma_3\cosh(\xi)\sinh(\xi)=\sigma_2\sinh(\xi)\cosh(\xi).
    \end{equation}
    Et cela nous enseigne que \( \sigma_3=\sigma_1\sigma_2\).

    Jusqu'à présent nous avons prouvé qu'il existe un unique \( \xi\in \eR\) et \( \sigma_1,\sigma_2\in \{ \pm 1 \}\) tels que
    \begin{equation}        \label{EQooYZIVooCTdmSh}
        A=\begin{pmatrix}
            \sigma_1\cosh(\xi)    &   \frac{ \sigma_1\sigma_2 }{ c }\sinh(\xi)    \\ 
            c\sinh(\xi)    &   \sigma_2\cosh(\xi)    
        \end{pmatrix}.
    \end{equation}
    
Nous utilisons à présent la bijection du lemme \ref{LEMooPZPZooVAdPVj}. Il existe un unique \( v\in \mathopen] -v , v \mathclose[\) tel que \( \sinh(\xi)=\varphi(v)\). En utilisant \( \cosh(\xi)^2=1+\varphi(v)^2\), nous trouvons
    \begin{equation}
        \cosh(\xi)^2=\frac{1}{ 1-\frac{ v^2 }{ c^2 } }.
    \end{equation}
    Mais comme le cosinus hyperbolique est toujours strictement positif, nous pouvons prendre la racine carrée des deux côtés :
    \begin{equation}
        \cosh(\xi)=\frac{1}{ \sqrt{ 1-\frac{ v^2 }{ c^2 } } }.
    \end{equation}
    En substituant dans \eqref{EQooYZIVooCTdmSh}, nous trouvons le résultat annoncé.
\end{proof}


%--------------------------------------------------------------------------------------------------------------------------- 
\subsection{Sous-groupe fini d'isométries du plan}
%---------------------------------------------------------------------------------------------------------------------------

\begin{theorem}[\cite{BIBooULRWooPsjtBE}]       \label{THOooKDMUooUxQqbB}
    Soit un groupe fini \( G\) d'isométries de \( (\eR^2,d)\) contenant \( n \) éléments.
    \begin{enumerate}
        \item       \label{ITEMooYEONooCOMpeb}
            Il existe un point \( C\in \eR^2\) fixé par tous les éléments de \( G\).
        \item       \label{ITEMooGELWooFFAqkc}
            Si \( G\) ne contient pas de réflexions, alors il est cyclique\footnote{Définition \ref{DefHFJWooFxkzCF}.} et engendré par la rotation d'angle \( 2\pi/n\) autour de \( C\).
        \item       \label{ITEMooDHKEooFpCfmX}
            Si \( G\) contient au moins une réflexion, et si \( C\) est un point fixe de \( G\), alors
            \begin{enumerate}
                \item       \label{ITEMooGQZTooJIPPLtyf}
                    toutes les réflexions ont un axe qui passe par \( C\),
                \item       \label{ITEMooKPQRooLquSiQ}
                    \( n\) est pair,
                \item       \label{ITEMooCHSWooHpDGHf}
                    Si \( \sigma\) est une réflexion dans \( G\), alors nous avons $G=\gr\big(\sigma,R_C(4\pi/n)\big)$ où \( R_C(\theta)  \) est la rotation d'angle \( \theta\) autour de \( C\),
                \item       \label{ITEMooROUYooRghvMv}
                    \( G\) est isomorphe au groupe diédral \( D_{n/2}\).
            \end{enumerate}
    \end{enumerate}
\end{theorem}

\begin{proof}
    Soit un groupe fini \( G\) constitué d'isométries de \( (\eR^2,d)\). Nous prouvons le théorème point par point.
    \begin{subproof}
        \item[Pour \ref{ITEMooYEONooCOMpeb}]
            C'est la proposition \ref{PROPooLAEBooWdcBoe}.
        \item[Questions de réflexions]
            Le théorème \ref{THOooRORQooTDWFdv} nous dit que les éléments de \( G\) sont des compositions d'au maximum \( 3\) réflexions. 
        \item[Exclure trois réflexions]
            Il n'est pas possible que \( G\) contienne un élément composé de trois réflexions. En effet, les composées de trois réflexions, par le théorème \ref{THOooVRNOooAgaVRN} sont des réflexions glissées\footnote{Définition \ref{DEFooJEOYooNwYtuQ}.}, c'est-à-dire des transformations de la forme \( g=\tau_v\circ \sigma_{\ell}\) où \( v\) est un vecteur parallèle à la droite \( \ell\). Si \( x\in \ell\), alors
            \begin{equation}
                g(x)=\tau_v(x)=x+v,
            \end{equation}
            de telle sorte que \( g^k(x)=x+kv\), qui signifie que tous les \( g^k\) sont différents. Le groupe \( G\) ne peut pas être fini si il contient une réflexion glissée.

        \item[\( G^+\) et \( G^-\)]
            Pour la même raison que celle qui exclu les réflexions glissées, \( G\) ne peut pas contenir de translations. Le théorème \ref{THOooVRNOooAgaVRN} nous donne la liste des possibilités. Après exclusion des translations et des réflexions glissées, il reste :
            \begin{itemize}
                \item l'identité
                \item les rotations,
                \item les réflexions.
            \end{itemize}
            Nous notons \( G^+\) la partie de \( G\) contenant l'identité et les rotations et \( G^-\) celle contenant les réflexions. Notons que \( G^+\) n'est pas vide parce qu'il contient au moins l'identité, tandis que \( G^-\) peut être vide, mais n'est certainement pas un groupe.
        \item[Même nombre d'éléments]
            Nous prouvons à présent que si \( G^-\) est non vide, alors il a le même nombre d'éléments que \( G\). Un élément de \( G^-\) est une réflexion. Soit \( \sigma\in G^-\). Nous prouvons que
            \begin{equation}        \label{EQooWRVVooBQCtPg}
                \begin{aligned}
                    \varphi\colon G^+&\to G^- \\
                    f&\mapsto \sigma\circ f 
                \end{aligned}
            \end{equation}
            est une bijection.

            \begin{subproof}
                \item[Surjective]
                    Soit \( s\in G^-\). Posons \( f=\sigma^{-1}\circ s\). Vu que \( \sigma^{-1}\) et \( s\) sont des réflexions, \( f\) est une rotation. Donc \( f\in G^+\) et \( \varphi(f)=s\).
                \item[Injective]
                    La condition \( \varphi(f)=\varphi(g)\) dit que \( \sigma\circ f=\sigma\circ g\). En composant par \( \sigma^{-1}\) nous obtenons \( f=g\).
            \end{subproof}
        \item[\( G=\gr\big(R_C(2\pi /p)\big)\)]
            Nous nommons \( p\) le nombre d'éléments de \( G^+\). Si \( G^-\) est vide, \( p=n\), et sinon \( p=n/2\). Dans les deux cas, \( G^+\) est un groupe de rotations à \( p\) éléments. 

            Le groupe \( G^+\) contient seulement des rotations; or le centre d'une rotation est l'unique point fixe. Donc tous les éléments de \( G^+\) sont des rotations autour de \( C\).
            
            Le corolaire \ref{CorpZItFX} au théorème de théorème de Lagrange nous indique que tous les éléments de \( G^+\) vérifient \( g^p=\id\). Seules les rotations d'angle \( 2k\pi/p\) autour de \( C\) satisfont la condition \( g^p=\id\). Or il n'y a que \( p\) telles rotations. Donc elles sont toutes dans \( G^+\). Nous en déduisons que
            \begin{equation}        \label{EQooUWTVooEMqkVH}
                G^+=\gr\big( R_C(2\pi/p) \big).
            \end{equation}
        \item[Pour \ref{ITEMooGELWooFFAqkc}]
            Dans le cas où \( G\) ne contient pas de réflexions, \( G^-\) est vide et \( G\) contient \( n\) éléments. La relation \eqref{EQooUWTVooEMqkVH} devient
            \begin{equation}
                G=G^+=\gr\big( R_C(2\pi/n) \big).
            \end{equation}
        \item[Pour \ref{ITEMooDHKEooFpCfmX}]
            Nous supposons maintenant que \( G\) contienne au moins une réflexion. De la sorte \( G^-\neq \emptyset\).
            \begin{subproof}
            \item[Pour \ref{ITEMooGQZTooJIPPLtyf}]
                Les seuls points fixes d'une réflexions sont ceux de l'axe. Donc \( C\) soit être sur tous les axes des réflexions contenues dans \( G^-\).

                Notons au passage que deux réflexions d'axes qui se coupent forment une rotation. Donc \( G^-\) ne forme pas un groupe, mais même pas en rêve.
            \item[Pour \ref{ITEMooKPQRooLquSiQ}]
                Vu que l'union \( G=G^+\cup G^-\) est disjointe et que \( G^+\) et $G^-$ ont le même nombre d'éléments par la bijection \ref{EQooWRVVooBQCtPg}, si \( G^-\) est non vide, \( G\) possède un nombre pair d'éléments.
            \item[Pour \ref{ITEMooCHSWooHpDGHf}]
                Si \( \sigma\in G\) est une réflexion, nous savons que \( G^+\) possède \( p=n/2\) éléments et que
                \begin{equation}
                    G^+=\{  R_C(2k\pi/p)  \}=\{  R_C(4k\pi/n) \}_{k=1,\ldots, n/2}.
                \end{equation}
                L'élément \( \sigma\in G^- \) étant fixé, la bijection \eqref{EQooWRVVooBQCtPg} nous indique que tous les éléments de \( G^-\) sont de la forme \( \sigma\circ f\) avec \( f\in G^+\). Donc
                \begin{equation}
                    G^{-}\subset \gr\big( \sigma, R_C(4\pi/n)  \big).
                \end{equation}
                Nous avons aussi
                \begin{equation}
                    G^+\subset \gr\big( \sigma,  R_C(4\pi/n)  \big).
                \end{equation}
                Et comme \( \sigma\) et \(  R_C(4\pi/n)  \) sont dans \( G\) nous avons \( \gr\big( \sigma ,  R_C(4\pi/n)  \big)\subset G\). Tout cela pour dire que 
                \begin{equation}
                    G=\gr\big( \sigma,  R_C(4\pi/n) \big).
                \end{equation}

            \item[\( R\sigma = \sigma R^{-1}\)]

                Nous restons dans le cas où \( G^-\) n'est pas vide. Nous considérons \( R\), la rotation d'angle \( \theta\) autour de \( C\). Si \( R_0\) est la rotation d'angle \( \theta\) autour de \( (0,0)\), nous avons
                \begin{equation}
                    R=\tau_C\circ R_0\circ \tau_C^{-1},
                \end{equation}
                et si \( \sigma_0\) est la symétrie d'axe parallèle à l'axe de \( \sigma\), mais passant par \( (0,0)\) nous avons : 
                \begin{equation}
                    \sigma=\tau_C\circ\sigma_0\circ\tau_C^{-1}.
                \end{equation}
                Si \( v\) est le vecteur directeur de la réflexion \( \sigma_0\), nous considérons enfin \( \alpha\), la rotation qui fait \( \alpha(v)=(1,0)\). Nous avons alors
                \begin{equation}
                    \sigma_0=\alpha^{-1}\circ s\circ \alpha
                \end{equation}
                où \( s\) est la symétrie autour de l'axe horizontal. En n'ayant pas peur d'identifier \( \eR^2\) à \( \eC\), l'applicaiton \( s\) est la conjugaison complexe. Avec tout ça nous avons
                \begin{equation}
                    R\sigma=\tau_CR_0\tau_C^{-1}\tau_C\sigma_0\tau_C^{-1}=\tau_CR_0\alpha^{-1}s\alpha\tau_C^{-1}=\tau_C\alpha^{-1}R_0s\alpha\tau_C^{-1}
                \end{equation}
                où nous avons utilisé le fait que les rotations autour de \( (0,0)\) forment un groupe abélien pour commuter \( \alpha^{-1}\) avec \( R_0\). Nous utilisaons à présent le lemme \ref{LEMooBNJFooAbhsUa} pour commuter \( R\) avec \( s\) :
                \begin{subequations}
                    \begin{align}
                        R\sigma&=\tau_X\alpha^{-1}sR_0^{-1}\alpha\tau_C^{-1}\\
                        &=\tau_C\underbrace{\alpha^{-1}s\alpha}_{\sigma_0} R_0^{-1}\tau_C^{-1}\\
                        &=\tau_C\sigma_0\tau_C^{-1}\tau_CR_0^{-1}\tau_C^{-1}\\
                        &=\sigma R^{-1}.
                    \end{align}
                \end{subequations}
                Nous avons utilisé le fat que \( \tau_CR_0^{-1}\tau_C^{-1}=R^{-1}\) comme on peut s'en convaincre en calculant le produit.

            \item[Table de multiplication]
                
                Nous considérons une réflexion \( \sigma\in G\). Les éléments de \( G^+ \) sont des rotations autour de \( C\) et ceux de \( G^-\) de la forme \( \sigma R\) où \( R\) est une rotation autour de \( C\). Pour savoir la table de multiplication de \( G\), nous devons écrire
                \begin{equation}
                    (\sigma^{\epsilon_1}R^k)(\sigma^{\epsilon_2}R^l)=\sigma^{\epsilon}R^m
                \end{equation}
                où \( \epsilon_1,\epsilon_2\in \{ 0,1 \}\), \( R\) est la rotation d'angle \( 4\pi/n\) autour de \( C\) et \( \alpha\) et \( m\) sont des constantes à exprimer en fonction de \( \epsilon_1\), \( \epsilon_2\), \( k\) et \( l\).

                Tous les éléments de \( G\) pouvant être écrits soit sous la forme \( R^m\) soit sous la forme \( \sigma R^m\), nous avons les possibilités suivantes :
                \begin{enumerate}
                    \item
                        \( R^mR^l=R^{m+l}\)
                    \item
                        \( (R^m)(\sigma R^l)=\sigma R^{-m}R^l=\sigma R^{l-m}\)
                    \item
                        \( (\sigma R^m)R^l=\sigma R^{m+l}\)
                    \item
                        \( (\sigma R^m)(\sigma R^l)=\sigma\sigma R^{l-m}=R^{l-m}\).
                \end{enumerate}
            \item[Pour \ref{ITEMooROUYooRghvMv}]
                Récoltons quelque faits.
                \begin{itemize}
                    \item 
                        Nous venons de prouver que \( R\sigma=\sigma R^{-1}\).
                    \item
                        Tout élément de \( G\) peut s'écrire soit sous la forme \( R^m\) soit sous la forme \( \sigma R^m\) suivant que l'élément soit dans \( G^+\) ou \( G^-\).
                    \item
                        Tout élément du groupe diédral \( D_n\) s'écrit soit sous la forme \( r^m\) soit sous la forme \( sr^m\) (proposition \ref{PropLDIPoZ}\ref{ITEMooOEBHooULRmZk}).
                \end{itemize}
                L'application \( \varphi\colon G\to D_n\) suivante est donc une bijection :
                \begin{subequations}
                    \begin{numcases}{}
                        \varphi(R^m)=r^m\\
                        \varphi(\sigma R^m)=sr^m.
                    \end{numcases}
                \end{subequations}
                Il nous reste à prouver que c'est un morphisme. Cela se fait en utilisant la table de multiplication du groupe diédral donnée dans la proposition \ref{PROPooPYDLooLgiUjk} et celle du groupe \( G\) que nous venons de faire.
            \end{subproof}
    \end{subproof}
\end{proof}

\begin{definition}[Groupe de symétrie d'une partie de \( \eR^n\)\cite{ooZYLAooXwWjLa}]
    Si \( Y\) est une partie de \( \eR^n\), nous définissons le \defe{groupe des symétries}{groupe!des symétries} de \( Y\) par
    \begin{equation}
        \Sym(Y)=\{ f\in\Isom(\eR^n)\tq f(Y)=Y \}.
    \end{equation}
    Nous définissons aussi le groupe des symétries propres de \( Y\) par
    \begin{equation}
        \Sym^+(Y)=\{ f\in\Isom^+(\eR^n)\tq f(Y)=Y \}.
    \end{equation}
\end{definition}

\begin{theorem}[\cite{ooZYLAooXwWjLa}]      \label{THOooAYZVooPmCiWI}
    Soit \( Y\subset \eR^2\) tel que le groupe \( \Sym^+(Y)\) soit fini d'ordre \( n\). Alors c'est un groupe cyclique d'ordre \( n\).

    Si \( \Sym^+(Y)\) est fini, alors \( \Sym(Y)\) est soit cyclique\footnote{Définition \ref{DefHFJWooFxkzCF}.} d'ordre \( n\) soit isomorphe au groupe diédral\footnote{Définition \ref{DEFooIWZGooAinSOh}.} d'ordre \( 2n\).
\end{theorem}

\begin{proof}
    Nous savons déjà par la proposition~\ref{PROPooEUFIooDUIYzi} que \( \Sym^+(Y)\) est isomorphe à un sous-groupe \( H^+\) d'ordre \( n\) de \( \SO(2)\). Vérifions que ce groupe est cyclique. Si \( n=1\), c'est évident. Si \( n\geq 2\) alors nous savons que \( H^+\) est constitué de rotations d'angles dans \( \mathopen[ 0 , 2\pi \mathclose[\) et vu que c'est un ensemble fini, il possède une rotation d'angle minimal (à part zéro). Notons \( \alpha_0\) cet angle.

        Nous montrons que \( H^+\) est engendré par la rotation d'angle \( \alpha_0\). Soit une rotation d'angle \( \alpha\). Étant donné que \( \alpha_0<\alpha\) nous pouvons effectuer la division euclidienne\footnote{Théorème~\ref{ThoDivisEuclide}.} de \( \alpha\) par \( \alpha_0\) et obtenir
        \begin{equation}
            \alpha=k\alpha_0+\beta
        \end{equation}
        avec \( \beta<\alpha_0\). Mézalors \( R(\beta)=R(\alpha)R(\alpha_0)^{-k}\) est également un élément du groupe. Cela contredit la minimalité dès que \( \beta\neq 0\). Avoir \( \beta=0\) revient à dire que \( \alpha\) est un multiple de \( \alpha_0\), ce qui signifie que le groupe \( H^+\) est cyclique engendré par \( \alpha_0\).

        Notons au passage que nous avons automatiquement \( \alpha_0=\frac{ 2\pi }{ n }\) parce qu'il faut \( R(\alpha_0)^n=\id\). Nous avons prouvé que \( \Sym^+(Y) \) est cyclique d'ordre \( n\).

        Nous étudions maintenant le groupe \( \Sym(Y)\). Par la proposition~\ref{PROPooEUFIooDUIYzi} nous avons un homomorphisme injectif
        \begin{equation}
            \phi\colon \Sym(Y)\to \gO(2),
        \end{equation}
        et en posant \( H=\phi\big( \Sym(Y) \big)\) nous avons un isomorphisme de groupes \( \phi\colon \Sym(Y)\to H\). Nous savons aussi que ce \( \phi\) se restreint en
        \begin{equation}
            \phi\colon   \Sym^+(Y) \to H^+\subset\SO(2)
        \end{equation}
        où \( H^+=\phi\big( \Sym^+(Y) \big)=H\cap\SO(2)\). Le groupe \( H^+\) est cyclique et est engendré par la rotation \( R(2\pi/n)\).

        Supposons un instant que \( H\subset \SO(2)\). Alors nous avons \( H=H^+\) et \( \phi\) est un isomorphisme entre \( \Sym(Y)\) et le groupe cyclique engendré par \( R(2\pi/n)\).

        Nous supposons à présent que \( H\) n'est pas un sous-ensemble de \( \SO(2)\). Quelles sont les isométries de \( \eR^2\) qui ne sont pas de déterminant \( 1\) ? Il faut regarder dans le théorème~\ref{THOooVRNOooAgaVRN} quelles sont les isométries contenant un nombre impair de réflexions. Ce sont les réflexions et les réflexions glissées. Or il ne peut pas y avoir de réflexions glissées dans un groupe fini parce que si \( f\) est une réflexion glissée, tous les \( f^k\) sont différents.

        Nous en déduisons que si \( H\) n'est pas inclus dans \( \SO(2)\), il contient une réflexion que nous nommons \( \sigma\). Nous allons en déduire que \( H\simeq H^+\times_{\AD}C_2\) où \( C_2=\{ \id,\sigma \}\). Si \( h\in H\) nous pouvons écrire
        \begin{equation}
            h=(h\sigma^{\epsilon})\sigma^{\epsilon}
        \end{equation}
        pour n'importe quelle valeur de \( \epsilon\), et en particulier pour \( \epsilon=\pm 1\).

        Si \( h\in \SO(2)\) alors nous écrivons \( h=h\epsilon^{0}\) et si \( h\notin\SO(2)\) nous écrivons \( h=(h\sigma)\sigma\). Vu que \( h\sigma\in\SO(2)\), cette dernière écriture est encore de la forme \( \SO(2)\times C_2\). Quoi qu'il en soit tout élément de \( H\) s'écrit comme un produit
        \begin{equation}
            H=H^+C_2.
        \end{equation}
        Cette décomposition est unique parce que si \( h_1c_1=h_2c_2\) alors \( h_2^{-1}h_1=c_2c_1^{-1}\), et comme \( h_2^{-1}h_1\in H^+\) nous avons \( c_2c_1^{-1}\in H^+\) et donc \( c_1=c_2\). Partant nous avons aussi \( h_1=h_2\). Pour avoir le produit semi-direct il faut encore montrer que \( \AD(C_2)H^+\subset H^+\). Le seul cas à vérifier est \( \AD(\sigma)H^+\subset H^+\). Vu que les éléments de \( H^+\) sont caractérisés par le fait d'avoir un déterminant positif, nous avons
        \begin{equation}
            \AD(\sigma)R(\alpha)=\sigma R(\alpha)\sigma^{-1}\in H^+.
        \end{equation}
\end{proof}

\begin{remark}
    Tout ceci est cohérent avec le théorème de Burnside~\ref{ThooJLTit} parce que le sous-groupe fini de \( \SO(n)\) engendré par la rotation \( R(2\pi/n)\) est un groupe d'exposant fini, à savoir que si \( h\) est dans ce groupe, \( h^n=\id\).
\end{remark}

%--------------------------------------------------------------------------------------------------------------------------- 
\subsection{Relations trigonométriques dans un triangle rectangle}
%---------------------------------------------------------------------------------------------------------------------------

Nous donnons maintenant quelques relations trigonométriques classiques dans un triangle rectangle. Le théorème de Pythagore est déjà le théorème \ref{THOooHXHWooCpcDan}; nous nous concentrons ici sur les angles.

\begin{proposition}
    Soient \( A,B,S\in \eR^2\) des points distincts et non alignés formant un triangle rectangle en \( A\) :
    \begin{equation}
        (A-S)\cdot (B-A)=0.
    \end{equation}
    En posant \( \theta=\reallywidehat{ASB}\) nous avons
    \begin{equation}
        \cos(\theta)=\frac{ \| A-S \| }{ \| B-S \| }
    \end{equation}
    et
    \begin{equation}        \label{EQooEKZEooFeNImX}
        \sin(\theta)=\pm\frac{ \| B-A \| }{ \| B-S \| }.
    \end{equation}
    
\end{proposition}

\begin{proof}
    Nous posons \( C=A-S\) et \( D=B-S\). Vu que \( C\neq 0\), il existe une rotation \( R_{\alpha}\) telle que
    \begin{subequations}
        \begin{numcases}{}
            (R_{\alpha}C)_x>0\\
            (R_{\alpha}C)_y=0.
        \end{numcases}
    \end{subequations}
    Nous posons \( X=R_{\alpha}C\) et \( Y=R_{\alpha}D\).

    Le triangle formé de \( O\), \( X\) et \( Y\) est «posé» sur l'axe des abscisses et est rectangle en \( X\), c'est-à-dire
    \begin{equation}
        X\cdot (Y-X)=0.
    \end{equation}
    De ce fait, le point \( Y\) satisfait à \( Y_x=X_x\). Et enfin, grâce aux propositions \ref{PROPooKVSHooRODGWE} et \ref{PROPooYWKJooRjybUJ} nous avons \( \widehat{ASB}=\widehat{XOY}\).

    Nous écrivons les relations qui définissent l'angle \( \widehat{XOY}\). Pour cela nous posons \( X'=X/\| X \|\) et \( Y'=Y/\| Y \|\) et nous avons
    \begin{subequations}        \label{SUBEQooVHNDooPOfbjC}
        \begin{numcases}{}
            \cos(\theta)=X'_xY'_x\\
            \sin(\theta)=X'_xY'_y.
        \end{numcases}
    \end{subequations}
    Vu que \( X=(X_x,0)\) nous avons \( X'_x=1\). De plus
    \begin{equation}
        \| Y \|=\| R_{\alpha}(D) \|=\| D \|=\| B-S \|.
    \end{equation}
    En substituant les valeurs dans \eqref{SUBEQooVHNDooPOfbjC},
    \begin{equation}
        \cos(\theta)=Y'_x=\frac{ Y_x }{ \| Y \| }=\frac{ X_x }{ \| B-S \| }=\frac{ \| C \| }{ \| B-S \| }=\frac{ \| A-S \| }{ \| B-S \| }.
    \end{equation}
    Voila déjà une chose de prouvée.

    Pour la seconde, nous avons \( \sin(\theta)=Y'_y\). Selon le signe de \( Y_y\) nous avons \( Y_y=\pm\| Y-X \|\) et donc
    \begin{equation}
        \sin(\theta)=Y'_y=\frac{ Y_y }{ \| Y \| }=\frac{ \pm\| Y-X \| }{ \| Y \| }=\pm\frac{ \| D-C \| }{ \| Y \| }=\pm\frac{ \| B-A \| }{ \| B-S \| }.
    \end{equation}
\end{proof}

Le signe sur la formule du sinus revient au fait que la définition de l'angle \( \widehat{AOB}\) est de considérer la rotation qui fait aller \( A\) vers \( B\). Donc suivant la position relative de \( A\), \( O\) et \( B\), il se peut que l'angle mesuré soit l'angle \emph{extérieur} au triangle.

La proposition suivante est parfois prise comme définition de l'angle. 
\begin{proposition}
    Soient trois points non alignés \( A,S,B\in \eR^2\). Nous avons
    \begin{equation}        \label{EQooOWULooVQntyE}
        \cos\big( \widehat{ASB} \big)=\frac{ (A-S)\cdot(B-S) }{ \| A-S \|\| B-S \| }.
    \end{equation}
\end{proposition}

\begin{proof}
    Nous posons \( C=A-S\), \( D=B-S\), \( X=C/\| C \|\) et \( Y=D/\| D \|\). Avec cela, la définition \ref{DEFooUPUUooKAPFrh} donne l'équation
    \begin{equation}
        \begin{pmatrix}
            \cos(\theta)    &   -\sin(\theta)    \\ 
            \sin(\theta)    &   \cos(\theta)    
        \end{pmatrix}\begin{pmatrix}
            X_x    \\ 
            X_u    
        \end{pmatrix}=\begin{pmatrix}
            Y_x    \\ 
            Y_y    
        \end{pmatrix}
    \end{equation}
    que nous écrivons comme le système
    \begin{subequations}
        \begin{numcases}{}
            X_c\cos(\theta)-X_y\sin(\theta)=Y_x\\
            X_y\cos(\theta)+X_x\sin(\theta)=Y_y.
        \end{numcases}
    \end{subequations}
    Nous considérons maintenant cela comme un système pour \( \big( \cos(\theta), \sin(\theta) \big)\) :
    \begin{equation}
        \begin{pmatrix}
            X_x    &   -X_y    \\ 
            X_y    &   X_x    
        \end{pmatrix}\begin{pmatrix}
            \cos(\theta)    \\ 
            \sin(\theta)    
        \end{pmatrix}=\begin{pmatrix}
            Y_x    \\ 
            Y_y    
        \end{pmatrix}.
    \end{equation}
    Le déterminant de la dernière matrice est \( X_x+X_y^2=\| X \|^2=1\) parce que \( X\) est unitaire. Cette matrice est donc inversible et son inverse est vite calculée. Nous avons
    \begin{equation}
        \begin{pmatrix}
            \cos(\theta)    \\ 
            \sin(\theta)    
        \end{pmatrix}=\begin{pmatrix}
            X_x    &   X_y    \\ 
            -X_y    &   X_x    
        \end{pmatrix}\begin{pmatrix}
            Y_x    \\ 
            Y_y    
        \end{pmatrix}.
    \end{equation}
    Cela donne ce que nous voulions :
    \begin{equation}
        \cos(\theta)=X\cdot Y=\frac{ C\cdot D }{ \| C \|\| D \| }=\frac{ (A-S)\cdot(B-S) }{ \| A-S \|\| C-S \| }.
    \end{equation}
\end{proof}

\begin{remark}
    Prendre la formule \eqref{EQooOWULooVQntyE} comme définition de l'angle $\widehat{ASB}$ est cependant trompeur parce que ça ne permet de définir les angles que sur une partie de \( \mathopen[ 0 , 2\pi \mathclose[\) sur laquelle le cosinus est injectif. Pour réellement définir tous les angles, il faut alors un peu bricoler.

    Sans vouloir être méchant, je crois que ceux qui prennent ça pour définition d'angle sont ceux qui donnent un cours sur le produit scalaire sans avoir l'intention de lier la définition d'une rotation comme composée de réflexions aux matrices de \( \SO(2)\) et aux fonctions trigonométriques.
\end{remark}

% This is part of (everything) I know in mathematics
% Copyright (c) 2011-2020
%   Laurent Claessens
% See the file fdl-1.3.txt for copying conditions.

%--------------------------------------------------------------------------------------------------------------------------- 
\subsection{Pavages du plan}
%---------------------------------------------------------------------------------------------------------------------------

\begin{definition}      \label{DEFooHPKFooSIDhCM}
    Une application affine \( f\colon \eR^n\to \eR^n\) est un \defe{déplacement}{déplacement} lorsqu'elle est une isométrie de \( (\eR^n,d)\) qui préserve l'orientation\footnote{Définition \ref{DEFooOTFPooIVkHFP}.}.
\end{definition}

\begin{definition}[\cite{NHXUsTa}]      \label{DEFooJPHKooRgCBJs}
    Un \defe{pavage}{pavage du plan} de \( \eR^2\) est une paire \( (G,K)\) où \( G\) est un groupe de déplacements\footnote{Définition \ref{DEFooHPKFooSIDhCM}.} de \( \eR^2\) et \( K\) un compact de \( \eR^2\) d'intérieur non vide telle que
    \begin{enumerate}
        \item
            \( G\cdot K=\eR^2\),
        \item       \label{ITEMooOIJZooZMKLUm}
            Si \( g_1,g_2\in G\) satisfont \( g_1\cdot \Int(K)\cap g_2\cdot\Int(K)\neq 0\), alors \( g_1\cdot K=g_2\cdot K\).
    \end{enumerate}
    Nous disons qu'un groupe de déplacements de \( \eR^2\) est un \defe{groupe de pavage}{groupe de pavage} de \( \eR^2\) si il existe un compact \( K\) tel que la paire \( (G,K)\) soit un pavage.
\end{definition}

En termes de notations,
\begin{equation}
    G\cdot K=\bigcup_{g\in G}g(K)=\bigcup_{g\in G}\bigcup_{k\in K}g(k).
\end{equation}

\begin{lemma}[\cite{MonCerveau}]        \label{LEMooWZSWooZYkICn}
    Soient une bijection affine \( \varphi\colon \eR^2\to \eR^2\) ainsi qu'une droite \( d\) et un point \( A\). Nous notons \( S\) la partie de \( \eR^2\) située du côté de \( d\) contenant \( A\).

    Alors \( \varphi(S)\) est la partie de \( \eR^2\) située du côté de \( \varphi(d)\) contenant \( \varphi(A)\).
\end{lemma}

\begin{proof}
    La droite \( d\) est donnée par une application affine \( f\colon \eR^2\to \eR\) et la définition
    \begin{equation}
        f=\{ x\in \eR^2\tq f(x)=0 \}.
    \end{equation}
    Nous supposons que \( f(A)>0\); sinon, nous pouvons utiliser \( -f\) au lieu de \( f\). Donc
    \begin{equation}
        S=\{ x\in \eR^2\tq f(x)>0 \}.
    \end{equation}
    La partie \( \varphi(S)\) est alors donnée par
    \begin{equation}
        \varphi(S)=\{ \varphi(x)\tq f(x)>0 \}.
    \end{equation}
    Comme \( \varphi\) est une bijection, cela s'écrit aussi bien
    \begin{equation}
        \varphi(S)=\{ y\in \eR^2\tq f\big( \varphi^{-1}(y) \big)>0 \}.
    \end{equation}
    De même
    \begin{equation}
        \varphi(d)=\{ s\in \eR^2\tq f\big( \varphi^{-1}(s) \big)=0 \}.
    \end{equation}
    Donc les deux côtés de la droite \( \varphi(d)\) sont donnés par le signe de \( f\circ \varphi^{-1}\). Nous avons
    \begin{equation}
        (f\circ\varphi^{-1})\big( \varphi(A) \big)=f(A)>0.
    \end{equation}
    Donc \( \varphi(A)\in \varphi(S)\).
\end{proof}

\begin{lemma}           \label{LEMooZOXVooTJiLTF}
    Le groupe 
    \begin{equation}
        G=\gr(\tau_{e_1}, \tau_{e_2})
    \end{equation}
    est un groupe de pavage du plan.
\end{lemma}

\begin{proof}
    Il suffit de prendre le carré \( K=\mathopen[ 0 , 1 \mathclose]\times \mathopen[ 0 , 1 \mathclose]\). En appliquant les translations, nous recouvrons tout le plan, sans intersections des intérieurs des carrés. Notons toutefois qu'il y a un recouvrement des bords.
\end{proof}

\begin{lemma}[\cite{MonCerveau, BIBooWIEGooJlwsCW}]    \label{LEMooTMRGooChBzZg}
    Le groupe 
    \begin{equation}
        G=\gr(\tau_{e_1}, \tau_{e_2}, R_0(\pi))
    \end{equation}
    est un groupe de pavage du plan.
\end{lemma}

\begin{proof}
    Le compact à considérer est \( K=\mathopen[ 0 , \frac{ 1 }{2} \mathclose]\times \mathopen[ 0 , 1 \mathclose]\). Le compact \( K\) et son image par \( R_0(\pi)\) sont représentés sur la figure \ref{LabelFigATJSooefYkmCbP}. % From file ATJSooefYkmCbP

    En agissant sur \( K\) avec les translations verticales et horizontales, nous recouvrons des bandes verticales de largeur \( 1/2\). En agissant de même sur \( R_0(\pi)(K) \), nous recouvrons les autres bandes verticales.

    Donc \( G\cdot K\) recouvre bien \( \eR^2\). Il serait cependant un peu présomptueux de croire en avoir fini. Il faut vérifier la condition \ref{ITEMooOIJZooZMKLUm} de la définition \ref{DEFooJPHKooRgCBJs} d'un pavage.

    Supposons que \( g_1\cdot\Int(K)\cap g_2\cdot \Int(K)\neq \emptyset\). Cela signifie qu'il existe \( k_1,k_2\in \Int(K)\) tels que \( g_1(k_1)=g_2(k_2)\), ou encore que
    \begin{equation}        \label{EQooITJEooUkUKuu}
        (g_2^{-1}g_1)k_1=k_2\in \Int(K).
    \end{equation}

    Quelle est la forme d'un élément général de \( G\) ? Le lemme \ref{LemFUIZooBZTCiy} nous indique qu'un élément général de \( G\) est un produit fini de \( \tau_{e_1}\), \( \tau_{e_2}\) et \( R_0(\pi)\). Mais nous savons que si \( \alpha\) est linéaire,
    \begin{equation}
        \alpha\circ \tau_u=\tau_{\alpha(u)}\circ \alpha.
    \end{equation}
    Dans notre cas, dans un produit général, nous pouvons déplacer tous les facteurs \( R_0(\pi)\) à droite en changeant des \( \tau_{e_i}\) en \( \tau_{R_0(\pi)e_i}=\tau_{-e_i}\). Les translations par contre commutent sans faire d'histoires. Donc un élément général de \( G\) est de la forme
    \begin{equation}
        g=\tau_{e_1}^k\tau_{e_2}^lR_0(\pi)^m
    \end{equation}
    Nous pouvons évidemment restreindre \( m\) à \( \{ 0,1 \}\). Supposons \( k\in \Int(K)\) et \( g(k)\in \Int(K)\). Nous avons \( 0<k_x<0.5\). Si \(m=1 \), alors \( g(k)_x\in\mathopen] -1/2 , 0 \mathclose[\) et aucun \( \tau_{e_1}^kg(k)_x\) ne pourra plus être entre \( 0\) et \( 1/2\). Donc \( m=0\). À partir de là, pour avoir \( g(k)\in \Int(K)\) nous devons avoir également \( k=l=0\). Donc \( g=\id\).

    Deux éléments \( g_1\) et \( g_2\) vérifiant la condition \eqref{EQooITJEooUkUKuu} doivent donc vérifier \( g_2^{-1}g_1=\id\), et donc \( g_1=g_2\). Par conséquent \( g_1\cdot K=g_2\cdot K\).
\end{proof}

\newcommand{\CaptionFigATJSooefYkmCbP}{Le compact \( K\) et son image par \( R_0(\pi)\) pour le lemme \ref{LEMooTMRGooChBzZg}.}
\input{auto/pictures_tex/Fig_ATJSooefYkmCbP.pstricks}

\begin{lemma}[\cite{MonCerveau, BIBooWIEGooJlwsCW}]         \label{LEMooJPNDooHDCLnY}
    Le groupe
    \begin{equation}
        \gr(\tau_{e_1}, \tau_{e_2}, R_0(\pi/2))
    \end{equation}
    est un groupe de pavage.
\end{lemma}

\begin{proof}
    Le pavé à considérer est
    \begin{equation}
        K=\mathopen[ -\frac{ 1 }{2} , 0 \mathclose]\times \mathopen[ 0 , \frac{ 1 }{2} \mathclose].
    \end{equation}
    En lui appliquant trois fois la rotation \( R_0(\pi/2)\), nous reconstituons le carré \( \mathopen[ -\frac{ 1 }{2} , \frac{ 1 }{2} \mathclose]\times \mathopen[ -\frac{ 1 }{2} , \frac{ 1 }{2} \mathclose]\). Ensuite, avec les translations, nous pavons tout le plan.

    Pour la seconde condition, nous procédons comme dans la démonstration du lemme \ref{LEMooTMRGooChBzZg}. D'abord \( R_0(\pi/2)e_1=e_2\) et \( R_0(\pi/2)e_2=-e_1\). Donc dans un produit général de \( \tau_{e_1}\), \( \tau_{e_2}\) et \( R_0(\pi/2)\) (et de leurs inverses), toutes les rotations peuvent être mises à droite; nous avons donc un élément général de \( G\) sous la forme
    \begin{equation}
        g=\tau_a\circ R_0(\pi/2)^k
    \end{equation}
    avec \( a\in \eZ e_1+\eZ e_2\) et \( k\in \{ 0,1,2,3 \}\).
    
    Vu que les translations se font par nombres entiers tandis que les différences de coordonnées entre les \( R_0(\pi/2)^kK\) sont demi-entiers, si \( k\in \Int(K)\), alors aucun \( \tau_a\) ne permet d'avoir \( (\tau_a\circ R_O(\pi/2))k\in \Int(K)\).

    Bref, si \( (g_2^{-1}g_2)k\in K\), alors encore une fois \( g_2^{-1}g_1=\id\).
\end{proof}

Et c'est maintenant que les choses compliquées commencent.

\begin{lemma}       \label{LEMooMWWEooEbZXtb}
    Le groupe
    \begin{equation}
        \gr\big( \tau_{e_1},\tau_{(-\frac{ 1 }{2},\frac{ \sqrt{ 3 } }{2})},R_0(\pi/3) \big)
    \end{equation}
    est un groupe de pavage.
\end{lemma}

\begin{proof}
    Ici le compact \( K\) est le triangle de sommets \( A=(0,0)\), \( B=(\frac{ 1 }{2},\frac{ \sqrt{ 3 } }{ 6 })\) et \( C=(1,0)\). Plus précisément il s'agit de l'intersection des trois parties suivantes :
    \begin{itemize}
        \item le côté de la droite \( (AB)\) où est \( C\),
        \item le côté de la droite \( (AC)\) où est \( B\),
        \item le côté de la droite \( (BC)\) où est \( A\).
    \end{itemize}
    Il est bon d'écrire ces trois conditions sous forme d'inéquations.
    \begin{itemize}
        \item La droite \( AB\) est donnée par l'équation \( f_{AB}(x,y)=0\) pour \( f_{AB}(x,y)=x-\sqrt{ 3 }y\). Vu que \( f_{AB}(C)=1\), la première inéquation pour \( K\) est
            \begin{equation}
                x-\sqrt{ 3 }y\geq 0.
            \end{equation}
        \item
            Pour la droite \( (BC)\) nous avons \( f_{BC}(x,y)=-x-\sqrt{ 3 }y+1\) et \( f_{BC}(A)=1\). Donc la seconde inéquation pour \( K\) est
            \begin{equation}
                -x-\sqrt{ 3 }y+1\geq 0
            \end{equation}
        \item
            La droite \( AC\) est donnée par l'application \( f_{AC}(x,y)=y\). Vu que \( f_{AC}(B)=\sqrt{ 3 }/6\) nous avons la troisième inéquation pour \( K\):
            \begin{equation}
                y\geq 0.
            \end{equation}
    \end{itemize}
    En résumé, la définition de \( K\) est le système
    \begin{subequations}        \label{SUBEQSooECKFooOdneOA}
        \begin{numcases}{}
                x-\sqrt{ 3 }y\geq 0\\
                -x-\sqrt{ 3 }y+1\geq 0\\
                y\geq 0.
        \end{numcases}
    \end{subequations}

    La figure  \ref{LabelFigPWMCooGWYCczZnssLabelSubFigPWMCooGWYCczZn0} nous montre ce triangle et l'action des puissances de \( R_0(\pi/3)\) sur lui. Pour votre lanterne, la matrice de cette rotation est
    \begin{equation}        \label{EQooMRRXooTebLlt}
        R_0(\pi/3)=\begin{pmatrix}
            1/2    &   -\sqrt{ 3 }/2    \\ 
            \sqrt{ 3 }/2    &   1/2    
        \end{pmatrix}.
    \end{equation}
    La figure \ref{LabelFigPWMCooGWYCczZnssLabelSubFigPWMCooGWYCczZn2} vous montre une partie de ce pavage dans toute sa splendeur.

    Vu que les translations sont \( u_1=e_1\) et \( u_2=(\frac{ 1 }{2}, \frac{ \sqrt{ 3 } }{2})\), il est suffisant de montrer que notre pavage pave réellement le parallélogramme construit sur ces deux vecteurs. Nous vous avons très obligeamment dessiné ce parallélogramme pavé sur la figure  \ref{LabelFigPWMCooGWYCczZnssLabelSubFigPWMCooGWYCczZn1}.

    Pour montrer que les triangles dessinés pavent effectivement le parallélogramme, nous allons faire une exhaustion de cas. Soit \( (x,y)\) dans le parallélogramme. 
    
    Nous commençons par couper le parallélogramme en deux parties suivant la diagonale allant de \( u_1\) à \( u_2\). Dans la vie mon p'ti gars, il y a deux types de points : ceux qui vérifient \( x+\frac{1}{ \sqrt{ 3 }}y-1\leq 0\) et les autres.

    \begin{subproof}
        \item[Si \( x+y/\sqrt{ 3 }-1\leq 0\)]
            Ceci est le côté de \( (0,0)\). Nous subdivisons suivant la petite barre verticale (suivez le dessin), c'est-à-dire suivant les deux cas : \( x\geq \frac{ 1 }{2}\) et \( x\leq \frac{ 1 }{2}\).
            \begin{subproof}
                \item[Si \( x\leq \frac{ 1 }{2}\)]
                    Enfin nous coupons avec la droite diagonale partant de \( (0,0)\), c'est-à-dire selon que \( x-\sqrt{ 3 }y\leq 0\) ou \( x-\sqrt{ 3 }y\geq 0\).
                    \begin{subproof}
                        \item[Si \( x-\sqrt{ 3 }y\leq 0\)]
                            Les points dont nous parlons sont les \( (x,y)\in \eR^2\) vérifiant
                            \begin{subequations}        \label{SUBEQSooNYWDooYMNVad}
                                \begin{numcases}{}
                                    x+\frac{1}{ \sqrt{ 3 } }y-1\leq 0\\
                                    x\leq \frac{ 1 }{2}\\
                                    x-\sqrt{ 3 }y\leq 0.
                                \end{numcases}
                            \end{subequations}
                            En suivant le dessin, vous remarquerez que l'élément de \( G\) à considérer est
                            \begin{equation}
                                g=\tau_{e_1}\circ \tau_{(-\frac{ 1 }{2},\frac{ \sqrt{ 3 } }{2})}\circ R_0(\pi/3)^4.
                            \end{equation}
                            Nous avons
                            \begin{subequations}        \label{EQSooOJBFooCTaTtu}
                                \begin{align}
                                    g(A)&=\big( \frac{ 1 }{2},\frac{ \sqrt{ 3 } }{2} \big)\\
                                    g(B)&=\big( \frac{ 1 }{2},\frac{ \sqrt{ 3 } }{ 6 } \big)\\
                                    g(C)&=(0,0).
                                \end{align}
                            \end{subequations}
                            Ce que les équations \eqref{SUBEQSooNYWDooYMNVad} décrivent est l'intersection des trois parties suivanteas\quext{Personnellement, je n'ai pas vérifié, mais ça m'étonnerait que ce soit faux. Vérifiez et écrivez-moi.} :
                            \begin{itemize}
                                \item le côté de la droite \( \big( g(A)g(B)\big)\) contenant \( g(C)\),
                                \item le côté de la droite \( \big( g(A)g(C)\big)\) contenant \( g(B)\),
                                \item le côté de la droite \( \big( g(B)g(C)\big)\) contenant \( g(A)\),
                            \end{itemize}
                            Le lemme \ref{LEMooWZSWooZYkICn} nous permet d'exprimer ces trois parties en termes de \( K\) :
                            \begin{itemize}
                                \item l'image par \( g\) du côté de la droite \( (AB)\) content \( C\),
                                \item l'image par \( g\) du côté de la droite \( (AC)\) content \( B\),
                                \item l'image par \( g\) du côté de la droite \( (BC)\) content \( A\),
                            \end{itemize}
                            Comme \( g\) est une bijection, l'intersection des images par \( g\) et l'image par \( g\) de l'intersection. Bref, les équations \eqref{EQSooOJBFooCTaTtu} décrivent l'image par \( g\) de \( K\).

                            Cette partie est donc pavée par \( (G,K)\).
                        \item[Si \( x-\sqrt{ 3 }y\geq 0\)]
                            hop.
                    \end{subproof}
                     
                \item[Si \( x\geq \frac{ 1 }{2}\)]
                    hop.
            \end{subproof}
        \item[Si \( x+y/\sqrt{ 3 }-1\geq 0\)]
            hop.
    \end{subproof}
    Les cas listés se traitement surement de la même façon. Nous tenons pour prouvé que \( (G,K)\) est bien surjectif sur \( \eR^2\).

    Nous devons encore montrer la condition \ref{ITEMooOIJZooZMKLUm} de la définition \ref{DEFooJPHKooRgCBJs}. Commençons par déterminer la forme générale d'un élément de \( G\) sous la forme \( \tau_a\circ r_0\) où \( r_0\) est une application linéaire. Le lemme \ref{LemFUIZooBZTCiy} nous indique qu'un élément général de \( G=\gr(\tau_{e_1}, \tau_{(\frac{ 1 }{2},\frac{ \sqrt{ 3 } }{2})}, R_0(\pi/3))\) est un produit arbitraire (mais fini) de \( \tau_{e_1}\), \( \tau_{(\frac{ 1 }{2},\frac{ \sqrt{ 3 } }{2})}\) et de \( R_0(\pi/3)\) et de leurs inverses.

    Comme toujours nous avons \( \alpha\circ\tau_v=\tau_{\alpha(v)}\circ \alpha\). Donc nous pouvons passer toutes les rotations \( R_0(\pi/3)\) et \( R_0(\pi/3)^{-1}\) à droite du produit quitte à produire des translations des formes suivantes :
    \begin{subequations}
        \begin{align}
            R_0(\pi/3)^ke_1\\
            R_0(\pi/3)^k\big( \frac{ 1 }{2},\frac{ \sqrt{ 3 } }{2} \big)\\
            R_0(\pi/3)^{-k}e_1\\
            R_0(\pi/3)^{-k}\big( \frac{ 1 }{2},\frac{ \sqrt{ 3 } }{2} \big).
        \end{align}
    \end{subequations}
    En utilisant la matrice \eqref{EQooMRRXooTebLlt} nous trouvons assez vite que
    \begin{subequations}        \label{SUBEQooEMVIooNaaMqk}
        \begin{align}
            R_0(\pi/3)e_1&=\begin{pmatrix}
                1/2    \\ 
                \sqrt{ 3 }/2
            \end{pmatrix}\\
            R_0(\pi/3)^2e_1&=\begin{pmatrix}
                -1/2    \\ 
                \sqrt{ 3 }/2    
            \end{pmatrix}\\
            R_0(\pi/3)^3e_1&=-e_1.
        \end{align}
    \end{subequations}
    Les applications suivantes de \( R_0(\pi/3)\) ne donnent rien de nouveau, si ce n'est le signe. Les puissances de \( R_0(\pi/3)\) appliquées à \( \big( \frac{ 1 }{2},\frac{ \sqrt{ 3 } }{2} \big)\) sont déjà parmi celles listées en \eqref{SUBEQooEMVIooNaaMqk}. Quant aux inverses, \( R_0(\pi/3)^{-1}=R_0(\pi/3)^5\); donc rien de nouveau non plus.

    Un élément général de \( G\) est donc dans
    \begin{equation}
        \tau_v\circ R_0(\pi/3)^k
    \end{equation}
    avec \( v\) dans
    \begin{equation}
        \eZ e_1+\eZ\begin{pmatrix}
            1/2    \\ 
            \frac{ \sqrt{ 3 } }{2}    
        \end{pmatrix}
        +\eZ\begin{pmatrix}
            -1/2    \\ 
            \frac{ \sqrt{ 3 } }{2}    
        \end{pmatrix}.
    \end{equation}
    Remarquons que
    \begin{equation}
        \begin{pmatrix}
            -1/2    \\ 
            \sqrt{ 3 }/2
        \end{pmatrix}=-e_1+\begin{pmatrix}
            1/2    \\ 
            \sqrt{ 3 }/2    
        \end{pmatrix}.
    \end{equation}
    Donc un élément général de \( G\) est 
    \begin{equation}
        \tau_{e_1}^k\circ \tau_{(-\frac{ 1 }{2},\frac{ \sqrt{ 3 } }{2})}^l\circ R_0(\pi/3)^m.
    \end{equation}
    
    Nous pouvons maintenant prouver notre point. Pour cela nous allons successivement considérer les \( 5\) rotations de \( K \) présentées dans la sous-figure \ref{LabelFigPWMCooGWYCczZnssLabelSubFigPWMCooGWYCczZn0}. Pour chacune nous allons montrer qu'aucune translation ne permet d'obtenir une intersection avec \( K\).

    Nous allons en faire un seul en détail.
    \begin{subproof}
        \item[Pour \( L=R_0(\pi/3)K\)]
        \item[Pour \( L=R_0(\pi/3)^2K\)]
            Nous allons prouver que si \( (x,y)\in \Int(L)\), alors
            \begin{equation}
                \tau_{e_1}^k\circ \tau_{(-\frac{ 1 }{2},\frac{ \sqrt{ 3 } }{2})}^l(x,y)
            \end{equation}
            ne peut pas être dans \( \Int(K)\). Pour la simplicité des notations nous notons \( r=R_0(\pi/3)^2\); nous avons
            \begin{equation}
                r=\begin{pmatrix}
                    -1/2    &   -\sqrt{ 3 }/2    \\ 
                    \sqrt{ 3 }/2    &   -1/2    
                \end{pmatrix}.
            \end{equation}
            Nous considérons \( g=\tau_{e_1}^k\circ\tau_{(-\frac{ 1 }{2},\frac{ \sqrt{ 3 } }{2})}\circ r\). Un calcul nous donne l'image de \( (x,y)\) par \( g\) :
            \begin{equation}
                g(x,y)=\tau_{e_1}^k\circ\tau_{(-\frac{ 1 }{2},\frac{ \sqrt{ 3 } }{2})}^l\begin{pmatrix}
                    -\frac{ 1 }{2}x-\frac{ \sqrt{ 3 } }{2}y   \\
                    -\frac{ \sqrt{ 3 } }{2}x-\frac{ 1 }{2}y    
                \end{pmatrix}=\begin{pmatrix}
                    -\frac{ 1 }{2}x-\frac{ \sqrt{ 3 } }{2}y+k-\frac{ 1 }{2}l    \\ 
                    \frac{ \sqrt{ 3 } }{2}x-\frac{ 1 }{2}y+\frac{ \sqrt{ 3 } }{2}l    
                \end{pmatrix}.
            \end{equation}
            Nous considérons \( (x,y)\in \Int(K)\) tel que \( g(x,y)\in \Int(K)\), et nous allons trouver une contradiction. Le fait que \( (x,y)\in \Int(K)\) signifie que \( (x,y)\) satisfait les inéquations \eqref{SUBEQSooECKFooOdneOA} mais avec des inégalités strictes. Le fait que \( g(x,y)\in\Int(K)\) nous donne trois inéquations de plus. Assez rapide calcul :
            \begin{subequations}
                \begin{align}
                    f_{AB}\big( g(x,y) \big)&=-2x+k-2l,\\
                    f_{BC}\big( g(x,y) \big)&=-x-\sqrt{ 3 }y-k-l+1,\\
                    f_{AC}\big( g(x,y) \big)&=\frac{ \sqrt{ 3 } }{2}x-\frac{ 1 }{2}y+\frac{ \sqrt{ 3 } }{2}l.
                \end{align}
            \end{subequations}
            Au final, notre point \( (x,y)\) doit satisfaire le système suivant :
            \begin{subequations}
                \begin{numcases}{}
                    x-\sqrt{ 3 }y> 0\label{SUBEQooQJBRooSvcvfW}\\
                    -x-\sqrt{ 3 }y+1> 0\label{SUBEQooLWJQooAIQhCh}\\
                y> 0        \label{SUBEQooYCVNooHJHVWt}\\
                -2x+k-2l>0      \label{SUBEQooYYVUooORxLnp}\\
                -x-\sqrt{ 3 }y-k-l+1>0  \label{SUBEQooRRGQooYtSxso}\\
                    \frac{ \sqrt{ 3 } }{2}x-\frac{ 1 }{2}y+\frac{ \sqrt{ 3 } }{2}l>0        \label{SUBEQooSNVNooVrIVVy}.
                \end{numcases}
            \end{subequations}
        Les inéquations \ref{SUBEQooYCVNooHJHVWt} et \ref{SUBEQooQJBRooSvcvfW} donnent déjà \( x>0\). De même avec \ref{SUBEQooLWJQooAIQhCh} nous trouvons \( x<1\). Voila déjà \( x\in \mathopen] 0 , 1 \mathclose[\) qui est directement visible sur le dessin du triangle \( K\).

            Vu que \( x>0\), l'inéquation \eqref{SUBEQooYYVUooORxLnp} donne
            \begin{equation}
                k-2l>-2x+k-2l>0.
            \end{equation}
            Donc \( k-2l>0\).

            Vu que \( x<1\) et \( y>0\), l'inéquation \eqref{SUBEQooSNVNooVrIVVy} donne
            \begin{equation}
                \frac{ \sqrt{ 3 } }{ 2 }+\frac{ \sqrt{ 3 } }{2}l> \frac{ \sqrt{ 3 } }{2}x-\frac{ 1 }{2}y+\frac{ \sqrt{ 3 } }{2}l>0.
            \end{equation}
            Donc \( \frac{ \sqrt{ 3 } }{2}(1+l)>0\). Comme \( l\) est entier, cela donne \( l\geq 0\).

            Enfin de \eqref{SUBEQooRRGQooYtSxso} nous tirons
            \begin{equation}
                -k+1>-x-\sqrt{ 3 }y-k-l+1>0,
            \end{equation}
            Ce qui donne \( k<1\) et donc \( k\leq 0\).

            En résumé nous avons trouvé trois inéquations pour \( k\) et \( l\) :
            \begin{subequations}
                \begin{numcases}{}
                    k\leq 0\\
                    l\geq 0\\
                    k-2l>0.
                \end{numcases}
            \end{subequations}
            Ce système est impossible.

            Il n'existe donc pas de translations qui, appliquée à \( \Int(L)\), donne une intersection avec \( \Int(K)\).
            
        \item[Pour \( L=R_0(\pi/3)^3K\)]
            hop.
        \item[Pour \( L=R_0(\pi/3)^4K\)]
            hop.
        \item[Pour \( L=R_0(\pi/3)^5K\)]
            hop.
    \end{subproof}
    Voila. J'espère que toutes les idées sont en place, et que les parties manquantes sont seulement des vérifications qui se font mécaniquement de la même manière\quext{Je n'ai pas vérifié. Faites-le et écrivez-moi pour me dire ce qui en est.}.
\end{proof}

\newcommand{\CaptionFigPWMCooGWYCczZn}{Illustrations pour le pavage du lemme \ref{LEMooMWWEooEbZXtb}.}
\input{auto/pictures_tex/Fig_PWMCooGWYCczZn.pstricks}


\begin{lemma}       \label{LEMooGSQSooGSfkaL}
    Le groupe 
    \begin{equation}
      \gr\big(\tau_1,\tau_{(-\frac{ 1 }{2},\frac{ \sqrt{ 3 } }{2})},R_0(\pi/3\big))
    \end{equation}
    est un groupe de pavage.
\end{lemma}

D'après \cite{BIBooWIEGooJlwsCW}, la démonstration du lemme \ref{LEMooGSQSooGSfkaL} demande d'utiliser le losange de sommets \( (0,0)\), \( (1,0)\), \( (\frac{ 1 }{2},\frac{ \sqrt{ 3 } }{ 6 })\) et \( (\frac{ 1 }{2}, -\frac{  \sqrt{ 3 } }{ 6 })\)\quext{Je n'ai pas vérifié, mais à mon avis une preuve doit prendre les mêmes idées que celles du lemme \ref{LEMooMWWEooEbZXtb}.}.


\begin{lemma}[\cite{MonCerveau}]        \label{LEMooEKWZooYbcGBp}
    Soient \( u_1,u_2\in \eR^2\) non colinéaires, ainsi que \( \alpha,\beta\in \mathopen[ 0 , 1 \mathclose]\). Nous considérons \( v=\alpha u_1+\beta u_2\).

    Nous supposons pour fixer les idées que \( \| u_1 \|\geq \| u_2 \|\). Alors
    \begin{equation}
        \min\{ \| v \|, \| u_1+u_2-v \| \}\leq \| u_1 \|.
    \end{equation}
    Autrement dit, tout point intérieur d'un parallélogramme est plus proche d'un angle que la longueur du plus long côté.
\end{lemma}

\begin{proof}
    Les points \( \alpha u_1+\beta u_2\) (\( \alpha,\beta\in \mathopen[ 0 , 1 \mathclose]\)) se divisent en deux parties : ceux avec \( 0\leq\alpha+\beta\leq 1\) et ceux avec \( 1\leq\alpha+\beta\leq 2\).

    Si \( \alpha+\beta\leq 1\) alors
    \begin{equation}
        \| \alpha u_1+\beta u_2 \|<\| \alpha u_1 \|+\beta\| u_2 \|= \alpha\| u_1 \|+\beta\| u_2 \|\leq (\alpha+\beta)\| u_1 \|\leq \| u_1 \|.
    \end{equation}
    L'inégalité est stricte parce que \( u_1\) et \( u_2\) ne sont pas colinéaires.

    Si au contraire \( \alpha+\beta\geq 1\) nous avons
    \begin{equation}
        \| u_1+u_2-\alpha u_1-\beta u_2 \|<\| (1-\alpha)u_1 \|+\| (1-\beta)u_2 \|<(2-\alpha-\beta)\| u_1 \|\leq \| u_1 \|.
    \end{equation}
\end{proof}

\begin{lemma}       \label{LEMooWKTGooQlfuxm}
    Soient un ensemble \( E\), et deux bijections \( r,s\colon E\to E\) ayant chacune un unique point fixe. Si elles commutent, alors leurs points fixes sont égaux. 
\end{lemma}

\begin{proof}
    Nous nommons \( a\) le point fixe de \( r\) et \( b\) celui de \( s\). Pour tout \( x\in E\) nous avons \( (rs)(x)=(sr)(x)\). En particulier pour \( x=s^{-1}(a)\). D'une part
    \begin{equation}
        (rs)(x)=r(a)=a.
    \end{equation}
    Et d'autre part,
    \begin{equation}
        (sr)(a)=(srs^{-1})(a)
    \end{equation}
    Si nous imposons \( (srs^{-1})(a)=a\), nous avons, en appliquant \( s^{-1}\) est deux côtés : \( (rs^{-1})(a)=s^{-1}(a)\). Cela prouve que \( s^{-1}(a)\) est un point fixe de \( r\). Donc \( s^{-1}(a)=a\).

    Nous en déduisons que \( a\) est un point fixe de \( s\) et donc que \( a=b\).
\end{proof}

\begin{lemma}       \label{LEMooDGSJooCiBhZz}
    Soit un sous-groupe fini \( T\) de \( (\eR^2,+)\) tel que \( \| v \|>\delta \) pour tout \( v\neq 0\) dans \( T\).

    Alors si \( u_1\) et $u_2$ sont les plus petits éléments en norme de \( T\), nous avons
    \begin{equation}
        T=\eZ u_1+\eZ u_2.
    \end{equation}
\end{lemma}

\begin{proof}
    Nous décomposons en plusieurs parties.

    \begin{subproof}
        \item[\( \eR^+v\cap T=\eN v_m\)]
            Soit \( v\in T\). L'ensemble \( \{ \lambda \in \eR^+\tq \lambda v\in T\}\) a un minimum parce que tous les éléments de \( T\) sont en norme plus grands que \( \delta>0\). Soit \( \lambda_m\) ce minimum et \( v_m=\lambda_mv\).

            Nous prétendons à présent que \( \eR^+v\cap T=\eN v_m\). Nous ne faisons d'ailleurs pas que prétendre; nous \emph{prouvons}. En effet, doit \( \lambda v\in T\). Nous devons prouver que \( \lambda = l\lambda_m\) pour un certain \( l\in \eN\). Soit \( k\in \eN\) tel que \( k\leq \lambda<k+1\).

            Nous avons
            \begin{equation}
                (k+1)\lambda_m-\lambda\leq (k+1)\lambda_m-k\lambda_m=\lambda_m.
            \end{equation}
            Vu que \( T\) est un groupe pour l'addition, les faits que \( \lambda_mv\in T\) et que \( \lambda v\in T\) impliquent que \( \big( (k+1)\lambda_m-\lambda \big)v\in T\). Mais
            \begin{equation}
                | \big( (k+1)\lambda_m-\lambda \big)v |\leq \lambda_m.
            \end{equation}
            Vu la propriété de minimalité de \( \lambda_m\) nous avons forcément
            \begin{equation}
                (k+1)\lambda_m-\lambda = \lambda_m.
            \end{equation}
            Cela prouve que \( \lambda=k\lambda_m\). 
            
            Jusqu'ici nous avons prouvé que
            \begin{equation}
                \eR^+v\cap T=\eN v_m
            \end{equation}
            pour un certain multiple \( v_m\) de \( v\).

        \item[\( \| u_1 \|\leq \| v \|\) pour tout \( v\)]

            Nous montrons à présent qu'il existe \( u_1\in T\) tel que \( \| u_1 \|\leq \| v \|\) pour tout \( v\in T\). Si tel n'était pas le cas, il existerait une suite \( v_k\in T\) telle que \( \| v_{k+1} \|<\| v_k \|\). Toute cette suite serait contenue dans le couronne (compacte) de rayons \( \delta\) et \( \| u_1 \|\). Quitte à prendre une sous-suite, nous pouvons supposer que \( (v_k)\) converge\footnote{Proposition \ref{THOooRDYOooJHLfGq}.}. Cette suite serait de Cauchy et pour tout \( \epsilon\) (en particulier \( \epsilon<\delta\)), il existerait \( p,q\) tels que \( \| v_p-v_q \|<\epsilon\). Vu que \( T\) est un groupe pour l'addition, nous aurions \( v_p-v_q\in T\) avec \( \| v_p-v_q \|<\epsilon\leq \delta\).

        \item[Première pause]

            Si \( T\) est engendré seulement par \( u_1\), nous avons fini. Autrement dit, si tout \( T\) est dans un sous-espace de dimension \( 1\) de \( \eR^2\) nous avons terminé.

            Dans la suite, nous supposons donc que \( T\) n'est pas contenu dans un sous-espace de dimension \( 1\).

        \item[\( T=\eZ u_1 + \eZ u_2\)]

            Soient \( u_1\) et \( u_2\) les deux plus petits éléments de \( T\) en norme (peut-être ex-aequo). Ces deux éléments ne sont pas colinéaires, sinon leur différence serait plus petite.

            Soit \( v\in T\). Vu que \( \{ u_1,u_2 \}\) est une base de \( \eR^2\), il existe \( \alpha,\beta\in \eR\) tels que
            \begin{equation}
                v=\alpha u_1+\beta u_2.
            \end{equation}
            Notre but est à présent de prouver que \( \alpha,\beta\in \eZ\).

            Si ce n'était pas le cas, une simple translation nous mènerait dans les circonstances du lemme \ref{LEMooEKWZooYbcGBp}. Nous aurions alors que soit \( \| v \|\) soit \(\| u_1+u_2-v \|\) serait strictement plus petit que le plus grand entre \( \| u_1 \|\) et \( \| u_2 \|\). Cela contredirait le fait que \( \| u_1 \|\) et \( \| u_2 \|\) étaient les deux plus petits.
    \end{subproof}
\end{proof}

\begin{proposition}[\cite{MonCerveau}]      \label{PROPooPQYXooIDZlHy}
    Si \( G\) est un groupe de pavage\footnote{Définition \ref{DEFooJPHKooRgCBJs}.} de \( \eR^2\) et si \( f\colon \eR^2\to \eR^2\) est une isométrie affine, alors le groupe
    \begin{equation}
        G'=f\circ G\circ f^{-1}=\{f\circ g\circ f^{-1}\tq g\in G\}
    \end{equation}
    est un groupe de pavage.
\end{proposition}

\begin{proof}
    Soit un compact \( K\) tel que \( (G,K)\) soit un pavage. Nous notons \( K'=f(K)\). Nous devons prouver deux choses :
    \begin{itemize}
        \item \( f\circ G\circ f^{-1}\) est un groupe de déplacements;
        \item \( (G',K')\) est un pavage.
    \end{itemize}
    
    Ceci fait trois éléments à prouver.

    \begin{subproof}
        \item[\( G'\) est constitué de déplacements]

            Les éléments de \( G\) sont des isométries, ainsi que \( f\) et \( f^{-1}\). Donc les éléments de \( G'\) sont des isométries. 

            Soit \( g\in G\). Vu que \( g\) est affine, il existe une décomposition \( g=\tau_v\circ g_0\) où \( g_0\) est linéaire. De même \( f=\tau_w\circ f_0\). Les règles du produit et de l'inverse de la proposition \ref{PROPooBPKKooJRAMeT}\ref{ITEMooGUFRooMuhXds}\ref{ITEMooYOMSooRUDSdm} nous indiquent que la partie linéaire de \( fgf^{-1}\) est \( f_0g_0f_0^{-1}\).

            En ce qui concerne le déterminant de \( f_0g_0f_0^{-1}\), c'est la proposition \ref{PropYQNMooZjlYlA} qui nous indique que
            \begin{equation}
                \det(f_0g_0f_0^{-1})=\det(f_0)\det(g_0)\det(f_0^{-1})=\det(f_0)\det(g_0)\det(f_0)^{-1}=\det(g_0)=1.
            \end{equation}

        \item[\( G'\cdot K'=\eR^2\)]
            Vu que \( f\) est affine, \( K'\) est encore compact\footnote{Il suffit de prouver que \( f(K)\) est fermé et borné par le théorème de Borel-Lebesgue \ref{ThoXTEooxFmdI}.} Nous avons :
            \begin{subequations}
                \begin{align}
                    (f\circ G\circ f^{-1})\big( f(K) \big)&=\bigcup_{g\in G}\bigcup_{k\in K}(f\circ g\circ f^{-1})\big( \alpha(k) \big)\\
                        &=\bigcup_{g\in G}\bigcup_{k\in K}(f\circ g)(k)\\
                        &=\bigcup_{g\in G}f\big( \bigcup_{k\in K}g(k) \big)\\
                        &=\bigcup_{g\in G}(f\circ g)(K)\\
                        &=f\big( \bigcup_{g\in G}g(K) \big)\\
                        &=f\big( G\cdot K \big)\\
                        &=\eR^2.
                \end{align}
            \end{subequations}
            Pour la dernière égalité, nous avons utilisé le fait que \( f\) est bijective et que \( G\cdot K=\eR^2\).
                
        \item[L'autre condition]

            Deux éléments de \( G'\) s'écrivent \( fg_1f^{-1}\) et \( fg_2f^{-1}\). Nous les supposons tels que
            \begin{equation}
                (fg_1f^{-1})\cdot \Int(K')\cap(f g_2 f^{-1})\cdot\Int(K')\neq \emptyset.
            \end{equation}
            Nous avons :
            \begin{subequations}
                \begin{align}
                    (fg_1f^{-1})\cdot \Int(K')\cap(f g_2 f^{-1})\cdot\Int(K')&=f\big( g_1\cdot\Int(K) \big)\cap f\big( g_2\cdot\Int(K) \big) \label{SUBEQooWPMUooWvfdAw}\\
                    &=f\big(  \underbrace{g_1\cdot\Int(K)\cap g_2\cdot\Int(K)}_{\neq \emptyset}  \big)        \label{SUBEQooTOGSooNrArAk}\\
                    &\neq \emptyset.
                \end{align}
            \end{subequations}
            Justifications :
            \begin{itemize}
                \item Égalité \eqref{SUBEQooWPMUooWvfdAw} parce qu'un peu de topologie nous enseigne que \( f^{-1}\big( \Int(K') \big)=  \Int\big( f^{-1}(K') \big)=\Int(K)  \) parce que \( f\) est affine.
                \item Égalité \eqref{SUBEQooTOGSooNrArAk} parce que, \( f\) étant bijective, \( f(A)\cap f(B)=f(A\cap B)\); 
            \end{itemize}

    \end{subproof}
\end{proof}


\begin{theorem}[\cite{NHXUsTa,BIBooWIEGooJlwsCW}]       \label{THOooUPHQooYfeHAy}
    Nous notons \( \tau_v\) la translation de vecteur \( v\), \( r_{A,\theta}\) la rotation de centre \( A\) et d'angle \( \theta\) ainsi que \( \tau_i=\tau_{e_i}\).

    Un groupe \( G\) est un groupe de pavage de \( \eR^2\) si et seulement si il existe une bijection affine \( f\colon \eR^2\to \eR^2\) telle que \( f\circ G\circ f^{-1}\) est un groupe de la liste suivante :
    \begin{enumerate}
        \item
            \( \gr\big(\tau_1,\tau_2)\)
        \item
            \( \gr\big(\tau_1,\tau_2,R_0(\pi)\big)\)
        \item
            \( \gr\big(\tau_1,\tau_{(\frac{ 1 }{2},\frac{ \sqrt{ 3 } }{ 2 })},R_0(2\pi/3)\big)\)
        \item
            \( \gr\big(\tau_1,\tau_2,R_0(\pi/2)\big)\)
        \item
            \( \gr\big(\tau_1,\tau_{(-\frac{ 1 }{2},\frac{ \sqrt{ 3 } }{2})},R_0(\pi/3\big))\).
    \end{enumerate}
\end{theorem}

\begin{proof}
    Les lemmes \ref{LEMooZOXVooTJiLTF},    \ref{LEMooTMRGooChBzZg},    \ref{LEMooJPNDooHDCLnY},    \ref{LEMooMWWEooEbZXtb}, et    \ref{LEMooGSQSooGSfkaL} montrent que les groupes listés sont des groupes de pavage. La proposition \ref{PROPooPQYXooIDZlHy} nous montre alors que si \( H=f Gf^{-1}\) est dans la liste, alors \( G\) est un groupe de pavage. Il nous reste à montrer que si \( G\) est un groupe de pavage, alors il existe une application affine \( \alpha\) telle que \( \alpha G\alpha^{-1}\) est un groupe de la liste.

    

    Soit \( (G,K)\) un pavage de \( \eR^2\).  Nous notons \( T\) l'ensemble des translations dans \( G\), plus précisément,
    \begin{equation}
        T=\{ v\in \eR^2\tq  \tau_v\in G\}.
    \end{equation}
    \begin{subproof}
        \item[Une borne pour \( T\)]
            Nous prouvons qu'il existe \( \delta>0\) tel que \( \| v \|>\delta\) pour tout \( v\neq 0\in T\). En effet, soit \( m\in \Int(K)\) ainsi que \( r\) tel que \( B(m,r)\subset \Int(K)\). Alors si \( v\in T\) est tel que \( \| v \|<r\) nous avons \( \tau_v(m)\in \Int(K)\), ce qui donnerait
            \begin{equation}
                m\in\tau_v\big( \Int(K) \big)\cap\Int(K)
            \end{equation}
            par hypothèse, cette intersection est non vide seulement si \( v=0\).

            Donc il existe \( \delta\) tel que \( \| v \|\geq \delta\) pour tout \( v\in T\).
        \item[Utilisation du lemme]

            La partie \( T\) est donc dans la position du lemme \ref{LEMooDGSJooCiBhZz} et nous avons
            \begin{equation}
                T=\eZ u_1+\eZ u_2
            \end{equation}
            pour les vecteurs \( u_1\) et \( u_2\) les plus petits en norme de \( T\).

            En réalité, il se peut que \( T\) soit plus petit que ça, parce que \( G\) peut par exemple ne contenir aucune translations. Nous avons trois possibilités :
            \begin{itemize}
                \item \( T=\{ 0 \}\),
                \item \( T=\eZ u\) pour un certain \( u\neq 0\) dans \( \eR^2\),
                \item \( T=\eZ u_1+\eZ u_2\) pour certains \( u_1,u_2\in \eR^2\setminus\{ 0 \}\).
            \end{itemize}
        \item[Translation]
            Si \( r,s\in G\), alors l'élément \( rsr^{-1}s^{-1}\) est une translation. En effet, vu que les éléments de \( G\) sont des déplacements, ce sont des applications affines et donc il existe des applications linéaires \( A_r,A_s\) et des translations \( \tau_r,\tau_s\) telles que \( r=A_r\circ \tau_r\) et \( s=A_s\circ \tau_s\). Le lemme \ref{LEMooUBGZooBIlmAN} nous donne les inverses. Nous avons
            \begin{equation}
                rsr^{-1}s^{-1}=(A_r\circ \tau_r)(A_s\circ \tau_s)(A_r^{-1}\circ \tau_{-A_rv_r})(A_s^{-1}\circ \tau_{-A_sv_s}).
            \end{equation}
            La partie linéaire de cela est
            \begin{equation}
                A_r\circ A_s\circ A_r^{-1}\circ A_s^{-1}.
            \end{equation}
            C'est donc une composée de rotations centrées en \( (0,0)\). Mais ces rotations forment un groupe abélien (proposition \ref{PROPooWMESooNJMdxf}). Donc nous pouvons écrire
            \begin{equation}
                A_r\circ A_s\circ A_r^{-1}\circ A_s^{-1}=A_r\circ A_r^{-1}\circ A_s\circ A_s^{-1}=\id.
            \end{equation}

            Tout ceci pour dire que dès que \( r,s\in G\), l'élément \( rsr^{-1} s^{-1}\) est une translation.

        \item[Les parties linéaires\cite{MonCerveau}]

            Nous savons de l'exemple \ref{EXooAGINooYmvPML} que les éléments de \( G\) s'écrivent sous la forme \( f=\tau_v\circ \alpha\) où \( v\in \eR^2\) et \( \alpha\colon \eR^2\to \eR^2\) est linéaire.

            De plus, \( f\) étant une isométrie de \( (\eR^2,d)\), l'application \( \alpha\) est une isométrie de \( (\eR^2,\| . \|)\). Vu que \( \alpha\) est une isométrie, \( \det(\alpha)=\pm1\). Mais les déplacements conservent l'orientation; donc \( \alpha\) doit conserver l'orientation, et la proposition \ref{PROPooNBAXooKNUrnk} nous dit que \( \det(\alpha)>0\). Donc
            \begin{equation}
                \det(\alpha)=1.
            \end{equation}

            Le théorème \ref{THOooWBIYooCtWoSq} dit que \( \alpha\) est la composition d'un nombre pair de réflexions. Mais comme il y en a au plus trois (théorème \ref{THOooRORQooTDWFdv}), l'application \( \alpha\) est composée de zéro ou deux réflexions.

            Donc les parties linéaires des éléments de \( G\) sont des rotations.
           
        \item[Les autres]

            Les parties linéaires des éléments de \( G\) sont des rotations. Mais les éléments de \( G\) eux-mêmes ne sont pas tellement mystérieux. Vu que ce sont des isométries de \( (\eR^2,d)\), elles sont composées de \( 0\), \( 1\), \( 2\) ou \( 3\) réflexions.

            Mais ce sont des déplacements, donc ils préservent l'orientation et le théorème \ref{THOooWBIYooCtWoSq} dit qu'ils sont des composées de zéro ou deux réflexions (nombre pair). Ce sont donc des rotations.

        \item[Hein ?]

            Les éléments linéaires de \( G\) sont des rotations. Et les autres aussi ? Les linéaires sont des rotations autour de \( (0,0)\); les autres sont des rotations autour de points autres que \( (0,0)\).

            C'est pourquoi dans la suite, nous préciserons «rotation linéaire» pour une rotation autour de \( (0,0)\) et nous dirons «rotation» pour une rotation en général. Dans le contexte affine, il faut toujours faire attention à ça : une rotation peut très bien n'être pas linéaire\footnote{Lorsque, ailleurs dans le Frido, nous disons «rotation», souvent nous pensons «rotation linéaire». Gardez cependant à l'esprit qu'une rotation peut très bien être centrée ailleurs qu'en l'origine, et soyez toujours capable de préciser le cas échéant.}.

        \item[Les rotations linéaires stabilisent \( T\)]

            Nous prouvons maintenant que les rotations linéaires de \( G\) stabilisent \( T\), c'est-à-dire que si \( v\in T\) et si \( \alpha\) est une rotation linéaire de \( G\), alors \( \alpha(v)\in T\). La transformation \( \alpha\tau_v\alpha^{-1}\) est dans \( G\). Mais pour tout \( x\in \eR^2\) nous avons
            \begin{equation}        \label{EQooLLZVooUuabir}
                (\alpha\tau_v\alpha^{-1})(x)=\alpha\big( \alpha^{-1}(x)+v \big)=x+\alpha(v)=\tau_{\alpha(v)}(x).
            \end{equation}
            Donc \( \alpha\tau\alpha^{-1}=\tau_{\alpha(v)}\) et \( \alpha(v)\in T\).
            
        \item[Exclusion de \( T=\{ 0 \}\)]

            Le fait que \( T=\{ 0 \}\) ne signifie pas que tous les éléments de \( G\) sont des rotations; il peut encore y avoir des composées de rotations et de translations \( A\circ \tau\). Cela étant dit, si \( T=\{ 0 \}\), il n'en reste pas moins que \( rsr^{-1}s^{-1}\) est une translation, c'est-à-dire est égal à \( \id\). Mais \( rsr^{-1}s^{-1}=e\) implique \( rs=sr\).

            Donc \( G\) est abélien. Les éléments de \( G\) sont donc des rotations qui commutent deux à deux. Vu qu'une rotation a son centre comme unique point fixe, le lemme \ref{LEMooWKTGooQlfuxm} nous dit que tous les éléments de \( G\) sont des rotations de même centre.

            Soit \( c\) le centre commun de tous les éléments de \( G\). Vu que \( K\) est compact dans \( \eR^2\), il existe \( r>0\) tel que \( K\subset B(c,r)\). Vu que \( G\) stabilise toutes les boules centrées en \( c\), nous avons
            \begin{equation}
                G\cdot K\subset B(c,r).
            \end{equation}
            Donc nous n'avons pas un recouvrement de \( \eR^2\). Le cas \( T=\{0  \}\) est exclu.

        \item[Exclusion de \( T=\eZ u\)]

            Nous supposons à présent que \( T=\eZ u\) pour un certain \( u\in \eR^2\). 

            \begin{subproof}
                \item[\( r_0=\pm\id\)]
                    Nous savons que tous les éléments de \( G\) sont des rotations; soit un élément \( r\) de \( G\). La proposition
                    \ref{PROPooTFNSooFjiWHG}\ref{ITEMooSIHZooBEJhdu} nous indique qu'il existe un point \( a\in \eR^2\) ainsi qu'une rotation linéaire \( r_0\) telle que \( r=\tau_a^{-1}r_0\tau_a\). Nous allons prouver que \( r_0\) est \( \pm\id\). D'abord,
                    \begin{equation}
                        r\tau_u r^{-1}=\tau_a^{-1}r_0\tau_a\tau_u\tau_a^{-1}r_0^{-1}\tau_a=\tau_a^{-1}r_0\tau_ur_0^{-1}\tau_a.
                    \end{equation}
                    Ensuite, nous appliquons cela à \( x\in \eR^2\) :
                    \begin{subequations}
                        \begin{align}
                            (\tau_a^{-1}r_0\tau_ur_0^{-1}\tau_a)(x)&=(\tau_a^{-1}r_0\tau_u)\big( r_0^{-1}(x+a) \big)\\
                            &=(\tau_a^{-1}r_0)\big( r_0^{-1}(x+a)+u \big)\\
                            &=\tau_a^{-1}\big( x+a+r_0(u) \big)\\
                            &=x+r_0(u).
                        \end{align}
                    \end{subequations}
                    Donc \( r\circ\tau_u\circ r^{-1}=\tau_{r_0(u)}\), ce qui prouve que \( r_0(u)\in T\). Vu que \( \| r_0(u) \|=\| u \|\) nous avons forcément \( r_0(u)=\pm u\).

                    Si \( r_0(u)=u\), alors \( r_0=\id\) (parce que \( r_0\) est une rotation fixant plus que un seul point). Dans ce cas, \( r=\id\).

                    Si au contraire \( r_0(u)=-u\), alors \( r_0=-\id\).

                \item[Forme générale]

                    Donc si \( r\) est un élément non trivial de \( G\) nous avons \( r=\tau_a^{-1}\circ(-\id)\circ\tau_a\), et alors
                    \begin{equation}
                        \big( \tau_a^{-1}\circ(-\id)\circ\tau_a \big)(x)=\tau_a^{-1}  (-\id)(x+a)=\tau_a^{-1}(-x-a)=-x-2a.
                    \end{equation}
                    Donc pour tout \( r\in G\), il existe \( a\in \eR^2\) tel que
                    \begin{equation}        \label{EQooQGNVooKyCCYW}
                        r(x)=-x-2a.
                    \end{equation}
                    Pour information, le centre de cette rotation est \( -a\) (c'est le seul point fixe).

                \item[Les centres sont alignés]

                    Soit \( r\) une rotation de centre \( -a\) et \( s\) de centre \( -b\). Alors
                    \begin{equation}
                        (rs)(x)=r(-x-2b)=x+2b-2a=x+2(b-a).
                    \end{equation}
                    Donc \( rs=\tau_{2(b-a)}\).

                    Cela prouve que \(2(b-a)\in \eZ u\).

                \item[Une bande]

                    Soit la droite \( D=\eR u\). Nous considérons la bande
                    \begin{equation}
                        B_r=\{ x\in \eR^2\tq d(x,D)<r \}.
                    \end{equation}

                \item[Une inclusion]
                    Nous prouvons à présent que pour tout \( r\), nous avons
                    \begin{equation}
                        G\cdot B_r \subset B_r.
                    \end{equation}
                    Nous savons qu'un élément général de \( G\) est une rotation centrée en un point de \( D\), et que l'action d'une telle rotation est donnée par \eqref{EQooQGNVooKyCCYW}. Nous avons
                    \begin{subequations}
                        \begin{align}
                            d\big( r(x),D \big)&=d(x+2a,D)      \label{EQooNCQGooGGAhCR}\\
                            &=d\big( \tau_{2a}^{-1}(x+2a),\tau_{2a}^{-1}(D) \big)   \label{EQooNFJNooSAYtBD}\\
                            &=d(x,D)        \label{SUBEQooMGTMooAlRwWT}.
                        \end{align}
                    \end{subequations}
                    Justifications :
                    \begin{itemize}
                        \item Pour \eqref{EQooNCQGooGGAhCR}, nous avons \( D=-D\) et \(d(x,y)=d(-x,-y) \).
                        \item Pour \eqref{EQooNFJNooSAYtBD}, invariance par translation de la distance dans \( \eR^2\).
                        \item Pour \eqref{SUBEQooMGTMooAlRwWT}, les éléments de \( D\) sont les multiples de \( a\); donc cette droite est invariante par cette translation.
                    \end{itemize}
                    Bref, si \(  d(x,D)=d\big( r(x),D \big)   \). Donc pour tout \( r>0\)\footnote{Remarquez la notation malheureuse pour \( r\) qui est maintenant une distance alors que trois mots plus tôt, c'était un élément de \( G\).} et pour tout \( g\in G\), si \( x\in B_r\), alors \( g(x)\in B_r\).

                \item[Exclusion]

                    Vu que \( K\) est compact et que la fonction \( x\mapsto d(x,D)\) est continue, il existe \( r>0\) tel que \( K\subset B_r\). Avec ça, \( G\cdot K\subset G\cdot B_r\subset B_r\). Donc \( G\cdot K\) ne recouvre par tout \( \eR^2\) et \( G\) n'est pas un groupe de pavage.

            \end{subproof}
    \end{subproof}

    Et nous voila avec seulement \( T=\eZ u_1+\eZ u_2\) en lice.

    \begin{subproof}
        \item[Pause : quelques parties de \( G\) à ne pas confondre]

            Il convient de ne pas se perdre entre différentes parties de \( G\). Je vous laisse méditer quelque temps sur la liste suivante :
            \begin{enumerate}
                \item \( G\) est le groupe que nous cherchons à déterminer;
                \item \( T\) est le groupe des translations de \( G\);
                \item le groupe des rotations linéaires dans \( G\);
                \item l'ensemble des \( \tau_A^{-1}\circ r\circ\tau_A\) où \( r\) est une rotation de \( G\) centrée en \( A\).     \label{ITEMooFWWMooNzLUGy}
                \item \( L\) est l'ensemble des parties linéaires des éléments de \( G\).      \label{ITEMooIEJZooNaSKpc}
            \end{enumerate}
            En particulier les deux derniers ne sont pas les mêmes. 

            Dans le cas \ref{ITEMooFWWMooNzLUGy}, il s'agit de dire que \( r\) est une rotation centrée en \( A\in \eR^2\) et écrire \( r=\tau_A\circ r_0\circ \tau_A^{-1}\) (proposition \ref{PROPooTFNSooFjiWHG}\ref{ITEMooSIHZooBEJhdu}) et considérer \( r_0\). Dans ce cas, \( r_0\) est une rotation, mais elle n'est pas ce que nous appelons la «partie linéaire» de \( r\). Il n'y a pas de garanties que cela forme un groupe.
            
            Si \( r\) est une rotation dans \( G\), dans le cas \ref{ITEMooIEJZooNaSKpc} il s'agit de décomposer \( r=\tau_v\circ\alpha\) (lemme \ref{LEMooYJCDooOGAHkF}) et considérer \( \alpha\). Dans ce cas, \( \alpha\) est linéaire, mais il n'y a pas de garanties que \( \alpha\) soit une rotation. 
            
        \item[\( L\) est un groupe]
            En trois conditions.
            \begin{itemize}
                \item 
                    Si \( \alpha,\beta\in L\), il existe \( v,w\in \eR^2\) tels que \( \tau_v\circ \alpha\in G\) et \( \tau_w\circ \beta\in G\). La loi de produit \ref{PROPooBPKKooJRAMeT}\ref{ITEMooGUFRooMuhXds} dit que \( \tau_v\circ \alpha\circ\tau_w\circ \beta=\tau_{\alpha(w)+v}\circ \alpha\beta\). Donc \( \alpha\beta\) est la partie linéaire d'un élément de \( G\). 
                \item
                    De la même façon, en utilisant l'inverse \ref{PROPooBPKKooJRAMeT}\ref{ITEMooYOMSooRUDSdm}, \( (\tau_v\circ\alpha)^{-1}= \tau_{-\alpha^{-1}(v)}\circ \alpha^{-1}   \). Donc \( \alpha^{-1}\) est la partie linéaire d'un élément de \( G\).
                \item
                    Et enfin \( \id=\tau_0\circ \id\). Donc l'identité est dans \( L\).
            \end{itemize}
            Ok pour que \( L\) soit un groupe.

        \item[Précision]
            L'ensemble \( L\) est un groupe certes. Mais rien ne dit que \( L\) soit un sous-groupe de \( G\).
        \item[\( L\) préserve le réseau]
            Soit \( \alpha\in L\). Il existe \( v\in \eR^2\) tel que \( g=\tau_v\circ \alpha\in G\). Soit \( u\in T\). Nous allons montrer que \( \alpha(u)\in T\). Vu que \( g\) et \( \tau_u\) sont dans \( G\), l'élément \( g\tau_ug^{-1}\) est également dans \( G\). Nous l'appliquons à \( x\in \eR^2\) :
            \begin{subequations}
                \begin{align}
                    (\tau_v\alpha\tau_u\alpha^{-1}\tau_v^{-1})(x)&=(\tau_v\alpha\tau_u\alpha^{-1})\alpha^{-1}(x-v)\\
                    &=(\tau_v\alpha)\big( \alpha^{-1}(x-v)+u \big)\\
                    &=\tau_v\big( x-v+\alpha(u) \big)\\
                    &=x+\alpha(u)\\
                    &=\tau_{\alpha(u)}(x).
                \end{align}
            \end{subequations}
            Donc \( g\tau_ug^{-1}=\tau_{\alpha(u)}\in G\).
        \item[Question de trace]
            Soit \( \alpha\in L\); dans la base \( \{ u_1,u_2 \}\) la matrice de \( \alpha\) est
            \begin{equation}
                \begin{pmatrix}
                    a&b\\
                    c&d
                \end{pmatrix}
            \end{equation}
            avec \( a,b,c,d\in \eZ\). La trace de cette matrice est \( a+d\in \eZ\). Dans la base canonique de \( \eR^2\) par contre la proposition \ref{PROPooOTIVooZpvLnb} nous dit qu'il existe \( \theta\in \mathopen[ 0 , 2\pi \mathclose[\) tel que la matrice de \( \alpha\) soit
                \begin{equation}
                    \begin{pmatrix}
                        \cos(\theta)    &   -\sin(\theta)    \\ 
                        \sin(\theta)    &   \cos(\theta)    
                    \end{pmatrix}.
                \end{equation}
                La trace est \( 2\cos(\theta)\). La trace est invariante par changement de base\footnote{Proposition \ref{PROPooRMYQooWkEpJJ}.}, donc \( 2\cos(\theta)=a+d\in \eZ\). Les possibilités pour \( \cos(\theta)\) sont donc \( -1\), \( -1/2\), \( 0\), \( 1/2\) et \( 1\).

            \item[Les angles possibles]

                Nous savons que \( \cos(\frac{ \pi }{2})=0\) et \( \cos(\pi/3)=1/2\) (proposition \ref{PROPooMWMDooJYIlis}\ref{ITEMooWFNUooYAybDB} et lemme \ref{LEMooRMHAooDEAPMw}). La proposition \ref{PROPooTUUUooVrAGQo} nous dit alors que, dans notre cas, les valeurs possibles pour \( \theta\) dans \( \mathopen[ 0 , 2\pi \mathclose[\) sont
                    \begin{equation}        \label{EQooLMPIooPQoHUI}
                    \{ 0,\frac{ \pi }{ 3 },\frac{ \pi }{ 2 }, \frac{ 2\pi }{ 3 }, \pi,\frac{ 4\pi }{ 3 }, \frac{ 3\pi }{ 2 },\frac{ 5\pi }{ 3 } \}.
                \end{equation}
                Donc les rotations possibles dans \( L\) sont les rotations de ces angles.

                Nous devons trouver quels sont les groupes qui peuvent être formés seulement avec ces éléments.

            \item[Quelques combinaisons impossibles]

                Vu que \( L\) est un groupe, il y a des combinaisons impossibles. Par exemple si \( R_0(\pi/3)\) et \( R_0(\pi/2)\) sont dans \( L\), alors la composée\footnote{Proposition \ref{PROPooISUCooRYJcwo} pour l'addition des angles.} \( R_0(\pi/2)R_0(\pi/3)=R_0(5\pi/6)\) est également dans \( L\). Mais comme \( 5\pi/6\) n'est pas dans la liste \eqref{EQooLMPIooPQoHUI}, \( R_0(5\pi/6)\) n'est pas dans \( L\).
            
                En raisonnant de la sorte, nous voyons que si \( R_0(\pi/2)\in L\), alors \( L=\gr\big( R_0(\pi/2) \big)\).

            \item[La liste]

                Plus généralement, les possibilités pour \( L\) sont
                \begin{itemize}
                    \item \( \{ \id \}\)
                    \item \( \gr\big( R_0(\pi/2) \big)\)
                    \item \( \gr\big( R_0(\pi/3) \big)\)
                    \item \( \gr\big( R_0(\pi) \big)\)
                    \item \( \gr\big( R_0(2\pi/3) \big)\)
                \end{itemize}
                Une justification plus courte pour cette liste est d'invoquer le théorème \ref{THOooKDMUooUxQqbB}\ref{ITEMooGELWooFFAqkc} qui dit que \( L\) étant un sous-groupe fini des isométries de \( (\eR^2,d)\), il est cyclique et donc monogène. Notons pour cela que \( \gr\big( R_0(4\pi/3) \big)=\gr\big( R_0(2\pi/3) \big)\) parce que \( R_0(4\pi/3)=R_0(2\pi/3)^{-1}\). De la même manière \( R_0(3\pi/2)=R_0(\pi/2)^{-1}\) et \( R_0(5\pi/3)=R_0(\pi/3)^{-1}\).

            \item[Le cas un peu générique]

                Nous supposons que \( L=\gr\big( R_0(\theta) \big)\) pour un certain \( \theta\). Nous allons voir qu'à part dans les cas \( \theta=0\) et \( \theta=\pi\), il est possible de trouver une application affine \( \alpha\) telle que le groupe \( \alpha G\alpha^{-1}\) est alors dans la liste.

                Nous considérons un élément de \( G\) de la forme \( \tau_{v_0}\circ R_0(\theta)\). Pour être bien clair, il n'est absolument pas garanti que \( v_0\) soit dans \( T\). Nous allons chercher un élément \( w_0\in \eR^2\) tel que le groupe de pavage\footnote{Proposition \ref{PROPooPQYXooIDZlHy}.} \( G'=\tau_{w_0}G\tau_{w_0}^{-1}\) contienne \( R_0(\theta)\). Le groupe \( G'\) contient l'élément \( g=\tau_{w_0}\tau_{v_0}R_0(\theta)\tau_{w_0}^{-1}\); nous l'appliquons à \( x\in \eR^2\) :
                \begin{equation}
                    g(x)=R_0(\theta)x-R_0(\theta)w_0+w_0+v_0.
                \end{equation}
                Nous avons \( g(x)=R_0(\theta)x\) lorsque
                \begin{equation}
                    w_0=\big( R_0(\theta)-\mtu \big)^{-1}v_0.
                \end{equation}
                Voila pourquoi le cas \( \theta=0\) sera traité à part : dans le cas \( \theta=0\), l'opérateur \( R_0(\theta)-\mtu\) n'est pas inversible. Dans les autres cas, nous avons un groupe \( G'=\tau_{w_0}G\tau_{w_0}^{-1}\) qui contient \( R_0(\theta)\).

                Vu qu'un élément général de \( G\) est de la forme \( \tau_v\circ R_0(\theta)^k\), un élément général de \( G' \) est de la forme
                \begin{equation}
                    \tau_{w_0}\circ \tau_v\circ R_0(\theta)^k\circ \tau_{w_0}=\tau_{w_0+v-R_0(\theta)^kw}\circ R_0(\theta)^k.
                \end{equation}
                Donc tous les éléments de \( G'\) sont encore de la forme
                \begin{equation}
                    \tau_v\circ R_0(\theta)^k.
                \end{equation}
                Mais dans \( G'\) nous avons une information capitale : \( R_0(\theta)^k\) lui-même est dans \( G\). Donc si \( \tau_v\circ R_0(\theta)^k\in G'\), alors \( \tau_v\in G'\).

                Vu que \( G'\) est encore un groupe de pavage, tout ce qui a été dit précédemment tient et le groupe des translations dans \( G'\) est un réseau \( T'=\eZ u_1+\eZ u_2\) où \( u_1\) et \( u_2\) peuvent être choisis arbitrairement parmi les deux plus petits vecteurs non colinéaires de \( T\).

                Tous les éléments de \( G'\) sont de la forme \( \tau_v\circ R_0(\theta)^k\) avec \( v\in T'\). Donc
                \begin{equation}        \label{EQooUUDQooQpRcIi}
                    G'=\gr\big(\tau_{u_1}, \tau_{u_2}, R_0(\theta)\big).
                \end{equation}
                Nous savons que \( R_0(\theta)\) fixe \( T'\). Or l'élément \( R_0(\theta)u_1\) a la même norme que \( u_1\); donc nous pouvons choisir \( u_2=R_0(\theta)\) pour peu que \( R_0(\theta)u_1\) soit non colinéaire à \( u_1\). Et c'est ici que nous laissons le cas \( \theta=\pi\) de côté.

                Nous écrivons donc sans vergogne que
                \begin{equation}
                    G'=\gr\big(\tau_{u_1}, \tau_{R_0(\theta)u_1}, R_0(\theta)\big).
                \end{equation}
                
                Nous allons maintenant nous occuper de \( u_1\). Pour cela nous considérons une rotation suivie d'une homothétie \( \alpha\) telle que \( \alpha(u_1)=e_1\). Une telle opération \( \alpha\) d'une part commute avec \( R_0(\theta)\) et d'autre part fait \( \alpha\circ \tau_v\circ\alpha^{-1}=\tau_{\alpha(v)}\). Donc le groupe \( G''=\alpha G'\alpha^{-1}\) est, par le lemme \ref{LEMooCFTVooKvmyKN},
                \begin{equation}
                    G''=\gr\big( \tau_{e_1}, \tau_{\alpha R_0(\theta)u_1}, R_0(\theta) \big).
                \end{equation}
                Cela s'écrit aussi bien
                \begin{equation}
                    G''=\gr\big( \tau_{e_1}, \tau_{R_0(\theta)e_1}, R_0(\theta) \big).
                \end{equation}
                Et voila, ce groupe \( G''\) est un de la liste.

            \item[Le cas \( L=\{ \id \}\)]

                Dans ce cas, tous les éléments de \( G\) sont de la forme \( \tau_v\) avec \( v\in T\). Nous considérons l'application linéaire \( \alpha\colon \eR^2\to \eR^2\) telle que \( \alpha(u_1)=e_1\) et \( \alpha(u_2)=e_2\). Nous avons
                \begin{equation}
                    (\alpha\circ \tau_{u_i}\circ \alpha^{-1})(x)=(\alpha\tau_{u_i})\alpha^{-1}(x)=\alpha\big( \alpha^{-1}(x)+u_i \big)=x+\alpha(u_i)=\tau_i(x).
                \end{equation}
                Donc \( \alpha\circ \tau_{u_i}\circ \alpha^{-1}=\tau_i\). De la même façon, si \( a\in \eZ\) nous avons \( \alpha\tau_{au_i}\alpha^{-1}=\tau_{ae_i}\). Et avec tout ça, si \( v\in T\), alors \( v=au_1+bu_2\) et nous avons
                \begin{equation}
                    \alpha\tau_v\alpha^{-1}=\alpha\tau_{au_1}\tau_{au_2}\alpha^{-1}=\alpha\tau_{au_1}\alpha^{-1}\alpha\tau_{bu_2}\alpha^{-1}=\tau_{ae1}\tau_{be_2}.
                \end{equation}
                Nous avons donc
                \begin{equation}
                    \alpha G\alpha^{-1}=\gr(\tau_1,\tau_2).
                \end{equation}
                Cela est un des groupes de la liste.
                
                Notez qu'à la place de ces calculs, nous pouvions aussi invoquer la proposition \ref{LEMooCFTVooKvmyKN}.
            \item[Le cas \( L=\{ -\id \}\)]
                Dans le cas \( \theta=\pi\), nous pouvons aller jusqu'à \eqref{EQooUUDQooQpRcIi} et écrire
                \begin{equation}
                    G'=\gr\big( \tau_{u_1}, \tau_{u_2}, R_0(\pi) \big).
                \end{equation}
                Vu que \( R_0(\pi)=-\id\) commute avec toutes les applications linéaires et que \( u_1\) et \( u_2\) ne sont pas colinéaires, nous pouvons considérer une application linéaire \( \alpha\colon \eR^2\to \eR^2\) telle que \( \alpha(u_1)=e_1\), \( \alpha(u_2)=e_2\). Nous avons alors
                \begin{equation}
                    G''=\alpha G'\alpha^{-1}=\gr\big( \tau_{e_1}, \tau_{e_2},-\id \big),
                \end{equation}
                qui est encore dans la liste.
    \end{subproof}
\end{proof}

%+++++++++++++++++++++++++++++++++++++++++++++++++++++++++++++++++++++++++++++++++++++++++++++++++++++++++++++++++++++++++++
\section{Un peu de structure de \texorpdfstring{\( \gO(n)\)}{O(n)}}
%+++++++++++++++++++++++++++++++++++++++++++++++++++++++++++++++++++++++++++++++++++++++++++++++++++++++++++++++++++++++++++

%---------------------------------------------------------------------------------------------------------------------------
\subsection{Valeurs propres dans \( \gO(n)\)}
%---------------------------------------------------------------------------------------------------------------------------

\begin{proposition}[\cite{ooPWOHooHwgPzO}]      \label{PROPooVEJGooWnqtMm}
    Soit une matrice \( A\in O(n)\). Si \( \lambda\in \eC\) est une valeur propre de \( A\), alors \( \bar\lambda\) est également une valeur propre de \( A\), et de plus \( | \lambda |=1\).
\end{proposition}

\begin{proof}
    Dire que \( \lambda\in \eC\) est une valeur propre de \( A\) signifie qu'il existe \( x\in \eC^n\) (non nul) tel que \( Ax=\lambda x\). Vu que les éléments de la matrice \( A\) sont réels,
    \begin{equation}
        A\bar x=\bar A\bar x=\overline{ Ax }=\overline{ \lambda x }=\bar \lambda\bar x.
    \end{equation}
    Donc \( \bar \lambda\) est une valeur propre de \( A\) pour le vecteur propre \( \bar x\).

    Soit \( \lambda\)  une valeur propre de \( A\) de vecteur propre \( x\). Alors nous avons d'une part
    \begin{equation}
        \langle \overline{ Ax }, Ax\rangle =\langle A^tA\bar x, x\rangle =\langle x, \bar x\rangle ,
    \end{equation}
    et d'autre part
    \begin{equation}
        \langle \overline{ Ax }, Ax\rangle =\langle \bar \lambda \bar x, \lambda x\rangle =| \lambda |^2\langle \bar x, x\rangle .
    \end{equation}
    Vu que \( x\neq 0\) nous avons aussi \( \langle \bar x, x\rangle \neq 0\). Par conséquent \( | \lambda |^2=1\) et \( | \lambda |=1\).
\end{proof}

\begin{lemma}[\cite{ooPWOHooHwgPzO}]        \label{LEMooNEDQooNRmASH}
    Soit un espace vectoriel euclidien \( E\) de dimension finie et une isométrie \( f\) de \( E\). Soit \( F\) un sous-espace de \( E\) stable par \( f\). Alors \( F^{\perp}\) est stable par \( f\).
\end{lemma}

\begin{proof}
    La restriction \( f_F\colon F\to F\) est encore une isométrie; elle est donc inversible : pour tout $y\in F$, il existe \( x\in F\) tel que \( y=f(x)\). Soit \( a\in F^{\perp}\); nous montrons que \( f(a)\in F^{\perp}\). Soit donc \( y\in F\) et calculons :
    \begin{equation}
        \langle y, f(a)\rangle =\langle f(x), f(a)\rangle =\langle x,a, \rangle =0
    \end{equation}
    parce que \( x\in F^{\perp}\).
\end{proof}

\begin{proposition}[\cite{ooPWOHooHwgPzO}]      \label{PROPooOMORooWzsrDB}
    Soit une isométrie \( f\colon \eR^3\to \eR^3\).
    \begin{enumerate}
        \item
            L'application linéaire \( f\) possède au moins une valeur propre réelle qui vaut \( \pm 1\).
        \item
            Il existe une base orthonormée de \( \eR^3\) dans laquelle la matrice de \( f\) est de la forme
            \begin{equation}
                \begin{pmatrix}
                    \lambda    &   0    &   0    \\
                    0    &   \cos(\theta)    &   -\epsilon\cos(\theta)    \\
                    0    &   \sin(\theta)    &   \epsilon\cos(\theta)
                \end{pmatrix}
            \end{equation}
            avec \( \epsilon,\lambda=\pm 1\) et \( \theta\in \mathopen[ 0 , 2\pi \mathclose[\).
    \end{enumerate}
\end{proposition}

\begin{proof}
    Le polynôme caractéristique de \( f\), donné par \( \det(f-\lambda\id)\), est à coefficients réels et de degré \( 3\). Il possède dont au moins une solution réelle par le corolaire~\ref{CORooKKNWooWEQukb}. Soit donc une valeur propre réelle \( \lambda\) de \( \chi_f\); par le lemme~\ref{PROPooVEJGooWnqtMm} nous avons \( \lambda=\pm 1\). Soit \( u_1\) le vecteur propre correspondant. Nous notons \( F\) l'espace engendré par \( u_1\).

    Nous avons \( f(F)=F\) et donc \( f(F^{\perp})=F^{\perp}\) par le lemme~\ref{LEMooNEDQooNRmASH}. Soit une base orthonormée \( \{ u_2,u_3 \}\) de \( F^{\perp}\) et la matrice \( B\) de la restriction \( f_{p}\) à \( F^{\perp}\). Vu que l'application \( f_p\) est une isométrie de \( F^{\perp}\), la matrice \( B\) est, par le lemme~\ref{LEMooAJMAooXPSKtS}, de la forme
    \begin{equation}
        B=\begin{pmatrix}
            \cos(\theta)    &   -\epsilon\sin(\theta)    \\
            \sin(\theta)    &   \epsilon\cos(\theta)
        \end{pmatrix}
    \end{equation}
    pour un certain \( \theta\in\mathopen[ 0 , 2\pi \mathclose[\) et \( \epsilon=\pm 1\).

    Dans la base \( \{u_1,u_2,u_3\}\) de \( \eR^3\), la matrice de \( f\) est alors
    \begin{equation}
        \begin{pmatrix}
            \lambda    &   0    \\
            0    &   B
        \end{pmatrix},
    \end{equation}
    comme annoncé.
\end{proof}

Pour classifier les isométries de \( \eR^3\), nous pouvons nous baser sur les possibilités de la matrice donne dans le lemme~\ref{PROPooOMORooWzsrDB}. Il y a essentiellement quatre possibilités suivant les valeurs de \( \lambda=\pm 1\) et \( \epsilon=\pm 1\).

\begin{subproof}
    \item[Si \( \epsilon=\lambda=1\)] Alors la matrice est
    \begin{equation}
        A=\begin{pmatrix}
            1    &   0    &   0    \\
            0    &   \cos(\theta)    &   -\sin(\theta)    \\
            0    &   \sin(\theta)    &   \cos(\theta)
        \end{pmatrix}
    \end{equation}
    et l'isométrie correspondante est la rotation d'angle \( -\theta\) autour de la droite de \( u_1\).

    \item[Si \( \epsilon=\lambda=-1\)]
    Alors la matrice est
    \begin{equation}
        A=\begin{pmatrix}
            -1    &  0     &   0    \\
            0    &   \cos(\theta)    &   \sin(\theta)    \\
            0    &   \sin(\theta)    &   -\cos(\theta)
        \end{pmatrix}
    \end{equation}
    Cette application est plus subtile, parce que même dans le plan \( \Span(u_2,u_3)\), ce n'est pas une rotation. Nous allons montrer qu'il s'agit d'une réflexion autour de la droite d'angle \( \theta/2\) dans le plan \( \Span(u_2,u_3)\). Nous nommons \( D\) cette droite. Dans la base \( \{ u_1,u_2,u_3 \}\), les points de cette droite sont de la forme\footnote{Les plus acharnés remarqueront que \( \{ u_1,u_2,u_3 \}\) est un ensemble, qui est une base. Mais un ensemble n'est pas ordonné, alors que pour écrire l'équation de droite qui suit, nous supposons un ordre. Je laisse au tel lecteur le soin de trouver une bonne notation.}
    \begin{equation}
        \big( 0,\cos(\theta/2),\sin(\theta/2) \big).
    \end{equation}

    L'image de \( u_1\) par cette réflexion est \(-u_1\), c'est clair.

    Faisons en détail l'image de \( u_3\). Nous devons démontrer que la droite \( D\) coupe le segment \( \mathopen[ u_3 , A(u_3) \mathclose]\) perpendiculairement en son milieu.

    Dans le plan \( \Span(u_2,u_3)\) nous avons \( u_3=\begin{pmatrix}
        0    \\
        1
    \end{pmatrix}\) et \( A(u_3)=\begin{pmatrix}
        \sin(\theta)    \\
        \cos(\theta)
    \end{pmatrix}\). Le milieu du segment \( \mathopen[ u_3 , A(u_3) \mathclose]\) est le point
    \begin{equation}
        M=\left( \frac{ \sin(\theta) }{2},\frac{ 1-\cos(\theta) }{2} \right).
    \end{equation}
    Les formules de duplication d'angle du corolaire~\ref{CORooQZDQooWjMXTF} nous permettent d'écrire \( \sin(\theta)\) et \( \cos(\theta)\) en fonction de \( \sin(\theta/2)\) et \( \cos(\theta/2)\), et donc d'exprimer le point \( M\) de la façon suivante :
    \begin{subequations}
        \begin{align}
            M&=\left( \cos(\theta/2)\sin(\theta/2),\frac{ 1-\big( \cos^2(\theta/2)-\sin^2(\theta/2) \big) }{2} \right)\\
            &=\big( \cos(\theta/2)\sin(\theta/2),\sin^2(\theta/2) \big)\\
            &=\sin(\theta/2)\big( \cos(\theta/2),\sin(\theta/2) \big).
        \end{align}
    \end{subequations}
    Ce point fait donc partie de la droite \( D\). La droite \( D\) coupe le segment \( \mathopen[ u_3 , A(u_3) \mathclose]\) en son milieu.

    En ce qui concerne l'orthogonalité, nous calculons le produit scalaire
    \begin{equation}
            \big( A(u_3)-u_3 \big)\cdot\begin{pmatrix}
                \cos(\theta/2)    \\
                \sin(\theta/2)
            \end{pmatrix}
            =\sin(\theta)\cos(\theta/2)-\big( 1+\cos(\theta) \big)\sin(\theta/2)=0
    \end{equation}
    où nous avons encore utilisé les duplications d'angles et le fait que \( 1=\cos^2(\theta/2)+\sin^2(\theta/2)\) (lemme~\ref{LEMooAEFPooGSgOkF}).

    \item[Si \( \epsilon=-1\) et \( \lambda=1\)] Alors la matrice est
        \begin{equation}
            A=\begin{pmatrix}
                1    &   0    &   0    \\
                0    &   \cos(\theta)    &   \sin(\theta)    \\
                0    &   \sin(\theta)    &   -\cos(\theta)
            \end{pmatrix}.
        \end{equation}
        C'est la symétrie orthogonale par le plan engendré par \( u_1\) et \( v=\cos(\theta/2)u_2+\sin(\theta/2)u_3\).

        Le vecteur \( u_1\) est bien évidemment préservé par \( A\). En ce qui concerne le vecteur \( v\),
        \begin{equation}
            A(v)=\cos(\theta/2)\begin{pmatrix}
                0    \\
                \cos(\theta)    \\
                \sin(\theta)
            \end{pmatrix}+\sin(\theta/2)\begin{pmatrix}
                0    \\
                -\sin(\theta)    \\
                \cos(\theta)
            \end{pmatrix}=
            \begin{pmatrix}
                0    \\
                \cos(\theta/2)    \\
                \sin(\theta/2)
            \end{pmatrix}=v.
        \end{equation}
        Nous avons sauté quelques étapes de calcul mettant en scène les formules de duplication d'angle : exprimer \( \cos(\theta)=\cos^2(\theta/2)-\sin^2(\theta/2)\) et \( \sin(\theta)=2\cos(\theta/2)\sin(\theta/2)\).

        Pour achever, nous devons trouver un vecteur \( w\) perpendiculaire au plan, et montrer qu'il est envoyé par \( A\) sur \( -w\). Un vecteur \( w=xu_1+yu_2+zu_3\) est perpendiculaire au plan si les deux égalités suivantes sont satisfaites :
        \begin{subequations}
            \begin{align}
                \big( \cos(\theta/2)u_2+\sin(\theta/2)u_3 \big)\cdot (xu_1+yu_2+zu_3)=0\\
                u_1\cdot(xu_1+yu_2+zu_3)=0.
            \end{align}
        \end{subequations}
        Nous avons immédiatement \( x=0\) et ensuite la relation
        \begin{equation}        \label{EQooXQMDooTvwrWk}
            y\cos(\theta/2)+z\sin(\theta/2)=0.
        \end{equation}
        En ne regardant que les deux dernières composantes pour alléger l'écriture,
        \begin{equation}
            A(w)=y\begin{pmatrix}
                \cos(\theta)    \\
                \sin(\theta)
            \end{pmatrix}+z\begin{pmatrix}
                \sin(\theta)    \\
                -\cos(\theta)
            \end{pmatrix}.
        \end{equation}
        Le but est de montrer que cela est égal à \( -y\cos(\theta/2)-z\sin(\theta/2)\).

        Notons \( c=\cos(\theta/2)\) et \( s=\sin(\theta/2)\). Alors \( A(w)_2=y(c^2-s^2)+2zcs\). Évacuons tout de suite les deux cas limite : si \( c=0\) alors \( A(w)_2=-y\) (parce que \( s=\pm1\)) et c'est bon. Si \( s=0\), alors \( A(w)_2=y\), mais la relation \eqref{EQooXQMDooTvwrWk} donne \( y=0\), donc c'est bon aussi. Dans le cas générique, \( z=-yc/2\) et
        \begin{equation}
            A(w)_2=y(c^2-s^2)-2cs\frac{ yc }{ s }=-y(c^2+s^2)=-y.
        \end{equation}

        En ce qui concerne \( A(w)_3\), c'est très similaire :
        \begin{equation}
            A(w)_3=2ysc-z(c^2-s^2).
        \end{equation}
        Avec \( z=0\) c'est \( -z\), donc c'est bon. Avec \( c=0\) c'est \( z\) mais \( z=0\). Et pour le cas générique, la substitution \( y=-zs/c\) donne le résultat.


    \item[Si \( \epsilon=1\) et \( \lambda=-1\)] Alors la matrice est
        \begin{equation}
            A=\begin{pmatrix}
                -1    &   0    &   0    \\
                0    &   \cos(\theta)    &   -\sin(\theta)    \\
                0    &   \sin(\theta)    &   \cos(\theta)
            \end{pmatrix}.
        \end{equation}
        Cela est la composition entre la symétrie de plan \( \Span(u_2,u_3)\) et la rotation d'angle \( \theta\) dans ce plan.
\end{subproof}

%---------------------------------------------------------------------------------------------------------------------------
\subsection{Sous-groupes finis de \( \SO(3)\)}
%---------------------------------------------------------------------------------------------------------------------------

\begin{lemma}[\cite{MonCerveau}]       \label{LEMooWIMMooXOCfSt}
    Points fixes pour \( \SO(3)\).
    \begin{enumerate}
        \item
            Tout élément de \( \SO(3)\) possède une droite de points fixes.
        \item
            Tout élément non trivial de \( \SO(3)\) possède une seule droite de points fixes.
    \end{enumerate}
\end{lemma}

\begin{proof}
    Le polynôme caractéristique d'un élément de \( \SO(3)\) est de degré trois et possède dont (en comptant les multiplicités) trois racines dont une réelle par le corolaire~\ref{CORooKKNWooWEQukb}. Vu que nous sommes en dimension impaire, le coefficient du terme de degré \( 3\) est \( -1\) et le polynôme caractéristique de \( g\in\SO(3)\) s'écrit
    \begin{equation}
        \chi_g(X)=-(X-\lambda_1)(X-\bar\lambda_1)(X-s)
    \end{equation}
    avec \( s=\pm1 \) que nous allons tout de suite fixer. Nous savons que \( \det(g)=\chi_g(0)\) mais aussi que \( \det(g)=1\). Donc
    \begin{equation}
        1=\det(g)=\chi_g(0)=\lambda_1\bar\lambda_1 s=s.
    \end{equation}
    Tout cela pour dire que tout élément de \( \SO(3)\) possède une valeur propre égale à \( 1\), et donc une droite de points fixes.

    Pour continuer, supposons que \( g\) possède deux droites distinctes de points fixes. En particulier \( g\) fixe un plan. Une base orthonormée de \( \eR^3\) peut être choisie en prenant deux vecteurs \( e_1\), \( e_2\) dans ce plan et un vecteur \( e_3\) perpendiculaire au plan.

    Vu que \( g\) est une isométrie, la base reste orthonormée sous l'action de \( g\). Donc \( g\) a pour matrice
    \begin{equation}
        \begin{pmatrix}
            1    &   0    &   0    \\
            0    &   1    &   0    \\
            0    &   0    &   \pm 1
        \end{pmatrix}.
    \end{equation}
    Pour que le déterminant soit \( 1\), il faut que la matrice soit l'identité.
\end{proof}

\begin{proposition}[\cite{MonCerveau,ooYODPooHeNKiQ,fJhCTE,ooBWVZooJiWRvf,ytMOpe}]      \label{PROPooBHPNooHPlgwH}
    Les sous-groupes finis de \( \SO(3)\) sont :
    \begin{multicols}{2}
        \begin{enumerate}
            \item
                les groupes cycliques \( \eZ/n\eZ\),
            \item
                les groupes diédraux \( D_n\),
            \item
                le groupe alterné \( A_4\),
            \item
                le groupe alterné \( A_5\)
            \item
                le groupe symétrique \( S_4\).
        \end{enumerate}
    \end{multicols}
\end{proposition}

\begin{proof}
    Soit \( G\), un sous-groupe fini de \( \SO(3)\). Par la proposition~\ref{PropKBCXooOuEZcS}, les éléments de \( G\) sont des isométries de \( \eR^3\), et le lemme~\ref{LEMooWIMMooXOCfSt} dit que tout élément de \( G\) possède une droite de points fixes.

    Un point de la sphère unité fixé par \( g\in G\) est un \defe{pôle}{pôle} de \( g\). Nous nommons \( \Omega\) l'ensemble des pôles des éléments non triviaux de \( G\).
    \begin{subproof}
        \item[Une action]
            Le groupe \( G\) agit sur \( \Omega\). En effet si \( x\in \Omega\), alors \( x\) est fixé par un élément \( g\). Montrons que \( h(x)\) est également fixé par un élément de \( G\). Par dur : \( (hgh^{-1})h(x)=h(x)\); donc \( h(x)\) est un pôle de \( h gh^{-1}\).

        \item[Les fixateurs sont cycliques]

            Nous montrons à présent que pour tout \( x\in\Omega\), le sous-groupe \( \Fix(x)\) est cyclique. Soit donc \( x\in\Omega\), le plan orthogonal \( \sigma=\Span(x)^{\perp}\) et \( h\in \Fix(x)\). Nous avons \( h(\sigma)=\sigma\). En effet si \( y\in \sigma\) nous avons
            \begin{equation}
                0=y\cdot x=h(y)\cdot h(x)=h(y)\cdot x,
            \end{equation}
            donc \( h(y)\) est perpendiculaire à \( x\). L'inclusion inverse se démontre de même : si \( y\in \sigma\) alors \( y=h\big( h^{-1}(y) \big)\) alors que \( h^{-1}(y)\in \sigma\).

            La restriction de \( h\) à \( \sigma\) est une isométrie de \( \sigma\). Prenant une isométrie \( f\colon \sigma\to \eR^2\), l'application
            \begin{equation}
                \begin{aligned}
                    \varphi\colon \Fix(x)&\to \SO(2) \\
                    h&\mapsto f\circ h\circ f^{-1}.
                \end{aligned}
            \end{equation}
            est un morphisme injectif de groupes. En effet nous avons d'une part
            \begin{equation}
                \varphi(hh')=f\circ h\circ h'\circ f^{-1}=fhf^{-1}fh'f^{-1}=\varphi(h)\varphi(h'),
            \end{equation}
            d'où le morphisme. Et d'autre part, si \( \varphi(h)=\varphi(h')\) alors \( f\circ h\circ f^{-1}=g\circ h'\circ f^{-1}\), qui donne immédiatement \( h=h'\).

            Nous en déduisons que \( \Fix(x)\) est isomorphe à un sous-groupe de \( SO(2)\) (l'image de \( \varphi\)). Le lemme~\ref{LEMooUKEVooAEWvlM} en fait un groupe cyclique.
        \item[Taille des fixateurs]

            Soient \( \Omega_i\) les orbites. Si \( x,y\in \Omega_i\) alors nous montrons que \( | \Fix(x) |=| \Fix(y) |\) avec la bijection
            \begin{equation}
                \begin{aligned}
                    \varphi\colon \Fix(x)&\to \Fix(y) \\
                    h&\mapsto g^{-1} hg
                \end{aligned}
            \end{equation}
            où \( g\) est choisi de façon à avoir \( y=g(x)\) (possible parce que \( x\) et \( y\) sont dans la même orbite). Cela est surjectif parce que si \( k\in\Fix(x)\) alors \( k=\varphi(gkg^{-1})\) et l'on vérifie que \( gkg^{-1}\in\Fix(y)\). L'application \( \varphi\) est également injective parce que si \( ghg^{-1}=gh'g^{-1}\) alors \( h=h'\).

        \item[Un peu de notations]
            Vu que tous les fixateurs des éléments d'une orbite ont la même taille (finie), nous pouvons noter
            \begin{equation}
                n_i=| \Fix(x_i) |
            \end{equation}
            pour \( x_i\in \Omega_i\). Nous notons également \( r\) le nombre d'orbites de \( G\).

            La formule de Burnside du théorème~\ref{THOooEFDMooDfosOw}, avec les notations d'ici, donne
            \begin{equation}
                r=\frac{1}{ | G | }\sum_{g\in G}| \Fix(g) |.
            \end{equation}

        \item[Une belle formule]

            Soit l'ensemble
            \begin{equation}
                A=\{ (g,x)\tq g\in G\setminus\{ e \}, x\in \Fix(g) \}
            \end{equation}
            où par \( \Fix(g)\) nous entendons les pôles de \( G\) fixés par \( g\).

            Il y a \( | G |-1\) possibilités pour la composante \( g\), mais chaque élément \( g\neq e\) possède exactement deux pôles, donc l'ensemble \( A\) contient exactement \( 2(| G |-1)\) éléments.

            Nous pouvons calculer le nombre d'éléments dans \( A \) d'une autre façon : pour chaque \( x\in \Omega\) nous avons \( | \Fix(x) |-1\) éléments de \( G\setminus\{ e \}\) qui fixent \( x\). Donc
            \begin{equation}
                | A |=\sum_{x\in\Omega}\big( | \Fix(x) |-1 \big).
            \end{equation}
            Mais \( | \Fix(x) |\) est constant sur les orbites. Nous coupons donc la somme sur \( \Omega\) en plusieurs sommes sur les orbites \( \Omega_i\) :
            \begin{equation}
                | A |=\sum_{i=1}^r\sum_{x\in \Omega_i}\big( | \Fix(x) |-1 \big)=\sum_i| \Omega_i |(n_i-1).
            \end{equation}
            En égalisant les deux façons de calculer \( | A |\), nous déduisons la formule
            \begin{equation}        \label{EQooHMLJooTGRBAl}
                2\big( | G |-1 \big)=\sum_i| \Omega_i |(n_i-1).
            \end{equation}

            Nous utilisons ensuite la relation orbite-stabilisateur, proposition~\ref{Propszymlr} : \( | \Fix(x_i) | |\Omega_i |=| G |\); la formule \eqref{EQooHMLJooTGRBAl} devient
            \begin{equation}
                2\big( | G |-1 \big)=\sum_i| G |-\sum_i\frac{ | G | }{ n_i }=r| G |+| G |\sum_i\frac{1}{ n_i },
            \end{equation}
            ou encore, en simplifiant par \( | G |\) :
            \begin{equation}        \label{EQooAMVBooDVcYeG}
                2-\frac{ 2 }{ | G | }=r-\sum_i\frac{1}{ n_i }=\sum_{i=1}^r\left( 1-\frac{1}{ n_i } \right).
            \end{equation}

            Nous pouvons aussi repartir de \eqref{EQooHMLJooTGRBAl} et sommer de façon plus simple \( \sum_i| \Omega_i |=| \Omega |\) pour obtenir
            \begin{equation}        \label{EQooTHUIooUEXsNl}
                2\big( | G |-1 \big)=r| G |-| \Omega |
            \end{equation}
            où \( \Omega\) est l'ensemble des pôles de \( G\setminus\{ e \}\).

        \item[Quelles sont les possibilités ?]

            Les nombres \( | G |\), \( r\) et \( n_i\) sont des entiers. Nous allons voir qu'il n'y a pas des centaines de possibilités pour satisfaire la relation \eqref{EQooAMVBooDVcYeG}. D'abord, pour toute valeur de \( | G |\) (strictement plus grande que \( 1\)),
            \begin{equation}
                1\leq 2-\frac{ 2 }{ | G | }<2.
            \end{equation}
            Ensuite, si \( g\) fixe \( x\) alors \( g^{-1}\) fixe également \( x\), de sorte que \( n_i=| \Fix(x_i) |\geq 2\) pour tout \( i\). Donc tous les termes dans la somme à droite de \eqref{EQooAMVBooDVcYeG} sont dans \( \mathopen[ \frac{ 1 }{2} , 1 \mathclose[\). Nous avons donc au minimum deux termes, et au maximum trois. Autrement dit : \( r=2\) ou \( r=3\).

            \item[Si \( r=2\)]

                Le plus simple est de repartir de \eqref{EQooTHUIooUEXsNl}. En posant \( r=2\) nous trouvons tout de suite \( | \Omega |=2\). Il y a donc exactement deux pôles pour l'action de \( G\) sur la sphère unité.

                Tous les éléments de \( G\) laissent donc le même axe invariant et \( G\) est un sous-groupe des isométries du plan qui lui est perpendiculaire. Autrement dit, \( G\) est un sous-groupe fini de \( \SO(2)\) et donc cyclique par le lemme~\ref{LEMooUKEVooAEWvlM}.

            \item[Les possibilités pour \( r=3\)]

                Nous devons voir les solutions entières \( (n_1,n_2,n_3,| G |)\) de
                \begin{equation}
                    2-\frac{ 2 }{ | G | }=3-\frac{1}{ n_1 }-\frac{1}{ n_2 }-\frac{1}{ n_3 }<2.
                \end{equation}
                Il faut en particulier que
                \begin{equation}
                    \frac{1}{ n_1 }+\frac{1}{ n_2 }+\frac{1}{ n_3 }>1,
                \end{equation}
                ce qui signifie qu'au moins un des \( n_i\) doit être \( 1\) ou \( 2\), mais qu'il n'est pas possible que tous les \( n_i\) soient plus grands ou égaux à \( 3\). Vu que \( n_i=| \Fix(x_i) |\geq 2\), nous en déduisons qu'au moins un des \( n_i\) doit valoir \( 2\). Nous posons donc \( n_1=2\).

                De plus, nous savons que les \( n_i\) doivent diviser \( | G |\). Donc \( | G |\) est pair.

            \item[Si \( n_2=2\)]

                Nous sommes dans le cas \( r=3\), \( n_1=2\), \( n_2=2\). Nous avons
                \begin{equation}
                    \frac{1}{ n_3 }=\frac{ 2 }{ | G | },
                \end{equation}
                mais aussi \( n_3=| G |/| \Omega_3 |\) d'où nous déduisons que \( | \Omega_3 |=2\). Nous avons donc une orbite à deux éléments. Soit \( \Omega_3=\{ x,y \}\) avec \( x\neq y\).

                Le groupe \( \Fix(x)\) est un groupe à \( | G |/2\) éléments. Il est donc normal par le lemme~\ref{LemSkIOOG}. Si \( g\in G\) est tel que \( g(x)=y\) alors nous avons \( \Fix(y)=g\Fix(x)g^{-1}\), mais comme \( \Fix(x)\) est normal nous avons \( \Fix(x)=\Fix(y)\). Donc tous les éléments de \( \Fix(x)\) fixent \( x\) et \( y\). Le groupe \( \Fix(x)\) est donc un sous-groupe de \( \SO(2)\) est est cyclique comme vu plus haut.

                Mais de plus nous avons forcément \( y=-x\) parce qu'un élément de \( G\) qui fixe un point fixe également l'opposé. Vu que \( \Omega_3=\{ x,-x \}\), il existe \( s\in G\) tel que \( s(x)=-x\). Évidemment, \( s\) n'est pas dans \( \Fix(x)\) et les points fixes de \( s\) ne sont pas parmi \( x\) et \( -x\). Donc l'élément \( s^2\) a au moins \( 4\) points fixes : les deux de \( s\) ainsi que \( x\) et \( -x\). Il a donc au moins deux droites de points fixes et est donc l'identité : \( s^2=e\).

                De plus, vu que \( s(y)\) doit être égal soit à \( x\) soit à \( y\), et vu que \( s(x)=y\), l'injectivité de \( s\) donne \( s(y)=x\).

                Soit \( a\), un générateur de \( \Fix(x)\). Nous allons montrer que \( G=\gr(s,sa)\). Nous avons déjà
                \begin{subequations}
                    \begin{align}
                        (sa)(x)=s(x)=y\\
                        (sa)(y)=s(y)=x.
                    \end{align}
                \end{subequations}
                Donc \( sa\) inverse \( x\) et \( y\). Mais \( sa\) a ses propres deux points fixes (qui ne sont ni \( x\) ni \( y\)). L'élément \( (sa)^2\) a deux quatre points fixes sur la sphère unité : \( x\), \( y\) et les deux de \( sa\). Nous en déduisons que \( (sa)^2=e\).

                Nous nous souvenons que \( a\) est un générateur \( \Fix(x)\). Mais \( a=s\cdot sa\), donc \( a^k=(ssa)^k\). Nous en déduisons que \( \gr(s,sa)\) contient au moins \( \Fix(x)\).

                D'autre part si \( h\) et \( h'\) sont des éléments distincts dans \( \Fix(x)\), alors \( sh\) et \( sh'\) sont des éléments distincts de \( \gr(s,sa)\) qui ne sont pas dans \( \Fix(x)\). Autrement dit, la partie
                \begin{equation}
                    A=\{ sh\tq h\in Fix(x) \}
                \end{equation}
                est une partie de même cardinal que \( \Fix(x)\) tout en n'ayant aucune intersection avec \( \Fix(x)\) (note : l'identité n'est pas dans \( A\)). Mais \( | \Fix(x) |=| G |/2\), donc \( A\cup\Fix(x)=G\). Et justement \( A\cup G\subset \gr(s,sa)\). Nous en déduisons que \( \gr(s,sa)=G\).

                Le théorème~\ref{THOooYITHooTNTBuG} nous assure que le groupe \( G\) est alors le groupe diédral parce que les éléments \( s\) et \( sa\) vérifient les relations données en~\ref{NORMooCCUEooRRENed}.

            \item[Si \( r=3\), les autres cas possibles]

                Nous repartons de \eqref{EQooAMVBooDVcYeG} en posant \( r=3\). Nous obtenons ceci :
                \begin{equation}
                    1+\frac{ 2 }{ | G | }=\frac{1}{ n_1 }+\frac{1}{ n_2 }+\frac{1}{ n_3 }.
                \end{equation}
                Nous avons déjà vu que \( n_1=2\) était obligatoire, et que tous les cas où deux des \( n_i\) sont égaux à \( 2\) sont déjà couverts. Donc \( n_2\) et \( n_3\) valent \( 3\) ou plus.

                Nous trions les \( n_i\) dans l'ordre croissant. Si \( n_2=4\) ou plus, alors \( n_3\) vaut \( 4\) ou plus. Mais
                \begin{equation}
                    \frac{ 1 }{2}+\frac{1}{ 4 }+\frac{1}{ 4 }=1<1+\frac{ 2 }{ | G | }.
                \end{equation}
                Donc \( n_3=3\) est obligatoire. Nous avons alors l'inégalité suivante qui restreint \( n_3\) :
                \begin{equation}
                    \frac{1}{ n_3 }=\frac{1}{ 6 }+\frac{ 3 }{ | G | }>\frac{1}{ 6 }.
                \end{equation}
                Donc \( n_3\) est \( 3\), \( 4\) ou \( 5\).

                Les derniers cas à couvrir sont :
                \begin{itemize}
                    \item \( (n_1,n_2,n_3)=(2,3,3)\). Dans ce cas, \( \frac{ 7 }{ 6 }=1+\frac{ 2 }{ | G | }\), donc \( | G |=12\).
                    \item \( (n_1,n_2,n_3)=(2,3,4)\). Dans ce cas, \( | G |=24\).
                    \item \( (n_1,n_2,n_3)=(2,3,5)\). Dans ce cas, \( | G |=60\).
                \end{itemize}

            \item[Le cas \( (2,3,3)\)]

                Nous utilisons les relations \( n_i| \Omega_i |=| G |\) pour savoir la taille des orbites. Nous avons :
                \begin{enumerate}
                    \item
                        \( 2| \Omega_1 |=12\), donc \( | \Omega_1 |=6\),
                        \( 3| \Omega_2 |=12\), donc \( | \Omega_2 |=4\),
                        \( 3| \Omega_3 |=12\), donc \( | \Omega_3 |=4\).
                \end{enumerate}

                Nous avons \( G\cdot \Omega_2=\Omega_2\). D'une part parce que, par définition d'une orbite, \( G\cdot\Omega_2\subset\Omega_2\), et d'autre part parce que si \( x\in\Omega_2\), alors \( g^{-1}(x)\in\Omega_2\) et \( g\big( g^{-1}(x) \big)=x\); donc \( \Omega_2\) est bien dans l'image de \( \Omega_2\) par \( G\). Nous avons donc un morphisme \( s\colon G\to S_{\Omega_2}\) que nous allons immédiatement voir comme
                \begin{equation}
                    s\colon G\to S_4
                \end{equation}
                où \( S_4\) est le groupe des permutations de \( \{ 1,2,3,4 \}\).

                Voyons que \( s\) est injective. Si \( s(g)=s(h)\), alors \( s(gh^{-1})=\id\). Autrement dit, l'élément \( sh^{-1}\) de \( G\) est l'identité sur \( \Omega_2\) qui contient \( 4\) éléments. Fixant \( 4\) points (au moins), l'élément \( sh^{-1}\) est l'identité. Par conséquent
                \begin{equation}
                    s\colon G\to s(G)\subset S_4
                \end{equation}
                est un isomorphisme entre \( G\) et un sous-groupe de \( S_4\). Mais \( | G |=12\) et \( | S_4 |=24\), donc \( G\) est d'indice deux dans \( S_4\) et est donc le groupe alterné \( A_4\) par la proposition~\ref{PROPooCPXOooVxPAij}\ref{ITEMooGGAHooRYgNqq}.

            \item[Le cas \( (2,3,4)\)]

                Nous avons \( | G=24 |\) et les orbites ont pour taille :
                \begin{itemize}
                    \item \( 2| \Omega_1 |=24\), donc \( | \Omega_1 |=12\),
                    \item \( 3| \Omega_2 |=24\), donc \( | \Omega_2 |=8\),
                    \item \( 4| \Omega_3 |=24\), donc  \( | \Omega_3 |=6\).
                \end{itemize}

                \begin{subproof}
                    \item[\( \Omega_2\) vient par paires]

                        Soit \( x\in \Omega\) tel que \( | \Fix(x) |=3\). Alors \( x\in\Omega_2\) parce que \( x\) est forcément dans un des \( \Omega_i\) et tout élément \( x_i\) de \( \Omega_i\) vérifie \( | \Fix(x_i) |=n_i\). Mais comme les éléments de \( \SO(3)\) sont des applications linéaires, ceux qui fixent \( x\) fixent également \( -x\). Cela pour dire que si \( x\in\Omega_2\), alors \( -x\in\Omega_2\). Nous avons donc quatre éléments distincts \( a_1\), \( a_2\), \( a_3\) et \( a_4\) tels que
                    \begin{equation}
                        \Omega_2=\{ \pm a_1,\pm a_2,\pm a_3,\pm a_4 \}.
                    \end{equation}

                \item[Action sur les couples]

                    Nous prétendons que \( G\) agit sur l'ensemble des couples \( \{ \pm a_i \}\). C'est encore la linéarité qui joue : l'élément \( g(a_i)\) est forcément un des \( \pm a_k\) (éventuellement \( k=i\)). Si \( g(a_i)=a_k\), alors \( g(-a_i)=-a_k\). Autrement dit, pour tout \( i\), il existe un \( k\) tel que \( g\big( \{ a_i,-a_i \} \big)=\{ a_k,-a_k \}\). Cette association \( i\mapsto k\) est bijective (sinon \( g\) ne serait pas bijective), et fournit donc un morphisme de groupes
                    \begin{equation}
                        s\colon G\to S_4.
                    \end{equation}

                \item[\( s\) est injective]

                    Nous prouvons à présent que \( s(g)=\id\) si et seulement si \( g=e\). Dans un sens c'est évident : \( g(e)=\id\). Dans l'autre sens, nous devons prouver que si \( g(a_i)\in \pm a_i\) pour tout \( i\) alors \( g=e\).

                    Si \( g(a_i)=a_i\) pour tout \( i\), alors \( g\) stabilise \( 4\) points et l'affaire est pliée. Nous supposons qu'au moins un des \( a_i\) n'est pas stabilisé par \( g\). Pour fixer les idées nous disons que c'est \( a_1\). Nous avons donc \( g(a_1)=-a_1\). (oui : \( g(a_1)=-a_1\) et non \( \pm a_k\) pour un autre \( k\) parce que nous sommes sous l'hypothèse que \( g\) stabilise les couples)

                    L'élément \( g^2\) fixe tout \( \Omega_2\); donc \( g^2=e\). Nommons \( \pm b\) les points fixes de \( g\). Si \( b\in \Omega_2\) alors \( | \Fix(b) |=3\), c'est-à-dire que les éléments de \( G\) qui fixent \( b\) sont dans un groupe d'ordre \( 3\), et le corolaire~\ref{CorpZItFX} nous indique que ces éléments ne peuvent être que d'ordre \( 1\) ou \( 3\), pas deux. Nous en déduisons que \( b\) n'est pas dans \( \Omega_2\) et donc que \( g(a_i)=-a_i\) pour tout \( i\).

                    Jusqu'à présent nous avons prouvé que si \( g\in \ker(s)\) est non trivial,  alors \( g(a_i)=-a_i\) pour tout \( i\).

                    Soit maintenant \( h\in G\). Vu que \( \Omega_2\) est une orbite, \( h(a_i)\in \Omega_2\) et nous notons \( h(a_i)=\epsilon a_k\) avec \( \epsilon=\pm 1\) et éventuellement \( k=i\) ou éventuellement pas. Nous avons :
                    \begin{equation}
                        (h^{-1}gh)(a_i)=\epsilon (h^{-1} g)(a_k)=-\epsilon h^{-1}(a_k)=-\epsilon^2 a_i=-a_i.
                    \end{equation}
                    Donc \( g\) et \( h^{-1} g h\) ont même restriction à \( \Omega_2\). En particulier \( h^{-1} ghg^{-1}\) est l'identité sur \( \Omega_2\) et est donc l'identité.

                    Pour tout \( h\) nous avons \( g=h^{-1} gh\). Les points fixes de \( h^{-1}g h\) sont \( \pm h^{-1}(b)\), mais aussi \( \pm b\). Nous avons donc égalité d'ensemble \( \{ h(b),-h(b) \}=\{ b,-b \}\) pour tout \( h\in G\) (notez le changement de notation \( h\to h^{-1}\)). Cela signifie que \( \{ b,-b \}\) est une orbite de \( G\). Maizon'a pas d'orbites de cardinal deux; contradition. Nous en déduisons que \( e\) est l'unique élément de \( \ker(s)\).

                    \item[Conclusion]

                        La partie \( s(G)\) est un sous-groupe de \( S_4\) isomorphe à \( G\). Mais au niveau des cardinaux, \( | G |=24\) en même temps que \( | S_4 |=24\). Donc \( G\simeq s(G)\simeq S_4\).

                \end{subproof}
                \end{subproof}

        Nous passons au cas \( (2,3,5)\), et comme ça va être long et douloureux\footnote{Mais pas autant que le théorème~\ref{THOooSTHXooXqLBoT}, cependant.}, nous sautons un niveau d'indentation.

                Au niveau du cardinal de \( G\),
                \begin{equation}
                    \frac{1}{ 2 }+\frac{1}{ 3 }+\frac{1}{ 5 }=1+\frac{ 2 }{ | G | },
                \end{equation}
                donc \( | G |=60\). Et pour les orbites, \( | \Omega_1 |=30\), \( | \Omega_2 |=20\), \( | \Omega_3 |=12\).

                La proposition~\ref{PROPooUBIWooTrfCat} nous indique que le seul groupe simple d'ordre \( 60\) est le groupe \( A_5\). Nous allons donc nous atteler à prouver que \( G\) est simple. Vous êtes prêts ?


                \begin{subproof}
                    \item[Fixateurs et ordres]

                Tous les éléments de \( G\) sont dans un fixateur de type \( \Fix(x)\), et comme l'ordre d'un élément divise l'ordre du groupe (corolaire~\ref{CorpZItFX}), tous les éléments de \( G\) ont un ordre \( 2\), \( 3\) ou \( 5\). Nous sommes dans un cas très particulier parce que
                \begin{itemize}
                    \item Les trois nombres \( 2\), \( 3\) et \( 5\) sont des nombres premiers distincts. Donc «diviser \( n_i\)» signifie pratiquement «être égal à \( n_i\)», surtout lorsqu'on parle de l'ordre d'un élément, qui ne peut pas être \( 1\).
                    \item Il existe une seule orbite de chaque taille.
                \end{itemize}
            Nous notons \( G(n_i)\) l'ensemble des éléments de \( G\) d'ordre \( n_i\). Les parties \( G(n_i)\) ne contiennent pas l'identité.

            \item[\( g\in G(n_i)\) implique \( \Fix(g)\subset \Omega_i\)]

                Si \( g\in G(n_i)\) et \( x\in\Fix(g)\) alors \( x\in \Omega_i\). En effet \( x\in Fix(g)\) signifie \( g(x)=x\) et donc aussi \( g\in\Fix(x)\). Donc l'ordre de \( g\) divise \( | Fix(x) |\), alors que l'ordre de \( g\) est \( n_i\) et que les possibilités pour \( | \Fix(x) |\) sont exactement les \( n_i\), lesquels sont premiers entre eux. Donc \( | \Fix(x) |=n_i\) et \( x\in \Omega_i\).

            \item[\( | \Fix(x) |=n_i\) implique \( x\in \Omega_i\)]

                Comme plus haut, \( g\in\Fix(x)\) implique que l'ordre de \( g\) divise \( n_i\) et est donc égal à \( n_i\). Autrement dit, \( g\in G(n_i)\). De plus \( g\in\Fix(x)\) implique \( x\in\Fix(g)\). Par le cas juste au-dessus nous déduisons \( x\in\Omega_i\).

            \item[\( a\) et \( -a\) dans la même orbite]

                Nous avons évidemment \( \Fix(a)=\Fix(-a)\) du fait que les éléments de \( G\) sont des applications linéaires. Si \( | \Fix(a) |=n_i\) alors \( a\in\Omega_i\) et aussi \( | \Fix(-a) |=| \Fix(a) |=n_i \) et aussi \( -a\in \Omega_i\).

            \item[Nombre de \( \Fix(x_i)\)]

                Soient \( a,b\in \Omega_i\). Nous avons \( | \Fix(a) |=| \Fix(b) |=n_i\) et \( \Fix(a)=\Fix(b)\) si et seulement si \( b=-a\) parce qu'un élément qui fixe \( a\) et \( b\) fixe automatiquement \( a\), \( b\), \( -a\), et \( -b\). Aucun élément non trivial ne peut fixer quatre points distincts. Autrement dit,
                \begin{equation}
                    \Fix(a)\cap\Fix(b)=\begin{cases}
                        \Fix(a)    &   \text{si } a=\pm b\\
                        \{ e \}    &    \text{sinon. }
                    \end{cases}
                \end{equation}
                Chaque élément \( x_i\in \Omega_i\) a son fixateur (il y en aurait \( | \Omega_i |=60/n_i\)), mais ces fixateurs sont égaux deux à deux, donc il y a seulement \( \frac{ 60 }{ 2n_i }\) groupes distincts de la forme \( | \Fix(x_i) |\) avec \( x_i\in \Omega_i\).

            \item[Récapitulatif]

                En reprenant ce que nous venons de dire avec \( i=1,2,3\) nous trouvons :
                \begin{enumerate}
                    \item
                        \( n_1=2\), avec \( | \Omega_1 |=30\) et \( 15\) groupes du type \( \Fix(x_1)\) avec \( x_1\) parcourant \( \Omega_1\).
                    \item
                        \( n_2=3\), avec \( | \Omega_2 |=20\) et \( 10\) groupes du type \( \Fix(x_2)\) avec \( x_2\) parcourant \( \Omega_2\).
                    \item
                        \( n_3=5\), avec \( | \Omega_3|=12\) et \( 6\) groupes du type \( \Fix(x_3)\) avec \( x_3\) parcourant \( \Omega_3\).
                \end{enumerate}
                Un élément non trivial de \( G\) se trouve forcément dans un et un seul de ces sous-groupes. Plus précisément, si \( g\in G(n_i)\) alors \( g\) est dans un des \( \Fix(x_i)\) avec \( x_i\in \Omega_i\).

                Comptons pour être sûr de ne pas s'être trompé. Chacune des lignes décrit \( 30\) éléments de \( G\); par exemple pour la seconde ligne donne \( 10\) groupes de taille \( | \Fix(x_2) |=n_2=3\). Mais tous ces groupes ont pour intersection exactement \( \{ e \}\). Donc le comptage des éléments se fait comme suit :
                \begin{equation}
                    3\times 30-15-10-6+1.
                \end{equation}
                Le dernier \( +1\) est parce que nous aurions décompté l'identité une fois de trop. Bref, on a bien \( 60\) éléments comme il se doit.

            \item[Un ensemble à calculer deux fois]

                Soient les ensembles \( A_2\), \( A_3\) et \( A_5\) définis par
                \begin{equation}
                    A_i=\{ (g,a)\in G(n_i)\times \Omega_i\tq g(a)=a  \}
                \end{equation}
                où \( G(n_i)\) est la partie de \( G\) des éléments d'ordre \( n_i\).

                Nous avons
                \begin{equation}
                    | A_i |=\sum_{g\in G(n_i)}| \Fix(g)\cap \Omega_i |.
                \end{equation}
                Mais les éléments de \( G(n_i)\) sont d'ordre \( n_i\), et par ce que nous avons dit plus haut, tous les éléments de \( \Fix(g)\) sont dans \( \Omega_i\). Donc \( \Fix(g)\cap \Omega_i=\Fix(g)\). Nous avons alors
                \begin{equation}
                    | A_i |=\sum_{g\in G(n_i)}| \Fix(g) |=2|G(n_i)|
                \end{equation}
                parce que \( | \Fix(g) |=2\) pour tout \( g\).

                En compatant \( | A_i |\) dans l'autre sens, nous avons
                \begin{equation}        \label{EQooBHIIooVcGgFd}
                    | A_i |=\sum_{x\in \Omega_i}|  \Fix(x)\cap G(n_i) |
                \end{equation}
                Vu que \( x\in \Omega_i\), les éléments de \( \Fix(x)\) sont d'ordre \( n_i\)\footnote{Encore et toujours parce que les éléments de \( \Fix(x)\) ont un ordre qui divise \( | \Fix(x) |=n_i\) et que \( n_i\) est premier, et que nous avons exclu l'identité.} (sauf \( e\)), et comme \( G(n_i)\) est justement l'ensemble des éléments d'ordre \( n_i\) dans \( G\) nous avons \( \Fix(x)\cap G(n_i):\Fix(x)\setminus\{ e \}\). Cela pour dire que
                \begin{subequations}
                    \begin{align}
                        | A_i |&=\sum_{x\in \Omega_i}\Big( |\Fix(x)|-1\Big)\\
                        &=\sum_{x\in \Omega_i}| \Fix(x) |-\sum_{x\in \Omega_i}1\\
                        &=\sum_{x\in \Omega_i}n_i-| \Omega_i |  & | \Fix(x) |=n_i \text{ pcq }x\in\Omega_i\\
                        &=| \Omega_i |n_i-| \Omega_i |=| G |-| \Omega_i |.
                    \end{align}
                \end{subequations}
                En égalisant cela à la valeur \( 2|G(n_i)|\) déjà trouvée, nous déduisons les valeurs des \( | G(n_i) |\) :
                \begin{equation}
                    | G(n_i) |=\frac{ | G |-| \Omega_i | }{2}.
                \end{equation}
                Nous avons alors
                \begin{enumerate}
                    \item
                        \( | G(2) |=15\)
                    \item
                        \( | G(3) |=20\)
                    \item
                        \( | G(5) |=24\)
                \end{enumerate}

            \item[Les Sylow de \( G\)]

                Les \( p\)-Sylow sont définis en~\ref{DEFooPRCHooVZdwST}, et le super théorème qui répond à toutes les questions est le théorème~\ref{ThoUkPDXf}. Dans notre cas, les diviseurs premiers de \( | G |=60\) sont \( 2\), \( 3\) et \( 5\). Il faut faire attention au $2$ parce que sa plus haute puissance dans la décomposition de \( 60\) est \( 4\) et non \( 2\). Nous avons :
                \begin{enumerate}
                    \item
                        Un \( 2\)-Sylow est un sous-groupe d'ordre \( 4\).
                    \item
                        Un \( 3\)-Sylow est un sous-groupe d'ordre \( 3\).
                    \item
                        Un \( 5\)-Sylow est un sous-groupe d'ordre \( 5\).
                \end{enumerate}
                Entre autres :
                \begin{enumerate}
                    \item
                        Les \( 10\) sous-groupes \( \Fix(x_2)\) avec \( x_2\in \Omega_2\) sont des \( 3\)-Sylow de \( G\).
                    \item
                        Les \( 6\) sous-groupes \( \Fix(x_3)\) avec \( x_3\in \Omega_3\) sont des \( 5\)-Sylow de \( G\).
                    \item
                        Les \( 15\) sous-groupes \( \Fix(x_1)\) avec \( x_1\in \Omega_1\) sont d'ordre $2$ et ne sont donc pas des \( 2\)-Sylow de \( G\).
                \end{enumerate}

            \item[Tous les \( 3\)-Sylow et les \( 5\)-Sylow]

                Nous avons déjà trouvé \( 10\) \( 3\)-Sylow et \( 6\) \( 5\)-Sylow. Nous montrons à présent qu'il n'y en a pas d'autres. Le théorème de Sylow~\ref{ThoUkPDXf}\ref{ItemkYbdzZ} nous indique que le nombre \( n_3\) de \( 3\)-Sylow est :
                \begin{itemize}
                    \item diviseur de \( 60\),
                    \item dans \( [1]_3\)
                    \item au moins \( 10\).
                \end{itemize}
                Les diviseurs de \( 60\) sont :
                \begin{equation}
                    1,5,3,15,2,10,6,30,4,20,12,60.
                \end{equation}
                Le seul qui vérifie toutes les conditions est \( 10\). Donc \( G\) possède seulement \( 10\) \( 3\)-Sylow et ils sont tous de la forme \( \Fix(x_2)\) avec \( x_2\in \Omega_2\).

                Même raisonnement pour les \( 5\)-Sylow : il faut
                \begin{itemize}
                    \item diviseur de \( 60\),
                    \item dans \( [1]_5\)
                    \item au moins \( 6\).
                \end{itemize}
                La seule possibilité est \( 6\).

            \item[Sous-groupe normal]

                Soit \( H\), un sous-groupe normal de \( G\). Notre but étant de prouver que \( G\) est simple, nous voulons prouver que \( H\) est soit \( \{ e \}\) soit \( G\). Nous supposons que \( H\) est non trivial, et nous allons prouver que \( H=G\).

                Le théorème de Lagrange~\ref{ThoLagrange}\ref{ITEMooDPKSooNpOusd} nous dit que \( | H |\) divise \( | G |\). Le nombre \( | H |\) ne peut donc avoir que \( 2\), \( 3\) et \( 5\) comme facteurs premiers. Avec une mention spéciale pour le \( 2\) : \( | H |\) pourrait être divisible aussi par \( 4\).

            \item[Diviseurs de \( | H |\)]

                Soit un sous-groupe normal \( H\) de \( G\). Vu que c'est un sous-groupe sont ordre divise celui de \( G\) (encore et toujours le théorème de Lagrange~\ref{ThoLagrange}), et donc les facteurs premiers de \( | H |\) ne peuvent être que \( 2\), \( 3\) et \( 5\).

            \item[Si \( | H |\) est divisible en \( 3\)]

                Alors \( H\) contient au moins un \( 3\)-Sylow. Mais nous avons vu que les \( 3\)-Sylow de \( H\) sont les \( 3\)-Sylow de \( G\). Donc \( H\) contient tous les \( 3\)-Sylow de \( G\), parce que les \( 3\)-Syow sont conjugués et \( H\) est normal.

                Soit \( E\) l'ensemble des sous-groupes de \( H\). Vu qu'il est normal, \( H\) agit sur \( E\) par conjugaison, et les \( 3\)-Sylow forment une orbite. Si \( \alpha\) est un \( 3\)-Sylow, la formule des classes (proposition~\ref{Propszymlr}\ref{ITEMooCWUGooCOFHYk}) nous donne
                \begin{equation}
                    | H |=| \Fix(\alpha) | |\mO_{\alpha} |.
                \end{equation}
                Mais l'orbite \( \mO_{\alpha}\) de \( \alpha\) est l'ensemble des \( 3\)-Sylow, de sorte que \( | \mO_{\alpha} |=10\). Donc \( | H |\) est divisible en \( 10\).

                Mais il y a pire : \( H\) contient au moins les \( 10\) sous-groupes \( \Fix(x_2)\) pour \( x_2\in \Omega_2\). Ce sont \( 10\) groupes de \( | \Fix(x_2) |=3\) éléments. En décomptant \( e\) qui est dans l'intersection, cela fait
                \begin{equation}
                    10\times | \Fix(x_2) |-10+1=21
                \end{equation}
                éléments. Donc \( H\) contient au moins \( 21\) éléments. Le nombre \( | H |\) est donc :
                \begin{itemize}
                    \item diviseur de \( 60\)
                    \item multiple de \( 10\)
                    \item au moins \( 21\).
                \end{itemize}
                Donc c'est \( 30\) ou \( 60\).

            \item[Si \( | H |\) est divisible en \( 5\)]

                Le même raisonnement tient et \( | H |\) est \( 30\) ou \( 60\).

    \end{subproof}

        Nous restons avec les possibilités \( | H |\) égal à \( 2\), \( 4\), \( 30\) ou \( 60\).

    \begin{subproof}

            \item[Si \( | H | = 4\)]

                Alors \( H\) contient au moins un \( 2\)-Sylow. Un \( 2\)-Sylow de \( H\) est un sous-groupe contenant \( 4\) éléments qui sont d'ordre \( 2^m\). Le seul \( m\) possible dans \( G\) est \( m=1\). Vu qu'un \( 2\)-Sylow de \( H\) contient \( 4\) éléments, nous sommes dans le cas où \( H\) est un \( 2\)-Sylow. Il est donc le seul \( 2\)-Sylow de \( H\) parce que \( H\) est normal et que tous les \( 2\)-Sylow sont conjugués.

                Mais tous les sous-groupes d'ordre \( 2\) sont contenus dans un \( 2\)-Sylow. En particulier tous les \( 15\) groupes \( \Fix(x_1)\) sont dans l'unique \( 2\)-Sylow \( H\) qui est soit-disant d'ordre \( 4\). IL y a là une belle impossibilité.

                Donc le cas \( | H |=4\) est hors-concours.

            \item[Si \( | H |=2\)]

                Alors \( H=\{ e,g \}\) avec \( g^2=e\). Si \( h\in G\), l'élément \( hgh^{-1}\) ne peut être que \( e\) ou \( g\) (parce que \( H\) est normal). Le premier cas est \( g=e\), et le second donne \( gh=hg\). Donc \( g\) est dans le centre de \( G\) : il commute avec tous les éléments de \( G\).

                Vu que \( g\in G(2)\), nous avons que les éléments \( a\in\Fix(g)\) sont forcément dans \( \Omega_1\) parce que les points dont les fixateurs sont formés d'éléments d'ordre \( 2\) sont dans \( \Omega_1\). Soit \( h\in G\). Nous avons \( g=hgh^{-1}\) et donc aussi
                \begin{equation}
                    \big( hgh^{-1} \big)\big( h(a) \big)=hg(a)=h(a),
                \end{equation}
                donc \( h(a)\) et \( -h(a)\) sont des points fixes de \( hgh^{-1}\). Ce sont donc également de points fixes de \( g\). Nous en déduisons que \( g\) a pour points fixes les points \( a\), \( -a\), \( h(a)\) et \( -h(a)\). Vu que \( g\) n'est pas \( e\), ces quatre points ne peuvent pas être distincts. Vu que \( h(a)\) ne peut pas être \( -h(a)\), nous avons forcément \( h(a)=\pm a\).

                Donc l'orbite de \( a\) ne contiendrait que \( 2\) éléments. Pas possible.

            \item[Si \( | H |=30\)]

               À part \( | H |=60\), le dernier cas à traiter est \( | H |=30\). Nous rappelons obligeamment que
               \begin{enumerate}
                   \item
                       \( | G(2) |=15\)
                   \item
                       \( | G(3) |=20\)
                   \item
                       \( | G(5) |=24\).
               \end{enumerate}
               Si \( H\) possède \( 30\) éléments, le théorème de Sylow dit que \( H\) contient au moins un \( 3\)-Sylow et un \( 5\)-Sylow, et donc tous. Vu que pour \( 3\) et \( 5\), les Sylow de \( H\) et de \( G\) sont les mêmes et bien identifiés, nous allons nous baser dessus. Le sous-groupe \( H\) contient tous les \( 3\) et \( 5\)-Sylow, donc le comptage des éléments est :
                \begin{equation}
                    10\times | \Fix(x_2) |+6\times | \Fix(x_3) |-15=45.
                \end{equation}
                Nous aurions aussi pu ajouter \( +4-1\) pour compter au moins un \( 2\)-Sylow.

                Donc dès que \( H\) compte \( 30\) éléments, il en compte \( 45\) et donc \( 60\) parce qu'il n'y a pas de diviseurs de \( 60\) entre \( 45\) et \( 60\).

    \end{subproof}

\end{proof}

\begin{probleme}
    La démonstration des groupes finis de \( \SO(3)\) est longue. Je me demande si il n'y a pas moyen de faire plus court. Par exemple \cite{ooYODPooHeNKiQ} utilise le théorème de Cauchy~\ref{THOooSUWKooICbzqM} que je n'utilise pas. D'autre part, toutes les références me semblent utiliser plus ou moins implicitement le fait que si le sous-groupe normal \( H\) contient un élément de \( G(n_i)\), alors il les contiennent tous. J'avoue ne pas trop comprendre pourquoi.
\end{probleme}

%+++++++++++++++++++++++++++++++++++++++++++++++++++++++++++++++++++++++++++++++++++++++++++++++++++++++++++++++++++++++++++ 
\section{Systèmes de coordonnées}
%+++++++++++++++++++++++++++++++++++++++++++++++++++++++++++++++++++++++++++++++++++++++++++++++++++++++++++++++++++++++++++
\label{SECooWTPRooZbOSzO}

La trigonométrique nous offre de nouveaux systèmes de coordonnées qui peuvent se révéler pratiques de certains cas : les coordonnées polaires sur \( \eR^2\) ainsi que les coordonnées cylindriques et sphériques sur \( \eR^3\).

%--------------------------------------------------------------------------------------------------------------------------- 
\subsection{Coordonnées polaires}
%---------------------------------------------------------------------------------------------------------------------------

%///////////////////////////////////////////////////////////////////////////////////////////////////////////////////////////
\subsubsection{Ce que ça signifie intuitivement}
%///////////////////////////////////////////////////////////////////////////////////////////////////////////////////////////

On a vu qu'un point $M$ dans $\eR^2$ peut être représenté par ses abscisses $x$ et ses ordonnées $y$. Nous pouvons également déterminer le même point $M$ en donnant un angle et une distance comme montré sur la figure~\ref{LabelFigJWINooSfKCeA}.
\newcommand{\CaptionFigJWINooSfKCeA}{Un point en coordonnées polaires est donné par sa distance à l'origine et par l'angle qu'il faut avec l'horizontale.}
\input{auto/pictures_tex/Fig_JWINooSfKCeA.pstricks}


Le même point $M$ peut être décrit indifféremment avec les coordonnées $(x,y)$ ou bien avec $(r,\theta)$.

\begin{remark}
	L'angle $\theta$ d'un point n'étant à priori défini qu'à un multiple de $2\pi$ près, nous convenons de toujours choisir un angle $0\leq\theta<2\pi$. Par ailleurs l'angle $\theta$ n'est pas défini si $(x,y)=(0,0)$.

	La coordonnée $r$ est toujours positive.
\end{remark}

Nous avons dans l'idée de définir \( r\) et \( \theta\) par les formules
\begin{subequations}		\label{EqrthetaxyPoal}
	\begin{numcases}{}
		x=r\cos(\theta)\\
		y=r\sin(\theta).
	\end{numcases}
\end{subequations}

%///////////////////////////////////////////////////////////////////////////////////////////////////////////////////////////
\subsubsection{Coordonnées polaires : le théorème}
%///////////////////////////////////////////////////////////////////////////////////////////////////////////////////////////

\begin{theorem}[Coordonnées polaires\cite{MonCerveau}]     \label{THOooBETSooXSQhdX}
    Soit l'application
    \begin{equation}
        \begin{aligned}
            T\colon \mathopen[ 0 , \infty \mathclose[\times \mathopen[ 0 , 2\pi \mathclose[&\to \eR^2 \\
                (r,\theta)&\mapsto \begin{pmatrix}
                    r\cos(\theta)    \\ 
                    r\sin(\theta)    
                \end{pmatrix}.
        \end{aligned}
    \end{equation}
    \begin{enumerate}
        \item       \label{ITEMooNGOKooFCXmwy}
            L'application \( T\) est surjective.
        \item       \label{ITEMooMCIOooJiBvug}
            L'application
            \begin{equation}
                T\colon \mathopen] 0 , \infty \mathclose[\times \mathopen[ 0 , 2\pi \mathclose[\to \eR^2\setminus\{ (0,0) \}
            \end{equation}
            est bijective.
        \item       \label{ITEMooZFRGooQPDUtX}
            En considérant la demi-droite \( D=\{ (x,0) \}_{x\geq 0}\), l'application
            \begin{equation}
                T\colon \mathopen] 0 , \infty \mathclose[\times \mathopen] 0 , 2\pi \mathclose[\to \eR^2\setminus D
            \end{equation}
            est un \(  C^{\infty}\)-difféomorphisme\footnote{L'application est de classe \(  C^{\infty}\) et son inverse est également de classe \(  C^{\infty}\). Le plus souvent, vous voulez seulement utiliser ce théorème dans le but de faire un changement de variables dans une intégrale; vous n'avez donc besoin que d'un \( C^1\)-difféomorphisme.}.
    \end{enumerate}
\end{theorem}

\begin{proof}
    Une bonne partie de ce théorème est une conséquence de \ref{PROPooKSGXooOqGyZj}. La vraie nouveauté de ce théorème sera la régularité.  Nous démontrons point par point.
    \begin{enumerate}
        \item
            Pour \ref{ITEMooNGOKooFCXmwy}. Soit \( a=(x,y)\in \eR^2\). Nous avons \( a/\| a \|\in S^1\). Par la proposition \ref{PROPooKSGXooOqGyZj}, il existe \( \theta\in \mathopen[ 0 , 2\pi \mathclose]\) tel que
            \begin{equation}
                \frac{ a }{ \| a \| }=\big( \cos(\theta),\sin(\theta) \big).
            \end{equation}
            Alors \( a=  \| a \|\big( \cos(\theta),\sin(\theta) \big)= T(\| a \|,\theta)\). Voila. L'application \( T\) est surjective.
        \item
            Pour \ref{ITEMooMCIOooJiBvug}. En ce qui concerne la surjectifivé,
            \begin{equation}
                T\big( 0,\mathopen[ 2 , 2\pi \mathclose[ \big)=\{ (0,0) \}.
            \end{equation}
            Donc le point \ref{ITEMooNGOKooFCXmwy} donne le surjectif lorsque nous enlevons d'un côté les points avec \( r=0\) et de l'autre le point \( (0,0)\). 
            
            Pour l'injectivité, nous supposons \( T(r_1,\theta_1)=T(r_2,\theta_2)\). Vu que \( \| T(t,\theta) \|=r\), nous avons tout de suite \( r_1=r_2\). Nous restons donc avec l'égalité
            \begin{equation}
                \begin{pmatrix}
                    \cos(\theta_1)    \\ 
                    \sin(\theta_1)    
                \end{pmatrix}=\begin{pmatrix}
                    \cos(\theta_2)    \\ 
                    \sin(\theta_2)    
                \end{pmatrix}.
            \end{equation}
            La proposition \ref{PROPooKSGXooOqGyZj} dit alors que \( \theta_1=\theta_2\).
        \item
            Pour \ref{ITEMooZFRGooQPDUtX}. L'application \( T\) est injective en tant que restriction d'une application injective. Pour le surjectif, soit \( a\in \eR^2\setminus D\). Vu que \( a\notin D\), nous avons \( \| a \|\neq 0\) et il est légitime de dire, comme plus haut, qu'il existe \( \theta\in \mathopen[ 0 , 2\pi \mathclose[\) tel que
                \begin{equation}
                    \frac{ a }{ \| a \| }=\begin{pmatrix}
                        \cos(\theta)    \\ 
                        \sin(\theta)    
                    \end{pmatrix}.
                \end{equation}
                Ce \( \theta\) n'est pas zéro parce que \( \theta=0\) donne le point \( (1,0)\) qui est sur \( D\).

                En ce qui concerne l'inverse, nous n'allons pas nous lancer dans une étude subtile de la fonction \eqref{EQooSAYFooRFVSPc}; nous avons déjà démontré la continuité dans le lemme \ref{LEMooEQVRooMAffCw}, et monter dans les dérivées nous semble un peu compliqué. Au lieu de cela, nous allons faire en deux étapes :
                \begin{itemize}
                    \item Prouver que \( T\) est de classe \( C^p\) pour tout \( p\) en invoquant seulement des théorèmes à proposition de différentielle,
                    \item
                        En déduire que \( T^{-1}\) est également \( C^p\) pour tout \( p\) en invoquant le théorème d'inversion locale \ref{ThoXWpzqCn}.
                \end{itemize}

                Les applications \( (r,\theta)\mapsto r\), \( (r,\theta)\mapsto \sin(\theta)\) et \( (r,\theta)\mapsto \cos(\theta)\) sont de classe \(  C^{\infty}\) grâce au lemme \ref{LEMooDDUZooLwXkRp}. Le lemme \ref{LemDiffProsuid} sur la différentiabilité du produit montre alors que les fonctions \( T_1\) et \( T_2\) données par
                \begin{subequations}
                    \begin{align}
                        T_1(r,\theta)=r\cos(\theta)\\
                        T_2(r,\theta)=r\sin(\theta)
                    \end{align}
                \end{subequations}
                sont différentiables\footnote{Si vous voulez seulement avoir un \( C^1\)-difféomorphisme, calculez explicitement la différentielle et montrez que c'est continu. Vous n'avez pas à utiliser la proposition \ref{PROPooWNCGooHbmcVb} ni rien des produits tensoriels.}. Mieux, la proposition \ref{PROPooWNCGooHbmcVb} montre que ces fonctions \( T_1\) et \( T_2\) sont de classe \( C^p\) pour tout \( p\), c'est-à-dire qu'elles sont de classe \(  C^{\infty}\). Cela montre que les coordonnées polaires sont de classe \(  C^{\infty}\), et il faut encore parler de l'inverse.

                En ce qui concerne la différentielle,
                \begin{equation}
                    dT_{(r,\theta)}(u,v)=\begin{pmatrix}
                        u\cos(\theta)-rv\sin(\theta)    \\ 
                        u\sin(\theta)+rv\cos(\theta)    
                    \end{pmatrix}.
                \end{equation}
                Donc la matrice de la différentielle est
                \begin{equation}
                    dT_{(r,\theta)}=\begin{pmatrix}
                        \cos(\theta)    &  -r\sin(\theta)    \\ 
                        \sin(\theta)    &   r\cos(\theta),    
                    \end{pmatrix}
                \end{equation}
                dont le déterminant est \( r\) (lemme \ref{LEMooAEFPooGSgOkF} utilisé). Donc la différentielle en \( (r,\theta)\) est une application linéaire inversible parce que \( r\neq 0\) aux points que nous considérons. L'application \( dT_{(r,\theta)}\) est bicontinue parce que nous sommes en dimension finie. Tout cela pour dire que le théorème d'inversion local \ref{ThoXWpzqCn} fonctionne et \( T^{-1}\) est \( C^p\) dès que \( T\) est \( C^p\). 

                Vu que \( T\) est de classe \( C^p\) pour tout \( p\), l'inverse \( T^{-1}\) est également \( C^p\) pour tout \( p\), c'est-à-dire que \( T^{-1}\) est de classe \(  C^{\infty}\).
    \end{enumerate}
\end{proof}

\begin{definition}
    Ce que nous appelons \defe{les coordonnées polaires}{coordonnées polaires} est l'application 
    \begin{equation}
        \begin{aligned}
            T\colon \mathopen[ 0 , \infty \mathclose[\times \mathopen[ 0 , 2\pi \mathclose[&\to \eR^2 \\
                (r,\theta)&\mapsto \begin{pmatrix}
                    r\cos(\theta)    \\ 
                    r\sin(\theta)    
                \end{pmatrix}.
        \end{aligned}
    \end{equation} 
    du théorème \ref{THOooBETSooXSQhdX}\ref{ITEMooZFRGooQPDUtX}. Selon les circonstances, nous considérons l'une ou l'autre des restrictions pour avoir une bijection ou un difféomorphisme.
\end{definition}

\begin{example}     \label{EXooSDHDooJzDioW}
	Soit à calculer
	\begin{equation}
		\lim_{(x,y)\to(0,0)}\frac{ x^2+y^2 }{ x-y }.
	\end{equation}

    Nous introduisons la fonction
    \begin{equation}
        \begin{aligned}
            f\colon \eR^2\setminus\{ x=y \}&\to \eR \\
            (x,y)&\mapsto \frac{ x^2+y^2 }{ x-y }. 
        \end{aligned}
    \end{equation}
    Une idée souvent fructueuse pour traiter ce genre de limite est de passer aux coordonnées polaires. Attention, si on veut faire les choses très explicitement, c'est un peu lourd en notations. Il s'agit de poser
    \begin{equation}
        \begin{aligned}
        f\colon \big( \mathopen] 0 , \infty \mathclose[\times\mathopen[ 0 , 2\pi \mathclose[ \big)\setminus\big\{ \eR\times\{ \frac{ \pi }{ 4 }\}\cup\eR\times \{ \frac{ 5\pi }{ 4 } \} \}&\to \eR \\
            (r,\theta)&\mapsto \frac{ r^2 }{ r\big( \cos(\theta)-\sin(\theta) \big) }. 
        \end{aligned}
    \end{equation}
    Bon. À strictement parler, nous aurions pu dire que \( g\) est définie pour \( r=0\), mais vu que nous voulons seulement calculer la limite pour \( r\to 0\), on n'a pas besoin de la valeur en zéro. De plus les coordonnées polaires ne sont pas bijectives en l'origine. Donc bon \ldots on s'en passe.

    Quel est le lien entre \( f\) et \( g \) ? Du point de vue du calcul, le lien est qu'on a remplacé \( x\) par \( r\cos(\theta)\) et \( y\) par \( r\sin(\theta)\). Le vrai lien est l'égalité
    \begin{equation}
        g=f\circ T
    \end{equation}
    où \( T\) est l'application de coordonnées polaires dont les principales propriétés sont données dans le théorème \ref{THOooBETSooXSQhdX}\ref{ITEMooMCIOooJiBvug}.

    Soit un voisinage \( B\big( (0,0), R \big)\) de \( (0,0)\) dans \( \eR^2\). Le but est de montrer que les valeurs \( f(B)\) se regroupent autour d'une valeur \( \ell\) lorsque \( R\to 0\). Soyons plus précis et nommons \( \ell\) le candidat limite. Soit \( \epsilon>0\); nous devons trouver \( R>0\) tel que \( f\Big( B\big( (0,0),R \big) \Big)\subset B(\ell,\epsilon)\).

    Pour \( R>0\), nous avons
    \begin{equation}
        B\big( (0,0),R \big)=T\big( \mathopen[ 0 , R \mathclose[\times \mathopen[ 0 , 2\pi \mathclose[ \big),
    \end{equation}
    donc
    \begin{equation}
        f(B)=g\big( \mathopen[ 0 , R \mathclose[\times \mathopen[ 2 , 2\pi \mathclose[ \big).
    \end{equation}
    Soit \( r<R\). Nous avons
    \begin{equation}
        \lim_{\theta\to \pi/4} g(r,\theta)=\infty.
    \end{equation}
    Donc \( f(B)\) contient des valeurs arbitrairement grandes, quelle que soit la valeur de \( R\). Il n'y a donc pas de limite possible.
    
    Si vous voulez un argument un peu plus imagé, en voici un\footnote{Qui satisfera tous vos professeurs, pourvu que vous ayez compris que ce qui se cache est une histoire de valeurs de \( f\) prises sur un voisinage de \( (0,0)\).} basé sur une combinaison entre la méthode des coordonnées polaires et la méthode des chemins.
    
	Certes \emph{pour chaque $\theta$} nous avons $\lim_{r\to 0} g(r,\theta)=0$, mais il ne faut pas en déduire trop vite que la limite $\lim_{(x,y)\to(0,0)}g(x,y)$ vaut zéro parce que prendre la limite $r\to 0$ avec $\theta$ fixé revient à prendre la limite le long de la droite d'angle $\theta$.

	Il n'est pas possible de majorer $g(r,\theta)$ par une fonction ne dépendant pas de $\theta$ parce que cette fonction tend vers l'infini lorsque $\theta\to\pi/4$. Est-ce que cela veut dire que la limite n'existe pas ? Cela veut en tout cas dire que la méthode des coordonnées polaires ne parvient pas à résoudre l'exercice. Pour conclure, il faudra encore un peu travailler.

    Nous pouvons essayer de calculer le long d'un chemin plus général \( (r(t),\theta(t))\). Choisissons \( r(t)=t\) puis cherchons \( \theta(t)\) de telle sorte à avoir
    \begin{equation}        \label{EqICrDSe}
        \cos\theta(t)-\sin\theta(t)=t^2.
    \end{equation}
    Le mieux serait de résoudre cette équation pour trouver \( \theta(t)\). Mais en réalité il n'est pas nécessaire de résoudre : montrer qu'il existe une solution suffit. Nous pouvons supposer que \( t^2<1\). Pour \( \theta=\pi/4\) nous avons \( \cos(\theta)-\sin(\theta)=0\) et pour \( \theta=0\) nous avons \( \cos(\theta)-\sin(\theta)=1\). Le théorème des valeurs intermédiaires nous enseigne alors qu'il existe une valeur de \( \theta\) qui résout l'équation \eqref{EqICrDSe}.

    % Laisser en deux lignes, parce que la seconde référence est ok vers le futur.
    Pour être rigoureux, nous devons aussi montrer que la fonction \( \theta(t)\) est continue. Pour cela il faudrait utiliser le théorème de la fonction implicite~\ref{ThoRYN_jvZrZ}.
    Nous verrons dans l'exemple~\ref{ExmeASDLAf} comment s'en sortir sans théorème de la fonction implicite, au prix de plus de calculs.
\end{example}

Les coordonnées polaires sont données par le difféomorphisme
\begin{equation}
	\begin{aligned}
		g\colon \mathopen]0,\infty\mathclose[\times\mathopen]0,2\pi\mathclose[ &\to\eR^2\setminus D\\
		(r,\theta)&\mapsto \big( r\cos(\theta),r\sin(\theta) \big)
	\end{aligned}
\end{equation}
où $D$ est la demi-droite $y=0$, $x\geq 0$. Le fait que les coordonnées polaires ne soient pas un difféomorphisme sur tout $\eR^2$ n'est pas un problème pour l'intégration parce que le manque de difféomorphisme est de mesure nulle dans $\eR^2$. Le jacobien est donné par
\begin{equation}
	Jg=\det\begin{pmatrix}
	\partial_rx	&	\partial_{\theta}x	\\
	\partial_ry	&	\partial_{\theta}y
\end{pmatrix}=\det\begin{pmatrix}
	\cos(\theta)	&	-r\sin(\theta)	\\
	\sin(\theta)	&	r\cos(\theta)
\end{pmatrix}=r.
\end{equation}

La fonction qui donne les coordonnées polaires est
\begin{equation}
    \begin{aligned}
        \varphi\colon \eR^+\times\mathopen] 0 , 2\pi \mathclose[&\to \eR^2 \\
        (r,\theta)&\mapsto\begin{pmatrix}
            r\cos(\theta)    \\
            r\sin(\theta)
        \end{pmatrix}.
    \end{aligned}
\end{equation}
Son Jacobien vaut
\begin{equation}
    J_{\varphi}(r,\theta)=\det\begin{pmatrix}
        \frac{ \partial x(r,\theta) }{ \partial r }    &   \frac{ \partial x(r,\theta) }{ \partial \theta }    \\
        \frac{ \partial y(r,\theta) }{ \partial r }    &   \frac{ \partial y(r,\theta) }{ \partial \theta }
    \end{pmatrix}=
    \begin{vmatrix}
        \cos(\theta)    &   -r\sin(\theta)    \\
        \sin(\theta)    &   r\cos(\theta)
    \end{vmatrix}=r.
\end{equation}

\begin{proposition}     \label{PROPooFLUAooDsyMXO}
    Soit la fonction
    \begin{equation}
        \begin{aligned}
        T\colon \mathopen] O , +\infty \mathclose[\times \eR&\to \eR^2\setminus\{(0,0)\} \\
            (r,\theta)&\mapsto \big( r\cos(\theta),r\sin(\theta) \big).
        \end{aligned}
    \end{equation}
    \begin{enumerate}
        \item
            Elle est surjective.
        \item
        Pour tout \( a\in \eR\), l'application \( T\) est bijective sur la bande \( \mathopen] 0 , +\infty \mathclose[\times \mathopen[ a-\pi , a+\pi \mathclose[\).
        \item
            Si \( a=0\), la fonction inverse est donnée par
            \begin{equation}
                T^{-1}(x,y)=\big( \sqrt{ x^2+y^2 },\arctg(y/x) \big).
            \end{equation}
    \end{enumerate}
\end{proposition}

    Soit $P=(x,y)$ un élément dans $\eR^2$, on dit que $r=\sqrt{x^2+y^2}$ est le rayon de $P$ et que $\theta=\arctg (y/x) $ est son argument principal. L'origine ne peut pas être décrite en coordonnées polaires parce que si son rayon est manifestement zéro, on ne peut pas lui associer une valeur univoque de l'angle $\theta$.

\begin{example}
L'équation du cercle de rayon $a$ et centre $(0, 0)$ en coordonnées polaires est $r=a$.
\end{example}

\begin{example}
	Une équation possible pour la demi-droite $x=y$, $x>0$,  est $\theta=\pi/4$.
\end{example}
%///////////////////////////////////////////////////////////////////////////////////////////////////////////////////////////
\subsubsection{Transformation inverse : théorie}
%///////////////////////////////////////////////////////////////////////////////////////////////////////////////////////////

Voyons la question inverse : comment retrouver $r$ et $\theta$ si on connait $x$ et $y$ ? Tout d'abord,
\begin{equation}
	r=\sqrt{x^2+y^2}
\end{equation}
parce que la coordonnée $r$ est la distance entre l'origine et $(x,y)$. Comment trouver l'angle ? Nous supposons $(x,y)\neq (0,0)$. Si $x=0$, alors le point est sur l'axe vertical et nous avons
\begin{equation}
	\theta=\begin{cases}
		\pi/2	&	\text{si }y>0\\
		3\pi/2	&	 \text{si }y<0
	\end{cases}
\end{equation}
Notez que si $y<0$, conformément à notre convention $\theta\geq 0$, nous avons noté $\frac{ 3\pi }{2}$ et non $-\frac{ \pi }{ 2 }$.

Supposons maintenant le cas général avec $x\neq 0$. Les équations \eqref{EqrthetaxyPoal} montrent que
\begin{equation}
	\tan(\theta)=\frac{ y }{ x }.
\end{equation}
Nous avons donc
\begin{equation}
	\theta=\tan^{-1}\left( \frac{ y }{ x } \right).
\end{equation}
La fonction inverse de la fonction tangente est celle définie plus haut.

%///////////////////////////////////////////////////////////////////////////////////////////////////////////////////////////
\subsubsection{Transformation inverse : pratique}
%///////////////////////////////////////////////////////////////////////////////////////////////////////////////////////////

Le code suivant utilise \href{http://www.sagemath.org}{Sage}.

\lstinputlisting{tex/frido/calculAngle.py}

Son exécution retourne :
\begin{verbatim}
(sqrt(2), 1/4*pi)
(sqrt(5), pi - arctan(1/2))
(6, 1/6*pi)
\end{verbatim}
Notez que ce sont des valeurs \emph{exactes}. Ce ne sont pas des approximations, Sage travaille de façon symbolique.

%///////////////////////////////////////////////////////////////////////////////////////////////////////////////////////////
\subsubsection{Coordonnées polaires : dérivées partielles}
%///////////////////////////////////////////////////////////////////////////////////////////////////////////////////////////

Le changement de coordonnées pour les polaires est la fonction
\begin{equation}
    f\begin{pmatrix}
        r    \\
        \theta
    \end{pmatrix}=\begin{pmatrix}
        x    \\
        y
    \end{pmatrix}=\begin{pmatrix}
        r\cos\theta    \\
        r\sin\theta
    \end{pmatrix}.
\end{equation}
Considérons une fonction $g$ sur $\eR^2$, et définissons la fonction $\tilde g$ par
\begin{equation}
    \tilde g(r,\theta)=g(r\cos\theta,r\sin\theta).
\end{equation}
La formule \eqref{EqDerCompofg} permet de trouver les dérivées partielles de $g$ par rapport à $r$ et $\theta$ en termes de celles par rapport à $x$ et $y$ de $g$.

Pour faire le lien avec les notations du point précédent, nous avons
\begin{equation}
    \begin{aligned}[]
        f_1(r,\theta)&=r\cos(\theta)\\
        f_2(r,\theta)&=r\sin(\theta)\\
        (x_1,x_2)&\to(r,\theta)\\
        (y_1,y_2)&\to(x,y).
    \end{aligned}
\end{equation}
Nous avons donc
\begin{equation}
    \begin{aligned}[]
        \frac{ \partial \tilde g }{ \partial r }(r,\theta)&=\sum_{i=1}^2\frac{ \partial g }{ \partial x_i }\big( f(r,\theta) \big)\frac{ \partial f_i }{ \partial r }(r,\theta)\\
        &=\frac{ \partial g }{ \partial x }(r\cos\theta,r\sin\theta)\frac{ \partial \big( r\cos\theta \big) }{ \partial r }(r,\theta)\\
        &\quad+\frac{ \partial g }{ \partial y }(r\cos\theta,r\sin\theta)\frac{ \partial \big( r\sin\theta\big) }{ \partial r }(r,\theta)\\
        &=\cos\theta\frac{ \partial g }{ \partial x }(r\cos\theta,r\sin\theta)+\sin\theta\frac{ \partial g }{ \partial y }(r\cos\theta,r\sin\theta).
    \end{aligned}
\end{equation}

Prenons par exemple $g(x,y)=\frac{1}{ x^2+y^2 }$. Étant donné que
\begin{equation}
    \frac{ \partial g }{ \partial x }=\frac{ -2x }{ (x^2+y^2)^2 },
\end{equation}
nous avons
\begin{equation}
    \frac{ \partial g }{ \partial x }(r\cos\theta,r\sin\theta)=\frac{ -2\cos\theta }{ r^3 }.
\end{equation}
En utilisant la formule,
\begin{equation}
    \frac{ \partial \tilde g }{ \partial r }(r,\theta)=\cos(\theta)\left( \frac{ -2\cos\theta }{ r^3 } \right)+\sin(\theta)\left( \frac{ -2\sin\theta }{ r^3 } \right)=-\frac{ 2 }{ r^3 }.
\end{equation}
Nous pouvons vérifier directement que cela est correct. En effet
\begin{equation}
    \tilde g(r,\theta)=g(r\cos\theta,r\sin\theta)=\frac{1}{ r^2 },
\end{equation}
dont la dérivée par rapport à $r$ vaut $-2/r^3$.

En ce qui concerne la dérivée par rapport à $\theta$, nous avons
\begin{equation}
    \begin{aligned}[]
    \frac{ \partial \tilde g }{ \partial \theta }&=\frac{ \partial g }{ \partial x }(r\cos\theta,r\sin\theta)\frac{ \partial \big( r\cos(\theta) \big) }{ \partial \theta }+\frac{ \partial g }{ \partial y }(r\cos\theta,r\sin\theta)\frac{ \partial \big( r\sin(\theta) \big) }{ \partial \theta }\\
    &=\left( \frac{ -2\cos\theta }{ r^3 } \right)(-r\sin\theta)+\left( \frac{ -2\sin\theta }{ r^3 } \right)(r\cos\theta)\\
    &=0.
    \end{aligned}
\end{equation}

En résumé et avec quelques abus de notation :
\begin{equation}
    \begin{aligned}[]
        \frac{ \partial \tilde g }{ \partial r }&=\cos(\theta)\frac{ \partial g }{ \partial x }+\sin(\theta)\frac{ \partial g }{ \partial y }\\
        \frac{ \partial \tilde g }{ \partial \theta }&=-r\sin(\theta)\frac{ \partial g }{ \partial x }+r\cos(\theta)\frac{ \partial g }{ \partial y }\\
    \end{aligned}
\end{equation}

%--------------------------------------------------------------------------------------------------------------------------- 
\subsection{Coordonnées cylindriques}
%---------------------------------------------------------------------------------------------------------------------------

Les \defe{coordonnées cylindriques}{coordonnées!cylindrique} sont un perfectionnement des coordonnées polaires. Il s'agit simplement de donner le point $(x,y,z)$ en faisant la conversion $(x,y)\mapsto(r,\theta)$ et en gardant le $z$. Les formules de passage sont
\begin{subequations}
	\begin{numcases}{}
		x=r\cos(\theta)\\
		y=r\sin(\theta)\\
		z=z.
	\end{numcases}
\end{subequations}

Soit $T$ la fonction de $]0, +\infty[\times \eR^2$ dans $\eR^3\setminus\{(0,0,0)\}$ définie par
\begin{equation}
  \begin{array}{lccc}
    T: &]0, +\infty[\times \eR\times \eR & \to & \eR^3\setminus\{(0,0,0)\}\\
 & (r, \theta, z)&\mapsto& (r\cos \theta, r \sin \theta, z),
  \end{array}
\end{equation}
Cette fonction est surjective. Elle est bijective sur chaque bande de la forme  $]0, +\infty[\times [a-\pi,a+\pi[\times \eR$, $a$ dans $\eR$. Il n'y a presque rien de nouveau par rapport aux coordonnées polaires. Les coordonnées  cylindriques sont intéressantes si on décrit un objet invariant par rapport aux rotations autour de l'axe des $z$.

\begin{example}
Il faut savoir ce que décrivent les équations les plus simples en coordonnées cylindriques,
\begin{itemize}
\item $r\leq a$, pour $a$ constant dans  $]0, +\infty[$, est le cylindre de hauteur infinie qui a pour axe l'axe des $z$ et pour base le disque de rayon $a$ centré à l'origine,
\item $r= a$ est  la surface du cylindre,
\item $\theta = b$ est un demi-plan ouvert et sa fermeture contient l'axe des $z$,
\item $z=c$ est un plan parallèle au plan $x$-$y$.
\end{itemize}
\end{example}

\begin{example}
  Un demi-cône qui a  son sommet en l'origine et  pour axe l'axe des $z$ est décrit par $z=d r$.  Si $d$ est positif  il s'agit  de la moitié supérieure du cône, si $d<0$ de la moitié inférieure.
\end{example}

\begin{example}
 De même,  la sphère de rayon $a$ et centrée à l'origine est l'assemblage des calottes $z=\sqrt{a^2-r^2}$ et $z=-\sqrt{a^2-r^2}$.
\end{example}

En ce qui concerne les coordonnées cylindriques, le Jacobien est donné par
\begin{equation}
    J(r,\theta,z)=\begin{vmatrix}
        \frac{ \partial x }{ \partial r }    &   \frac{ \partial x }{ \partial \theta }    &   \frac{ \partial x }{ \partial z }    \\
        \frac{ \partial y }{ \partial r }    &   \frac{ \partial y }{ \partial \theta }    &   \frac{ \partial y }{ \partial z }    \\
        \frac{ \partial z }{ \partial r }    &   \frac{ \partial z }{ \partial \theta }    &   \frac{ \partial z }{ \partial z }
    \end{vmatrix}=
    \begin{vmatrix}
        \cos\theta    &   -r\sin\theta    &   0    \\
        \sin\theta    &   r\cos\theta    &   0    \\
        0    &   0    &   1
    \end{vmatrix}=r.
\end{equation}
Nous avons donc $dx\,dy\,dz=r\,dr\,d\theta\,dz$.

\begin{subequations}
    \begin{numcases}{}
        x=r\cos\theta\\
        y=r\sin\theta\\
        z=z
    \end{numcases}
\end{subequations}
avec \( r\in\mathopen] 0 , \infty \mathclose[\), \( \theta\in\mathopen[ 0 , 2\pi [\) et \( z\in\eR\). Le jacobien vaut \( r\).

%++++++++++++++++++++++++++++++++++++++++++++++++++++++++++++++++++++++++
\subsection{Coordonnées sphériques}
%++++++++++++++++++++++++++++++++++++++++++++++++++++++++++++++++++++++++

Soit $T$ la fonction de $]0, +\infty[\times \eR^2$ dans $\eR^3\setminus\{(0,0,0)\}$ définie par
\begin{equation}
  \begin{array}{lccc}
    T: &]0, +\infty[\times \eR\times \eR & \to & \eR^3\setminus\{(0,0,0)\}\\
 & (\rho, \theta, \phi)&\mapsto& (\rho\cos \theta\sin \phi, \rho \sin \theta\sin \phi, \rho\cos \phi),
  \end{array}
\end{equation}
Cette fonction est surjective. Elle est bijective sur chaque bande de la forme  $]0, +\infty[\times [a-\pi,a+\pi[\times [b-\pi/2, b+\pi/2[$, $a$ et $b$ dans $\eR$.  Si $a =0$ et $b=-\pi/2$ la fonction inverse $T^{-1}$ est donnée donnée
\begin{equation}
  \begin{array}{lccc}
    T: &\eR^3\setminus\{(0,0,0)\} & \to & ]0, +\infty[\times [-\pi,\pi[\times [0, \pi[\\
 & (x,y,z)&\mapsto& \left(\sqrt{x^2+y^2+z^2}, \arctg \frac{y}{x}, \arccos \left(\frac{z}{\sqrt{x^2+y^2+z^2}}\right)\right).
  \end{array}
\end{equation}
Soit $ P$ un point dans $\eR^3$. L'angle $\phi$ est l'angle entre le demi-axe positif des $z$ et le vecteur $\overrightarrow{OP}$, $\rho$ est la norme de $\overrightarrow{OP}$ et $\theta$ est l'argument en coordonnées polaires de la projection de $\overrightarrow{OP}$ sur le plan $x$-$y$.

\begin{remark}
	Dans la littérature, les angles $\theta$ et $\phi$ sont parfois inversés (voire, changent de nom, par exemple $\varphi$ au lieu de $\phi$). Il faut donc être très prudent lorsqu'on veut utiliser dans un cours des formules données dans un autre cours.
\end{remark}

\begin{example}
Il faut connaitre le sens des équations plus simples,
\begin{itemize}
\item $\rho\leq a$, pour $a$ constant dans  $]0, +\infty[$, est la boule fermée de rayon $a$ centrée à l'origine,
\item $\rho= a$ est  la sphère de rayon $a$ centrée à l'origine,
\item $\theta = b$ est un demi-plan ouvert et sa fermeture contient l'axe des $z$,
\item $\phi= c$ est un demi-cône qui a  son sommet à l'origine et  pour axe l'axe des $z$.  Si $c$ est positif  il s'agit  de la moitié supérieure du cône, si $d<0$ de la moitié inférieure.
\end{itemize}
 \end{example}

Les \defe{coordonnées sphériques}{coordonnées!sphériques} sont ce qu'on appelle les «méridiens» et «longitudes» en géographie. Les formules de transformation sont
\begin{subequations}		\label{SubEqsCoordSphe}
	\begin{numcases}{}
		x=\rho\sin(\theta)\cos(\varphi)\\
		y=\rho\sin(\theta)\sin(\varphi)\\
		z=\rho\cos(\theta)
	\end{numcases}
\end{subequations}
avec $0\leq\theta\leq\pi$ et $0\leq\varphi<2\pi$.

\begin{remark}
	Attention : d'un livre à l'autre les conventions sur les noms des angles changent. N'essayez donc pas d'étudier par cœur des formules concernant les coordonnées sphériques trouvées autre part. Par exemple sur le premier dessin de \href{http://fr.wikipedia.org/wiki/Coordonnées_sphériques}{wikipédia}, l'angle $\varphi$ est noté $\theta$ et l'angle $\theta$ est noté $\Phi$. Mais vous noterez que sur cette même page, les conventions de noms de ces angles changent plusieurs fois.
\end{remark}

Les coordonnées sphériques sont données par
\begin{equation}		\label{EqChmVarSpherique}
	\left\{
\begin{array}{lllll}
x=r\cos\theta\sin\varphi	&			&r\in\mathopen] 0 , \infty \mathclose[\\
y=r\sin\theta\sin\varphi	&	\text{avec}	&\theta\in\mathopen] 0 , 2\pi \mathclose[\\
z=r\cos\varphi			&			&\phi\in\mathopen] 0 , \pi \mathclose[.
\end{array}
\right.
\end{equation}
Le jacobien associé est $Jg(r,\theta,\varphi)=-r^2\sin\varphi$. Rappelons que ce qui rentre dans l'intégrale est la valeur absolue du jacobien.

\begin{subequations}
    \begin{numcases}{}
        x=\rho\cos\theta\sin\phi\\
        y=\rho\sin\theta\sin\phi\\
        z=\rho\cos\phi
    \end{numcases}
\end{subequations}
avec \( \rho\in\mathopen] 0 , \infty \mathclose[\), \( \theta\in\mathopen[ 0 , 2\pi [\) et \( \phi\in\mathopen[ 0 , \pi [\). Le jacobien vaut \( -\rho^2\sin(\phi)\).

N'oubliez pas que lorsqu'on effectue un changement de variables dans une intégrale, la \emph{valeur absolue} du jacobien apparaît.

Cependant notre convention de coordonnées sphériques fait venir \( \sin(\phi)\) avec \( \phi\in\mathopen[ 0 , \pi [\); vu que le signe de \( \sin(\phi)\) y est toujours positif, cette histoire de valeur absolue est sans grandes conséquent. Ce n'est pas le cas de toutes les conventions possibles.

%///////////////////////////////////////////////////////////////////////////////////////////////////////////////////////////
    \subsubsection{Coordonnées sphériques : inverse}
%///////////////////////////////////////////////////////////////////////////////////////////////////////////////////////////

Trouvons le changement inverse, c'est-à-dire trouvons $\rho$, $\theta$ et $\varphi$ en termes de $x$, $y$ et $z$. D'abord nous avons
\begin{equation}
	\rho=\sqrt{x^2+y^2+z^2}.
\end{equation}
Ensuite nous savons que
\begin{equation}
	\cos(\theta)=\frac{ z }{ \rho }
\end{equation}
détermine de façon unique\footnote{Le problème $\rho=0$ ne se pose pas; pourquoi ?} un angle $\theta\in\mathopen[ 0 , \pi \mathclose]$. Dès que $\rho$ et $\theta$ sont connus, nous pouvons poser $r=\rho\sin\theta$ et alors nous nous trouvons avec les équations
\begin{subequations}
	\begin{numcases}{}
		x=r\cos(\varphi)\\
		y=r\sin(\varphi),
	\end{numcases}
\end{subequations}
qui sont similaires à celles déjà étudiées dans le cas des coordonnées polaires.

% This is part of (everything) I know in mathematics
% Copyright (c) 2011-2019
%   Laurent Claessens
% See the file fdl-1.3.txt for copying conditions.


%+++++++++++++++++++++++++++++++++++++++++++++++++++++++++++++++++++++++++++++++++++++++++++++++++++++++++++++++++++++++++++
\section{Calcul de limites}
%+++++++++++++++++++++++++++++++++++++++++++++++++++++++++++++++++++++++++++++++++++++++++++++++++++++++++++++++++++++++++++

Beaucoup de techniques de calcul de limites fonctionnent bien avec les fonctions trigonométriques, entre autres grâce à l'utilisation des coordonnées polaires de la proposition~\ref{PROPooFLUAooDsyMXO}. De plus, le théorème de la fonction implicite Nous en voyons quelques exemples à présent.

\begin{example}[Limite et prolongement par continuité] \label{ExQWHooGddTLE}
    La fonction
    \begin{equation}
        f(x)=\frac{ \cos(x)-1 }{ x }
    \end{equation}
    n'est pas définie en \( x=0\).

    Nous avons vu dans l'équation \eqref{SUBEQooTTNNooXzApSM} que \( \cos(0)=1\), donc la limite
    \begin{equation}
        \lim_{x\to 0} \frac{ \cos(x)-1 }{ x }
    \end{equation}
    est la limite définissant la dérivée de cosinus en \( 0\) (ici, le \( x\) joue le rôle de \( \epsilon\)). Le lemme~\ref{LEMooBBCAooHLWmno} nous donne la dérivée du cosinus comme étant le sinus. Nous avons donc :
    \begin{equation}
        \lim_{x\to 0} \frac{ \cos(x)-1 }{ x }=\sin(0)=0,
    \end{equation}
    et nous définissons le prolongement par continuité :
    \begin{equation}
        \tilde f(x)=\begin{cases}
            \frac{ \cos(x)-1 }{ x }    &   \text{si } x\neq 0\\
            0    &    \text{sinon}.
        \end{cases}
    \end{equation}

    Encore une fois, le graphe de la fonction \(\tilde f\) ne présente aucune particularité autour de \( x=0\).
    \begin{center}
        \input{auto/pictures_tex/Fig_RPNooQXxpZZ.pstricks}
    \end{center}
\end{example}

\begin{example}[Un calcul heuristique de limite]        \label{EXooINLRooPzRWEA}
    Soit à calculer la limite suivante :
    \begin{equation}
        \lim_{x\to 0} \frac{  e^{-2\cos(x)+2}\sin(x) }{ \sqrt{ e^{2\cos(x)+2}}-1 }.
    \end{equation}
    La stratégie que nous allons suivre pour calculer cette limite est de développer certaines parties de l'expression en série de Taylor, afin de simplifier l'expression. La première chose à faire est de remplacer $ e^{y(x)}$ par $1+y(x)$ lorsque $y(x)\to 0$. La limite devient
    \begin{equation}
        \lim_{x\to 0} \frac{ \big( -2\cos(x)+3 \big)\sin(x) }{ \sqrt{-2\cos(x)+2} }.
    \end{equation}
    Nous allons maintenant remplacer $\cos(x)$ par $1$ au numérateur et par $1-x^2/2$ au dénominateur. Pourquoi ? Parce que le cosinus du dénominateur est dans une racine, donc nous nous attendons à ce que le terme de degré deux du cosinus donne un degré un en dehors de la racine, alors que du degré un est exactement ce que nous avons au numérateur : le développement du sinus commence par $x$.

    Nous calculons donc
    \begin{equation}
        \begin{aligned}[]
            \lim_{x\to 0} \frac{ \sin(x) }{ \sqrt{-2\left( 1-\frac{ x^2 }{ 2 } \right)+2} }=\lim_{x\to 0} \frac{ \sin(x) }{ x }=1.
        \end{aligned}
    \end{equation}
    Tout ceci n'est évidemment pas très rigoureux, mais en principe vous avez tous les éléments en main pour justifier les étapes.
\end{example}

%---------------------------------------------------------------------------------------------------------------------------
\subsection{Méthode des coordonnées polaires}
%---------------------------------------------------------------------------------------------------------------------------
\label{SUBSECooWCGMooPrXSpt}

La proposition suivante exprime la définition de la limite en d'autres termes, et va être pratique dans le calcul de certaines limites.
\begin{proposition}		\label{PropMethodePolaire}
	Soit $f\colon D\subset\eR^m\to \eR^n$, $a$ un point d'accumulation de $D$ et $\ell\in \eR^n$. Nous définissons
	\begin{equation}
		E_r=\{ f(x)\tq x\in B(a,r)\cap D \},
	\end{equation}
	et
	\begin{equation}
		s_r=\sup\{ \| v-\ell \|\tq v\in E_r \}.
	\end{equation}
	Alors nous avons $\lim_{x\to a} f(x)=\ell$ si et seulement si $\lim_{r\to 0} s_r=0$.
\end{proposition}

Dans cette proposition, $E_r$ représente l'ensemble des valeurs atteintes par $f$ dans un rayon $r$ autour de $a$. Le nombre $s_r$ sélectionne, parmi toutes ces valeurs, celle qui est la plus éloignée de $\ell$ et donne la distance. En d'autres termes, $s_r$ est la distance maximale entre $f(x)$ et $\ell$ lorsque $x$ est à une distance au maximum $r$ de $a$.

Lorsque nous avons affaire à une fonction $f\colon \eR^2\to \eR$, cette proposition nous permet de calculer facilement les limites en passant aux coordonnées polaires.

\begin{example}		\label{ExempleMethodeTrigigi}
	Reprenons la fonction de l'exemple~\ref{ExFNExempleMethodeTrigigi}:
	\begin{equation}
		f(x,y)=\frac{ xy }{ x^2+y^2 }.
	\end{equation}
	Son domaine est $\eR^2\setminus\{ (0,0) \}$. Nous voulons calculer $\lim_{(x,y)\to(0,0)}f(x,y)$. Écrivons la définition de $E_r$~:
	\begin{equation}
		E_r=\{ f(x,y)\tq (x,y)\in B\big( (0,0),r \big) \}.
	\end{equation}
	Les points de la boule sont, en coordonnées polaires, les points de la forme $(\rho,\theta)$ avec $\rho<r$. La chose intéressante est que $f(\rho,\theta)$ est relativement simple (plus simple que la fonction départ). En effet en remplaçant tous les $x$ par $\rho\cos(\theta)$ et tous les $y$ par $\rho\sin(\theta)$, et en utilisant le fait que $\cos^2(\theta)+\sin^2(\theta)=1$, nous trouvons
	\begin{equation}		\label{Eq2807fpolairerhodeuxcossin}
		f(\rho,\theta)=\frac{ \rho^2\cos(\theta)\sin(\theta) }{ \rho^2 }=\cos(\theta)\sin(\theta).
	\end{equation}
	Cela signifie que
	\begin{equation}
		E_r=\{ \cos(\theta)\sin(\theta)\tq\theta\in\mathopen[ 0 , 2\pi [ \}.
	\end{equation}
	Prenons $\ell$ quelconque. Le nombre $s_r$ est le supremum des
	\begin{equation}
		\| \ell-\cos(\theta)\sin(\theta) \|
	\end{equation}
	lorsque $\theta$ parcours $\mathopen[ 0 , 2\pi \mathclose]$. Nous ne sommes pas obligés calculer la valeur exacte de $s_r$. Ce qui compte ici est que $s_r$ ne vaut certainement pas zéro, et ne dépend pas de $r$. Donc il est impossible d'avoir $\lim_{r\to 0} s_r=0$, et la fonction donnée n'a pas de limite en $(0,0)$.
\end{example}

Nous pouvons retenir cette règle pour calculer les limites lorsque $(x,y)\to(0,0)$ de fonctions $f\colon \eR^2\to \eR$ :
\begin{enumerate}
	\item
		passer en coordonnées polaires, c'est-à-dire remplacer $x$ par $\rho\cos(\theta)$ et $y$ par $\rho\sin(\theta)$;
	\item
		nous obtenons une fonction $g$ de $\rho$ et $\theta$. Si la limite $\lim_{r\to 0} g(r,\theta)$ n'existe pas ou dépend de $\theta$, alors la fonction n'a pas de limite. Si on peut majorer $g$ par une fonction ne dépendant pas de $\theta$, et que cette fonction a une limite lorsque $r\to 0$, alors cette limite est la limite de la fonction.
\end{enumerate}

La vraie difficulté de la technique des coordonnées polaires est de trouver le supremum de $E_r$, ou tout au moins de montrer qu'il est borné par une fonction qui a une limite qui ne dépend pas de $\theta$. Une des situations classiques dans laquelle c'est facile est lorsque la fonction se présente comme une fonction de $r$ multiplié par une fonction de $\theta$.

\begin{example}		\label{Exemplexyxsqysq}
	Soit à calculer la limite
	\begin{equation}
		\lim_{(x,y)\to(0,0)}xy\left( \frac{ x^2-y^2 }{ x^2+y^2 }\right).
	\end{equation}
	Le passage aux coordonnées polaires donne
	\begin{equation}
		f(r,\theta)=r^2\sin\theta\cos\theta(\cos^2\theta-\sin^2\theta).
	\end{equation}
	Déterminer le supremum de cela est relativement difficile. Mais nous savons que de toutes façons, la quantité $\sin\theta\cos\theta(\cos^2\theta-\sin^2\theta)$ est bornée par $1$. Donc
	\begin{equation}
		\| f(r,\theta) \|\leq r^2.
	\end{equation}
	Maintenant la règle de l'étau montre que $\lim_{(x,y)\to(0,0)}f(x,y)$ est zéro.

	La situation vraiment gênante serait celle avec une fonction de $\theta$ qui risque de s'annuler dans un dénominateur.
\end{example}

L'exemple~\ref{EXooSDHDooJzDioW} donnera un cas où la méthode fonctionne plus difficilement. Entre autres parce qu'il utilisera en même temps la méthode des chemins et celle des coordonnées polaires.

\begin{example}\label{ExmeASDLAf}
	Considérons fonction
	\begin{equation}
		f(x,y)=\frac{ x^2+y^2 }{ x-y }.
	\end{equation}
	Une mauvaise idée pour prouver que la limite n'existe pas pour $(x,y)\to(0,0)$ est de considérer le chemin $(t,t)$. En effet, la fonction n'existe pas sur ce chemin. Or la méthode des chemins parle uniquement de chemins contenus dans le domaine de la fonction.

	Nous prouvons que la limite n'existe pas en trouvant des chemins le long desquels les limites sont différentes. Si nous essayons le chemin \( (t,kt)\) avec \( k\) constant, nous trouvons
    \begin{equation}
        f(t,kt)=\frac{ t(1+k^2) }{ 1-k }.
    \end{equation}
    La limite \( t\to 0\) est hélas toujours \( 0\). Nous ne pouvons donc pas conclure.

    Nous allons maintenant utiliser la même technique que celle utilisée en coordonnées polaires. Vous noterez que dans ce cas, travailler en cartésiennes donne lieu à des calculs plus longs.  L'astuce consiste à prendre \( k\) non constant et à chercher par exemple \( k(t)\) de façon à avoir
    \begin{equation}
        \frac{ 1+k(t)^2 }{ 1-k(t) }=\frac{1}{ t }.
    \end{equation}
    Avec une telle fonction \( k\), la fonction \( t\mapsto f(t,tk(t))\) serait la constante \( 1\). L'équation à résoudre pour \( k\) est
    \begin{equation}
        tk^2+k+(t-1)=0,
    \end{equation}
    et les solutions sont
    \begin{equation}
        k(t)=\frac{ -1\pm\sqrt{1-4t(t-1)} }{ 2t }.
    \end{equation}
    Nous proposons donc les chemins
    \begin{equation}
        \begin{pmatrix}
            x    \\
            y
        \end{pmatrix}=\begin{pmatrix}
            t    \\
            \frac{ -1\pm\sqrt{1-4t(t-1)}    }{2}
        \end{pmatrix}
    \end{equation}
    Nous devons vérifier deux points. D'abord que ce chemin est bien défini, et ensuite que \( tk(t)\) tend bien vers zéro lorsque \( t\to 0\) (sinon \( (t,k(t)t)\)) n'est pas un chemin passant par \( (0,0)\). Lorsque \( t\) est petit, ce qui se trouve sous la racine est proche de \( 1\) et ne pose pas de problèmes. Ensuite,
    \begin{equation}
        \lim_{t\to 0} tk(t)=\frac{ -1\pm 1 }{ 2 }.
    \end{equation}
    En choisissant le signe \( +\), nous trouvons un chemin qui nous convient.

    Ce que nous avons prouvé est que
    \begin{equation}
        f\left( t,   \frac{ -1+\sqrt{1-4t(t-1)}    }{2}\right)=1
    \end{equation}
    pour tout \( t\). Le long de ce chemin, la limite de \( f\) est donc \( 1\). Cette limite est différente des limites obtenues le long de chemins avec \( k\) constant. La limite \( \lim_{(x,y)\to (0,0)} f(x,y)\) n'existe donc pas.
\end{example}

\begin{example}\label{seno}
	Considérons la fonction (figure~\ref{LabelFigsenotopologo})

	\begin{equation}
		f(x,y)=\begin{cases}
			\sqrt{x^2+y^2}\sin\frac{1}{ x^2+y^2 }	&	\text{si }(x,y)\neq(0,0)\\
			0	&	 \text{si }(x,y)=(0,0),
		\end{cases}
	\end{equation}
    et cherchons la limite $(x,y)\to(0,0)$. Le passage en coordonnées polaires\footnote{Proposition~\ref{PROPooFLUAooDsyMXO}.} donne
	\begin{equation}		\label{EqFoncRho2907}
		f(\rho,\theta)=\rho\sin\frac{1}{ \rho }.
	\end{equation}
	Pour calculer la limite de cela lorsque $\rho\to 0$, nous remarquons que
	\begin{equation}
		0\leq|\rho\sin\frac{1}{ \rho }|\leq\rho
	\end{equation}
	parce que $\sin(\frac{1}{ \rho })\leq 1$ quel que soit $\rho$. Or évidemment $\lim_{\rho\to 0} \rho=0$, donc la limite de la fonction \eqref{EqFoncRho2907} est zéro et ne dépend pas de $\theta$. Nous en concluons que $\lim_{(x,y)\to(0,0)}f(x,y)=0$.
\end{example}
\newcommand{\CaptionFigsenotopologo}{La fonction de l'exemple~\ref{seno}.}
\input{auto/pictures_tex/Fig_senotopologo.pstricks}

%---------------------------------------------------------------------------------------------------------------------------
\subsection{Méthode du développement asymptotique}
%---------------------------------------------------------------------------------------------------------------------------
\label{SUBSECooRAKKooAnpvkE}

Nous savons  que nous pouvons développer certaines fonctions en série grâce au développement de Taylor (théorème~\ref{ThoTaylor}). Lorsque nous avons une limite à calculer, nous pouvons remplacer certaines parties de la fonction à traiter par la formule \eqref{subeqfTepseqb}. Cela est très utile pour comparer des fonctions trigonométrique à des polynômes.

\begin{lemma}       \label{LEMooZYNEooYkwsWD}
    Nous avons la limite
    \begin{equation}
        \lim_{x\to 0} \frac{ \sin(x) }{ x }=1.
    \end{equation}
\end{lemma}

\begin{proof}
    Une manière de prouver cela est d'écrire
    \begin{equation}
		\sin(x)=x+h(x)
	\end{equation}
	avec $h\in o(x)$, c'est-à-dire $\lim_{x\to 0} h(x)/x=0$. Alors nous avons
	\begin{equation}
		\lim_{x\to 0} \frac{ \sin(x) }{ x }=\lim_{x\to 0} \frac{ x+h(x) }{ x }=\lim_{x\to 0} \frac{ x }{ x }+\lim_{x\to 0} \frac{ h(x) }{ x }=1.
	\end{equation}
\end{proof}

L'utilisation de la proposition~\ref{PropLimCompose} permet d'utiliser cette technique dans le cadre de limites à plusieurs variables. Reprenons le lemme \ref{LEMooZYNEooYkwsWD} un tout petit peu modifié :

\begin{lemma}       \label{LEMooSFALooVRBdNb}
    Pour tout \( x>0\) nous avons \( \sin(x)<x\).
\end{lemma}

\begin{proof}
    Nous posons \( f(x)=x-\sin(x)\). Cette fonction vérifie \( f(0)=0\) et
    \begin{equation}
        f'(x)=1-\cos(x).
    \end{equation}
    Vu que \( | \cos(x) |\leq 1\), nous avons toujours \( f'(x)\geq 0\) et même \( f'(x)>0\) pour \( x\in \mathopen] 0 , \delta \mathclose]\). Donc \( f\) est au moins strictement croissante sur \( \mathopen] 0 , \delta \mathclose]\) et ensuite strictement croissante presque partout.
\end{proof}

\begin{example}
	Soit à calculer $\lim_{(x,y)\to(0,0)}f(x,y)$ où
	\begin{equation}
		f(x,y)=\frac{ \sin(xy) }{ xy }.
	\end{equation}
	La première chose à faire est de voir $f$ comme la composée de fonctions $f=f_1\circ f_2$ avec
	\begin{equation}
		\begin{aligned}
			f_1\colon \eR&\to \eR \\
			t&\mapsto \frac{ \sin(t) }{ t }
		\end{aligned}
	\end{equation}
	et
	\begin{equation}
		\begin{aligned}
			f_2\colon \eR^2&\to \eR \\
			(x,y)&\mapsto xy.
		\end{aligned}
	\end{equation}
	Étant donné que $\lim_{(x,y)\to(0,0)}f_2(x,y)=0$, nous avons $\lim_{(x,y)\to(0,0)}f(x,y)=\lim_{t\to 0} f_1(t)=1$.
\end{example}

\begin{example}     \label{EXooETZYooYsKPDJ}
Les dérivées partielles de la fonction $f(x,y)=xy^3+\sin y$ au point $(0,\pi)$ sont
\[
\partial_xf(0,\pi)=\frac{ \partial f }{ \partial x }(0,\pi)=\lim_{\begin{subarray}{l}
    t\to 0\\ t\neq 0
  \end{subarray}} \frac{(t\pi^3+\sin \pi)-(\sin \pi)}{t}= \pi^3,
\]
\[
\partial_yf(0,\pi)=\frac{ \partial f }{ \partial y }(0,\pi)=\lim_{\begin{subarray}{l}
    t\to 0\\ t\neq 0
  \end{subarray}} \frac{0(\pi+t)^3+\sin (t+\pi)-0\cdot \pi^3}{t}= \cos \pi=-1,
\]
\end{example}

%+++++++++++++++++++++++++++++++++++++++++++++++++++++++++++++++++++++++++++++++++++++++++++++++++++++++++++++++++++++++++++
\section{Quelques intégrales avec de la trigonométrie}
%+++++++++++++++++++++++++++++++++++++++++++++++++++++++++++++++++++++++++++++++++++++++++++++++++++++++++++++++++++++++++++
\label{SECooOOPPooZLbaEH}

Le théorème~\ref{THOooUMIWooZUtUSg} manque un peu d'exemples. Nous allons en voir quelques-uns maintenant.

%---------------------------------------------------------------------------------------------------------------------------
\subsection{Changement de variables}
%---------------------------------------------------------------------------------------------------------------------------

Le domaine $E=\{ (x,y)\in\eR^2\tq x^2+y^2<1 \}$ s'écrit plus facilement $E=\{ (r,\theta)\tq r<1 \}$ en coordonnées polaires. Le passage aux coordonnées polaires permet de transformer une intégration sur un domaine rond à une intégration sur le domaine rectangulaire $\mathopen]0,2\pi\mathclose[\times\mathopen]0,1\mathclose[$. La question est évidemment de savoir si nous pouvons écrire
\begin{equation}
    \int_Ef=\int_{0}^{2\pi}\int_0^1f(r\cos\theta,r\sin\theta)drd\theta.
\end{equation}
Hélas, non; la vie n'est pas aussi simple.

\begin{definition}
    Un \defe{difféomorphisme}{difféomorphisme} est une application $g\colon A\to B$ telle que $g$ et $g^{-1}\colon B\to A$ soient de classe $C^1$.
\end{definition}

\begin{theorem}
Soit $g\colon A\to B$ un difféomorphisme. Soient $F\subset B$ un ensemble mesurable et borné et $f\colon F\to \eR$ une fonction bornée et intégrable. Supposons que $g^{-1}(F)$ soit borné et que $Jg$ soit borné sur $g^{-1}(F)$. Alors
\begin{equation}
	\int_Ff(x)dy=\int_{g^{-1}(F)f\big( g(x) \big)}| Jg(x) |dx
\end{equation}
\end{theorem}
Pour rappel, $Jg$ est le déterminant de la matrice \href{http://fr.wikipedia.org/wiki/Matrice_jacobienne}{jacobienne} (aucun lien de \href{http://fr.wikipedia.org/wiki/Jacob}{parenté}) donnée par
\begin{equation}
	Jg=\det\begin{pmatrix}
	\partial_xg_1	&	\partial_yg_1	\\
	\partial_xg_2	&	\partial_tg_2
\end{pmatrix}.
\end{equation}

Comme dans les intégrales simples, il y a souvent moyen de trouver un changement de variables qui simplifie les expressions.  Le domaine $E=\{ (x,y)\in\eR^2\tq x^2+y^2<1 \}$ par exemple s'écrit plus facilement $E=\{ (r,\theta)\tq r<1 \}$ en coordonnées polaires. Le passage aux coordonnées polaires permet de transformer une intégration sur un domaine rond à une intégration sur le domaine rectangulaire $\mathopen]0,2\pi\mathclose[\times\mathopen]0,1\mathclose[$. La question est évidemment de savoir si nous pouvons écrire
\begin{equation}
	\int_Ef=\int_{0}^{2\pi}\int_0^1f(r\cos\theta,r\sin\theta)drd\theta.
\end{equation}
Hélas ce n'est pas le cas. Il faut tenir compte du fait que le changement de base dilate ou contracte certaines surfaces.

Soit $\varphi\colon D_1\subset\eR^2\to D_2\subset \eR^2$ une fonction bijective de classe $C^1$ dont l'inverse est également de classe $C^1$. On désigne par $x$ et $y$ ses composantes, c'est-à-dire que
\begin{equation}
    \varphi(u,v)=\begin{pmatrix}
        x(u,v)    \\
        y(u,v)
    \end{pmatrix}
\end{equation}
avec $(u,v)\in D_1$.

\begin{theorem}     \label{ThoChamDeVarIntDDf}
    Soit une fonction continue $f\colon D_2\to \eR$. Alors
    \begin{equation}
        \int_{\varphi(D_1)}f(x,y)dxdy=\int_{D_1}f\big( x(u,v),y(u,v) \big)| J_{\varphi}(u,v) |dudv
    \end{equation}
    où $J_{\varphi}$ est le Jacobien de $\varphi$ c'est-à-dire
    \begin{equation}
        J_{\varphi}(u,v)=\det\begin{pmatrix}
            \frac{ \partial x }{ \partial u }    &   \frac{ \partial x }{ \partial v }    \\
            \frac{ \partial y }{ \partial u }    &   \frac{ \partial u }{ \partial v }
        \end{pmatrix}.
    \end{equation}
\end{theorem}
Ne pas oublier de prendre la valeur absolue lorsqu'on utilise le Jacobien dans un changement de variables.

%--------------------------------------------------------------------------------------------------------------------------- 
\subsection{Coordonnées polaires}
%---------------------------------------------------------------------------------------------------------------------------

\begin{example}
    Calculons la surface du disque $D$ de rayon $R$. Nous devons calculer
    \begin{equation}
        \int_Ddxdy.
    \end{equation}
    Pour passer au polaires, nous savons que le disque est décrit par
    \begin{equation}
        D=\{ (r,\theta)\tq 0\leq r\leq R,0\leq\theta\leq 2\pi \}.
    \end{equation}
    Nous avons donc
    \begin{equation}
        \int_Ddxdy=\int_{D}r\,drd\theta=\int_0^{2\pi}\int_0^Rr\,drd\theta=2\pi\int_0^Rr\,dr=\pi R^2.
    \end{equation}
\end{example}

\begin{example}     \label{ExpmfDtAtV}
    Montrons comment intégrer la fonction $f(x,y)=\sqrt{1-x^2-y^2}$ sur le domaine délimité par la droite $y=x$ et le cercle $x^2+y^2=y$, représenté sur la figure~\ref{LabelFigHFAYooOrfMAA}. Pour trouver le centre et le rayon du cercle $x^2+y^2=y$, nous commençons par écrire $x^2+y^2-y=0$, et ensuite nous reformons le carré : $y^2-y=(y-\frac{ 1 }{2})^2-\frac{1}{ 4 }$.

\newcommand{\CaptionFigHFAYooOrfMAA}{Passage en polaire pour intégrer sur un morceau de cercle.}
\input{auto/pictures_tex/Fig_HFAYooOrfMAA.pstricks}

    Le passage en polaire transforme les équations du bord du domaine en
    \begin{equation}
        \begin{aligned}[]
            \cos(\theta)&=\sin(\theta)\\
            r^2&=r\sin(\theta).
        \end{aligned}
    \end{equation}
    L'angle $\theta$ parcours donc $\mathopen] 0 , \pi/4 \mathclose[$, et le rayon, pour chacun de ces $\theta$ parcours $\mathopen] 0 , \sin(\theta) \mathclose[$. La fonction à intégrer se note maintenant $f(r,\theta)=\sqrt{1-r^2}$. Donc l'intégrale à calculer est
    \begin{equation}		\label{PgOMRapIntMultFubiniBoutCercle}
        \int_{0}^{\pi/4}\left( \int_0^{\sin(\theta)}\sqrt{1-r^2}r\,rd \right).
    \end{equation}
    Remarquez la présence d'un $r$ supplémentaire pour le jacobien.

    Notez que les coordonnées du point $P$ sont $(1,1)$.
\end{example}

En pratique, lors du passage en coordonnées polaires, le «$dxdy$» devient «$r\,drd\theta$».

\begin{example}
    On veut évaluer l'intégrale de la fonction $f(x,y)= x^2+y^2$ sur la région $V$ suivante :
    \[
    V=\{(x,y) \in \eR^2\,\vert\, x^2+y^2\leq 1,\, x>0,\, y>0\}.
    \]
    On peut faire le calcul directement,
    \[
    \int_{V}f(x,y)\, dV=\int_0^1\int_0^{\sqrt{1-x^2}}x^2+y^2\, dy\,dx=\int_0^1\left(x^2\sqrt{1-x^2} + \frac{(1-x^2)^{3/2}}{3}\right) dx
    \]
    mais c'est un peu ennuyeux. On peut simplifier beaucoup les calculs avec un changement de variables vers les coordonnées polaires. Dans ce cas, on sait bien que le difféomorphisme à utiliser est $\phi(r,\theta)=(r\cos \theta, r\sin\theta)$. Le jacobien  $J_{\phi}$ est
    \begin{equation}
     J_{\phi}(r, \theta)= \left\vert\begin{array}{cc}
    \cos \theta & \sin \theta \\
    -r\sin \theta  & r\cos \theta
    \end{array}\right\vert= r,
    \end{equation}
    qui est toujours positif. D'une part, la fonction $f$ peut s'écrire sous la forme $f(\phi(r,\theta))=r^2$ et d'autre part, $\phi^{-1}(V)=]0,1]\times]0, \pi/2[$. Par conséquent, la formule du changement de variables nous donne
    \[
    \int_{V}f(x,y)\, dV=\int_0^{\pi/2}\int_0^{1}r^3 dr\,d\theta=\int_0^{\pi/2}\frac{1}{4}\,d\theta=\frac{\pi}{8}.
    \]
\end{example}

\begin{example}
    Montrons comment intégrer la fonction $f(x,y)=\sqrt{1-x^2-y^2}$ sur le domaine délimité par la droite $y=x$ et le cercle $x^2+y^2=y$, représenté sur la figure~\ref{LabelFigQXyVaKD}. Pour trouver le centre et le rayon du cercle $x^2+y^2=y$, nous commençons par écrire $x^2+y^2-y=0$, et ensuite nous reformons le carré : $y^2-y=(y-\frac{ 1 }{2})^2-\frac{1}{ 4 }$.
    \newcommand{\CaptionFigQXyVaKD}{Passage en polaire pour intégrer sur un morceau de cercle.}
\input{auto/pictures_tex/Fig_QXyVaKD.pstricks}

    Le passage en polaire transforme les équations du bord du domaine en
    \begin{equation}
        \begin{aligned}[]
            \cos(\theta)&=\sin(\theta)\\
            r^2&=r\sin(\theta).
        \end{aligned}
    \end{equation}
    L'angle $\theta$ parcours donc $\mathopen] 0 , \pi/4 \mathclose[$, et le rayon, pour chacun de ces $\theta$ parcours $\mathopen] 0 , \sin(\theta) \mathclose[$. La fonction à intégrer se note maintenant $f(r,\theta)=\sqrt{1-r^2}$. Donc l'intégrale à calculer est
    \begin{equation}		\label{PgRapIntMultFubiniBoutCercle}
        \int_{0}^{\pi/4}\left( \int_0^{\sin(\theta)}\sqrt{1-r^2}r\,rd \right).
    \end{equation}
    Remarquez la présence d'un $r$ supplémentaire pour le jacobien.

    Notez que les coordonnées du point $P$ sont $(1,1)$.
\end{example}

%--------------------------------------------------------------------------------------------------------------------------- 
\subsection{Coordonnées cylindriques}
%---------------------------------------------------------------------------------------------------------------------------

\begin{example}
    On veut calculer le volume de la région $A$ définie par  l'intersection entre la boule unité et le cylindre qui a pour base un disque de rayon $1/2$ centré en $(0, 1/2)$
    \[
    A=\{(x,y,z) \in\eR^3 \,\vert\, x^2+y^2+z^1\leq 1\}\cap\{(x,y,z) \in \eR^3\,\vert\, x^2+(y-1/2)^2\leq 1/4\}.
    \]
    On peut décrire $A$ en coordonnées cylindriques
    \begin{equation}
      \begin{aligned}
        A=\Big\{(r,\theta,z) &\in ]0, +\infty[\times [-\pi,\pi[\times \eR\,\vert\,\\
    & -\pi/2<\theta<\pi, \, 0<r\leq \sin\theta, \, -\sqrt{1-r^2}\leq z\leq\sqrt{1-r^2} \Big\}.
      \end{aligned}
    \end{equation}
    Le jacobien de ce changement de variables,  $J_{cyl}$, est
    \begin{equation}
     J_{cyl}(r, \theta), z= \left\vert\begin{array}{ccc}
    \cos \theta & \sin \theta & 0\\
    -r\sin \theta  & r\cos \theta &0 \\
    0&0&
    \end{array}\right\vert= r,
    \end{equation}
    qui est toujours positif. Le volume de $A$ est donc
    \[
    \int_{\eR^3}\chi_{A}(x,y,z)\, dV=\int_{-\pi/2}^{\pi/2}\int_0^{\sin\theta}\int_{-\sqrt{1-r^2}}^{\sqrt{1-r^2}} r dz\,dr\,d\theta=\frac{2\pi}{8}+\frac{8}{9}.
    \]
\end{example}

\begin{example}[Volume d'un solide de révolution]
Soit $g:[a,b]\to\eR_+$ une fonction continue et positive. On dit que le solide $A$ décrit par
\[
A=\left\{(x,y,z)\in\eR^3\, \vert \, z\in[a,b], \,\sqrt{x^2+y^2}\leq g^2(z) \right\}
\]
est un solide de révolution. Afin de calculer son volume, on peut décrire $A$ en coordonnées cylindriques,
\[
A=\left\{(r,\theta,z) \in ]0, +\infty[\times [-\pi,\pi[\times \eR\,\vert\, a\leq z\leq b, \, 0<r^2\leq g^2(z) \right\}.
\]
Le jacobien de ce changement de variables est  $J_{cyl}=r$, comme dans l'exemple précédent. Le volume de $A$ est donc
\[
\int_{\eR^3}\chi_{A}(x,y,z)\, dV=\int_a^{b}\int_{-\pi}^{\pi}\int_{0}^{g(z)} r  \,dr\,d\theta\, dz=\int_a^{b} \pi g^2(z) \, dz.
\]
Cette formule peut être utilisée pour tout solide de révolution.
\end{example}

%///////////////////////////////////////////////////////////////////////////////////////////////////////////////////////////
\subsubsection{Coordonnées sphériques}
%///////////////////////////////////////////////////////////////////////////////////////////////////////////////////////////

Le calcul est un peu plus long :
\begin{equation}
    \begin{aligned}[]
        J(\rho,\theta,\varphi)&=\begin{vmatrix}
            \frac{ \partial x }{ \partial \rho }    &   \frac{ \partial x }{ \partial \theta }    &   \frac{ \partial x }{ \partial \varphi }    \\
            \frac{ \partial y }{ \partial \rho }    &   \frac{ \partial y }{ \partial \theta }    &   \frac{ \partial y }{ \partial \varphi }    \\
            \frac{ \partial z }{ \partial \rho }    &   \frac{ \partial z }{ \partial \theta }    &   \frac{ \partial z }{ \partial \varphi }
        \end{vmatrix}\\
        &=
        \begin{vmatrix}
            \sin\theta\cos\varphi    &   \rho\cos\theta\cos\varphi    &   -\rho\sin\theta\sin\varphi    \\
            \sin\theta\sin\varphi    &   \rho\cos\theta\sin\varphi    &   -\rho\sin\theta\cos\varphi    \\
            \cos\theta               &   -\rho\sin\theta              &   0
        \end{vmatrix}\\
        &=\rho^2\sin\theta.
    \end{aligned}
\end{equation}
Donc
\begin{equation}
    dx\,dy\,dz=\rho^2\sin(\theta)\,d\rho\,d\theta\,d\varphi.
\end{equation}

%--------------------------------------------------------------------------------------------------------------------------- 
\subsection{Coordonnées sphériques}
%---------------------------------------------------------------------------------------------------------------------------

\begin{example}
    On veut calculer le volume du cornet de glace  $A$
    \[
    A=\left\{(x,y,z)\in\eR^3\, \vert \, (x,y)\in \mathbb{S}^2, \,\sqrt{x^2+y^2}\leq z\leq \sqrt{1-x^2-y^2} \right\}.
    \]
    On peut décrire $A$ en coordonnées sphériques.
    \[
    A=\{(\rho,\theta,\phi) \in ]0, +\infty[\times [-\pi,\pi[\times [0,\pi[\,\vert\, 0<\phi\leq\pi/4, \, 0<\rho\leq 1 \}.
    \]
    Le jacobien de ce changement de variables  $J_{sph}$ est
    \begin{equation}
     J_{sph}(\rho, \theta, \phi)= \left\vert\begin{array}{ccc}
    \cos \theta \sin\phi & \sin \theta\sin\phi & \cos\phi\\
    -\rho\sin \theta\sin\phi  & \rho\cos \theta\sin\phi & 0 \\
    \rho\cos\theta\cos\phi&\rho\sin\theta\cos\phi& -\rho\sin\phi
    \end{array}\right\vert= \rho^2\sin\phi,
    \end{equation}
    Le volume de $A$ est donc
    \[
    \int_{\eR^3}\chi_{A}(x,y,z)\, dV=\int_{-\pi}^{\pi}\int_0^{\pi/4}\int_{0}^{1}\rho^2\sin\phi \,d\rho\,d\phi\,d\theta=\frac{2\pi}{3}\left(1-\frac{1}{\sqrt{2}}\right).
    \]
\end{example}

\begin{example}[Une petite faute à ne pas faire]
    Si nous voulons calculer le volume de la sphère de rayon $R$, nous écrivons donc
    \begin{equation}
        \int_0^Rdr\int_{0}^{2\pi}d\theta\int_0^{\pi}r^2 \sin(\phi)d\phi=4\pi R=\frac{ 4 }{ 3 }\pi R^3.
    \end{equation}
    Ici, la valeur absolue n'est pas importante parce que lorsque $\phi\in\mathopen] 0,\pi ,  \mathclose[$, le sinus de $\phi$ est positif.

    Des petits malins pourraient remarquer que le changement de variable \eqref{EqChmVarSpherique} est encore un paramétrage de $\eR^3$ si on intervertit le domaine des angles :
    \begin{equation}
        \begin{aligned}[]
            \theta&\colon 0 \to \pi\\
            \phi	&\colon 0\to 2\pi,
        \end{aligned}
    \end{equation}
    alors nous paramétrons encore parfaitement bien la sphère, mais hélas
    \begin{equation}		\label{EqVolumeIncorrectSphere}
        \int_0^Rdr\int_{0}^{\pi}d\theta\int_0^{2\pi}r^2 \sin(\phi)d\phi=0.
    \end{equation}
    Pourquoi ces «nouvelles» coordonnées sphériques sont-elles mauvaises ? Il y a que quand l'angle $\phi$ parcours $\mathopen] 0 , 2\pi \mathclose[$, son sinus n'est plus toujours positif, donc la \emph{valeur absolue} du jacobien n'est plus $r^2\sin(\phi)$, mais $r^2\sin(\phi)$ pour les $\phi$ entre $0$ et $\pi$, puis $-r^2\sin(\phi)$ pour $\phi$ entre $\pi$ et $2\pi$. Donc l'intégrale \eqref{EqVolumeIncorrectSphere} n'est pas correcte. Il faut la remplacer par
    \begin{equation}
        \int_0^Rdr\int_{0}^{\pi}d\theta\int_0^{\pi}r^2 \sin(\phi)d\phi- \int_0^Rdr\int_{0}^{\pi}d\theta\int_{\pi}^{2\pi}r^2 \sin(\phi)d\phi = \frac{ 4 }{ 3 }\pi R^3
    \end{equation}

\end{example}

%--------------------------------------------------------------------------------------------------------------------------- 
\subsection{Un autre système utile}
%---------------------------------------------------------------------------------------------------------------------------

Un changement de variables que l'on voit assez souvent est
\begin{subequations}
    \begin{numcases}{}
        u=x+y\\
        v=x-y.
    \end{numcases}
\end{subequations}
Afin de calculer son jacobien, il faut d'abord exprimer $x$ et $y$ en fonctions de $u$ et $v$ :
\begin{subequations}
    \begin{numcases}{}
        x=(u+v)/2\\
        y=(u-v)/2.
    \end{numcases}
\end{subequations}
La matrice jacobienne est
\begin{equation}
    \begin{pmatrix}
        \frac{ \partial x }{ \partial u }    &   \frac{ \partial x }{ \partial v }    \\
        \frac{ \partial y }{ \partial u }    &   \frac{ \partial y }{ \partial v }
    \end{pmatrix}=
    \begin{pmatrix}
        \frac{ 1 }{2}    &   \frac{ 1 }{2}    \\
        \frac{ 1 }{2}    &   -\frac{ 1 }{2}
    \end{pmatrix}.
\end{equation}
Le déterminant vaut $-\frac{1}{ 2 }$. Nous avons donc
\begin{equation}
    dxdy=\frac{ 1 }{2}dudv.
\end{equation}
Nous insistons sur le fait que c'est $\frac{ 1 }{2}$ et non $-\frac{ 1 }{2}$ qui intervient parce que que la formule du changement de variable demande d'introduire la \emph{valeur absolue} du jacobien.

\begin{example}
    Calculer l'intégrale de la fonction $f(x,y)=x^2-y^2$ sur le domaine représenté sur la figure~\ref{LabelFigVWFLooPSrOqz}. % From file VWFLooPSrOqz
\newcommand{\CaptionFigVWFLooPSrOqz}{Un domaine qui s'écrit étonnament bien avec un bon changement de coordonnées.}
\input{auto/pictures_tex/Fig_VWFLooPSrOqz.pstricks}

    Les droites qui délimitent le domaine d'intégration sont
    \begin{equation}
        \begin{aligned}[]
            y&=-x+2\\
            y&=x-2\\
            y&=x\\
            y&=-x
        \end{aligned}
    \end{equation}
    Le domaine est donc donné par les équations
    \begin{subequations}
        \begin{numcases}{}
            y+x<2\\
            y-x>-2\\
            y-x<0 \\
            y+x>0.
        \end{numcases}
    \end{subequations}
    En utilisant le changement de variables $u=x+y$, $v=x-y$ nous trouvons le domaine $0<u<2$, $0<v<2$. En ce qui concerne la fonction, $f(x,y)=(x+y)(x-y)$ et par conséquent
    \begin{equation}
        f(u,v)=uv.
    \end{equation}
    L'intégrale à calculer est simplement
    \begin{equation}
        \int_0^2\int_0^2 uv\,dudv=\int_0^2 u\,du\left[ \frac{ v^2 }{ 2 } \right]_0^2=2\int_0^2u\,du=4.
    \end{equation}
\end{example}

%+++++++++++++++++++++++++++++++++++++++++++++++++++++++++++++++++++++++++++++++++++++++++++++++++++++++++++++++++++++++++++ 
\section{Aire d'une surface de révolution}
%+++++++++++++++++++++++++++++++++++++++++++++++++++++++++++++++++++++++++++++++++++++++++++++++++++++++++++++++++++++++++++

Soit $\gamma$ une courbe dans le plan $xy$, paramétrée par
\begin{equation}
    \gamma(u)=\begin{pmatrix}
        x(u)    \\
        y(u)    \\
        0
    \end{pmatrix}
\end{equation}
avec $u\in\mathopen[ a , b \mathclose]$. Nous supposons que la courbe est toujours positive, c'est-à-dire $y(u)>0$ pour tout $u$.

Nous voulons considérer la surface obtenue en effectuant une rotation de cette ligne autour de l'axe $X$. Chaque point de la courbe va parcourir un cercle de rayon $y(u)$ dans le plan $YX$ et centré en $(x(u),0,0)$. La surface est donc donnée par
\begin{equation}
    \varphi(u,\theta)=\begin{pmatrix}
        x(u)    \\
        y(u)\cos\theta    \\
        y(u)\sin\theta
    \end{pmatrix}
\end{equation}
avec $(u,\theta)\in\mathopen[ a , b \mathclose]\times \mathopen[ 0 , 2\pi \mathclose]$. Notez que la courbe de départ correspond à $\theta=0$.

Les vecteurs tangents à la surface pour ce paramétrage sont
\begin{equation}
    \begin{aligned}[]
        T_u&=\frac{ \partial \varphi }{ \partial u }=\begin{pmatrix}
            x'(u)    \\
            y'(u)\cos\theta    \\
            y'(u)\sin\theta
        \end{pmatrix}&
        T_{\theta}&=\frac{ \partial \varphi }{ \partial \theta }=\begin{pmatrix}
            0    \\
            -y(u)\sin\theta    \\
            y(u)\cos\theta
        \end{pmatrix}.
    \end{aligned}
\end{equation}
Le produit vectoriel de ces deux vecteurs vaut
\begin{equation}
    \begin{aligned}[]
        T_u\times T_{\theta}&=\begin{vmatrix}
            e_x    &   e_y    &   e_z    \\
            x'    &   y'\cos\theta    &   y'\sin\theta    \\
            0    &   -y\sin\theta    &   y\cos\theta
        \end{vmatrix}\\
        &=y'(u)y(u)\,e_x-x'(u)y(u)\cos\theta\, e_y+x'(u)y(u)\sin\theta\, e_z.
    \end{aligned}
\end{equation}
En ce qui concerne la norme :
\begin{equation}
    dS=\| T_u\times T_{\theta} \|=\sqrt{(y'y)^2+(x'y)^2}=| y(u) |\sqrt{y'(u)^2+x'(u)^2}.
\end{equation}
Étant donné que nous avons supposé que $y(u)>0$, nous pouvons supprimer les valeurs absolues, et l'aire de la surface de révolution devient :
\begin{equation}
    \begin{aligned}[]
        Aire(S)&=\int_0^{2\pi}d\theta\int_a^b y(u)\sqrt{x'(u)^2+y'(u)^2}du\\
        &=2\pi\int_a^b y(u)\sqrt{x'(u)^2+y'(u)^2}du.
    \end{aligned}
\end{equation}

\begin{example}     \label{EXooZCLXooVmXQgY}
    Calculons la surface du cône de révolution de rayon (à la base) $R$ et de hauteur $h$. La courbe de départ est le segment droite qui part de $(0,0)$ et qui termine en $(R,h)$ de la figure~\ref{LabelFigYHJYooTEXLLn}. % From file YHJYooTEXLLn
\newcommand{\CaptionFigYHJYooTEXLLn}{En faisant tourner cette droite autour de l'axe $X$, nous obtenons un cône.}
\input{auto/pictures_tex/Fig_YHJYooTEXLLn.pstricks}

    Ce segment peut être paramétré par
    \begin{equation}
        \gamma(u)=\begin{pmatrix}
            Ru    \\
            hu    \\
            0
        \end{pmatrix}
    \end{equation}
    avec $u\in\mathopen[ 0 , 1 \mathclose]$. Cela donne $x(u)=Ru$, $y(u)=hu$ et par conséquent
    \begin{equation}
        Aire=2\pi\int_0^1hu\sqrt{R^2+h^2}=\pi h\sqrt{R^2+h^2}.
    \end{equation}
    Ce résultat peut aussi être exprimé en fonction de l'angle, grâce à la formule \eqref{EQooEKZEooFeNImX}. En sachant que $h=\sqrt{h^2+R^2}\sin(\alpha)$, nous trouvons
    \begin{equation}
        Aire=\pi(R^2+h^2)\sin(\alpha).
    \end{equation}

\end{example}

\begin{example}
    Calculons la surface latérale du tore obtenu par révolution du cercle de la figure ~\ref{LabelFigROAOooPgUZIt}. % From file ROAOooPgUZIt
\newcommand{\CaptionFigROAOooPgUZIt}{Si nous tournons ce cercle autour de l'axe $X$, nous obtenons un tore de rayon «externe» $a$ et de rayon «interne» $R$.}
\input{auto/pictures_tex/Fig_ROAOooPgUZIt.pstricks}

    Le chemin qui détermine le cercle de départ est
    \begin{equation}
        \gamma(u)=\begin{pmatrix}
            R\cos(u)    \\
            a+R\sin(u)    \\
            0
        \end{pmatrix},
    \end{equation}
    c'est-à-dire $x(u)=R\cos(u)$, $y(u)=a+R\sin(u)$ avec $u\in\mathopen[ 0 , 2\pi \mathclose]$. Nous avons donc l'aire
    \begin{equation}
        \begin{aligned}[]
            Aire&=2\pi\int_0^{2\pi}\big( a+R\sin(u) \big)R\,du\\
            &=2\pi R\big( 2\pi a+R[-\cos(u)]_0^{2\pi} \big)\\
            &=4\pi^2aR.
        \end{aligned}
    \end{equation}
\end{example}

%+++++++++++++++++++++++++++++++++++++++++++++++++++++++++++++++++++++++++++++++++++++++++++++++++++++++++++++++++++++++++++
\section{Table de caractères du groupe diédral}
%+++++++++++++++++++++++++++++++++++++++++++++++++++++++++++++++++++++++++++++++++++++++++++++++++++++++++++++++++++++++++++
\label{SecWMzheKf}
Cette section vient de \cite{KXjFWKA}; nous avons comme but d'établir la table des caractères des représentations complexes du groupe diédral \( D_n\).
\index{groupe!de permutation}
\index{groupe!diédral!générateurs (utilisation)}
\index{représentation!groupe diédral}
\index{caractère!groupe diédral}

%---------------------------------------------------------------------------------------------------------------------------
\subsection{Représentations de dimension un}
%---------------------------------------------------------------------------------------------------------------------------

Nous nous occupons des représentations de \( D_n\) sur \( \eC\). Les applications linéaires \( \eC\to \eC\) sont seulement les multiplications par des nombres complexes. Nous cherchons donc \( \psi\colon D_n\to \eC^*\).

Nous savons que \( D_n\) est généré\footnote{Voir proposition~\ref{PropLDIPoZ} et tout ce qui suit.} par \( s\) et \( r\). Vu que \( s^2=1\), nous avons
\begin{equation}
    \psi(s)^2=\psi(s^2)=\psi(1)=1,
\end{equation}
donc \( \psi(s)\in\{ -1,1 \}\). Nous savons aussi que \( srsr=1\), donc
\begin{equation}
    \psi(s)^2\psi(r)^2=1,
\end{equation}
ce qui donne \( \psi(r)\in\{ -1,1 \}\).

Nous avons donc quatre représentations de dimension un données par
\begin{equation*}
    \begin{array}[]{|c||c|c|}
        \hline
        &\psi(r)=1&\psi(r)=-1\\
        \hline\hline
        \psi(s)=1&\rho^{++}&\rho^{+-}\\
        \hline
        \psi(s)=-1&\rho^{-+}&\rho^{--}\\
        \hline
    \end{array}
\end{equation*}
Attention au fait que nous devons aussi avoir la relation \( \psi(r)^n=\psi(r^n)=1\). Donc \( \psi(r)\) doit être une racine \( n\)\ieme\ de l'unité. Nous allons donc devoir avoir un compte différent selon la parité de \( n\). Nous en reparlerons à la fin, au moment de faire les comptes. En ce qui concerne les caractères correspondants,
\begin{equation*}
    \begin{array}[]{|c||c|c|}
        \hline
        &r^k&sr^k\\
        \hline\hline
        \chi^{++}&1&1\\
        \hline
        \chi^{+-}&(-1)^k&(-1)^k\\
        \hline
        \chi^{-+}&1&-1\\
        \hline
        \chi^{--}&(-1)^k&(-1)^{k+1}\\
        \hline
    \end{array}
\end{equation*}
Étant donné qu'ils sont tous différents, ce sont des représentations deux à deux non équivalentes, lemme~\ref{LempUSOlo}.

%---------------------------------------------------------------------------------------------------------------------------
\subsection{Représentations de dimension deux}
%---------------------------------------------------------------------------------------------------------------------------

Nous cherchons maintenant les représentations \( \rho\colon D_n\to \End(\eC^2)\). Ici nous supposons connue la liste des éléments de \( D_n\) donnée par le corolaire~\ref{CorWYITsWW}. Soit \( \omega= e^{2i\pi/n}\) et \( h\in \eZ\); nous considérons la représentation \( \rho^{(h)}\) de \( D_n\) définie par
\begin{subequations}
    \begin{align}
        \rho^{(h)}(r^k)&=\begin{pmatrix}
            \omega^{hk}    &   0    \\
            0    &   \omega^{-hk}
        \end{pmatrix}\\
        \rho^{(h)}(st^k)&=\begin{pmatrix}
            0    &   \omega^{-hk}    \\
            \omega^{hk}    &   0
        \end{pmatrix}.
    \end{align}
\end{subequations}
Cela donne bien \( \rho^{(h)}\) sur tous les éléments de \( D_n\) par la proposition~\ref{PropLDIPoZ}. Nous pouvons restreindre le domaine de \( h\) en remarquant d'abord que \( \rho^{(h)}=\rho^{(h+n)}\), et ensuite que les représentations \( \rho^{(h)}\) et \( \rho^{(-h)}\) sont équivalentes. Un opérateur d'entrelacement est donné par \( T=\begin{pmatrix}
    0    &   1    \\
    1    &   0
\end{pmatrix}\), et il est facile de vérifier que \( T\rho^{(h)}(x)=\rho^{-h}(x)T\) avec \( x=r^k\) puis avec \( x=sr^k\).

Donc \( \rho^{(h)}\simeq\rho^{(-h)}\simeq\rho^{(n-h)}\) et nous pouvons restreindre notre étude à \( 0\leq h\leq \frac{ n }{2}\).

Nous allons séparer les cas \( n=0\), \( h=n/2\) et les autres. En effet si nous notons par commodité \( a=\omega^h\), alors un vecteur \( (x,y)\) est vecteur propre de \( \rho^{(h)}(s)\) et de \( \rho^{(h)}(r)\) si et seulement s'il vérifie les systèmes d'équations
\begin{subequations}        \label{SubEqsGXZoxLq}
    \begin{numcases}{}
        ax=\lambda x\\
        \frac{1}{ a }y=\lambda y
    \end{numcases}
\end{subequations}
et
\begin{subequations}    \label{SubEqsFYZmzhT}
    \begin{numcases}{}
        \frac{1}{ a }y=\mu x\\
        ax=\mu y
    \end{numcases}
\end{subequations}
avec \( \lambda\) et \( \mu\) des nombres non nuls. Une représentation sera réductible si et seulement si ces deux systèmes acceptent une solution non nulle commune. Il est vite vu que si \( x\neq 0\) et \( y\neq 0\), alors \( a^2=1\), ce qui signifie \( h=0\) ou \( h=n/2\). Sinon, il n'y a pas de solutions, et la représentation associée est irréductible.

\begin{enumerate}
    \item
        \( h=0\). Nous avons
        \begin{equation}
            \begin{aligned}[]
                \rho^{(0)}(r^k)&=\begin{pmatrix}
                    1    &   0    \\
                    0    &   1
                \end{pmatrix}& \rho^{(0)}(sr^k)=\begin{pmatrix}
                    0    &   1    \\
                    1    &   0
                \end{pmatrix},
            \end{aligned}
        \end{equation}
        donc le caractère de cette représentation est \( \chi^{(0)}(r^k)=2\) et \( \chi^{(0)}(sr^k)=0\). Donc nous avons
        \begin{equation}
            \chi^{(0)}=\chi^{++}+\chi^{-+}.
        \end{equation}
        Il y a maintenant (au moins) quatre façons de voir que la représentation \( \rho^{(0)}\) est réductible.
        \begin{description}

            \item[Première méthode]
                Trouver un opérateur d'entrelacement. Pour cela nous calculons les matrices :
        \begin{subequations}
            \begin{align}
                S(r)&=(\rho^{++}\oplus \rho^{-+})(r^k)=\begin{pmatrix}
                    \rho^{++}(r^k)    &   0    \\
                    0  &   \rho^{-+}(r^k)
                \end{pmatrix}=\begin{pmatrix}
                    1    &   0    \\
                    0    &   1
                \end{pmatrix}\\
                S(sr^k)&=(\rho^{++}\oplus \rho^{-+})(sr^k)=\begin{pmatrix}
                    \rho^{++}(sr^k)    &   0    \\
                    0  &   \rho^{-+}(sr^k)
                \end{pmatrix}=\begin{pmatrix}
                    1    &   0    \\
                    0    &   -1
                \end{pmatrix}\\
            \end{align}
        \end{subequations}
        Nous cherchons une matrice \( T\) telle que \( TS(r^k)=\rho^{(0)}(r^k)T\) et \( TS(sr^k)=\rho^{(0)}(sr^k)T\). Étant donné que \( S(r^k)=\mtu=\rho^{(0)}(r^k)\), la première contrainte n'en est pas une. Nous pouvons vérifier qu'avec \( T=\begin{pmatrix}
            1    &   1    \\
            1    &   -1
        \end{pmatrix}\), nous avons bien
        \begin{equation}
            T\begin{pmatrix}
                1    &   0    \\
                0    &   -1
            \end{pmatrix}=\begin{pmatrix}
                0    &   1    \\
                1    &   0
            \end{pmatrix}.
        \end{equation}
        Donc ce \( T\) entrelace \( \rho^{++}\oplus \rho^{-+}\) avec \( \rho^{(0)}\) qui sont donc deux représentations équivalentes. Donc \( \rho^{(0)}\) est réductible et ça ne nous intéresse pas de la lister.
            \item[Seconde méthode]
                Invoquer le théorème \ref{ThoWGkfADd}\ref{ItemZReOWoHi} et dire que les représentations sont équivalentes parce que les caractères sont égaux.

    \item[Troisième méthode]
        Utiliser le théorème~\ref{ThoWGkfADd}\ref{ItemZReOWoHii} et calculer \( \langle \chi^{(0)}, \chi^{(0)}\rangle \) :
        \begin{subequations}
            \begin{align}
                \langle \chi^{(0)}, \chi^{(0)}\rangle &=\frac{1}{ | D_n | }\sum_{g\in D_n}| \chi^{(0)}(g) |^2\\
                &=\frac{1}{ 2n }\big(4+0+4(n-1)\big)\\
                &=2.
            \end{align}
        \end{subequations}
        Ici le \( 4\) est pour le \( 1\), le zéro est pour les termes \( sr^k\) et \( 4(n-1)\) est pour les \( n-1\) termes \( r^k\). Vu que le résultat n'est pas \( 1\), la représentation \( \rho^{(0)}\) n'est pas irréductible.

    \item[Quatrième méthode]
        Regarder les solutions des systèmes \eqref{SubEqsGXZoxLq} et \eqref{SubEqsFYZmzhT} dont nous avons parlé plus haut.

    \end{description}

    La première méthode a l'avantage d'être simple et ne demander aucune théorie particulière à part les définitions. La seconde méthode est la plus rapide, mais demande un théorème très puissant. La troisième utilise également un théorème assez avancé, mais a l'avantage sur les deux autres méthodes de ne pas avoir besoin de savoir à priori un candidat décomposition de \( \rho^{0)}\); cette méthode est applicable même sans faire la remarque que \( \chi^{(0)}=\chi^{++}+\chi^{-+}\).

    Quoi qu'il en soit, nous ne listons pas \( \chi^{(0)}\) dans notre \href{http://fr.wikipedia.org/wiki/Aide:Unicode}{table de caractères}.

    \item
        \( h=n/2\). Vu que \( \omega^{n/2}= e^{i\pi}=-1\), nous avons
        \begin{equation}
            \begin{aligned}[]
                \rho^{(n/2)}(r^k)&=\begin{pmatrix}
                    (-1)^k    &   0    \\
                    0    &   (-1)^k
                \end{pmatrix}&
                \rho^{(n/2)}(sr^k)&=\begin{pmatrix}
                    0   &   (-1)^k    \\
                    (-1)^k    &  0
                \end{pmatrix}&
            \end{aligned},
        \end{equation}
        et donc
        \begin{subequations}
            \begin{align}
                \chi^{(n/2)}(r^k)&=2(-1)^k\\
                \chi^{(n/2)}(sr^k)&=0.
            \end{align}
        \end{subequations}
        Il est vite vu que \( \chi^{(n/2)}=\chi^{+-}+\chi^{-+}\). Ergo la représentation \( \rho^{(n/2)}\) n'est pas irréductible.

    \item
        \( 0<h<\frac{ n }{2}\). Dans ce cas nous avons \( \omega^h\neq \omega^{-h}\), et en regardant les systèmes d'équations donnés plus haut, nous voyons que \( \rho^{(h)}(s)\) et \( \rho^{(h)}(r)\) n'ont pas de vecteurs propres communs. Donc ces représentations sont irréductibles.

        Nous devons cependant encore vérifier si elles sont deux à deux non équivalentes. Supposons que pour \( h\neq h'\) nous ayons une matrice \( T\in \GL(2,\eC)\) telle que \( T\rho^{(h)}(r)T^{-1}=\rho^{(h')}(r)\). Cela impliquerait en particulier que les matrices \( \rho^{(h)}(r)\) et \( \rho^{(h')}(r)\) aient même valeurs propres. Nous aurions donc \( \{ \omega^h,\omega^{-h} \}=\{ \omega^{h'},\omega^{-h'} \}\). Mais cela est impossible avec \( 0<h<h'<\frac{ n }{2}\). Donc toutes ces représentations sont distinctes.

\end{enumerate}

Le caractère de la représentation \( \rho^{(h)}\) est \( \chi^{(h)}(r^k)=\omega^{hk}+\omega^{-hk}=2\cos\left( \frac{ 2\pi hk }{ n } \right)\).

Nous ajoutons donc la ligne suivante à notre liste :
\begin{equation*}
    \begin{array}[]{|c||c|c|}
        \hline
        &r^k&sr^k\\
        \hline\hline
        \chi^{(h)}&2\cos\left( \frac{ 2\pi hk }{ n } \right)&0\\
        \hline
    \end{array}
\end{equation*}

%---------------------------------------------------------------------------------------------------------------------------
\subsection{Le compte pour \texorpdfstring{$ n$}{n} pair}
%---------------------------------------------------------------------------------------------------------------------------

Nous avons \( 4\) représentations de dimension \( 1\) puis \( \frac{ n }{2}-1\) représentations de dimension \( 2\). En tout nous avons
\begin{equation}
 \frac{ n }{2}+3
\end{equation}
représentations irréductibles modulo équivalence. Cela fait le compte en vertu des classes de conjugaisons listées en~\ref{SubsubsecROVmHuM}. Pour rappel, le nombre de représentations non équivalentes est égal au nombre de classes de conjugaison par le corolaire~\ref{CorbdcVNC}. Notons que c'est cela qui justifie le fait que nous ne devons pas chercher d'autres représentations. Nous sommes sûrs de les avoir toutes trouvées.

%---------------------------------------------------------------------------------------------------------------------------
\subsection{Le compte pour \texorpdfstring{$ n$}{n} impair}
%---------------------------------------------------------------------------------------------------------------------------

Nous avions fait mention plus haut du fait que si \( \psi\) est une représentation de dimension \( 1\), le nombre \( \psi(r)\) devait être une racine \( n\)\ieme\ de l'unité. Donc en dimension \( 1\) nous avons seulement les représentations \( \rho^{++}\) et \( \rho^{-+}\). Pour celles de dimension \( 2\), nous en avons \( \frac{ n-1 }{2}\). En tout nous avons donc
\begin{equation}
    \frac{ n+3 }{2}
\end{equation}
représentations irréductibles modulo équivalence. Cela fait le compte en vertu des classes de conjugaisons listées en~\ref{GJIzDEP}.


\chapter{Corps finis, racines de l'unité}
\input{197_racines}

\chapter{Intégration sur des variétés}
\input{72_Integration}
\input{155_Integration}
\input{73_Integration}
\input{38_Integration}
% This is part of Le Frido
% Copyright (c) 2011,2017-2019
%   Laurent Claessens,Carlotta Donadello
% See the file fdl-1.3.txt for copying conditions.

%+++++++++++++++++++++++++++++++++++++++++++++++++++++++++++++++++++++++++++++++++++++++++++++++++++++++++++++++++++++++++++
\section{L'aire en dessous d'une courbe}
%+++++++++++++++++++++++++++++++++++++++++++++++++++++++++++++++++++++++++++++++++++++++++++++++++++++++++++++++++++++++++++

Soit $f$ une fonction à valeurs dans $\eR^+$.

Nous voudrions pouvoir calculer l'aire au-dessous du graphe de la fonction \( f\). Nous notons $S_f(x)$ l'aire là-dessous de la fonction $f$ entre l'abscisse $0$ et $x$, c'est-à-dire l'aire bleue de la figure~\ref{LabelFigKKRooHseDzC}.

\newcommand{\CaptionFigKKRooHseDzC}{L'aire en dessous d'une courbe. Le rectangle rouge d'aire $f(x)\Delta x$ approxime de combien la surface augmente lorsqu'on passe de $x$ à $x+\Delta x$.}
\input{auto/pictures_tex/Fig_KKRooHseDzC.pstricks}

Si la fonction $f$ est continue et que $\Delta x$ est assez petit, la fonction ne varie pas beaucoup entre $x$ et $x+\Delta x$. L'augmentation de surface entre $x$ et $x+\Delta x$ peut donc être approximée par le rectangle de surface $f(x)\Delta x$. Ce que nous avons donc, c'est que quand $\Delta x$ est très petit,
\begin{equation}
	S_f(x+\Delta x)-S_f(x)=f(x)\Delta x,
\end{equation}
ou encore
\begin{equation}
	f(x)=\frac{  S_f(x+\Delta x)-S_f(x)}{ \Delta x }.
\end{equation}
Nous formalisons la notion de «lorsque \( \Delta x\) est très petit» par une limite :
\begin{equation}
	f(x)=\lim_{\Delta x\to 0}\frac{  S_f(x+\Delta x)-S_f(x)}{ \Delta x }.
\end{equation}
Donc, la fonction $f$ est la dérivée de la fonction qui représente l'aire là-dessous de $f$. Calculer des surfaces revient donc au travail inverse de calculer des dérivées.

%+++++++++++++++++++++++++++++++++++++++++++++++++++++++++++++++++++++++++++++++++++++++++++++++++++++++++++++++++++++++++++
\section{Propriétés des intégrales}
%+++++++++++++++++++++++++++++++++++++++++++++++++++++++++++++++++++++++++++++++++++++++++++++++++++++++++++++++++++++++++++

\begin{lemma}			\label{LemIneqnormeintintnorm}
	Pour toute fonction $u\colon \mathopen[ a , b \mathclose]\to \eR^n$, nous avons
	\begin{equation}
		\| \int_a^bu(t)dt\|\leq\int_a^b\| u(t) \|dt
	\end{equation}
	pourvu que le membre de gauche ait un sens.
\end{lemma}

\begin{proof}
	Étant donné que $\int_a^bu(t)dt$ est un élément de $\eR^n$, par la proposition~\ref{LemSclNormeXi}, il existe un $\xi\in\eR^n$ de norme $1$ tel que
	\begin{equation}
		\| \int_a^bu(t)dt \|=\xi\cdot\int_a^b u(t)dt=\int_a^b u(t)\cdot\xi dt\leq\int_a^b\| u(t) \|   \| \xi \|=\int_a^b\| u(t) \|dt.
	\end{equation}
\end{proof}

\begin{proposition}[Relations de Chasles]
    Soit  \( f\) une fonction continue sur l'intervalle \( I\). Si \( a,b,c\in I\) nous avons
    \begin{equation}
        \int_a^cf(x)dx=\int_a^bf(x)dx+\int_b^cf(x)dx.
    \end{equation}
\end{proposition}
\index{relations!de Chasles}

Sur la figure~\ref{LabelFigNWDooOObSHB}, la surface de \( a\) à \( c\) est évidemment égale à la somme des surfaces de \( a\) à \( b\) et de \( b\) à \( c\).
\newcommand{\CaptionFigNWDooOObSHB}{Illustration pour les relations de Chasles.}
\input{auto/pictures_tex/Fig_NWDooOObSHB.pstricks}

\begin{corollary}
  \begin{equation}
        \int_a^bf(x)dx=-\int_b^af(x)dx.
    \end{equation}
\end{corollary}

\begin{proposition}[Linéarité de l'intégrale]\label{lineariteintegrale}
    Si $f$ et $g$ sont deux fonctions continues sur $I\subset\eR$, $a, \, b\in I$ et \( \lambda\in \eR\) nous avons
    \begin{equation}
        \int_a^b\big( f(x)+g(x) \big)dx=\int_a^bf(x)dx+\int_a^bg(x)dx,
    \end{equation}
    et
    \begin{equation}
        \int_a^b \lambda f(x)dx=\lambda\int_a^bf(x)dx.
    \end{equation}
\end{proposition}

\begin{proposition}[L'intégrale est monotone]   \label{PropCJIooHqECbq}
    Soient \( a,b\in I\) avec \( a<b\). Si \( f\geq g\) sur \( \mathopen[ a , b \mathclose]\) alors
    \begin{equation}
        \int_a^bf(x)dx\geq \int_a^bg(x)dx.
    \end{equation}
\end{proposition}

\begin{corollary}[Positivité] \label{PropHVWooBDRhCX}
    Si \( a<b\) et \( f\geq 0\) sur \( \mathopen[ a , b \mathclose]\) alors
    \begin{equation}
        \int_a^bf(x)dx\geq 0.
    \end{equation}
\end{corollary}

Ce résultat n'est qu'une application de la proposition~\ref{PropCJIooHqECbq} car il consiste à prendre comme fonction $g$ la fonction nulle.

%+++++++++++++++++++++++++++++++++++++++++++++++++++++++++++++++++++++++++++++++++++++++++++++++++++++++++++++++++++++++++++
\section{Techniques d'intégration}
%+++++++++++++++++++++++++++++++++++++++++++++++++++++++++++++++++++++++++++++++++++++++++++++++++++++++++++++++++++++++++++

Par le théorème \ref{ThoRWXooTqHGbC}, la calcul d'une intégrale consiste essentiellement à trouver une primitive de la fonction à intégrer.  Il est donc indispensable de bien connaitre les dérivées des fonctions usuelles.

Voici un tableau des primitives à connaitre.

\label{PageLCHooMbWjOj}
\begin{equation*}
    \begin{array}[]{|c||c|c|c|}
        \hline
        \text{Fonction}&\text{Primitive}&\text{Ensemble de définition}&\text{Remarques}\\
        f(x)&\int f(x)\, dx& \text{de } f&\\
        \hline\hline
        x^{\alpha}&\frac{ x^{\alpha+1} }{ \alpha+1 } + C& \text{dépend de }\alpha&  \alpha\in \eR\setminus\{ -1 \}  \\
        \hline
        \frac{1}{ x }&\ln\big( | x | \big) + C&x\neq 0&\\
        \hline
        \frac{1}{ 1+x^2 }&\arctan(x) + C&\eR&\\
        \hline
        \frac{1}{ \sqrt{1-x^2} }&\arcsin(x) + C&\mathopen] -1 , 1 \mathclose[&\\
        \hline
        \frac{-1}{ \sqrt{1-x^2} }&\arccos(x) + C&\mathopen] -1 , 1 \mathclose[&\\
        \hline
        e^x&e^x + C&\eR&\\
        \hline
        \sin(x)&-\cos(x) + C&\eR&\\
        \hline
        \cos(x)&\sin(x) + C&\eR&\\
        \hline
    1+\tan^2(x)&\tan(x) + C&\text{in intervalle de la forme }\mathopen] -\frac{ \pi }{2} , \frac{ \pi }{2} \mathclose[+k\pi&\\
        \hline
    \end{array}
\end{equation*}



Notez que au signe près, les fonctions \( \arcsin \) et \( \arccos\) ont la même dérivée.

Si la fonction à intégrer est une combinaison linéaire de fonctions usuelles alors sa primitive peut \^etre calculée en utilisant la proposition~\ref{lineariteintegrale}. Dans les sections suivantes on abordera deux autres cas où la fonction à intégrer peut s'écrire en termes de fonctions dont on connaît une primitive.

%---------------------------------------------------------------------------------------------------------------------------
\subsection{Intégration par parties}
%---------------------------------------------------------------------------------------------------------------------------

\begin{proposition}     \label{PROPooRLFIooQHnyJY}
    Si \( u\) et \( v\) sont deux fonctions dérivables de dérivées continues sur l'intervalle \( \mathopen[ a , b \mathclose]\) alors
    \begin{equation}        \label{EQooKISBooQvGMQT}
        \int_a^b u(x)v'(x)dx=\big[ u(x)v(x) \big]_a^b-\int_a^bu'(x)v(x)dx.
    \end{equation}
\end{proposition}

\begin{proof}
    Il s'agit d'utiliser à l'envers la formule de dérivation d'un produit :
    \begin{equation}
        uv'=(uv)'-u'v.
    \end{equation}
    Les fonctions à gauche et à droite étant égales, elles ont même intégrale sur \( \mathopen[ a , b \mathclose]\) et par linéarité, voir  proposition~\ref{lineariteintegrale}, on a :
    \begin{equation}
        \int_a^b u(x)v'(x)dx=\int_a^b (uv)'(x)-\int_a^b u'(x)v(x)dx.
    \end{equation}
    La fonction \( uv\) est évidemment une primitive de \( (uv)'\), de telle sorte que l'on puisse un peu simplifier cette expression :
    \begin{equation}
        \int_a^b u(x)v'(x)dx= \Big[ u(x)v(x) \Big]_a^b -\int_a^b u'(x)v(x)dx,
    \end{equation}
    ce qu'il fallait démontrer.
\end{proof}

\begin{example} \label{ExWIEooVUgvSp}
    Un cas typique d'utilisation de l'intégrale par parties est le suivant. Soit à calculer
    \begin{equation}
       \int_0^{\pi}x\cos(x)dx.
    \end{equation}
    Nous devons écrire \( x\cos(x)\) comme un produit \( u(x)v'(x)\). Il y a (au moins) deux moyens de le faire :
    \begin{subequations}
        \begin{numcases}{}
            u=x\\
            v'=\cos(x).
        \end{numcases}
    \end{subequations}
    ou
    \begin{subequations}
        \begin{numcases}{}
            u=\cos(x)\\
            v'=x.
        \end{numcases}
    \end{subequations}
    Nous allons choisir le premier\footnote{Mais nous conseillons vivement au lecteur d'essayer le deuxième pour se rendre compte qu'il ne fonctionne pas.}. Nous avons donc
    \begin{equation}
        \begin{aligned}[]
            u&=x,&v'&=\cos(x)\\
            u'&=1&v&=\sin(x).
        \end{aligned}
    \end{equation}
    En utilisant la formule d'intégration par parties,
    \begin{equation}
        \int_0^{\pi}x\cos(x)dx=\Big[ x\sin(x) \Big]_0^{\pi}-\int_0^{\pi} 1\times \sin(x)dx=\pi\sin(\pi)-\Big[ -\cos(x) \Big]_0^{\pi}=-2.
    \end{equation}
\end{example}

Le plus souvent, pour alléger les notations, il est plus pratique d'utiliser l'intégration par parties pour déterminer une primitive. Nous utilisons pour cela la formule (sans doute plus simple à retenir)
\begin{equation}
    \int uv'=uv-\int u'v.
\end{equation}

\begin{example} \label{ExLTJooDZIYWP}
    Nous reprenons l'exemple~\ref{ExWIEooVUgvSp} en déterminant cette fois une primitive de \( x\cos(x)\) :
    \begin{equation}\label{EqTQNooVTYkZX}
        \int x\cos(x)dx=x\sin(x)-\int \sin(x)dx=x\sin(x)+\cos(x) + C, \qquad C \in\eR.
    \end{equation}
    Nous retrouvons le résultat numérique de l'exemple précédent en ajoutant les extrêmes d'intégration
    \begin{equation}
        \int_0^{\pi} x\cos(x)dx=\big[ x\sin(x)+\cos(x) \big]_0^{\pi}=-2.
    \end{equation}
\end{example}


\begin{remark}
    Lorsqu'on calcule des intégrales, il est bon de passer par la primitive (c'est-à-dire en suivant l'exemple~\ref{ExLTJooDZIYWP} et non~\ref{ExWIEooVUgvSp}) parce qu'il est alors facile de vérifier le résultat en calculant la dérivée de la primitive trouvée.

    Par exemple pour vérifier si \eqref{EqTQNooVTYkZX} est correct, il suffit de dériver \( x\sin(x)+\cos(x)\) :
    \begin{equation}
        \big( x\sin(x)+\cos(x) \big)'=\sin(x)+x\cos(x)-\sin(x)=x\cos(x).
    \end{equation}
    La fonction \( x\sin(x)+\cos(x)\) est donc bien une primitive de \( x\cos(x)\).
\end{remark}

\begin{example}[Primitive du logarithme]\label{primln}
    La primitive de la fonction logarithme définie en~\ref{DEFooELGOooGiZQjt} nous offre un bon moment d'intégration par partie.

    Trouver la primitive de la fonction \( x\mapsto \ln(x)\). Pour calculer
    \begin{equation}
        \int\ln(x)dx
    \end{equation}
    nous écrivons \( \ln(x)=1\times \ln(x)\) et nous posons \( u'=1\) et \( v=\ln(x)\), c'est-à-dire
    \begin{equation}
        \begin{aligned}[]
            u'&=1&v=\ln(x)\\
            u&=x&v'=\frac{1}{ x }.
        \end{aligned}
    \end{equation}
    La formule d'intégration par parties \eqref{EQooKISBooQvGMQT} donne donc
    \begin{equation}
        \int \ln(x)=x\ln(x)-\int x\times \frac{1}{ x }=x\ln(x)-\int 1=x\ln(x)-x+C, \qquad C\in\eR.
    \end{equation}
    Il est facile de vérifier par un petit calcul que
    \begin{equation}
        \big( x\ln(x)-x \big)'=\ln(x).
    \end{equation}
\end{example}


%---------------------------------------------------------------------------------------------------------------------------
\subsection{Changement de variables -- pour trouver des primitives}
%---------------------------------------------------------------------------------------------------------------------------

De la même manière que l'utilisation «à l'envers» de la formule de dérivation du produit avait donné la méthode d'intégration par parties, nous allons voir que que l'utilisation «à l'envers» de la formule de dérivation d'une fonction composée donne lieu à la méthode d'intégration par changement de variables.
\begin{proposition}     \label{PROPooMVIUooZmvHxS}
    Soient \( I\) et \( J\) des intervalles de \( \eR\), \( u\colon I\to J\) une fonction qui est dérivable de dérivée continue et \( f\colon J\to \eR\) une fonction admettant une primitive \( F\). Alors la fonction
    \begin{equation}
        x\mapsto F\big( u(x) \big)
    \end{equation}
    est une primitive de
    \begin{equation}\label{changvar}
        f\big( u(x) \big)u'(x).
    \end{equation}
\end{proposition}

\begin{proof}
    Cela est une utilisation immédiate de la formule de dérivée des fonctions composées.
\end{proof}

\begin{example}
    Soit à calculer
    \begin{equation}
        \int x\sqrt{1-x^2}dx.
    \end{equation}
La fonction $g(x) = x\sqrt{1-x^2}$ est le produit de $x$ et de $\sqrt{1-x^2}$. On remarque que la dérivée de $1-x^2$ est $-2x$ : nous avons alors, à un facteur $-2$ près, une expression de la forme \eqref{changvar} où la racine carrée joue le r\^ole de $f$, \( f(t)=\sqrt{t}\),   et $1-x^2$ le r\^ole de $u$.  Une primitive de la fonction \( f(t)=\sqrt{t}\) est $F(t) = 2t^{3/2}/3$.

    Donc la fonction
      $  \frac{ 2u(x)^{3/2} }{ 3 }=\frac{ 2 }{ 3 }(1-x^2)^{3/2}$
    est primitive de
     $   -2x\sqrt{1-x^2} = -2 g(x)$.
    Autrement dit,
    \begin{equation}
        \int -2x\sqrt{1-x^2}\,dx=\frac{ 2 (1-x^2)^{3/2}}{ 3 } + C,
    \end{equation}
    et en divisant par \( -2\) nous trouvons la primitive demandée :
    \begin{equation}
        \int x\sqrt{1-x^2}\,dx=-\frac{ (1-x^2)^{3/2} }{ 3 } + C.
    \end{equation}
\end{example}

L'exemple suivant donne une façon plus économe de retenir la méthode du changement de variables.

\begin{example}\label{exempleprimitivechangvar}
    Soit à calculer
    \begin{equation}
        \int \cos(x) e^{\sin(x)}dx.
    \end{equation}
    Vu qu'il y a beaucoup de fonctions trigonométriques dans la fonction à intégrer, nous allons poser \( u(x)=\sin(x)\), et remplacer élément par élément tout ce qui contient du «$x$»  dans l'intégrale demandée par la quantité correspondante en termes de \( u\).

    La difficulté est de savoir ce que nous allons faire du «\( dx\)» dans l'intégrale. Ce \( dx \) marque une variation (infinitésimale) de \( x\). La formule des accroissements finis dit que si \( x\) augmente de la valeur \( dx\), alors \( u(x)\) augmente de $u'(x)dx$, c'est-à-dire que
    \begin{equation}
        du=\cos(x)dx.
    \end{equation}

    Nous avons donc les substitutions suivantes à faire :
    \begin{subequations}
        \begin{align}
            \sin(x)&=u\\
            du&=\cos(x)dx\\
            dx&=\frac{ du }{ \cos(x) }.
        \end{align}
    \end{subequations}
    La chose «magique» est que le \( \cos(x)\) se trouvant dans la fonction se simplifie avec le cosinus qui arrive lorsqu'on remplace \( dx\) par \( \frac{ du }{ \cos(x) }\). Les substitutions faites nous restons avec
    \begin{equation}
        \int\cos(x) e^{\sin(x)}dx=\int e^{u}du=e^u + C, \qquad \text{où } u= \sin(x).
    \end{equation}
   Attention : la réponse doit \^etre impérativement donnée en termes de \( x\) et non de \( u\). Nous écrivons donc
    \begin{equation}
        \int \cos(x) e^{\sin(x)}= e^{\sin(x)}+C.
    \end{equation}
\end{example}

%---------------------------------------------------------------------------------------------------------------------------
\subsection{Changement de variables -- pour calculer des intégrales}
%---------------------------------------------------------------------------------------------------------------------------

Le théorème \ref{ThoRWXooTqHGbC} fixe la relation entre la recherche des primitives de $f $ et la calcul de l'intégrale de $f$ sur l'intervalle d'extrêmes $a$ et $b$. On a vu dans la section précédente comment utiliser le changement de variable pour trouver une primitive de $f$. Il faut maintenant comprendre comment appliquer ce qu'on a vu dans le calcul d'une intégrale.

En effet nous avons le choix entre
\begin{itemize}
\item trouver une primitive de $f$ comme dans la section précédente et appliquer ensuite la formule du corolaire \ref{ThoRWXooTqHGbC} ;
\item écrire une intégrale pour la nouvelle variable $u = u(x)$ sur l'intervalle entre $u(a)$ et $u(b)$.
\end{itemize}

Nous allons voir ce deux méthodes dans des exemples.

\begin{example}
    Soit à  calculer
    \begin{equation}
        \int_{1/3}^{1/2}x\sqrt{1-x^2}dx.
    \end{equation}
   Les primitives $\int x\sqrt{1-x^2}dx$ ont été trouvé dans l'exemple~\ref{exempleprimitivechangvar}. Une primitive est
    \begin{equation}
        F(x)=\int x\sqrt{1-x^2}dx=-\frac{(1-x^2)^{3/2}}{ 3 }.
    \end{equation}
    Nous pouvons maintenant calculer l'intégrale de $x\sqrt{1-x^2}$ sur l'intervalle $[1/3, 1/2]$ par la définition
    \begin{equation}
        \int_{1/3}^{1/2}x\sqrt{1-x^2}dx=F\left(\frac{ 1 }{2}\right)-F\left(\frac{1}{ 3 }\right)=-\frac{ \sqrt{3} }{ 8 }+\frac{ 16\sqrt{2} }{ 81 }.
    \end{equation}
\end{example}
\begin{remark}
  Pour que le calcul d'intégrale donne quelque chose de sensé il faut absolument que la primitive soit écrite en tant que fonction de $x$ et non comme fonction de $u$. La méthode que nous allons voir dans l'exemple suivant réduit grandement la probabilité d'oublier ce détail, d'où le fait qu'elle soit de loin la plus utilisée.
\end{remark}
\begin{example}
    Calculons à nouveau
    \begin{equation}
        \int_{1/3}^{1/2}x\sqrt{1-x^2}dx.
    \end{equation}
    Cette fois nous allons toucher à l'intervalle d'intégration en même temps que faire le changement de variables. Nous savons déjà les substitutions
    \begin{subequations}
        \begin{numcases}{}
            u=1-x^2\\
            du=-2xdx\\
            dx=\frac{ du }{ -2x }.
        \end{numcases}
    \end{subequations}
    En ce qui concerne les extrêmes d'intégration, si \( x=1/3\) alors \( u=1-\frac{1}{ 9 }=\frac{ 8 }{ 9 }\) et si \( x=\frac{ 1 }{2}\) alors \( u=\frac{ 3 }{ 4 }\). Nous avons donc encore les substitutions suivantes  :
    \begin{subequations}
        \begin{numcases}{}
            x=1/3\to u=8/9\\
            x=1/2\to u=3/4
        \end{numcases}
    \end{subequations}
    Le calcul est alors
    \begin{equation}
        \int_{1/3}^{1/2}x\sqrt{1-x^2}dx=-\frac{ 1 }{2}\int_{8/9}^{3/4}\sqrt{u}du=-\frac{ 1 }{2}\left[  \frac{ u^{3/2} }{ 3/2 }    \right]_{8/9}^{3/4}=-\frac{ \sqrt{3} }{ 8 }+\frac{ 16\sqrt{2} }{ 81 }.
    \end{equation}
    Attention : la dernière égalité n'est pas immédiate; elle demande quelques calculs et une bonne utilisation des règles de puissances.
\end{example}

La deuxième méthode est plus utilisée et, avec un peu d'exercice, plus rapide à mettre en place que la première.

Jusqu'à présent nous avons utilisé des changements de variables dans lesquels nous exprimions \( u\) en termes de \( x\). Comme le montre l'exemple suivant, il est parfois fructueux d'utiliser le changement de variable dans le sens inverse : avec \( x\) exprimé en termes d'un paramètre.

\begin{example}\label{exemplepassagepolaires}
    À calculer :
    \begin{equation}
        \int_{1/2}^{\sqrt{3}/2}\sqrt{1-x^2}dx.
    \end{equation}
    Nous posons \( x=\sin(\theta)\) parce que nous savons que \( 1-\sin^2(\theta)=\cos^2(\theta)\); nous espérons que le changement de variables simplifie l'expression\footnote{Lorsqu'on fait un changement de variables, il s'agit toujours d'\emph{espérer} que l'expression se simplifie. Il n'y a pas moyen de savoir à priori si tel changement de variable va être utile. Il faut essayer.}. Les substitutions à faire dans l'intégrale sont :
    \begin{subequations}
        \begin{numcases}{}
            x=\sin(\theta)\\
            dx=\cos(\theta)d\theta,
        \end{numcases}
    \end{subequations}
    et en ce qui concerne les bornes, si \( x=1/2\) alors \( \sin(\theta)=\frac{ 1 }{2}\), c'est-à-dire \( \theta=\frac{ \pi }{ 6 }\). Si \( x=\sqrt{3}/2\) alors \( \theta=\frac{ \pi }{ 3 }\). Donc
    \begin{equation}
        \int_{1/2}^{\sqrt{3}/2}\sqrt{1-x^2}dx=\int_{\pi/6}^{\pi/3}\sqrt{1-\sin^2(\theta)}\cos(\theta)dt.
    \end{equation}
    Nous avons \( 1-\sin^2(\theta)=\cos^2(\theta)\) et vu que \( \theta\in\mathopen[ \frac{ \pi }{ 6 } , \frac{ \pi }{ 3 } \mathclose]\) nous avons toujours \( \cos(\theta)>0\), ce qui donne \( \sqrt{\cos^2(\theta)}=\cos(\theta)\). Nous devons donc calculer
    \begin{equation}
        \int_{\pi/6}^{\pi/3}\cos^2(\theta)d\theta.
    \end{equation}
    Pour celle-là, il faut utiliser une formule de trigonométrie\footnote{En fait, il y a moyen de terminer le calcul en intégrant deux fois par parties, mais c'est plus compliqué.} :
    \begin{equation}
        \cos^2(\theta)=\frac{ 1+\cos(2\theta) }{ 2 }.
    \end{equation}
    Donc
    \begin{equation}
        \int_{\pi/6}^{\pi/3}\cos^2(\theta)d\theta=\int_{\pi/6}^{\pi/3}\frac{ 1+\cos(2\theta) }{2}d\theta=\left[ \frac{ \theta }{2}\right]_{\pi/6}^{\pi/3}+\int_{\pi/6}^{\pi/3}\frac{ \cos(2\theta) }{2}d\theta,
    \end{equation}
    Pour calculer proprement la dernière intégrale nous effectuons un autre changement de variable (facile) en posant $t = 2\theta$, $dt = 2 d\theta$, $t(\pi/6) = \pi/3$ et $t(\pi/3) = 2\pi/3$, nous avons alors
    \begin{equation}
        \int_{\pi/6}^{\pi/3}\cos^2(\theta)d\theta=\left[ \frac{ \theta }{2}\right]_{\pi/6}^{\pi/3}+\int_{\pi/3}^{2\pi/3}\frac{ \cos(t) }{4}dt  = \left[ \frac{ \theta }{2}\right]_{\pi/6}^{\pi/3}=\frac{ \pi }{ 6 }-\frac{ \pi }{ 12 }=\frac{ \pi }{ 12 },
    \end{equation}
    parce que \( \sin\big( \frac{ 2\pi }{ 3 } \big)=\sin\big( \frac{ \pi }{ 3 } \big)\). Au final,
    \begin{equation}
        \int_{1/2}^{\sqrt{3}/2}\sqrt{1-x^2}dx=\frac{ \pi }{ 12 }.
    \end{equation}
\end{example}

%---------------------------------------------------------------------------------------------------------------------------
\subsection{Intégrations des fractions rationnelles réduites}
%---------------------------------------------------------------------------------------------------------------------------

\begin{definition}
    Une \defe{fraction rationnelle}{fraction rationnelle} est un quotient de deux polynômes à coefficients réels ou complexes.
\end{definition}
Par exemple
\begin{equation}
    \frac{ x^5+7x^4-\frac{ x^3 }{2}+x }{ x^2-1 }
\end{equation}
est une fraction rationnelle.

Il sera expliqué dans le cours d'algèbre que toute fraction rationnelle peut être écrite sous forme d'une somme d'éléments simples, c'est-à-dire de fractions rationnelles d'un des deux types suivants :
\begin{subequations}
    \begin{align}
        \frac{ \alpha }{ (x-a)^m },&& \alpha,a\in \eR,m\in \eN  \label{CasMMIooZnZpUWi}\\
        \frac{ \alpha x+\beta }{ (x^2+ax+b)^m };&&\alpha, \beta ,a,b\in \eR,m\in \eN,a^2-4b<0. \label{CasMMIooZnZpUWii}
    \end{align}
\end{subequations}
Nous allons nous contenter de donner un exemple de chaque type.

\begin{enumerate}
    \item
        En ce qui concerne le cas \eqref{CasMMIooZnZpUWi} avec \( m=1\), nous avons par exemple
        \begin{equation}
            \int\frac{1}{ x-3 }dx=\ln\big( | x-3 | \big)+C .
        \end{equation}
        Si vous voulez en être tout à fait sûr, effectuez d'abord le changement de variables \( u=x-3\) qui donne \( dx=du\).
    \item
        En ce qui concerne le cas \eqref{CasMMIooZnZpUWi} avec \( m\neq 1\), nous avons par exemple
        \begin{equation}
            \int\frac{1}{ (x-1)^4 }dx=-\frac{1}{ 3(x-1)^3 }+C.
        \end{equation}
        Encore une fois, pour s'en convaincre, utiliser le changement de variables \( u=x-1\), \( dx=du\) :
        \begin{equation}
            \int\frac{1}{ (x-1)^4 }dx=\int\frac{1}{ u^4 }du=\int u^{-4}du=-\frac{ u^{-3} }{ 3 }+C=-\frac{1}{ 3 }\frac{1}{ (x-1)^3 }+C.
        \end{equation}
    \item
        En ce qui concerne le cas \eqref{CasMMIooZnZpUWii} avec \( \alpha\neq 0\), nous avons par exemple
        \begin{equation}
            \int\frac{ x }{ x^2+4 }dx=\frac{ 1 }{2}\ln(x^2+4)+C.
        \end{equation}
        Pour ce faire, il faut faire le changement de variables \( u=x^2+4\), \( du=2xdx\), \( dx=\frac{ du }{ 2x }\) qui donne
        \begin{equation}
            \int \frac{ x }{ x^2+4 }dx=\frac{ 1 }{2}\int\frac{ du }{ u }=\frac{ 1 }{2}\ln(| u |)+C=\frac{ 1 }{2}\ln(| x^2+4 |)+C.
        \end{equation}
        Dans ce cas nous pouvons oublier d'écrire la valeur absolue dans le logarithme parce que de toutes façons, \( x^2+4\) est toujours positif.
    \item
        En ce qui concerne le cas \eqref{CasMMIooZnZpUWii} avec \( \alpha= 0\), nous avons par exemple
        \begin{equation}
            \int\frac{ dx }{ x^2+4 }=\frac{1}{ 4 }\int\frac{ dx }{ (\frac{ x }{2})^2+1 }=\frac{ 1 }{2}\arctan(\frac{ x }{2})+C.
        \end{equation}
        où nous avons utilisé la primitive \( \int \frac{dx}{ x^2+1 }dx=\arctan(x)\) du tableau de la page \pageref{PageLCHooMbWjOj}. Pour vous en convaincre vous pouvez faire la dernière étape avec le changement de variables \( u=x/2\), \( dx=2du\).
\end{enumerate}

%---------------------------------------------------------------------------------------------------------------------------
\subsection{Quelques formules à connaitre}
%---------------------------------------------------------------------------------------------------------------------------

\begin{Aretenir}
  \begin{subequations}
    \begin{equation}
      \int \left(\alpha f(x) + \beta g(x)\right) \, dx = \alpha \int f(x) \, dx + \beta \int g(x) \, dx.
    \end{equation}
    \begin{equation}
      \int f(x) g'(x) \, dx = f(x)g(x) - \int f'(x) g(x) \, dx.
    \end{equation}
    \begin{equation}
      \int f'(u(x))u'(x)\, dx = \int f(t)\, dt, \qquad \text{avec } t = u(x).
    \end{equation}
    \begin{equation}
      \int \frac{f'(x)}{f(x)} \, dx = \log |f(x)| + C, \qquad \text{c'est un cas particulier de la formule précédente.}
    \end{equation}
  \end{subequations}
\end{Aretenir}

%--------------------------------------------------------------------------------------------------------------------------- 
\subsection{Approximation de \texorpdfstring{$ \ln(2)$}{ln(2)}}
%---------------------------------------------------------------------------------------------------------------------------

\begin{theorem}     \label{THOooDGCJooXKmFTT}
    Soit une fonction \( f\colon \mathopen[ a , b \mathclose]\to \eR\) de classe \( C^{n+1}\). Alors pour tout \( N\leq n\) nous avons
    \begin{equation}        \label{EQooSCKCooXcKzCc}
        f(b)=f(a)+\sum_{k=1}^N\frac{ f^{(k)}(a) }{ k! }(b-a)^k+\frac{1}{ N! }\int_a^b(b-t)^Nf^{(N+1)}(t)dt.
    \end{equation}
\end{theorem}

\begin{proof}
    Notons que dans l'énoncé, \( n\) est fixé; nous faisons une récurrence sur \( N\). Ça ne change pas grand chose, mais il faut être conscient de ce qui est exactement dans l'hypothèse du théorème et ce qui est dans l'hypothèse de récurrence.

    Bref, \( n\) est fixé, la fonction \( f\) est de classe \( C^{n+1}\) et nous vérifions d'abord la formule avec \( N=1\). À droite dans \eqref{EQooSCKCooXcKzCc} nous avons
    \begin{equation}        \label{EQooETPRooJHcOXh}
        f(a)+f'(a)(b-a)\int_a^b(b-t)f''(t)dt.
    \end{equation}
    Nous évaluons l'intégrale à part en faisant une intégration par parties\footnote{Proposition \ref{PROPooRLFIooQHnyJY}.}. Il s'agit de poser
    \begin{subequations}
        \begin{align}
            u&=b-t\\
            v'&=f'',
        \end{align}
    \end{subequations}
    de déduire
    \begin{subequations}
        \begin{align}
            u'&=-1\\
            v&=f'
        \end{align}
    \end{subequations}
    et d'écrire
    \begin{subequations}
        \begin{align}
            \int_a^b(b-t)f''(t)dt&=\left[ (b-t)f'(t) \right]^b_a-\int_a^b(-)f'(t)dt\\
            &=-(b-a)f'(a)+\int_a^bf'(t)dt\\
            &=-(b-a)f'(a)+f(b)-f(a).
        \end{align}
    \end{subequations}
    Dans le calcul nous avons utilisé le théorème fondamental du calcul intégral \ref{ThoRWXooTqHGbC}. En remettant ça dans \eqref{EQooETPRooJHcOXh} nous trouvons \( f(b)\) comme il se doit.
    
    En ce qui concerne la récurrence, nous devons calculer
    \begin{equation}        \label{EQooKQWZooGBvtlZ}
        f(a)+\sum_{k=1}^{N+1}\frac{ f^{(k)(a)} }{ k! }(b-a)^k+\frac{1}{ (N+1)! }\int_a^b(b-t)^{N+1}f^{(N+2)}(t)dt.
    \end{equation}
    Ici encore, il s'agit de faire une intégration par partie, et sortir de la somme le terme \( k=N+1\). L'intégration par partie donne
    \begin{equation}
        \int_a^b(b-t)^{N+1}f^{(N+2)}(t)dt=-(b-a)^{N+1}f^{(N+1)}(a)+(N+1)\int_a^b(b-t)^Nf^{(N+1)}(t)dt.
    \end{equation}
    En remettant tout ensemble, il y a encore deux termes qui se simplifient, et des termes qui se remettent pour former la formule de récurrence. Bref, on obtient que \eqref{EQooKQWZooGBvtlZ} se réduit bien à \( f(b)\).
\end{proof}

Cette formule avec reste intégral sert par exemple à prouver un encadrement pour \( \ln(2)\), voir la proposition \ref{PROPooHOMYooFclkCU}.

\begin{proposition}[Approximation de \( \ln(2)\)\cite{BIBooSJDCooHjuWeU}]       \label{PROPooHOMYooFclkCU}
    Pour tout \( n\) nous avons
    \begin{equation}
        \left| \ln(2)-\sum_{k=1}^n\frac{ (-1)^{k+1} }{ k } \right|\leq \frac{1}{ n+1 }.
    \end{equation}
\end{proposition}

\begin{proof}
    Nous écrivons la formule de Taylor avec reste intégral du théorème \ref{THOooDGCJooXKmFTT} pour la fonction \( f=\ln\) et pour \( a=1\) et \( b=x\). Cela donne :
    \begin{equation}
        \ln(x)=\ln(1)+\sum_{k=1}^N\frac{ \ln^{(k)}(1) }{ k! }(x-1)^k+\frac{1}{ N! }(x-t)^N\ln^{(N+1)}(t)dt.
    \end{equation}
    Sachant que la dérivée du logarithme\footnote{Voir la proposition \ref{ExZLMooMzYqfK}.} est \( 1/x\) et faisant une petite récurrence, pour \( k\geq 1\) nous avons
    \begin{equation}
        \ln^{(k)}(x)=\frac{ (-1)^{k+1}(k-1)! }{ x^k }.
    \end{equation}
    En remplaçant,
    \begin{equation}
        \ln(x)=\sum_{k=1}^N\frac{ (-1)^{k+1} }{ k }(x-1)^k+\int_1^x\frac{ (-1)^N(x-t)^N }{ t^{N+1} }dt.
    \end{equation}
    C'est le moment de poser \( x=2\) et de faire les simplifications qui s'imposent,
    \begin{equation}
        \ln(2)=\sum_{k=1}^N\frac{ (-1)^{k+1} }{ k }+\int_1^2\frac{ (-1)^N(2-t)^N }{ t^{N+1} }dt.
    \end{equation}
    Nous déplaçons la somme à gauche, et nous prenons la valeur absolue des deux côtés :
    \begin{subequations}
        \begin{align}
            | \ln(2)-\sum_{k=1}^N\frac{ (-1)^{k+1} }{ k } |&=| \int_1^2\frac{ (-1)^N(2-t)^N }{ t^{N+1} }dt |\\
            &\leq\int_1^2\int_1^2\frac{ (2-t)^N }{ t^{N+1} }dt \label{SUBEQooZSXEooVcbJpd}\\
            &\leq \int_1^2(2-t)^N       \label{SUBEQooHGZLooGIhoVt}.
        \end{align}
    \end{subequations}
    Justifications :
    \begin{itemize}
        \item Pour \ref{SUBEQooZSXEooVcbJpd}. Majoration en rentrant la valeur absolue dans l'intégrale, suppression de \( (-1)^N\), et le fait que pour \( t\in \mathopen[ 1 , 2 \mathclose]\), \( 2-t\geq 0\).
        \item Pour \ref{SUBEQooHGZLooGIhoVt}. Majoration en supprimant purement et simplement le dénominateur \( t^N+1\geq 1\).
    \end{itemize}
    Ais-je vraiment besoin de vous dire que la dernière intégrale se calcule en posant le changement de variables\footnote{Proposition \ref{PROPooMVIUooZmvHxS}.} \( u=2-t\) ? Le résultat est que
    \begin{equation}
        \int_1^2(2-t)^Ndt=\frac{1}{ N+1 }.
    \end{equation}
\end{proof}

\begin{example}[\cite{MonCerveau}]      \label{EXooYMEEooMGpUNM}
    La convergence de l'encadrement \eqref{PROPooHOMYooFclkCU} n'est pas terrible. Pour avoir une erreur de \( \frac{1}{ 10 }\), il faut
    \begin{equation}
        \frac{1}{ 10 }=\frac{1}{ n+1 },
    \end{equation}
    ce qui demande \( n=9\). Ça reste jouable, même pour les jeunes d'aujourd'hui. Écrivons \( 9\) termes :
    \begin{equation}
        | \ln(2)-1+\frac{ 1 }{2}-\frac{1}{ 3 }+\frac{1}{ 4 }-\frac{1}{ 5 }+\frac{1}{ 6 }-\frac{ 1 }{ 7 }+\frac{1}{ 8 }-\frac{1}{ 9 } |\leq \frac{1}{ 10 }.
    \end{equation}
    En calculant\footnote{Moi j'ai utilisé Sage, mais si tu es au tableau, débrouilles-toi.},
    \begin{equation}
        | \ln(2)-\frac{ 1879 }{ 2520 } |\leq\frac{1}{ 10 }.
    \end{equation}
    Voici donc un bel encadrement
    \begin{equation}
        \frac{ 1879 }{ 2520 }-\frac{1}{ 10 }\leq \ln(2)\leq \frac{ 1879 }{ 2520 }+\frac{1}{ 10 }.
    \end{equation}
    Pour avoir quelque chose avec des virgules, d'abord un peu de calcul mental donne
    \begin{equation}
        \frac{ 1879 }{ 2520 }\simeq 0.745634920634921.
    \end{equation}
    Donc en majorant et minorant, disons, la troisième décimale\footnote{Notez ici que nous utilisons le fait que la division euclidienne, elle, donne un encadrement pour les fractions. Pensez-y.}, on n'est pas moins précis que le $\frac{1}{ 10 }$. On a
    \begin{equation}
        0.744-\frac{1}{ 10 }\leq \ln(2)\leq 0.746+\frac{1}{ 10 }.
    \end{equation}
    Bref, on retient l'approximation
    \begin{equation}
        0.644\leq \ln(2)\leq 0.846.
    \end{equation}
    
    Pour la quantité de travail, avouez que ce n'est pas terrible comme résultat. Eh oui; le calcul numérique c'est tout un métier; il existe des méthodes nettement plus efficaces que ce que nous venons de faire.
\end{example}


\begin{proposition}[Formule de Taylor avec reste intégral\cite{VBYOJrU,ooSZKEooLejXAh}]\label{PropAXaSClx}
    Soient \( X\) et \( Y\) des espaces normés et un ouvert \( \mO\subset X\). Si \( f\in C^m(\mO,Y)\) et si \( [p,x]\subset \mO\) alors
    \begin{equation}
        \begin{aligned}[]
            f(x)=f(p)&+\sum_{k=1}^{m-1}\frac{1}{ k! }(d^kf)_p (x-p)^k \\
            &+\frac{1}{ (m-1)! }\int_0^1(1-t)^{m-1}(d^mf)_{ p+t(x-p) }(x-p)^m\,dt 
        \end{aligned}
    \end{equation}
    où \( \omega_pu^k\) signifie \( \omega_p(u,\ldots, u)\) lorsque \( \omega\in \Omega^k\).
\end{proposition}
\index{formule!Taylor!reste intégral}

Notez que l'intégrale n'est pas une intégrale faisant intervenir les espaces \( X\) ou \( Y\). Elle est une simple intégrale d'une fonction \( \eR\to \eR\), comme définie par la mesure de Lebesgue de la définition \ref{DefooYZSQooSOcyYN}.

Comme expliqué dans l'exemple \ref{ExZHZYcNH}, toute ces applications de différentielles se réduisent à des termes de la forme
\begin{equation}
    f^{(k)}(p)(x-p)^k
\end{equation}
dans le cas d'une fonction \( \eR\to\eR\).


%+++++++++++++++++++++++++++++++++++++++++++++++++++++++++++++++++++++++++++++++++++++++++++++++++++++++++++++++++++++++++++
\section{Constructions plus naïves de l'intégrale dans le cas réel}
%+++++++++++++++++++++++++++++++++++++++++++++++++++++++++++++++++++++++++++++++++++++++++++++++++++++++++++++++++++++++++++

Les sections~\ref{SecSLOooeMaig} et~\ref{SecZTFooXlkwk} ont donné une construction très complète de la mesure de Lebesgue, et nous avons définit la théorie de l'intégration sur un espace mesuré quelconque dans la définition~\ref{DefTVOooleEst}.

Dans cette section nous allons donner différentes choses plus rapides qui servent souvent de définition dans les cours moins avancés.

%---------------------------------------------------------------------------------------------------------------------------
\subsection{Mesure de Lebesgue, version rapide}
%---------------------------------------------------------------------------------------------------------------------------

Nous construisons à présent la mesure de Lebesgue sur \( \eR^n\). Un \defe{pavé}{pavé} dans \( \eR^n\) est un ensemble de la forme
\begin{equation}
    B=\prod_{i=1}^n\mathopen[ a_i , b_i \mathclose];
\end{equation}
le volume d'un tel pavé est défini par \( \Vol(B)=\prod_i(b_i-a_i)\). Soit maintenant \( A\subset \eR^n\). La \defe{mesure externe}{mesure!externe} de \( A\) est le nombre
\begin{equation}
    m^*(A)=\inf\{ \sum_{B\in\mF}\Vol(B)\text{ où } \mF\text{ est un ensemble dénombrable de pavés dont l'union recouvre } A\text{.} \}
\end{equation}

\begin{definition}  \label{DefKTzOlyH}
Nous disons que \( A\) est \defe{mesurable}{mesurable!Lebesgue} au sens de Lebesgue si pour tout ensemble \( S\subset \eR^n\) nous avons l'égalité
\begin{equation}
    m^*(S)=m^*(A\cap S)+m^*(S\setminus A).
\end{equation}
Dans ce cas nous disons que la mesure de Lebesgue de \( A\) est \( m(A)=m^*(A)\).
\end{definition}

\begin{proposition}     \label{PropNCMToWI}
    Deux fonctions continues égales presque partout pour la mesure de Lebesgue\footnote{Définition~\ref{DefKTzOlyH}.} sont égales.
\end{proposition}

\begin{proof}
    Soient \( f\) et \( g\) deux fonctions continues telles que \( f(x)=g(x)\) pour presque tout \( x\in D\). La fonction \( h=f-g\) est alors presque partout nulle et nous devons prouver qu'elle est nulle sur tout \( D\). La fonction \( h\) est continue; si \( h(a)\neq 0\) pour un certain \( a\in D\) alors \( h\) est non nulle sur un ouvert autour de \( a\) par continuité et donc est non nulle sur un ensemble de mesure non nulle.
\end{proof}

%---------------------------------------------------------------------------------------------------------------------------
\subsection{Pavés et subdivisions}
%---------------------------------------------------------------------------------------------------------------------------

\begin{definition}
 Nous appelons \defe{pavé}{pavé} de $\eR^p$ toute partie de $\eR^p$ obtenue comme produit de $p$ intervalles de $\eR$. Plus explicitement, une partie $R$ est un pavé de $\eR^p$ s'il s'écrit sous la forme
\[
R=\left\{(x_1,\ldots, x_p)\in\eR^p \,\big\vert\,x_i\in \mathcal{I}_i,  i=1,\ldots, p  \right\},
\]
où $\mathcal{I}_i$ est un intervalle de $\eR$ pour tout $i=1,\ldots, p$.
\end{definition}
On appelle pavé fermé de $\eR^p$ le produit de $p$ intervalles fermés
\[
R=\prod_{i=1}^{p}[a_i,b_i].
\]
On définit de même le pavé ouvert
\[
S=\prod_{i=1}^{p}]a_i,b_i[.
\]
Un pavé $ R=\prod_{i=1}^{p}\mathcal{I}_i$ est dit borné si tous les intervalles $\mathcal{I}_i$ sont bornés dans $\eR$. Les pavés non bornés sont des produits d'intervalles où un (ou plusieurs) des intervalles n'est pas borné. Par exemple,
\[
N=]-\infty, 5]\times [0,13].
\]
L'espace $\eR^p$, lui-même, est un pavé de $\eR^p$.
\begin{definition}
  Une partie $A$ de $\eR^p$ est dite  \defe{pavable}{pavable} s'il existe une famille finie de pavés bornés $R_j$, $j=1,\ldots, n$, et deux à deux disjoints tels que
\[
A=\bigcup_{j=1}^{n}R_j.
\]
\end{definition}
Un exemple d'ensemble pavable dans $\eR^2$ est donné à la figure~\ref{LabelFigPolirettangolo}. Il existe beaucoup d'ensembles dans $\eR^2$ qui ne sont pas pavables, par exemple les ellipses.
\newcommand{\CaptionFigPolirettangolo}{Un ensemble pavable.}
\input{auto/pictures_tex/Fig_Polirettangolo.pstricks}

Le complémentaire d'un pavé est  un ensemble pavable et, en particulier, tout complémentaire d'un pavé borné est une réunion de  pavés non bornés. Toute union finie et toute intersection d'ensemble pavables est pavable.
\begin{definition}
	Soit $R$ un pavé borné de $\eR^p$, pour fixer les idées on peut penser $R=\prod_{i=1}^{p}[a_i,b_i]$. On appelle \defe{longueur}{longueur!d'une arrête} de l'$i$-ème arrête de $R$ le nombre $b_i-a_i$. La \defe{mesure $p$-dimensionnelle de $R$}{}, $m(R)$, est le produit des longueurs
\[
m(R)=\prod_{i=1}^{p}(b_i-a_i).
\]
\end{definition}
\begin{example}
  Dans $\eR^3$, l'ensemble $R=[-1,1]\times[3,4]\times[0,2]$ est un pavé fermé de mesure
\[
m(R)= (1+1)\cdot(4-3)\cdot(2-0)=4.
\]
Il s'agit du volume usuel du parallélépipède rectangle.
\end{example}

\begin{example}
 L'ensemble $R=\mathopen] -1 , 1 \mathclose[\times[3,4]\times[0,2]$ est un pavé de $\eR^3$. Il n'est ni fermé ni ouvert, sa mesure est encore $4$.
\end{example}

Si $R$ est un pavé non borné on peut encore définir sa mesure. La notion de mesure se généralise en deux étapes. D'abord on dit que la longueur d'une arête non bornée est $\infty$. Ensuite, on adopte la convention $0\cdot \infty=0$. Il faut remarquer que avec cette généralisation tout point et toute droite dans $\eR^2$ ont mesure nulle.

Afin de définir les intégrales, nous allons intensivement faire appel à la notion de subdivision d'intervalles, voir définition~\ref{DefSubdivisionIntervalle} et la discussion qui suit.

Lorsqu'on considère un pavé borné $R=\prod_{i=1}^p\mI_i$ de $\eR^p$, on note $\sdS_i$ l'ensemble des subdivisions de l'intervalle $\mI_i$. La notion de subdivision de généralise au cas des pavés.
\begin{definition}
	Soir $R$ un pavé fermé borné de $\eR^p$, pour fixer les idées on peut penser à $R=\prod_{i=1}^p\mathopen[ a_i , b_i \mathclose]$. On appelle \defe{subdivision}{subdivision} finie de $R$ les éléments de l'ensemble $\mathcal{S}=\prod_{i=1}^{p}\mathcal{S}_i$,
\[
\mathcal{S}=\left\{ (Y_{1},\ldots, Y_{p})\,\big\vert\, Y_{i}=(y_{i,j})_{j=1}^{n_i}\in\mathcal{S}_i,\, i=1,\ldots,p\right\}.
\]
On peut définir de même l'ensemble des subdivisions d'un pavé non borné.
 \end{definition}
 Souvent, une subdivision d'un pavé $R=\prod_{i=1}^p\mI_i$ sera noté $\sigma=(y_{i,j})_{j=1}^{n_i}$. Dans cette notation, on sous-entend que pour chaque $i$ fixé, les nombres $y_{i,j}$ (il y en a $n_i$) forment une subdivision de l'intervalle $\mI_i$. Afin de vous familiariser avec ces notations, repérez bien tous les éléments de la figure~\ref{LabelFigUneCellule}.
\newcommand{\CaptionFigUneCellule}{Une cellule d'une subdivision d'un pavé de $\eR^2$. La cellule grisée est $R_{(4,2)}$.}
\input{auto/pictures_tex/Fig_UneCellule.pstricks}

%On désigne par
%\[
%\delta(Y_i)=\max_{0\leq j\leq n}| y_{i,j}- y_{i,j-1}|,
%\]
%le pas de la subdivision $Y_i$ dans $\mathcal{S}_i$ et par
%\[
%\delta(\sigma)=\max_{0\leq i\leq p}\delta(Y_i),
%\]
%le pas de la subdivision $\sigma$ dans $\mathcal{S}$.

\begin{definition}
	Si $\sigma$ est une subdivision d'un pavé $R$, un \defe{raffinement}{raffinement!subdivision d'un pavé} de $\sigma$ est une subdivision de $R$ obtenue en fixant plus de points dans chaque intervalle.
\end{definition}

La subdivision $\sigma$ de $R$ détermine $n_1n_2\ldots n_p$ pavés fermés de la forme
\[
R_{(k_1,\ldots,k_p)}=\{(x_1,\ldots, x_p)\in\eR^p\,\big\vert\, y_{i,k_{i-1}}\leq x_i\leq y_{i,k_i}\},
\]
où $k_i$ est dans $\{1,\ldots, n_i\}$ et $i$ dans $\{1,\ldots, p\}$. On les appelles \defe{cellules}{cellule d'un pavage} de $\sigma$. On remarque que les cellules de $\sigma$ sont toujours deux à deux disjointes (sauf au plus sur leurs bords).
\begin{lemma}\label{meas_sous}
	Soit $R$ un pavé borné de $\eR^p$ et soit $\sigma=(y_{i,j})_{j=1}^{n_i}$ une subdivision de $R$.
On a
\[
m(R)=\sum_{(k_1,\ldots,k_p)\in K} m(R_{(k_1,\ldots,k_p)}),
\]
où $K=\{1,\ldots,n_1\}\times\{1,\ldots,n_2\}\times\ldots \times\{1,\ldots,n_p\}$.
\end{lemma}
Le lemme~\ref{meas_sous} suggère de définir la mesure d'un ensemble borné pavable $P=\bigcup_{j=1}^{n}R_j$ comme la somme des mesures des pavés disjoints $R_j$, $j=1,\ldots, n$.
\begin{definition}
Une application $f:\eR^p\to\eR$ est dite \defe{application en escalier}{application!en escalier} sur $\eR^m$ si
  \begin{itemize}
  \item $f$ est une application bornée,
\item il existe une subdivision $\sigma$ de $\eR^p$ telle que la restriction de $f$  est une application constante sur toute cellule $R_k$ de $\sigma$
\[
f_{\vert_{R_k}}=C_k, \qquad C_k\in\eR,
\]
%Pour tout $k=(k_1,\ldots,k_p)$ dans $ K=\{1,\ldots,n_1\}\times\{1,\ldots,n_2\}\times\ldots \times\{1,\ldots,n_p\}$.

Une telle subdivision $\sigma$ est dite \defe{associée}{associée!subdivision}\index{subdivision!associée à une fonction} à $f$.
  \end{itemize}
\end{definition}
\begin{example}
  La fonction $f$ de $\eR^2$ dans $\eR$ définie par
  \begin{equation}
    f(x,y)=\left\{
    \begin{array}{ll}
      1&\qquad \textrm{si } (x,y) \in [0,3]\times[-1,2],\\
2 &\textrm{sinon.}
    \end{array}\right.
  \end{equation}
est une application en escalier. Exercice : donner une subdivision de $\eR^2$ associée à cette fonction.
\end{example}

\begin{example}
  La fonction $f$ de $\eR^2$ dans $\eR$ définie par
  \begin{equation}
    f(x,y)=\left\{
    \begin{array}{ll}
      \frac{1}{m^2+n^2},&\qquad \textrm{si } (x,y) \in [m,m+1]\times[n,n+1], \quad m,\,n\in\eN_0,\\
0, &\textrm{sinon}
    \end{array}\right.
  \end{equation}
est une application en escalier. Observez que, dans ce cas, il n'existe pas une subdivision finie de $\eR^2$ associée à $f$.
\end{example}
\begin{remark}
 Si la subdivision $\sigma$ est associée à $f$ alors tout raffinement de $\sigma$ (c'est-à-dire, toute subdivision obtenue en fixant plus de points dans chaque intervalle) a la même propriété.

Si $f$ et $g$ sont deux applications en escalier sur $R$ et $\sigma_f$ et $\sigma_g$ sont des subdivisions de $R$ associées respectivement à $f$ et $g$, alors on peut construire une troisième subdivision de $R$ qui est associée à $f$ et à $g$ en même temps. Soient $\sigma_f=(Y_{1},\ldots, Y_{p})$ et $\sigma_g=(Z_{1},\ldots, Z_{p})$, où $Y_{i}=(y_{i,j})_{j=1}^{m_i}$ et $Z_{i}=(z_{i,j})_{j=1}^{n_i}$ sont des subdivisions de l'intervalle $[a_i, b_i]$, pour $i=1,\ldots, p$. La subdivision de $[a_i, b_i]$ obtenue par l'union de $Y_i$ et $Z_i$ est encore une subdivision finie, qu'on appellera $\bar Y_i$. La subdivision $\bar \sigma = (\bar Y_{1},\ldots,\bar Y_{p})$ de $R$ est un raffinement de $\sigma_f $ et de $\sigma_g$, donc elle est associée à la fois à $f$ et à $g$.

Cela nous permet de prouver que si $f$ et $g$ sont des applications en escalier, alors $f+g$, $fg$, $\min\{f,g\}$, $\max\{f,g\}$ et $|f|$ sont des applications en escalier.
\end{remark}

%---------------------------------------------------------------------------------------------------------------------------
\subsection{Intégrale d'une fonction en escalier}
%---------------------------------------------------------------------------------------------------------------------------

\begin{definition}
  Soit $f$ une fonction de $\eR^m$ dans $\eR^n$. Le \defe{support}{support} de $f$ est la fermeture de l'ensemble des points $x$ tels que $f(x)\neq 0$.
\end{definition}
\begin{definition}
Une application en escalier $f$ est dite \defe{intégrable}{fonction!en escalier intégrable} si son support est compact.
\end{definition}
Soit $f$ une application en escalier sur $\eR^p$. Soit $\sigma$ une subdivision de  $\eR^p$ associée à $f$ et appelons $R_k$ les cellules de $\sigma$, avec $k=(k_1,\ldots,k_p)$ dans $ K=\{1,\ldots,n_1\}\times\{1,\ldots,n_2\}\times\ldots \times\{1,\ldots,n_p\}$. Alors
\[
f_{\vert_{R_k}}=C_k, \qquad C_k\in\eR.
\]

\begin{definition}
On définit l'\defe{intégrale}{intégrale!fonction en escalier} de $f$ sur $\eR^p$ par
\[
\int_{\eR^p}f\,dV=\sum_{k\in K}C_km(R_k).
\]
\end{definition}
L'intégrale ainsi définie est un nombre réel. La proposition suivante nous dit que l'intégrale est «bien définie», au sens que sa valeur ne dépend pas de la subdivision associée à $f$ qu'on utilise dans le calcul.
\begin{proposition}
Soit $f$ une application en escalier intégrable sur $\eR^p$. Soient $\sigma_1$ et $\sigma_2$ deux subdivisions de $\eR^p$ associées à  $f$. L'intégrale de $f$ ne dépend pas de la subdivision choisie.
\end{proposition}
On ne donne pas une preuve complète de cette proposition. En fait elle est une conséquence de la formule de réduction introduite dans la suite de ce chapitre.


%%%%%%%%%%%%%%%%%%%%%%%%%%%%%%%%%%%%%%%%%%%%%%%%%%%%%%%%%%%%%%%%%%%%%%%%%%%%%%%%
\subsection{Intégrales partielles}
%%%%%%%%%%%%%%%%%%%%%%%%%%%%%%%%%%%%%%%%%%%%%%%%%%%%%%%%%%%%%%%%%%%%%%%%%%%%%%%%
Soit $f$ de $\eR^p$ dans $\eR$ une fonction continue, nulle hors du pavé borné $R$. Posons  $R=\prod_{i=1}^{p}[a_i,b_i]$, pour fixer les idées. Pour chaque $i$ dans $\{1,\ldots, p\}$ fixé, on peut associer à $f$ la fonction $F_i$ de $p-1$ variables définie par
\[
F_i(x_1,\ldots, x_{i-1}, x_{i+1}, \ldots, x_p)=\int_{a_i}^{b_i}f(x_1,\ldots, x_{i-1},y, x_{i+1}, \ldots, x_p)\, dy.
\]
La fonction $F_i$ est l'intégrale partielle de $f$ par rapport à la $i$-ème variable.
En particulier, si $f(x_1,\ldots, x_p)=g(x_i)h(x_1,\ldots, x_{i-1}, x_{i+1}, \ldots, x_p)$ on obtient
\[
F_i=\int_{a_i}^{b_i}g(y)h(x_1,\ldots, x_{i-1}, x_{i+1}, \ldots, x_p)\, dy= h\cdot\int_{a_i}^{b_i}g \, dy.
\]
La fonction d'une seule variable qu'on obtient à partir de $f$ en fixant $x_1,\ldots, x_{i-1}, x_{i+1}, \ldots, x_p$ et qui associe à $x_i$ la valeur $f(x_1,\ldots, x_{i-1}, x_i, x_{i+1}, \ldots, x_p)$, est appelée $x_i$-ème section de $f$ en $x_1,\ldots, x_{i-1}, x_{i+1}, \ldots, x_p$.

\begin{example}
  Soit $f$ la fonction de $\eR^2$ dans $\eR$ définie par
  \begin{equation}
	  f(x,y)=\begin{cases}
		  x+3y	&	\text{si }(x,y)\in\mathopen[ 9 , 10 \mathclose]\times\mathopen] \pi , 5 \mathclose]\\
		  0	&	 \text{sinon}.
	  \end{cases}
  \end{equation}
 Les intégrales partielles de $f$ sont
\[
F_1(y)=\int_{9}^{10}x+3y\,dx=\left[\frac{x^2}{2}+3xy\right]_{x=9}^{x=10}=\frac{19}{2}+3y,
\]
\[
F_2(x)=\int_{\pi}^{5}x+3y\,dy=\left[xy+\frac{3y^2}{2}\right]_{y=\pi}^{y=5}=x(5-\pi)+\frac{3}{2}(25-\pi^2).
\]
\end{example}
%%%%%%%%%%%%%%%%%%%%%%%%%%%%%%%%%%%%%%%%%%%%%%%%%%%%%%%%%%%%%%%%%%%%%%%%%%%%%%%%
\subsection{Réduction d'une intégrale multiple}
%%%%%%%%%%%%%%%%%%%%%%%%%%%%%%%%%%%%%%%%%%%%%%%%%%%%%%%%%%%%%%%%%%%%%%%%%%%%%%%%

Soit $R=[a,b]\times[c,d]$ un pavé fermé et borné de $\eR^2$ et soit $f$ une application en escalier intégrable sur $\eR^2$ telle que le support de $f$ soit contenu dans $R$. On considère la subdivision $\sigma$ de $R$ définie par les subdivisions
\[
a=x_0\leq x_1\leq\ldots\leq x_m=b,
\]
 \[
c=y_0\leq y_1\leq\ldots\leq y_n=d.
\]
Les cellules de $\sigma$ sont
\[
R_{i,j}=[x_{i},x_{i+1}]\times[y_{j},y_{j+1}], \quad\qquad i=0,\ldots,m-1, \quad j=0,\ldots,n-1.
\]
La mesure de $R$ est la somme des mesures des $R_{i,j}$
\begin{equation}
  \begin{aligned}
    m(R)=&\sum_{(i,j)\in \{0,\ldots, m-1\}\times\{0,\ldots, n-1\}} m(R_{i,j})=\\
&=\sum_{j=0}^{n-1}\sum_{i=0}^{m-1}(x_{i+1}-x_{i})\cdot(y_{i+1}-y_{i})=\\
&=\sum_{i=0}^{m-1}(x_{i+1}-x_{i})\cdot\sum_{j=0}^{n-1}(y_{i+1}-y_{i})=\\
&= (b-a)\cdot(d-c).
  \end{aligned}
\end{equation}
Si $f$ est constante sur chaque cellule de $\sigma$ on peut écrire $f$ de la forme suivante
\[
f(x,y)=\sum_{j=0}^{n-1}\sum_{i=0}^{m-1}C_{i,j}\,\chi_{R_{i,j}}
\]
où les $C_{i,j}$ sont des constantes réelles et $\chi_{R_{i,j}}$ est la \defe{fonction caractéristique}{fonction!caractéristique} de $R_{i,j}$
\begin{equation}
  \chi_{R_{i,j}}(x,y)=\left\{
      \begin{array}{ll}
      1,\qquad &\textrm{si } (x,y)\in R_{i,j} ,\\
0, & \textrm{sinon}.
      \end{array}\right.
\end{equation}
Comme $(x,y)$ est dans $R_{i,j}$ si et seulement si $x\in[x_{i},x_{i+1}]$ et $ y\in[y_{j},y_{j+1}]$, on vérifie que la fonction $\chi_{R_{i,j}}$ est égal au produit des fonctions caractéristiques des intervalles $[x_{i},x_{i+1}]$ et $[y_{j},y_{j+1}]$
\[
 \chi_{R_{i,j}}(x,y)=\chi_{[x_{i},x_{i+1}]}(x)\cdot\chi_{[y_{j},y_{j+1}]}(y).
\]
On peut donc écrire la fonction $f$ de la façon suivante
\[
f(x,y)=\sum_{j=0}^{n-1}\sum_{i=0}^{m-1}C_{i,j}\,\chi_{[x_{i},x_{i+1}]}(x)\cdot\chi_{[y_{j},y_{j+1}]}(y).
\]
Comme on suppose que le support de $f$ est une partie de $R$, l'intégrale de $f$ sur $\eR^2$ est
\begin{equation}
  \begin{aligned}
\int_{\eR^2}f \,dV = \sum_{j=0}^{n-1}\sum_{i=0}^{m-1}C_{i,j}\,m(R_{i,j})=\sum_{j=0}^{n-1}\sum_{i=0}^{m-1}C_{i,j}\,(x_{i+1}-x_i)\cdot(y_{j+1}-y_j).
 \end{aligned}
\end{equation}
Cette intégrale peut être réduite à la composition de deux intégrales partielles. Il suffit de remarquer que la valeur de l'intégrale de la fonction caractéristique d'un intervalle est la longueur de l'intervalle,
\begin{equation}
  \begin{aligned}
    C_{i,j}(x_{i+1}-x_i)&\cdot(y_{j+1}-y_j)=\\
&=C_{i,j}\left(\int_{x_i}^{x_{i+1}}\chi_{[x_{i},x_{i+1}]}(x)\, dx \right)\cdot \left(\int_{y_j}^{y_{j+1}}\chi_{[y_{ j},y_{ j+1}]}(y)\, dy \right)=\\
&=C_{i,j}\left(\int_{a}^{b}\chi_{[x_{i},x_{i+1}]}(x)\, dx \right)\cdot \left(\int_{c}^{d}\chi_{[y_{ j},y_{ j+1}]}(y)\, dy \right),
  \end{aligned}
\end{equation}
et utiliser les propriétés de linéarité de l'intégrale
\begin{equation}
  \begin{aligned}
   \int_{\eR^2}f \,dV =& \sum_{j=0}^{n-1}\sum_{i=0}^{m-1}C_{i,j}\,\left(\int_{a}^{b}\chi_{[x_{i},x_{i+1}]}(x)\, dx \right)\cdot \left(\int_{c}^{d}\chi_{[y_{ j},y_{ j+1}]}(y)\, dy \right)=\\
&=\int_{c}^{d}\int_{a}^{b}\sum_{j=0}^{n-1}\sum_{i=0}^{m-1}C_{i,j}\,\chi_{[x_{i},x_{i+1}]}(x)\cdot \chi_{[y_{ j},y_{ j+1}]}(y)\, dx dy=\\
&=\int_{c}^{d}\int_{a}^{b} f\, dx dy.
  \end{aligned}
\end{equation}
De même on obtient
\begin{equation}
  \begin{aligned}
   \int_{\eR^2}f \,dV =&\int_{a}^{b}\int_{c}^{d}\sum_{j=0}^{n-1}\sum_{i=0}^{m-1}C_{i,j}\,\chi_{[x_{i},x_{i+1}]}(x)\cdot \chi_{[y_{ j},y_{ j+1}]}(y)\, dx dy=\\
&=\int_{a}^{b}\int_{c}^{d} f\, dx dy.
  \end{aligned}
\end{equation}
En général, on prouve la proposition suivante
\begin{proposition}
 Soit $f$ une application en escalier intégrable sur $\eR^p$ et soit $R$ un pavé borné dans $\eR^p$ qui contient le support de $f$. Comme d'habitude, pour fixer les idées nous écrivons $=\prod_{i=1}^p[a_i,b_i]$. Alors
 \begin{equation}
   \begin{aligned}
     \int_{\eR^p}f(x_1,\ldots, x_p) \, dV =& \int_{a_p}^{b_p}\int_{a_{p-1}}^{b_{p-1}}\cdots\int_{a_1}^{b_1} f(x_1,\ldots, x_p) \, dx_1\cdots dx_p=\\
&=\int_{a_{s_p}}^{b_{s_p}}\int_{a_{s_{p-1}}}^{b_{s_{p-1}}}\cdots\int_{a_{s_1}}^{b_{s_1}} f(x_1,\ldots, x_p) \, dx_1\cdots dx_p,
   \end{aligned}
 \end{equation}
pour toute permutation $(s_1,\ldots,s_p)$ de l'ensemble $\{1,\ldots p\}$.
\end{proposition}

%---------------------------------------------------------------------------------------------------------------------------
\subsection{Propriétés de l'intégrale}
%---------------------------------------------------------------------------------------------------------------------------

Soient $f$ et $g$ deux fonctions en escalier intégrables de $\eR^p$ dans $\eR$, et soient $a$ et $b$ dans $\eR$.
\begin{description}
\item[Linéarité de l'intégrale] :
  \begin{itemize}
  \item Additivité : $f+g$ est intégrable et
\[
\int_{\eR^p} (f+g)\, dV = \int_{\eR^p} f\, dV+ \int_{\eR^p} g\, dV,
\]
\item Homogénéité : $\lambda f$ est intégrable pour tout réel $\lambda$
\[
\int_{\eR^p} \lambda  f\, dV = \lambda\int_{\eR^p} f\, dV,
\]
  \end{itemize}
\item[Monotonie] Si $f\leq g$ alors
\[
 \int_{\eR^p} f\, dV\leq \int_{\eR^p} g\, dV,
\]
\item[Inégalité fondamentale]
  \[
\lvert \int_{\eR^p}f\,dV\rvert \leq\int_{\eR^p}\lvert f\rvert\,dV.
\]
Cette dernière inégalité s'obtient de la façon suivante :
\[
\lvert\int_{\eR^p}f\,dV\rvert =\lvert \sum_{k\in K} C_k m(R_k)\rvert \leq\sum_{k\in K}\lvert C_k\rvert m(R_k)=\int_{\eR^p}|f|\,dV.
\]
\item[Inégalité de Čebičeff]  Si $f$ est une application en escalier alors pour tout $a>0$ dans $\eR$ l'ensemble $\{x\in\eR^p\,:\, |f(x)|\geq a\}$ est pavable et borné, et l'inégalité suivante est satisfaite
\[
m\left(\{x\in\eR^p\,:\, |f(x)|\geq a\}\right)\leq \frac{1}{a} \int_{\eR^p}\lvert f\rvert\,dV.
\]
\end{description}

%---------------------------------------------------------------------------------------------------------------------------
\subsection{Intégrales multiples, cas général}
%---------------------------------------------------------------------------------------------------------------------------

Nous voulons généraliser la définition d'intégrale multiple au cas des domaines non pavables et de fonctions qui ne sont pas en escalier. Il y a plusieurs méthodes de le faire et ici on ne considère qu'une seule, introduite par Riemann.
\begin{definition} Soit $f: \eR^p\to \eR$ une fonction.
  \begin{itemize}
	  \item Pour toute application en escalier intégrable $f_*$ telle que $f_*\leq f$, l'intégrale de $f_*$ est dit une \defe{somme inférieure}{somme!inférieure} de $f$.
	  \item Pour toute application en escalier intégrable $f^*$ telle que $f_*\geq f$, l'intégrale de $f^*$ est dit une \defe{somme supérieure}{somme!supérieure} de $f$.
  \end{itemize}
\end{definition}
Soient $\sum_* f$ et  $\sum^* f$ les ensembles des sommes inférieures et supérieures de $f$. Grâce à la propriété de  monotonie de l'intégrale on sait que si $a$ est dans $\sum_* f$ et  $b$ est dans $\sum^* f$ alors $a\leq b$.
\begin{definition}
  La fonction $f$ est intégrable (au sens de Riemann) si $\sum_* f$ et  $\sum^* f$ ne sont pas vides et
\[
\inf \Sigma^* f=I =\sup \Sigma_* f.
\]
Dans ce cas, la valeur $I$ est appelée intégrale de $f$ sur $\eR^p$.
\end{definition}
\begin{remark}
  Toute fonction intégrable est bornée et à support compact. En effet, si le support de la  fonction n'est pas compact alors soit $\sum_* f$ soit $\sum^* f$ doit être vide !
\end{remark}
L'intégrale qu'on vient de définir possède toutes les propriétés de l'intégrale pour les fonctions en escalier. Le produit de deux fonctions intégrables est intégrable.

Il y a des cas où l'intégrabilité d'une fonction n'est pas évidente. Cependant, dans la plupart des exercices et des exemples de ce cours, nous nous aidons avec le critère suivant
\begin{proposition}
  Toute fonction continue à support compact est intégrable.
\end{proposition}
Cette proposition n'est à priori pas étonnante, vu qu'une fonction continue sur un support compact est bornée (théorème de Weierstrass~\ref{ThoWeirstrassRn}).

%%%%%%%%%%%%%%%%%%%%%%%%%%%%%%%%%%%%%%%%%%%%%%%%%%%%%%%%%%%%%%%%%%%%%%%%%%%%%%%%
\subsection{Réduction d'une intégrale multiple}
%%%%%%%%%%%%%%%%%%%%%%%%%%%%%%%%%%%%%%%%%%%%%%%%%%%%%%%%%%%%%%%%%%%%%%%%%%%%%%%%
On n'utilise jamais la définition pour calculer la valeur d'une intégrale multiple. La méthode plus efficace, en pratique, est de réduire l'intégrale à la composition de plusieurs intégrales d'une variable.
\begin{theorem}[de Fubini]\label{fub}
 Soit $f$ une fonction intégrable de $\eR^2$ dans $\eR$. Si pour tout $x$ dans $\eR$ la section $f(x,\cdot)$ est intégrable par rapport à $y$, alors
\[
\int_{\eR^2}f(x,y)\,dV=\int_{\eR}\left(\int_{\eR}f(x,y)\,dx\right)\,dy.
\]
De même, si pour tout $y$ dans $\eR$ la section $f(\cdot, y)$ est intégrable par rapport à $x$, alors
\[
\int_{\eR^2}f(x,y)\,dV=\int_{\eR}\left(\int_{\eR}f(x,y)\,dy\right)\,dx.
\]
\end{theorem}		\label{ThoSectionINte}
En général, on ne peut pas dire que les sections d'une fonction intégrable sont intégrables, donc il faut vraiment se souvenir des hypothèses du théorème~\ref{fub}. En dimension plus haute, on a le même résultat
\begin{theorem}
 Soit $f$ une fonction intégrable de $\eR^p$ dans $\eR$. Si pour tout $(p-1)$-uple $(x_1,\ldots, x_{i-1},x_{i+1}, \ldots, x_p)$ dans $\eR^{p-1}$ la section $f(x_1,\ldots, x_{i-1},\cdot,x_{i+1}, \ldots, x_p)$ est intégrable par rapport à $x_i$, alors
\[
\int_{\eR^p}f \,dV=\int_{\eR}\left(\int_{\eR^{p-1}}f \,dV\right)\,dx_i.
\]
\end{theorem}

 Si $f$ est une fonction positive et intégrable de $\eR^2$ dans $\eR$ on peut interpréter l'intégrale de $f$ comme le volume du solide au-dessous du graphe de $f$.  Avec cette interprétation,  l'intégrale partielle par rapport à $x$ pour $y=y_0$ fixé est l'aire de la tranche qu'on obtient en coupant le solide par le plan $y=y_0$.

 \begin{example}
   Le premier exemple à faire est celui d'une fonction en escalier intégrable et positive. Soit $f\colon \eR^2\to \eR$ la fonction
\begin{equation}
	f(x,y)=\begin{cases}
		1	&	\text{si }(x,y)\in R_1=\mathopen] -1 , 3 \mathclose]\times\mathopen[ 4 , 5 \mathclose]\\
		3	&	 \text{si }(x,y)\in R_2=\mathopen] 13 , 15 \mathclose[\times\mathopen[ 0 , 2 \mathclose[\\
		0	&	 \text{dans les autres cas.}
	\end{cases}
\end{equation}
L'intégrale de $f$ sur $\eR^2$ est $1\cdot m(R_1)+ 3\cdot m(R_2)= 16$. On voit tout de suite qu'il s'agit de la somme du volume des deux parallélépipèdes de hauteurs respectives $1$ et $3$ et bases $R_1$ et $R_2$.
 \end{example}

\begin{example}
On veut calculer le volume du solide $S$, borné par le paraboloïde elliptique $x^2+2y^2+z=16$ et le plans $x=2$, $x=0$, $y=2$ $y=0$, $z=0$. On observe que la portion de  paraboloïde elliptique qui nous intéresse est le graphe de la fonction $f(x,y)=16-x^2-2y^2$ pour $(x,y)$ dans $R=[0,2]\times[0,2]$. La fonction $f$ est continue ainsi que ses sections, donc on peut appliquer le théorème~\ref{fub} et décomposer l'intégrale double en deux intégrales simples :
\begin{equation}
  \begin{aligned}
   & \int_R 16-x^2-2y^2 \,dV= \int_{0}^2\int_{0}^2f(x,y)\,dx dy= \\
&=\int_0^2 \left[(16-2y^2)x-\frac{x^3}{3}\right]_{x=0}^{x=2}\, dy =\\
& = \left[ \left(32-\frac{8}{3}\right) y -\frac{4y^3}{3}\right]_{x=0}^{x=2}= 64- \frac{16+32}{3}=48.
  \end{aligned}
\end{equation}
Vérifiez, comme exercice, qu'on obtient le même résultat en intégrant d'abord par rapport à $y$ et puis par rapport à $x$.
\end{example}

\begin{example}
  Dans les hypothèses du théorème~\ref{fub}  l'ordre des intégrations partielles ne change pas la valeur de l'intégrale. En fait, si les calculs sont faits par des êtres humains l'ordre d'intégration peut faire une certaine différence comme dans cet exemple. On veut évaluer la valeur de l'intégrale
\[
\int_{\eR^2}f(x,y)\, dV
\]
où
\begin{equation}
	f(x,y)=\begin{cases}
		y\sin(x,y)	&	\text{si }(x,y)\in\mathopen[ 1,2 ,  \mathclose]\times\mathopen[ 0 , \pi \mathclose]\\
		0	&	 \text{sinon.}
	\end{cases}
\end{equation}
Les deux sections de $f(x,y)=y\sin(xy)$ sont continues. Si on intègre d'abord par rapport à $y$ on obtient
\[
-\int_1^2\frac{ \pi\cos(\pi x) }{ x }dx+\int_1^2\frac{ \sin(\pi x) }{ x^2 }dx,
\]
qui n'est pas du tout immédiat, alors que, si on intègre d'abord par rapport à $x$ on obtient
\[
\int_0^\pi \cos y - \cos(2y)\,dy.
\]
\end{example}

%%%%%%%%%%%%%%%%%%%%%%%%%%%%%%%%%%%%%%%%%%%%%%%%%%%%%%%%%%%%%%%%%%%%%%%%%%%%%%%%
\subsection{Intégrales sur des parties de $\eR^2$ }
%%%%%%%%%%%%%%%%%%%%%%%%%%%%%%%%%%%%%%%%%%%%%%%%%%%%%%%%%%%%%%%%%%%%%%%%%%%%%%%%

On veut évaluer l'intégrale de la fonction $f(x,y)=\sqrt{1-x^2}$ sur son domaine, la boule unité $B((0,0),1)$. La théorie introduite jusqu'ici n'est pas suffisante pour résoudre  ce problème, parce que $B((0,0),1)$ n'est pas pavable. Les parties bornées de $\eR^p$ sur lesquelles on peut intégrer des fonctions sont dites mesurables (au sens de Riemann) parce que, comme on verra dans la suite, la mesure d'une partie de $\eR^p$ est l'intégrale (s'il existe) de sa fonction caractéristique.

On peut dire qu'une partie de $\eR^p$  est mesurable si son bord est <<assez régulier>>. Dans $\eR^2$ il est suffisant que le bord de $A$ soit une réunion finie de courbes paramétrées continues. En particulier, on est très souvent dans un des deux cas suivants
\begin{description}
\item[Régions du premier type] $A$ est borné et contenu entre les graphes de deux fonctions continues de $x$
\[
A=\{(x,y)\in\eR^2 \,:\, a\leq x\leq b, \, g_1(x)\leq y\leq g_2(x)\},
\]
avec $g_1$ et $g_2$ continues.
\item[Régions du deuxième type] $A$ est borné et contenu entre les graphes de deux fonctions continues de $y$
\[
A=\{(x,y)\in\eR^2 \,:\, c\leq y\leq d, \, h_1(y)\leq x\leq h_2(y)\},
\]
avec $h_1$ et $h_2$ continues.
\end{description}
%\ref{LabelFigRegioniPrimoeSecondoTipo}
\newcommand{\CaptionFigRegioniPrimoeSecondoTipo}{Régions du premier et du deuxième type}
\input{auto/pictures_tex/Fig_RegioniPrimoeSecondoTipo.pstricks}

\begin{example}
 Il y a des régions qui sont des deux types au même temps, comme les boules centrées à l'origine, le triangle de sommets  $(0,0)$, $(0,a)$ et $(b,0)$, ou la région $C$ délimité par les courbes $y=2x$ et $y=x^2$. Cette dernière admet les représentations suivantes
\[
C= \{(x,y)\in\eR^2 \,:\, 0\leq x\leq 1, \, x^2\leq y\leq 2x\},
\]
et
\[
C= \{(x,y)\in\eR^2 \,:\, 0\leq y\leq 1, \, y/2\leq x\leq \sqrt{y}\}.
\]
\end{example}
\begin{definition}
  Soit $f$ une fonction de $\eR^2$ dans $\eR$ dont le support  $A$ est une région du premier ou du deuxième type. On définit la fonction $\bar f$ comme
 \begin{equation}
 \bar f(x,y) = \left\{ \begin{array}{ll}
     f(x,y), \qquad & \textrm{si } (x,y)\in A,\\
  0 , & \textrm{sinon.}
    \end{array}\right.
  \end{equation}
  La fonction $f$ est dite \defe{intégrable}{intégrable!fonction non en escalier} si $\bar f$ est intégrable, et la valeur de son intégrale est
\[
\int_A f\, dV=\int_{\eR^2} \bar f\, dV.
\]
\end{definition}
Une fonction continue définie sur une région du premier ou du deuxième type est toujours intégrable.

Pour fixer les idées on suppose ici que $A$ est du premier type et contenue dans le pavé borné $R=[a,b]\times [c,d]$. En suivant la définition on obtient
\begin{equation}
  \begin{aligned}
    \int_A f\, dV&=\int_{\eR^2} \bar f\, dV=\\
    &= \int_a^b\int_c^d \bar f\, dy dx=\\
&= \int_a^b\left(\int_c^{g_1(x)} \bar f\, dy+\int_{g_1(x)}^{g_2(x)} \bar f\, dy+\int_{g_2(x)}^d \bar f\, dy\right)\, dx= \\
&= \int_a^b\int_{g_1(x)}^{g_2(x)}  f\, dy dx.
  \end{aligned}
\end{equation}
De même, si $A$ est du deuxième type on obtient
\begin{equation}
     \int_A f\, dV=\int_c^d\int_{h_1(y)}^{h_2(y)}  f\, dx dy.
\end{equation}
\begin{example}
	On peut maintenant résoudre notre problème de départ, évaluer l'intégrale de la fonction $f(x,y)=\sqrt{1-x^2}$ sur $B((0,0),1)$. Nous choisissons de décrire la boule unité de $\eR^2$ comme une région du premier type : $B((0,0),1)=\{(x,y)\, :\, x\in[-1,1], \, -\sqrt{1-x^2}\leq y\leq \sqrt{1-x^2} \}$.
	\begin{equation}
		I=\int_{B}\sqrt{1-x^2}\, dV=\int_{-1}^1\int_{-\sqrt{1-x^2}}^{\sqrt{1-x^2}}\sqrt{1-x^2}dydx
	\end{equation}
	La première intégrale à effectuer, par rapport à $y$, est l'intégrale d'une fonction constante. Ne pas oublier que l'on intègre $\sqrt{1-x^2}$ par rapport à $y$; c'est bien une constante et l'intégrale consiste seulement à multiplier par $y$ :
	\begin{equation}
		I=\int_{-1}^1\left[ y\sqrt{1-x^2} \right]_{y=-\sqrt{1-x^2}}^{y=\sqrt{1-x^2}}dx=2\int_{-1}^1(1-x^2)dx.
	\end{equation}
	Cela est à nouveau une intégrale simple à effectuer. Le résultat est
	\begin{equation}
		2\int_{-1}^1(1-x^2)dx=2\left[ x-\frac{ x^3 }{ 3 } \right]_{x=-1}^{x=1}=\frac{ 8 }{ 3 }.
	\end{equation}
\end{example}
\begin{remark}
	Toutes les techniques d'intégration à une variable restent valables. Par exemple, lorsqu'une des intégrales est l'intégrale d'une fonction impaire sur un intervalle symétrique par rapport à zéro, l'intégrale vaut zéro.
\end{remark}

\begin{normaltext}   \label{NORMooDSNXooFhyHkx}
Par le lemme~\ref{LemooPJLNooVKrBhN} nous savons que la mesure d'une région bornée de \( \eR^2\) est l'intégrale de sa fonction caractéristique, si elle existe.

La mesure d'une région bornée de $\eR^2$ est dite son \defe{aire}{aire}, et celle d'une région bornée de $\eR^3$ est son \defe{volume}{volume!région bornée dans $\eR^3$}. Voir aussi la remarque~\ref{RemLongIntUn}.
\end{normaltext}

\begin{example}\label{exint}
  On veut calculer l'aire de la région de la figure~\ref{LabelFigExampleIntegration} définie par
\[
A=\{(x,y)\in\eR^2\,\vert\, 0\leq x\leq 1, x^3-1\leq y\leq x \}.
\]
On considère l'intégrale
\[
\int_{\eR^2} \chi_{A}\, dV= \int_0^1\int^{x}_{x^3+1} 1 \, dy\, dx= \int_0^1 -x^3+x+1\, dx= -\frac{1}{4}+\frac{1}{2}+1=\frac{5}{4}.
\]
\end{example}
\newcommand{\CaptionFigExampleIntegration}{La région $A$ de l'exemple~\ref{exint}}
\input{auto/pictures_tex/Fig_ExampleIntegration.pstricks}

\begin{example}
Parfois la région sur laquelle on veut intégrer peut être décrite indifféremment de deux façons, mais la fonction à intégrer nous force a choisir un ordre particulier. Vérifiez que la fonction $f(x,y)=\sin(y^2)$ sur la région triangulaire de sommets $(0,0)$, $(0, 2)$, $(2,2)$ doit être intégrée d'abord par rapport à $x$.
\end{example}

Si une région bornée n'est pas de premier ou de deuxième type on peut normalement la découper en morceaux plus faciles à décrire. On utilise alors la propriété suivante.
\begin{lemma}
  Soit $A$ un sous-ensemble borné de $\eR^2$ et soient $B_1$ et $B_2$ deux parties de $A$ telles que $B_1\cap B_2=\emptyset$ et $B_1\cup B_2= A$. Alors, pour toute fonction $f$ intégrable sur $A$ (et en particulier pour sa fonction caractéristique) on a
\[
\int_{A}f \, dV= \int_{B_1}f \, dV+\int_{B_2}f \, dV.
\]
\end{lemma}

\begin{example}\label{exint2}
La région $D$ que nous voyons sur la figure~\ref{LabelFigExampleIntegrationdeux} est bornée par la parabole $y^2=2x+6$ et la droite $y=x-1$. La région $D$ est une région du deuxième type. Nous pouvons aussi la décrire comme l'union de deux régions du premier type $D_1$ et $D_2$,
\[
D_1=\{(x,y)\,:\, -3\leq x \leq -1,\, -\sqrt{2x+6}\leq y \leq \sqrt{2x+6}\},
\]
 et
\[
D_2=\{(x,y)\,:\, -3\leq x \leq -1, \, x-1\leq y \leq \sqrt{2x+6}\}.
\]
\newcommand{\CaptionFigExampleIntegrationdeux}{La région $D$ de l'exemple~\ref{exint2}}
\input{auto/pictures_tex/Fig_ExampleIntegrationdeux.pstricks}
\end{example}

%%%%%%%%%%%%%%%%%%%%%%%%%%%%%%%%%%%%%%%%%%%%%%%%%%%%%%%%%%%%%%%%%%%%%%%%%%%%%%%%
\subsection{Intégrales sur des parties de $\eR^3$}
%%%%%%%%%%%%%%%%%%%%%%%%%%%%%%%%%%%%%%%%%%%%%%%%%%%%%%%%%%%%%%%%%%%%%%%%%%%%%%%%
Dans ces notes nous n'avons pas l'ambition de traiter d'une façon rigoureuse l'étude des ensembles mesurables de $\eR^3$. Comme dans la section précédente on se limitera à considérer des cas particuliers.
\begin{definition}\label{primotipo_solida}
	Soit $E$ une région de  $\eR^3$. On dit que $E$ est une \defe{région solide de premier type}{premier type!région solide} si $E$ est contenue entre les graphes de deux fonctions continues de $x$ et $y$.
\[
E=\{(x,y,z)\in\eR^3\, \vert \, (x,y)\in A\subset \eR^2, u_1(x,y)\leq z\leq u_2(x,y) \}.
\]
\end{definition}
Le sous-ensemble de $A$  de $\eR^2$ qui apparaît dans la définition~\ref{primotipo_solida} est la projection (ou l'ombre) de $E$ sur le plan $x$-$y$.

\begin{example}\label{cornet}
 La région $E$ donnée par une portion de sphère collée à un cône est une région solide de premier type
 \begin{equation}
     E=\{(x,y,z)\in\eR^3\, \vert \, (x,y)\in \overline{  B\big((0,0),1\big)}, \sqrt{x^2+y^2}\leq z\leq \sqrt{1-x^2-y^2} \}.
 \end{equation}
L'ombre de $E$ est la boule unité de $\eR^2$. L'ensemble $\sqrt{x^2+y^2}\leq z$ est un cône posé sur sa pointe tandis que l'ensemble $z\leq\sqrt{ 1-x^2-y^2 }$ est la demi-sphère. L'ensemble $E$ contient les points entre les deux, voir la figure~\ref{LabelFigCornetGlace}.
\newcommand{\CaptionFigCornetGlace}{Il faut voir ça en trois dimensions.}
\input{auto/pictures_tex/Fig_CornetGlace.pstricks}

\end{example}

Si la fonction $f$, à intégrer sur $E$, et ses sections sont intégrables  alors on peut réduire l'intégrale
\begin{equation}
  \begin{aligned}
     \int_E  f(x,y,z)\, dV&=\int_A\left(\int_{u_1(x,y)}^{u_2(x,y)}f(x,y,z)\, dz \right) \, dV=\\
&=\int_A\left(F(x,y,u_2(x,y))-F(x,y,u_1(x,y))\right)\, dV,
  \end{aligned}
\end{equation}
où $F$ est une primitive de $f$ par rapport à la variable $z$, c'est-à-dire en considérant $x$ et $y$ comme des constantes. Il faut ensuite évaluer la partie qui reste comme dans la section précédente. Comme le calcul des aires dans $\eR^2$, le calcul des volumes dans $\eR^3$ est fait par des intégrales. En fait le \defe{volume}{volume!d'une région solide} d'une région solide dans $\eR^3$ est sa mesure.
\begin{definition}
   La mesure d'une région de  $\eR^3$ est l'intégrale de sa fonction caractéristique.
\end{definition}
Soit $E$ une région solide du premier type, nous pouvons évaluer son volume par l'intégrale
\[
\int_A\left(u_2(x,y)-u_1(x,y)\right)\, dV.
\]
Parfois c'est plus intéressant de calculer le volume avec la formule de réduction contraire : l'intégrale double d'abord et puis l'intégrale simple par rapport à $z$. On parle alors de calcul de volume «par tranche».

\begin{example}
On veut calculer le volume de la boule de rayon $a$, centrée à l'origine $B=\{(x,y,z)\in\eR^3\,\vert\, x^2+y^2+z^2\leq a^2 \}$. On peut décrire $B$ par
\[
  B=\left\{(x,y,z)\in\eR^3\,\vert\, (x,y)\in D_a, -\sqrt{a^2-x^2-y^2}\leq z\leq \sqrt{a^2-x^2-y^2}  \right\},
\]
où $D_a$ est le disque de rayon $a$ centré en $(0,0)$, donc le volume $B$ sera
\[
2 \int_{D_a}\sqrt{a^2-x^2-y^2} dV.
\]
Cette intégrale est un peu ennuyeuse à calculer. On peut simplifier le calcul en observant que pour $\bar z$ fixé dans l'intervalle $[-a,a]$ la section de la boule au niveau $\bar z$ est un disque de rayon $\sqrt{a^2-z^2}$. L'aire d'un tel disque est  $\pi (a^2+z^2)$. Si on réduit l'intégrale de volume de la façon
\[
\int_{B} 1\, dV=\int_{-a}^{a}  \sqrt{a^2-z^2}\, dz,
\]
on obtient tout de suite la valeur cherchée : le volume de $B$ est $4/3 \pi a^3$.
\end{example}
\begin{example}
	On calcule l'intégrale de $f(x,y,z)=z$ sur la pyramide $P$ bornée par le plans $x=0$, $y=0$, $x+y+z=1$, $x+y+z/2=1$. On remarque tout de suite que le plans $x+y+z=1$, $x+y+z/2=1$ se coupent en la droite $x+y=1$, $z=0$ (on se souvient qu'\emph{une} droite dans $\eR^3$, c'est \emph{deux} équations). Cela veut dire que la projection de $P$ sur le plan $x$-$y$ est le  triangle $T$ borné par les droites $x=z=0$, $y=z=0$ et $x+y=1$, $z=0$.
On  décrit donc $P$ par
\[
P=\{(x,y,z)\in\eR^3\,\vert\, (x,y)\in T, \, 1-2x-2y\leq z\leq 1-x-y\}
\]
et $T$ par
\[
T=\{(x,y)\in\eR^2\,\vert\, 0\leq x\leq 1,\,  0\leq y\leq 1-x\},
\]
donc l'intégrale de $f$ sur $P$ est
\[
\int_pf(x,y,z)\, dV= \int_{0}^{1}\int_{0}^{1-x}\int_{1-2x-2y}^{1-x-y}z \,dz\,dy\,dx=-\frac{1}{ 24 }.
\]
Notez que lorsque $x$ et $y$ sont entre $0$ et $1$, nous avons bien $1-2x-2y<1-x-y$, d'où le fait que nous mettons $1-2x-2y$ dans la borne inférieure de l'intégrale.
\end{example}

De façon analogue on définit les régions solides du deuxième et du troisième type.

%---------------------------------------------------------------------------------------------------------------------------
					\subsection[Fonctions et ensembles non bornés]{Intégrales de fonctions non bornées sur des ensembles non bornés}
%---------------------------------------------------------------------------------------------------------------------------

Soit $f\colon \eR^n\to \overline{ \eR }$, une fonction positive. On dit qu'elle est \defe{intégrable}{intégrable!fonction positive} sur $E\subset\eR^n$ si
\begin{enumerate}
    \item $\forall r>0$, la fonction $f_r(x)=f(x)\mtu_{f<r}$ est intégrable sur $E_r$;
\item la limite $\lim_{r\to\infty}\int_{E_r}f_r$ est finie.
\end{enumerate}
Dans ce cas, on pose
\begin{equation}
	\int_Ef=\lim_{r\to\infty}\int_{E_r}f_r.
\end{equation}

\begin{theorem}	\label{ThoFnTestIntnnBorn}
Soit $E$ mesurable dans $\eR^n$ et $f\colon E\to \overline{ \eR }$. Si $f$ est mesurable et s'il existe $g\colon E\to \overline{ \eR }$ intégrable sur $E$ telle que $| f(x) |\leq g(x)$ pour tout $x\in E$, alors $f$ est intégrable sur~$E$.

Réciproquement, si $f$ est intégrable sur $E$, alors $f$ est mesurable.
\end{theorem}

\begin{lemma}\label{LemTHBSEs}
    Si \( f\) est une fonction sur \( \mathopen[ a , \infty [\), alors nous avons la formule
    \begin{equation}
        \lim_{b\to \infty}\int_a^bf(x)dx=\int_a^{\infty}f(x)dx
    \end{equation}
    au sens où si un des deux membres existe, alors l'autre existe et est égal.
\end{lemma}

\begin{proof}
    Supposons que le membre de gauche existe. Cela signifie que la fonction
    \begin{equation}
        \psi(x)=\int_a^xf
    \end{equation}
    est bornée. Soit \( M\), un majorant. Pour toute fonction simple \( \varphi\) dominant \( f\), on a \( \int\varphi\leq M\), donc l'ensemble sur lequel on prend le supremum pour calculer \( \int_a^{\infty}f\) est majoré par \( M\) et possède donc un supremum. Nous avons donc
    \begin{equation}
        \int_a^{\infty}f\leq\lim_{b\to\infty}\int_a^bf.
    \end{equation}
\end{proof}


%---------------------------------------------------------------------------------------------------------------------------
\subsection{Lemme de Morse}
%---------------------------------------------------------------------------------------------------------------------------

\begin{lemma}[Lemme de Morse]     \label{LemNQAmCLo}
    Soit \( f\in C^3(\mU,\eR)\) où \( \mU\) est un ouvert de \( \eR^n\) contenant \( 0\). Nous supposons que \( df_0=0\) et que \( d^2f_0\) est non dégénérée\footnote{En tant qu'application bilinéaire.} et de signature \( (p,n-p)\). Alors il existe un \( C^1\)-difféomorphisme \( \varphi\) entre deux voisinages de \( 0\) dans \( \eR^n\) tel que
    \begin{enumerate}
        \item
            \( \varphi(0)=0\),
        \item
            si \( \varphi(x)=u\) alors
            \begin{equation}
                f(x)-f(0)=u_1^2+\cdots +u_p^2-u_{p+1}^2-\ldots-u_n^2.
            \end{equation}
    \end{enumerate}
    Une autre façon de dire est qu'il existe un \( C^1\)-difféomorphisme local \( \psi\) tel que
    \begin{equation}
        (f\circ\psi)(x)-f(0)=x_1^2+\cdots +x_p^2-x_{p+1}^2-\ldots-x_n^2.
    \end{equation}
\end{lemma}
\index{lemme!de Morse}
\index{développement!Taylor}
\index{application!différentiable}
\index{forme!quadratique}
\index{théorème!inversion locale!utilisation}
\index{action de groupe!sur des matrices}
\index{extrémum}

\begin{proof}
    Nous allons noter \( Hf\) la matrice hessienne de \( f\), c'est-à-dire \( Hf_a=d^2f_a\in\aL^{(2)}(\eR^n,\eR)\). Écrivons la formule de Taylor avec reste intégral (proposition~\ref{PropAXaSClx} avec \( p=0\) et \( m=2\)) :
    \begin{equation}
        f(x)-f(0)=\underbrace{df_0(x)}_{=0}+\int_0^1(1-t)\underbrace{d^2f_{tx}(x,x)}_{x^t(Hf)_{tx}x=\langle Hf_{tx}x, x\rangle }dt=x^tQ(x)x
    \end{equation}
    avec
    \begin{equation}
        Q(x)=\int_0^1(1-t)(Hf)_{tx}dt
    \end{equation}
    qui est une intégrale dans \( \aL^{(2)}(\eR^n,\eR)\). Nous prouvons à présent que \( Q\) est de classe \( C^1\) en utilisant le résultat de différentiabilité sous l'intégrale~\ref{PropAOZkDsh}. Pour cela nous passons aux composantes (de la matrice) et nous considérons
    \begin{equation}
        \begin{aligned}
            h_{kl}\colon U\times\mathopen[ 0 , 1 \mathclose]&\to \eR \\
            h_{kl}(x,t)&=(1-t)\frac{ \partial^2f  }{ \partial x_k\partial x_l }(tx).
        \end{aligned}
    \end{equation}
    Étant donné que \( f\) est de classe \( C^3\), la dérivée de \( h_{kl}\) par rapport à \( x_i\) ne pose pas de problèmes :
    \begin{equation}
        \frac{ \partial h_{kl} }{ \partial x_i }=t(t-1)\frac{ \partial^3f  }{ \partial x_i\partial x_k\partial x_l }(tx),
    \end{equation}
    qui est encore continue à la fois en \( t\) et en \( x\). La proposition~\ref{PropAOZkDsh} nous montre à présent que
    \begin{equation}
        Q_{kl}(x)=\int_0^1(1-t)h_{kl}(tx)dt
    \end{equation}
    est une fonction \( C^1\). Étant donné que les composantes de \( Q\) sont \( C^1\), la fonction \( Q\) est également \( C^1\).

    Nous avons \( Q(0)=\frac{ 1 }{2}(Hf)_0\in S_n\cap \GL(n,\eR)\), d'abord parce que \( f\) est \( C^2\) (et donc la matrice hessienne est symétrique), ensuite par hypothèse \( d^2f_0\) est non dégénérée.
    %TODO : prouver que la matrice hessienne est symétrique lorsque f est C^2 (ou vérifier que c'est déjà fait), et référentier ici.

    À partir de là, le lemme~\ref{LemWLCvLXe} donne un voisinage \( V\) de \( Q(0)\) dans \( S_n\) et une application \( \phi\) de classe \( C^1\)
    \begin{equation}
            \phi\colon V\to \GL(n,\eR) \\
    \end{equation}
    telle que pour tout \( A\in V\),
    \begin{equation}
        \phi(A)^tQ(0)\phi(A)=A.
    \end{equation}
    Si on pose \( M=\phi\circ Q\), et si \( x\) est dans un voisinage de zéro, \( Q\) étant continue nous avons \( Q(x)\in V\) et donc
    \begin{equation}
        Q(x)=M(x)^tQ(0)M(x).
    \end{equation}
    Notons que l'application \( \eM\colon \eR\to \GL(n,\eR)\) est de classe \( C^1\) parce que \( Q\) et \( \phi\) le sont.

    Nous avons
    \begin{equation}
        f(x)-f(0)=x^tQ(x)x=x^tM(x)^tQ(0)M(x)x=y(x)^tQ(0)y(x)
    \end{equation}
    où \( y(x)=M(x)x=(\phi\circ Q)(x)x\) est encore une fonction de classe \( C^1\) parce que la multiplication est une application \(  C^{\infty}\).

    D'un autre côté le théorème de Sylvester~\ref{ThoQFVsBCk} nous donne une matrice inversible \( P\) telle que
    \begin{equation}
        Q(0)=P^t\begin{pmatrix}
            \mtu_p    &       \\
            &   -\mtu_{n-p}
        \end{pmatrix}P.
    \end{equation}
    Et nous posons enfin \( u=\varphi(x)=Py(x)\) qui est toujours de classe \( C^1\) et qui donne
    \begin{subequations}
        \begin{align}
            f(x)-f(0)&=y^tQ(0)y\\
            &=y^tP^t\begin{pmatrix}
                \mtu    &       \\
                    &   -\mtu
            \end{pmatrix}Py\\
            &=u^t\begin{pmatrix}
                \mtu    &       \\
                    &   -\mtu
            \end{pmatrix}u\\
            &=u_1^2+\cdots +u_p^2-u_{p+1}^2-\ldots -u_n^2.
        \end{align}
    \end{subequations}

    Nous devons maintenant montrer que, quitte à réduire son domaine à un ouvert plus petit, \( \varphi\) est un \( C^1\)-difféomorphisme. Dans la chaine qui donne \( \varphi\), seule l'application
    \begin{equation}
        \begin{aligned}
            g\colon U\subset \eR^n&\to \eR^n \\
            x&\mapsto M(x)x
        \end{aligned}
    \end{equation}
    est sujette à caution. Nous allons appliquer le théorème d'inversion locale. Nous savons que \( g\) est de classe \( C^1\) et donc différentiable; calculons la différentielle en utilisant la formule \eqref{EqOWQSoMA} :
    \begin{equation}
        dg_0(x)=\Dsdd{ g(tx) }{t}{0}=\Dsdd{ tM(tx)x }{t}{0}=M(0)x.
    \end{equation}
    Note que nous avons utilisé la règle de Leibnitz pour la dérivée d'un produit, mais le second terme s'est annulé. Donc \( dg_0=M(0)\in \GL(n,\eR)\) et \( g\) est localement un \( C^1\)-difféomorphisme.

    Il suffit de restreindre \( \varphi\) au domaine sur lequel \( g\) est un \( C^1\)-difféomorphisme pour que \( \varphi\) devienne lui-même un \( C^1\)-difféomorphisme.

\end{proof}

\begin{definition}
    Un point \( a\) est un \defe{point critique}{point critique!définition} de la fonction différentiable \( f\) si \( df_a=0\).
\end{definition}

\begin{corollary}[\cite{XPautfO}]
    Les points critiques non dégénérés d'une fonction \( C^3\) sont isolés.
\end{corollary}

\begin{proof}
    Soit \( a\) un point critique non dégénéré. Par le lemme de Morse~\ref{LemNQAmCLo}, il existe un \( C^1\)-difféomorphisme \( \psi\) et un entier \( p\) tel que
    \begin{equation}
        (f\circ \psi)(x)=x_1^2+\cdots +x_p^2-x_{p+1}^2-\ldots -x_n^2+f(a)
    \end{equation}
    sur un voisinage \( \mU\) de \( a\). Vue la formule générale \( df_x(u)=\nabla f(x)\cdot u\), si \( x\) est un point critique de \( f\), alors \( \nabla f(x)=0\). Dans notre cas, les points critiques de \( f\circ \psi\) dans \( \mU\) doivent vérifier \( x_i=0\) pour tout \( i\), et donc \( x=a\).

    Nous devons nous assurer que la fonction \( f\) elle-même n'a pas de points critiques dans \( \mU\). Pour cela nous utilisons la formule générale de dérivation de fonction composée :
    \begin{equation}
        \nabla(f\circ\psi)(x)=\sum_k \frac{ \partial f }{ \partial y_k }\big( g(x) \big)\nabla g_k(x).
    \end{equation}
    Si \( \psi(x)\) est une point critique de \( f\), alors le membre de droite est le vecteur nul parce que tous les \( \partial_kf\big( \psi(x) \big)\) sont nuls. Par conséquent le membre de gauche est également nul, et \( x\) est un point critique de \( f\circ\psi\). Or nous venons de voir que \( f\circ\psi\) n'a pas de points critiques dans \( \mU\).

    Donc \( f\) n'a pas de points critiques dans un voisinage d'un point critique non dégénéré.
\end{proof}


%+++++++++++++++++++++++++++++++++++++++++++++++++++++++++++++++++++++++++++++++++++++++++++++++++++++++++++++++++++++++++++ 
\section{Autres intégrales sympathiques}
%+++++++++++++++++++++++++++++++++++++++++++++++++++++++++++++++++++++++++++++++++++++++++++++++++++++++++++++++++++++++++++

%--------------------------------------------------------------------------------------------------------------------------- 
\subsection{Intégrale de Wallis}
%---------------------------------------------------------------------------------------------------------------------------

\begin{lemma}[\cite{BIBooBKPHooAHRmLD}]     \label{LEMooMGUVooIIQSmC}
    Soit \( n>0\). En posant \( I_n=\int\sin^n(x)dx\), nous avons :
    \begin{equation}
        I_n=\frac{ \cos(x)\sin^{n-1}(x) }{ n }+\frac{ n-1 }{ n }I_{n-2},
    \end{equation}
    c'est-à-dire
    \begin{equation}
        \int\sin^n(x)dx=\frac{ n-1 }{ n }\int\sin^{n-2}(t)dt-\frac{ \sin^{n-1}(x)\cos(x) }{ n }.
    \end{equation}

    De la même manière,
    \begin{equation}        \label{EQooWJMIooSgBbJx}
        \int\cos^n(x)dx=\frac{ \cos^{n-1}(x)\sin(x) }{ n }+\frac{ n-1 }{ n }\int\cos^{n-2}(x)dx.
    \end{equation}
\end{lemma}

\begin{proof}
    Nous posons \( I_n=\int\sin^n(t)dt\), et nous y allons par récurrence. D'abord pour \( n=1\). Dans ce cas, la formule à démontrer se réduit à
    \begin{equation}
        \int\sin(x)dx=\cos(x).
    \end{equation}
    Pas de problèmes.

    Pour \( n\geq 2\), nous évaluons l'intégrale \( I_n\) en utilisant une intégration par partie\footnote{Proposition \ref{PROPooRLFIooQHnyJY}.} en posant
    \begin{subequations}
        \begin{numcases}{}
            u=\sin^{n-1}(t)\\
            v'=\sin(t)
        \end{numcases}
    \end{subequations}
    et en déduisant
    \begin{subequations}
        \begin{numcases}{}
            u'=(n-1)\sin^{n-2}(t)\cos(t)\\
            v=-\cos(t).
        \end{numcases}
    \end{subequations}
    Nous avons alors le calcul
    \begin{subequations}
        \begin{align}
            I_n&=-\cos(x)\sin^{n-1}(x)+\int(n-1)\sin^{n-2}(x)\cos^2(x)dx\\
            &=-\cos(x)\sin^{n-1}(x)+(n-1)\int\sin^{n-2}\underbrace{\cos^2(x)}_{=1-\sin^2(x)}dx\\
            &=\cos(x)\sin^{n-1}(x)+(n-1)\underbrace{\int\sin^{n-2}(x)dx}_{=I_{n-2}}-(n-1)\underbrace{\int\sin^n(x)}_{=I_n}dx
        \end{align}
    \end{subequations}
    Nous avons donc déjà prouvé que
    \begin{equation}
        I_n=\cos(x)\sin^{n-1}(x)+(n-1)(I_{n-2}-I_n).
    \end{equation}
    En isolant \( I_n\),
    \begin{equation}
        I_n=\frac{ \cos(x)\sin^{n-1}(x) }{ n }+\frac{ n-1 }{ n }I_{n-2}.
    \end{equation}

    La formule \eqref{EQooWJMIooSgBbJx} se démontre de la même façon.
\end{proof}

\begin{lemma}[\cite{BIBooFMRSooRYnhNf,BIBooSZLCooXWYESD}]       \label{LEMooUOIBooLyMDft}
    Nous posons
    \begin{equation}
        W_n=\int_0^{\pi/2}\cos^n(t)dt.
    \end{equation}
    Alors :
    \begin{enumerate}
        \item
            une formule de récurrence :
            \begin{equation}        \label{EQooILMZooBUgJpk}
                W_n=\frac{ n-1 }{ n }W_{n-2},
            \end{equation}
        \item et une formule un peu explicite :
            \begin{equation}        \label{EQooUYIDooEpHCnP}
                W_{2n}=\frac{ (2n)! }{ (2^nn!)^2 }\frac{ \pi }{2}.
            \end{equation}
    \end{enumerate}
\end{lemma}

\begin{proof}
    Le nombre \( W_n\) est seulement la seconde intégrale du lemme \ref{LEMooMGUVooIIQSmC}, évaluée entre \( 0\) et \( \pi/2\). En partant donc de \eqref{EQooWJMIooSgBbJx}, nous avons
    \begin{equation}
        W_n=\int_0^{\pi/2}\cos^n(x)dx=\left[ \frac{ \cos^{n-1}(x)\sin(x) }{ n } \right]_0^{\pi/2}+\frac{ n-1 }{ n }\int_0^{\pi/2}\cos^{n-2}(x)dx.
    \end{equation}
    Le terme aux bords disparaît grâce aux valeurs trigonométriques remarquables\footnote{Par exemple la proposition \ref{PROPooMWMDooJYIlis}\ref{ITEMooQKPKooEPeHER}.}. Il reste immédiatement
    \begin{equation}
        W_n=\frac{ n-1 }{ n }W_{n-2}.
    \end{equation}
    À partir de là, nous démontrons \eqref{EQooUYIDooEpHCnP} par récurrence. D'abord pour \( n=0\), c'est l'égalité \( W_0=\pi/2\) qui est correcte parce que
    \begin{equation}
        \int_0^{\pi/2}\cos^0(x)dx=\int_0^{\pi/2}1dx.
    \end{equation}
    Pour la récurrence elle-même,
    \begin{subequations}
        \begin{align}
            W_{2(n+1)}&=\frac{ 2n+1 }{ 2(n+1) }W_{2n}\\
            &=\frac{ 2n+1 }{ 2n+2 }\frac{ (2n)! }{ (2^nn!)^2 }\frac{ \pi }{ 2 }\\
            &\text{\ldots pas mal de petits calculs \ldots}\\
            &=\frac{ (2n+2)! }{ \big( (n+1)!2^{n+1} \big)^2 }\frac{ \pi }{2}.
        \end{align}
    \end{subequations}
    Voila.
\end{proof}

Maintenant, la suite \( (W_n)\) se divise en ses termes pairs et ses termes impairs. Pour les pairs, nous avons une formule assez explicite donnée par le lemme \ref{LEMooUOIBooLyMDft}. Pour les termes impairs, nous n'avons rien. Dans tous les cas, nous avons la formule de récurrence
\begin{equation}
    W_n=\frac{ n-1 }{ n }W_{n-2}
\end{equation}
qui ne sert à rien pour déduire des choses sur les termes impairs à partir de ce que l'on sait des termes pairs.

Sommes-nous perdus ? Non. La situation se débloque grâce au lemme suivant.

\begin{lemma}       \label{LEMooZFBVooQsOuOx}
    La suite donnée par
    \begin{equation}
        W_n=\int_0^{\pi/2}\cos^n(x)dx
    \end{equation}
    est décroissante.
\end{lemma}

\begin{proof}
    Vu que l'intégrale est sur \( \mathopen[ 0 , \pi/2 \mathclose]\), le nombre \( \cos(x)\) prend ses valeurs dans \( \mathopen[ 0 , 1 \mathclose]\). Nous avons donc
    \begin{equation}
        \cos^{n+1}(x)\leq \cos^n(x).
    \end{equation}
    Les intégrales suivent les mêmes inégalités.
\end{proof}

Ce lemme permet de relancer le jeu parce que les termes impairs sont coincés entre les termes pairs, qui décroissent. Les termes impairs doivent donc décroître à la même vitesse. Le lemme suivant met cela en musique.

\begin{lemma}       \label{LEMooAXTEooLBXQuM}
    Nous posons
    \begin{equation}
        W_n=\int_0^{\pi/2}\cos^n(x)dx
    \end{equation}
    La fonction \( \alpha\colon \eN\to \eR\) définie par \( W_n=\alpha(n)W_{n-1}\) vérifie \( \lim_{n\to \infty} \alpha(n)=1\).
\end{lemma}

\begin{proof}
    Parce que nous en aurons besoin, nous triturons d'abord un peu la formule de récurrence \eqref{EQooILMZooBUgJpk}. D'abord nous l'inversons un peu pour avoir
    \begin{equation}
        W_n=\frac{ n+2 }{ n }W_{n+2},
    \end{equation}
    et ensuite nous écrivons
    \begin{equation}        \label{EQooYINNooBUKUYB}
        W_{n-1}=\frac{ n+1 }{ n }W_{n+1}.
    \end{equation}
    Ne vous posez pas de questions, ça va être utile. Le fait que la suite soit décroissante (lemme \ref{LEMooZFBVooQsOuOx}) nous permet d'écrire
    \begin{equation}
        W_{n+2}\leq W_n\leq W_{n-1}.
    \end{equation}
    En y remplaçant \( W_n\) par \( \alpha(n)W_{n+1}\) et \( W_{n-1}\) par \eqref{EQooYINNooBUKUYB},
    \begin{equation}
        W_{n+1}\leq \alpha(n)W_{n+1}\leq \frac{ n+1 }{ n }W_{n+1}.
    \end{equation}
    Nous simplifions par \( W_{n+1}\) et nous trouvons l'encadrement, valable pour tout \( n\) :
    \begin{equation}
        1\leq \alpha(n)\leq \frac{ n+1 }{ n }.
    \end{equation}
    Nous en déduisons par la règle de l'étau que \( \alpha(n)\to 1\).
\end{proof}

\begin{lemma}       \label{LEMooWQZAooOXAPQO}
    Soit
    \begin{equation}
        W_n=\int_0^{\pi/2}\cos^n(x)dx.
    \end{equation}
    Encore plusieurs choses à dire.
    \begin{enumerate}
        \item
            Pour tout \( n\) nous avons
            \begin{equation}        \label{EQooLOLFooMIwMXN}
                nW_nW_{n-1}=\frac{ \pi }{2}.
            \end{equation}
            \item
                Nous avons l'équivalence de suites\footnote{Définition \ref{DEFooEWRTooKgShmT}.}
                \begin{equation}
                    W_n\sim\sqrt{ \frac{ \pi }{ 2n } }.
                \end{equation}
    \end{enumerate}
\end{lemma}

\begin{proof}
    En deux parties
    \begin{subproof}
        \item[La suite constante]
            Nous posons \( K_n=nW_nW_{n-1}\). Grâce aux formules de récurrence \eqref{EQooILMZooBUgJpk} que nous écrivons sous la forme
            \begin{subequations}
                \begin{align}
                    W_n&=\frac{ n-1 }{ n }W_{n-2}\\
                    W_{n+1}&=\frac{ n }{ n+1 }W_{n-1},
                \end{align}
            \end{subequations}
            nous avons
            \begin{equation}
                K_{n+1}=(n+1)W_{n+1}W_n=(n+1)\frac{ n }{ n+1 }\frac{ n-1 }{ n }W_{n-2}W_{n_2}=(n-1)W_{n-1}W_{n-2}=K_{n-1}.
            \end{equation}
            Nous avons montré que \( K_{n+1}=K_{n-1}\).

            Il nous reste à prouver que \( K_1=K_2=\frac{ \pi }{2}\). Pour cela nous avons immédiatement \( W_0=\frac{ \pi }{2}\) ainsi que
            \begin{equation}
                W_1=\int_0^{\pi/2}\cos(t)dt=1.
            \end{equation}
            Pour \( W_2\), il ne faut pas calculer d'intégrales, mais seulement utiliser la formule \eqref{EQooUYIDooEpHCnP}. Nous trouvons vite \( W_2=\frac{ \pi }{ 4 }\). Donc
            \begin{equation}
                K_1=W_1W_0=\frac{ \pi }{2}
            \end{equation}
            et
            \begin{equation}
                K_2=2W_2W_1=\frac{ \pi }{2}.
            \end{equation}
        \item[L'équivalence de suites]
            Soit la fonction \( \alpha\)  définie par \( W_n=\alpha(n)W_{n-1}\). Le lemme \ref{LEMooAXTEooLBXQuM} nous dit que \( \alpha(n)\to 1\). Nous l'utilisons dans \eqref{EQooLOLFooMIwMXN} :
            \begin{equation}
                \frac{ \pi }{2}=nW_nW_{n-1}=n\alpha(n)W_{n-1}^2.
            \end{equation}
            Donc
            \begin{equation}
                W_{n-1}^2=\frac{ \pi }{ 2n\alpha(n) },
            \end{equation}
            et
            \begin{equation}
                W_{n-1}\sqrt{ \frac{1}{ \alpha(n) } }\sqrt{ \frac{ \pi }{ 2n } }.
            \end{equation}
            Vu que le coefficient \( \sqrt{ 1/\alpha(n) }\) tend vers \( 1\) pour \( n\to 1\), nous avons l'équivalence demandée.
    \end{subproof}
\end{proof}

%--------------------------------------------------------------------------------------------------------------------------- 
\subsection{Formule de Stirling}
%---------------------------------------------------------------------------------------------------------------------------

\begin{lemma}[\cite{BIBooFRPFooFDvXrp}]     \label{LEMooDXJOooOGFcIv}
    Si \( n\in \eN\) nous avons la formule
    \begin{equation}
        \ln(1+\frac{1}{ n })=2\sum_{k=0}^{\infty} \frac{1}{ 2k+1 } \left( \frac{1}{ 2n+1 } \right)^{2k+1}
    \end{equation}
\end{lemma}

\begin{proof}
    Il s'agit d'utiliser astucieusement le développement de la proposition \ref{PROPooKPBIooJdNsqX}. Nous avons d'une part
    \begin{equation}
        \ln(1+t)=\sum_{k=1}^{\infty}(-1)^{k+1}\frac{ x^k }{ k }
    \end{equation}
    et d'autre part,
    \begin{equation}
        -\ln(1-t)=\sum_{k=1}^{\infty}\frac{ x^k }{ k }.
    \end{equation}
    Cela permet de calculer, en utilisant l'associativité de la série\footnote{Proposition \ref{PROPooUEBWooUQBQvP}.}
    \begin{subequations}\label{EQooYVXHooNDetVx}
        \begin{align} 
            \ln\left( \frac{ 1+t }{ 1-t } \right)&=\ln(1+t)-\ln(1-t)\\
            &=\sum_{k=1}^{\infty}\left( \frac{ (-1)^{k+1}x^k }{ k }+\frac{ x^k }{ k } \right)\\
            &=2\sum_{k=0}^{\infty}\frac{ x^{2k+1} }{ 2k+1 }     \label{SUBEQooIIOBooXYoHkK}
        \end{align}
    \end{subequations}
    parce que tous les termes pairs s'annulent, tandis que les termes impairs sont doublés. Notez que la somme dans \eqref{SUBEQooIIOBooXYoHkK} commence à \( k=0\), contrairement aux autres qui commencent à \( 1\).
   
    Et là c'est l'astuce : on pose écrit l'égalité \eqref{EQooYVXHooNDetVx} avec le \( t\) qu'il faut pour que
    \begin{equation}
        \frac{ 1+t }{ 1-t }=1+\frac{1}{ n }.
    \end{equation}
    Nous posons donc \( t=\frac{1}{ 2n+1 }\) nous avons
    \begin{equation}
        \ln\left( 1+\frac{1}{ n } \right)=2\sum_{k=0}^{\infty}  \frac{1}{ 2k+1 } \left( \frac{1}{ 2n+1 } \right)^{2k+1},
    \end{equation}
    ce qu'il fallait.
\end{proof}


\begin{lemma}[Formule de Stirling\cite{MEHuVnb,BIBooZRVYooGKkFmy,BIBooFMRSooRYnhNf,BIBooFRPFooFDvXrp,BIBooEFYIooBvytzQ}]        \label{LemCEoBqrP}
    Nous avons l'équivalence de suites\footnote{Définition \ref{DEFooEWRTooKgShmT}.}
    \begin{equation}
        n!\sim \left( \frac{ n }{ e } \right)^n\sqrt{2\pi n}.
    \end{equation}
\end{lemma}
\index{formule de Stirling}

\begin{proof}
    Nous posons
    \begin{equation}
        a_n=\frac{ n! }{ \sqrt{ 2n } \left( \frac{ n }{ e } \right)^n }
    \end{equation}
    et \( b_n=\ln(a_n)\).
    \begin{subproof}
        \item[Une formule pour \( b_n-b_{n+1}\)]
            Nous faisons un beau calcul qui utilise les formules de la proposition \ref{PROPooLAOWooEYvXmI} ainsi que \( \ln(e)=1\) :
            \begin{subequations}        \label{SUBEQSooZTABooBukFGM}
                \begin{align}
                    b_n-b_{n+1}&=\ln\left( \frac{ a_n }{ a_{n+1} } \right)\\
                    &=\ln\left(     \frac{ (n+1)^{n+1/2} }{ n^{n+1/2} }\frac{1}{ e }    \right)\\
                    &=\ln\left( (\frac{ n+1 }{ n })^{n+1/2} \right)-1\\
                    &=\left( n+\frac{1}{ 2 } \right)\ln(1+\frac{1}{ n })-1.
                \end{align}
            \end{subequations}
        \item[La suite \( (b_n)\) est décroissante]
            Nous écrivons l'égalité \eqref{SUBEQSooZTABooBukFGM} en utilisant le lemme \ref{LEMooDXJOooOGFcIv} :
            \begin{subequations}
                \begin{align}
                    b_n-b_{n+1}&=(n+\frac{ 1 }{2})\ln(1+\frac{1}{ n })-1\\
                    &=\frac{ 1 }{2}(2n+1)2\sum_{k=0}^{\infty}\frac{1}{ 2k+1 }\left( \frac{1}{ 2n+1 } \right)^{2k+1}-1\\
                    &=(2n+1)\sum_{k=1}^{\infty}\frac{1}{ 2k+1 }\left( \frac{1}{ 2n+1 } \right)^{2k+1}\\
                    &=\sum_{k=1}^{\infty}  \frac{1}{ 2k+1 }  \left( \frac{1}{ 2n+1 } \right)^{2k}.
                \end{align}
            \end{subequations}
            Notez que le terme \( k=0\) s'est simplifié avec le \( -1\). Vu que le tout est une somme de termes positifs, nous avons
            \begin{equation}
                b_n-b_{n+1}>0
            \end{equation}
            et la suite est décroissante.
        \item[Majoration pour \( b_n-b_{n+1}\)]
            Vu que \( \frac{1}{ 2k+1 }<1\), nous pouvons majorer :
            \begin{equation}
                b_n-b_{n+1}\leq \sum_{k=1}^{\infty}\left( \frac{1}{ (2n+1)^2 } \right)^k.
            \end{equation}
            Nous remarquons que cela est une série géométrique déjà traitée dans l'exemple \ref{ExZMhWtJS}. Nous faisons un peu de calcul en partant de 
            \begin{equation}
                \sum_{k=1}^{\infty}q^n=\frac{ q }{ 1-q }
            \end{equation}
            avec \( q=\frac{1}{ (2n+1)^2 }\). Après quelques simplifications,
            \begin{equation}
                b_n-b_{n+1}\leq \frac{1}{ 4 }\frac{ 1 }{ n(n+1) }
            \end{equation}
        \item[Le coup de la somme télescopique]
            Nous avons
            \begin{equation}
                b_1-b_n=(b_1-b_2)+(b_2-b_3)+\ldots+(b_{n_1}-b_n).
            \end{equation}
            Chacun de ces termes est majoré; nous avons donc
            \begin{equation}
                b_1-b_n<\frac{1}{ 4 }\sum_{m=1}^{n-1}\frac{1}{ m(m+1) }\leq \frac{1}{ 4 }\sum_{m=1}^{\infty}\frac{1}{ m(m+1) }=\frac{1}{ 4 }
            \end{equation}
            grâce au lemme \ref{LEMooKDHPooPlFTIT} pour la dernière somme.
        \item[La suite \( b_n\) est bornée vers le bas]

            Vu que \( b_1=1-\frac{ \ln(2) }{2}\), nous avons
            \begin{equation}
                b_n>b_1-\frac{1}{ 4 }=\frac{ 3 }{ 4 }-\frac{ \ln(2) }{ 2 }.
            \end{equation}
            En utilisant la majoration de l'exemple \ref{EXooYMEEooMGpUNM} nous trouvons
            \begin{equation}
                0.327\leq b_n\leq 0.427.
            \end{equation}
            Cet encadrement n'est pas très important. Le point est que la suite \( (b_n)\) soit bornée vers le bas; savoir que la borne est strictement positive n'est pas indispensable.

        \item[Une limite pour \( (a_n)\)]
            La suite \( (b_n)\) est décroissante et bornée vers le bas, donc elle est convergente par le lemme \ref{LemSuiteCrBorncv}. Vu que l'exponentielle est une fonction continue\footnote{Parce qu'elle est dérivable, voir par exemple le théorème \ref{ThoKRYAooAcnTut}.}, la suite \( a_n= e^{b_n}\) est également convergente.

            Vu que \( \lim_{n\to \infty} b_n>0\), nous avons \( \lim_{n\to \infty} a_n= e^{\lim_{n\to \infty} b_n}>1\).

            L'important est que nous sachions que \( (a_n)\) est une suite convergente. Nous notons \( L\) sa limite :
            \begin{equation}
                \lim_{n\to \infty} \frac{ n! }{ \sqrt{ 2n }\left( \frac{ n }{ e } \right)^n }=L.
            \end{equation}
            Ou encore :
            \begin{equation}
                \lim_{n\to \infty} \frac{ n! }{ L\sqrt{ 2n }\left( \frac{ n }{ e } \right)^n }=1.
            \end{equation}
            Autrement dit, en posant
            \begin{equation}        \label{EQyiBooFVCCooJbGpsW}
                n!=\alpha(n)L\sqrt{ 2n }\left( \frac{ n }{ e } \right)^n,
            \end{equation}
            nous avons \( \lim_{n\to \infty} \alpha(n)=1\).

        \item[Introduction de Wallis]
            Nous avons déjà parlé dans le lemme \ref{LEMooUOIBooLyMDft} du nombre
            \begin{equation}
                W_{2n}=\frac{ (2n)! }{ (2^nn!)^2 }\frac{ \pi }{2}.
            \end{equation}
            Le lemme \ref{LEMooWQZAooOXAPQO} implique que la suite \( I_n=W_{2n}\) est équivalente à \( \sqrt{ \pi/4n }\). 
            
            Cela étant dit, nous faisons un gros calcul en remplaçant les factorielles dans \( W_{2n}\) par la formule \eqref{EQyiBooFVCCooJbGpsW}. Après pas mal de calculs\footnote{Si vous êtes en manque de papier de brouillon, c'est le moment de vous inquiéter.}, nous trouvons
            \begin{equation}
                I_n=\frac{ \alpha(2n) }{ \alpha(n)^2 }\frac{1}{ \sqrt{ n } }\frac{ \pi }{ 2n }.
            \end{equation}
            Nous avons donc
            \begin{equation}
                \sqrt{ \frac{ \pi }{ 4n } }\sim \frac{ \alpha(2n) }{ \alpha(n)^2 }\frac{ \pi }{ \sqrt{ n } }\frac{1}{ 2L }.
            \end{equation}
            Il existe donc une fonction \( \beta(n)\) avec \( \beta(n)\to 1\) telle que
            \begin{equation}
                \frac{ \sqrt{ \pi } }{2}\frac{1}{ \sqrt{ n } }=\beta(n)\frac{ \alpha(2n) }{ \alpha(n)^2 }\frac{ \pi }{ \sqrt{ n } }\frac{1}{ 2L }.
            \end{equation}
            En simplifiant par \( 1/\sqrt{ n }\),
            \begin{equation}
                \frac{ \sqrt{ \pi } }{2}=\beta(n)\frac{ \alpha(2n) }{ \alpha(n)^2 }\frac{\pi}{ 2L }.
            \end{equation}
            Nous prenons à présent la limite \( n\to \infty\) en nous rappelant que \( \alpha\) et \( \beta \) donnent \( 1\). Après simplifications, nous trouvons
            \begin{equation}
                L=\sqrt{ \pi }.
            \end{equation}
            
        \item[La fin]
            Nous introduisons la valeur \( L=\sqrt{ \pi }\) dans l'expression \eqref{EQyiBooFVCCooJbGpsW} de la factorielle :
            \begin{equation}
                n!=\alpha(n)\sqrt{ 2\pi n }\left( \frac{ n }{ e } \right)^n.
            \end{equation}
    \end{subproof}
    La dernière équation est exactement ce qui signifie l'équivalence de suite demandée.
\end{proof}

%--------------------------------------------------------------------------------------------------------------------------- 
\subsection{La fonction sinus cardinal, intégrale de Dirichlet}
%---------------------------------------------------------------------------------------------------------------------------

\begin{definition}
    La fonction \defe{sinus cardinal}{sinus cardinal} est
    \begin{equation}
        f(t)=\begin{cases}
            1    &   \text{si } t=0 \\
            \frac{ \sin(t) }{ t }    &    \text{sinon. }
        \end{cases}
    \end{equation}
\end{definition}
Elle sert à plein de choses. Entre autres, le lemme \ref{LEMooEEWSooZwLSAP} montrera que la fonction \( x\mapsto | \sin(x)/x |\) a une intégrale sur \( \eR\) qui vaut \( \infty\). Cela nous permettra de donner un exemple d'une fonction dans \( L^1(\eR)\) dont la transformée de Fourier n'est pas dans \( L^1(\eR)\) (lemme \ref{LEMooROPHooOSguhN}).

\begin{normaltext}
    Le but que nous nous fixons maintenant est de prouver que
    \begin{equation}
        \int_{0}^{\infty}\frac{ \sin(t) }{ t }dt=\frac{ \pi }{2}.
    \end{equation}

    Un adage dit que si un théorème est trop long, c'est qu'il n'a pas assez de lemmes. Nous allons faire plein de lemmes.
\end{normaltext}

\begin{lemma}
    La fonction sinus cardinal est continue.
\end{lemma}

\begin{proof}
    Elle est continue en zéro parce que le lemme \ref{LEMooZYNEooYkwsWD} nous donne
    \begin{equation}
        \lim_{t\to 0}\frac{ \sin(t) }{ t }=1.
    \end{equation}
\end{proof}

Nous commençons par une mauvaise nouvelle.
\begin{lemma}[\cite{MonCerveau}]           \label{LEMooEEWSooZwLSAP}
    Nous avons
    \begin{equation}
        \int_{\eR}| \frac{ \sin(x) }{ x } |dx=\infty.
    \end{equation}
\end{lemma}

\begin{proof}
    Soit \( \delta>0\) tel que sur l'intervalle \( \mathopen[ \frac{ \pi }{2}-\delta , \frac{ \pi }{ 2 }+\delta \mathclose]\), nous ayons \( \sin(x)>0.9\)\footnote{Ça existe par une astucieuse combinaison du théorème \ref{ThoValInter} des valeurs intermédiaires, de la valeur remarquable \( \sin(\pi/2)=1\) (de \eqref{SUBEQSooBTNPooSvCAHO}) et du fait que \( \sin\) est continue (proposition \ref{PROPooZXPVooBjONka}).} .

    Les intervalles \( I_k=\mathopen[ \frac{ \pi }{2}-\delta+2k\pi , \frac{ \pi }{2}+\delta+2k\pi \mathclose]\) sont disjoints et la fonction que nous intégrons est partout positive. Nous découpons
    \begin{equation}
        \eR=C+\bigcup_{k=0}^{\infty}\mathopen[ \frac{ \pi }{2}-\delta+2k\pi , \frac{ \pi }{2}+\delta+2k\pi \mathclose]
    \end{equation}
    où \( C\) est le complémentaire qu'il faut pour faire \( \eR\).

    La \( \sigma\)-additivité de l'intégrale de Lebesgue (proposition \ref{PROPooDWYNooWKJmEV}) nous indique que
    \begin{equation}
        \int_{\eR}\big| \frac{ \sin(x) }{ x } \big|= \int_C\big| \frac{ \sin(x) }{ x } \big|+  \sum_{k=0}^{\infty}\int_{I_k}\big| \frac{ \sin(x) }{ x } \big|dx
    \end{equation}
    Vu que tous les termes sont positifs, nous obtenons une majoration en en supprimant un. Allons-y :
    \begin{subequations}        \label{SUBEQSooSRAYooEOBwiC}
        \begin{align}
            \int_{\eR}\big| \frac{ \sin(x) }{ x } \big|dx&\geq \sum_{k=0}^{\infty}\int_{I_k}\big| \frac{ \sin(x) }{ x } \big|dx\\
            &\geq\sum_{k=0}^{\infty}0.9\int_{I_k}\frac{ 1 }{ x }dx\\
            &\geq 0.9\sum_{k=0}^{\infty}2\delta\frac{1}{ \frac{ \pi }{2}-\delta+2k\pi }     \label{SUBEQooKMURooVuIpCo}\\
            &\geq 1.8\delta\sum_{k=0}^{\infty}\frac{1}{ 2\pi(k+1) }     \label{SUBEQooJCQOooYqUCps}\\
            &=\frac{ 1.8\delta }{ 2\pi }\sum_{k=0}^{\infty}\frac{1}{ k+1 }.
        \end{align}
    \end{subequations}
    Justifications :
    \begin{itemize}
        \item Pour \eqref{SUBEQooKMURooVuIpCo}, nous avons majoré \( \frac{1}{ x }\) par \( \frac{1}{ \frac{ \pi }{ 2 }+\delta+2k\pi }\) sur \( I_k\).
        \item Pour \eqref{SUBEQooJCQOooYqUCps}, nous avons dit que \( \frac{ \pi }{2}+\delta<2\pi\).
    \end{itemize}
    La dernière somme dans \eqref{SUBEQSooSRAYooEOBwiC} diverge.

    Donc la fonction sinus cardinal n'est pas dans \( L^1(\eR)\).
\end{proof}

La mauvaise nouvelle suivante en est un corolaire immédiat.
\begin{lemma}       \label{LEMooBEQRooHaugKj}
    La fonction \( t\mapsto \frac{ \sin(t) }{ t }\) n'est pas intégrable sur \( \mathopen[ 0 , \infty \mathclose[\) au sens de Lebesgue.
\end{lemma}

\begin{proof}
    Le lemme \ref{LEMooEEWSooZwLSAP} nous dit que \( \int_0^{\infty}| f |=\infty\). Dans ce cas, \( \int_0^{\infty}f\) n'existe pas par le lemme \ref{LEMooMWKTooIKomSw}.
\end{proof}

Donc l'intégrale \( \int_0^{\infty}\frac{ \sin(t) }{ t }dt\) n'existe pas parce que la définition de l'intégrale de Lebesgue ne permet pas de profiter des compensations qui arrivent entre les valeurs positives et négatives.

Nous définissons donc
\begin{equation}        \label{EQooWAQLooTFOPbl}
    \int_0^{\infty}\frac{ \sin(t) }{ t }dt=\lim_{b\to \infty} \int_0^b\frac{ \sin(t) }{ t }dt.
\end{equation}
Pour chaque \( b\), l'intégrale existe sans problèmes (fonction continue sur le compact \( \mathopen[ 0 , b \mathclose]\)), et les compensations se font. Il n'est pas pas sans espoir que la limite \eqref{EQooWAQLooTFOPbl} existe et valle un nombre fini.

\begin{lemma}[\cite{BIBooCFXJooWrArNT}]     \label{LEMooTFVZooRAmjUN}
    La limite
    \begin{equation}
        \lim_{b\to \infty}\int_0^b\frac{ \sin(t) }{ t }dt
    \end{equation}
    existe dans \( \eR\).
\end{lemma}

\begin{proof}
    En plusieurs parties.
    \begin{subproof}
        \item[Découpage]

            Nous découpons l'intervalle \( \mathopen[ 0 , b \mathclose]\) en morceaux du type \( \mathopen[ k\pi , (k+1)\pi \mathclose]\) et un morceau restant lorsque \( b\) n'est pas un multiple de \( \pi\) :
            \begin{equation}
                \mathopen[ 0 , b \mathclose]=\bigcup_{k=0}^{N(b)-1}\mathopen[ k\pi , (k+1)\pi \mathclose]\cup\mathopen[ N(b)\pi , b \mathclose]
            \end{equation}
            où \( N(b)\) est un entier bien choisi\quext{Il me semble que le traitement de ce terme manque dans \cite{BIBooCFXJooWrArNT}.}. En tout cas \( \lim_{b\to \infty}N(b)=\infty\).

        \item[Majoration 1]

            Pour chaque \( b\in \eR^+\) nous avons
            \begin{equation}
                \int_0^b\frac{ \sin(t) }{ t }dt=\sum_{k=0}^{N(b)}\int_{\mathopen\big[ k\pi , (k+1)\pi \mathclose\big]}\frac{ \sin(t) }{ t }dt+\int_{\mathopen[ \big( N(b)+1 \big)\pi , b \mathclose]}\frac{ \sin(t) }{ t }
            \end{equation}
            Dans le dernier terme, nous majorons \( | \sin(t) |\leq 1\) et \( \frac{1}{ t }\leq \big( N(b)+1 \big)\pi\). Cela donne
            \begin{equation}
                \left|  \int_{\mathopen\big[ \big(N(b)+1\big)\pi  , b \mathclose\big]}\frac{ \sin(t) }{ t } \right|\leq \frac{ b-\big( N(b)+1 \big)\pi }{ \big( N(b)+1 \big)\pi }\leq \frac{1}{ N(b)+1 }
            \end{equation}
            où nous avons encore majoré \( b\leq \big( N(b)+2 \big)\pi\).

        \item[Majoration 2]

            En ce qui concerne les autres termes, sur l'intervalle \( \mathopen[ k\pi , (k+1)\pi \mathclose]\), nous avons \( \sin(t)=(-1)^k| \sin(t) |\). Nous avons alors
            \begin{equation}        \label{EQooHZPRooFuwRWQ}
                \int_0^b\frac{ \sin(t) }{ t }dt=\sum_{k=0}^{N(b)}(-1)^k\int_{k\pi}^{(k+1)\pi}\frac{ | \sin(t) | }{ t }dt+\alpha(b)
            \end{equation}
            où \( | \alpha(b) |\leq \frac{1}{ N(b)+1 }\); l'important est que \( \lim_{b\to \infty}\alpha(b)=0\).
            
        \item[Une suite alternée]

            L'inégalité \eqref{EQooHZPRooFuwRWQ} nous incite à étudier la série \( \sum_{k=0}^{\infty}(-1)^ka_k\) en ayant posé
            \begin{equation}
                a_k=\int_{k\pi}^{(k+1)\pi}\frac{ | \sin(t) | }{ t }dt.
            \end{equation}
            Nous montrons à présent que la suite \( (a_k)\) vérifie les conditions du critère des séries alternées \ref{THOooOHANooHYfkII}.

            D'abord, \( a_{k+1}\leq a_k\). En effet en utilisant le changement de variables\footnote{Le théorème \ref{THOooUMIWooZUtUSg} est toujours bon à citer.} \( u=t-\pi\),
            \begin{equation}
                a_{k+1}=\int_{(k+1)\pi}^{(k+2)\pi}\frac{ | \sin(t) | }{ t }dt=\int_{k\pi}^{(k+1)\pi}\frac{ | \sin(u+\pi) | }{ u+\pi }du=\int_{k\pi}^{(k+1)\pi}\frac{ | \sin(u) | }{ u+\pi }du<a_k.
            \end{equation}
            Nous avons utilisé le fait que \( | \sin(u+\pi) |=| \sin(u) |\) pour tout \( u\).

            De plus, vu que \( | \sin(t) |\leq 1\) et que \( t\in\mathopen[ k\pi , (k+1)\pi \mathclose]\), nous avons
            \begin{equation}
                \int_{k\pi}^{(k+1)\pi}\frac{ | \sin(t) | }{ t }\leq \int_{k\pi}^{(k+1)\pi}\frac{1}{ k\pi }=\frac{ (k+1)\pi-k\pi }{ k\pi }=\frac{1}{ k }.
            \end{equation}
            Donc \( a_k\leq\frac{1}{ k }\to 0\).

            Le critère des séries alternées \ref{THOooOHANooHYfkII} nous dit que
            \begin{equation}
                \sum_{k=0}^{\infty}(-1)^ka_k<\infty.
            \end{equation}

        \item[Conclusion]
            Nous repartons de \eqref{EQooHZPRooFuwRWQ} :
            \begin{equation} 
                \int_0^b\frac{ \sin(t) }{ t }dt=\sum_{k=0}^{N(b)}(-1)^k\int_{k\pi}^{(k+1)\pi}\frac{ | \sin(t) | }{ t }dt+\alpha(b).
            \end{equation}
            Cette égalité est valable pour tout \( b\in \eR^+\). Le passage à la limite \( b\to 0\) à droite donne un nombre fini; donc à gauche aussi, et nous avons prouvé que
            \begin{equation}
                \lim_{b\to\infty} \int_0^b\frac{ \sin(t) }{ t }dt<\infty.
            \end{equation}
    \end{subproof}
\end{proof}
Notre tache n'est donc pas sans espoir. Au moins l'intégrale que nous cherchons à évaluer est finie.

\begin{lemma}       \label{LEMooARPIooDPSGwR}
    Soit \( x\in \mathopen] 0 , \infty \mathclose[\). L'intégrale
    \begin{equation}
        F(x)=\int_0^{\infty} e^{-tx}\frac{ \sin(t) }{ t }dt
    \end{equation}
    existe au sens de Lebesgue usuel.
\end{lemma}

\begin{proof}
    Vu qu'en \( t=0\) nous avons \( \frac{ \sin(t) }{ t }=1\), il n'y a pas de problèmes de ce côté. Lorsque \( t>1\) nous avons la majoration
    \begin{equation}
        |  e^{-tx}\frac{ \sin(t) }{ t } |\leq |  e^{-tx} |.
    \end{equation}
    Lorsque \( t\) est assez grand, le lemme \ref{LEMooNYFVooXjFShk} nous donne aussi la majoration
    \begin{equation}
        |  e^{-tx} |\leq \frac{1}{ t^2 }.
    \end{equation}
    La proposition \ref{PropBKNooPDIPUc}\ref{ITEMooJFSXooHmgmEj} implique que \( \int_1^{\infty}\frac{1}{ t^2 }<\infty\). Et les majorations font que la proposition \ref{PROPooGTMVooPHcrRl} nous donne le résultat.
\end{proof}

\begin{lemma}[\cite{BIBooCFXJooWrArNT}]     \label{LEMooRDCSooBrWmep}
    Il existe une constante \( C\in \eR\) telle que
    \begin{equation}
        I(x)=\int_0^{\infty} e^{-tx}\frac{ \sin(t) }{ t }dt=-\arctan(x)+C
    \end{equation}
    pour tout \( x>0\).
\end{lemma}

\begin{proof}
    En permutant dérivée et intégrale, nous allons prouver que \( I'(x)=-\frac{1}{ 1+x^2 }\).

    \begin{subproof}
        \item[Permuter]
            Nous posons
            \begin{equation}
                f(x,t)= e^{-tx}\frac{ \sin(t) }{ t },
            \end{equation}
            et nous vérifions les hypothèses du théorème \ref{ThoMWpRKYp}.

            \begin{enumerate}
                \item
                    Pour chaque \( x>0\) fixé, la fonction \( t\mapsto f(x,t)\) est intégrable sur \( \mathopen[ 0 , \infty \mathclose[\), c'est le lemme \ref{LEMooARPIooDPSGwR}.
                    \item
                        Pour \( t>0\) fixé, la fonction \( x\mapsto f(x,t)\) est dérivable.
                    \item
                        Nous avons la dérivée partielle
                        \begin{equation}
                            \frac{ \partial f }{ \partial x }(x,t)=- e^{-tx}\sin(t)
                        \end{equation}
                        qui vérifie
                        \begin{equation}
                            | \frac{ \partial f }{ \partial x }(x,t) |\leq  e^{-tx},
                        \end{equation}
                        alors que la fonction \( t\mapsto  e^{-tx}\) est intégrable sur \( \mathopen[ 0 , \infty \mathclose[\).
            \end{enumerate}
            Nous pouvons donc dériver sous l'intégrale et obtenir
            \begin{equation}
                F'(x)=-\int_0^{\infty} e^{-xt}\sin(t)dt.
            \end{equation}
        \item[Quelques intégrations par partie]
            Nous posons
            \begin{equation}
                J(x)=\int_0^{\infty} e^{-xt}\sin(t)dt,
            \end{equation}
            et nous allons la faire par parties\footnote{Proposition \ref{PROPooRLFIooQHnyJY}.}, en deux fois.

            D'abord en posant \( u= e^{-xt}\) et \( v'=\sin(t)\) nous avons
            \begin{equation}
                J(x)=\left[ - e^{-xt}\cos(t) \right]_{t=0}^{t=\infty}-\int_0^{\infty}(-)x e^{-xt}(-)\cos(t)dt=1-x\int_0^{\infty} e^{-xt}\cos(t)dt.
            \end{equation}
            Nous faisons l'intégrale encore par parties en posant \( u= e^{-xt}\) et \( v'=\cos(t)\) :
            \begin{equation}
                \int_0^{\infty} e^{-xt}\cos(t)dt=\left[  e^{-xt}\sin(t) \right]_0^{\infty}-\int_0^{\infty}(-) e^{-xt}\sin(t)dt=J(x).
            \end{equation}
            Donc
            \begin{subequations}
                \begin{align}
                    J(x)&=1-x\int_0^{\infty} e^{-xt}\cos(t)dt\\
                    &=1-x\int_0^{\infty} e^{-xt}\cos(t)dt\\
                    &=1-x\Big( x\underbrace{\int_0^{\infty} e^{-xt}\sin(t)dt}_{J(x)} \Big)\\
                    &=1-x^2J(x).
                \end{align}
            \end{subequations}
            Voila qui prouve que \( J(x)=\frac{1}{ 1+x^2 }\), et donc que
            \begin{equation}
                F'(x)=-J(x)=-\frac{1}{ 1+x^2 }.
            \end{equation}
            
        \item[Et enfin]

            Le théorème \ref{THOooUSVGooOAnCvC}\ref{ITEMooMNHLooOVhIIb} nous dit que la dérivée de la fonction \( \arctan\) est précisément \( 1/(1+x)\). Donc \( F\) et \( \arctan\) ont la même dérivée (au signe près). Donc il existe \( C\in \eR\) tel que
            \begin{equation}
                F(x)=-\arctan(x)+C
            \end{equation}
            pour tout \( x>0\).

    \end{subproof}
\end{proof}

\begin{lemma}       \label{LEMooEOYHooVIMCCa}
    Nous avons
    \begin{equation}
        F(x)=\int_0^{\infty} e^{-xt}\frac{ \sin(t) }{ t }dt=-\arctan(x)+\frac{ \pi }{2}
    \end{equation}
    pour tout \( x>0\).
\end{lemma}

\begin{proof}
    Le but de ce lemme est de fixer la constante laissée arbitraire dans le lemme \ref{LEMooRDCSooBrWmep}. Nous savons qu'il existe \( C\in \eR\) tel que
    \begin{equation}        \label{EQooTZGXooUxfAjT}
        F(x)=-\arctan(x)+C.
    \end{equation}
    Le but est de prendre la limite \( x\to\infty\) des deux côtés.

    Par le lemme \ref{LEMooSFALooVRBdNb}, nous avons \( | \frac{ \sin(t) }{ t } |\leq 1\) sur \( \mathopen[ 0 , \infty \mathclose[\). Donc
    \begin{equation}
        F(x)\leq \int_0^{\infty} e^{-xt}dt=\left[ -\frac{1}{ x } e^{-xt} \right]_{t=0}^{t=\infty}=\frac{1}{ x }.
    \end{equation}
    Vu que \( F(x)\leq \frac{1}{ x }\) pour tout \( x\), nous avons certainement \( \lim_{x\to \infty} F(x)=0\).

    D'autre part,
    \begin{equation}
        \lim_{x\to \infty} \arctan(x)=\frac{ \pi }{2}.
    \end{equation}
    
    En passant à la limite dans \eqref{EQooTZGXooUxfAjT}, nous avons
    \begin{equation}
        0=-\frac{ \pi }{2}+C,
    \end{equation}
    et donc \( C=\pi/2\).
\end{proof}

Nous avons maintenant la formule
\begin{equation}
    F(x)=\int_0^{\infty} e^{-tx}\frac{ \sin(t) }{ t }dt=-\arctan(x)+\frac{ \pi }{2}
\end{equation}
qui est valable pour tout \( x>0\).

Notre but sera de prendre la limite \( x\to 0\) des deux côtés. Vu que \( \arctan\) est continue, le membre de droite ne pose pas de problèmes et donne \( \pi/2\). Pour le membre de gauche, il faut encore permuter une limite et une intégrale.

Pour la suite, nous allons étudier\cite{BIBooCFXJooWrArNT}
\begin{equation}
    \int_0^{\infty}(1- e^{-xt})\frac{ \sin(t) }{ t }dt.
\end{equation}
Cette intégrale n'existe pas au sens de Lebesgue et est définie par
\begin{equation}
    L(x)=\lim_{b\to \infty} \int_0^b(1- e^{-xt})\frac{ \sin(t) }{ t }dt.
\end{equation}
Rien n'indique cependant pour l'instant que cette limite existe.

\begin{lemma}[\cite{BIBooCFXJooWrArNT}]     \label{LEMooZGODooLaBuHo}
    Soit \( x>0\). Nous posons
    \begin{equation}        \label{EQooJXWMooRbbCtt}
        L_k(x)=\int_{k\pi}^{(k+1)\pi}(1- e^{-tx})\frac{ | \sin(t) | }{ t }dt.
    \end{equation}
    La suite \( (L_k(x))_{k\in \eN}\) satisfait le critère des séries alternées\footnote{Théorème \ref{THOooOHANooHYfkII}.}, c'est-à-dire que cette suite est positive, décroissante à limite nulle.
\end{lemma}

\begin{proof}
    Notons que chacune des intégrales \( L_k(x)\) est sans problèmes : fonction continue sur un compact. Trois éléments à prouver.
    \begin{subproof}
        \item[Positive]
            Vu que dans toute notre histoire, \( x,t>0\), nous avons \( 1- e^{-tx}>0\) et donc toute la fonction intégrée est positive.
        \item[Tend vers zéro]
            Vu que \( 1- e^{-tx}<1\), nous avons
            \begin{equation}        \label{EQooCGAQooDSvbln}
                L_k(x)\leq \int_{k\pi}^{(k+1)\pi}\frac{1}{ t }dt\leq \pi\frac{1}{ k\pi }=\frac{1}{ k }.
            \end{equation}
            Donc \( \lim_{k\to \infty} L_k(x)=0\).
        \item[Décroissante]
            Nous devons à présent prouver que \( L_k(x)\) est décroissante en \( k\) lorsque \( x\) est fixé.

            Nous avons \( L_{k+1}(0)=L_k(0)\) pour tout \( k\). Nous allons montrer que \( L'_{k+1}(x)<L'_k(x)\) pour tout \( x>0\). De cette façon nous aurons bien \( L_{k+1}(x)<L_k(x)\) pour tout \( k\) et \( x\).

            En permutant (encore) intégrale et dérivée,
            \begin{subequations}
                \begin{align}
                    L_{k+1}'(x)&=\int_{(k+1)\pi}^{(k+2)\pi} e^{-tx}| \sin(t) |dt        \label{EQooAGGCooSoPHnz}\\
                    &=\int_{k\pi}^{(k+1)\pi} e^{-(u+\pi)x}| \sin(u+\pi) |       \label{SUBEQooEWZSooQtZBYI}\\
                    &= e^{-\pi x}\int_{k\pi}^{(k+1)\pi} e^{-ux}| \sin(u) |du    \label{SUBEQooYGDQooLWqrvg}\\
                    &= e^{-\pi x}L_k'(x).
                \end{align}
            \end{subequations}
            Justifications :
            \begin{itemize}
                \item Pour \eqref{EQooAGGCooSoPHnz}, permuter dérivée et intégrale; je ne donne pas tout le détail. Ça a déjà été fait.
                \item Pour \eqref{SUBEQooEWZSooQtZBYI}, nous avons fait le changement de variables \( u=t-\pi\).
                \item Pour \eqref{SUBEQooYGDQooLWqrvg}, nous avons utilisé le fait que \( | \sin(u+\pi) |=| \sin(u) |\) ainsi que \(  e^{-(u+\pi)x}= e^{-ux} e^{-\pi x}\) par \eqref{EQooEWIHooDRAQGR}.
            \end{itemize}
            Nous avons donc prouvé que
            \begin{equation}
                L_{k+1}'(x)= e^{-\pi x}L'_{k}(x)<L_k'(x).
            \end{equation}
    \end{subproof}
\end{proof}

\begin{lemma}       \label{LEMooSWFDooGLfwoD}
    Pour chaque \( x>0\), nous avons la limite
    \begin{equation}
        \lim_{b\to \infty} \int_0^{b}(1- e^{-tx})\frac{ \sin(t) }{ t }dt=\sum_{k=0}^{\infty}(-1)^kL_k(x)<\infty.
    \end{equation}
\end{lemma}

\begin{proof}
    Nous fixons (provisoirement) \( b\) et nous découpons l'intervalle d'intégration comme
    \begin{equation}
        \mathopen[ 0 , b \mathclose]=\bigcup_{k=1}^N\mathopen[ k\pi , (k+1)\pi \mathclose]\cup\mathopen\big[ (N+1)\pi  , b \mathclose\big]
    \end{equation}
    où \( N\) est une fonction de \( b\); quelque chose comme \( N(b)\) est le plus grand entier tel que \(\big( N(b)+1 \big)\pi\leq \pi\). Sur chacun des intervalles nous avons \( \sin(t)=(-1)^k| \sin(t) |\). Nous avons donc
    \begin{equation}       \label{EQooGBEDooSeuwMN}
        \begin{aligned}[]
            \int_0^b(1- e^{-tx})\frac{ \sin(t) }{ t }dt&=\sum_{k=0}^{N(b)}(-1)^k\int_{k\pi}^{(k+1)\pi}(1- e^{-tx})\frac{ | \sin(t) | }{ t }dt\\
                &\quad+\int_{(N+1)\pi}^b(1- e^{-tx})\frac{ \sin(t) }{ t }dt
        \end{aligned}
    \end{equation}
    Le premier terme est \( \sum_{k=0}^{N(b)}L_k(x)\), dont nous savons que la limite \( b\to \infty\) existe parce que \( L_k(x)\) vérifie le critère des séries alternées (lemme \ref{LEMooZGODooLaBuHo}). En ce qui concerne le second terme,
    \begin{equation}
        \big|\int_{(N+1)\pi}^b (1- e^{-tx})\frac{ \sin(t) }{ t }\big| <  \frac{  b-( N+1 )\pi    }{ ( N+1 )\pi }<\frac{1}{ N(b)+1 }.
    \end{equation}
    La dernière inégalité est le fait que \( N(b)\) est choisi pour avoir \( b-\big( N(b)+1 \big)\pi<\pi\).

            Les deux termes de \eqref{EQooGBEDooSeuwMN} ont donc une limite lorsque \( b\to \infty\). Nous pouvons donc passer à la limite en sommant les deux limites :
            \begin{equation}
                \int_0^{\infty}(1- e^{-tx})\frac{ \sin(t) }{ t }dt=\sum_{k=0}^{\infty}(-1)^kL_k(x).
            \end{equation}
            Cette égalité est valable pour chaque \( x>0\). 
            
            Le fait que la limite soit finie est dans le critère des séries alternées. Pour chaque \( x\), la suite \( L_k(x)\) vérifie ce critère par le lemme \ref{LEMooZGODooLaBuHo}.
\end{proof}

\begin{lemma}       \label{LEMooNZVSooDbZCZx}
    Nous avons
    \begin{equation}
        \lim_{x\to 0^+} \int_0^{\infty}(1- e^{-tx})\frac{ \sin(t) }{ t }dt=0.
    \end{equation}
\end{lemma}

\begin{proof}
    La définition de l'intégrale ainsi que le lemme \ref{LEMooSWFDooGLfwoD} nous ont déjà donné
    \begin{equation}
        \int_0^{\infty}(1- e^{-tx})\frac{ \sin(t) }{ t }dt=\lim_{b\to \infty} \int_0^b(1- e^{-tx})\frac{ \sin(t) }{ t }dt=\sum_{k=0}^{\infty}(-1)^kL_k(x)
    \end{equation}
    ainsi que l'assurance que le tout est un nombre réel fini\footnote{De toutes façons, il n'existe pas de nombres réels infinis, mais vous voyez ce que je veux dire.}.

    \begin{subproof}
        \item[Majoration pour la série alternée]

            Nous majorons un peu. Pour \( x>0\) et \( N\in \eN\) nous avons
            \begin{subequations}        \label{SUBEQSQooLIGNooNAzpmi}
                \begin{align}
                    | \sum_{k=0}^N(-1)^kL_k(x) |&\leq \sum_{k=0}^N| L_k(x) |\\
                    &=\sum_{k=0}^N\big|   \int_{k\pi}^{(k+1)\pi}(1- e^{-tx})\frac{ | \sin(t) | }{ t }dt   \big| \label{EQooIIVVooJvHFhS}\\
                    &\leq \sum_{k=0}^N\int_{k\pi}^{(N+1)\pi}(1- e^{-tx})\frac{ | \sin(t) | }{ t }dt\\
                    &=\int_0^{(N+1)\pi}(1- e^{-tx})\frac{ | \sin(t) | }{ t }dt\\
                    &\leq \int_0^{(N+1)\pi}tx\frac{ | \sin(t) | }{ t }dt    \label{SUBEQooZPVVooTBZzdl}\\
                    &\leq \int_0^{(N+1)\pi}xdt\\
                    &=x(N+1)\pi.
                \end{align}
            \end{subequations}
            Justifications :
            \begin{itemize}
                \item Pour \eqref{EQooIIVVooJvHFhS} c'est la définition \eqref{EQooJXWMooRbbCtt}.
                \item Pour \eqref{SUBEQooZPVVooTBZzdl}, c'est le fait que \( 0\leq 1- e^{-u}\leq u\) pour tout \( u\geq 0\) ainsi que la sous-additivité de l'intégrale de la proposition \ref{PropOPSCooVpzaBt}.
            \end{itemize}

        \item[Majoration pour l'intégrale]
            
            Nous fixons \( N\in \eN\), et nous avons :
            \begin{subequations}        \label{SUBEQSooSBSJooMAkJPh}
                \begin{align}
                    \big| \lim_{b\to \infty} \int_0^b(1-e^{-tx})\frac{ \sin(t) }{ t } dt\big|&=\big| \sum_{k=0}^{\infty}(-1)^kL_k(x) \big|\\
                    &=\big| \sum_{k=0}^N(-1)^kL_k(x)+\sum_{k=N+1}^{\infty}(-1)^kL_k(x) \big|\\
                    &\leq\big| \sum_{k=0}^N(-1)^kL_k(x) \big|+\big| \sum_{k=N+1}^{\infty}(-1)^kL_k(x) \big|\\
                    &\leq\big| \sum_{k=0}^N(-1)^kL_k(x) \big|+L_{N+1}(x)    \label{SUBEQooYFWDooYKhYtd}\\
                    &\leq\big| \sum_{k=0}^N(-1)^kL_k(x) \big|+\frac{1}{ N+1 }    \label{SUBEQooDDRGooNDfxqO}\\
                \end{align}
            \end{subequations}
            Justifications :
            \begin{itemize}
                \item Pour \eqref{SUBEQooYFWDooYKhYtd} c'est le reste du critère des séries alternées, théorème \ref{THOooOHANooHYfkII}\ref{ITEMooWEPWooXhLMYL}.
                \item Pour \eqref{SUBEQooDDRGooNDfxqO} c'est la majoration \eqref{EQooCGAQooDSvbln} déjà faite.
            \end{itemize}

        \item[Les deux ensemble]
            Pour chaque \( N\) et pour chaque \( x>0\) nous avons, en mettant \eqref{SUBEQSQooLIGNooNAzpmi} au bout de \eqref{SUBEQSooSBSJooMAkJPh} :
            \begin{equation}
                \big| \int_0^{\infty}(1- e^{-tx})\frac{ \sin(t) }{ t }dt \big|\leq x(N+1)\pi+\frac{1}{ N+1 }.
            \end{equation}
            En prenant la limite \( x\to 0\) nous trouvons
            \begin{equation}
                \lim_{x\to 0} \int_0^{\infty}(1- e^{-tx})\frac{ \sin(t) }{ t }dt\leq \frac{1}{ N+1 }
            \end{equation}
            pour tout \( N\). Donc cette limite est nulle :
            \begin{equation}
                \lim_{x\to 0} \int_0^{\infty}(1- e^{-tx})\frac{ \sin(t) }{ t }dt=0.
            \end{equation}
    \end{subproof}
\end{proof}

Maintenant que nous avons fait plein de lemmes, nous pouvons énoncer notre résultat principal, et le démontrer facilement.

\begin{theorem}[Intégrale de Dirichlet\cite{BIBooCFXJooWrArNT}]
    Nous avons
    \begin{equation}
        \lim_{b\to \infty} \int_0^b\frac{ \sin(t) }{ t }dt=\frac{ \pi }{2}.
    \end{equation}
\end{theorem}
\index{intégrale de Dirichlet}

Nous avons écrit \( \lim_{b\to \infty} \int_0^b\frac{ \sin(t) }{ t }dt\) et non \( \int_0^{\infty}\frac{ \sin(t) }{ t }dt\) parce que cette dernière intégrale n'existe pas vraiment au sens de Lebesgue, voir le lemme \ref{LEMooBEQRooHaugKj}. Dans la suite nous écrirons cependant \( \int_0^{\infty}\frac{ \sin(t) }{ t }dt\), en gardant en tête que cela n'est défini que via la limite.

\begin{proof}
    Nous nommons \( D\) la valeur que nous cherchons. Le lemme \ref{LEMooTFVZooRAmjUN} nous assure que
    \begin{equation}
        D=\lim_{b\to \infty} \int_0^b\frac{ \sin(t) }{ t }<\infty.
    \end{equation}
    Le lemme \ref{LEMooEOYHooVIMCCa} nous donne, quant à lui,
    \begin{equation}
        \lim_{b\to \infty} \int_0^b e^{-tx}\frac{ \sin(t) }{ t }dt=-\arctan(x)+\frac{ \pi }{2}.
    \end{equation}
    Vu que les deux limites existent, on peut permuter somme et limite\footnote{C'est une phrase un peu grandiloquente pour dire que \( \lim_{b\to a} f(b)-\lim_{b\to a} g(b)=\lim_{b\to a} (f(b)-g(b))  \). Ici nous avons \( a=\infty\) et les fonctions \( f\) et \( g\) sont celles définies par les intégrales.} :
    \begin{subequations}
        \begin{align}
            D+\arctan(x)-\frac{ \pi }{2}&=\lim_{b\to \infty} \big(    \int_0^b\frac{ \sin(t) }{ t } +  \int_0^b e^{-tx}\frac{ \sin(t) }{ t }dt  \big)\\
            &=\lim_{b\to \infty} \int_0^b(1- e^{-xt})\frac{ \sin(t) }{ t }dt.    \label{SUBEQooXIUNooPZnCPb}
        \end{align}
    \end{subequations}
     Pour \eqref{SUBEQooXIUNooPZnCPb}, nous avons des fonctions bornées sur un intervalle borné (\( \mathopen] 0 , b \mathclose[\)), donc il n'y a pas de mal à sommer les intégrales.
         
         Donc pour tout \( x>0\), nous avons
         \begin{equation}
             D+\arctan(x)-\frac{ \pi }{2}=\int_0^{\infty}(1- e^{-xt})\frac{ \sin(t) }{ t }dt.
         \end{equation}
         Nous passons à la limite \( x\to 0\) en utilisant le lemme \ref{LEMooNZVSooDbZCZx} et le fait que \( \arctan(0)=0\) (lemme \ref{LEMooPQNCooDkEUyw}) :
         \begin{equation}
             D-\frac{ \pi }{2}=0,
         \end{equation}
         c'est à dire que résultat annoncé.
\end{proof}



\chapter{Arcs paramétrés}
% This is part of Mes notes de mathématique
% Copyright (c) 2010-2018, 2020
%   Laurent Claessens, Carlotta Donadello
% See the file fdl-1.3.txt for copying conditions.

La structure de ce chapitre, comme beaucoup de choses dans le Frido, est fortement liée au choix de présenter toutes les matières dans l'ordre mathématiquement logique. Nous devons donc le placer après la trigonométrie; les propriétés principales des fonctions trigonométriques étant dans la proposition~\ref{PROPooMWMDooJYIlis}, et c'est la proposition~\ref{PROPooKSGXooOqGyZj} qui nous permet de dire que \( \big( \cos(t),\sin(t) \big)\) décrit le cercle.

Et enfin nous n'avons pas encore calculé la circonférence du cercle, et pour cause : nous n'avons pas encore donné de définition à la longueur d'un chemin dans \( \eR^2\). C'est pourquoi ce chapitre va aller droit à la longueur avant de donner des exemples.

%+++++++++++++++++++++++++++++++++++++++++++++++++++++++++++++++++++++++++++++++++++++++++++++++++++++++++++++++++++++++++++
\section{Définitions}        \label{SecDeExCPar}
%+++++++++++++++++++++++++++++++++++++++++++++++++++++++++++++++++++++++++++++++++++++++++++++++++++++++++++++++++++++++++++

\begin{definition}
    Un \defe{arc paramétré}{arc!paramétré} dans $\eR^p$ est un couple $(I,\gamma)$ où $I$ est un intervalle de $\eR$ et $\gamma$ est une application continue de $I$ dans $\eR^p$. Nous disons que $(I,\gamma)$ est un arc paramétré \defe{compact}{compact!arc paramétré} (ou un \defe{chemin}{chemin!dans $\eR^p$} dans $\eR^p$) lorsque $I$ est compact dans $\eR$.
\end{definition}
L'intervalle $I$ d'un arc paramétré compact est toujours de la forme $[a,b]$, étant donné que tous les intervalles compacts de $\eR$ sont de cette forme. Un \defe{sous arc}{sous arc} de $(I,\gamma)$ est un arc de la forme $(I_0,\gamma)$ avec $I_0\subset I$.

\begin{definition}
    Un \defe{chemin}{chemin} dans $\eR$ est une application continue
    \begin{equation}
        \begin{aligned}
            \sigma\colon [a,b]&\to \eR^3 \\
            t&\mapsto \sigma(t).
        \end{aligned}
    \end{equation}
\end{definition}

La fonction $\sigma'(t)$ est la \defe{vitesse}{vitesse d'un chemin} du chemin $\sigma$. Si la fonction $t\mapsto\sigma(t)$ est dérivable, on dit que $\sigma''(t)$ est l'\defe{accélération}{accélération d'un chemin}. Les points $\sigma(a)$ et $\sigma(b)$ sont les extrémités du chemin. L'ensemble
\begin{equation}
    \{ \sigma(t)\tq t\in\mathopen[ a , b \mathclose] \}
\end{equation}
est la \defe{courbe}{courbe} $\sigma$.

%+++++++++++++++++++++++++++++++++++++++++++++++++++++++++++++++++++++++++++++++++++++++++++++++++++++++++++++++++++++++++++
\section{Longueur d'arc}        \label{SecLongArc}
%+++++++++++++++++++++++++++++++++++++++++++++++++++++++++++++++++++++++++++++++++++++++++++++++++++++++++++++++++++++++++++

Nous voulons définir et étudier la notion de \wikipedia{fr}{Arc_rectifiable}{longueur} d'un arc paramétré. Pour cela, le plus raisonnable est d'approcher l'arc par des petits segments de droites (dont les longueurs sont évidentes), et d'extraire la «meilleure» approximation.

Une des notions clefs pour la suite est celle de subdivision d'intervalles. Cette notion sera encore utilisée par la suite à propos des intégrales.
\begin{definition}      \label{DefSubdivisionIntervalle}
    Si $I$ est un intervalle d'extrêmes $a$ et $b$ avec $a<b$, nous appelons \defe{subdivision finie}{subdivision!d'un intervalle} de $I$ un choix de nombres $t_i$ tels que
    \begin{equation}
        a=t_0<t_1<\ldots<t_n=b.
    \end{equation}
    Nous disons qu'une subdivision $\sigma'$ est \defe{plus fine}{fine!subdivision} que la subdivision $\sigma$ si l'ensemble des points de $\sigma$ est inclus dans celui des points de $\sigma'$. Dans ce cas, la subdivision $\sigma'$ est un \defe{raffinement}{raffinement} de $\sigma$. Nous désignons par $\sdS(I)$ l'ensemble des subdivisions finies de l'intervalle $I$.
\end{definition}
Dans la suite, toutes les subdivisions que nous considérons seront des subdivisions finies. Aussi nous parlerons simplement de \emph{subdivisions} sans préciser. Nous allons souvent noter $\sigma=(t_i)_{i=1}^n$ pour désigner la subdivision formée par les nombres $t_i$. Il faut garder en tête que dans une subdivision, les nombres \emph{sont ordonnés}.

% TODOooHLDQooGIwacO Dans cette figure, les segments rouges ne s'affichent pas tous.
\newcommand{\CaptionFigCourbeRectifiable}{La longueur d'un découpage. La somme des longueurs des segments droits est facile à calculer.}
\input{auto/pictures_tex/Fig_CourbeRectifiable.pstricks}
\begin{definition}      \label{DEFooDNZWooXmxhsU}
    Soit un arc paramétré compact $(I,\gamma)$ et une subdivision $\sigma=(t_i)_{i=1}^n$ de $I=[a,b]$. À partir de $\gamma$ et du découpage $\sigma$ nous définissons le nombre (voir figure~\ref{LabelFigCourbeRectifiable})
    \begin{equation}        \label{Eqlsigmagammasss}
        l_{\sigma}(\gamma)=\sum_{i=1}^n\big\| \gamma(t_i)-\gamma(t_{i-1}) \big\|.
    \end{equation}
    On appelle \defe{longueur}{longueur!d'un arc paramétré compact} de l'arc $\gamma$ le nombre
    \begin{equation}
        l(\gamma)=\sup_{\sigma}l_{\sigma}(\gamma)\in\mathopen[ 0 , \infty \mathclose].
    \end{equation}
    Nous disons que $\gamma$ est \defe{rectifiable}{rectifiable} lorsque $l(\gamma)<\infty$.
\end{definition}
Lorsque nous voulons spécifier sur quel intervalle nous considérons l'arc, nous noterons $l(I,\gamma)$ au lieu de $l(\gamma)$ pour être plus précis.

Par l'inégalité triangulaire, si $\sigma_1$ est plus fine que $\sigma$, nous avons
\begin{equation}
    l_{\sigma}(\gamma)\leq l_{\sigma_1}(\gamma),
\end{equation}
Comme cela peut être vu sur la figure~\ref{LabelFigArcLongueurFinesse}.
\newcommand{\CaptionFigArcLongueurFinesse}{Il est visible que la longueur donnée par l'approximation par des petits segments (verts) est plus longue et plus précise que celle donnée par les longs segments (rouge).}
\input{auto/pictures_tex/Fig_ArcLongueurFinesse.pstricks}
%TODO : questa figura e' invisibile quando stampiamo il pdf.

\begin{proposition}     \label{PROPooCXLYooRpKDMs}
    Si \( P\) et \( Q\) sont des points de \( \eR^2\), alors le segment de droite joignant \( P\) à \( Q\) est le plus court des arcs paramétrés passant par \( P\) et \( Q\).
\end{proposition}

\begin{proof}
    Si \( \gamma\) est un arc paramétré joignant \( P\) et \( Q\), la longueur de \( \gamma\) est donné par un supremum dont un des éléments est la longueur du segment de droite.
\end{proof}

Dans la vie réelle, il est souvent difficile et peu pratique de calculer le supremum «à la main». C'est pourquoi nous allons travailler à exprimer la longueur d'un arc à l'aide d'une intégrale (théorème~\ref{ThoLongueurIntegrale}).

\begin{lemma}
    Nous avons $l(\gamma)=0$ si et seulement si $\gamma(t)$ est un vecteur constant.
\end{lemma}

\begin{proof}
    Si l'application $\gamma(t)$ est constante, le résultat est évident. Supposons maintenant que $\gamma$ ne soit pas constante. Cela signifie qu'il existe $t_1$ et $t_2$ dans $I$ tels que $\gamma(t_1)\neq \gamma(t_2)$. Dans ce cas, si nous prenons le découpage $\sigma=\{ a,t_1,t_2,b \}$, la somme \eqref{Eqlsigmagammasss} contient au moins le terme non nul $\| \gamma(t_2)-\gamma(t_1) \|$, et donc $l_{\sigma}(\gamma)>0$. Par définition du supremum, nous avons alors $l(\gamma)\geq l_{\sigma}(\gamma)>0$.
\end{proof}

\begin{proposition}     \label{Propletautredecop}
    Soit $(I,\gamma)$ un arc paramétré compact.
    \begin{enumerate}
        \item
            Si $\gamma'=(I',\gamma)$ avec $I'\subset I$, alors $l(\gamma')\leq l(\gamma)$.
        \item
            Soit $c\in\mathopen[ a , b \mathclose]$, et considérons les arcs $\gamma_1=\big( \mathopen[ a , c \mathclose],\gamma \big)$ et $\gamma_2=\big( \mathopen[ c , b \mathclose],\gamma \big)$. Alors
            \begin{equation}
                l(\gamma)=l(\gamma_1)+l(\gamma_2).
            \end{equation}
            En particulier, $\gamma$ est rectifiable si et seulement si $\gamma_1$ et $\gamma_2$ le sont.
    \end{enumerate}
\end{proposition}

\begin{proof}
    \begin{enumerate}
        \item
            Nous notons $I=\mathopen[ a , b \mathclose]$ et $I'=\mathopen[ a' , b' \mathclose]$. Étant donné que $I'\subset I$, nous avons
            \begin{equation}
                a\leq a'<b'\leq b.
            \end{equation}
            Pour chaque subdivision $\sigma_0:a'=t_0<t_1<\ldots<t_n=b'$ de $I'$, nous pouvons construire une subdivision de $I$ en «ajoutant» les points $a$ et $b$, c'est-à-dire
            \begin{equation}
                \sigma:a\leq t_0<\ldots<t_n\leq b.
            \end{equation}
            Si nous calculons $l_{\sigma}(\gamma)$, nous avons tous les termes qui arrivent dans $l_{\sigma_0}(\gamma')$ plus le premier et dernier terme : $\| \gamma(t_0)-\gamma(a) \|$ et $\| \gamma(b)-\gamma(t_n)\|$. Nous avons donc
            \begin{equation}
                l_{\sigma_0}(\gamma')\leq l_{\sigma}(\gamma)\leq\sup_{\sigma}l_{\sigma}(\gamma)=l(\gamma).
            \end{equation}
            Étant donné que pour toute subdivision $\sigma_0$ nous avons $l_{\sigma_0}(\gamma')\leq l(\gamma)$, en prenant le supremum sur les subdivisions $\sigma_0$ de $I'$, nous avons comme annoncé
            \begin{equation}
                l(\gamma')\leq l(\gamma).
            \end{equation}
        \item
            Soit $\sigma=\{ t_i \}$ une subdivision de $\mathopen[ a , b \mathclose]$. Nous considérons les subdivisions $\sigma_1$ et $\sigma_2$ définies comme suit:
            \begin{equation}
                \begin{aligned}[]
                    \sigma_1&:\{ t_i\tq t_i< c \}\cup\{ c \},\\
                    \sigma_2&:\{ t_i\tq t_i> c \}\cup\{ c \}.
                \end{aligned}
            \end{equation}
            L'inégalité triangulaire implique que
            \begin{equation}
                l_{\sigma}(\gamma)\leq l_{\sigma\cup\{ c \}}(\gamma)=l_{\sigma_1}(\gamma_1)+l_{\sigma_2}(\gamma_2)\leq l(\gamma_1)+l(\gamma_2).
            \end{equation}
            Nous avons donc
            \begin{equation}    \label{EqIneglglglgud}
                l(\gamma)\leq l(\gamma_1)+l(\gamma_2).
            \end{equation}

            Nous prouvons maintenant l'inégalité inverse. Soit $\varepsilon>0$. Étant donné que $l(\gamma_1)$ est le supremum des quantités $l_{\sigma_1}(\gamma_1)$ lorsque $\sigma_1$ parcours toutes les subdivisions possibles, il existe une partition $\sigma_1^{\varepsilon}$ telle que (idem pour $\gamma_2$)
            \begin{equation}        \label{EqAllsigmaepsgammaufd}
                \begin{aligned}[]
                    l_{\sigma_1^{\varepsilon}}(\gamma_1)+\frac{ \varepsilon }{2}>l(\gamma_1),\\
                    l_{\sigma_2^{\varepsilon}}(\gamma_2)+\frac{ \varepsilon }{2}>l(\gamma_2),
                \end{aligned}
            \end{equation}
            où $\sigma_1^{\varepsilon}$ est une subdivision de $\mathopen[ a , c \mathclose]$ et $\sigma_2^{\varepsilon}$ en est une de $\mathopen[ c , b \mathclose]$. En faisant la somme des deux équations \eqref{EqAllsigmaepsgammaufd}, nous trouvons
            \begin{equation}
                l(\gamma_1)+l(\gamma_2)<l_{\sigma_1^{\varepsilon}}(\gamma_1)+l_{\sigma_2^{\varepsilon}}(\gamma_2)+\varepsilon=l_{\sigma_1^{\varepsilon}\cup\sigma_2^{\varepsilon}}(\gamma)\leq l(\gamma)+\varepsilon.
            \end{equation}
            L'inégalité $l(\gamma_1)+l(\gamma_2)<l(\gamma)+\varepsilon$ étant valable pour tout $\varepsilon$, nous avons
            \begin{equation}
                l(\gamma_1)+l(\gamma_2)\leq l(\gamma).
            \end{equation}
            Cette inégalité, combinée avec l'inégalité \eqref{EqIneglglglgud}, donne bien $l(\gamma)=l(\gamma_1)+l(\gamma_2)$.
    \end{enumerate}
\end{proof}


%+++++++++++++++++++++++++++++++++++++++++++++++++++++++++++++++++++++++++++++++++++++++++++++++++++++++++++++++++++++++++++
\section{Abscisse curviligne}
%+++++++++++++++++++++++++++++++++++++++++++++++++++++++++++++++++++++++++++++++++++++++++++++++++++++++++++++++++++++++++++

\begin{definition}
    Soit $(I,\gamma)$ un arc rectifiable compact avec $I=\mathopen[ a , b \mathclose]$. L'application
    \begin{equation}
        \begin{aligned}
            \varphi\colon \mathopen[ a , b \mathclose]&\to \eR^+ \\
            t&\mapsto l\big( \mathopen[ a , t \mathclose],\gamma \big)
        \end{aligned}
    \end{equation}
    est la \defe{longueur d'arc}{longueur d'arc} de $\gamma$.
\end{definition}
Cette fonction nous permet de calculer la distance (suivant la courbe) entre deux points arbitraires parce que si $a\leq t<u\leq b$, nous avons
\begin{equation}
    l\big( [t,u],\gamma \big)=\varphi(u)-\varphi(t).
\end{equation}
En effet,
\begin{equation}
    \varphi(u)-\varphi(t)=l\big( [a,u],\gamma \big)-l\big( [a,t],\gamma \big),
\end{equation}
mais en utilisant la proposition~\ref{Propletautredecop}, nous avons
\begin{equation}
    l\big( [a,u],\gamma \big)=l\big( [a,t],\gamma \big)+l\big( [t,u],\gamma \big).
\end{equation}

\begin{proposition}
    La longueur d'arc d'un arc rectifiable compact est une fonction continue et croissante.
\end{proposition}

\begin{proof}
    Soit $(I,\gamma)$ un arc paramétré rectifiable compact avec $I=[a,b]$. Afin de montrer que $\varphi$ est croissante, prenons $t\in I$ ainsi que $h>0$ et montrons que $\varphi(t+h)\geq \varphi(t)$. La proposition~\ref{Propletautredecop} implique que
    \begin{equation}
        l\big( \mathopen[ a , t+h \mathclose],\gamma \big)=l\big( \mathopen[ a ,t  \mathclose],\gamma \big)+l\big( \mathopen[ t , t+h \mathclose],\gamma \big),
    \end{equation}
    c'est-à-dire
    \begin{equation}
        \varphi(t+h)=\varphi(t)+l\big( \mathopen[ t , t+h \mathclose],\gamma \big)\geq \varphi(t).
    \end{equation}

    Pour la continuité, soit $t$ fixé dans $\mathopen[ a , b \mathclose]$ et $\varepsilon>0$. Il nous faut démontrer qu'il existe $\eta>0$ tel que si $s$ est dans $[0,\eta]$ alors
\[
|\varphi(t+s)-\varphi(t)|\leq \varepsilon, \qquad \forall t \in [a,b].
\]
Étant donné que $l\big( \mathopen[ t , b \mathclose],\gamma \big)$ est le supremum des $l_{\sigma}\big( \mathopen[ t , b \mathclose],\gamma \big)$, il existe une subdivision $\sigma$ donnée par les points  $t,t_1,\cdots,t_{n-1},b$ telle que
    \begin{equation}
        l_{\sigma}\big( \mathopen[ t , b \mathclose],\gamma \big)>l\big( \mathopen[ t , b \mathclose],\gamma \big)-\frac{ \varepsilon }{2}=\varphi(b)-\varphi(t)-\frac{ \varepsilon }{2}.
    \end{equation}
    La continuité de $\gamma$ implique qu'il existe un $\eta$ tel que
    \begin{equation}
        s\in\mathopen[ 0 , \eta \mathclose]\Rightarrow\| \gamma(t+s)-\gamma(t) \|<\frac{ \varepsilon }{2}
    \end{equation}
    Quitte à prendre $\eta$ encore plus petit, nous supposons que $t+\eta<t_1$. Soit $s\in\mathopen[ 0 , \eta \mathclose]$ et considérons la subdivision de $\mathopen[ t , b \mathclose]$ donnée par $\sigma'=\sigma\cup\{ t+s \}$. Étant donné que $\sigma'$ est plus fine que $\sigma$, le nombre $l_{\sigma}\big( \mathopen[ t , b \mathclose],\gamma \big)$ est inférieur ou égal à $l_{\sigma'}\big( \mathopen[ t , b \mathclose],\gamma \big)$. Nous avons donc les inégalités
    \begin{equation}
        \begin{aligned}[]
            \varphi(b)-\varphi(t)-\frac{ \varepsilon }{2}&\leq l_{\sigma}\big( \mathopen[ t , b \mathclose],\gamma \big)\\
            &\leq l_{\sigma'}\big( \mathopen[ t , b \mathclose],\gamma \big)\\
            &= \big\| \gamma(t+s)-\gamma(t) \big\|+l_{\sigma'\setminus\{ t \}}\big( \mathopen[ t+s , b \mathclose]\gamma \big)\\
            &\leq\| \gamma(t+s)-\gamma(t) \|+\varphi(b)-\varphi(t+s)\\
            &\leq \frac{ \varepsilon }{2}+\varphi(b)-\varphi(t+s).
        \end{aligned}
    \end{equation}
    Au final, nous avons trouvé que
    \begin{equation}
        \varphi(t+s)-\varphi(t)\leq\varepsilon,
    \end{equation}
    ce qui prouve que $\varphi$ est continue au point $t$.
\end{proof}

En guise de paramètre sur un arc, nous pouvons utiliser la longueur d'arc elle-même. En effet si $(I,\gamma)$ est un arc de longueur $l$, nous pouvons donner le même arc avec le couple $\big( \mathopen[ 0 , l \mathclose],g \big)$ où $g$ est la fonction qui au réel $s$ fait correspondre l'élément $\gamma\big( \varphi^{-1}(s) \big)$ de $\eR^n$. Dire
\begin{equation}
    P=(\gamma\circ\varphi^{-1})(s)
\end{equation}
revient à dire que le point $P$ est le point sur la courbe sur lequel on tombe après avoir marché une distance $s$ sur la courbe.

Nous allons revenir sur ce «changement de paramètre» plus tard, en particulier dans la section~\ref{SecArcGeometrique}.

%---------------------------------------------------------------------------------------------------------------------------
\subsection{Formule intégrale de la longueur}
%---------------------------------------------------------------------------------------------------------------------------

Nous pouvons voir un chemin $\gamma$ comme étant la trajectoire d'une particule en fonction du temps. Sa vitesse à l'instant $t$ est le vecteur $\gamma'(t)$, tandis que sa vitesse \emph{scalaire} est le nombre $\| \gamma'(t) \|$. Une question naturelle est de savoir quelle est la longueur de la trajectoire parcourue entre $t=a$ et $t=b$.

Si nous prenons un petit intervalle de temps $dt$, nous pouvons supposer que le mobile avance à la vitesse constante $\| \gamma'(t) \|$. Cela ferait un trajet parcouru de longueur $\| \gamma'(t) \|dt$. Nous nous attendons donc à une formule de la forme suivante pour la longueur de \( \gamma\) :
\begin{equation}        \label{EqDefLongueurChemin}
    l(\gamma)=\int_a^b\| \gamma'(t) \|dt.
\end{equation}
Plus explicitement, si $\gamma(t)=\big( x(t),y(t),z(t) \big)$, alors nous aurions la formule
\begin{equation}
    l(\gamma)=\int_a^b\sqrt{x'(t)^2+y'(t)^2+z'(t)^2}dt.
\end{equation}


\begin{theorem}     \label{ThoLongueurIntegrale}
    Soit $(I,\gamma)$ un arc paramétré compact de classe $\mathcal{C}^1$. Alors $\gamma$ est rectifiable et
    \begin{equation}        \label{EqLongGammalInt}
        l(\gamma)=\int_a^b\| \gamma'(t) \|dt=\int_{\gamma}1,
    \end{equation}
    où $I=\mathopen[ a , b \mathclose]$.
\end{theorem}

\begin{proof}
    L'égalité avec l'intégrale le long de \( \gamma\) de la fonction \( 1\) est simplement la définition~\ref{DEFooFAYUooCaUdyo} de l'intégrale curviligne.

    Si $\sigma=\{ t_i \}$ est une subdivision de l'intervalle $\mathopen[ a , b \mathclose]$, alors
    \begin{equation}
        \begin{aligned}[]
            l_{\sigma}(\gamma)&=\sum_{i=1}^n\| \gamma(t_i)-\gamma(t_{i-1}) \|\\
                &=\sum_{i=1}^n\| \int_{t_{i-1}}^{t_i}\gamma'(t)dt \|\\
                &\leq\sum_{i=1}^n\int_{t_{i-1}}^{t_i}\| \gamma'(t) \|dt\\
                &=\int_a^b\| \gamma'(t) \|dt.
        \end{aligned}
    \end{equation}
    Cela prouve déjà que
    \begin{equation}        \label{Eq_0208lsigsigmmintifp}
        l(\gamma)=\sup_{\sigma}l_{\sigma}(\gamma)\leq\int_a^b\| \gamma'(t) \|dt.
    \end{equation}
    Nous devons maintenant prouver l'inégalité inverse.

    Notons $\varphi$ l'abscisse curviligne $\varphi(t)=l\big( \mathopen[ a , t \mathclose],\gamma \big)$. Cette dernière vérifie
    \begin{equation}
        \varphi(t+h)-\varphi(t)=l\big( \mathopen[ t , t+h \mathclose],\gamma \big)\geq \| \gamma(t+h)-\gamma(t) \|,
    \end{equation}
    et en particulier
    \begin{equation}     \label{Eq_0208intervpvpintfrach}
        \left\| \frac{ \gamma(t+h)-\gamma(t) }{ h } \right\|\leq \frac{ \varphi(t+h)-\varphi(t) }{ h }.
    \end{equation}
    D'autre part, en utilisant \eqref{Eq_0208lsigsigmmintifp} sur le segment $\mathopen[ t , t+h \mathclose]$, nous avons
    \begin{equation}
        \varphi(t+h)-\varphi(t)=l\big( \mathopen[ t , t+h \mathclose],\gamma \big)\leq\int_{t}^{t+h}\| \gamma'(u) \|du.
    \end{equation}
    Cela nous permet de continuer l'inéquation \eqref{Eq_0208intervpvpintfrach} en
    \begin{equation}
        \left\| \frac{ \gamma(t+h)-\gamma(t) }{ h } \right\|\leq\frac{ \varphi(t+h)-\varphi(t) }{ h }\leq\frac{1}{ h }\int_t^{t+h}\| \gamma'(u) \|du.
    \end{equation}
    Prenons la limite $h\to 0$. À gauche nous reconnaissons la formule de la dérivée, et nous obtenons $\| \gamma'(t) \|$; au centre nous avons $\varphi'(t)$ et à droite, si $n(u)$ représente une primitive de la fonction $u\mapsto\| \gamma'(u) \|$,
    \begin{equation}
        \lim_{h\to 0}\frac{ n(t+h)-n(t) }{ h }=n'(t)=\| \gamma'(t) \|.
    \end{equation}
    Au final,
    \begin{equation}
        \| \gamma'(t) \|\leq \varphi'(t)\leq\| \gamma'(t) \|,
    \end{equation}
    c'est-à-dire $\varphi'(t)=\| \gamma'(t) \|$ et donc par le théorème fondamental du calcul intégral~\ref{ThoRWXooTqHGbC},
    \begin{equation}
        \varphi(t)-\varphi(a)=\int_a^t\| \gamma'(u) \|du.
    \end{equation}
    Par construction de la longueur d'arc, $\varphi(a)=0$ et en posant $t=b$ nous obtenons la relation recherchée:
    \begin{equation}
        l(\gamma)=\varphi(b)=\int_a^b\| \gamma'(u) \|du.
    \end{equation}
\end{proof}

\begin{remark}  \label{RemLongIntUn}
    Cela est cohérent avec~\ref{NORMooDSNXooFhyHkx}, mais il faut garder en tête que \( l(\gamma)\) n'est pas le mesure de Lebesgue de l'image de \( \gamma\) dans \( \eR^2\). Cette dernière est nulle.
\end{remark}

\begin{example}
Soient donc $a$ et $b$ deux points de $\eR^m$, et $\gamma$ la droite joignant $a$ à $b$, c'est-à-dire
\begin{equation}
    \gamma(t)=(1-t)a+tb
\end{equation}
avec $t\in\mathopen[ 0 , 1 \mathclose]$. Le théorème~\ref{ThoLongueurIntegrale} nous enseigne que la longueur de ce chemin est
\begin{equation}
    l\big( [0,1],\gamma \big)=\int_0^1\| \gamma'(t) \|dt=\int_0^1\| -a+b \|=\| b-a \|,
\end{equation}
qui est bien la distance entre $a$ et $b$.
\end{example}

\begin{example}[Circonférence du cercle]
    Nous savons que l'image de
    \begin{equation}
        \begin{aligned}
            \gamma\colon \mathopen[ 0 , 2\pi \mathclose[&\to \eR^2 \\
            t&\mapsto \big( R\cos(t),R\sin(t) \big)
        \end{aligned}
    \end{equation}
    est le cercle de centre \( (0,0)\) et de rayon \( R>0\). Et de plus cet arc est de classe \( C^1\) (et même \(  C^{\infty}\)) par la proposition~\ref{PROPooZXPVooBjONka}. La longueur sera, d'après la formule \eqref{ThoLongueurIntegrale}
    \begin{equation}
        l_{\gamma}=\int_0^{2\pi}\| \gamma'(t) \|dt=2\pi R
    \end{equation}
    grâce à la formule \( \sin^2+\cos^2=1\) du lemme~\ref{LEMooAEFPooGSgOkF}.

    Mais tout cela n'est pas satisfaisant parce que nous n'avons pas encore de valeur numérique de \( \pi\).

    Il y a une autre façon de faire en considérant le quart de cercle dont la longueur en fonction de \( \pi\) est vite calculée par
    \begin{equation}
        l_{\gamma}=\int_0^{pi/2}\| \gamma'(t) \|dt=\frac{ \pi R }{ 2 }.
    \end{equation}

    Cette même longueur est calculée en termes de fonctions plus courantes avec le chemin
    \begin{equation}
        \begin{aligned}
        \sigma\colon \mathopen] 0 , 1 \mathclose[&\to \eR^2 \\
            t&\mapsto \begin{pmatrix}
                t    \\
                \sqrt{ R^2-t^2 }
            \end{pmatrix}
        \end{aligned}
    \end{equation}
    La longueur s'exprime avec
    \begin{equation}        \label{EQooIIKSooRQMgWY}
        l_{\sigma}=\int_0^1\sqrt{ \frac{ R^2 }{ R^2+t^2 } }dt.
    \end{equation}
    Notons que le changement de variables \( t=R\sin(u)\) permet de retrouver l'expression \( l_{\sigma}=\pi R/2\).

    Pour avoir une approximation de \( \pi\), il est loisible de calculer une approximation numérique de l'intégrale \eqref{EQooIIKSooRQMgWY} (avec \( R=1\)) et de l'égaler à \( \pi/2\).
\end{example}

\begin{example}
    Considérons l'arc de cercle de rayon $R$ interceptée par l'angle $\theta$ présenté sur la figure~\ref{LabelFigAMDUooZZUOqa}. % From file AMDUooZZUOqa
\newcommand{\CaptionFigAMDUooZZUOqa}{Quelle est la longueur de la partie bleue de ce cercle de rayon $R$ ?}
\input{auto/pictures_tex/Fig_AMDUooZZUOqa.pstricks}

    Par définition, cette longueur sera
    \begin{equation}
        \int_{\theta_0}^{\theta_1}\sqrt{R^2\sin^2(t)+R^2\cos^2(t)}dt=R(\theta_1-\theta_0).
    \end{equation}
    Le radian comme unité de mesure d'angle est donc l'unité parfaite : elle est la longueur d'arc interceptée (si le rayon est $R=1$).
\end{example}

Une conséquence à peine indirecte de ce que nous venons de voir à propos de longueur d'arc de cercle est la proposition suivante\quext{À mon avis il y a moyen de prouver ça avec un développement limité, mais je ne sais pas trop comment majorer l'erreur sans accepter que \( x\) soit arbitrairement proche de \( y\). Si vous savez comment faire, écrivez-moi.}.
\begin{proposition}     \label{PROPooYMMKooSUBtoo}
    Pour tout \( x,y\in \eR\), nous avons
    \begin{equation}
        |  e^{ix}- e^{iy} |\leq | x-y |.
    \end{equation}
\end{proposition}

\begin{proof}
    Évacuons tout de suite la différence entre \( \eR^2\) et \( \eC\) : ils sont isométriques. Si vous n'êtes pas convaincu que tout se passe bien, vous pouvez récrire toute la démonstration en écrivant systématiquement \( \big( \cos(x),\sin(x) \big)\) au lieu de \(  e^{ix}\). Cela serait au passage un bon exercice pour voir que les formules de dérivation fonctionnent bien.
    
    Nous considérons les points \(  e^{ix}\) et \(  e^{iy}\) dans \( \eC\) et deux chemins différents les joignant. Le premier est le segment de droite
    \begin{equation}
        \begin{aligned}
            \sigma_1\colon \mathopen[ 0 , 1 \mathclose]&\to \eC \\
            t&\mapsto t e^{ix}+(1-t) e^{iy}. 
        \end{aligned}
    \end{equation}
    Le second est l'arc de cercle
    \begin{equation}
        \begin{aligned}
            \sigma_1\colon \mathopen[ 0 , 1 \mathclose]&\to \eC \\
            t&\mapsto  e^{i\big( tx+(1-t)y \big)}. 
        \end{aligned}
    \end{equation}
    Nous avons \( \sigma_1'(t)= e^{ix}- e^{iy}\) qui ne dépend pas de \( t\), et donc la longueur est facile à calculer à partir de la formule intégrale du théorème \ref{ThoLongueurIntegrale} :
    \begin{equation}
        l(\sigma_1)=\int_0^1| \sigma_1'(t) |=|  e^{ix}- e^{iy} |.
    \end{equation}
    En ce qui concerne le second chemin,
    \begin{equation}
        \sigma_2'(t)=(x-y) e^{i\big( tx+(1-t)y \big)}.
    \end{equation}
    Nous avons\footnote{Si vous voulez citer des résultats, lemme \ref{LEMooHOYZooKQTsXW} et proposition \ref{PROPooXLARooYSDCsF}.} \( | \sigma_2'(t) |=| x-y |\) qui ne dépend pas non plus de \( t\). Donc
    \begin{equation}
        l(\sigma_2)=| x-y |.
    \end{equation}
    Étant donné la proposition \ref{PROPooCXLYooRpKDMs} qui dit que le chemin le plus court est le segment de droite,
    \begin{equation}
        l(\sigma_1)<l(\sigma_2) 
    \end{equation}
    et donc le résultat annoncé.
\end{proof}

\begin{normaltext}
    Si on veut savoir la longueur d'une courbe donnée sous la forme d'une fonction $y=y(x)$, un chemin qui trace la courbe est évidemment donné par
    \begin{equation}
        \gamma(t)=(t,y(t)),
    \end{equation}
    et le vecteur tangent au chemin est $\gamma'(t)=(1,y'(t))$. Donc
    \begin{equation}
        \| \gamma'(t) \|=\sqrt{1+y'(t)^2},
    \end{equation}
    et
    \begin{equation} \label{EqLongFonction}
        L=\int_a^b\sqrt{1+y'(t)^2}.
    \end{equation}
\end{normaltext}

\begin{example}
    La longueur de l'hélice
    \begin{equation}
        \sigma(t)=\begin{pmatrix}
            \cos(2t)    \\
            \sin(2t)    \\
            \sqrt{5}t
        \end{pmatrix}
    \end{equation}
    pour $t\in\mathopen[ 0 , 2\pi \mathclose]$ est donnée par
    \begin{equation}
        l(\sigma)=\int_0^{4\pi}\sqrt{4\sin^2(2t)+4\cos^2(2t)+5}dt=\int_0^{4\pi}\sqrt{9}=12\pi.
    \end{equation}
\end{example}

\begin{definition}
    Soit $\sigma_1\colon \mathopen[ a , b \mathclose]\to \eR^3$, un chemin et $\sigma_2\colon \mathopen[ c , d \mathclose]\to \eR^3$, un autre chemin. On dit que ces chemins sont \defe{équivalents}{equivalence@équivalence!chemin} s'il existe une fonction $\varphi\colon \mathopen[ a , b \mathclose]\to \mathopen[ c , d \mathclose]$ strictement croissante telle que $\sigma_1(t)=\sigma_2\big( \varphi(t) \big)$.
\end{definition}

Deux chemins équivalents parcourent la même courbe dans le même sens. Ils ne le parcourent toutefois pas à la même vitesse. On dit que les chemins sont \defe{opposée}{opposés!chemins} si la fonction $\varphi$ de la définition est strictement décroissante. Dans ce cas, ils ont la même image, mais parcourue dans le sens opposés. Nous disons que deux chemins équivalents sont un \defe{changement de paramétrage}{paramétrage} pour la même courbe.

 Dans le cas d'un paramétrage équivalente, nous avons $\varphi(a)=c$ et $\varphi(b)=d$. Les points de départ et d'arrivée des deux paramètres coïncident. Dans le cas d'un paramètre qui va dans le sens opposé par contre nous avons automatiquement $\varphi(a)=d$ et $\varphi(b)=c$.

\begin{proposition}
    La longueur d'une courbe ne dépend pas du paramètre (équivalent ou opposé) choisi.
\end{proposition}

\begin{proof}
    Soient $\sigma_1\colon \mathopen[ a , b \mathclose]\to \eR^3$ et $\sigma_2\colon \mathopen[ c , d \mathclose]\to \eR^3$ tels que
    \begin{equation}     \label{EqChmsigmaundeuxvp}
        \sigma_1(t)=\sigma_2\big( \varphi(t) \big)
    \end{equation}
    où $\varphi\colon \mathopen[ a , b \mathclose]\to \mathopen[ a , d \mathclose]$ est une bijection strictement monotone. Par définition on a
    \begin{equation}
        l(\sigma_1)=\int_a^b\| \sigma_1'(t) \|dt.
    \end{equation}
    Nous pouvons exprimer la dérivée de $\sigma_1$ en termes de celle de $\sigma_2$ en dérivant la relation \eqref{EqChmsigmaundeuxvp} :
    \begin{equation}
        \sigma_1'(t)=\varphi'(t)\sigma_2'\big( \varphi(t) \big).
    \end{equation}
    En ce qui concerne la norme,
    \begin{equation}
        \| \sigma_1'(t) \|=| \varphi'(t) |\| \sigma_2'(t) \|.
    \end{equation}
    Notez dans cette relation que $\varphi'(t)$ est un nombre (et non un vecteur). Étant donné que nous avons supposé que $\varphi$ était monotone, soit elle est monotone croissante et $\| \varphi'(t) \|=\varphi'(t)$ pour tout $t$, soit elle est monotone décroissante et $\| \varphi'(t) \|='\varphi(t)$ pour tout $t$.

    Considérons d'abord le premier cas, c'est-à-dire $\| \varphi'(t) \|=\varphi'(t)$. Nous posons $s=\varphi(t)$, $ds=\varphi'(t)dt$. En remplaçant cela dans la formule de la longueur est
    \begin{equation}
        \begin{aligned}[]
            l(\sigma_1)&=\int_a^b\varphi'(t)\| \sigma_2\big( \varphi(t) \big) \|dt\\
            &=\int_{\varphi(a)}^{\varphi(b)}\| \sigma_2'(s) \|ds\\
            &=\int_c^d\| \sigma_2'(s) \|ds\\
            &=l(\sigma_2).
        \end{aligned}
    \end{equation}

    Si nous considérons maintenant un paramétrage strictement décroissante. Dans ce cas, $\varphi'(t)\leq 0$ et $\| \varphi'(t) \|=-\varphi'(t)$. Nous posons encore une fois $s=\varphi(t)$, $ds=\varphi'(t)ds$. Ici il ne faut pas oublier que $\varphi(a)=d$ et $\varphi(b)=c$. Le calcul est à part cela le même en faisant attention au singe :
    \begin{equation}
        \begin{aligned}[]
            l(\sigma_1)&=\int_a^b\varphi'(t)\| \sigma_2\big( \varphi(t) \big) \|dt\\
            &=-\int_{\varphi(a)}^{\varphi(b)}\| \sigma_2'(s) \|ds\\
            &=-\int_d^c\| \sigma_2'(s) \|ds\\
            &=\int_c^d\| \sigma_2'(s) \|ds\\
            &=l(\sigma_2).
        \end{aligned}
    \end{equation}
    Nous avons changé le signe en changeant l'ordre des bornes.
\end{proof}

%+++++++++++++++++++++++++++++++++++++++++++++++++++++++++++++++++++++++++++++++++++++++++++++++++++++++++++++++++++++++++++
\section{Suite du chapitre}
%+++++++++++++++++++++++++++++++++++++++++++++++++++++++++++++++++++++++++++++++++++++++++++++++++++++++++++++++++++++++++++

Le grand avantage des arcs paramétrés par rapports aux graphes de fonctions est le le graphe peut «faire des retours en arrière», ou bien des auto intersections. Outre les deux exemples typiques de la la figure~\ref{LabelFigExempleArcParam}, un exemple classique est la droite verticale. Les fonctions $y=ax+b$ permettent de décrire toutes les droites, sauf les droites verticales. Dans le cadre des courbes paramétrées, les droites verticales et horizontales sont sur pied d'égalité. Quelques exemples classiques :
\begin{description}
    \item[Droite horizontale] Une droite horizontale à la hauteur $a$ est donnée par la courbe paramétrée $\gamma(t)=(t,a)$, avec $t\in I=\eR$.
    \item[Droite verticale] Une droite verticale à la distance $b$ de l'origine est donnée par la courbe paramétrée $\gamma(t)=(b,t)$, avec $t\in I=\eR$.
    \item[Graphe d'une fonction]\label{PgGrqFnGamma} Le graphe d'une fonction $f\colon \eR\to \eR$ est donné par l'arc $\gamma(t)=\big( t,f(t) \big)$.
    \item[Un cercle] Le cercle de rayon $R$ est donné par l'arc $\gamma(t)=\big( R\cos(t),R\sin(t) \big)$.
\end{description}

\newcommand{\CaptionFigExempleArcParam}{Des exemples d'arcs paramétrées. Ceux ne sont pas des graphes.}
\input{auto/pictures_tex/Fig_ExempleArcParam.pstricks}

\begin{remark}
    Afin d'alléger la notation, nous allons le plus souvent désigner l'arc $(I,\gamma)$ simplement par la fonction $\gamma$. Il est cependant toujours \emph{très} important de savoir sur quel intervalle nous considérons le chemin. Cela dépendra le plus souvent du contexte, et nous indiquerons l'intervalle $I$ explicitement lorsqu'une ambigüité est à craindre.

    Par exemple, lorsque nous considérons le cercle $\gamma(t)=\big( R\cos(t),R\sin(t) \big)$, le plus souvent l'intervalle de variation de $t$ sera $I=\mathopen[ 0 , 2\pi \mathclose]$. Par contre, si nous considérons la droite $\gamma(t)=(t,2t)$, l'intervalle de variation de $t$ sera naturellement $I=\eR$.
\end{remark}

%+++++++++++++++++++++++++++++++++++++++++++++++++++++++++++++++++++++++++++++++++++++++++++++++++++++++++++++++++++++++++++
\section{Autres exemples}
%+++++++++++++++++++++++++++++++++++++++++++++++++++++++++++++++++++++++++++++++++++++++++++++++++++++++++++++++++++++++++++

\begin{example}
    Soit $v\in\eR^3$ et $x_0\in\eR^3$. Le chemin
    \begin{equation}
        \sigma(t)=x_0+tv
    \end{equation}
    est une droite. Sa vitesse est $\sigma'(t)=v$.
\end{example}

\begin{example}
    La courbe
    \begin{equation}
        \sigma(t)=\begin{pmatrix}
            \cos(t)    \\
            \sin(t)
        \end{pmatrix}\in\eR^2
    \end{equation}
    avec $t\in\mathopen[ 0 , 2\pi [$ est le cercle unité parcouru une fois dans le sens trigonométrique.

    Notez que si on prend $t\in\mathopen[ 0 , 4\pi [$, nous avons un \emph{autre} chemin; c'est le même cercle unité, mais parcouru \emph{deux} fois. Même si le «dessin» (le graphe) des deux est le même, le chemin n'est pas le même.

    Le chemin
    \begin{equation}
        \gamma(t)=\begin{pmatrix}
            \cos(2\pi-t)    \\
            \sin(2\pi-t)
        \end{pmatrix}
    \end{equation}
    est le cercle unité parcouru une fois dans le sens inverse. Encore une fois le «dessin» est le même, mais le chemin n'est pas le même.
\end{example}

\begin{example}
    Le chemin
    \begin{equation}
        \sigma(t)=\begin{pmatrix}
            t    \\
            t^2
        \end{pmatrix}
    \end{equation}
    est un chemin dont l'image est la parabole d'équation $y=x^2$.
\end{example}

L'importance de la dérivée du chemin réside en le fait qu'elle donne la tangente. En effet le vecteur $\sigma'(t)$ est tangent au graphe de $\sigma$ au point $\sigma(t)$.

\begin{corollary}       \label{CorKBEMooRvYAcJ}
    La tangente à un cercle est perpendiculaire au rayon.
\end{corollary}

\begin{proof}
    Nous savons que pour un cercle,
	\begin{equation}
		y'(x)=\frac{ -x }{ \sqrt{R^2-x^2} }.
	\end{equation}
	Un point général du cercle a pour abscisse $x=R\cos(\theta)$. En remplaçant nous trouvons le coefficient directeur suivant pour la tangente :
	\begin{equation}
		y'\big( R\cos(\theta) \big)=-\frac{1}{ \tan(\theta) }.
	\end{equation}
	Par conséquent une droite perpendiculaire à la tangente aurait comme coefficient directeur le nombre $\tan(\theta)$. Or cela est bien le coefficient directeur du rayon qui joint le point $(0,0)$ au point $\big( R\cos(\theta),R\sin(\theta) \big)$.

\end{proof}

\begin{example}
    Pour le cercle,
    \begin{equation}
        \sigma(t)=\begin{pmatrix}
            \cos(t)    \\
            \sin(t)
        \end{pmatrix},
    \end{equation}
    la dérivée est donnée par
    \begin{equation}
        \sigma'(t)=\begin{pmatrix}
            -\sin(t)    \\
            \cos(t).
        \end{pmatrix}
    \end{equation}
    Le produit scalaire $\sigma(t)\cdot \sigma'(t)$ est nul. Le vecteur $\sigma'(t)$ est donc bien tangent (corolaire~\ref{CorKBEMooRvYAcJ}).
\end{example}

\begin{example}
    Le courbe donnée par le chemin
    \begin{equation}
        \sigma(t)=\begin{pmatrix}
            \cos(t)    \\
            \sin(t)    \\
            t
        \end{pmatrix}
    \end{equation}
    est une hélice. Sa vitesse est
    \begin{equation}
        \sigma'(t)=\begin{pmatrix}
            -\sin(t)    \\
            \cos(t)    \\
            1
        \end{pmatrix}.
    \end{equation}
    Notez que pour tout $t\in\eR$, nous avons $\| \sigma'(t) \|=\sqrt{2}$.
\end{example}

\begin{remark}
    Lorsqu'on parle d'une courbe dans l'espace, l'intervalle sur lequel on considère la variation du paramètre est une donné fondamentale. Elle fait partie intégrante de la définition de la courbe.
\end{remark}


%+++++++++++++++++++++++++++++++++++++++++++++++++++++++++++++++++++++++++++++++++++++++++++++++++++++++++++++++++++++++++++
\section{Élément de longueur}
%+++++++++++++++++++++++++++++++++++++++++++++++++++++++++++++++++++++++++++++++++++++++++++++++++++++++++++++++++++++++++++

%---------------------------------------------------------------------------------------------------------------------------
\subsection{Élément de longueur : cartésiennes}
%---------------------------------------------------------------------------------------------------------------------------

Étant donné que la longueur d'arc d'une courbe paramétrée $(I,\gamma)$ est donnée par l'intégrale de $\| \gamma'(t) \|$, il est naturel d'appeler le nombre $\| \gamma'(t) \|\,dt$, \defe{l'élément de longueur}{longueur!élément de} de la courbe $\gamma$ au point $\gamma(t)$.

En coordonnées cartésiennes dans le plan, une courbe paramétrée est donnée par
\begin{equation}
    \gamma(t)=\big( x_1(t),x_2(t) \big),
\end{equation}
et l'élément de longueur est
\begin{equation}        \label{EqElLongCart}
    \| x'(t) \|\, dt =\sqrt{(x_1')^2+(x_2')^2} \, dt.
\end{equation}

%---------------------------------------------------------------------------------------------------------------------------
\subsection{Élément de longueur : polaires (1)}
%---------------------------------------------------------------------------------------------------------------------------

En coordonnées polaires, une courbe est donnée par
\begin{equation}
    \gamma(t)=\big( \rho(t),\theta(t) \big),
\end{equation}
et le passage aux cartésiennes se fait via les formules
\begin{subequations}
    \begin{numcases}{}
        x(t)=\rho(t)\cos\big( \theta(t) \big)\\
        y(t)=\rho(t)\sin\big( \theta(t) \big).
    \end{numcases}
\end{subequations}
L'élément de longueur se trouve directement en remplaçant $x(t)$ et $y(t)$ dans la formule \eqref{EqElLongCart}. Les dérivées sont données par
\begin{equation}
    \begin{aligned}[]
        x'(t)&=\rho'(t)\cos\theta(t)-\rho(t)\theta'(t)\sin\theta(t)\\
        y'(t)&=\rho'(t)\sin\theta(t)+\rho(t)\theta'(t)\cos\theta(t),
    \end{aligned}
\end{equation}
et un calcul montre que
\begin{equation}        \label{EqElLongEnPolaires}
    \big( x'(t) \big)^2+\big( y'(t) \big)^2=\big( \rho'(t) \big)^2+\big( \rho(t) \big)^2\big( \theta'(t) \big)^2.
\end{equation}

Nous reviendrons plus en détail sur le concept de changement de paramétrage (ici, les polaires) à la section~\ref{SecArcGeometrique}.

%---------------------------------------------------------------------------------------------------------------------------
\subsection{Élément de longueur : polaires (2)}
%---------------------------------------------------------------------------------------------------------------------------

Parfois on utilise $\theta$ comme paramètre. L'équation de la courbe est alors donnée en coordonnées polaires sous la forme
\begin{equation}        \label{Eqgenereformepolaire}
    \rho(\theta)=f(\theta),
\end{equation}
où $f$ est une fonction réelle et  il faut comprendre que nous parlons de la courbe $\big( \rho(\theta),\theta \big)$ en coordonnées polaires. En coordonnées cartésiennes, cette courbe est donnée par
\begin{subequations}        \label{EqPolaireSemiGen}
    \begin{numcases}{}
        x(t)=\rho(t)\cos(t)\\
        y(t)=\rho(t)\sin(t)
    \end{numcases}
\end{subequations}
avec $t$ qui parcours le plus souvent l'intervalle $\mathopen[ 0 , 2\pi \mathclose]$. Notez qu'il se peut que le domaine ne soit pas toujours $\mathopen[ 0 , 2\pi \mathclose]$; cela peut dépendre des circonstances. Quoi qu'il en soit, la donnée du domaine fait partie de la donnée d'une courbe, et il ne peut donc pas y avoir d'équivoques à ce niveau.

Nous utilisons à nouveau la formule \eqref{EqElLongCart} en y mettant les valeurs \eqref{EqPolaireSemiGen} :
\begin{subequations}
    \begin{numcases}{}
        x'(t)=\rho'(t)\cos(t)-\rho(t)\sin(t)\\
        y'(t)=\rho'(t)\sin(t)+\rho(t)\cos(t),
    \end{numcases}
\end{subequations}
et
\begin{equation}        \label{EqElemOngPOldeux}
    \big( x'(t) \big)^2+\big( y'(t) \big)^2=\rho'(t)^2+\rho(t)^2.
\end{equation}
% position 55702
%Si vous avez bien compris ce passage, vous pouvez jeter un œil à l'exercice~\ref{exoGeomAnal-0004}.

\begin{remark}
    N'oubliez pas, en utilisant ces formules, que ce qui rentre dans l'intégrale est la \emph{racine carré} de $(x')^2+(y')^2$.
\end{remark}

\begin{example}     \label{ExempleLongCercle}
    Calculons la circonférence du cercle. En coordonnées polaires, le graphe du cercle correspond à l'équation
    \begin{equation}
        \big( \rho(t),\theta(t) \big)=(R,t)
    \end{equation}
    où $R$ est constante (le rayon du cercle) et $t$ va de $0$ à $2\pi$. En substituant dans l'équation \eqref{EqElLongEnPolaires}, l'élément de longueur à intégrer est seulement
    \begin{equation}
        \sqrt{R^2}=R
    \end{equation}
    parce que $\rho'(t)=0$ et $\theta'(t)=1$. La longueur du cercle est alors directement donnée par
    \begin{equation}
        l=\int_0^{2\pi}Rdt=2\pi R.
    \end{equation}

    Nous pouvions aussi faire le calcul en coordonnées cartésiennes. Alors la courbe est donnée par les équations
    \begin{equation}
        \begin{aligned}[]
            x(t)&=R\cos(t)\\
            y(t)&=R\sin(t)
        \end{aligned}
    \end{equation}
    et $t\in\mathopen[ 0 , 2\pi \mathclose]$. La circonférence du cercle est alors
    \begin{equation}
        l=\int_0^{2\pi}\sqrt{R^2\sin^2(t)+R^2\cos^2(t)}\,dt=\int_0^{2\pi}R\,dt=2\pi R.
    \end{equation}
\end{example}
\begin{remark}
  Il faut bien comprendre que quand on parle de courbes paramétrées en  coordonnées cartésiennes on pense à une courbe dont le paramètre est, par exemple, $t$ et les équations de la courbe sont $(x(t), y(t))$. Cela ne veut pas dire que $x$ ou $y$ soit le paramètre. La cas où $x$ ou $y$ est le paramètre est un cas particulier qui est possible seulement pour certaines courbes et notamment pour les graphes. Le cercle de rayon $1$ n'est pas un graphe, donc si on veut utiliser $x$ ou $y$ comme paramètre il faut d'abord découper la courbe en deux morceaux, par exemple, la moitié inférieure ($y<0$) et la moitié supérieure ($y>0$).
\end{remark}
\begin{example}     \label{ExCycloLong}
    Une \defe{cycloïde}{cycloïde!longueur} est une courbe paramétrée par
    \begin{subequations}
        \begin{numcases}{}
            x(t)=a(t-\sin(t))\\
            y(t)=a(1-\cos(t))
        \end{numcases}
    \end{subequations}
    avec $a>0$ et $t\in\eR$. Comme montré sur la figure~\ref{LabelFigCycloideA}, la cycloïde donne lieu à un graphe périodique. Il est possible de montrer (le faire) que le premier arc correspond à $t\in\mathopen[ 0 , 2\pi \mathclose]$. Nous voulons donc calculer la longueur de l'arc sur cet intervalle.
    \newcommand{\CaptionFigCycloideA}{La cycloïde de paramètre $a=1$ entre $0$ et $4\pi$.}
    \input{auto/pictures_tex/Fig_CycloideA.pstricks}

    Nous avons $x'(t)=a(1-\cos(t))$ et $y'(t)=a\sin(t)$, de telle façon que
    \begin{equation}    \label{Eq_0508dlcycloide}
        \sqrt{(x')^2+(y')^2}=a\sqrt{2-2\cos(t)}=a\sqrt{4\sin^2\left( \frac{ t }{ 2 } \right)}=2a\Big| \sin\frac{ t }{2} \Big|.
    \end{equation}
    La longueur est donc donnée par
    \begin{equation}
        \int_0^{2\pi}2a| \sin\frac{ t }{2} | dt=4a\int_0^{\pi}\sin(t)dt=8a.
    \end{equation}

\end{example}

\begin{example}
    La \defe{cardioïde}{cardioïde} est la courbe donnée par
    \begin{equation}        \label{EqCardioide}
        \rho(\theta)=a(1+\cos(\theta)).
    \end{equation}
    avec $\theta\in\mathopen[ -\pi , \pi \mathclose]$. Le nom de cette courbe provient de son graphe illustré à la figure~\ref{LabelFigCardioid}.
    \newcommand{\CaptionFigCardioid}{Une cardioïde, $\rho=1+\cos(\theta)$.}
    \input{auto/pictures_tex/Fig_Cardioid.pstricks}

    L'équation \eqref{EqCardioide} est donnée sous la forme \eqref{Eqgenereformepolaire}, c'est-à-dire que $\theta(t)=t$ et $\theta'(t)=1$, et par conséquent l'élément de longueur est donné par
    \begin{equation}
        \begin{aligned}[]
            (\rho')^2+(\rho)^2&=\big( -a\sin(\theta) \big)^2+a^2\big( 1+\cos(\theta) \big)^2\\
                    &=a^2\sin^2(\theta)+a^2\big( 1+2\cos(\theta)+\cos^2(\theta) \big)\\
                    &=a^2\big( 1+1+2\cos(\theta) \big)\\
                    &=2a^2\big( 1+\cos(\theta) \big)\\
                    &=4a^2\cos^2\frac{ \theta }{2}.
        \end{aligned}
    \end{equation}
    La longueur d'arc est donc donnée par
    \begin{equation}
        l=\int_{-\pi}^{\pi}2a\cos\frac{ \theta }{2}d\theta=2a\int_{-\pi/2}^{\pi/2}\cos(t)2dt=8a.
    \end{equation}
\end{example}

%---------------------------------------------------------------------------------------------------------------------------
\subsection{Approximation de la longueur par des cordes}
%---------------------------------------------------------------------------------------------------------------------------

\begin{definition}
    Soit un arc paramétré $(I,\gamma)$. Un point $t\in I$ est dit \defe{régulier}{régulier!point d'un arc} si $\gamma'(t)\neq 0$, et il est dit \defe{critique}{critique!point d'un arc} si $\gamma'(t)=0$. Le point $t\in I$ est dit \defe{\href{http://c.caignaert.free.fr/chapitre15/node1.html}{birégulier}}{birégulier!point sur une courbe} si les vecteurs $\gamma'(t)$ et $\gamma''(t)$ sont linéairement indépendants et non nuls.

    Par extension, nous dirons également que le point $\gamma(t)$ lui-même est régulier, critique ou birégulier. Un arc est dit \emph{régulier}\index{régulier!arc} lorsque tous ses points sont réguliers.
\end{definition}
Note : dans le lemme~\ref{LEMooUECMooNBDGiR} et ses dépendances, nous utilisons effectivement que l'arc \( \gamma\) est de classe \( C^2\).

Nous savons que la longueur d'une courbe est donné par le supremum sur toutes les subdivisions de la longueur des cordes correspondantes. De plus l'inégalité triangulaire nous enseigne que plus la subdivision est fine, plus la longueur sera grande. Il est donc naturel de penser que sur un petit intervalle, la longueur de la courbe ne doit pas être très différente de la longueur de la corde correspondante.

La proposition suivante est un énoncé précis et quantitatif de ce fait.
\begin{proposition}
    Soit $(I,\gamma)$ un arc de classe $\mathcal{C}^1$ et $t_0\in I$ un point régulier (c'est-à-dire $\gamma'(t_0)\neq 0$). Alors pour tout $\varepsilon>0$, il y a un $\delta>0$ tel que on trouve  $t,t'\in I\cap(t_0,\delta)$ tels que
    \begin{equation}
        \left| \int_t^{t'}\| \gamma'(u) \|du-\| \gamma(t)-\gamma(t') \| \right| \leq 2\varepsilon| t-t' |.
    \end{equation}
\end{proposition}
Intuitivement, cette proposition signifie qu'au voisinage de $t_0$, la longueur d'arc est équivalente à celle de la corde.

\begin{proof}
    Par la continuité de $\gamma'$ (parce que $\gamma$ est $\mathcal{C}^1$), pour tout $\varepsilon$, il existe un $\delta$ tel que
    \begin{equation}
        | t-t_0 |<\delta\Rightarrow\big\| \gamma'(t)-\gamma'(t_0) \big\|\leq \varepsilon.
    \end{equation}
    Nous considérons la fonction
    \begin{equation}
        u\mapsto \gamma(u)-\gamma(t_0)-(u-t_0)\gamma'(t_0),
    \end{equation}
    dont la dérivée (par rapport à $u$) est
    \begin{equation}
        \gamma'(u)-\gamma'(t_0).
    \end{equation}
    Nous y appliquons la formule des accroissements finis entre $t$ et $t'$ choisis dans $I\cap\mathopen] t_0-\delta , t_0+\delta \mathclose[$. Il existe un $u$ entre $t$ et $t'$ tel que
    \begin{equation}
        \begin{aligned}[]
            \big\| \gamma(t)-\gamma(t_0)-(t-t_0)\gamma'(t_0)&-\gamma(t')+\gamma(t_0)+(t'-t_0)\gamma'(t_0) \big\|\\
                    &=| t-t' | \|\gamma'(u)-\gamma'(t_0) \|\\
                    &\leq\varepsilon| t-t' |.
        \end{aligned}
    \end{equation}
    En simplifiant ce qui peut être simplifié dans le membre de gauche, nous trouvons
    \begin{equation}
        \big\| \gamma(t)-\gamma(t')-(t-t')\gamma'(t_0) \big\|\leq\varepsilon| t-t' |.
    \end{equation}
    Le membre de gauche peut être minoré en utilisant la proposition~\ref{PropNmNNm} :
    \begin{equation}        \label{Eq0308ffttttftt}
        \Big| \| \gamma(t)-\gamma(t')\| -\|(t-t')\gamma'(t_0) \| \Big|\leq\varepsilon| t-t' |.
    \end{equation}
    D'autre part, les inégalités \eqref{EqNleqNNleqNvqlqbs} montrent que
    \begin{equation} \label{EqNleqNNleqNvqlqbsgamma}
        -\| \gamma'(u)-\gamma'(t_0) \|\leq \| \gamma'(u) \|-\| \gamma'(t_0) \|\leq\| \gamma'(u)-\gamma'(t_0) \|.
    \end{equation}
    Si de plus $u$ est compris entre $t$ et $t'$, ces inégalités sont encore coincées entre $-\varepsilon$ et $\varepsilon$. En intégrant \eqref{EqNleqNNleqNvqlqbsgamma} par rapport à $u$ entre $t$ et $t'$, nous obtenons
    \begin{equation}
        \left| \int_t^{t'}\big\| \gamma'(u) \big\|-(t-t')\big\| \gamma'(t_0) \big\| \right| \leq\varepsilon| t-t' |.
    \end{equation}
    Afin d'alléger les notations pour la ligne suivante, nous notons $A$ le nombre positif $\int_t^{t'}\| \gamma'(u) \|du$. Nous avons
    \begin{equation}        \label{Eq0308Afffgelleqinegs}
        \begin{aligned}[]
        \Big| A-\| \gamma(t)-\gamma(t')\| \Big| &=\Big| A-| t-t' |\,\| \gamma'(t_0) \|+| t-t' |\,\| \gamma'(t_0) \|-\| \gamma(t)-\gamma(t') \| \Big| \\
                &\leq\Big|  A-| t-t' |\,\| \gamma'(t_0) \|  \Big|+\Big| | t-t' |\,\| \gamma'(t_0) \|-\| \gamma(t)-\gamma(t') \|    \Big|.
        \end{aligned}
    \end{equation}
    L'équation \eqref{Eq0308ffttttftt} montre que le second terme est plus petit ou égal à $\varepsilon| t-t' |$. En ce qui concerne le premier terme, étant donné que $A$ est positif,
    \begin{equation}
        \Big|   A-| t-t' |\,\| \gamma'(t_0) \|   \Big| \leq\Big|  A-(t-t')\,\| \gamma'(t_0) \|  \Big|\leq \varepsilon| t-t' |.
    \end{equation}
    Au final, l'inéquation \eqref{Eq0308Afffgelleqinegs} donne
    \begin{equation}
            \Big| A-\| \gamma(t)-\gamma(t')\| \Big| \leq 2\varepsilon\,| t-t' |,
    \end{equation}
    ce qu'il fallait démontrer.
\end{proof}

%+++++++++++++++++++++++++++++++++++++++++++++++++++++++++++++++++++++++++++++++++++++++++++++++++++++++++++++++++++++++++++
\section{Arc géométrique}
%+++++++++++++++++++++++++++++++++++++++++++++++++++++++++++++++++++++++++++++++++++++++++++++++++++++++++++++++++++++++++++
\label{SecArcGeometrique}

\begin{definition}      \label{DefAcrEquiva}
Soient $(I,\gamma)$ et $(J,g)$ deux arcs paramétrés de classe $\mathcal{C}^k$. On dit qu'il sont \defe{équivalents}{equivalence@équivalence!arcs paramétrés} s'il existe une bijection $\theta\colon I\to J$ de classe $\mathcal{C}^k$, d'inverse de classe $\mathcal{C}^k$ telle que $g=\gamma\circ\theta$. Nous notons $\gamma\sim g$\nomenclature[C]{$\gamma\sim g$}{Équivalence d'arcs paramétrés} lorsque $\gamma$ et $g$ sont équivalents (les ensembles $I$ et $J$ sont sous-entendus).
\end{definition}

Le passage d'un paramétrage $(I,\gamma)$ à une autre $(J,g)$ se fait selon le diagramme suivant:
\begin{equation}
\xymatrix{%
I \ar[r]^{\gamma}   &   \eR^n\\
J \ar[ru]_{g}\ar[u]^{\theta}
   }
\end{equation}

\begin{proposition}
La relation donnée dans la définition~\ref{DefAcrEquiva} est une relation d'équivalence.
\end{proposition}

\begin{proof}
Les trois points d'une relation d'équivalence se vérifient en utilisant le fait que $\theta$ est inversible, et que l'inverse $\theta^{-1}$ jouit des mêmes propriétés de continuité ($\mathcal{C}^k$) que $\theta$.
\begin{description}
    \item[Réflexivité] Nous avons $\gamma\sim \gamma$ avec $\theta=\id$.
    \item[Symétrie] Si $\gamma\sim g$, alors nous avons une application $\theta$ telle que $g=\gamma\circ\theta$, et donc $\gamma=g\circ\theta^{-1}$, ce qui montre que $g\sim \gamma$.
    \item[Transitivité] Si $\gamma\sim g$ et $g\sim h$ avec $g=\gamma\circ\theta$ et $h=g\circ\omega$, alors $h=\gamma\circ(\theta\circ\omega)$, ce qui montre que $\gamma\sim h$.
\end{description}
\end{proof}
Si les arcs $(I,\gamma)$ et $(J,g)$ sont équivalents, les images dans $\eR^n$ sont identiques, et décrivent donc «le même dessin». Nous allons préciser cette notion plus loin.

\begin{definition}
    Pour cette raison les classes d'équivalences sont appelées des \defe{arcs géométriques}{arc!géométriques} (de classe $\mathcal{C}^k$).
\end{definition}

Si $\Gamma$ est une arc géométrique, ses représentants sont dits des \defe{paramétrages admissibles}{paramétrages!admissible} ou, plus simplement \emph{paramétrage}. On dit que l'application $\theta\colon J\to I$ est un \defe{changement de variable}{changement de variable}. Nous disons que un arc géométrique est \emph{compact} quand ses représentants sont compacts.

%Voir l'exercice~\ref{exoGeomAnal-0001} position 31124

\begin{lemma}       \label{LemChamVarsStriMomnot}
Dans le cas d'un arc $\mathcal{C}^1$, les changements de variables sont strictement monotones (croissants ou décroissants).
\end{lemma}

\begin{proof}
Nous considérons $(I,\gamma)$ et $(J,g)$, deux paramétrages différents du même arc géométrique, et $\theta\in \mathcal{C}^1(J,I)$ le changement de variable. Nous allons noter $t$ la variable sur $I$ et $s$ la variable sur $J$. Par définition, $\theta\big( \theta^{-1}(t) \big)=t$, et par conséquent,
\begin{equation}
    \theta'\big( \theta^{-1}(t) \big)(\theta^{-1})'(t)=1.
\end{equation}
En particulier $\theta'\big( \theta^{-1}(t) \big)$ ne s'annule pas pour aucune valeur de $t$. Mais $\theta^{-1}(t)$ peut prendre n'importe quelle valeur dans $J$, donc nous avons $\theta'(s)\neq 0$ pour tout $s\in J$. Cela signifie bien que $\theta$ est strictement monotone. En effet, $\theta'$ étant continue, elle ne peut pas changer de signe sans passer par zéro (théorème~\ref{ThoValInter} des valeurs intermédiaires).
\end{proof}

\begin{theorem}     \label{ThoLongArcGeom}
La longueur d'un arc est indépendante de son paramétrage, c'est-à-dire que les représentants d'un arc géométrique compact de classe $\mathcal{C}^1$ ont même longueur.
\end{theorem}

\begin{proof}
Nous utilisons les mêmes notations que celles du lemme~\ref{LemChamVarsStriMomnot}. Nous savons déjà que le changement de variable $\theta \colon J\to I$ est strictement monotone. Supposons que $\theta$ soit croissante.
%    (voir exercice~\ref{exoGeomAnal-0002}). Position 23657
En effectuant un changement de variable dans l'intégrale qui donne la longueur\footnote{Théorème~\ref{ThoLongueurIntegrale}.} nous avons
\begin{equation}
    \begin{aligned}[]
        l(\gamma)&=\int_I\| \gamma'(t) \|dt\\
            &=\int_J\| \gamma'\big( \theta(s) \big) \|\theta'(s)ds\\
            &=\int_J\| \gamma'\big( \theta(s) \big)\theta'(s) \|ds\\
            &=\int_J\| \frac{ d }{ ds }(\gamma\circ\theta)(s) \|ds\\
            &=\int_J\| g'(s) \|ds\\
            &=l(J,g).
    \end{aligned}
\end{equation}
\end{proof}

\begin{definition}
    Nous nommons \defe{longueur}{longueur!arc géométrique} d'un arc géométrique la longueur commune de tous ses représentants. On dit que l'arc géométrique est \defe{rectifiable}{rectifiable!arc géométrique} si sa longueur est $<\infty$.
\end{definition}

%---------------------------------------------------------------------------------------------------------------------------
\subsection{Abscisse curviligne et paramétrage normal}     \label{SubSecAbsCurv}
%---------------------------------------------------------------------------------------------------------------------------
\index{paramétrage!normale}

\begin{definition}
    Soit $(I,\gamma)$ un arc paramétré continu rectifiable. Nous appelons \defe{abscisse curviligne}{abscisse!curviligne} de $\gamma$ toute application $\phi\colon I\to \eR$ telle que pour tout $t,t'\in I$ avec $t<t'$, nous ayons
    \begin{equation}
        l\big( \mathopen[ t,t'  \mathclose],\gamma\big) = \big|  \phi(t')-\phi(t) \big|.
    \end{equation}
    Si il existe un $t_0\in I$ tel que $\phi(t_0)=0$, alors nous disons que $t_0$ est l'\defe{origine}{origine!abscisse curviligne} de l'abscisse $\phi$.
\end{definition}

\begin{definition}      \label{DEFooJJQFooEITCvG}
    Un arc paramétré $(I,\gamma_N)$ continu rectifiable est dit \defe{normal}{normal!arc paramétré} si l'identité est une abscisse curviligne.
\end{definition}

%
% Une abscisse curviligne est une fonction qui vérifie cette propriété. Les abscisse curvilignes sont notées \phi.
% Le nom de longueur d'arc est réservé à l'abscisse curviligne qui commence en 0. C'est celle définie plus haut.
% La longueur d'arc est notée \varphi.
%

\begin{lemma}       \label{LEMooLADUooBlHjuT}
    Si \( \gamma\) est de classe \( C^1\) et est un paramétrage normal, alors 
    \begin{enumerate}
        \item
    pour tout choix de $t$ et $t'$ dans $I$ avec $t<t'$, nous avons
    \begin{equation}
        l\big( \mathopen[ t , t' \mathclose],\gamma_N \big)=t'-t.
    \end{equation}
\item
    \( \| \gamma'(t) \|=1\) pour tout \( t\).
    \end{enumerate}
\end{lemma}

\begin{proof}
    Pour tout \( x_1,x_2\) dans le domaine nous avons
    \begin{equation}
        l\big( [x_1,x_2],\gamma \big)=\int_{x_1}^{x_2}\| \gamma'(t) \|dt=x_2-x_1.
    \end{equation}

    Cela implique \( \| \gamma'(t) \|=1\) pour tout \( t\). En effet, pour fixer les idées, supposons que \( \| \gamma'(t) \|>1\) en un point, par continuité, cela reste strictement supérieur à \( 1\) sur un intervalle. L'intégrale sur cet intervalle ne peut alors pas être la taille de l'intervalle.
\end{proof}

\begin{example}     \label{ExCerlceRadNorm}
Le cercle unitaire est donné par l'arc
\begin{equation}
    \gamma(t)=\big( \cos(t),\sin(t) \big)
\end{equation}
et $t\in\mathopen[ 0 , 2\pi \mathclose]$. Pour tout choix de $t$ et $t'$ dans $\mathopen[ 0 , 2\pi \mathclose]$, nous avons
\begin{equation}
    l\big( \mathopen[ t , t' \mathclose],\gamma \big)=\int_t^{t'}\sqrt{\sin^2(u)+\cos^2(u)}du=t'-t.
\end{equation}
Les angles exprimés en radians forment donc un paramétrage normal du cercle de rayon~$1$.
% position 28183
%Voir aussi les exercices~\ref{exoGeomAnal-0003} et~\ref{exoCourbesSurfaces0008}.
\end{example}

\begin{lemma}
Pour un arc paramétré compact, la longueur d'arc est une abscisse curviligne.
\end{lemma}

\begin{proof}
Par définition de la longueur d'arc $\varphi$, nous avons
\begin{equation}
    \varphi(t')-\varphi(t)=l\big( [a,t'],\gamma \big)-l\big( [a,t],\gamma \big)=\diamondsuit.
\end{equation}
Supposons pour fixer les idées que $t'>t$. En utilisant la proposition~\ref{Propletautredecop}, nous avons
\begin{equation}
    l\big( [a,t'],\gamma \big)=l\big( [a,t],\gamma \big)+l\big( [t,t'],\gamma \big),
\end{equation}
et donc après simplification de deux termes,
\begin{equation}
    \diamondsuit=l\big( [t,t'],\gamma \big),
\end{equation}
ce qui est précisément la propriété demandée pour être une abscisse curviligne.
\end{proof}

\begin{proposition}     \label{PropExisteChmNorm}
Pour tout arc paramétré $C^1$ sans points critiques, il existe un changement de coordonnées qui rend l'arc normal.
\end{proposition}

\begin{proof}
Soit $(I,\gamma)$ un arc de classe $\mathcal{C}^1$. Nous devons montrer qu'il existe un intervalle $J$ et une application $\theta\colon J\to I$ de classe $\mathcal{C}^1$ et d'inverse $\mathcal{C}^1$ tel que l'arc $(J,\gamma_N)$ soit $\mathcal{C}^1$ où $\gamma_N=\gamma\circ\theta$.

Si $I=\mathopen[ a ,b \mathclose]$, nous considérons la fonction
\begin{equation}        \label{EqDevVarPhi}
    \begin{aligned}
        \phi\colon I&\to \eR^+ \\
        t&\mapsto \int_a^t\| \gamma'(u) \|du.
    \end{aligned}
\end{equation}
Étant définie par l'intégrale d'une fonction $\mathcal{C}^0$, la fonction $\phi$ est $\mathcal{C}^1$, et nous avons $\phi'(t)=\| \gamma'(t) \|>0$ pour tout $t\in I$. Vue comme application $\phi\colon \mathopen[ a , b \mathclose]\to \mathopen[ 0 , l(\gamma) \mathclose]$, l'application $\phi$ est bijective et d'inverse $\mathcal{C}^1$. Voyons cela point par point.
\begin{enumerate}
    \item
        La fonction $\phi$ est injective parce que strictement croissante.
    \item
        Elle est surjective parce que $\phi(a)=0$ et $\phi(b)=l(\gamma)$.
    \item
        La continuité de l'inverse est plus délicate. Soit $l\in\mathopen[ 0 , l(\gamma) \mathclose]$ et $\varepsilon>0$. Pour prouver la continuité de $\phi^{-1}$ en $s$, nous devons trouver un $\delta$ tel que
        \begin{equation}
            | s-s' |<\delta\Rightarrow\big| \phi^{-1}(s)-\phi^{-1}(s') \big|<\varepsilon.
        \end{equation}
        Étant donné que $s$ et $s'$ sont dans l'image de $\phi$, nous considérons les uniques $t$ et $t'$ tels que $s=\phi(t)$ et $s'=\phi(t')$. La quantité $\phi(t)-\phi(t')$ devient
        \begin{equation}        \label{EqCondvpemuCont}
            \int_a^t\big\| \gamma'(u) \big\|du-\int_a^{t'}\big\| \gamma'(u) \big\|du=\int_{t}^{t'}\big\| \gamma'(u) \big\|du.
        \end{equation}
        D'autre part, $\phi^{-1}(s)=t$ et $\phi^{-1}(s')=t'$, donc la condition  \eqref{EqCondvpemuCont} devient
        \begin{equation}
            |   \int_{t'}^t\big\| \gamma'(u) \big\|du  |\leq\delta\Rightarrow | t-t' |<\varepsilon.
        \end{equation}
        Cela revient à la continuité des fonctions définies par une intégrale.
    \item
        La dérivée de son inverse est donnée par\footnote{Pour obtenir cette formule, dérivez les deux membres de l'équation $\phi\big( \phi^{-1}(s) \big)=s$.}
        \begin{equation}
            (\phi^{-1})'(s)=\frac{1}{\phi'\big( \phi^{-1}(s) \big)}.
        \end{equation}
        Nous avons vu que $\phi^{-1}$ et $\phi'$ étaient continues. La fonction $(\phi^{-1})'$ étant exprimée en termes de ces deux fonctions elle est également continue.
\end{enumerate}

Nous considérons l'arc paramétré $(J,\gamma_N)$ avec $J=\mathopen[ 0 , l(\gamma) \mathclose]$ et
\begin{equation}
    \gamma_N(s)=(\gamma\circ\phi^{-1})(s).
\end{equation}
Nous montrons maintenant que ce nouveau paramétrage est normal. Soient $0\leq s\leq s'\leq l(\gamma)$,
\begin{equation}
    \begin{aligned}[]
        l\big( \mathopen[ s , s' \mathclose],g \big)&=\int_s^{s'}\big\| \gamma_N'(u) \big\|du\\
        &=\int_{\phi^{-1}(s)}^{\phi^{-1}(s')}\big\| (\gamma_N'\circ\phi)(t) \big\|\phi'(t)dt\\
        &=\int_{\phi^{-1}(s)}^{\phi^{-1}(s')}\big\| (\gamma_N\circ\phi)'(t) \big\|dt\\
        &=\int_{\phi^{-1}(s)}^{\phi^{-1}(s')}\big\| \gamma'(t) \big\|dt\\
        &=\int_{0}^{\phi^{-1}(s')}\big\| \gamma'(t) \big\|\,dt -\int_0^{\phi^{-1}(s)}\big\| \gamma'(t) \big\|\,dt \\
        &=\phi\big( \phi^{-1}(s') \big)-\phi\big( \phi^{-1}(s) \big)\\
        &=s'-s,
    \end{aligned}
\end{equation}
ce qui prouve que le paramétrage $(J,\gamma_N)$ est normale.
\end{proof}

Nous retenons que le paramétrage normal de $\gamma$ est donnée par $(J,\gamma_N)$ avec $J=\mathopen[ 0 , l(\gamma) \mathclose]$ et
\begin{equation}        \label{EqFomVPcogammaN}
\gamma_N(s)=(\gamma\circ\phi^{-1})(s)
\end{equation}
où
\begin{equation}        \label{EqFomVPcoordnorm}
\begin{aligned}
    \phi\colon I&\to \eR^+ \\
    t&\mapsto \int_a^t\| \gamma'(u) \|du.
\end{aligned}
\end{equation}
Notons aussi que $\phi$ est une fonction croissante, étant l'intégrale d'une fonction positive.

\begin{example}
Trouvons les coordonnées normales pour la cycloïde\index{cycloïde!coordonnées normales} donnée par
\begin{subequations}
    \begin{numcases}{}
        x(t)=a(t-\sin(t)),\\
        y(t)=a(1-\cos(t))
    \end{numcases}
\end{subequations}
et $t\in\mathopen] 0 , 2\pi \mathclose[$. Relire l'exemple~\ref{ExCycloLong}.

D'abord nous trouvons $\phi$ avec la formule \eqref{EqFomVPcoordnorm} avec $a=0$. En utilisant le bout de calcul \eqref{Eq_0508dlcycloide}, nous avons
\begin{equation}
    \phi(t)=2a\int_0^t\sin\frac{ u }{2}du=4a\left( 1-\cos\frac{t}{2} \right).
\end{equation}
Pour trouver $\phi^{-1}(s)$, nous résolvons l'équation
\begin{equation}
    s=\phi\big( \phi^{-1}(s) \big)
\end{equation}
par rapport à $\phi^{-1}(s)$. Dans un premier temps, nous trouvons
\begin{equation}
    1-\frac{ s }{ 4a }=\cos\frac{ \phi^{-1}(s) }{ 2 },
\end{equation}
donc $\frac{ \phi^{-1}(s) }{2}=\arccos(\frac{ 4a-s }{ 4a })$, et finalement
\begin{equation}
    \phi^{-1}(s)=2\arccos\left(\frac{ 4a-s }{ 4a }\right).
\end{equation}
Il nous reste à injecter cela dans les expressions de $x(t)$ et $y(t)$ pour trouver $(\gamma_N)_x(s)$ et $(\gamma_N)_y(s)$. D'abord,
\begin{equation}
    (\gamma_N)_x(s)=a\big[ \phi^{-1}(s)-\sin\big( \phi^{-1}(s) \big) \big].
\end{equation}
Nous utilisons maintenant la formule trigonométrique $\sin(x)=2\sin\frac{ x }{ 2 }\cos\frac{ x }{2}$ afin de simplifier les expressions :
\begin{equation}
    \begin{aligned}[]
        (\gamma_N)_x&=a\Big[ 2\arccos\left( \frac{ 4a-s }{ 4a } \right)-2\sin\big( \arccos\left( \frac{ 4a-s }{ 4a } \right) \big)\cos\big( \arccos\left( \frac{ 4a-s }{ 4a } \right) \big) \Big]\\
        &=a\Big[ 2\arccos\left( \frac{ 4a-s }{ 4a } \right)-\frac{ 4a-s }{ 2a } \sqrt{1-\left( \frac{ 4a-s }{ 4a } \right)^2}\Big]\\
        &=2a\arccos\left( \frac{ 4a-s }{ 4a } \right)-\sqrt{8as-s^2}\,\frac{ 4a-s }{ 8a }
    \end{aligned}
\end{equation}
où nous avons utilisé la formule $\sin\big( \arccos(x) \big)=\sqrt{1-x^2}$. Ensuite, pour obtenir $(\gamma_N)_y$ nous devons calculer
\begin{equation}
    (\gamma_N)_y(s)=a\big[ 1-\cos\big( \phi^{-1}(s) \big) \big].
\end{equation}
Encore une fois, il est intéressant d'exprimer le cosinus en termes des angles divisés par deux : $\cos(x)=\cos^2\frac{ x }{2}-\sin^2\frac{ x }{2}$.
\begin{equation}
    \begin{aligned}[]
        (\gamma_N)_y&=a\Big[ 1-\cos^2\frac{ \phi^{-1}(s) }{2}+\sin^2\frac{ \phi^{-1}(s) }{2} \Big]\\
        &=a\Big[ 2-2\cos^2\frac{ \phi^{-1}(s) }{2} \Big]\\
        &=2a\Big[ 1-\left( \frac{ 4a-s }{ 4a } \right)^2 \Big].
    \end{aligned}
\end{equation}
Dans ce paramétrage, $s\in\mathopen] 0 , 8a \mathclose[$.
\end{example}

\begin{example}
La cardioïde $\rho(\theta)=a\big(1+\cos(\theta)\big)$ avec $\theta$ entre $-\pi$ et $\pi$. Avant d'utiliser la formule \eqref{EqFomVPcoordnorm}, nous devons trouver l'élément de longueur de la cardioïde. Étant donné la façon dont l'équation de la cardioïde nous est donnée, l'élément de longueur est donné par\footnote{Nous vous déconseillons d'étudier cette formule par cœur. Sachez cependant la retrouver assez vite.} \eqref{EqElemOngPOldeux} :
\begin{equation}
    \begin{aligned}[]
        \| \gamma'(u) \|^2&=a^2\sin^2(u)+a^2(1+\cos(u))^2\\
            &=2a^2\big( 1+\cos(u) \big),
    \end{aligned}
\end{equation}
et par conséquent\footnote{L'utilisation stricte de la formule \eqref{EqFomVPcoordnorm} demanderait d'intégrer à partir de $-\pi$. Pour plus de simplicité, nous intégrons à partir de zéro, et nous verrons plus tard comment adapter l'intervalle du nouveau paramètre.}
\begin{equation}
    \begin{aligned}[]
        \phi(t)&=\int_0^t\sqrt{2a^2\big( 1+\cos(u) \big)}du\\
        &=\int_0^t\sqrt{2a^2\left( 1+\cos^2\frac{ u }{2}-\sin^2\frac{ u }{2} \right)}du\\
        &=2a\int_0^t\cos\frac{ u }{2}du\\
        &=4a\sin\frac{ t }{2}.
    \end{aligned}
\end{equation}
Pour trouver l'inverse, nous résolvons $\phi\big( \phi^{-1}(s) \big)=s$ par rapport à $\phi^{-1}(s)$ :
\begin{equation}
    \begin{aligned}[]
        4a\sin\left( \frac{ \phi^{-1}(s) }{2} \right)&=s,\\
        \phi^{-1}(s)&=2\arcsin\left( \frac{ s }{ 4a } \right).
    \end{aligned}
\end{equation}

Avant d'écrire trop brutalement $\gamma_N(s)=(\gamma\circ\phi^{-1})(s)$, il faut comprendre comment est $\gamma$. Nous avons reçu la courbe sous forme polaire, c'est-à-dire
\begin{equation}
    \gamma(t)=\big( \gamma_r(t),\gamma_{\theta}(t) \big)=\Big( a\big( 1+\cos(t) \big),t \Big).
\end{equation}
C'est comme cela qu'il faut comprendre la donnée $\rho(\theta)=a\big( 1+\cos(\theta) \big)$. Maintenant la formule $\gamma_N(s)=(\gamma\circ\phi^{-1})(s)$ devient
\begin{subequations}
    \begin{numcases}{}
        (\gamma_N)_r(s)=\gamma_r\big( \phi^{-1}(s) \big)\\
        (\gamma_N)_{\theta}(s)=\gamma_{\theta}\big( \phi^{-1}(s) \big).
    \end{numcases}
\end{subequations}
Étant donné que $\gamma_{\theta}(t)=t$, la seconde est facile :
\begin{equation}
    (\gamma_N)_{\theta}(s)=2\arcsin\left( \frac{ s }{ 4a } \right).
\end{equation}
Pour la première,
\begin{equation}
    (\gamma_N)_r(s)=a\big[ 1+\cos\big( 2\arcsin\frac{ s }{ 4a } \big) \big]=\frac{ 16a^2-s^2 }{ 8a }.
\end{equation}
Nous écrivons donc le nouveau paramétrage en coordonnées polaires sous la forme
\begin{equation}
    \left( \frac{ 16a^2-s^2 }{ 8a },2\arcsin\frac{ s }{ 4a } \right).
\end{equation}
La question qui arrive maintenant est de savoir quel intervalle parcours la nouvelle variable $s$. D'après le résultat de l'exemple~\ref{EqCardioide}, la longueur de la cardioïde est de $8a$ et nous avons donc $s\in\mathopen[ 0 , 8a \mathclose]$. Cependant, la condition d'existence de $\arcsin$ nous interdit d'avoir $s$ plus grand que $4a$ en valeur absolue. Où est le problème ?

Le problème est que nous avons changé l'origine de notre paramètre en donnant $\phi(t)$ comme une intégrale à partir de $0$ au lieu de $-\pi$. Cela se voit en regardant de quel point nous partons : en $s=0$ nous sommes sur le point $(2a,0)$ tandis qu'avec le paramètre original, c'est-à-dire $\theta\in\mathopen[ -\pi , \pi \mathclose]$, nous avons pour $\theta=-\pi$ le point $(0,-\pi)$.

Il se passe donc que si nous commençons à parcourir la cardioïde avec $s=0$, nous partons du milieu, et nous ne parcourons donc pas tout. Étant donné que le «premier» point de la cardioïde est le point $(0,-\pi)$, le paramètre $s$ commence en $s=-4a$, et nous avons comme intervalle :
\begin{equation}
    s\in\mathopen[ -4a , 4a \mathclose],
\end{equation}
ce qui est en accord avec la conditions d'existence.
\end{example}

Quel enseignement tirer de cet exemple ? Lorsqu'on calcule $\phi(t)$ pour trouver les coordonnées normales, il y a deux solutions.
\begin{enumerate}
\item
    Utiliser strictement la formule $\phi(t)=\int_a^t\| \gamma'(u) \|du$, en prenant bien comme borne de départ le point de départ de le paramétrage de $\gamma$. À ce moment la coordonnée normale construite aura $\mathopen[ 0 , l(\gamma) \mathclose]$ comme intervalle de variation.
\item
    Faire commencer l'intervalle d'intégration en zéro (ou ailleurs). Un bon choix peut simplifier quelques calculs, mais alors il faudra bien choisir la valeur de départ de la nouvelle coordonnées pour que le «premier» point de la courbe soit correct. Dans ce cas, la longueur de l'intervalle sera quand même $l(\gamma)$. Il n'y a donc pas de problèmes pour trouver la valeur du bout de l'intervalle de variation du paramètre normal.
\end{enumerate}
Dans tous les cas, il faut bien préciser l'intervalle de variation du paramètre lorsqu'on donne une courbe paramétrée.

% This is part of Mes notes de mathématique
% Copyright (c) 2010-2016,2018-2020
%   Laurent Claessens, Carlotta Donadello
% See the file fdl-1.3.txt for copying conditions.

%---------------------------------------------------------------------------------------------------------------------------
\subsection{Tangente à une courbe paramétrée}
%---------------------------------------------------------------------------------------------------------------------------

\begin{definition}
Soit $(I,\gamma)$ un arc paramétré de classe $\mathcal{C}^k$ avec $k\geq 1$. Nous disons que la courbe admet une \defe{tangente}{tangente} en $\gamma(t_0)\in\eR^n$ lorsque les deux conditions suivantes sont remplies
\begin{enumerate}
    \item
        $\gamma(t)\neq \gamma(t_0)$ pour tout $t$ dans un voisinage de $t_0$;
    \item
        la direction de la droite qui passe par $\gamma(t)$ et $\gamma(t_0)$ admet une limite lorsque $t\to t_0$.
\end{enumerate}
Dans ce cas, la tangente sera la droite passant par le point $\gamma(t_0)$ et dont la direction est donnée par la limite.
\end{definition}
Dans cette définition, par \defe{direction}{direction} d'une droite, nous entendons le vecteur de norme $1$ parallèle à celle-ci sans tenir compte du signe. La tangente sera donc la droite passant par $\gamma(t_0)$ et parallèle au vecteur
\begin{equation}
\lim_{t\to t_0}\frac{ \gamma(t)-\gamma(t_0) }{ \| \gamma(t)-\gamma(t_0) \| }.
\end{equation}
Évidemment si nous avions écrit $\gamma(t_0)-\gamma(t)$, ça n'aurait pas changé la droite. Par abus de langage, nous parlerons souvent de «la direction $u$» même lorsque $u$ n'est pas de norme $1$.

Formellement, une direction est une classe d'équivalence de vecteurs pour la relation $u\sim v$ s'il existe $\lambda\neq 0$ tel que $u=\lambda v$, mais nous n'aurons pas besoin de cette précision ici.

Sans surprises, la tangente est à peu près toujours donnée par la dérivée lorsqu'elle existe. Plus précisément nous avons le
\begin{theorem}
Soit $(I,\gamma)$, un arc paramétré de classe $\mathcal{C}^k$ ($k\geq 1$) et $t_0\in I$ tel que
\begin{equation}
    \gamma'(t_0)=\gamma''(t_0)=\ldots=\gamma^{(q-1)}(t_0)=0
\end{equation}
et
\begin{equation}
    \gamma^{(q)}(t_0)\neq 0
\end{equation}
pour un entier $1\leq q\leq k$. Alors $\gamma$ admet une tangente en $\gamma(t_0)$ de direction $\gamma^{(q)}(t_0)$.
\end{theorem}

\begin{proof}

Le développement de $\gamma(t_0)$ en série de Taylor autour de $t$ jusqu'à l'ordre $q$ est
\begin{equation}        \label{EqDevTaylfttzq}
    \begin{aligned}[]
        \gamma(t)&=\gamma(t_0)+\gamma'(t_0)| t-t_0 |+\frac{ \gamma't(t_0) }{2}| t-t_0 |^2+\cdots +\frac{ \gamma^{(q)}(t_0) }{ q! }| t-t_0 |^q\\
            &\quad+\varepsilon(t)| t-t_0 |^q
    \end{aligned}
\end{equation}
où $\varepsilon$ est une application $\varepsilon\colon \eR\to \eR^n$ telle que $\lim_{t\to t_0} \varepsilon(t)=0$. En utilisant les hypothèses, nous éliminons la majorité des termes dans le développement \eqref{EqDevTaylfttzq} :
\begin{equation}
    \gamma(t)-\gamma(t_0)=\frac{1}{ q! }\gamma^{(q)}(t_0)| t-t_0 |^q+\varepsilon(t)| t-t_0 |^q.
\end{equation}
La direction de la droite qui joint $\gamma(t)$ à $\gamma(t_0)$ est donc donnée par
\begin{equation}
    \frac{ \gamma(t)-\gamma(t_0) }{ \| \gamma(t)-\gamma(t_0) \| }=\frac{ \frac{1}{ q! }\gamma^{(q)}(t_0)| t-t_0 |^q+\varepsilon(t)| t-t_0 |^q }{ \| \frac{1}{ q! }\gamma^{(q)}(t_0)| t-t_0 |^q+\varepsilon(t)| t-t_0 |^q\|  }
\end{equation}
et la limite lorsque $t\to t_0$ donne $\gamma^{(q)}(t_0)$ comme direction de la tangente.

\end{proof}

Lorsque le théorème s'applique, le vecteur
\begin{equation}
\tau=\frac{ \gamma^{(q)}(t_0) }{ \| \gamma^{(q)}(t_0) \| }
\end{equation}
est appelé le \defe{vecteur unitaire tangent}{vecteur!unitaire tangent} en $\gamma(t_0)$ à l'arc paramétré $\gamma$.


\begin{corollary}       \label{CorTgSoCun}
Si $(I,\gamma)$ est un arc paramétré de classe $\mathcal{C}^1$ régulier (c'est-à-dire $\gamma'(t)\neq 0$ pour tout $t$) alors l'arc admet une tangente en tout point et le vecteur unitaire de la tangente est donné par
\begin{equation}
    \tau(t)=\frac{ \gamma'(t) }{ \| \gamma'(t) \| },
\end{equation}
pour tout $t$ dans $I$.
\end{corollary}

\begin{corollary}       \label{CorUnitTgtaugpnorma}
Si $\gamma=(J,\gamma_N)$ est un arc paramétré de classe $\mathcal{C}^1$, normal, alors le vecteur unitaire de la tangente au point $\gamma_N(s)$ est donné par $\tau(s)=\gamma_N'(s)$.
\end{corollary}

\begin{proof}
Nous devons démontrer que dans le cas d'un paramétrage normal nous avons $\| \gamma_N'(s) \|=1$ pour tout $s$. Par définition,
\begin{equation}
    l\big( \mathopen[ s , s' \mathclose],g \big)=\int_s^{s'}\| \gamma_N'(u) \|du=s'-s.
\end{equation}
Par conséquent,
\begin{equation}
    \lim_{h\to 0} \frac{1}{ h }\int_s^{s+h}\| \gamma_N'(u) \|du=\lim_{y\to 0} \frac{ s+h-s }{ h }=1.
\end{equation}
Cela implique que $\| \gamma_N'(s) \|=1$, et donc en particulier que $(J,\gamma_N)$ est un arc régulier. Le corolaire précédent montre alors que $\tau(s)=\gamma_N'(s)/\| \gamma_N'(s) \|=\gamma_N'(s)$.
\end{proof}

\begin{example}
Considérons la courbe $\gamma(t)=(t^2,t^3)$, et cherchons la tangente en $t_0=0$. En dérivant nous avons successivement
\begin{equation}
    \begin{aligned}[]
        \gamma(t)&=(t^2,t^3)\\
        \gamma'(t)&=(2t,3t^2)\\
        \gamma''(t)&=(2,6t).
    \end{aligned}
\end{equation}
En posant $t=0$, nous trouvons que $\gamma'(0)=0$ mais $\gamma''(0)=(2,0)\neq 0$. Le théorème nous dit donc que la direction de la tangente est horizontale. Nous pouvons faire le calcul directement :
\begin{equation}
    \frac{ \gamma(t)-\gamma(t_0) }{ \| \gamma(t)-\gamma(t_0) \| }=\frac{ (t^2,t^3) }{ \sqrt{t^4+t^6} }=\frac{ (t^2,t^3) }{ t^2\sqrt{1+t^2} }=\frac{ (1,t) }{ \sqrt{1+t^2} },
\end{equation}
dont la limite \( t\to 0\) est bien le vecteur horizontal $(1,0)$.

% Laisser la coupure à la ligne suivante, parce que le ref vers la figure peut être vers le futur; les éventuelles autres, non.
La figure~\ref{LabelFigParamTangente} montre quelques tangentes,
c'est-à-dire quelques vecteurs dans la direction $\gamma'(t)$ (pour les $t\neq 0$, il ne faut pas aller à la dérivée seconde). Nous remarquons que de part et d'autres du sommet, les vecteurs ne sont pas dirigés dans le même sens. \emph{En tant que vecteurs} de norme $1$, ces vecteurs n'ont pas de limites quand $t\to 0$. Ce sont bien les \emph{directions} qui ont une limite, parce que la direction ne tient pas compte du sens.
\newcommand{\CaptionFigParamTangente}{Quelques tangentes de la courbe $\gamma(t)=(t^2,t^3)$.}
\input{auto/pictures_tex/Fig_ParamTangente.pstricks}

\end{example}

%+++++++++++++++++++++++++++++++++++++++++++++++++++++++++++++++++++++++++++++++++++++++++++++++++++++++++++++++++++++++++++ 
\section{Un peu de topologie}
%+++++++++++++++++++++++++++++++++++++++++++++++++++++++++++++++++++++++++++++++++++++++++++++++++++++++++++++++++++++++++++


La proposition \ref{PROPooJYGVooShNewy} donne une sorte de théorème des valeurs intermédiaires pour le cas d'une application à valeurs dans un chemin.
\begin{proposition}     \label{PROPooJYGVooShNewy}
    Soit une application continue et injective \( \gamma\colon \mathopen[ 0 , 1 \mathclose]\to \eR^n\). Nous posons \( \Gamma=\gamma\big( \mathopen[ 0 , 1 \mathclose] \big)\) sur lequel nous considérons la topologie induite\footnote{Définition \ref{DefVLrgWDB}.} de \( \eR^n\).

    Nous supposons que \( \gamma^{-1}\colon \Gamma\to \mathopen[ 0 , 1 \mathclose]\) est continue\quext{Je ne suis pas certain que cette hypothèse soit indispensable. Voiir la question \ref{ITEMooVTBHooVoMzcq} dans \ref{SUBSEUBSECooCCKFooOXsnwG}.}

    Nous considérons un chemin \( \alpha\colon \mathopen[ 0 , 1 \mathclose]\to \eR^n\) tel que
    \begin{enumerate}
        \item
            \( \alpha\) est continu,
        \item
            \( \alpha(0)=\gamma(0)\)
        \item
            \( \alpha\big( \mathopen[ 0 , 1 \mathclose] \big)\subset \Gamma\)
        \item
            \( \alpha(1)=\gamma(t_0)\) pour un certain \( t_0\in \mathopen[ 0 , 1 \mathclose]\).
    \end{enumerate}
    Alors pour tout \( t\in\mathopen[ 0 , t_0 \mathclose]\), il existe \( u\in\mathopen[ 0 , 1 \mathclose]\) tel que \( \alpha(u)=\gamma(1)\).
\end{proposition}

\begin{proof}
    Nous commençons par montrer que l'application \( \alpha\colon \mathopen[ 0 , 1 \mathclose]\to \Gamma\) est encore continue lorsque nous voyons bien l'espace d'arrivée comme \( \Gamma\) muni de sa propre topologie et non comme \( \eR^2\) muni de sa topologie usuelle.

    Soit un ouvert \( \mO\) de \( \Gamma\); il existe un ouvert \( \mO'\) de \( \eR^2\) tel que \( \mO=\mO'\cap \Gamma\). Vu que \( \alpha\) ne prend ses valeurs que dans \( \Gamma\), nous avons \( \alpha^{-1}(\Gamma\cap \mO')=\alpha^{-1}(\mO')\) et comme \( \alpha\) est continue pour la topologie de \( \eR^2\), la partie \( \alpha^{-1}(\mO')\) est un ouvert de \( \mathopen[ 0 , 1 \mathclose]\).

    Nous considérons donc l'application
    \begin{equation}
        \gamma^{-1}\circ\alpha\colon \mathopen[ 0 , 1 \mathclose]\to \mathopen[ 0 , 1 \mathclose]
    \end{equation}
    qui est continue et vérifie donc le théorème des valeurs intermédiaires \ref{ThoValInter}. Les hypothèses \( \alpha(0)=\gamma(0)\) et \( \alpha(1)=\gamma(t_0)\) donnent
    \begin{subequations}
        \begin{align}
            (\gamma^{-1}\circ\alpha)(0)&=0\\
            (\gamma^{-1}\circ\alpha)(1)&=t_0.
        \end{align}
    \end{subequations}
    Donc pour tout \( t\in \mathopen[ 0 , t_0 \mathclose]\), il existe \( u\in \mathopen[ 0 , 1 \mathclose]\) tel que \( (\gamma^{-1}\circ \alpha)(u)=t\). En appliquant \( \gamma\) des deux côtés, nous voyons que ce \( u\) vérifie \( \alpha(u)=\gamma(t)\) comme demandé.
\end{proof}

La proposition suivant dit essentiellement que la longueur d'un chemin est minoré par la longueur de son graphe.
\begin{proposition}
    
\end{proposition}
<++>

%\ref{LabelFigYWxOAkh}. % From file YWxOAkh
\newcommand{\CaptionFigYWxOAkh}{La figure de la proposition \ref{PROPooVXDNooPZYKPr}.}
\input{auto/pictures_tex/Fig_YWxOAkh.pstricks}

\begin{proposition}     \label{PROPooVXDNooPZYKPr}
    Nous considérons la partie suivante de \( \eR^2\) :
    \begin{equation}
        A_2=\big\{   \big( t,\sin(1/t) \big)   \big\}_{t\in \mathopen] 0 , 1 \mathclose]}
    \end{equation}
    qui est dessinée est sur la figure \ref{LabelFigYWxOAkh}. Ensuite nous posons
    \begin{equation}
        A=\big\{  (0,0)  \big\}.
    \end{equation}
    La partie \( A\) est connexe, mais pas connexe par arcs.
\end{proposition}

\begin{proof}
    En plusieurs points.
    \begin{subproof}
        \item[\( A_2\) est connexe]
            La partie \( A_2\) est l'image de la fonction
            \begin{equation}
                \begin{aligned}
                f\colon \mathopen] 0 , 1 \mathclose]&\to \eR^2 \\
                    t&\mapsto \big( t,\sin(1/t) \big). 
                \end{aligned}
            \end{equation}
            Vu que \( f\) est continue et que son ensemble de départ est connexe, \( A_2\) est connexe (proposition \ref{PropGWMVzqb}).
        \item[\( A\) est connexe]
            Soient deux ouverts disjoints \( \mO_1\) et \( \mO_2\) dont l'union contient \( A\). Nous supposons que \( (0,0)\in \mO_1\). Si \( \mO_1\) contient \( B\big( (0,0),r \big)\), alors il contient tous les points de la forme \( \big( \frac{1}{ 2k\pi },0 \big)\) pour \( k\) assez grand. Ces points sont dans \( A_1\).

            Vu que \( \mO_1\) contient des points de \( A_1\), il doit contenir tous les points de \( A_1\); sinon les ouverts \( \mO_1\) et \( \mO_2\) contrediraient la connexité de \( A_1\). Finalement, \( A\subset \mO_1\) et \( A\) est connexe.
    \end{subproof}
    <++>
\end{proof}
<++>

%+++++++++++++++++++++++++++++++++++++++++++++++++++++++++++++++++++++++++++++++++++++++++++++++++++++++++++++++++++++++++++
\section{Repère de Frenet}      \label{SecFrenet}
%+++++++++++++++++++++++++++++++++++++++++++++++++++++++++++++++++++++++++++++++++++++++++++++++++++++++++++++++++++++++++++

Dans cette section, nous ne considérons que des courbes dans $\eR^3$.

\begin{proposition}     \label{Proptausclataupzero}
    Soit $\gamma=(J,\gamma_N)$ un arc paramétré normal de classe $\mathcal{C}^2$. Alors pour toute valeur de $s$ dans $J$, nous avons
    \begin{equation}
        \tau(s)\cdot\tau'(s)=0
    \end{equation}
    où $\tau(s)=\gamma_N'(s)$. C'est-à-dire que la dérivée seconde est perpendiculaire à la dérivée première.
\end{proposition}

\begin{proof}
    Le paramétrage étant normal, nous avons
    \begin{equation}
        \| \gamma_N'(s) \|^2=\sum_{i=1}^nx'_i(s)^2=1;
    \end{equation}
    ce qui implique, en dérivant les deux membres, que
    \begin{equation}
        0=2\sum_{i=1}^nx_i'(s)x''_i(s),
    \end{equation}
    c'est-à-dire exactement $\gamma_N'(s)\cdot \gamma_N''(s)=0$; d'où la thèse.
\end{proof}

\begin{remark}
    Si nous n'utilisons pas des coordonnées normales, la proposition~\ref{Proptausclataupzero} n'est pas spécialement vraie. Prenons par exemple la courbe qui donne la parabole :
    \begin{subequations}
        \begin{align}
            \gamma(t)&=(t,t^2)\\
            \gamma'(t)&=(1,2t)\\
            \gamma''(t)&=(0,2)
        \end{align}
    \end{subequations}
    Nous avons $\gamma'(t)\cdot \gamma''(t)=4t$. Par conséquent, la dérivée seconde n'est la normale à la courbe que en $t=0$. Cela est une propriété très intéressante des coordonnées normales : la dérivée seconde d'une coordonnées normale donne un vecteur normal à la courbe, c'est-à-dire perpendiculaire à la tangente.
\end{remark}

\begin{definition}      \label{DefCourbureNormleUnit}
    Soit $\gamma=(J,\gamma_N)$ un arc paramétré normal de classe $\mathcal{C}^2$.
    \begin{enumerate}
        \item
            Le \defe{vecteur unitaire tangent}{tangent!vecteur unitaire} est donné par le corolaire~\ref{CorTgSoCun} : \( \tau(t)=\gamma'(t)/\| \gamma'(t) \|\).
        \item
    La \defe{normale principale}{normale!principale} est le vecteur $\tau'(s)$. Le \defe{vecteur unitaire normal}{unitaire!normale principale}\index{vecteur!unitaire normal} est le vecteur\nomenclature[C]{$\nu(s)$}{Vecteur unitaire de la normale principale}
    \begin{equation}
        \nu(s)=\frac{ \tau'(s) }{ \| \tau'(s) \| }=\frac{ \gamma_N''(s) }{ \| \gamma_N''(s) \| }.
    \end{equation}
    Nous déduirons une formule plus pratique en dehors des coordonnées normales en \eqref{EqCourburetermf}.
\item
    La \defe{courbure}{courbure} au point $\gamma_N(s)$ est le réel\nomenclature[C]{$c(s)$}{rayon de courbure}
    \begin{equation}
        c(s)=\| \tau'(s) \|=\| \gamma_N''(s) \|.
    \end{equation}
    Note : il y a une notion de courbure signée qui sera donnée dans la définition~\ref{DEFooJFWEooXcIVUs}.
\item
    Le \defe{rayon de courbure}{rayon!de courbure} est le réel
    \begin{equation}
        R(s)=\frac{1}{ c(s) }=\frac{1}{ \| \gamma_N''(s) \| }.
    \end{equation}
    \end{enumerate}
\end{definition}

Par la proposition~\ref{Proptausclataupzero}, nous avons $\nu(s)\cdot\tau(s)=0$. En combinant toutes les formules, nous avons les différentes expressions suivantes pour le vecteur normal unitaire :
\begin{equation}        \label{Eq0908nufractauRc}
    \nu(s)=\frac{ \gamma_N''(s) }{ c(s) }=\frac{ \tau'(s) }{ \| \tau'(s) \| }=\frac{ \tau'(s) }{ c(s) }=R(s)\tau'(s)=R(s)\gamma_N''(s).
\end{equation}

\begin{proposition}
    La fonction courbure s'écrit $c=\| \gamma_N'\times \gamma_N'' \|$.
\end{proposition}

\begin{proof}
    Par la proposition \ref{PROPooMXAIooJureOD} nous avons :
    \begin{equation}
        \langle \gamma_N', \gamma_N''\rangle^2 + \| \gamma_N'\times \gamma_N'' \|^2=\| \gamma_N' \|^2\| \gamma_N'' \|^2=\| \gamma_N'' \|^2
    \end{equation}
    parce que, le paramétrage étant normal, $\| \gamma_N' \|=1$. Mais $\langle \gamma_N', \gamma_N''\rangle =0$, donc il reste $\| \gamma_N'\times \gamma_N'' \|^2=\| \gamma_N'' \|^2$, d'où
    \begin{equation}        \label{Eqcsnormgpgpps}
        c(s)=\| \gamma_N''(s) \|=\| \gamma_N'(s)\times \gamma_N''(s) \|
    \end{equation}
    pour chaque $s$ dans $J$.
\end{proof}

\begin{definition}
    Soit $s$ un point birégulier (c'est-à-dire $\gamma_N'(s)\neq 0$ et $\gamma_N''(s)\neq 0$) de l'arc normal $\gamma=(J,\gamma_N)$. Le \defe{vecteur unitaire de la binormale}{binormale} est le vecteur\nomenclature[C]{$\beta(s)$}{Vecteur unitaire de la binormale}
    \begin{equation}
        \beta(s)=\tau(s)\times\nu(s)
    \end{equation}
\end{definition}

Par leurs définitions, $\tau$ et $\nu$ sont unitaires, tandis que la proposition~\ref{Proptausclataupzero} montre qu'ils sont également orthogonaux. Les propriétés du produit vectoriel font que $\beta$ est également unitaire, et simultanément orthogonal à $\tau$ et à $\nu$.

\begin{definition}
    Le repère orthonormal $\{ \gamma_N(s),\tau(s),\beta(s) \}$ est le \defe{repère de Frenet}{repère!de Frenet} au point $\gamma_N(s)$.
\end{definition}

\begin{lemma}
    Le  vecteur unitaire normal est donné par $\nu(s)=\beta(s)\times \tau(s)$.
\end{lemma}

\begin{proof}
    Ceci est une application de la formule d'expulsion \eqref{EqFormExpluxxx} et de l'orthonormalité de la base de Frenet :
    \begin{equation}
        \beta\times\tau=(\tau\times\nu)\times\tau=-(\nu\cdot\tau)\tau+(\tau\cdot\tau)\nu=\nu.
    \end{equation}
\end{proof}

%---------------------------------------------------------------------------------------------------------------------------
\subsection{Torsion}
%---------------------------------------------------------------------------------------------------------------------------

Décomposons le vecteur $\beta'(s)$ dans la base de Frenet. Pour cela nous allons utiliser la proposition~\ref{PropScalCompDec} et montrer que $\beta'(s)\cdot \tau(s)=\beta'(s)\cdot\beta(s)=0$, ce qui voudra dire que, dans la base de Frenet, les composantes de $\beta'$ le long de $\tau$ et $\beta$ sont nulles. Le vecteur $\beta'$ sera donc colinéaire à $\nu$.

D'abord, étant donné que la norme de $\beta(s)$ est constante par rapport à $s$, nous avons
\begin{equation}
    0=\frac{ d }{ ds }\| \beta(s) \|^2=2\beta'(s)\cdot\beta(s).
\end{equation}
Ensuite, nous dérivons la définition $\beta(s)=\tau(s)\times\nu(s)$ en utilisant la formule de Leibnitz \eqref{EqFormLeibProdscalVect} :
\begin{equation}
    \beta'(s)=\tau'(s)\times\nu(s)+\tau(s)\times\nu'(s).
\end{equation}
Mais $\tau'(s)=\gamma_N''(s)$ tandis que $\nu(s)=\frac{ \gamma_N''(s) }{ \| \gamma_N''(s) \| }$, de telle sorte que $\tau'(s)\times\nu(s)=0$. Nous restons donc avec $\beta'(s)=\tau(s)\times\nu'(s)$, ce qui prouve que $\beta'(s)$ est perpendiculaire à $\tau(s)$ et donc que $\beta'(s)\cdot\tau(s)=0$.

Le vecteur $\beta'(s)$ est donc un multiple de $\nu(s)$. Nous notons $t(s)$\nomenclature[C]{$t(s)$}{Torsion} le facteur de proportionnalité :
\begin{equation}
    \beta'(s)=t(s)\nu(s).
\end{equation}

\begin{definition}      \label{DefTorsion}
    Soit $\gamma=(J,\gamma_N)$ un arc paramétré normal de classe $\mathcal{C}^3$. La \defe{torsion}{torsion} de $\gamma$ au point $\gamma_N(s)$ est le réel
    \begin{equation}
        t(s)=\| \beta'(s) \|=\| \tau(s)\times\nu'(s) \|.
    \end{equation}
    Lorsque $t(s)\neq 0$, le réel $T(s)=\frac{1}{ t(s) }$ est le \defe{rayon de torsion}{rayon!de torsion} de $\gamma$ en $\gamma_N(s)$.
\end{definition}

Étant donné que pour chaque $s$, l'ensemble $\{ \tau(s),\nu(s),\beta(s) \}$ est une base, il est naturel de vouloir décomposer leurs dérivées dans cette base. D'abord, par définition de $c$ et de $t$, nous avons
\begin{equation}
    \begin{aligned}[]
        \tau'(s)&=c(s)\nu(s)\\
        \beta'(s)&=t(s)\nu(s).
    \end{aligned}
\end{equation}
Il reste à décomposer $\nu'(s)$. Définissons $\alpha_{\tau}$, $\alpha_{\nu}$ et $\alpha_{\beta}$ (qui peuvent dépendre de $s$) par
\begin{equation}
    \nu'(s)=\alpha_{\tau}\tau(s)+\alpha_{\nu}\nu(s)+\alpha_{\beta}\beta(s).
\end{equation}
En vertu de la proposition~\ref{PropScalCompDec}, nous avons
\begin{equation}
    \begin{aligned}[]
        \alpha_{\tau}=\langle \nu'(s), \tau(s)\rangle&=-\langle \nu(s), \tau'(s)\rangle =-\langle \nu(s), c(s)\nu(s)\rangle =-c(s) ,\\
        \alpha_{\nu}=\langle \nu'(s),  \nu(s)\rangle &=0,\\
        \alpha_{\beta}=\langle \nu'(s), \beta(s)\rangle &=-\langle \nu(s), \beta'(s)\rangle =-t(s),
    \end{aligned}
\end{equation}
où nous avons utilisé le fait que $\langle \nu(s), \nu(s)\rangle =\| \nu(s) \|^2=1$. Si nous mettons ces résultats sous forme matricielle, nous avons les \defe{formules de Frenet}{Frenet!formules} :
\begin{equation}
    \begin{pmatrix}
        \tau'(s)    \\
        \nu'(s) \\
        \beta'(s)
    \end{pmatrix}=
    \begin{pmatrix}
        0   &   c(s)    &   0   \\
        -c(s)   &   0   &   -t(s)   \\
        0   &   t(s)    &   0
    \end{pmatrix}
    \begin{pmatrix}
        \tau(s) \\
        \nu(s)  \\
        \beta(s)
    \end{pmatrix}.
\end{equation}


\begin{proposition}
    Si $s$ est un point birégulier, alors la torsion est donnée par
    \begin{equation}
        t(s)=-\frac{ (\gamma_N'\times \gamma_N'')\times \gamma_N''' }{ \| \gamma_N'(s)\times \gamma_N''(s) \|^2 }.
    \end{equation}
\end{proposition}

\begin{proof}
    Par l'équation \eqref{Eq0908nufractauRc}, nous avons $\gamma_N''(s)=c'(s)\nu(s)$, et par conséquent
    \begin{equation}
        \gamma_N'''(s)=c'(s)\nu(s)+c(s)\nu'(s)=c'(s)\nu(s)+c(s)\big[ -c(s)\tau(s)-t(s)\beta(s) \big],
    \end{equation}
    où nous avons utilisé la formule de Frenet pour $\nu'(s)$. Par ailleurs, sachant le corolaire~\ref{CorUnitTgtaugpnorma} et la formule de Frenet pour $\tau'$, nous avons
    \begin{equation}
        \gamma_N'\times \gamma_N''=\tau(s)\times \tau'(s)=\tau(s)\times c(s)\nu(s)=c(s)\beta(s).
    \end{equation}
    En combinant les deux dernières équations, et en se souvenant que la base de Frenet et orthonormale,
    \begin{equation}
        (\gamma_N'\times \gamma_N'')\cdot \gamma_N'''(s)=-c(s)^2t(s),
    \end{equation}
    et donc, en remplaçant $c(s)$ par la formule \eqref{Eqcsnormgpgpps},
    \begin{equation}
        t(s)=-\frac{  (\gamma_N'\times \gamma_N'')\cdot \gamma_N'''   }{ \| \gamma_N'\times \gamma_N'' \|^2 }.
    \end{equation}
\end{proof}

%+++++++++++++++++++++++++++++++++++++++++++++++++++++++++++++++++++++++++++++++++++++++++++++++++++++++++++++++++++++++++++
\section{Hors des coordonnées normales}
%+++++++++++++++++++++++++++++++++++++++++++++++++++++++++++++++++++++++++++++++++++++++++++++++++++++++++++++++++++++++++++

\begin{remark}      \label{Remfougnormoupad}
    Notons que la définition de $\tau$ est donnée pour tout arc $\mathcal{C}^1$ régulier $(I,\gamma)$ par $\tau(t)=\gamma'(t)/\| \gamma'(t) \|$. La propriété $\tau=\gamma_N'$ n'est valable que lorsque le paramétrage est normal. Les autres définitions ont toutes été données dans le cas d'un paramétrage normal.
\end{remark}

La remarque~\ref{Remfougnormoupad} nous incite à exprimer toute la base de Frenet en termes de $\gamma$ lorsque le paramétrage n'est pas normal. Étant donné que nous pouvons toujours faire le changement de variable $\gamma(t)=\gamma_N\big( \phi(t) \big)$ (proposition~\ref{PropExisteChmNorm}), il est possible d'exprimer les vecteurs $\tau$, $\nu$ et $\beta$ ainsi que les réels $c$ et $t$ en fonction de $\gamma$ et de ses dérivées.

Nous allons maintenant travailler à écrire les formules.

Pour plus de facilité, nous collectons les définitions. Afin d'alléger la notation, nous n'exprimons pas explicitement les dépendances en $s$ :
\begin{description}
    \item[Vecteur unitaire tangent]
        Par le corolaire~\ref{CorUnitTgtaugpnorma}, $\tau$ est donné par $\tau=\gamma_N'$.
    \item[Vecteur unitaire normal]
        Par la définition~\ref{DefCourbureNormleUnit}, $\nu$ est donné par
        $\nu=\frac{ \tau' }{ \| \tau' \| }$.
    \item[Vecteur unitaire de la binormale]
        Par la définition~\ref{DefCourbureNormleUnit}, $\beta$ est donné par
            $\beta=\tau\times\nu$.
    \item[Courbure]
        Par la définition~\ref{DefCourbureNormleUnit}, $c$ est donné par
            $c=\| \tau' \|$.
    \item[Torsion]
        Par la définition~\ref{DefTorsion}, $t$ est donné par
            $t=\| \beta' \|$.
\end{description}


Le schéma du changement de variable est
\begin{equation}        \label{EqDiagIJstgvpR}
    \xymatrix{%
    t\in I \ar[r]^{f}\ar[d]_{\phi}      &   \eR^3\\
    s\in J \ar[ru]_{g}  &
       }
\end{equation}
La difficulté ne sera pas d'éliminer $\gamma_N$ de toutes les formules, mais bien de se débarrasser des fonctions $\phi$ qui arrivent quand nous exprimons $\gamma_N$ en termes de $\gamma$, et en particulier lorsque nous voulons exprimer les dérivées de $\gamma_N$ en termes de $\gamma$ et de ses dérivées.

Regardons d'abord comment les dérivées de $\gamma_N$ s'expriment en termes de $\gamma$. En utilisant le fait que $\gamma_N(s)=(\gamma\circ\phi^{-1})(s)$ et que $\| \gamma_N'(s) \|=1$, nous avons
\begin{equation}        \label{EqgpNgpnNnr}
    \gamma_N'(s)=\frac{ \gamma_N'(s) }{ \| \gamma_N'(s) \| }
    =\frac{ (\gamma\circ\phi^{-1})'(s) }{ \| (\gamma\circ\phi^{-1})'(s) \| }
    =\frac{ \gamma'\big( \phi^{-1}(s) \big)   (\phi^{-1})'(s)   }{ \| \gamma'\big( \phi^{-1}(s) \big) \|  |(\phi^{-1})'(s) |}
    =\frac{ \gamma'(t) }{ \| \gamma'(t) \| }
\end{equation}
où nous avons utilisé le fait que $\phi^{-1}$ étant croissante (parce que l'inverse d'une fonction croissante est croissante), $(\phi^{-1})'(s)=| (\phi^{-1})'(s) |$. Pourquoi écrivons nous $| \phi^{-1}(s) |$ et non $\| \phi^{-1}(s) \|$ ?

Pour la dérivée seconde, nous dérivons la relation \eqref{EqgpNgpnNnr} :
\begin{equation}
    \gamma_N''(s)=\frac{ \gamma''\big( \phi^{-1}(s) \big)(\phi^{-1})'(s) }{ \| \gamma'\big( \phi^{-1}(s) \big) \| }+\gamma'\big( \phi^{-1}(s) \big)\frac{ d }{ ds }\Big[ \| \gamma'\big( \phi^{-1}(s) \big) \| \Big].
\end{equation}
Le petit calcul suivant va nous permettre de simplifier cette expression :
\begin{equation}        \label{Eavpemuetfpnorm}
    (\phi^{-1})'(s)=(\phi^{-1})'\big( \phi(t) \big)=\frac{1}{ \phi'(t) }=\frac{1}{ \| \gamma'(t) \| }.
\end{equation}
Donc
\begin{equation}
    \gamma_N''(s)=\frac{ \gamma''(t) }{ \| \gamma'(t) \|^2 }+\gamma'(t)\frac{ d }{ ds }\Big[ \| \gamma'(t) \| \Big]
\end{equation}
où il est entendu que $t=\phi^{-1}(s)$. Avec cette expression, nous ne nous sommes pas encore débarrassés de la fonction $\phi$, mais nous allons voir que cela nous sera suffisant.

Pour le vecteur unitaire tangent $\tau(s)$, nous avons donc immédiatement
\begin{equation}        \label{EqTauavect}
    \tau(s)=\gamma_N'(s)=\frac{ \gamma'(t) }{ \| \gamma'(t) \| }.
\end{equation}
Ici encore il est sous-entendu que le $t$ dans le membre de droite est lié au $s$ du membre de gauche par $t=\phi^{-1}(s)$. Il est donc naturel de nous demander si nous avons gagné quelque chose, étant donné que la formule \eqref{EqTauavect} contient encore la fonction $\phi$.

Géométriquement, le vecteur $\tau(s)$ est le vecteur normal unitaire de la courbe au point $\gamma_N(s)$. En utilisant les relations du diagramme \eqref{EqDiagIJstgvpR}, nous avons en réalité $\gamma_N(s)=\gamma_N\big( \phi(t) \big)=\gamma(t)$. Le vecteur $\frac{ \gamma'(t) }{ \| \gamma'(t) \| }$ représente donc le vecteur normal tangent au point $\gamma(t)$.

Pour calculer la courbure, nous devons d'abord calculer le produit vectoriel
\begin{equation}        \label{eqProdvectogpgpp}
    \begin{aligned}[]
        \gamma_N'(s)\times \gamma_N''(s) &=  \frac{ \gamma'(t) }{ \| \gamma'(t) \| }\times \left( \frac{ \gamma''(t) }{ \| \gamma'(t) \|^2 }+\gamma'(t)\frac{ d }{ ds }\Big[ \| \gamma'(t) \| \Big] \right)\\
        &=\frac{ \gamma'(t)\times \gamma''(t) }{ \| \gamma'(t) \|^3 }
    \end{aligned}
\end{equation}
parce que le deuxième terme dans la parenthèse est un multiple de $\gamma'(t)$, de telle sorte à ce que son produit vectoriel avec $\gamma'(t)/\| \gamma'(t) \|$ soit nul. En prenant la norme,
\begin{equation}        \label{EqCourburetermf}
    c(s)=\frac{ \| \gamma'(t)\times \gamma''(t) \| }{ \| \gamma'(t) \|^3 }.
\end{equation}
Encore une fois, cette équation nous enseigne que la courbure au point $\gamma(t)\in\eR^3$ est donnée par le membre de droite, qui ne dépend que de $t$.

Le vecteur unitaire binormal est donné par $\beta(s)=\tau(s)\times \nu(s)$. En utilisant \eqref{EqTauavect} et \eqref{Eq0908nufractauRc},
\begin{equation}
    \beta(s)=\tau(s)\times\nu(s)=\gamma_N'(s)\times \frac{ \gamma_N''(s) }{ c(s) }.
\end{equation}
Les formules \eqref{eqProdvectogpgpp} pour le produit vectoriel et \eqref{EqCourburetermf} pour la courbure donnent ensuite
\begin{equation}
    \beta(s)=\frac{ \gamma'(t)\times \gamma''(t) }{ \| \gamma'(t) \|^3 }\cdot\frac{1}{ c(s) }=\frac{ \gamma'(t)\times \gamma''(t) }{ \|  \gamma'(t)\times \gamma''(t)  \| }.
\end{equation}
Cela donne le vecteur unitaire binormal au point $\gamma(t)$ en termes de $\gamma'(t)$ et $\gamma''(t)$.

La torsion demande d'utiliser la dérivée troisième de $\gamma_N$. Nous avons
\begin{equation}
    \begin{aligned}[]
        \gamma_N'''(s)&=(\gamma\circ\phi^{-1})'''(s)\\
        &=\Big( \gamma'\big( \phi^{-1}(s) \big)(\phi^{-1})'(s) \Big)''\\
        &=\Big( \gamma''\big( \phi^{-1}(s) \big)(\phi^{-1})'(s)^2+\gamma'\big( \phi^{-1}(s) \big)(\phi^{-1})''(s) \Big)'\\
        &=\gamma'''\big( \phi^{-1}(s) \big)(\phi^{-1})'(s)^3+ v\\
        &=\frac{ \gamma'''\big( \phi^{-1}(s) \big) }{ \| \gamma'(t) \|^3 }+v&&\text{par \eqref{Eavpemuetfpnorm}}
    \end{aligned}
\end{equation}
où $v$ est un élément de $\langle \gamma''\big( \phi^{-1}(s) \big),\gamma'\big( \phi^{-1}(s) \big)\rangle$. Le vecteur $v$ est donc perpendiculaire à $\gamma'\times \gamma''$ et donc à $\gamma_N'\times \gamma_N''$ à cause de la relation \eqref{eqProdvectogpgpp} qui montre que $\gamma'\times \gamma''$ est parallèle à $\gamma_N'\times \gamma_N''$. De ce fait, lorsque nous calculons $(\gamma_N'\times \gamma_N'')\cdot \gamma_N'''$, la partie $v$ de $\gamma_N'''$ n'entre pas en ligne de compte.

Nous avons donc le calcul suivant, en remplaçant les diverses occurrences de $\gamma_N'\times \gamma_N''$ par sa valeur \eqref{eqProdvectogpgpp} en termes de $\gamma$,
\begin{equation}
    \begin{aligned}[]
        t(s)&=-\frac{ (\gamma_N'\times \gamma_N'')\cdot \gamma_N''' }{ \| \gamma_N'\times \gamma_N'' \|^2 }\\
        &=-\frac{ (\gamma_N'\times \gamma_N'')\cdot \gamma'''(t) }{ \| \gamma_N'\times \gamma_N'' \|^2\,\| \gamma'(t) \|^2 }\\
        &=-\frac{ (\gamma'\times \gamma'')\cdot \gamma''' }{ \| \gamma'\times \gamma'' \|^2 }.
    \end{aligned}
\end{equation}
Dans cette expression, il est sous-entendu que tous les $\gamma_N$ sont fonctions de $s$ et tous les $\gamma$ sont fonction de $t$ où $s$ et $t$ sont liés par $s=\phi(t)$.

Ce que nous avons prouvé est le
\begin{theorem}
    Pour tout représentant $(I,\gamma)$, les éléments métriques $(\tau,\nu,\beta,c,t)$ au point $\gamma(t)$ s'expriment en fonction de $\gamma(t)$, $\gamma'(t)$, $\gamma''(t)$ et $\gamma'''(t)$.
\end{theorem}

\begin{lemma}
    Si \( \gamma\) est le graphe de la fonction \( y\) alors la courbure de \( \gamma\) est donnée par la formule
    \begin{equation}
        c\big( \gamma(t) \big)=\frac{ | y''(t) | }{ \big( 1+y'(t)^2 \big)^{3/2} }
    \end{equation}
\end{lemma}

\begin{proof}
    Nous avons :
    \begin{subequations}
        \begin{align}
            \gamma(t)=\big( t,y(t) \big)\\
            \gamma'(t)=\big( 1,y'(t) \big)\\
            \gamma''(t)=\big( 0,y''(t) \big).
        \end{align}
    \end{subequations}
    Il s'agit maintenant seulement d'utiliser la formule \eqref{EqCourburetermf} en se souvenant comment on calcule un produit vectoriel\footnote{Définition~\ref{DEFooTNTNooRjhuJZ}.} :
    \begin{equation}
        \gamma'\times \gamma''=
        \begin{vmatrix}
            e_1&e_2&e_3\\
            1&y'&0\\
            0&y''&0
        \end{vmatrix}=y''.
    \end{equation}
\end{proof}

%+++++++++++++++++++++++++++++++++++++++++++++++++++++++++++++++++++++++++++++++++++++++++++++++++++++++++++++++++++++++++++
\section{Tracer des courbes paramétriques dans $\eR^2$}     \label{SecTracerParmCourbe}
%+++++++++++++++++++++++++++++++++++++++++++++++++++++++++++++++++++++++++++++++++++++++++++++++++++++++++++++++++++++++++++

Nous allons maintenant voir comment les concepts introduits nous aident à effectivement tracer des courbes dans le plan. Les courbes que nous regardons sont de la forme $\gamma(t)=\big( x(t),y(t) \big)$, et nous supposons que ces fonctions soient suffisamment régulières (disons trois fois continument dérivables). Nous ne supposons pas que la courbe soit donnée en coordonnées normales, en particulier, $\gamma''(t)$ n'est pas le vecteur normal en $\gamma(t)$.

La notion clef qui va jouer est le \defe{cercle osculateur}{osculateur (cercle)} de la courbe $\gamma$ au point $\gamma(t)$. Sans rentrer dans les détails, disons que c'est le cercle qui «colle» le mieux possible la courbe. Le rayon de ce cercle est le rayon de courbure :
\begin{equation}
    R(t)=\frac{ \| \gamma(t) \|^3 }{ \| \gamma'(t)\times\gamma''(t) \| }.
\end{equation}
En pratique, le produit vectoriel se calcule comme ceci :
\begin{equation}
    \gamma'(t)\times\gamma''(t)=\begin{vmatrix}
        e_x &   e_y &   e_z \\
        x'(t)   &   y'(t)   &   0   \\
        x''(t)  &   y''(t)  &   0
    \end{vmatrix}=(x'y''-x''y')e_z.
\end{equation}
Le centre du cercle osculateur va se trouver quelque part sur la normale. Le vecteur normal est donné par
\begin{equation}
    n(t)=J\frac{\gamma'(t) }{ \| \gamma'(t) \| }
\end{equation}
où $J$ est la rotation d'angle $\frac{ \pi }{2}$ :
\begin{equation}
    J\begin{pmatrix}
        x'(t)   \\
        y'(t)
    \end{pmatrix}=
    \begin{pmatrix}
        0   &   1   \\
        -1  &   0
    \end{pmatrix}\begin{pmatrix}
        x'(t)   \\
        y'(t)
    \end{pmatrix}=\begin{pmatrix}
        y'(t)   \\
        -x'(t)
    \end{pmatrix}.
\end{equation}
Cela nous laisse deux possibilités pour le centre du cercle osculateur : $\gamma(t)+R(t)n(t)$ ou bien $\gamma(t)-R(t)n(t)$. Il faut savoir de quel côté de la courbe est situé le centre du cercle osculateur. Il faut choisir le côté de la concavité, c'est-à-dire le côté de la dérivée seconde.

\newcommand{\CaptionFigQuelCote}{De quel côté de $\gamma'(t)$ se trouvent $n(t)$ et $-n(t)$ ?}
\input{auto/pictures_tex/Fig_QuelCote.pstricks}

La difficulté maintenant est de savoir qui de $n(t)$ ou $-n(t)$ est du côté de $\gamma''(t)$. Il faut savoir si $n(t)$ est du même côté de la droite tangente que $\gamma''(t)$ ou non. Par construction, si nous regardons la figure ~\ref{LabelFigQuelCote}, le vecteur $n(t)$ sera toujours à gauche de $\gamma'(t)$. Le fait que $\gamma''(t)$ soit à gauche ou à droite de $\gamma'(t)$ est donné par le signe du produit vectoriel $\gamma'(t)\times \gamma''(t)$. Si ce produit vectoriel est positif, il faut choisir $-n(t)$ et s'il est négatif, il faut choisir $n'(t)$.

Le truc pour obtenir le signe de $x'y''-x''y'$ est de faire
\begin{equation}
    \frac{ (\gamma'\times\gamma'')\cdot e_z}{\| \gamma'\times\gamma'' \|}.
\end{equation}

Le centre de courbure sera donc situé à la position
\begin{equation}
    \Omega(t)=\gamma(t)-n(t)\frac{ \| \gamma(t) \|^3 }{ \| \gamma'(t)\times\gamma''(t) \|^2 } (\gamma'\times\gamma'')\cdot e_z
\end{equation}
Nous pouvons écrire cela plus explicitement en nous souvenant que $\gamma'\times\gamma''=(x'y''-x''y')e_z$, par conséquent $\frac{ (\gamma'\times\gamma'')\cdot e_z}{\| \gamma'\times\gamma'' \|^2}=\frac{1}{ x'y''-x''y' }$. Nous avons
\begin{subequations}
    \begin{align}
        \Omega_x(t)&=x(t)-y'(t)\frac{ x'^2+y'^2 }{ x'y''-x''y' }\\
        \Omega_y(t)&=y(t)+x'(t)\frac{ x'^2+y'^2 }{ x'y''-x''y' }.
    \end{align}
\end{subequations}

Quelques exemples de cercles osculateurs sont sur la figure~\ref{LabelFigOsculateur}.
\newcommand{\CaptionFigOsculateur}{Exemple de cercles osculateurs.}
\input{auto/pictures_tex/Fig_Osculateur.pstricks}

% TODO : Écrire quelque chose sur les points de rebroussement et d'inflexion, ainsi que sur les asymptotes.
%   Quand ce sera fait, il y a des choses à décommenter dans l'exerice exoCourbesSurfaces0002.tex

%+++++++++++++++++++++++++++++++++++++++++++++++++++++++++++++++++++++++++++++++++++++++++++++++++++++++++++++++++++++++++++
\section{Courbes planes}
%+++++++++++++++++++++++++++++++++++++++++++++++++++++++++++++++++++++++++++++++++++++++++++++++++++++++++++++++++++++++++++

\begin{definition}
    Une courbe \( \gamma\colon \mathopen[ a , b \mathclose]\to \eR^n\) est \defe{fermée}{courbe!fermée} si \( \gamma(a)=\gamma(b)\). Elle est \defe{simple}{courbe!simple} si \( \gamma(t)\neq \gamma(t')\) dès que \( t,t'\in\mathopen] a , b \mathclose[\) et \( t\neq  t'\).
\end{definition}

\begin{definition}      \label{DEFooSAZTooZGQrQG}
    Nous disons qu'une courbe fermée est continue, de classe \( C^1\), de classe \( C^2\) ou autre condition de régularité si son extension périodique comme application \( \gamma\colon \eR\to \eR^2\) a cette régularité.
\end{definition}

%---------------------------------------------------------------------------------------------------------------------------
\subsection{Angle}
%---------------------------------------------------------------------------------------------------------------------------

\begin{lemma}[\cite{ooIEJXooIYpBbd}]        \label{LEMooUECMooNBDGiR}
    Soient des courbes régulières \( \gamma\) et \( \sigma\) de classe \( C^2\) de l'intervalle ouvert \( I\) vers \( \eR^2\). Soit \( \theta_0\in \eR\) tel que
    \begin{subequations}
        \begin{align}
            \frac{ \gamma'(t_0)\cdot\sigma'(t_0) }{ \| \gamma'(t_0) \|\| \sigma'(t_0) \| }=\cos(\theta_0)\\
            \frac{ \gamma'(t_0)\cdot J\sigma'(t_0) }{ \| \gamma'(t_0) \|\| \sigma'(t_0) \| }=\cos(\theta_0).
        \end{align}
    \end{subequations}
    Alors il existe une unique fonction différentiable \( \theta\colon I\to \eR\)
    \begin{subequations}
        \begin{align}
            \frac{ \gamma'(t)\cdot\sigma'(t) }{ \| \gamma'(t) \|\| \sigma'(t) \| }=\cos\big( \theta(t) \big)\\
            \frac{ \gamma'(t)\cdot J\sigma'(t) }{ \| \gamma'(t) \|\| \sigma'(t) \| }=\sin\big( \theta(t) \big).
        \end{align}
    \end{subequations}
\end{lemma}

\begin{proof}
    Il suffit de prendre
    \begin{equation}
        f(t)=\frac{ \gamma'(t)\cdot\sigma'(t) }{ \| \gamma'(t) \|\| \sigma'(t) \| }
    \end{equation}
    et
    \begin{equation}
        g(t)=\frac{ \gamma'(t)\cdot J\sigma'(t) }{ \| \gamma'(t) \|\| \sigma'(t) \| }
    \end{equation}
    dans la proposition~\ref{PROPooWZFGooMVLtFz}. Ces courbes sont de classe \( C^1\) parce que \( \gamma\) et \( \sigma\) sont de classe \( C^2\).
\end{proof}

%---------------------------------------------------------------------------------------------------------------------------
\subsection{Courbure signée}
%---------------------------------------------------------------------------------------------------------------------------

Nous avons déjà défini la courbure d'une courbe en la définition~\ref{DefCourbureNormleUnit}. Nous introduisons maintenant la courbure signée qui est propre à la dimension deux.

\begin{definition}      \label{DEFooTSJXooTIyRXf}
    La \defe{structure complexe}{structure!complexe} sur \( \eR^2\) est l'application
    \begin{equation}
        \begin{aligned}
            J\colon \eR^2&\to \eR^2 \\
            (x,y)&\mapsto (-y,x).
        \end{aligned}
    \end{equation}
\end{definition}
\ifbool{isGiulietta}{Cette définition est un cas très particulier des structures complexes sur les variétés, voir la définition~\ref{DefSymHermMGKalg}.}{}

\begin{definition}      \label{DEFooJFWEooXcIVUs}
    La \defe{courbure signée}{courbure!signée} de la courbe \( \gamma\colon I\to \eR^2\) (\( I\) est un intervalle dans \( \eR\)) est la fonction
    \begin{equation}        \label{EQooWOUQooXrVzGx}
        \kappa(t)=\frac{ \gamma''(t)\cdot J\gamma'(t) }{ \| \gamma'(t) \|^3 }
    \end{equation}
    où \( J\) est la structure complexe de la définition~\ref{DEFooTSJXooTIyRXf}.
\end{definition}
Cette définition est motivée par le fait qu'en identifiant \( \eR^2\) à \( \eC\), l'application \( J\) revient à l'application \( z\mapsto iz\).

Si \( v,w\in \eR^2\) nous avons formellement
\begin{equation}
    v\times w=-(v\cdot Jw)e_3.
\end{equation}
En particulier pour tout \( v\in \eR^2\) nous avons
\begin{equation}
    v\cdot Jv=0.
\end{equation}

\begin{lemma}
    Soir une courbe régulière \( \gamma\colon \mathopen[ a , b \mathclose]\to \eR^2\) et un difféomorphisme \( h\colon \mathopen[ c , d \mathclose]\to \mathopen[ a , b \mathclose]\). Si nous posons \( \sigma=\gamma\circ h\) alors
    \begin{equation}        \label{EQooSQNMooUKGhPd}
        \kappa_{\sigma}(u)=\signe\big( h'(u) \big)\kappa_{\gamma}\big( h(u) \big).
    \end{equation}
\end{lemma}

\begin{proof}
    Nous utilisons la définition \eqref{EQooWOUQooXrVzGx} de la courbure signée. La règle de dérivation en chaine donne :
    \begin{subequations}
        \begin{align}
            \sigma'(u)&=\gamma'\big( h(u) \big)h'(u)\\
            \sigma''(u)&=\gamma''\big( h(u) \big)h'(u)^2+\gamma'\big( h(u) \big)h''(u).
        \end{align}
    \end{subequations}
    La numérateur de \( \kappa_{\sigma}(u)\) est :
    \begin{equation}
        (\gamma''\circ h)h'^2\cdot J(\gamma'\circ h)h'+h''(\gamma'\circ h)\cdot J(\gamma'\cdot h)h'
    \end{equation}
    dont le second terme est nul parce que \( v\cdot Jv=0\). Il nous reste donc
    \begin{equation}
        \kappa_{\sigma}(u)=\frac{ (h')^3 }{ | h' |^3} \frac{ (\gamma''\circ h)\cdot J(\gamma'\circ h) }{ \| \gamma'\circ h \|^3 }=\signe(h')\kappa_{\gamma}\big( h(u) \big).
    \end{equation}
\end{proof}

\begin{lemma}[\cite{ooIEJXooIYpBbd}]        \label{LEMooKPORooEGJCRm}
    Si \( \gamma_N\) est un arc paramétré normal, alors
    \begin{equation}
        \gamma_N''(s)=\kappa(s)J\gamma_N'(s).
    \end{equation}
\end{lemma}

\begin{proof}
    Vu que le paramétrage est normal, \( \gamma'_N\cdot \gamma'_N=1\), et en dérivant, \( \gamma_N''\cdot\gamma_N'=0\). Donc \( \gamma''_N\) est un multiple de \( J\gamma'\). En tenant compte du fait que le paramétrage est normal, la courbure est
    \begin{equation}
        \kappa(s)=\gamma_N''(s)\cdot J\gamma'(s).
    \end{equation}
    En y injectant \( \gamma''(s)=\lambda(s)J\gamma'(s)\) nous trouvons
    \begin{equation}
        \kappa(s)=\lambda(s)J\gamma'(s)\cdot J\gamma'(s)=\lambda(s).
    \end{equation}
    Donc le facteur de proportionnalité est \( \kappa(s)\).
\end{proof}

\begin{theorem}     \label{THOooDLDVooFQnLWn}
    Soit une courbe \( \gamma\colon I\to \eR^2\) de classe \( C^2\).
    \begin{enumerate}
        \item
            \( \gamma\) est une partie de droite si et seulement si \( \kappa(t)=0\) pour tout \( t\).
        \item
            \( \gamma\) est une partie d'un cercle de rayon \( r>0\) si et seulement si \( | \kappa(s) |=\frac{1}{ r }\).
    \end{enumerate}
\end{theorem}

\begin{proof}
    Si \( \gamma\) est une droite, la dérivée seconde est nulle et la courbure est nulle. Supposons pour la réciproque que \( \kappa(t)=0\) pour tout \( t\). Nous utilisant un paramétrage normal de \( \gamma\), ce qui ne change pas que la courbure reste nulle. Nous avons par le lemme~\ref{LEMooKPORooEGJCRm} que \( \gamma''(s)=0\) et donc l'existence de \( a,b\in \eR^2\) tels que \( \gamma(t)=at+b\).

    Passons au cas du cercle. Si \( \gamma\) est un cercle, le paramétrage normal est
    \begin{subequations}
        \begin{align}
            \gamma(t)&=R\begin{pmatrix}
                \cos(\frac{ t }{ R })    \\
                \sin(\frac{ t }{ R })
            \end{pmatrix}\\
            \gamma'(t)&=\begin{pmatrix}
                -\sin(\frac{ t }{ R })    \\
                \cos(\frac{ t }{ R })
            \end{pmatrix}\\
            \gamma''(t)=-\frac{1}{ R }\begin{pmatrix}
                \cos(\frac{ t }{ R })    \\
                \sin(\frac{ t }{ R })
            \end{pmatrix}.
        \end{align}
    \end{subequations}
    Avec tout cela nous avons \( \kappa(s)=\gamma''(s)\cdot J\gamma'(s)=\frac{1}{ R }\).

    Nous supposons enfin que \( \kappa(t)=1/R\) et que le paramétrage soit normal (encore une fois, un reparamétrage ne change pas la courbure lorsqu'elle est constante). Nous définissons la courbe
    \begin{equation}
        \begin{aligned}
            \beta\colon \mathopen[ b , c \mathclose]&\to \eR^2 \\
            t&\mapsto \gamma(t)+rJ\gamma'(t).
        \end{aligned}
    \end{equation}
    Nous avons \( \beta'(t)=\gamma'(t)+rJ\gamma''(t)\). Mais par le lemme~\ref{LEMooKPORooEGJCRm} nous avons \( \gamma''=kJ\gamma'=\frac{1}{ r }J\gamma'\). Donc
    \begin{equation}
        \beta'(t)=\gamma'(t)-\gamma'(t)=0.
    \end{equation}
    Du coup \( \beta\) est constante : \( \beta(t)=a\). Alors \( a=\gamma(t)+rJ\gamma'(t)\) et en particulier
    \begin{equation}
        \| \gamma(t)-a \|=\| rJ\gamma'(t) \|=r.
    \end{equation}
    Donc effectivement \( \gamma\) reste sur un cercle de rayon \( r\) et de centre \( a\).
\end{proof}

\begin{definition}[\cite{ooIEJXooIYpBbd}]
    La \defe{courbure totale}{courbure!totale} de la courbe \( \gamma\colon \mathopen[ a , b \mathclose] \to \eR^2 \) est le nombre
    \begin{equation}        \label{EQooTIFWooQflOfd}
        K=\int_a^b\kappa(t)\| \gamma'(t) \|dt.
    \end{equation}
\end{definition}

\begin{lemma}
    La courbure signée ne change pas sous reparamétrage positif, et change de signe sous reparamétrage négatif.
\end{lemma}

\begin{proof}
    Soit la courbe \( \gamma\colon \mathopen[ a , b \mathclose]\to \eR^2\) et un difféomorphisme \( h\colon \mathopen[ a , b \mathclose]\to \mathopen[ c , d \mathclose]\). Il s'agit d'intégrer la relation \eqref{EQooSQNMooUKGhPd} en effectuant le changement de variables \( t=h(u)\) :
    \begin{subequations}
        \begin{align}
            K_{\gamma}&=\int_a^b\kappa_{\gamma}(t)\| \gamma'(t) \|dt\\
            &=\int_c^d\underbrace{\kappa_{\gamma}\big( h(u) \big)}_{\kappa_{\sigma}(u)}\| \gamma'\big( h(u) \big) \|h'(u)du.
        \end{align}
    \end{subequations}
    En utilisant le fait que \( \sigma'(u)=\gamma'\big( h(u) \big)h'(u)  \) nous avons alors
    \begin{subequations}
        \begin{align}
            K_{\gamma}&=\int_{c}^d\kappa_{\sigma}(u)\left\| \frac{ \sigma'(u) }{ h'(u) }   \right\|h'(u)du\\
            &=\int_c^d\kappa_{\sigma}(u)\| \sigma'(u) \|\frac{ h'(u) }{ | h'(u) | }du\\
            &=\signe(h')K_{\sigma}.
        \end{align}
    \end{subequations}
\end{proof}

\begin{lemmaDef}[\cite{ooIEJXooIYpBbd}]     \label{LEMDEFooLPWJooAnWZjb}
    Soit une courbe régulière \( \gamma\colon I\to \eR^2\) de classe \( C^2\) et \( t_0\) dans l'intérieur de \( I\). Soit \( \theta_0 \in \eR\) tel que
    \begin{equation}
        \frac{ \gamma'(t_0)  }{ \| \gamma'(t_0) \| }=\big( \cos(\theta_0),\sin(\theta_0) \big).
    \end{equation}
    Alors il existe une unique fonction différentiable \( \theta\colon I\to \eR\) telle que \( \theta(t_0)=\theta_0\) et
    \begin{equation}
        \frac{ \gamma'(t) }{ \| \gamma'(t) \| }=\begin{pmatrix}
            \cos\big( \theta(t) \big)    \\
            \sin\big( \theta(t) \big)
        \end{pmatrix}
    \end{equation}
    pour tout \( t\in I\).

    Cette fonction est l'\defe{angle}{angle!d'une courbe} de \( \gamma\) déterminé par \( \theta_0\).
\end{lemmaDef}

\begin{proof}
    Soit \(  \beta(t)=(t,0)  \); alors \( \beta'(t)=(1,0)\) et nous avons
    \begin{subequations}
        \begin{align}
            \gamma'\cdot \beta'=\gamma_x'\\
            \gamma'\cdot J\beta'=\gamma_y'.
        \end{align}
    \end{subequations}
    Par la proposition~\ref{LEMooUECMooNBDGiR} il existe une unique fonction \( \theta\colon I\to \eR\) telle que \( \theta(t_0)=\theta_0\) et
    \begin{subequations}
        \begin{numcases}{}
            \cos\big( \theta(t) \big)=\frac{ \gamma'\cdot \beta' }{ \| \gamma' \|\| \beta' \| }=\frac{ \gamma_x' }{ \| \gamma' \| }\\
            \sin\big( \theta(t) \big)=\frac{ \gamma'\cdot J\beta' }{ \| \gamma' \|\| \beta' \| }=\frac{ \gamma_y' }{ \| \gamma' \| }.
        \end{numcases}
    \end{subequations}
    Une telle fonction est bien celle que l'on demande ici.
\end{proof}

\begin{lemma}       \label{LEMooWLAUooKetUiW}
    Si \( \gamma\) est une courbe régulière de classe \( C^2\), alors sa courbure et son angle vérifient la relation
    \begin{equation}
        \theta'(t)=\| \gamma'(t) \|\kappa(t).
    \end{equation}
\end{lemma}

\begin{proof}
    Par définition de l'angle (lemme~\ref{LEMDEFooLPWJooAnWZjb}) nous avons
    \begin{equation}
        \frac{ \gamma'(t) }{ \| \gamma'(t) \| }=\big( \cos\theta(t),\sin\theta(t) \big).
    \end{equation}
    Dérivant cela,
    \begin{subequations}
        \begin{align}
            \frac{ \gamma''(t) }{ \| \gamma'(t) \| }+\gamma'(t)\frac{ d  }{ dt }\left( \frac{1}{ \| \gamma'(t) \| } \right)&=\theta'(t)
            \begin{pmatrix}
                -\sin\big( \theta(t) \big)\\
                \cos\big( \theta(t) \big)
            \end{pmatrix}\\
            &=\theta'(t)J\begin{pmatrix}
                \cos\big( \theta(t) \big)   \\
                \sin\big( \theta(t) \big)
            \end{pmatrix}\\
            &=\theta'(t)\frac{ J\gamma'(t) }{ \| \gamma'(t) \| }.
        \end{align}
    \end{subequations}
    Nous prenons le produit scalaire de cette égalité avec \( J\gamma'\) en tenant compte du fait que \( \gamma'\cdot J\gamma'=0\) :
    \begin{equation}
        \frac{ \gamma''\cdot J\gamma' }{ \| \gamma' \| }=\frac{ \theta' }{ \| \gamma' \| }J\gamma'\cdot J\gamma'.
    \end{equation}
    En remarquant que \( Jv\cdot Jv=\| v \|^2\) nous trouvons \( \theta'\| \gamma' \|=\| \gamma' \|^2\kappa_{\gamma}\) et donc
    \begin{equation}
        \theta'(t)=\| \gamma'(t) \|\kappa_{\gamma}(t),
    \end{equation}
    ce qu'il fallait prouver.
\end{proof}
Ce lemme nous fournit la formule attendue pour la courbure totale.

\begin{lemma}[\cite{ooIEJXooIYpBbd}]
    Soit une courbe régulière de classe \( C^2\) \( \gamma\colon \mathopen[ a , b \mathclose]\to \eR^2\). Sa courbure totale est donnée par
    \begin{equation}
        K=\theta(b)-\theta(a).
    \end{equation}
\end{lemma}

\begin{proof}
    Il suffit de remplacer dans la définition \eqref{EQooTIFWooQflOfd} de la courbure totale l'intégrante par son expression du lemme~\ref{LEMooWLAUooKetUiW} :
    \begin{equation}
        K=\int_a^b\kappa(t)\| \gamma'(t) \|dt=\int_a^b\theta'(t)dt=\theta(b)-\theta(a)
    \end{equation}
    par le théorème~\ref{ThoRWXooTqHGbC}.
\end{proof}

%---------------------------------------------------------------------------------------------------------------------------
\subsection{Degré, indice et homotopie}
%---------------------------------------------------------------------------------------------------------------------------

\begin{definition}[\cite{ooIEJXooIYpBbd}]
    Le \defe{nombre de tours}{nombre!tours d'une courbe plane} d'une courbe fermée de classe \( C^1\) \( \gamma\colon \mathopen[ a , b \mathclose]\to \eR^2\) est le nombre
    \begin{equation}
        \Turn(\gamma)=\frac{1}{ 2\pi }\int_a^b\kappa(t)\| \gamma'(t) \|dt
    \end{equation}
    où \( \kappa\) est la courbure signée définie en~\ref{DEFooJFWEooXcIVUs}.
\end{definition}

\begin{lemmaDef}        \label{DEFooTKBUooNVcheO}
    Soit une application continue \( \phi\colon S^1\to S^1\). Un \defe{relèvement}{relèvement} de \( \sigma\) est une application \( \varphi\colon \eR\to \eR\) une application continue telle que
    \begin{equation}
        \phi\big( \cos(t),\sin(t) \big)=\begin{pmatrix}
            \cos\big( \varphi(t) \big)    \\
            \sin\big( \varphi(t) \big)
        \end{pmatrix}.
    \end{equation}
    Le \defe{degré}{degré!application \( S^1\to S^1\)} est l'entier \( \deg(\phi)\) tel que
    \begin{equation}
        \varphi(2\pi)-\varphi(0)=2\deg(\phi)\pi.
    \end{equation}
    Ce nombre ne dépend pas du choix du relèvement \( \varphi\).
\end{lemmaDef}

\begin{proof}
    Soient \( \varphi_1\) et \( \varphi_2\), deux applications qui satisfont les contraintes. Nous avons une fonction continue \( n\colon S^1\to \eR\) telle que
    \begin{equation}
        \varphi_1(t)-\varphi_2(t)=2\pi n(t).
    \end{equation}
    La fonction \( n\) ne pouvant prendre que des valeurs entières et étant continue, elle est constante. Par conséquent
    \begin{equation}
        \varphi_1(2\pi)-\varphi_1(0)=\varphi_2(2\pi)-\varphi_2(0),
    \end{equation}
    ce que nous voulions.
\end{proof}

\begin{definition}      \label{DEFooOCUQooUAlbLo}
    Soit une courbe fermée \( \gamma\colon \mathopen[ 0 , L \mathclose]\to \eR^2\) de classe \( C^1\). Nous posons
    \begin{equation}
        \Phi_{\gamma}(t)=\frac{ \tilde \gamma'(t) }{ \| \tilde \gamma'(t) \| }
    \end{equation}
    où \( \tilde \gamma(u)=\gamma\big( \frac{ L }{ 2\pi }u \big)\). L'\defe{indice de rotation}{indice!de rotation} de \( \gamma\) est le degré de \( \Phi_{\gamma}\), c'est-à-dire
    \begin{equation}
        \IR(\gamma)=\deg(\Phi_{\gamma}).
    \end{equation}
\end{definition}
Notons que dans cette définition, \( \Phi_{\gamma}\) n'est rien d'autre que le vecteur unitaire tangent à \( \gamma\), ramené à \( \mathopen[ 0 , 2\pi \mathclose]\).

\begin{proposition}     \label{PROPooXHSDooDDnlJQ}
    Pour une courbe fermée de classe\footnote{La dérivée seconde arrive dans la définition de la courbure; il faudrait donc supposer au moins \( C^2\) pour avoir la continuité de la courbure.} \( C^{2}\), l'indice de rotation est égal au nombre de tours.
\end{proposition}

\begin{proof}
    Soit \( \gamma\colon \mathopen[ 0 , L \mathclose]\to \eR^2\) une courbe vérifiant les hypothèses. Par définition, \( \IR(\gamma)=\deg(\Phi_{\gamma})\) où
    \begin{equation}        \label{EQooGTAPooDRCFPG}
        \Phi_{\gamma}(t)=\frac{ \tilde \gamma'(t) }{ \| \tilde \gamma'(t) \| }=\begin{pmatrix}
            \cos \tilde \theta(t)    \\
            \sin \tilde \theta(t)
        \end{pmatrix}
    \end{equation}
    où nous avons noté \( \tilde \theta\) l'angle tournant\footnote{Définition~\ref{LEMDEFooLPWJooAnWZjb}.} de \( \tilde \gamma\) et \( \theta\) celui de \( \gamma\). Vu que \( \tilde \theta\) vérifie les hypothèses de la définition~\ref{DEFooTKBUooNVcheO} nous pouvons calculer le degré de \( \Phi_{\gamma}\) par
    \begin{equation}        \label{EQooKNCJooTwTBMO}
        \deg(\Phi_{\gamma})=\frac{1}{ 2\pi }\big( \tilde \theta(2\pi)-\tilde \theta(0) \big)=\frac{1}{ 2\pi }\int_0^{2\pi}\tilde \theta'(s)ds.
    \end{equation}
    Il faut trouver le lien entre \( \theta\) et \( \tilde \theta\). Pour cela nous notons que
    \begin{equation}
        \frac{ \tilde \gamma'(t) }{ \| \tilde \gamma'(t) \| }=\frac{ \gamma'\left( \frac{ L }{ 2\pi }t \right) }{ \| \gamma'\left( \frac{ L }{ 2\pi } \right) \| }.
    \end{equation}
    En comparant avec \eqref{EQooGTAPooDRCFPG} il vient
    \begin{equation}
        \tilde \theta(t)=\theta\big( \frac{ L }{ 2\pi }t \big)
    \end{equation}
    et
    \begin{equation}
        \tilde \theta'(s)=\frac{ L }{ 2\pi }\theta'\left( \frac{ L }{ 2\pi }s \right).
    \end{equation}
    Le changement de variables \( t=\frac{ L }{ 2\pi }s\) est donc tout vu dans l'intégrale \eqref{EQooKNCJooTwTBMO} :
    \begin{equation}
        \deg(\Phi_{\gamma})=\frac{1}{ 2\pi }\int_0^{2\pi}\frac{ L }{ 2\pi }\theta'\big( \frac{ L }{ 2\pi }s \big)ds=\frac{1}{ 2\pi }\int_0^L\theta'(t)dt=\frac{1}{ 2\pi }\int_0^L\kappa_{\gamma}(t)\| \gamma'(t) \|dt=\Turn(\gamma)
    \end{equation}
    où nous avons aussi utilisé le lemme~\ref{LEMooWLAUooKetUiW} qui donne le lien entre \( \theta'\) et \( \kappa\).
\end{proof}

\begin{definition}[homotopie de chemins fermés]     \label{DEFooHJQTooYUFcee}
    Les courbes fermées \( \gamma_0,\gamma_1\colon \mathopen[ 0 , L \mathclose]\to Y\) (\( Y\) est un espace topologique) sont \defe{homotopes}{homotopie} s'il existe une application continue
    \begin{equation}
        F\colon \mathopen[ 0 , 1 \mathclose]\times \mathopen[ 0 , L \mathclose]\to Y
    \end{equation}
    telle que
    \begin{enumerate}
        \item
           \( F(0,t)=\gamma_0(t)\) pour tout \( t\),
       \item
           \( F(1,t)=\gamma_1(t)\) pour tout \( t\),
       \item
           \( F(u,L)=F(u,0)\) pour tout \( u\).
    \end{enumerate}
    L'application \( F\) est l'homotopie entre \( \gamma_0\) et \( \gamma_1\).
\end{definition}

\begin{proposition}[Homotopie, degré et indice\cite{ooIEJXooIYpBbd}]      \label{PROPooZIAKooHqtnZj}
    Il y a deux résultats à ne pas confondre.
    \begin{enumerate}
        \item   \label{ITEMooLEHFooXEyTHY}
            Si \( \phi_i\colon S^1\to S^1\) sont homotopes, alors \( \deg(\phi_1)=\deg(\phi_2)\).
        \item
            Si \(\gamma_i\colon \mathopen[ 0 , L \mathclose]\to \eR^2 \) sont homotopes, alors \( \IR(\phi_1)=\IR(\phi_2)\).
    \end{enumerate}
\end{proposition}

\begin{proof}
    Nous allons décomposer la preuve en deux parties.
    \begin{subproof}
    \item[Le degré pour les applications \( S^1\to S^1\)]
        Soient deux applications homotopes \( \gamma_i\colon S^1\to S^1\). Si \( F\) est l'homotopie\footnote{Définition~\ref{DEFooHJQTooYUFcee}.} entre \( \gamma_0\) et \( \gamma_1\), nous posons \( \gamma_u(t)=F(u,t)\) qui est encore une courbe fermée \( \gamma_u\colon S^1\to S^2\). Nous pouvons donc considérer le degré de \( \gamma_u\). C'est un entier \( n_u\) qui vérifie
        \begin{equation}
            \tilde \gamma_u(2\pi)-\tilde \gamma_u(0)=2\pi n_u.
        \end{equation}
        Le membre de gauche est une fonction continue de \( u\); dans le membre de droite \( n_u\) ne pouvant prendre que des valeurs entières, elle est alors constante.
    \item[Indice de rotation pour des courbes dans $\eR^2$]
        Soient maintenant \( \gamma_0\) et \( \gamma_1\) deux courbes homotopes de \( \mathopen[ 0 , L \mathclose]\) dans \( \eR^2\). Par définition, \( \IR(\gamma_i)=\deg(\Phi_{\gamma_i})\). Prouvons alors que \( \Phi_{\gamma_0}\) et homotope à \( \Phi_{\gamma_1}\). De cette façon, la première partie de la preuve conclura à
        \begin{equation}
            \IR(\gamma_0)=\deg(\Phi_{\gamma_0})=\deg(\Phi_{\gamma_1})=\IR(\gamma_1).
        \end{equation}
        Nous savons que
        \begin{equation}
            \Phi_{\gamma}=\frac{ \gamma'\left( \frac{ L }{ 2\pi }t \right) }{ \| \gamma'\left( \frac{ L }{ 2\pi }t \right) \| },
        \end{equation}
        donc en posant
        \begin{equation}
            \Phi_F(u,t)=\frac{ \gamma_u'\big( \frac{ L }{ 2\pi }t \big) }{ \| \gamma'_u\big( \frac{ L }{ 2\pi }t \big) \| }
        \end{equation}
        nous avons une homotopie entre \( \Phi_{\gamma_0}\) et \( \Phi_{\gamma_1}\).
    \end{subproof}
\end{proof}

\begin{theorem}[\cite{ooIEJXooIYpBbd}]      \label{THOooEQWOooBCRMMZ}
    Le nombre de tours d'une courbe simple fermée de classe \(  C^{2}\) dans \( \eR^2\) est \( \pm 1\).
\end{theorem}

\begin{proof}
    Soit une telle courbe \( \gamma\) et son image \( \Gamma\).
    \begin{subproof}
        \item[Choix d'un point et d'une tangente]

            Si \( \ell\) est une droite dans le plan, soit \( p\) le point de \( \Gamma\) le plus proche de \( \ell\). Il n'est peut-être pas unique, mais la parallèle à \( \ell\) passant par \( p\) est une tangente à \( \Gamma\) telle que tout \( \Gamma\) se trouve d'un seul côté de \( \ell_p\). Pour voir cela, il suffit de choisir un système d'axes pour lequel \( \ell\) est l'axe \( y=0\). La distance entre \( \ell\) et les points de \( \Gamma\) est donnée par \( \gamma_y\), et donc les extrémums sont atteints là où \( \gamma_y'=0\), c'est-à-dire pour les points sur lesquels la tangente est parallèle à \( \ell\).

            Étant donné qu'il y a (au moins) un maximum et un minimum distincts, pour pouvons choisir le point \( p\) de telle sorte que pour tout point \( q\in \Gamma\), l'angle du vecteur \( q-p\) soit entre \( \alpha_0\) et \( \alpha_0+\pi\) et non entre \( \alpha_0-pi\) et \( \alpha_0\). Ce choix revient à choisir \( p\) de telle sorte que \( \Gamma\) soit d'un côté ou de l'autre de \( \ell_p\).

            Pour simplifier les notations plus tard nous choisissons \( \ell_p\) horizontale, de telle sorte que \( \alpha_0=0\), et que \( \Gamma\) est au dessus de \( \ell_p\). Les angles des vecteurs \( q-p\) pour \( q\in \Gamma\) sont donc tous entre \( 0\) et \( \pi\).

        \item[Définition de \( \Sigma\)]

            Soit \( L\) la longueur\footnote{Définition~\ref{DEFooDNZWooXmxhsU}.} de \( \gamma\) et \( \beta\), un paramétrage de \( \gamma\) telle que \( \beta(0)=p\). Nous considérons le triangle
            \begin{equation}
                \mT=\{ (t_1,t_2)\in \eR^2\tq 0\leq t_1\leq t_2\leq L \}
            \end{equation}
            et l'application sécante \( \Sigma\colon \mT\to S^1\) définie par
            \begin{equation}
                \Sigma(t_1,t_2)=\begin{cases}
                    \dfrac{ \beta'(t) }{ \| \beta'(t) \| }    &   \text{si } t_1=t_2=t\\
                    -\dfrac{ \beta'(0) }{ \| \beta'(0) \| }    &   \text{si } t_1=0\text{ et } t_2=L\\
                    \dfrac{ \beta(t_2)-\beta(t_1) }{ \| \beta(t_2)-\beta(t_1) \| }    &   \text{sinon}.
                \end{cases}
            \end{equation}
    \item[Continuité de \( \Sigma\)]
        Nous devons prouver les limites\footnote{Limite au sens de la définition~\ref{DefYNVoWBx}, en sachant qu'elle est unique par la proposition~\ref{PropFObayrf}.} suivantes :
        \begin{subequations}
            \begin{align}
                \lim_{\substack{(t_1,t_2)\to (t,t)\\0\leq t_1\leq t\leq t_2\leq L}}\frac{ \beta(t_1)-\beta(t_2) }{ \| \beta(t_1)-\beta(t_2) \| }=\frac{ \beta'(t) }{ \| \beta'(t) \| }      \label{SUBEQooDXJDooOIxcBD}  \\
                \lim_{\substack{(t_1,t_2)\to (0,L)\\0\leq t_1\leq t\leq t_2\leq L}}\frac{ \beta(t_1)-\beta(t_2) }{ \| \beta(t_1)-\beta(t_2) \| }=-\frac{ \beta'(0) }{ \| \beta'(0) \| }        \label{SUBEQooOXGSooXHEHHh}
            \end{align}
        \end{subequations}
        Nous commençons par \eqref{SUBEQooDXJDooOIxcBD} en multipliant et divisant par \( t_2-t_1\) :
        \begin{equation}
            \frac{ \beta(t_2)-\beta(t_1) }{ \|  \beta(t_2)-\beta(t_1)  \| }=\frac{  \beta(t_2)-\beta(t_1)  }{ t_2-t_1 }\frac{ t_2-t_1 }{ \|  \beta(t_2)-\beta(t_1)  \| }.
        \end{equation}
        Si chacune des limites des deux facteurs existent dans \( \eR\), la limite du produit sera le produit des limites. Pour le premier facteur nous développons \( \beta(t_2)\) autour de \( t=t_1\) via la formulation \eqref{SUBEQooPYABooKpDgdu} :
        \begin{equation}
            \beta(t_2)=\beta(t_1)+(t_2-t_1)\beta'(t_1)+\alpha(t_2-t_1)
        \end{equation}
        où \( \alpha\) est une fonction ayant la propriété \( \lim_{t\to 0} \frac{ \alpha(t) }{ t }=0\). Nous avons à calculer la limite de
        \begin{equation}
            \frac{ (t_1-t_1)\beta'(t_1)+\alpha(t_2-t_1) }{ t_2-t_1 }=\beta'(t_1)+\frac{ \alpha(t_2-t_1) }{ t_2-t_1 }.
        \end{equation}
        Prendre la limite \( (t_1,t_2)\to (t,t)\) donne bien \( \beta'(t)\) parce que \( \beta\) est de classe \( C^1\). En ce qui concerne la limite de la deuxième partie, nous allons la faire plus en détail pour la limite \eqref{SUBEQooOXGSooXHEHHh}.

        La limite \eqref{SUBEQooOXGSooXHEHHh}. Nous prenons la prolongation périodique de \( \beta\). Alors si \( t_2=L-\epsilon\) nous pouvons écrire \( \beta(-\epsilon)\) au lieu de \( \beta(t_2)\). Nous développons \( \beta(t_1)\) autour de \( t=-\epsilon\) (parce que \( t_1\) est petit) :
        \begin{equation}
            \beta(t_1)=\beta(-\epsilon)+(t_1+\epsilon)\beta'(-\epsilon)+\alpha(t_1+\epsilon).
        \end{equation}
        Après multiplication et division par \( t_1+\epsilon\), la première limite à calculer est celle de
        \begin{equation}
            \frac{ \beta(-\epsilon)-\beta(t_1) }{ t_1+\epsilon }=-\beta'(-\epsilon)+\frac{ \alpha(t_1+\epsilon) }{ t_1+\epsilon },
        \end{equation}
        pour \( (\epsilon,t_1)\to (0,0)\). Cela donne bien \( -\beta'(0)\). La seconde limite à calculer est celle de
        \begin{equation}
            \frac{ t_1+\epsilon }{ \| \beta(-\epsilon)-\beta(t_1) \| }=\frac{ t_1+\epsilon }{ \| -(t_1+\epsilon)\beta'(-\epsilon)-\alpha(t_1+\epsilon) \| }=\frac{ t_1+\epsilon }{ \| (t_1+\epsilon)\beta'(-\epsilon)+\alpha(t_1+\epsilon) \| }.
        \end{equation}
        Nous calculons la limite de l'inverse (qui, si elle est non nulle donnera la réponse en inversant à nouveau) en nous souvenant de la formule
        \begin{equation}
           \big|  a-| b | \big|  \leq | a+b |\leq \big|    a+| b |\big|.
        \end{equation}
        Nous avons l'encadrement
        \begin{subequations}
            \begin{align}
                \frac{  \big| (t_1+\epsilon)\beta'(-\epsilon)-| \alpha(t_1+\epsilon) | \big| }{ t_1+\epsilon }&\leq \frac{ \|  (t_1+\epsilon)\beta'(-\epsilon)+ \alpha(t_1+\epsilon) \|  }{ t_1+\epsilon }\\
                &\leq  \frac{ \big|   (t_1+\epsilon)\beta'(-\epsilon)+| \alpha(t_1+\epsilon) | \big| }{ t_1+\epsilon }
                v
            \end{align}
        \end{subequations}
        Les limites des deux extrêmes existent et valent \( \beta'(0)\); la règle de l'étau~\ref{ThoRegleEtau} conclu.

    \item[Deux chemins homotopes]
            Nous considérons dans \( \mT\) les points \( A=(0,0)\), \( B=(0,L)\) et \( C=(L,L)\). Nous allons considérer les chemins direct de \( A\) à \( C\) et celui passant via \( B\). Et comme ces chemins doivent être paramétrés de \( 0\) à \( 2 \pi\), il faut faire un peu attention. Nous définissons les chemins
            \begin{equation}
                \sigma_i\colon S^1\to S^1
            \end{equation}
            par
            \begin{subequations}
                \begin{align}
                    \sigma_1(t)&=\Sigma\left( \frac{ L }{ 2\pi }t,\frac{ L }{ 2\pi }t \right)=\frac{ \beta'(tL/2\pi) }{ \| \beta'(tL/2\pi) \| }=\frac{ \tilde \beta'(t) }{ \|\tilde  \beta'(t) \| }=\Phi_{\beta}\\
                    \sigma_2(t)&=\begin{cases}
                        \Sigma(0,\frac{ L }{ \pi }t)    &   \text{si } 0\leq t\leq \pi\\
                        \Sigma\big( \frac{ L }{ \pi }(t-\pi),L  \big)    &    \text{si }\pi\leq t\leq 2\pi
                    \end{cases}
                \end{align}
            \end{subequations}
            où nous avons repris les notations de la définition~\ref{DEFooOCUQooUAlbLo}. Notons que pour \( \sigma_2\), en \( t=\pi\) les deux expressions donnent
            \begin{equation}
                \sigma_2(\pi)=\Sigma(0,L)=-\frac{ \beta'(0) }{ \| \beta'(0) \| }.
            \end{equation}
            Ce sont des chemins fermés parce que
            \begin{equation}
                \sigma_1(2\pi)=\Sigma(L,L)=\frac{ \beta'(L) }{ \| \beta'(L) \| }=\frac{ \beta'(0) }{ \| \beta'(0) \| }=\Sigma(0,0)=\sigma_1(0).
            \end{equation}
            Notons que dans toutes ces définitions et calculs, nous avons utilisé de façon assez cruciale la définition~\ref{DEFooSAZTooZGQrQG} pour définir la dérivée de \( \beta\) en \( t=0\).

            Les chemins \( \sigma_1\) et \( \sigma_2\) sont homotopes par construction.
        \item[Indices et degrés]
            La proposition~\ref{PROPooZIAKooHqtnZj}\ref{ITEMooLEHFooXEyTHY} nous donne \( \deg(\sigma_1)=\deg(\sigma_2)\). Nous avons alors la chaine d'égalités
            \begin{equation}        \label{EQooSXOAooZVOVxc}
                \deg(\sigma_2)=\deg(\sigma_1)=\deg(\Phi_{\beta})=\IR(\beta)=\IR(\gamma)=\Turn(\gamma)
            \end{equation}
            où les justifications sont :
            \begin{enumerate}
                \item
                    \( \deg(\sigma_2)=\deg(\sigma_1)\) par homotopie : proposition~\ref{PROPooZIAKooHqtnZj}.
                \item
                    \( \deg(\sigma_1)=\deg(\Phi_{\beta})\) parce que \( \sigma_1=\Phi_{\beta}\).
                \item
                    \( \deg(\Phi_{\beta}=\IR(\beta)\) par définition~\ref{DEFooOCUQooUAlbLo} de l'indice.
                \item
                    \(  \IR(\beta)=\IR(\gamma) \) par invariance de l'indice sous reparamétrage.
                \item
                    \( \IR(\gamma)=\Turn(\gamma)\) par la proposition~\ref{PROPooXHSDooDDnlJQ}.
            \end{enumerate}
            Il nous reste à montrer que \( \deg(\sigma_2)=\pm 1\).
        \item[La géométrie de \( \sigma_2\)]
            Notons que par définition les valeurs de \( \sigma_2(t)\) sont les vecteurs (unitaires) joignant \( p\) aux points de \( \Gamma\) lorsque \( 0\leq t\leq \pi\), et les vecteurs inverses pour \( \pi\leq t\leq 2\pi\). Plus précisément nous avons, si \( 0<a<\pi\) :
            \begin{equation}
                \sigma_2(a)=\Sigma(0,\frac{ L }{ \pi }a)=\frac{ \beta(\frac{ L }{ \pi }a)-\beta(L) }{ \|    \beta(\frac{ L }{ \pi }a)-\beta(L)   \| }
            \end{equation}
            et
            \begin{equation}
                \sigma_2(\pi+a)=\Sigma(\frac{ L }{\pi  }a,L)=\frac{ \beta(L)-\beta(\frac{ L }{ \pi }a) }{ \|   \beta(L)-\beta(\frac{ L }{ \pi }a) \| }.
            \end{equation}
            En sachant que \( \beta(0)=\beta(L)\) nous avons alors
            \begin{equation}        \label{EQooKMSRooEzWkyL}
                \sigma_2(\pi+a)=-\sigma_2(a).
            \end{equation}

            Notons aussi que le fait que \( \gamma\) soit une courbe simple assure que le numérateur et le dénominateur de \( \sigma_2\) ne s'annulent pas autrement que pour \( t=L\) ou \( t=0\).

        \item[Un relèvement pour \( \sigma_2\)]
            Ces propriétés motivent cette idée pour le relèvement de \( \sigma_2\) :
            \begin{equation}
                \begin{aligned}
                    \theta\colon \mathopen[ 0 , 2\pi \mathclose[&\to\mathopen[ 0 , 2\pi \mathclose[  \\
                        t&\mapsto \text{l'angle du vecteur } \sigma_2(t),
                \end{aligned}
            \end{equation}
            avec \( \theta(2\pi)\) définit par continuité. Nous allons cependant voir, en étant plus prudent, que cette définition n'assure pas la continuité (surtout en \( t=\pi\)).

            Soyons donc plus prudent et construisons \( \theta\) petit à petit.

            Étant donné que \( \ell_p\) est tangente à \( \Gamma\) au point \(p \), la droite \( \ell_p\) est parallèle à \( \beta'(0)\), et l'angle entre \( \ell_p\) et \( \beta'(0)\) est soit \( 0\) soit \( \pi\). Donc \( \theta(0)\) devrait valoir soit \(0\) soit \(\pi\).

            Nous commençons par définir ceci :
            \begin{equation}
                \begin{aligned}
                    \theta\colon \mathopen[ 0 , \pi \mathclose[&\to \mathopen[ 0 , \pi \mathclose] \\
                    t&\mapsto \text{angle de } \sigma_2(t)=\arccos\big(  \sigma_2(t)_x   \big).
                \end{aligned}
            \end{equation}
            C'est le choix d'avoir \( \ell_p\) horizontale et \( \Gamma\) au dessus de \( \ell_p\) qui nous assure que pour tout \( t\in\mathopen[ 0 , \pi \mathclose]\), l'angle de \( \sigma_2(t)\) peut être choisi entre \( 0\) et \( \pi\). De plus cette fonction est continue en tant que partie de la fonction \( \arccos\colon \mathopen[ -1 , 1 \mathclose]\to \mathopen[ 0 , \pi \mathclose]\) qui est elle-même continue par la proposition~\ref{PropoInvCompCont}.

            Le fait que \( \theta\) soit continue est assuré par le fait que \( \sigma_2\) est \(  C^{\infty}\).

            \begin{subproof}
                \item[Si \( \theta(0)=0\)]

                    Cela correspond à la situation des vecteurs rouges sur la figure~\ref{LabelFigERPMooZibfNOiU}.

                    Vu que \( \sigma_2(\pi)=-\sigma_2(0)\), l'angle de \( \sigma_2\) en \( t=\pi\) est le supplémentaire de celui en \( t=0\). Mais pour \( 0\leq t<\pi\), \( \theta(t)\) prend ses valeurs entre \( 0\) et \( \pi\), le seul supplémentaire de \( 0\) à être disponible est \( \theta(\pi)=\pi\) (et non \( \theta(\pi)=-\pi\) par exemple). Nous définissons donc \( \theta(\pi)=\pi\) pour la continuité.

                En ce qui concerne les \( \pi<t< 2\pi\) nous savons que \( \sigma_2(\pi+a)=-\sigma_2(a)\). Les angles sont donc les supplémentaires de ceux pour \( 0\leq t\leq \pi\). Pour assurer la continuité en \( t=pi\) nous sélectionnons la place \( \mathopen] \pi , 2\pi \mathclose]\); et nous définissons
                \begin{equation}
                    \begin{aligned}
                    \theta\colon \mathopen] \pi , 2\pi \mathclose[&\to \mathopen] \pi , 2\pi \mathclose] \\
                    t&\mapsto \text{angle de } \sigma_2(t).
                    \end{aligned}
                \end{equation}
                Le nombre \( \theta(2\pi)\) est défini par continuité. Il doit valoir \( \theta(2\pi)=2\pi\).

                La fonction \( \theta\) ainsi définie est un relèvement pour \( \sigma_2\), et le degré peut être calculé :
                \begin{equation}
                    \deg(\sigma_2)=\frac{1}{ 2\pi }\big( \theta(2\pi)-\theta(0) \big)=1.
                \end{equation}

\newcommand{\CaptionFigERPMooZibfNOiU}{Les vecteurs représenant \( \sigma_2\) dans le cas où \( \beta'(0)\) est dans le sens de \( \ell_p\) ou dans le sens inverse. Pour le sport nous avons dessiné la situation avec une droite \( \ell\) quelconque plutôt que horizontale.}
\input{auto/pictures_tex/Fig_ERPMooZibfNOiU.pstricks}

                \item[Si \( \theta(0)=\pi\)]
                    Cela correspond à la situation des vecteurs bleus sur la figure~\ref{LabelFigERPMooZibfNOiU}.

                Alors \( \theta(\pi)=0\) est obligatoire parce qu'il doit être supplémentaire à \( \theta(0)\). Les angles atteints par \( \sigma_2(t)\) pour \( t\in\mathopen] \pi , 2\pi \mathclose[ \) sont encore les complémentaires, mais cette fois la continuité en \( t=\pi\) nous impose de les chercher dans \( \mathopen] -\pi , 0 \mathclose]\) et nous définissons
                \begin{equation}
                    \begin{aligned}
                    \theta\colon \mathopen] \pi , 2\pi \mathclose[&\to \mathopen] -\pi , 0 \mathclose] \\
                    t&\mapsto \text{angle de } \sigma_2(t).
                    \end{aligned}
                \end{equation}
                Par continuité nous devons avoir \( \theta(2\pi)=\theta(\pi)-\pi\) et donc \( \theta(2\pi)=-\pi\).

                Le degré de \( \sigma_2\) est alors
                \begin{equation}
                    \deg(\sigma_2)=\frac{1}{ 2\pi }\big( \theta(2\pi)-\theta(0) \big)=-1.
                \end{equation}

            \end{subproof}

        \item[Conclusion]

            Nous avons montré que le degré de \( \sigma_2\) est \( 1\) ou \( -1\), et en remontant les égalités \eqref{EQooSXOAooZVOVxc} nous déduisons que \( \Turn(\gamma)=\pm 1\).

    \end{subproof}
\end{proof}

%+++++++++++++++++++++++++++++++++++++++++++++++++++++++++++++++++++++++++++++++++++++++++++++++++++++++++++++++++++++++++++
\section{Courbes fermées planes}
%+++++++++++++++++++++++++++++++++++++++++++++++++++++++++++++++++++++++++++++++++++++++++++++++++++++++++++++++++++++++++++

%---------------------------------------------------------------------------------------------------------------------------
\subsection{Cercle circonscrit}
%---------------------------------------------------------------------------------------------------------------------------

La proposition suivante est dans le même esprit que l'ellipse de John-Loewer\footnote{Proposition~\ref{PropJYVooRMaPok}.}.
\begin{propositionDef}[Cercle circonscrit\cite{ooCQFXooOLEkIr,MonCerveau}]      \label{PROPDEFooCWESooVbDven}
    Soit une courbe fermée simple et continue \( \gamma\colon \mathopen[ 0 , 1 \mathclose]\to \eR^2\). Soit \( \Gamma\) son image. Il existe un unique cercle de rayon minimum contenant \( \Gamma\). Ce cercle est le \defe{cercle circonscrit}{cercle!circonscrit à une courbe} à \( \gamma\).

    Il a les propriétés suivantes :
    \begin{enumerate}
        \item
            Le cercle circonscrit à \( \gamma\) coupe \( \Gamma\) en au moins deux points distincts.
        \item
            Tout arc du cercle circonscrit plus grand que le demi-cercle intersection \( \Gamma\).
    \end{enumerate}
\end{propositionDef}

\begin{proof}
    Division de la preuve.
    \begin{subproof}
        \item[Existence]
        L'application \( \gamma\) étant continue, l'ensemble \( \Gamma\) est compact (théorème~\ref{ThoImCompCotComp}). Nous considérons l'ensemble \( Q\) des formes quadratiques de la forme
        \begin{equation}
            q_{a,r}(x)=\| a-x \|^2-r^2
        \end{equation}
        avec \( a\in \eR^2\) et \( r\in \eR\). Nous mettons sur cet ensemble la topologie de \( \eR^2\times \eR=\eR^3\). Le nombre \( q_{a,r}(x)\) est continu en \( a\), \( r\) et \( x\). Soit \( A\) l'ensemble des formes quadratiques de cette forme et vérifiant
        \begin{equation}
            q\big( \gamma(t) \big)\leq 0
        \end{equation}
        pour tout \( t\in\mathopen[ 0 , 1 \mathclose]\).

        Cet ensemble \( A\) est non vide parce que \( \Gamma \) est compact et donc borné; il existe donc une boule qui contient \( \Gamma\) en son intérieur.

        Montrons que \( A\) est fermé dans \( Q\). Si \( q_{a,r}\notin A\) alors il existe \( t_0\) tel que \( q_{a,r}\big( \gamma(t_0) \big)>0\). Par continuité, il existe un voisinage de \( (a,r)\) dans \( \eR^3\)  et donc de \( q_{a,r}\) dans \( Q\) tel que \( q_{a',r'}\big( \gamma(t_0) \big)\) reste strictement positif pour tout \( (a',r')\) dans ce voisinage.

        En particulier l'ensemble
        \begin{equation}
            \{ r\in \eR^+\tq \exists\,(a,r)\tq q_{a,r}\in A \}
        \end{equation}
        est fermé et borné vers le base. De plus \( r=0\) n'est pas dans cet ensemble. Il possède donc un minimum strictement positif. Le cercle correspondant donne l'existence.
    \item[Unicité]

        En ce qui concerne l'unicité, si \( \Gamma\) est contenu dans les boules \( B(a,R)\) et \( B(b,R)\) alors
        \begin{equation}
            \Gamma\subset B(a,R)\cap B(b,R).
        \end{equation}

        \begin{center}
            \input{auto/pictures_tex/Fig_QMWKooRRulrgcH.pstricks}
        \end{center}

        Nous choisissons les axes comme indiqué sur le dessin et nous montrons que l'intersection est dans le cercle de centre \( O\) et de rayon \( \| OI \|<R\). Soit \( Q\) un point de l'intersection; par symétrie il est suffisant de supposer \( Q_x<0\) et \( Q_y>0\). Vu que \( Q\) est dans le cercle de centre \( B\) et de rayon \( R\), il doit satisfaire
        \begin{equation}
            (B_x-Q_x)^2+Q_{y}^2<R.
        \end{equation}
        D'autre part le point \( I\) est d'abscisse \( I_x=0\) et d'ordonnée donnée par \( I_y^2=R^2-B_x^2\).

        Nous devons prouver que \( Q_x^2+Q_y^2\leq I_y^2\). Il s'agit simplement de calculer
        \begin{equation}
            Q_x^2+Q_y^2\leq Q_x^2+R^2-B_x^2+2B_xQ_x-Q_x^2=R^2-B_x^2+2B_xQ_x\leq R^2-B_x^2=I_y^2
        \end{equation}
        parce que \( B_x>0\) et \( Q_x<0\).

        Nous concluons que \( \Gamma\) est inclus dans un cercle de rayon plus petit que \( R\) et donc que \( R\) n'est pas minimum. D'où l'unicité.

    \item[Au moins deux intersections]

        Nous nommons \( C\) le cercle circonscrit à \( \gamma\), et nous écrivons, pour \( p\in \Gamma\)
        \begin{equation}
            r(p)=d(p,C)
        \end{equation}
        la distance entre \( p\) et \( C\). Cela est une fonction continue sur le compact \( \Gamma\). Elle atteint donc ses bornes sur \( \Gamma\).

        Si \( C\cap \Gamma=\emptyset\) alors \( r(p)>0\) pour tout \( p\) et le minimum est également strictement positif. Soit \( r_0\) ce minimum. Alors le cercle \( C_2\) même centre que \( C\) mais de rayon \( r_0/2\) n'intersecte pas non plus \( \Gamma\) parce que
        \begin{equation}
            d(p,C)\leq d(p,C_2)+d(C_2,C)
        \end{equation}
        où \( d(p,C)=r(p)\) et \( d(C_2,C)=r_0/2\).
        \begin{equation}
            d(p,C_2)\geq r(p)-\frac{ r_0 }{2}\geq \frac{ r_0 }{2}>0.
        \end{equation}

        Si \( C\cap \Gamma=\{ q \}\) alors un choix d'axe place le centre de \( C\) en \( (0,0)\), le point \( q\) en \( (1,0)\) et fixe le rayon de \( C\) à \( 1\). Pour \( p\in \Gamma\) nous notons \( r(p)\) la distance entre \( p\) et le point de \( C\) situé sur sa gauche, c'est-à-dire, si \( p=(p_x,p_y)\),
        \begin{equation}
            r(p)=\sqrt{ 1-p_y^2 }+p_x.
        \end{equation}
        Cela est encore une fonction continue sur \( \Gamma\) qui atteint son minimum valant \( r_0\). Alors le cercle de centre \( (\frac{ r_0 }{2},0)\) et de même rayon contient encore \( \Gamma\) mais n'a plus de points d'intersection avec \( \Gamma\).

        Enfin, tout nombre de points d'intersection entre \( C\) et \( \Gamma\) est possible à partir de \( 2\). Pour en avoir deux, prendre une ellipse, et pour en avoir plus, prendre des polynômes dont les angles sont un peu modifiés de façon à rester \( C^1\).

    \item[Intersection avec les demi-arcs]

        Supposons, en fixant encore les axes, que le cercle circonscrit soir encore centré en \( (0,0)\) et que \( \Gamma\) n'intersecte pas le demi-cercle \( x<0\). Alors pour tout \( p\in \Gamma\) la distance entre \( p\) et ce demi-cercle est strictement positive. Il y a un minimum \( r_0\). En décalant le centre du cercle de \( r_0/2\) vers la droite, nous obtenons un nouveau cercle contenant \( \Gamma\) mais ne l'intersectant pas.
    \end{subproof}
\end{proof}

%---------------------------------------------------------------------------------------------------------------------------
\subsection{Description locale}
%---------------------------------------------------------------------------------------------------------------------------

\begin{definition}      \label{DEFooVQODooJSNYLw}
    Une courbe plane différentiable est \defe{convexe}{convexe!courbe plane} si son graphe est en tout point d'un seul côté de sa tangente.
\end{definition}

\begin{lemma}[\cite{MonCerveau}]        \label{LEMooGEVEooHxPTMO}
    Soit une courbe simple, convexe \( \gamma\colon \mathopen[ 0 , 1 \mathclose]\to \eR^2\) que nous supposons être de classe \( C^1\). Nous notons \( \Gamma\) l'image de \( \gamma\). Alors \( \Gamma\) est localement le graphe d'une fonction convexe (définition~\ref{DefVQXRJQz}).
\end{lemma}

\begin{proof}
    Soit \( p\in \Gamma\) et \( l_p\) la tangente à \( \Gamma\) en \( p\). Nous considérons un système d'axe centré en \( p\) de telle sorte que \( l_p\) soit l'axe des abscisses et l'axe des ordonnées soit dirigé de tel manière que \( \Gamma\) se trouve dans la partie \( y>0\). De plus nous paramétrons \( \gamma\) de telle sorte à avoir \( \gamma(0)=p=(0,0)\).

    Vu que \( l_p\equiv y=0\) est la tangente à \( \Gamma\) nous avons \( \gamma'_y=0\) et \( \gamma_x'(0)>0\). Nous en déduisons que \( \gamma_x'(t)>0\) pour tout \( t\in B(0,\delta)\) pour \( \delta\) suffisamment petit. Nous posons alors
    \begin{equation}
        g(x)=\gamma_y\big( \gamma_x^{-1}(x) \big)
    \end{equation}
    qui est bien définie parce que \( \gamma_x\) est une bijection entre \( B(0,\delta)\) et son image. La fonction \( g\) est continue par la proposition~\ref{PropIntContMOnIvCont} et même dérivable par la proposition~\ref{PROPooSGTBooFxUuXK}. De plus, vu la formule \eqref{EQooELIHooDxUFxH}, la fonction \( g^{-1}\) est de classe \( C^1\) parce que \( (g^{-1})'\) est une composée d'applications continues.

    \begin{subproof}
        \item[Si \( g\) est \( C^2\)]
            Dans ce cas, \( g''\) ne peut pas changer de signe, sinon la tangente coupe le graphe. Par positivité de \( g\) (et le fait que \( g(0)=g'(0)=0\)), il n'est pas possible d'avoir \( g''<0\) partout. Donc \( g''\geq 0\) partout. Cela prouve que \( g\) est convexe par la caractérisation~\ref{ThoGXjKeYb}.
        \item[Si \( g\) est seulement de classe \( C^1\)]
            Le graphe de \( g\) correspond au graphe de \( \Gamma\). Nous montrons que \( g\) est convexe en utilisant la caractérisation de la proposition~\ref{PROPooQPOSooDZlUAJ}.

            La tangente au graphe de \( g\) en \( x=x_0\), que nous notons \( l_0\), est la tangente à \( \Gamma\) en \( t=\gamma_x^{-1}(x_0)\). Le graphe de \( g\), qui est une partie de \( \Gamma\) se trouve donc d'un seule côté de \(l_0\).

            Nous nous restreignons \( g\) à un compact  \( I\) et nous considérons la fonction
            \begin{equation}
                d_a(x)=g(x)-l_a(x)
            \end{equation}
            qui donne la distance entre le graphe de \( g\) et la tangente à \( g\) en \( x=a\). Cela est une fonction continue en \( x\) et en \( a\). Le graphe de \( g\) est au dessus de la tangente en \( x=0\) (par construction des axes). Supposons que le graphe de \( g\) soit en dessous de la tangente en \( x=x_2\). Alors nous avons, pour tout \( x\) :
            \begin{subequations}
                \begin{numcases}{}
                    d_0(x)\geq 0\\
                    d_{x_1}(x)\leq 0.
                \end{numcases}
            \end{subequations}
            Nous posons
            \begin{equation}
                s(a)=\sup_{x\in I}d_a(x),
            \end{equation}
            qui est une fonction continue par la proposition~\ref{PROPooWXBAooAEweSF}. Vu le définitions, \( s(0)\geq 0\) et \( s(x_2)\leq 0\). Il existe donc \( m\in \mathopen[ 0 , m_2 \mathclose]\) tel que \( s(m)=0\). À ce moment nous avons \( g(x)=l_m(x)\) pour tout \( x\in I\) et donc \( g\) est une droite, et en réalité toutes les inégalités sont des égalités. La fonction \( g\) est alors bien convexe (mais pas strictement).
    \end{subproof}
\end{proof}

%---------------------------------------------------------------------------------------------------------------------------
\subsection{Enveloppe convexe}
%---------------------------------------------------------------------------------------------------------------------------

\begin{proposition}[\cite{ooHJQTooKVMAdi,MonCerveau}]       \label{PROPooWZITooTFiWsi}
    Soit une courbe simple, fermée et convexe \( \gamma\colon \mathopen[ 0 , 1 \mathclose]\to \eR^2\) que nous supposons être de classe \( C^2\). Nous notons \( \Gamma\) l'image de \( \gamma\). Alors il existe un convexe \( D\) tel que \( \partial D=\Gamma\).
\end{proposition}

\begin{proof}
    Pour \( p\in \Gamma\) nous notons \( l_p\) la tangente à \( \Gamma\) en \( p\) et \( H_p\) le demi-plan (fermé) contenant \( \Gamma\). Nous posons
    \begin{equation}        \label{EQooDYFTooCHRbsD}
        D=\bigcap_{p\in \Gamma}H_p.
    \end{equation}
    Cet ensemble est convexe comme intersection de convexes et fermé comme intersection de fermés. Nous prouvons que \( \Gamma=\partial D\).


    L'inclusion \( \Gamma\subset\partial D\) est la plus facile. Si \( p\in \Gamma\) alors \( p\) est dans chacun des \( H_q\) et donc dans \( D\). De plus tout voisinage de \( p\) contient des points en dehors de \( H_p\), donc \( p\) n'est pas dans l'intérieur de \( D\). Ce dernier étant fermé, un point hors de l'intérieur est sur le bord. Ergo \( p\in\partial D\).

    Pour l'inclusion inverse, soit \( p\in\partial D\).
    \begin{subproof}
        \item[Il existe \( q\) tel que \( p\in l_q\)] Vu que \( D\) est fermé, le point \( p\) est dans \( D\), et donc dans tous les \( H_q\). Supposons qu'il soit dans l'intérieur de tous les \( H_q\). Alors nous considérons la fonction
            \begin{equation}
                r(q)=\frac{ d(p,H_q) }{ 2 }
            \end{equation}
            définie sur \( \Gamma\). C'est une fonction continue\footnote{L'équation de la droite \( l_q\) a des coefficients continus parce que \( \gamma\) est de classe \( C^1\).} strictement positive définie sur le compact \( \Gamma\) qui possède donc un minimum strictement positif. Si \( r_0\) est ce minimum, alors \( B(p,r_0)\) est inclue à tous les \( H_q\), ce qui ferait que \( p\) est à l'intérieur de \( D\). Nous concluons à l'existence de \( q\in \Gamma\) tel que \( p\in l_q\).
        \item[Le point où ça décolle]
            Nous supposons que \( p\notin \Gamma\), sinon ce serait trop facile. Nous paramétrons \( \gamma\) de telle sorte à avoir \( q=\gamma(0)\) et nous posons
            \begin{equation}
                T=\{ t\in \eR^+\tq l_{\gamma(t)}=l_q \}.
            \end{equation}
            Cela est un fermé dans \( \eR\) parce que \( \gamma\) est de classe \( C^1\). Nous posons \( t_0=\inf(T^c)\) et pour tout \( \epsilon\) suffisamment petit,
            \begin{subequations}
                \begin{numcases}{}
                    l_{\gamma(t_0+\epsilon)}\neq l_q,\\
                    l_{\gamma(t_0-\epsilon)}=l_q.
                \end{numcases}
            \end{subequations}
            La seconde est parce que si \( l_{\gamma(t_0-\epsilon)}\neq l_q\) nous aurions \( t_0-\epsilon\in T^c\). Soit \( r=\gamma(t_0)\); nous avons \( l_r=l_q\) parce que si \( l_r\neq l_q\) alors par continuité de \( \gamma'\) nous aurions \( l_{r'}\neq l_q\) pour tout \( r'\in \gamma\big( B(t_0,\delta) \big)\).

        \item[Graphe d'une fonction strictement convexe] En suivant le lemme~\ref{LEMooGEVEooHxPTMO}, l'ensemble \( \Gamma\) est localement (autour de \( r\)) le graphe d'une fonction convexe au-dessus de \( l_r\). Soit \( g\colon B(0,\delta)\to \eR^2\) cette fonction convexe.

        Soit \( \epsilon>0\). Si \( g''(x)=0\) sur \( \mathopen] 0 , \epsilon \mathclose]\) alors \( g'\) y est constante. Mais \( g'(0)=\), ce qui signifierait que sur \( \mathopen[ 0 , \epsilon \mathclose]\) nous ayons \( g'(x)=0\) et donc \( g(x)=0\). Cela ferait que \( l_{r'}\equiv y=0\) pour tout \( r'\) de la forme \( (x,0)\) avec \( x\in \mathopen[ 0 , \epsilon \mathclose]\) (qui sont des points de \( \Gamma\)). Cela est en contradiction avec la définition de \( r\). Donc il existe un point \(x\in \mathopen[ 0 , \epsilon \mathclose[\) tel que \( g''(x)>0\).

        Rappelons que \( p\in l_q\), ce qui fait que \( p\) a pour coordonnées \( (p_x,0)\). Nous restreignons \( \delta\) et \( \epsilon\) de telle sorte que \( p_x\) soit plus grand à la fois que \( \epsilon\) et \(\delta\).

        Il existe donc un intervalle \( \mathopen[ a , b \mathclose]\) avec \( a,b\geq 0\) et \( a,b<p_x\) sur lequel \( g\) est strictement convexe.

    \item[La tangente qui tue]

        En particulier \( g(b)>0\) et le théorème des accroissements finis~\ref{ThoAccFinis}\ref{ITEMooFZONooXJqLyX} nous donne l'existence de \( m\in\mathopen[ 0 , b \mathclose]\) tel que la tangente à \( g\) en \( x=m\) est parallèle au segment joignant \( (0,0)\) à \( (b,f(b))\). Cette tangente, que nous nommons \( l_m\), est en dessous de la corde, par strict convexité. En particulier, son point d'intersection avec \( y=0\) est strictement entre \( 0\) et \( m\).

        L'ensemble \( D\) est d'un seul côté de \(l_m\). Ce côté est forcément celui de \( q=(0,0)\) (parce que \( q\in \Gamma\subset D\)), et donc le points de coordonnées \( (x,0)\) avec \( x>m\) ne sont pas dans \(D\). Pas de chance, \( p\) est un point de ce type.

    \item[La contradiction]
        Nous avons prouvé que si \( p\in \partial D\setminus \Gamma\) alors \( p\) n'est pas dans \( D\), ce qui est impossible parce que, l'ensemble \( D\) étant fermé, nous avons \( \partial D\subset D\).
    \end{subproof}
\end{proof}

\begin{remark}
   Bien que cela puisse paraitre évident dès le début, nous ne démontrerons que dans la proposition~\ref{PROPooOORPooCXrIQi} que \( D\) est l'enveloppe convexe de \( \Gamma\).
\end{remark}

\begin{corollary}       \label{CORooSXDGooJEmVcf}
    Si \( p\in\Int(D)\) alors toute droite passant par \( p\) intersecte \( \Gamma\) en exactement \( 2\) points.
\end{corollary}

\begin{proof}
    Vu que la partie \( D\) est bornée, toute droite passant par son intérieur coupe \( \partial D\) en au moins deux points (un dans chaque sens, et en utilisant le lemme de passage de douane~\ref{LEMooLKWEooItGnkP}).

    Soit \( \ell\) une droite passant par \( p\) et supposons qu'elle coupe \( \Gamma\) en trois points distincts. Alors au moins deux d'entre eux sont du même côté de \( p\). Soient \( q_1\) et \( q_2\) ces points. Nous avons donc dans l'ordre \( p\in \Int(D)\), \( q_1\in\partial D\) et \( q_2\in \partial D\).

    Vu que tout \( \Gamma\) est d'un seul côté de ses tangentes, lesdites tangentes ne passent pas par l'intérieur de \( D\). Ni \( p\) ni \( q_2\) ne sont sur \( \ell_{q_1}\), parce que si \( q_2\in\ell_{q_1}\) alors \( \ell_{q_1}=\ell\), ce qui est impossible parce que \( \ell\) passe par \( p\in\Int(D)\).

    Or \( p\) et \( q_2\) sont de part et d'autres de \( \ell_{q_1}\), ce qui est impossible parce que \( \Gamma\) est d'un seul côté de cette droite.
\end{proof}

\begin{proposition}     \label{PROPooOORPooCXrIQi}
    Soit \( D\) l'ensemble définit en \eqref{EQooDYFTooCHRbsD}.
    \begin{enumerate}
        \item
            \( D=\Conv(\Gamma)\) (\( \Conv(\Gamma)\) désigne l'enveloppe convexe de \( \Gamma\))
        \item
            \( \partial\Conv(\Gamma)=\Gamma\).
    \end{enumerate}
    Pour la définition d'enveloppe convexe, voir la définition~\ref{DefNLYYooXUHFUY}.
\end{proposition}

\begin{proof}
    L'ensemble \( D\) est un convexe contenant \( \Gamma\). Donc \( \Conv(\Gamma)\subset D\). L'inclusion inverse est à prouver.

    Soit \( x\in D\). Si \( x\in \partial D\) alors \( x\in \Gamma\) (proposition~\ref{PROPooWZITooTFiWsi}) et donc \( x\in\Conv(\Gamma)\). Nous ne devons donc traiter que le cas \( x\in\Int(D)\).

    Le corolaire~\ref{CORooSXDGooJEmVcf} nous dit que toute droite passant par \( x\) coupe \( \Gamma\) en exactement deux points. Soient \( p\) et \( q\) ces points. Alors \( p,q\in \Gamma\subset \Conv(\Gamma)\). Vu que \( \Conv(\Gamma) \) est convexe, tout le segment \( [p,q]\) est dans \( \Conv(\Gamma)\), et en particulier \( p\in\Conv(\Gamma)\).

    Nous passons à la seconde affirmation. Nous savons que \( D=\Conv(\Gamma)\). En prenant le bord des deux côtés, \( \partial D=\partial\Conv(\Gamma)\), donc \( \Gamma=\partial\Conv(\Gamma)\).
\end{proof}

\begin{lemma}[Des tangentes parallèles\cite{ooYGXBooTzOAtL}]        \label{LEMooUEKQooWhGyKn}
    Une courbe fermée \( \gamma\) de classe \( C^1\) est convexe si et seulement si elle ne possède pas \( 3\) tangentes parallèles distinctes.
\end{lemma}

\begin{proof}
    Si le graphe \( \Gamma\) possédait trois tangentes parallèles distinctes, une serait entre les deux autres et le graphe \( \Gamma\) serait de part et d'autres de cette tangente. Dans ce cas, \( \gamma\) n'est pas convexe.

    Nous montrons maintenant que si \( \gamma\) n'est pas convexe, alors elle possède trois tangentes parallèles distinctes. Pour \( p\in \Gamma\) tel que \( \Gamma\) soit des deux côtés de \( l_p\).

    Soient \( \Gamma_1\) et \( \Gamma_2\) les parties de \( \Gamma\) délimitées par \( l_p\). Nous notons \( q_i\) le point de \( \Gamma_i\) le plus éloigné de la droite \( l_p\), il existe parce que \( G\) est compact et que la fonction distance à \( l_p\) est continue sur \( \Gamma\). Les points \( p\), \( q_1\) et \( q_2\) sont distincts (sinon \( l_p\) ne couperait pas \( \Gamma\) en deux parties).

    Montrons que \( l_{q_i}\parallel l_p\). Pour cela nous choisissons un système d'axe dans lequel \( l_p\equiv y=0\). Dans ce système, la distance entre \( \gamma(t)\) et \( l_p\) est \( \gamma_y(t)\) et les extrémums de cette fonction ont lieu aux points \( t\) tels que \( \gamma_y'(t)=0\), c'est-à-dire aux points sur lesquels la tangente est parallèle à \( l_p\).

    À quel moment avons nous utilisé le fait que la courbe soit fermée ? Au moment de dire que le point le plus éloigné devait vérifier \( \gamma'_y(t)=0\). En effet un extrémum peut ne pas vérifier cette condition s'il n'est pas à l'intérieur du domaine. Dans notre cas, nous avons \( \gamma\colon \mathopen[ 0 , 1 \mathclose]\to \eR^2\) qui est fermée :  \( \gamma(0)=\gamma(1)\). Donc en réalité nous pouvons considérer \( \gamma\colon \eR\to \eR^2\) et tous les points du domaine sont intérieurs au domaine. L'extrémum doit donc vérifier la condition d'annulation de la dérivée.
\end{proof}

\begin{example}
    Si la courbe n'est pas fermée, alors le lemme~\ref{LEMooUEKQooWhGyKn} ne tient pas comme le montre le contre-exemple du graphe de \( f(x)=x^3-x\). Il est non convexe et pourtant ne présente que \( 2\) tangentes parallèles (dans chaque directions).
\end{example}

\begin{lemma}       \label{LEMooCSXCooIDPiKW}
    Soit \( \gamma\) une courbe convexe et \( \ell\) une droite qui intersecte \( \Gamma\) mais qui n'est pas tangente. Alors l'intersection entre \( \ell\) et \( \Gamma\) comprend au maximum \( 2\) points.
\end{lemma}

\begin{proof}
    Supposons que \( \ell\) coupe \( \Gamma\) en trois points \( p,q,r\) (dans cet ordre). Vu que \( \ell\) n'est pas tangente à \( \Gamma\), la tangente \( \ell_q\) est distincte de \( \ell\) (et intersecte \( \ell\) en l'unique point \( q\)). Les points \( p\) et \( r\) sont dans \( \Gamma\) et sont pourtant de deux côtés différents de \( \ell_q\). Contradiction avec la convexité de \( \gamma\).
\end{proof}

La proposition suivante nous dit que si deux points de \( \Gamma\) ont la même tangente, alors entre ces deux points, \( \Gamma\) est le segment de droite les joignant.
\begin{proposition}[\cite{MonCerveau}]     \label{PROPooCKTZooIPcUca}
    Soit \( \gamma\colon \mathopen[ 0 , L \mathclose]\to \eR^2\) une courbe fermée simple et convexe de classe \( C^1\). Si la tangente en \( p=\gamma(s_p)\) et la tangente en \( q=\gamma(s_q)\) sont identiques (pas seulement parallèles), alors soit
    \begin{equation}
        \gamma\big( \mathopen[ s_p , s_q \mathclose] \big)=[p,q]
    \end{equation}
    soit
    \begin{equation}
        \gamma\big( \mathopen[ s_q , L \mathclose] \big)=[p,q]
    \end{equation}
\end{proposition}

\begin{proof}
    Nous considérons un système d'axe dans lequel \( p=(0,0)\), \( \ell_p=\ell_q\equiv y=0\) et tel que \( \gamma_y(t)\geq 0\) pour tout \( t\). Nous choisissons enfin un paramétrage de \( \gamma\) telle que \( p=\gamma(0)\).

    Si \( \gamma_y\big( \mathopen[ 0 , s_q \mathclose] \big)=\{ 0 \}\) alors par le théorème des valeurs intermédiaires~\ref{ThoValInter}, tous les points du type \( (t,0)\) avec \( 0\leq t\leq q_x\) sont atteints par \(  \gamma(t)   \) avec \( 0\leq t\leq s_q\). De plus aucun autre point ne peut être atteint parce que \( \gamma\) étant simple, elle ne peut pas faire de retour en arrière.

    Nous supposons donc que \( \gamma_y\) n'est pas identiquement nulle sur \( \mathopen[ 0 , s_q \mathclose]\); il existe donc un \( s_M\) avec  \( 0<s_M<q_q\) qui maximise \( \gamma_y\) sur \( \mathopen[ 0 , s_q \mathclose]\). La tangente à \( \Gamma\) en \( \gamma(s_M)\) est horizontale. Les droites \( y=0\) et \( y=\gamma_y(s_M)\) sont donc deux tangentes parallèles à \( \Gamma\). Par le lemme des tangentes parallèles~\ref{LEMooUEKQooWhGyKn}, il n'y a pas d'autres tangentes horizontales. Donc pour tout \( n\in \eN\) la droite \( y=\frac{1}{ n }\) n'est tangente nulle part\footnote{À part \( n=0\) et si par manque de chance, \( \gamma_y(s_M) \) est un nombre de la forme \( 1/n\).} à \( \Gamma\).

    Par le lemme~\ref{LEMooCSXCooIDPiKW}, la droite \( y=\frac{1}{ n }\) ne peut intersecter \( \Gamma\) qu'en seulement deux points. Or le théorème des valeurs intermédiaires appliqué à \( \gamma_y\) sachant que \( \gamma_y(0)=\gamma_y(s_q)=0\) et \( \gamma_y(s_M)>0\) nous donne \( a_n,b_n\) tels que
    \begin{subequations}
        \begin{align}
            0<a_n<s_M\\
            s_M<b_n<s_q
        \end{align}
    \end{subequations}
    et \( \gamma_y(a_n)=\gamma_y(b_n)=\frac{1}{ n }\). Donc la droite \( y=\frac{1}{ n }\) intersecte \( \gamma\) deux fois dans \( \mathopen[ 0 , s_q \mathclose]\). En conséquence de quoi \( \gamma_y(t)<\frac{1}{ n }\) pour tout \( t\in\mathopen[ q_q , L \mathclose]\). Cela étant valable pour tout \( n\) nous avons \( \gamma_y\big( \mathopen[ s_q , L \mathclose] \big)=\{ 0 \}\) et nous sommes ramenés essentiellement au premier cas.
\end{proof}

\begin{proposition}     \label{PROPooKHUQooIOUxFw}
    Soit une courbe fermée simple \( \gamma\) de classe \( C^1\) et une droite \( \ell\). Alors il existe au moins deux points distincts \( q_1,q_2\) tels que \( \ell_{q_1}\parallel \ell_{q_2}\parallel\ell\) avec \( \ell_p\neq \ell_q\).
\end{proposition}

\begin{proof}
    Pour \( p\in \Gamma\) nous considérons le nombre \( r(p)=d(p,\ell)\). En tant que fonction continue sur un compact, elle possède un minimum et un maximum. Dans un système d'axe pour lequel \( \ell\equiv y=0\), la fonction \( r\) s'écrit \( r(p)=p_y\) et les extrémums arrivent en \( \gamma(s)\) avec \( \gamma_y'(s)=0\), ce qui signifie que les tangentes aux extrémums sont parallèles à \( \ell\). Vu que le maximum et le minimum ne peuvent pas être égaux (sinon la courbe serait horizontale et pas simple), les tangentes en ces points sont distinctes.
\end{proof}

\begin{corollary}
    Soit une courbe convexe fermée simple \( \gamma\) de classe \( C^2\). L'ensemble $\Conv(\Gamma)$ est compact.
\end{corollary}

\begin{proof}
    Nous savons que \( \Conv(\Gamma)\) n'est autre que \( D\) par la proposition~\ref{PROPooOORPooCXrIQi}. Nous savons déjà que \( D\) est fermé. Il nous suffit donc de prouver qu'il est borné (théorème de Borel-Lebesgue~\ref{ThoXTEooxFmdI}). Nous considérons deux droites perpendiculaires et les \( 4\) tangentes correspondantes par la proposition~\ref{PROPooKHUQooIOUxFw}. Vu que \( \Gamma\) est d'un seul côté de chacune de ces tangentes, elle est contenue dans le rectangle délimité par ces \( 4\) droites.
\end{proof}

%---------------------------------------------------------------------------------------------------------------------------
\subsection{Courbure et convexité}
%---------------------------------------------------------------------------------------------------------------------------
\label{SUBSECooNJOLooYuGRjA}

\begin{lemma}       \label{LEMooHMFSooFlhanD}
    Soit \(   \gamma\colon \mathopen[ 0 , L \mathclose]\to \eR^2    \) une courbe simple, fermée en paramétrage normal. Alors l'application
    \begin{equation}
        \begin{aligned}
            \sigma&\colon \mathopen[ 0 , L \mathclose]\to S^1\\
            s&\mapsto \gamma(s)
        \end{aligned}
    \end{equation}
    est surjective.
\end{lemma}

\begin{proof}
    Le théorème~\ref{THOooEQWOooBCRMMZ} nous dit que si \( \theta(0)=a\) alors \( \theta(2\pi)=a+2\pi\) ou \( a-2\pi\). Le théorème des valeurs intermédiaires nous dit alors que \( \theta\) prend toutes les valeurs entre \( a\) et \( a+2\pi\) ou \( a-2\pi\).
\end{proof}

\begin{proposition}[\cite{ooYGXBooTzOAtL}]      \label{PROPooWXUKooPOtPdj}
    Une courbe fermée simple de classe \(  C^{2}\) est convexe si et seulement si sa courbure est soit toujours positive, soit toujours négative.
\end{proposition}

\begin{proof}
    Nous considérons la courbe \( \gamma\) munie d'un paramétrage de vitesse \( 1\), c'est-à-dire avec \( \| \gamma'(t) \|=1\) pour tout \( t\). Si \( \theta\) est sa fonction d'angle, alors nous avons \( \theta'=\kappa\) par le lemme~\ref{LEMooWLAUooKetUiW}. Donc la fonction \( \theta\) est monotone si et seulement si la courbure ne change pas de signe. Nous allons montrer que \( \theta\) est monotone si et seulement si \( \gamma\) est convexe.

    \begin{subproof}
        \item[\( \Rightarrow\)]

        Nous supposons que \( \theta\) est monotone et \( \gamma\) non convexe. Soient \( p\in \Gamma\) tel que \( \Gamma\) soit des deux côtés de \( \ell_p\), et soient les points \( q_1\), \( q_2\) donnés par le lemme des tangentes parallèles~\ref{LEMooUEKQooWhGyKn} tels que \( \ell_p\parallel\ell_{q_1}\parallel\ell_{q_2}\). Parmi les vecteurs tangents en \( p\), \( q_1\) et \( q_2\), deux au moins ont la même direction; supposons que ce sont \( q_1\) et \( q_2\). C'est-à-dire que si \( p=\gamma(s_0)\), \( q_1=\gamma(s_1)\) et \( q_2=\gamma(s_2)\) alors nous avons \( \gamma'(s_1)=\gamma'(s_2)\) et donc aussi
        \begin{equation}
            \theta(s_1)=\theta(s_2)+2n\pi
        \end{equation}
        pour un certain \( n\). Mais \( \theta\) est monotone et la différence entre sa première et sa dernière valeur doit valoir \( 2\pi\) ou \( -2\pi\) par le théorème~\ref{THOooEQWOooBCRMMZ}. Donc \( n\) ne peut valoir que \( -1\), \( 0\) et \( 1\).

        Si \( n=0\) alors \( \theta\) est constante sur \( \mathopen[ s_1 , s_2 \mathclose]\). Si \( n=1\) alors \( \theta(s_1)=\sigma(s_2)+2\pi\) alors que sur toute la courbe, \( \theta\) ne peut faire que \( 2\pi\). Donc \( \theta\) est constant sur \( \mathopen[ 0 , s_1 \mathclose]\) et sur \( \mathopen[ s_2 , L \mathclose]\) (où \( L\) est le bord de le paramétrage). Si \( n=-1\), même conclusion.

        Dans tous les cas, \( \Gamma\) contient une ligne droite, soit de \( q_1\) à \( q_2\), soit de \( q_2\) à \( q_1\). Et dans ces cas nous avons \( \ell_{q_1}=\ell_{q_2}\), ce qui est contraire à la construction de \( q_i\).

        Nous concluons que \( \gamma\) est convexe.

        \item[\( \Leftarrow\)]

            Nous supposons que \( \gamma\) est convexe, mais que \( \theta\) n'est pas monotone. Il existe donc \( s_1<s_0<s_2\) tels que
            \begin{equation}
                \theta(s_1)=\theta(s_2)\neq \theta(s_0).
            \end{equation}
            Et vu le lemme~\ref{LEMooHMFSooFlhanD}, il existe \( s_3\) tel que \( \gamma'(s_3)=-\gamma'(s_1)\).

            Donc en \( s_1\), \( s_2\) et \( s_3\) nous avons trois tangentes parallèles. La proposition~\ref{LEMooUEKQooWhGyKn} est alors formelle, \( \gamma\) étant convexe, deux de ces tangentes doivent être identiques.

            La proposition~\ref{PROPooCKTZooIPcUca} dit qu'entre deux points dont les tangentes sont identiques, la courbe doit être un segment de droite. Or sur un segment de droite, \( \kappa=0\) et \( \theta\) est constante.

            \begin{itemize}
                \item
            La partie \( \gamma\big( \mathopen[ s_1 , s_2 \mathclose] \big)\) ne peut pas être droite parce que nous avions supposé l'existence d'un \( s_0\in \mathopen] s_1 , s_2 \mathclose[\) tel que \( \theta(s_1)=\theta(s_2)\neq \theta(s_0)\).

            \item
            La partie \( \gamma\big( \mathopen[ s_1 , s_3 \mathclose] \big)\) ne peut pas être droite parce que \( \theta(s_3)\neq \theta(s_1)\).
            \item
            La partie \( \gamma\big( \mathopen[ s_2 , s_3 \mathclose] \big)\) ne peut pas être droite parce que \( \theta(s_2)\neq \theta(s_1)\).
            \end{itemize}
            Nous sommes donc devant une contradiction.

            Nous en concluons que \( \theta\) doit être monotone.
    \end{subproof}
\end{proof}

%---------------------------------------------------------------------------------------------------------------------------
\subsection{Théorème des quatre sommets}
%---------------------------------------------------------------------------------------------------------------------------

\begin{lemma}       \label{LEMooELIRooNDVXPh}
    Soit une droite \( \ell\) du plan. Il existe \( a,c\in \eR^2\) avec \( c\neq 0\) tels que \( z\in \ell\) si et seulement si \( (z-a)\cdot c=0\).
\end{lemma}

\begin{proof}
    Une droite est paramétrée par \( \gamma(t)=p+tq\). En posant \( a=p\) et \( c=Jq\) nous avons la réponse. En effet nous allons montrer qu'avec ces valeurs de \( a\) et \( c\), nous avons \( z\in \Gamma\) si et seulement si \( (z-a)\cdot c=0\).

    D'abord un point de \( \gamma\) est de la forme \( z=\gamma(t)=p+tq\). Nous avons :
    \begin{equation}
        \big( \gamma(t)-a \big)\cdot c=\big( \gamma(t)-p \big)\cdot Jq=tq\cdot Jq=0.
    \end{equation}

    Et dans l'autre sens, si \( (z-a)\cdot c=0\) nous devons prouver que \( z\in \Gamma\). Nous avons : \( (z-p)\cdot Jq=0\), ce qui fait que \( z-p\) est un multiple de \( q\). Autrement dit : \( z-p=\lambda q\) ou encore \( z=\alpha q+p\), qui est sur la droite \( \Gamma\).
\end{proof}

\begin{definition}
    Un \defe{sommet}{sommet} d'une courbe est un point d'extrémum local de la courbure.
\end{definition}

\begin{theorem}[Théorème des quatre sommets\cite{KXjFWKA,ooIEJXooIYpBbd}]       \label{THOooFRBBooWKZcfY}
    Soit un arc paramétrique \( \gamma\colon \eR\to \eR^2\) fermé, simple et convexe\footnote{Par la proposition~\ref{PROPooWXUKooPOtPdj} nous pouvons aussi bien demander à la courbure d'être toujours strictement positive, comme le fait \cite{KXjFWKA}.} de classe \( C^3\) et \( T\)-périodique.

    Alors \( \gamma\) possède au moins \( 4\) points critiques sur chaque période.
\end{theorem}

\begin{proof}
    Nous supposons que le paramétrage de \( \gamma\) soit normale.

    Si la courbure \( \kappa\) est constante sur une partie ouverte de la (du paramétrage de la) courbe, alors tous les points de cette partie sont des sommets et le théorème est fait. Nous supposons que \( \kappa\) n'est pas constante et en particulier que \( \Gamma\) ne contient ni bouts de droites ni bouts de cercles (théorème~\ref{THOooDLDVooFQnLWn}).

    La fonction \( \kappa\) étant de classe \( C^1\) sur le compact \( \Gamma\), elle admet au moins un maximum et un minimum distincts. Vu que ces points sont intérieurs, ils correspondent au changement de signe de \( \kappa'\). Soient \( p\) et \( q\) ces points. Pour la simplicité nous supposons que \( \gamma\) est paramétré de telle sorte que \( \gamma(0)=p\), et \( q=\gamma(s_q)\) avec \( 0<s_q<T\).

    Nous supposons que \( p\) et \( q\) sont les seuls points de changement de signe de \( \kappa'\).

    Soit \( \ell\) la droite passant par \( p\) et \( q\). Tous les points du segment \( [p,q]\) (qui sont dans \( \Conv(\Gamma)\)) ne peuvent pas être sur \( \Gamma\) (sinon nous aurions un morceau de droite). Donc certains points sont dans l'intérieur de \( \Conv(\Gamma)\). Donc la droite \( \ell\) passe par l'intérieur de \( \Conv(\Gamma)\) et le corolaire~\ref{CORooSXDGooJEmVcf} nous dit que la droite \( \ell\) ne coupe \( \Gamma\) en seulement deux points.

    Par conséquent, les ensembles \( \gamma\big( \mathopen[ 0 , s_q \mathclose] \big)\) et \( \gamma\big( \mathopen[ s_q , T \mathclose] \big)\) sont de part et d'autre de \( \Gamma\). Vu qu'en ces points, \( \kappa'\) change de signe et qu'il ne change de signe en aucun autre points, la fonction \( \kappa'\) est positive d'un côté de \( \ell\) et négative de l'autre.

    D'autre part par le lemme~\ref{LEMooELIRooNDVXPh}, il existe \( a\in \eR^2\) et \( c\neq 0\) tels que \( z\in\ell\) si et seulement si \( (z-a)\cdot c=0\). La fonction \( z\mapsto (z-a)\cdot c\) est donc positive d'un côté de \( \ell\) et négative de l'autre.

    En résumé les fonctions
    \begin{subequations}
        \begin{align}
            s&\mapsto \kappa'(s)\\
            s&\mapsto \big( \gamma(s)-a \big)\cdot c
        \end{align}
    \end{subequations}
    changent de signe en même temps et le produit a donc un signe constant. Ce produit n'est de plus pas nul parce que \( \kappa'\) n'est nul sur aucun intervalle (sinon \( \kappa\) y serait constant et \( \Gamma\) un segment de droite) et \( \big( \gamma(s)-a \big)\cdot c\) ne s'annule pour aucun \( s\) sauf ceux qui correspondent à \( p\) et \( q\).

    Nous avons donc
    \begin{subequations}
        \begin{align}
            0&\neq \int_0^T\kappa'(s)\big( \gamma(s)-a \big)\cdot c\,ds\\
            &=\underbrace{\Big[ \big( \gamma(s)-a \big)\cdot c\kappa(s) \Big]_0^{T}}_{A=0}-\int_0^T\kappa(s)\gamma'(s)\cdot c\,ds\\
            &=-\int_0^T\kappa(s)\big( \gamma'(s)\cdot c \big)ds\\
            &=\int_0^TJ\gamma''(s)\cdot c\,ds\\
            &=J\int_0^T\gamma''(s)\cdot c\,ds\\
            &=J\big[ \gamma'(s)\cdot c \big]_0^T\\
            &=0.
        \end{align}
    \end{subequations}
    Justifications :
    \begin{itemize}
        \item
    L'expression \( A\) est nulle parce que les valeurs en \( 0\) et en \( T\) sont identiques.
        \item
            Nous utilisons le lemme~\ref{LEMooKPORooEGJCRm} pour faire \( -\kappa(s)\gamma'(s)=J\gamma''(s)\).
    \end{itemize}
    Le tout est une contradiction de la forme \( 0\neq a=0\).

    Nous avons donc au moins un troisième point de changement de signe de \( \kappa'\). Vu que la courbe est périodique, il en faut un nombre pair et donc un quatrième.
\end{proof}

L'exemple de l'ellipse montre qu'il n'y a pas lieu de chercher d'autres extrémums de \( \kappa\) à part les \( 4\) déjà trouvés.

\begin{example}
    Nous trouvons les sommets de l'ellipse.
    \begin{subequations}
        \begin{align}
            \gamma(t)&=\big( a\cos(t),b\sin(t) \big)\\
            \gamma'(t)&=\big( -a\sin(t),b\cos(t) \big)\\
            \gamma''(t)&=-\big( a\cos(t),b\sin(t) \big)\\
        \end{align}
    \end{subequations}
    La courbure est
    \begin{subequations}
        \begin{align}
        \kappa(t)&=\frac{ \gamma''\cdot J\gamma' }{ \| \gamma' \|^3 }\\
        &=\frac{-1}{ [a^2\sin^2(t)+b^2\cos^2(t)]^{3/2} }\begin{pmatrix}
            a\cos(t)    \\
            b\sin(t)
        \end{pmatrix}\cdot\begin{pmatrix}
            -b\cos(t)    \\
            -a\sin(t)
        \end{pmatrix}\\
        &=\frac{ ab }{  [a^2\sin^2(t)+b^2\cos^2(t)]^{3/2}  }.
        \end{align}
    \end{subequations}
    Vu que \( ab>0\), les extrémums de cela sont ceux du dénominateur et il suffit donc d'étudier les extrémums de
    \begin{equation}
        f(t)=a^2\sin^2(t)+b^2\cos^2(t).
    \end{equation}
    Nous avons
    \begin{equation}
        f'(t)=2(a^2-b^2)\cos(t)\sin(t),
    \end{equation}
    fonction qui s'annule effectivement $4$ fois sur une période. Deux maximums et deux minima.
\end{example}

%---------------------------------------------------------------------------------------------------------------------------
\subsection{Le théorème de Jordan}
%---------------------------------------------------------------------------------------------------------------------------

\begin{definition}[\cite{ooTXKNooIgJrPw}]       \label{DEFooCJCWooLNrHFd}
    Une \defe{courbe de Jordan}{courbe!de Jordan} est une courbe simple fermée dans le plan.
\end{definition}

\begin{definition}      \label{DEFooBYRDooTBgsui}
    Une \defe{courbe de Jordan}{courbe!de Jordan}\index{Jordan!courbe} dans le plan est une application \( \gamma\colon S^1\to \eR^2\) qui est continue et injective.
\end{definition}
Une telle courbe peut évidemment être vue comme une application \( \gamma\colon \mathopen[ 0 , 2\pi ]\to \eR^2\) telle que \( \gamma(0)=\gamma(2\pi)\). En particulier il n'est jamais mauvais de se rappeler qu'on peut choisir un paramétrage normal par la proposition~\ref{PropExisteChmNorm}.

Le théorème suivant a un énoncé relativement simple, mais la démonstration est en réalité très longue.
\begin{theorem}[Théorème de Jordan\cite{ooTXKNooIgJrPw, HDJTbua}]\label{ThoHSPWBuh}
     Le complémentaire d'une courbe de Jordan \( \Gamma\) dans un plan affine réel est formé de exactement deux composantes connexes distinctes, dont l'une est bornée et l'autre non. Toutes deux ont pour frontière la courbe \( \Gamma\).
\end{theorem}
\index{théorème!de Jordan}
% Si un jour on travaille sur ce théorème, il y a moyen de revoir la réponse de Alphago dans
% http://math.stackexchange.com/questions/1727310/convex-curve-as-boundary-of-a-convex-set



\chapter{Géométrie hyperbolique}
\input{160_hyperbolique}

\chapter{Espaces projectifs}
\input{161_EspacesProjectifs}
\input{170_EspacesProjectifs}


\chapter{Analyse vectorielle}
\input{37_anal_vectorielle}

\chapter{Espaces de Hilbert}
\input{81_Hilbert}

\chapter{Analyse complexe}          \label{ChapICHIooXbLccl}
% This is part of Mes notes de mathématique
% Copyright (c) 2012-2019
%   Laurent Claessens
% See the file fdl-1.3.txt for copying conditions.

%+++++++++++++++++++++++++++++++++++++++++++++++++++++++++++++++++++++++++++++++++++++++++++++++++++++++++++++++++++++++++++
\section{Fonctions holomorphes}
%+++++++++++++++++++++++++++++++++++++++++++++++++++++++++++++++++++++++++++++++++++++++++++++++++++++++++++++++++++++++++++

La dérivée complexe est discutée à la section~\ref{SECooJWNOooOgMiWR}, et la définition d'une fonction holomorphe est \ref{DefMMpjJZ}.

%---------------------------------------------------------------------------------------------------------------------------
\subsection{Équations de Cauchy-Riemann}
%---------------------------------------------------------------------------------------------------------------------------

Notons que la formule \eqref{EqYFmoiM} donne un \defe{développement limité}{développement!limité!fonction holomorphe} pour les fonctions holomorphes. Si \( f\) est holomorphe en \( z_0\) alors si \( z\) est dans un voisinage de \( z_0\), il existe une fonction \( s\colon \eR\to \eC\) telle que \( \lim_{t\to 0} s(t)/t=0\) et
\begin{equation}    \label{EqptwBFG}
    f(z)=f(z_0)+f'(z_0)(z-z_0)+s(| z-z_0 |).
\end{equation}

Nous introduisons les opérateurs\nomenclature[Y]{\( \partial_z\),\( \partial_{\bar z}\)}{dérivées partielles d'une fonction complexe}
\begin{subequations}
    \begin{align}
        \frac{ \partial  }{ \partial z }=\partial=\frac{ 1 }{2}\left( \frac{ \partial  }{ \partial x }-i\frac{ \partial  }{ \partial y } \right)\\
        \frac{ \partial  }{ \partial \bar z }=\bar\partial=\frac{ 1 }{2}\left( \frac{ \partial  }{ \partial x }+i\frac{ \partial  }{ \partial y } \right)
    \end{align}
\end{subequations}
Si \( f\) est une fonction $\eC$-dérivable représentée par la fonction \( F=P+iQ\), les équations de Cauchy-Riemann signifient que \( \Delta P=\Delta Q=0\), c'est-à-dire que la fonction \( f\) a des composantes harmoniques.

\begin{theorem}
    Si \( f\in C^1(\Omega)\) alors nous avons équivalence des faits suivants :
    \begin{enumerate}
        \item
            \( f\) est holomorphe sur \( \Omega\),
        \item
            \( f\) vérifie \( \partial_{\bar z}f=0\).
    \end{enumerate}
\end{theorem}
%TODO : une preuve.

\begin{proposition}\label{PropkwIQwg}
    Une application \( f\colon \Omega\to \eC\) est $\eC$-dérivable sur \( \Omega\) si et seulement si elle est différentiable et
    \begin{subequations}        \label{EqmblExI}
        \begin{numcases}{}
            \frac{ \partial u }{ \partial x }=\frac{ \partial v }{ \partial y }\\
            \frac{ \partial u }{ \partial y }=-\frac{ \partial v }{ \partial x }
        \end{numcases}
    \end{subequations}
    où \( f(x+iy)=u(x,y)+iv(x,y)\). Ces équations se notent de façon plus compacte
    \begin{equation}
        \frac{ \partial f }{ \partial \bar z }=0.
    \end{equation}
\end{proposition}
Les équations \eqref{EqmblExI} sont les équations de \defe{Cauchy-Riemann}{Cauchy-Riemann}.

\begin{proof}
    La différentielle de \( f\colon \eR^2\to \eR^2\) est donnée par la matrice
    \begin{equation}        \label{EQwtagsz}
        T=\begin{pmatrix}
            \partial_xu(a)    &   \partial_yu(a)    \\
            \partial_xv(a)    &   \partial_yv(a)
        \end{pmatrix}.
    \end{equation}
    Cette matrice est une similitude si et seulement si les équations de Cauchy-Riemann sont satisfaites. En effet si \( 1=\begin{pmatrix}
        1    \\
        0
    \end{pmatrix}\) et \( i=\begin{pmatrix}
        0    \\
        1
    \end{pmatrix}\), la matrice \( T\) est une similitude (écrivons \( \alpha+i\beta\) son coefficient) si
    \begin{subequations}
        \begin{numcases}{}
            T(1)=\alpha+i\beta\\
            T(i)=-\beta+i\alpha,
        \end{numcases}
    \end{subequations}
    c'est-à-dire
    \begin{equation}
        T=\begin{pmatrix}
            \alpha    &   -\beta    \\
           \beta    &   \alpha
        \end{pmatrix}.
    \end{equation}
    Identifier cette matrice à \eqref{EQwtagsz} fournit le résultat annoncé.
\end{proof}

\begin{proposition}     \label{PROPooCHUEooYsGcQK}
    Si \( f\colon \eC\to \eC\) est holomorphe, alors nous avons
    \begin{equation}
        df_{z_0}=(\partial_zf)(z_0)
    \end{equation}
    au sens où l'opérateur linéaire \( df_{z_0}\colon \eC\to \eC\) est l'opération de multiplication par le nombre complexe \( (\partial_zf)(z_0)\).
\end{proposition}

\begin{proof}
    Soit \( f(x+iy)=f_1(x,y)+if_2(x,y)\) une fonction holomorphe\footnote{Définition \ref{DefMMpjJZ}.}. Les fonctions réelles \( f_1\) et \( f_2\) sont assujetties aux équations de Cauchy-Riemann de la proposition~\ref{PropkwIQwg} :
    \begin{subequations}
        \begin{numcases}{}
            \partial_xf_1=\partial_yf_2\\
            \partial_xf_2=-\partial_yf_1.
        \end{numcases}
    \end{subequations}
    Nous avons, en recourant à un petit abus de notation entre \( f_i\colon \eR^2\to \eR\) et \( f_i\colon \eC\to \eR\) :
    \begin{subequations}
        \begin{align}
            df_{z_0}(u)&=\Dsdd{ f(z_0+tu) }{t}{0}\\
            &=\Dsdd{ f_1(z_0+tu)+if_2(z_0+tu) }{t}{0}\\
            &=\partial_xf_1u_1+\partial_yf_1u_2+i\big( \partial_xf_2u_1+\partial_yf_2u_2 \big)\\
            &=\begin{pmatrix}
                \partial_xf_1    &   \partial_yf_1    \\
                \partial_xf_2    &   \partial_yf_2
            \end{pmatrix}\begin{pmatrix}
                u_1    \\
                u_2
            \end{pmatrix}\\
            &=\begin{pmatrix}
                \partial_xf_1    &   \partial_yf_1    \\
                -\partial_yf_1    &   \partial_yf_2
            \end{pmatrix}
            \begin{pmatrix}
                u_1    \\
                u_2
            \end{pmatrix}.
        \end{align}
    \end{subequations}
    En utilisant le lemme~\ref{LEMooJNFEooZCbJMo} nous reconnaissons la matrice de multiplication par le nombre \( \partial_xf_1-i\partial_yf_1\). Or justement,
    \begin{equation}
        \partial_zf=\frac{ 1 }{2}\left( \frac{ \partial  }{ \partial x }-i\frac{ \partial  }{ \partial y } \right)f=\frac{ 1 }{2}\big( \partial_xf_1+i\partial_xf_2-i\partial_yf_1+\partial_yf_2 \big),
    \end{equation}
    qui se réduit à \( \partial_xf_1-i\partial_yf_1\) lorsque nous y appliquons les équations de Cauchy-Riemann.
\end{proof}

%---------------------------------------------------------------------------------------------------------------------------
\subsection{Intégrales sur des chemins fermés}
%---------------------------------------------------------------------------------------------------------------------------

\begin{lemma}       \label{LemtpEOmi}
    Si \( g\) est une fonction continue dans un ouvert \( \Omega\subset \eC\) et si \( g\) admet une primitive complexe sur \( \Omega\) alors
    \begin{equation}
        \int_{\gamma}g(z)dz=0
    \end{equation}
    pour tout chemin fermé \( \gamma\) de classe \( C^1\) contenu dans \( \Omega\).
\end{lemma}

\begin{proof}
    Nommons \( G\) une primitive de \( g\). Par définition,
    \begin{subequations}
        \begin{align}
            \int_{\gamma}g&=\int_{\gamma}G'\\
            &=\int_0^1G'\big( \gamma(t) \big)\gamma'(t)dt\\
            &=\int_0^1 (G\circ g\gamma)'(t)dt\\
            &=G(\gamma(1))-G\big( \gamma(0) \big)\\
            &=0
        \end{align}
    \end{subequations}
    parce que le chemin est fermé : \( \gamma(0)=\gamma(1)\).
\end{proof}

\begin{lemma}[Goursat\cite{Holomorphieus}]  \label{LemwbwbUR}
    Soit \( \Omega\) un ouvert dans \( \eC\) et \( f\) une fonction continue sur \( \Omega\), holomorphe sur \( \Omega\) moins éventuellement un point (nommé \( z_1\in\Omega\)). Soit \( T\), un triangle\footnote{Nous considérons ici le triangle «plein».} fermé inclus dans \( \Omega\). Alors nous avons
    \begin{equation}
        \int_{\partial T}f(z)dz=0.
    \end{equation}
\end{lemma}

\begin{proof}
    Nous notons \( \gamma=\partial T\). Dans la suite nous allons définir une suite de triangles \( T^{(n)}\) et nous noterons \( \gamma_n=\partial T^{(n)}\) avec une orientation que nous allons expliquer. Pour commencer nous posons \( T^{(0)}=T\) et \( \gamma_0=\partial T^{(0)}\).

    Nous considérons le cas \( z_1\notin T\), et nous posons
    \begin{equation}
        c=l(\gamma)^{-2}| \int_{\gamma}f |.
    \end{equation}
    Notre objectif est de montrer que \( c=0\). Soit \( A,B,C\) les trois sommes du triangle; nous divisons le triangle de la façons suivante. D'abord nous considérons les points \( A',B,C'\) respectivement milieux de \( BC\), \( AC\) et \( AB\). En traçant le triangle \( A'B'C'\), nous construisons quatre triangles que nous nommons \( T^{(0)}_i\). Le théorème de Thalès assure que le périmètre de chacun des quatre triangles est la moitié du périmètre du grand triangle \( T\).

    Sur \( T\) nous choisissons l'orientation \( ABC\). De façon à être «compatible», nous choisissons les orientations \( AC'B'\), \( BA'C'\) et \( A'CB'\). La somme de ces trois triangles donne \( T\) plus le triangle \( A'C'B'\). Par conséquent nous choisissons sur le triangle central l'orientation (inverse) \( AB'C'\) de façon à avoir
    \begin{equation}
        \int_{\gamma}f=\sum_{i=1}^4\int_{\partial T^{(0)}_i}f.
    \end{equation}
    Cela implique que pour au moins un des quatre triangles (disons \( T^{(0)}_k\) pour fixer les idées) nous ayons
    \begin{equation}
        \int_{\partial T^{(0)}_k}f\geq \frac{1}{ 4 }\int_{\partial T^{(0)}}f
    \end{equation}
    Nous notons \( T^{(1)}\) ce triangle. Comme noté précédemment nous avons
    \begin{equation}
        l(\partial T^{(1)})=\frac{ 1 }{2}l(\partial T^{(0)}),
    \end{equation}
    et donc
    \begin{equation}
        l(\gamma_1)^{-2}| \int_{\gamma_1} |f=4l(\gamma_0)^{-2}| \int_{\gamma_1}f |\geq 4l(\gamma_0)^{-2}\frac{1}{ 4 }| \int_{\gamma_0}f |=c.
    \end{equation}
    En répétant le procédé nous construisons une suite de triangles \( T^{(n)}\) qui satisfont toujours
    \begin{equation}
        l(\partial T^{(n)})=\frac{1}{ 2^n }l(\partial T^{(0)}).
    \end{equation}
    Ces triangles forment une suite de fermés emboités dont le diamètre tend vers zéro. Leur intersection contient donc exactement un point (lemme~\ref{LemdCOMQM}) que nous nommons \( z_0\) (et qui appartient évidemment à \( \Omega\)). Étant donné que \( f\) est holomorphe nous utilisons le développement limité \eqref{EqptwBFG} autour de \( z_0\) :
    \begin{equation}
        f(z)=f(z_0)+f'(z_0)(z-z_0)+s(| z-z_0 |)(z-z_0)
    \end{equation}
    avec \( \lim_{t\to 0} s(t)=0\). Nous posons \( g(z)=f(z_0)+f'(z_0)(z-z_0)\) et nous considérons \( \epsilon>0\). Soit \( \alpha>0\) tel que
    \begin{equation}
        | f(z)-g(z) |<\epsilon| z-z_0 |
    \end{equation}
    pour tout \( | z-z_0 |<\alpha\). Le \( \alpha\) à choisir pour obtenir cet effet est celui qui donne \( s(| z-z_0 |)<\epsilon\). Soit \( N\in \eN\) tel que \( l(\gamma_n)<\alpha\) pour tout \( n>N\). D'autre part, deux points dans un triangle sont toujours à distance moindre que la longueur d'un côté, donc pour tout \( z\in T^{(n)}\) nous avons \( | z-z_0 |<\alpha\) et par conséquent pour tout \( z\) dans \( T^{(n)}\) nous avons
    \begin{equation}
        | f(z)-g(z) |<\epsilon| z-z_0 |.
    \end{equation}
    Notons que la fonction \( g\) est une dérivée : c'est la dérivée de la fonction
    \begin{equation}
        G(z)=zf(z_0)+\frac{ 1 }{2}f'(z_0)(z-z_0)^2.
    \end{equation}
    Par conséquent nous avons
    \begin{equation}
        \int_{\gamma_n}g=0
    \end{equation}
    par le lemme~\ref{LemtpEOmi}. Nous avons donc
    \begin{subequations}
        \begin{align}
            | \int_{\gamma_n}f |&=|\int_{\gamma_n}(f-g)|\\
            &\leq l(\gamma_n)\max\{ | f(z)-g(z) |\tq z\in T^{(n)} \}\\
            &\leq \epsilon l(\gamma_n)^2,
        \end{align}
    \end{subequations}
    et par conséquent
    \begin{equation}
        c\leq l(\gamma_n)^{-2}| \int_{\gamma_n}f |\leq \epsilon,
    \end{equation}
    ce qui signifie que \( c=0\) parce que \( \epsilon\) est arbitraire. Nous avons donc prouvé le lemme de Goursat dans le cas où le point de non holomorphie \( z_1\) est en dehors de \( T\).

    Si \( z_1\) est sur un côté, disons sur le côté \( AB\), alors nous considérons un vecteur \( v\in \eC\) tel que \( T_{\epsilon}=T+\epsilon v\) ne contienne \( z_1\) pour aucun \( \epsilon\). Le vecteur \( v=z_1-C\) fait par exemple l'affaire. En vertu du point précédent nous avons
    \begin{equation}
        \int_{\partial T_{\epsilon}}f=0
    \end{equation}
    pour tout \( \epsilon>0\). Étant donné que la fonction \( f\) est continue (y compris en \( z_1\)), l'intégrale sur \( \partial T\) est également nulle.

    Si maintenant le point \( z_1\) est à l'intérieur de \( T\) nous décomposons \( T\) en trois triangles ayant \( z_1\) comme sommet commun. Si nous considérons les orientations \( Az_1C\), \( ABz_1\) et \( BCz_1\), alors nous avons
    \begin{equation}
        \int_Tf=\int_{Az_1C}f+\int_{ABz_1}f+\int_{BCz_1}f,
    \end{equation}
    alors que par le point précédent les trois intégrales du membre de droite sont nulles.
\end{proof}

\begin{proposition}[\cite{Holomorphieus}]   \label{PrpopwQSbJg}
    Soient \( \Omega\) un ouvert étoilé et \( f\) une fonction holomorphe sur \( \Omega\) sauf éventuellement en un point \( z_1\) où \( f\) est seulement continue. Alors si \( \gamma\) est un chemin fermé dans \( \Omega\), nous avons
    \begin{equation}
        \int_{\gamma}f=0.
    \end{equation}
\end{proposition}

\begin{proposition}     \label{PropRZCKeO}
    Si \( f(z)=\sum_na_nz^n\) a pour rayon de convergence \( R\), alors \( f\) est $\eC$-dérivable et nous pouvons dériver terme à terme dans la boule ouverte \( B(0,R)\).
\end{proposition}

\begin{proof}
    Cela est exactement la proposition~\ref{ProptzOIuG}.
\end{proof}

%---------------------------------------------------------------------------------------------------------------------------
\subsection{Lacets, indice et homotopie}
%---------------------------------------------------------------------------------------------------------------------------

\begin{propositionDef}      \label{DEFooLFBNooGlvJmp}
    Soit \( \gamma\) un chemin fermé\footnote{Par abus de langage, nous désignerons par \( \gamma\) à la fois le chemin et son image.} dans \( \eC\). L'\defe{indice}{indice!d'une courbe dans $\eC$} de la courbe \( \gamma\) est la fonction
    \begin{equation}
        \begin{aligned}
            \Ind_{\gamma}\colon \eC\setminus \gamma&\to \eZ \\
            z&\mapsto \frac{1}{ 2\pi i }\int_{\gamma}\frac{ d\omega }{ \omega-z }.
        \end{aligned}
    \end{equation}
    Un chemin continu et fermé (au sens \( \gamma(1)=\gamma(0)\)) est un \defe{lacet}{lacet}.
    \begin{enumerate}
        \item
            La fonction \( \Ind_{\gamma}\) est continue et prend effectivement des valeurs entières.
        \item
            La fonction indice est constante sur chaque composante connexe\footnote{Définition \ref{}.} de \( \eC\setminus \gamma\) et est nulle sur la composante non bornée.
    \end{enumerate}
\end{propositionDef}

%TODO : une preuve. Si cette preuve ne demande pas vraiment d'analyse complexe, alors on peut la mettre plus haut et éventuellement remettre le théorème de Brouwer~\ref{ThoLVViheK} à sa place.
Le second point est en partie la proposition~\ref{PropHSjJcIr}.
\index{connexité!indice d'une courbe}

\begin{definition}  \label{DefECnFJQp}
    Si \( \gamma_1\) et \( \gamma_2\) sont deux lacets en \( x_0\in X\) (un espace topologique), une \defe{équivalence d'homotopie}{equivalence@équivalence!homotopie} est une application \( f\colon \mathopen[ 0 , 1 \mathclose]\times \mathopen[ 0 , 1 \mathclose]\to X\) telle que
    \begin{enumerate}
        \item
            \( f(0,t)=\gamma_1(t)\) pour tout \( t\);
        \item
            \( f(1,t)=\gamma_1(t)\) pour tout \( t\);
        \item
            pour chaque \( t\in \mathopen[ 0 , 1 \mathclose]\), l'application \( s\mapsto f(s,t)\) est continue;
        \item
            pour chaque \( s\in \mathopen[ 0 , 1 \mathclose]\), l'application \( t\mapsto f(s,t)\) est un lacet basé en \( x_0\).
    \end{enumerate}
\end{definition}

\begin{example} \label{ExradygL}
    Si \( \gamma\) est un cercle de centre \( z_0\in \eC\) et de rayon \( r\), alors
    \begin{equation}
        \Ind_{\gamma}(z)=\begin{cases}
            2\pi i    &   \text{si } z\in B(z_0,r)\\
            0    &    \text{sinon}.
        \end{cases}
    \end{equation}
    La seconde ligne provient directement de la proposition~\ref{DEFooLFBNooGlvJmp}. Pour la première, le cercle \( \gamma\) se paramètre par
    \begin{equation}
        \gamma(\theta)=z_0+r e^{i\theta},
    \end{equation}
    et l'intégrale vaut
    \begin{equation}
        \int_{\gamma}\frac{ d\omega }{ \omega-z_0 }=\int_0^{2\pi}\frac{1}{ r e^{i\theta} }ir e^{i\theta}d\theta=2\pi i.
    \end{equation}
    L'indice de ce chemin va évidemment jouer un rôle particulier dans la suite.
\end{example}

\begin{theorem}[Cauchy, version homotopique\cite{ADEyNiz}]
    Soient \( \Omega\) un ouvert de \( \eC\) et \( f\) une fonction holomorphe sur \( \Omega\). Si \( \gamma_1\) et \( \gamma_2\) sont deux lacets homotopes de classe \( C^1\) dans \( \Omega\), alors
    \begin{equation}
        \int_{\gamma_1}f(z)dz=\int_{\gamma_2}f(z)dz.
    \end{equation}
\end{theorem}

\begin{corollary}[\cite{ADEyNiz}]   \label{CorGZXzuZR}
    Soient \( a\in \eC\) ainsi que deux chemins \( \gamma_1\) et \( \gamma_2\) homotopes dans \( \eC\setminus\{ a \}\). Alors \( \Int(\gamma_1,a)=\Ind(\gamma_2,a)\).
\end{corollary}
Il y a aussi des choses sur l'indice dans \cite{Holomorphieus}.


%---------------------------------------------------------------------------------------------------------------------------
\subsection{Théorème de Cauchy et analycité}
%---------------------------------------------------------------------------------------------------------------------------

Cette sous-section veut prouver le théorème de Cauchy. Comme d'habitude, une référence qui ne peut pas rater est \cite{Holomorphieus}.


\begin{theorem}[Formule de Cauchy]    \label{ThoUHztQe}
    Soient \( \Omega\) ouvert dans \( \eC\), \( z_0\in \Omega\) et \( f\) une fonction holomorphe sur \( \Omega\). Soit \( r>0\) tel que \( B(z_0,r)\subset \Omega\). Alors pour tout \( z\in B(z_0,r)\) nous avons
    \begin{equation}    \label{EqPzUABM}
        f(z)=\frac{1}{ 2\pi i }\int_{\partial B(z_0,r)}\frac{ f(\omega) }{ \omega-z }d\omega.
    \end{equation}
\end{theorem}
\index{formule!de Cauchy}
\index{Cauchy!formule}

\begin{proof}
    Soit \( z\in B(z_0,r)\). Considérons la fonction
    \begin{equation}
        g(\omega)=\begin{cases}
            \frac{ f(\omega)-f(z) }{ \omega-z }    &   \text{si } \omega\neq z\\
            f'(z)    &    \text{si } \omega=z.
        \end{cases}
    \end{equation}
    Cette fonction est holomorphe sur \( B(z_0,r)\setminus\{ z \}\) et continue en \( z\). Elle vérifie donc la proposition~\ref{PrpopwQSbJg} et nous avons
    \begin{equation}
        \int_{\gamma}g=0
    \end{equation}
    où \( \gamma\) est le cercle de centre \( z_0\) et de rayon \( r\). Nous avons donc
    \begin{equation}
        0=\int_{\gamma}\frac{ f(\omega) }{ \omega-z }-\int_{\gamma}\frac{ f(z) }{ \omega-z },
    \end{equation}
    et ayant déjà calculé la seconde intégrale dans l'exemple~\ref{ExradygL} nous en déduisons
    \begin{equation}
        \int_{\gamma}\frac{ f(\omega) }{ \omega-z }d\omega=2\pi if(z),
    \end{equation}
    ce qu'il fallait.
\end{proof}

\begin{theorem}     \label{ThomcPOdd}
    Soient \( \Omega\) ouvert dans \( \eC\) et \( f\) holomorphe sur \( \Omega\). Soient encore \( z_0\in \Omega\) et \( r_0\) tels que \( B(z_0,r_0)\subset \Omega\). Alors :
    \begin{enumerate}
        \item       \label{ITEMooYWSOooHJtxGr}
            Sur \( B(z_0,r_0)\), la fonction \( f\) s'écrit
    \begin{equation}
        f(z)=\sum_{n=0}^{\infty}a_n(z-z_0)^n.
    \end{equation}
    \item
        Nous avons
        \begin{equation}
            a_n=\frac{ f^{(n)}(z_0) }{ n! }=\frac{1}{ 2\pi i }\int_{\gamma}\frac{ f(\omega) }{ (\omega-z_0)^{n+1} }d\omega
        \end{equation}
        où \( \gamma=\partial B(z_0,r)\) avec \( | z-z_0 |<r<r_0\).
    \item   \label{ItemMRRTooMChmuZ}
        En particulier \( f\) est infiniment dérivable.
    \end{enumerate}
\end{theorem}
\index{série!entière!fonctions holomorphes}

\begin{proof}
    Soit \( r>0\) tel que \( | z-z_0 |<r<r_0\). La formule de Cauchy (théorème~\ref{ThoUHztQe}) nous dit que
    \begin{equation}
        f(z)=\frac{1}{ 2\pi i }\int_{\gamma}\frac{ f(\omega)}{ \omega-z }d\omega
    \end{equation}
    où \( \gamma=\partial B(z_0,r)\). Nous pouvons paramétrer ce chemin par \( \omega=z_0+r e^{i\theta}\) et \( \theta\in \mathopen[ 0 , 2\pi \mathclose]\). Nous avons
    \begin{subequations}
        \begin{align}
            f(z)&=\frac{1}{ 2\pi i }\int_0^{2\pi}\frac{ f(z_0+r e^{i\theta}) }{ z_0+r e^{i\theta}-z }ri e^{i\theta}d\theta\\
            &=\frac{1}{ 2\pi }\int_0^{2\pi}\frac{ f(z_0+r e^{i\theta}) }{ 1- e^{-i\theta}(z-z_0)/r }d\theta.
        \end{align}
    \end{subequations}
    Nous pouvons développer l'intégrante en puissance de \( (z-z_0)\) en utilisant la formule~\ref{EqVmuaqT}. Ici le rôle de \( x\) est tenu par
    \begin{equation}
        e^{-i\theta}(z-z_0)/r
    \end{equation}
    dont le module est bien plus petit que \( 1\), par hypothèse sur \( r\). Nous avons donc
    \begin{equation}
        f(z)=\frac{1}{ 2\pi }\int_0^{2\pi}\sum_{n=0}^{\infty}f(z_0+r e^{i\theta}) e^{-in\theta}r^{-n}(z-z_0)^nd\theta.
    \end{equation}
    L'art est maintenant de permuter la somme et l'intégrale. Pour cela nous remarquons que ce qui se trouve dans la somme est majoré en module par
    \begin{equation}        \label{EqbykTLD}
        M\left| \frac{ z-z_0 }{ r } \right|^n
    \end{equation}
    où \( M\) est le maximum de \( | f |\) sur \( \gamma\). La borne \eqref{EqbykTLD} ne dépend pas de \( \theta\); par conséquent la convergence de la somme est uniforme en \( \theta\) par le critère de Weierstrass (théorème~\ref{ThoCritWeierstrass}). Le théorème~\ref{ThoCciOlZ} s'applique\footnote{Étant donné que nous savions déjà que la somme était une fonction intégrable, nous sommes loin d'avoir utilisé toute la puissance du théorème.} et nous pouvons permuter la somme avec l'intégrale.

    Ce que nous trouvons est que
    \begin{equation}
        f(z)=\sum_{n=0}^{\infty}a_n(z-z_0)^n
    \end{equation}
    où
    \begin{equation}
        a_n=\frac{1}{ 2\pi }\int_0^{2\pi}f(z_0+r e^{i\theta}) e^{-in\theta}r^{-n}d\theta=\frac{1}{ 2\pi i }\int_{\gamma}\frac{ f(\omega) }{ (\omega-z_0)^{n+1} }.
    \end{equation}
    Cette formule est valable pour \( | z-z_0 |<r\). Sur cette boule, la fonction est donc une série entière. Le théorème de Taylor~\ref{ThoTGPtDj} nous permet donc d'affirmer que \( f\) est partout infiniment continument dérivable (parce que en chaque point on a un voisinage sur lequel c'est vrai), et d'identifier les coefficients (qui, eux, ne sont valables que localement) sous la forme
    \begin{equation}
        a_n=\frac{ f^{(n)}(z_0) }{ n! }.
    \end{equation}
\end{proof}

\begin{corollary}       \label{CorwfHtJu}
    Soit \( f\) une fonction continue sur un ouvert \( \Omega\) telle que pour toute boule \( B(a,r)\) contenue dans \( \Omega\), nous ayons
    \begin{equation}
        f(a)=\frac{1}{ 2\pi i }\int_{\partial B(a,r)}\frac{ f(\xi) }{ \xi-a }d\xi.
    \end{equation}
    Alors \( f\) est holomorphe.
\end{corollary}

\begin{proof}
    Il suffit de recopier la démonstration du théorème~\ref{ThomcPOdd} pour savoir que \( f\) se développe en série de puissances et est donc en particulier dérivable.
\end{proof}

Le fait qu'une fonction holomorphe soit \(  C^{\infty}\) comme dit dans la proposition~\ref{ThomcPOdd} permet de démonter un résultat de dérivation sous l'intégrale, qui dépend de pouvoir majorer la différentielle.

\begin{proposition}     \label{PROPooZCLYooUaSMWA}
    Soit une fonction continue \( g\colon \eR\times \eC\to \eC\). Nous supposons que pour tout \( t\), la fonction \( z\mapsto g(t,z)\) est \( \eC\)-dérivable (définition~\ref{DEFooVJVXooKlnFkh}) et différentiable. Soit \( B\) compact dans \( \eR\) et la fonction
    \begin{equation}
        G(z)=\int_B g(t,z)dt.
    \end{equation}
    que nous supposons exister pour tout \( z\).

    Alors
    \begin{equation}
        G'(z)=\int_Bg'(t,z)dt
    \end{equation}
    où le prime réfère à la \( \eC\)-dérivée par rapport à la variable \( z\) à \( t\) fixé.
\end{proposition}

\begin{proof}
    Nous fixons \( z\in \eC\) et nous considérons la suite de fonctions
    \begin{equation}
        g_i(t)=\frac{ g(t,z+\epsilon_i)-g(t,z) }{ \epsilon_i }
    \end{equation}
    où \( \epsilon_i\) est une suite de nombres complexes tendant vers zéro (\( \epsilon_i\stackrel{\eC}{\longrightarrow}0\)). Si la limite existe et ne dépend pas de la suite choisie, alors \( \lim_{i\to \infty} g_i(t)=g'(t,z)\). Et vu que \( g\) est supposée dérivable, c'est le cas.

    Nous avons aussi, par linéarité de l'intégrale :
    \begin{equation}
        G'(z)=\lim_{i\to \infty} \int_B g_i(t)dt.
    \end{equation}
    La difficulté est de permuter la limite et l'intégrale. Pour cela nous allons utiliser la convergence dominée de Lebesgue (théorème~\ref{ThoConvDomLebVdhsTf}). Afin de majorer \( | g_i(t) |\) par une fonction intégrable en \( t\) (uniformément en \( i\)), nous exploitons le théorème des accroissements finis, théorème~\ref{ThoNAKKght}. En notant \( dg\) la différentielle de \( g\) par rapport à \( z\) à \( t\) fixé, pour chaque \( t\) et chaque \( i\) nous avons
    \begin{equation}
        | g(t,z+\epsilon_i)-g(t,z) |\leq \sup_{\xi\in\mathopen[ z , z+\epsilon_i \mathclose]}\| dg_{\xi} \|\| \epsilon_i \|.
    \end{equation}
    Vu que \( z\) est fixé et que \( \xi\) est dans le compact \( \mathopen[ z , z+\epsilon_i \mathclose]\) et que \( dg\) est continue (parce que la \( \eC\)-dérivabilité implique la continuité de la différentielle parce que nous avons l'analycité par le théorème~\ref{ThomcPOdd}), nous pouvons majorer \( \| dg_{\xi} \|\) par une constante \( M_i(z)\) qui dépend à priori de \( i\) et de \( z\).

    Heureusement, nous pouvons prendre a fortiori le supremum sur \( \overline{ B(z,|\epsilon_i|) }\) (qui est tout autant compact) et supposer que \( | \epsilon_i |\) est strictement décroissante; de toutes façons, il y a un maximum parce que \( | \epsilon_i |\to 0\). Dans ce cas, il suffit de prendre le supremum de \( \| dg_{\xi} \|\) pour \( \xi\in \overline{ B(z,| \epsilon_1 |) }\) et ça contente tout le monde.

    Quoi qu'il en soit nous avons une constante \( M(z)\) telle que
    \begin{equation}
        | g(t,z+\epsilon_i)-g(t,z) |\leq M(z)\| \epsilon_i \|
    \end{equation}
    et donc \( | g_i(t) |\leq M(z)\). La constante (par rapport à \( t\)) \( M(z)\) est évidemment intégrable sur le compact \( B\) et nous pouvons permuter la limite avec l'intégrale :
    \begin{equation}
        G'(z)=\lim_{i\to \infty} \int_Bg_i(t)dt=\int_B\lim_{i\to \infty} g_i(t)dt=\int_Bg'(t,z)dt.
    \end{equation}
\end{proof}

\begin{proposition}\label{PropZOkfmO}
    Une fonction continue \( f\) est holomorphe si et seulement si la \( 1\)-forme différentielle \( f(z)dz\) est localement exacte.
\end{proposition}

\begin{proof}
    Si \( f\) est holomorphe, alors nous avons vu que \( f\) était différentiable et que \( df_{z}=f(z)dz\) par la formule~\ref{EqPropZOkfmO}.

    Dans le sens inverse, supposons que \( f(z)dz\) est localement exacte, et soit \( F\) telle que \( dF=f(z)dz\). Ce que nous allons faire est montrer que la dérivée de \( F\) existe et vaut \( f\). En effet, la définition de la différentielle nous dit que
    \begin{equation}
        \lim_{h\to 0} \left| \frac{ F(z+h)-F(z)-dF_z(h) }{ h } \right| =0.
    \end{equation}
    La limite vaut évidemment encore zéro si nous enlevons les modules :
    \begin{subequations}
        \begin{align}
            0&=\lim_{h\to 0} \frac{ F(z+h)-F(z)-f(z)h }{ h }\\
            &=\lim_{h\to 0} \frac{ F(z+h)-F(z) }{ h }-f(z).
        \end{align}
    \end{subequations}
    Donc \( F'=f\). Cela montre que \( F\) est \( \eC\)-dérivable et donc holomorphe. En conséquence du théorème~\ref{ThomcPOdd}, la fonction \( F\) est infiniment dérivable et \( f\) l'est alors aussi. La fonction \( f\) est donc holomorphe\footnote{Dire que la dérivée d'une fonction holomorphe est holomorphe est un raisonnement classique.}.
\end{proof}

%---------------------------------------------------------------------------------------------------------------------------
\subsection{Théorème de Brouwer en dimension \texorpdfstring{$ 2$}{2}}
%---------------------------------------------------------------------------------------------------------------------------
Pour d'autres versions du théorème de Brouwer, voir la sous-section~\ref{subSecZCCmMnQ}.

\begin{theorem}[Brouwer en dimension \( 2\)\cite{KXjFWKA}]     \label{ThoLVViheK}
    Soit \( \mB\) la boule unité fermée de \( \eR^2\). Alors toute application continue de \( \mB\) dans elle-même admet un point fixe.
\end{theorem}
\index{théorème!Brouwer!dimension \( 2\)}
\index{connexité!utilisation!Brouwer}
\index{théorème!point fixe!Brouwer}

\begin{proof}
    Supposons que la fonction \( f\in C^0(\mB,\mB)\) n'admette pas de points fixes sur \( \mB=\overline{ B(0,1) }\). Pour \( x\in \mB\) nous notons \( g(x)\) l'intersection entre \( \partial \mB\) et la demi-droite allant de \( f(x)\) vers \( x\). C'est bien parce que \( f\) n'a pas de points fixes que \( g\) est bien définie.

    En reprenant le même début de la preuve de la proposition~\ref{PropDRpYwv} nous savons que la fonction
    \begin{equation}
        \begin{aligned}
            g\colon \overline{ B(0,1) }&\to \partial B(0,1) \\
            x&\mapsto \lambda(x)\big( x-f(x) \big)+f(x)
        \end{aligned}
    \end{equation}
    est continue. De plus \( g(x)=x\) sur \( \partial B(0,1)\). Nous allons montrer qu'une telle fonction\footnote{Qui est nommée \emph{rétraction} de la sphère sur elle-même.} ne peut pas exister.

    Pour \( s\in\mathopen[ 0 , 1 \mathclose]\) nous paramétrons le cercle \( \partial B(0,s)\) par
    \begin{equation}
        \begin{aligned}
            x_s\colon \mathopen[ 0 , 1 \mathclose]&\to \partial B(0,1) \\
            t&\mapsto \big( s\cos(2\pi t),s\sin(2\pi t) \big).
        \end{aligned}
    \end{equation}
    Ensuite nous considérons les chemins
    \begin{equation}
        \begin{aligned}
            \gamma_s\colon \mathopen[ 0 , 1 \mathclose]&\to \partial B(0,s) \\
            t&\mapsto g\circ x_s.
        \end{aligned}
    \end{equation}
    L'application \( \gamma_s\) est continue et \( \gamma_s(0)=\gamma_s(1)\). Les chemins \( \gamma_s\) sont des lacets; nous nous intéressons maintenant à l'indice au point \( 0\) de \( \gamma_0\) et \( \gamma_1\). D'une part \( \gamma_0(t)=g(0)\) (lacet constant) et \( \gamma_1(t)= e^{2i\pi t}\) (parce que \( g(x)=x\) sur le bord). Nous avons donc, en utilisant l'indice de la définition~\ref{DEFooLFBNooGlvJmp},
    \begin{equation}
        \Ind_{\gamma_0}(0)=\frac{1}{ 2\pi i }\Ind_{\gamma_0}\frac{ d\omega }{ \omega }=\frac{1}{ 2\pi i }\int_0^1\frac{ \gamma_0'(t) }{ \gamma_0(t) }dt=0,
    \end{equation}
    alors que
    \begin{equation}
        \Ind_{\gamma_1}(0)=\frac{1}{ 2\pi i }\int_0^1\frac{ 2i\pi e^{2i\pi t} }{  e^{2i\pi t} }dt=1.
    \end{equation}

    Nous considérons l'homotopie
    \begin{equation}
        \begin{aligned}
            \gamma\colon \mathopen[ 0 , 1 \mathclose]\times \mathopen[ 0 , 1 \mathclose]&\to \overline{ B(0,1) } \\
            (s,t)&\mapsto \gamma_s(t)=(g\circ x_s)(t).
        \end{aligned}
    \end{equation}
    Nous avons \( g(0)\neq 0\) parce que \( g\) prend ses valeurs sur le bord. Vu que c'est une équivalence d'homotopie\footnote{Définition~\ref{DefECnFJQp}} entre \( \gamma_1\) et \( \gamma_2\), les indices devraient être égaux par le corolaire~\ref{CorGZXzuZR}.
\end{proof}

%---------------------------------------------------------------------------------------------------------------------------
\subsection{Principe des zéros isolés}
%---------------------------------------------------------------------------------------------------------------------------

\begin{lemma}   \label{LEMooYYZQooClmOgG}
    Si \( f\) est une fonction holomorphe\footnote{Définition \ref{DEFooQSMCooOoWVZk}.} sur le compact \( K\), alors il existe une fonction polynôme \( P_f\) et une fonction holomorphe \( h_f \) ne s'annulant pas sur \( K\) telles que \( f=h_fP_f \).
\end{lemma}

\begin{proof}
    Soit une fonction $f$ vérifiant les conditions. Si \( f\) est identiquement nulle, alors il suffit de prendre \( P_f=0\) et c'est fait. Nous supposons donc que \( f\) n'est pas identiquement nulle.

    \begin{subproof}
    \item[Quantité finie de racines]

    D'abord \( f\) ne peut s'annuler qu'un nombre fini de fois sur \( K\). Sinon, on pourrait considérer une suite des racines\quext{Notez l'utilisation de la proposition~\ref{PROPooBYKCooGDkfWy} que je vous invite à ne pas considérer comme une trivialité absolue.} de \( f\) dans \( K\). Vu qu'une suite dans un compact contient une sous-suite convergente (théorème~\ref{ThoBWFTXAZNH}), la fonction \( f\) aurait un point d'accumulation de racines. Alors le principe des zéros isolés (théorème~\ref{ThoukDPBX}) nous donne un ouvert sur lequel \( f\) est nulle et donc le corolaire~\ref{CORooFBXXooZyfUQi} nous dit que \( f\) est identiquement nulle.

\item[Autour d'une racine]

    Bref, la fonction \( f\) possède un nombre fini de racines sur \( K\). Soit \( z_0\) l'un d'eux.

    Par le théorème~\ref{ThomcPOdd}\ref{ITEMooYWSOooHJtxGr}, nous avons, sur un voisinage de \( z_0\) :
    \begin{equation}
        f(z)=\sum_{n=0}^{\infty}a_n(z-z_0)^n.
    \end{equation}
    En particulier, \( 0=f(z_0)=a_0\). Donc \( a_0=0\). Soit \( k\), le plus petit naturel pour lequel \( a_k\neq 0\). Nous avons
    \begin{equation}
        f(z)=(z-z_0)^kg(z)
    \end{equation}
    avec \( g(z)= \sum_{n=0}^{\infty}a_{k+n}(z-z_0)^n.\). Vu que \( a_{k}\neq 0\) nous avons \( g(z_0)\neq 0\). Montrons à présent que \( g\) est holomorphe sur un voisinage de \( z_0\). Vu que la série définissant \( g\) est une sous-série d'une série convergente sur un voisinage, elle converge sur un voisinage et la proposition~\ref{PropRZCKeO} nous dit que \( g\) est \( \eC\)-dérivable. C'est-à-dire holomorphe par définition.

        \item[Autour de toutes les racines]

            Soient \( (z_i)\) les racines (en nombre fini). Pour chaque \( i\) nous avons une boule \( B(z_i,r_i)\) sur laquelle \( f=P_ig_i\) où \( P_i\) est un polynôme de la forme \( (z-z_i)^k\) et \( g_i\) est holomorphe sur \( B(z_i,r_i)\). Nous définissons la fonction suivante :
            \begin{equation}
                h(z)=\begin{cases}
                    \dfrac{ f(z) }{ \prod_kP_k(z) }    &   \text{si } z\neq z_i\\
                    \dfrac{ g_i }{ \prod_{k\neq i}P_k(z) }    &    \text{si } z=z_i.
                \end{cases}
            \end{equation}
            Cette fonction ne s'annule jamais. Mais est-elle holomorphe ?

            Si \( z\neq z_i\) (sous-entendu : pour tout \( i\)), alors sur un voisinage, \( h=f/\prod P_k\) qui est un quotient de fonctions holomorphes dont le dénominateur ne s'annule pas. Elle est donc holomorphe sur ce voisinage par le lemme~\ref{LEMooVDXOooUyFHXZ}.

            Pour les autres notons que pour tout \( z\in B(z_i,r_i)\),
            \begin{equation}
                h=\frac{ g_i }{\prod_{k\neq i}P_k}.
            \end{equation}
            Cela est encore un quotient dont le dénominateur ne s'annule pas\footnote{Nous avons choisi les \( r_i\) de telle sorte que les boules ne s'intersectent pas.}.

        \item[La réponse]

            Nous avons, pour tout \( z\in K\) :
            \begin{equation}
                f(z)=h(z)\prod_{k}P_k(z).
            \end{equation}

    \end{subproof}
\end{proof}

Afin de détendre l'atmosphère, nous allons laisser tomber l'analyse quelques instants et prouver un résultat d'algèbre.
\begin{proposition}[\cite{MonCerveau,ooACALooBfyhba}]       \label{PROPooVWRPooGQMenV}
    L'anneau des fonctions holomorphes sur un compact\footnote{Être holomorphe sur un compact signifie qu'il existe une extension holomorphe à un ouvert contenant le compact.} donné de \( \eC\) est principal\footnote{Définition~\ref{DEFooGWOZooXzUlhK}}.
\end{proposition}

\begin{proof}
    Nous nommons \( A\) l'ensemble des fonctions holomorphes sur le compact \( K\), et \( J\) un anneau de \( A\).

    \begin{subproof}

        \item[\( A\) est un anneau]

            Le point délicat de la définition \ref{DefHXJUooKoovob} est le fait que la somme et le produit d'éléments de \( A\) sont des éléments de \( A\) parce que les résultats type «la somme de deux fonctions holomorphes est holomorphes» sont valides sur des ouverts alors que nous sommes ici sur un compact. Soient \( f\) et \( g\) dans \( A\); nous nommons \( \Omega_f\) et \( \Omega_g\) des ouverts contenant \( K\) tels que \( f\) est holomorphe sur \( \Omega_f\) et \( g\) sur \( \Omega_g\).

            L'ensemble \( \Omega_f\cap\omega_g\) est un ouvert (intersection d'ouverts) contenant \( K\) et sur lequel \( f\) et \( g\) sont holomorphes. Donc \( f+g\) et \( fg\) y sont holomophes.

        \item[Engendré par des polynômes]

            Pour chaque \( f\in J\) nous écrivons \( f=P_fh_f\) en vertu de la décomposition donnée par le lemme~\ref{LEMooYYZQooClmOgG}. Vu que \( h_f\) ne s'annule pas, \( 1/h_f\) est encore holomorphe sur \( K\) et nous déduisons que \( P_f=f/h_f\) est dans \( J\).  La partie
            \begin{equation}
                S=\{ P_f\tq f\in J \}
            \end{equation}
            est génératrice de \( J\) parce que, par construction, tous les éléments de \( J\) sont des produits d'éléments de \( S\) par des fonctions holomorphes sur \( K\) (donc, des éléments de \( A\)). Mais tous les éléments de \( S\) sont dans \( J\), donc \( (S)=J\).

        \item[Un polynôme pour tous les engendrer]

            Soit \( M\), l'idéal de \( \eC[X]\) engendré par \( S\). Attention : \( J\) est l'idéal de \( A\) engendré par \( S\). Mais l'idéal de \( \eC[X]\) engendré par \( S\) est peut-être autre chose.  Vu que \( \eC\) est un corps, le lemme~\ref{LEMooIDSKooQfkeKp} dit que \( \eC[X]\) est principal. Donc \( M\) est un idéal principal de \( \eC[X]\) et nous avons un polynôme \( p\in \eC[X]\) tel que
            \begin{equation}
                M=\eC[X]p.
            \end{equation}
            Si vous avez compris le chausse trappe, vous saurez pourquoi il faut écrire \( M=\eC[X]p\) et non utiliser l'écriture plus simple «\( M=(p)\)».

        \item[\( A\eC\lbrack X\rbrack=A\)]

            L'inclusion \( A\subset A\eC[X]\) est dûe au fait que \( 1\in \eC[X]\), et l'autre inclusion est le fait que \( \eC[X]\subset A\) alors que \( A\) est un anneau.

        \item[Suite des opérations]

    Nous avons :
    \begin{equation}
        J=AS\subset A\eC[X]p.
    \end{equation}
    Voilà une inclusion de montrée. Reste à faire l'autre.

    Vu que \( p\in J\) nous avons aussi \( Ap\subset J\). Et donc
    \begin{equation}
        A\eC[X]p = Ap\subset J.
    \end{equation}

    Avec ces deux inclusions, \( J=A\eC[X]p=Ap\). Donc \( J\) est engendré par un seul élément et est principal.
    \end{subproof}
\end{proof}

%---------------------------------------------------------------------------------------------------------------------------
\subsection{Prolongement de fonctions holomorphes}
%---------------------------------------------------------------------------------------------------------------------------

\begin{proposition} \label{PropDRnYkKP}
    Soient \( \Omega\) un ouvert de \( \eC\) et \( f\colon \Omega\to \eC\) une fonction holomorphe sur \( \Omega\setminus\{ a \}\) (\( a\in \Omega\)). Nous supposons qu'il existe \( r>0\) tel que \( f\) est bornée sur \( B(a,r)\cap\Omega\). Alors \( f\) se prolonge en une fonction holomorphe sur \( \Omega\).
\end{proposition}
Le théorème de prolongement de Riemann~\ref{ThoTLQOEwW} donnera plus d'informations.

\begin{proof}
    Nous définissons la fonction \( g\colon \Omega\to \eC\) par
    \begin{equation}
        g(z)=\begin{cases}
            (z-a)f(z)    &   \text{si } z\neq a\\
            0    &    \text{si } z=a.
        \end{cases}
    \end{equation}
    Sur \( \Omega\setminus\{ a \}\), la fonction \( g\) est holomorphe (produit de fonctions holomorphes), et elle est continue en \( a\). Par conséquent elle est holomorphe sur \( \Omega\). Nous la développons en série entière sur une boule \( B(a,r)\) :
    \begin{equation}
        g(z)=\sum_{n=0}^{\infty}c_n(z-a)^n.
    \end{equation}
    Nous avons \( g(a)=c_0=0\). Nous posons
    \begin{equation}
        \varphi(z)=\sum_{n=0}^{\infty}c_{n+1}(z-a)^n.
    \end{equation}
    Si \( z\neq a\), alors \( \varphi(z)=f(a)\) parce que \( \varphi(z)=g(z)/(z-a)\). Mais \( \varphi\) est continue en \( a\), et donc holomorphe en \( a\).

    La fonction \( \varphi\) est par conséquent un prolongement holomorphe de \( f\) en \( a\).
\end{proof}

%---------------------------------------------------------------------------------------------------------------------------
\subsection{Théorème de Runge}
%---------------------------------------------------------------------------------------------------------------------------

Le théorème que nous allons prouver n'est en réalité qu'une partie de ce qui est usuellement appelle le théorème de Runge.
\begin{theorem}[Théorème de Runge]\index{théorème!Runge}     \label{ThoMvMCci}
    Soit \( K\), un compact de \( \eC\) tel que \( \complement K\) soit connexe. Si \( a\in \complement K\) alors la fonction
    \begin{equation}
        \varphi_a(z)=\frac{1}{ z-a }
    \end{equation}
    est limite uniforme de polynômes sur \( K\).
\end{theorem}
\index{connexité!théorème de Runge}
\index{approximation!polynomiale}

\begin{proof}
    Nous considérons \( P(K)\), l'adhérence des polynômes sur \( K\) pour la norme uniforme (sur \( K\)). Nous devons montrer que pour tout \( a\in \complement K\), la fonction \( \varphi_a\) est dans \( P(K)\). Pour cela nous considérons l'ensemble
    \begin{equation}
        A=\{ a\in\complement K\tq \varphi_a\in P(K) \}
    \end{equation}
    et nous allons montrer qu'il est à la fois non vide, ouvert et fermé dans le connexe \( \complement K\).

    Je répète : nous allons prouver l'ouverture et la fermeture \emph{pour la topologie de \( \complement K\)}. Nous n'allons pas prouver que \( A\) est un ouvert de \( \eC\). Ce qui sera par conséquent prouvé est que \( A=\complement K\).

    \begin{subproof}
    \item[Non vide] Soit \( R=\sup_{z\in K}| z |\) et \( a\in \complement K\) tel que \( | a |>R\). Nous avons
        \begin{equation}
                \varphi_a(z)=\frac{1}{ a }\frac{1}{ \frac{ z }{ a }-1 }
                =-\frac{1}{ a }\frac{1}{ 1-\frac{ z }{ a } }
                =-\frac{1}{ a }\sum_{k=0}^{\infty}\left( \frac{ z }{ a } \right)^k
                =\sum_{k=0}^{\infty}\frac{ z^k }{ a^{k+1} }.
        \end{equation}
        Ici la convergence de la série et sa limite sont assurées par le fait que \( | z/a |<1\) par choix de \( R\) et \( a\). La suite de polynômes
        \begin{equation}
            P_n(z)=\sum_{k=0}^n\frac{ z^k }{ a^{k+1} }
        \end{equation}
        converge uniformément sur \( B(0,R)\) et en particulier sur \( K\). Donc \( P_n\to \varphi_a\).

    \item[Fermé]

        Nous allons montrer que la fermeture de \( A\) (dans \( \complement K\)) est inclue dans \( A\), et donc qu'elle est égale à \( A\) et donc que \( A\) est fermé. Par le lemme~\ref{LemkUYkQt}, la fermeture de \( A\) dans \( \complement K\) est l'ensemble \( \bar A\cap\complement K\) où \( \bar A\) est la fermeture de \( A\) au sens usuel.

        Bref, soit \( a\in \bar A\cap\complement K\), et montrons que \( \varphi_a\in \overline{ P(K) }\). Vu que \( P(K)\) est déjà une fermeture, nous aurons en fait \( \varphi_a\in P(K)\) et donc \( a\in A\), ce qui signifierait que \( \bar A\cap\complement A=A\) et donc que \( A\) est fermé.

        Au travail.

        Soit \( (a_n)\in A\) une suite convergente vers \( a\). Soit aussi \( d=d(a,K)\); on a \( d>0\) parce que \( K\) est compact et \( a\) est hors de \( a\) alors le complémentaire de \( K\) est ouvert. Nous choisissons en plus la suite \( a_n\) pour avoir \( | a_n-a |<\frac{ d }{2}\); au pire on prend la queue de suite. Soit \( z\in K\); nous avons
        \begin{equation}    \label{EqYHWQhI}
            | \varphi_{a_n}(z)-\varphi_a(z) |=\left| \frac{1}{ z-a_n }-\frac{1}{ z-a } \right| =  \left| \frac{ a_n-a }{ (z-a_n)(z-a) } \right|.
        \end{equation}
        Vu que \( a_n\in B(a,\frac{ d }{2})\) et que \( z\in K\) et \( d=d(a,K)\) nous avons \( | a_n-z |\geq \frac{ d }{2}\); et aussi \( | a-z |\geq \frac{ d }{2}\). Nous pouvons donc majorer \eqref{EqYHWQhI} par
        \begin{equation}
            | \varphi_{a_n}(z)-\varphi_a(z) |\leq 2\frac{ | a_n-a | }{ d^2 }.
        \end{equation}
        Donc nous avons
        \begin{equation}
            \| \varphi_a-\varphi_{a_n} \|_K\leq 2\frac{ | a_n-a | }{ d^2 }\to 0
        \end{equation}
        où la norme \( \| . \|_K\) est la norme supremum sur \( K\). Donc \( a\in \overline{ P(K) }=P(K)\) et \( A\) est fermé.

    \item[Ouvert] Vu que \( K\) est compact, il est fermé et donc \( \complement K\) est ouvert. Par conséquent, ainsi que précisé dans l'exemple~\ref{ExloeyoR}, les ouverts de \( \complement K\) sont les ouverts de \( \eC\) contenus dans \( \complement K\). Afin de prouver que \( A\) est ouvert, nous prenons  \( a\in A\) et nous cherchons une boule (au sens de \( \eC\)) autour de \( a\) qui serait incluse dans \( A\).

        Soit donc \( h\in \eC\) «petit» dans un sens que nous allons préciser plus tard. Encore une fois nous posons \( d=d(a,K)\). Nous avons
        \begin{equation}        \label{EqgBSxFB}
            \varphi_{a+h}(z)=\frac{1}{ z-a-h }=\frac{1}{ z-a }\frac{1}{ 1-\frac{ h }{ z-a } }=\sum_{k=0}^{\infty}\frac{ h^k }{ (z-a)^{k+1} }.
        \end{equation}
        Déjà ici nous demandons \( h<\sup_{z\in K}| z-a |\). Puisque \( | z-a |>d\), nous avons alors
        \begin{equation}
            | \varphi_{a+h}(z) |\leq \sum_{k=0}^{\infty}\frac{ h^k }{ d^{k+1} }<\infty.
        \end{equation}
        Cela pour dire que la somme à droite de \eqref{EqgBSxFB} converge bien pourvu que \( h\) soit bien petit. Nous pouvons donc poursuivre :
        \begin{equation}    \label{EqTSSdttylSDX}
            \varphi_{a+h}(z)=\sum_{k=0}^{\infty}\frac{ h^k }{ (z-a)^{k+1} }=\sum_{k=0}^{\infty}h^k\varphi_a(z)^{k+1}.
        \end{equation}
        Nous montrons maintenant que la convergence de la somme \eqref{EqTSSdttylSDX} est en réalité uniforme en \( z\). En effet
        \begin{subequations}
            \begin{align}
                \big| \varphi_{a+h}(z)-\sum_{k=0}^Nh^k\varphi_a(z)^{k+1} \big|&=\big| \sum_{k=N+1}^{\infty}h^k\varphi_a(z)^{k+1} \big|\\
                &\leq\sum_{k=N+1}^{\infty}| h |^k| \varphi_a(z) |^{k+1}.
            \end{align}
        \end{subequations}
        Étant donné que \( \varphi_a\) est continue sur le compact \( K\), elle y est majorée en module; on peut même être plus précis :
        \begin{equation}
            |\varphi_a(z)|=\frac{1}{ | z-a | }\leq \frac{1}{ d }.
        \end{equation}
        Nous pouvons donc écrire
        \begin{equation}
            \big| \varphi_{a+h}(z)-\sum_{k=0}^Nh^k\varphi_a(z)^{k+1} \big|\leq\frac{1}{ d }\sum_{k=N+1}^{\infty}\left| \frac{ h }{ d } \right|^k.
        \end{equation}
        Étant donné que la somme \( \sum_{k=0}^{\infty}| h/d |^k\) converge, la limite \( N\to \infty\) est nulle et nous avons
        \begin{equation}
            \lim_{N\to \infty} \| \varphi_{a+h}-\sum_{k=0}^Nh^k\varphi_a^{k+1} \|_K=0.
        \end{equation}
        Pour avoir \( \varphi_{a+h}\in P(K)\), il faut encore savoir si les fonctions \( \varphi_a^{k}\) sont dans \( P(K)\) pour tout \( k\). Dans ce cas pour chaque \( N\) la somme sera encore dans \( P(K)\) et \( \varphi_{a+h}\) sera limite uniforme d'éléments de \( P(K)\).

        Par hypothèse, \( \varphi_a\in P(K)\); soit \( P_n\) une suite de polynômes qui converge uniformément vers \( \varphi_a\). Nous allons montrer qu'alors la suite de polynômes \( P_n^k\) converge uniformément vers \( \varphi_a^k\). Soit \( n\) tel que \( \| P_n-\varphi_a \|_{K}\leq \epsilon\) et utilisons le produit remarquable\index{produit remarquable}
        \begin{equation}
            a^k-b^k=(a-b)\sum_{i=0}^{k-1}a^ib^{k-1-i}
        \end{equation}
        pour obtenir
        \begin{equation}
            | P_n(z)^k-\varphi_a(z)^k |\leq | P_n(z)-\varphi_a(z) |\sum_{i=0}^{k-1}| P_n(z)^i\varphi_a(z)^{k-1-i} |.
        \end{equation}
        Vu que \( P_n\) et \( \varphi_a\) sont continues sur le compact \( K\), on peut majorer la somme par une constante \( M\), et il restera
        \begin{equation}
            | P_n(z)^k-\varphi_a(z)^k |\leq M | P_n(z)-\varphi_a(z) |,
        \end{equation}
        ou encore
        \begin{equation}
            \| P_n^k-\varphi_a^k \|\leq M\epsilon.
        \end{equation}
        Cela prouve que \( \varphi_a^{k}\in P(K)\) et donc que \( \varphi_{a+h}\) est limite uniforme (sur \( K\)) d'éléments de \( P(K)\) et donc fait partie de \( P(K)\) lui aussi.

        Ceci achève de prouver que \( A\) est ouvert dans \( \complement K\).
    \item[Conclusion]

        L'ensemble \( A\) est non vide, ouvert et fermé dans \( \complement K\), donc il est égal à \( \complement K\). Le théorème est ainsi démontré.
    \end{subproof}
\end{proof}

%+++++++++++++++++++++++++++++++++++++++++++++++++++++++++++++++++++++++++++++++++++++++++++++++++++++++++++++++++++++++++++
\section{Intégrales de fonctions holomorphes}
%+++++++++++++++++++++++++++++++++++++++++++++++++++++++++++++++++++++++++++++++++++++++++++++++++++++++++++++++++++++++++++

Nous commençons par le lemme technique.
\begin{lemma}[\cite{Holomorphieus}]       \label{LemNAnweA}
    Soit \( f\) une fonction holomorphe sur \( B(z_0,r_0)\). Pour tout \( z\in B(z_0,r)\) (avec \( r<r_0\)) nous avons
    \begin{equation}
        | f'(z) |\leq \frac{ r }{ \big( r-| z-z_0 | \big)^2 }\max\big\{ f(z_0+r e^{i\theta}) \big\}_{\theta\in \eR}.
    \end{equation}
\end{lemma}

\begin{proof}
    Par translation nous pouvons supposer que \( z_0=0\). Étant donné que \( f\) est holomorphe, elle admet un développement en séries entières
    \begin{equation}
        f(z)=\sum_{n=0}^{\infty}a_nz^n
    \end{equation}
    et nous notons \( M=\max\{ f(z)\tq z\in \overline{ B(0,r) } \}\). Nous avons \( r^n| a_n |\leq M\). Par conséquent
    \begin{subequations}
        \begin{align}
            | f'(z) |&=\left| \sum_{n=1}^{\infty}na_nz^{n-1} \right| \\
            &\leq\frac{1}{ r }\sum r^n| a_n |n\left( \frac{ | z | }{ r } \right)^{n-1}\\
            &<\frac{ M }{ r }\sum n\left( \frac{ | z | }{ r } \right)^{n-1}.
        \end{align}
    \end{subequations}
    À ce point nous devons utiliser la série de l'exemple~\ref{ExGxzLlP}. Nous avons alors
    \begin{equation}
        | f'(z) |\leq \frac{ M }{ r }\frac{ 1 }{ \left( 1-\frac{ | z | }{ r } \right)^2 }=\frac{ Mr }{ (r-| z |)^2 }.
    \end{equation}
\end{proof}

\begin{theorem}[Holomorphie sous l'intégrale\cite{Holomorphieus}] \label{ThopCLOVN}
    Soit un espace mesuré \( (\Omega,\mu)\), un ouvert \( A\) dans \( \eC\) et une fonction \( f\colon A\times \Omega\to \eC\). Nous voulons étudier la fonction
    \begin{equation}
        F(z)=\int_{\Omega}f(z,\omega)d\mu(\omega)
    \end{equation}
    pour tout \( z\in A\). Nous supposons que
    \begin{enumerate}
        \item
            la fonction \( f(.,\omega)\) est holomorphe sur \( A\) pour chaque \( \omega\).
        \item
            La fonction \( f(z,.)\) est mesurable sur \( (\Omega,\mu)\).
        \item
            Pour tout compact \( K\subset A\), il existe une fonction \( g_K\colon \Omega\to \eR\) telle que \( | f(z,\omega) |\leq g_K(\omega)\) et telle que
            \begin{equation}
                \int_{\Omega}g_K(\omega)d\mu(\omega)
            \end{equation}
            existe.
    \end{enumerate}
    Alors la fonction \( F\) est holomorphe et
    \begin{equation}
        F'(z)=\int_{\Omega}\frac{ \partial f }{ \partial z }(z,\omega)d\mu(\omega).
    \end{equation}
\end{theorem}

\begin{proof}
    Soient \( z_0\in A\) et \( r>0\) tels que \( K=\overline{ B(z_0,r) }\subset A\). Pour chaque \( \omega\in \Omega\) nous considérons la fonction
    \begin{equation}
        \begin{aligned}
            f_{\omega}\colon \overline{ B(z_0,r) }&\to \eC \\
            z&\mapsto f(z,\omega).
        \end{aligned}
    \end{equation}
    Étant donné que \( \overline{ B(z_0,r) }\) est compacte, la fonction \( | f_{\omega} |\) est majorée par un nombre que nous notons \( f_K(\omega)\) qui est indépendant de \( z\) (pour autant que $z\in K$). Nous désignons par \( S(z_0,r)\) la frontière de la boule \( B(z_0,r)\). Étant donné que la majoration est valable sur \( \overline{ B(z_0,r) }\), nous avons en particulier
    \begin{equation}
        | f_{\omega}(z) |\leq f_K(\omega)
    \end{equation}
    pour tout \( z\in S\). En utilisant la lemme~\ref{LemNAnweA} nous avons
    \begin{subequations}
        \begin{align}
            | f'_{\omega}(z) |&\leq \frac{ r }{ (r-| z-z_0 |)^2 }\max\{ f(z_0+r e^{i\theta}) \}_{\theta\in \eR}\\
            &\leq \frac{ rf_K(\omega) }{ (r-| z-z_0 |)^2 }.
        \end{align}
    \end{subequations}
    Cette majoration est valable pour tout \( z\in B(z_0,r)\). Si nous supposons de plus que \( z\in B(z_0,r/2)\)  nous avons
    \begin{equation}
        | f'(z) |\leq \frac{ rf_K(\omega) }{ \left( r-\frac{ r }{2} \right)^2 }=\frac{ 4 }{ r }f_K(\omega).
    \end{equation}
    Étant donné que la boule \( B(z_0,r/2)\) est convexe, la fonction \( f_{\omega}\) est Lipschitz et pour tout \( h\in \eC\) tel que \( | h |<r/2\) nous avons
    \begin{equation}
        \left| \frac{ f_{\omega}(z_0+h)-f_{\omega}(z_0) }{ h } \right| \leq \frac{ 4f_K(\omega) }{ r }.
    \end{equation}
    Soit maintenant une suite \( (h_n)\) qui converge vers \( 0\) dans \( \eC\). Nous considérons la suite de fonctions correspondantes
    \begin{equation}
        g_n(\omega)=\frac{ f(z_0+h_n,\omega)-f(z_0,\omega) }{ h_n }.
    \end{equation}
    Cette suite de fonctions vérifie la convergence ponctuelle
    \begin{equation}
        g_n(\omega)\to\frac{ \partial f }{ \partial z }(z_0,\omega).
    \end{equation}
    De plus \( g_n\) est une fonction (de \( \omega\)) dominée par \( \frac{ 4f_K }{ r }\) qui est intégrable. Par conséquent le théorème de la convergence dominée\footnote{Lebesgue, théorème \ref{ThoConvDomLebVdhsTf}.} nous indique que
    \begin{equation}
        \int_{\Omega}g_n(\omega)d\mu(\omega)\to \int_{\Omega}\frac{ \partial f }{ \partial z }(z_0,\omega)d\mu(\omega),
    \end{equation}
    tandis que
    \begin{equation}
        F'(z)=\lim_{n\to \infty} \frac{ F(z_0+h_n)-F(z_0) }{ h_n }=\lim_{n\to \infty} \int_{\Omega}g_N(\omega)d\mu(\omega).
    \end{equation}
\end{proof}

\begin{corollary}       \label{CorNxTjEj}
    Si \( f\) est une fonction holomorphe sur l'ouvert \( \xi\) contenant la fermeture de la boule \( B(z_0,r)\), alors pour tout \( z\) dans \( B(z_0,\rho)\) (\( \rho<r\)) les dérivées de \( f\) s'expriment pas la formule suivante :
    \begin{equation}        \label{EQooBPIQooNhOTtB}
        f^{(k)}(z)=\frac{1}{ 2\pi i }\int_{\partial B(z_0,r)}\frac{ f(\xi) }{ (\xi-z)^{k+1} }d\xi.
    \end{equation}
\end{corollary}
\index{compacité}

\begin{proof}
    Nous faisons par récurrence.
    \begin{subproof}
    \item[Pour la dérivée première]
        Nous appliquons le théorème~\ref{ThopCLOVN} à la fonction
        \begin{equation}
            g(z,\xi)=\frac{ f(\xi) }{ \xi-z }
        \end{equation}
        avec \( \xi=\partial B(z_0,r)\) et \( A=B(z_0,\rho)\) avec \( \rho<r\). Étant donné que \( f\) est holomorphe, elle est continue et donc bornée sur tout compact \( K\subset A\) par une constante \( M\) (qui dépend du compact choisi).  D'autre part, nous avons toujours \( | \xi-z |>r-\rho\) et donc
        \begin{equation}
            | g(z,\xi) |\leq \frac{ M }{ r-\rho }.
        \end{equation}
        La fonction constante \( g_K=\frac{ M }{ r-\rho }\) est évidemment intégrable. Le théorème conclut que \( f\) est holomorphe (cela, nous le savions déjà\footnote{Et cela fournit une preuve alternative à la réciproque du théorème de Cauchy : une fonction continue qui vérifie la formule de Cauchy est holomorphe.}), et
        \begin{equation}
            f'(z)=\frac{1}{ 2i\pi }\int_{\partial B}\frac{ f(\xi) }{ (\xi-z)^2 }d\xi.
        \end{equation}

        \item[Les dérivées suivantes]
            Pour la récurrence\cite{ooKZJHooZhNpkf} nous supposons que
            \begin{equation}
                f^{(k)}(z)=\frac{k!}{ 2i\pi }\int_{\partial B(z_0,r)}\frac{ f(\xi) }{ (\xi-z)^{k+1} }d\xi,
            \end{equation}
            et nous tentons de calculer \( f^{(k+1)}(z)\). Pour cela nous paramétrons l'intégrale de façon très usuelle :
            \begin{equation}
                f^{(k)}(z)=\frac{ k! }{ 2\pi i }\int_0^{2\pi}\frac{ f(r e^{it}) }{ (r e^{it}-z) }ir e^{it}dt.
            \end{equation}
            Nous permettons de permuter la \( \eC\)-dérivation (par rapport à \( z\)) et l'intégrale en vertu de la proposition~\ref{PROPooZCLYooUaSMWA} appliquée à la fonction
            \begin{equation}
                g(t,z)=\frac{ f(r e^{it}) }{ (r e^{it}-z)^k+1 }ir e^{it}.
            \end{equation}
            Cela donne
            \begin{equation}
                f^{(k+1)}(z)=\frac{ k! }{ 2i\pi }\int_0^{2\pi}f(r e^{it})ir e^{it}\frac{ k+1 }{ (r e^{it}-z)^{k+1} }dt=\frac{ (k+1)! }{ 2i\pi }\int_{\partial B}\frac{ f(\xi) }{ (\xi-z)^{k+2} }d\xi.
            \end{equation}
    \end{subproof}
\end{proof}

\begin{theorem}     \label{THOooSULFooHTLRPE}
    Si \( f\) est une fonction holomorphe sur le disque ouvert \( B(z_0,R)\) alors
    \begin{equation}
        f(z)=\sum_{n=0}^{\infty}\frac{ f^{(n)}(z_0) }{ n! }(z-z_0)^n
    \end{equation}
    et cette série converge uniformément sur tout compact.
\end{theorem}

\begin{proof}
    Sans perte de généralité nous supposons que \( z_0=0\). La formule de Cauchy (théorème~\ref{ThoUHztQe}) fournit, pour \( z\in B(0,R)\),
    \begin{equation}
        f(z)=\frac{1}{ 2\pi i }\int_{\partial B}\frac{ f(\xi) }{ \xi-z }d\xi=\frac{1}{ 2\pi i }\int_{\partial B}\frac{ f(\xi) }{ 1-(z/\xi) }\frac{ d\xi }{ \xi }.
    \end{equation}
    En particulier notons que \( z\in B(0,R)\) alors que \( \xi\) est sur le bord de cette boule ouverte. Donc \( | \xi |>| z |\) pour tous les \( \xi\) et \( z\) qui interviennent. Nous utilisons la série géométrique
    \begin{equation}
        \frac{1}{ 1-(z/\xi) }=\sum_{n=0}^{\infty}\left( \frac{ z }{ \xi } \right)^n.
    \end{equation}

    \begin{subproof}
        \item[Permuter une intégrale et une somme]
            En utilisant la mesure de comptage\footnote{Définition~\ref{DEFooILJRooByDzhs}.} sur \( \eN\) (qui est \( \sigma\)-finie), nous pouvons écrire
            \begin{equation}        \label{EQooWOLOooFHSrsx}
                \int_{\partial B}\sum_{n=0}^{\infty}\frac{ z^nf(\xi) }{ \xi^{n+1} }d\xi= \int_{\partial B}\left( \int_{\eN}g(\xi,n)dm(n) \right)d\xi
            \end{equation}
            où \begin{equation}
                \begin{aligned}
                    g\colon \partial B\times \eN&\to \eC \\
                    (\xi,n)&\mapsto \frac{ z^nf(\xi) }{ \xi^{n+1} }.
                \end{aligned}
            \end{equation}
            Nous allons permuter les intégrales en utilisant le théorème de Fubini, selon la procédure décrite en~\ref{NORMooKIRJooPvyPWQ}. Nous commençons par l'intégrale sur \( \eN\) :
            \begin{equation}        \label{EQooLQLOooKiSAKH}
                \int_{\eN}| g(n,\xi) |=| \frac{ f(\xi) }{ \xi } |\sum_{n\in \eN}| \frac{ z }{ \xi } |^n=\frac{1}{ R }| f(\xi) |\frac{1}{ 1-| z |/R }.
            \end{equation}
            Ici nous avons utilisé \( | \xi |=R\). Notons que \( z\) est fixé depuis longtemps à l'intérieur de la boule de rayon \( R\) de telle sorte que \( | z/\xi |\) est une constante strictement inférieure à \( 1\).

            L'intégrale sur \( \xi\in \partial B\) n'a pas à être effectuée explicitement : nous nous contentons de prouver qu'elle est finie.
            La fonction \( f\) est continue sur le compact \( \partial B\). Cela parce que \( B\) est une boule fermée dans l'ouvert \( \Omega\) sur lequel \( f\) est continue. Au final l'expression à droite de \eqref{EQooLQLOooKiSAKH} est bornée sur le compact \( \partial B\) et son intégrale donne un nombre fini.

            Tout ceci pour invoquer le corolaire~\ref{CorTKZKwP} qui nous indique que \( g\in L^1(\eN\times \partial B)\).

            Une fois \( g\) intégrable sur l'espace produit \( \eN\times \partial B\), nous pouvons utiliser Fubini~\ref{ThoFubinioYLtPI} pour permuter les intégrales.
    \end{subproof}

    Une fois la somme et l'intégrale permutées, nous avons
    \begin{equation} \label{EqXSgZGw}
            f(z)=\frac{1}{ 2\pi i }\sum_{n=0}^{\infty}\int_{\partial B}\frac{ z^nf(\xi) }{ \xi^{n+1} }
            =\sum_{n=0}^{\infty}\left( \frac{1}{ 2\pi i }\int_{\partial B}\frac{ f(\xi) }{ \xi^{n+1} } \right)z^n.
    \end{equation}
    Nous devons maintenant montrer que ce qui se trouve dans la grande parenthèse vaut \( f^{(n)}(0)/n!\). Cela est immédiat en comparant avec la formule \eqref{EQooBPIQooNhOTtB}.

\end{proof}

\begin{proposition}[Morera \cite{NEBgfg}]   \label{ThoRckxes}
    Soit \( \Omega\) ouvert dans \( \eC\) et \( f\) continue. Si
    \begin{equation}
        \int_{\partial T}f=0
    \end{equation}
    pour tout triangle (plein) \( T\) contenu dans \( \Omega\), alors \( f\) est holomorphe sur \( \Omega\).
\end{proposition}

\begin{proof}
    Il est suffisant de prouver que \( f\) est holomorphe sur toute boule ouverte \( B(a,r)\) inclue dans \( \Omega\). Nous posons, pour tout \( z\in B(a,r)\),
    \begin{equation}
        F(z)=\int_{[p,z]}f,
    \end{equation}
    et nous considérons le chemin triangulaire \( a\to z\to z+h\to a\) où \( h\in \eC\) est choisi assez petit pour que \( z+h\in B(a,r)\). L'intégrale sur le triangle étant nulle, nous avons
    \begin{equation}
        0=\int_{a\to z}f+\int_{z\to z+h}f+\int_{z+h\to a}f,
    \end{equation}
    c'est-à-dire
    \begin{equation}
        F(z+h)-F(z)=\int_{z\to z+h}f.
    \end{equation}
    En paramétrant le chemin par \( z+th\) avec \( t\in\mathopen[ 0 , 1 \mathclose]\), et en tenant compte de la remarque~\ref{RemiqswPd},
    \begin{subequations}
        \begin{align}
            F'(z)&=\lim_{h\to 0} \frac{ F(z+h)-F(z) }{ h }\\
            &=\lim_{h\to 0} \frac{1}{ h }\int_0^1f(z+th)hdt,
        \end{align}
    \end{subequations}
    ce qui prouve que \( F\) est dérivable et \( F'=f\). Par définition (\ref{DefMMpjJZ}), \( F\) est holomorphe, et donc \( C^{\infty}\) par le théorème~\ref{ThomcPOdd}. Du coup \( f\) est également \(  C^{\infty}\) et donc en particulier holomorphe.
\end{proof}

%---------------------------------------------------------------------------------------------------------------------------
\subsection{Mesure de Radon}
%---------------------------------------------------------------------------------------------------------------------------

\begin{definition}
    Une \defe{mesure de Radon}{mesure!de Radon} sur un compact \(  K\) de \( \eC\) est une forme linéaire continue sur \( C(K)\). Si \( \mu\) est une mesure de Radon, on définit la \defe{transformée de Cauchy}{transformée!de Cauchy} de \( \mu\) par
    \begin{equation}
        \begin{aligned}
            \hat \mu\colon \eC\setminus K&\to \eC \\
            z&\mapsto -\frac{1}{ \pi }\mu\left( \frac{1}{ \xi-z } \right).
        \end{aligned}
    \end{equation}
\end{definition}

\begin{theorem}     \label{ThoJVNTzn}
    Si \( \mu\) est une mesure de Radon sur \( K\) alors \( \hat \mu\) est infiniment \( \eC\)-dérivable sur \( \Omega=\eC\setminus K\) et nous avons
    \begin{equation}
        \hat\mu^{(n)}(z)=-\frac{ n! }{ \pi }\mu\left( \frac{1}{ (\xi-z)^{n+1} } \right).
    \end{equation}
\end{theorem}

Cette théorie permet de fournir une démonstration plus technologique du corolaire~\ref{CorNxTjEj}.
\begin{lemma}
    Si \( f\) est holomorphe sur \( \Omega\) et si \( B\) est une boule fermée dans \( \Omega\) alors pour tout \( z\in \Int(B)\) nous avons
    \begin{equation}
        f^{(k)}(z)=\frac{ k! }{ 2i\pi }\int_{\partial B}\frac{ f(\xi) }{ (\xi-z)^{k+1} }d\xi.
    \end{equation}
\end{lemma}

\begin{proof}
    Appliquer le théorème~\ref{ThoJVNTzn} à la mesure de Radon
    \begin{equation}
        \mu(\phi)=\int_{\partial B}\phi(\xi)d\xi.
    \end{equation}
\end{proof}

Tout ce petit monde à propos de la mesure de Radon permet également de redémontrer que

    \begin{equation}
        \left( \frac{1}{ 2\pi i }\int_{\partial B}\frac{ f(\xi) }{ \xi^{n+1} } \right)=f^{(n)}(0)/n!,
    \end{equation}
    comme nous l'avons déjà fait autour de l'équation \eqref{EqXSgZGw}. Nous utilisons le théorème de Radon~\ref{ThoJVNTzn} à la mesure
    \begin{equation}
        \mu(\phi)=\int_{\partial B}\phi(\xi)d\xi.
    \end{equation}
    La transformée de Cauchy est
    \begin{equation}        \label{EqTzkmeL}
        \hat \mu(z)=-\frac{1}{ \pi }\mu\left( \frac{1}{ \xi-z } \right)=-\frac{1}{ \pi }\int_{\partial B}\frac{1}{ \xi-z }d\xi,
    \end{equation}
    et le théorème assure que
    \begin{equation}
        \hat\mu^{(n)}(z)=-\frac{ n! }{ \pi }\mu\left( \frac{1}{ (\xi-z)^{n+1} } \right)=-\frac{ n! }{ \pi }\int_{\partial B}\frac{ 1 }{ (\xi-z)^{n+1} }d\xi.
    \end{equation}
    En comparant la formule \eqref{EqTzkmeL} avec la formule de Cauchy nous voyons que \( \hat\mu(z)=-2i f(z)\). Par conséquent
    \begin{equation}
        f^{(n)}(z)=-\frac{1}{ 2i }\hat\mu^{(n)}(z)=\frac{ n! }{ 2\pi i }\int_{\partial B}\frac{1}{ (\xi-z)^{n+1} }d\xi,
    \end{equation}
    et
    \begin{equation}
        f^{(n)}(0)=\frac{ n! }{ 2\pi i }\int_{\partial B}\frac{1}{ \xi^{n+1} }d\xi.
    \end{equation}

%+++++++++++++++++++++++++++++++++++++++++++++++++++++++++++++++++++++++++++++++++++++++++++++++++++++++++++++++++++++++++++
\section{Conditions équivalentes à l'holomorphie}
%+++++++++++++++++++++++++++++++++++++++++++++++++++++++++++++++++++++++++++++++++++++++++++++++++++++++++++++++++++++++++++

Nous nous proposons de lister les conditions que nous avons vues être équivalentes à l'holomorphie.

\begin{theorem}
    Soient \( \Omega\) un ouvert de \( \eC\) et \( f\colon \Omega\to \eC\) une fonction continue. Les conditions suivantes sont équivalentes.
    \begin{enumerate}
        \item   \label{ItemOtPcTb}
            \( f\) est holomorphe.
        \item   \label{ItemHWRnxx}
            Pour tout triangle (plein) \( T\) contenu dans \( \Omega\), \( \int_Tf=0\).
        \item   \label{ItempBBPVv}
            \( f\) est \( \eC\)-dérivable.
        \item   \label{ItemmLhzbB}
            \( f\) est \(  C^{\infty}\)
        \item   \label{ItemCCrSrLj}
            \( \frac{ \partial f }{ \partial \bar z }=0\); ce sont les équations de Cauchy-Riemann.
        \item   \label{ItemEvxRSn}
            La \( 1\)-forme différentielle \( f(z)dz\) est localement exacte.
        \item   \label{ItemVSCHtY}
            Pour toute boule \( B(a,r)\) contenue dans \( \Omega\) nous avons
            \begin{equation}
                f(a)=\frac{1}{ 2\pi i }\int_{\partial B(a,r)}\frac{ f(z) }{ z-a }dz.
            \end{equation}
    \end{enumerate}

    La fonction \( f\) est holomorphe en \( z_0\) si et seulement si il existe un voisinage \( B(z_0,r)\) de \( z_0\) et des nombres \( a_k\) tels que sur la boule,
    \begin{equation}
        f(z)=\sum_{n=0}^{\infty}a_n(z-z_0)^n.
    \end{equation}
    Dans ce cas, \( f\) est holomorphe sur toute la boule.
\end{theorem}

\begin{proof}
   ~\ref{ItemOtPcTb} implique~\ref{ItemHWRnxx} est le lemme de Goursat~\ref{LemwbwbUR}.~\ref{ItemHWRnxx} implique~\ref{ItemOtPcTb} est le théorème de Morera~\ref{ThoRckxes}.

   ~\ref{ItempBBPVv} est la définition de l'holomorphie, définition~\ref{DefMMpjJZ}.

   ~\ref{ItemmLhzbB} implique~\ref{ItemOtPcTb} est un a fortiori sur la définition.~\ref{ItemOtPcTb} implique~\ref{ItemmLhzbB} est contenu dans le théorème de développement en série entière~\ref{ThomcPOdd}.

    L'équivalence entre~\ref{ItemCCrSrLj} et l'holomorphie est le théorème~\ref{PropkwIQwg}.

    L'équivalence entre~\ref{ItemEvxRSn} et~\ref{ItemOtPcTb} est la proposition~\ref{PropZOkfmO}.

    L'équivalence entre~\ref{ItemOtPcTb} et~\ref{ItemVSCHtY} est d'une part le théorème~\ref{ThomcPOdd} et d'autre part le corolaire~\ref{CorwfHtJu}.

    En ce qui concerne la dernière affirmation, si \( f\) est holomorphe en \( z_0\), alors le théorème~\ref{ThomcPOdd}\ref{ITEMooYWSOooHJtxGr} donne la série. Si au contraire nous avons la série, la proposition~\ref{PropRZCKeO} nous donne le résultat.
\end{proof}

% This is part of Mes notes de mathématique
% Copyright (c) 2012-2013,2016-2019
%   Laurent Claessens
% See the file fdl-1.3.txt for copying conditions.

%+++++++++++++++++++++++++++++++++++++++++++++++++++++++++++++++++++++++++++++++++++++++++++++++++++++++++++++++++++++++++++
\section{Singularités, pôles et méromorphe}
%+++++++++++++++++++++++++++++++++++++++++++++++++++++++++++++++++++++++++++++++++++++++++++++++++++++++++++++++++++++++++++

\begin{definition}
    Si \( f\) est holomorphe sur un ouvert \( \Omega\), alors une \defe{singularité}{singularité} de \( f\) est un point isolé du bord de \( \Omega\).
    \begin{enumerate}
        \item
            La singularité est \defe{effaçable}{singularité!effaçable} si la fonction \( f\) s'y prolonge en une fonction holomorphe.
        \item
            La singularité \( Z\) est un \defe{pôle}{singularité!pôle} d'ordre \( k\) de \( f\) si elle n'est pas effaçable et si la fonction \( z\mapsto (z-Z)^kf(z)\) se prolonge en une fonction holomorphe en \( Z\).
    \end{enumerate}
\end{definition}

\begin{example}
    La fonction
    \begin{equation}
        z\mapsto \frac{ \sin(z) }{ z }
    \end{equation}
    n'est pas définie en \( z=0\), mais elle s'y prolonge en une fonction continue en posant \( f(0)=1\).
\end{example}
%TODO : dans \eC je ne sais pas si c'est facile à montrer. De toutes façons, il faudrait déjà définir le sinus.

\begin{proposition}
    Une singularité de \( f\) est un pôle si et seulement si
    \begin{equation}
        \lim_{z\to Z}f(z)=\infty.
    \end{equation}
\end{proposition}

Le théorème suivant compète la proposition~\ref{PropDRnYkKP}.
\begin{theorem}[Prolongement de Riemann]    \label{ThoTLQOEwW}
    Soient \( f\colon \Omega\to \eC\) et \( Z\) une singularité de \( f\). Nous avons équivalence de
    \begin{enumerate}
        \item
            la singularité \( Z\) est effaçable;
        \item
            \( f\) possède un prolongement continue en \( Z\);
        \item
            il existe un voisinage épointé de \( Z\) sur lequel \( f\) est bornée;
        \item
            \( \lim_{z\to Z}(z-Z)f(z)=0\).
    \end{enumerate}
\end{theorem}
\index{théorème!prolongement de Riemann}

\begin{definition}[Fonction méromorphe\cite{ooBBIEooFYzkzz}]
    Soient \( \mU\) un ouvert de \( \eC\) et \( \{ p_i \}\) une suite de points dans \( \mU\) sans points d'accumulation (éventuellement il y a un nombre fini de \( p_i\)). Si la fonction \( f\) est holomorphe sur \( \mU\setminus\{ p_i \}\) et si chaque \( p_i\) est un point régulier ou un pôle de \( f\), alors nous disons que \( f\) est \defe{méromorphe}{méromorphe} sur \( \mU\).
\end{definition}

\begin{proposition} \label{PropPUZTQKl}
    Soient \( \Omega\) un ouvert de \( \eC\) et \( f_n\colon \Omega\to \eC\) une suite de fonctions telles que pour tout compact \( K\) de \( \Omega\) il existe \( N_K\geq 0\) tel que
    \begin{enumerate}
        \item
            \( f_n\) n'a pas de pôle dans \( K\) dès que \( n\geq N_K\);
        \item
            la série \( \sum_{n\geq N_K}f_n\) converge uniformément sur \( K\).
    \end{enumerate}
    Alors
    \begin{enumerate}
        \item
            La fonction
            \begin{equation}
                f(z)=\sum_{n=0}^{\infty}f_n(z)
            \end{equation}
            est méromorphe sur \( \Omega\) et ses pôles sont l'union de ceux des \( f_n\).
        \item
            Nous pouvons permuter la somme et la dérivée :
            \begin{equation}
                f'(z)=\sum_{n=0}^{\infty}f'_n(z).
            \end{equation}
    \end{enumerate}
\end{proposition}

\begin{theorem}[Série de Laurent]       \label{THOooMKJOooVghZyG}
    Soient \( C\) une couronne de rayons \( r_1<r_2\) centrée en zéro et une fonction \( f\) holomorphe dans cette couronne. Alors nous avons la \defe{série de Laurent}{série!de Laurent}
    \begin{equation}
        f(z)=\sum_{n\in \eZ}a_nz^n.
    \end{equation}
    \begin{enumerate}
        \item
            Cette série converge uniformément sur tout compact de \( C\).
        \item
            Les coefficients sont donnés par
            \begin{equation}
                a_n=\frac{1}{ 2\pi i }\int_{\gamma}\frac{ f(z) }{ z^{n+1} }dz
            \end{equation}
            où \( \gamma\) est un cercle centré en zéro.
        \item
            Ce développement en série est unique.
        \item
            La valeur des \( a_n\) ne dépend pas du choix du rayon du cercle \( \gamma\).
    \end{enumerate}
\end{theorem}

%+++++++++++++++++++++++++++++++++++++++++++++++++++++++++++++++++++++++++++++++++++++++++++++++++++++++++++++++++++++++++++
\section{Fonctions d'Euler}
%+++++++++++++++++++++++++++++++++++++++++++++++++++++++++++++++++++++++++++++++++++++++++++++++++++++++++++++++++++++++++++

\begin{theorem}[Prolongement méromorphe de la fonction \( \Gamma\) d'Euler\cite{KXjFWKA}]   \label{ThoZJYooWKfbVz}
    Nous considérons la formule
    \begin{equation}
        \Gamma(z)=\int_0^{\infty} e^{-t}t^{z-1}dt.
    \end{equation}
    Alors
    \begin{enumerate}
        \item
            Cette formule définit une fonction holomorphe sur
            \begin{equation}
                \mP=\{ z\in \eC\tq \Re(z)>0 \}.
            \end{equation}
        \item
            La fonction \( \Gamma\colon \mP\to \eC\) admet un unique prolongement méromorphe sur \( \eC\), lequel a des pôles sur les entiers négatifs.
    \end{enumerate}
\end{theorem}
\index{fonction!\( \Gamma\) d'Euler}
\index{prolongement!méromorphe de la fonction \( \Gamma\)}
\index{fonction!définie par une intégrale!\( \Gamma\) d'Euler}
\index{fonction!méromorphe!\( \Gamma\) d'Euler}

\begin{proof}
    \begin{subproof}
        \item[Holomorphie sous l'intégrale]

            Pour étudier l'holomorphie de la fonction \( \Gamma\) sur \( \mP\) nous utilisons le théorème~\ref{ThopCLOVN}.

            Nous considérons la fonction
            \begin{equation}
                \begin{aligned}
                    g\colon \mP\times \eR^+&\to \eC \\
                    (z,t)&\mapsto  e^{-t}z^{z-1}
                \end{aligned}
            \end{equation}
            et nous commençons par montrer que c'est holomorphe en \( z\) pour chaque \( t>0\) fixé. Nous le vérifions par le critère de \( \partial_{\bar zf=0}\)\footnote{Théorème~\ref{PropkwIQwg}.} et en nous souvenant que \( t^i= e^{\ln(t^i)}= e^{i\ln(t)}\). Nous obtenons rapidement que
            \begin{equation}
                \frac{ \partial g }{ \partial \bar z }=0.
            \end{equation}

            Le fait que la fonction \( t\mapsto g(z,t)\) soit mesurable pour tout \( z\) est d'accord.

            Et enfin soit \( K\) compact dans \( \mP\). Il faut trouver une fonction \( g_K(t)\) intégrable sur \( \mathopen[ 0 , \infty [\) telle que pour tout \( z\in K\) et \( t\in\mathopen[ 0 , \infty [\) nous ayons \( | f(z,t)\leq g(t) |\). Pour cela nous majorons séparément les parties \( t\in\mathopen] 0 , 1 \mathclose[\) et \( t\geq 1\).

            Soit donc \( K\) compact dans \( \mP\); nous posons \( M=\max_{z\in K}\Re(z)\) et \( \epsilon=\min_{z\in K}\Re(z)\).

            Si \( t\in \mathopen] 0 , 1 \mathclose[\) alors nous avons
            \begin{equation}
                e^{-t}t^{z-1}= e^{-t} e^{(z-1)\ln(t)},
            \end{equation}
            de telle façon à que que
            \begin{subequations}
                \begin{align}
                    |  e^{-t}t^{z-1} |&\leq|  e^{(x-1+iy)\ln(t)} |\\
                    &=|   e^{(\Re(z)-1)\ln(t)} |\\
                    &=| t^{\Re(z)-1} |\\
                    &\leq | t^{\epsilon-1} |\\
                    &=\frac{1}{ t^{1-\epsilon} }.
                \end{align}
            \end{subequations}
            Cette dernière fonction est intégrable sur \( \mathopen] 0 , 1 \mathclose[\).

            Nous considérons maintenant \( t\geq 1\). Dans ce cas nous avons
            \begin{equation}
                |  e^{-t}z^{z-1} |= e^{-t}t^{\Re(z)-1}\leq  e^{-t}t^{M-1}.
            \end{equation}
            Cette dernière fonction est un produit d'une exponentielle décroissante avec un polynôme. C'est donc intégrable entre \( 1\) et l'infini.

            La fonction \( g_K\) que nous considérons est donc
            \begin{equation}
                g_K(t)=\begin{cases}
                    \frac{1}{ t^{1-\epsilon} }    &   \text{si } t<1\\
                    \text{borné}    &    \text{si } 1\leq t\leq b\\
                    e^{-t}t^{M-1}    &    \text{si } t>b.
                \end{cases}
            \end{equation}
            Cela est une fonction intégrable sur \( \mathopen] 0    \infty ,  \mathclose[\) et qui majore \( f\) uniformément en \( z\) sur le compact \( K\) de \( \mP\). Le théorème~\ref{ThopCLOVN} nous permet donc de conclure que
            \begin{equation}
                \Gamma(z)=\int_0^{\infty}f(z,t)dt
            \end{equation}
            est holomorphe en \( z\) sur \( \mP\) et que
            \begin{equation}
                \Gamma'(z)=\int_0^{\infty}\frac{ \partial f }{ \partial z }(z,t)dt.
            \end{equation}

        \item[En deux morceaux] Nous passons maintenant à la seconde partie du théorème. Pour \( z\in \mP\) nous coupons l'intégrale en deux :
            \begin{equation}
                \Gamma(z)=\int_0^1 e^{-t}t^{z-1}dt+\int_1^{\infty} e^{-t}t^{z-1}dt
            \end{equation}

        \item[Première partie] Nous commençons par parler de la première partie : \( \int_0^1 e^{-t}t^{z-1}dt\) dans laquelle nous voulons utiliser le développement en série de l'exponentielle \(  e^{-t}\). Nous devons donc traiter
            \begin{equation}
                \int_0^1\sum_{n=0}^{\infty}\frac{ (-1)^n }{ n! }t^{n+z-1}dt.
            \end{equation}
            Nous allons permuter la somme avec l'intégrale à l'aide du théorème de Fubini~\ref{ThoFubinioYLtPI} en posant la fonction
            \begin{equation}
                g(n,t)=\frac{ (-1)^n }{ n! }t^{n+z-1}
            \end{equation}
            et en considérant le produit entre la mesure de Lebesgue sur \( \eC\) et la mesure de comptage sur \( \eN\), c'est-à-dire que nous étudions
            \begin{equation}
                \int_0^1\int_{\eN}g(n,t)dndt.
            \end{equation}
            Pour permuter il suffit de prouver que \( | g |\) est intégrable pour la mesure produit, c'est-à-dire que
            \begin{equation}
                \int_0^1\int_{\eN}\left| \frac{ (-1)^n }{ n! }t^{n+z-1} \right| <\infty.
            \end{equation}
            Nous avons \( | t^z=t^{\Re(z)} |\), donc
            \begin{equation}
                \sum_{n=0}^{\infty}\left| \frac{ t^{n+z-1} }{ n! } \right| =t^{\Re(z)-1}\sum_{n=0}^{\infty}\frac{ t^n }{ n! }=t^{\Re(z)-1} e^{t}.
            \end{equation}
            Étant donné que nous avons fixé \( z\in\mP\), nous avons \( \Re(z)-1>-1\) et donc \( t^{\Re(z)-1}\) est intégrable entre \( 0\) et \( 1\).
            %TODO : il faudrait prouver et citer ici le coup du 1/x^alpha qui est intégrable ou non.
            La partie \(  e^{t}\) se majore sur \( \mathopen[ 0 , 1 \mathclose]\) par une constante quelconque. Nous avons donc payé le droit d'inverser la somme et l'intégrale :
            \begin{equation}
                \int_0^1 e^{-t}t^{z-1}dt=\sum_{n=0}^{\infty}\int_0^1\frac{ (-1)^n }{ n! }t^{n+z-1}dt=\sum_{n=0}^{\infty}\frac{ (-1)^n }{ n! }[t^{n+z}]_0^1=\sum_{n=0}^{\infty}\frac{ (-1)^n }{ n!(n+z) }.
            \end{equation}
            Nous avons donc l'intéressante formule suivante, valable pour tout \( z\in\mP\) :
            \begin{equation}
                \Gamma(z)=\sum_{n=0}^{\infty}\frac{ (-1)^n }{ n!(n-z) }+\int_1^{\infty} e^{-t}t^{z-1}dt.
            \end{equation}

        \item[Prolongation de la première partie] Nous voudrions montrer maintenant que la fonction
            \begin{equation}
                \sum_{n=0}^{\infty}\frac{ (-1)^n }{ n!(n-z) }
            \end{equation}
            est méromorphe sur \( \eC\) avec des pôles en les entiers négatifs. Pour cela nous considérons la suite de fonctions
            \begin{equation}
                f_n(z)=\frac{ (-1)^n }{ n!(z+n) }
            \end{equation}
            et nous allons utiliser la proposition~\ref{PropPUZTQKl}. Si \( n\geq 0\), la fonction \( f_n\) est méromorphe sur \( \eC\) avec un pôle simple en \( z=-n\). Soit \( K\) compact de \( \eC\) et \( N_K\) tel que \( K\subset\overline{ B(0,N_K) }\). Pour \( n\geq N_K+1\), la fonction \( f_n\) n'a pas de pôle dans \( K\) et de plus pour tout \( z\in K\) nous avons
            \begin{equation}
                | z+n |=| z-(-z) |\geq\big| n-| z | \big|\geq n-| z |\geq n-N_K,
            \end{equation}
            et par conséquent
            \begin{equation}
                | f_n(z) |\leq \frac{1}{ n!(n-N) },
            \end{equation}
            ou pour le dire de façon plus snob :
            \begin{equation}
                \| f_n \|_{\infty,K}\leq \frac{1}{ n!(n-N) },
            \end{equation}
            dont la série converge. Cela signifie que la série \( \sum_{n\geq N}f_n\) converge normalement\footnote{Définition~\ref{DefVBrJUxo}.} sur \( K\), donc la fonction
            \begin{equation}
                f(z)=\sum_{n=0}^{\infty}f_n(z)
            \end{equation}
            est une fonction méromorphe dont les pôles sont ceux des \( f_n\), c'est-à-dire les entiers négatifs (proposition~\ref{PropPUZTQKl}).

        \item[La seconde partie]

            Nous allons à présent prouver que la fonction
            \begin{equation}
                g(z)=\int_1^{\infty} e^{-t}t^{z-1}dt
            \end{equation}
            est holomorphe sur \( \eC\). Pour cela nous considérons la fonction de deux variables \( f(z,t)= e^{-t}t^{z-1}\) et nous utilisons le théorème d'holomorphie sous l'intégrale~\ref{ThopCLOVN}. D'abord pour \( z_0\) fixé dans \( \eC\) nous avons
            \begin{equation}
                \int_1^{\infty}|  e^{-t}t^{z_0-1} |\leq \int_1^{\infty} e^{-t}t^{\Re(z_0)-1}dt,
            \end{equation}
            donc l'intégrale converge parce que c'est polynôme contre exponentielle. Par ailleurs pour chaque \( t_0\) fixé sur \( \mathopen[ 0 , \infty [\), la fonction \( z\mapsto  e^{-t_0}t_0^{z-1}\) est holomorphe sur \( \eC\) comme en témoigne le calcul suivant :
                \begin{equation}
                    \frac{ 1 }{2}\left( \frac{ \partial  }{ \partial x }+i\frac{ \partial  }{ \partial y } \right)t_0^{x+iy-1}=0.
                \end{equation}
                Et enfin si \( K\) est compact dans \( \eC\) nous avons
                \begin{equation}
                    | f(z,t) |=|  e^{-t}t^{z-1} |= e^{-t}| t^{\Re(z)-1} |\leq  e^{-t}t^{M-1}
                \end{equation}
                où \( M=\max_{z\in K}\Re(z)\). Nous en déduisons que la fonction
                \begin{equation}
                    z\mapsto\int_1^{\infty} e^{-t}t^{z-1}dt
                \end{equation}
                est une fonction holomorphe sur \( \eC\).

            \item[Conclusion]

                Au final nous avons prouvé que la fonction \( \Gamma\) d'Euler admet le prolongement méromorphe sur \( \eC\) donné par
                \begin{equation}
                    \Gamma(z)=\sum_{n=0}^{\infty}\frac{ (-1)^n }{ n!(z+n) }+\int_1^{\infty} e^{-t}t^{z-1}dt.
                \end{equation}
    \end{subproof}
\end{proof}

%---------------------------------------------------------------------------------------------------------------------------
\subsection{Euler et factorielle}
%---------------------------------------------------------------------------------------------------------------------------

\begin{proposition}
    Nous avons la formule \( \Gamma(n)=(n-1)!\) pour tout \( n\in \eN\).
\end{proposition}

\begin{proof}
    Nous partons de la formule
    \begin{equation}
        \Gamma(n)=\int_0^{\infty} e^{-t}t^{n-1}dt
    \end{equation}
    que nous intégrons par partie en posant
    \begin{equation}
        \begin{aligned}[]
            u&=t^{n-1}&u'&=(n-1)t^{n-1}\\
            v&= e^{-t}&v'&=- e^{-t}.
        \end{aligned}
    \end{equation}
    Les termes au bord s'annulent (ici il y a un passage à la limite qui n'est pas écrit) et nous trouvons
    \begin{equation}
        \Gamma(n)=\int_0^{\infty}(n-1) e^{-t}t^{n-2}dt=(n-1)\Gamma(n-1).
    \end{equation}

    Pour conclure il suffit de remarquer que
    \begin{equation}
        \Gamma(1)=\int_0^{\infty}=-[ e^{-t}]_0^{\infty}=1.
    \end{equation}
\end{proof}

%+++++++++++++++++++++++++++++++++++++++++++++++++++++++++++++++++++++++++++++++++++++++++++++++++++++++++++++++++++++++++++
\section{Partition d'un entier en parts fixées}
%+++++++++++++++++++++++++++++++++++++++++++++++++++++++++++++++++++++++++++++++++++++++++++++++++++++++++++++++++++++++++++
\index{partition!d'un entier en parts fixées}

\begin{proposition}[\cite{KXjFWKA}]     \label{PropWUFpuBR}
    Soient \( a_1,\ldots, a_k\in \eN^*\) des entiers premiers entre eux dans leur ensemble. Pour \( n\geq 1\) nous posons
    \begin{equation}
        u_n=\Card\left\{  (x_1,\ldots, x_k)\in \eN^*\tq \sum_{i=1}^ka_ix_i=n \right\},
    \end{equation}
    et \( u_0=1\).

    Alors nous avons l'équivalence de suite (pour \( n\to \infty\)) :
    \begin{equation}
        u_n\sim\frac{1}{ a_1\ldots a_k }\frac{ n^{k-1} }{ (k-1)! }.
    \end{equation}
\end{proposition}
\index{série!entière!utilisation}
\index{série!génératrice d'une suite!utilisation}
\index{anneau!de séries formelles!utilisation}
\index{corps!des fractions rationnelles!utilisation}

\begin{proof}
    Pour chacun des \( i\in\{ 1,\ldots, k \}\) nous considérons la série entière
    \begin{equation}
        \sum_{x=0}^{\infty}z^{xa_i}=\sum_k(z^{a_i})^x.
    \end{equation}
    Étant donné que \( | z^{a_i} |<1\) si et seulement si \( | z<1 |\), cette série a un rayon de convergence égal à \( 1\). Nous allons calculer le produit de Cauchy de ces \( k\) séries, en nous souvenant que le théorème~\ref{ThokPTXYC} nous assure que la série résultante aura un rayon de convergence au moins égal à \( 1\) et vaudra le produit des différentes séries.

    Le coefficient de \( z^n\) dans cette série vaut
    \begin{equation}
        \sum_{\substack{x\in \eN^k\\\sum x_ia_i=n}}1=u_n
    \end{equation}
    parce que dans chacune des séries, le coefficient de tous les \( z^{xa_i}\) est \( 1\). Nous définissons la fonction
    \begin{equation}    \label{EqKTRNFSl}
        f(z)=\sum_{n=0}^{\infty}u_nz^n=\prod_{i=1}^k\left( \sum_{x=0}^{\infty}z^{xa_i} \right)=\prod_{i=1}^k\frac{1}{ 1-z^{a_i} }.
    \end{equation}
    La fonction \( f\) existe sur \( | z |<1\) parce que nous venons de voir qu'elle peut  s'exprimer comme un produit de Cauchy; et la dernière égalité est simplement la somme de la série harmonique. D'autre part la fonction \( f\) est la série génératrice de la suite \( (u_n)\).

    Nous sommes en présence d'une fonction ayant des pôles aux racines \( a_1\),\ldots, \( a_k\)\ieme\ de l'unité. Étant donné que \( 1\) est une racine de l'unité de tous les ordres, le pôle en \( z=1\) est de multiplicité \( k\). Les autres pôles sont de multiplicité strictement inférieure; en effet soit \( \omega\in \eC\) tel que \( \omega^{a_i}=1\) pour tout \( i\). Alors Bezout\footnote{Théorème~\ref{ThoBuNjam}.} nous donne des entiers \( v_i\in \eZ\) tels que \( \sum_iv_ia_i=1\). Alors nous avons
%TODO : il faudrait citer ici un théorème de Bezout pour les ensembles de nombres premiers dans leur ensemble; le théorème cité ici n'est pas suffisant.
    \begin{equation}
        \omega=\omega^{\sum_{v_ia_i}}=\prod_{i=1}^k(\omega^{a_i})^{v_i}=1.
    \end{equation}
    Donc nous voyons que \( 1\) est le seul à être racine de tous les ordres en même temps. Nous notons
    \begin{equation}
        P=\{ \omega_1,\ldots, \omega_p \}
    \end{equation}
    l'ensemble des pôles avec \( \omega_1=1\). Par ailleurs la fonction \( f\) est une fraction rationnelle dont nous connaissons les racines du dénominateur (ce sont les \( \omega_i\)) et à peu près leurs ordres. Nous utilisons le truc de la décomposition en fractions simples
%TODO : après avoir fait la décomposition en fractions simples, il faut mettre une référence ici.
    en séparant le terme de puissance \( k\) qui n'existe que pour la racine \( \omega_1=1\) :
    \begin{equation}    \label{EqDLTJaYr}
        f(z)=\frac{ \alpha }{ (1-z)^k }+\sum_{i=1}^p\sum_{j=1}^{k-1}\frac{ c_{ij} }{ (\omega_i-z)^j }.
    \end{equation}
    Ce développement est valable pour tout \( | z |<1\). Nous considérons maintenant \( \omega\in P\) et \( j\in \eN\) et nous étudions la fonction
    \begin{equation}
        g(z)=\frac{1}{ \omega-z }.
    \end{equation}
    Un rapide calcul (par exemple par récurrence) montre que
    \begin{equation}    \label{EqEJLDIFJ}
        g^{(k)}(z)=\frac{ k! }{ (\omega-z)^{k+1} },
    \end{equation}
    et étant donné que \( | \omega |=1\) nous pouvons écrire la série
    \begin{equation}
        \frac{1}{ \omega-z }=\sum_{k=0}^{\infty}\frac{ z^k }{ \omega^{k+1} },
    \end{equation}
    valable pour \( | z |<1\). Ce qui nous intéresse, c'est d'exprimer une série pour \( 1/(\omega-z)^j\); et voyant \eqref{EqEJLDIFJ}, nous voyons qu'il suffit de calculer les dérivées de la série de \( g\). Nous dériver terme à terme à l'intérieur du rayon de convergence. Avec quelques abus d'écriture, et en utilisant la bête formule \eqref{EqSOFdwhw} nous avons\quext{À ce niveau j'ai pas exactement le même coefficient binomial que dans \cite{KXjFWKA}, mais je n'exclus absolument pas que ce soit moi qui me trompe. Écrivez-moi si vous pouvez infirmer ou confirmer l'erreur. Quoi qu'il en soit, cela ne change pas le résultat asymptotique que nous cherchons.}
    \begin{subequations}
        \begin{align}
            \frac{1}{ (\omega-z)^j }&=\frac{ g^{(j-1)}(z) }{ (j-1)! }\\
            &=\frac{1}{ (j-1)! }\left( \frac{1}{ \omega-z } \right)^{(j-1)}\\
            &=\frac{1}{ (j-1)! }\sum_{k=0}^{\infty}\frac{1}{ \omega^{k+1} }(z^k)^{(j-1)}\\
            &=\sum_{k=j-^{\infty}}\frac{1}{ (j-1)! }\frac{1}{ \omega^{k+1} }\frac{ k! }{ (k-j+1)! }z^{k-j+1}\\
            &=\sum_{n=0}^{\infty}\frac{1}{ \omega^{n+j} }\frac{ (n+j-1)! }{ n!(j-1)! }z^n\\
            &=\sum_{n=0}^{\infty}\frac{1}{ \omega^{n+j} }{n+j-1\choose n}z^n.
        \end{align}
    \end{subequations}
    Nous pouvons utiliser cela pour récrire la formule \eqref{EqDLTJaYr} de façon considérablement plus compliquée :
    \begin{equation}
            f(z)=\alpha\sum_{n=0}^{\infty}{n+j-1\choose n}z^n
            +\sum_{i=1}^p\sum_{j=1}^{k-1}\sum_{n=0}^{\infty}c_{ij}{n+j-1\choose n}\frac{ z^n }{ \omega_i^{n+j} }.
    \end{equation}
    Mais nous savons que ce \( f\) est la série génératrice de la suite \( (u_n)\) et que nous pouvons donc utiliser la formule \eqref{EqNGhVCpP} pour exprimer les nombres \( u_l\) : \( u_l\) est simplement le coefficient de \( z^l\) divisé par \( l!\). C'est-à-dire
    \begin{equation}
        u_l=\alpha{l+k-1\choose l}+\sum_{i=1}^p\sum_{j=1}^{k-1}c_{ij}{l+j-1\choose l}\frac{1}{ \omega_i^{l+j} }.
    \end{equation}
    Notre boulot est d'examiner le comportement de cela lorsque \( l\to\infty\), c'est-à-dire regarder quels sont les puissances de \( l\) en présence. Notons que

    En ce qui concerne le premier terme, la puissance dominante dans le coefficient binomial est \( l^{k-1}\). Dans les autres termes\footnote{Attention : les termes \( i=1\) ont \( \omega_1=1\) et il n'est donc pas possible de conclure simplement en disant que \( \omega_i^{l-j}\to 0\) pour \( l\to \infty\); bien que cela soit vrai pour tous les \( i\neq 1\).}, c'est \( l^{j-1}\) qui est de degré moins grand. Donc le comportement de \( u_l\) en termes de \( l\) est
    \begin{equation}
        u_l\sim \alpha\frac{ l^{k-1} }{ (k-1)! }.
    \end{equation}
    Il nous reste à voir ce que vaut \( \alpha\). Pour cela nous repartons de l'expression~\ref{EqKTRNFSl} que nous écrivons sous la forme
    \begin{equation}
        (1-z)^kf(z)=\prod_{i=1}^{k}\frac{ 1-z }{ 1-z^{a_i} }.
    \end{equation}
    Nous reconnaissons l'inverse d'une somme harmonique partielle :
    \begin{equation}    \label{EqTIAxvHE}
        (1-z)^kf(z)=\prod_{i=1}^k\frac{1}{ 1+z+z^2+\cdots +z^{a_i-1} }.
    \end{equation}
    Par ailleurs, nous ne savons pas si \( f(1)\) existe parce que son rayon de convergence n'est que de \( 1\); et nous savons même qu'elle n'existe pas (parce que ce serait la somme des \( u_n\)). Mais nous savons aussi que le pôle de plus grande multiplicité de \( f\) est en \( z=1\) et est de multiplicité \( k\). Donc \( (1-z)^kf(z)\) devrait converger pour \( z\to 1\). Pour tout \( | z<1 |\) nous avons
    \begin{equation}
        (1-z)^kf(z)=\alpha+\sum_{i=1}^p\sum_{j=1}^{k-1}c_{ij}\frac{ (1-z)^k }{ (\omega_i-z)^j }.
    \end{equation}
    Lorsque \( z\to 1\), tous les termes des sommes tendent vers zéro, y compris ceux avec \( i=1\) parce que \( j<k\). Il reste donc
    \begin{equation}
        \lim_{z\to 0} (1-z)^kf(z)=\alpha.
    \end{equation}
    En calculant la même limite avec \eqref{EqTIAxvHE} nous trouvons
    \begin{equation}
        \lim_{z\to 1}(1-z)^kf(z)=\lim_{z\to 1}\prod_{i=1}^k\frac{1}{ 1+z+z^2+\cdots +z^{a_i-1} }=\frac{1}{ a_1\ldots a_k }.
    \end{equation}
    Donc
    \begin{equation}
        \alpha=\frac{1}{ a_1\ldots a_k },
    \end{equation}
    et le résultat est prouvé.

\end{proof}

%+++++++++++++++++++++++++++++++++++++++++++++++++++++++++++++++++++++++++++++++++++++++++++++++++++++++++++++++++++++++++++
\section{Exponentielle et logarithme complexe}
%+++++++++++++++++++++++++++++++++++++++++++++++++++++++++++++++++++++++++++++++++++++++++++++++++++++++++++++++++++++++++++

%---------------------------------------------------------------------------------------------------------------------------
\subsection{Propriétés de l'exponentielle}
%---------------------------------------------------------------------------------------------------------------------------

\begin{proposition}
    Soit \( z\in\eC\) fixé. La fonction
    \begin{equation}
        \begin{aligned}
            E\colon \eR&\to \eC \\
            t&\mapsto  e^{tz}
        \end{aligned}
    \end{equation}
    est  \(  C^{\infty}\), sa dérivée est
    \begin{equation}
        E'(t)=z e^{tz}.
    \end{equation}
    La fonction \( E\) est développable en série entière (voir définition~\ref{DefwmRzKh}) sur \( \eR\) en \( t=0\) et
    \begin{equation}
        e^{tz}=\sum_{n=0}^{\infty}\frac{ z^n }{ n! }t^n.
    \end{equation}
\end{proposition}

\begin{proof}
    Nous fixons \( z\in \eC\). Par définition~\ref{DefJilXoM}, la série suivante est \(  e^{tz}\) :
    \begin{equation}
        f(t)=\sum_{n=0}^{\infty}\frac{ z^n }{ n! }t^n.
    \end{equation}
    Cette série a un rayon de convergence infini et la fonction \( f\) est donc \(  C^{\infty}\) sur \( \eR\). Nous pouvons la dériver terme à terme :
    \begin{equation}
            f'(t)=\sum_{n=1}^{\infty}\frac{ z^n }{ n! }nt^{n-1}
            =z\sum_{n=1}^{\infty}\frac{ z^{n-1} }{ (n-1)! }t^{n-1}
            =z e^{tz}.
    \end{equation}
\end{proof}

\begin{theorem}     \label{THOooNGOIooEECfAv}
    La fonction exponentielle vérifie les propriétés suivantes.
    \begin{enumerate}
        \item
            \( \exp\) est holomorphe\footnote{Définition \ref{DefMMpjJZ}.}.
        \item
            \( (e^z)'=e^z\).
        \item
            L'exponentielle est développable en série entière,
            \begin{equation}
                e^z=\sum_{n=0}^{\infty}\frac{ z^n }{ n! }
            \end{equation}
            et la série converge normalement sur tout compact de \( \eC\).
    \end{enumerate}
\end{theorem}

\begin{proof}
    En tant que application \( E\colon \eR^2\to \eC\), la fonction
    \begin{equation}
        E(x,y)=e^x(\cos y+i\sin y)
    \end{equation}
    est \( C^{\infty}\). De plus nous avons
    \begin{subequations}
        \begin{align}
            \frac{ \partial E }{ \partial x }(x,y)= e^{x+iy}=E(x,y)\\
            \frac{ \partial E }{ \partial y }(x,y)=iE(x,y),
        \end{align}
    \end{subequations}
    et par conséquent la fonction \( E\) vérifie les équations de Cauchy-Riemann.

    Si \( r\) est fixé, par le critère d'Abel appliqué à la suite \(r/n!\) nous savons que la série \( \sum z^n/n!\) converge normalement sur le compact \( B(0,r)\).
\end{proof}

%---------------------------------------------------------------------------------------------------------------------------
\subsection{Intégrale de Fresnel}
%---------------------------------------------------------------------------------------------------------------------------

Nous allons calculer l'\defe{intégrale de Fresnel}{intégrale!Fresnel}\index{Fresnel!intégrale}
\begin{equation}
    \int_0^{\infty} e^{-ix^2}dx=\frac{ \sqrt{\pi} }{ 2 } e^{-i\pi/4}
\end{equation}
en suivant la démarche présentée par Wikipédia\cite{ooOXWGooGhLJvX}. Nous commençons par prouver que l'intégrale est convergente en nous contentant de justifier la convergence de
\begin{equation}
    \int_0^{\infty}\sin(x^2)dx.
\end{equation}
Pour chaque \( a>0\) fixé, l'intégrale \( \int_0^a\sin(x^2)dx\) ne pose pas de problèmes. Le lemme \ref{LemTHBSEs} nous permet de passer à la limite; nous devons donc seulement calculer
\begin{equation}
    \lim_{b\to \infty}\int_a^b\sin(x^2)dx
\end{equation}
où \( a\) est une constante strictement positive. Nous effectuons une intégration par partie en posant
\begin{subequations}
    \begin{align}
        u&=\frac{1}{ x }&   u'&=-\frac{1}{ x^2 }\\
        v'&=x\sin(x)    & v&=\frac{ 1-\cos(x) }{2}.
    \end{align}
\end{subequations}
Notons que la primitive \( v\) a été choisie pour avoir \( v(0)=0\). Nous avons
\begin{equation}    \label{EqOdeKye}
    \int_a^b\sin(x^2)dx=\left[ \frac{ 1-\cos(x^2) }{ 2x } \right]_a^b-\int_a^b\frac{ \cos(x^2)-1 }{ 2x^2 }dx
\end{equation}
Pour le premier terme nous avons
\begin{equation}
    \lim_{b\to \infty}\left[ \frac{ 1-\cos(x^2) }{ 2x } \right]_a^b=\lim_{b\to \infty}\frac{ 1-\cos(b^2) }{ 2b }-\frac{ 1-\cos(a^2) }{ 2a }=-\frac{ 1-\cos(a^2) }{ 2a }.
\end{equation}
C'est borné. Pour le second terme de \eqref{EqOdeKye}, la fonction
\begin{equation}
    \frac{ \cos(x^2)-1 }{ 2x^2 }
\end{equation}
est majorée par la fonction \( 1/x^2\) qui est intégrable entre \( a\) et \( \infty\).


Nous allons calculer l'intégrale demandée en passant par la fonction
\begin{equation}
    f(x)= e^{-z^2}
\end{equation}
définie sur le plan complexe. Nous l'intégrons sur le chemin \( \gamma=\gamma_1+\gamma_2-\gamma_3\) indiqué à la figure~\ref{LabelFigCheminFresnel}.
\newcommand{\CaptionFigCheminFresnel}{Chemin d'intégration pour l'intégrale de Fresnel}
\input{auto/pictures_tex/Fig_CheminFresnel.pstricks}
Ces chemins sont donnés par
\begin{equation}
    \begin{aligned}
        \gamma_1\colon \mathopen[ 0 , R \mathclose]&\to \eC \\
        t&\mapsto t,
    \end{aligned}
\end{equation}
\begin{equation}
    \begin{aligned}
        \gamma_2\colon \mathopen[ 0 , \frac{ \pi }{ 4 } \mathclose]&\to \eC \\
        t&\mapsto R e^{it},
    \end{aligned}
\end{equation}
\begin{equation}
    \begin{aligned}
        \gamma_3\colon \mathopen[ 0 , R \mathclose]&\to \eC \\
        t&\mapsto t e^{i\pi/4}.
    \end{aligned}
\end{equation}
Tout d'abord la fonction \( f\) est bien holomorphe par le critère du théorème~\ref{PropkwIQwg}. Le calcul de \( \frac{ \partial f }{ \partial \bar z }\) se fait simplement en posant \( f(x,y)= e^{-(x+iy)^2}\). Le calcul est usuel :
\begin{verbatim}
----------------------------------------------------------------------
| Sage Version 4.8, Release Date: 2012-01-20                         |
| Type notebook() for the GUI, and license() for information.        |
----------------------------------------------------------------------
sage: f(x,y)=exp(-(x+I*y)**2)
sage: A=f.diff(x)+I*f.diff(y)
sage: A.simplify_full()
(x, y) |--> 0
\end{verbatim}
Nous avons donc
\begin{equation}    \label{EqfaoRgU}
    0=\int_{\gamma}f=\underbrace{\int_0^R e^{-t^2}dt}_{I_1(R)}+\underbrace{\int_0^{\pi/4} e^{-R^2 e^{2it}}Ri e^{it}dt}_{I_2(R)}+\underbrace{\int_0^R e^{-t^2 e^{i\pi/2}} e^{i\pi/4}dt}_{I_3(R)}.
\end{equation}
L'intégrale est nulle pour tout \( R\) en vertu de la proposition~\ref{PrpopwQSbJg}. L'intégrale \( I_1\) est une gaussienne et nous avons
\begin{equation}
    \lim_{R\to\infty}I_1(R)=\frac{ \sqrt{\pi} }{ 2 }
\end{equation}
par l'exemple~\ref{EXooLUFAooGcxFUW}. Nous montrons maintenant que \( \lim_{R\to\infty}| I_2(R) |=0\)\footnote{Il y a moyen de démontrer cela via le lemme de Jordan\cite{FresnelDavidS}. Nous donnons ici une démonstration moins technologique.}. D'abord nous majorons en prenant la norme puis nous effectuons le changement de variables \( u=2t\) :
\begin{subequations}
    \begin{align}
        | I_2(R) |&\leq \int_{0}^{\pi/4}R e^{-R^2\cos(2t)}dt\\
        &=\frac{ R }{ 2 }\int_0^{\pi/2} e^{-R^2\cos(u)}du.
    \end{align}
\end{subequations}
Nous savons que le graphe du cosinus est concave : il reste au dessus de la droite que joint \( (0,1)\) à \( (\frac{ \pi }{2},0)\). Du coup \( \cos(u)\geq 1-\frac{ 2 }{ \pi }u\) et par conséquent
\begin{equation}
        e^{-R^2\cos(u)}\leq  e^{-R^2(1-\frac{ 2 }{ \pi }u)}= e^{R^2(\frac{ 2 }{ \pi }u-1)}.
\end{equation}
Nous effectuons l'intégrale
\begin{subequations}
    \begin{align}
        | I_2(R) |&\leq \frac{ R }{2}\int_0^{\pi/2} e^{-R^2} e^{\frac{ 2R^2 }{ \pi }u}du\\
        &=\frac{ R }{2} e^{-R^2}\left[ \frac{ \pi }{ 2R^2 } e^{2R^2 u/\pi} \right]_0^{\pi/2}\\
        &=\frac{ \pi }{ 4R }-\frac{ \pi e^{-R^2} }{ 4R },
    \end{align}
\end{subequations}
et nous avons bien \( \lim_{R\to\infty}| I_2(R) |=0\). Nous passons à la troisième intégrale. En tenant compte que \(  e^{i\pi/2}=i\), nous avons
\begin{subequations}
    \begin{align}
        I_3(R)&=-\int_0^R e^{-\gamma_3(t)^2} e^{i\pi/4}dt\\
        &=-\frac{ 1+i }{ \sqrt{2} }\int_0^R e^{-t^2} e^{2i\pi/4}\\
        &=-\frac{ 1+i }{ \sqrt{2} }\int_0^R e^{-it^2}.
    \end{align}
\end{subequations}
En passant à la limite \( R\to 0 \), de l'équation \eqref{EqfaoRgU} il ne reste que
\begin{equation}
    0=\frac{ \sqrt{2} }{2}-\frac{ 1+i }{ \sqrt{2} }\int_0^{\infty} e^{-it^2}dt,
\end{equation}
ce qui signifie que
\begin{equation}
    \int_0^{\infty} e^{-it^2}dt=\frac{ \sqrt{2\pi} }{ 2(1+i) }=\frac{ \sqrt{\pi} }{2} e^{-i\pi/4}.
\end{equation}

%---------------------------------------------------------------------------------------------------------------------------
\subsection{Logarithme complexe}
%---------------------------------------------------------------------------------------------------------------------------


%///////////////////////////////////////////////////////////////////////////////////////////////////////////////////////////
\subsubsection{La fonction argument}
%///////////////////////////////////////////////////////////////////////////////////////////////////////////////////////////

Nous savons la définition~\ref{DefJilXoM} de l'exponentielle complexe.

\begin{definition}
    Un \defe{logarithme}{logarithme!dans \( \eC\)} de \( \alpha\in \eC\) est une solution de l'équation \(  e^{z}=\alpha\).
\end{definition}
Notons bien que cela définit \emph{un} logarithme, et non \emph{le} logarithme.

\begin{lemma}       \label{LEMooUMESooJVzeDb}
    Si \( z_1\) et \( z_2\) sont des logarithmes de \( \alpha\) alors il existe \( k\in \eZ\) tel que \( z_1=z_2+2ik\pi\).
\end{lemma}

\begin{proof}
    Nous commençons par déterminer les logarithmes de \( \alpha=1\). Nous avons besoin de \(  e^{a+bi}=1\) (\( a,b\in \eR\)). Nous avons
    \begin{equation}
        e^{a} e^{bi}=1,
    \end{equation}
    et en prenant la norme nous trouvons \( | e^a |=1\), ce qui donne \( a=0\). Ensuite \(  e^{bi}=1\), qui signifie \( b=2k\pi\). Les logarithmes de \( 1\) sont donc les nombres de la forme \( 2ik\pi\).

    Soient maintenant \( z_1\) et \( z_2\) des logarithmes de \( \alpha\). Alors \(  e^{z_1}= e^{z_2}\), donc\footnote{C'est facile de dire «donc». Il faut surtout citer la proposition~\ref{PropdDjisy}\ref{ITEMooRLHCooJTuYKV}.} \(  e^{z_1-z_2}=1\), ce qui signifie que \( z_1-z_2\) est un logarithme de \( 1\). Donc il existe un \( k\in \eZ\) tel que \( z_1-z_2=2ik\pi\).
\end{proof}

\begin{remark}
    Jusqu'ici nous n'avons pas donné de conditions donnant l'existence d'un logarithme. Nous avons seulement supposé des existences et donné des propriétés sur ces hypothétiques objets.
\end{remark}

\begin{definition}[\cite{ooDQKTooXNjklV}]
Si \( z\in \eC^*\) nous définissons la \defe{valeur principale}{valeur principale} de son argument le nombre \( \theta\in \mathopen] -\pi , \pi \mathclose]\) tel que
\begin{equation}
    z=| z | e^{i\theta}
\end{equation}
Nous le notons \( \arg(z)\)\nomenclature[Y]{\( \arg(z)\)}{La valeur principale de l'argument de \( z\in \eC\)}.
\end{definition}

\begin{normaltext}      \label{NORMooOGHNooYriCBH}
    Il ne faut pas se ruer sur \( \arg(x+iy)=\arctan(y/x)\). Pour rappel, la fonction \( \arctan\) a été définie dans le théorème~\ref{THOooUSVGooOAnCvC}, et elle prend ses valeurs dans \( \mathopen] -\pi/2 , \pi/2 \mathclose[\). La formule \( v(x,y)=\arctan(y/x)\) n'est donc valable que pour \( x>0\). Les valeurs sont :
        \begin{equation}        \label{EQooPJVFooSEKTny}
            \arg(x+iy)=\begin{cases}
                 \arctan(y/x)   &   \text{si } x>0\\
                 \pi+\arctan(y/x)    &    \text{si } x<0\text{ et }y\geq 0 \\
                 -\pi+\arctan(y/x)    &    \text{si } x<0 \text{ et }y<0\\
                 \frac{ \pi }{ 2 }    &    \text{si } x=0 \text{ et }y>0\\
                 \frac{- \pi }{ 2 }    &    \text{si } x=0 \text{ et }y<0.
            \end{cases}
        \end{equation}

    Pour \( x>0\) nous avons \( \arg(x+iy)=\arctan(y/x)  \) parce que justement la fonction \( \arctan\) prend ses valeurs en particulier entre \( -\pi\) et \( \pi\). Pour \( x<0\) et \( y>0 \) nous avons \( \arg(x+iy)=\pi+\arctan(y/x)\) (dans ce cas, \( \arctan(y/x)<0\)) et si \( x<0\), \( y<0\) nous avons \( \arg(x+iy)=-\pi+\arctan(y/x)\).
\end{normaltext}

\begin{normaltext}[Les dérivées partielles de la fonction argument]     \label{NORMooMRBEooVtTcIA}
    Vu que nous en aurons besoin plusieurs fois, nous calculons maintenant les dérivées partielles de la fonction
    \begin{equation}
        \begin{aligned}
            \varphi\colon \eR^2&\to \eR \\
            (x,y)&\mapsto \arg(x+iy).
        \end{aligned}
    \end{equation}
    Nous commençons par la dérivée \( \partial_x\varphi(x,y)\). Et il y a de nombreux cas à séparer.
    \begin{subproof}

         \item[\( x>0\)]

             Nous avons
             \begin{equation}
                 \frac{ \partial \varphi }{ \partial x }(x,y)=\lim_{\epsilon\to 0}\frac{ \arctan(y/(x+\epsilon))-\arctan(y/x) }{ \epsilon },
             \end{equation}
             qui n'est autre que la dérivée de la fonction \( x\mapsto\arctan(y/x)\). Nous pouvons la calculer facilement avec le théorème~\ref{THOooUSVGooOAnCvC}\ref{ITEMooMNHLooOVhIIb} :
             \begin{equation}
                 \frac{ \partial \varphi }{ \partial x }(x,y)=-\frac{ y }{ x^2+y^2 }.
             \end{equation}

         \item[\( x<0\)]

             Nous avons
             \begin{equation}
                 \frac{ \partial \varphi }{ \partial x }(x,y)=\lim_{\epsilon\to 0}\frac{ \pm\pi+\arctan(y/(x+\epsilon))-\big( \pm\pi+\arctan(y/x) \big) }{ \epsilon }
             \end{equation}
             où les signes \( \pm\) dépendent du signe de \( y\). De toutes façons, les termes en \( \pi\) se simplifient et le calcul est le même que celui du cas \( x>0\). Encore une fois nous avons
             \begin{equation}
                 \frac{ \partial \varphi }{ \partial x }(x,y)=-\frac{ y }{ x^2+y^2 }.
             \end{equation}

         \item[\( x=0\)]

             Nous devons calculer
             \begin{equation}
                 \frac{ \partial \varphi }{ \partial x }(0,y)=\lim_{\epsilon\to 0}\frac{ \arg(\epsilon+ iy)-\arg(iy) }{ \epsilon }.
             \end{equation}
             Il y a quatre cas d'après les signes de \( \epsilon\) (séparer limite à gauche et à droite) et \( y\).

             Si \( \epsilon>0\) et \( y>0\) alors nous avons à faire le calcul
             \begin{equation}
                 \lim_{\epsilon\to 0^+}\frac{ \arctan(y/\epsilon)-\pi/2 }{ \epsilon }
             \end{equation}
             qui se traite par la règle de l'Hospital. Cela donne \( -1/y\).

             Les trois autres cas ne se distinguent que par des constantes au numérateur, lesquelles disparaissent en appliquant la règle de l'Hospital\footnote{Nonobstant le fait que ces constantes se mettent bien pour avoir un vrai cas d'indétermination \( 0/0\), sinon la règle de l'Hospital ne s'applique pas.}. Au final,
             \begin{equation}
                 \frac{ \partial \varphi }{ \partial x }(0,y)=-\frac{1}{ y }.
             \end{equation}
    \end{subproof}

    Nous avons calculé jusqu'ici :
    \begin{equation}        \label{EQooAOJPooOrvUBR}
        \frac{ \partial \varphi }{ \partial x }(x,y)=\frac{ -y }{ x^2+y^2 }
    \end{equation}
    pour tout \( (x,y)\in \eR^2\setminus\{ (0,0) \}\). En particulier vous avez noté que cette dérivée partielle est continue sur \( \eR^2\setminus\{ (0,0) \}\).

    Nous calculons à présent la dérivée partielle par rapport à \( y\) :
    \begin{equation}
        \frac{ \partial \varphi }{ \partial y }(x,y)=\lim_{\epsilon\to 0}\frac{ \arg(x+iy+i\epsilon)-\arg(x+iy) }{ \epsilon }.
    \end{equation}

    \begin{subproof}

        \item[\( x>0\)]

            Nous avons à calculer
            \begin{equation}
                \lim_{\epsilon\to 0}\frac{ \arctan\frac{ y+\epsilon }{ x }-\arctan\frac{ y }{ x } }{ \epsilon },
            \end{equation}
            qui n'est autre que la dérivée de la fonction \( t\mapsto\arctan\frac{ t }{ x }\) en \( t=y\). Résultat :
            \begin{equation}
                \frac{ \partial \varphi }{ \partial y }(x,y)=\frac{ x }{ x^2+y^2 }.
            \end{equation}

        \item[\( x<0  \) et \( y\neq 0\)]

            Le calcul à faire est :
            \begin{equation}
                \lim_{\epsilon\to 0}\frac{ \pm\pi+\arctan\frac{ y+\epsilon }{ x }-\left( \pm\pi+\arctan\frac{ y }{ x } \right) }{ \epsilon }
            \end{equation}
            Une chose importante à remarquer est que dans le calcul de la limite nous pouvons supposer que \( y\) et \( y+\epsilon\) aient le même signe, quelle que soit la valeur et le signe de \( \epsilon\) (assez petit). C'est pour cela que les deux termes \( \pm\pi\) arrivent avec le même signe des deux côtés de la différence, et se simplifient. Nous tombons sur une limite déjà faite et
            \begin{equation}
                \frac{ \partial \varphi }{ \partial y }(x,y)=\frac{ x }{ x^2+y^2 }
            \end{equation}

        \item[\( x<0\) et \( y=0\)]

            Vu que \( x<0\) nous avons \( \arg(x)=\pi\) et nous devons calculer
            \begin{equation}
                \lim_{\epsilon\to 0}\frac{ \arg(x+i\epsilon)-\pi }{ \epsilon }.
            \end{equation}
            La limite \( \epsilon\to 0^+\) est classique et donne \( 1/x\).

            Mais la limite \( \epsilon\to 0^-\) n'existe pas :
            \begin{equation}
                \lim_{\epsilon\to 0^-}\frac{ -\pi+\arctan(\epsilon/x)-\pi }{ \epsilon }
            \end{equation}
            n'existe pas.

            Donc
            \begin{equation}
                \frac{ \partial \varphi }{ \partial y }(x,0)
            \end{equation}
            n'existe pas pour \( x<0\).

        \item[\( x=0\) et \( y\neq 0\)]

            Le calcul est immédiat
            \begin{equation}
                \lim_{\epsilon\to 0}\frac{ \arg(iy+i\epsilon)-\arg(iy) }{ \epsilon }=0,
            \end{equation}
            donc
            \begin{equation}
                \frac{ \partial \varphi }{ \partial y }(0,y)=0.
            \end{equation}


    \end{subproof}
    En ce qui concerne la continuité, nous avons que \( \partial_y\varphi\) est continue partout sauf sur la demi-droite \(  \{ (x,0)\tq x\leq 0 \}   \) où elle n'existe pas.

\end{normaltext}

%///////////////////////////////////////////////////////////////////////////////////////////////////////////////////////////
\subsubsection{Une définition possible du logarithme}
%///////////////////////////////////////////////////////////////////////////////////////////////////////////////////////////

\begin{definition}      \label{DEFooWDYNooYIXVMC}
    Nous définissons la fonction \defe{logarithme}{logarithme!complexe} par
    \begin{equation}
        \begin{aligned}
            \ln\colon \eC^*&\to \eC \\
            z&\mapsto \ln\big( | z | \big)+i\arg(z)
        \end{aligned}
    \end{equation}
    où le \( \ln\) à droite est le logarithme usuel sur \( \eR^+\).
\end{definition}

\begin{remark}
Cette fonction généralise le logarithme déjà vu sur \( \mathopen] 0 , \infty \mathclose[\subset \eR\). En effet pour des valeurs de \( z\) dans cette partie nous avons \( \arg(z)=0\) et \( | z |=z\).
\end{remark}

\begin{lemma}
    Le nombre \( \ln(z)\) est un logarithme de \( z\).
\end{lemma}

\begin{proof}
    Nous avons
    \begin{equation}
        e^{\ln(z)}= e^{\ln| z |} e^{i\arg(z)}=| z | e^{i\arg(z)}=z.
    \end{equation}
    Nous avons utilisé le fait que \(  e^{\ln(x)}=x\) pour \( x\in\eR^+\) et \( | z | e^{i\arg(z)}=z\) par définition de la fonction \( \arg\).
\end{proof}

Notons que si on avait pris d'autres conventions pour définir \( \arg\), nous aurions eu d'autres définitions possibles de \( \ln\).

\begin{example}
    Nous avons
    \begin{equation}
        \ln(-1)=\ln(1)+i\arg(-1).
    \end{equation}
    Mais \( \ln(1)=0\) et \( \arg(-1)=\pi\) (et non \( -\pi\)), donc
    \begin{equation}
        \ln(-1)=i\pi.
    \end{equation}

    C'est cette définition du logarithme qui est prise par Sage, et c'est cela qui lui permet de donner la primitive de \( 1/x\) comme \( \ln(x)\) et non \( \ln(| x |)\), parce que Sage connaît les logarithmes de nombres réels négatifs :
\lstinputlisting{tex/sage/sageSnip010.sage}
\end{example}

Nous avons jusqu'ici défini une fonction sur \( \eC^*\) qui fait correspondre à chaque nombre complexe un de ses logarithmes. Il reste quelques questions à régler :
\begin{itemize}
    \item Est-ce que cette fonction est continue ? Holomorphe ? (réponses : non et non)
    \item Si non, est-ce qu'il y avait moyen de trouver une définition plus efficace ? (réponse : non)
\end{itemize}

\begin{lemma}       \label{LEMooMUOIooCnoWwq}
La fonction \( \ln\) n'est pas continue sur \( \mathopen] -\infty , 0 \mathclose]\).
\end{lemma}

\begin{proof}
    Attention à bien comprendre l'énoncé. La fonction
    \begin{equation}
        \begin{aligned}
        f\colon \mathopen] -\infty , 0 \mathclose[&\to \eC \\
            x&\mapsto \ln(x)
        \end{aligned}
    \end{equation}
    est continue. D'ailleurs c'est \( \ln(x)=\ln(| x |)+i\pi\). Ce dont il est question dans l'énoncé, c'est de la fonction \( \ln\) vue comme fonction sur \( \eC^*\).

    Soit \( x>0\) dans \( \eR\); nous avons
    \begin{equation}
        \ln(-x)=\ln(x)+i\pi.
    \end{equation}
    Cependant \( \lim_{\substack{\lambda\to 0^-\\\lambda\in \eR}}\ln(-x+\lambda i) \) va valoir \( \ln(| x |-i\pi)\). En effet lorsque \( \lambda<0\) est petit, l'argument de \( -x+\lambda i\) se rapproche de \( -\pi\) (et non de \( \pi\)).

\begin{center}
   \input{auto/pictures_tex/Fig_CWKJooppMsZXjw.pstricks}
\end{center}

Donc
\begin{equation}
    \lim_{\substack{\lambda\to 0^-\\\lambda\in \eR}}\ln(-x+\lambda i)=\lim \ln(| x+\lambda i |)+i\arg(-x+\lambda i)=\ln(| x |)-i\pi.
\end{equation}
Nous n'avons donc pas continuité de la fonction logarithme comme fonction sur \( \eC^*\).
\end{proof}

\begin{theorem}     \label{THOooWUXOooYKvLbJ}
    La restriction
    \begin{equation}
        \ln\colon \eC\setminus\mathopen] -\infty , 0 \mathclose]\to \eC
    \end{equation}
    est holomorphe.
\end{theorem}

\begin{proof}
    Nous allons utiliser la proposition~\ref{PropKJUDooJfqgYS} et considérer la fonction
    \begin{equation}
        \begin{aligned}
            F\colon S&\to \eR^2 \\
            (x,y)&\mapsto \big( \ln(| x+iy |),\arg(x+iy) \big)
        \end{aligned}
    \end{equation}
    où \( S=\eR^2\setminus\{ (x,0)\tq x\leq 0 \}\). Nous devons vérifier que \( F\) est différentiable et que sa différentielle en un point de \( S\) est une similitude.

    Nous posons
    \begin{equation}
        u(x,y)=\ln\big( \sqrt{ x^2+y^2 } \big)
    \end{equation}
    et
    \begin{equation}
        v(x,y)=\arg(x+iy).
    \end{equation}
    Les dérivées partielles de \( u\) ne sont pas très compliquées :
    \lstinputlisting{tex/sage/sageSnip011.sage}
    c'est-à-dire
    \begin{subequations}
        \begin{align}
            \frac{ \partial u }{ \partial x }=\frac{ x }{ x^2+y^2 }\\
            \frac{ \partial u }{ \partial y }=\frac{ y }{ x^2+y^2 }.
        \end{align}
    \end{subequations}

    Pour celles de \( v \) par contre, il faut se poser des questions, par exemples résister à la tentation d'écrire \( v(x,y)=\arctan(y/x)\) et lire~\ref{NORMooOGHNooYriCBH}.

    Nous avons déjà calculé les dérivées partielles de \( v\) dans~\ref{NORMooMRBEooVtTcIA}, et nous avons vu qu'elles étaient continues sur \( \eR^2\) privé de la demi-droite.

    Vu que les dérivées partielles sont continues, le théorème \ref{THOooBEAOooBdvOdr} nous dit que \( F\) est différentiable. La matrice de la différentielle est alors la matrice des dérivées partielles
    \begin{equation}
        \begin{pmatrix}
            \frac{ x }{ x^2+y^2 }    &   \frac{ y }{ x^2+y^2 }    \\
            \frac{ -y }{ x^2+y^2 }    &   \frac{ x }{ x^2+y^2 }
        \end{pmatrix},
    \end{equation}
    qui a la forme requise \eqref{EQooWZGKooLDEHGr} pour que la proposition~\ref{PropKJUDooJfqgYS} nous assure que \( \ln\) soit \( \eC\)-dérivable, c'est-à-dire holomorphe.
\end{proof}

%///////////////////////////////////////////////////////////////////////////////////////////////////////////////////////////
\subsubsection{Pas plus de continuité}
%///////////////////////////////////////////////////////////////////////////////////////////////////////////////////////////

Bon. La fonction logarithme que nous avons définie est holomorphe sur \( \eC^*\) privé d'une demi-droite \( U\). Et elle n'est pas continue sur \( U\); elle y est cependant continue «par le haut». Pouvons-nous faire mieux ? Nous allons maintenant prouver quelques résultats d'impossibilité de faire mieux que holomorphe partout sauf une partie pas si petite que ça.

\begin{proposition}
    Il n'existe pas de fonctions continues \( f\colon \eC^*\to \eC\) telle que \(  e^{f(z)}=z\) pour tout \( z\in \eC^*\).
\end{proposition}

\begin{proof}
    Pour tout \( z\), le nombre \( f(z)\) est un logarithme de \( z\). Or \( \ln(z)\) en est également un. Donc par le lemme~\ref{LEMooUMESooJVzeDb}
    \begin{equation}
        f(z)=\ln(z)+2i k(z)\pi
    \end{equation}
    pour une certaine fonction \( k\colon \eC^*\to \eZ\). Sur le domaine d'holomorphie de \( \ln\), les fonctions \( \ln\) et \( f\) étant continues, la fonction \( k\) l'est aussi. Mais une fonction continue à valeurs dans \( \eZ\) est constante (son domaine est connexe).

    Il existe donc \( k\in \eZ\) tel que
    \begin{equation}
         f(z)=\ln(z)+2ik\pi
    \end{equation}
    au moins pour tout \( z\in \eC^*\setminus U\). Une telle fonction ne peut pas être continue sur \( U\) parce que \( \ln\) ne l'est pas.
\end{proof}

Ok. Pas continue sur tout \( \eC\). Mais continue sur un peu plus que \( \eC\) privé de toute une demi-droite ? La proposition suivante répond que bof.

\begin{proposition}
    Soit \( \Omega\) un ouvert de \( \eC\) contenant \( S(0,r)\) (le cercle centré en \( 0\) et de rayon \( r>0\)). Il n'existe pas de fonction continue \( f\colon \Omega\to \eC\) telle que \(  e^{f(z)}=z\) pour tout \( z\in \Omega\).
\end{proposition}

\begin{proof}
    Encore une fois, pour tout \( z\in \Omega\) nous avons
    \begin{equation}
        f(z)=\ln(z)+2i\pi k(z)
    \end{equation}
    pour une certaine fonction \( k\colon \Omega\to \eZ\). Sur \( \Omega\setminus U\), la fonction \( \ln\) est continue et \( k\) doit également l'être. Donc \( k\) est constante sur les composantes connexes de \( \Omega\setminus U\).

    Vu que \( S(0,r)\) est compact, on peut le recouvrir par un nombre fini de boules centrées en des points de \( S(0,r)\). En prenant le minimum des rayons de ces boules, nous voyons que \( \Omega\) contient une couronne
    \begin{equation}
        \{ z\in \eC\tq r-\delta\leq | z |\leq r+\delta \}.
    \end{equation}
    Soit le point \( x_0=-r\). C'est un point de \( \Omega\) contenu dans \( U\). Nous allons prouver que \( B(x_0,\delta)\setminus U\) est dans une seule composante connexe de \( \Omega\).

    Soit un point \( z_1\in B(x_0,\delta)\) situé au-dessus de \( U\), et \( z_2\) un point de \( B(x_0,\delta)\) situé en dessous de \( U\). Le cercle \( S(0,r)\) coupe \( B(x_0,\delta)\) en deux points : un au-dessus et un en-dessous de \( U\). On peut lier \( z_1\) au point de «sortie» supérieur de \( S(0,r)\) en restant dans \( B(x_0,\delta)\); ce point est ensuite relié en suivant le cercle au point d'entrée inférieur du cercle dans \( B(x_0,\delta)\). Ce dernier point est lié à \( z_2\) par un chemin restant dans la boule.

    Tout cela pour dire que \( z_1\) et \( z_2\) sont dans la même composante connexe de \( \Omega\) et que \( k(z_1)=k(z_2)\). Il existe donc \( k\in \eZ\) tel que
    \begin{equation}
        f(z)=\ln(z)+2ik\pi
    \end{equation}
    sur \( B(x_0,\delta)\setminus U\). Une telle fonction \( f\) ne peut pas être continue.
\end{proof}

%///////////////////////////////////////////////////////////////////////////////////////////////////////////////////////////
\subsubsection{Pas d'unicité : autres déterminations de l'argument}
%///////////////////////////////////////////////////////////////////////////////////////////////////////////////////////////

\begin{normaltext}      \label{NORMooFCDOooFDzAjp}
    Nous avons pris la fonction d'argument \( \arg\colon \eC\to \mathopen] -\pi , \pi \mathclose]\). Il y en a évidemment beaucoup d'autres de possibles. Par exemple pour \( \alpha\in \eR\) nous pouvons considérer
    \begin{equation}        \label{EQooNKKDooOuJxXe}
        \arg_{\alpha^+}\colon \eC\to \mathopen] \alpha , \alpha+2\pi \mathclose]
    \end{equation}
    ou
    \begin{equation}
        \arg_{\alpha^-}\colon \eC\to \mathopen[ \alpha , \alpha+2\pi \mathclose[.
    \end{equation}
    En posant
    \begin{equation}
        \ln_{\alpha^{\pm}}(z)=\ln(| z |)+i\arg_{\alpha^{\pm}}(z)
    \end{equation}
nous avons une fonction réciproque de l'exponentielle définie sur \( \eC^*\) et holomorphe sur \( \eC^*\) privé d'une demi-droite \( D_{\alpha}\) (dépendante de la valeur de \( \alpha\)).
\end{normaltext}

La différence entre \( \ln_{\alpha^+}\) et \( \ln_{\alpha^-}\) est seulement la valeur sur la demi-droite de non-holomorphie. L'une sera semi-continue d'un côté et l'autre, de l'autre côté.

\begin{remark}
    La fonction \( \arg_{0^-}\) a déjà été utilisée en \ref{SUBSECooWFNMooOuZBRN} pour écrire un inverse de la fonction
    \begin{equation}
        \begin{aligned}
            \varphi\colon \mathopen[ 0 , 2\pi \mathclose[  &\to S^1 \\
                t&\mapsto  e^{it}. 
        \end{aligned}
    \end{equation}
\end{remark}

\begin{definition}[\cite{ooXDXQooWXsXlk}]
    Soit un ouvert \( \Omega\subset \eC^*\). Nous disons que la fonction \( f\colon \Omega\to \eC\) est une \defe{détermination}{détermination!logarithme} sur \( \Omega\) si elle est continue et vérifie
    \begin{equation}
        e^{f(z)}=z
    \end{equation}
    pour tout \( z\in \eC\).
\end{definition}

Les différents résultats vus jusqu'ici montrent qu'il n'existe pas de détermination du logarithme sur \( \eC^*\).

\begin{definition}
    La \defe{détermination principale}{détermination!logarithme!principale} du logarithme est la restriction de notre logarithme~\ref{DEFooWDYNooYIXVMC}
    \begin{equation}
        \begin{aligned}
            \ln\colon \eC^*&\to \eC \\
            z&\mapsto \ln(| z |)+i\arg(z)
        \end{aligned}
    \end{equation}
    à l'ouvert \( \eC^*\setminus U\) où \( U\) est la partie \( \Re(z)\leq 0\) de \( \eC\).
\end{definition}

\begin{remark}      \label{REMooFBLLooDnkmjR}
    Beaucoup de sources\cite{ooGUROooApafph} ne définissent pas \( \ln_{\alpha^{\pm}}\) sur la droite \( D_{\alpha}\). C'est-à-dire qu'ils notent \( \ln_{\alpha}\) notre fonction \( \ln_{\alpha^+}\) restreinte à \( \eC^*\setminus D_{\alpha}\). Dans ce cas, les fonctions \( \ln_{\alpha^+}\) et \( \ln_{\alpha^-}\) sont identiques\footnote{Cela n'est pas tout à fait évident; vous devriez y penser.}.

    Cette remarque est importante parce que certains vont vous dire «le logarithme n'est pas définit sur la demi-droite»; de leur point de vue, la fonction que nous avons définie est une prolongation (non continue) à \( U\) du logarithme, qui est continu.

    \begin{enumerate}
        \item
            Certaines personnes pourraient vous dire que notre logarithme «n'est pas bien définit parce que si on fait le tour dans un sens ou dans l'autre nous n'obtenons pas la même valeur pour \( \ln(z)\) lorsque \( z\) est sur \( U\)». Et cela avec des arguments aussi forts que «\( 2\pi\) et \( 0\), c'est le même point».

            Nous préférons être bien clairs\quext{Est-ce qu'il faut vraiment un pluriel ici ?} sur ce point : notre fonction \( \ln\) est parfaitement définie sur \( \eC^*\) et \( 2\pi\) n'est pas la même chose que zéro. En particulier \( \arg( e^{2i\pi})=0\) et \(  \arg(e^{-i\pi})=\pi\) et non \( -\pi\).
        \item
            Il n'en reste pas moins que Sage donne \( \ln(-1)=I\pi\) et que nous avons choisi de faire de même, parce que le Frido n'est pas un cours d'agrégation, mais un texte qui donne quelques éléments de mathématique dans le but d'utiliser Sage efficacement.
        \item
            Tout ceci pour dire que si vous utilisez ce livre pour l'agrégation, vous devriez sérieusement considérer l'option de ne pas donner du logarithme la définition donnée ici, mais bien sa restriction.
    \end{enumerate}

    En fait notre logarithme est maximum pour la propriété «être une réciproque de l'exponentielle» alors que beaucoup de monde préfère avoir une fonction maximale pour la propriété «être réciproque de l'exponentielle tout en étant continue».

\end{remark}

De toutes les fonctions ayant le droit de vouloir être appelée «logarithme», celle que nous avons choisie (un peu arbitrairement) pour s'appeler «logarithme» et accaparer de la notation «\( \ln\)» est \( \ln_{\pi^+}\). Elle est d'une certaine manière celle qui arrive le plus naturellement.

En effet si nous pensons au logarithme népérien \( \ln\colon \eR^+\to \eR\) que nous voulons prolonger sur \( \eR\), nous devons poser
\begin{equation}
    \ln(-x)=\ln(-1)+\ln(x)
\end{equation}
pour \( x>0\). Que peut valoir \( \ln(-1)\) ? Il doit vérifier \(  e^{\ln(-1)}=-1\). La première valeur qui nous tombe sous la main est \( \ln(-1)=\pi\). Bien entendu, d'autres possibilités étaient possibles, comme \( \ln(-1)=2017\pi\) par exemple.

%///////////////////////////////////////////////////////////////////////////////////////////////////////////////////////////
\subsubsection{Pas d'unicité : développement en série}
%///////////////////////////////////////////////////////////////////////////////////////////////////////////////////////////

Pour \( z_0\in \eC^*\) nous pouvons écrire un développement en série de la réciproque de l'exponentielle autour de \( z_0\). La fonction ainsi définie est holomorphe sur la boule \( B(z_0,| z_0 |)\) et diverge en dehors de cette boule.

Voilà encore une fonction «logarithme» pour chaque point de \( \eC^*\). Nous nommons \( \ln_{z_0}\) la fonction
\begin{equation}
    \ln_{z_0}\colon B(z_0,| z_0 |)\to \eC
\end{equation}
donnée par la série.

En général nous n'avons pas \( \ln_{z_1}=\ln_{z_2}\) sur l'intersection des disques de convergence. Si c'était le cas, de proche en proche nous pourrions construire une fonction continue réciproque du logarithme sur \( \eC^*\), ce qui est impossible.

%///////////////////////////////////////////////////////////////////////////////////////////////////////////////////////////
\subsubsection{Pas d'unicité : laquelle choisir ?}
%///////////////////////////////////////////////////////////////////////////////////////////////////////////////////////////

Bon. Pour chaque demi-droite \( D\) nous avons une détermination du logarithme sur \( \eC^*\setminus D\). Et pour tout \( z_0\in \eC^*\) nous en avons une sur \( B(z_0,| z_0 |)\).

En pratique, quel logarithme choisir ? Cela dépend du problème.

Si vous avez besoin ou envie de travailler avec des série entières, le mieux est de choisir une détermination donnée par un développement autour d'un point bien choisi par rapport à votre problème.

Si vous avez surtout besoin d'holomorphie, et que vous en avez besoin sur un grand domaine, vous devriez choisir une détermination sur un des ensembles \( \eC^*\setminus D_{\alpha}\) en choisissant \( \alpha\) de telle sorte que la demi-droite maudite ne passe pas par la zone sur laquelle vous travaillez.

Dans tous les cas, vous devez préciser très explicitement la détermination choisie. Dans ce texte, sauf mention du contraire, nous utiliserons la détermination principale, et même son extension (non continue) à \( \eC^*\). Lorsque nous aurions besoin d'holomorphie, nous préciserons que nous considérons la restriction.

%///////////////////////////////////////////////////////////////////////////////////////////////////////////////////////////
\subsubsection{Logarithme comme primitive}
%///////////////////////////////////////////////////////////////////////////////////////////////////////////////////////////

Tout le monde sait que le logarithme \( \ln\colon \eR^+\to \eR\) est une primitive de la fonction \( x\mapsto 1/x\). Qu'en est-il dans le cas complexe ? Tout d'abord précisons que nous ne comptons pas encore parler d'intégrale sur \( \eC\), mais seulement d'intégrales sur \( \eR\) d'une fonction à valeur complexes.

\begin{proposition}     \label{PROPooNIJVooKueuYJ}
    Si \( z\in \eC\) alors
    \begin{equation}        \label{EQooAHYXooTPGXDS}
        \int\frac{1}{ x+z }dx=\ln(x+z)
    \end{equation}
\end{proposition}

\begin{proof}
    Il est important de comprendre que la formule \eqref{EQooAHYXooTPGXDS} est un abus de notation pour dire que si nous considérons la fonction
    \begin{equation}
        \begin{aligned}
            \varphi\colon \eR&\to \eC \\
            x&\mapsto \ln(x+z)
        \end{aligned}
    \end{equation}
    alors nous avons \( \varphi'(x)=\frac{1}{ x+z }\). Ici la dérivation est une dérivation sur \( \eR\) et l'intégrale est une intégrale sur \( \eR\), c'est-à-dire «composante par composantes». La fonction \(  \varphi\) se décompose en partie réelle et imaginaire qui sont à dériver séparément :
    \begin{equation}
        \varphi(x)=\ln(| x+z |)+i\arg(x+z).
    \end{equation}

    \begin{subproof}

        \item[Si \( z\) est imaginaire pur]

            Nous posons \( z=\lambda i\) avec \( \lambda\in \eR^*\). D'abord nous avons
            \begin{equation}
                \frac{1}{ x+\lambda i }=\frac{ x }{ x^2+\lambda^2 }-i\frac{ \lambda }{ x^2+\lambda^2 }.
            \end{equation}
            La partie réelle de \( \varphi(x)\) est
            \begin{equation}
                \varphi_1(x)=\ln\big( \sqrt{ x^2+\lambda^2 } \big),
            \end{equation}
            dont la dérivée est
            \begin{equation}
                \varphi_1'(x)=\frac{ x }{ x^2+\lambda^2 },
            \end{equation}
            qui correspond bien à la partie réelle de \( \frac{1}{ x+\lambda i }\).

            En ce qui concerne la partie imaginaire, \( \varphi_2(x)=\arg(x+\lambda i)\), et sa dérivée n'est rien d'autre que la dérivée partielle par rapport à \( x\) de la fonction argument, déjà calculée en \eqref{EQooAOJPooOrvUBR} :
            \begin{equation}
                \varphi_2'(x)=\frac{ -\lambda }{ x^2+\lambda }.
            \end{equation}
            Cela est bien la partie imaginaire de \( \frac{1}{ x+\lambda i }\).

    Notons que nous n'avons pas de problèmes sur la demi-droite des réels négatifs parce que nous ne considérons au final que la dérivée partielle par rapport à \( x\) de la fonction argument, laquelle existe et est continue, même sur cette partie.

        \item[Pour \( z\) quelconque]

            Soit \( z=s+\lambda i\) avec \( s,\lambda\in \eR\). En posant \( \varphi_0(x)=\ln(x+\lambda i)\) nous avons \( \varphi(x)=\varphi_0(x+s)\) et donc
            \begin{equation}
                \varphi'(x)=\varphi_0'(x+s)=\frac{ 1 }{ x+s+\lambda i }=\frac{1}{ x+z }.
            \end{equation}
            Tout va bien.

    \end{subproof}
\end{proof}

\begin{example}     \label{EXooAKEDooZgjocX}
    Un petit calcul d'intégrale, que nous avions déjà faite dans l'exemple~\ref{EXooIPEQooGKDjea} (avec la méthode de Rothstein-Trager). En passant par une décomposition en fractions simples :
    \begin{subequations}
        \begin{align}
            \int\frac{1}{ x^3+x }&=\int\left( \frac{1}{ x }-\frac{ 1/2 }{ x-i }-\frac{ 1/2 }{ x+i } \right)\\
            &=\ln(x)-\frac{ 1 }{2}\ln(x-i)-\frac{ 1 }{2}\ln(x+i)\\
            &=\ln(x)-\frac{ 1 }{2}\ln(x^2+1).       \label{SUBEQooRNQLooScfSlG}
        \end{align}
    \end{subequations}
    Attention aux justifications. Il n'est pas vrai en général dans le cas de nombres complexes \( a\) et \( b\) que \( \ln(ab)=\ln(a)+\ln(b)\). En effet, pour la partie réelle, ça passe parce que \( | ab |=| a | |b |\). Mais en ce qui concerne la partie imaginaire,
    \begin{equation}
        \arg(ab)\neq \arg(a)+\arg(b)
    \end{equation}
lorsque la somme dépasse les bornes de \( \mathopen] -\pi , \pi \mathclose]\). Le passage à \eqref{SUBEQooRNQLooScfSlG} fonctionne parce que dans le cas particulier des nombres \( x+i\) et \( x-i\), les arguments se somment à zéro : \( \arg(x+i)+\arg(x-i)=0\).
\end{example}

%+++++++++++++++++++++++++++++++++++++++++++++++++++++++++++++++++++++++++++++++++++++++++++++++++++++++++++++++++++++++++++
\section{Théorème de Weierstrass}
%+++++++++++++++++++++++++++++++++++++++++++++++++++++++++++++++++++++++++++++++++++++++++++++++++++++++++++++++++++++++++++

\begin{theorem}[Théorème de Weierstrass\cite{uTyBDj}]       \label{ThoArYtQO}
    Soit \( (f_n)\) une suite de fonctions holomorphes sur un ouvert \( \Omega\) de \( \eC\) que nous supposons converger uniformément sur tout compact vers \( f\). Alors \( f\) est holomorphe sur \( \Omega\) et pour tout \( k\) nous avons
    \begin{equation}
        f^{(k)}_n\to f^{(k)}
    \end{equation}
    uniformément sur tout compact.

    Dit en peu de mots, la limite uniforme d'une suite de fonctions holomorphes est holomorphe, et on peut permuter la limite avec la dérivation.
\end{theorem}
\index{compacité}
\index{suite!de fonctions intégrables}
\index{fonction!définie par une intégrale}
\index{fonction!holomorphe}
\index{limite!inversion}
\index{limite!de fonctions holomorphes}

\begin{proof}
    Chacune des fonctions \( f_n\) étant holomorphes, si \( a\in \Omega\) et \( r\) est tel que \( B(a,r)\subset \Omega\), nous avons par la formule de Cauchy~\ref{ThoUHztQe} :
    \begin{equation}
        f_n(z)=\frac{1}{ 2\pi i }\int_{\partial B(a,r)}\frac{ f_n(\xi) }{ \xi-z }d\xi
    \end{equation}
    pour tout \( z\) dans un boule \( B(a,\rho)\) incluse dans \( B(a,r)\). Étant donné que le cercle \( \partial B\) est compact, elle y est majorée par une constante \( M\). Montrons que de plus nous pouvons choisir \( M\) de telle façon à avoir \( | f_n(\xi) |\leq M\) pour tout \( n\) et tout \( \xi\) en même temps. D'abord nous utilisons la continuité de la limite \( f\) sur le compact \( \partial B \) pour poser \( A=\max_{z\in\partial B}| f(z) |\). Ensuite nous considérons un \( \epsilon>0\) et \( N\) tel que \( |\ f_n-f \|_{\partial B}\leq \epsilon\) pour tout \( n\geq N\). Nous savons maintenant que
    \begin{equation}
        \{ | f_n(\xi) |\tq n \geq N,\xi\in\partial B \}
    \end{equation}
    est majoré par \( A+\epsilon\). Nous posons enfin
    \begin{equation}
        B=\max_{n\leq N}\max_{\xi\in\partial B}| f_n(z) |,
    \end{equation}
    et alors le nombre \( M=\max\{ A+\epsilon,B \}\) majore \( | f_n(\xi) |\) pour tout \( n\) et tout \( \xi\in\partial B\).

    De plus pour tout \( \xi\in\partial B\) et pour tout \( z\) dans la petite boule, nous avons \( | \xi-z |>r-\rho\), donc  la fonction dans l'intégrale est majorée par une constante ne dépendant ni de \( n\) ni de \( \xi\). Nous pouvons donc permuter l'intégrale et la limite sur \( n\) :
    \begin{equation}
        f(z)=\frac{1}{ 2i\pi }\int_{\partial B}\frac{ f(\xi) }{ \xi-z }.
    \end{equation}
    Cela implique que la fonction \( f\) est holomorphe par le corolaire~\ref{CorwfHtJu}.

    Nous voudrions maintenant parler des dérivées des \( f_n\) et de \( f\). Pour cela nous voulons permuter l'intégrale et les dérivées, ce qui est fait au corolaire~\ref{CorNxTjEj} :
    \begin{equation}
        f_n^{(k)}=\frac{1}{ 2\pi i }\int_{\partial B(z_0,r)}\frac{ f(\omega) }{ (\omega-z)^{k+1} }d\omega.
    \end{equation}
    Nous voulons la convergence sur tout compact contenu dans l'ouvert \( \Omega\). Pour ce faire, nous allons considérer un compact \( K\subset \Omega\) et prouver la convergence uniforme dans toute boule de la forme \( B(z_0,r)\) avec \( z_0\in K\) et \( B(z_0,r)\subset \Omega\). Pour chaque tel couple \( (z_0,r)\), nous aurons un \( N_{(z_0,r)}\in \eN\) tel que si \( n\geq N_{(z_0,r)}\),
    \begin{equation}
        \| f_n^{(k)}-f^{(k)} \|_{B(z_0,r)}\leq \epsilon.
    \end{equation}
    Vu que ces boules \( B(z_0,r)\) forment un recouvrement de \( K\) par des ouverts, nous pouvons en retirer un sous-recouvrement fini et prendre, comme \( N\), le maximum des \( N_{(z_0,r)}\) correspondants. Pour ce \( N\) nous aurons
    \begin{equation}
        \| f_n^{(k)}-f^{(k)} \|_K\leq \epsilon.
    \end{equation}
    Au travail !

    Pour \( z\in B(z_0,r)\) nous considérons \( r'>r\) tel que \( B(z_0,r')\subset \Omega\) et nous avons
    \begin{subequations}
        \begin{align}
            | f^{(k)}_n(z)-f^{(k)}(z) |&=\left| \frac{1}{ 2\pi i }\int_{\partial B(z_0,r')}\frac{ f_n(\xi)-f(\xi) }{ (\xi-z)^{k+1} }d\xi \right| \\
            &\leq\frac{1}{ 2\pi }\int_{\partial B(z_0,r')}\frac{ | f_n(\xi)-f(\xi) | }{ | r-r' |^{k+1} }d\xi.
        \end{align}
    \end{subequations}
    Nous avons pris ce \( r'\) de telle manière que \( | \xi-z |\) soit borné par le bas par \( | r-r' |\); sinon la majoration que nous venons de faire ne marche pas. Étant donné que \( f_n\to f\) uniformément, nous pouvons considérer \( n\) assez grand pour que le numérateur soit plus petit que \( \epsilon\) indépendamment de \( \xi\) et de \( z\). Donc pour un \( n\) assez grand,
    \begin{equation}
        | f^{(k)}_n(z)-f^{(k)}(z) |\leq \frac{ \epsilon }{ 2\pi }\frac{ 2\pi r' }{ | r-r' |^{k+1} }
    \end{equation}
    pour tout \( z\in B(z_0,r)\). Donc nous avons convergence uniforme \( f_n^{(k)}\to f^{(k)}\) sur cette boule. Par l'argument de compacité donné plus haut, nous avons la convergence uniforme sur tout compact.
\end{proof}


\chapter{Analyse fonctionnelle}
\input{82_analyse_fonctionnelle}
% This is part of Mes notes de mathématique
% Copyright (c) 2011-2020
%   Laurent Claessens
% See the file fdl-1.3.txt for copying conditions.

%---------------------------------------------------------------------------------------------------------------------------
\subsection{Approximation}
%---------------------------------------------------------------------------------------------------------------------------

\begin{lemma}[Théorème fondamental d'approximation \cite{TribuLi}]      \label{LempTBaUw}
    Soit \( \Omega\) un espace mesurable et \( f\colon \Omega\to \mathopen[ 0 , \infty \mathclose]\) une application mesurable. Alors il existe une suite croissante d'applications étagées \( \varphi_n\colon \Omega\to \eR^+\) dont la limite est \( f\).

    De plus si \( f\) est bornée, la convergence est uniforme.
\end{lemma}

\begin{theorem}[\cite{HilbertLi}]       \label{ThoJsBKir}
    Soit \( I\) un intervalle de \( \eR\). L'espace \( \swD(I)\)\nomenclature[Y]{\( C_c(I)\)}{fonctions continues à support compact dans \( I\)} des fonctions continues à support compact sur \( I\) est dense dans \( L^2(I)\).
\end{theorem}
Ce théorème sera généralisé à tous les \( L^p(\eR^d)\) par le théorème~\ref{ThoILGYXhX}. Cependant \( L^p\) n'étant pas un Hilbert, il faudra travailler sans produit scalaire.

\begin{proof}
    Soit \( g\in L^2(I)\) une fonction telle que \( g\perp f\) pour toute fonction \( f\in C_c(I)\). Nous avons donc
    \begin{equation}
        \langle f, g\rangle =\int_If\bar g=0.
    \end{equation}
    En passant éventuellement aux composantes réelles et imaginaires nous pouvons supposer que les fonctions sont toutes réelles. Nous décomposons \( g\) en parties positives et négatives : \( g=g^+-g^-\). Notre but est de montrer que \( g^+=g^-\), c'est-à-dire que \( g\) est nulle. La proposition~\ref{PropqiWonByiBmc} conclura que \( C_c(I)\) est dense dans \( L^2(I)\).

    Soit un intervalle \( \mathopen[ a , b \mathclose]\subset I\) et une suite croissante de fonctions \( f_n\in C_c(I)\) qui converge vers \( \mtu_{\mathopen[ a , b \mathclose]}\). Par hypothèse pour chaque \( n\) nous avons
    \begin{equation}
        \int_If_ng^+=\int_I f_ng^-.
    \end{equation}
    La suite étant croissante, le théorème de la convergence monotone (théorème~\ref{ThoRRDooFUvEAN}) s'applique et nous avons
    \begin{equation}
        \lim_{n\to \infty} \int_I f_ng^+=\int_a^bg^+,
    \end{equation}
    de telle sorte que nous ayons, pour tout intervalle \( \mathopen[ a , b \mathclose]\subset I\) l'égalité
    \begin{equation}        \label{EqYlErAM}
        \int_a^bg^+=\int_a^bg^-.
    \end{equation}
    De plus ces intégrales sont finies parce que
    \begin{equation}
        \int_a^b g^+\leq\int_a^b| g |=\int_I| g |\mtu_{\mathopen[ a , b \mathclose]}=\langle | g |, \mtu_{\mathopen[ a , b \mathclose]}\rangle \leq \| g \|_{L^2}\sqrt{b-a}<\infty
    \end{equation}
    par l'inégalité de Cauchy-Schwarz.

    Soit maintenant un ensemble mesurable \( A\subset I\). La fonction caractéristique \( \mtu_A\) est mesurable et il existe une suite croissante de fonctions étagées \( (\varphi_n)\) convergente vers \( a\) par le lemme~\ref{LempTBaUw}. À multiples près, les fonctions \( \varphi_n\) sont des sommes de fonctions caractéristiques du type \( \mtu_{\mathopen[ a , b \mathclose]}\), par conséquent, en vertu de \eqref{EqYlErAM} nous avons
    \begin{equation}
        \int_I\varphi_ng^+=\int_I\varphi_ng^-.
    \end{equation}
    Une fois de plus nous pouvons utiliser le théorème de la convergence monotone et obtenir
    \begin{equation}
        \int_Ag^+=\int_A g^-
    \end{equation}
    pour tout ensemble mesurable \( A\subset I\). Si nous notons \( dx\) la mesure de Lebesgue, les mesures \( g^+dx\) et \( g^-dx\) sont par conséquent égales et dominées par \( dx\). Par le corolaire~\ref{CorZDkhwS} du théorème de Radon Nikodym, les fonctions \( g^+\) et \( g^-\) sont égales.
\end{proof}

%+++++++++++++++++++++++++++++++++++++++++++++++++++++++++++++++++++++++++++++++++++++++++++++++++++++++++++++++++++++++++++
\section{Convolution}
%+++++++++++++++++++++++++++++++++++++++++++++++++++++++++++++++++++++++++++++++++++++++++++++++++++++++++++++++++++++++++++


\begin{definition}      \label{DEFooHHCMooHzfStu}
    Pour toutes fonctions \( f,g\colon \eR^n\to \eC\) et pour tout \( x\in \eC\) tels que l'intégrale de droite ait un sens\footnote{Attention divlgâchi : ce sera le cas pour \( f,g\in L^1(\eR^n)\) par le théorème \ref{THOooMLNMooQfksn}.}, nous définissons
    \begin{equation}
        (f*g)(x)=\int_{\eR^n} f(y)g(x-y)dy.
    \end{equation}
    L'éventuelle fonction \( f*g\) ainsi définie est le \defe{produit de convolution}{produit!de convolution} de \( f\) et \( g\).
\end{definition}

Le théorème qui permet de dire que le produit de convolution n'est pas tout à fait ridicule est le suivant.

\begin{theorem}[\cite{MesIntProbb}]     \label{THOooMLNMooQfksn}
    Soient \( f,g\in L^1(\eR^n)\).
    \begin{enumerate}
        \item
            Pour presque tout \( x\in \eR^n\), la fonction
            \begin{equation}
                y\mapsto g(x-y)f(y)
            \end{equation}
            est dans \( L^1(\eR^n)\).
        \item
            \( f*g\in L^1(\eR^n)\).
        \item
            \( \| f*g \|_1\leq \| f \|_1\| g \|_1\).
    \end{enumerate}
\end{theorem}

\begin{proposition}     \label{PROPooNBHNooInwoar}
    L'ensemble \( L^1(\eR^n)\) devient alors une algèbre de Banach.
\end{proposition}

\begin{lemma}
    Le produit de convolution est commutatif : \( f*g=g*f\).
\end{lemma}

\begin{proof}
    Le théorème de Fubini (théorème~\ref{ThoFubinioYLtPI}) permet d'écrire
    \begin{equation}
        (f*g)(x)=\int_{\eR^n}f(y)g(x-y)dy=\int_{-\infty}^{\infty}dy_1\ldots \int_{-\infty}^{\infty}dy_nf(y)g(x-y).
    \end{equation}
    En effectuant le changement de variable \( z_i=x_i-y_i\) dans chacune des intégrales nous obtenons
    \begin{equation}
        (f*g)(x)=\int_{\eR^n}g(z)f(x-z)dz=(g*f)(x).
    \end{equation}
    Attention : on pourrait croire qu'un signe apparaît du fait que \( z=x-y\) donne \( dz=-dy\). Mais en réalité, l'intégrale \( \int_{-\infty}^{+\infty}\) devient par le même changement de variables \( \int_{+\infty}^{-\infty}\) qui redonne un nouveau signe au moment de remettre dans l'ordre.
\end{proof}

La proposition suivante est une conséquence de l'inégalité de Minkowski sous forme intégrale de la proposition \ref{PropInegMinkKUpRHg}\ref{ItemDHukLJiii}.
\begin{proposition}     \label{PROPooDMMCooPTuQuS}
    Si \( 1\leq p\leq \infty\) et si \( f\in L^p(\eR^d)\) et \( g\in L^1(\eR^d)\) alors
    \begin{enumerate}
        \item
            \( f*g\in L^p\)
        \item
            \( \| f*g \|_p\leq \| f \|_p\| g \|_1\).
    \end{enumerate}
\end{proposition}

\begin{proposition}[\cite{CXCQJIt}] \label{PropHNbdMQe}
    Si \( f\in L^1(\eR)\) et si \( g\) est dérivable avec \( g'\in L^{\infty}\), alors \( f*g\) est dérivable et \( (f*g)'=f*g'\).
\end{proposition}

\begin{proof}
    La fonction qu'il faut intégrer pour obtenir \( f*g\) est $f(t)g(x-t)$, dont la dérivée par rapport à \( x\) est \( f(t)g'(x-t)\). La norme de cette dernière est majorée (uniformément en \( x\)) par \( G(t)=| f(t) | \| g' \|_{\infty}\). La fonction \( f\) étant dans \( L^1(\eR)\), la fonction \( G\) est intégrable et le théorème de dérivation sous l'intégrale (théorème~\ref{ThoMWpRKYp}) nous dit que \( f*g\) est dérivable et
    \begin{equation}
        (f*g)'(x)=\frac{ d }{ dx }\int_{\eR}f(t)g(x-t)dt=\int_{\eR}f(t)g'(x-t)dt=(f*g')(x).
    \end{equation}
\end{proof}

\begin{corollary}       \label{CORooBSPNooFwYQrc}
    Si \( f\in L^1(\eR^d)\) et si \( g\) est de classe \(  C^{\infty}\), alors \( f*g\) est de classe \(  C^{\infty}\).
\end{corollary}

\begin{proof}
    Il s'agit d'itérer la proposition~\ref{PropHNbdMQe}.
\end{proof}

\begin{lemma}       \label{LemDQEKNNf}
    Soit \( f\in L^2(I)\) telle que
    \begin{equation}
        \int_If\varphi=0
    \end{equation}
    pour toute fonction \( \varphi\in C^{\infty}_c(I)\). Alors \( f=0\) presque partout sur \( I\).
\end{lemma}

\begin{proof}
    Nous considérons la forme linéaire
    \begin{equation}
        \begin{aligned}
            \phi\colon L^2(I)&\to \eC \\
            g&\mapsto \langle f, g\rangle=\int_If\bar g .
        \end{aligned}
    \end{equation}
    Par densité\footnote{Théorème~\ref{ThoILGYXhX}\ref{ItemYVFVrOIv}.} nous pouvons aussi considérer une suite \( (\varphi_n)\) dans \(  C^{\infty}_c(I)\) convergeant dans \( L^2\) vers \( f\). Alors nous avons pour tout \( n\) :
    \begin{equation}
        \langle f, \varphi_n\rangle =0.
    \end{equation}
    En passant à la limite, \( \langle f, f\rangle =0\), ce qui implique \( f=0\) dans \( L^2\) et donc \( f=0\) presque partout en tant que bonne fonction.
\end{proof}
Ce résultat est encore valable dans les espaces \( L^p\) (proposition~\ref{PropUKLZZZh}), mais il demande le théorème de représentation de Riesz\footnote{Théorème~\ref{ThoLPQPooPWBXuv}.}.

%---------------------------------------------------------------------------------------------------------------------------
\subsection{Approximation de l'unité}
%---------------------------------------------------------------------------------------------------------------------------


\begin{definition}[\cite{MonCerveau,TUEWwUN}]       \label{DEFooEFGNooOREmBb}
Nous considérons \( \Omega=\eR^d\) ou \( (S^1)^d\). Une \defe{approximation de l'unité}{approximation!de l'unité} sur \( \Omega\) autour de \( a\in \Omega\) est une suite \( (\varphi_n)\) de fonctions à valeurs réelles dans \( L^1(\Omega)\) telle que
    \begin{enumerate}
        \item
            $\sup_k \| \varphi_k \|_1 <\infty$,
        \item   \label{ITEMooGVRQooHDbrcf}
            pour chaque \( n\) nous avons $\int_{\Omega}\varphi_n=1$,
        \item
            si \( V\) est un voisinage de \( a\), alors
            \begin{equation}
                \lim_{k\to \infty} \int_{\Omega\setminus V}| \varphi_k |=0.
            \end{equation}
    \end{enumerate}
    En pratique, nous allons, sur \( \eR^d\) toujours considérer des approximations de l'unité autour de \( 0\), même si nous ne le préciserons pas. Vous noterez que dans le cas de \( S^1\), le choix du «point de base» est plus arbitraire.
\end{definition}
%TODO : voir si ça n'approxime pas un delta de Dirac d'une façon ou d'une autre.
Ce sont des fonctions dont la masse vient s'accumuler autour de zéro. En effet quel que soit le voisinage \( B(0,\alpha)\), si \( k\) est assez grand, il n'y a presque plus rien en dehors.

Pour le point \eqref{ITEMooTFFQooOUajFw}, si \( \Omega\) est \( S^1\), la mesure que nous considérons est \( \frac{ dx }{ 2\pi }\).


\begin{example}
    Une façon de construire une approximation de l'unité sur \( \eR\) est de considérer une fonction \( \varphi\in L^1(\Omega)\) telle que \( \int\varphi=1\) puis de poser
    \begin{equation}
        \varphi_k(x)=k^d\varphi(kx).
    \end{equation}
    Ici, \( \Omega\) peut être \( \eR\) ou \( S^1\).
\end{example}

Le lemme suivant permet de construire des approximations de l'unité intéressantes. Nous aurons une version pour \( S^1\) dans le lemme \ref{LEMooUNFBooRCzwIn}.
\begin{lemma}[\cite{TUEWwUN}]   \label{LemCNjIYhv}
    Soit \( \varphi\) est une fonction continue et positive à support compact sur \( \eR^d\) telle que \( \varphi(x)>\varphi(0)\) pour tout \( x\neq 0\). Si nous posons
    \begin{equation}
        \varphi_n(x)=\left( \int\varphi(y)^n \right)^{-1}\varphi(x)^n,
    \end{equation}
    alors la suite \( (\varphi_n)\) est une approximation de l'unité.
\end{lemma}

Voici un théorème qui donne les propriétés à propos du produit de convolution avec une approximation de l'unité dans \( \eR^d\). Une version pour \( S^1\) sera le théorème \ref{THOooIAOPooELSNxq}.
\begin{theorem}[\cite{TUEWwUN}] \label{ThoYQbqEez}
    Soit \( (\varphi_k)\) une approximation de l'unité sur \( \eR^d\).
    \begin{enumerate}
        \item
            Si \( g\) est mesurable et bornée sur \( \eR^d\) et si \( g\) est continue en \( x_0\) alors
            \begin{equation}
                (\varphi_k*g)(x_0)\to g(x_0).
            \end{equation}
        \item
            Si \( g\in L^p(\eR^d)\) (\( 1\leq p<\infty\)) alors
            \begin{equation}
                \varphi_k*g\stackrel{L^p}{\to}g.
            \end{equation}
        \item
            Si \( g\) est uniformément continue et bornée, alors
            \begin{equation}
                \varphi_k*g\stackrel{L^{\infty}}{\to}g
            \end{equation}
    \end{enumerate}
\end{theorem}

\begin{proof}
    En plusieurs points.
    \begin{enumerate}
        \item
            Nous notons \( d_k=(\varphi_k*g)(x_0)-g(x_0)\) et nous devons prouver que \( d_k\to 0\). Vu que \( \varphi_k\) est d'intégrale \( 1\) sur \( \eR^d\) nous pouvons écrire
            \begin{equation}
                d_k=\int_{\eR^d}\varphi_k(y)g(x_0-y)dy-\int_{\eR^d}g(x_0)\varphi_k(y)dy,
            \end{equation}
            et donc
            \begin{equation}
                |d_k|=\big| \int_{\eR^d}\big( g(x_0-y)-g(x_0) \big)\varphi_k(y)dy\big|\leq\int_{\eR^d}\big| g(x_0-y)-g(x_0) \big| |\varphi_k(y) |dy.
            \end{equation}
            Nous notons \( M=\sup_k\| \varphi_k \|_1\), et nous considérons \( \alpha>0\) tel que
            \begin{equation}
                \big| g(x_0-y)-g(x_0) \big|\leq \epsilon
            \end{equation}
            pour tout \( y\in B(0,\alpha)\). Nous nous restreignons maintenant aux \( k\) suffisamment grands pour que \( \int_{\complement B(0,\alpha)}| \varphi_k(y) |dy\leq \epsilon\). Alors en découpant l'intégrale en \( B(0,\alpha)\) et son complémentaire dans \( \eR^d\),
            \begin{equation}
                | d_k |\leq \epsilon M+\int_{\complement B(0,\alpha)} 2\| g \|_{\infty}| \varphi_k(y) |dy  \leq \epsilon M+2\| g \|_{\infty}\epsilon\leq \epsilon C.
            \end{equation}
            Donc oui, nous avons \( | d_k |\to 0\), et donc le premier point du théorème.

        \item

            Cette fois \( g\in L^p(\eR^d)\) et nous cherchons à montrer que \( \| d_k \|_p\to 0\). Encore qu'ici \( d_k\) soit défini à partir d'un représentant dans la classe de \( g\) et que d'ailleurs, nous allons travailler avec ce représentant.

            D'abord nous développons un peu ce \( d_k\) :
            \begin{subequations}
                \begin{align}
                \| d_k \|_p&=\left[ \int_{\eR^d}\left|     \int_{\eR^d}\big( g(x-y)-g(x) \big)\varphi_k(y)dy  \right|^pdx \right]^{1/p}\\
                &\leq\left[    \int_{\eR^d}\Big( \int_{\eR^d}| g(x-y)-g(x) |\cdot |\varphi_k(y) |dy \Big)^pdx \right]^{1/p}.
                \end{align}
            \end{subequations}
            À cette dernière expression nous appliquons l'inégalité de Minkowski (théorème~\ref{PropInegMinkKUpRHg}) sous la forme \eqref{EqZSiTZrH} pour la mesure \( d\nu(y)=| \varphi_k(y) |dy\) et \( f(x,y)=g(x-y)-g(x)\) :
            \begin{equation}
                \| d_k \|_p\leq \int_{\eR^d}\Big( \int_{\eR^d}\big| g(x-y)-g(x) \big|^pdx \Big)^{1/p}| \varphi_k(y) |dy=\int_{\eR^d}\| \tau_yg-g \|_p| \varphi_k(y) |dy.
            \end{equation}
            Par le lemme~\ref{LemCUlJzkA} nous pouvons trouver \( \alpha>0\) tel que \( \| \tau_yg-g \|_p\leq \epsilon\) pour tout \( y\in B(0,\alpha)\). Avec cela nous découpons encore le domaine d'intégration :
            \begin{equation}
                \| d_k \|_p\leq \int_{B(0,\alpha)}\underbrace{\| \tau_yg-g \|_p}_{\leq \epsilon}| \varphi_k(y) |dy+\int_{\complement B(0,\alpha)}  \underbrace{\| \tau_yg-g \|_p}_{\leq 2\| g \|_p}| \varphi_k(y) |dy\leq \epsilon M+2\epsilon\| g \|_p.
            \end{equation}

        \item

            Nous posons \( d_k(x)=(\varphi_k*g)(x)-g(x)\) et nous voulons prouver que \( \| d_k \|_{\infty}\to 0\), c'est-à-dire que \( d_k(x)\) converge vers zéro uniformément en \( x\). Nous posons aussi
            \begin{equation}
                \tau_y(g)\colon x\mapsto g(x-y).
            \end{equation}
            En récrivant le produit de convolution, une petite majoration donne
            \begin{equation}
                | d_k(x) |\leq \int_{\eR^d}\| \tau_y(g)-g \|_{\infty}| \varphi_k(y) |dy.
            \end{equation}
            L'uniforme continuité de \( g\) signifie que pour tout \( \epsilon\), il existe un \( \alpha\) tel que pour tout \( y\in B(0,\alpha)\),
            \begin{equation}
                \| \tau_y(g)-g \|_{\infty}\leq \epsilon.
            \end{equation}
            Encore une fois nous découpons le domaine d'intégration en \( B=B(0,\alpha)\) et son complémentaire :
            \begin{subequations}
                \begin{align}
                    \| d_k \|_{\infty}&\leq\int_B\underbrace{\| \tau_y(g)-g \|_{\infty}}_{\leq \epsilon}| \varphi_k(y) |dy+\int_{\complement B}\underbrace{\| \tau_y(g)-g \|_{\infty}}_{\leq 2\| g \|_{\infty}}| \varphi_k(y) |\\
                    &\leq \epsilon M+2\| g \|_{\infty}\epsilon
                \end{align}
            \end{subequations}
            où la seconde ligne est justifiée par le choix d'un \( k\) assez grand pour que \( \int_{\complement B}| \varphi_k(y) |dy\leq \epsilon\).

            Nous avons donc bien \( \| d_k \|_{\infty}\to 0\).
    \end{enumerate}
\end{proof}

\begin{example}
    Une petite remarque en passant : aussi triste que cela en ait l'air, la convergence uniforme n'implique pas la convergence \( L^p(\Omega)\) si \( \Omega\) n'est pas borné. En effet si \( f\in L^p\), la suite donnée par
    \begin{equation}
        f_n(x)=f(x)+\frac{1}{ n }
    \end{equation}
    converge uniformément vers \( f\), mais
    \begin{equation}
        \| f_n-f \|_p=\int_{\Omega}\frac{1}{ n }
    \end{equation}
    n'existe même pas si le domaine \( \Omega\) n'est pas borné.
\end{example}

%---------------------------------------------------------------------------------------------------------------------------
\subsection{Densité des polynômes trigonométriques}
%---------------------------------------------------------------------------------------------------------------------------

\begin{definition}      \label{DEFooGCZAooFecAHB}
    Le \defe{système trigonométrique}{système!trigonométrique} donné par \( \{ e_n \}_{n\in \eZ}\) est
    \begin{equation}
        e_n(t)= \frac{1}{ \sqrt{ 2\pi } } e^{int}.
    \end{equation}
\end{definition}

Une bonne partie de la douleur qu'évoque mot « densité » consiste à montrer que ce système est total dans \( L^2(S^1)=L^2(\mathopen[ 0 , 2\pi \mathclose])\), et donc en est une base hilbertienne.

\begin{definition}
    Un \defe{polynôme trigonométrique}{polynôme!trigonométrique} est une fonction de la forme
    \begin{equation}
        P(t)=\sum_{n=-N}^Nc_n e_n(t).
    \end{equation}
\end{definition}

\begin{definition}[Coefficients de Fourier]
    Pour toute fonction pour laquelle ça a un sens (que ce soit des fonctions \( L^2\) ou non), nous posons
    \begin{equation}\label{EqhIPoPH}
        c_n(f)=\langle f, e_n\rangle .
    \end{equation}
    Ces nombres sont les \defe{coefficients de Fourier}{coefficients!de Fourier} de \( f\). 
\end{definition}

Ces trois définitions n'ont à priori aucun rapport entre elles, et rien en particulier ne devrait vous faire penser à une égalité du type
\begin{equation}
    f(x)=\sum_{n=-\infty}^{\infty}c_n(f)e_n(x).
\end{equation}
Nous avons toutefois quelque liens.

\begin{lemma}   \label{LemZVfZlms}
    Deux petits résultats simples mais utiles à propos des polynômes trigonométriques.
    \begin{enumerate}
        \item
    Si \( f\in L^1(S^1)\), alors nous avons la formule
    \begin{equation}
        f*e_n=c_n(f)e_n.
    \end{equation}
\item

    Si \( P\) est un polynôme trigonométrique et si \( f\in L^1(S^1)\) alors \( f*P\) est encore un polynôme trigonométrique.
    \end{enumerate}
\end{lemma}

\begin{proof}
    Le premier point est un simple calcul :
    \begin{subequations}
        \begin{align}
            (f*e_n)(x)=\int_0^{2\pi}f(x-t)e_n(t)
        \end{align}
    \end{subequations}

    En ce qui concerne le second point, nous notons \( P=\sum_{k=-N}^NP_ke_k\), et par linéarité de la convolution,
    \begin{equation}
        f*P=\sum_{k=-N}^NP_kf*e_k=\sum_{k=-N}^nP_kc_k(f)e_k,
    \end{equation}
    qui est encore un polynôme trigonométrique.
\end{proof}

\begin{example} \label{ExDMnVSWF}
    Sur \( S^1\) nous construisons alors l'approximation de l'unité basée sur la fonction \( 1+\cos(x)\) et le lemme~\ref{LemCNjIYhv}. Cette fonction est évidemment un polynôme trigonométrique parce que
    \begin{equation}
        \cos(x)=\frac{  e^{ix}+ e^{-ix} }{2}.
    \end{equation}
    Ensuite les puissances le sont aussi à cause de la formule du binôme :
    \begin{equation}
        \big( 1+\cos(x) \big)^n=\sum_{k=0}^n\binom{ n }{ k }\cos^n(x),
    \end{equation}
    dans laquelle nous pouvons remettre \( \cos(x)\) comme un polynôme trigonométrique et développer à nouveau la puissance avec (encore) la formule du binôme. La chose importante est qu'il existe une approximation de l'unité \( (\varphi_n)\) formée de polynômes trigonométrique.

    Ce qui fait la spécificité des polynômes trigonométriques est qu'ils sont à la fois stables par convolution (lemme~\ref{LemZVfZlms}) et qu'ils permettent de créer une approximation de l'unité sur \( \mathopen[ 0 , 2\pi \mathclose]\). Ce sont ces deux choses qui permettent de prouver l'important théorème suivant.
\end{example}

\begin{theorem} \label{ThoQGPSSJq}
    Les polynôme trigonométriques sont dense dans \( L^p(S^1)\) pour \( 1\leq p<\infty\).
\end{theorem}

\begin{proof}
    
    \begin{equation}
        \varphi_k*f\stackrel{L^p}{\to}f
    \end{equation}
    par le théorème~\ref{ThoYQbqEez}. Nous avons donc convergence \( L^p\) d'une suite de polynômes trigonométrique, ce qui prouve que l'espace de polynômes trigonométriques est dense dans \( L^p(S^1)\).
\end{proof}

\begin{remark}
    Deux remarques.
    \begin{itemize}
        \item
            Il n'est pas possible que les polynômes trigonométriques soient dense dans \( L^{\infty}\) parce qu'une limite uniforme de fonctions continues est continue (c'est le théorème~\ref{ThoUnigCvCont}). Donc les polynômes trigonométriques ne peuvent engendrer que des fonctions continues.
        \item
            Nous donnerons au théorème~\ref{ThoDPTwimI} une démonstration indépendante de la densité des polynômes trigonométriques dans \( L^p(S^1)\).
    \end{itemize}
\end{remark}

%+++++++++++++++++++++++++++++++++++++++++++++++++++++++++++++++++++++++++++++++++++++++++++++++++++++++++++++++++++++++++++
\section{Espaces \texorpdfstring{$L^2$}{$L^2$}, généralités}
%+++++++++++++++++++++++++++++++++++++++++++++++++++++++++++++++++++++++++++++++++++++++++++++++++++++++++++++++++++++++++++
\label{SECooEVZSooLtLhUm}

L'espace \( L^2\) est l'espace \( L^p\) définit en \ref{DEFooKMJQooXeaUtp} avec \( p=2\). Cependant il possède une propriété extraordinaire par rapport aux autres \( L^p\), c'est que la norme \( | . |_2\) dérive d'un produit scalaire. Il sera donc un espace de Hilbert.

\begin{normaltext}  \label{NORMooUEIEooYtlFse}
    Nous en rappelons la construction. Soit \( (\Omega,\tribA,\mu)\) un espace mesuré. Nous considérons l'opération
    \begin{equation}    \label{DefProdScalLubrgTj}
        \langle f, g\rangle =\int_{\Omega}f(\omega)\overline{ g(\omega)}d\mu(\omega)
    \end{equation}
    et la norme associée
    \begin{equation}
        \| f \|_2=\sqrt{\langle f, f\rangle }.
    \end{equation}
    Nous considérons l'ensemble
    \begin{equation}
        \mL^2(\Omega,\mu)=\{ f\colon \Omega\to \eC\tq \| f \|_2<\infty \}
    \end{equation}
    et la relation d'équivalence \( f\sim g\) si et seulement si \( f(x)=g(x)\) pour \( \mu\)-presque tout \( x\).

    Et enfin, nous considérons le quotient
    \begin{equation}
        L^2(\Omega,\mu)=\mL^2(\Omega,\mu)/\sim.
    \end{equation}
\end{normaltext}


\begin{lemma}   \label{LemIVWooZyWodb}
    Soit un espace mesuré\quext{Est-ce qu'il ne faudrait pas un peu plus d'hypothèses, comme \( \sigma\)-fini par exemple ? Vérifiez et écrivez-moi quand vous avez la réponse.} \( (\Omega,\tribA,\mu)\).
    \begin{enumerate}
        \item
            Pour tout \( f,g\in L^2(\Omega,\tribA,\mu)\), le produit 
            \begin{equation}        \label{EQooGLVUooObPmaX}
                \langle f, g\rangle =\int_{\Omega}f\bar g\,d\mu 
            \end{equation}
            est bien défini et est un nombre complexe\footnote{Par opposition au fait que ce serait l'infini.}.
        \item
            L'opération \( (f,g)\mapsto \langle f, g\rangle \) est un produit hermitien\footnote{Définition \ref{DefMZQxmQ}. Pour rappel, nous considérons des fonctions à valeurs complexes. Si au contraire nous avions considéré seulement des fonctions à valeurs réelles, nous aurions eu un produit scalaire.}.
        \item
            Le couple \( \big( L^2(\Omega,\tribA,\mu),\langle ., .\rangle  \big)\) est un espace de Hilbert\footnote{Définition \ref{DefORuBdBN}.}.
    \end{enumerate}
\end{lemma}

\begin{proof}
    Que \( L^2(\Omega)\) soit un espace vectoriel est un cas particulier de la proposition \ref{PROPooTYCYooAKJWOX}. Voyons cette histoire de produit scalaire.

    \begin{subproof}
        \item[Pour de vraies fonctions]
            Nous commençons par analyser l'intégrale \eqref{EQooGLVUooObPmaX} dans le cas où \( f\) et \( g\) sont des fonctions, c'est-à-dire des représentants d'éléments de \( L^2\).

            Dans ce cas, l'inégalité de Hölder (proposition~\ref{ProptYqspT}) avec \( p=q=2\) nous indique que le produit \( f\bar g\) est un élément de \( L^1\). Par conséquent la formule a un sens.

        \item[Passage aux classes]

            Ensuite nous montrons que la formule passe au quotient. Pour cela, nous considérons des fonctions \( \alpha\) et \( \beta\) nulles presque partout et nous regardons le produit de \( f_1=f+\alpha\) par \( g_1=g+\beta\) :
            \begin{equation}
                \langle f_1, g_1\rangle =\int fg+\beta f+\alpha g+ \alpha\beta.
            \end{equation}
            Les fonctions \( \beta f\), \( \alpha g\) et \( \alpha\beta\) étant nulles presque partout, leur intégrale est nulle et nous avons bien \( \langle f_1, g_1\rangle =\langle f,g \rangle \). Nous pouvons donc considérer le produit sur l'ensemble des classes.

        \item[Produit hermitien]
            Pour vérifier que la formule est un produit hermitien, le seul point non évidement est de prouver que \( \langle f, f\rangle =0\) implique \( f=0\). Cela découle du fait que
            \begin{equation}
                \langle f, f\rangle =\int_{\Omega}| f |^2.
            \end{equation}
            La fonction \( x\mapsto | f(x) |^2\) vérifie les hypothèses du lemme~\ref{Lemfobnwt}. Par conséquent \( | f(x) |^2\) est presque partout nulle.

        \item[Espace de Hilbert]
            En ce qui concerne le fait que \( L^2(\Omega)\) soit un espace de Hilbert, il s'agit simplement de se remémorer que c'est un espace complet (théorème ~\ref{ThoUYBDWQX}) et dont la norme dérive d'un produit scalaire ou hermitien. Nous sommes donc bien dans la définition~\ref{DefORuBdBN}.
    \end{subproof}
\end{proof}

\begin{normaltext}
    Ces espaces seront utilisés pour de nombreuses applications. Nous en aurons besoin pour plusieurs combinaisons d'ensembles \( \Omega\) et de mesures \( \mu\).
    \begin{itemize}
        \item Pour \( \eR^d\)
        \item Pour \( S^1\)
        \item Pour \( \mathopen[ a , b \mathclose]\)
        \item Pour \( \mathopen[ 0 , 2\pi \mathclose[\)
            \item Pour \( \mathopen[ -T , T \mathclose[\)
    \end{itemize}
    Le premier est non compact et il est raisonnable de penser qu'il sera foncièrement différents des autres. À isomorphismes assez triviaux près, les espaces des fonctions sur les trois autres sont identiques. Nous nous attendons donc à ce qu'ils aient les mêmes propriétés. Notons que du point de vue de \( L^2\), étant donné qu'il y a un quotient par les parties de mesures nulles, prendre \( \mathopen] 0 , 2\pi \mathclose[\) ou \( \mathopen[ 0 , 2\pi \mathclose]\) ou n'importe quelle autre possibilité de ce genre revient au même.

    Afin de pouvoir utiliser ces espaces de façon optimale, et entre autres y définir les séries de Fourier, nous avons besoin, pour chacun d'entre eux de définir les éléments suivants :
    \begin{itemize}
        \item mesure
        \item produit de convolution
        \item le système trigonométrique (que nous allons montrer être une base hilbertienne)
        \item coefficients de Fourier
    \end{itemize}
    Ça fait pas mal de choses à définir. Il n'est pas besoin de définir un produit scalaire parce que le lemme \ref{LemIVWooZyWodb} nous en donne un générique.

    Les définitions qui viennent sont à prendre «tant que les formules ont un sens». Nous parlons donc de fonctions dans \( \Fun(\Omega,\eC)\), l'ensemble de toutes les fonctions sur \( \Omega\) à valeurs dans \( \eC\). Nous verrons plus tard les espaces de fonctions sur lesquels tout a un sens.
\end{normaltext}

%+++++++++++++++++++++++++++++++++++++++++++++++++++++++++++++++++++++++++++++++++++++++++++++++++++++++++++++++++++++++++++ 
\section{L'espace \( L^2(\eR^d)\)}
%+++++++++++++++++++++++++++++++++++++++++++++++++++++++++++++++++++++++++++++++++++++++++++++++++++++++++++++++++++++++++++

La mesure est celle de Lebesgue. Le produit de convolution est donné, pour \( f,g\in\Fun(\eR^d,\eC)\), par
\begin{equation}
    (f*g)(x)=\int_{\eR^d}f(y)g(x-y)dy
\end{equation}
Certaines de ses propriétés ont déjà été vues dans le théorème \ref{THOooMLNMooQfksn}.

En ce qui concerne le système trigonométrique, pour tout \( \xi\in \eR^d\) nous définirions bien
\begin{equation}
    e_{\xi}(x)= e^{i\xi\cdot x},
\end{equation}
genre pour faire que les transformations de Fourier sont des séries continues \ldots mais bon. Nous n'allons pas tenter le diable plus que ça, et nous ne définissons 
\begin{itemize}
    \item pas de système trigonométrique,
    \item pas de coefficients de Fourier non plus,
    \item pas de théorie des séries de Fourier sur \( \eR^d\).
\end{itemize}
Quand je disais que la non-compacité de \( \eR^d\) allait un peu changer les choses par rapport aux autres, je ne rigolais pas.

%+++++++++++++++++++++++++++++++++++++++++++++++++++++++++++++++++++++++++++++++++++++++++++++++++++++++++++++++++++++++++++ 
\section{L'espace \( L^2(S^1)\)}
%+++++++++++++++++++++++++++++++++++++++++++++++++++++++++++++++++++++++++++++++++++++++++++++++++++++++++++++++++++++++++++

L'espace \( S^1\) sera fait avec forces détails, parce qu'il va servir de base pour les espaces \( L^2(\mathopen[ 0 , 2\pi \mathclose[)\), \( L^2(\mathopen[ -T , T \mathclose[)\) ainsi que pour l'étude des fonctions périodiques sur \( \eR\).

En tant qu'ensemble,
\begin{equation}
    S^1=\{  e^{it} \}_{t\in \eR},
\end{equation}
sans garanties que ce paramétrage soit une bijection.

Il y a essentiellement deux façons de définir une intégrale sur \( S^1\).
\begin{enumerate}
    \item Voir \( S^1\) comme une sous-variété de \( \eR^2\) et utiliser la définition \ref{PROPooOAHWooAfxvyv}. Cette façon a cependant deux inconvénients :
        \begin{itemize}
            \item Elle ne donne pas la tribu des mesurables sur \( S^1\), c'est-à-dire que cette méthode ne donne pas de façon évidente une théorie de la mesure sur \( S^1\).
            \item Il faut au moins deux cartes pour paramétrer le cercle. La fainéantise nous prévient que ça va être technique.
        \end{itemize}
    \item
        Rapporter la structure d'espace mesuré de \( \mathopen[ 0 , 2\pi \mathclose[\) vers \( S^1\), de force via le premier difféomorphisme qui nous passe par la tête, à savoir \( t\mapsto  e^{it}\).
\end{enumerate}
Nous allons choisir la seconde possibilité, en gardant en tête qu'elle fonctionne de façon très simple un peu par coup de chance, voir la remarque \ref{REMooOMYYooNFiKOs}\ref{ITEMooJTKCooYQknqo}.

%--------------------------------------------------------------------------------------------------------------------------- 
\subsection{Espace mesuré}
%---------------------------------------------------------------------------------------------------------------------------

Plusieurs choses sont déjà faites.
\begin{itemize}
    \item Les boréliens de \( S^1\) sont décrits dans la proposition \ref{PROPooHMSCooRIjcJq},
    \item la tribu de Lebesgue de \( S^1\) est décrite dans la proposition \ref{PROPooDLBCooUfQZOa}. Non, ce n'est pas la tribu induite de la tribu de Lebesgue de \( \eC\).
\end{itemize}

\begin{proposition}[Espaces de fonctions sur \( S^1\)\cite{MonCerveau}]     \label{PROPooDJERooYirMru}
    Soit l'espace mesuré \( \big( S^1,\Lebesgue(S^1), \mu \big)\).
    \begin{enumerate}
            \item
                La formule
                \begin{equation}        \label{EQooHPFQooEaujfZ}
                    \langle f, g\rangle =\int_{S^1}f\bar gd\mu
                \end{equation}
                est un produit hermitien\footnote{Définition \ref{DefMZQxmQ}.} sur \( L^2(S^1,\Lebesgue(S^1),\mu)\).
            \item
                L'espace \( L^2(S^1)\) est un espace de Hilbert.
            \item       \label{ITEMooQZAPooKEeQBW}
    L'application
    \begin{equation}
        \begin{aligned}
            \phi\colon L^2(S^1)&\to L^2\big( \mathopen[ 0 , 2\pi \mathclose[ \big) \\
                f&\mapsto \frac{1}{ \sqrt{ 2\pi } }f\circ \varphi.
        \end{aligned}
    \end{equation}
            est une bijection isométrique (isomorphisme d'espaces de Hilbert)
            \begin{equation}
            L^2(S^1,\Lebesgue(S^1),\mu)=L^2\big( \mathopen] 0  , 2\pi \mathclose[,\Lebesgue(\eR),\lambda \big)
            \end{equation}
        où nous avons fait un minuscule abus de notations : ici \( \Lebesgue(\eR)\) est en réalité la tribu induite sur \( \mathopen] 0 , 2\pi \mathclose[\).
    \end{enumerate}
\end{proposition}

\begin{proof}
    Le fait que la formule \eqref{EQooHPFQooEaujfZ} donne bien un produit hermitien est le lemme \ref{LemIVWooZyWodb}. Ce même lemme assure que le tout donne un espace de Hilbert.

    Il nous reste à prouver le point \ref{ITEMooQZAPooKEeQBW}. En ce qui concerne l'isométrie, nous posons\footnote{Notez que cette définition passe aux classes. Nous le répéterons pas.}
    \begin{equation}
        \begin{aligned}
            \phi\colon L^2(S^1)&\to L^2\big( \mathopen[ 0 , 2\pi \mathclose[ \big) \\
                f&\mapsto \frac{1}{ \sqrt{ 2\pi } }f\circ \varphi.
        \end{aligned}
    \end{equation}
    \begin{subproof}
        \item[Injection]
            Si \( \phi(f)=\phi(g)\), alors pour tout \( x\in\mathopen[ 0 , 2\pi \mathclose[\) nous avons \( f\big( \varphi(x) \big)=g\big( \varphi(x) \big)\). Vu que \( \varphi\colon \mathopen[ 0 , 2\pi \mathclose[\to S^1\) est une bijection nous avons alors \( f(s)=g(s)\) pour tout \( s\in S^1\).
            \item[Surjection]
            Si \( f\in L^2\big( \mathopen] 0 , 2\pi \mathclose] \big)\), nous posons \( g\colon S^1\to \eC\) par
            \begin{equation}
                g(s)=\sqrt{ 2\pi }f\big( \varphi^{-1}(s) \big).
            \end{equation}
            Nous avons alors bien \( \phi(g)(x)=f(x)\).
        \item[Isométrie]
            Nous montrons que \( \phi\) préserve le produit scalaire :
            \begin{subequations}        \label{SUBEQSooRYYHooPcLXHN}
                \begin{align}
                    \langle \phi(f), \phi(g)\rangle &=\int_0^{2\pi}\phi(f)(x)\overline{ \phi(g)(x) }d\lambda(x)\\
                    &=\frac{1}{ 2\pi }\int_0^{2\pi}(f\circ\varphi)(x)\overline{ (g\circ\varphi)(x) }\,d\lambda(x)\\
                    &=\frac{1}{ 2\pi }\int_0^{2\pi}(fg)\circ\varphi\, d\lambda
                \end{align}
            \end{subequations}
            Pour la suite nous devons invoquer la proposition \ref{PROPooILOEooBiumKD} pour passer d'une intégrale sur \( \big( \mathopen[ 0 , 2\pi \mathclose[,\Lebesgue\big( \mathopen[ 0 , 2\pi \mathclose[ \big),\lambda \big)\) à une intégrale sur \( \big( S^1,\Lebesgue(S^1), \mu \big)\). La première condition de cette proposition est que \( \Lebesgue(S^1)=\varphi\big( \Lebesgue(\mathopen[ 0 , 2\pi \mathclose[) \big)\). Cela est la proposition \ref{PROPooDLBCooUfQZOa}\ref{ITEMooNIRNooKSeyCa}. La condition sur la mesure dans la proposition \ref{PROPooILOEooBiumKD} n'est vraie ici qu'à un facteur \( 2\pi\) près. Nous avons :
                \begin{equation}
                    \int_{\mathopen[ 0 , 2\pi \mathclose[}fd\lambda=2\pi\int_{S^1}(f\circ \varphi^{-1})d\mu.
                \end{equation}
                Nous continuons le calcul \eqref{SUBEQSooRYYHooPcLXHN} :
                \begin{equation}
                    \langle \phi(f), \phi(g)\rangle =\frac{1}{ 2\pi }\int_0^{2\pi}(fg)\circ\varphi\, d\lambda=\int_{S^1}f\bar gd\mu=\langle f, g\rangle .
                \end{equation}
    \end{subproof}
\end{proof}

%--------------------------------------------------------------------------------------------------------------------------- 
\subsection{Topologie}
%---------------------------------------------------------------------------------------------------------------------------

Nous considérons sur \( S^1\) la topologie induite de \( \eC\). Vu que \( S^1\) est fermé et borné dans \( \eC\), il en est une partie compacte. Par le lemme \ref{LEMooVYTRooKTIYdn}, l'espace \( S^1\) muni de sa topologie est un espace topologique compact.

Nous pouvons donc sans crainte affirmer que toute fonction continue \( f\colon S^1\to \eK\) est bornée et atteint ses bornes.

\begin{propositionDef}      \label{PROPooEQDBooDfOrTZ}
    Soit la formule
    \begin{equation}
        d( e^{ix},  e^{iy})=\inf_{k\in \eZ}| x-y+2k\pi |.
    \end{equation}
    \begin{enumerate}
        \item
            Elle est bien définie (ne dépend pas des choix de \( x\) et \( y\) donnant les mêmes points dans \( S^1\))
        \item
            L'infimum est en réalité un minimum : il est atteint par un certain \( k\in \eZ\) (qui, lui, dépend des choix).
        \item
            La formule définit une distance\footnote{Définition \ref{DefMVNVFsX}.} sur \( S^1\).
    \end{enumerate}
    Nous considérons sur \( S^1\) la topologie \( \tau_d\) découlant de cette distance.
\end{propositionDef}


\begin{proof}
    Point par point.
    \begin{enumerate}
        \item
            Soient \( x',y'\in \eR\) tels que \(  e^{ix'}= e^{ix}\) et \(  e^{iy'}= e^{iy}\). Alors \( x'=x+2l\pi\) et \( y'=y+2l'\pi\) pour certains entiers \( l,l'\in \eZ\) (corolaire \ref{CORooTFMAooHDRrqi}). Nous avons alors \( | x'-y'+2k\pi |=| x-y+2\pi(k+l-l') |\) et
            \begin{equation}
                \inf_{k\in \eZ}| x'-y'+2k\pi |=\inf_{k\in \eZ}| x-y+2k\pi |.
            \end{equation}
        \item
            Quels que soient \( x\) et \( y\) fixés, nous avons
            \begin{equation}
                \lim_{k\to \pm\infty} | x-y+2k\pi |=\infty.
            \end{equation}
            Donc l'infimum est forcément atteint par un \( k\in \eZ\).
        \item
            Pour la distance, il y a plusieurs points à prouver. 
            \begin{itemize}
                \item 
                    Pour tout \( z,z'\in S^1\) nous avons \( d(z,z')\geq 0\) parce que la distance est donnée par une valeur absolue.
                \item
                    Si \( d(z,z')=0\), alors il existe \( k\) tel que \( x=y+2k\pi\). Alors \(  e^{ix}= e^{i(y+2k\pi)}= e^{iy} e^{2ki\pi}= e^{iy}\). C'est-à-dire \( z=z'\).
                \item
                    Pour la symétrie, nous avons
                    \begin{equation}
                        | x-y+2k\pi |=| y-x-2k\pi |=| y-x+2k'\pi |
                    \end{equation}
                    en posant \( k'=-k\). L'infimum étant pris sur \( k\in \eZ\), nous avons al symétrique \( d( e^{ix},  e^{iy})=d( e^{iy}, e^{ix})\).
                \item
                    Pour attaquer l'inégalité triangulaire, nous considérons \( z_1= e^{ix_1}\), \( z_2= e^{ix_2}\) et \( z_3= e^{ix_3}\). Nous posons également \( k_1,k_2, k_3\in \eZ\) tels que \( d(z_1,z_3)=| x_1-x_3+2k_1\pi | \), \( d(z_1,z_2)=| x_1-x_2+2k_2\pi |\) et \( d(z_2,z_3)=| x_2-x_3+2k_3\pi |\). Nous avons alors
                    \begin{subequations}
                        \begin{align}
                            d(z_1,z_3)&=| x_1-x_3+2k_1\pi |=| x_1-x_2+x_2-x_3+2k_1\pi |\\
                            &=\inf_{k\in \eZ}| (x_1-x_2+2k_2\pi)+(x_2-x_3+2k_3\pi)+2k\pi |\\
                            &=\inf_{k\in \eZ}\Big( | x_1-x_2+2k_1\pi |+| x_2-x_3+2k_3\pi |+2k\pi \Big)\\
                            &=d(z_1,z_3)+d(z_2,z_3)
                        \end{align}
                    \end{subequations}
                    parce que le dernier infimum est réalisé par \( k=0\).
            \end{itemize}
        \end{enumerate}
\end{proof}

Le cercle est bien connu pour être symétrique et en particulier avoir une symétrie sous les rotations. Nous allons voir quelques résultats qui vont dans le sens de dire que la distance définie sur \( S^1\) respecte cette symétrie.

\begin{lemma}       \label{LEMooCQCAooAEctbe}
    Plusieurs points à propos de l'invariance de la topologie sous les rotations.
    \begin{enumerate}
        \item
            La distance est invariante sous les rotations, c'est-à-dire que si \( a,b\in S^1\) et si \( s\in \eR\), alors
            \begin{equation}
                d( e^{is}a, e^{is}b)=d(a,b).
            \end{equation}
        \item       \label{ITEMooCIPYooTyPQLj}
            Les boules sont préservées sous les rotations\footnote{Pas chaque boule séparément, mais l'ensemble des boules}, c'est-à-dire que
            \begin{equation}
                e^{is}B_d(a,r)=B_d( e^{is}a,r).
            \end{equation}
        \item
            La topologie est invariante sous les rotations : \(  e^{is}\tau_d=\tau_d\).
    \end{enumerate}
\end{lemma}

\begin{proof}
    Point par point.
    \begin{enumerate}
        \item
            Si \( a= e^{ix}\) et \( b= e^{iy}\), nous avons
            \begin{equation}
                d(a,b)=\inf_{k\in \eZ}| x-y+2k\pi |=\inf_{k\in \eZ}| (x-s)-(y-s)+2k\pi |=d( e^{is}a, e^{is}b).
            \end{equation}
            Et de un.
        \item
            Il faut une inclusion dans chaque sens.
            \begin{subproof}
                \item[\(  e^{is}B_d(a,r)\subset B_d( e^{is}a,r)\)]
                    Soit \( b\in  e^{is}B_d(a,r)\). Alors \( b= e^{is}b'\) pour un certain \( b'\in B_d(a,r)\). Nous avons alors, en utilisant le premier point,
                    \begin{equation}
                        d(b, e^{is})=d( e^{-is}b,a)=d(b',a)<r.
                    \end{equation}
                    Donc \( b\in B_d( e^{is}a,r)\).
                \item[\(  B_d( e^{is}a,r)\subset e^{is} B_d( a,r)\)]
                    Soit \( b\in B_d( e^{is}a,r)\). Nous devons prouver que \( b\in  e^{is}B_d(a,r)\), c'est-à-dire que \( b= e^{is}b'\) pour un certain \( b'\in B_d(a,r)\) ou encore que \(  e^{-is}b\in B_d(a,r)\). En utilisant encore le premier point,
                    \begin{equation}
                        d( e^{-is}b,a)=d(b, e^{is}a)<r.
                    \end{equation}
                    Donc oui, \(  e^{-is}b\in B(a,r)\).
            \end{subproof}
        \item
            Soit \( A\in\tau_d\). Si \( a\in  e^{is}A\), alors \( a= e^{is}a'\) pour un certain \( a'\in A\). Notre but est de prouver que \(  e^{is}A\) contient un voisinage de \( a\).

            Vu que \( a'\in A\), il existe \( r>0\) tel que \( B_d(a',r)\subset A\). Nous avons alors
            \begin{equation}
                e^{is}aB_d(a',r)\subset  e^{is}A,
            \end{equation}
            et comme \(  e^{is}B_d(a',r)=B_d( e^{is}a',r)=B_d(a,r)\) nous avons bien
            \begin{equation}
                B_d(a,r)\subset  e^{is}A.
            \end{equation}
    \end{enumerate}
\end{proof}

Nous allons voir maintenant quelques résultats à propos de \(B_d(1,r)\) qui a la bonne figure d'être un ouvert qui s'étale symétriquement en partant de \( 1\) (le point le plus à droite du cercle). Par rapport à la figure \ref{LabelFigJOQVoolPTsYPZK}, il s'agit ni plus ni moins que de voir qu'une boule de rayon \( r\) autour de \( 1\) est bien la partie indiquée (symétrique par rapport à \( 1\) et de longueur d'arc \( r\) des deux côtés). De plus, ce voisinage n'est autre que la partie du cercle située à droite de la ligne en pointillés.
\newcommand{\CaptionFigJOQVoolPTsYPZK}{Un voisinage de \( 1\) dans \( S^1\).}
\input{auto/pictures_tex/Fig_JOQVoolPTsYPZK.pstricks}

Ce lemme-ci montre que \( B_d(1,r)\) est une partie de \( S^1\) qui s'étale symétriquement autour de \( 1\).
\begin{lemma}       \label{LEMooMYNVooIWWsiV}
    Soit l'application
    \begin{equation}
        \begin{aligned}
            \varphi\colon \eR&\to S^1 \\
            x&\mapsto  e^{ix}.
        \end{aligned}
    \end{equation}
    Nous avons \( B_d(1,r)=\varphi\big( \mathopen] -r , r \mathclose[ \big)\).
\end{lemma}

\begin{proof}
    Soit \( b\in B_d(1,r)\) de la forme \( b= e^{iy}\) avec \( y\) choisi de telle sorte que \( d(1,b)=| y |\). Vu que \( d(1,b)<r\), nous avons \( | y |<r\) et donc \( b\in \varphi\big( \mathopen] -r , r \mathclose[ \big)\).

    Dans l'autre sens, si \( y\in\mathopen] -r , r \mathclose[\), alors
        \begin{equation}        \label{EQooRWASooVZnQCJ}
            d(1, e^{iy})=\inf_{k\in \eZ}| y+2k\pi |\leq | y |<r.
        \end{equation}
        Nous avons utilisé le fait que l'infimum sur \( k\in \eZ\) est plus petit ou égal à la valeur pour \( k=0\). Les inégalités \eqref{EQooRWASooVZnQCJ} montrent que \(  e^{iy}\in B_d(1,r)\).
\end{proof}

Le lemme suivant montre que que les boules autour de \( 1\) sont délimitées par la droite en pointillé de la figure \ref{LabelFigJOQVoolPTsYPZK}.
\begin{lemma}       \label{LEMooLINCooHJmJWx}
    Soit \( r\in \mathopen[ 0 , \pi \mathclose]\). Nous avons
    \begin{equation}
        B_d(1,r)=S^1\cap\{x+iy\tq x>\cos(r)\}.
    \end{equation}
\end{lemma}

\begin{proof}
    Si \( r=\pi\), alors \( B(1,r)=S^1\setminus\{ -1 \}\), alors que \( \cos(\pi)=-1\).

Si \( r<\pi\), alors nous partons de la formule \eqref{EQooRVPJooTMwNTU} qui dit que \(  e^{ir}=\cos(e)+i\sin(r)\). D'après le lemme \ref{LEMooMYNVooIWWsiV}, un élément de \( B_d(1,r)\) est de la forme \(  e^{iy}\) avec \( y\in \mathopen] -r , r \mathclose[\). Nous voudrions donc prouver que \( \cos(y)>\cos(r)\) dès que \( y\in\mathopen] -r , r \mathclose[\) et \( r<\pi\).

    %TODOooBZPCooAFVbcT
Sur \( \mathopen] -r , 0 \mathclose[\), la fonction \( \cos\) est croissante, donc si \( y<0\) alors 
    \begin{equation}
        \cos(y)>\cos(-r)=\cos(r).
    \end{equation}
De la même façon, sur \( \mathopen] 0,r \mathclose[\), la fonction \( \cos\) est décroissante, de telle sorte que si \( y>0\), alors \( \cos(y)>\cos(r)\).

    Nous avons prouvé que \( B_d(1,r)\subset S^1\cap  \{ x+iy\tq x>\cos(r) \}\). 

    Lançons nous dans la preuve de l'inclusion inverse.

    Soit \( x>\cos(r)\). Si \( x+iy\in S^1\), nous avons \( x+iy= e^{is}=\cos(s)+i\sin(s)\) pour un certain \( s\in\mathopen[ -\pi , \pi \mathclose[\). Notons que \( s=-\pi\) correspondrait au point \( -1\in S^1\), qui est exclu de notre étude parce que nous supposons \( r<\pi\). Donc \( s\in\mathopen] -\pi , \pi \mathclose[\).

    Nous avons donc \( \cos(r)<x=\cos(s)\). Et voila.
\end{proof}

\begin{proposition}
    La topologie \( \tau_d\) sur \( S^1\)\footnote{Définition \ref{PROPooEQDBooDfOrTZ}.} est la topologique induite depuis \( \eC\).
\end{proposition}

\begin{proof}
    Nous notons \( \tau_i\) la topologie induite (c'est-à-dire l'ensemble des ouverts) et \( \tau_d\) la topologie de la distance fraichement définie. Nous allons également noter \( B_{\eC}(a,r)\) la boule dans \( \eC\) de centre \( a\) et de rayon \( r\), et \( B_d(z,r)\) celle dans \( S^1\), de centre \( z\in S^1\) et de rayon \( r\) pour notre distance \( d\).
    \begin{subproof}
        
    \item[\( \tau_i\subset\tau_d\)]
        Un élément général de \( \tau_i\) est de la forme \( \mO\cap S^1\) où \( \mO\) est un ouvert de \( \eC\). Soit \( a\in \mO\cap S^1\) et prouvons qu'il existe un ouvert de \( \tau_d\) contenant \( a\) et contenu dans \( \mO\cap S^1\); cela prouvera que \( \mO\cap S^1\) est ouvert de \( \tau_d\) par le théorème \ref{ThoPartieOUvpartouv}.

        Soient \( a= e^{ix}\) et \( r\) tel que \( B_{\eC}(a,r)\subset \mO\). Nous allons montrer que \( B_{d}(a,r)\subset B_{\eC}(a,r)\). Un élément général de \( B_d(a,r)\) est \( b= e^{iy}\) tel que
        \begin{equation}
            d(a,b)=\inf_{k\in \eZ}| x-y+2k\pi |\leq r.
        \end{equation}
        Quitte à redéfinir \( x\) ou \( y\) nous pouvons supposer que l'infimum est atteint en \( k=0\). En utilisant la proposition \ref{PROPooYMMKooSUBtoo} nous majorons :
        \begin{equation}
            | a-b |=|  e^{ix}- e^{iy} |\leq | x-y |=d(a,b)\leq r.
        \end{equation}
        Donc nous avons bien \( b\in B_{\eC}(a,r)\) dès que \( b\in B_d(a,r)\).

    \item[\( \tau_d\subset \tau_i\)]

        Ce sens est plus délicat parce que, si nous voulons suivre les mêmes pas que le premier sens, nous devrons nous appuyer sur la continuité de l'application \( \ln\colon \eC^*\to \eC\), laquelle n'est pas vraie en \( -1\) (voir par exemple le lemme \ref{LEMooMUOIooCnoWwq}).

        Soient \( A\in \tau_d\) et \( a\in A\). Nous devons prouver l'existence d'un ouvert \( \mO\) de \( \eC\) tel que \( S^1\cap\mO\) soit inclus dans \( A\) et contienne \( a\). Nous allons prouver cela dans le cas \( a=1\) et ensuite propager le résultat en utilisant la symétrie de \( S^1\).

        \begin{subproof}
            \item[Si \( a=1\)]
                Vu que \( A\) est ouvert pour la topologie de la distance \( d\), et vu que \( 1\in A\), il existe \( r>0\) tel que \( B_d(1,r)\subset A\). Pour ce \( r\) le lemme \ref{LEMooLINCooHJmJWx} donne
                \begin{equation}
                    B_d(1,r)=S^1\cap\{ x+iy\tq x>\cos(r) \}.
                \end{equation}
                Nous montrons que \( \mO=B_{\eC}(1,\delta)\) avec \( \delta<1-\cos(r)\) fait l'affaire. Si \( x+iy\in B_{\eC}(1,\delta)\), alors \( x>1-\delta\) et donc
                \begin{equation}
                    1-\delta>1-(1-\cos(r))=\cos(r),
                \end{equation}
                ce qui prouve que la partie de \( \mO\) qui est dans \( S^1\) est bien dans \( B_d(1,r)\).
            \item[Si \( a\neq 1\)]

                Soient un ouvert quelconque \( A\in\tau_d\) ainsi que \( a= e^{ix}\in A\). Nous considérons \( r>0\) tel que \( B_d(a,r)\subset A\); nous avons \(  e^{-ix}B_d(d,r)\subset  e^{-ix}A\) et donc, en tenant compte du lemme \ref{LEMooCQCAooAEctbe}\ref{ITEMooCIPYooTyPQLj} :
                \begin{equation}
                    B_d(1,r)\subset  e^{-ix}A.
                \end{equation}
                Par le premier point, il existe un ouvert \( \mO\) de \( \eC\) tel que \( 1\in \mO\) et
                \begin{equation}
                    \mO\cap S^1\subset B_d(1,r)\subset  e^{-ix}A.
                \end{equation}
                Nous avons évidemment que \( a\in e^{ix}\mO\) et
                \begin{equation}
                    e^{ix}(\mO\cap S^1)\subset  e^{ix}B_d(1,r)\subset A.
                \end{equation}
                Donc \(  e^{ix}\mO\cap S^1\subset A\). Vu que \(  e^{ix}\mO\) est un ouvert de \( \eC\), l'ensemble \(  e^{ix}\mO\cap S^1\) est un ouvert de \( \tau_i\).
        \end{subproof}
\end{subproof}
\end{proof}

\begin{lemma}[\cite{MonCerveau}]        \label{LEMooTKFHooJaeMyc}
    Deux résultats de limites dans \( S^1\).
    \begin{enumerate}
        \item       \label{ITEMooEUDIooDuynRg}
            Pour tout \( a_0\in S^1\), nous avons
            \begin{equation}
                \lim_{s\to 1} d(a,as)=0.
            \end{equation}
        \item       \label{ITEMooXCBUooUxQldB}
            Si \( f\colon S^1\to \eC\) est continue en \( a\in S^1\), alors 
            \begin{equation}
                \lim_{s\to 1} f(as)=f(a).
            \end{equation}
    \end{enumerate}
\end{lemma}

\begin{proof}
    Point par point.
    \begin{enumerate}
        \item
            Soient \( a,s\in S^1\). Donnons une formule pour \( d(a,as)\). Si \( a= e^{ix}\) et \( s= e^{iy}\) nous avons \( as= e^{i(x+y)}\) et donc
            \begin{equation}
                d(a,as)=\inf_{k\in \eZ}| x-(x+y)+2k\pi |=\int_{k\in \eZ}| y+2k\pi |.
            \end{equation}
            Voila pour la formule. Maintenant la preuve de notre point.

            Soit \( \epsilon>0\). Si \( \delta<\epsilon\) et si \( s\in B(1,\delta\), alors il existe \( y\in \mathopen] -\delta , \delta \mathclose[\) tel que \( s= e^{iy}\) par le lemme \ref{LEMooMYNVooIWWsiV}. Pour un tel \( s\) nous avons
                \begin{equation}
                    d(a,sa)=\inf_{k\in \eZ}| y+2k\pi |\leq | y |<\delta<\epsilon.
                \end{equation}
                Nous avons trouvé \( \delta>0\) tel que \( s\in B(1,\delta)\) implique \( d(a,as)<\epsilon\). Cela est la limite que nous devions prouver.
            \item
                Soit \( \epsilon>0\). Soit \( r>0\) tel que si \( b\in B(a,r)\), alors \( | g(b)-g(a) |<\epsilon\); l'existence d'un tel \( r\) est la continuité de \( g\) en \( a\). Nous considérons \( \delta>0\) tel que \( s\in B(1,\delta)\) implique \( sa\in B(a,r)\); l'existence d'un tel \( \delta\) est le point \ref{ITEMooEUDIooDuynRg} de ce lemme.

                Avec tout cela nous avons \( | g(as)-g(a) |<\epsilon\) dès que \( s\in B(1,\delta)\). Nous avons donc, comme nous le voulions, la limite \( \lim_{s\to 1} g(as)=g(a)\).
    \end{enumerate}
\end{proof}

\begin{lemma}
    Pour \( s\in S^1\), nous considérons l'application
    \begin{equation}
        \begin{aligned}
            \alpha_s\colon \Fun(S^1)&\to \Fun(S^1) \\
            \alpha_s(g)(u)&=g(u\bar s)-g(u).
        \end{aligned}
    \end{equation}
    Quelques propriétés avec \( 1\leq p<\infty\) :
    \begin{enumerate}
        \item
            Si \( f\in L^p(S^1)\), alors \( \alpha_s(f)\in L^p(S^1)\).
        \item
            Si \( f\) est continue dans \( L^p(S^1)\) nous avons la limite
            \begin{equation}
                \lim_{s\to 1} \alpha_s(f)=0
            \end{equation}
            dans \( L^p(S^1)\).
    \end{enumerate}
\end{lemma}

\begin{proof}
    D'abord un calcul de norme :
    \begin{equation}
        \| \alpha_s(f) \|_p^p=\int_{S^1}| f(u\bar s)-f(u) |^pdu\leq\int_{S^1}| f(u\bar s) |^pdu+\int_{S^1}| f(u) |^pdu=2\| f \|_p^p.
    \end{equation}
    Donc oui pour que \( \alpha_s(f)\in L^p(S^1)\).

    La fonction \( f\) étant supposée continue sur le compact \( S^1\), elle est majorée. Nous savons qu'en posant \( \| f \|_{\infty}\) nous avons \( | \alpha_s(f) |\leq 2M\). Donc la fonction constante
    \begin{equation}
        \begin{aligned}
            g\colon S^1&\to \eC \\
            u&\mapsto 2M 
        \end{aligned}
    \end{equation}
    est une fonction intégrable sur \( S^1\) qui majore \( | \alpha_s(f) |\) uniformément en \( s\). Soit une suite \( s_i\to 1\) dans \( S^1\), et posons \( f_i=\alpha_{s_i}(f)\). Alors nous avons 
    \begin{equation}
        \| f_i \|^p_p=\int_{S^1}| f_i(u) |^pdu
    \end{equation}
    et aussi \( | f_i |^p\leq (2M)^p\). Le théorème de la convergence dominée de Lebesgue \ref{ThoConvDomLebVdhsTf} nous permet de permuter limite et intégrale :
    \begin{equation}
        \lim_{i\to \infty} \| f_i \|_p^p=\int_{S^1}\lim_{i\to \infty} | f_i(u) |^pdu.
    \end{equation}
    Mais
    \begin{equation}
        \lim_{i\to \infty} f_i(u)=\lim_{i\to \infty} \alpha_{s_i}(f)(u)=\lim_{i\to \infty} \big( f(u\bar s_i)-f(u) \big)=0.
    \end{equation}
    La dernière limite est due au fait que \(  \lim_{s\to 1} g(us)=g(u) \) (lemme \ref{LEMooTKFHooJaeMyc}\ref{ITEMooXCBUooUxQldB}).
\end{proof}

%--------------------------------------------------------------------------------------------------------------------------- 
\subsection{Système trigonométrique}
%---------------------------------------------------------------------------------------------------------------------------

\begin{definition}
    La \defe{famille trigonométrique}{famille trigonométrique!sur \( S^1\)} sur \( S^1\) est l'ensemble de fonctions \( \{ e_n \}_{n\in \eZ}\) données par
    \begin{equation}
        \begin{aligned}
            e_n\colon S^1&\to \eC \\
            z&\mapsto z^n
        \end{aligned}
    \end{equation}
    avec \( n\in \eZ\). Un \defe{polynôme trigonométrique}{polynôme trigonométrique} est une application \( S^1\to \eC\) de la forme
    \begin{equation}
        \sum_{k=-n}^na_ke_k
    \end{equation}
    pour des nombres \( a_k\in \eC\), peut-être pas tous non-nuls (autrement dit, il n'est pas forcé d'avoir autant de termes négatifs que positifs).
\end{definition}
Le but de \( z\mapsto z^n\) dans cette définition est d'être lu \(  t\mapsto e^{in t}\) lorsqu'on considère les fonctions sur \( \mathopen[ 0 , 2\pi \mathclose[\).

\begin{proposition}     \label{PROPooOMGFooROFFFr}
    La famille trigonométrique est une famille orthonormale pour le produit scalaire \( L^2(S^1,\tribA,\mu)\).
\end{proposition}

\begin{proof}
    En utilisant la proposition \ref{PROPooDJERooYirMru}\ref{ITEMooQZAPooKEeQBW} nous avons :
    \begin{subequations}
        \begin{align}
            \langle e_n, e_n\rangle =\int_{S^1}e_n\overline{ e_n }&=\frac{1}{ 2\pi }\int_{\mathopen[ 0 , 2\pi \mathclose[}e_n\big( \varphi(x) \big)\overline{ e_n\big( \varphi(x) \big) }\\
                &=\frac{1}{ 2\pi }\int_{\mathopen[ 0 , 2\pi \mathclose[} e^{inx} e^{-inx}dx=\frac{1}{ 2\pi }\int_{0}^{2\pi}1dx=1.
        \end{align}
    \end{subequations}

    Et nous avons également, pour \( m\neq n\) :
    \begin{equation}
        \langle e_n, e_m\rangle =\frac{1}{ 2\pi }\int_{\mathopen[ 0 , 2\pi \mathclose[} e^{i(n-m)x}dx=\frac{1}{ 2\pi }\left[ \frac{1}{ i(n-m) e^{i(n-m)x} } \right]_0^{2\pi}=0.
    \end{equation}
\end{proof}

\begin{remark}
    Vous aurez noté que le facteur \( \frac{1}{ 2\pi }\) qui permet d'avoir \( \langle e_n, e_n\rangle=1 \) ne provient ni de la définition du produit scalaire ni de celle de la famille trigonométrique, mais bien de la mesure, voir la définition \ref{EQooKHZRooSrFMdo}.
\end{remark}

\begin{remark}      \label{REMooUCANooVyXPxj}
    Notez aussi que nous avons bien \( \langle e_n, e_{-n}\rangle =0\). Il faut donc bien prendre tous les \( e_n\) avec \( n\in \eZ\) et non seulement \( n\in \eN\).
\end{remark}

\begin{proposition}     \label{PROPooTGBHooXGhdPR}
    Les polynômes trigonométriques forment une partie dense dans \( \big( C(S^1,\eC),\| . \|_{\infty} \big)\).
\end{proposition}

\begin{proof}
    Pour préciser les notations, \( C(S^1,\eC)\) est l'ensemble des fonctions continues de \( S^1\) vers \( \eC\), et l'espace topologique que nous considérons est cet ensemble sur lequel nous considérons la distance supremum.

    Nous utilisons le théorème de Stone-Weierstrass \ref{ThoWmAzSMF}.

    Le système contient une fonction constante non nulle, à savoir \( e_0\).

    Il sépare les points grâce à la fonction \( e_1\) qui n'est autre que la fonction identité \( z\mapsto z\). De plus l'ensemble des polynômes trigonométriques est stable par conjugaison parce que si
    \begin{equation}
        P=\sum_{k=-n}^na_ke_k,
    \end{equation}
    alors \( \bar P=\sum_{k=-n}^n\overline{ a_k e_ka}=\sum_{k=-n}^n\overline{ a_k }e_{-k}\) qui est encore un polynôme trigonométrique.
\end{proof}

\begin{definition}
    Si nous avons une fonction \( f\colon S^1\to \eC\), nous définissons ses \defe{coefficients de Fourier}{coefficients de Fourier} par
    \begin{equation}
        c_n(f)=\langle f, e_n\rangle 
    \end{equation}
    pourvu que l'intégrale existe.
\end{definition}

%--------------------------------------------------------------------------------------------------------------------------- 
\subsection{Convolution}
%---------------------------------------------------------------------------------------------------------------------------

La convolution sur \( \eR^n\) est donnée par la définition \ref{DEFooHHCMooHzfStu}. Nous voyons maintenant comment cela s'adapte à \( S^1\).
\begin{definition}      \label{DEFooSKWOooEdIHoH}
    Si \( f\) et \( g\) sont des fonctions sur \( S^1\) à valeurs dans \( \eC\), nous définissons la \defe{convolution}{convolution sur \( S^1\)} de \( f\) et \( g\) comme étant la fonction sur \( S^1\) définie par
    \begin{equation}        \label{EQooILQNooBKtSBj}
        (f*g)(z)=\int_{S^1}f(s)g(z\bar s)d\mu(s).
    \end{equation}
\end{definition}

Cette définition appelle plusieurs remarques.
\begin{itemize}
    \item
        Dès que \( z,s\in S^1\), nous avons \( z\bar s\in S^1\), de telle sorte qu'au moins l'intégrande ait un sens.
    \item Nous ne prétendons pas que l'intégrale \eqref{EQooILQNooBKtSBj} converge pour toutes les fonctions \( f\) et \( g\). Cela est une définition «pour tous les couples \( f,g\) pour lesquels l'intégrale fonctionne».
    \item
        Le lemme \ref{LEMooTYSSooItOiYE} nous dira que \( L^1(S^1)\) est stable par convolution : si \( f\) et \( g\) sont dans \( L^1\), alors \( f*g\) y est aussi.
    \item
        Dans la formule \eqref{EQooILQNooBKtSBj}, la variable \( s\) est vraiment une variable muette. Cette formule aurait également pu être écrite
    \begin{equation}        
        (f*g)(z)=\int_{S^1} \big[ s\mapsto f(s)g(z\bar s) \big]d\mu.
    \end{equation}
\end{itemize}

\begin{lemma}[\cite{ooPIYUooRQCQRz}]        \label{LEMooTYSSooItOiYE}
    Si \( f,g\in L^1(S^1)\), alors pour presque tout \( z\in S^1\), la fonction \( s\mapsto f(s)g(z\bar s)\) est dans \( L^1(S^1)\).
\end{lemma}

\begin{proof}
    Nous considérons la fonction
    \begin{equation}
        \begin{aligned}
            \psi\colon S^1\times S^1&\to \eC \\
            (z,s)&\mapsto f(s)g(z\bar s). 
        \end{aligned}
    \end{equation}
    \begin{subproof}
    \item[\( \psi\in L^1(S^1\times S^1)\)]
        Nous utilisons le corolaire \ref{CorTKZKwP}, et pour cela nous calculons les intégrales en chaine\footnote{Dans les expressions suivantes, les symboles «\( ds\)» et «\( dz\)» n'ont pas d'autres valeurs que purement de notation pour indiquer le nom de la variable d'intégration.} :
        \begin{subequations}
            \begin{align}
                \int_{S^1}\left[ \int_{S^1}| f(s)g(z\bar s) |dz \right]]ds&=\int_{S^1}| f(s) |\underbrace{\left[ \int_{S^1}| g(z\bar s) |dz \right]}_{=A<\infty}ds\\
                &=A\int_{S^1}| f(s)ds |\\
                &<\infty.
            \end{align}
        \end{subequations}
        Le fait que \( A<\infty\) provient directement de l'hypothèse \( g\in L^1(S^1)\)\footnote{Avec un changement de variables \( z\mapsto z\bar s\) que je vous conseille d'être capable de justifier.}.
        
        Par le corolaire sus-cité nous avons bien \( \psi\in L^1(S^1\times S^1)\).
    \item[Et par Fubini]
        Le théorème de Fubini \ref{ThoFubinioYLtPI}\ref{ITEMooVFGWooZTePQS} nous renseigne que pour presque tout \( z\in S^1\), l'application
        \begin{equation}
            s\mapsto \psi(z,s)
        \end{equation}
        est dans \( L^1(S^1)\). Et la partie \ref{ThoFubinioYLtPI}\ref{ITEMooCYMKooUdizni} ajoute que l'application
        \begin{equation}        \label{EQooPLLBooJZsZzu}
            z\mapsto \int_{S^1}\psi(s,z)ds
        \end{equation}
        est également \( L^1(S^1)\).
    \item[Conclusion]
        L'application donnée en \eqref{EQooPLLBooJZsZzu} est précisément \( (f*g)\). Donc \( f*g\in L^1(S^1)\).
    \end{subproof}
\end{proof}

\begin{lemma}[\cite{MonCerveau}]
    Si \( f\in L^1(S^1)\) et si \( g\) est continue sur \( S^1\), alors \( f*g\) existe et est continue sur \( S^1\).
\end{lemma}

\begin{proof}
    Vu que \( S^1\) est compact, la continuité de \( g\) implique que \( g\) est bornée et donc dans \( L^1(S^1)\). Le lemme \ref{LEMooTYSSooItOiYE} dit alors que \( f*g\) est bien définie sur \( S^1\).

    Soit \( z_0\in S^1\). Nous montrons que \( f*g\) est continue en \( z_0\); pour cela nous considérons \( \epsilon>0\) et ensuite nous réfléchissons un peu. 
    
    Vu que \( g\) est continue sur \( S^1\) qui est compact, \( g\) y est uniformément continue par le théorème de Heine\ref{PROPooBWUFooYhMlDp}. Il existe donc un \( \delta>0\) tel que pour tout \( z_0\in S^1\), si \( z\in B(z_0,\delta)\), alors \( | g(z_0)-g(z) |<\epsilon\).

    Soit \( s\in S^1\). Si \( z\in B(z_0,\delta)\), alors \( \bar sz\in B(\bar sz_0,\delta)\) par le lemme \ref{LEMooCQCAooAEctbe}\ref{ITEMooCIPYooTyPQLj}. Dans ce cas nous avons aussi
    \begin{equation}
        | g(\bar s z_0)-g(\bar sz) |<\epsilon.
    \end{equation}
    Un peu de calcul maintenant. D'une part
    \begin{equation}
        (f*g)(z_0)-(f*g)(z)=\int_{S^1}f(s)\big( g(z_0\bar s)-g(z\bar s) \big)ds,
    \end{equation}
    et donc
    \begin{subequations}
        \begin{align}
            |(f*g)(z_0)-(f*g)(z)|&\leq\int_{S^1}|f(s)| \underbrace{\big|  g(z_0\bar s)-g(z\bar s) \big|}_{<\epsilon} ds\\
            &\leq \epsilon\int_{S^1}| f |\\
            &=A\epsilon
        \end{align}
    \end{subequations}
    pour une certaine constante \( A\) ne dépendant pas de \( z_0\).

    Nous avons prouvé que pour tout \( \epsilon>0\), il existe un \( \delta\) tel que \( z\in B(z_0,\alpha)\) implique 
    \begin{equation}
            |(f*g)(z_0)-(f*g)(z)|\leq A\epsilon,
    \end{equation}
    ce qui signifie que \( f*g\) est continue en \( z_0\).
\end{proof}

Notez que dans cette démonstration, l'uniforme continuité de \( g\) a été utilisée pour effectuer d'un seul coup la majoration pour tout \( s\) dans l'intégrale.

\begin{proposition}     \label{PROPooCSRNooDyClBY}
    Si \( f\in L^1(S^1)\), nous avons
    \begin{equation}
        f*e_n=c_n(f)e_n.
    \end{equation}
\end{proposition}

\begin{proof}
    Il s'agit d'un bon calcul. En considérant \( z= e^{i\theta}\) nous avons
    \begin{subequations}
        \begin{align}
            (f*e_n)(z)&=\int_{S^1}f(s)e_n(z\bar s)ds\\
            &=\frac{1}{ 2\pi }\int_{\mathopen[ 0 , 2\pi \mathclose[}f( e^{ix})e_n(  e^{i(\theta-x)}) dx\\
            &= e^{in\theta}\frac{1}{ 2\pi }\int_{\mathopen[ 0 , 2\pi \mathclose[}f( e^{ix}) e^{-inx}dx\\
            &=e_n( e^{i\theta})\frac{1}{ 2\pi }\int_{\mathopen[ 0 , 2\pi \mathclose[}f( e^{ix})\overline{ e_n( e^{ix}) }dx\\
                &=e_n(z)\int_{S^1}f(s)\overline{ e_n(s) }ds\\
                &=e_n(z)\langle f, e_n\rangle.
        \end{align}
    \end{subequations}
    Donc \( (f*e_n)(z)=\langle f, e_n\rangle e_n(z)\), c'est-à-dire que
    \begin{equation}
        f*e_n=c_n(f)e_n.
    \end{equation}
\end{proof}

\begin{lemma}       \label{LEMooDGHJooRAnwpy}
    Si \( P\) est un polynôme trigonométrique et si \( f\in L^1(S^1)\), alors \( f*P\) est également un polynôme trigonométrique.
\end{lemma}

\begin{proof}
    Soit \( P=\sum_{k=-n}^na_ke_k\). Par la linéarité du produit de convolution,
    \begin{equation}
        f*P=\sum_{k=-n}^na_kf*e_k=\sum_ka_kc_k(f)e_k
    \end{equation}
    où nous avons également utilisé la proposition \ref{PROPooCSRNooDyClBY}. Nous avons donc un polynôme trigonométrique dont les coefficients sont \( a_kc_k(f)\) au lieu de \( a_k\).
\end{proof}

%--------------------------------------------------------------------------------------------------------------------------- 
\subsection{Approximation de l'unité}
%---------------------------------------------------------------------------------------------------------------------------

\begin{lemma}[\cite{TUEWwUN}]       \label{LEMooUNFBooRCzwIn}
    Soient une fonction continue \( f\colon S^1\to \mathopen[ 0 , \infty \mathclose[\) et \( a\in S^1\) telle que \( f(z)<f(a)\) pour tout \( z\in S^1\setminus\{ a \}\). Alors la suite de fonctions \( f_n\colon S^1\to \eR\) donnée par
    \begin{equation}        \label{EQooTQYPooLZprJj}
        f_n(z)=\left( \int_{S^1}f^n \right)^{-1}f(z)^n
    \end{equation}
    est une approximation de l'unité\footnote{Définition \ref{DEFooEFGNooOREmBb}.} autour de \( a\).
\end{lemma}

\begin{proof}
    En plusieurs points, dont d'abord une série de vérifications pour voir que la formule a un sens.
    \begin{subproof}
    \item[Strictement positive]
    D'abord, vu que \( f\) prend ses valeurs dans \( \mathopen[ 0 , \infty \mathclose[\) et vu que \( f(z)<f(a)\), nous avons \( f(a)>0\) (strict). Peut-être que \( f\) s'annule à certains endroits de \( S^1\), mais pas \( a\).
    \item[\( f^n\) est intégrable sur \( S^1\)]
        La fonction \( f^n\) est dans les hypothèses de la proposition \ref{PROPooKFRSooANzglT} parce que \( S^1\) est compact, \( f^n\) y est continue et la mesure sur \( S^1\) est compatible avec la topologie (voir les hypothèses précises).
    \item[L'intégrale n'est pas nulle]
        Vu que \( f(a)>0\), il existe un ouvert \( A\) contenant \( a\) sur lequel \( f>0\). Nous avons alors
        \begin{equation}
            \int_Kf^n\geq \int_Af^n>0.
        \end{equation}
        Cela pour dire que l'inverse dans \eqref{EQooTQYPooLZprJj} ne pose pas de problèmes.
    \item[Norme]
        Vu que toutes les fonctions tant \( f\) que \( f_n\) sont positives, les valeurs absolues ne jouent aucun rôle et nous avons
        \begin{equation}
            \| f_n \|_1=\int_{S^1}f_nd\mu=\left( \int_{S^1}f^n \right)^{-1}\int_{S^1}| f(z) |^nd\mu=1.
        \end{equation}
        Ce calcul donne d'un seul coup les deux conditions
        \begin{itemize}
            \item \( \sup_k\| f_k \|=1\)
            \item \( \int_{S^1}f_n=1\) pour tout \( n\).
        \end{itemize}
    \end{subproof}
    Nous passons maintenant au vrai travail.
    Soit un voisinage \( V\) de \( a\) dans \( S^1\). Soit une suite croissante \( (t_k)\) qui converge vers \( f(a)\), c'est-à-dire \( 0<t_k<f(a)\). Nous posons
    \begin{equation}
        A_k=\{ x\in S^1\setminus V\tq f(x)\geq t_k \}.
    \end{equation}
    Cet ensemble est contenu dans \( S^1\) et est donc borné (pour la métrique de \( S^1\)).    
    \begin{subproof}
    \item[\( A_k\) est fermé]
        Attention : ici nous démontrons que \( A_k\) est fermé dans \( S^1\), et les complémentaires sont pris dans \( S^1\).

        Nous montrons que le complémentaire est ouvert en prenant \( y\in A^c\) et en montrant que \( y\) admet un voisinage contenu dans \( A^c\) (le fameux théorème \ref{ThoPartieOUvpartouv} que nous ne nous lasserons jamais de citer). Si \( y\in A^c\), il y a deux possibilités (non exclusives) : soit \( y\in V\) soit \( f(y)<t_k\). Si \( y\in V\), alors le voisinage \( V\) lui-même est encore dans \( A^c\). Si par contre \( f(y)<t_k\), alors par continuité, il existe un voisinage de \( y\) sur lequel \( f<t_k\).
    \item[\( A_k\) est compact]
        L'espace \( S^1\) est compact, par exemple grâce au lemme \ref{LEMooVYTRooKTIYdn}. La partie \( A_k\) est fermée dans le compact \( S^1\), donc elle est compacte par le lemme \ref{LemnAeACf}.
    \item[\( A_{k+1}\subset A_k\)]
        Si \( x\in A_{k+1}\), alors \( f(x)\geq t_{k+1}>t_k\). Donc \( f(x)>t_k\) et \( x\in A_k\).
    \item[Intersection vide]
        Si \( x\in\bigcap_{k\in \eN}A_k\), alors \( f(x)\geq t_k\) pour tout \( k\). En passant à la limite et en sachant que \( \lim_{k\to \infty} t_k=f(a)\), nous avons \( f(x)\geq f(a)\). Par hypothèse, cela n'est pas. Donc
        \begin{equation}
            \bigcap_{k\in \eN}A_k=\emptyset.
        \end{equation}
        Nous avons, dans un compact, des fermés emboîtés dont l'intersection est vide. Le corolaire \ref{CORooQABLooMPSUBf} nous dit qu'il existe un indice à partir duquel tous les \( A_k\) sont vides.
    \end{subproof}
    
    Soit \( \delta=t_k\) pour un \( k\) tel que \( A_k\) est vide. Nous avons
    \begin{equation}
        \{ x\in S^1\setminus V\tq f(x)\geq \delta \}=\emptyset,
    \end{equation}
    c'est-à-dire que sur \( S^1\setminus V\), nous avons \( f<\delta\) et donc
    \begin{equation}
        \int_{S^1\setminus V}f(s)^n<\int_{S^1\setminus V}\delta^n=\vol(S^1\setminus V)\delta^n
    \end{equation}
    où \( \vol(S^1\setminus V)=\int_{S^1\setminus V}1=\mu'(S^1\setminus V)\) est une constante réelle strictement positive.

    Nous avons aussi \( \delta<f(a)\) parce que \( \delta\) est un des \( t_k\) (et que cette suite croissante converge vers \( f(a)\) sans l'atteindre par hypothèse). Soit \( \delta_1\) tel que \( \delta<\delta_1<f(a)\)\quext{Dans \cite{TUEWwUN}, il prend \( \delta<\delta_1<1\) et je crois qu'il aurait dû écrire \( \varphi(0)\) au lieu de \( 1\).}.

    Nous posons
    \begin{equation}
        W=\{ x\in S^1\tq f(x)>\delta_1 \}.
    \end{equation}
    Cet ensemble n'est pas vide parce qu'il contient \( a\) et est ouvert parce que \( f\) est continue. Nous avons
    \begin{equation}
        \int_{S^1}f(s)^nds\geq \int_{W}f(s)^nds\geq \delta_1^n\vol(W).
    \end{equation}
    
    Nous avons donc déjà ces deux inégalités :
    \begin{equation}
        \int_{S^1\setminus V}f(s)^n<=\vol(S^1\setminus V)\delta^n
    \end{equation}
    et
    \begin{equation}
        \int_{S^1}f(s)^nds\geq \delta_1^n\vol(W).
    \end{equation}

    En ce qui concerne les fonctions \( f_n\) que nous voulions étudier,
    \begin{equation}
        f_n(z)=\left( \int_{S^1}f(s)^nds \right)^{-1}f(z)^n\leq \big( \vol(W)\delta_1^n \big)^{-1}f(z)^n,
    \end{equation}
    et donc
    \begin{equation}
        \int_{S^1\setminus V}f_n\leq\big( \vol(W)\delta_1^n \big)^{-1}\vol(S^1\setminus V)\sigma^n=\frac{ \vol(S^1\setminus V) }{ \vol(W) }\left( \frac{ \delta }{ \delta_1 } \right)^n.
    \end{equation}
    Étant donné que \( \delta<\delta_1\), nous avons \( (\delta/\delta_1)^n\to 0\). Donc aussi
    \begin{equation}
        \lim_{n\to \infty} \int_{S^1\setminus V}f_n(z)dz=0.
    \end{equation}
\end{proof}

Le théorème suivant est une version pour \( S^1\) du théorème \ref{ThoYQbqEez}. Le produit de convolution dans \( S^1\) est la définition \ref{DEFooSKWOooEdIHoH}.
\begin{theorem}[\cite{TUEWwUN,MonCerveau}]         \label{THOooIAOPooELSNxq}
    Soient \( (\varphi_k)\) une approximation de l'unité sur \( \Omega=S^1\) ainsi qu'une fonction \( g\colon S^1\to \eC\).
    \begin{enumerate}
        \item       \label{ITEMooNUDFooYLFIwR}
            Si \( g\) est mesurable et bornée sur \( \Omega\) et si \( g\) est continue en \( a_0\) alors
            \begin{equation}
                (\varphi_k*g)(a_0)\to g(a_0).
            \end{equation}
    \end{enumerate}
\end{theorem}

\begin{proof}
    Pour chaque \( k\in \eN\) nous posons 
    \begin{equation}
        d_k=\varphi_k*g-g.
    \end{equation}
    Le but de ce théorème est de montrer que \( d_k\to 0\) pour diverses notions de convergence.
    \begin{subproof}
        \item[Preuve du point \ref{ITEMooNUDFooYLFIwR}]
            Soit \( a_0\in S^1\). Par définition de l'approximation de l'unité, \( \int_{S^1}\varphi_k=1\) et donc on peut écrire \( g(a_0)=\int_{S^1}g(a_0)\varphi_k(s)ds\). En ce qui concerne \( d_k(a_0)\) nous avons alors
            \begin{subequations}
                \begin{align}
                    d_k(a_0)&=\int_{S^1}\varphi_k(s)g(a_0\bar s)ds-\int_{S^1}g(a_0)\varphi_k(s)ds\\
                    &=\int_{S^1}\varphi_k(s)\big( g(a_0\bar s)-g(a_0) \big).
                \end{align}
            \end{subequations}
            Nous pouvons passer à la norme (et non la valeur absolue parce que \( d_k\) prend ses valeurs dans \( \eC\)) :
            \begin{equation}
                | d_k(a_0) |\leq \int_{S^1}| \varphi_k(s) |\big| g(a_0\bar s)-g(a_0) \big|ds.
            \end{equation}
            La définition d'une approximation de l'unité nous permet de considérer \( M=\sup_k\| \varphi_k \|_1<\infty\). Le lemme \ref{LEMooTKFHooJaeMyc}\ref{ITEMooXCBUooUxQldB} nous permet, lui, de considérer \( \alpha>0\) tel que 
            \begin{equation}
                | g(a_0\bar s)-g(a_0) |<\epsilon
            \end{equation}
            dès que \( s\in B(1,\alpha)\)\footnote{Notez que \( s\in B(1,\alpha)\) si et seulement si \( \bar s\in B(1,\alpha)\). Il n'y a donc pas d'incohérence entre l'hypothèse sur \( s\) et notre condition sur \( g(a_0\bar s)\)}. Vu que la suite \( (\varphi_k)\) est une approximation de l'unité, nous avons
            \begin{equation}
                \int_{S^1\setminus B(1,\alpha)}| \varphi_k |=0.
            \end{equation}
            Soit \( k\) suffisamment grand pour avoir \( \int_{S^1\setminus B(1,\alpha)}| \varphi_k |<\epsilon\). Avec tout cela nous avons les majorations
            \begin{subequations}
                \begin{align}
                    | d_k(a_0) |&\leq \int_{S^1\setminus B(1,\alpha)}| \varphi_k(s) |\big| g(a_0\bar s)-g(a_0) \big|ds\\
                    &=\int_{S^1\setminus B(1,\alpha)}| \varphi_k(s) |\underbrace{\big| g(a_0\bar s)-g(a_0) \big|}_{\leq 2\| g \|_{\infty}}ds+\int_{B(1,\alpha)}| \varphi_k(s) |\underbrace{\big| g(a_0\bar s)-g(a_0) \big|}_{<\epsilon}ds\\
                    &\leq 2\| g \|_{\infty}\int_{S^1\setminus B(1,\alpha)}| \varphi_k(s) |ds+\epsilon\int_{S^1}| \varphi_k(s) |ds\\
                    &\leq \epsilon\big( 1+2\| g \|_{\infty} \big).
                \end{align}
            \end{subequations}
            Nous avons donc bien \( \lim_{k\to \infty} | d_k(a_0) |=0\) et donc la continuité de \( \varphi_k*g\) en \( a_0\).

    \end{subproof}
\end{proof}

Voici une version un peu forte sous l'hypothèse de continuité. Vu que \( S^1\) est compact, la continuité est en réalité une hypothèse assez forte : ça implique l'uniforme continuité et l'existence d'un maximum et d'un minimum.
\begin{proposition}
    Soient \( (\varphi_k)\) une approximation de l'unité sur \( \Omega=S^1\) ainsi qu'une fonction continue \( g\colon S^1\to \eC\).
    \begin{enumerate}
        \item       \label{ITEMooPHBJooOHDVoW}
        Si \( g\in L^p(\Omega)\) (\( 0\leq p<\infty\)) et si \( g\) est continue, alors\footnote{Vous noterez les \( p\in \mathopen] 0 , 1 \mathclose[\) en bonus par rapport au cas de \( \eR^n\).}
            \begin{equation}
                \varphi_k*g\stackrel{L^p}{\to}g.
            \end{equation}
        \item       \label{ITEMooLOSVooDtaugF}
            Si \( g\) est continue sur \( S^1\), alors
            \begin{equation}
                \varphi_k*g\stackrel{L^{\infty}}{\to}g
            \end{equation}
    \end{enumerate}
\end{proposition}

\begin{proof}
    
    En plusieurs points
    \begin{subproof}

        \item[\( \lim_{u\to 1} \| \tau_u(g)-g \|=0\)]
            Ceci est un petit point intermédiaire. Pour des besoins de notations, nous posons
            \begin{equation}
                \begin{aligned}
                    \tau_u(g)\colon S^1&\to \eC \\
                    s&\mapsto g(s\bar u)
                \end{aligned}
            \end{equation}
            pour \( u\in S^1\). 

            La fonction \( g\) est continue sur le compact \( S^1\), et y est donc uniformément continue\footnote{Théorème de Heine \ref{PROPooBWUFooYhMlDp}. C'est fondamentalement ce fait qui unifie les parties \ref{ITEMooPHBJooOHDVoW} et \ref{ITEMooLOSVooDtaugF} de cette preuve.}. Nous allons en déduire que \( \lim_{u\to 1} \| \tau_u(g)-g \|_{\infty}=0\).

            Soit \( \epsilon>0\). L'uniforme continuité de \( g\) signifie qu'il existe \( \delta>0\) tel que pour tout \( a\in S^1\) si \( s\in B(a,\delta)\), alors \( \big| g(s)-g(a) \big|<\epsilon\). Si \( u\in B(1,\delta)\) nous avons aussi \( \bar u\in B(1,\delta)\) et donc \( s\bar u\in B(s,\delta)\); ça c'est le lemme \ref{LEMooCQCAooAEctbe}\ref{ITEMooCIPYooTyPQLj}.

            Pour tout \( a\in S^1\) nous avons la chaine
            \begin{equation}
                u\in B(1,\delta)\Rightarrow a\bar u\in B(a,\delta)\Rightarrow \big| g(a\bar u)-g(a) \big|<\epsilon.
            \end{equation}
            Cela étant valable pour tout \( a\), c'est encore valable en passant au supremum\footnote{Notez l'inégalité qui n'est plus stricte.} :
            \begin{equation}
                u\in B(1,\delta)\Rightarrow\sup_{a\in S^1}\big| g(a\bar u)-g(a) \big|\leq\epsilon
            \end{equation}
            et donc d'accord pour
            \begin{equation}
                \lim_{u\to 1} \| \tau_u(g)-g \|=0.
            \end{equation}

        \item[\( \| d_k \|_{\infty}\to 0\)]
            Nous prouvons la convergence uniforme sur \( S^1\) de \( d_k\) vers zéro. Ensuite nous verrons que la compacité de \( S^1\) permet d'en déduire les points \ref{ITEMooPHBJooOHDVoW} et \ref{ITEMooLOSVooDtaugF}.
            
            En utilisant la notation \( \tau_u\), nous pouvons écrire
            \begin{equation}
                d_k(s)=\int_{S^1}\varphi_k(u)g(s\bar u)du-g(s)=\int_{S^1}\varphi_k(u)\big( \underbrace{ g(s\bar u)}_{=\tau_u(g)(s)}-g(s) \big)du,
            \end{equation}
            et donc
            \begin{equation}
                | d_k(s) |\leq \int_{S^1}| \varphi_k(u) |\| \tau_u(g)-g \|_{\infty}du.
            \end{equation}
            Nous posons \( M=\sup_k\| \varphi_k \|_1\), et nous considérons \( \delta\) tel que \( \| \tau_u(g)-g \|_{\infty}<\epsilon\) pour tout \( u\in B(1,\delta)\). Ensuite nous subdivisons \( S^1\) en \( B(1,\delta)\) et \( B(1,\delta)^c\) :
            \begin{subequations}
                \begin{align}
                    \| d_k \|_{\infty}&\leq \int_{B(1,\delta)}| \varphi_k(u) |\underbrace{\| \tau_u(g)-g \|_{\infty}}_{\leq \epsilon}du+\int_{B(1,\delta)^c}| \varphi_k(u) |\| \tau_u(g)-g \|_{\infty}du\\
                    &\leq\epsilon M+2\| g \|_{\infty}\int_{B(1,\delta)^c}| \varphi_k(u) |du\\
                    &\leq \epsilon\big( M+2\| g \|_{\infty} \big)
                \end{align}
            \end{subequations}
            parce que pour chaque \( s\in S^1\) nous avons \( \tau_u(g)(s)-g(s)\) et donc \( \| \tau_u(g)-g \|_{\infty}\leq 2\| g \|_{\infty}\).

            Tout cela montre que \( d_k\stackrel{\| . \|_{\infty}}{\longrightarrow}0\).

        \item[Convergence \( L^p\), \( 0<p<\infty\)]

            Soit \( \epsilon>0\). Nous avons
            \begin{equation}
                \| d_k \|_p^p\leq \int_{S^1}\big| (\varphi_k*g)(s)-g(s) \big|^pds.
            \end{equation}
            Il existe un \( k\) à partir duquel \( \| \varphi_k*g-g \|_{\infty}<\epsilon\). Pour de tels \( k\) nous avons 
            \begin{equation}
                \| d_k \|_p^p<\epsilon^p.
            \end{equation}
            Ce passage est très possible dans le cas de \( S^1\) parce que \( \int_{S^1}1=1\). Dans le cas de \( \eR^d\), c'est pas du tout bon; c'est pour cela que nous avons un résultat un peu plus fort dans \( S^1\). La croissance de la fonction puissance (proposition \ref{PROPooRXLNooWkPGsO}) nous permet de conclure que \( \| d_k \|_p<\epsilon\).

            Nous avons donc la convergence \( L^p\) pour \( 0<p<\infty\).
        \item[Convergence \( L^{\infty}\)]
 
            Non, la convergence \( L^{\infty}\) n'est pas la convergence pour la norme \( \| . \|_{\infty}\). Voir la sous-section \ref{SUBSECooYFJTooBqrLXv}. Il n'en reste pas moins que si \( \epsilon>0\) et si \( k\) est assez grand pour que \( \| f_k-f \|_{\infty}<\epsilon\), nous aurons
            \begin{equation}
                N_{\infty}(f_k-f)\leq \| f_k-f \|<\epsilon.
            \end{equation}
    \end{subproof}
\end{proof}

%--------------------------------------------------------------------------------------------------------------------------- 
\subsection{Base hilbertienne (suite des polynômes trigonométriques)}
%---------------------------------------------------------------------------------------------------------------------------

Voici le plan pour la suite :
\begin{itemize}
    \item Construire un polynôme trigonométrique qui vérifie les hypothèse du lemme \ref{LEMooUNFBooRCzwIn}.
    \item En déduire une approximation de l'unité constituée de polynômes trigonométriques.
    \item Dire que si \( f\in L^2(S^1)\), alors \( f*\varphi_k\) est un polynôme trigonométrique dès que \( \varphi_k\) en est un.
    \item Invoquer le théorème \ref{THOooIAOPooELSNxq}\ref{ITEMooPHBJooOHDVoW} pour déduire que \( \varphi_k*f\) est une suite de polynômes trigonométriques dans \( L^2(S^1)\) qui converge \( \varphi_k*f\stackrel{L^2}{\longrightarrow}f\).
\end{itemize}

\begin{lemma}       \label{LEMooQQILooWlhntZ}
    La fonction \( P=e_1+e_{-1}\) est continue à valeurs réelles sur \( S^1\).
\end{lemma}

\begin{proof}
    Nous avons \( e_1(z)=z\) et \( e_{-1}(z)=z^{-1}\), c'est-à-dire que pour \( z= e^{ix}\) (\( x\in \eR\)), nous avons \( e_{-1}( e^{ix})= e^{-ix}\), de telle sorte que, en utilisant le lemme \ref{LEMooHOYZooKQTsXW} qui donne \(  e^{ix}\) en termes des fonctions trigonométriques usuelles :
    \begin{equation}
        (e_1+e_{-1})( e^{ix})= e^{ix}+ e^{-ix}=\cos(x)+i\sin(x)+\cos(x)-i\sin(x)=2\cos(x).
    \end{equation}
    Nous avons donc la continuité et les valeurs réelles.
\end{proof}

\begin{lemma}       \label{LEMooIDTVooYTpfEm}
    Il existe un polynôme trigonométrique à valeurs dans \( \mathopen[ 0 , \infty \mathclose[\) et tel que \( f(a)<f(1)\) pour tout \( a\neq 1\) dans \( S^1\).
\end{lemma}

\begin{proof}
    Le lemme \ref{LEMooQQILooWlhntZ} nous dit déjà que \( P=e_1+e_{-n}\) est continue à valeurs réelles. Or qui est continue sur un compact (ici \( S^1\)), atteint donc ses bornes. Il est donc facile de considérer\footnote{Par exemple, \( M\) est le maximum de \( | P |\).} \( M>0\) tel que \( Q=M+e_1+e_{-1}\) est à valeurs dans \( \mathopen[ 0 , \infty \mathclose[\).

    Une forme explicite de \( Q\) est que
    \begin{equation}
        Q( e^{ix})=M+2\cos(x).
    \end{equation}
    Le maximum de \( \cos(x)\) est obtenu en \( x=0\) et vaut \( 1\). Le maximum de \( Q\) est alors \( Q(1)=2+M\). Il n'est atteint qu'une seule fois sur \( S^1\) parce que pour avoir \( Q( e^{ix})=2+M\), il faut avoir \( 2\cos(x)=2\), c'est-à-dire \( x=2k\pi\). Mais \(  e^{i2k\pi}=1\).

    Donc \( Q(a)<M+2=Q(1)\) pour tout \( a\neq 1\) dans \( S^1\).
\end{proof}

\begin{proposition}
    Les polynômes trigonométriques \( \{ e_n \}_{n\in \eZ}\) forment une base hilbertienne de \( L^2(S^1)\).
\end{proposition}

\begin{proof}
    Le fait que les \( e_n\) soient orthonormée est la proposition \ref{PROPooOMGFooROFFFr}. Il reste à prouver que ce soit un système total. 
    
    Soit \( f\in L^2(S^1)\). Soit un polynôme \( Q\) vérifiant le lemme \ref{LEMooIDTVooYTpfEm}; nous posons
    \begin{equation}
        \varphi_k(z)=\left( \int_{S^1}Q^n \right)^{-1}Q(z)^n.
    \end{equation}
    Cela est une approximation de l'unité par la proposition \ref{LEMooUNFBooRCzwIn}. Les \( \varphi_k\) sont des polynômes trigonométriques parce que les \( Q^n\) le sont et que que \( \int_{S^1}Q^n\) est seulement un nombre.

    Le lemme \ref{LEMooDGHJooRAnwpy} nous dit alors que pour tout \( k\), la fonction
    \begin{equation}
        \varphi_k*f
    \end{equation}
    est un polynôme trigonométrique.
\end{proof}

Nous nous permettons de confirmer la remarque \ref{REMooUCANooVyXPxj} comme quoi il faut bien tous les \( e_n\) avec \( n\in \eZ\), parce que le polynôme trigonométrique \( Q\) est bien construit à partir de \( e_1+e_{-1}\).

%--------------------------------------------------------------------------------------------------------------------------- 
\subsection{Convolution, bis}
%---------------------------------------------------------------------------------------------------------------------------

\begin{lemma}       \label{LEMooLUBQooWLMFrN}
    Nous considérons l'application
    \begin{equation}
        \begin{aligned}
            \varphi\colon \eR&\to S^1 \\
            t&\mapsto e^{it}.
        \end{aligned}
    \end{equation}
    Soient \( t,u\in \eR\) tels que \( \varphi(t)=\varphi(u)\). Alors pour toutes fonctions pour lesquelles les intégrales convergent,
    \begin{equation}
        \int_0^{2\pi}(f\circ \varphi)(\theta)(g\circ\varphi)(t-\theta)\frac{ d\theta }{ 2\pi }=\int_0^{2\pi}(f\circ \varphi)(\theta)(g\circ\varphi)(u-\theta)\frac{ d\theta }{ 2\pi }.
    \end{equation}
\end{lemma}

\begin{proof}
    Si \( \varphi(u)=\varphi(t)\), alors \( u=t+2k\pi\) pour un certain \( k\in \eZ\). Cette condition implique que \( \varphi(t-\theta)=\varphi(u-\theta)\), et donc l'égalité 
    \begin{equation}
        \int_0^{2\pi}(f\circ \varphi)(\theta)(g\circ\varphi)(t-\theta)\frac{ d\theta }{ 2\pi }=\int_0^{2\pi}(f\circ \varphi)(\theta)(g\circ\varphi)(u-\theta)\frac{ d\theta }{ 2\pi }.
    \end{equation}
\end{proof}

\begin{definition}[Convolution sur \( S^1\)]
    Le lemme \ref{LEMooLUBQooWLMFrN} permet de définir
    \begin{equation}
        (f*g)\big( \varphi(t) \big)=\int_0^{2\pi}(f\circ \varphi)(\theta)(g\circ\varphi)(t-\theta)\frac{ d\theta }{ 2\pi }
    \end{equation}
    pour toutes les paires de fonctions \( f,g\in \Fun(S^1,\eC)\) pour lesquelles l'intégrale converge.
\end{definition}

%+++++++++++++++++++++++++++++++++++++++++++++++++++++++++++++++++++++++++++++++++++++++++++++++++++++++++++++++++++++++++++ 
\section{L'espace \( L^2\big( \mathopen[ a , b \mathclose] \big)\)}
%+++++++++++++++++++++++++++++++++++++++++++++++++++++++++++++++++++++++++++++++++++++++++++++++++++++++++++++++++++++++++++

L'espace \( L^2\big( \mathopen[ a , b \mathclose] \big)\) est l'espace générique sur lequel nous allons construire les espaces \( L^2\) sur \( \mathopen[ -T , T \mathclose]\) et \( \mathopen[ 0 , 2\pi \mathclose]\).

Si \( f\) et \( g\) sont dans \( L^2\big( \mathopen[ a , b \mathclose] \big)\), il n'est pas possible de définir \( f*g\) par la formule intégrale usuelle parce que \( f(x_0+t)\) n'existe pas pour tout \( x_0\) et \( t\) dans \( \mathopen[ a , b \mathclose]\). Donc soit nous utilisons un truc pas très net comme étendre les fonctions sur \( \mathopen[ a , b \mathclose]\) en fonctions périodiques sur \( \eR\), soit nous intégrons vraiment seulement sur \( \mathopen[ a , b \mathclose]\).

Nous n'allons suivre aucune de ces deux voies ou plutôt les deux en même temps. Nous allons seulement tout ramener de \( S^1\) que nous venons de travailler. 

\begin{definition}
    Sur \( \mathopen[ a , b \mathclose]\) nous considérons la mesure de Lebesgue et le produit
    \begin{equation}
        \langle f, g\rangle =\frac{1}{ b-a }\int_a^bf(x)\bar g(x)dx.
    \end{equation}
\end{definition}

Si vous voulez une raison inavouable pour justifier ce facteur \( \frac{1}{ b-a }\), remarquez que \( dx\) a pour unité le mètre. En mettant la facteur \( b-a\) (qui a aussi le mètre comme unité), le tout a les unités de \( fg\), comme il se doit pour le produit scalaire.

\begin{proposition}
    L'application
    \begin{equation}
        \begin{aligned}
            \phi\colon L^2\big( \mathopen[ a , b \mathclose] \big)&\to L^2(S^1) \\
            \phi(f)(z)&=f\big( (s^{-1}\circ \varphi^{-1})(z) \big)
        \end{aligned}
    \end{equation}
    où \( s\) est donnée par
    \begin{equation}
        \begin{aligned}
            s\colon \mathopen[ a , b \mathclose]&\to \mathopen[ 0 , 2\pi \mathclose] \\
            x&\mapsto 2\pi\frac{ x-a }{ b-a }
        \end{aligned}
    \end{equation}
    et \( \varphi\) est l'application usuelle
    \begin{equation}
        \begin{aligned}
            \varphi\colon \mathopen[ 0 , 2\pi \mathclose[&\to S^1 \\
                t&\mapsto  e^{it}
        \end{aligned}
    \end{equation}
    est une bijection isométrique.
\end{proposition}

\begin{proof}
    La preuve du fait que \( \phi\) est isométrique suffira pour prouver qu'elle prend bien ses valeurs dans \( L^2(S^1)\).
    \begin{subproof}
        \item[Isométrique]
            C'est un calcul :
            \begin{subequations}
                \begin{align}
                    \| \phi(f) \|^2&=\langle \phi(f), \phi(f)\rangle \\
                    &=\int_{S^1}| \phi(f) |^2\\
                    &=\frac{1}{ 2\pi }\int_{\mathopen[ 0 , 2\pi \mathclose[}| \phi(f)\big( \varphi(u) \big) |^2du\\
                        &=\frac{1}{ 2\pi }\int_0^{2\pi}| f\big( (s^{-1}\circ\varphi^{-1}\circ\varphi)(u) \big) |^2du\\
                        &=\frac{1}{ 2\pi }\int_0^{2\pi}| f\big( s^{-1}(u) \big) |^2du.
                \end{align}
            \end{subequations}
            Il est temps de faire le changement de variables\footnote{Nous le faisons de façon un peu informelle; soyez capable de bien justifier.} \( y=s^{-1}(u)\), c'est-à-dire
            \begin{equation}
                y=\frac{ b-a }{ 2\pi }u+a.
            \end{equation}
            En ce qui concerne la différentielle,
            \begin{equation}
                dy=\frac{ b-a }{ 2\pi }du
            \end{equation}
            et pour les bornes, si \( u=0\) alors \( y=a\) et si \( u=2\pi\), \( y=b\). Donc
            \begin{subequations}
                \begin{align}
                    \| \phi(f) \|^2&=\frac{1}{ 2\pi }\int_a^b| f(y) |^2\frac{ 2\pi }{ b-a }dy\\
                    &=\frac{1}{ b-a }\int_a^b| f |^2\\
                    &=\| f \|^2.
                \end{align}
            \end{subequations}
        \item[Injectif]
            Soit \( f\) telle que \( \phi(f)=0\). Alors pour tout \( z\in S^1\) nous avons
            \begin{equation}
                f\big( (s^{-1}\circ\varphi^{-1})(z) \big)=0.
            \end{equation}
            Vu que \( s^{-1}\circ\varphi^{-1}\colon S^1 \to \mathopen[ a , b \mathclose[\) est une bijection, pour tout \( u\in\mathopen[ a , b \mathclose[\) nous avons \( f(u)=0\). Donc \( f=0\) dans \( L^2\big( \mathopen[ a , b \mathclose] \big)\) parce que du point de vue de \( L^2\), que l'on prenne ou non les bornes, ce n'est pas important.
        \item[Surjectif]
            Si \( g\in L^2(S^1)\), alors en posant
            \begin{equation}
                f(u)=g\big( (\varphi\circ s)(u) \big)
            \end{equation}
            nous avons \( g=\phi(f)\).
    \end{subproof}
\end{proof}

\begin{definition}
    En ce qui concerne le produit de convolution, si \( f\) et \( g\) sont des fonctions sur \( \mathopen[ a , b \mathclose]\) nous définissons
    \begin{equation}
        f*g=\phi^{-1}\big( \phi(f)*\phi(g) \big)
    \end{equation}
    tant que les formules ont un sens.
\end{definition}

%+++++++++++++++++++++++++++++++++++++++++++++++++++++++++++++++++++++++++++++++++++++++++++++++++++++++++++++++++++++++++++ 
\section{Sur \( \mathopen[ 0 , 2\pi \mathclose[\)}
%+++++++++++++++++++++++++++++++++++++++++++++++++++++++++++++++++++++++++++++++++++++++++++++++++++++++++++++++++++++++++++

Le produit de convolution est un peut subtil parce que \( f(t-x)\) n'est pas défini à priori pour tout \( t,x\in \mathopen[ 0 , 2\pi \mathclose[\), vu que \( f\) n'est définie que sur \( \mathopen[ 0 , 2\pi \mathclose[\). Au moins trois solutions s'offrent à nous :
\begin{itemize}
    \item 
       considérer implicitement la fonction prolongée par périodicité.
   \item
       considérer les fonctions sur \( \eR/2\pi\), et définir un peut toutes les opérations modulo \( 2\pi\) (fastidieux)
   \item
       utiliser une bijection ayant les bonnes propriétés avec \( S^1\) sur lequel tout est déjà fait.
\end{itemize}
Nous sélectionnons la troisième voie. Pour cela nous considérons la fonction (attention, elle n'est pas tout à fait la même que celle plus haut)
\begin{equation}
    \begin{aligned}
        \varphi\colon \mathopen[ 0 , 2\pi \mathclose[&\to S^1 \\
            t&\mapsto  e^{it} 
    \end{aligned}
\end{equation}
qui est une bijection par la proposition \ref{PROPooZEFEooEKMOPT}. Pour le produit de convolution,
\begin{equation}
    (f * g)(x)=(f\circ \varphi^{-1})*(g\circ\varphi^{-1})\big( \varphi(x) \big)
\end{equation}
pour toutes les fonctions \( f,g\colon \mathopen[ 0 , 2\pi \mathclose[\to \eC\) pour lesquelles les intégrales en jeu ont un sens.

%+++++++++++++++++++++++++++++++++++++++++++++++++++++++++++++++++++++++++++++++++++++++++++++++++++++++++++++++++++++++++++ 
\section{Sur \( \mathopen[ -T , T \mathclose[\)}
%+++++++++++++++++++++++++++++++++++++++++++++++++++++++++++++++++++++++++++++++++++++++++++++++++++++++++++++++++++++++++++

Pour rappel, les éléments de \( L^2\) sont des classes de fonctions à valeurs dans \( \eC\).

\begin{proposition}     \label{PROPooHNJZooGfRCfU}
    Les fonctions    
    \begin{equation}
        \begin{aligned}
            e_n\colon \mathopen[ -T , T \mathclose]&\to \eC \\
            t&\mapsto \frac{1}{ \sqrt{ 2T } } e^{2\pi int/T}. 
        \end{aligned}
    \end{equation}
    forment une base hilbertienne de \( L^2\big( \mathopen[ -T , T \mathclose[ \big)\).
\end{proposition}

\begin{proof}
    Nous utilisons le théorème de Stone-Weierstrass \ref{ThoWmAzSMF}. Pour cela, nous considérons \( X=\mathopen[ -T , T \mathclose]\) qui est compact et Hausdorff ainsi que \( A\), l'ensemble des polynômes trigonométrique :
    \begin{equation}
        A=\{ \sum_{k=-n}^na_ke_k\tq a_k\in \eC \}_{n\in \eN}.
    \end{equation}
    Nous vérifions que ce \( A\) satisfait aux hypothèses de Stone-Weierstrass \ref{ThoWmAzSMF}.
    \begin{subproof}
        \item[\( A\) est une algèbre]
            Il s'agit seulement de vérifier que
            \begin{equation}
                e_ke_l=\frac{1}{ \sqrt{ 2T } }e_{k+l}.
            \end{equation}
        \item[\( A\) sépare les points]
            Soient \( x,y\in \mathopen[ -T , T \mathclose]\). Si \( e_k(x)=e_k(y)\)
    \end{subproof}
\end{proof}

%--------------------------------------------------------------------------------------------------------------------------- 
\subsection{Le cas dans \( \mathopen[ 0 , 2\pi \mathclose]\)}
%---------------------------------------------------------------------------------------------------------------------------

En pratique, nous n'allons pas souvent travailler avec des fonctions sur intervalle symétrique \( \mathopen[ -T , T \mathclose]\), mais le plus souvent nous serons sur \( \mathopen[ 0 , 2\pi \mathclose]\).

Nous notons ici une conséquence du théorème~\ref{ThoGVmqOro} dans le cas de l'espace \( L^2\). La proposition suivante est une petite partie du corolaire~\ref{CorQETwUdF}, qui sera d'ailleurs démontré de façon indépendante.

\begin{proposition}
    Si nous avons une suite de réels \( (a_k)\) telle que \( \sum_{k=0}^{\infty}| a_k |^2<\infty\) alors la suite
    \begin{equation}
        f_n(x)=\sum_{k=0}^na_k e^{ikx}
    \end{equation}
    converge dans \( L^2\big( \mathopen] 0 , 2\pi \mathclose[ \big)\).
\end{proposition}

\begin{proof}
    Quitte à séparer les parties réelles et imaginaires, nous pouvons faire abstraction du fait que nous parlons d'une série de fonctions à valeurs dans \( \eC\) au lieu de \( \eR\).

    Un simple calcul est :
    \begin{equation}    \label{EqHVdJxZT}
        \| f_n-f_m \|^2\leq\int_0^{2\pi}\sum_{k=n}^m| a_k |^2dx\leq 2\pi\sum_{k=n}^m| a_k |^2.
    \end{equation}
    Par hypothèse le membre de droite est \( | s_m-s_n |\) où \( s_k\) dénote la suite des sommes partielles de la série des \( | a_k |^2\). Cette dernière est de Cauchy (parce que convergente dans \( \eR\)) et donc la limite \( n\to\infty\) (en gardant \( m>n\)) est zéro. Donc la suite des \( f_n\) est de Cauchy dans \( L^2\) et donc converge dans \( L^2\).
\end{proof}

Adaptons tout cela pour l'espace \( L^2\big( \mathopen[ 0 , 2\pi \mathclose] \big)\). Nous posons
\begin{equation}        \label{EQooBFKDooMkCZOt}
    \langle f, g\rangle =\int_0^{2\pi}f(t)\overline{ g(t) }dt
\end{equation}
et
\begin{equation}        \label{EQooKMYOooLZCNap}
    e_n(t)=\frac{1}{ \sqrt{ 2\pi } } e^{int}.
\end{equation}


L'importance du système trigonométrique défini en \ref{DEFooGCZAooFecAHB} est d'être une base de \( L^2\big( \mathopen[ 0 , 2\pi \mathclose] \big)\), comme précisé dans le lemme suivant.
\begin{lemma}       \label{LEMooBJDQooLVPczR}
    Le système trigonométrique \( \{ e_n \}_{n\in \eZ}\) est une base hilbertienne de \( L^2\big( \mathopen[ 0 , 2\pi \mathclose] \big)\).
\end{lemma}

\begin{proof}
    Pour rappel, une base hilbertienne est la définition~\ref{DEFooADQXooFoIhTG}. Nous prouvons d'abord que le système est orthogonal. Nous avons
    \begin{equation}
        \langle e_n, e_m\rangle =\frac{1}{ 2\pi }\int_0^{2\pi} e^{i(n-m)t}dt.
    \end{equation}
    Si \( n=m\), alors cela est égal à \( 1\). Sinon c'est une intégrale simple :
    \begin{equation}
        \langle e_n, e_m\rangle =\left[ \frac{ i }{ n-m } e^{i(n-m)t} \right]_0^{2\pi}=0.
    \end{equation}
    Cela est pour l'orthogonalité.

    Pour que le système soit total, il faut que son espace vectoriel engendré soit dense. Cela est le théorème~\ref{ThoQGPSSJq}.
\end{proof}

Note : le théorème~\ref{ThoDPTwimI} donné aussi la densité, mais sera démontré plus tard, indépendamment. Voir aussi les thèmes~\ref{THEooPUIIooLDPUuq} et~\ref{THEMooNMYKooVVeGTU}.

Pour un élément donné \( f\in L^2\big( \mathopen[ 0 , 2\pi \mathclose] \big)\), nous définissons\nomenclature[Y]{\( S_nf\)}{somme partielle de série de Fourier}
\begin{equation}
    S_nf=\sum_{k=-n}^n\langle f, e_k\rangle e_k
\end{equation}
et nous avons le théorème suivant, qui récompense les efforts consentis à propos de la densité des polynômes trigonométriques dans \( L^2\).

\begin{theorem} \label{ThoYDKZLyv}
    Soit \( f\in L^2\big( \mathopen[ 0 , 2\pi \mathclose] \big)\). Nous avons égalité\footnote{Notons que la somme sur \( \eZ\) dans \eqref{EqXMMRpSN} est commutative; il n'est donc pas besoin d'être plus précis.}
    \begin{equation}    \label{EqXMMRpSN}
        f=\sum_{n\in \eZ}c_n(f)e_n
    \end{equation}
    dans \( L^2\).

    Nous avons aussi la convergence
\begin{equation}    \label{EqRBWKsYP}
    S_nf\stackrel{L^2}{\to} f.
\end{equation}
\end{theorem}

\begin{proof}
    Le système trigonométrique \( \{ e_n \}_{n\in \eZ}\) est total pour l'espace de Hilbert \( L^2\big( \mathopen[ 0 , 2\pi \mathclose] \big)\) (sans périodicité particulière). Donc le point~\ref{ItemQGwoIxi} du théorème~\ref{ThoyAjoqP} nous donne l'égalité demandée.

    La convergence \eqref{EqRBWKsYP} est une reformulation de l'égalité \eqref{EqXMMRpSN}.
\end{proof}

\begin{normaltext}
    Obtenir la convergence \( L^2\) ne demande pas d'hypothèses de périodicité : la convergence \eqref{EqRBWKsYP} est automatique du fait que le système trigonométrique soit total. Ce n'est cependant pas plus qu'une convergence \( L^2\) et elle ne demande pas \( f(0)=f(2\pi)\), même si pour chacun des \( e_k\) nous avons \( e_k(0)=e_k(2\pi)\).

    Si \( f(2\pi)\neq f(0)\), alors il existe tout de même une suite \( (f_n)\) convergente vers \( f\) au sens \( L^2\) telle que \( f_n(0)=f_n(2\pi)\). Cela ne contredit en rien le fait que \( e_k(0)=e_k(2\pi)\) parce que dans \( L^2\), la valeur d'un point seul n'a pas d'importance.

    Si nous voulons une vraie convergence ponctuelle voir uniforme \( (S_nf)(x)\to f(x)\), alors il faut ajouter des hypothèses sur la continuité de \( f\), sa périodicité ou le comportement des coefficients \( c_n\). Voir aussi le thème~\ref{THMooHWEBooTMInve}.
\end{normaltext}

\begin{example}     \label{EXooQDWUooLtuIOm}
    Si \( f\in L^2\big( \mathopen[ 0 , 2\pi \mathclose] \big)\) est (la classe de) une fonction à valeurs réelles, alors on peut la développer avec nettement moins de termes. D'abord nous savons que \( e_{-n}=\overline{ e_n }\), et donc
    \begin{equation}
        \langle f, e_n\rangle =\overline{ \langle f, e_{-n}\rangle  },
    \end{equation}
    ce qui donne
    \begin{equation}
        f=\sum_{n\in\eZ}\langle f, e_n\rangle e_n=\sum_{n\in \eN}\langle f, e_n\rangle e_n +\overline{ \langle f, e_n\rangle e_n }=\sum_{n\in \eN}\Re\big( \langle f, e_n\rangle e_n \big).
    \end{equation}
    Or
    \begin{equation}        \label{EQooMWJNooSjPCpR}
        \Re\big( \langle f, e_n\rangle e_n \big)=\frac{1}{ (2\pi)^{3/2} }\cos(nx)\int_0^{2\pi}f(t)\cos(nt)dt-\frac{1}{ (2\pi)^{3/2} }\sin(nx)\int_0^{2\pi}f(t)\sin(nt)dt.
    \end{equation}

    Considérons la fonction impaire \( \tilde f\in\L^2\big( [-2\pi,2\pi] \big)\) créée à partir de \( f\). Elle se développe de même et nous avons la même formule \eqref{EQooMWJNooSjPCpR} à part quelques coefficients et le fait que les intégrales sont entre \( -2\pi\) et \( 2\pi\). Vu que \( \tilde f\) est impaire, l'intégrale avec \( \cos(nt)\) s'annule et
    \begin{equation}
        \tilde f(x)=\sum_{n\in \eN}c_n\sin(nx)
    \end{equation}
    pour certains coefficients réels \( c_n\). Cette égalité est à considérer dans \( L^2\), c'est-à-dire presque partout et en particulier presque partout sur \( \mathopen[ 0 , 2\pi \mathclose]\).

    Donc les fonctions réelles sur \( \mathopen[ 0 , 2\pi \mathclose]\) peuvent être écrites sous la forme d'une série de seulement des sinus.

    Note : en choisissant \( \tilde f\) paire, nous aurions eu une série de cosinus.
\end{example}

%+++++++++++++++++++++++++++++++++++++++++++++++++++++++++++++++++++++++++++++++++++++++++++++++++++++++++++++++++++++++++++ 
\section{Théorème de la projection normale}
%+++++++++++++++++++++++++++++++++++++++++++++++++++++++++++++++++++++++++++++++++++++++++++++++++++++++++++++++++++++++++++

%--------------------------------------------------------------------------------------------------------------------------- 
\subsection{Espace uniformément convexe}
%---------------------------------------------------------------------------------------------------------------------------

\begin{definition}[Espace uniformément convexe\cite{BIBooPYEZooTxohAd}]     \label{DEFooOPQBooBhufew}
Un espace de Banach \( B\) est \defe{uniformément convexe}{uniformément convexe} si il existe une fonction \( \delta\colon \mathopen] 0 , \infty \mathclose[\to \eR^+\) telle que si
    \begin{enumerate}
        \item
            \( \| x \|\leq \| y \|\leq 1\),
        \item
            \( \| x-y \|\geq \epsilon\),
    \end{enumerate}
    alors
    \begin{equation}
        \| \frac{ x+y }{ 2 } \|\leq \| y \|-\delta(\epsilon).
    \end{equation}
\end{definition}

\begin{lemma}[\cite{MonCerveau}]
    Si \( B\) est un espace de Banach uniformément convexe, alors pour tout \( k>0\), il existe une fonction \( \delta_k\colon \mathopen] 0 , \infty \mathclose[\to \eR^2\) telle que si
    \begin{enumerate}
        \item
            \( \| x \|\leq \| y \|\leq k\),
        \item
            \( \| x-y \|\geq \epsilon\),
    \end{enumerate}
    alors
    \begin{equation}
        \| \frac{ x+y }{ 2 } \|\leq \| y \|-\delta_k(\epsilon).
    \end{equation}
\end{lemma}
 
\begin{proof}
    Nous posons \( x'=x/\| x \|\) et \( y'=y/\| y \|\). Nous avons alors
    \begin{equation}
        \| x'-y' \|=\frac{ \| x-y \| }{ k }>\frac{ \epsilon }{ k }.
    \end{equation}
    L'uniforme convexité de \( B\) dit alors que
    \begin{equation}
        \| \frac{ x'+y' }{2} \|>\| y' \|-\delta(\epsilon/k).
    \end{equation}
    En multipliant cette inégalité par \( k\) nous trouvons
    \begin{equation}
        \| \frac{ x+y }{2} \|>\| y \|-k\delta(\epsilon/k).
    \end{equation}
    Donc en posant \( \delta_k(\epsilon)=k\delta(\epsilon/k)\), nous avons le résultat escompté.
\end{proof}

\begin{definition}[Projection normale\cite{BIBooPYEZooTxohAd}]      \label{DEFooMYYLooJyACPL}
    Soient un espace de Banach \( B\) ainsi que \( V\subset B\). Soit \( a\in B\). La fonction
    \begin{equation}
        \begin{aligned}
            f\colon V&\to \eR \\
            x&\mapsto d(x,a) 
        \end{aligned}
    \end{equation}
    possède un infimum\footnote{Toute fonction à valeurs positives possède un infimum, c'est la proposition \ref{DefSupeA}.} \( m\). Si \( x\in V\) est tel que \( d(x,a)=m\), alors \( x\) est une \defe{projection normale}{projection normale} de \( a\) sur \( V\).
\end{definition}

\begin{proposition}[\cite{BIBooPYEZooTxohAd}]       \label{PROPooDKXVooUoYPgz}
    Soient un espace de Banach \( B\) et un sous-espace vectoriel \( V\subset B\). Si une projection normale de \( a\in B\) sur \( V\) existe, alors elle est unique.
\end{proposition}

\begin{proof}
    Soient deux projections normales \( b,b'\) de \( a\) sur \( V\).

    Si \( m=0\), alors \( \| a-b \|=0\) et \( \| a-b' \|=0\), ce qui donne \( a=b\) et \( a=b'\). Donc d'accord pour \( b=b'\).

    Si \( m>0\) alors nous utilisons l'inégalité \( \| x+y \|\leq \| x \|+\| y \|\) sous la forme
    \begin{equation}        \label{EQooQWJWooVaWMCL}
        \| a-\frac{ b+b' }{ 2 } \|=\| \frac{ a-b }{2}+\frac{ a-b' }{2} \|\leq \| \frac{ a-b }{2} \|+\| \frac{ a-b' }{2} \|=\frac{ m }{ 2 }+\frac{ m }{2}=m.
    \end{equation}
    Mais \( \frac{ b+b' }{2}\in V\), donc
    \begin{equation}
        \| a-\frac{ b+b' }{2} \|\geq m.
    \end{equation}
    Nous en déduisons que dans \eqref{EQooQWJWooVaWMCL}, toutes les inégalités sont des égalités et en particulier
    \begin{equation}
        \| \frac{ b+b' }{2}-a \|=m.
    \end{equation}
    Nous avons donc les deux égalités suivantes :
    \begin{equation}
        2m=\| a-b \|+\| a-b' \|
    \end{equation}
    et
    \begin{equation}
        2m=\| b+b'-2a \|.
    \end{equation}
    Cela donne
    \begin{equation}
        \| a-b \|+\| a-b' \|=\| (a-b)+(a-b') \|.
    \end{equation}
    Vu que \( B\) est strictement convexe, cela n'est possible que si \( a-b=a-b'\), ce qui signifie que \( b=b'\).
\end{proof}

\begin{theorem}[\cite{BIBooVHQSooTrLCzQ,BIBooPYEZooTxohAd}]     \label{THOooOOVVooMhzHqd}
    Si \( B\) est un espace de Banach uniformément convexe, si \( V\subset B\) est un sous-espace vectoriel complet et si \( a\in B\), alors la projection normale de \( a\) sur \( V\) existe et est unique.
\end{theorem}

\begin{proof}
    En deux parties.
    \begin{subproof}
        \item[Unicité]
            Soient deux projections normales \( b\) et \( b'\) de \( a\) sur \( V\). Nous avons \( \| a-b \|=\| a-b' \|=m\). Si \( b\neq b'\), il existe \( \epsilon>0\) tel que
            \begin{equation}
                \| b-b' \|>\epsilon>0.
            \end{equation}
            En posant
            \begin{equation}
                \begin{aligned}[]
                    x&=\frac{ 1 }{2}\frac{ b-a }{ m },&y=\frac{ 1 }{2}\frac{ b'-a }{ m },
                \end{aligned}
            \end{equation}
            nous avons \( \| x \|=\| y \|=\frac{ 1 }{2}<1\). L'uniforme convexité de \( B\) donne alors
            \begin{equation}
                \| \frac{ x+y }{2} \|\leq \| y \|-\delta(\epsilon).
            \end{equation}
            Mais
            \begin{equation}
                x+y=\frac{ \frac{ 1 }{2}(b+b')-a }{ m }
            \end{equation}
            et
            \begin{equation}
                \| x+y \|\leq 2\| y \|-2\delta(\epsilon)=1-2\delta(\epsilon)<1.
            \end{equation}
            Nous avons donc prouvé que
            \begin{equation}
                \| \frac{ 1 }{2}(b+b')-a \|<m,
            \end{equation}
            ce qui est impossible parce que cela dirait que \( \frac{ b+b' }{2}\) est une «meilleure» projection normale que \( b\) et \( b'\).

        \item[Existence]
            Soient \( b_k\) dans \( V\) tels que \( \| a-b_k \|\to m\). Nous supposons (quitte à passer à une sous-suite) que
            \begin{equation}
                \| a-b_{k+1} \|\leq \| a-b_k \|.
            \end{equation}
            \begin{subproof}
                \item[La suite \( (b_k)\) converge]
                    Nous supposons qu'elle ne converge pas. Elle n'est donc pas de Cauchy parce que \( B\) est de Banach\footnote{Définition \ref{DefVKuyYpQ}.} et donc complet (ce n'est pas la complétude de \( V\) qui joue ici, mais bien celle de \( B\)). Il existe \( \epsilon>0\) tel que pour tout \( N\in \eN\) il existe \( p,q>N\) tels que 
                    \begin{equation}
                        \| b_p-b_q \|>\epsilon.
                    \end{equation}

                    Nous effectuons quelque choix.
                    \begin{enumerate}
                        \item
                            nous choisissons \( q>p\) de telle sorte que $\| a-b_p \|\leq\| a-b_q \|$,
                        \item
                            nous choisissons \( N\) assez grand pour avoir
                            \begin{equation}
                                \| a-b_p \|\leq \| a-b_q \|<m.
                            \end{equation}
                    \end{enumerate}

                    
                    Nous avons \( \| (a-b_p)-(a-b_q) \|=\| a_q-a_p \|>\epsilon\), ce qui avec l'uniforme convexité donne
                    \begin{equation}
                        \frac{ \| (a-b_p)+(a-b_q) \| }{2}\leq \| a-b_q \|-\delta(\epsilon).
                    \end{equation}
                    Donc
                    \begin{equation}
                        m\leq \| a-\frac{ b_p-b_q }{ 2 } \|=\| \frac{ (a-b_p)+(a-b_q) }{2} \|\leq \| a-b_q \|-\delta(\epsilon)<m-\delta(\epsilon)<m.
                    \end{equation}
                    Cela signifie que \( m<m\), ce qui est impossible.
                \item[Conclusion]
                    La suite \( (b_k)\) converge dans \( B\). Vu que \( V\) est complet, la limite est dans \( V\). Cette limite, que nous nommons \( b\), vérifie
                    \begin{equation}
                        \| a-b \|\leq \| a-b_k \|
                    \end{equation}
                    pour tout \( k\). Mais comme nous avons \( m\leq \| a-b_k \|\to m\), nous avons \( \| a-b \|=m\), c'est-à-dire que \( b\) est une projection normale de \( a\) sur \( V\).
            \end{subproof}
    \end{subproof}
\end{proof}

%--------------------------------------------------------------------------------------------------------------------------- 
\subsection{Des inégalités}
%---------------------------------------------------------------------------------------------------------------------------

Avant d'entrer dans le vif du sujet, nous nous fendons d'une petite étude de fonction. Soit
\begin{equation}
    \begin{aligned}
        \phi\colon \mathopen[ 0 , 1 \mathclose]&\to \eR \\
        x&\mapsto \frac{ (1+x)^r }{ 1+x^r }. 
    \end{aligned}
\end{equation}
Un peu de calcul montre que
\begin{equation}
    \frac{ \phi'(x) }{ \phi(x) }=\frac{ r(1-x^{r-1}) }{ (1+x^r)(1+x) }.
\end{equation}

\begin{lemma}       \label{LEMooFKKEooDTypUd}
    Soient \( a,b>0\) et \( r>1\). Nous avons les inégalités
    \begin{equation}
        a^r+b^r\leq (a+b)^r\leq 2^{r-1}(a^r+b^r).
    \end{equation}
\end{lemma}

\begin{proof}
    Pour la première inégalité, nous posons \( f(x)=a^r+x^r\) et \( g(x)=(a+x)^r\). Nous avons \( f(0)=g(0)=a^r\), et
    \begin{subequations}
        \begin{align}
            f'(x)&=rx^{r-1}\\
            g'(x)&=r(a+x)^{r-1}.
        \end{align}
    \end{subequations}
    Vu que \( r>1\), la fonction \( t\mapsto t^{r-1}\) est croissante par la proposition \ref{PROPooRXLNooWkPGsO}.

    Nous passons à la seconde inégalité. Le lemme \ref{LEMooSXTXooZOmtKq} nous dit que la fonction \( f\colon x\mapsto x^r \) est convexe. Donc
    \begin{equation}
        f\left( \frac{ a }{2}+\frac{ b }{2} \right)\leq\frac{ 1 }{2}f(a)+\frac{ 1 }{2}f(b).
    \end{equation}
    De là nous déduisons
    \begin{equation}
        \frac{ (a+b)^r }{ 2^r }\leq \frac{ 1 }{2}(a^r+b^r),
    \end{equation}
    c'est-à-dire la seconde inégalité.
\end{proof}

Nous allons démontrer les inégalités de Hanner dans le théorème \ref{THOooZRRYooBTBQKW}. Vu que ce sera un peu longuet, nous faisons un lemme.
\begin{lemma}       \label{LEMooDHRCooQiSpyC}
    Soient \( z_1,z_2\in \eC\). Nous avons
    \begin{equation}        \label{EQooMUXVooSpGSyG}
        | z_1+z_2 |^p+| z_1-z_2 |^p\geq \big( | z_1 |+| z_2 | \big)^p+\big| | z_1 |-| z_2 | \big|^p.
    \end{equation}
\end{lemma}

\begin{proof}
    Soient \( z_1,z_2\in \eC\). Nous posons
    \begin{equation}        \label{EQooJKYZooFzbETG}
        d=| z_1+z_2 |^p+| z_1-z_2 |^p.
    \end{equation}
    Pour \( | z_1 |\) et \( | z_2 |\) fixés, nous nous demandons quel est le minimum possible de \( d\).

    Si \( | z_1 |=0\), alors le minimum est \( 2| z_2 |^p\) et si \( | z_2 |=0\) alors il est \( 2| z_1 |^p\). Pour les autres cas, nous posons $| z_1 |=a>0$ ainsi que \( b\in \eR\) et \( \theta\in \eR\) tels que\footnote{Proposition \ref{PROPooRFMKooURhAQJ}}
    \begin{equation}
        z_2=z_1a^{-1}b e^{i\theta}.
    \end{equation}
    Nous avons déjà que \( z_1+z_2=z_1(1+a^{-1}b e^{i\theta})\) et donc
    \begin{equation}
        | z_1+z_2 |=a| 1+a^{-1}b e^{i\theta} |=| a+b e^{i\theta} |
    \end{equation}
    parce que \( a>0\). De plus,
    \begin{equation}
        | a+b e^{i\theta} |^2= (a+b e^{i\theta})(a+b e^{-i\theta})=a^2+b^2+2ab\cos(\theta)
    \end{equation}
    parce que \(  e^{i\theta}+ e^{-i\theta}=\cos(\theta)\). Nous posons
    \begin{equation}
        d(\theta)=| a+b e^{i\theta} |^p+| a-b e^{i\theta} |^p.
    \end{equation}
    En développant,
    \begin{equation}
        d(\theta)=\big(a^2+b^2+2ab\cos(\theta)\big)^{p/2}+\big(a^2+b^2-2ab\cos(\theta)\big)^{p/2}.
    \end{equation}
    Trouvons le minimum de cette fonction de \( \theta\). D'abord sa dérivée :
    \begin{subequations}
        \begin{align}
            ad'(\theta)&=pab\sin(\theta)\big[ \big( a^2+b^2-2ab\cos(\theta) \big)^{p/2-1}-\big( a^2+b^2+2ab\cos(\theta) \big)^{p/2-1}  \big]\\
            &=pab\sin(\theta)s(\theta).
        \end{align}
    \end{subequations}
    Nous avons \( s(\theta)=0\) pour \( \theta=\pi/2\) et \( \theta=3\pi/2\). Il faut surtout remarquer que \( 1<p<2\), ce qui donne \( \frac{ p }{2}-1<0\). La fonction \( x\mapsto x^{p/2-1}\) est donc décroissante. Cela pour dire que
    \begin{equation}
        s(0)=\left( | a-b |^2 \right)^{p/2-1}-\left( | a+b |^2 \right)^{p/2-1}>0.
    \end{equation}
    De la même façon, \( s(\pi)=-s(0)<0\). Cela permet d'écrire un petit tableau de signe de \( d'\), et de conclure que \( d(\theta)\) a un minimum en \( 0\) et en \( \pi\). Calcul fait, nous avons
    \begin{equation}
        d(0)=d(\pi)=| a+b |^p+| a-b |^p.
    \end{equation}
    En reliant à \eqref{EQooJKYZooFzbETG} nous avons l'inégalité
    \begin{equation}        \label{EQooVHQOooJcheCR}
        | z_1+z_2 |^p+| z_1-z_2 |^p\geq (a+b)^p-| a-b |^p.
    \end{equation}
    Nous rappelons que \( a=| z_1 |\) et que \( z_2=z_1a^{-1}b e^{i\theta}\). Notons au passage que \( | z_2 |=b\), donc que ce que nous dit l'équation \eqref{EQooVHQOooJcheCR} est que
    \begin{equation}    
        | z_1+z_2 |^p+| z_1-z_2 |^p\geq \big( | z_1 |+| z_2 | \big)^p+\big| | z_1 |-| z_2 | \big|^p.
    \end{equation}
\end{proof}

Encore dans la catégorie des lemmes pour les inégalités de Hanner, nous avons celui-ci.
\begin{lemma}[\cite{MonCerveau,ooKGWWooAybolH}]     \label{LEMooTCNEooADpNai}
    La fonction
    \begin{equation}
        \begin{aligned}
        \eta\colon \mathopen] 0 , \infty \mathclose[&\to \eR \\
            a&\mapsto (a^{1/p}+1)^p+| a^{1/p}-1 |^p 
        \end{aligned}
    \end{equation}
    est strictement convexe.
\end{lemma}

\begin{proof}
    La fonction \( \eta\) est une fonction de classe \(  C^{\infty}\) sur \( \mathopen] 0 , \infty \mathclose[\setminus\{ 1 \}\). Quelle est sa régularité en \( a=1\) ? Le fait qu'elle y soit dérivable pas clair à cause de la valeur absolue. En tout cas, la fonction \( x\mapsto| x-1 |\) n'est pas dérivable en \( x=1\), mais peut-être que les exposants aident à lisser. Nous y reviendrons.

    Afin de  suivre les calculs nous introduisons quelques fonctions :
    \begin{subequations}
        \begin{align}
            so(x)&=1+x^{1/p}\\
            di(x)&=1-x^{1/p}\\
            dj(x)&=x^{1/p}-1
        \end{align}
    \end{subequations}
    Pour les dérivées, nous avons
    \begin{subequations}
        \begin{align}
            so'(x)&=\frac{1}{ p }x^{1/p-1}\\
            di'(x)&=-so'(x)\\
            dj'(x)&=so'(x).
        \end{align}
    \end{subequations}
    En divise les cas selon \( a<1\) ou \( a>1\).
    \begin{subproof}
    \item[Pour \( a<1\)]
        Nous avons
        \begin{equation}
            \eta(a)=so(a)^p+di(a)^p,
        \end{equation}
        et la première dérivée donne :
        \begin{equation}        \label{EQooCLXZooXClOwd}
            \eta'(a)=p\,so'(a)\big( so(a)^{p-1}-di(a)^{p-1} \big).
        \end{equation}
        Pour la seconde dérivée nous trouvons d'abord
        \begin{equation}
            \begin{aligned}[]
            \eta''(a)&=\left( \frac{ 1-p }{ p } \right)a^{\frac{ 1 }{ p }-2}\big( so(a)^{p-1}-di(a)^{p-1} \big)\\
            &\quad+\frac{ p-1 }{ p }a^{\frac{ 2 }{ p }-2}\big( so(a)^{p-2}+di(a)^{p-2} \big).
            \end{aligned}
        \end{equation}
        À partir de là, le truc est de substituer les expressions suivantes :
        \begin{subequations}
            \begin{align}
                so(a)^{p-1}&=so(a)^{p-2}so(a)=so(a)^{p-2}+so(a)^{p-2}a^{1/p}\\
                di(a)^{p-1}&=di(a)^{p-2}-x^{1/p}di(a)^{p-2}. 
            \end{align}
        \end{subequations}
        Plein de trucs se simplifient et nous obtenons
        \begin{equation}
            \eta''(a)=\frac{ p-1 }{ p }a^{\frac{1}{ p }-2}\big( di(a)^{p-1}-so(a)^{p-2} \big).
        \end{equation}
    \item[Pour \( a>1\)]
            Les calculs sont essentiellement les mêmes, en partant de
            \begin{equation}
                \eta(a)=so(a)^p+dj(a)^p.
            \end{equation}
            Les résultats sont :
            \begin{equation}    \label{EQooAJLHooGWjPlz}
                \eta'(a)=p\,so'(a)\big( so(a)^{p-1}+dj(a)^{p-1} \big),
            \end{equation}
            et
            \begin{equation}
                \eta''(a)=\frac{ p-1 }{ p }a^{\frac{1}{ p }-2}\big( dj(a)^{p-2}-so(a)^{p-2} \big).
            \end{equation}
    \end{subproof}
    Au final, nous avons pour tout \( a\neq 1\) :
    \begin{equation}
        \eta''(a)=\frac{ p-1 }{ p }a^{\frac{1}{ p }-2}\big( | 1-a^{1/p} |^{p-2}-(1+a^{1/p})^{p-2} \big).
    \end{equation}
    Ce qu'il se passe en \( a=1\) est encore une question ouvert que nous traitons maintenant.
    \begin{subproof}
        \item[Pour \( a=1\)]
            Les limites des expressions \eqref{EQooCLXZooXClOwd} et \eqref{EQooAJLHooGWjPlz} en \( a=1\) sont vite calculées et c'est \( 2^{p-1}\) dans les deux cas. Donc la dérivée admet une prolongation continue en \( a=1\). Nous allons prouver que la fonction \( \eta\) est en réalité dérivable en \( a=1\) et que la dérivée vaut \( 2^{p-1}\).

            Nous nous concentrons sur la partie difficile donnée par \( f(x)=| x^{1/p}-1 |^p\). Elle est donnée par
            \begin{equation}
                f(x)=\begin{cases}
                    di(x)^p    &   \text{si } x<1\\
                    dj(x)^p    &    \text{si } x>1\\
                    0    &    \text{si } x=1.
                \end{cases}
            \end{equation}
            Si \( f'(1)\) existe, alors elle est égale à la limite
            \begin{equation}
                f'(1)=\lim_{\epsilon\to 0}\frac{ f(1)-f(1-\epsilon) }{ \epsilon }.
            \end{equation}
            Les deux limites à calculer sont :
            \begin{equation}
                \lim_{\epsilon\to 0^+}\frac{ \big( (1+\epsilon)^{1/p}-1 \big)^p }{ \epsilon }
            \end{equation}
            et
            \begin{equation}
                \lim_{\epsilon\to 0^-}\frac{ \big( 1-(1+\epsilon)^{1/p} \big)^p }{ \epsilon }.
            \end{equation}
            La première se traite par la règle de l'Hospital\footnote{Proposition \ref{PROPooBZHTooHmyGsy}}, et le résultat est zéro. Pour la seconde, il faut juste transformer
            \begin{equation}
                \lim_{\epsilon\to 0^+}\frac{ \big( (1+\epsilon)^{1/p}-1 \big)^p }{ \epsilon }=\lim_{h\to 0^+} \frac{ \big( 1-(1-h)^{1/p} \big)^p }{ -h },
            \end{equation}
            qui se traite également par la règle de l'Hospital. Le résultat est également zéro.

            Donc \( \eta\) est dérivable en \( a=1\) et la dérivée vaut \(\eta'(1)= 2^{p-1}\).
    \end{subproof}
En récapitulant, nous avons \( \eta''>0\) sur $\mathopen] 0  , \infty \mathclose[\setminus\{ 0 \}$, donc \( \eta'\) est croissante sur cette partie (proposition \ref{PropGFkZMwD}). Vu que \( \eta'\) est continue sur \( \mathopen] 0 , \infty \mathclose[\), elle est même croissante (strictement) sur tout \( \mathopen] 0 , \infty \mathclose[\).

La proposition \ref{PropYKwTDPX} conclu que \( \eta\) est strictement convexe sur \( \mathopen] 0 , \infty \mathclose[\).
\end{proof}

Toujours dans la catégorie des lemmes pour les inégalités de Hanner, nous avons celui-ci.
\begin{lemma}[\cite{ooKGWWooAybolH}]
    Soit \( 1<p<2\). La fonction
    \begin{equation}
        \begin{aligned}
            \xi\colon \eR^+\times \eR^+&\to \eR \\
            (a,b)&\mapsto \big( a^{1/p}+b^{1/p} \big)^p+| a^{1/p}-b^{1/p} |^p
        \end{aligned}
    \end{equation}
    est convexe.
    
    Pour rappel, les conventions de données en \ref{REMooOCXLooKQrDoq} donnent \( \eR^+=\mathopen[ 0 , \infty \mathclose[\).
\end{lemma}

\begin{proof}
    La fonction \( \xi\) vérifie facilement les conditions suivante :
    \begin{itemize}
        \item \( \xi(a,b)=\xi(b,a)\),
        \item \( \xi(0,0)=0\),
        \item \( \xi(ta,tb)=t\xi(a,b)\) pour tout \( t\geq 0\).
    \end{itemize}
    Nous posons
    \begin{equation}
        \begin{aligned}
            \eta\colon \eR^+&\to \eR \\
            a&\mapsto \xi(a,1).
        \end{aligned}
    \end{equation}
    Le lemme \ref{LEMooTCNEooADpNai} dit que \( \eta\) est strictement convexe, et le lemme \ref{LEMooNUDOooVfVPkw} conclu que \( \xi\) est convexe.
\end{proof}

\begin{lemma}[\cite{BIBooGPACooYtOhPP}]     \label{LEMooWIPYooMZqjbn}
    Soit \( 1<p<2\). Nous considérons les fonctions
    \begin{equation}
        \begin{aligned}
            \alpha\colon \mathopen[ 0 , 1 \mathclose]&\to \eR \\
            x&\mapsto (1+x)^{p-1}+(1-x)^{p-1} 
        \end{aligned}
    \end{equation}
    et
    \begin{equation}
        \begin{aligned}
            \beta\colon \mathopen[ 0 , 1 \mathclose]&\to \eR \\
            x&\mapsto x^{1-p}\big( (1+x)^{p-1}-(1-x)^{p-1} \big).
        \end{aligned}
    \end{equation}
    Soient \( A,B\in \eR\). Nous avons
    \begin{equation}
        \alpha(x)| A |^p+\beta(x)| B |^p\leq | A+B |^p+| A-B |^p.
    \end{equation}
\end{lemma}

\begin{proof}
    Plusieurs étapes.
    \begin{subproof}
        \item[\( \beta(x)\leq \alpha(x)\)]
            Nous avons \( \alpha(1)=\beta(1)=2^{p-1}\). Pour les autres valeurs de \( x\), nous allons raisonner avec la dérivée. La valeur de \( \alpha'(x)\) est facile à calculer
            \begin{equation}
                \alpha'(x)=(p-1)(x+1)^{p-2}-(p-1)(1-x)^{p-2}.
            \end{equation}
            Pour \( \beta'(x)\) c'est un peu plus lourd. En substituant \( (1+x)^{p-1}=(1+x)^{p-2}(1+x)\) et \( (1-x)^{p-1}=(1-x)^{p-2}(1-x)\) nous pouvons regrouper les termes en \( (1+x)^{p-2}\) et \( (1-x)^{p-2}\). Après un peu de travail,
            \begin{equation}
                \beta'(x)=\frac{ p-1 }{ x^p }\big( (1-x)^{p-2}-(1+x)^{p-2} \big).
            \end{equation}
            Cela nous permet de calculer \( \alpha'-\beta'\) :
            \begin{equation}
                \alpha'(x)-\beta'(x)=(p-1)\big( 1+\frac{1}{ x^p } \big)\big( (1+x)^{p-2}-(1-x)^{p-2} \big).
            \end{equation}
            Vu que \( 1<p<2\), le nombre \( p-2\) est strictement négatif; afin de travailler avec des exposants positifs, nous écrivons
            \begin{equation}
                \alpha'(x)-\beta'(x)=\underbrace{(p-1)}_{>0}\underbrace{\big( 1+\frac{1}{ x^p } \big)}_{>0}\underbrace{\left( \frac{1}{ (1+x)^{2-p}}-\frac{1}{ (1-x)^{2-p} }  \right)}_{<0}.
            \end{equation}
            Nous avons \( \alpha'(x)-\beta'(x)<0\) pour tout \( x\in \mathopen] 0 , 1 \mathclose]\). Du fait qu'en plus nous ayons \( \alpha(1)=\beta(1)\), nous déduisons que \( \alpha(x)\geq \beta(x)\).
        \item[Une petite étude de fonction]
            Soit \( R\in \mathopen[ 0 , 1 \mathclose]\). Nous considérons la fonction
            \begin{equation}
                \begin{aligned}
                    F\colon \mathopen[ 0 , 1 \mathclose]\eR&\to \eR \\
                    x&\mapsto \alpha(x)+R^p\beta(x). 
                \end{aligned}
            \end{equation}
            Nous montrons maintenant que cette fonction a un maximum global pour \( x=R\). D'abord sa dérivée :
            \begin{equation}
                F'(x)=\underbrace{(p-1)}_{>0}\underbrace{\Big( (1-x)^{p-1}-(1+x)^{p-2} \Big)}_{<0}\Big( 1-\left( \frac{ R }{ x } \right)^p \Big)
            \end{equation}
            Nous avons
            \begin{itemize}
                \item \( F'(x)=0\) pour \( x=R\),
                \item \( F'(x)<0\) pour \( x>R\),
                \item \( F'(x)>0\) pour \( x<R\).
            \end{itemize}
            Donc \( x=R\) est bien un maximum global.
        \item[Pause]
            Nous avons les petits résultats utiles pour commencer à prouver. Petite pause avant de commencer; pas de panique, ça ne va pas être trop violent.
        \item[Pour \( 0<B<A\)]
            Nous devons prouver que
            \begin{equation}        \label{EQooEPKRooBYJDSF}
                \alpha(x)A^p+\beta(x)B^p\leq (A+B)^p+(A-B)^p.
            \end{equation}
            En divisant par \( A^p\) et en posant \( R=B/A\), l'inéquation \eqref{EQooEPKRooBYJDSF} est équivalente à
            \begin{equation}
                \alpha(x)+\beta(x)R^p\leq (1+R)^p+(1-R)^p
            \end{equation}
            où \( R\in \mathopen] 0 , 1 \mathclose[\) parce que nous avons supposé \( 0<B<A\). Nous avons (il y a un petit calcul pour \( F(R)\))
            \begin{equation}
                (1+R)^p+(1-R)^p=F(R)\geq F(x)=\alpha(x)+\beta(x)R^p.
            \end{equation}
            ok.
        \item[Pour \( 0<A<B\)]
            Lorsque \( 0<A<B\) nous avons 
            \begin{subequations}
                \begin{align}
                    \alpha(x)| A |^p+\beta(x)| B |^p&=\alpha(x)A^p+\beta(x)^p\\
                    &\leq \alpha(x)B^p+\beta(x)A^p      \label{SUBEQooSHNUooCoWMFB}\\
                    &\leq (B+A)^p+(B-A)^p       \label{SUBEQooBPYVooPsAjbq}\\
                    &=| A+B |^p+| A-B |^p.
                \end{align}
            \end{subequations}
            Justification :
            \begin{itemize}
                \item Pour \eqref{SUBEQooSHNUooCoWMFB}, c'est parce que \( \alpha(x)>\beta(x)\); alors en mettant le plus grand de \( A\) et \( B\) devant le \( \alpha\) au lieu du \( \beta\), nous majorons.
                \item Pour \eqref{SUBEQooBPYVooPsAjbq}, c'est l'inégalité dans le cas \( 0<B<A\), mais en inversant les noms de \( A\) et \( B\).
            \end{itemize}
        \item[Pour \( 0<A=B\)]
            Toutes les expressions sont continues par rapport à \( B\) (fixons \( x\) et \( A\)). Nous avons prouvé pour \( B<A\) et pour \( B>A\). Par continuité, l'inégalité est encore valide pour \( A=B\).
        \item[Pour \( A<0\), \( B>0\)]
            En posant \( A'=-A\) nous avons \( A'>0\) et nous pouvons écrire
            \begin{equation}
                | A+B |^p+| A-B |^p=| -A'+B |^p+| -A'-B |^p=| B-A' |^p+| B+A' |^p\geq \alpha(x)| A' |^p+\beta(x)| B |^p.
            \end{equation}
            Nous avons utilisé, avec \( A'\) et \( B\) le cas déjà prouvé \( A',B>0\).
        \item[Pour \( A>0\), \( B<0\)]
            Celui-là, je vous le laisse.
        \item[Pour \( A<0\), \( B<0\)]
            Posez \( A'=-A\) et \( B'=-B\) et hop.
    \end{subproof}
\end{proof}


\begin{theorem}[Inégalités de Hanner\cite{ooKGWWooAybolH,BIBooGPACooYtOhPP}]       \label{THOooZRRYooBTBQKW}
    Soit un espace mesuré \(  (\Omega,\tribA,\mu)\). Soit \( 1<p<2\) and \( f,g\in L^p(\Omega,\tribA,\mu)\); nous avons
    \begin{equation}
        \big( \| f \|_p+\| g \|_p \big)^p+\Big| \| f \|_p-\| g \|_p \Big|^p
                \leq \| f+g \|_p^p+\| f-g \|_p^p
     %           \leq 2\| f \|_p^p+2\| g \|_p^p.        
     % Je laisse tomber cette partie parce qu'elle est -je crois- inutile pour le théorème de Weienersmith
    \end{equation}
    Il y a égalité si et seulement si \( f(t) \) et \( g(t)\) sont colinéaires pour presque tout \( t\).
\end{theorem}

\begin{proof}
    Nous supposons que \( \| f \|_p\geq \| g \|_p\) pour fixer les idées. De toutes façons, la symétrie des formules nous fait passer de ce cas à l'autre sans difficulté.

    Soit \( x\in \mathopen[ 0 , 1 \mathclose]\). Nous écrivons l'inégalité du lemme \ref{LEMooWIPYooMZqjbn} pour \( A=| f(\omega) | \) et \( B=| g(\omega) |\) :
    \begin{equation}
        \alpha(x)| f(\omega) |^p+\beta(x)| g(\omega) |^p\leq \big| f(\omega)+g(\omega) \big|^p+\big| f(\omega)-g(\omega) \big|^p.
    \end{equation}
    Nous intégrons cela par rapport à \( \omega\) sur \( \Omega\) :
    \begin{equation}
        \alpha(x)\| f \|_p^p+\beta(x)\| g \|_p^p\leq \| f+g \|^p_p+\| f-g \|_p^p.
    \end{equation}
    Et là vient l'idée qu'on se demande ce qui est passé par l'esprit du mec qui a tout combiné : nous évaluons cela pour \( x=\frac{ \| g \|_p }{ \| f \|_p }\), ce qui est permis parce que nous avons supposé \( \| f \|_p\geq \| g \|_p \). Faites le calcul, collectez les termes identiques, vous obtiendrez
    \begin{equation}
        \big( \| f \|_p+\| g \|_p \big)^p+\big( \| f \|_p-\| g \|_p \big)^p\leq \| f+g \|^p_p+\| f-g \|_p^p.
    \end{equation}
    Et vu que \( \| f \|_p\geq \| g \|_p\), nous pouvons gratuitement faire
    \begin{equation}
        \| f \|_p-\| g \|_p=\big| \| f \|_p-\| g \|_p \big|.
    \end{equation}
    Fini pour Hanner.
\end{proof}

%--------------------------------------------------------------------------------------------------------------------------- 
\subsection{Inégalités de Clarkson}
%---------------------------------------------------------------------------------------------------------------------------

\begin{lemma}[\cite{BIBooUNSIooQCLkzT}]     \label{LEMooWEODooLHeVrP}
    Si \( p\geq 2\) et si \( a,b\in \eC\), alors
    \begin{equation}
        \left| \frac{ a+b }{2} \right|^p+\left| \frac{ a-b }{2} \right|^p\leq \frac{ 1 }{2}\big( | a |^p+| b |^p \big).
    \end{equation}
\end{lemma}

\begin{proof}
    Nous prouvons l'inégalité en prenant montent petit à petit en généralité.
    \begin{subproof}
        \item[Avec \( x>0\)]
            Soit \( x\geq 0\). Nous montrons dans cette partie l'inégalité
            \begin{equation}        \label{EQooDJBNooEyfNtq}
                x^p+1\leq (x+1)^{p/2}.
            \end{equation}
            Pour cela nous considérons la fonction
            \begin{equation}
                \begin{aligned}
                    f\colon \mathopen[ 0 , \infty \mathclose[&\to \eR \\
                        t&\mapsto (t^2+1)^{p/2}-t^p-1. 
                \end{aligned}
            \end{equation}
            Nous avons \( f(0)=0\), mais aussi, en utilisant les règle de dérivation\footnote{Par exemple celle de la proposition \ref{PROPooKIASooGngEDh}.} nous trouvons vite
            \begin{equation}
                f'(t)=p(t^2+1)^{p/2-1}t-pt^{p-1}.
            \end{equation}
            Vu que \( (t^2+1)^{p/2-1}\geq t^{p-2}\), le signe de \( f'(t)\) est toujours strictement positif pour \( t>0\). La proposition \ref{PropGFkZMwD} fait que \( f\) est strictement croissante et que \( f(t)>0\) pour tout \( t>0\).

        \item[Avec \( x,y\geq 0\)]
            Soient \( x,y\geq 0\) dans \( \eR\). Nous prouvons dans cette partie que
            \begin{equation}        \label{EQooGFGMooSiDfKX}
                (x^2+y^2)^{p/2}\geq x^p+y^p.
            \end{equation}
            Il s'agit d'appliquer l'inégalité \eqref{EQooDJBNooEyfNtq} à \( x/y\) :
            \begin{equation}
                \left( \left( \frac{ x }{ y } \right)^2+1 \right)^{p/2}\geq \left( \frac{ x }{ y } \right)^p+1.
            \end{equation}
            En multipliant par \( y^p\) et en simplifiant un peu, nous trouvons le résultat \eqref{EQooGFGMooSiDfKX}.
        \item[Avec \( a,b\in \eC\)]
            Nous appliquons l'inégalité \eqref{EQooGFGMooSiDfKX} à \( x=| \frac{ a+b }{ 2 } |\) et \( y=| \frac{ a-b }{2} |\). Cela donne :
            \begin{subequations}
                \begin{align}
                    \left| \frac{ a+b }{2} \right|^p+\left| \frac{ a-b }{2} \right|^p&\leq \left( \left| \frac{ a+b }{2} \right|^2+\left| \frac{ a-b }{2} \right|^2 \right)^{p/2}\\
                    &=\left( \frac{ 2| a |^2+2| b |^2 }{ 4 } \right)^{p/2}\\
                    &=\frac{ 1 }{2}| a |^p+\frac{ 1 }{2}| b |^p.
                \end{align}
            \end{subequations}
    \end{subproof}
\end{proof}

\begin{lemma}[\cite{BIBooHJQOooJsInho}]       \label{LEMooLTROooVusGte}
    Soient \( a,b\in \eC\) ainsi que \( 1<p<2\). Nous notons \( q\) l'exposant conjugué de \( p\). Nous avons l'inégalité
    \begin{equation}
        | a+b |^q+| a-b |^q\leq 2\big( | a |^p+| b |^p \big)^{q-1}.
    \end{equation}
\end{lemma}

\begin{proof}
    Soient \( a,b\in \eC\); nous nommons \( | a |\) le module du nombre complexe \( a\). Nous avons :
    \begin{subequations}
        \begin{align}
            | a+b |^q+| a-b |^q&=\left\|   \begin{pmatrix}
                | a+b |    \\ 
                | a-b |    
            \end{pmatrix}\right\|_q^q\\
            &\leq \left( 2^{\frac{1}{ q }-\frac{1}{ 2 }}\left\|  \begin{pmatrix}
                | a+b |    \\ 
                | a-b |    
            \end{pmatrix}\right\|_2 \right)^q    \label{SUBEQooYYRIooQTWyZd}   \\
            &=2^{1-q/2}\left\|   \begin{pmatrix}
                | a+b |    \\ 
                | a-b |    
            \end{pmatrix}\right\|_2^q\\
            &= 2^{1-q/2}\big( | a |^2+| b |^2 \big)^{q/2}       \label{SUBEQooRMVZooLlLvmf}\\
            &=2\left\| \begin{pmatrix}
                | a |    \\ 
                | b |    
            \end{pmatrix}\right\|_2^q\\
            &\leq 2\left\| \begin{pmatrix}
                | a |    \\ 
                | b |    
            \end{pmatrix}\right\|_p^q       \label{SUBEQooKZKSooEjwQpm}\\
            &=2\big( | a |^p+| b |^p \big)^{q/p}\\
            &=2\big( | a |^p+| b |^p \big)^{q-1}.       \label{SUBEQooUHJCooJKwGxe}
        \end{align}
    \end{subequations}
    Quelques justifications. 
    \begin{itemize}
        \item Pour \eqref{SUBEQooYYRIooQTWyZd}. C'est Hölder de la proposition \ref{PROPooQZTNooGACMlQ}.
        \item Pour \eqref{SUBEQooRMVZooLlLvmf}. C'est le calcul suivant, basé sur le fait que \( | z |^2=z\bar z\) :
    Un petit calcul pour la norme \( 2\) :
    \begin{subequations}
        \begin{align}
        \left\|  \begin{pmatrix}
            | a+b |    \\ 
            | a-b |    
        \end{pmatrix}\right\|_2^2&=| a+b |+| a-b |^2\\
        &=| a |^2+a\bar b+\bar a b+| b |^2\\
        &\quad +| a |^2-a\bar b-\bar a b+| b |^2\\
        &=2| a |^2+2| b |^2.
        \end{align}
    \item Pour \eqref{SUBEQooKZKSooEjwQpm}. Du fait que \( p<2<q\), la proposition \ref{PROPooQZTNooGACMlQ} donne \( \| x \|_p\geq\| x \|_2\).
    \item Pour \eqref{SUBEQooUHJCooJKwGxe}. Simplement multiplier par \( q\) l'équation \( \frac{1}{ p }+\frac{1}{ q }=1\) fournit \( \frac{ q }{ p }=q-1\).
    \end{subequations}
    \end{itemize}
\end{proof}

\begin{proposition}[Inégalité de Clarkson\cite{BIBooVHQSooTrLCzQ}]      \label{PROPooJDOQooWsGlkr}
    Soient \( f,g\in L^p(\Omega,\tribA,\mu)\).
    \begin{enumerate}
        \item
            Si \( p\geq 2\), alors
            \begin{equation}        \label{EQooBWDJooGXzdxz}
                \| \frac{ f+g }{2} \|_p^p+\| \frac{ f-g }{2} \|_p^p\leq \frac{ 1 }{2}\Big( \| f \|_p^p+\| g \|_p^p \Big).
            \end{equation}
        \item
            Si \( 1<p<2\) et si \( q\) est l'exposant conjugué de \( p\), alors
            \begin{equation}        \label{EQooXMWBooYrvaoV}
                \| f+g \|_p^q+\| f-g \|_p^q\leq 2\Big( \| f \|_p^p +\| g \|_p^p \Big)^{q-1},
            \end{equation}
            ou
            \begin{equation}        \label{EQooZCWDooBnaMom}
                \| \frac{ f+g }{2} \|_p^q+\| \frac{ f-g }{2} \|_p^q\leq 2^{1-q}\big( \| f \|_p^p+\| g \|_p^p \big)^{q-1}.
            \end{equation}
    \end{enumerate}
\end{proposition}

\begin{proof}
    En deux parties.
    \begin{subproof}
        \item[Pour \( p\geq 2\)]
            Soient \( f,g\in L^p(\Omega,\tribA,\mu)\); ce sont des fonctions à valeurs dans \( \eC\). Pour chaque \( \omega\in \Omega\) nous considérons les nombres complexes \( f(\omega)\) et \( g(\omega)\); nous pouvons écrire l'inégalité du lemme \ref{LEMooWEODooLHeVrP} :
            \begin{equation}        \label{EQooIGNKooKFUpKO}
                \left| \frac{ f(\omega)+g(\omega) }{2} \right|^p+\left| \frac{ f(\omega)-g(\omega) }{2} \right|^p\leq \frac{ 1 }{2}\big( | f(\omega) |^p+| g(\omega) |^p \big).
            \end{equation}
            Nous avons les substitutions évidentes \( f(\omega)+g(\omega)=(f+g)(\omega)\) et \( f(\omega)-g(\omega)=(f-g)(\omega)\). En intégrant alors \eqref{EQooIGNKooKFUpKO} sur \( \Omega\) nous trouvons l'inégalité demandée.
        \item[Pour \( 1<p<2\)]
            Il s'agit de faire la même chose, en utilisant l'inégalité de Clarkson du lemme \ref{LEMooLTROooVusGte}.

            Pour obtenir \eqref{EQooZCWDooBnaMom}, il s'agit simplement de multiplier et diviser le member de gauche de \eqref{EQooXMWBooYrvaoV} par \( 2^q\).
    \end{subproof}
\end{proof}

%--------------------------------------------------------------------------------------------------------------------------- 
\subsection{Uniforme convexité des espaces de Lebesgue}
%---------------------------------------------------------------------------------------------------------------------------

\begin{proposition}[\cite{BIBooRISHooBcPPKQ}]     \label{PROPooFNLJooDlyIKV}
    Si \( 1<p<\infty\), l'espace \( L^p(\Omega,\tribA, \mu)\) est uniformément convexe\footnote{Définition \ref{DEFooOPQBooBhufew}.}.
\end{proposition}

\begin{proof}
    En deux parties.

    \begin{subproof}
        \item[\( 1<p\leq 2\)]
            Nous montrons que la fonction \( \delta(\epsilon)=2^{-q}\epsilon^q\) fonctionne.
            
            Soient \( f,g\in L^p\) telles que \( \| f \|_p\leq \| g \|_p\leq 1\) et \( \| f-g \|_p\geq \epsilon\). Nous commençons par écrire l'inégalité de Clarkson \eqref{EQooXMWBooYrvaoV} :
            \begin{equation}        \label{EQooOWVEooGGfCpy}
                \| \frac{ f+g }{2} \|_p^q+\| \frac{ f-g }{2} \|_p^q\leq 2^{1-q}\big( \| f \|_p^p+\| g \|_p^p \big)^{q-1}.
            \end{equation}
            Par hypothèse, \( \| f \|_p\) et \( \| g \|_p\) sont plus petites que \( 1\). Vu que \( p>1\), nous avons
            \begin{equation}
                \| f \|_p^p+\| g \|_p^p\leq 1+1=2.
            \end{equation}
            En remplaçant dans le membre de droite de \eqref{EQooOWVEooGGfCpy} nous avons
            \begin{equation}        
                \| \frac{ f+g }{2} \|_p^q+\| \frac{ f-g }{2} \|_p^q\leq 2^{1-q}2^{q-1}=1,
            \end{equation}
            et donc
            \begin{equation}        \label{EQooKARVooDrOuJI}
                \| \frac{ f+g }{2} \|_p^q\leq 1-\| \frac{ f-g }{2} \|_p^q.
            \end{equation}
            
            Par ailleurs nous avons supposé \( \| f-g \|_p\geq 1\). Donc aussi\quext{Ici j'ai un coefficient un peu différent que celui de \cite{BIBooRISHooBcPPKQ}. Écrivez-moi pour confirmer ou infirmer mes calculs.}
            \begin{equation}        \label{EQooCGDDooWtDokf}
                \| \frac{ f-g }{2} \|_p^q\geq 2^{-q}\epsilon^q.
            \end{equation}
            
            Et par un autre ailleurs,
            \begin{equation}        \label{EQooOFWYooLVrNDc}
                \| \frac{ f+g }{2} \|_p=\frac{ 1 }{2}\| f+g \|_p\leq \frac{ 1 }{2}\big( \| f \|_p+\| g \|_p \big)\leq 1.
            \end{equation}
            Vu que nous avons \( q\geq 2\), cela donne aussi
            \begin{equation}        \label{EQooGMPRooGiLSss}
                \| \frac{ f+g }{2} \|_p\leq \| \frac{ f+g }{2} \|^q.
            \end{equation}
            
            Avec les inégalités \eqref{EQooCGDDooWtDokf} et \ref{EQooGMPRooGiLSss} nous finissons l'inégalité \eqref{EQooKARVooDrOuJI} :
            \begin{equation}
                \| \frac{ f+g }{2} \|_p\leq \| \frac{ f+g }{2} \|_p^q\leq 1-2^{-q}\epsilon^q\leq \| g \|_p-\delta(\epsilon).
            \end{equation}
            Okay, c'est bon.

        \item[\( 2\leq p<\infty\)]
            Il s'agit de faire la même chose en partant de Clarkson \eqref{EQooBWDJooGXzdxz}. Le résultat est que la fonction \( \delta(\epsilon)=(\sigma/2)^p\), ça fonctionne.
    \end{subproof}
\end{proof}

%--------------------------------------------------------------------------------------------------------------------------- 
\subsection{Théorème de la projection normale}
%---------------------------------------------------------------------------------------------------------------------------

\begin{proposition}     \label{PROPooTZMRooCvQtGg}
    Si \( 1<p<\infty\), et si \( V\) est un sous-espace vectoriel fermé de \( L^p(\Omega,\tribA, \mu)\), alors la projection normale\footnote{Définition \ref{DEFooMYYLooJyACPL}.} de \( a\in L^p\) sur \( V\) existe et est unique.
\end{proposition}

\begin{proof}
    La proposition \ref{PROPooFNLJooDlyIKV} nous indique que l'espace \( L^p(\Omega,\tribA, \mu)\) est uniformément convexe. Or le théorème \ref{THOooOOVVooMhzHqd} nous indique que les espaces uniformément convexes vérifient la présente proposition.
\end{proof}

Nous pouvons donner une preuve directe, sans passer par l'uniforme convexité, dans les cas \( p\geq 2\).
\begin{theorem}[Théorème de la projection normale\cite{BIBooRYTOooYjaNkX}] \label{THOooRJFUooQivDKm}
    Nous considérons \( p\geq 2\). Soit un sous-espace vectoriel fermé \( W\subset L^p(\Omega,\tribA,\mu)\) et \( u_0\in L^p\). Nous notons
    \begin{equation}
        d(u_0,W)=\inf_{w\in W}d(u_0,W).
    \end{equation}
    Alors il existe \( w_0\in W\) tel que \( \| u_0-w_0 \|=d(u_0,W)\).
\end{theorem}

\begin{proof}
    Nous allons séparer trois cas : \( p=2\) et \( p>2\).
    \begin{subproof}
        \item[\( p=2\)]
            Pour \( p=2\), nous savons que \( L^2\) est un espace de Hilbert\footnote{Lemme \ref{LemIVWooZyWodb}.}, et nous avons déjà le théorème de la projection \ref{ThoProjOrthuzcYkz}.
        \item[\( p>2\)]
            Pour chaque \( x\in \Omega\) nous avons \( f(x), g(x)\in \eC\) et donc l'identité du parallélogramme\footnote{Proposition \ref{PropEQRooQXazLz} en remarquant que $(z_1,z_2)\mapsto z_1\bar z_2$ est un produit scalaire hermitien sur $\eC$.} :
            \begin{equation}        \label{EQooUBFEooDUjLnb}
                \big| f(x)-g(x) \big|^2+\big| f(x)+g(x) \big|=2| f(x) |+2| g(x) |^2.
            \end{equation}
            Vu que \( p>2\), la fonction \( s\colon x\mapsto  x^{p/2}\) est convexe (lemme \ref{LEMooSXTXooZOmtKq}). Calcul :
            \begin{subequations}
                \begin{align}
                    | f(x)-g(x) |^p+| f(x)+g(x) |^p&=\big( | f(x)-g(x) |^2 \big)^{p/2}+\big( | f(x)+g(x) |^2 \big)^{p/2}\\
                    &=s\big( | \ldots |^2 \big)+s\big( | \ldots |^2 \big)\\
                    &\leq \big( | f(x)-g(x) |^2+| f(x)+g(x) |^2 \big)^{p/2}     \label{SUBEQooRHAEooHkYNLH}\\
                    &=\big( 2| f(x) |^2+2| g(x) |^2 \big)^{p/2}                 \label{SUBEQooQFSLooJkoeqN}\\
                    &=2^{p/2}\big( | f(x) |^2+| g(x) |^2 \big)^{p/2}\\
                    &\leq  2^{p/2}2^{p/2-1}\big( | f(x) |^p+| g(x) |^p \big)     \label{SUBEQooQSUHooXKaWwO}\\
                    &=2^{p-1}\big( | f(x) |^p+| g(x) |^p \big)    
                \end{align}
            \end{subequations} 
            Justifications : 
            \begin{itemize}
                \item Pour \eqref{SUBEQooRHAEooHkYNLH} : la convexité de \( s\).
                \item Pour \eqref{SUBEQooQFSLooJkoeqN} : la relation \eqref{EQooUBFEooDUjLnb}.
                \item Pour \eqref{SUBEQooQSUHooXKaWwO} : la seconde inégalité du lemme \ref{SUBEQooQSUHooXKaWwO}.
            \end{itemize}
            Nous isolons \( | f(x)-g(x) |^p\) :
            \begin{subequations}
                \begin{align}
                    | f(x)-g(x) |^p&\leq 2^{p-1}\big( | f(x) |^p+| g(x) |^p \big)-| f(x)+g(x) |^p\\
                    &=2^p\left( \frac{ | f(x) |^p+| g(x) |^p }{2}-\left| \frac{ | f(x) |+| g(x) | }{2} \right|^p \right)
                \end{align}
            \end{subequations}
            Cette inégalité étant valable pour tout \( x\), nous pouvons intégrer sur \( \Omega\) et découper l'intégrale en petits morceaux :
            \begin{equation}        \label{EQooVNHSooPXjFNC}
                \| f-g \|^p_p\leq 2^p\left( \frac{ \| f \|_p^p+\| g \|_p^p }{2}- \| \frac{ f+g }{2} \|_p^p \right).
            \end{equation}
            Voila une bonne chose de prouvée. Nous pouvons maintenant passer au vif du sujet.

            Soit une suite \( w_j\) dans \( W\) telle que \( \| u_0-w_j \|\to d(u_0,W)\). Trois choses à savoir sur cette suite :
            \begin{enumerate}
                \item
                    Une telle suite existe parce que \( d(u_0,W)\) est défini comme un infimum.
                \item
                    Rien ne garanti qu'elle converge.
                \item
                    Même si elle convergeait, rien ne garantirait que la limite soit encore dans \( W\).
            \end{enumerate}
            Le troisième point est facile à régler : vu que \( W\) est fermé par hypothèse, une suite convergente contenue dans \( W\) a sa limite dans \( W\). Nous allons régler la convergence de \( w_j\) en prouvant qu'elle est de Cauchy.
            
            Remarquons que \( W\) est vectoriel, donc \( (w_j+w_k)/2\) est dans \( W\) pour tout \( j\) et \( k\); donc
            \begin{equation}
                \| \frac{ w_j+w_k }{2}-u_0 \|\geq d(u_0,W).
            \end{equation}
            En tenant compte de cela, nous écrivons l'inégalité \eqref{EQooVNHSooPXjFNC} avec \( f=w_j-u_0\) et \( g=w_k-u_0\) :
            \begin{equation}
                \| f-g \|_p^p=\| w_j-w_k \|_p^p\leq 2^p\left( \frac{ \| w_j-u_0 \|^p+\| w_k-u_0 \|^p }{2}-d(u_0,W) \right).
            \end{equation}
            Soit \( \epsilon>0\) et \( 0<\epsilon_1,\epsilon_2<\epsilon\) tels que \( \epsilon_1+\epsilon_2<\epsilon\). Il existe un \( N\) tel que si \( j,k>N\) alors \( \| w_j-u_0 \|^p\leq d(u_0,W)^p+\epsilon_1\) et \( \| w_k-u_0 \|^p\leq d(u_0,W)^p+\epsilon_2\). Pour de telles valeurs de \( j\) et \( k\), nous avons
            \begin{equation}
                \| w_j-w_k \|_p\leq 2\left( \frac{ \epsilon_1+\epsilon_2 }{2} \right)<2\epsilon^{1/p}.
            \end{equation}
            Donc la suite \( (w_j)\) est de Cauchy.

            L'espace \( L^p\) étant complet par le théorème \ref{ThoUYBDWQX}, nous en déduisons que \( (w_j)\) converge dans \( L^p\). Mais comme \( W\) est fermé, nous avons \( w_j\stackrel{L^p}{\longrightarrow}w\in W\).

            En termes de normes, nous avons
            \begin{equation}
                \| w-u_0 \|=\lim_j\| w_j-u_0 \|=d(W,u_0).
            \end{equation}
    \end{subproof}
\end{proof}

%+++++++++++++++++++++++++++++++++++++++++++++++++++++++++++++++++++++++++++++++++++++++++++++++++++++++++++++++++++++++++++
\section{Dualité et théorème de représentation de Riesz}
%+++++++++++++++++++++++++++++++++++++++++++++++++++++++++++++++++++++++++++++++++++++++++++++++++++++++++++++++++++++++++++

Dans la suite \( E'\) est le dual topologique, c'est-à-dire l'espace des formes linéaires et continues sur \( E\).

Voici déjà un bel énoncé. Pour des espaces mesurés \( (\Omega,\tribA,\mu)\) plus généraux, voir l'arme totale en el théorème \ref{ThoLPQPooPWBXuv}.

\begin{proposition}[\cite{PAXrsMn}, thème~\ref{THEMEooULGFooPscFJC}] \label{PropOAVooYZSodR}
    Soit \( 1<p<2\) et \( q\) tel que \( \frac{1}{ p }+\frac{1}{ q }=1\). L'application
    \begin{equation}
        \begin{aligned}
            \Phi\colon L^q\big( \mathopen[ 0 , 1 \mathclose] \big)&\to  L^p\big( \mathopen[ 0 , 1 \mathclose] \big)'  \\
            \Phi_g(f)&= \int_{\mathopen[ 0 , 1 \mathclose]}f\bar g.
        \end{aligned}
    \end{equation}
    est une isométrie linéaire surjective.
\end{proposition}

\begin{proof}
    Pour la simplicité des notations nous allons noter \( L^2\) pour \( L^2\big( \mathopen[ 0 , 1 \mathclose] \big)\), et pareillement pour \( L^p\).
    \begin{subproof}
        \item[\( \Phi_g\) est un élément de \( (L^p)'\)]

            Si \( f\in L^p\) et \( g\in L^q\) nous devons prouver que \( \Phi_q(f)\) est bien définie. Pour cela nous utilisons l'inégalité de Hölder\footnote{Proposition~\ref{ProptYqspT}.} qui dit que \( fg\in L^1\); par conséquent la fonction \( f\bar g\) est également dans \( L^1\) et nous avons
            \begin{equation}
                | \Phi_g(f) |\leq\int_{\mathopen[ 0 , 1 \mathclose]}| f\bar g |=\| fg \|_1\leq \| f \|_p\| g \|_q.
            \end{equation}
            En ce qui concerne la norme de l'application \( \Phi_g\) nous avons tout de suite
            \begin{equation}
                \| \Phi_g \|=\sup_{\| f\|_p=1}\big| \Phi_g(f) \big|\leq \| g \|_q.
            \end{equation}
            Cela signifie que l'application \( \Phi_g\) est bornée et donc continue par la proposition~\ref{PROPooQZYVooYJVlBd}. Nous avons donc bien \( \Phi_g\in (L^p)'\).

        \item[Isométrie]

            Afin de prouver que \( \| \Phi_g \|=\| g \|_q\) nous allons trouver une fonction \( f\in L^p\) telle que \( \frac{ | \Phi_g(f) | }{ \| f \|_p }=\| g \|_q\).  De cette façon nous aurons prouvé que \( | \Phi_g |\geq \| g \|_q\), ce qui conclurait que \( | \Phi_g |=\| g \|_q\).

            Nous posons \( f=g| g |^{q-2}\), de telle sorte que \( | f |=| g |^{q-1}\) et
            \begin{equation}
                \| f \|_p=\left( \int| g |^{p(q-1)} \right)^{1/p}=\left( \int | g |^q \right)^{1/p}=\| g \|_q^{q/p}
            \end{equation}
            où nous avons utilisé le fait que \( p(q-1)=q\). La fonction \( f\) est donc bien dans \( L^p\). D'autre part,
            \begin{equation}
                \Phi_g(f)=\int f\bar g=\int g| g |^{q-2}\bar g=\int | g |^q=\| g \|_q^q.
            \end{equation}
            Donc
            \begin{equation}
                \frac{ | \Phi_g(f) | }{ \| f \|_p }=\| g \|_q^{q-\frac{ q }{ p }}=\| g \|_q
            \end{equation}
            où nous avons encore utilisé le fait que \( q-\frac{ q }{ p }=\frac{ q(p-1) }{ p }=1\).

        \item[Surjectif]

            Soit \( \ell\in (L^p)'\); c'est une application \( \ell\colon L^p\to \eC\) dont nous pouvons prendre la restriction à \( L^2\) parce que la proposition~\ref{PropIRDooFSWORl} nous indique que \( L^2\subset L^p\). Nous nommons \( \phi\colon L^2\to \eC\) cette restriction.

            \begin{subproof}

                \item[\( \phi\in (L^2)'\)]

                    Nous devons montrer que \( \phi\) est continue pour la norme sur \( L^2\). Pour cela nous montrons que sa norme opérateur (subordonnée à la norme de \( L^2\) et non de \( L^p\)) est finie :
                    \begin{equation}
                        \sup_{f\in L^2}\frac{ | \phi(f) | }{ \| f \|_{2} }\leq \sup_{f\in L^2}\frac{ | \ell(f) | }{ \| f \|_p }<\infty.
                    \end{equation}
                    Nous avons utilisé l'inégalité de norme \( \| f \|_p\leq \| f \|_2\) de la proposition~\ref{PropIRDooFSWORl}\ref{ItemWSTooLcpOvXii}.

                \item[Utilisation du dual de \( L^2\)]

                    Étant donné que \( L^2\) est un espace de Hilbert (lemme~\ref{LemIVWooZyWodb}) et que \( \phi\in (L^2)'\), le théorème~\ref{ThoQgTovL} nous donne un élément \( g\in L^2\) tel que \( \phi(f)=\Phi_g(f)\) pour tout \( f\in L^2\).

                    Nous devons prouver que \( g\in L^q\) et que pour tout \( f\in L^p\) nous avons \( \ell(f)=\Phi_g(f)\).

                \item[\( g\in L^q\)]

                    Nous posons \( f_n=g| g |^{q-2}\mtu_{| g |<n}\). Nous avons d'une part
                    \begin{equation}    \label{EqEBUooOnlRHj}
                        \Phi_g(f_n)=\int_0^1f_n\bar g=\int_{| g |<n}| g |^q.
                    \end{equation}
                    Et d'autre part comme \( f_n\in L^2\) nous avons aussi \( \phi(f_n)=\Phi_g(f_n)\) et donc
                    \begin{subequations}
                        \begin{align}
                            0\leq \Phi(f_n)= \phi(f_n)&\leq \| \ell \|\| f_n \|_p\\
                            &=\| \ell \|\left( \int_{| g |<n}| g |^{(q-1)p} \right)^{1/p}\\
                            &=\| \ell \|\left( \int_{| g |<n}| g |^q \right)^{1/p}.
                        \end{align}
                    \end{subequations}
                    où nous avons à nouveau tenu compte du fait que \( p(q-1)=q\). En combinant avec \eqref{EqEBUooOnlRHj} nous trouvons
                    \begin{equation}
                        \int_{| g |<n}| g |^q\leq \| \ell \|\left( \int_{| g |<n}| g |^q \right)^{1/p},
                    \end{equation}
                    et donc
                    \begin{equation}
                        \left( \int_{| g |<n}| g |^{q} \right)^{1-\frac{1}{ p }}\leq \| \ell \|,
                    \end{equation}
                    c'est-à-dire
                    \begin{equation}
                        \Big( \int_{| g |<n}| g |^q \Big)^{1/q}\leq \| \ell \|.
                    \end{equation}

                    Si ce n'était pas encore fait nous nous fixons un représentant de la classe \( g\) (qui est dans \( L^2\)), et nous nommons également \( g\) ce représentant. Nous posons alors
                    \begin{equation}
                        g_n=| g |^q\mtu_{| g |<n}
                    \end{equation}
                    qui est une suite croissante de fonctions convergeant ponctuellement vers \( | g |^q\). Le théorème de Beppo-Levi~\ref{ThoRRDooFUvEAN} nous permet alors d'écrire
                    \begin{equation}
                        \lim_{n\to \infty} \int_{| q |<n}| g |^q=\int_{0}^1| g |^q.
                    \end{equation}
                    Mais comme pour chaque \( n\) nous avons \( \int_{| g |<n}| q |^q\leq \| \ell \|^q\), nous conservons l'inégalité à la limite et
                    \begin{equation}
                        \int_0^1| g |^q\leq \| \ell \|^q.
                    \end{equation}
                    Cela prouve que \( g\in L^p\).

                \item[\( \ell(f)=\Phi_g(f)\)]

                    Soit \( f\in L^p\). En vertu de la densité de \( L^2\) dans \( L^p\) prouvée dans le corolaire~\ref{CorFZWooYNbtPz} nous pouvons considérer une suite \( (f_n)\) dans \( L^2\) telle que \( f_n\stackrel{L^p}{\longrightarrow}f\). Pour tout \( n\) nous avons
                    \begin{equation}
                        \ell(f_n)=\Phi_g(f_n).
                    \end{equation}
                    Mais \( \ell\) et \( \Phi_g\) étant continues sur \( L^p\) nous pouvons prendre la limite et obtenir
                    \begin{equation}
                        \ell(f)=\Phi_g(f).
                    \end{equation}
            \end{subproof}
        \end{subproof}
\end{proof}

\begin{lemma}[\cite{MathAgreg}] \label{LemHNPEooHtMOGY}
    Soit \( (\Omega,\tribA,\mu)\) un espace mesuré fini. Soit \( g\in L^1(\Omega)\) et \( S\) fermé dans \( \eC\). Si pour tout \( E\in \tribA\) nous avons
    \begin{equation}
        \frac{1}{ \mu(E) }\int_Egd\mu\in S,
    \end{equation}
    alors \( g(x)\in S\) pour presque tout \( x\in \Omega\).
\end{lemma}

\begin{proof}
    Soit \( D=\overline{ B(a,r) }\) un disque fermé dans le complémentaire de \( S\) (ce dernier étant fermé, le complémentaire est ouvert). Posons \( E=g^{-1}(D)\). Prouvons que \( \mu(E)=0\) parce que cela prouverait que \( g(x)\in D\) pour seulement un ensemble de mesure nulle. Mais \( S^c\) pouvant être écrit comme une union dénombrable de disques fermés\footnote{Tout ouvert peut être écrit comme union dénombrable d'éléments d'une base de topologie par la proposition~\ref{PropMMKBjgY} et $\eC$ a une base dénombrable de topologie par la proposition~\ref{PropNBSooraAFr}.}, nous aurions \( g(x)\in S^c\) presque nulle part.

    Vu que \( \frac{1}{ \mu(E) }\int_Ea=a\) nous avons
    \begin{subequations}
        \begin{align}
            \big| \frac{1}{ \mu(E) }gd\mu-a \big|=\big| \frac{1}{ \mu(E) }\int_E(g-a) \big|\leq  \frac{1}{ \mu(E) }\int_E| g-a |\leq\frac{1}{ \mu(E) }\mu(E)r=r.
        \end{align}
    \end{subequations}
    Donc
    \begin{equation}
        \frac{1}{ \mu(E) }\int_Egd\mu\in D,
    \end{equation}
    ce qui est une contradiction avec le fait que \( D\subset S^c\).
\end{proof}

Dans toute la partie d'analyse fonctionnelle, sauf mention du contraire, nous considérons dans \( L^p\) des fonctions à valeurs complexes, et donc les éléments du dual sont des applications linéaires continues à valeurs dans \( \eC\).

\begin{theorem}[Théorème de représentation de Riesz, thème~\ref{THEMEooULGFooPscFJC}, \cite{MathAgreg,TLRRooOjxpTp,LRBWftc,ooRCYWooNAeaTA}]
                \label{ThoLPQPooPWBXuv}
    Soit un espace mesuré \( (\Omega,\tribA,\mu)\). Soit \( q\) tel que \( \frac{1}{ p }+\frac{1}{ q }=1\) avec la convention que \( q=\infty\) si \( p=1\). Alors l'application
    \begin{equation}
        \begin{aligned}
            \Phi\colon L^q&\to (L^p)' \\
            \Phi_g(f)&=\int_{\Omega}f\bar gd\mu
        \end{aligned}
    \end{equation}
    est une bijection isométrique dans les cas suivants :
    \begin{enumerate}
        \item       \label{ITEMooSQQBooWSFBmX}
            si \( 1<p<\infty\) et \( (\Omega,\tribA,\mu)\) est un espace mesuré quelconque,
        \item       \label{ITEMooCQGJooOWzjoV}
            si \( p=1\) et \( (\Omega,\tribA, \mu)\) est \( \sigma\)-fini.
    \end{enumerate}
\end{theorem}
\index{dual!de $L^p$}

\begin{proof}
    Par petits bouts.
    \begin{subproof}
        \item[\( \Phi\) est injective]
        Nous commençons par prouver que \( \Phi\) est injectif. Soient \( g,g'\in L^q\) tels que \( \Phi_g=\Phi_{g'}\). Alors pour tout \( f\in L^p\) nous avons
                \begin{equation}
                    \int_{\Omega}f(g-g')d\mu=0.
                \end{equation}
                Soient des parties \( A_i\) de mesures finies telles que \( \Omega=\bigcup_{i=1}^{\infty}A_i\). Étant donné que \( \mu(A_i)\) est fini, nous avons \( \mtu_{A_i}\in L^p(\Omega)\) et donc
                \begin{equation}
                    \int_{A_i}(g-g')d\mu=\int_{\Omega}\mtu_{A_i}(x)(g-g')(x)d\mu(x)=0.
                \end{equation}
                La proposition~\ref{PropRERZooYcEchc} nous dit alors que \( g-g'=0\) dans \( L^q(A_i)\). Pour chaque \( i\), la partie \( N_i=\{ x\in A_i\tq (g-g')(x)=0 \}\) est de mesure nulle.

                Vu que \( \Omega\) est l'union de tous les \( A_i\), la partie de \( \Omega\) sur laquelle \( g-g'\) est non nulle est l'union des \( N_i\) et donc de mesure nulle parce que une réunion dénombrable de parties de mesure nulle est de mesure nulle. Donc \( g-g'=0\) presque partout dans \( \Omega\), ce qui signifie \( g-g'=0\) dans \( L^q(\Omega)\).

            \item[La suite]

        La partie difficile est de montrer que \( \Phi\) est surjective.

        Soit \( \phi\in L^p(\Omega)'\). Si \( \phi=0\), c'est bien dans l'image de \( \Phi\); nous supposons donc que non. Nous allons commencer par prouver qu'il existe une (classe de) fonction \( g\in L^1(\Omega)\) telle que \( \Phi_g(f)=\phi(f)\) pour tout \(f\in L^{\infty}(\Omega,\mu)\); nous montrerons ensuite que \( g\in L^q\) et que le tout est une isométrie.

        \item[Une mesure complexe]

            Si \( E\in\tribA\) nous notons \( \nu(E)=\phi(\mtu_E)\). Nous prouvons maintenant que \( \nu\) est une mesure complexe\footnote{Définition~\ref{DefGKHLooYjocEt}.} sur \( (\Omega,\tribA)\). La seule condition pas facile est la condition de dénombrable additive. Il est déjà facile de voir que \( A\) et \( B\) sont disjoints, \( \nu(A\cup B)=\nu(A)+\nu(B)\). Soient ensuite des ensembles \( A_n\) deux à deux disjoints et posons \( E_k=\bigcup_{i\leq k}A_i\) pour avoir \( \bigcup_kA_k=\bigcup_kE_k\) avec l'avantage que les \( E_k\) soient emboîtés. Cela donne
            \begin{equation}
                \| \mtu_E-\mtu_{E_k} \|_p=\mu(E\setminus E_k)^{1/p},
            \end{equation}
            mais vu que \( 1\leq p<\infty\), avoir \( x_k\to 0\) implique d'avoir \( x_k^{1/p}\to 0\). Prouvons que \( \mu(E\setminus E_k)\to 0\). En vertu du lemme~\ref{LemPMprYuC} nous avons pour chaque \( k\) :
            \begin{equation}
                \mu(E\setminus E_k)=\mu(E)-\mu(E_k),
            \end{equation}
            et vu que \( E_k\to E\) est une suite croissante, le lemme~\ref{LemAZGByEs}\ref{ItemJWUooRXNPci}, sachant que \( \mu\) est une mesure « normale », donne
            \begin{equation}
                \lim_{n\to \infty} \mu(E_k)=\mu\big( \bigcup_kE_k \big).
            \end{equation}
            Donc effectivement \( \mu(E_k)\to \mu(E)\) et donc oui : \( \mu(E\setminus E_k)\to 0\). Jusqu'à présent nous avons
            \begin{equation}
                \lim_{k\to \infty} \| \mtu_E-\mtu_{E_k} \|_p=0,
            \end{equation}
            c'est-à-dire \( \mtu_{E_k}\stackrel{L^p}{\longrightarrow}\mtu_E\). La continuité de \( \phi\) sur \( L^p\) donne alors
            \begin{equation}
                \lim_{k\to \infty} \nu(E_k)=\lim_{k\to \infty} \phi(\mtu_{E_k})=\phi(\lim_{k\to \infty} \mtu_{E_k})=\phi(\mtu_E)=\nu(E).
            \end{equation}
            Par additivité finie de \( \nu\) nous avons
            \begin{equation}
                \nu(E_k)=\sum_{i\leq k}\nu(A_i)
            \end{equation}
            et en passant à la limite, \( \sum_{i=1}^{\infty}\nu(A_i)=\nu(\bigcup_{i}A_i)\). L'application \( \nu\) est donc une mesure complexe.

        \item[Mesure absolument continue]

            En prime, si \( \mu(E)=0\) alors \( \nu(E)=0\) parce que
            \begin{equation}
                \mu(E)=0\Rightarrow \| \mtu_E \|_p=0\Rightarrow \mtu_E=0\text{ (dans } L^p\text{)}\Rightarrow\phi(\mtu_E)=0
            \end{equation}

        \item[Utilisation de Radon-Nikodym]

            Nous sommes donc dans un cas où \( \nu\ll\mu\) et nous utilisons le théorème de Radon-Nikodym~\ref{ThoZZMGooKhRYaO} sous la forme de la remarque~\ref{RemSYRMooZPBhbQ} : il existe une fonction intégrable \( g\colon \Omega\to \eC\)\footnote{On peut écrire, pour utiliser de la notation compacte que $ g\in L^1(\Omega,\eC)$.} telle que pour tout \( A\in\tribA\),
            \begin{equation}
                \nu(A)=\int_A\bar gd\mu.
            \end{equation}
            C'est-à-dire que
            \begin{equation}
                \phi(\mtu_A)=\int_A\bar gd\mu=\int_{\Omega}\bar g\mtu_Ad\mu.
            \end{equation}
            Nous avons donc exprimé \( \phi\) comme une intégrale pour les fonctions caractéristiques d'ensembles.

        \item[Pour les fonctions étagées]

            Par linéarité si \( f\) est mesurable et étagée nous avons aussi
            \begin{equation}
                \phi(f)=\int f\bar gd\mu=\Phi_g(f).
            \end{equation}

        \item[Pour \( f\in L^{\infty}(\Omega)\)]

            Une fonction \( f\in L^{\infty}\) est une fonction presque partout bornée. Nous supposons que \( f\) est presque partout bornée par \( M\). Par ailleurs cette \( f\) est limite uniforme de fonctions étagées : \( \| f_k-f \|_{\infty}\to 0\) en posant \( f_k=f\mtu_{| f |\leq k}\). Pour chaque \( k \) nous avons l'égalité
            \begin{equation}    \label{EqPDCJooGNjuAO}
                \Phi_g(f_k)=\phi(f_k).
            \end{equation}
            Par ailleurs la fonction \( f_k\bar g\) est majorée par la fonction intégrable \( M\bar g\) et le théorème de la convergence dominée~\ref{ThoConvDomLebVdhsTf} nous donne
            \begin{equation}
                \lim_{k\to \infty} \Phi_g(f_i)=\lim_{k\to \infty} \int f_k\bar g=\int f\bar g=\Phi_g(f).
            \end{equation}
            Et la continuité de \( \phi\) sur \( L^p\) couplée à la convergence \( f_k\stackrel{L^p}{\longrightarrow}f\) donne \( \lim_{k\to \infty} \phi(f_k)=(f)\). Bref prendre la limite dans \eqref{EqPDCJooGNjuAO} donne
            \begin{equation}
                \Phi_g(f)=\phi(f)
            \end{equation}
            pour tout \( f\in L^{\infty}(\Omega)\).

        \item[La suite \ldots]

            Voici les prochaines étapes.
            \begin{itemize}
                \item Nous avons \( \int f\bar g=\phi(f)\) tant que \( f\in L^{\infty}\). Nous allons étendre cette formule à \( f\in L^p\) par densité. Cela terminera de prouver que notre application est une bijection.
                \item Ensuite nous allons prouver que \( \| \phi \|=\| \Phi_g \|\), c'est-à-dire que la bijection est une isométrie.
            \end{itemize}

        \item[De \( L^{\infty}\) à \( L^p\)]

            Soit \( f\in L^p\). Si nous avions une suite \( (f_n) \) dans \( L^{\infty}\) telle que \( f_n\stackrel{L^p}{\longrightarrow}f\) alors \( \lim \phi(f_n)=\phi(f)\) par continuité de \( \phi\). La difficulté est de trouver une telle suite de façon à pouvoir permuter l'intégrale et la limite :
            \begin{equation}    \label{EqLYYAooUQnbfV}
                \lim_{n\to \infty} \int_{\Omega}f_n\bar g=\int_{\Omega}\lim_{n\to \infty} f_n\bar g=\int_{\Omega}f\bar g=\Phi_g(f).
            \end{equation}
            Nous allons donc maintenant nous atteler à la tâche de trouver \( f_n\in L^{\infty}\) avec \( f_n\stackrel{L^p}{\longrightarrow}f\) et telle que \eqref{EqLYYAooUQnbfV} soit valide.

            Nous allons d'abord supposer que \( f\in L^p\) est positive à valeurs réelles. Nous avons alors par le théorème \ref{THOooXHIVooKUddLi} qu'il existe une suite croissante de fonction étagées (et donc \( L^{\infty}\)) telles que \( f_n\to f\) ponctuellement. De plus étant donné que \( | f_n |\leq | f |\), la proposition~\ref{PropBVHXycL} nous dit que \( f_n\stackrel{L^p}{\longrightarrow}f\). Pour chaque \( n\) nous avons
            \begin{equation}
                \int_{\Omega}f_n\bar g=\phi(f_n).
            \end{equation}
            Soit \( g^+\) la partie réelle positive de \( \bar g\). Alors nous avons la limite croissante ponctuelle \( f_ng^+\to fg^+\) et le théorème de la convergence monotone~\ref{ThoRRDooFUvEAN} nous permet d'écrire
            \begin{equation}
                \lim_{n\to \infty} \int f_ng^+=\int fg^+.
            \end{equation}
            Faisant cela pour les trois autres parties de \( \bar g\) nous avons prouvé que si \( f\in L^p\) est réelle et positive,
            \begin{equation}
                \int f\bar g=\phi(f),
            \end{equation}
            c'est-à-dire que \( \Phi_g(f)=\phi(f)\).

            Refaisant le tout pour les trois autres parties de \( f\) nous montrons que
            \begin{equation}
                \Phi_g(f)=\phi(f)
            \end{equation}
            pour tout \( f\in L^p(\Omega)\). Nous avons donc égalité de \( \phi\) et \( \Phi_g\) dans \(  (L^p)' \) et donc bijection entre \( (L^p)'\) et \( L^q\).

        \item[Isométrie : mise en place]

            Nous devons prouver que cette bijection est isométrique. Soit \( \phi\in (L^p)'\) et \( g\in L^q\) telle que \( \Phi_g=\phi\). Il faut prouver que
            \begin{equation}
                \| g \|_q=\| \phi \|_{(L^p)'}.
            \end{equation}

        \item[ \( \| \phi \|\leq \| g \|_q\) ]

            Nous savons que \( \phi(f)=\int f\bar g\), et nous allons écrire la définition de la norme dans \( (L^p)'\) :
            \begin{subequations}
                \begin{align}
                    \| \phi \|_{(L^p)'}&=\sup_{\| f \|_p=1}\big| \phi(f) \big|\\
                    &=\sup| \int f\bar g |\\
                    &\leq\sup\underbrace{\int| f\bar g |}_{=\| f\bar g \|_1}.
                \end{align}
            \end{subequations}
            Il s'agit maintenant d'utiliser l'inégalité de Hölder~\ref{ProptYqspT} :
            \begin{equation}
                \| \phi \|\leq \sup_{\| f \|_p=1}\| f \|_p\| \bar g \|_q=\| g \|_q.
            \end{equation}

            L'inégalité dans l'autre sens sera démontrée en séparant les cas \( p=1\) et \( 1<p<\infty\).

        \item[Si \( p=1\), une formule]
            Si \( E\) est un ensemble mesurable de mesure finie, alors
            \begin{equation}
                | \int_Egd\mu |=\big| \phi(\mtu_E) \big|.
            \end{equation}
            Mais le fait que \( \mu(E)<\infty\) donne que \( \mtu_E\in L^1(\Omega)\). Donc \( \mtu_E\in L^{\infty}\cap L^1\); nous pouvons alors écrire \( \phi(\mtu_E)=\int_{\Omega}\mtu_E\bar gd\mu\) et donc
            \begin{equation}    \label{EqUPCTooJvoKKI}
                | \int_{\Omega}\mtu_E\bar gd\mu |=|\int_Egd\mu |=\big| \phi(\mtu_E) \big|\leq \| \phi \|_{(L^1)'}\| \mtu_E \|_1=\| \phi \|\mu(E).
            \end{equation}
            Nous écrivons cela dans l'autre sens :
            \begin{equation}
                \| \phi \|\geq \frac{1}{ \mu(E) }| \int_{\Omega}\mtu_E\bar gd\mu |=| \frac{1}{ \mu(E) }\int_E\bar gd\mu |.
            \end{equation}
            Si nous prenons \( S=\{ t\in \eC\tq | t |\leq \| \phi \| \}\), c'est un fermé vérifiant que
            \begin{equation}        \label{EQooMRLGooYPEjUo}
                \frac{1}{ \mu(E) }\int_E\bar gd\mu\in S.
            \end{equation}

            Voila une petite formule qui va nous aider à utiliser le lemme \ref{LemHNPEooHtMOGY}. Nous ne pouvons cependant pas l'utiliser immédiatement parce que l'appartenance \eqref{EQooMRLGooYPEjUo} n'est vraie que pour les parties de mesure finie.

        \item[Si \( p=1\), conclusion\cite{MonCerveau}]

            Pour utiliser le lemme~\ref{LemHNPEooHtMOGY}, nous utilisons l'hypothèse que \( \Omega\) est \( \sigma\)-fini. Soient des mesurables \( A_i\) de mesure fine tels que \( \bigcup_{i\in \eN}A_i=\Omega\).

            Pour chaque \( i\) nous considérons la restriction \( g_i\colon A_i\to \eC\) de \( g\) à \( A_i\). Par le point précédent, elle vérifie
            \begin{equation}
                \frac{1}{ \mu(A_i) }\int_{A_i}\bar g_id\mu=\frac{1}{ \mu(A_i) }\int_{A_i}\bar gd\mu\in S.
            \end{equation}
            En appliquant le lemme \ref{LemHNPEooHtMOGY} à l'espace restreint \( (A_i,\tribA_i,\mu_i)\), nous concluons \( \bar g_i\in S\) presque partout, ce qui signifie que \( \| g_i \|_{\infty}\in S\). Nous en concluons que
            \begin{equation}
                \| g_i \|_{\infty}\leq \| \phi \|
            \end{equation}
            où, dans ce contexte, \( \| g_i \|_{\infty}\) signifie \( \sup_{x\in A_i}| g_i(x) |\).
            
            Nous avons alors
            \begin{equation}
                \| g \|_{\infty}=\sup_{x\in \Omega}| g(x) |=\sup_{i\in \eN}\| g_i \|_{\infty}\leq \| \phi \|.
            \end{equation}
            Une petite justification pour cela ? Prenons une suite \( x_k\) telle que \( | g(x_k) |\to \| g \|_{\infty}\). Vu que les \( A_i\) recouvrent \( \Omega\), existe un naturel \( i(k)\) tel que \( x_k\in A_{i(k)}\). Nous avons alors
            \begin{equation}
                | g(x_k) |\leq \| g_{i(k)} \|_{\infty}\leq \| \phi \|.
            \end{equation}
            Cela pour conclure que \( g\in L^{\infty}\).

            Notons que cet argument ne tient pas avec \( p> 1\) parce que l'équation \eqref{EqUPCTooJvoKKI} terminerait sur \( \| \phi \|\mu(E)^{1/p}\). Du coup l'ensemble \( S\) à prendre serait \( S=\{ t\in \eC\tq | t |\leq \| \phi \|\mu(E)^{1/p-1} \}\) et nous sommes en dehors des hypothèses du lemme parce qu'il n'y a pas d'ensemble \emph{indépendant} de \( E\) dans lequel l'intégrale \( \frac{1}{ \mu(E) }\int_{E}\bar gd\mu\) prend ses valeurs.

        \item[\( 1<p<\infty\)]

            La fonction
            \begin{equation}
                \alpha(x)=\begin{cases}
                    \frac{ g(x) }{ | g(x) | }    &   \text{si } g(x)\neq 0\\
                    1    &    \text{si } g(x)=0
                \end{cases}
            \end{equation}
            a la propriété de faire \( \alpha g=| g |\) en même temps que \( | \alpha(x) |=1\) pour tout \( x\). Nous définissons
            \begin{equation}
                E_n=\{ x\tq | g(x) |\leq n \}
            \end{equation}
            et
            \begin{equation}
                f_n=\mtu_{E_n}| g^{q-1} |\alpha.
            \end{equation}
            Ce qui est bien avec ces fonctions c'est que\footnote{C'est ici que nous utilisons le lien entre $p$ et $q$. En l'occurrence, de $1/p+1/q=1$ nous déduisons $q(p-1)=p$.}
            \begin{equation}
                | f_n |^p=| g^{p(q-1)} | \alpha |^p=| g |^q
            \end{equation}
            sur \( E_n\). Dans \( E_n\) nous avons \( | f_n |=| g^{q-1} |\leq n^{q-1}\) et dans \( E_n\) nous avons \( f_n=0\). Au final, \( f_n\in L^{\infty}\). Par ce que nous avons vu plus haut, nous avons alors
            \begin{equation}
                \phi(f_n)=\Phi_g(f_n).
            \end{equation}
            Par ailleurs,
            \begin{equation}
                f_n\bar g=\mtu_{E_n}| g^{q-1} |\frac{ g }{ | g | }\bar g,
            \end{equation}
            donc\quext{Dans \cite{MathAgreg}, cette équation arrive sans modules, ce qui me laisse entendre que \( \phi(f_n)\) est réel et positif pour pouvoir écrire que \( \phi(f_n)\leq \| \phi \|\| f_n \|_p\), mais je ne comprends pas pourquoi.}
            \begin{subequations}
                \begin{align}
                    \left|\int_{E_n}| g |^qd\mu\right|&=|\int_{\Omega}f_n\bar gd\mu|\\
                    &=|\phi(f_n)|\\
                    &\leq \| \phi \|\| f_n \|_p\\
                    &=\| \phi \|\left( \int_{E_n}| f_n |^p \right)^{1/p}\\
                    &=\| \phi \|\left( \int_{E_n}| g |^q \right)^{1/p}.
                \end{align}
            \end{subequations}
            Nous avons de ce fait une inégalité de la forme \( A\leq \| \phi \|A^{1/p}\) et donc aussi \( A^{1/p}\leq \| \phi \|^{1/p}A^{1/p^2}\), et donc \( A\leq \| \phi \|\| \phi \|^{1/p}A^{1/p^2}\). Continuant ainsi à injecter l'inégalité dans elle-même, pour tout \( k\in \eN\) nous avons :
            \begin{equation}
                \left| \int_{E_n}| g |^qd\mu \right| \leq\| \phi \|^{1+\frac{1}{ p }+\cdots+\frac{1}{ p^k }}\left( \int_{E_n}| g |^qd\mu \right)^{1/p^k}.
            \end{equation}
            Nous pouvons passer à la limite \( k\to \infty\). Sachant que \( p>1\) nous savons \( A^{1/k}\to 1\) et
            \begin{equation}
                1+\frac{1}{ p }+\cdots+\frac{1}{ p^k }\to\frac{ p }{ p-1 }=q.
            \end{equation}
            Nous avons alors
            \begin{equation}
                \int_{E_n}| g |^qd\mu\leq \| \phi \|^q.
            \end{equation}
            L'intégrale s'écrit tout aussi bien sous la forme \( \int_{\Omega}| g  |^q\mtu_{E_n}\). La fonction dans l'intégrale est une suite croissante de fonctions mesurables à valeurs dans \( \mathopen[ 0 , \infty \mathclose]\). Nous pouvons alors permuter l'intégrale et la limite \( n\to \infty\) en utilisant la convergence monotone (théorème~\ref{ThoRRDooFUvEAN}) qui donne alors \( \int_{\Omega}| g |^q\leq \| \phi \|^q\) ou encore
            \begin{equation}
                \| g \|_q\leq \| \phi \|.
            \end{equation}

            Ceci achève de prouver que l'application \( \phi\mapsto \Phi_g\) est une isométrie, et donc le théorème.
    \end{subproof}
\end{proof}

\begin{theorem}     \label{THOooXMVTooBAbyvr}
    Soit un espace mesuré \( (\Omega,\tribA,\mu)\).
    \begin{enumerate}
        \item   \label{ITEMooNCVEooTyNsoJ}
            Si \( 1<p<\infty\), alors \( L^p(\Omega,\tribA,\mu)\) est réflexif\footnote{Définition \ref{PROPooMAQSooCGFBBM}.}.
        \item   \label{ITEMooTQDJooFShTiA}
            Si \( (\Omega,\tribA,\mu)\) est \( \sigma\)-finie, alors
            \begin{enumerate}
                \item       \label{ITEMooHMMZooMQxWgB}
                    \( (L^1)'=L^{\infty}\)
                \item       \label{ITEMooBFFZooNxoHER}
                    \( L^1\subset (L^{\infty})' \).
            \end{enumerate}
    \end{enumerate}
\end{theorem}
\index{dual!de $L^p(\Omega)$}

\begin{proof}
    En plusieurs parties, en notant toujours \( p\) et \( q\) les exposants conjugués, c'est-à-dire \( \frac{1}{ p }+\frac{1}{ q }=1\).
    \begin{subproof}
        \item[Pour \ref{ITEMooNCVEooTyNsoJ}]
            Le théorème \ref{ThoLPQPooPWBXuv}\ref{ITEMooSQQBooWSFBmX} nous indique que
            \begin{equation}        \label{EQooQHBIooWrkiYC}
                (L^p)'=L^q
            \end{equation}
            au sens d'une bijection isométrique. Vu que \( 1<p<\infty\), nous avons aussi \( 1<q<\infty\) et donc \( (L^q)'=L^p\). En prenant le dual des deux côtés de \eqref{EQooQHBIooWrkiYC}, 
            \begin{equation}
                (L^p)''=(L^q)'=L^p,
            \end{equation}
            et nous avons prouvé que \( L^p\) est réflexif.
        \item[Pour \ref{ITEMooHMMZooMQxWgB}]
            Il s'agit du théorème \ref{ThoLPQPooPWBXuv}\ref{ITEMooCQGJooOWzjoV}.
        \item[Pour \ref{ITEMooBFFZooNxoHER}\cite{MonCerveau}]
            Il nous reste à couvrir le cas de \( (L^{\infty})'\). Pour \( g\in L^1\) nous prouvons que \( \Phi_g\in (L^{\infty})'\). 
            
            \begin{subproof}
                \item[\( \Phi_g(f)\) est bien définie]
                    Nous prouvons d'abord que si \( f\in L^{\infty}\), alors l'intégrale \( \int_{\Omega}f\bar g\) est bien définie. Par définition du supremum essentiel\footnote{Voir les définitions \ref{DEFooIQOOooLpJBqi} et \ref{DEFooXUKHooXYrlYq}.}, il existe \( M>0\) tel que \( | f(x) |<M\) pour tout \( x\) hors d'une partie \( A\) de mesure nulle. Nous avons alors
                    \begin{equation}
                        \int_{\Omega}|f\bar g|=\int_{\Omega\setminus A}| f\bar g |\leq M\int_{\Omega\setminus A}| f |= M\int_{\Omega}| f |<\infty.
                    \end{equation}
                \item[\( \Phi_g\) est continue]
                    Soit une suite \( f_k\stackrel{L^{\infty}}{\longrightarrow}f\) ainsi que \( g\in L^1\). Pour chaque \( k\), il existe une partie de mesure nulle \( A_k\) et un nombre \( M_k=\| f_k \|_{L^{\infty}}\) tel que \( | f_k(x) |<\| f_k \|_{L^{\infty}}\) pour tout \( x\) hors de \( A_k\). Nous avons alors
                    \begin{equation}
                        | \Phi_g(f_k) |\leq \int_{\Omega\setminus A_k}| f_k\bar g |d\mu\leq \| f_k \|_{L^{\infty}}\int_{\Omega\setminus A_k}| g |\leq \| f_k \|_{L^{\infty}}\| g \|_1.
                    \end{equation}
                    Vu que par hypothèse \( f_k\to 0\) dans \( L^{\infty}\), nous avons \( \| f_k \|_{L^{\infty}\to 0}\), et donc aussi
                    \begin{equation}
                        |\Phi_g(f_k)|\to 0.
                    \end{equation}
            \end{subproof}
    \end{subproof}
\end{proof}

\begin{proposition} \label{PropUKLZZZh}
    Soit \( f\in L^p(\Omega)\) telle que
    \begin{equation}
        \int_{\Omega}f\varphi=0
    \end{equation}
    pour tout \( \varphi\in C^{\infty}_c(\Omega)\). Alors \( f=0\) presque partout.
\end{proposition}

\begin{proof}
    Nous considérons la forme linéaire \( \Phi_f\in (L^q)'\) donnée par
    \begin{equation}
        \begin{aligned}
            \Phi_f\colon L^p&\to \eC \\
            u&\mapsto \int_{\Omega}fu
        \end{aligned}
    \end{equation}
    Par hypothèse cette forme est nulle sur la partie dense \(  C^{\infty}_c(\Omega)\). Si \( (\varphi_n)\) est une suite dans \(  C^{\infty}_c(\Omega)\) convergente vers \( u\) dans \( L^p\), nous avons pour tout \( n\) que
    \begin{equation}
        0=\Phi_f(\varphi_n)
    \end{equation}
    En passant à la limite, nous voyons que \( \Phi_f\) est la forme nulle. Elle est donc égale à \( \Phi_0\). La partie « unicité » du théorème de représentation de Riesz~\ref{ThoLPQPooPWBXuv} nous indique alors que \( f=0\) dans \( L^p\) et donc \( f=0\) presque partout.
\end{proof}

\begin{proposition} \label{PropLGoLtcS}
    Si \( f\in L^1_{loc}(I)\) est telle que
    \begin{equation}
        \int_If\varphi'=0
    \end{equation}
    pour tout \( \varphi\in  C^{\infty}_c(I)\), alors il existe une constante \( C\) telle que \( f=C\) presque partout.
\end{proposition}

\begin{proof}
    Soit \( \psi\in C^{\infty}_c(I)\) une fonction d'intégrale \( 1\) sur \( I\). Si \( w\in C^{\infty}_c(I)\) alors nous considérons la fonction
    \begin{equation}
        h=w-\psi\int_Iw,
    \end{equation}
    qui est dans \(  C^{\infty}_c(I)\) et dont l'intégrale sur \( I\) est nulle. Par la proposition~\ref{PropHFWNpRb}, la fonction \( h\) admet une primitive dans \(  C^{\infty}_c(I)\); et nous notons \( \varphi\) cette primitive. L'hypothèse appliquée à \( \varphi\) donne
    \begin{equation}
        0=\int_If\varphi'=\int_If\left( w-\psi\int_Iw \right)=\int_Ifw-\underbrace{\left( \int_If(x)\psi(x)dx \right)}_C\left( \int_Iw(y)dy \right)=\int_Iw(f-C).
    \end{equation}
    L'annulation de la dernière intégrale implique par la proposition~\ref{PropUKLZZZh} que \( f-C=0\) dans \( L^2\), c'est-à-dire \( f=C\) presque partout.
\end{proof}

Dans \cite{ooHGADooNGZnbt}, il est dit que « la preuve [du lemme suivant], un peu fastidieuse mais en rien ingénieuse, est laissée en exercice ». La preuve est donc de moi; elle est un tout petit peu ingénieuse mais en rien fastidieuse. J'espère ne pas m'être trompé et me demande bien ce que l'auteur avait en tête. Ma preuve s'appuie sur la proposition \ref{PROPooLIGIooPrHYlb} dont la preuve ne me paraît pas non plus «fastidieuse mais en rien ingénieuse».

\begin{lemma}[\cite{ooHGADooNGZnbt,MonCerveau}]        \label{LEMooLDQRooEGWDlm}
    Soient \( r>0\). Il existe \( \delta>0\) tel que pour tout \( s,t\in \eC\) vérifiant \( | s |\leq 1\), \( | t |\leq 1\) et \( | s-t |\geq r\) nous ayons
    \begin{equation}
        \left| \frac{ s+t }{ 2 } \right|^p\leq (1-\delta)\frac{ | s |^p+| t |^p }{2}.
    \end{equation}
\end{lemma}

\begin{proof}
    Soit \( r>0\). La partie de \( \eC^2\) donnée par
    \begin{equation}
        D=\{ (s,t)\in \eC^2\tq | s |\leq 1, | t |\leq 1,| s-t |\geq r \}
    \end{equation}
    est compacte. En effet elles est bornée (par la sphère de rayon \( \sqrt{ 2 }\)) et fermée comme intersection de fermée\footnote{Lemme \ref{LemQYUJwPC} suivit du théorème de Borel-Lebesgue \ref{ThoXTEooxFmdI}.}. Nous considérons la fonction \( \Delta\colon D\to \eR\) donnée par
    \begin{equation}
        \left| \frac{ s+t }{2} \right|^p=\Delta(s,t)\frac{ | s |^p+| t |^p }{2}.
    \end{equation}
    Si vous voulez une expression explicite,
    \begin{equation}
        \Delta(s,t)=\frac{ 2^{p-1}| s+t |^p }{ | s |^p+| t |^p }.
    \end{equation}
    Cela est bien défini et continu sur \( D\) parce que le complémentaire \( D^c\) (qui est ouvert) contient \( (0,0)\) et donc aussi un voisinage de \( (0,0)\).

    La proposition \ref{PROPooLIGIooPrHYlb} nous dit que la fonction \( z\mapsto | z |^p\) est strictement convexe. En prenant la définition \ref{DEFooKCFPooLwKAsS} de la stricte convexité avec \( \theta=\frac{ 1 }{2}\), nous trouvons que
    \begin{equation}
        \Delta(s,t)<1
    \end{equation}
    pour tout \( (s,t)\in D\). Vu que par ailleurs \( \Delta\) est une fonction continue sur le compact \( D\), elle atteint un minimum dans \( D\). Soit \( \Delta_0\) ce minimum quivérifie forcément \( \Delta_0<1\).

    En posant \( 1-\delta=\Delta_0\) nous avons le résutat.
\end{proof}

%+++++++++++++++++++++++++++++++++++++++++++++++++++++++++++++++++++++++++++++++++++++++++++++++++++++++++++++++++++++++++++
\section{Théorèmes de Hahn-Banach}
%+++++++++++++++++++++++++++++++++++++++++++++++++++++++++++++++++++++++++++++++++++++++++++++++++++++++++++++++++++++++++++

\begin{theorem}[Hahn-Banach\cite{brezis,TQSWRiz}]
    Soit \( E\), un espace vectoriel réel et une application \( p\colon E\to \eR\) satisfaisant
    \begin{enumerate}
        \item
            \( p(\lambda x)=\lambda p(x)\) pour tout \( x\in E\) et pour tout \( \lambda>0\),
        \item
            \( p(x+y)\leq p(x)+p(y)\) pour tout \( x,y\in E\).
    \end{enumerate}
    Soit de plus \( G\subset E\) un sous-espace vectoriel muni d'une application \( g\colon G\to \eR\) vérifiant \( g(x)\leq p(x)\) pour tout \( x\in G\). Alors il existe \( f\in\aL(E,\eR)\) telle que \( f(x)=g(x)\) pour tout \( x\in G\) et \( f(x)\leq p(x)\) pour tout \( x\in E\).
\end{theorem}
\index{théorème!Hahn-Banach}

\begin{proof}
    Si \( h\) une application linéaire définie sur un sous-espace de \( E\), nous notons \( D_h\) ledit sous-espace.

    \begin{subproof}
    \item[Un ensemble inductif]

        Nous considérons \( P\), l'ensemble des fonctions linéaires suivant
        \begin{equation}
            P=\Big\{  h\colon D_h\to \eR\tq
            \begin{cases}
                G\subset D_h\\
                h(x)=g(x)&\forall x\in G\\
                h(x)\leq p(x)&\forall x\in D_h
            \end{cases}
        \Big\}
        \end{equation}
        Cet ensemble est non vide parce que \( g\) est dedans. Nous le munissons de la relation d'ordre \( h_1\leq h_2\) si et seulement si \( D_{h_1}\subset D_{h_2}\) et \( h_2\) prolonge \( h_1\). Nous montrons à présent que \( P\) est un ensemble inductif. Soit un sous-ensemble totalement ordonné \( Q\subset P\); nous définissons une fonction \( h\) de la façon suivante. D'abord \( D_h=\sup_{l\in Q}D_l\) et ensuite
        \begin{equation}
            \begin{aligned}
                h\colon D_h&\to \eR \\
                x&\mapsto l(x)&\text{si } x\in D_l
            \end{aligned}
        \end{equation}
        Cela est bien définit parce que si \( x\in D_l\cap D_{l'}\) alors, vu que \( Q\) est totalement ordonné (i.e. \( l\leq l'\) ou \( l'\leq l\)), on a obligatoirement \( D_l\subset D_{l'}\) et \( l'\) qui prolonge \( l\) (ou le contraire). Donc \( h\) est un majorant de \( Q\) dans \( P\) parce que \( h\geq l\) pour tout \( l\in Q\). Cela montre que \( P\) est inductif (définition~\ref{DefGHDfyyz}). Le lemme de Zorn~\ref{LemUEGjJBc} nous dit alors que \( P\) possède un maximum \( f\) qui va être la réponse à notre théorème.

    \item[Le support de \( f\)]

        La fonction \( f\) est dans \( P\); donc \( f(x)\leq p(x)\) pour tout \( x\in D_h\) et \( f(x)=g(x)\) pour tout \( x\in G\). Pour terminer nous devons montrer que \( D_f=E\). Supposons donc que \( D_f\neq E\) et prenons \( x_0\notin D_f\). Nous allons contredire la maximalité de \( f\) en considérant la fonction \( h\) donnée par \( D_h=D_f+\eR x_0 \) et
        \begin{equation}
            h(x+tx_0)=f(x)+t\alpha
        \end{equation}
        où \( \alpha\) est une constante que nous allons fixer plus tard.

        Nous commençons par prouver que \( f\) est dans \( P\). Nous devons prouver que
        \begin{equation}    \label{EqOIXrlFe}
            h(x+tx_0)=f(x)+t\alpha\leq p(x+tx_0)
        \end{equation}
        Pour cela nous allons commencer par fixer \( \alpha\) pour avoir les relations suivantes :
        \begin{subequations}    \label{EqMDNkcQk}
            \begin{numcases}{}
                f(x)+\alpha\leq p(x+x_0)    \label{EqDYmRWEY}\\
                f(x)-\alpha\leq p(x-x_0)
            \end{numcases}
        \end{subequations}
        pour tout \( x\in D_f\). Ces relations sont équivalentes à demander \( \alpha \) tel que
        \begin{subequations}
            \begin{numcases}{}
                \alpha\leq p(x+x_0)-f(x)\\
                \alpha\geq f(x)-p(x-x_0)
            \end{numcases}
        \end{subequations}
        Nous nous demandons donc s'il existe un \( \alpha\) qui satisfasse
        \begin{equation}
            \sup_{y\in D_f}\big( f(y)-p(y-x_0) \big)\leq \alpha\leq \inf_{z\in D_f}\big( p(z+x_0)-f(z) \big).
        \end{equation}
        Ou encore nous devons prouver que pour tout \( y,z\in D_f\),
        \begin{equation}
            p(z+x_0)-f(x)\geq f(y)-p(y-x_0)\geq 0.
        \end{equation}
        Par les propriétés de \( p\) et de \( f\),
        \begin{equation}
        p(z+x_0)+p(y-x_0)-f(z)-f(y)\geq p(z+y)-f(z+y)\geq 0.
        \end{equation}
        La dernière inégalité est le fait que \( f\in P\). Un choix de \( \alpha\) donnant les inéquations \eqref{EqMDNkcQk} est donc possible.

        À partir des inéquations \eqref{EqMDNkcQk} nous obtenons la relation \eqref{EqOIXrlFe} de la façon suivante. Si \( t>0\) nous multiplions l'équation \eqref{EqDYmRWEY} par \( t\) :
        \begin{equation}
            tf(x)+t\alpha\leq tp(x+x_0).
        \end{equation}
        Et nous écrivons cette relation avec \( x/t\) au lieu de \( x \) en tenant compte de la linéarité de \( f\) :
        \begin{equation}
            f(x)+t\alpha\leq  tp\big( \frac{ x }{ t }+x_0 \big)=p(x+tx_0).
        \end{equation}
        Avec \( t<0\), c'est similaire, en faisant attention au sens des inégalités.

        Nous avons donc construit \( h\colon D_h\to \eR\) avec \( h\in P\), \( D_f\subset D_h\) et \( h(x)=f(x)\) pour tout \( x\in D_f\). Cela pour dire que \( h>f\), ce qui contredit la maximalité de \( f\). Le domaine de \( f\) est donc \( E\) tout entier.

        La fonction \( f\) est donc une fonction qui remplit les conditions.

    \end{subproof}
\end{proof}

\begin{definition}  \label{DefPJokvAa}
    Un espace topologique est \defe{localement convexe}{convexité!locale} si tout point possède un système fondamental de voisinages formé de convexes.
\end{definition}
%TODO : il faudrait parler de système fondamental de voisinages.

\begin{definition}[Hyperplan qui sépare]
    Soit \( E\) un espace vectoriel topologique ainsi que \( A\), \( B\) des sous-ensembles de \( E\). Nous disons que l'hyperplan d'équation \( f=\alpha\) \defe{sépare au sens large}{hyperplan!séparer!au sens large} les parties \( A\) et \( B\) si \( f(x)\leq \alpha\) pour tout \( x\in A\) et \( f(x)\geq \alpha\) pour tout \( x\in B\).

    La séparation est \defe{au sens strict}{hyperplan!sépare!au sens strict} s'il existe \( \epsilon>0\) tel que
    \begin{subequations}
        \begin{align}
            f(x)\leq \alpha-\epsilon&&\text{pour tout } x\in A\\
            f(x)\geq \alpha+\epsilon&&\text{pour tout } x\in B.
        \end{align}
    \end{subequations}
\end{definition}

\begin{theorem}[Hahn-Banach, première forme géométrique\cite{TQSWRiz}]  \label{ThoSAJjdZc}
    Soit \( E\) un espace vectoriel topologique et \( A\), \( B\) deux convexes non vides disjoints de \( E\). Si \( A\) est ouvert, il existe un hyperplan fermé qui sépare \( A\) et \( B\) au sens large.
\end{theorem}

\begin{theorem}[Hahn-Banach, seconde forme géométrique] \label{ThoACuKgtW}
    Soient un espace vectoriel topologique localement convexe\footnote{Définition~\ref{DefPJokvAa}.} ainsi que des convexes non vides disjoints \( A\) et \( B\) tels que \( A\) soit compact et \( B\) soit fermé. Alors il existe un hyperplan fermé qui sépare strictement \( A\) et \( B\).
\end{theorem}

\begin{proof}
    Vu que \( B\) est fermé, \( A\) est dans l'ouvert \( E\setminus B\). Donc si \( a\in A\), il existe un voisinage ouvert convexe de \( a\) inclus dans \( A\). Soit \( U_a\) un voisinage ouvert et convexe de \( 0\) tel que \( (a+U_a)\cap B=\emptyset\).

    Vu que la fonction \( (x,y)\mapsto x+y\) est continue, nous pouvons trouver un ouvert convexe \( V_a\) tel que \( V_a+V_a\subset U_a\). L'ensemble \( a+V_a\) est alors un voisinage ouvert de \( a\) et bien entendu \( \bigcup_a(a+V_a)\) recouvre \( A\) qui est compact. Nous en extrayons un sous-recouvrement fini, c'est-à-dire que nous considérons \( a_1,\ldots, a_n\in A\) tels que
    \begin{equation}
        A\subset \bigcup_{i=1}^n(a_i+V_{a_i}).
    \end{equation}
    Nous posons alors
    \begin{equation}
        V=\bigcap_{i=1}^nV_{a_i}.
    \end{equation}
    Cet ensemble est non vide parce et il contient un voisinage de zéro parce que c'est une intersection finie de voisinages de zéro. Soit \( x\in A+V\). Il existe \( i\) tel que
    \begin{equation}
        x\in a_i+U_{a_i}+V\subset a_i+V_{a_i}+V_{a_i}\subset a_i+U_{a_i}\subset E\setminus B.
    \end{equation}
    Donc \( (A+V)\cap B=\emptyset\). L'ensemble \( A+V\) est alors un ouvert convexe disjoint de \( B\). Par la première forme géométrique du théorème de Hahn-Banach~\ref{ThoSAJjdZc} nous avons un hyperplan qui sépare \( A+V\) de \( B\) au sens large : il existe \( f\in E'\setminus\{ 0 \}\) tel que \( f(a)+f(v)\leq f(b)\) pour tout \( a\in A\), \( v\in V\) et \( b\in B\).

    Il suffit donc de trouver un \( v\in V\) tel que \( f(v)\neq 0\) pour avoir la séparation au sens strict. Cela est facile : \( V\) étant un voisinage de zéro et \( f\) étant linéaire, si elle était nulle sur \( V\), elle serait nulle sur \( E\).
\end{proof}

%+++++++++++++++++++++++++++++++++++++++++++++++++++++++++++++++++++++++++++++++++++++++++++++++++++++++++++++++++++++++++++
\section{Théorème de Tietze}
%+++++++++++++++++++++++++++++++++++++++++++++++++++++++++++++++++++++++++++++++++++++++++++++++++++++++++++++++++++++++++++

\begin{definition}
Si \( E\) et \( F\) sont des espaces normés, une application \( f\colon E\to F\) est \defe{presque surjective}{presque!surjective} s'il existe \( \alpha\in\mathopen] 0 , 1 \mathclose[\) et \( C>0\) tels que pour tout \( y\in \overline{ B_F(0,1) }\), il existe \( x\in\overline{ B_E(0,C) }\) tel que \( \| y-f(x) \|\leq \alpha\).
\end{definition}

\begin{lemma}[\cite{KXjFWKA}]   \label{LemBQLooRXhJzK}
    Soient \( E\) et \( F\) des espaces de Banach et \( f\in\cL(E,F)\)\footnote{L'ensemble des applications linéaires continues}. Si \( f\) est presque surjective, alors
    \begin{enumerate}
        \item   \label{ItemTSOooYkxvBui}
            \( f\) est surjective
        \item\label{ItemTSOooYkxvBuii}
            pour tout \( y\in \overline{ B_F(0,1) }\), il existe \( x\in\overline{ B_E(0,\frac{ C }{ 1-\alpha }) }\) tel que \( y=f(x)\).
    \end{enumerate}
\end{lemma}
Le point~\ref{ItemTSOooYkxvBuii} est une précision du point~\ref{ItemTSOooYkxvBui} : il dit quelle est la taille de la boule de \( E\) nécessaire à obtenir la boule unité dans \( F\).

\begin{proof}
    Soit \( y\in \overline{ B_F(0,1) }\). Nous allons construire \( x\in B\big( 0,\frac{ C }{ 1-\alpha } \big)\) qui donne \( f(x)=y\). Ce \( x\) sera la limite d'une série que nous allons construire par récurrence. Pour \( n=1\) nous utilisons la presque surjectivité pour considérer \( x_1\in\overline{ B_E(0,C) } \) tel que \( \| y-f(x_1) \|\leq \alpha\). Ensuite nous considérons la récurrence
    \begin{equation}
        x_n\in \overline{ B_E(0,C) }
    \end{equation}
    tel que
    \begin{equation}
        \big\| y-\sum_{i=1}^n\alpha^{i-1}f(x_i) \big\|\leq \alpha^n
    \end{equation}
    Pour montrer que cela existe nous supposons que la série est déjà construire jusqu'à \( n>1\) :
    \begin{equation}
        \frac{1}{ \alpha^n }\Big( y-\sum_{i=1}^n\alpha^{i-1}f(x_i) \Big)\in \overline{ B_F(0,1) }
    \end{equation}
    À partir de là, par presque surjectivité il existe un \( x_{n+1}\in \overline{ B_E(0,C) }\) tel que
    \begin{equation}
        \big\| \frac{ y-\sum_{i=1}^n\alpha^{i-1}f(x_i) }{ \alpha^n }-f(x_{n+1}) \big\|\leq \alpha.
    \end{equation}
    En multipliant par \( \alpha^{n}\), le terme \( \alpha^nf(x_{n+1})\) s'intègre bien dans la somme :
    \begin{equation}
        \big\| y=\sum_{i=1}^{n+1}\alpha^{i-1}f(x_i) \big\|\leq \alpha^{n+1}.
    \end{equation}
    Nous nous intéressons à une éventuelle limite à la somme des \( \alpha^{n-1}x_n\). D'abord nous avons la majoration \( \| \alpha^{n-1}x_n \|\leq \alpha^{n-1}C\), et vu que par la définition de la presque surjectivité \( 0<\alpha<1\), la série
    \begin{equation}
        \sum_{n=1}^{\infty}\alpha^{n-1}x_n
    \end{equation}
    converge absolument\footnote{Définition~\ref{DefVFUIXwU}.} parce que la suite des normes est une suite géométrique de raison \( \alpha\). Vu que \( E\) est de Banach, la convergence absolue implique la convergence simple (la suite des sommes partielles est de Cauchy et Banach est complet). Nous posons
    \begin{equation}
        x=\sum_{n=1}^{\infty}\alpha^{n-1}x_n\in E,
    \end{equation}
    et en termes de normes, ça vérifie
    \begin{equation}
        \| x \|\leq\sum_{n=1}^{\infty}\alpha^{n-1}\| x_n \|\leq C\sum_{n=1}^{\infty}\alpha^{n-1}=\frac{ C }{ 1-\alpha }.
    \end{equation}
    Donc c'est bon pour avoir \( x\in B\big( 0,\frac{ C }{ 1-\alpha } \big)\). Nous devons encore vérifier que \( y=f(x)\). Pour cela nous remarquons que
    \begin{equation}
        \| y-f\Big( \sum_{n=1}^N\alpha^{n-1}x_n \Big) \|\leq \alpha^N.
    \end{equation}
    Nous pouvons prendre la limite \( N\to \infty\) et permuter \( f\) avec la limite (par continuité de \( f\)). Vu que \( 0<\alpha<1\) nous avons
    \begin{equation}
        \| y-f(x) \|=0.
    \end{equation}
\end{proof}

\begin{theorem}[Tietze\cite{KXjFWKA,ytMOpe}]   \label{ThoFFQooGvcLzJ}
    Soit un espace métrique \( (X,d)\) et un fermé \( Y\subset X\). Soit \( g_0\in C^0(Y,\eR)\). Alors \( g_0\) admet un prolongement continu sur \( X\).
\end{theorem}

\begin{proof}
    Soit l'opération de restriction
    \begin{equation}
        \begin{aligned}
            T\colon (C^0_b(X,\eR),\| . \|_{\infty})&\to (C^0_b(Y,\eR),\| . \|_{\infty}) \\
            f&\mapsto f|_Y.
        \end{aligned}
    \end{equation}
    L'application \( T\) est évidemment linéaire. Elle est de plus borné pour la norme opérateur usuelle donnée par la proposition~\ref{DefNFYUooBZCPTr} parce que \( \| T(f) \|\leq \| f \|<\infty\). L'application \( T\) est alors continue par la proposition~\ref{PROPooQZYVooYJVlBd}.

    \begin{subproof}
    \item[Presque surjection]

    Soit \( g\in C^0_b(Y,\eR)\) avec \( \| g \|_{\infty}\leq 1\). Nous posons
    \begin{subequations}
        \begin{align}
            Y^+=\{ x\in Y\tq \frac{1}{ 3 }\leq g(x)\leq 1 \}\\
            Y^-=\{ x\in Y\tq -1\leq g(x)\leq -\frac{1}{ 3 } \}.
        \end{align}
    \end{subequations}
    Nous considérons alors
    \begin{equation}
        \begin{aligned}
            f\colon X&\to \eR \\
            x&\mapsto \frac{1}{ 3 }\frac{ d(x,Y^-)-d(x,Y^+) }{ d(x,Y^-)+d(x,Y^+) }
        \end{aligned}
    \end{equation}
    Vu qu'en valeur absolue le dénominateur est plus grand que le numérateur nous avons \( \| f \|_{\infty}\leq \frac{1}{ 3 }\). Notons que
    \begin{itemize}
        \item Si \( x\in Y^+\) alors \( f(x)=\frac{1}{ 3 }\) et \( g(x)\in\mathopen[ \frac{1}{ 3 } , 1 \mathclose]\);
        \item Si \( x\in Y^-\) alors \( f(x)=-\frac{1}{ 3 }\) et \( g(x)\in\mathopen[-1,-\frac{1}{ 3 } \mathclose]\);
        \item Si \( x\) n'est ni dans \( Y^+\) ni dans \( Y^-\) alors nous avons\footnote{Nous rappelons que \( \| g \|=1\), donc \( g(x)\) est forcément ente \( -1\) et \( 1\).} \( g(x)\in\mathopen[ -\frac{1}{ 3 } , \frac{1}{ 3 } \mathclose]\) et donc \( \big| f(x)-g(x) \big|\leq \big| f(x) \big|+\big| g(x) \big|\leq \frac{ 2 }{ 3 }\).
    \end{itemize}
    Dans les deux cas nous avons \( \big| f(x)-g(x) \big|\in\mathopen[ 0 , \frac{ 2 }{ 3 } \mathclose]\) pour tout \( x\in X\). Cela prouve que
    \begin{equation}
        \| T(f)-g \|_{Y,\infty}\leq \frac{ 2 }{ 3 }.
    \end{equation}
    En résumé nous avons pris \( g\) dans la boule \( \overline{ B(0,1) }\) de \( \big( C^0_b(Y,\eR), \| . \|_{\infty} \big)\) et nous avons construit une fonction \( f\) dans la boule \( \overline{ B(0,\frac{1}{ 3 }) }\) de \( \big( C^0_b(X,\eR),\| . \|_{\infty} \big)\) telle que \( \| T(f)-g \|_{\infty}\leq \frac{ 2 }{ 3 }\). L'application \( T\) est donc une presque surjection avec \( \alpha=\frac{1}{ 3 }\) et \( C=\frac{ 2 }{ 3 }\).

\item[Prolongement dans les boules unité fermées]

    La proposition~\ref{PropSYMEZGU} nous assure que les espaces \( C^0_b(X,\eR)\) et \( C_b^0(Y,\eR)\) sont de Banach (complets), et le lemme~\ref{LemBQLooRXhJzK} nous dit alors que \( T\) est surjective et que pour tout \( g\in\overline{ B(0,1) }\), il existe
    \begin{equation}
        f\in\overline{ B\left( 0,\frac{ 1/3 }{ 1-\frac{ 2 }{ 3 } } \right) }=\overline{ B(0,1) }.
    \end{equation}
    telle que \( g=T(f)\).


\item[Prolongement pour les boules ouvertes]

    Jusqu'à présent nous avons montré qu'une fonction \( g\in\overline{ B(0,1) }\) admet une prolongement continu dans \( \overline{ B(0,1) }\). Nous allons montrer que si \( g\) est dans la boule ouverte \( B(0,1)\) de \( \big( C^0_b(Y,\eR),\| . \|_{\infty} \big)\) alors \( g\) admet un prolongement dans la boule ouverte \( B(0,1)\) de \( \big( C_b^0(X,\eR),\| . \|_{\infty} \big)\).

    Soit \( g\in B_{C^0_b(Y)}(0,1) \) et son prolongement \( h\in \overline{ B_{C_b^0(X)}(0,1) }\). Si \( \| h \|_{\infty}<1\) alors le résultat est vrai. Sinon nous considérons l'ensemble
    \begin{equation}
        Z=\{ x\in X\tq | h(x) |=1 \}.
    \end{equation}
    Nous avons \( Y\cap Z=\emptyset\) parce que nous avons \( h=g\) sur \( Y\) et nous avons choisi \( \| g \|_{\infty}<\infty\). Par ailleurs \( Y\) est fermé par hypothèse et \( Z\) est fermé parce que \( h\) est continue; par conséquent \( Y\cap Z\) est fermé, donc\footnote{Si vous avez l'intention de dire que \( \overline{ Y\cap Z }=\bar Y\cap\bar Z=Y\cap Z=\emptyset\), allez d'abord voir l'exemple~\ref{ExBFLooUNyvbw}. Ici c'est correct parce que \( Y\) et \( Z\) sont fermés.}
    \begin{equation}
        \bar Y\cap\bar Z=Y\cap Z=\emptyset.
    \end{equation}
    Nous posons
    \begin{equation}
        \begin{aligned}
            u\colon X&\to \eR^+ \\
            x&\mapsto \frac{ d(x,Z) }{ d(x,Y)+d(x,Z) }
        \end{aligned}
    \end{equation}
    Le dénominateur n'est pas nul parce qu'il faudrait \( d(x,Y)=d(x,Z)=0\), ce qui demanderait \( x\in \bar Y\cap\bar Z\), ce qui n'est pas possible. Nous posons \( f=uh\). Si \( x\in Y\) alors \( u(x)=1\), donc \( f\) est encore un prolongement de \( g\). De plus \( f\) est encore continue, et donc encore un bon candidat. Enfin si \( x\) est hors de \( Y\) alors \( d(x,Y)>0\) (strictement parce que \( Y\) est fermé) et donc \( 0<u(x)<1\), ce qui donne \( | f(x) |<| h(x) |\leq 1\). Donc \( \| f \|_{\infty}<1\).

    Nous avons donc trouvé qu'une fonction dans la boule ouverte \( B_{C^0_b(Y)}(0,1)\) se prolonge en une fonction dans la boule ouverte \( B_{C^0_b(X)}(0,1)\).

\item[Le cas non borné]

Soit enfin \( g_0\in C^0(Y,\eR)\). Nous allons nous ramener au cas de la boule unité ouverte en utilisant un homéomorphisme \( \phi\colon \eR\to \mathopen] -1 , 1 \mathclose[\). L'application \( g=\phi\circ g_0\) est dans la boule unité ouvert de \( C^0(Y,\eR)\) et donc admet un prolongement \( f\) dans la boule unité ouverte de \( C^0(X)\). L'application \( f_0=\phi^{-1}\circ f\) est un prolongement continu de \( g_0\).

    \end{subproof}
\end{proof}

Un homéomorphisme \( \phi\colon \eR\to \mathopen] -1 , 1 \mathclose[\) est donné par exemple par la fonction \( \phi(t)=\frac{ 2 }{ \pi }\arctan(t)\) dont le graphique est donné ci-dessous :
\begin{center}
    \input{auto/pictures_tex/Fig_FXVooJYAfif.pstricks}
\end{center}

%+++++++++++++++++++++++++++++++++++++++++++++++++++++++++++++++++++++++++++++++++++++++++++++++++++++++++++++++++++++++++++
\section{Espace de Schwartz}
%+++++++++++++++++++++++++++++++++++++++++++++++++++++++++++++++++++++++++++++++++++++++++++++++++++++++++++++++++++++++++++

Pour un multiindice \( \alpha=(\alpha_1,\ldots, \alpha_d)\in \eN^d\), nous notons
\begin{equation}
    \partial^{\alpha}\varphi=\partial_{x_1}^{\alpha_1}\ldots\partial_{x_d}^{\alpha_d}\varphi
\end{equation}
pour peu que la fonction \( \varphi\) soit \( | \alpha |=\alpha_1+\cdots +\alpha_d\) fois dérivable.

\begin{definition}  \label{DefHHyQooK}
    Soit \( \Omega\subset\eR^d\). L'\defe{espace de Schwartz}{espace!de Schwartz} \( \swS(\Omega)\) est le sous-ensemble de \(  C^{\infty}(\Omega)\) des fonctions dont toutes les dérivées décroissent plus vite que tout polynôme :
    \begin{equation}
        \swS(\Omega)=\big\{   \varphi\in C^{\infty}(\Omega)\tq\forall \alpha,\beta\in \eN^d, p_{\alpha,\beta}(\varphi)<\infty   \big\}
    \end{equation}
    où nous avons considéré
    \begin{equation}    \label{EqOWdChCu}
        p_{\alpha,\beta}(\varphi)=\sup_{x\in \Omega}| x^{\beta}(\partial^{\alpha}\varphi)(x) |=\| x^{\beta}\partial^{\alpha}\varphi \|_{\infty}.
    \end{equation}
\end{definition}

Pour simplifier les notations (surtout du côté de Fourier), nous allons parfois écrire \( M_i\varphi\)\nomenclature[Y]{\( M_i\varphi\)}{La fonction \( x\mapsto x_i\varphi(x)\)} pour la fonction \( x\mapsto x_i\varphi(x)\).

\begin{example}
    La fonction \(  e^{-x^2}\) est une fonction à décroissance rapide sur \( \eR\).
\end{example}

\begin{definition}
    Une fonction \( f\colon \eR^d\to \eC\) est dite à \defe{décroissance rapide}{fonction!à décroissance rapide} si elle décroît plus vite que n'importe quel polynôme. Plus précisément, si pour tout polynôme \( Q\), il existe un \( r>0\) tel que \(  | f(x) |<\frac{1}{ | Q(x) | } \) pour tout \( \| x \|\geq r\).
\end{definition}

\begin{proposition} \label{PropCSmzwGv}
    Une fonction Schwartz est à décroissance rapide.
\end{proposition}

\begin{proof}
    Nous commençons par considérer un polynôme \( P\) donné par
    \begin{equation}
        P(x)=\sum_kc_kx^{\beta_k}
    \end{equation}
    où les \( \beta_k\) sont des multiindices, les \( c_k\) sont des constantes et la somme est finie. Nous avons la majoration
    \begin{equation}
        \sup_{x\in \eR^d}| \varphi(x)P(x) |\leq\sum_k\sup_x\big| c_k\varphi(x) x^{\beta_k} \big|\leq\sum_k| c_k |p_{0,\beta_k}(\varphi)<\infty.
    \end{equation}
    Nous allons noter \( M_P\) la constante \( \sum_k| c_k |p_{0,\beta_k}(\varphi)\), de sorte que pour tout \( x\in \eR^d\) nous ayons \( | \varphi(x)P(x) |\leq M_P\) et donc
    \begin{equation}
        | \varphi(x) |\leq \frac{ M_P }{ | P(x) | }=\frac{1}{ | \frac{1}{ M_P }P(x) | }.
    \end{equation}
    Notons que cette inégalité est a fortiori correcte pour les \( x\) sur lesquels \( P\) s'annule.

    Soit maintenant un polynôme \( Q\). Nous considérons le polynôme \( P(x)=\| x \|Q(x)\). Étant de plus haut degré, pour toute constante \( C\) il existe un rayon \( r_C\) tel que \( | P(x) |\geq C| Q(x) |\) pour tout \( | x |\geq r_C\). En particulier pour \( | x |\geq r_{M_P}\) nous avons
    \begin{equation}
        | P(x) |\geq M_P| Q(x) |
    \end{equation}
    et donc, pour ces \( x\),
    \begin{equation}
        | \varphi(x) |\leq \frac{1}{ | \frac{1}{ M_P }P(x) | }\leq \frac{1}{ | Q(x) | }.
    \end{equation}
    La première inégalité est valable pour tout \( x\), et la seconde pour \( \| x \|\geq r_{M_P}\).
\end{proof}

\begin{corollary}[\cite{MonCerveau}]        \label{CORooZFPSooHCFUSH}
    Soit \( \varphi\) une fonction Schwartz sur \( \eR^m\times \eR^n\). Alors la fonction
    \begin{equation}
        y\mapsto \sup_{x\in \eR^n}| \varphi(x,y) |
    \end{equation}
    est intégrable.
\end{corollary}

\begin{proof}
    Soit un polynôme \( Q\) en la variable \( y\). Par la proposition~\ref{PropCSmzwGv}, il existe \( r>0\) tel que
    \begin{equation}
        | \varphi(x,t) |<\frac{1}{ Q(y) }
    \end{equation}
    pour tout \( \| (x,y) \|>r\). A fortiori l'inégalité tient pour tout \( | y |>r\). Donc
    \begin{equation}
        \int_{\eR^m}\sup_{x\in \eR^n}| \varphi(x,y) |dy=\int_{\| y \|\leq r}\sup_{x}| \varphi(x,y) |dy+\int_{ \| y \|>r  }\sup_{x}| \varphi(x,y) |dy.
    \end{equation}
    La première intégrale est bornée par \( \Vol\big( B(0,r) \big)\| f \|_{\infty}\) tandis que la seconde est bornée par l'intégrale de \( \frac{1}{ Q(y) }\). En prenant \( Q\) de degré suffisamment élevé en toutes les composantes de \( y\) nous avons intégrabilité.
\end{proof}

%---------------------------------------------------------------------------------------------------------------------------
\subsection{Topologie}
%---------------------------------------------------------------------------------------------------------------------------

\begin{lemmaDef}        \label{LEMDEFooZEFVooMMmiBr}
    Les \( p_{\alpha,\beta}\) donnés par l'équation \eqref{EqOWdChCu} ci-dessus sont des semi-normes\footnote{Définition~\ref{DefPNXlwmi}.}. La topologie considérée sur \( \swS(\eR^d)\) est celle des semi-normes \( p_{\alpha,\beta}\).
\end{lemmaDef}
%TODO : une preuve pour égayer la galerie.

\begin{normaltext}      \label{NORMooVQESooRwJShl}
Nous avons un enchainement de résultats qui nous aident à prouver la continuité d'une application \( T\colon \swS(\eR^d)\to X\).
\begin{enumerate}
    \item
        La topologie de \( \swS(\eR^d)\) est donnée par une famille dénombrable de semi-normes. Donc la proposition~\ref{PROPooMJEQooHtIyeX} nous dit que \( \swS(\eR^d)\) est métrisable.
    \item
        La proposition~\ref{PROPooKNVUooMbLZoy} nous dit alors que si \( X\) est métrique, toute application séquentiellement continue \( T\colon \swS(\eR^d)\to X\) est continue.
    \item
        Donc si \( X\) est métrique, il suffit de prouver que pour \( f_n\stackrel{\swS(\eR^d)}{\longrightarrow}0\) nous avons \( T(f_n)\stackrel{X}{\longrightarrow} 0\) où \( f_n\colon \swS(\eR^d)\to X\). Dans les cas usuels, \( T\) sera une distribution et \( X=\eC\).
    \item
        En vertu de la proposition~\ref{PropQPzGKVk}, la convergence \( f_n\stackrel{\swS(\eR^d)}{\longrightarrow}0\) signifie que pour tout choix de multiindice \( \alpha\) et \( \beta\),  \( p_{\alpha,\beta}(f_n)\to 0\), c'est-à-dire
        \begin{equation}        \label{EQooPUJPooNbtNFh}
            \| x^{\beta}\partial^{\alpha}f_n \|_{\infty}\to 0.
        \end{equation}
    \item
        Et enfin, la technique pour montrer que \( T\colon \swS(\eR^d)\to \eC\) est continue est de montrer que sous l'hypothèse d'avoir \eqref{EQooPUJPooNbtNFh} pour tout choix de \( \alpha\) et \( \beta\), nous avons \( T(f_n)\to 0\) dans \( \eC\).
\end{enumerate}
\end{normaltext}

\begin{lemma}[\cite{OEVAuEz}]   \label{LemRJhCbkO}
    La topologie sur \( \swS(\eR^d)\) est donnée aussi par les semi-normes
    \begin{equation}
        q_{n,m}=\max_{| \alpha |\leq n}\sup_{x\in \eR^d}\big( 1+\| x \| \big)^m\big| \partial^{\alpha}\varphi(x) \big|.
    \end{equation}
    Autrement dit, une suite \( \varphi_n\stackrel{\swS(\eR^d)}{\to}0\) si et seulement si \( q_{n,m(\varphi)}\to 0\) pour tout \( n\) et \( m\).
\end{lemma}
Le fait que les \( q_{n,m}(\varphi)\) restent bornés est la proposition~\ref{PropCSmzwGv}. Cependant ce lemme est plus précis parce qu'en disant seulement que \( \varphi\) est majoré par des polynôme, nous ne disons pas que les polynômes correspondants aux \( \varphi_n\) tendent vers zéro si \( \varphi_n\stackrel{\swS}{\to}0\). Et d'ailleurs on ne sait pas très bien ce que signifierait \( P_n\to 0\) pour une suite de polynômes.

\begin{proposition}     \label{PropGNXBeME}
    Pour \( p\in\mathopen[ 1 , \infty \mathclose]\), l'espace \( \swS(\eR^d)\) s'injecte continument dans \( L^p(\eR^d)\).
\end{proposition}

\begin{proof}
    L'injection dont nous parlons est l'identité ou plus précisément l'identité suivie de la prise de classe. Il faut vérifier que cela est correct et continu, c'est-à-dire d'abord qu'une fonction à décroissance rapide est bien dans \( L^p\) et ensuite que si \( f_n\stackrel{\swS}{\to}0\), alors \( f_n\stackrel{L^p}{\to}0\).

    Commençons par \( p=\infty\). Alors \( \| f_n \|_{\infty}=p_{0,0}(f_n)\to 0\) parce que si \( f_n\stackrel{\swS}{\to}0\), alors en particulier \( p_{0,0}(f_n)\to 0\).

    Au tour de \( p<\infty\) maintenant. Nous savons qu'en dimension \( d\), la fonction
    \begin{equation}
        x\mapsto \frac{1}{ (1+\| x \|)^s }
    \end{equation}
    est intégrable dès que \( s>d\).
    %TODO : il faudrait une petite preuve de ça.
    Pour toute valeur de \( m\) nous avons
    \begin{equation}
        \| \varphi \|_p^p=\int_{\eR^d}| \varphi(x) |^pdx=\int_{\eR^d}\frac{ \big|    (1+\| x \|)^m\varphi(x)   \big|^p }{ \big( 1+\| x \| \big)^{mp} }\leq\int_{\eR^d}\frac{q_{0,m}(\varphi)^p}{ \big( 1+\| x \| \big)^{mp} }.
    \end{equation}
    En choisissant \( m\) de telle sorte que \( mp>d\), nous avons convergence de l'intégrale et donc \( \| \varphi \|_p<\infty\). Nous retenons que
    \begin{equation}    \label{EqVWfEFMk}
        \| \varphi \|_p^p\leq Cq_{0,m}(\varphi)^p
    \end{equation}
    pour une certaine constance \( C\) et un bon choix de \( m\).

    Ceci prouve que \( \swS(\eR^d)\subset L^p(\eR^d)\). Nous devons encore vérifier que l'inclusion est continue. Si \( \varphi_n\stackrel{\swS}{\to}0\), alors en particulier nous avons \( q_{0,m}(\varphi_n)\to 0\) par le lemme~\ref{LemRJhCbkO}. Par conséquent la majoration \eqref{EqVWfEFMk} nous dit que \( \| \varphi_n \|_p\to 0\) également.

\end{proof}
En résumé, si \( \varphi_n\stackrel{\swS(\eR^d)}{\to}\varphi\) alors \( \varphi_n\stackrel{L^p}{\to}\varphi\).

\begin{theorem}[\cite{MesIntProbb}]      \label{ThoRWEoqY}
    Soit \( \mu\) une mesure sur les boréliens de \( \eR^n\) finie sur les compacts. Alors \( \swD(\eR^n)\) est dense dans \( L^1(\eR^n,\Borelien(\eR^n),\mu)\).
\end{theorem}
\index{densité!de \( \swD(\eR^n)\) dans \( L^1(\eR^n)\)}

\begin{proposition}[\cite{ooIKXSooRlKVJR}]      \label{PROPooJNQZooIRbJei}
    La partie \( \swD(\eR^d)\) est dense dans \( \swS(\eR^d)\).
\end{proposition}

\begin{proof}
    Soit \( f\in \swS(\eR^d)\), et \( \phi\), une fonction de \( \swD(\eR^d)\) telle que \( \phi(x)=1\) pour \(| x |\leq 1 \) (l'existence de telles fonctions est discutée en~\ref{subsecOSYAooXXCVjv}). Soit aussi \( \phi_k(x)=\phi(x/k)\). Nous posons
    \begin{equation}
        f_k(x)=\phi_k(x)f(x),
    \end{equation}
    et nous allons prouver que pour tout multiindices \( \alpha\) et \( \gamma\),
    \begin{equation}
        p_{\alpha,\gamma}(f_k-f)=\| x^{\gamma}\partial^{\alpha}(f_k-f)  \|_{\infty}\to 0.
    \end{equation}
    Pour cela nous allons noter \(  \beta\leq \alpha  \) lorsque \( \beta\) est un multiindice contenu dans \( \alpha\). En utilisant la dérivée du produit nous avons
    \begin{subequations}
        \begin{align}
            (\partial^{\alpha}f_k)(x)&=\sum_{\beta\leq \alpha}(\partial^{\alpha-\beta}\phi_k)(x)\partial^{\beta}f(x)\\
            &=\sum_{\beta\leq \alpha}k^{-| \alpha-\beta |}(\partial^{\alpha-\beta}\phi)(x/k)(\partial^{\beta}f)(x)\\
            &=\sum_{\beta< \alpha}k^{-| \alpha-\beta |}(\partial^{\alpha-\beta}\phi)(x/k)(\partial^{\beta}f)(x) + \phi(x/k)(\partial^{\alpha}f)(x).
        \end{align}
    \end{subequations}
    Nous devons donc étudier et majorer
    \begin{equation}
        \begin{aligned}[]
        \sup_{x\in \eR^d}| x^{\gamma}\partial^{\alpha}(f_k-f) |&\leq \sup\big| x^{\gamma}  \sum_{\beta< \alpha}k^{-| \alpha-\beta |}(\partial^{\alpha-\beta}\phi)(x/k)(\partial^{\beta}f)(x)  \big|\\
        &\quad+\sup \big| x^{\gamma}\big( \phi(x/k)-1 \big)(\partial^{\alpha}f)(x) \big|\\
        \end{aligned}
    \end{equation}
    En ce qui concerne le second terme, soit \( \epsilon>0\), vu que \( f\) est Schwartz, il existe \( R\) tel que
    \begin{equation}
        | x^{\gamma}(\partial^{\alpha}f)(x) |<\epsilon
    \end{equation}
    dès que \( \| x \|>R\). En prenant \( k>R\),
    \begin{equation}
        | x^{\gamma}(\partial^{\alpha}f)(x) |\begin{cases}
            =0    &   \text{si } \| x \|<R\\
            \leq \epsilon    &    \text{si } \| x \|>R\text{.}
        \end{cases}
    \end{equation}
    En ce qui concerne le premier terme,
    \begin{subequations}
        \begin{align}
            \sup_{x\in \eR^d}\big| x^{\gamma}&\sum_{\beta<\alpha}k^{-|\alpha-\beta |}(\partial^{\alpha-\beta}\phi)(x/k)(\partial^{\beta}f)(x) \Big|\\
            &\leq \frac{1}{ k }\sup_{x}\big| \sum_{\beta<\alpha}(\partial^{\alpha-\beta}\phi)(x/k)(x^{\gamma}\partial^{\beta}f)(x) \big|\\
            &= \frac{1}{ k }\sup_{x}\big| \sum_{\beta<\alpha}(\partial^{\alpha-\beta}\phi)(x/k)  p_{\beta,\gamma}(f)   \big|
        \end{align}
    \end{subequations}
    La somme ne contient qu'un nombre fini de \( \beta\) différents, donc nous pouvons considérer un nombre \( K\) qui majore tous les \( p_{\beta,\gamma}(f)\) en même temps. La partie avec \( \phi\) peut être majorée par \( \| \partial^{\alpha-\beta}\phi \|_{\infty}\) (qui est fini) dont nous pouvons prendre le maximum sur \(\beta<\alpha\). Toute l'expression dans la somme est donc majorée par un nombre qui ne dépend ni de \( x\) ni de \( \beta\). Vu que la somme est finie, elle est majorée par ce nombre multiplié par le nombre de termes dans la somme et au final
    \begin{equation}
        \sup_{x\in \eR^d}\big| x^{\gamma}\sum_{\beta<\alpha}k^{-|\alpha-\beta |}(\partial^{\alpha-\beta}\phi)(x/k)(\partial^{\beta}f)(x) \Big|\leq \frac{ K' }{ k }.
    \end{equation}
    La limite \( k\to \infty\) ne fait alors plus de doutes.
\end{proof}

\begin{remark}
    Vu la topologie de \( \swS(\eR^d)\) (définition~\ref{LEMDEFooZEFVooMMmiBr}), la convergence \( f_k\stackrel{\swS(\eR^d)}{\longrightarrow}f\) peut être exprimée par le fait que pour tout \( k,l\),
    \begin{equation}
        t^kf_n^{(l)}\stackrel{unif}{\longrightarrow}t^kf^{(l)}.
    \end{equation}
    C'est-à-dire convergence uniforme de toutes les dérivées multipliées par n'importe quel polynôme.
\end{remark}

%---------------------------------------------------------------------------------------------------------------------------
\subsection{Produit de convolution}
%---------------------------------------------------------------------------------------------------------------------------

\begin{proposition}[Stabilité de Schwartz par convolution\footnote{Définition~\ref{DEFooHHCMooHzfStu}.} \cite{CXCQJIt}]     \label{PROPooUNFYooYdbSbJ}
    Si \( \varphi\in L^1(\eR^d)\) et \( \psi\in\swS(\eR^d)\), alors \( \varphi * \psi\in \swS(\eR^d)\).
\end{proposition}

\begin{proof}
    Nous devons prouver que
    \begin{equation}
        p_{\alpha,\beta}(\varphi*\psi)=\sup_{x\in \eR^d}| x^{\beta}(\partial^{\alpha}(\varphi*\psi))(x) |
    \end{equation}
    est borné pour tout multiindices \( \alpha\) et \( \beta\). En appliquant \( | \alpha |\) fois la proposition~\ref{PropHNbdMQe}, nous mettons toutes les dérivées sur \( \psi\) : \( \partial^{\alpha}(\varphi*\psi)=(\varphi*\partial^{\alpha}\psi)\). Cela étant fait, nous majorons
    \begin{subequations}
        \begin{align}
            \big| x^{\beta}(\varphi*\partial^{\alpha}\psi)(x) \big|&\leq| x^{\beta} |\int_{\eR^d} |\varphi(y)|\underbrace{\big| (\partial^{\alpha}\psi)(x-y)\big|}_{\leq\| \partial^{\alpha}\psi \|_{\infty}} dy \big|\\
            &\leq | x^{\beta} |  \| \partial^{\alpha}\psi \|_{\infty}\int_{\eR^d}| \varphi(y) |dy\\
            &\leq p_{\alpha,\beta}(\psi)\| \varphi \|_{_{L^1}}.
        \end{align}
    \end{subequations}
    Par conséquent, \( p_{\alpha,\beta}(\varphi*\psi)\leq \| \varphi \|_{L^1}p_{\alpha,\beta}(\psi)<\infty\).
\end{proof}

%+++++++++++++++++++++++++++++++++++++++++++++++++++++++++++++++++++++++++++++++++++++++++++++++++++++++++++++++++++++++++++
\section{Théorème de Montel}
%+++++++++++++++++++++++++++++++++++++++++++++++++++++++++++++++++++++++++++++++++++++++++++++++++++++++++++++++++++++++++++

\begin{theorem}[Montel\cite{KXjFWKA}]   \label{ThoXLyCzol}
    Soient \( \Omega\) un ouvert de \( \eC\) et \( \mF\) une famille de fonctions holomorphes sur \( \Omega\), uniformément bornée sur tout compact de \( \Omega\). Alors de toute suite dans \( \mF\) nous pouvons extraire une sous-suite convergeant uniformément sur tout compact de \( \Omega\).
\end{theorem}
\index{théorème!Montel}
\index{compacité!utilisation!théorème de Montel}
\index{suite!de fonctions!théorème de Montel}
\index{fonction!holomorphe!théorème de Montel}

\begin{proof}

    \begin{subproof}
    \item[Un ensemble équicontinu]

        Nous commençons par prendre une suite de compacts dans \( \Omega\) comme dans le lemme~\ref{LemGDeZlOo}, et une suite \( \delta_n\) de réels strictement positifs tels que
        \begin{equation}
            B(z,2\delta_n)\subset K_{n+1}
        \end{equation}
        pour tout \( z\in K_n\). Soient \( x,y\in K_n\) tels que \( | x-y |<\delta_n\); nous notons \( \partial B(x,2\delta_n)\) le cercle de rayon \( 2\delta_n\) autour de \( x\), parcouru dans le sens positif. La formule de Cauchy~\ref{EqPzUABM} nous donne
        \begin{equation}
                f(x)-f(y)=\frac{1}{ 2\pi i }\int_{\partial B}\left( \frac{ f(\xi) }{ \xi-x }-\frac{ f(\xi) }{ \xi-y } \right)d\xi
                =\frac{ x-y }{ 2\pi i }\int_{\partial B}\frac{ f(\xi) }{ (\xi-x)(\xi-y) }d\xi
        \end{equation}
        Nous majorons ça par
        \begin{equation}
            \big| f(x)-f(y) \big|\leq\frac{ | x-y | }{ 2\pi }\int_{\partial B}\frac{ | f(\xi) | }{ 2\delta_n^2 }d\xi\leq \frac{ | x-y | }{ \delta_n }M_n.
        \end{equation}
        Justifications :
        \begin{itemize}
            \item
                \( | \xi-x |=2\delta_n\) et \( | \xi-y |\geq \delta_n\) parce que \( \xi\) est au mieux sur le rayon passant par \( x\) et \( y\).
            \item
                \( | f(\xi) |\leq M_n\) où \( M_n\) est la borne uniforme de \( \mF\) sur le compact \( K_n\).
            \item
                Nous avons aussi fini par calculer l'intégrale dans laquelle il ne restait plus rien, ça a donné la circonférence du cercle de rayon \( 2\delta_n\).
        \end{itemize}
        Jusqu'à présent nous avons prouvé que l'ensemble
        \begin{equation}
            \mF_n=\{ f|_{K_n}\tq f\in\mF \}
        \end{equation}
        est équicontinu. Il est aussi équiborné par hypothèse.

    \item[Application du théorème d'Ascoli]

        L'ensemble \( \mF_n\) vérifie les hypothèses du théorème d'Ascoli~\ref{ThoKRbtpah}. Donc l'ensemble \( \mF_n\) est relativement compact dans \( C(K_n,\eC)\) pour la norme uniforme. Autrement dit l'ensemble \( \bar\mF\) est compact et si nous avons une suite de fonctions dans \( \mF_n\), il existe une sous-suite convergeant dans \( \bar\mF_n\), c'est-à-dire uniformément. Autrement dit il existe une fonction strictement croissante \( \varphi\colon \eN\to \eN\) telle que la suite \( k\mapsto f_{\varphi(k)}\) converge uniformément sur \( K_n\). La limite n'est cependant pas spécialement dans \( \mF_n\).

    \item[L'argument diagonal]

        La suite \( k\mapsto f_{\varphi_1\circ\ldots\varphi_k(k)}\) converge uniformément sur tous les \( K_n\). Si \( K\) est un compact de \( \Omega\), alors les petites propriétés sympas du lemme~\ref{LemGDeZlOo} nous disent que \( K\subset \Int(K_m)\) pour un certain \( m\). Ladite suite convergeant uniformément sur \( K_m\), elle converge uniformément sur \( K\) et nous avons montré la convergence uniforme sur tout compact de \( \Omega\).

    \end{subproof}
\end{proof}

\begin{corollary}[\cite{KXjFWKA}]
    Soient \( \Omega\) un ouvert connexe borné de \( \eC\) et \( a\in \Omega\). Soit \( f\) holomorphe sur \( \Omega\) telle que \( f(a)=a\) et \( | f'(a) |<1\).

    Alors de \( (f^n)\) on peut extraire une sous-suite convergeant uniformément sur tout compact de \( \Omega\) vers la fonction constante \( a\).
\end{corollary}
\index{prolongement!analytique!utilisation}

\begin{proof}
    Nous considérons un voisinage de \( a\) inclus dans \( \Omega\); sachant que \( | f(a) |<1\), nous trouvons un voisinage encore plus petit de \( a\) sur lequel \( | f'(z) |<1\).  Soit donc \( r\) tel que \( \overline{ B(a,r) }\subset \Omega\) et tel que \( | f'(z) |<1\) sur \( \overline{ B(a,r) }\). Étant donné que \( f'(z)\) est continue sur le compact \( \overline{ B(a,r) }\), nous en prenons le maximum \( \lambda\) (qui est strictement inférieur à \( 1\)) et nous avons au final
    \begin{equation}
        | f'(z) |\leq \lambda< 1
    \end{equation}
    pour tout \( z\in \overline{ B(a,r) }\). Le théorème des accroissements finis~\ref{val_medio_2} nous dit que
    \begin{equation}
        \big| f(z)-a \big|\leq \lambda| z-a |
    \end{equation}
    pour tout \( z\in\overline{ B(a,r) }\). C'est ici que nous utilisons l'hypothèse de convexité de \( \Omega\). Nous montrons alors par récurrence que
    \begin{equation}    \label{EqIQUzKpg}
        \big| f^n(z)-a \big|\leq \lambda^n| z-a |\leq \lambda^nr\leq r.
    \end{equation}
    L'ensemble \( A=\{ f^n\tq n\geq 1 \}\) est donc uniformément borné sur \( \overline{ B(a,r) }\) par \( a+r\). Autre manière de le dire : pour tout \( z\in\overline{ B(a,r) }\) nous avons
    \begin{equation}
        f^n(z)\in\overline{ B(a,r) }.
    \end{equation}
    La suite \( (f^n)\) est donc uniformément bornée sur tout compact de \( B(a,r)\). Le théorème de Montel~\ref{ThoXLyCzol} nous indique que l'on peut extraire une sous-suite convergente uniformément sur tout compact. Au vu de \eqref{EqIQUzKpg} cette convergence ne peut avoir lieu que vers une fonction \( g\) qui vaut la constante \( a\) sur \( B(a,r)\).

    D'autre par la fonction \( g\) est holomorphe en tant que limite uniforme de fonctions holomorphes, théorème~\ref{ThoArYtQO}. Or une fonction holomorphe constante sur un ouvert est constante sur tout son domaine d'holomorphie (principe d'extension analytique, théorème~\ref{ThoAVBCewB}).
\end{proof}


%+++++++++++++++++++++++++++++++++++++++++++++++++++++++++++++++++++++++++++++++++++++++++++++++++++++++++++++++++++++++++++
\section{Espaces de Bergman}
%+++++++++++++++++++++++++++++++++++++++++++++++++++++++++++++++++++++++++++++++++++++++++++++++++++++++++++++++++++++++++++

Source : \cite{ytMOpe}.

Soit \( \Omega\) un borné dans \( \eC\) et \( D\) le disque unité ouvert de \( \eC\).

\begin{definition}
    L'\defe{espace de Bergman}{espace!de Bergman}\index{Bergman (espace)} sur \( \Omega\), noté \( A^2(\Omega)\)\nomenclature[Y]{\( A^2(\Omega)\)}{espace de Bergman} est l'espace des fonctions holomorphes sur \( \Omega\) qui sont en même temps dans \( L^2(\Omega)\).
\end{definition}
Nous mettons sur \( A^2(\Omega)\) le produit scalaire usuel hérité de \( L^2\) :
\begin{equation}
    \langle f, g\rangle =\int_{\Omega}f(z)\overline{ g(z) }dz.
\end{equation}

\begin{lemma}   \label{LemIZxKfB}
    Soient un compact \( K\subset \Omega\) et une fonction \( f\in A^2(\Omega)\). Alors
    \begin{equation}
        \max_{z\in K}| f(z) |\leq \frac{1}{ \sqrt{\pi} }\frac{1}{ d(K,\partial \Omega) }\| f \|_2.
    \end{equation}
\end{lemma}

\begin{proof}
    Soient \( a\in \Omega\) et \( r>0\) tels que \( B(a,r)\subset\Omega\). Nous considérons aussi \( \rho\leq r\). La formule de Cauchy \eqref{EqPzUABM} nous donne
    \begin{equation}
        f(a)=\frac{1}{ 2\pi i }\int_{B(a,\rho)}\frac{ f(\xi) }{ \xi-a }f\xi=\frac{1}{ 2\pi }\int_0^{2\pi}f(a+\rho e^{i\theta})d\theta
    \end{equation}
    où nous avons utilisé le chemin \( \gamma(\theta)=a+\rho e^{i\theta}\), \( \gamma'(\theta)=i\rho e^{i\theta}\) et \( \rho=| \xi-a |\). Maintenant une astuce est d'écrire
    \begin{equation}
        \frac{ r^2 }{2}f(a)=\int_0^rf(a)\rho d\rho,
    \end{equation}
    et d'y substituer la valeur de \( f(a)\) que nous venons de calculer :
    \begin{subequations}
        \begin{align}
            \frac{ r^2 }{2}f(a)&=\int_0^r\frac{1}{ 2\pi }\int_0^{2\pi}f(a+\rho e^{i\theta})d\theta\rho d\rho\\
            &=\frac{1}{ 2\pi }\int_{B(a,r)}f(z)dz   &   \text{passage aux polaires}\\
            &=\frac{1}{ 2\pi }\langle 1, f\rangle_B   &   \text{produit scalaire sur } B(a,r)\\
            &\leq\frac{1}{ 2\pi }\sqrt{\langle 1, 1\rangle_B\langle f, f\rangle_B }
        \end{align}
    \end{subequations}
    Nous avons donc
    \begin{equation}
        r^2f(a)\leq \frac{1}{ \pi }\sqrt{\langle 1, 1\rangle_B\langle f, f\rangle_B},
    \end{equation}
    et donc
    \begin{equation}
        \pi r^2 f(a)\leq \sqrt{\pi r^2}\| f \|_2,
    \end{equation}
    parce que \( \langle f, f\rangle_B\leq \| f \|_2^2\). En effet le produit scalaire \( \| . \|_2\) est donné par une intégrale sur \( \Omega\) alors que \( B(a,r)\subset \Omega\) et que la fonction qu'on y intègre est positive (c'est \( | f(z) |^2\)). En simplifiant,
    \begin{equation}
        f(a)\leq \frac{1}{ \sqrt{\pi}r }\| f \|_2.
    \end{equation}
    Mais \( r\) a été choisi pour avoir \( B(a,r)\subset\Omega\), donc \( r\leq d(a,\partial \Omega)\) et
    \begin{equation}
        | f(a) |\leq \frac{1}{ d(a,\partial\Omega)\sqrt{\pi} }\| f \|_2.
    \end{equation}

    Maintenant si nous prenons \( a\in K\), nous avons encore la minoration \( d(a,\partial K)\leq d(a,\partial \Omega)\) et donc
    \begin{equation}
        | f(a) |\leq\frac{1}{ d(a,\partial K)\sqrt{\pi} }\| f \|_2.
    \end{equation}

\end{proof}

\begin{theorem}
    Soit \( \Omega\) un ouvert de \( \eC\).
    \begin{enumerate}
        \item
            L'espace \( A^2(\Omega)\) est un espace de Hilbert.
        \item
            Si \( D\) est la boule unité dans \( \eC\), une base hilbertienne de \( A^2(D)\) est donnée par les fonctions
            \begin{equation}
                e_n(z)=\sqrt{\frac{ n+1 }{ \pi }}z^n
            \end{equation}
            pour \( n\geq 0\).
    \end{enumerate}
\end{theorem}

\begin{proof}
    Nous commençons par montrer que \( A^2(\Omega)\) est complet. Pour cela nous considérons une suite de Cauchy \( (f_n)\) dans \( A^2(\Omega)\) et un compact \( K\subset \Omega\). Nous savons par le lemme~\ref{LemIZxKfB} que
    \begin{equation}
        \max_{z\in K}\big| f_n(z)-f_m(z) \big|\leq \frac{1}{ \sqrt{\pi}d(K,\partial\Omega) }\| f_n-f_m \|_2.
    \end{equation}
    Donc \( f_n\) converge uniformément sur \( K\). Par le théorème de Weierstrass~\ref{ThoArYtQO}, la fonction \( f\) est holomorphe. Il existe donc une fonction holomorphe \( f\) qui est limite uniforme sur tout compact de \( \Omega\) de la suite \( (f_n)\).

    Mais \( L^2(\Omega)\) étant complet, la suite \( (f_n)\) a une limite \( g\in L^2(\Omega)\). Ce que nous voudrions faire est prouver que \( f=g\). Notons que tel quel, ce n'est pas vrai parce que \( f\) est une vraie fonction alors que \( g\) est une classe. Ce que nous enseigne la proposition~\ref{PropWoywYG} est qu'il existe une sous-suite (qu'on note \( (g_n)\)) qui converge vers \( g\) presque partout. Dans cette dernière phrase, \( g_n\) et \( g\) sont de vraies fonctions, des représentants des classes dans \( L^2\).

    Nous déduisons que \( f=g\) presque partout (ici \( f\) et \( g\) sont les fonctions) parce que la sous-suite converge uniformément vers \( f\) en même temps que presque partout vers \( g\). Donc \( f=g\) dans \( L^2(\Omega)\) (ici \( f\) et \( g\) sont les classes). Donc \( f\in L^2(\Omega)\) et l'espace \( A^2(\Omega)\) est de Hilbert.

    Il nous faut encore prouver que \( (e_n)_{n\geq 0}\) est une base orthonormale. En ce qui concerne les produits scalaires,
    \begin{subequations}
        \begin{align}
            \langle e_m, e_n\rangle &=\sqrt{\frac{ (m+1)(n+1) }{ \pi }}\int_Dz^n\overline{ z^m }dz\\
            &=\sqrt{\frac{ (m+1)(n+1) }{ \pi^2 }}\int_0^1\rho\,d\rho\int_0^{2\pi}d\theta \rho^{m+n} e^{i\theta(n-m)}\\
            &=\sqrt{\frac{ (m+1)(n+1) }{ \pi^2 }}\frac{1}{ m+n+2 }\underbrace{\int_{0}^{2\pi} e^{i\theta(n-m)}d\theta}_{2\pi \delta_{mn}}\\
            &=\sqrt{\frac{ (n+1)^2 }{ \pi^2 }}\frac{1}{ 2n+2 }2\pi \delta_{nm}\\
            &=\delta_{nm}.
        \end{align}
    \end{subequations}
    Donc les fonctions données sont bien orthonormales. Nous devons montrer qu'elles sont denses dans \( A^2(D)\). Soit \( f\in A^2(D)\) et \( c_n(f)=\langle f, e_n\rangle \); nous allons montrer que
    \begin{equation}
        \| f \|_2^2=\sum_{n=0}^{\infty}| \langle f, e_n\rangle  |^2,
    \end{equation}
    parce que le point~\ref{ItemQGwoIx} du théorème~\ref{ThoyAjoqP} nous indique que ce sera suffisant pour avoir une base hilbertienne.

    Étant donné que \( f\) est holomorphe sur \( D\), le théorème~\ref{ThoUHztQe} nous développe \( f\) en série entière :
    \begin{equation}    \label{EqObkbPK}
        f(z)=\sum_{k=0}^{\infty}a_kz^k.
    \end{equation}
    En permutant la somme avec le produit scalaire,
    \begin{equation}
        c_n(f)=\int_Df(z)\bar e_n(z)=\sqrt{\frac{ n+1 }{ \pi }}\int_Df(z)\bar z^ndz.
    \end{equation}
    Afin de profiter de la convergence uniforme de la série \eqref{EqObkbPK} à l'intérieur de \( D\), nous allons exprimer l'intégrale sur \( D\) comme une intégrale sur \( | z |<r\) en faisant tendre \( r\) vers \( 1\) (par le bas). Pour ce faire nous considérons les fonctions
    \begin{equation}
        g_k(z)=\begin{cases}
            f(z)\bar z^n    &   \text{si } | z |<1-1/k\\
            0    &    \text{sinon.}
        \end{cases}
    \end{equation}
    Ces fonctions sont intégrables sur \( D\) et dominées par \( f(z)\bar z^n\) qui est intégrable sans dépendre de \( k\). Mais nous avons évidemment \( g_k(z)\to f(z)\bar z^n\). Le théorème de la convergence dominée permet alors de permuter l'intégrale et la limite \( k\to \infty\). Cela nous permet d'écrire
    \begin{equation}
        c_n(f)=\sqrt{\frac{ n+1 }{ \pi }}\lim_{r\to 1^-}\int_{| z |<r}\bar z^nf(z)dz=\sqrt{\frac{ n+1 }{ \pi }}\lim_{r\to 1^-}\int_{| z |<r}\sum_{k=0}^{\infty}a_kz^k\bar z^n.
    \end{equation}
    Par la convergence uniforme de la série entière \emph{à l'intérieur} du disque \( D\) nous pouvons permuter l'intégrale et la somme (proposition~\ref{PropfeFQWr}) :
    \begin{equation}
        c_n(f)=\sqrt{\frac{ n+1 }{ \pi }}\lim_{r\to 1^-}\sum_{k=0}^{\infty}a_k\int_{| z |<r}z^k\bar z^ndz.
    \end{equation}
    L'intégrale proprement dite est vite calculée et vaut
    \begin{equation}
        \int_{| z |<1}\bar z^nz^kdz=\frac{ \pi r^{2n+2} }{ n+1 }\delta_{kn}.
    \end{equation}
    Nous pouvons donc continuer le calcul de \( c_n(f)\) en effectuant la somme sur \( k\) qui se réduit à changer \( k\) en \( n\) puis en effectuant la limite :
    \begin{equation}
        c_n(f)=\sqrt{\frac{ n+1 }{ \pi }}\lim_{r\to 1^-}\sum_ka_k\frac{ \pi r^{2n+2} }{ n+1 }\delta_{kn}=\sqrt{\frac{ \pi }{ n+1 }}a_n.
    \end{equation}

    Nous effectuons le même genre de calculs pour évaluer \( \| f \|^2_2\) :
    \begin{subequations}
        \begin{align}
            \| f \|_2^2&=\int_D| f(z) |^2dz\\
            &=\lim_{r\to 1^-}\int_{| z |<r}f(z)\sum_{k=0}^{\infty}\bar a_k\bar z_kdz\\
            &=\lim_{r\to 1^-}\sum_{k=0}^{\infty}\bar a_k\int_{| z |<r}f(z)\bar z^kdz&\text{permuter } \sum\text{ et } \int\\
            &=\lim_{r\to 1^-}\sum_{k=0}^{\infty}\bar a_ka_k\frac{ \pi r^{2k+2} }{ k+1 }&\text{intégrale déjà faite}.
        \end{align}
    \end{subequations}
    Mais nous savons déjà que \( c_n(f)=\sqrt{\pi/(n+1)}\), donc ce qui est dans la somme est \( \pi\bar a_ka_k/(n+1)=| c_k(f) |^2\). Nous avons donc
    \begin{equation}
        \| f \|^2_2=\lim_{r\to 1^-}\sum_{k=0}^{\infty}| c_k(f) |^2 r^{2k+2}.
    \end{equation}
    La fonction (de \( r\)) constante \( | c_k(f) |^2\) domine \( | c_k(f)r^{2k+2} |\) tout en ayant une somme (sur \( k\)) qui converge; en effet la proposition~\ref{PropHKqVHj} nous indique que \( \sum_j| c_k(f) |^2\leq \| f \|_2^2\). Le théorème de la convergence dominée nous permet d'inverser la limite et la somme pour obtenir le résultat attendu :
    \begin{equation}
        \| f \|_2^2=\sum_{k=0}^{\infty}| c_k(f) |^2.
    \end{equation}
\end{proof}


\chapter{Séries de Fourier}
% This is part of Mes notes de mathématique
% Copyright (c) 2011-2015,2017-2019
%   Laurent Claessens
% See the file fdl-1.3.txt for copying conditions.

%+++++++++++++++++++++++++++++++++++++++++++++++++++++++++++++++++++++++++++++++++++++++++++++++++++++++++++++++++++++++++++
\section{Densité des polynômes trigonométriques}
%+++++++++++++++++++++++++++++++++++++++++++++++++++++++++++++++++++++++++++++++++++++++++++++++++++++++++++++++++++++++++++

%---------------------------------------------------------------------------------------------------------------------------
\subsection{Convergence pour les fonctions continues (via Weierstrass)}
%---------------------------------------------------------------------------------------------------------------------------

Le résultat fondamental qui nous permet d'utiliser les polynômes trigonométriques comme base pour les fonctions \emph{continues} périodiques est le suivant. Notons que pour les fonctions non continues, il y a encore du travail.
\begin{lemma}   \label{LemXGYaRlC}
    Si \( f\colon \eR\to \eC\) est une fonction continue \( 2\pi\)-périodique et si \( \epsilon>0\), alors il existe un polynôme trigonométrique \( P\) tel que \( \| f-P \|_{\infty}\leq \epsilon\).
\end{lemma}

\begin{proof}
    Nous allons utiliser le théorème de Stone-Weierstrass~\ref{ThoWmAzSMF}. Soit le compact Hausdorff
    \begin{equation}
        S^1=\{ z\in \eC\tq | z |=1 \},
    \end{equation}
    et \( C(S^1,\eC)\) l'algèbre des fonctions continues de \( S^1\) vers \( \eC\). Il suffit de vérifier que les polynômes trigonométriques vérifient les hypothèse du théorème de Stone-Weierstrass. Un polynôme trigonométrique est un polynôme en \( z\) et \( \bar z\) défini sur \( S^1\).
    \begin{enumerate}
        \item
            Le polynôme constant est dans l'algèbre, ok.
        \item
            Pour la séparation des points, le polynôme trigonométrique \( x\mapsto  e^{ix}\).
        \item
            Si \( P\) est un polynôme en \( z\) et \( \bar z\), alors \( \bar P\) l'est encore.
    \end{enumerate}
    Donc si \( \epsilon>0\) et \( \tilde f\in C(S^1,\eC)\) sont donnés, il existe un polynôme trigonométrique \( P\) tel que
    \begin{equation}
        \sum_t| \tilde f( e^{it})-P(t) |<\epsilon.
    \end{equation}
    Soit \( f\colon \eR\to \eC\) une fonction continue \( 2\pi\)-périodique. Nous considérons \( \tilde f\in C(S^1,\eC)\) donnée par \( \tilde f( e^{it})=f(t)\). Alors \( \sup_t| f(t)-P(t) |\leq \epsilon\).
\end{proof}



%---------------------------------------------------------------------------------------------------------------------------
\subsection{Convergence pour les fonctions continues (via Fejér)}
%---------------------------------------------------------------------------------------------------------------------------
Si nous ne voulons pas passer par le gros théorème de Stone-Weierstrass pour prouver la densité des polynômes trigonométrique dans \( \big( C^0_{2\pi},\| . \|_{\infty} \big)\), nous pouvons passer par le gros théorème de Fejér. C'est ce que nous faisons maintenant.

Le \defe{noyau de Dirichlet}{noyau!Dirichlet}\index{Dirichlet!noyau} est la fonction
\begin{equation}
    D_n(t)=\sum_{k=-n}^n e^{int}.
\end{equation}
Le \defe{noyau de Fejér}{noyau!Fejér}\index{Fejér!noyau} est la moyenne de Cesaro des noyaux de Dirichlet :
\begin{equation}
    F_n(t)=\frac{1}{ n }\sum_{k=0}^{n-1}D_k(t).
\end{equation}

\begin{lemma}   \label{LemHPoIkwu}
    Le noyau de Dirichlet s'exprime sous la forme
    \begin{equation}
        D_n(t)=\sum_{k=-n}^n e^{-ikt}=\frac{ \sin\left( \frac{ 2n+1 }{ 2 }t \right) }{ \sin(t/2) }
    \end{equation}
\end{lemma}
Note : ce noyau n'est pas positif.

\begin{proof}
    Nous commençons par mettre en facteur le premier terme :
    \begin{equation}
        D_n(t)=\sum_{k=-n}^n e^{int}= e^{-int}\sum_{k=0}^{2n} e^{ikt}.
    \end{equation}
    En utilisant la formule de la somme géométrique,
    \begin{subequations}
        \begin{align}
            D_n(t)&= e^{-int}\frac{ 1-( e^{it})^{2n+1} }{ 1- e^{it} }\\
            &= e^{-int}\frac{ 1- e^{(2n+1)it} }{ 1- e^{it} }\\
            &= e^{-int}\frac{  e^{(2n+1)it/2} }{  e^{i\frac{ t }{ 2 }} }\frac{  e^{-(2n+1)it/2}- e^{(2n+1)it/2} }{  e^{-it/2}- e^{it/2} }\\
            &=\frac{ (-2i)\sin\left( \frac{ 2n+1 }{ 2 }t \right) }{ (-2i)\sin\left( \frac{ t }{2} \right) }.
        \end{align}
    \end{subequations}
\end{proof}

\begin{theorem}[Théorème de Dirichlet]\index{théorème!Dirichlet}\index{Dirichlet!théorème}
    Soit \( f\) une fonction \( 2\pi\)-périodique et \( C^1\) par morceaux. Pour tout \( x\in \eR\) nous posons
    \begin{equation}
        s_n(x)=\sum_{k=-n}^nc_k(f) e^{ikx}.
    \end{equation}
    Alors nous avons
    \begin{equation}
        \lim_{n\to \infty} s_n(x)=\frac{ f(x^+)+f(x^-) }{ 2 }.
    \end{equation}
\end{theorem}


\begin{lemma}   \label{LemtCAjJz}
    Le noyau de Fejér s'exprime sous la forme
    \begin{equation}    \label{EqLOtzCf}
        F_n(t)=\frac{1}{ n }\left( \frac{ \sin\frac{ nt }{2} }{ \sin\frac{ t }{2} } \right)^2.
    \end{equation}
\end{lemma}
Note : ce noyau est positif. C'est important parce qu'on s'en sert dans la preuve du théorème de Fejér.

\begin{proof}
    L'astuce est de noter \( \sin(x)=\Im( e^{ix})\) et de repartir du résultat à propos du noyau de Dirichlet. En utilisant encore la formule de la série géométrique partielle\footnote{Voir l'exemple \ref{ExZMhWtJS}.},
    \begin{subequations}
        \begin{align}
            F_n(t)&=\frac{1}{ n\sin(t/2) }\Im\sum_{k=0}^{n-1} e^{(2k+1)it/2}\\
            &=\frac{1}{ n\sin(t/2) }\Im e^{\frac{ it }{ 2 }}\sum_{k=0}^{n-1}\\
            &=\frac{1}{ n\sin(t/2) }\Im e^{\frac{ it }{ 2 }}\left( \frac{ 1- e^{nit} }{ 1- e^{it} } \right)\\
            &=\frac{1}{ n\sin(t/2) }\Im e^{it/2}\frac{  e^{\frac{ nit }{ 2 }}\left(  e^{-\frac{ int }{2}}- e^{\frac{ nit }{2}} \right) }{  e^{\frac{ it }{2}}\left(  e^{-it/2}- e^{it/2} \right) }\\
            &=\frac{1}{ n\sin(t/2) }\underbrace{\Im e^{\frac{ nit }{2}}}_{\sin(nt/2)}\frac{ \sin\left( \frac{ nt }{ 2 } \right) }{ \sin(\frac{ t }{2}) }\\
            &=\frac{1}{ n }\left( \frac{ \sin\frac{ nt }{2} }{ \sin\frac{ t }{2} } \right)^2.
        \end{align}
    \end{subequations}
\end{proof}


\begin{theorem}[Fejèr]      \label{ThoJFqczow}
    Soit \( f\colon \eR\to \eC\) une fonction continue et \( 2\pi\)-périodique. Pour tout \( k\in \eZ\) nous notons
    \begin{equation}
        \begin{aligned}
            e_k\colon \eR&\to \eC \\
            x&\mapsto  e^{ikx}.
        \end{aligned}
    \end{equation}
    Pour chaque \( n\in \eN\) nous posons
    \begin{subequations}
        \begin{align}
            D_n&=\sum_{k=-n}^ne_k& \tilde S_n(f)&=\sum_{k=-n}^nc_k(f)e_k\\
            F_n&=\frac{  D_0+\cdots + D_{n-1} }{ n }&  \tilde F_n=\sigma_n(f)&=\frac{1}{ n }\sum_{k=0}^{n-1}S_k(f).
        \end{align}
    \end{subequations}
    Alors
    \begin{enumerate}
        \item
            $\frac{1}{ 2\pi }\int_{-\pi}^{\pi}F_n(t)dt=1$.
        \item
            Pour tout \( \alpha\in \mathopen] 0 , \pi \mathclose[\), \( F_n\) converge uniformément vers \( 0\) sur \( \mathopen[ -\pi , \pi \mathclose]\setminus\mathopen[ -\alpha , \alpha \mathclose]\).
        \item
            La suite \( \tilde F_n \) converge uniformément sur \( \eR\) vers \( f\).
        \item   \label{ItemUNQSPmyiv}
            Le système trigonométrique \( \{ e_k \}_{k\in\eZ}\) est total pour l'espace \( \big( C^0(S^1),\| . \|_{\infty} \big)\) des fonctions continues \( 2\pi\)-périodiques.
    \end{enumerate}
\end{theorem}
\index{théorème!Fejér}

\begin{proof}
    Un calcul usuel montre que
    \begin{equation}
        \int_{-\pi}^{\pi}e_l(t)dt=\begin{cases}
            0    &   \text{si } l\neq 0\\
            2\pi    &    \text{si } l=0
        \end{cases}
    \end{equation}
    Nous avons alors
    \begin{equation}
        \frac{1}{ 2\pi }\int_{-\pi}^{\pi}F_n(t)dt=\frac{1}{ 2\pi }\frac{1}{ n }\sum_{k=0}^{n-1}\sum_{l=-k}^k\underbrace{\int_{-\pi}^{\pi}e_l(t)dt}_{2\pi\delta_l}=\frac{1}{ n }\sum_{k=0}^{n-1}1=1.
    \end{equation}
    Cela prouve déjà le premier point.

    Pour le second point, en partant de l'expression \eqref{EqLOtzCf} et en considérant \( x\in\mathopen[ -\pi, \pi ,  \mathclose]\setminus\mathopen[ -\alpha , \alpha \mathclose]\) (ce qui nous évite l'annulation du dénominateur),
    \begin{equation}
        | F_n(x) |\leq\frac{1}{ (n+1)\sin^2(\alpha/2) },
    \end{equation}
    et donc \( F_n\to 0\) uniformément sur l'ensemble considéré.

    Nous passons maintenant à cette histoire de convergence uniforme de la moyenne de Cesaro vers \( f\). Pour tout \( n\in \eN\) nous avons
    \begin{subequations}
        \begin{align}
            \tilde  D_n(x)&=\frac{1}{ 2\pi }\sum_{k=-n}^n\left( \int_{-\pi}^{\pi}f(t) e^{-ikt}dt \right) e^{ikx}\\
            &=\frac{1}{ 2\pi }\int_{-\pi}^{\pi}f(t)\sum_{k=-n}^ne_k(x-t)\\
            &=\frac{1}{ 2\pi }\int_{-\pi}^{\pi}f(t)D_k(x-t).
        \end{align}
    \end{subequations}
    Par conséquent, en effectuant le changement de variable \( u=x-t\) et la périodicité,
    \begin{subequations}    \label{EqkDsyAc}
        \begin{align}
            \tilde F_n(x)&=\int_{-\pi}^{\pi}f(t)F_n(x-t)dt\\
            &=-\int_{x+\pi}^{x-\pi}f(x-u)F_n(u)du\\
            &=\int_{-\pi}^{\pi}f(x-u) F_n(u)du.
        \end{align}
    \end{subequations}
    Nous prouvons à présent l'uniforme continuité. Soit \( \epsilon>0\); étant donné que \( f\) est continue et \( 2\pi\)-périodique, elle est uniformément continue et nous considérons \( \delta>0\) tel que \( | x-y |<\delta\) implique \( \big| f(x)-f(y) \big|<\epsilon\). Soit \( M\) un majorant de \( | f |\) sur \( \eR\). L'équation \eqref{EqkDsyAc} nous donne
    \begin{subequations}
        \begin{align}
            \big| f(x)-\tilde F_n(x) \big|&=\big| \frac{1}{ 2\pi }\int_{-\pi}^{\pi}\big( f(x-t)-f(x) \big)F_n(t)dt \big|    \label{ykuGGh}\\
            &\leq\frac{1}{ 2\pi }\int_{\delta\leq| t |\leq \pi}| 2MF_n(t) |dt+\frac{1}{ 2\pi }\int_{-\delta}^{\delta}\epsilon| F_n(t) |dt\\
            &\leq\frac{ 2M }{ 2\pi }\int_{\delta\leq | t |\leq\pi}F_n(t)dt+\epsilon'    \label{uRAMyq}
        \end{align}
    \end{subequations}
    Pour obtenir \eqref{ykuGGh} nous avons pu rentrer \( f(x)\) dans l'intégrale en utilisant le premier point. Pour obtenir \eqref{uRAMyq} nous avons d'abord utilisé la positivité de \( F_n\) (lemme~\ref{LemtCAjJz}) pour enlever les valeurs absolues, et nous avons ensuite utilisé le fait que son intégrale valait \( 2\pi\).

    Étant donné que \( F_n\to 0\) uniformément sur \( \mathopen[ -\pi,\pi ,  \mathclose]\setminus\mathopen[ -\alpha , \alpha \mathclose]\), il existe un \( N\) tel que
    \begin{equation}
        \int_{\delta\leq| t |\leq \pi}F_n(t)dt\leq \epsilon
    \end{equation}
    dès que \( n>N\). Le résultat découle.

    Pour le point~\ref{ItemUNQSPmyiv}, il suffit de remarquer que chacun des \( \tilde F_n\) est une combinaison finie d'éléments du système trigonométrique.
\end{proof}

%---------------------------------------------------------------------------------------------------------------------------
\subsection{Densité dans \texorpdfstring{$ L^p$}{Lp}}
%---------------------------------------------------------------------------------------------------------------------------

Nous venons de voir (de deux façons différentes) que les polynômes trigonométriques étaient dense dans \( \big( C^0_{2\pi}(\eR),\| . \|_{\infty} \big)\). Nous avons aussi déjà vu par le théorème~\ref{ThoQGPSSJq} que ces polynômes trigonométriques étaient denses dans \( L^p(S^1)\). Nous présentons à présent une autre façon de prouver cette dernière densité.

\begin{theorem}     \label{ThoDPTwimI}
    Les polynômes trigonométriques sont denses dans \( L^p(S^1)\) pour \( 1\leq p <\infty\).
\end{theorem}

\begin{proof}
    Par les théorèmes~\ref{LemXGYaRlC} ou~\ref{ThoJFqczow} (au choix), nous savons que les polynômes trigonométriques sont denses dans \( \big( C^0_{2\pi}(S^1),\| . \|_{\infty} \big)\). Vu que \( S^1\) est compact, la densité est également au sens \( L^p\). En effet si \( \| f_n-f \|_{\infty}\leq \epsilon\), alors
    \begin{equation}
        \| f_n-f \|_{\infty}=\int_0^{2\pi}| f_n-f |^p\leq\int_0^{2\pi}\epsilon^p=2\pi\epsilon^p.
    \end{equation}
    Donc les polynômes trigonométriques sont denses dans \( \big( C^0_{2\pi}(S^1),\| . \|_p \big)\). Mais nous savons par (un a fortiori sur) le théorème~\ref{ThoILGYXhX} que les fonctions continues sont denses dans \( L^p(S^1)\).

    Par densité de la densité, les polynômes trigonométriques sont denses dans \( L^p(S^1)\).
\end{proof}

%---------------------------------------------------------------------------------------------------------------------------
\subsection{Suite équirépartie, critère de Weyl}
%---------------------------------------------------------------------------------------------------------------------------

\begin{definition}
    Soit \( u\) une suite dans \( \mathopen[ 0 , 1 \mathclose]\). Pour \( 0\leq a\leq b\leq 1\) nous posons
    \begin{equation}
        X_n(a,b)=\Card\big\{  k\in\{ 1,\ldots, n \}\tq u_k\in\mathopen[ a , b \mathclose] \big\}.
    \end{equation}
    Nous disons que la suite \( u\) est \defe{équirépartie}{suite!équirépartie} si pour tout \( 0\leq a<b<1\), on a
    \begin{equation}
        \lim_{n\to \infty} \frac{ X_n(a,b) }{ n }=b-a.
    \end{equation}
\end{definition}
Voir aussi la remarque~\ref{RemUXAkcuH} sur les nombres normaux.

\begin{proposition}[Critère de Weyl\cite{ytMOpe,KXjFWKA}]  \label{PropDMvPDc}
    Soit \( (x_n)\) une suite dans \( \mathopen[ 0 , 1 [\). Les conditions suivantes sont équivalentes.
    \begin{enumerate}
        \item   \label{ItemKWcZTHqi}
            La suite \( (x_n)\) est équirépartie.
        \item\label{ItemKWcZTHqii}
            Pour toute fonction continue à valeurs réelles sur \( \mathopen[ 0 , 1 \mathclose]\),
            \begin{equation}    \label{EqBSqdjpn}
                \lim_{n\to \infty} \frac{1}{ n }\sum_{k=1}^nf(x_k)=\int_0^1f(x)dx.
            \end{equation}
        \item\label{ItemKWcZTHqiii}
            Pour tout \( p\in\eN^*\) nous avons
            \begin{equation}
                \lim_{n\to \infty} \frac{1}{ n }\sum_{k=1}^n e^{2i\pi px_k}=0.
            \end{equation}
    \end{enumerate}
\end{proposition}
\index{convergence!suite numérique}
\index{intégrale!calcul}
\index{densité!dans un espace de fonction!critère de Weyl}
\index{suite!équirépartie!critère de Weyl}

\begin{proof}
    On pose
    \begin{equation}
        S_n(f)=\frac{1}{ n }\sum_{k=1}^nf(x_k).
    \end{equation}


    \begin{subproof}
    \item[Une espèce de lemme]

        Supposons connaitre un ensemble de fonctions \( A\) dense dans \( C^0(\mathopen[ 0 , 1 \mathclose])\) pour toutes les fonctions duquel nous avons la limite \eqref{EqBSqdjpn}. Alors la limite a lieu pour toute fonction de \( C^0(\mathopen[ 0 , 1 \mathclose])\). En effet, soit \( f\in C^0(\mathopen[ 0 , 1 \mathclose])\) et \( g\in A\) tel que \( \| f-g \|_{\infty}<\epsilon\). Alors
        \begin{subequations}
            \begin{align}
                \left\|   \frac{1}{ n }\sum_{k=1}^nf(x_k)-\int_0^1f(t)dt  \right\|&\leq \left\| \frac{1}{ n }\sum_{k=1}^n\big( f(x_k)-g(x_k)\big) \right\|\\
                &\quad+ \left\| \frac{1}{ n }\sum_{k=1}^n  g(x_k)-\int_0^1g(t)dt   \right\|\\
                &\quad+ \left\| \int_0^1g(t)dt-\int_0^1f(t)dt \right\|.
            \end{align}
        \end{subequations}
        Le premier terme se majore par \( \epsilon\). Le troisième est la même majoration : \( \int_0^1\big(  f(t)-g(t)\big)dt\leq \| f-g \|_{\infty}=\epsilon\). Par hypothèse sur l'espace \( A\), le second terme se majore par \( \epsilon\) lorsque \( n\) est grand.


    \item[\ref{ItemKWcZTHqi}\( \Rightarrow\)\ref{ItemKWcZTHqii}]
    Nous supposons que la suite est équirépartie et nous commençons par montrer le résultat pour les fonctions en escalier. Soit donc la fonction en escalier \( \eta(x)=c_j\) sur \( a_{j-1}< x<a_j\). Sur le point \( a_j\) lui-même, la fonction \( \eta\) vaut soit \( c_j\) soit \( c_{j+1}\). Nous avons
    \begin{equation}    \label{EqohMuel}
        \frac{1}{ n }\sum_{k=1}^n\eta(x_k)=\frac{1}{ n }\left[  \sum_{j=1}^mc_jX_n(a_j,a_{j+1})-\sum_{j=1}^mc_jX_n(a_j,a_j)+\sum_{j=1}^m\eta(a_j)X_n(a_j,a_j) \right].
    \end{equation}
    À la limite \( n\to\infty\), les deux derniers termes tombent\quext{J'en profite pour mentionner que mon équation \eqref{EqohMuel} n'est pas la même que celle de \cite{ytMOpe} dans laquelle il me semble voir une faute; quoi qu'il en soit, les termes litigieux tombent.} et il reste
    \begin{equation}
        \lim_{n\to \infty} \frac{1}{ n }\sum_{k=1}^n\eta(x_k)=\sum_{j=1}^mc_j(a_{j-1}-a_j).
    \end{equation}
    Or par construction, pour une fonction en escalier,
    \begin{equation}
        \sum_{j=1}^mc_j(a_{j-1}-a_j)=\int_0^1\eta.
    \end{equation}

    Étant donné que les fonctions en escalier sont denses dans les fonctions continues, l'espèce de lemme plus haut conclut.

    \item[\ref{ItemKWcZTHqii}\( \Rightarrow\)\ref{ItemKWcZTHqi}]
    Nous prouvons maintenant le sens inverse. C'est-à-dire que pour toute fonction continue sur \( \mathopen[ 0 , 1 \mathclose]\), nous avons
    \begin{equation}
        \int_0^1f(x)dx=\lim_{n\to \infty} \frac{1}{ n }\sum_{k=1}^nf(x_k).
    \end{equation}
    Nous devons en déduire que \( (x_n)\) est équirépartie. Pour ce faire, soit \( x\in \mathopen[ 0 , 1 [\) et \( \epsilon>0\) tel que \( x+\epsilon<1\). Nous considérons \( \varphi=\mtu_{\mathopen[ x , 1 [}\) et
    \begin{equation}
        \varphi_{\epsilon(t)}=\begin{cases}
            0    &   \text{si } t\in\mathopen[ 0 , x [\\
            \frac{ t-x }{ \epsilon }    &   \text{si } t\in \mathopen[ x , x+\epsilon [\\
            1    &    \text{si } t\geq x+\epsilon.
        \end{cases}
    \end{equation}
    Cela est une fonction continue, donc
    \begin{equation}
        \lim_{n\to \infty} S_n\big( \varphi_{\epsilon}(t) \big)=\int_0^1\varphi_{\epsilon}(t)dt=\int_{x}^{x+\epsilon}\frac{ t-x }{ \epsilon }dt+\int_{x+\epsilon}^11dt=1-x-\frac{ \epsilon }{2}.
    \end{equation}
    Mais \( \varphi_{\epsilon}\leq \varphi\), donc \( S_n(\varphi_{\epsilon})\leq S_n(\varphi)\) et donc
    \begin{equation}
        \liminf_{n\to \infty}S_n(\varphi)\geq 1-x.
    \end{equation}
    Notons que nous ne savons pas si la \emph{vraie} limite de gauche existe; c'est pourquoi nous prenons la limite inférieure, qui existe toujours.

    Nous définissons aussi
    \begin{equation}
        \psi_{\epsilon}(t)=\begin{cases}
            0    &   \text{si } t\in \mathopen[ 0 , x-\epsilon [\\
            \frac{ t-x+\epsilon }{ \epsilon }    &   \text{si } t\in\mathopen[ x-\epsilon , x [\\
            1    &    \text{si } t>x.
        \end{cases}
    \end{equation}
    C'est encore une fonction continue et nous trouvons\footnote{Je recommande chaudement de dessiner les fonctions \( \varphi_{\epsilon}\) et \( \psi_{\epsilon}\) pour avoir une idée de la situation.}
    \begin{equation}
        \int_0^1\psi_{\epsilon}(t)dt=1-x+\frac{ \epsilon }{2}.
    \end{equation}
    Vu que \( \psi_{\epsilon}\geq\varphi\), nous avons \( S_n(\psi_{\epsilon})\geq S_n(\varphi)\) et donc
    \begin{equation}
        \limsup_{n}S_n(\varphi)\leq 1-x.
    \end{equation}
    Nous avons déjà obtenu que
    \begin{equation}
        1-x\leq\liminf S_n(\varphi)\leq \limsup S_n(\varphi)\leq 1-x,
    \end{equation}
    donc la limite existe et vaut
    \begin{equation}
        \lim_{n\to \infty} S_n(\varphi)=1-x.
    \end{equation}
    Cela est pour la fonction caractéristique \( \varphi=\mtu_{\mathopen[ x , 1 [}\). Si nous prenons une fonction caractéristique \( \mtu_{\mathopen[ a , b \mathclose]}\), nous avons la même chose parce que \( \mtu_{\mathopen[ a , b [}\) est une combinaisons linéaire de fonctions du type \( \mtu_{\mathopen[ x , 1 [}\).

    Nous avons donc
    \begin{equation}
        \lim_{n\to \infty} S_n\big( \mtu_{\mathopen[ a , b \mathclose]} \big)=b-a,
    \end{equation}
    alors que le membre de gauche n'est autre que
    \begin{equation}
        S_n\big( \mtu_{\mathopen[ a , b \mathclose]} \big)=\frac{1}{ n }\sum_{k=1}^n\mtu_{\mathopen[ a , b \mathclose]}(x_k)=\frac{1}{ n }N(n,a,b).
    \end{equation}

    \item[\ref{ItemKWcZTHqii}\( \Rightarrow\)\ref{ItemKWcZTHqiii}]

        Vu que \(  e^{2i\pi px_k}=\cos(2\pi px_k)+\sin(2\pi ix_k)\) est une fonction périodique, c'est immédiat.
    \item[\ref{ItemKWcZTHqiii}\( \Rightarrow\)\ref{ItemKWcZTHqii}]
        Par linéarité, le point~\ref{ItemKWcZTHqii} montre que si \( f\) est un polynôme trigonométrique, alors
        \begin{equation}
            \lim_{n\to \infty} \frac{1}{ n }\sum_{k=1}^nf(x_k)=\int_0^1f(t)dt.
        \end{equation}
    \item[Densité des polynômes trigonométriques]

        Il nous reste à prouver que les polynômes trigonométriques sont denses dans les fonctions continues sur \( \mathopen[ 0 , 1 \mathclose]\). Soit une fonction continue sur \( \mathopen[ 0 , 1 \mathclose]\) avec \( f(0)=f(1)\). Alors le théorème de Stone-Weierstrass dans sa version trigonométrique (lemme~\ref{LemXGYaRlC}) nous donne la densité.

        Si \( f(1)\neq f(0)\) c'est pas très grave : on peut trouver une fonction \( g\) vérifiant \( g(0)=g(1) \) et \( \| f-g \|_{\infty}\leq \epsilon\). Ensuite un polynôme trigonométrique approxime très bien \( g\).
        .
    \end{subproof}
\end{proof}

%+++++++++++++++++++++++++++++++++++++++++++++++++++++++++++++++++++++++++++++++++++++++++++++++++++++++++++++++++++++++++++
\section{Fonctions de Dirichlet}
%+++++++++++++++++++++++++++++++++++++++++++++++++++++++++++++++++++++++++++++++++++++++++++++++++++++++++++++++++++++++++++

\begin{definition}
    Une fonction \( f\colon \eR\to \eC\) est une \defe{fonction de Dirichlet}{fonction!de Dirichlet} si
    \begin{enumerate}
        \item
            elle est \( 2\pi\)-périodique,
        \item
            elle est continue par morceaux,
        \item
            pour tout \( x\in \eR\) nous avons
            \begin{equation}
                f(x)=\frac{ f(x^+)+f(x^-) }{2}.
            \end{equation}
    \end{enumerate}
    Nous notons \( \mD\) l'ensemble des fonctions de Dirichlet.
\end{definition}

\begin{lemma}[\cite{NJsYInp}]   \label{LemVIwMsTC}
    L'ensemble \( C^0(S^1)\) est dense dans \( \big( \mD,\| . \|_2 \big)\).
\end{lemma}

\begin{proof}
    Nous commençons par supposer que \( f\in\mD\) n'ait qu'un seul point de discontinuité, \( x_0\). Alors nous considérons la fonction \( f_n\) qui est égale à \( f\) sur \( S^1\setminus B(x_0,\frac{1}{ n })\) et qui sur \( B(x_n,\frac{1}{ n })\) est le segment de droite joignant \( f(x_0-\frac{1}{ n })\) et \( f(x_0+\frac{1}{ n })\). Cela est une fonction continue, et de plus nous avons
    \begin{equation}
        | f_n(x) |\leq \| f \|_{\infty}
    \end{equation}
    pour tout \( x\). En effet si \( x\) est en dehors de \( B(x_0,\frac{1}{ n })\) c'est évident, et si \( x\in B(x_0,\frac{1}{ n })\), alors \( | f_n(x) |\) est majoré soit par \( f(x_0-\frac{1}{ n })\) soit par \( f(x_0+\frac{1}{ n })\) suivant que le raccord affin soit croissant ou décroissant. Avec ça nous avons
    \begin{equation}
        \| f_n-f \|_2^2=\int_{x_0-1/n}^{x_0+1/n}| f(x)-f_n(x) |^2dx\leq \int_{x_0-1/n}^{x_0+1/n}4\| f \|_{\infty}=\frac{ 8\| f \|_{\infty} }{ n }.
    \end{equation}
    Et nous voyons que \( \| f_n-f \|_2\to 0\).

    Si \( f\) contient plusieurs points de continuité, on fait le même coup autour de chaque point, en prenant \( n\) assez grand pour que si \( x_0\) est un point de discontinuité, \( B(x_0,\frac{1}{ n })\) n'en contienne pas d'autres.
\end{proof}

Notons que la densité de \( C^0(S^1)\) dans \( \big( \mD,\| . \|_{\infty} \big)\) est impossible parce qu'une limite uniforme de fonctions continues est continue.

\begin{theorem}
    Le système trigonométrique \( \{ e_n \}_{n\in \eZ}\) est total dans \( \big( \mD,\| . \|_2 \big)\).
\end{theorem}

\begin{proof}
    Soit \( f\in\mD\). Si elle est continue, le théorème de Fejèr~\ref{ThoJFqczow} nous donne convergence uniforme sur \( S^1\) d'une suite de polynômes trigonométriques vers \( f\). Cette convergence est également une convergence \( L^2\) parce que \( S^1\) est compact.

    Prenons donc \( f\in \mD\) non continue et \( \epsilon>0\)\footnote{Par exemple \( \epsilon=0.4\), mais ce n'est qu'un exemple hein. Si vous en voulez un autre, prenez \( p\), un nombre premier puis calculez \( \epsilon=1/p\).}. Par le lemme~\ref{LemVIwMsTC}, il existe une fonction \( g\in C^0(S^1)\) telle que
    \begin{equation}
        \| g-f \|_2\leq \epsilon.
    \end{equation}
    Le théorème de Fejèr donne aussi un polynôme trigonométrique \( P\) tel que \( \| P-g \|_2<\epsilon\); nous avons alors
    \begin{equation}
        \| P-f \|_2\leq \| P-g \|_{2}+\| g-f \|_2\leq 2\epsilon.
    \end{equation}
\end{proof}

Notons que cette histoire de fonctions de Dirichlet n'a pas attaquée le vrai fond du problème de la densité des polynômes trigonométriques dans \(  L^2(S^1)\) parce que nous restons avec une hypothèse de continuité, alors que les représentants des éléments de \( L^2(S^1)\) n'ont strictement aucune régularité à priori.

%+++++++++++++++++++++++++++++++++++++++++++++++++++++++++++++++++++++++++++++++++++++++++++++++++++++++++++++++++++++++++++
\section{Coefficients et série de Fourier}
%+++++++++++++++++++++++++++++++++++++++++++++++++++++++++++++++++++++++++++++++++++++++++++++++++++++++++++++++++++++++++++

\begin{definition}
    La \defe{série de Fourier}{série!de Fourier} associée à \( f\) est
    \begin{equation}
        f(x)\sim\sum_{n=-\infty}^{\infty}c_n(f) e^{2\pi i\frac{ n }{ T }x}.
    \end{equation}
\end{definition}
Cette expression est pour l'instant purement formelle. Cela ne présume ni de la convergence de la série, ni, au cas où elle serait convergente, que la limite soit \( f\).

Pour la suite nous allons considérer des fonctions périodiques de période \( 2\pi\), et les coefficients de Fourier de \( f\) (quand ils existent) sont alors
\begin{equation}    \label{EqNDBaXRL}
    c_n(f)=\frac{1}{ 2\pi }\int_0^{2\pi}f(t) e^{-int}dt
\end{equation}

\begin{proposition}[\cite{DupFourEsdgKEI}]  \label{PropmrLfGt}
    Soit \( f\) une fonction continue et périodique telle que sa série de Fourier converge uniformément. Alors la convergence est vers \( f\).
\end{proposition}
%TODO : ajouter ce théorème à Wikipédia, et le lier dans l'article sur la formule sommatoire de Poisson.

\begin{proof}
    Notons d'abord que \( f\) étant continue sur \(\mathopen[ 0 , 2\pi \mathclose]\), elle y est bornée et \( L^2\). Par conséquent Parseval nous enseigne que
    \begin{equation}
        \| S_N(f)-f \|_{L^2}\to 0.
    \end{equation}
    Cela signifie que
    \begin{equation}
        \lim_{N\to \infty} \frac{1}{ 2\pi }\int_{0}^{2\pi}| f(t)-S_N(t) |^2dt=0.
    \end{equation}
    L'hypothèse de convergence uniforme nous dit que la fonction \( | f(t)-S_N(t) |^2\) converge uniformément vers la fonction \( | f(t)-S(t) |^2\) où nous avons écrit \( S\) la limite de \( S_N\). En permutant la limite et l'intégrale,
    \begin{equation}
        \frac{1}{ 2\pi }\int_0^{2\pi}| f(t)-S(t) |^2dt=0,
    \end{equation}
    ce qui signifie que la fonction \( t\mapsto | f(t)-S(t) |^2\) est la fonction nulle. Nous en déduisons que \( f=S\).
\end{proof}

\begin{proposition}     \label{PropSgvPab}
    Soit \( f\) une fonction \( 2\pi\)-périodique. Si \( \sum_{n\in \eZ}| c_n(f) |<\infty\), alors pour tout \( x\in \eR\) nous avons
    \begin{equation}
        f(x)=\sum_{n\in \eZ}c_n(f) e^{inx}.
    \end{equation}
    De plus, la suite \( (S_nf)\) converge uniformément vers \( f\).
\end{proposition}

\begin{proof}
    Nous posons
    \begin{equation}
        g(x)=\sum_{n\in \eZ}c_n(f) e^{inx}.
    \end{equation}
    Étant donné les hypothèses, la série de droite converge absolument, la fonction \( g\) est continue sur \( \eR\). Nous avons
    \begin{equation}
        \big| g(x)-(S_nf)(x) \big|\leq \sum_{| k |> n}| c_k(f) |,
    \end{equation}
    mais le terme de droite tend vers zéro lorsque \( n\to \infty\) parce que c'est le reste d'une série convergente. Cela signifie que \( S_nf\) converge uniformément vers \( g\).

    Par ailleurs nous savons que dans \( L^2\) nous avons la convergence \( S_nf\to f\) (parce que \( f\) est continue sur le compact \( \mathopen[ 0 , 2\pi \mathclose]\) et donc y est bornée et \( L^2\)), ce qui signifie que \( g=f\) presque partout au sens \( L^2\). Ces deux fonctions étant continues, elles sont égales partout.
\end{proof}

\begin{theorem}     \label{ThozHXraQ}
    Soit \( f\), une fonction \( C^1\) et \( 2\pi\)-périodique. Nous notons \( (c_n)_{n\in \eZ}\) la suite de ses coefficients de Fourier. Alors \( (c_n)\in \ell^1(\eZ)\) et pour tout \( x\in \eR\) nous avons
    \begin{equation}
        f(x)=\sum_{n\in \eZ}c_n(f) e^{inx}.
    \end{equation}
\end{theorem}

\begin{proof}
    Soit \( n\in \eZ\). Nous posons \( g(t)=f(t) e^{-int}\). Nous avons
    \begin{equation}
        0=g(2\pi)-g(0)=\int_0^{2\pi}g'(t)dt=\int_0^{2\pi}\big[ f'(t) e^{-int}-inf(t) e^{-int} \big].
    \end{equation}
    Du coup, \( c_n(f')=inc_n(f)\). La fonction \( f'\) étant bornée (parce que continue sur \( \mathopen[ 0 , 2\pi \mathclose]\)), elle est de carré intégrable sur \( \mathopen[ 0 , 2\pi \mathclose]\) et par les inégalités de Parseval (théorème~\ref{ThoyAjoqP}) nous avons
    \begin{equation}
        \sum_{n\in \eZ}| c_n(f') |^2<\infty.
    \end{equation}
    Par conséquent \( (c_n(f'))\in \ell^2(\eZ)\) et a fortiori \( (c_n(f'))_{n\in \eN}\in \ell^2(\eN)\). L'inégalité de Cauchy-Schwarz nous indique alors
    \begin{equation}
        \sum_{n\in \eN}| c_n(f) |=\sum_{n\in \eN}\frac{1}{ n }| c_n(f') |\leq \left( \sum_n\frac{1}{ n^2 } \right)^{1/2}\left( \sum_{n}| c_n(f') |^2 \right)^{1/2}<\infty.
    \end{equation}
    Nous procédons de même pour \( n<0\). Cela prouve que
    \begin{equation}
        \sum_{n\in \eZ}| c_n(f) |<\infty.
    \end{equation}
\end{proof}

\begin{corollary}   \label{CordgtXlC}
    Soient \( f,g\) deux fonctions continues et \( 2\pi\)-périodiques. Si \( c_n(f)=c_n(g)\) alors \( f=g\).
\end{corollary}

\begin{proof}
    Dans le cas de fonctions continues, le théorème de Fejér nous enseigne que si nous posons
    \begin{equation}
        S_n(x)=\sum_{k=-n}^{n}c_k(f) e^{ikx}
    \end{equation}
    alors nous avons la convergence
    \begin{equation}
        \frac{1}{ N+1 }\sum_{n=0}^NS_n(f)(x)\to f(x).
    \end{equation}
    C'est-à-dire qu'une fonction continue est déterminée par ses coefficients de Fourier.
\end{proof}

\begin{example}
    Considérons la fonction
    \begin{equation}
        f(x)=1-\frac{ x^2 }{ \pi^2 }
    \end{equation}
    sur \( \mathopen[ -\pi , \pi \mathclose]\). Nous la développons en série trigonométrique, et étant paire il n'y a pas de sinus. Un calcul montre que
    \begin{equation}
        a_0=\frac{ 4 }{ 3 }
    \end{equation}
    et
    \begin{equation}
        a_n=(-1)^{n+1}\frac{ 4 }{ n^2\pi^2 },
    \end{equation}
    de telle sorte que
    \begin{equation}
        f(x)=\frac{ 2 }{ 3 }-\frac{ 4 }{ \pi^2 }\sum_{n=1}^{\infty}(-1)^n\frac{ \cos(nx) }{ n^2 }.
    \end{equation}
    Nous avons \( f(\pi)=0\), mais vu le développement,
    \begin{equation}
        f(\pi)=\frac{ 2 }{ 3 }-\frac{ 4 }{ \pi^2 }\sum_{n=1}^{\infty}\frac{1}{ n^2 },
    \end{equation}
    donc
    \begin{equation}
        \sum_{n=1}^{\infty}\frac{1}{ n^2 }=\frac{ \pi^2 }{ 6 }.
    \end{equation}
\end{example}

%---------------------------------------------------------------------------------------------------------------------------
\subsection{Le contre-exemple que nous attendions tous}
%---------------------------------------------------------------------------------------------------------------------------

Nous montrons maintenant que la continuité et la périodicité ne sont pas suffisantes pour avoir convergence de la série de Fourier.

\begin{proposition}[\cite{KXjFWKA}] \label{PropREkHdol}
    Soit \( C^0_{2\pi}\) l'ensemble des fonctions continues muni de la norme uniforme. Nous définissons
    \begin{equation}
        S_n(f)(x)=\sum_{k=-n}^nc_k(f) e^{ikx}.
    \end{equation}
    Alors il existe \( f\in C^0_{2\pi}\) tel que la suite \(n\mapsto S_n(f)(0)\) soit divergente. En particulier \( f\) n'est pas la somme de sa série de Fourier.
\end{proposition}

\begin{proof}
    Nous considérons la forme linéaire
    \begin{equation}
        \begin{aligned}
            l_n\colon C^0_{2\pi}&\to \eC \\
            f&\mapsto S_n(f)(0)=\sum_{k=-n}^nc_k(f).
        \end{aligned}
    \end{equation}
    \begin{subproof}
        \item[La forme est continue]

            Nous montrons d'abord que \( \| l_n \|\) est continue en montrant que \( \| l_n \|<\infty\) et en utilisant la proposition~\ref{PROPooQZYVooYJVlBd}. Pour cela nous calculons un peu :
            \begin{equation}    \label{EqBELHGya}
                l_n(f)=\sum_{k=-n}^n\frac{1}{ 2\pi }\int_{-\pi}^{\pi}f(t) e^{-ikt}dt=\frac{1}{ 2\pi }\int_{-\pi}^{\pi}f(t)\sum_{k=-n}^n e^{-ikt}dt=\frac{1}{ 2\pi }\int_{-\pi}^{\pi}f(t)D_n(t)dt
            \end{equation}
            où \( D_n(t)\) est le noyaux de Dirichlet dont nous savons une formule par le lemme~\ref{LemHPoIkwu}. Nous avons donc
            \begin{equation}
                | l_n(f) |\leq \frac{1}{ 2\pi }\int_{-\pi}^{\pi}| D_n(t) |\| f \|_{\infty}dt.
            \end{equation}
            En prenant \( \| f \|_{\infty}=1\) nous avons la borne suivante pour la norme de \( l_n\) :
            \begin{equation}        \label{EqBXoIUiD}
                \| l_n \|\leq \frac{1}{ 2n }\int_{-\pi}^{\pi}| D_n(t) |dt<\infty.
            \end{equation}
            Notons que la convergence de l'intégrale vient de la continuité de la fonction
            \begin{equation}
                t\mapsto \frac{ \sin\left( \frac{ 2n+1 }{2}t \right) }{ \sin\left( \frac{ t }{ 2 } \right) }
            \end{equation}
            qui, elle même, se prouve avec une règle de l'Hospital :
            \begin{equation}
                \lim_{t\to 0} \frac{ \sin(at) }{ \sin(t) }=\lim_{t\to 0} \frac{ a\cos(at) }{ \cos(t) }=a.
            \end{equation}
            Donc \( D_n(t)\) a une limite bien définie pour \( t\to 0\) et est alors une fonction continue sur le compact \( \mathopen[ -\pi , \pi \mathclose]\).

        \item[La norme de \( l_n\) (début)]

            Nous avons prouvé que \( \| l_n \|\leq \frac{1}{ 2\pi }\int_{-\pi}^{\pi}| D_n(t) |dt\). Nous allons à présent prouver que cela est effectivement la norme de \( l_n\). Pour \( \epsilon>0\) nous considérons la fonction
            \begin{equation}
                \begin{aligned}
                    f_{\epsilon}\colon \eR&\to \eC \\
                    x&\mapsto \frac{ D_n(x) }{ | D_n(x) |+\epsilon }.
                \end{aligned}
            \end{equation}
            C'est une fonction continue et \( 2\pi\)-périodique satisfaisant \( \| f_{\epsilon} \|\leq 1\) parce que le dénominateur est toujours plus grand que le numérateur. Nous nous proposons de calculer
            \begin{equation}
                l_n(f_{\epsilon})=\sum_{k=-n}^n\frac{1}{ 2\pi }\int_{-\pi}^{\pi}f_{\epsilon}(t) e^{-ikt}dt.
            \end{equation}
            Vu que \( f_{\epsilon}(t) e^{-ikt}\) vaut en norme \( | f_{\epsilon}(t) |\) qui est une fonction intégrable (ne dépendant pas de \( k\)) sur \( \mathopen[ -\pi , \pi \mathclose]\), le théorème de la convergence dominée~\ref{ThoConvDomLebVdhsTf} nous permet de permuter la somme et l'intégrale :
            \begin{equation}
                l_n(f_{\epsilon})=\frac{1}{ 2\pi }\int_{-\pi}^{\pi}\frac{ D_n(t) }{ | D_n(t) |+\epsilon }\underbrace{\sum_{k=-n}^n e^{-ikt}}_{=D_n(t)}dt=\frac{1}{ 2\pi }\int_{-\pi}^{\pi}\frac{ \big| D_n(t) \big|^2 }{ | D_n(t) |+\epsilon }dt.
            \end{equation}
            Nous avons donc
            \begin{equation}
                \lim_{\epsilon\to 0}l_n(f_{\epsilon})=\frac{1}{ 2\pi }\int_{-\pi}^{\pi}| D_n(t) |dt.
            \end{equation}
            Mais vue l'inégalité \eqref{EqBXoIUiD} nous avons
            \begin{equation}
                \| l_n \|=\frac{1}{ 2\pi }\int_{-\pi}^{\pi}| D_n(t) |dt.
            \end{equation}
            Notre tâche est maintenant de donner une valeur à cette intégrale.

        \item[Norme de \( l_n\) tend vers \( \infty\)]
            D'abord nous écrivons
            \begin{equation}
                \| l_n \|=\frac{1}{ 2\pi }\int_{-\pi}^{\pi}\frac{ \left| \sin\left( \frac{ 2n+1 }{2}t \right) \right|  }{ \big| \sin(t/2) \big| }dt,
            \end{equation}
            ensuite nous nous souvenons que \( | \sin(x) |\leq | x |\) pour tout \( x\), ce qui nous permet de changer le dénominateur :
            \begin{equation}
                \| l_n \|\geq \frac{ 2 }{ \pi }\int_0^{\pi}\frac{ \left| \sin\left( \frac{ 2n+1 }{2}t \right) \right|  }{ | t | }dt
            \end{equation}
            Nous y effectuons le changement de variable \( u=\frac{ 2n+1 }{2}t\) qui donne
            \begin{equation}
                \| l_n \|\geq \frac{ 2 }{ \pi }\int_{0}^{(n+\frac{ 1 }{2})\pi}\frac{ \big| \sin(u) \big| }{ | u | }.
            \end{equation}
            Nous y reconnaissons l'intégrale \eqref{EqKNOmLEd} du sinus cardinal que nous savons diverger. Cela donne
            \begin{equation}
                \lim_{n\to \infty} \| l_n \|=\infty.
            \end{equation}
        \item[La conclusion]

            L'espace \( \big( C^0_{2\pi},\| . \|_{\infty} \big)\) est complet\footnote{Parce qu'une limite uniforme de fonctions continues est continue, théorème~\ref{ThoUnigCvCont}.}, donc le théorème de Banach-Steinhaus~\ref{ThoPFBMHBN} s'applique. Par rapport aux notations de l'énoncé de Banch-Steinhaus, nous posons
            \begin{subequations}
                \begin{align}
                    E=\big( C^0_{2\pi},\| . \|_{\infty} \big)\\
                    F=\eR\\
                    H=\{ l_n \}_{n\in \eN}.
                \end{align}
            \end{subequations}
            Vu que la suite \( (\| l_n \|)\) n'est pas bornée, il existe \( f\in C^0_{2\pi}\) tel que
            \begin{equation}
                \sup_n\| l_n(f) \|=\infty.
            \end{equation}
            Pour cette fonction nous avons
            \begin{equation}
                \sup_{n\geq 0}S_n(f)(0)=\infty,
            \end{equation}
            et donc la série de Fourier de \( f\) ne converge pas en zéro.

        \end{subproof}
\end{proof}


%---------------------------------------------------------------------------------------------------------------------------
\subsection{Inégalité isopérimétrique}
%---------------------------------------------------------------------------------------------------------------------------

Le théorème suivant dit que parmi les courbes \( C^1\), le cercle a la plus grande surface possible à périmètre donné.
\begin{theorem}[Inégalité isopérimétrique\cite{KXjFWKA}]    \label{ThoIXyctPo}
    Soit \( f\colon S^1\to \eC \) une courbe de Jordan\footnote{Définition \ref{DEFooCJCWooLNrHFd}} de classe \( C^1\). Nous notons \( L\) sa longueur et \( S\) l'aire contenue de la surface délimitée\footnote{C'est la partie connexe bornée de \( \eC\setminus\gamma\) dont l'existence est donnée par le théorème de Jordan~\ref{ThoHSPWBuh}.} par \( f\). Alors
    \begin{enumerate}
        \item
            Nous avons l'\defe{inégalité isopérimétrique}{inégalité!isopérimétrique} : \( L^2\geq 4\pi S\).
        \item
            Nous avons l'égalité \( L^2=4\pi S\) si et seulement si la courbe donnée par \( f\) est un cercle.
    \end{enumerate}
\end{theorem}
\index{base!hilbertienne!utilisation}
\index{inégalité!isopérimétrique}
\index{géométrique!avec des nombres complexes}
\index{courbe!étude métrique}
\index{série!de Fourier!utilisation}
\index{Fourier!série!utilisation}

\begin{proof}
    Nous commençons par considérer un chemin dont la longueur est \( 2\pi\) et nous en considérons son paramétrage normal. Nous allons exprimer l'aire \( S\) en utilisant le théorème de Green, et plus particulièrement la formule de surface \eqref{EqAJGrtOk}.

    Si \( f(s)=x(s)+iy(s)\), nous devons intégrer \( y'x-x'y\), qui n'est rien d'autre que la partie imaginaire de \( f'(s)\overline{ f(s) }\). Donc
    \begin{equation}    \label{EqCSWKbPX}
        S=\frac{ 1 }{2}\imag\int_0^{2\pi}f'(s)\overline{ f(s) }ds
    \end{equation}
    Nous considérons les coefficients de Fourier de \( f\) donnés par la formule \eqref{EqNDBaXRL} :
    \begin{equation}
        c_n(f)=\frac{1}{ 2\pi }f(s) e^{-ins}.
    \end{equation}
    Ceux de \( f'\) (qui est aussi continue sur le compact \( S^1\) et donc tout autant \( L^2\)) sont donnés par
    \begin{equation}
        c_n(f')=inc_n(f).
    \end{equation}

    D'autre part en vertu du théorème~\ref{ThoLongueurIntegrale}, la longueur de \( \gamma\) s'exprime en termes de l'intégrale de la norme de sa dérivée :
    \begin{equation}
        2\pi=L=\int_0^{2\pi}| f'(s) |ds=\int_0^{2\pi}| f'(s) |^2ds
    \end{equation}
    parce que nous avons choisi un paramétrage normal qui vérifie automatiquement \( | f'(s) |=1\) pour tout \( s\). L'identité de Parseval sous sa forme \eqref{EqMIuCSfz} appliquée à \( f'\) nous enseigne que
    \begin{equation}        \label{EqXSpHuZI}
        L=2\pi=\int_0^{2\pi}| f'(s) |^2ds=2\pi\langle f', f'\rangle=2\pi\sum_{n=-\infty}^{\infty}| c_n(f') |^2=2\pi\sum_nn^2| c_n(f) |^2.
    \end{equation}
    Par ailleurs le système trigonométrique étant une base hilbertienne, et les fonctions \( f\) et \( f'\) étant dans \( L^2\big( \mathopen[ 0 , 2\pi \mathclose] \big)\) (parce que continues sur un compact), elles sont égales à leurs séries de Fourier (au sens \( L^2\)), c'est-à-dire que nous avons l'égalité \eqref{EqXMMRpSN}. Nous avons alors
    \begin{subequations}
        \begin{align}
            \langle f', f\rangle_{L^2} &=\langle \sum_{n\in \eZ}c_n(f')e_n, \sum_{m\in \eZ}c_m(f)e_m\rangle \\
            &=\sum_m\sum_nc_n(f')\overline{ c_m(f) }\underbrace{\langle e_n, e_m\rangle }_{\delta_{m,n}}\\
            &=\sum_{n\in \eZ}c_n(f')\overline{ c_n(f) }\\
            &=\sum_nin| c_n(f) |^2
        \end{align}
    \end{subequations}
    où nous avons utilisé la continuité du produit scalaire pour sortir les sommes. Avec cela nous pouvons exprimer l'aire \eqref{EqCSWKbPX} en termes de coefficients de Fourier :
    \begin{equation}    \label{EqOZBMiat}
        S=\frac{ 1 }{2}\imag2\pi\langle f', f\rangle =\pi\sum_{n\in \eZ}n| c_n(f) |^2.
    \end{equation}
    En utilisant les expressions \eqref{EqXSpHuZI} et \eqref{EqOZBMiat} pour \( L\) et \( S\), et en écrivant \( L=2\pi L\), nous avons
    \begin{equation}
        L^2-4\pi S=4\pi^2\sum_{n\in \eZ}(n^2-n)| c_n(f) |^2\geq 0.
    \end{equation}
    Cela prouve l'inégalité demandée dans le cas où \( L=2\pi\).

    Si \( \gamma\) n'est pas de longueur \( 2\pi\) mais \( L\), alors nous considérons le chemin \( \sigma(t)=\frac{ 2\pi\gamma(t) }{ L }\). Sa longueur est \( 2\pi\) et son aire, au vu de la formule de Green \eqref{EqCSWKbPX}, son aire est \( 4\pi^2\frac{ S }{ L^2 }\). L'inégalité isopérimétrique appliquée au chemin \( \sigma\) donne alors \( L^2\geq 4\pi S\).

    Le cas d'égalité s'obtient uniquement si \( c_n=0\) pour tout \( n\) différent de \( 0\) ou \( 1\). Dans ce cas nous avons
    \begin{equation}
        f(s)=c_0(f)+c_1(f) e^{is},
    \end{equation}
    qui est un cercle de centre \( c_0(f)\) et de rayon \( | c_1(f) |\).
\end{proof}

%---------------------------------------------------------------------------------------------------------------------------
\subsection{À propos des coefficients}
%---------------------------------------------------------------------------------------------------------------------------

Nous considérons l'application
\begin{equation}
    \begin{aligned}
        c\colon \big( L^1_{2\pi},\| . \|_1 \big)&\to \big( C_0,\| .\|_{\infty} \big) \\
        f&\mapsto (c_n(f))_{n\in \eZ}
    \end{aligned}
\end{equation}
qui à une fonction \( 2\pi\)-périodique fait correspondre la suite (bornée) de ses coefficients de Fourier. Nous rappelons la définition
\begin{equation}
    c_n(f)=\frac{1}{ 2\pi }\int_0^{2\pi}f(t) e^{-int}.
\end{equation}
Nous allons montrer que cette application est linéaire, continue, injective et non surjective. Pour la continuité, par la linéarité il suffit de la montrer en \( 0\). Nous devons donc montrer que si nous avons une suite de fonctions \( f_k\) qui tend vers \( 0\) au sens \( L^1\), alors \( c(f_k)\to 0\) au sens de la norme \( \| . \|_{\infty}\) sur l'ensemble des suites.

Si nous posons \( r_k=\int_0^{2\pi}| f_k(t) |dt\), alors \( r_k=\| f_k \|_1\) et nous avons \( r_k\to 0\). Mais par définition
\begin{equation}
    | c_n(f_k) |\leq r_k,
\end{equation}
et donc \( \| c(f_k) \|_{\infty}\leq r_k\). L'application \( c\) est donc continue. L'injectivité est donnée par le corolaire~\ref{CordgtXlC}.

Si nous supposons que l'application \( c\) est continue, alors le théorème d'isomorphisme de Banach (\ref{ThofQShsw}) nous dit que cela devrait être un homéomorphisme, c'est-à-dire que \( c^{-1}\) serait également continue. Nous allons montrer qu'il n'en est rien.

Nous considérons la suite de suite
\begin{equation}    \label{EqdMtbOB}
    (c_n)_k=\begin{cases}
        1    &   \text{si } k<n\\
        0    &    \text{sinon}.
    \end{cases}
\end{equation}
Ici \( (c_n)_k\) est le terme numéro \( k\) de la suite \( n\). Par injectivité de l'application qui à une fonction fait correspondre la suite de ses coefficients de Fourier, la seule fonction qui possède ces coefficients est
\begin{equation}
    f_n(t)=\sum_{k\in \eN}c_{n,k} e^{ikt}.
\end{equation}
Étant donné que \( \| f_n \|_1=n\), la suite \( (\| f_n \|_1)\) n'est pas bornée alors que a suite de suites \eqref{EqdMtbOB} est bornée dans l'ensemble des suites parce que \( \| c_n \|_{\infty}=1\).


\chapter{Transformation de Fourier}
\input{87_Fourier}

\chapter{Distributions}
\input{88_distributions}

\chapter{Espaces de Sobolev, équations elliptiques}      \label{CHAPooVTIIooGOEvXT}
\input{158_Sobolev}

\chapter{Équations différentielles ordinaires}
\input{89_Equations_diff}
% This is part of Mes notes de mathématique
% Copyright (c) 2008-2009,2011-2019
%   Laurent Claessens
% See the file fdl-1.3.txt for copying conditions.

%+++++++++++++++++++++++++++++++++++++++++++++++++++++++++++++++++++++++++++++++++++++++++++++++++++++++++++++++++++++++++++
\section{Autour de Cauchy-Lipschitz}
%+++++++++++++++++++++++++++++++++++++++++++++++++++++++++++++++++++++++++++++++++++++++++++++++++++++++++++++++++++++++++++
\label{SECooNKICooDnOFTD}

Dans cette section nous étudions les équations différentielles du type
\begin{subequations}
    \begin{numcases}{}
        y'(t)=f\big( t,y(t) \big)\\
        y(t_0)=y_0.
    \end{numcases}
\end{subequations}

%---------------------------------------------------------------------------------------------------------------------------
\subsection{Fuite des compacts et explosion en temps fini}
%---------------------------------------------------------------------------------------------------------------------------

\begin{theorem}[Fuite des compacts\cite{GPRooZkclFA,ZPNooLNyWjX}]
Nous considérons l'équation différentielle
\begin{subequations}
    \begin{numcases}{}
        y'(t)=f\big( t,y(t) \big)\\
        y(t_0)=y_0,
    \end{numcases}
\end{subequations}
où \( f\colon I\times \Omega\to \eR^n\) est continue et \( \Omega\) ouvert dans \( \eR^n\). Soit la solution maximale \( y_M\colon J_M=\mathopen] t_{min} , t_{max} \mathclose[\to \Omega\). Si \( t_{max}<\sup(I)\) alors \( y_M(t)\) sort de tout compact de \( \Omega\) lorsque \( t\to t_{max}\).
\end{theorem}
\index{théorème!fuite des compacts}

\begin{proof}
Soit \( K\) un compact de \( \Omega\) et nous considérons une suite \( (t_m)\) dans \( \mathopen] t_{min} , t_{max} \mathclose[\) telle que \( t_m\to t_{max}\). Si nous supposons que \( y_M(t)\) ne sort pas de \( K\) alors nous avons \( y_M(t_m)\in K\), c'est-à-dire une suite dans un compact. Quitte à passer à une sous-suite, nous supposons qu'elle est convergente. Soit \( x_1\in K\) la limite \( \lim_{m\to \infty}y_M(t_m)=x_1\).

    Vu que \( t_{max}\in I\), la condition initiale \( y(t_{max})=x_1\) est valide et le théorème de Cauchy-Lipschitz~\ref{ThokUUlgU} nous donne une unique solution maximale \( y_P\) définie sur un ouvert \( J_P\) autour de \( t_{max}\).

    Nous allons maintenant construire une solution au problème initial qui contredit la maximalité de \( y_M\). Attention : il n'est pas évident à priori que \( y_P(t)=y_M(t)\) sur l'intersection des domaines. Si c'était évident, la proposition serait démontrée.

    Soit \( \tilde J=J_M\cup J_P\cap\mathopen] t_{min} , +\infty \mathclose[\) et la fonction
        \begin{equation}
            \tilde y(t)=\begin{cases}
                y_M(t)    &   \text{si } t<t_{max}\\
                y_P(t)    &    \text{si } t\geq t_{max}.
            \end{cases}
        \end{equation}
        La fonction \( \tilde y\) est continue par construction parce que
        \begin{equation}
            \lim_{t\to t_{max}} y_M(t)=x_1=y_P(t_{max}).
        \end{equation}
        Nous vérifions à présent que \( \tilde y\) est une solution : \( \tilde y'(t_{max})=f\big( t_{max},y(t_{max}) \big)\) :
        \begin{subequations}
            \begin{align}
                \lim_{\epsilon\to 0}\frac{ \tilde y(t_{max}-\epsilon)-\tilde y(t_{max}) }{ \epsilon }&=\lim_{\epsilon\to 0}\frac{ y_M(t_{max}-\epsilon)-y_P(t_{max}) }{ \epsilon }\\
                &=\lim_{\epsilon\to 0}\frac{ y_M(t_{max}-\epsilon)-y_P(t_{max}-\epsilon)+y_P(t_{max}-\epsilon)-y_P(t_{max}) }{ \epsilon }\\
                &=\lim_{\epsilon\to 0}\frac{ y_P(t_{max}-\epsilon)-y_P(t_{max}) }{ \epsilon }\\
                &=y'_P(t_{max}).
            \end{align}
        \end{subequations}
        Donc \( \tilde y\) est solution pour la condition initiale \( \tilde y(t_{max})=x_1\) et coïncide avec \( y_P\) en \( t_{max}\) et avec \( y_M\) avant \( t_{max}\). Donc en réalité \( y_P\), \( y_M\) et \( \tilde y\) sont identiques et cela contredit la maximalité de \( y_M\).
\end{proof}

\begin{corollary}[Explosion en temps fini]      \label{CorGDJQooNEIvpp}
    Soit \( (y_m,J)\) la solution maximale du problème de Cauchy \eqref{XtiXON} :
    \begin{subequations}
        \begin{numcases}{}
            y'=f(t,y)\\
            y(t_0)=y_0,
        \end{numcases}
    \end{subequations}
    avec \( f\colon U=I\times \Omega\to \eR^n\) où \( I\) est ouvert dans \( \eR\) et \( \Omega\) ouvert dans \( \eR^n\). Nous supposons que \( f\) est continue sur \( U\) et localement Lipschitz par rapport à \( y\).

    Si la solution maximale est définie sur \( J=\mathopen] t_{min} , t_{max} \mathclose[\) alors nous avons l'alternative suivante :
    \begin{enumerate}
        \item   \label{ItemOLYYooJVkRfj}
            Soit \( t_{max}=\sup(I)\),
        \item       \label{ITEMooUKFAooXwRNSB}
            soit \( t_{max}<\sup(I)\) et \( \lim_{t\to t_{max}}  \| y(t) \|= \infty\).
    \end{enumerate}

    Le résultat tient aussi \emph{mutatis mutandis} pour \( t_{\min}\).
\end{corollary}

\begin{remark}
    Attention : ceci n'est pas une simple paraphrase de la fuite des compacts. L'information supplémentaire que ce corolaire donne est que la solution sort de tout compact \emph{pour ne plus y retourner}.
\end{remark}

\begin{proof}
    L'hypothèse \( t_{max}<\sup(I)\) signifie que la solution finit d'exister avant que les hypothèses sur \( f\) cessent d'être vraies. C'est-à-dire que la solution maximale est moindre que ce que nous aurions pu espérer.

Soit un compact \( K\). Supposons que que pour tout \( t_0<t_{max}\) il existe \( t\in\mathopen] t , t_{max} \mathclose[\) tel que \( y_M(t)\in K\). Alors cela crée une suite \( t_k\) dans \( J\) telle que \( y_M(t_k)\) est dans \( K\). Comme dans le théorème de la fuite des compacts nous concluons l'impossibilité de la chose.

    Donc pour tout compact \( K\) de \( \Omega\), il existe \( T<t_{max}\) tel que \( y_M(t)\in \Omega\setminus K\) pour tout \( t\in\mathopen[ T , t_{max} [\). En prenant des boules fermées de plus en plus grandes en guise de compacts nous concluons que
        \begin{equation}
            \lim_{t\to t_{max}} \| y_M(t) \|=\infty.
        \end{equation}
\end{proof}

\begin{normaltext}      \label{NORMooZROGooZfsdnZ}
    Notons que si \( t_{max}<\infty\), si nous sommes dans l'alternative~\ref{CorGDJQooNEIvpp}\ref{ITEMooUKFAooXwRNSB} et si la solution maximale \( y\) est de classe \( C^1\) (ce qui est le cas lorsqu'on utilise Cauchy-Lipschitz~\ref{ThokUUlgU}) alors la dérivée de \( y\) est également non bornée dans un voisinage de \( t_{max}\).

    Mais si \( f\) est globalement bornée, alors dans l'équation \( y'=f(t,y)\), la dérivée \( y'\) sera globalement bornée. Dans ce cas, la solution ne peut pas exploser en temps fini et existe donc globalement.
\end{normaltext}

\begin{probleme}
    Êtes-vous d'accord avec~\ref{NORMooZROGooZfsdnZ} ?
\end{probleme}

\begin{example}
    Soit l'équation différentielle
    \begin{subequations}
        \begin{numcases}{}
            y'=y(y-1)\sin(yt)\\
            y(0)=\frac{1}{2}.
        \end{numcases}
    \end{subequations}
    La fonction \( f(t,y)=y(y-1)\sin(yt)\) ayant une dérivée bornée partout, elle est localement Lipschitz et le théorème de Cauchy-Lipschitz~\ref{ThokUUlgU} s'applique. Pour toute condition initiale, une solution maximale unique existe.

    Si nous oublions la condition initiale, il est facile de trouver des solutions constantes : \( y'=0\) avec \( y(t)=k\) donne l'équation
    \begin{equation}
        0=k(k-1)\sin(kt).
    \end{equation}
    Les solutions \( y_1(t)=0\) et \( y_2(t)=1\) sont des solutions existant pour tout \( t\).

    Le graphe de la solution correspondante à la condition initiale \( y(0)=\frac{ 1 }{2}\) ne pouvant pas croiser les graphes de \( y_1\) et \( y_2\), elle est obligée d'exister pour tout \( t\) parce qu'elle ne peut pas exploser en temps fini.
\end{example}

%---------------------------------------------------------------------------------------------------------------------------
\subsection{Écart entre deux conditions initiales}
%---------------------------------------------------------------------------------------------------------------------------

\begin{proposition}[\cite{ooBSUCooIKClhZ,ooUQOJooSPNjlt}]      \label{PROPooOPRRooQgYFDk}
    Soit une fonction \( f\colon I\times \Omega\to \eR^n\) continue et globalement Lipschitz en sa seconde variable (\( \Omega\) est un ouvert de \( \eR^n\)). Soient deux solutions \( y_1\colon I_1\to \eR^n\) et \( y_2\colon I_2\to \eR^n\) aux problèmes
    \begin{subequations}
        \begin{numcases}{}
            y_i'(t)=f\big( t,y_i(t) \big)\\
            y_i(t_0)=a_i.
        \end{numcases}
    \end{subequations}
    Alors pour tout \( t\in I_1\cap I_2\) nous pouvons estimer l'écart entre \( y_1\) et \( y_2\) par la formule
    \begin{equation}
        \| y_1(t)-y_2(t) \|\leq  e^{L| t-t_0 |}\| a_1-a_2 \|,
    \end{equation}
    où \( L\) est la constante de Lipschitz de \( f\).
\end{proposition}

\begin{proof}
    Nous avons d'abord les majorations suivantes, qui semblent juste jouer avec les notations, mais qui utilisent le fait (contenu dans le théorème de Cauchy-Lipschitz) que \( y_i\) soit de classe \( C^1\) :
    \begin{subequations}
        \begin{align}
            \| y_1(t)-y_2(t) \|&=\| \int_{t_0}^t\big( y'_1(s)-y'_1(s) \big) \|\\
            &\leq L\int_{t_0}^t\| f\big( s,y_1(s) \big)-f\big( s,y_2(s) \big) \|ds\\
            &=L\int_{t_0}^t\| y_1(s)-y_2(s) \|ds.
        \end{align}
    \end{subequations}
    C'est à ce moment que nous utilisons le lemme de Grönwall. Vu que
    \begin{equation}
            \| y_1(t)-y_2(t) \|\leq L\int_{t_0}^t\| y_1(s)-y_2(s) \|ds,
    \end{equation}
    nous sommes dans les hypothèses de Grönwall~\ref{LEMooUGZGooCczAmKa} en posant
    \begin{subequations}
        \begin{align}
            u(t)&=\| y_1(t)-y_2(t) \|\\
            b(t)&=\| y_1(0)-y_2(0) \|\\
            a(t)&=L.
        \end{align}
    \end{subequations}
    Nous avons la majoration
    \begin{equation}
        \| y_1(t)-y_2(t) \|\leq \| y_1(0)-y_2(0) \|+L\int_0^t\| y_1(0)-y_2(0) \| e^{L(t-s)}ds.
    \end{equation}
    Le calcul de l'intégrale intérieure donne
    \begin{equation}
        \int_0^t e^{L(t-s)}ds=-\frac{1}{ L }( e^{-Lt}-1).
    \end{equation}
    Avec ça, nous avons
    \begin{equation}
        \| y_1(t)-y_2(t) \|\leq  e^{Lt}\| y_1(0)-y_2(0) \|.
    \end{equation}
\end{proof}

\begin{normaltext}
    Notons que la proposition~\ref{PROPooOPRRooQgYFDk} est plutôt une mauvaise nouvelle parce que les solutions restent seulement linéairement proches l'une de l'autre lorsqu'on rapproche les conditions initiales, mais elle divergent exponentiellement vite avec le temps. Donc deux trajectoires arbitrairement proches au départ finissent assez vite par être bien séparées.

   Cette proposition est cependant cruciale parce qu'elle explique que pour des petits \( t\), les solutions ne s'écartent pas beaucoup, c'est-à-dire que pour \( t\) fixé, l'application qui à une donnée initiale fait correspondre la solution en \( t\) est continue. C'est le premier pas pour parler de régularité du flot.

\end{normaltext}

%---------------------------------------------------------------------------------------------------------------------------
\subsection{Flot d'un champ de vecteurs}
%---------------------------------------------------------------------------------------------------------------------------

Nous reprenons l'équation différentielle du théorème de Cauchy-Lipschitz~\ref{ThokUUlgU}. En ce qui concerne les notations, \( I\) est un intervalle ouvert de \( \eR\) contenant \( 0\) et l'application \( f\colon I\times \eR^n\to \eR^n \) est continue et localement Lipschitz en sa seconde variable. Pour \( a\in \eR^n\), nous notons \( (J_a,y_a)\) la solution maximale (donc \( y_a\colon J_a\to \eR^n\)) du problème
\begin{subequations}        \label{EQooADJMooRAZWfm}
    \begin{numcases}{}
        y_a(t)=f\big( t,y_a(t) \big)\\
        y_a(0)=a.
    \end{numcases}
\end{subequations}
Nous noterons aussi de temps en temps \( \varphi(t,a)=y_a(t)\).

Nous savons que \( t\mapsto y_a(t)\) est de classe \( C^1\), et cela est directement dans le théorème de Cauchy-Lipschitz. Une question d'une toute autre difficulté est la régularité de \( a\mapsto y_a(t)\) pour \( t\) fixé, et encore pire : celle de \( (t,a)\mapsto y_a(t)\).

Il se fait que l'application \( (t,a)\mapsto y_a(t)\) a la même régularité que celle de \( f\), mais cela va être un peu long à prouver. En ce qui concerne la régularité \( C^1\), ce sera le théorème~\ref{THOooSTHXooXqLBoT} dont la démonstration, comme vous pouvez le voir sera copieuse et demandera des propositions intermédiaires pas simples.

\begin{definition}
    Si \( t\) est fixé, l'application
    \begin{equation}
        \begin{aligned}
            \varphi_t\colon \eR^n&\to \eR^n \\
            x&\mapsto \varphi(t,x) = y_x(t),
        \end{aligned}
    \end{equation}
    est le \defe{flot}{flot} du problème de Cauchy \eqref{EQooADJMooRAZWfm}.

    L'application \( t\mapsto \varphi_t\) est ce qui est appelé le groupe à un paramètre de flot, pour des raisons qui arriverons plus tard\quext{ou pas\ldots}.
\end{definition}

Le but est d'étudier les propriétés du flot : est-il continu, un difféomorphisme, existe, pour quels \( t\) ? Où se cache le champ de vecteurs du titre dans l'équation différentielle ?

Nous posons
\begin{equation}
    \mD=\bigcup_{x\in\Omega}\big( J_x\times \{ x \} \big).
\end{equation}

Comme tout produit d'espaces métrique, l'ensemble \( \mD\) est muni d'une métrique via la définition~\ref{DefZTHxrHA}.

\begin{proposition}[\cite{ooBSUCooIKClhZ}]      \label{PROPooUDQWooNFrNOQ}
    Soit un intervalle \( I\) ouvert de \( \eR\) contenant \( 0\) et \( \Omega\) un ouvert connexe de \( \eR^n\). Soit une application \( f\colon I\times \eR^n\to \eR^n \) continue et localement Lipschitz en sa seconde variable. Pour \( a\in \eR^n\), nous notons \( (J_a,y_a)\) la solution maximale (donc \( y_a\colon J_a\to \eR^n\)) du problème
    \begin{subequations}
        \begin{numcases}{}
            y_a(t)=f\big( t,y_a(t) \big)\\
            y_a(0)=a.
        \end{numcases}
    \end{subequations}

    Nous posons
    \begin{equation}
        \mD=\bigcup_{x\in\Omega}\big( J_x\times \{ x \} \big).
    \end{equation}
    Nous définissons la fonction \( \varphi\) par \( \varphi(t,x)=y_x(t)\) là où ça existe.

    L'ensemble \( \mD\) est ouvert. L'application \( \varphi\colon \mD\to \Omega\) est localement Lipschitz.
\end{proposition}

\begin{proof}
    Soit \( (s,a)\in\mD\) et \( (J_a,y_a)\) la solution maximale passant par \( a\) en \( t=0\). Par définition de \( \mD\) nous avons \( s\in J_a\). Nous considérons \( J\), un compact inclus dans \( J_a\) et contenant \( 0\) et \( s\) en son intérieur. Nous posons
    \begin{equation}
        K=J\times y_a(J).
    \end{equation}
    Vu que \( y_a\) est continue, cela est un compact. Chaque point de \( K\) possède un voisinage ouvert sur lequel \( f\) est Lipschitz\footnote{Cela est à peu près la définition d'être localement Lipschitz :~\ref{DefJSFFooEOCogV}, voir aussi~\ref{NORMooYNRAooBgobcK}.}; nous considérons un sous recouvrement fini et le maximum des constantes de Lipschitz. Cela nous crée un voisinage \( V\) de \( K\) dans \( I\times \Omega\) dans lequel \( f\) est Lipschitz.

    Vu que \( V\) est ouvert et \( K\) est compact avec \( K\subset V\), nous pouvons trouver un ouvert \( V'\) et un compact \( K'\) tels que
    \begin{equation}
        K\subset V'\subset K'\subset V.
    \end{equation}
    Sur ce \( V'\), la fonction \( f\) est de plus bornée parce que continue sur le compact \( K'\). Nous renommons \( V'\) en \( V\). Sur \( V\) nous avons :
    \begin{itemize}
        \item \( \| f \|_{\infty,V}\leq M\),
        \item \( f\) est Lipschitz en sa seconde variable, de constante de Lipschitz \( L\).
    \end{itemize}


    En tant qu'espace produit, nous avons une distance sur \( I\times \Omega\) donnée en~\ref{DefZTHxrHA} :
    \begin{equation}
        d\big( (t,y),(t',y') \big)=\max\big\{  | t-t' |,\| y-y' \|   \big\}.
    \end{equation}
    Nous posons
    \begin{subequations}
        \begin{align}
            V_{\epsilon}(K)=\{ z\in I\times \Omega \tq\,d(z,K)<\epsilon  \}\\
            W_{\epsilon}=\{ (t,y)\in J\times \Omega\tq\,\| y- y_a(t) \|<\epsilon \}.
        \end{align}
    \end{subequations}

    \begin{subproof}
    \item[\( \overline{ W_{\epsilon} }\subset \overline{  V_{\epsilon}(K) }\)]

        Soit \( (t,y)\in W_{\epsilon}\). Nous avons :
        \begin{subequations}
            \begin{align}
                d\big( (t,y),K \big)&=\inf_{(t',y')\in K}d\big( (t,y),(t',y') \big)\\
                &=\inf_{(t',y')\in K}\max\{ | t-t' |,\| y-y' \| \}.     \label{SUBEQooUCZXooRpAitk}
            \end{align}
        \end{subequations}
        Mais demander \( (t,y)\in W_{\epsilon}\) signifie que \( t\in J\) et \( \| y-y_a(t) \|\leq \epsilon\). Dans \( K \) nous avons l'élément \( \big( t,y_a(t) \big)\) qui vérifie
        \begin{equation}
            d\big( (t,y),(t,y_a(t)) \big)=\| y-y_a(t) \|\leq \epsilon.
        \end{equation}
        Donc l'infimum de \eqref{SUBEQooUCZXooRpAitk} est majoré par \( \epsilon\). Nous avons prouvé que \( W_{\epsilon}\subset V_{\epsilon}(K)\) et donc même inclusion pour les fermetures.

    \item[Il existe \( \epsilon>0\) tel que \( \overline{ V_{\epsilon}(K) }\subset V\)]

        Supposons que \( \overline{ V_{\epsilon}(K) }\) ne soit inclus dans \( V\) pour aucun \( \epsilon\). Alors nous considérons
        \begin{equation}
            z_n\in \overline{ V_{1/n}(K) }\setminus V.
        \end{equation}
        Nous avons par définition \( d(z_n,K)\leq \frac{1}{ n }\). Vu que \( K\) est compact, il comprend (au moins) un élément réalisant la distance : soit \( z'_n\in K\) tel que
        \begin{equation}
            d(z_n,z'_n)=d(z_n,K).
        \end{equation}
        Nous avons \( d(z_n,z'_n)\to 0\), de telle sorte que les valeurs d'adhérence de \( (z_n)\) et \( (z'_n)\) sont les mêmes. Et comme \( (z'_n)\) est une suite dans un compact, elle a des valeurs d'adhérence. Soit \( z_{\infty}\) l'une d'elles. Vu que c'est une valeur d'adhérence d'une suite contenue dans le compact \( K\), elle est également dans \( K\) : \( z_{\infty}\in K\). Mais en même temps, \( z_n\) est hors de l'ouvert \( V\), et donc dans le fermé \( V^c\). Les valeurs d'adhérences restent dans le fermé, c'est-à-dire \( z_{\infty}\notin V\). Vu que \( K\subset V\), il y a contradiction.

        Donc il existe \( \epsilon>0\) tel que \( \overline{ V_{\epsilon}(K) }\subset V\).

    \item[Il existe \( \epsilon\) tel que \( \overline{ W_{\epsilon} }\subset V\)]

        Il suffit de prendre le \( \epsilon\) dont nous venons de parler pour avoir
        \begin{equation}
            \overline{ _{\epsilon} }\subset \overline{ V_{\epsilon}(K) }\subset V.
        \end{equation}

    \end{subproof}
    Soit le \( \epsilon\) en question, et \( T>0\) tel que \( J\subset \mathopen[ -T , T \mathclose]\). Nous posons \( r=\epsilon e^{-LT}\). Soit \( b\in \overline{ B(a,r) } \) et
    \begin{equation}
    X=\{ \tau\in J_{+}\tq\,\mathopen] 0 , \tau \mathclose]\subset J_b\text{ et }    \big( t,y_b(t) \big)\in \overline{ W_{\epsilon} }\,\forall\,t\in\mathopen[ 0 , \tau \mathclose]     \}.
    \end{equation}
Nous allons prouver que \( X=J_{+}\) en prouvant qu'il est ouvert, fermé et non vide dans \( J_+=J\cap\mathopen] 0 , \infty \mathclose[\). Nous parlons bien de la topologie de \( J_+\), celle induite\footnote{Définition~\ref{DefVLrgWDB}.} de \( \eR\). Vu que \( 0\in J\), l'ensemble \( J_+\) est ouvert à gauche, mais comme il est compact, il ne va certainement pas jusqu'à \( +\infty\), de telle sorte qu'il est fermé à gauche. Les ouverts de \( J_+\) sont les ensembles de la forme \( \mO\cap J_+\) où \( \mO\) est ouvert de \( \eR\). Il y en a de la forme \( \mathopen] 0 , m \mathclose]\).

    \begin{subproof}
        \item[\( X\) est fermé]

            C'est parce que \( \overline{ W_{\epsilon} }\) et \( J_b\) sont fermés.

        \item[\( X\) est ouvert]

            Soit \( \tau\in X\). Si \( \tau=\sup J_+\) alors \( X=J_+\) est un ouvert de \( J_+\). Supposons donc que \( 0<\tau<\sup J_+\). Dans ce cas nous avons
            \begin{equation}
                \big( \tau,y_b(\tau) \big)\in\overline{ W_{\epsilon} }\subset V,
            \end{equation}
            et nous pouvons résoudre localement le problème de Cauchy
            \begin{subequations}
                \begin{numcases}{}
                    y'(t)=f\big( t,y(t) \big)\\
                    y(\tau)=\varphi(\tau,b)=y_b(\tau).
                \end{numcases}
            \end{subequations}
            Ce \( y\) existe jusqu'à \( \tau+\eta\) (pour au moins un petit \( \eta\)), et par l'unicité de la solution, \( y=y_b\) sur \( \mathopen[ \tau , \tau+\eta \mathclose[\). Ceci pour dire que le flot \( \varphi(.,b)\) existe au moins jusqu'à \( \tau+\eta\).

                Grâce à la proposition~\ref{PROPooOPRRooQgYFDk} nous pouvons évaluer
                \begin{equation}
                    \| \varphi(\tau,b)-\varphi(\tau,a) \|=\| y_a(\tau)-y_b(\tau) \|\leq  e^{L\tau}\| b-a \|.
                \end{equation}
                Comme nous avions choisi \( r=\epsilon e^{-LT}\) et \( b\in\overline{ B(a,r) }\) nous avons aussi \( \| b-a \|\leq \epsilon e^{-LT}\) et donc
                \begin{equation}
                    \| \varphi(\tau,b)-\varphi(\tau,a) \|\leq\epsilon e^{L(\tau-T)}<\epsilon
                \end{equation}
                parce que nous avions \( \tau<\sup J_+\leq T\), ce qui garantit que \(  e^{L(\tau-T)}<1\).

                Est-ce que ceci nous garantit que \( \tau+\eta\in X\) ? Il faudrait \( \big( \tau+\eta,y_b(\tau+\eta) \big)\in \overline{ W_{\epsilon} }\), c'est-à-dire \(  \| y_b(\tau+\eta)-y_a(\tau+\eta) \|\leq\epsilon   \). L'ensemble \( J_+\) étant fermé dans l'ouvert \( J_a\), ce dernier déborde certainement. Prenons donc \( \eta\) assez petit pour que \( y_a\) existe jusqu'en \( \tau+\eta\).

                Vu que \( y_a\) et \( y_b\) sont continues, et qu'en \( \tau\) elles sont distantes de moins de \( \epsilon\), en \( \tau+\eta\), elles restent distantes de moins de \( \epsilon\) (quitte à prendre encore \( \eta\) plus petit).

                Ceci nous permet de conclure que \( X\) est ouvert.

            \item[\( X\) est non vide]

                La solution \( y_b\) au problème
                \begin{subequations}
                    \begin{numcases}{}
                        y_b'(t)=f\big( t,y_b(t) \big)\\
                        y_b(0)=b
                    \end{numcases}
                \end{subequations}
                existe au moins localement et vérifie \( \| y_b(0)-y_a(0) \|=\| b-a \|\leq \epsilon e^{-LT}<\epsilon\). Par continuité nous avons
                \begin{equation}
                    \| y_b(t)-y_a(t) \|<\epsilon
                \end{equation}
                pour tout \( t\) dans un voisinage de \( 0\). Donc \( X\) est non vide.

            \item[Conclusion pour \( X\)]

                La partie \( X\) est ouverte, fermée et non vide dans \( J_+\) qui est connexe. Donc \( X=J_+\) par la proposition~\ref{PropHSjJcIr}\ref{ITEMooNIPZooIDPmEf}.

    \end{subproof}

    La conclusion \( X=J_+\) nous enseigne que pour tout \( t\in J_+\) nous avons \( \mathopen] 0 , t \mathclose]\in J_b\) et \( \big( t,y_b(t) \big)\in \overline{ W_{\epsilon} }\). Nous pouvons faire la même chose pour \( J_-\) et au final nous avons que pour tout \( \tau\in J\) nous avons d'abord \( \tau\in J_b\), ce qui prouve \( J\subset J_b\). De plus pour tout \( t\in J\) nous avons aussi
    \begin{equation}
        \big( t,y_b(t) \big)\in\overline{ W_{\epsilon} }\subset V.
    \end{equation}
    Nous en concluons que
    \begin{equation}
        J\times \overline{ B(a,r) }\subset V.
    \end{equation}

    Nous savons de plus que pour tout \( b\in \overline{ B(a,r) }\), \( J\subset J_b\). Cela signifie que
    \begin{equation}
        J\times \overline{ B(a,r) }\subset \mD.
    \end{equation}

    Mais \( J\times \overline{ B(a,r) }\) est un voisinage de \( (s,a)\) qui était au début de la preuve un point générique choisi dans \( \mD\). Donc \( \mD\) est ouvert parce qu'il contient un voisinage de chacun de ses points.

    Il nous reste à voir que \( \varphi\colon \mD\to \Omega\) est localement Lipschitz. Soit donc le point générique \( (s,a)\) dans \( \mD\) et l'ensemble $V$ qui avait été construit plus haut. Nous allons montrer que \( \varphi\) est Lipschitz sur \( J\times \overline{ B(a,r) }\subset V\). D'abord sur \( V\), l'application \( f\) est Lipschitz, donc
    \begin{equation}
         \| \varphi(t,b_1)-\varphi(t,b_2) \|\leq  e^{Lt}\| b_1-b_2 \|
         \leq  e^{LT}\| b_1-b_2 \|
    \end{equation}
    pour tout \( t\in J\) et \( b_1,b_2\in \overline{ B(a,r) }\).

    Ensuite, \( f\) est bornée, majorée par \( M\) sur $V$, donc
    \begin{subequations}
        \begin{align}
            \| \varphi(t_1,b)-\varphi(t_2,b) \|&=| \int_{\mathopen[ t_1 , t_2 \mathclose]} y'_(s)ds |\\
            &=| \int_{\mathopen[ t_1 , t_2 \mathclose]}f\big( s,y_b(s) \big) |\\
            &\leq \int_{\mathopen[ t_1 , t_2 \mathclose]}| f\big( s,y_b(s) \big) |ds\\
            &\leq M| t_1-t_2 |.
        \end{align}
    \end{subequations}
    Et enfin nous prouvons que \( \varphi\) est localement Lipschitz. En posant \( k=\max\{  e^{LT},M \}\) nous avons
    \begin{subequations}
        \begin{align}
            \| \varphi(t_1,b_1)-\varphi(t_2,b_2) \|&\leq \| \varphi(t_1,b_1)-\varphi(t_1,b_2) \|+\| \varphi(t_1,b_2)-\varphi(t_2,b_2) \|\\
            &\leq  e^{LT}\| b_1-b_2 \|+M| t_1-t_2 |\\
            &\leq k\big( \| b_1-b_2 \|+| t_1-t_2 | \big)\\
            &\leq 2k\max\{ \| b_1-b_2 \|,| t_1-t_2 | \}\\
            &=2kd\big(  (b_1,t_2),(b_2,t_2)  \big).
        \end{align}
    \end{subequations}
    Le flot \( \varphi\) est donc Lipschitz de constante \( 2k\).
\end{proof}

\begin{example}

        \begin{probleme}
            Cet exemple doit être lu attentivement. Il me semble prouver que le flot n'est pas dérivable en la condition initiale sans que \( f\) le soit. Le document \cite{ooPMPXooEpbDkm} semble dire le contraire. Je ne suis pas assez sûr de mon coup pour contredire.
        \end{probleme}

    Il n'y a pas de raisons de penser que \( a\mapsto y_a(t)\) soit mieux que continue en sans hypothèses supplémentaires sur \( f\). Pour illustrer cela nous considérons l'équation différentielle
    \begin{subequations}
        \begin{numcases}{}
            \frac{ \partial X }{ \partial s }=f\big( X(s),s \big)\\
            X(t)=x
        \end{numcases}
    \end{subequations}
    où \( t\) et \( x\) sont des paramètres fixés. Nous allons étudier la dérivabilité de \( X\) en \( x\) lorsque
    \begin{equation}
        f(x,t)=| x |.
    \end{equation}
    Cela est un exemple typique de fonction autant Lipschitz que l'on veut sans être dérivable. L'équation différentielle est
    \begin{equation}
        \frac{ \partial X }{ \partial s }(s)=| X(s) |.
    \end{equation}
    Si \( x>0\) alors \( X(s)>0\) dans un voisinage de \( s=t\) et nous avons \( X(s)=K e^{s}\). La constante \( K\) se fixe par la condition initiale \( X(t)=x\) :
    \begin{equation}
        X(s)=x e^{s-t}.
    \end{equation}
    Et cette solution tient en réalité pour tout \( s\) parce que \( X(s)\) est alors toujours positif.

    Si au contraire \( x<0 \) nous avons la solution
    \begin{equation}
        X(s)=x e^{t-s}.
    \end{equation}
    Au final,
    \begin{equation}
        X(s;x,t)=
        \begin{cases}
            x e^{t-s}    &   \text{si } x<0\\
            0    &    \text{si }x=0\\
            x e^{s-t}    &    \text{si }x>0
        \end{cases}
    \end{equation}
    L'application \( (s,x,t)\mapsto X(s;x,t)\) est continue. En ce qui concerne la dérivée partielle \( \partial_xX\) en \( x=0\) nous avons :
    \begin{equation}        \label{EQooTQXQooAtRxNT}
        \frac{ \partial X }{ \partial x }(s,0,t)=\lim_{\epsilon\to 0}\frac{ X(s,\epsilon,t)-X(s,0,t) }{ \epsilon }=\lim_{\epsilon\to 0}\frac{ X(s,\epsilon,t) }{ \epsilon }.
    \end{equation}
    La limite à droite donne :
    \begin{equation}
        \lim_{\epsilon\to 0^+}\frac{ X(s,\epsilon,t) }{ \epsilon }=\frac{ \epsilon e^{s-t} }{ \epsilon }= e^{s-t}.
    \end{equation}
    La limite à gauche donne :
    \begin{equation}
        \lim_{\epsilon\to 0^-}\frac{ X(s,\epsilon,t) }{ \epsilon }= e^{t-s}.
    \end{equation}
    Les deux limites n'étant pas égales, la limite \eqref{EQooTQXQooAtRxNT} n'existe pas\footnote{Si vous comptez donner ça à manger au jury d'un concours, soyez prudent et n'écrivez pas l'équation \eqref{EQooTQXQooAtRxNT} au tableau. Réfléchissez comment rédiger cela correctement.} et l'application \( (s,x,t)\mapsto X(s,x,t) \) n'est pas dérivable par rapport à \( x\).
\end{example}

\begin{lemma}[\cite{ooGQTBooJKpoVP}]        \label{LEMooOJSNooXTJoEf}
    Soit un application \( A\colon \overline{ B(t_0,\tau) }\times \overline{ B(a,R) }\to \aL(\eR^n)\) continue par rapport à sa première variable (\( t_0\in \eR\) et \( a\in \eR^n\)). Alors en posant l'équation
    \begin{subequations}
        \begin{numcases}{}
            \frac{ \partial \psi }{ \partial t }(t,b)=A(t,b)\psi(t,b)\\
            \psi(t_0,b)=\psi_0.
        \end{numcases}
    \end{subequations}
    Nous avons l'estimation
    \begin{equation}
        \begin{aligned}[]
        \| \psi(t,v)-\psi(t,w) \|\leq \| \psi_0 \|\tau\max_{s\in\overline{ B(t_0,\tau) }}\| A(s,v)&-A(s,w) \|\times\\
        &\times \exp\left( \tau\max_{s\in\overline{ B(t_0,\tau) }}\max\{ \| A(s,v) \|,\| A(s,w) \| \} \right)
        \end{aligned}
    \end{equation}
    pour tout \( t\in\overline{ B(t_0,\tau) }\) et \( v,w\in V\).
\end{lemma}

\begin{theorem}[Régularité \( C^1\) du flot \cite{ooGQTBooJKpoVP}]      \label{THOooSTHXooXqLBoT}
    Soit un intervalle ouvert \( I \) de \( \eR\) et un ouvert connexe \( \Omega\) de \( \eR^n\). Soit une fonction \( f\in C^1\big( I\times \Omega,\eR \big)\), \( a\in \Omega\) et \( t_0\in I\).

    Il existe un voisinage \( W\times V = \overline{ B(t_0,\tau) }\times \overline{ B(a,r) }\) de \( (t_0,a)\) dans \( I\times \Omega\) et une unique application \( \varphi\colon W\times V\to \Omega\) telle que
    \begin{subequations}
        \begin{numcases}{}
            \frac{ \partial \varphi }{ \partial t }(t,x)=f\big( t,\varphi(t,x) \big)\\
            \varphi(t_0,x)=x
        \end{numcases}
    \end{subequations}
    pour tout \( x\in V\).

    L'application \( (t,x)\mapsto \varphi(t,x)\) est de classe \( C^1\).
\end{theorem}

\begin{probleme}
    La preuve qui suit doit être lue avec beaucoup d'attention, en particulier sur les incohérences possibles de notations, et sur les oublis possibles de précautions oratoires type «quitte à encore réduire les voisinages \( V\) et $W$».
\end{probleme}

\begin{proof}
    En termes de notations, pour \( x\in \Omega \) fixé nous écrivons \( y_x(t)\) pour \( \varphi(t,x)\) et pour \( t\in I\) fixé nous notons \( \varphi_t(x)\) pour \( \varphi(t,x)\).

    De plus lorsque nous écrirons des choses comme \( g\colon \eR\to \eR\), nous n'entendrons pas que \( g\) est effectivement définie sur tout \( \eR\). La notation «\( g\colon \eR\to \eR\)» indiquera seulement que la variable de \( g\) est réelle, et que nous comptons préciser le domaine plus tard. Cette remarque s'applique seulement à cette démonstration et non à l'ensemble du livre.

    Nous considérons \( R>0\) tel que \( \overline{ B(a,2R) }\subset\Omega\) et ensuite nous posons \( V=\overline{ B(a,R) }\). La fonction \( y_x\), solution pour la condition initiale \( y_x(t_0)=x\) est définie sur \( W=\mathopen[ t_0-\tau , t_0+\tau  \mathclose]\) et prend ses valeurs dans \( \overline{ B(x,R) }\). Ceci est parce que \( y_x\) est continue, alors en prenant \( \tau\) assez petit, la valeur de \( y_x(t)\) ne va pas s'éloigner de \( x\) lorsque \( t\) ne s'éloigne pas de \( t_0\).

    Nous savons déjà de la proposition~\ref{PROPooUDQWooNFrNOQ} que \( \varphi\) est \( C^1\) en \( t\) et localement Lipschitz en sa seconde variable, avec une constante Lipschitz uniforme sur \( W\times V\). Elle est donc continue en tant que fonction
    \begin{equation}
        \varphi\colon V\times W\to \eR^d.
    \end{equation}

    \begin{subproof}
        \item[La différentielle partielle \( Df\)]
            Pout \( t\) fixé nous notons \( Df_{(t,x)}\) la différentielle de \( f\) par rapport à \( x\). C'est-à-dire que
            \begin{equation}
                \begin{aligned}
                    Df_{(t,x)}\colon \eR^n&\to \eR^n \\
                    u&\mapsto \Dsdd{ f(t,x+su) }{s}{0}.
                \end{aligned}
            \end{equation}
            C'est un élément de \( \aL(\eR^n)\), l'ensemble des applications linéaires de \( \eR^n\) vers \( \eR^n\). Nous allons montrer que
            \begin{equation}
                (t,x)\mapsto Df_{(t,x)}
            \end{equation}
            est continue en tant qu'application \( \eR\times \eR^n\to\aL(\eR^n)\). Pour cela nous introduisons l'application d'inclusion \( i\colon \eR^n\to \eR\times \eR^n\), \( i(u)=(0,u)\). Elle donne
            \begin{equation}
                Df_{(t,x)}(u)=\Dsdd{ f\big( (t,x)+s(0,u) \big) }{s}{0}=df_{(t,x)}\circ i (u).
            \end{equation}
            Autrement dit
            \begin{equation}
                Df_{(t,x)}=df_{(t,x)}\circ i.
            \end{equation}
            Or l'application \( (t,x)\mapsto df_{(t,x)} \) est continue par hypothèse (\( f\) est de classe \( C^1\)) et l'application
            \begin{equation}        \label{EQooZTAPooCduWcl}
                \begin{aligned}
                     \aL(\eR\times \eR^n,\eR^n)&\to \aL(\eR^n,\eR^n) \\
                    A&\mapsto A\circ i
                \end{aligned}
            \end{equation}
            est également continue. Donc \( (t,x)\mapsto Df_{(t,x)}\) est continue\quext{Si quelqu'un peut prouver ça de façon moins verbeuse, je suis preneur. Il me semble que quel que soit la façon dont on s'y prend, sous le capot, on passe par la continuité de l'application \eqref{EQooZTAPooCduWcl}.}.

        \item[L'équation aux variations]

            Soit \( x\in \Omega\). Nous introduisons l'opérateur
            \begin{equation}
                \begin{aligned}
                    S_x\colon \eR\times \aL(\eR^n)&\to \aL(\eR^n) \\
                    S_x(t,\psi)&=Df_{(t,y_x(t))}\circ \psi.
                \end{aligned}
            \end{equation}
            Par ce que nous avons raconté, cela est une fonction continue en sa première variable et Lipschitz en sa seconde variable. Nous identifions \( \aL(\eR^n)\) à \( \eR^{2^n}\).

            Toujours pour chaque \( x\) considéré nous posons l'équation différentielle ordinaire
            \begin{subequations}        \label{EQooQONGooBrxuSA}
                \begin{numcases}{}
                    \frac{ \partial\psi }{ \partial t }(t,x)=S_x\big( t,\psi(t,x) \big)\\
                    \psi(t_0,x)=\id.
                \end{numcases}
            \end{subequations}
            qui est une équation différentielle ordinaire pour \( \psi\colon \eR\times \eR^n\to \aL(\eR^n)\) rentrant dans le cadre de Cauchy-Lipschitz.

            Quel est le domaine de définition de \( \psi\) pour sa première variable ? C'est un ouvert autour de \( t_0\). Nous réduisons \( W\) de telle sorte que la solution \( \psi\) soit définie sur \( W\). Idem pour la variable \( x\) qui est dans un voisinage de \( a\).

            L'équation \eqref{EQooQONGooBrxuSA} s'appelle l'\defe{équation aux variations}{équation!aux variations}. Nous allons montrer dans la douleur que \( \psi\) est continue et est la différentielle de \( \varphi_t\), c'est-à-dire que
            \begin{equation}
                (d\varphi_t)_b=\psi(t,b).
            \end{equation}

        \item[\( \psi\) est continue en \( (t,x)\) (début)]


            Il s'agit de majorer les deux termes de
            \begin{equation}        \label{EQooVUNUooExeQba}
                \| \psi(t_1,a_1)-\psi(t_2,a_2) \|\leq \| \psi(t_1,a_1)-\psi(t_2,a_1) \|+\| \psi(t_2,a_1)-\psi(t_2,a_2) \|.
            \end{equation}

            \begin{subproof}

        \item[Premier terme]

            Nous avons
            \begin{subequations}
                \begin{align}
                    \| \psi(t_1,b)-\psi(t_2,b) \|&=\| \int_{\mathopen[ t_1 , t_2 \mathclose]}\frac{ \partial \psi }{ \partial t }(s,b)ds \|\\
                    &\leq\int_{\mathopen[ t_1 , t_2 \mathclose]}\| Df_{(s,y_b(s))}\circ\psi(s,b) \|ds\\
                    &\leq\int_{\mathopen[ t_1 , t_2 \mathclose]}\| Df_{(s,t_b(s))} \|\| \psi(s,b) \|ds\\
                    &\leq | t_1-t_2 |\max_{s\in\mathopen[ t_1 , t_2 \mathclose]}\| Df_{(s,y_b(s))} \|\max_{s\in\mathopen[ t_1 , t_2 \mathclose]}\| \psi(s,b) \|.\label{SUBEQooLYMAooRMaMhn}
                \end{align}
            \end{subequations}

            Nous allons majorer le second maximum. Prenons \( t\in\mathopen[ 0 , \tau \mathclose]\); et posons \( A(u,b)=Df_{(u,y_b(u))}\) pour alléger les notations. Par l'équation de définition de \( \psi\) nous avons
            \begin{equation}
                \psi(t,b)=\psi(0,b)+\int_{\mathopen[ 0 , t \mathclose]}A(u,b)\psi(u,b)du,
            \end{equation}
            et donc
            \begin{equation}
                \| \psi(t,b) \|\leq \| \psi_0 \|+\int_{\mathopen[ 0 , t \mathclose]} \| A(u,b) \|  \| \psi(u,b) \|du.
            \end{equation}
            En y appliquant le lemme de Grönwall dans sa version~\ref{LemuBVozy} nous trouvons
            \begin{subequations}
                \begin{align}
                \| \psi(s,b) \|&\leq \| \psi_0 \|\exp\left( \int_{\mathopen[ 0 , s \mathclose]}\| A(u,b) \|du \right)\\
                &\leq \| \psi_0 \|\exp\left( s\max_{u\in\mathopen[ 0 , s \mathclose]}\| A(u,b) \| \right).
                \end{align}
            \end{subequations}
            En retournant à \eqref{SUBEQooLYMAooRMaMhn} nous avons \( \psi_0=\id\) et donc \( \| \psi_0 \|=1\) et
            \begin{equation}
                \max_{s\in\mathopen[ t_1 , t_2 \mathclose]}\| \psi(s,b) \|\leq \max_{s\in \mathopen[ t_1 , t_2 \mathclose]}\exp\left( s\max_{u\in \mathopen[ 0 , t \mathclose]}\| Df_{(u,y_b(u))} \| \right)
            \end{equation}
            Là dedans nous pouvons remplacer \( t\) par \( \max\{ | t_1 |,| t_2 | \}\). Posons enfin, pour alléger les expressions
            \begin{equation}
                a(t_1,t_2,b)=\max_{s\in\mathopen[ t_1 , t_2 \mathclose]}\| Df_{(s,y_b(s))} \|.
            \end{equation}
            La majoration que nous retenons est :
            \begin{equation}
                \| \psi(t_1,b)-\psi(t_2,b) \|\leq | t_1-t_2 |a(t_1,t_2,b)\exp\big( \max\{ | t_1 |,| t_2 | \}a(0,t,b) \big).
            \end{equation}
            Cela tend vers zéro lorsque \( t_1\to t_2\).

        \item[Deuxième terme]

            En ce qui concerne le second terme,
            \begin{equation}
                \| \psi(t,b_1)-\psi(t,b_2) \|
            \end{equation}
            nous utilisons le lemme~\ref{LEMooOJSNooXTJoEf} qui donne, pour \( t\in\mathopen[ t_0 -\tau, t_0+\tau \mathclose]\),
            \begin{equation}
                \begin{aligned}[]
                    \| \psi(t,b_1)-\psi(t,b_2) \|\leq\tau\max_{s\in \overline{ B(0,\tau) }}&\| Df_{s,y_{b_1}(s)}-Df_{s,y_{b_2}(s)} \|\times\\
                    &\times \exp\left( \tau\max\{ \| Df_{s,y_{b_1}(s)},\| Df_{s,y_{b_2}(s)} \| \| \} \right).
                \end{aligned}
            \end{equation}
            Dans notre cas, \( t_0=0\), donc \( t\in\mathopen[ -\tau , \tau \mathclose]\). Vu la continuité de \( Df\), nous avons
            \begin{equation}
                    \max_{s\in \overline{ B(0,\tau) }}\| Df_{s,y_{b_1}(s)}-Df_{s,y_{b_2}(s)} \|\to 0
            \end{equation}
            lorsque \( b_1\to b_2\).

        \item[\( \psi\) est continue en \( (t,x)\) (fin)]

            Les deux bons calculs faits, nous avons, en repartant de \eqref{EQooVUNUooExeQba},
            \begin{equation}
                \lim_{(t_1,b_1)\to(t_2,b_2)}\| \psi(t_1,b_1)-\psi(t_2,b_2) \|=0,
            \end{equation}
            ce qui signifie que \( \psi\) est une fonction continue de ses deux variables en même temps.
            \end{subproof}

        \item[Différentiabilité de \( \varphi\) (début)]

            Nous montrons maintenant que \( D\varphi(t,x)\) existe. Pour rappel, \( D\) est la différentielle par rapport à la seconde variable. Nous sommes à étudier l'existence de \( D\varphi_{(t,b)}=d(\varphi_t)_b\). Nous posons
            \begin{equation}
                \theta(t,h)=\varphi(t,b+h)-\varphi(t,b)=y_{b+h}(t)-y_b(t)
            \end{equation}
            où \( b\) est le point où nous étudions la différentiabilité. Il est dans un voisinage du point \( a\) fixé depuis le début et autour duquel il existe un voisinage qui donne un sens à tout ce que nous avons fait jusqu'à présent. La dépendance de \( \theta\) en \( b\) est implicite. Vu que \( \varphi\) est Lipschitz en sa seconde variable, nous avons la majoration
            \begin{equation}        \label{EQooKYELooZlfeed}
                \| \theta(t,h) \|\leq C\| h \|
            \end{equation}
            dès que \( t\in V\) et \( b,b+h\in W\).

            De plus, parce que \( t_0\) est le temps de la condition initiale nous avons
            \begin{equation}
                \theta(t_0,h)=y_{b+h}(t_0)-y_{b}(t_0)=a+h-a=h.
            \end{equation}
            Et aussi, par définition de \( \psi\) :
            \begin{equation}
                    \psi(t,b)=\psi_0+\int_{t_0}^t\frac{ \partial \psi }{ \partial t }(s,b)ds =\psi_0+\int_{t_0}^tDf_{(s,y_b(s))}\circ\psi(s,b)
            \end{equation}
            En appliquant à \( h\) et en se souvenant que \( \psi_0=\id\),
            \begin{equation}
                    \psi(t,b)h=h+\int_{t_0}^t\Big( Df_{s,y_b(s)}\circ\psi(s,b)\Big)h\,ds.
            \end{equation}
            Puis on peut faire un calcul assez classique en se souvenant que \( \theta(t_0,h)=h\) :
            \begin{subequations}
                \begin{align}
                    \theta(t,h)&=\theta(t_0,h)+\int_{t_0}^t\big[ \frac{ \partial \varphi }{ \partial t }(s,b+h)-\frac{ \partial \varphi }{ \partial t }(s,b) \big]ds\\
                    &=h+\int_{t_0}^t\big[   f\big( s,y_{b+h}(s) \big)-f\big( s,y_b(s) \big)   \big]ds.
                \end{align}
            \end{subequations}
            On fait la différence entre les deux :
            \begin{equation}
                \theta(t,h)-\psi(t,b)h=-\int_{t_0}^t\big[ Df_{s,y_b(s)}\circ\psi(s,b)h-f\big( s,y_{b+h}(s)\big)+f\big( s,y_b(s) \big) \big]ds.
            \end{equation}
            Nous y ajoutons et soustrayons \( Df_{s,y_{b}(s)}\theta(s,h)\) et nous retenons la majoration suivante :
            \begin{equation}        \label{EQooODHPooDYyBoH}
                \begin{aligned}[]
                    \| \theta(t,h)-\psi(t,b)h \|\leq &\int_{t_0}^t\| Df_{(s,y_b(s))}\psi(s,b)-Df_{(s,y_b(s))} \theta(s,h)\| ds\\
                    &+\int_{t_0}^t\| f(s,y_{b+h}(s))-f(s,y_b(s))+Df_{(s,y_b(s))}\theta(s,h)  \|ds.
                \end{aligned}
            \end{equation}
            Nous allons encore majorer ces deux termes séparément. Soit \( \epsilon>0\).

            \begin{subproof}

        \item[Premier terme]

             Ce qui est dans la norme à majorer est
             \begin{equation}
                 Df_{(s,y_b(s))}\big( \psi(s,b)h-\theta(s,h) \big).
             \end{equation}
             Vu que \( Df\) est continue et que \( y_b\) est continue\footnote{Il faut encore réduire les voisinages \( V\) et \( W\) pour que ceci ait un sens.}, l'application \( s\mapsto Df_{s,y_b(s)}\) est continue et donc de norme majorée sur le compact \( \mathopen[ t_0 , t \mathclose]  \). Nous rapellons la notation
             \begin{equation}
                 a(t_0,t,b)=\max_{s\in\mathopen[ t_0 , t \mathclose]}\| Df_{s,y_b(s)} \|,
             \end{equation}
             et nous majorons encore et toujours. D'abord
             \begin{equation}
                 \int_{t_0}^t\| Df_{(s,y_b(s))}\big( \psi(s,b)h-\theta(s,h) \big) \|ds\leq a(t_0,t,b)\int_{t_0}^t\| \psi(s,b)-\theta(s,h) \|ds.
             \end{equation}

        \item[Deuxième terme]

            Pour traiter le deuxième terme, nous allons provisoirement noter \( x=y_b(s)\) et \( y=y_{b+h}(s)\); entre autres, \( y-x=\theta(s,h)\). Ce qui est écrit dans le second terme de \eqref{EQooODHPooDYyBoH} est
            \begin{equation}
                f(s,y)-f(s,x)+Df_{(s,x)}\theta(s,h)=f(s,y)-f(s,x)+Df_{(s,x)}(y-x)
            \end{equation}
            Comme \( D\) ne s'applique pas à la variable \( s\), nous pouvons alléger la notation et déduire de la différentiabilité de \( f\) qu'il existe un \( \eta>0\) tel que \( x,y\in W\) avec \( \| y-w \|\leq \eta\) implique
            \begin{equation}
                \| f(y)-f(x)-Df_x(y-x) \|\leq \epsilon\| y-x \|.
            \end{equation}
            Prenons \( \| h \|\leq \eta/C\) (le \( C\) de \eqref{EQooKYELooZlfeed}); en déballant les notations,
            \begin{equation}
                \| f\big( s,y_{b+h}(s) \big)-f\big( s,y_b(s) \big) -Df_{(s,y_b(s))}\theta(s,h)\|\leq \epsilon\| \theta(s,h) \|\leq \epsilon C\| h \|.
            \end{equation}


        \item[Les deux termes ensemble]

            En remettant les deux dans \eqref{EQooODHPooDYyBoH} nous trouvons la majoration
            \begin{equation}
                \| \theta(t,h)-\psi(t,b)h \|\leq | t-t_0 |\epsilon C\| h \|+a(t_0,t,v)\int_{t_0}^t\| \psi(s,b)h-\theta(s,h) \|ds
            \end{equation}
            qui est encore de la graine à Grönwall avec
            \begin{subequations}
                \begin{numcases}{}
                    u(t)=\| \theta(t,h)-\psi(t,b) \|\\
                    b(t)=| t-t_0 |\epsilon C\| h \|\\
                    a(s)=a(t_0,t,v),
                \end{numcases}
            \end{subequations}
            la troisième étant une fonction constante. Cela donne, pour \( t\in\mathopen[ t_0-\tau , t_0+\tau \mathclose]\),
            \begin{equation}
                \| \theta(t,h)-\psi(t,b)h \|\leq| t-t_0 |\epsilon C\| h \|+\int_{t_0}^t(s-t_0)\epsilon C\| h \|a(t_0,t,b)\exp\left( \int_{s}^ta(t_0,t,b)du \right)ds.
            \end{equation}
            En valeur absolue, la différence \( s-t_0\) est majorée par \( \tau\), l'intégrale dans l'exponentielle vaut \( (t-s)a(t_0,t,b)\), et restons avec
            \begin{equation}
                \| \theta(t,h)-\psi(t,b)h \|\leq \tau \epsilon C\| h \|+\tau\int_{t_0}^t\epsilon C\| h \|a(t_0,t,b) e^{a(t_0,t,b)(t-s)}ds.
            \end{equation}
            En supposant \( t>t_0\) nous pouvons calculer l'intégrale. Si vous m'avez suivi jusqu'ici, vous devriez avoir de tels maux de tête que je vous donne la réponse :
            \begin{equation}
                \int_{t_0}^ta(t_0,t,b) e^{(t-s)a(t_0,t,b)}ds= e^{(t_0-t)a(t_0,t,b)}-1.
            \end{equation}
            En remettant dans l'expression,
            \begin{equation}
                \| \theta(t,h)-\psi(t,b)h \|\leq \tau\epsilon C\| h \|+\epsilon C\| h \|a(t_0,t,b)\tau\big(  e^{(t-t_0)}-1 \big)=\tau\epsilon C\| h \| e^{(t-t_0)a(t,t_0,b)}.
            \end{equation}
            Nous pouvons majorer \( t-t_0\) par \( \tau\) et \( a(t,t_0,b) \) par \( a(t_0-\tau,t_0+\tau,b)\) pour avoir la majoration
            \begin{equation}
                \| \theta(t,h)-\psi(t,b)h \|\leq \tau\epsilon C\| h \| e^{\tau a(t_0-\tau,t_0+\tau,b)}.
            \end{equation}

        \item[Différentiabilité de \( \varphi(t,b)\) (fin)]

            Nous écrivons la définition~\ref{DefDifferentiellePta} de la différentiabilité : nous voulons vérifier que
            \begin{equation}
                \lim_{h\to 0} \frac{ \varphi(t,b+h)-\varphi(t,b)-\psi(t,b)h }{ \| h \| }=0.
            \end{equation}
            Nous remplaçons \( \varphi(t,b+h)-\varphi(t,b)\) par \( \theta(t,h)\) et prenons la norme avec les majorations données :
            \begin{equation}
                \lim_{h\to 0} \frac{ \|  \varphi(t,b+h)-\varphi(t,b)-\psi(t,b)h  \|   }{ \| h \| }\leq \lim_{h\to 0} \tau\epsilon C e^{\tau e(t_0-\tau,t_0+\tau,b)}.
            \end{equation}
            Cela étant valable pour tout \( \epsilon\), nous en déduisons la nullité de la limite.

            Nous avons démontré que \( \varphi\) était différentiable par rapport à sa deuxième variable et que
            \begin{equation}        \label{EQooPJFOooHuOIuw}
                D\varphi_{(t,b)}=\psi(t,b).
            \end{equation}
            \end{subproof}

        \item[Conclusion : \( \varphi\) est de classe \( C^1\)]

            Nous avons déjà prouvé que \( (t,b)\mapsto \psi(t,b)\) est continue. Donc de \eqref{EQooPJFOooHuOIuw} nous déduisons que les dérivées partielles \(  (t,b)\mapsto\frac{ \partial \varphi }{ \partial x_i } (t,b)\) sont continues. Mais comme \( \varphi\) est Lipschitz en \( t\), la dérivée partielle \( (t,b)\mapsto \frac{ \partial \varphi }{ \partial t }(t,b)\) est également continue.

            La continuité de toutes les dérivées partielles de \( \varphi\) nous donne la classe \( C^1\) pour \( \varphi\) par le théorème \ref{THOooBEAOooBdvOdr}.
    \end{subproof}
\end{proof}

\begin{proposition}[Régularité \( C^p\) du flot\cite{ooEGXQooRwPKcC}]        \label{PROPooINLNooDVWaMn}
    Soit un intervalle ouvert \( I \) de \( \eR\) et un ouvert connexe \( \Omega\) de \( \eR^n\). Soit une fonction \( f\in C^p\big( I\times \Omega,\eR \big)\),  ainsi que \( a\in \Omega\) et \( t_0\in I\).

    Il existe un voisinage \( W\times V = \overline{ B(t_0,\tau) }\times \overline{ B(a,r) }\) de \( (t_0,a)\) dans \( I\times \Omega\) et une unique application \( \varphi\colon W\times V\to \Omega\) telle que
    \begin{subequations}
        \begin{numcases}{}
            \frac{ \partial \varphi }{ \partial t }(t,x)=f\big( t,\varphi(t,x) \big)\\
            \varphi(t_0,x)=x
        \end{numcases}
    \end{subequations}
    pour tout \( x\in V\).

    L'application \( (t,x)\mapsto \varphi(t,x)\) est de classe \( C^p\).
\end{proposition}

\begin{proof}
    Nous savons déjà par le théorème~\ref{THOooSTHXooXqLBoT} que \( (t,x)\mapsto \varphi(t,x)\) est de classe \( C^1\). Nous supposons que \( f\) est de classe \( C^p\) avec \( p\geq 2\).

    Vu que \( \varphi\) et \( f\) sont de classe \( C^1\), nous avons aussi que l'application \( (t,x)\mapsto f\big( t,\varphi(t,x) \big)\) est de classe \( C^1\). L'équation donne alors immédiatement le fait que
    \begin{equation}
        (t,x)\mapsto\frac{ \partial \varphi }{ \partial t }(t,x)
    \end{equation}
    est de classe \( C^1\).

    En ce qui concerne la régularité par rapport aux autres variables, il faudra travailler un peu plus.

    \begin{subproof}
    \item[Une équation différentielle pour le flot]
    Nous allons commencer par un habile jeu d'écriture : la formule
    \begin{equation}
        \varphi(t,x)=x+\int_{t_0}^tf\big( s,\varphi(s,x) \big)ds
    \end{equation}
    devient
    \begin{equation}        \label{EQooQGVOooYMEgcM}
        \varphi_t(x)=x+\int_{t_0}^tf\big( s,\varphi_s(x) \big)ds.
    \end{equation}
    Dans le même ordre d'idée nous notons \( f_s(x)=f(s,x)\), et ce qui se trouve dans l'intégrale \eqref{EQooQGVOooYMEgcM} n'est autre que la fonction
    \begin{equation}
        g_s(x)=(f_s\circ\varphi_s)(x).
    \end{equation}
    Tout cela pour différentier l'égalité \eqref{EQooQGVOooYMEgcM} par la proposition~\ref{PropAOZkDsh} :
    \begin{subequations}
        \begin{align}
            (d\varphi_t)_x&=\id+\int_{t_0}^t(dg_s)_xds\\
            &=\id+\int_{t_0}^t(df_s)_{\varphi_s(x)}\circ(d\varphi_s)_sds.
        \end{align}
    \end{subequations}
    Nous dérivons ensuite cela par rapport à \( t\) :
    \begin{equation}        \label{EQooBETGooXKWRxX}
        \frac{ \partial  }{ \partial t }\Big( (d\varphi_t)_x \Big)=(df_t)_{\varphi_t(x)}\circ(d\varphi_t)_x.
    \end{equation}
    Cela est une égalité dans \( \aL(\eR^n)\).

    Nous introduisons la fonction
    \begin{equation}
        \begin{aligned}
            F\colon I\times \aL(\eR^n)\times \Omega&\to \aL(\eR^n)  \\
            (t,A,x)&\mapsto (df_t)_{\varphi_t(x)}\circ A.
        \end{aligned}
    \end{equation}
    En fait, à la place de \( I\) et \( \Omega\) il faut prendre des petits voisinages dans lesquels les choses ont un sens. Ce que dit l'équation~\ref{EQooBETGooXKWRxX} est que l'application
    \begin{equation}
        \begin{aligned}
            A\colon I \times \Omega&\to \aL(\eR^n) \\
            (t,x)&\mapsto (d\varphi_t)_x
        \end{aligned}
    \end{equation}
    vérifie l'équation différentielle
        \begin{subequations}\label{EQooTOJMooLXLfVv}
            \begin{numcases}{}
                    \frac{ \partial A }{ \partial t }(t,x)=F\big( t,A(t,x),x \big)\\
                    A(t_0,x)=\id.
            \end{numcases}
        \end{subequations}

    \item[Une autre équation différentielle]

        Nous n'oublions pas l'équation différentielle pour la dérivée par rapport à \( t\) :
        \begin{equation}        \label{EQooYOJPooKEgiec}
            \frac{ \partial \varphi }{ \partial t }(t,x)=f\big( t,\varphi(t,x) \big).
        \end{equation}

    \item[Réécriture pour la différentielle]

        Nous allons récrire l'équation \eqref{EQooTOJMooLXLfVv} de façon à ce que le paramètre \( x\) soit inclus dans la condition initiale. De cette manière, la solution pourra profiter de la régularité \( C^1\) du flot déjà prouvée dans le théorème~\ref{THOooSTHXooXqLBoT}.

        Soit
        \begin{equation}
            \begin{aligned}
                g\colon I\times \big( \aL(\eR^n)\times \Omega \big)&\to \aL(\eR^n)\times \Omega \\
                \big( t,(A,x) \big)&\mapsto \Big( F(t,a,x),0 \Big).
            \end{aligned}
        \end{equation}
        Nous posons \( E=\aL(\eR^n)\times \Omega\); c'est cet espace qui va jouer le rôle de \( \Omega\). Nous considérons à présent l'équation différentielle suivante pour \( z_x\colon I\to E\) :
        \begin{subequations}        \label{EQooSMYOooDaxgQx}
            \begin{numcases}{}
                \begin{pmatrix}
                    z_1'(t)    \\
                    z_2'(t)
                \end{pmatrix}=
                z'(t)=g(t,z(t))=
                \begin{pmatrix}
                F\big( t,z_1(t),z_2(t) \big)\\
                    0
                \end{pmatrix}\\
                z(t_0)=(\id,x).
            \end{numcases}
        \end{subequations}
        Il devrait y avoir un indice \( (\id,x)\) à \( z\) parce que c'est sa condition initiale. La fonction \( g\) est de classe \( C^p\), donc cette équation admet une unique solution dont le flot est de classe \( C^1\). Autrement dit, si \( S\) est dans un voisinage de \( \id\), l'application
        \begin{equation}
            (t,x)\mapsto z(t)
        \end{equation}
        est de classe \( C^1\). Nous allons montrer qu'en posant \( A(t,x)=z_1(t)\), nous avons une solution de \eqref{EQooTOJMooLXLfVv} (l'unicité de la solution impose que cette solution est effectivement la différentielle de \( d\varphi_t\)). D'abord, la seconde ligne de l'équation différentielle est \( z'_2(t)=0\), c'est-à-dire \( z_2(t)=x\) pour tout \( t\).

        Sachant cela, la première équation devient
        \begin{subequations}
            \begin{numcases}{}
                z_1'(t)=F\big( t,z_1(t),x \big)\\
                z_1(t_0)=\id,
            \end{numcases}
        \end{subequations}
        qui est l'équation différentielle pour \( A\). Rappel : il y a partout une dépendance de \( z\) en sa condition initiale \( x\) que nous n'avons pas écrite pour des raisons de légèreté notionnelle. Il n'en reste pas moins que le flot de l'équation différentielle pour \( z\) est \( C^1\), c'est-à-dire que \( (t,x)\mapsto z_1(t)\) est de classe \( C^1\).

        Par conséquent, \( (t,x)\mapsto A(t,x)\) est également \( C^1\).


    \item[Régularité \( C^2\) du flot]

        Le fait que \( A\) soit \( C^1\) n'implique pas que le flot le soit parce que le flot suit la même équation différentielle que \( A\) ne signifie pas que il soit égal. Il y a un raisonnement à faire.

        Le fait est que si \( A\) est une solution de \eqref{EQooTOJMooLXLfVv}, alors \( z(t)=\big( A(t,x),x \big)\) est solution de \eqref{EQooSMYOooDaxgQx}. C'est l'unicité de cette dernière qui permet de déduire l'unicité de la solution pour \( A\).

        Nous avons donc que l'unique solution \( A\) du système \eqref{EQooTOJMooLXLfVv} est égale à \( A(t,x)=(d\varphi_t)_x\) et est de classe \( C^1\) par rapport à \( (t,x)\).

        Donc \( (t,x)\mapsto  \varphi(t,x)\) est de classe \( C^2\).

    \item[Régularité \( C^p\)]

        Nous avons vu que le flot de \( y'=f(t,y)\) est de classe \( C^2\) dès que \( f\) est de classe \( C^2\). Supposons que \( f\) soit de classe \( C^p\) et montrons que si le flot est de classe \( C^k\) (\( k<p\)) alors il est de classe \( C^{k+1}\).

        Vu que le flot d'une équation différentielle de classe \( C^p\) est de classe \( C^k\), en particulier celui de \eqref{EQooSMYOooDaxgQx} est de classe \( C^k\). Donc aussi la solution pour \( A(t,x)=(d\varphi_t)_x\) est de classe \( C^k\). Et vu que \(  (t,x)\mapsto (d\varphi_t)_x   \) est de classe \( C^k\), l'application \( \varphi\) est de classe \( C^{k+1}\).

    \end{subproof}
\end{proof}

\begin{normaltext}      \label{NORMooWEWVooXbGmfE}
    % Attention : ce 'normaltext' est référentié dans l'index thématique sur l'inversion locale. Si on développe ici, il faudra modifier là-bas.
    % position 1051229132
    Le théorème d'inversion locale~\ref{ThoXWpzqCn} nous permet de dire que, pour \( t\) fixé, le flot \( x\mapsto \varphi_t(x)\) est un \( C^p\)-difféomorphisme local.
\end{normaltext}

\begin{proposition}[Cauchy-Lipschitz avec paramètre, régularité \( C^p\)\cite{ooCMJNooHgCXBS,ooSDCGooACTQQH,ooZWJZooBocVPb}]       \label{PROPooPYHWooIZhQST}
    Soit un intervalle ouvert \( I\) de \( \eR\), un connexe ouvert \( \Omega\) de \( \eR^n\) et un intervalle ouvert \( \Lambda\) de \( \eR^d\). Soit une fonction \( f\in C^p( I\times \Omega\times \Lambda, \eR^n)\) localement Lipschitz en \( \Omega\). Soient \( t_0\in I\), \( y_0\in \Omega\) et \( \lambda_0\in \Lambda\). Il existe un voisinage compact de \( (t_0,y_0,\lambda_0)\) sur lequel le problème
    \begin{subequations}
        \begin{numcases}{}
            y'_{\lambda}(t)=f\big( t,y_{\lambda}(t),\lambda \big)\\
            y_{\lambda}(t_0)=y_0
        \end{numcases}
    \end{subequations}
    possède une unique solution. De plus \( (t,\lambda)\mapsto y_{\lambda}(t)\) est de classe \( C^p\) par rapport à ses deux variables.
\end{proposition}

\begin{proof}
    Nous récrivons immédiatement le problème pour la fonction \( y\colon I\times \Lambda\to \eR^n\) donné par \( y(t,\lambda)=y_{\lambda}(t)\) :
    \begin{subequations}        \label{SUBEQooXMYMooKfpqQW}
        \begin{numcases}{}
            \frac{ \partial y }{ \partial t }(t,\lambda)=f\big( t,y(t,\lambda),\lambda \big)\\
            y(t_0)=y_0.
        \end{numcases}
    \end{subequations}
    Nous allons montrer que ce problème est en réalité équivalent à un problème sans paramètre. Nous posons \( E=\Omega\times \Lambda\) et
    \begin{equation}
        \begin{aligned}
            g\colon I\times E&\to E \\
            (t,x)&\mapsto \big(f(t,x_1,x_2) ,0\big)
        \end{aligned}
    \end{equation}
    où \( x_1\) est la composante \( \Omega\) de \( x\) et \( x_2\) est la composante \( \Lambda\) de \( x\). Pour une valeur \( \mu\in \Lambda\) donnée nous considérons le problème au condition initiales
    \begin{subequations}        \label{SUBEQSooIBTNooHzYImh}
        \begin{numcases}{}
            x'(t)=g\big( t,x(t) \big)\\
            x(t_0)=(y_0,\mu).
        \end{numcases}
    \end{subequations}
    Le théorème de Cauchy-Lipschitz que nous prenons sous la forme~\ref{PROPooINLNooDVWaMn} nous indique que ce problème admet une unique solution maximale et que le flot \(   \big( t,(y_0,\mu) \big)     \mapsto  x_{(y_0,\mu)}(t)   \) est de classe \( C^p\).

    Nous passons maintenant à la résolution du problème \eqref{SUBEQooXMYMooKfpqQW}

    \begin{subproof}
    \item[Existence d'une solution \( C^1\)]

    Nous montrons à présent que la fonction \( y\) donnée par
    \begin{equation}
        y(t,\mu)=x_{(t_0,\mu)}(t)_1
    \end{equation}
    est solution de \eqref{SUBEQooXMYMooKfpqQW}. Vu que \( x\) a deux composantes, nous pouvons un peu déballer l'équation. Afin d'éviter les notations laborieuses nous allons noter \( x\) pour \( x_{(t_0,\mu)}\) et donc \( x_1(t)\) pour \( x_{(t_0,\mu)}(t)\). Nous avons l'équation différentielle
    \begin{equation}
        \begin{pmatrix}
            x_1'(t)    \\
            x_2'(t)
        \end{pmatrix}=\begin{pmatrix}
            f\big( t,x_1(t),x_2(t) \big)    \\
            0
        \end{pmatrix}
    \end{equation}
    avec la condition initiale
    \begin{equation}
        \begin{pmatrix}
            x_1(t_0)    \\
            x_2(t_0)
        \end{pmatrix}=\begin{pmatrix}
            y_0    \\
            \mu
        \end{pmatrix}.
    \end{equation}
    La seconde ligne de l'équation donne immédiatement \( x_2(t)=\mu\) pour tout \( t\). En injectant dans la première ligne :
    \begin{equation}
        x_1'(t)=f\big( t,x_1(t),\mu \big).
    \end{equation}
    Or vue la définition de \( y\), le nombre \( x_1'(t)\) n'est autre que \( \frac{ \partial y }{ \partial t }(t,\mu)\). La fonction \( y\) que nous avons définie vérifie donc
    \begin{equation}
        \frac{ \partial y }{ \partial t }(t,\mu)=f\big( t,y(t,\mu),\mu \big)
    \end{equation}
    et la condition initiale \( y(t_0)=x_1(t_0)=y_0\). Elle est donc bien solution du problème initial.

    De plus l'application \( (t,\mu)\mapsto y(t,\mu)=x_{(t_0,\mu)}(t)_1\) est de classe \( C^p\).

    \item[Unicité]

        Pour l'unicité, soit on invoque la proposition~\ref{THOooDTCWooSPKeYu} qui donne l'unicité dans les fonctions continues et a fortiori dans les fonctions \( C^1\). Soit on fait le jeu inverse : on prouve qu'à chaque solution de \eqref{SUBEQooXMYMooKfpqQW} correspond une solution de \eqref{SUBEQSooIBTNooHzYImh}, et l'unicité de la solution \( x\) donne l'unicité du côté de \( y\).
    \end{subproof}
\end{proof}

\begin{lemma}           \label{LEMooQWDNooOjNXhl}
    Soit le problème
    \begin{subequations}        \label{EQooKHSKooEFCsMQ}
        \begin{numcases}{}
            \frac{ \partial y }{ \partial s }(s)=f\big( y(s),s \big)\\
            y(t)=x
        \end{numcases}
    \end{subequations}
    avec \( t\) et \( x\) fixés. Nous supposons que \( f\) est de classe \( C^p\).

    Alors l'application \( t\mapsto  y_x(s)   \) est de classe \( C^p\).
\end{lemma}

\begin{proof}
    Soit \( t\) fixé, et l'équation différentielle
    \begin{subequations}
        \begin{numcases}{}
            \frac{ \partial z }{ \partial s }(s)a-=f\big( z(s),t-s \big)\\
            z(0)=x.
        \end{numcases}
    \end{subequations}
    Par le théorème~\ref{PROPooINLNooDVWaMn}, La solution \( z\) est de classe \( C^p\) en \( (s,x)\). En posant \( y(s)=z(t-s)\) il est vite vérifié que \( y\) est solution de \eqref{EQooKHSKooEFCsMQ}. C'est alors bien de classe \( C^p\) en \( t\).
\end{proof}

% This is part of Le Frido
% Copyright (c) 2008-2009,2011-2017, 2019
%   Laurent Claessens
% See the file fdl-1.3.txt for copying conditions.

%---------------------------------------------------------------------------------------------------------------------------
\subsection{Stabilité de Lyapunov}
%---------------------------------------------------------------------------------------------------------------------------

\begin{definition}  \label{DefKMCGooOeFKlA}
    Dans le cas de l'équation différentielle \( y'(t)=f\big( y(t),t \big)\) pour \( y\colon \eR\to \eR^n\), un point \( a\in \eR^n\) est un \defe{point d'équilibre}{équilibre!point point une équation différentielle} lorsque la fonction constante \( y(t)=a\) est une solution.

    Le point d'équilibre \( a\in \eR^n\) est \defe{stable}{point!d'équilibre!stable}\index{stabilité!d'un point d'équilibre} si pour tout \( \epsilon>0\), il existe \( \delta>0\) tel que \( \| y(0)-a \|<\delta\) implique \( \| y(t)-a \|<\epsilon\) pour tout \( t\).
\end{definition}

\begin{theorem}[Théorème de stabilité de Lyapunov\cite{MonCerveau,MJEooXxBFFY,PAXrsMn,GPRooZkclFA}]    \label{ThoBSEJooIcdHYp}
    Soit l'équation différentielle
    \begin{subequations}    \label{EqZZLBooKBkZkG}
        \begin{numcases}{}
            y'(t)=f(y)\\
            y(0)=y_0
        \end{numcases}
    \end{subequations}
    avec une fonction \( f\colon \eR^n\to \eR^n\) de classe \( C^1\) vérifiant \( f(0)=0\) et \( y_0\in \eR^n\). Nous supposons que l'application linéaire \( df_0\) n'a que des valeurs propres dont la partie réelle est strictement négative.

    Alors
    \begin{enumerate}
        \item   \label{ItemGZEAooAhxuDQi}
            Il existe \( k>0\) tel que si \( \| y_0 \|<k\) alors la solution maximale est définie sur \( \eR\),
        \item   \label{ItemGZEAooAhxuDQii}
            pour le même nombre \( k>0\), si \( \| y_0 \|<k\) alors \( y(t)\stackrel{t\to\infty}{\longrightarrow} 0\) exponentiellement vite,
        \item
            la solution \( y=0\) est un point d'équilibre attractif.
    \end{enumerate}

\end{theorem}
\index{théorème!stabilité de Lyapunov}
\index{stabilité!Lyapunov}

\begin{proof}
    Placer ici une phrase intelligente\footnote{Parce que sinon l'environnement \info{description} qui suit donne un mauvais effet.}.
    \begin{subproof}
        \item[Prolégomène]

    Le théorème de Cauchy-Lipschitz~\ref{ThokUUlgU} nous enseigne que l'équation différentielle considérée possède une unique solution maximale (entre autres parce qu'une fonction de classe \( C^1\) est localement Lipschitz) et nous nommons \( J\) l'intervalle sur lequel elle est définie.

\item[Système linéarisé]

    Nous posons \( A=df_0\). La fonction \( y_L(t)= e^{tA}y_0\) est solution du système linéarisé
    \begin{subequations}
        \begin{numcases}{}
            y'(t)=Ay(t)\\
            y(0)=y_0.
        \end{numcases}
    \end{subequations}
    Pour évaluer la norme de \( y_L\) nous utilisons le lemme~\ref{LemQEARooLRXEef} : il existe un polynôme \( P\) tel que
    \begin{equation}
        \| y_L(t) \|\leq P\big( | t | \big)\sum_{i=1}^r e^{\real{\lambda_i}t}\| y_0 \|.
    \end{equation}
    Mais par hypothèse, \( \real(\lambda_i)<0\) et si nous posons \( \lambda=\max\{ \real(\lambda_i) \}\) nous avons \( \lambda<0\) et
    \begin{equation}
        \| y_L(t) \|\leq P\big( | t | \big) e^{\lambda t}\| y_0 \|.
    \end{equation}
    Donc quel que soit \( y_0\) nous avons \( \lim_{t\to \infty} \| y_L(t) \|=0\) c'est-à-dire \( \lim_{t\to \infty} y_L(t)=0\).

\item[Une forme linéaire]

    Nous définissons la forme bilinéaire suivante sur \( \eR^n\) :
    \begin{equation}
        b(x,y)=\int_0^{\infty}\langle  e^{tA}x,  e^{tA}y\rangle dt.
    \end{equation}
    D'abord cela est bien défini pour tout \( x,y\in \eR^n\) parce que
    \begin{equation}
        \big| \langle  e^{tA}x,  e^{tA}y\rangle  \big|\leq \|  e^{tA}x \|\|  e^{tA}y \|\leq P_1\big( | t | \big)P_2\big( | t | \big) e^{2\lambda t}\| x \|\| y \|,
    \end{equation}
    qui est intégrable entre \( 0\) et \( \infty\) à cause de la décroissance exponentielle\footnote{Proposition~\ref{PropBQGBooHxNrrf}.}. Montrons que \( b\) est définie positive. Soit donc \( x\neq 0\) et calculons
    \begin{equation}
        b(x,x)=\int_0^{\infty}\|  e^{tA}x \|^2dt.
    \end{equation}
    Ce qui est dans l'intégrale est forcément (pas strictement) positif pour tout \( t\). Mais si \( x\neq 0\) alors \( \| x \|^2\) est strictement positif et sur un voisinage de \( t=0\) nous avons aussi \( \|  e^{tA}x \|^2\) qui est strictement positif. Ergo \( b(x,x)>0\) dès que \( x\neq 0\), ce qui signifie que \( b\) est strictement définie positive (lemme~\ref{LemWZFSooYvksjw}).

    Nous notons \( q\colon V\to \eR\) la forme quadratique associée à \( b\) et aussi la norme qui va avec : \( \| x \|_q=\sqrt{q(x)}\). En ce qui concerne le gradient \( \nabla q\colon V\to V\), nous avons le petit calcul suivant\cite{MJEooXxBFFY} qui se base sur une des nombreuses formules du lemme \ref{LemdfaSurLesPartielles}\footnote{Le fait que \( q\) soit différentiable est simplement le fait que \( b\) soit bilinéaire.} :
    \begin{subequations}
        \begin{align}
            \nabla q(x)\cdot y&=\Dsdd{ q(x+ty) }{t}{0}\\
            &=\Dsdd{ q(x)+t^2q(y)+2tb(x,y) }{t}{0}\\
            &=2b(x,y).
        \end{align}
    \end{subequations}
    Nous avons aussi
    \begin{subequations}
        \begin{align}
            \nabla q(x)\cdot Ax&=2b(x,Ax)\\
            &=2\int_0^{\infty}\langle  e^{tA}x,  e^{tA}Ax\rangle \\
            &=\int_0^{\infty}\frac{ \partial  }{ \partial t }\Big( \langle  e^{tA}x,  e^{tA}x\rangle  \Big)(t)dt\\
            &=\lim_{T\to \infty} \Big[ \langle  e^{tA}x,  e^{tA}x\rangle  \Big]_{t=0}^{t=T}.
        \end{align}
    \end{subequations}
    Mais vu que \( \|  e^{tA}x \|\to 0\), pour \( t\to \infty\) il ne reste que terme \( t=0\) de la différence, c'est-à-dire
    \begin{equation}    \label{EqUCOGooEFxZSO}
        \nabla q(x)\cdot Ax=2b(x,Ax)=-\| x \|^2.
    \end{equation}
    Étant donné que \( \nabla q(x)\) est le vecteur dirigé vers l'extérieur de l'ellipsoïde de la courbe de niveau de \( q\) au point \( x\), le vecteur \( Ax\) est dirigé vers l'intérieur.

\begin{center}
   \input{auto/pictures_tex/Fig_FNBQooYgkAmS.pstricks}
\end{center}

\item[Majoration de \(  q\big( y(t) \big)'  \)]
    Nous posons
    \begin{equation}
        \begin{aligned}
            r\colon \eR^n&\to \eR^n \\
            x&\mapsto f(x)-Ax.
        \end{aligned}
    \end{equation}

    Soit \( y\) la solution maximale au problème \eqref{EqZZLBooKBkZkG} que nous pouvons aussi écrire sous la forme
    \begin{equation}
        y'(t)=r\big( y(t) \big)+Ay(t).
    \end{equation}
    Calculons un peu \ldots
    \begin{subequations}    \label{subeqsZCOLooOzTBLr}
        \begin{align}
            q\big( y(t) \big)'&=b\big( y(t),y(t) \big)'\\
            &=2b(y,y')\\
            &=2b\big( y,Ay \big)+2b\big( y,r(y) \big)\\
            &=-\| y \|^2+2b\big( y,r(y) \big)       &\text{\eqref{EqUCOGooEFxZSO} avec } x=y(t)\\
            &\leq -\| y \|^2+2\| y(t) \|_q\| r\big( y(t) \big) \|_q &\text{Cauchy-Schwarz : } | b(a,b) |\leq \| a \|_q\| b \|_q\text{.}
        \end{align}
    \end{subequations}
    Chacun des deux termes peut encore être majoré. En ce qui concerne le premier, par équivalence des normes\footnote{Définition~\ref{DefEquivNorm} et théorème~\ref{ThoNormesEquiv}.}, il existe une constante \( C\) telle que \( \| y \|\geq C \| y \|_q\). En renommant immédiatement $C^2$ en \( C\), \( \| y \|^2\geq C\| y \|_q^2=Cq(y)\).

    Pour le second, nous allons utiliser la différentiabilité de \( r\) et le théorème des accroissements finis. Vu que \( df_0=A\) nous avons \( dr_0=df_0-A=0\) et de plus \( r\) est de classe \( C^1\) parce que \( f\) l'est. Toutes les normes étant équivalentes\footnote{Théorème~\ref{ThoNormesEquiv}.} sur \( \eR^n\) nous pouvons exprimer la continuité de \( dr\) pour la norme \( \| . \|_q\) : si \( \epsilon>0\) est fixé alors il existe \( \alpha>0\) tel que \( \| x \|<\alpha\) implique \( \| dr_x \|_q<\epsilon\). Nous pouvons écrire les accroissements finis\footnote{Théorème~\ref{ThoNAKKght}.} pour la fonction \( r\) :
    \begin{equation}    \label{EqIDTHooCsMSVs}
        \| r(x)-r(0) \|_q\leq \sup_{a\in\mathopen[ 0 , x \mathclose]}\| df_a \|\| x \|_q.
    \end{equation}
    La chose facile à remarquer est que \( r(0)=f(0)=0\). En ce qui concerne les choses difficiles, vu que \( dr\) est continue (parce que \( r\) est \( C^1\)) il existe un \( \delta>0\) tel que \( \| dr_a \|_q<\epsilon\) dès que \( a\in B_q(0,\delta)\). Si nous prenons \( \| x \|_q<\delta\) alors cette majoration est valable pour tous les éléments sur lequel est pris le supremum dans la formule \eqref{EqIDTHooCsMSVs}. Donc
    \begin{equation}
        \| r(x) \|_q\leq \epsilon\| x \|_q
    \end{equation}
    tant que \( \| x \|_q\leq \delta\). Par conséquent, tant que \(  \| y(t) \|_q\leq \delta\) nous avons \( \| r\big( y(t) \big) \|\leq \epsilon\| y(t) \|_q\). Nous continuons le calcul \eqref{subeqsZCOLooOzTBLr} :
    \begin{subequations}
        \begin{align}
            q\big( y(t) \big)'&\leq Cq(y)+2\epsilon\| y(t) \|_q^2\\
            &=-(C-2\epsilon)q(y).
        \end{align}
    \end{subequations}
    Si \( \epsilon\) est petit on a \( C-2\epsilon >0 \) et on pose \( \beta=C-2\epsilon\) pour écrire
    \begin{equation}    \label{EqEYJIooHvSBic}
        q\big( y(t) \big)'\leq -\beta q\big( y(t) \big)
    \end{equation}
    tant que \( \| y(t) \|_q<\delta\).

\item[\( t_1\) est un minimum]
    Le nombre \( t_1\) est bien défini et est bien un minimum. J'en veux pour preuve\quext{J'espère que ce passage est correct. faites moi savoir si vous trouvez une erreur ou si vous pouvez me confirmer que c'est bon.} que si \( q\big( y(t_s) \big)=\delta\), on peut prendre le minimum seulement sur les \( t\in\mathopen[ 0 , t_s \mathclose]\); or par continuité \( q\big( y(t) \big)=\delta\) définit un fermé. Bref \( t_1\) est un infimum sur un compact (fermé borné) et donc bien un minimum atteint.

\item[Si \( q(y_0)<\delta  \) alors \( q\big( y(t) \big) < \delta \)]

    Nous posons\footnote{}
    \begin{subequations}    \label{subeqsFNPJooERJkxO}
        \begin{align}
            t_1=\min\{ t>0\tq q\big( y(t) \big)=\delta \}\\
            t_2=\max\{ t<0\tq q\big( y(t) \big)=\delta \}.
        \end{align}
    \end{subequations}
    L'inégalité \eqref{EqEYJIooHvSBic} est valable pour \( t=0\), \( t=t_1\) et \( t=t_2\); nous l'écrivons pour \( t_1\) :
    \begin{equation}
        q\big( y(t) \big)'_{t=t_1}\leq-\beta q\big( y(t_1) \big)\leq-\beta \delta<0
    \end{equation}
    Nous avons donc \( q\big( y(t_1) \big)=\delta\) et \( q\big( y(t) \big)'_{t=t_1}<0\). Par conséquent pour tout \( t\) proche de \( t_1\) avec \( 0<t<t_1 \) il y a \( q\big( y(t) \big)>\delta\).


Pour la même raison, prise en \( t=0\) nous avons pour tout \( t\) proche de \( 0\) avec \( t>0\) que \( q\big( y(t) \big)<\delta\). Par continuité de \( t\mapsto q\big( y(t) \big)\) cette fonction doit passer par la valeur \( \delta\) dans \( \mathopen] t_2 , 0 \mathclose[\) et \( \mathopen] 0 , t_1 \mathclose[\), ce qui contredit la maximalité de \( t_2\) et la minimalité de \( t_1\).

    Ci-dessous, une partie de ce à quoi ressemble le graphe de \( t\mapsto q\big( y(t) \big)\) :
\begin{center}
   \input{auto/pictures_tex/Fig_ASHYooUVHkak.pstricks}
\end{center}

    Deux conclusions :
    \begin{itemize}
        \item
            Vu que \( q\big( y(t) \big)\) est borné pour tout \( t\in \eR\), nous sommes dans le cas~\ref{ItemOLYYooJVkRfj} de l'alternative du théorème d'explosion en temps fini~\ref{CorGDJQooNEIvpp}. Donc la solution \( y(t)\) existe sur tout \( \eR\) pourvu que \( \| y_0 \|\) soit assez petit. Plus précisément par équivalence des normes, il existe un nombre \( D>0\) tel que \( \| x \|\geq D\| x \|_q\) pour tout \( x\). Si \( \| y_0 \|\leq D\delta\) alors
            \begin{equation}
                D\| y_0 \|_q\leq \| y_0 \|\leq D\delta,
            \end{equation}
            qui donne immédiatement \( \| y_0 \|_q\leq \delta\), ce qui faut pour faire fonctionner l'existence de \( y(t)\) pour tout \( t\).
        \item
            Nous pouvons maintenant d'utiliser l'inégalité \eqref{EqEYJIooHvSBic} pour tout \( t\in \eR\) sous la seule hypothèse que \( q(y_0)<\delta\) au lieu de \( q\big( y(t) \big)<\delta\).
    \end{itemize}

    La partie~\ref{ItemGZEAooAhxuDQi} de ce théorème est prouvée; nous passons au reste à la partie~\ref{ItemGZEAooAhxuDQii}. Pour cela nous supposons que \( q(y_0)<\delta\).

\item[À propos de \(  e^{\beta t}q(y)\)]

    En sous-entendant la dépendance en \( t\) dans \( y\) nous avons
    \begin{equation}
        \Big(  e^{\beta t}q(y) \Big)'=\beta e^{\beta t}q(y)+ e^{\beta t}q(y)'= e^{\beta t}\big( \beta q(y)+q(y)' \big),
    \end{equation}
    mais nous avons déjà prouvé que \( q(y)'\leq -\beta q(y)\) (équation \eqref{EqEYJIooHvSBic}), donc
    \begin{equation}    \label{EqEJMEooFKuxTv}
        \Big(  e^{\beta t}q(y) \Big)'\leq 0
    \end{equation}

\item[Décroissance exponentielle]
    Si \(t\geq 0\), l'inégalité \eqref{EqEJMEooFKuxTv} donne
    \begin{equation}
        e^{\beta t}q\big( y(t) \big)\leq q(y_0),
    \end{equation}
    c'est-à-dire
    \begin{equation}
        q\big( y(t) \big)\leq  e^{-\beta t}q(y_0)
    \end{equation}
    lorsque \( t\geq 0\). Par équivalence des normes, nous avons des nombres \( D_1\) et \( D_2\) tels que
    \begin{equation}
        D_1\| x \|_q\leq \| x \|\leq D_2\| x \|_q
    \end{equation}
    pour tout \( x\in \eR^n\). Nous avons donc pour tout \( t\geq 0\) que
    \begin{equation}
        \| y(t) \|\leq D_2\| y(t) \|_q\leq D_2\| y_0 \|_q e^{-\beta t}.
    \end{equation}
    Pour rappel, \( \beta>0\), ce qui prouve la partie~\ref{ItemGZEAooAhxuDQii} du théorème.

\item[Point d'équilibre]

    Le point \( y=0\) est point d'équilibre (définition~\ref{DefKMCGooOeFKlA}) parce que \( f(0)=0\), donc \( y(t)=0\) fonctionne. Dans ce cas, \( y_0=0\).

\item[Stabilité]

La stabilité est le fait que \( \| y(t) \|_q\leq \delta\) dès que \( \| y_0 \|_q\leq \delta\).

\end{subproof}
\end{proof}

%---------------------------------------------------------------------------------------------------------------------------
\subsection{Système proies-prédateurs de Lotka-Volterra}
%---------------------------------------------------------------------------------------------------------------------------

Le système de \defe{Lotka-Volterra}{Lotka-Volterra} est l'équation différentielle suivante :
\begin{subequations}
    \begin{numcases}{}
        x'=ax-bxy\\
        y'=-cy+dxy
    \end{numcases}
\end{subequations}
où \( a,b,c,d\) sont des constantes positives, et avec la condition \( x(t_0)>0\), \( y(t_0)>0\).

En ce qui concerne l'interprétation des équations\cite{QUMHooCSloAC},
\begin{enumerate}
    \item
        \( x(t)\) est le nombres de proies,
    \item
        \( y(t)\) est le nombres de prédateurs,
    \item
        Les proies ont une reproduction rapide qui mène à une croissance exponentielle en absence de prédation (d'où le terme \( ax\)).
    \item
        Au contraire, les prédateurs meurent (ou migrent) rapidement lorsqu'ils n'ont pas de proies et nous supposons une décroissance exponentielle du nombre de prédateurs en l'absence de proies. D'où le terme \( -cy\) avec le signe négatif.
    \item
        Les termes \( -bxy\) et \( dxy\) sont les termes d'interaction entre les proies et les prédateurs. Ils sont proportionnels à la fréquence de leurs rencontres, lesquelles sont avantageuses pour les prédateurs et problématiques pour les proies.
\end{enumerate}

\begin{theorem}[Lotka-Volterra\cite{PAXrsMn}]            \label{ThoJHCLooHjeCvT}
    Soient des constantes positives \( a,b,c,d\) et le système d'équations différentielles
    \begin{subequations}
        \begin{numcases}{}
            x'=ax-bxy\\
            y'=-cy+dxy\\
            x(t_0)>0, y(t_0)>0.
        \end{numcases}
    \end{subequations}

    Alors
    \begin{enumerate}
        \item
            Les solutions sont positives sur leur domaines.
        \item
            Les solutions existent sur \( \eR\).
        \item
            Les solutions sont périodiques.
    \end{enumerate}
\end{theorem}
\index{théorème!Lotka-Volterra}

\begin{proof}
    Nous divisons la preuve.
    \begin{subproof}
    \item[Comment théorème de Cauchy-Lipschitz s'applique]
        Le théorème de Cauchy-Lipschitz\footnote{Théorème \ref{ThokUUlgU}.} ne peut pas s'appliquer tel quel parce qu'il demande une condition initiale pour avoir unicité. En ce qui concerne les notations, ce qui est noté «\( y\)» dans le théorème est ici le couple \( x,y\) et la fonction \( f\) est alors
        \begin{equation}
            f\big( t,\begin{pmatrix}
                x    \\
                y
            \end{pmatrix}\big)=\begin{pmatrix}
                ax-bxy    \\
                -cy+dxy
            \end{pmatrix}.
        \end{equation}
        C'est une fonction continue localement Lipschitz partout par le lemme~\ref{LemCFZUooVqZmpc} et la proposition~\ref{PropGIBZooVsIqfY}.

        Nous savons cependant que les solutions sont de classe \( C^1\) et que moyennant la donnée d'une condition initiale, la solution est unique.
    \item[Les solutions restent positives]
        Supposons \( x(s)=0\) pour un certain \( s>t_0\). Alors le solution
        \begin{subequations}
            \begin{numcases}{}
                x(t)=0\\
                y(t)=\exp(-ct)
            \end{numcases}
        \end{subequations}
        est une solution pour \( \mathopen[ t_0 , s+\epsilon \mathclose]\). Par unicité de la solution avec condition initiale \( s(s)=0\), nous avons aussi \( x(t_0)=0\) pour toutes les solutions, ce qui contredit notre condition.

        De la même façon, avoir \( y(s)=0\) donne une solution avec \( y(t)=0\) pour tout \( t\) et donc une contradiction.

    \item[Solutions sur \( \eR\)]

        Nous montrons maintenant que les solutions sont définies sur \( \eR\).

        Nous avons \( x'<ax\), donc pour tout \( t\) où la solution est définie,
        \begin{equation}
            0<x(t)<x(t_0) e^{a(t-t_0)},
        \end{equation}
        c'est-à-dire que la solution ne peut pas exploser en temps fini\footnote{Voir le corolaire~\ref{CorGDJQooNEIvpp}.} : elle est bornée par le haut et le bas. Elle doit donc exister pour tout \( t\in \eR\). Par ailleurs, \( y'<dxy\) donc
        \begin{equation}
            0<y(t)<y(t_0) e^{d\int_{t_0}^{t}x(s)ds}
        \end{equation}
        qui est également contraire à l'explosion en temps fini.

    \item[4 zones : monotonie]

        Nous divisons \( \eR^2\) en quatre zones d'après les signes de \( a-by\) et \( c-dx\). Nous montrons que dans chacune de ces zones, les solutions sont monotones. Prenons par exemple la partie
        \begin{equation}
            \{  (x,y)\in \eR^2\tq   a-by>0 \}\times\{ c-dx<0 \}.
        \end{equation}
        Vu l'équation \( x'=x(a-by)\), tant que \( \big( x(t),y(t) \big)\) est dans cette zone, la fonction \( x'\) a le signe de \( x\) et est donc positive. Donc \( x\) est croissante dans cette zone.

        De la même façon, \( y'=-y(c-dx)\), et \( y'\) a un signe constant dans la zone.

    \item[4 zones : on bouge]

        Nous prouvons à présent qu'une solution ne reste pas dans une zone.

        \begin{enumerate}
            \item
        Supposons que \( \big( x(t_0),y(t_0) \big) \) soit dans la zone
        \begin{subequations}
            \begin{align}
                \{ a-by>0 \}&&\times&& \{ c-dx>0 \}\\
                x'>0&&&&y'<0
            \end{align}
        \end{subequations}
        et que la solution reste dans cette zone (pour les \( t>t_0\)). Nous avons en particulier \( x'>0\), donc \( x\) est croissante tout en ayant la borne supérieure \(  x<c/d \). Par conséquent \( x\) a une limite que nous appelons \( x_1\in \mathopen[ 0 , \frac{ c }{ d } \mathclose]\).

        De la même façon,; \( y\) est décroissante et bornée vers le bas par zéro. Donc $y$ a une limite que nous notons \( y_1\in\mathopen[ 0 , y(t_0) \mathclose]\).

        Vu que \( x\) est bornée et de classe \( C^1\) nous avons forcément \( \lim_{t\to \infty} x'(t)=0\). Mais vu que \( x'=ax-bxy\) nous devons avoir
        \begin{equation}
            ax_1-bx_1y_1=0.
        \end{equation}
        Mais ni \( x_1>0\) donc \( a-by_1=0\), ce qui donne \( y_1=\frac{ a }{ b }\) et aussi \( x_1=\frac{ c }{ d }\). Bref, \( y\) est décroissante et tend vers \( a/b\); donc \( y(t_0)>a/b\), ce qui contredit que \( y(t_0)\) soit dans la zone considérée.

        Étant donné que \( x'>0\) et \( y'<0\), la solution sort de la zone pour entrer dans la zone \ldots
    \item
        Supposons que \( \big( x(t_0),y(t_0) \big) \) soit dans la zone
        \begin{subequations}
            \begin{align}
                \{ a-by>0 \}&&\times&& \{ c-dx<0 \}\\
                x'<0&&&&y'>0
            \end{align}
        \end{subequations}
        et que la solution reste dans cette zone (pour les \( t>t_0\)). Les fonctions \( x\) et \( y\) sont convergentes. Par conséquent \( \ln(y)\) converge aussi et vu que \( x\) est croissante,
        \begin{equation}
            \frac{ y' }{ y }=-c+dx\geq -x+dx(t_0)>0
        \end{equation}
        Cela signifie que \( \ln(y)'\) est toujours positive et bornée par le bas. Cela est impossible si \( y\) est borné.

        Donc on sort de la zone pour entrer dans \ldots
    \item
        Supposons que \( \big( x(t_0),y(t_0) \big) \) soit dans la zone
        \begin{subequations}
            \begin{align}
                \{ a-by<0 \}&&\times&& \{ c-dx<0 \}\\
                x'<0&&&&y'>0
            \end{align}
        \end{subequations}
        et que la solution reste dans cette zone (pour les \( t>t_0\)).

        Le même type de raisonnement fait passer à la zone\ldots
    \item
        Supposons que \( \big( x(t_0),y(t_0) \big) \) soit dans la zone
        \begin{subequations}
            \begin{align}
                \{ a-by<0 \}&&\times&& \{ c-dx>0 \}\\
                x'<0&&&&y'<0
            \end{align}
        \end{subequations}
        et que la solution reste dans cette zone (pour les \( t>t_0\)). Encore une fois, cela nous fait sortir de la zone et retourne vers la première zone.
        \end{enumerate}

       À ce moment nous voyons déjà que la relation entre proies et prédateurs, c'est un peu le mythe de Sisyphe\dots

   \item[Une intégrale première]

       Posons la fonction
       \begin{equation}
           H(x,y)=by+dx-a\ln(y)-c\ln(x).
       \end{equation}
       Une simple dérivation montre que  \(  x\mapsto H\big( x(t),y(t) \big) \) est constante. Nous considérons la fonction
       \begin{equation}
           \begin{aligned}
               f\colon \eR&\to \eR \\
               s&\mapsto H\big( \frac{ c }{ d },s \big)
           \end{aligned}
       \end{equation}
       dont la dérivée n'est autre que \( f'(s)=b-\frac{ a }{ s }\). La fonction \( f\) est donc décroissante sur l'intervalle \( \mathopen[ \frac{ a }{ b } , \infty [\) et donc injective. Sur les changements de zones, il existe un \( t_0\) tel que
           \begin{subequations}
               \begin{align}
                   x(t_0)&=\frac{ d }{ c }\\
                   y(t_0)&>0.
               \end{align}
           \end{subequations}
            Pour cette valeur \( t_0\) nous avons alors \( H\big( x(t_0),y(t_0) \big)=  f\big( y(t_0) \big)  \). En posant \( s_0=y(t_0)>0\) nous avons
            \begin{equation}
                H(x_0,y_0)=f(s_0)
            \end{equation}
            et \( f\) étant injective, ce \( s_0\) est la seule valeur de \( s\) à vérifier \( H(x_0,y_0)=f(s)\).

        \item[Conclusion]

            La fonction \( x\) passant d'une zone à l'autre, il existe un \( t_1>t_0\) tel que \( x(t_1)=a/b\). Nous avons évidemment
            \begin{equation}
                H\big( x(t_1),y(t_1) \big)=H(x_0,y_0)
            \end{equation}
            parce que \( H\) est constante le long du mouvement. Cela se traduit par
            \begin{equation}
                H\big( \frac{ a }{ b },y(t_1) \big)=f(s_0),
            \end{equation}
            et donc \( y(t_1)=f(s_0)=y(t_0)\). Avec tout cela nous avons
            \begin{subequations}
                \begin{numcases}{}
                    y(t_1)=y(t_0)\\
                    x(t_1)=x(t_2)=\frac{ a }{ b }.
                \end{numcases}
            \end{subequations}
            Cela est donc un point par lequel la solution repasse. Par unicité de la solution, elle est donc périodique.
    \end{subproof}
\end{proof}

%+++++++++++++++++++++++++++++++++++++++++++++++++++++++++++++++++++++++++++++++++++++++++++++++++++++++++++++++++++++++++++
\section{Équation du second ordre}
%+++++++++++++++++++++++++++++++++++++++++++++++++++++++++++++++++++++++++++++++++++++++++++++++++++++++++++++++++++++++++++

%---------------------------------------------------------------------------------------------------------------------------
\subsection{Wronskien}
%---------------------------------------------------------------------------------------------------------------------------

Nous considérons ici une équation différentielle de la forme
\begin{equation}    \label{EqJDAAnWY}
    y''(t)+q(t)y(t)=0
\end{equation}
Dans ce point nous allons considérer la fonction \( q\) sans hypothèse de périodicité. L'équation de Hill (sous-section~\ref{SubSecDWwVVPa}) sera la même équation, mais en supposant que \( q\) est périodique.

Nous commençons par argumenter que si \( q\) est continue, alors l'ensemble des solutions de l'équation \eqref{EqJDAAnWY} est un espace vectoriel de dimension deux. Pour cela il suffit d'appliquer la méthode de réduction de l'ordre (section~\ref{SecWGdleRM}) puis le théorème de dimension pour les systèmes linéaires (théorème~\ref{ThoNYEXqxO}). En effet si la fonction \( y_1\) est solution de \eqref{EqJDAAnWY} si et seulement si le vecteur \(Y= \begin{pmatrix}
    y_1    \\
    y_2
\end{pmatrix}\) est solution du système linéaire
\begin{equation}
    Y'(t)=\begin{pmatrix}
        0    &   1    \\
        -q(t)    &   0
    \end{pmatrix}Y(t).
\end{equation}

Soient deux solutions \( y_1\) et \( y_2\) de l'équation différentielle. Le \defe{Wronskien}{Wronskien} de ces deux solutions est le déterminant
\begin{equation}
    W(t)=\begin{vmatrix}
        y_1    &   y_2    \\
        y'_1    &   y'_2
    \end{vmatrix}.
\end{equation}
Si nous considérons l'équation différentielle
\begin{equation}
    y''+py'+qy=0,
\end{equation}
le Wronskien peut être déterminé sans savoir explicitement \( y_1\) et \( y_2\) parce que \( W=y_1y'_2-y'_1y_2\), et en dérivant,
\begin{subequations}
    \begin{align}
        W'&=y_1y_2''+y'_1y'_2-y''_1y_2-y'_1y'_2\\
        &=y_1(-py'_2-qy_2)-(-py'_1-qy_1)y_2\\
        &=-p\begin{vmatrix}
            y_1  & y_2    \\
            y'_1 & y'_2
        \end{vmatrix},
    \end{align}
\end{subequations}
c'est-à-dire
\begin{equation}    \label{EqHEMRgM}
    W'=-pW.
\end{equation}
Il suffit donc de savoir une condition initiale pour obtenir une équation différentielle pour \( W\).

%---------------------------------------------------------------------------------------------------------------------------
\subsection{Avec second membre}
%---------------------------------------------------------------------------------------------------------------------------

Une équation différentielle du second ordre avec un second membre se présente sous la forme
\begin{equation}
	ay''(t)+by'(t)+cy(t)=v(t)
\end{equation}
où $v(t)$ est une fonction donnée. Le truc est de commencer par résoudre l'équation différentielle sans second membre, c'est-à-dire trouver la fonction $y_H(t)$ telle que
\begin{equation}
	ay''_H(t)+by_H'(t)+cy_H(t)=0.
\end{equation}
Cela se fait en utilisant la méthode du polynôme caractéristique.

Ensuite, il faut trouver une solution particulière $y_P(t)$ de l'équation avec le second membre. Une seule. Pour y parvenir, il faut du doigté et un peu de technique. Il faut faire des essais en fonction de ce à quoi ressemble le $v(t)$ :
\begin{enumerate}

	\item
		Si $v(t)$ est un polynôme, alors il faut essayer un polynôme,

	\item
		Si $v(t)=\cos(\omega t)$ ou bien $v(t)=\sin(\omega t)$, alors essayer $y_P(t)=A\cos(t)+B\sin(\omega t)$,

	\item
		Si $v(t)= e^{\omega t}$, alors essayer $y_P(t)=A e^{\omega t}$.

\end{enumerate}

%---------------------------------------------------------------------------------------------------------------------------
\subsection{Équation \texorpdfstring{$y''+q(t)y=0$}{y''+q(t)y=0}}
%---------------------------------------------------------------------------------------------------------------------------
\label{subsecSyTwyM}


Nous allons donner quelques propriétés des solutions de l'équation
\begin{equation}
    y''+qy=0
\end{equation}
en fonction de telle ou telle hypothèse sur \( q\).

\begin{proposition}
    Si \( q\colon \eR^+\to \eR\) est continue et si
    \begin{equation}
        \int_0^{\infty}| q(t) |dt
    \end{equation}
    converge, alors
    \begin{enumerate}
        \item
            toute solution bornée de \( y''+qy=0\) vérifie \( \lim_{t\to \infty} y'(t)=0\),
        \item
            l'équation \( y''+qy=0\) admet des solutions non bornées.
    \end{enumerate}
\end{proposition}

\begin{proof}
    Soit \( y\) une solution bornée, et intégrons l'équation différentielle entre \( 0\) et \( \infty\) :
    \begin{equation}
        \int_0^{\infty}y''(t)dt=-\int_0^{\infty}q(t)y(t)dt.
    \end{equation}
    La fonction \( y\) étant bornée, l'hypothèse sur \( q\) permet de dire que l'intégrale de droite existe. Par ailleurs,
    \begin{equation}
        \int_0^{\infty}y''=\lim_{a\to \infty}\int_0^ay''=\lim_{a\to \infty}y'(a)-y'(0).
    \end{equation}
    Cela justifie que la limite \( \lim_{t\to \infty} y'(t)\) existe. Posons \( \alpha=\lim_{t\to \infty} y'(t)\) et supposons par l'absurde que \( \alpha\neq 0\). Soit \( \epsilon>0\) et \( \lambda\) assez grand pour que
    \begin{equation}
        \| y'-\alpha \|_{\mathopen[ \lambda , \infty [}<\epsilon.
    \end{equation}
    Soit aussi \( x>\lambda\). Nous avons
    \begin{subequations}
        \begin{align}
            y(x)&=y(\lambda) + \int_{\lambda}^x y'(t)dt\\
            &\geq y(\lambda) + \int_{\lambda}^x (\alpha-\epsilon) dt\\
            &=y(\lambda) + (\alpha-\epsilon)(x - \lambda).
        \end{align}
    \end{subequations}
    En prenant la limite des deux côtés on voit que \( y(x)\to \infty\) dès que \( \alpha\neq 0\), ce qui est contraire aux hypothèses. Donc \( \alpha=0\).

    Pour la seconde partie de la proposition, nous devons prouver que l'équation \( y''+qy=0\) possède des solutions non bornées. Si l'équation a seulement des solutions bornées et si \( \{ u,v \}\) est une base de solutions, alors nous avons \( u',v'\to 0\). Si nous reprenons l'équation \eqref{EqHEMRgM} avec \( p=0\) nous savons que dans notre cas le Wronskien satisfait à \( W'=0\), c'est-à-dire qu'il est constant. Mais vu que \( u\) et \( v\) sont bornées et que les dérivées tendent vers zéro, nous avons \( W(t)\to 0\) et donc \( W(t)=0\).

    Or l'annulation identique du Wronskien contredit que \( \{ u,v \}\) serait une base de solutions. Donc il existe des solutions non bornées.
\end{proof}

\begin{proposition} \label{PropMYskGa}
    Soit l'équation différentielle \( y''+qy=0\). Si \( q\) est \( C^1\), strictement positive et croissante, alors toutes les solutions sont bornées.
\end{proposition}
\index{monotonie}

\begin{proof}
    Soit \( y\) une solution et multiplions l'équation par \( 2y'\) (qui est non nulle par hypothèse) :
    \begin{equation}
        2y'y''+2qy'y=0.
    \end{equation}
    Nous allons intégrer cela en nous souvenant que \( 2y'y''\) est la dérivée de \( (y')^2\). Pour tout \( t>0\) nous avons
    \begin{subequations}
        \begin{align}
            0&=y'(t)^2-y'(0)^2+2\underbrace{\int_0^tq(t)y'(t)y(t)dt}_{\text{par partie}}\\
            &=y'(t)^2-y'(0)^2+2\left( [qy^2]_0^t-\int_0^tq'y^2 \right)\\
        \end{align}
    \end{subequations}
    Le terme qui nous intéresse est celui qui contient \( y(t)\) :
    \begin{equation}
        2q(t)y(t)^2=-y'(t)^2+y'(0)^2+2q(0)y(0)^2+2\int_0^t q'y^2
    \end{equation}
    Nous pouvons majorer \( -y'(t)^2\) par zéro et remplacer toutes les constantes par \( K\) :
    \begin{equation}
        q(t)y(t)^2\leq\int_0^tq'y^2+K=\int_0^t\frac{ q' }{ q }qy^2.
    \end{equation}
    C'est le moment d'utiliser le lemme de Grönwall~\ref{LemuBVozy} avec \( \phi=qy^2\) et \( \psi=q'/q\). Les hypothèses de croissance et de positivité ont été posées exprès. Bref, on a
    \begin{subequations}
        \begin{align}
            qy^2&\leq K\exp\left( \int_0^t\frac{ q'(s) }{ q(s) }ds \right)\\
            &=K\exp\left( \ln\frac{ q(t) }{ q(0) } \right)\\
            &=K\frac{ q(t) }{ q(0) }.
        \end{align}
    \end{subequations}
    Notons que \( q(0)\) est strictement positif. Nous déduisons que
    \begin{equation}
        y^2\leq \frac{ K }{ q(0) }
    \end{equation}
    et donc \( y\) est bornée.
\end{proof}

%---------------------------------------------------------------------------------------------------------------------------
\subsection{Équation de Hill}
%---------------------------------------------------------------------------------------------------------------------------
\label{SubSecDWwVVPa}

L'équation \defe{de Hill}{équation!différentielle!Hill} est une équation différentielle de la forme
\begin{equation}    \label{EqPQMvzEZ}
    y''+qy=0
\end{equation}
où
\begin{enumerate}
    \item
        \( q\in C^1(\eR,\eR)\),
    \item
        \( q\) est paire et \( \pi\)-périodique
\end{enumerate}
Nous nous intéressons aux solutions complexes de cette équation différentielle.

Nous nommons \( W\subset C^2(\eR,\eC)\) l'espace des solutions complexes de l'équation \eqref{EqPQMvzEZ}. Nous savons par ce qui a été dit en~\ref{subsecSyTwyM} que cet espace est de dimension deux. De plus avec les hypothèses faites ici sur \( q\), nous savons que les solutions sont de classe $C^3$ parce que si \( y\) est une solution, alors l'équation \( y''=qy\) nous indique que \( y\) est \( C^1\) parce que \( y''\) existe (\( y'\) est dérivable et donc continue). Mais si \( y\) est de classe \( C^1\), alors le membre de droite \( qy\) est \( C^1\) et donc \( y''\) est \( C^1\), ce qui prouve que \( y\) est de classe \( C^3\). La récurrence ne va pas plus loin parce que \( q\) est seulement de classe \( C^1\).

Nous considérons l'application de translation
\begin{equation}
    \begin{aligned}
        T\colon C^2(\eR,\eC)&\to C^2(\eR,\eC) \\
        (Ty)(x)&=y(x+\pi).
    \end{aligned}
\end{equation}
En utilisant la règle de dérivation de fonctions composées, \( (Ty)'=Ty'\) et \( (Ty)''=Ty''\), de telle sorte que si \( u\) est solution de l'équation \eqref{EqPQMvzEZ}, alors \( Tu\) est également solution. Donc \( W\) est un espace stable par \( T\).

Le théorème~\ref{ThoNYEXqxO} nous permet de choisir une base de \( W\) en imposant des conditions. Nous choisissons une base \( \{ y_1,y_2 \}\) telles que
\begin{equation}
    \begin{aligned}[]
        y_1(0)&=1       &&  y_2(0)=0\\
        y'_1(0)&=0      &&  y_2'(0)=1.
    \end{aligned}
\end{equation}
Le théorème~\ref{ThoNYEXqxO} nous assure que deux telles solutions existent et qu'elles forment une base de \( W\) parce que \( W\) est de dimension \( 2\).

\begin{lemma}[\cite{WNxwuWc}]   \label{IVLzNaU}
    Avec ce choix de base \( \{ y_1,y_2 \}\) la matrice de \( T\) est donnée par
    \begin{equation}
        T=\begin{pmatrix}
            y_1(\pi)    &   y_2(\pi)    \\
            y'_1(\pi)    &   y'_2(\pi)
        \end{pmatrix}.
    \end{equation}
    De plus la fonction \( y_1\) est paire et la fonction \( y_2\) est impaire.
\end{lemma}

\begin{proof}

Cherchons la matrice de \( T\) dans cette base en associant \( \begin{pmatrix}
    1    \\
    0
\end{pmatrix}\) à \( y_1\) et \( \begin{pmatrix}
    0    \\
    1
\end{pmatrix}\) à \( y_2\). Si \( T=\begin{pmatrix}
    a    &   b    \\
    c    &   d
\end{pmatrix}\), alors
\begin{equation}    \label{EqSZhBPGy}
    Ty_1=\begin{pmatrix}
        a    &   b    \\
        c    &   d
    \end{pmatrix}\begin{pmatrix}
        1    \\
        0
    \end{pmatrix}=\begin{pmatrix}
        a    \\
        c
    \end{pmatrix}=ay_1+cy_2.
\end{equation}
En évaluant cela en \( t=0\),
\begin{equation}
    (Ty_1)(0)=ay_1(0)+cy_2(0)=a,
\end{equation}
donc \(a=(Ty_1)(0)=y_1(\pi)\). En dérivant \eqref{EqSZhBPGy}, en tenant compte du fait que \( (Ty_1)'=Ty_1'\) et en évaluant en \( t=0\), nous trouvons de même \( c=y'_1(\pi)\). Puis le même cinéma avec \( y_2\) donne
\begin{equation}
    T=\begin{pmatrix}
        y_1(\pi)    &   y_2(\pi)    \\
        y'_1(\pi)    &   y'_2(\pi)
    \end{pmatrix}.
\end{equation}

Passons maintenant à la parité de \( y_1\) et \( y_2\). Nous posons \( \psi(t)=y_1(-t)\). Alors \( \psi'(t)=-y_1'(-t)\) et \( \psi''(t)=y_1''(t)\), tant et si bien que
\begin{equation}
    \psi''(t)+q(t)\psi(t)=y_1''(-t)+q(t)y_1(-t)=0.
\end{equation}
donc \( \psi\) est une solution de l'équation. Mais
\begin{subequations}
    \begin{numcases}{}
        \psi(0)=y_1(0)\\
       \psi'(0)=-y'_1(0)=0,
    \end{numcases}
\end{subequations}
donc \( \psi\) a les mêmes conditions initiales que \( y_1\). Par conséquent \( \psi=y_1\) (par le l'unicité donnée dans le théorème de Cauchy-Lipschitz~\ref{ThokUUlgU}) et \( y_1\) est paire. Nous procédons de même en partant de \( \varphi(t)=-y_2(-t)\) pour trouver que \( \varphi=y_2\) et que donc que \( y_2\) est impaire.


\end{proof}
Remémorons nous toutefois, pour calmer tout enthousiasme excessif, que \( T\) dépend de deux solutions et donc de la fonction \( q\) donnée dans l'équation.

\begin{proposition}[\cite{KXjFWKA}] \label{PropGJCZcjR}
    Nous considérons l'équation \( y''+qy=0\) et sa base de solutions \( \{ y_1,y_2 \}\) en suivant les notations données plus haut.
    \begin{enumerate}
        \item
            Si \( | \tr(T) |<2\), alors toutes les solutions de l'équation sont bornées.
        \item
            Si \( | \tr(T) |=2\) alors nous avons une solution non bornée.
        \item
            Si \( |\tr(T)|>2\) alors toutes les solutions de l'équation sont non bornées.
        \item
            Le cas \( | \tr(T) |=2\) se présente si et seulement si \( y'_1(\pi)y_2(\pi)=0\).
    \end{enumerate}
\end{proposition}
\index{endomorphisme!sous-espace stable}
\index{endomorphisme!diagonalisable}
\index{équation!différentielle!étude qualitative}
\index{équation!différentielle!système}


\begin{proof}
    Remarquons que le déterminant de la matrice \( T\) est égal au Wronskien des solutions \( y_1\) et \( y_2\) calculé en \( t=\pi\). Calculons sa valeur :
    \begin{equation}
        W(y_1,y_2)=\det\begin{pmatrix}
            y_1    &   y_2    \\
            y'_1    &   y'_2
        \end{pmatrix}=y_1y'_2-y'_1y'_2.
    \end{equation}
    En dérivant et en remplaçant \( y''_i\) par \( -qy_i\), nous trouvons tout de suite \( W(y_1,y_2)'=0\). Donc le Wronskien est constant et il est facile de le calculer en \( t=0\) :
    \begin{equation}
        W(y_1,y_2)(0)=1-0=1.
    \end{equation}
    Donc pour tout \( t\) nous avons \( W(y_1,y_2)(t)=1\). En particulier
    \begin{equation}
        \det(T)=W(y_1,y_2)(\pi)=1,
    \end{equation}
    et notons au passage que \( T\) est inversible.

    Nous écrivons le polynôme caractéristique de \( T\) sous la forme \( \chi_T=X^2-\tr(T)X+\det(T)\), c'est-à-dire
    \begin{equation}
        \chi_T=X^2-\tr(T)X+1,
    \end{equation}
    dont le discriminant est \( \Delta=\tr(A)^2-4\).

    Nous passons à présent aux différents points de la proposition.
    \begin{enumerate}
        \item
            Si \( | \tr(T) |<2\), alors \( \Delta<0\) et \( \chi_T\) a deux racines complexes conjuguées que nous notons \( \rho\) et \( \bar\rho\). De plus le produit des racines étant le terme indépendant, \( \rho\bar\rho=1\); en particulier \( | \rho |=| \bar \rho |=1\). Notons \( \{ u,v \}\) une base de vecteurs propres : \( Tu=\rho u\) et \( Tv=\bar \rho v\). Il est vite vu que la fonction \( | u |\) est \( \pi\)-périodique :
            \begin{equation}
                | u |(t+\pi)=| u(t+\pi) |=| (Tu)(t) |=| (\rho u)(t) |=| \rho | | u |(t)=| u |(t).
            \end{equation}
            La fonction \( | u |\) est continue\footnote{La fonction \( u\) elle-même n'est cependant pas garantie d'être périodique.} et périodique ergo bornée. La fonction \( | v |\) est bornée pour la même raison et par linéarité, toutes les fonctions de \( W\) sont bornées.

        \item

            Si \( \tr(T)=\pm 2\), alors \( \Delta=0\) et \( \chi_T\) a une racine réelle double\footnote{Ce qui n'implique pas le fait d'avoir deux vecteurs propres pour cette valeur propre, mais tout de même au moins un, voir l'exemple~\ref{ExICOJcFp}.} qui doit être \( \pm 1\). Soit \( u\) un vecteur propre de \( T\) pour la valeur propre \( \pm 1\). Nous avons
            \begin{equation}
                | u |(t+\pi)=| Tu(t) |=| \pm u(t) |,
            \end{equation}
            ce qui prouve encore que \( | u |\) est périodique et donc bornée.

            Notons que nous n'avons pas d'informations sur le fait qu'une autre solution soit ou non bornée.

        \item

            Si \( | \tr(T) |>2\), alors \( \chi_T\) a deux racines réelles distinctes \( r\) et \( r'\) avec \( rr'=1\) (toujours les relations coefficients-racines). En raison de quoi \( r'=r^{-1}\) et quitte à échanger \( r\) et \( r'\) nous supposons \( | r |>1\). L'opérateur est maintenant diagonalisable et nous considérons \( \{ u,v \}\) une base de vecteurs propres pour les valeurs propres \( r\) et \( r'\). Une solution non nulle de l'équation s'écrit donc sous la forme
            \begin{equation}
                y=\alpha u+\beta v
            \end{equation}
            avec \( (\alpha,\beta)\neq (0,0)\).

            \begin{itemize}
                \item Si \( \alpha=0\), alors \( \beta\neq 0\) et nous choisissons une valeur \( t\) telle que \( v(t)\neq 0\). Dans ce cas,
                    \begin{equation}
                        y(t+n\pi)=\beta v(t+n\pi)=\beta(T^nv)(t)=\beta (r')^n v(t),
                    \end{equation}
                    et en faisant \( n\to -\infty\) nous obtenons \( \pm \infty\) suivant le signe de \( \beta\).

                \item Si \( \alpha\neq 0\), alors nous fixons\footnote{Mais pas trop hein; nous aurons encore besoin d'assigner à \( t\) d'autres valeurs dans d'autres théorèmes.} \( t\) tel que \( u(t)\neq 0\). Alors
                    \begin{equation}
                        y(t+n\pi)=\alpha r^nu(t)+\beta (r')^n(t).
                    \end{equation}
                    En faisant \( n\to \infty\), nous avons \( (r')^n\to 0\) tandis que le premier terme tend vers \( \pm\infty\) suivant le signe de \( \alpha\).
            \end{itemize}

        \item

            D'abord le théorème de Cayley-Hamilton~\ref{ThoCalYWLbJQ} nous indique que \( \chi_T(T)=0\), c'est-à-dire que
            \begin{equation}    \label{EqFHVSsUO}
                T^2-\tr(T)T+1=0.
            \end{equation}
            Nous avons déjà mentionné le fait que \( T\) était inversible. Multiplions donc \eqref{EqFHVSsUO} par \( T^{-1}\) :
            \begin{equation}    \label{EqPNyjBOy}
                T+T^{-1}=\tr(T)\mtu_2.
            \end{equation}
            Vu que \( T^{-1}\) est l'endomorphisme \( T^{-1}u(t)=u(t-\pi)\), sa matrice est donnée par
            \begin{equation}
                T^{-1}=\begin{pmatrix}
                    y_1(-\pi)    &   y_2(-\pi)    \\
                    y'_1(-\pi)    &   y'_2(-\pi)
                \end{pmatrix}=\begin{pmatrix}
                    y_1(\pi)    &   -y_2(\pi)    \\
                    -y'_1(\pi)    &   y'_2(\pi)
                \end{pmatrix}
            \end{equation}
            où nous avons utilisé le fait que \( y_1\) était paire et \( y_2\) impaire (lemme~\ref{IVLzNaU}). Si nous notons \( T=\begin{pmatrix}
                a    &   b    \\
                c    &   d
            \end{pmatrix}\), alors \( T^{-1}=\begin{pmatrix}
                a    &   -b    \\
                -c    &   d
            \end{pmatrix}\) et
            \begin{equation}
                T+T^{-1}=\begin{pmatrix}
                    2a    &     0  \\
                       0 &   2b
                \end{pmatrix}.
            \end{equation}
            L'équation \eqref{EqPNyjBOy} donne alors, vu que \( \tr(T)=a+d\),
            \begin{equation}
                \begin{pmatrix}
                    2a    &   0    \\
                    0    &   2b
                \end{pmatrix}=\begin{pmatrix}
                    a+d    &   0    \\
                    0    &   a+d
                \end{pmatrix},
            \end{equation}
            ce qui donne immédiatement \( a=d\). La matrice de \( T\) a donc comme forme \( T=\begin{pmatrix}
                a    &   b    \\
                c    &   a
            \end{pmatrix}\) et \( \tr(T)=2a\).

            Donc \( \tr(T)=\pm 2\) si et seulement si \( a=\pm 1\) et vu que \( 1=\det(T)=a^2-bc\), nous avons \( a=\pm 1\) si et seulement si \( bc=0\), ce qui signifie exactement \( y'_1(\pi)y_2(\pi)=0\).
    \end{enumerate}
\end{proof}

%+++++++++++++++++++++++++++++++++++++++++++++++++++++++++++++++++++++++++++++++++++++++++++++++++++++++++++++++++++++++++++
\section{Différents types d'équations différentielles}
%+++++++++++++++++++++++++++++++++++++++++++++++++++++++++++++++++++++++++++++++++++++++++++++++++++++++++++++++++++++++++++

%---------------------------------------------------------------------------------------------------------------------------
\subsection{Équation homogène}
%---------------------------------------------------------------------------------------------------------------------------
\label{SubSecEqDiffHomo}

Une équation différentielle \defe{homogène}{équation!différentielle!homogène} est une équation de la forme
\begin{equation}
	y'=f(t,y)
\end{equation}
où $f(\lambda t,\lambda y)=f(t,y)$ pour tout $\lambda\neq 0$.

Elle se présente sous la forme
\begin{equation}
    y'=\frac{ \text{degré } n \text{ en } t,y }{ \text{degré } n \text{ en } t,y },
\end{equation}
avec pas de $y'$ à droite : juste du $y$ et du $t$.

\begin{lemma}
L'équation $y'=f(t,y)$ est homogène si et seulement si $f(t,y)$ est une fonction de $y/t$ seulement.
\end{lemma}
Pour résoudre l'équation homogène, on pose
\begin{equation}		\label{EqDiffHomoPoser}
	z(t)=\frac{ y(t) }{ t },
\end{equation}
donc $tz=y$, et
\begin{equation}
	y'(t)=tv'(t)+v(t),
\end{equation}
à remettre dans l'équation de départ.
%---------------------------------------------------------------------------------------------------------------------------
\subsection{Équation de Bernoulli}
%---------------------------------------------------------------------------------------------------------------------------
\label{SubSecBernh}

C'est une équation du type
\begin{equation}	\label{EqBerNDiffalp}
	y'=a(t)y+b(t)y^{\alpha}
\end{equation}
où $\alpha\neq 0$ ou $1$. Pour la résoudre, on divise l'équation par $y^{\alpha}$, et on pose $u=y^{1-\alpha}$, et on tombe sur une équation linéaire
\begin{equation}
	u'=(1-\alpha)\big( a(t)u+b(t) \big).
\end{equation}

%---------------------------------------------------------------------------------------------------------------------------
\subsection{Équation de \href{https://fr.wikipedia.org/wiki/Jacopo_Riccati}{Riccati}}
%---------------------------------------------------------------------------------------------------------------------------
\label{SubSecRicatti}

C'est une équation de la forme
\begin{equation}		\label{EqDiffGFeneRicatti}
	y'=a(t)y^2+b(t)y+c(t).
\end{equation}
\index{équation!de Riccati}

En général, on ne peut pas la résoudre, mais si on en connaît \emph{à priori} des solutions particulières, alors on peut s'en sortir.

\begin{enumerate}

\item
Si on sait que $y_1(t)$ est une solution, alors on pose
\begin{equation}
	y(t)=y_1(t)+\frac{1}{ u(t) },
\end{equation}
et on obtient une équation linéaire
\begin{equation}
	u'=-\big( 2y_1(t)a(t)+b(t) \big)u-a(t).
\end{equation}

\item
Si $y_1$ et $y_2$ sont solutions, alors nous avons $y$ sous forme implicite
\begin{equation}
	\frac{ y-y_1 }{ y-y_2 }=K e^{\int a(t)\big( y_1(t)-y_2(t) \big)dt}.
\end{equation}
\end{enumerate}

Pour résoudre une équation de Ricatti, il faut donc d'abord deviner une ou deux solutions.

%---------------------------------------------------------------------------------------------------------------------------
\subsection{Équation différentielle exacte}
%---------------------------------------------------------------------------------------------------------------------------
\label{SubSecEqDiffExacte}

%///////////////////////////////////////////////////////////////////////////////////////////////////////////////////////////
\subsubsection{Résolution lorsque tout va bien}
%///////////////////////////////////////////////////////////////////////////////////////////////////////////////////////////

Avant de vous lancer dans les équations différentielles exacte, vous devez lire la section sur les formes différentielles~\ref{SecFormDiffRappel}. Une équation différentielle exacte est de la forme $P(t,y)+Q(t,y)y'=0$ que nous allons écrire sous la forme
\begin{equation}		\label{EqExacteDiff}
	P(t,y)dt+Q(t,y)dy=0.
\end{equation}
Nous savons que si $\partial_yP=\partial_tQ$, alors il existe une fonction $f(t,y)$ telle que $Pdt+Qdy=df$. Pour trouver une telle fonction, nous pouvons simplement intégrer la forme $Pdt+Qdy$. En effet, si $\gamma\colon [0,1]\to \eR^2$ est un chemin tel que $\gamma(0)=(0,0)$ et $\gamma(1)=(t,y)$, alors en définissant
\begin{equation}
	f(t,y)=\int_{\gamma}[Pdt+Qdt]=\int_{0}^1\big[ (P\circ\gamma)(u)dt+(Q\circ\gamma)(u) \big]\big( \gamma'(u) \big)du,
\end{equation}
nous avons $df=Pdt+Qdy$. N'importe quel chemin fait l'affaire. Calculons avec $\gamma(u)=(tu,yu)$. La dérivée de ce chemin est donnée par
\begin{equation}
	\gamma'(u)=t\begin{pmatrix}
	1	\\
	0
\end{pmatrix}+y\begin{pmatrix}
	0	\\
	1
\end{pmatrix}.
\end{equation}
Étant donné que $dt\begin{pmatrix}
	a	\\
	b
\end{pmatrix}=a$ et $dy\begin{pmatrix}
	a	\\
	b
\end{pmatrix}=b$, nous avons
\begin{equation}
	\begin{aligned}[]
	f(t,y)&=\int_0^1[Pdt+Qdy]\big( \gamma(u) \big)\left( t\begin{pmatrix}
	1	\\
	0
\end{pmatrix}+y\begin{pmatrix}
	0	\\
	1
\end{pmatrix} \right)du\\
		&=\int_0^1P\big( \gamma(t) \big)tdu+\int_0^1Q\big( \gamma(t) \big)ydu\\
		&=\int_0^1\big[ tP(tu,uy)+yQ(tu,yu) \big]du.
	\end{aligned}
\end{equation}
Nous retrouvons exactement la formule \eqref{EqIMFormI33Fffdd}. Si ça t'étonne, c'est que tu n'as pas compris ;) Dans le cas où nous avons la fonction $f$ qui vérifie $P=\partial_tf$ et $Q=\partial_yf$, l'équation \eqref{EqExacteDiff} devient
\begin{equation}
	\frac{ \partial f }{ \partial t }+\frac{ \partial f }{ \partial y }\frac{ dy }{ dt }=0,
\end{equation}
c'est-à-dire
\begin{equation}
	\frac{ d }{ dt }\Big[ f\big( t,y(t) \big) \Big]=0,
\end{equation}
dont la solution
\begin{equation}
	f\big( t,y(t) \big)=C
\end{equation}
donne la solution $y(t)$ sous forme implicite.

%///////////////////////////////////////////////////////////////////////////////////////////////////////////////////////////
\subsubsection{Facteur intégrant (quand tout ne va pas bien)}
%///////////////////////////////////////////////////////////////////////////////////////////////////////////////////////////

Si la forme $Pdt+Qdy$ n'est pas exacte, il n'existe pas de fonction $f$ qui résolve l'affaire. Nous pouvons toutefois essayer de trouver un \defe{facteur intégrant}{facteur!intégrant}. Nous cherchons une fonction $M$ telle que
\begin{equation}
	(MP)dt+(MQ)dy
\end{equation}
soit exacte. Nous cherchons donc $M(t,y)$ telle que $\partial_y(MP)=\partial_t(MQ)$. En utilisant la règle de Leibnitz, nous trouvons l'équation suivante pour $M$ :
\begin{equation}		\label{EqDuFacteurIntegrant}
	M(\partial_yP-\partial_tQ)=Q(\partial_tM)-P(\partial_yM).
\end{equation}
Cette équation est en générale extrêmement difficile à résoudre, mais dans certains cas particuliers, il est possible d'en trouver une solution à tâtons.

%+++++++++++++++++++++++++++++++++++++++++++++++++++++++++++++++++++++++++++++++++++++++++++++++++++++++++++++++++++++++++++
\section{Distributions pour les équations différentielles}
%+++++++++++++++++++++++++++++++++++++++++++++++++++++++++++++++++++++++++++++++++++++++++++++++++++++++++++++++++++++++++++
\label{SecTNgeNms}

Nous commençons par définir l'espace \(  C^{\infty}\big( \eR,\swS'(\eR^d) \big)\)\nomenclature[Y]{$C^{\infty}\big( \eR,\swS'(\eR^d) \big)$}{Fonctions à valeurs dans les distributions.} en disant que \( t\mapsto u_t\) est dans cet espace si
\begin{enumerate}
    \item
        pour tout \( t\in \eR\) nous avons \( u_t\in \swS'(\eR^d)\),
    \item
        l'application \( t\mapsto u_t\) est de classe \(  C^{\infty}\).
\end{enumerate}
Pour définir ce que nous entendons par une fonction de classe \( C^k\) à valeurs dans \( \swS'(\eR^d)\) nous nous souvenons de la proposition~\ref{PropQAuJstI}.

%---------------------------------------------------------------------------------------------------------------------------
\subsection{Équation de Schrödinger}
%---------------------------------------------------------------------------------------------------------------------------

\begin{theorem}[Équation de Schrödinger\cite{KXjFWKA}]    \label{ThoLDmNnBR}
    Soit \( g\in\swS'(\eR^d)\) et le problème
    \begin{subequations}
        \begin{numcases}{}
            \partial_t\tilde u-i\Delta \tilde u=0   \label{EqIKhGuiq}\\
            u_0=g
        \end{numcases}
    \end{subequations}
    où \( \tilde u\in C^{\infty}\big( \eR,\swD'(\eR^d) \big)\) est lié à \( u\) par la remarque~\ref{RemZYVkHRT}. Alors
    \begin{enumerate}
        \item   \label{ItemVFracYji}
            Il existe une unique solution dans \( C^{\infty}\big( \eR,\swS'(\eR^d) \big)\).
        \item   \label{ItemVFracYjiii}
            Cette solution \( u\) vérifie de plus \( \tilde u\in\swS'(\eR\times \eR^d)\).
    \end{enumerate}
\end{theorem}
\index{Schrödinger}
\index{distribution!équation de Schrödinger}

\begin{proof}
    Nous allons donner explicitement une fonction \( u\in C^{\infty}\big( \eR,\swS'(\eR^d) \big)\) et nous allons vérifier l'équation \eqref{EqIKhGuiq} en testant sur une fonction \( \psi\in\swS'(\eR\times \eR^d)\). Cela prouvera le point~\ref{ItemVFracYjiii} ainsi que la partie existence de~\ref{ItemVFracYji}. Dans ce qui suit toutes les transformées de Fourier seront par rapport à la variable \( x\in \eR^d\) ou par rapport à \( \xi\). Jamais par rapport à \( t\in \eR\).

    \begin{subproof}
    \item[Existence]
        Pour \( t\in \eR\) nous posons\footnote{En utilisant la définition \eqref{DefTDkrqkA} du produit d'une distribution par une fonction.}
        \begin{equation}
            u_t=\TF^{-1}(f_t\hat g)
        \end{equation}
        où \( f_t\in\swS(\eR^d)\) est la fonction \( f_t(x)= e^{-it\| x \|^2}\). Pour toute fonction \( \varphi\in\swS(\eR^d)\) nous avons
        \begin{equation}
            u_t(\varphi)=(f\hat g)\big( \TF^{-1}(\varphi) \big)=\hat g\big( f\TF^{-1}(\varphi) \big)=g\Big( \TF\big( f\TF^{-1}(\varphi) \big) \Big).
        \end{equation}
        Le fait que \( \TF^{-1}(\varphi)\) soit une fonction Schwartz fait partie de la proposition~\ref{PROPooLWTJooReGlaN}. Pour chaque \( t\) nous avons bien \( u_t\in\swS'(\Omega)\).

        De plus la fonction \( h(t,x)= e^{-it\| x \|^2}(\TF^{-1}\varphi)(x)\) est dans \(  C^{\infty}(\eR\times \eR^d)\), et par conséquent l'application
        \begin{equation}
            t\mapsto \hat g\big( h(t,.) \big)
        \end{equation}
        est également \(  C^{\infty}\) par la proposition~\ref{PropBQUOcyw}. Ceci pour dire que \( u\in C^{\infty}\big( \eR,\swS'(\eR^d) \big)\). Il faut encore vérifier que cette fonction est bien une solution de notre problème. Nous testons cette équation sur \( \psi\in\swS(\eR\times\eR^d)\). Pour alléger les notations nous posons \( \psi_t\colon x\mapsto \psi(t,x)\) et par conséquent aussi \( (\partial_t\psi_t)(x)=(\partial_t\psi)(t,x)\). Nous avons :
        \begin{subequations}
            \begin{align}
                \heartsuit&=(\partial_t\tilde u-i\Delta\tilde u)(\psi)\\
                &=-\tilde u(\partial_t\psi)-i\tilde u(\Delta\psi)\\
                &=-\int_{\eR}u_t\big( (\partial_t\psi_t)+i(\Delta\psi_t) \big)dt
            \end{align}
        \end{subequations}
        Ici nous nous souvenons du lemme~\ref{LemYYjFZSa} qui nous dit que nous pouvons permuter \( \TF^{-1}\) et \( \partial_t\). Et pour l'autre terme il faut utiliser le lemme~\ref{LemQPVQjCx} avec \( | \alpha |=2\) et une somme pour obtenir que
        \begin{equation}
            \widehat{\Delta\varphi}(x)=-\| x \|^2\hat\varphi(x),
        \end{equation}
        qui dans notre cas s'écrit sous la forme
        \begin{equation}
            \TF^{-1}\Big( (\Delta\psi_t) \Big)(x)=-\| x \|^2\TF^{-1}\psi(t,x).
        \end{equation}
        En remettant bout à bout,
        \begin{subequations}
            \begin{align}
                \heartsuit&=-\int_{\eR}(f_t\hat g)\Big( (\partial_t-i\| . \|^2)\TF^{-1}\psi_t \Big)dt\\
                &=-\int_{\eR}\hat g\Big( x\mapsto  e^{-it\| x \|^2}(\partial_t-i\| x \|^2)(\TF^{-1}\psi)(t,x) \Big)dt
            \end{align}
        \end{subequations}
        Pour alléger les notations nous notons \( \check{\psi_t}(x)=(\TF^{-1}\psi)(t,x)\). Nous avons
        \begin{equation}
            \partial_t\left(  e^{-it\| x \|^2}\check\psi_t(x) \right)=-i\| x \|^2 e^{-it\| x \|^2}\check{\psi_t}(x)+ e^{-it\| x \|^2}(\partial_t\check{\psi_t}),x)= e^{-it\| x \|^2}\big( \partial_t-i\| x \|^2 \big)\check{\psi_t}(x);
        \end{equation}
        cela nous permet d'un peu factoriser une dérivée dans \( \heartsuit\) :
        \begin{subequations}
            \begin{align}
                \heartsuit&=-\int_{\eR}\hat g\left( \partial_t\Big(  e^{-it\| . \|^2}\check{\psi_t}(.) \Big) \right)dt\\
                &=-\int_{\eR}\partial_t\hat g\left(  e^{-it\| . \|^2}\check{\psi_t}(.) \right)dt\\
                &=-\lim_{N\to \infty} \left[ \hat g\Big(  e^{-i\| . \|^2}\check{\psi_t}(.) \Big) \right]_{t=-N}^{t=N}.
            \end{align}
        \end{subequations}
        Histoire de bien comprendre les notations, il ne s'agit pas de calculer \( \hat g\big(  e^{-it\| . \|^2}\check\psi_t \big)\) pour un \( t\) général et de remplacer ensuite \( t\) par \( N\) et \( -N\). En effet la valeur de \( \hat g\big(  e^{-it\| . \|^2}\check\psi_t \big)\) pour un \( t\) donné est celle qu'on obtient en calculant \( \hat g(\ldots)\) après avoir remplacé \( t\) par ce que l'on veut. Par conséquent, en posant \( \varphi(t,\xi)= e^{-i\| \xi \|^2}\check\psi_t(\xi)\) nous avons :
        \begin{subequations}
            \begin{align}
                \heartsuit&=\lim_{N\to \infty} \left[ g\left( x\mapsto\int_{\eR} e^{-ix\xi}\varphi(t,\xi )d\xi \right) \right]_{t=-N}^{t=N}\\
                &=\lim_{N\to \infty} g\left( x\mapsto\int_{\eR} e^{-ix\xi}\varphi(N,\xi )d\xi \right)-\lim_{N\to \infty} g\left( x\mapsto\int_{\eR} e^{-ix\xi}\varphi(-N,\xi )d\xi \right)
            \end{align}
        \end{subequations}
        La limite commute avec \( g\) parce que cette dernière est une distribution (continue). De plus la limite commute avec l'intégrale parce que ce qui est dedans est Schwartz. La fonction \( \varphi\) étant Schwartz, la limite est nulle. Donc
        \begin{equation}
            \heartsuit=0.
        \end{equation}
        Cela signifie que la fonction \( u\) proposée est bien une solution de l'équation de Schrödinger dans \(  C^{\infty}(\eR,\swS'(\eR^d))\).

    \item[Unicité]

        Nous considérons deux solutions \( u_1,u_2\in C^{\infty}\big( \eR,\swS'(\eR^d) \big)\) et la fonction \( u=u_1-u_2\) doit satisfaire au problème
        \begin{subequations}
            \begin{numcases}{}
                (\partial_t\tilde u-i\Delta\tilde u)(\psi)=0\\
                u_0=0.
            \end{numcases}
        \end{subequations}
        Nous allons montrer que seule la fonction \( u_t=0\) peut satisfaire à cela pour tout \( \psi\in\swS(\eR\times \eR^d)\). Nous allons même montrer qu'en imposant ces équations seulement sur la partie de \( \swS(\eR\times\eR^d)\) qui est à support compact par rapport à \( \eR\), la seule solution est \( u_t=0\). Soit donc \( \psi\in\swS(\eR\times \eR^d)\) à support compact vis-à-vis de sa variable \( t\). Alors
        \begin{equation}
            0=-\tilde u(\partial_t\psi+i\Delta\psi)=-\int_{\eR}u_t\Big( (\partial_t\psi_t)+i(\Delta\psi_t) \Big)dt
        \end{equation}
        où encore une fois \( \partial_t\psi_t\) est la fonction \( x\mapsto (\partial_t\psi)(t,x)\). Maintenant nous utilisons la proposition~\ref{PropUDkgksG} pour dire que
        \begin{equation}
            \frac{ d }{ dt }\Big( u_t(\psi_t) \Big)=u^{(1)}_t(\psi_t)+u_t\left( \frac{ \partial \psi }{ \partial t }(t,.) \right)
        \end{equation}
        pour écrire
        \begin{equation}
            0=-\int_{\eR}\frac{ d }{ dt }\big( u_t(\psi_t) \big)-u_{t}^{(1)}(\psi_t)+u_t\big( i(\Delta\psi)(t,.) \big)dt
        \end{equation}
        Le premier terme est facile :
        \begin{equation}
            \int_{\eR}\frac{ d }{ dt }\Big( u_t(\psi_t) \Big)dt=\lim_{N\to \infty} \Big[ u_t(\psi_t) \Big]_{t=-N}^{t=N}=0
        \end{equation}
        parce que \( \psi\) est à support compact par rapport à \( t\). Nous restons donc avec
        \begin{equation}
            \int_{\eR}u_{t}^{(1)}(\psi_t)-iu_t\big( (\Delta\psi)(t,.) \big)dt=0
        \end{equation}
        Nous traitons le terme en \( u_t^{(1)}\) en utilisant le fait évident \( T(\varphi)=(\TF T)(\TF^{-1}\varphi)\) et en remarquant le lemme~\ref{LemWRoRPIX} :
        \begin{equation}
            u_t^{(1)}(\psi_t)=(\TF u_t^{(1)})(\TF^{-1}\psi_t)=(\TF u)_t^{(1)}(\TF^{-1}\psi_t).
        \end{equation}
        Pour l'autre terme on fait un peu la même chose en nous souvenant ce que fait la transformée de Fourier en traversant le laplacien :
        \begin{equation}
            u_t(\Delta\psi_t)=(\TF u_t)(\TF^{-1}\Delta\psi_t)=(\TF u_t)\big( x\mapsto -\| x \|^2(\TF^{-1}\psi_t)(x) \big).
        \end{equation}
        En recollant encore :
        \begin{equation}    \label{EqHOGaGpt}
            \int_{\eR}(\TF u)^{(1)}_t(\TF^{-1}\psi_t)+i(\TF u_t)\big( \| . \|^2\TF^{-1}\psi_t \big)dt=0.
        \end{equation}
        Cette équation est valable tant que \( \psi\in \swS(\eR\times\eR^d)\) avec support compact en \( t\). Nous allons nous en créer une super cool. D'abord nous choisissons \( \varphi\in\swS(\eR^d)\) et \( \chi\in\swD(\eR)\) et nous considérons\footnote{Le candidat qui parvient à effectivement présenter ça comme développement, il est fort.}
        \begin{equation}    \label{EqEVtJcnz}
            \psi(t,x)=\TF\Big( \xi\mapsto  e^{it\| \xi \|^2}\varphi(\xi)\chi(t) \Big)(x).
        \end{equation}
        Notons que la transformée de Fourier conserve le fait qu'une fonction soit Schwartz\footnote{Proposition~\ref{PropKPsjyzT}.}, mais pas le fait d'avoir support compact. Cependant nous ne prenons que la transformée de Fourier par rapport à \( x\). Le résultat est donc une fonction \( \psi\) qui est Schwartz par rapport à \( x\) et support compact par rapport à \( t\). Nous pouvons donc écrire \eqref{EqHOGaGpt} en utilisant la fonction \eqref{EqEVtJcnz} :
        \begin{equation}    \label{EqHPUyZFz}
            0=\int_{\eR}(\TF u)_t^{(1)}\Big( x\mapsto e^{it\| x \|^2}\varphi(x)\chi(t) \Big)+i(\TF u_t)\Big( x\mapsto\| x \|^2 e^{it\| x \|^2}\varphi(x)\chi(t) \Big)dt.
        \end{equation}
        Là dedans, \( \chi(t)\) peut sortir à la fois de la transformée de Fourier et de l'application des distributions; il doit seulement rester dans l'intégrale. Dans le second terme nous allons utiliser l'égalité (due entre autre à la proposition~\ref{PropUDkgksG}) :
        \begin{subequations}    \label{EqCRGfbLU}
            \begin{align}
            \frac{ d }{ dt }\big( \hat u_t( e^{it\| . \|^2}\varphi) \big)&=\frac{ d }{ dt }\left( u_t\big( \TF e^{it\| . \|^2}\varphi \big) \right)\\
            &=u_t^{(1)}\big( \TF  e^{it\| . \|^2}\varphi \big)+u_t\left( \frac{ \partial  }{ \partial t }\TF e^{it\| . \|^2}\varphi \right)\\
            &=(\TF u_t^{(1)})\big( x\mapsto  e^{it\| x \|^2}\varphi(x) \big)+(\TF u_t)\big( x\mapsto i\| x \|^2 e^{it\| x \|^2}\varphi(x) \big)\\
            &=(\TF u)_t^{(1)}\big( x\mapsto  e^{it\| x \|^2}\varphi(x) \big)+(\TF u_t)\big( x\mapsto i\| x \|^2 e^{it\| x \|^2}\varphi(x) \big).
            \end{align}
        \end{subequations}
        Et là, magie c'est exactement ce qui est dans \eqref{EqHPUyZFz}. Donc
        \begin{equation}
            \int_{\eR}\frac{ d }{ dt }\hat u_t\big( x\mapsto  e^{it\| x \|^2}\varphi(x) \big)\chi(t)dt=0
        \end{equation}
        pour toute fonctions à support compact \( \chi\). Donc la proposition~\ref{PropAAjSURG} nous dit que
        % 13107277
        \begin{equation}
            \partial_t\hat u_t\big( x\mapsto e^{it\| x \|^2}\varphi(x) \big)=0.
        \end{equation}
        C'est zéro partout et non seulement presque partout parce qu'en plus nous avons la continuité. Par conséquent pour tout \( t\in \eR\) nous avons
        \begin{equation}
            \hat u_t\big( x\mapsto e^{it\| x \|^2}\varphi(x) \big)=\hat u_0\big( x\mapsto \varphi(x)\big)=0.
        \end{equation}
        Et cela est vrai pour toute fonction \( \varphi\in\swS(\eR^d)\). Nous considérons donc \( t_0\in \eR\) et une fonction \( \theta\in\swS(\eR^d)\) pour construire
        \begin{equation}
            \varphi(x)= e^{-it_0\| x \|^2}\theta(x).
        \end{equation}
        Nous avons alors \( \hat u_{t_0}\big( x\mapsto\theta(x) \big)=0\), ce qui signifie que \( \hat u_{t_0}=0\). Du coup pour tout \( \theta\in\swS(\eR^d)\) nous avons \( u_{t_0}(\TF\theta)=0\), mais comme la transformée de Fourier est une bijection de \( \swS(\eR^d)\) (proposition~\ref{PROPooLWTJooReGlaN}) nous avons en fait \( u_{t_0}(\theta)=0\) pour tout \( \theta\in\swS(\eR^d)\), c'est-à-dire \( u_{t_0}=0\) pour tout \( t_0\in \eR\) et au final \( u=0\).
    \end{subproof}
\end{proof}

% This is part of Analyse Starter CTU
% Copyright (c) 2014,2017, 2019
%   Laurent Claessens,Carlotta Donadello
% See the file fdl-1.3.txt for copying conditions.

%+++++++++++++++++++++++++++++++++++++++++++++++++++++++++++++++++++++++++++++++++++++++++++++++++++++++++++++++++++++++++++
\section{Équations différentielles du premier ordre}
%+++++++++++++++++++++++++++++++++++++++++++++++++++++++++++++++++++++++++++++++++++++++++++++++++++++++++++++++++++++++++++

\begin{definition}[Équation différentielle du premier ordre]
Une  \defe{équation différentielle du premier ordre}{équation différentielle!premier ordre} est une équation qui, sur un intervalle donné, \(I\), décrit la relation entre une variable réelle, notée \(x\) ou \(t\) dans \(I\), une fonction \(y \,:\,I\to\eR \), et la dérivée première de \(y\) qui on note \(y'\).
\end{definition}
Souvent on écrit <<\(y'(x) = \text{une formule contenant }x \text{ et }y(x)\)>>, c'est \`a dire
\begin{equation}\label{ed_generale}
  y'(x) = f(x,y(x)),\quad\text{pour }x\in I,
\end{equation}
où \(f\) est une fonction de deux variables réelles.
\begin{remark}
  La théorie des fonctions de deux variables ne sera pas abordée dans ce cours, nous allons nous contenter de prendre \(f\) dans \eqref{ed_generale} comme une simple notation.
\end{remark}
On peut presque toujours omettre d'écrire la dépendance de \(y\) en \(x\) et écrire simplement \eqref{ed_generale} sous la forme \(y' = f(x,y)\).
\begin{definition}[Solution particulière d'une équation différentielle du premier ordre]
  Une \defe{solution particulière}{solution!particulière} de l'équation \eqref{ed_generale} sur l'intervalle \(I\) est une fonction \(z\,:\, I\to\eR\) telle que :
  \begin{enumerate}
  \item \(z\) est dérivable sur \(I\) ;
  \item \(z'(x) = f(x, z(x)), \) pour tout \(x\in I\).
  \end{enumerate}
\end{definition}
\begin{definition}[Solution générale d'une équation différentielle du premier ordre]
  Résoudre une équation différentielle veut dire trouver l'ensemble qui contient toutes ses solutions particulières. Cet ensemble s'appelle \defe{solution générale}{solution!générale} de l'équation.
\end{definition}
\begin{example}
    \begin{enumerate}
        \item
        Résoudre une équation du type \( y'(x)=f(x)\)<++> revient à trouver l'ensemble des primitives de la fonction \(f\), qui est donc la solution générale de cette équation. Il y a donc une infinité de solutions particulières, déterminées par une constante additive.

        Si \(f (x) = \sin(x)\) alors la solution générale sera \(\mathcal{Y} = \{-\cos(x) + C\, : \, C\in\eR\}\).
        \item
        L'équation
        \begin{equation}\label{equation_exponentielle}
          y'= y, \qquad x \in\eR,
        \end{equation}
        a peut-être été abordée dans votre cours de terminale lors de la définition de la fonction exponentielle. Sa solution générale est \(\mathcal{Y} = \{Ce^x\, : \, C\in\eR\}\). Ici aussi il y a une infinité de solutions particulières.
    \end{enumerate}
\end{example}
\begin{remark}
  La solution générale d'une équation différentielle du premier ordre est une famille à un paramètre de fonctions.
\end{remark}
\begin{definition}[Équation différentielle du second ordre]
Une  \defe{équation différentielle du second ordre}{équation différentielle!second ordre} est une équation qui, sur un intervalle donne, \(I\), décrit la relation entre une variable réelle, notée \(x\) ou \(t\) dans \(I\), une fonction \(y \,:\,I\to\eR \), et les dérivées première et seconde de \(y\) qui on note \(y'\) et \(y''\) respectivement.

On utilise la forme générale
\begin{equation}\label{ed_generale_second_ordre}
  y'' = f(x,y, y'),\quad\text{pour }x\in I.
\end{equation}
o\`u \(f\) est une fonction de trois variables réelles.
\end{definition}
On peut définir de manière analogue les équations différentielles d'ordre supérieur. Les définitions de solution particulière et de solution générale se généralisent aux équations différentielles d'ordre supérieur à un.
\vspace{0.5cm}
\begin{definition}[Trajectoire]
  La trajectoire tracée par une solution particulière $y$ de l'équation \eqref{ed_generale} est le graphe de $y$ en tant que fonction de $x$.
\end{definition}
\begin{example}
  Nous allons regarder de plus près l'équation \eqref{equation_exponentielle}, $y'=y$, pour tout $x\in\eR$. Soient $y_1$ et $y_2$ deux solutions distinctes de cette équation. S'il existe un point $\bar x$ tel que $y_1(\bar x) = y_2 (\bar x)$ alors forcement $y_1(\bar x)/y_2 (\bar x)=1$. Or, la solution générale de l'équation est \(\mathcal{Y} = \{Ce^x\, : \, C\in\eR\}\), donc $y_i(x) = C_ie^x$, $i= 1,2$, o\`u les $C_i$ sont des constantes. Le rapport $y_1(\bar x)/y_2 (\bar x)$ vaut $C_1/C_2$ et par conséquent $C_1 = C_2$. Ce résultat contredit l'hypothèse que les deux solutions soient distinctes. On a donc montré que \emph{deux trajectoires distinctes de cette équation ne se croisent jamais}.

\newcommand{\CaptionFigSBTooEasQsT}{Quelques trajectoires de l'équation \( y'=y\).}
\input{auto/pictures_tex/Fig_SBTooEasQsT.pstricks}

La figure~\ref{LabelFigSBTooEasQsT} représente quelques trajectoires de l'équation. Si on les avait tracées toutes elles recouvriraient tout le plan $x$-$y$. Cela veut dire que \emph{par tout point $(x,y)$ passe une et une seule trajectoire de l'équation \eqref{equation_exponentielle}}.

\end{example}
\begin{definition}[Condition initiale]
  Une \defe{condition initiale}{condition initiale} pour l'équation \eqref{ed_generale} sur l'intervalle \(I\) est un point \((\bar x, \bar y)\in I\times\eR\).

On dit que la solution particulière \(z\) de \eqref{ed_generale} satisfait la condition initiale \((\bar x, \bar y)\in I\times\eR\) si \(z(\bar x) =\bar y\).
\end{definition}
\begin{definition}[Problème de Cauchy]
  L'association d'une équation différentielle et d'une condition initiale est appelée \defe{problème de Cauchy}{problème de Cauchy}
  \begin{equation}\label{plme_cauchy}
    \begin{cases}
      y'= f(x,y), \quad x\in I, \\
      y(\bar x) = \bar y.
    \end{cases}
  \end{equation}
\end{definition}
\begin{remark}
  Sous des conditions assez générales qui seront toujours vérifiées dans ce cours, tout problème de Cauchy admet une et une seule solution.
\end{remark}
Pour passer de la solution générale d'une équation différentielle de premier ordre \`a une solution particulière il faut choisir une valeur du paramètre. Comme il y a un seul paramètre une seule condition (la trajectoire de la solution doit passer par un point fixe du plan) peut suffire. Pour une équation différentielle de second ordre comme \eqref{ed_generale_second_ordre}, nous aurons besoin de plus de conditions. Sans rentrer dans les détails, nous allons constater ce fait dans l'exemple suivant.

\begin{example}
  La solution générale de l'équation
    \begin{equation}\label{eq_expcompl}
      y'' = -y,
    \end{equation}
    est \(\mathcal{Y}= \{C_1\cos(x) + C_2\sin(x) \, :\, C_1,\, C_2\in\eR\}\). Remarquez que l'équation est du second ordre et que sa solution générale est une famille d'équations \`a deux paramètres réels. Ce sera toujours les cas pour les équations abordées dans la section~\ref{Secordredeux}.
    Pour déterminer une solution particulière de \eqref{eq_expcompl} il faut fixer les valeurs des deux paramètres et donc, en général, il sera nécessaire de donner deux conditions.
\end{example}

    \begin{remark}
      Une condition comme \(y(0)=4\) nous dit que la constante $C_1 = 4$ mais elle ne nous permet pas de trouver $C_2$. Il y a donc une infinité de solutions de \eqref{eq_expcompl} qui satisfont \`a la condition \(y(0)=4\).
    \end{remark}

On peut fixer les deux conditions de deux manières différentes.
\begin{enumerate}
\item{Problème  de Cauchy :} on fixe une terne de valeurs réels \(\bar x, \bar y, \bar y'\) et on cherche la solution telle que \(y(\bar x) = \bar y\), \(y'(\bar x) = \bar y'\).

  \begin{example}
    Les conditions \(y(0)=4\), \(y'(0)=15\) permettent de trouver la solution \(z(x) = 4\cos(x) + 15\sin(x)\).
  \end{example}
\item{Problème aux bords :} on fixe deux points dans le plan $x$-$y$, \(A=(\bar x, \bar y\)) et \(B=(\tilde x, \tilde y)\), et on cherche la solution dont la trajectoire passe par $A$ et $B$, c'est \`a dire, on impose \(y(\bar x) = \bar y\), \(y(\tilde x) = \tilde y\).

\begin{example}
    Les conditions \(y(0)=4\), \(y(\pi/2)=15\) permettent de trouver la solution \(z(x) = 4\cos(x) + 15\sin(x)\).
  \end{example}
\end{enumerate}

%+++++++++++++++++++++++++++++++++++++++++++++++++++++++++++++++++++++++++++++++++++++++++++++++++++++++++++++++++++++++++++
\section{Premier ordre, variables séparables}
%+++++++++++++++++++++++++++++++++++++++++++++++++++++++++++++++++++++++++++++++++++++++++++++++++++++++++++++++++++++++++++

Pour certaines équations différentielles la recherche d'une solution particulière se réduit à une recherche de primitive moyennant un changement de variables.
\begin{definition}[Équation différentielle du premier ordre \`a variables separables]
Une  \defe{équation différentielle du premier ordre à variables séparables}{équation différentielle!variables séparables} est une équation qui, pour tout les \(x\) dans un intervalle donné, \(I\), peut se mettre sous la forme
\begin{equation}\label{eq_var_sep}
  f(y)y' = g(x),
\end{equation}
o\`u \(f\) et \(g\) sont deux fonctions de \(\eR\) dans \(\eR\).
\end{definition}
Nous pouvons intégrer les deux côtés de l'égalité par rapport à \(x\) et obtenir
\[
  \int f(y(x))y'(x)\, dx = G(x)+C,
\]
o\`u $G$ est une primitive de $g$ et $C$ une constante réelle. Il est facile \`a ce point d'effectuer un changement de variable dans le membre de gauche de l'équation en posant (sans surprise) \(y= y(x)\) et donc \(y'(x)\,dx = dy\).
\[
  \int f(y(x))y'(x)\, dx =  \int f(y)\, dy  = F(y(x)) + C ,
\]
o\`u $F$ est une primitive de $f$ et $C$ une constante réelle. En somme nous avons
\[
  F(y(x)) = G(x) + C ,
\]
et, si $F$ admet une fonction réciproque, alors
\begin{equation}
  y(x) = F^{-1} (G(x)+C).
\end{equation}
\begin{remark}
  L'expression de $F^{-1} $ peut être difficile à calculer. Il sera alors préférable de garder $y$ dans la forme implicite.
\end{remark}

\begin{example}
  L'équation
  \begin{equation}\label{ex_un_var_sep}
    3y^2 y' = x, \qquad\text{pour tout } x\in\eR,
  \end{equation}
est une équation à variables séparables. Pour reprendre les notations du début du chapitre, ici \(f(y) = 3y^2\) et \(g(x) = x\). En intégrant de deux côtés on trouve
\[
y^3 = \frac{x^2}{2} + C .
\]
La fonction $F(y) = y^3$ est une bijection de $\eR$ dans $\eR$, donc nous pouvons écrire la solution générale de l'équation \eqref{ex_un_var_sep} dans la forme
\[
\mathcal{Y} =\left\{ \left(\frac{x^2}{2} + C \right)^{1/3} \:\text{tel que } C\in\eR\right\}.
\]
\end{example}

\begin{example}
  En intégrant de deux côtés l'équation à variables séparables
  \begin{equation}
    2y y' = x, \text{pour tout } x\in\eR,
  \end{equation}
  on trouve
  \[
  y^2 = \frac{x^2}{2} + C .
  \]
La fonction $F(y) = y^2$ est \emph{n'est pas inversible} sur tout $\eR$, et on sait que \(\sqrt{y^2} = |y|\).  Au moment de rendre $y$ explicite on doit choisir entre
\[
y = \left(\frac{x^2}{2} + C \right)^{1/2}\qquad \text{ou}\qquad y = -\left(\frac{x^2}{2} + C \right)^{1/2} .
\]
Ce choix se fait suivant la condition initiale, si elle est donnée. S'il n'y a pas de condition initiale nous pouvons écrire que la solution générale est l'ensemble
\[
\mathcal{Y} =\left\{ y \,:\, \eR \to \eR\:\text{tels que }  y^2 = \frac{x^2}{2} + C \:\text{et } C\in\eR\right\}.
\]
\end{example}

\begin{example}
  On considère le problème de Cauchy
  \begin{equation}
    \begin{cases}
      e^y y' = \frac{1}{x+3}, \quad x\in ]-\infty, -3[,\\
      y(-4) = 0.
    \end{cases}
  \end{equation}
En intégrant des deux côtés nous trouvons
\[
e^y = \ln(|x+3|) +C.
\]
Nous pouvons alors imposer la condition initiale et obtenir $e^{0} =\ln(|-4+3|) +C $, c'est \`a dire $C = 1- \ln(1) = 1$.
\begin{remark}
  L'énoncé du problème de Cauchy dit que $x$ peut varier dans \(]-\infty, -3[\), mais nous voyons maintenant que la solution n'est pas définie sur toute la demi-droite, parce que $e^y$ est toujours positif  et $\ln(|x+3|) +1$ est positif seulement pour $x < -(1/e + 3)\approx -3,3679$.
\end{remark}
Donc la solution du problème de Cauchy est \(y(x) = \ln(|x+3|) +1\) pour tout $x\in ]-\infty, -(1/e +3)[$.
\end{example}

\begin{example}\label{exemple_eq_hom}
  \textbf{Attention, cet exemple est le plus important de la section !}

On considère l'équation à variables séparables
\begin{equation}    \label{EqYNXooFzYeZS}
  y' = \sin(x) y , \qquad x\in\eR.
\end{equation}
Dans ce cas, pour pouvoir écrire l'équation dans la forme \eqref{eq_var_sep} il faut pouvoir multiplier les deux c\^otés par $1/y$. Il faut donc éliminer tout de suite le cas o\`u $y = 0$.

Si $y= 0$ alors $y' =0$ et on a une solution constante (on dit souvent : une solution stationnaire) de l'équation. Par ailleurs les trajectoires des solutions ne peuvent pas se croiser; donc si \( y_G\) est une solution non nulle de l'équation \eqref{EqYNXooFzYeZS} alors \( y_G(x)\neq 0\) pour tout \( x\)\footnote{Ça vaut la peine de prendre un peu de temps pour bien comprendre cela.}. Il n'y a donc aucun danger à diviser par \( y\) dans la recherche d'une solution non identiquement nulle.

Supposons maintenant que $y\neq 0$ et écrivons $y'/y = \sin(x)$. En intégrant des deux côtés on trouve
\[
  \ln(|y|) =- \cos(x) +C,
\]
d'où
\[
|y| = e^{- \cos(x) +C}= e^{C}e^{- \cos(x)}.
\]
Si on avait impose une condition initiale alors on pourrait déterminer une solution particulière de l'équation en choisissant une valeur de la constante $C$. Nous pouvons observer cependant que la fonction exponentielle est bijective de $\eR$ dans $\eR^{+,\star}$ et par conséquent il n'y a pas de perte de généralité en disant que la solution générale de l'équation est
\begin{equation*}
  \mathcal{Y} = \left\{ y \,:\, |y| = Ke^{- \cos(x)}, \:\text{pour } K\in\eR^{+,\star}\right\}\cup\{y\equiv 0\}.
\end{equation*}
Il n'empêche qu'il serait plus élégant d'écrire la solution générale de l'équation sous une forme plus explicite, sans valeur absolue. Nous pouvons le faire en nous nous rappelant que
\begin{equation*}
 |x| =  \begin{cases}
    x & \quad\text{si } x \geq 0 ,\\
    -x & \quad\text{si } x <0 ,
  \end{cases}
\end{equation*}
Il suffit alors d'autoriser $K$ dans $\eR^{\star}$ pour éliminer la valeur absolue.

Pour écrire la solution générale de façon encore plus compacte nous observons que si $K=0$ alors $y \equiv 0$, c'est \`a dire, on retrouve la solution constante nulle.

Finalement, la solution générale de cette équation sera toujours écrite sous la forme suivante
\begin{equation}
  \mathcal{Y} = \left\{ y = Ke^{- \cos(x)}, \:\text{pour } K\in\eR\right\}.
\end{equation}
\end{example}

%+++++++++++++++++++++++++++++++++++++++++++++++++++++++++++++++++++++++++++++++++++++++++++++++++++++++++++++++++++++++++++
\section{Équations différentielles linéaires du premier ordre}
%+++++++++++++++++++++++++++++++++++++++++++++++++++++++++++++++++++++++++++++++++++++++++++++++++++++++++++++++++++++++++++

\begin{definition}[Équation différentielle linéaire du premier ordre]
Soit $I\subset\eR$ un intervalle .

Une  \defe{équation différentielle linéaire du premier ordre}{équation différentielle!linéaire du premier ordre} est une équation différentielle de la forme
\begin{equation}\label{eq_lin_ordre_un}
  a(x)y' + b(x) y = c(x), \quad\text{pour } x\in I,
\end{equation}
o\`u $a$, $b$, $c$ sont des fonctions de $\eR$ dans $\eR$ et $a\neq 0$ pour tout $x\in I$ .

On dit que $a$, $b$, $c$ sont les coefficients de l'équation \eqref{eq_lin_ordre_un}.
\end{definition}
\begin{remark}\label{remarque_lineaire}
  Une fonction $f: \eR\to\eR$ est dite \emph{linéaire} si pour tout $x_1$, $x_2$ dans $\eR$ et pour tout couple de constantes $\lambda$ et $\mu$ on a
  \begin{equation}\label{eq_linearite}
    f(\lambda x_1 + \mu x_2) = \lambda f(x_1) +\mu f (x_2).
  \end{equation}
Ces équations différentielles sont dites linéaires parce que la partie de l'équation qui contient $y$ (le membre de gauche) satisfait la propriété \eqref{eq_linearite} par rapport à $y$. En effet par les propriétés de la dérivée nous avons que
\[
 a(x)(\lambda y_1 + \mu y_2 )' + b(x) (\lambda y_1 + \mu y_2 ) =\lambda ( a(x)y'_1 + b(x) y_1 ) + \mu( a(x)y'_2 + b(x) y_2 ).
\]
\end{remark}
\begin{definition}
  L'équation \eqref{eq_lin_ordre_un} est dite \defe{homogène}{équation différentielle!linéaire du premier ordre, homogène} quand $c$ est la fonction nulle.
Si \eqref{eq_lin_ordre_un} n'est pas homogène on dit que l'équation
\begin{equation}\label{eq_lin_ordre_un_hom}
  a(x)y' + b(x) y =0,
\end{equation}
est son \defe{équation homogène associée}{équation homogène associée}.
\end{definition}
Toute équation linéaire du premier ordre homogène est une équation du premier ordre à variables séparables, comme nous en avons vu l'exemple \ref{exemple_eq_hom}. Nous n'allons pas répéter les détails du procédé pour trouver sa solution générale, qui aura la forme suivante
\begin{Aretenir}
  \begin{equation}\label{solgeneqlinordre1}
    \mathcal{Y}_h=\left\{Ke^{-\int\frac{b(x)}{a(x)}\, dx}\,:\, K\in\eR\right\}.
  \end{equation}
\end{Aretenir}
\begin{proposition}
  \begin{enumerate}
  \item Soit $y_p$ une solution particulière de l'équation \eqref{eq_lin_ordre_un} et $y_h$ une solution particulière de l'équation homogène associé \eqref{eq_lin_ordre_un_hom}. Alors la fonction somme $z= y_p+y_h$ est encore une solution particulière de l'équation \eqref{eq_lin_ordre_un}.
  \item Soient $y_1$ et $y_2$ deux solutions particulières de \eqref{eq_lin_ordre_un}. Alors la fonction différence $w = y_1-y_2$ est une solution particulière de \eqref{eq_lin_ordre_un_hom}.
  \end{enumerate}
\end{proposition}
\begin{proof}
  \begin{enumerate}
  \item
    \begin{equation}
      a(x)\left(y_p+y_h\right)' + b(x)\left(y_p+y_h\right)-c(x)  =\left( a(x)y'_p+ b(x)y_p-c(x)\right) + \left( a(x)y'_h+ b(x)y_h\right) = 0.
    \end{equation}
  \item
    \begin{equation}
      a(x)\left(y_1-y_2\right)' + b(x)\left(y_1-y_2\right) =\left( a(x)y'_1+ b(x)y_1-c(x)\right) -\left( a(x)y'_2+ b(x)y_2-c(x)\right) = 0.
    \end{equation}
  \end{enumerate}
\end{proof}
Cette proposition permet de démontrer le théorème suivant, qui est le plus important de cette section.
\begin{theorem}
  Soit $y_p$ une solution particulière de l'équation \eqref{eq_lin_ordre_un} et $\mathcal{Y}_h$ la solution générale de l'équation \eqref{eq_lin_ordre_un_hom}, alors la solution générale de l'équation \eqref{eq_lin_ordre_un} est l'ensemble
  \begin{equation}
    \mathcal{Y} = \mathcal{Y}_h +y_p = \left\{z= y_h + y_p\,:\, y=h \in\mathcal{Y}_h \right\}.
  \end{equation}
\end{theorem}
\begin{Aretenir}
  La résolution d'une équation différentielle linéaire du premier ordre comporte trois étapes :
  \begin{enumerate}
  \item résolution de l'équation homogène associée ;
  \item recherche d'une solution particulière de l'équation non homogène ;
  \item somme de la solution générale de l'équation homogène et de la solution particulière trouvée au point précédent.
  \end{enumerate}
\end{Aretenir}
La partie qui nous manque encore est de savoir comment trouver une solution particulière de l'équation non homogène \eqref{eq_lin_ordre_un}. Si la fonction $c$ dans \eqref{eq_lin_ordre_un} est une constante ou un polynôme simple, ou une exponentielle alors on peut essayer de deviner. Cette méthode cependant n'est pas la plus sure pour des débutants.

\begin{example}
  On considère l'équation
  \begin{equation}
    y'-5y = 10, \qquad x\in\eR.
  \end{equation}
Comme tous les coefficients de l'équation sont constants on peut essayer de trouver une solution constante.

Toutes les fonctions constantes ont une dérivée nulle, par conséquent, si une solution constante existe elle doit satisfaire $-5y = 10$, ce qui veut dire que la solution constante est $y(x)\equiv -2$.
\end{example}

\begin{example}
  On considère l'équation
  \begin{equation}
    xy'+y = x+1, \qquad x\in\eR^{+,\star}.
  \end{equation}
Comme le membre de droite de l'équation est un polynôme de degré un on cherche une solution de la forme $y(x) = Ax + B$ avec $A$ et $B$ dans $\eR$.

Par substitution on obtient $Ax + (Ax +B) = x+1$, c'est-à-dire que une solution particulière de l'équation est $y(x) = x/2+1$.
\end{example}

\begin{example}
   L'équation
  \begin{equation}
    xy'-y = x+1, \qquad x\in\eR^{+,\star}.
  \end{equation}
ressemble beaucoup à celle de l'exemple précédent, cependant il n'existe pas un polynôme de degré un qui en soit solution.

Dans un cas comme celui-ci, il faut rapidement abandonner la divination et replier sur la méthode, plus technique mais plus sure, dite  \emph{variation de la constante}.
\end{example}


\subsection{Méthode de variation de la constante}

\begin{itemize}
\item Soit $\mathcal{Y}_h$ la solution générale de l'équation homogène associé à \eqref{eq_lin_ordre_un}. Il s'agit d'une famille à un paramètre de fonctions. La première étape de cette méthode consiste à construire un candidat solution particulière $y_p$ en remplaçant le paramètre dans  $\mathcal{Y}_h$ par une fonction $C: \eR \to\eR$ à déterminer.

  \begin{example}
    L'équation homogène associée à $y'-y = \cos(x)$ est $y' - y = 0$, dont la solution générale est $\mathcal{Y}_h = \{Ce^x \,:\, C\in\eR\}$. Le candidat solution sera alors $y_p = C(x)e^x$, avec $C$ fonction à déterminer.
  \end{example}

 \item  La deuxième étape de cette méthode consiste à injecter $y_p$ dans l'équation. Cela permet de trouver une équation différentielle  à variables séparables pour $C$, en principe plus facile à résoudre que l'équation de départ.

  \begin{example}
    On continue avec l'exemple précédent. On a $y_p' = C'(x) e^x + C(x) e^x$, d'où
    \[
    (C'(x) e^x + C(x) e^x) - C(x) e^x = \cos(x),
    \]
    c'est-à-dire
    \[
    C'(x)  = \cos(x)e^{-x}.
    \]
  \end{example}
\item  La troisième étape de la méthode consiste à trouver une solution particulière de l'équation différentielle pour $C$ et, par conséquent déterminer une $y_p$.

  \begin{example}
    La solution générale de
    \[
    C'(x)  = \cos(x)e^{-x}.
    \]
    est $\mathcal{C} = \left\{e^{-1}\frac{(\sin(x)-\cos(x))}{2} +K \,:\, K\in\eR\right\}$. Il nous suffit une solution particulière, nous pouvons donc choisir $K=0$ et alors la solution particulière de \eqref{eq_lin_ordre_un} sera $y_p (x)= \frac{\sin(x)-\cos(x)}{2} $.
  \end{example}
\end{itemize}
\begin{remark}
  Le plus souvent en intégrant l'équation pour $C$ on en trouvera la solution générale. Dans ce cas on peut remplacer $C$ par cette solution générale et obtenir d'un seul coup la solution générale de l'équation \eqref{eq_lin_ordre_un} , c'est-à-dire sans faire la somme entre la solution générale de l'homogène associée et la solution particulière.

  \begin{example}
    Dans l'exemple qu'on vient de voir la solution générale de \eqref{eq_lin_ordre_un} est
    \begin{equation}
      \mathcal{Y} = \mathcal{Y}_h + y_p = \left\{Ce^x + \frac{(\sin(x)-\cos(x))}{2} \,:\, C\in\eR\right\}.
    \end{equation}
On obtient le m\^eme résultat est écrivant $\mathcal{Y} = \left\{e^{-x}\left(e^{-1}\frac{(\sin(x)-\cos(x))}{2} +K \right) \,:\, K\in\eR\right\}$. Notez qu'on a changé le nom du paramètre de $C$ à $K$ seulement pour souligner qu'on obtient de m\^eme résultat par deux chemins différents, sinon les deux expressions sont équivalentes !
  \end{example}
\end{remark}

%+++++++++++++++++++++++++++++++++++++++++++++++++++++++++++++++++++++++++++++++++++++++++++++++++++++++++++++++++++++++++++
\section{Équations différentielles linéaires du second ordre}
%+++++++++++++++++++++++++++++++++++++++++++++++++++++++++++++++++++++++++++++++++++++++++++++++++++++++++++++++++++++++++++
\label{Secordredeux}

\begin{definition}[Équation différentielle linéaire du second ordre]
Une  \defe{équation différentielle linéaire du second ordre}{équation différentielle!linéaire du second ordre} est une équation différentielle de la forme
\begin{equation}\label{eq_lin_ordre_deux}
  a(x)y'' + b(x) y' + c(x)y = d(x), \quad\text{pour } x\in I,
\end{equation}
o\`u $a$, $b$, $c$ et $d$ sont des fonctions de $\eR$ dans $\eR$ et $a\neq 0$ pour tout $x\in I$ .

On dit que $a$, $b$, $c$ et $d$ sont les coefficients de l'équation \eqref{eq_lin_ordre_deux}.
\end{definition}
Dans ce cours nous allons étudier exclusivement le cas où $a$, $b$ et $c$ sont des fonctions constantes.
\begin{definition}[Équation différentielle linéaire du second ordre homogène]
Une  \defe{équation différentielle linéaire du second ordre homogène}{équation différentielle!linéaire du second ordre, homogène} est une équation différentielle de la forme \eqref{eq_lin_ordre_deux}, telle que le coefficient $d$ est nul.
\end{definition}
À toute équation de la forme \eqref{eq_lin_ordre_deux} on peut associer une équation homogène exactement comme on a fait dans la section précédente pour les équations linéaires du premier ordre.

%--------------------------------------------------------------------------------------------------------------------------- 
\subsection{Équations différentielles linéaires du second ordre homogènes à coefficients constants}
%---------------------------------------------------------------------------------------------------------------------------

\begin{remark}
 L'application qui à la fonction $y$ fait correspondre $a(x)y'' + b(x) y' + c(x)y$ est linéaire, au sens de la remarque~\ref{remarque_lineaire}.

Cela nous dit en particulier, que si $y_1$ et $y_2$ sont deux solutions de l'équation homogène alors toute leur combinaison de la forme $z = \lambda y_1 + \mu y_2$, avec $\lambda$ et $\mu$ dans $\eR$, est encore une solution.
 \end{remark}

\begin{framed}
  Jusqu'ici nous avons toujours travaillé avec des fonctions définies sur $\eR$ et à valeurs dans $\eR$. Dans cette section nous nous autorisons à passer par des fonctions définies sur $\eR$ et à valeurs dans $\eC$, mais cela sera uniquement une étape dans nos calculs. Au final toutes les solutions que nous allons considérer sont des fonctions à valeurs dans $\eR$.
\end{framed}

La solution générale \textbf{à valeurs dans les complexes} d'une équation de ce type a la forme
\begin{equation}\label{sol_gen_ordre_deux_hom}
  \mathcal{Y}_h^\eC  = \left\{C_1 e^{r_1x} +C_2 e^{r_2x} \,:\, C_1,\, C_2 \in \eC, \: x\in I \right\},
\end{equation}
où $r_1$ et $r_2$ sont aussi des nombres complexes. Remarquez que la solution générale est une famille à deux paramètres. Il faut aussi observer que en tout cas l'intervalle $I$ dans lequel varie $x$ est un intervalle dans $\eR$, parce que $I$ est une des données du problème.

À partir de cette information nous pouvons, pour toute équation donnée, chercher la solution générale \textbf{complexe} par substitution. Il suffit de remplacer $y$ dans l'équation par $e^{rx}$ et chercher les valeurs de $r$ qui nous conviennent.

Si notre équation de départ est
\begin{equation}\label{eq_lin_ordre_deux_hom}
  ay'' + by' + cy = 0, \quad\text{pour } x\in I,
\end{equation}
alors la substitution nous donne
\[
e^{rx}\left(ar^2+br+c\right)=0.
\]
Il est connu que la fonction exponentielle ne prend pas la valeur $0$, par conséquent ce qui s'annule est le polynôme de degré deux $ar^2+br+c$. Il est donc très facile de trouver les valeurs de $r$ qu'on pourra utiliser comme $r_1$ et $r_2$ dans la solution générale \textbf{complexe}.
\begin{description}
  \item[Si $b^2 - 4ac >0$ :] le polynôme admet deux solutions réelles et distinctes, $r_1$ et $r_2$ ;
  \item[Si $b^2 - 4ac <0$ :] le polynôme admet deux solutions complexes conjuguées, $r_1 = \alpha + i \beta$ et $r_2 = \alpha - i \beta$ ;
  \item[Si $b^2 - 4ac =0$ :] le polynôme admet une solution réelle double $r=r_1 = r_2$.
\end{description}
Il faut maintenant écrire la solution générale \textbf{réelle} de l'équation, qui est celle que nous intéresse vraiment. La façon de l'obtenir est différente dans les trois cas.
\begin{description}
  \item[Si $b^2 - 4ac >0$ :] la solution générale réelle a la m\^eme forme que la solution complexe, \eqref{sol_gen_ordre_deux_hom}, il suffit de prendre les paramètres $C_1$ et $C_2$ dans $\eR$ plut\^ot que dans $\eC$.
\begin{equation}\label{sol_gen_reelle_ordre_deux_hom}
  \mathcal{Y}_h  = \left\{C_1 e^{r_1x} +C_2 e^{r_2x} \,:\, C_1,\, C_2 \in \eR, \: x\in I\right\},
\end{equation}
  \item[Si $b^2 - 4ac <0$ :] le polynôme admet deux solutions complexes conjuguées, $r_1 = \alpha + i \beta$ et $r_2 = \alpha - i \beta$ ; Il faut alors utiliser les formules suivantes
    \begin{equation}
      \begin{array}{l}
        e^{\alpha + i \beta} =e^{\alpha}(\cos(\beta) + i \sin(\beta))\\
        e^{\alpha - i \beta} =e^{\alpha}(\cos(\beta) - i \sin(\beta)).
      \end{array}
    \end{equation}
    La somme $e^{r_1x} +e^{r_2x}$, où $x$ est dans $I\in\eR$, vaut
    \[
    e^{(\alpha + i \beta)x} + e^{(\alpha - i \beta)x}=e^{\alpha x}(\cos(\beta x) + i \sin(\beta x )) + e^{\alpha x}(\cos(\beta x) - i \sin(\beta x)) =2 e^{\alpha x}\cos(\beta x)
    \]
    et la différence $e^{r_1x} -e^{r_2x}$ vaut
    \[
    e^{(\alpha + i \beta)x} - e^{(\alpha - i \beta)x}=e^{\alpha x}(\cos(\beta x) + i \sin(\beta x )) - e^{\alpha x}(\cos(\beta x) - i \sin(\beta x)) =2 e^{\alpha x}\sin(\beta x).
    \]
    Par ces deux calculs élémentaires nous avons trouvé deux fonctions à valeurs dans $\eR$ qui n'ont pas de zéros en commun. Elles sont les génératrices de la famille des solutions réelles de l'équation différentielle (la solution générale)
    \begin{equation}\label{sol_gen_reelle_ordre_deux_hom_complconj}
      \mathcal{Y}_h  = \left\{ e^{\alpha x}\left(C_1\cos(\beta x) +C_2\sin(\beta x)\right)  \,:\, C_1,\, C_2 \in \eR, \: x\in I\right\},
    \end{equation}
  \item[Si $b^2 - 4ac =0$ :] le polynôme admet une solution réelle double $r=r_1 = r_2$. Dans ce cas la solution générale de l'équation est la famille
    \begin{equation}\label{sol_gen_reelle_ordre_deux_hom_doublerac}
      \mathcal{Y}_h  = \left\{(C_1  +C_2x) e^{r x} \,:\, C_1,\, C_2 \in \eR, \: x\in I\right\}.
    \end{equation}
    Pour justifier cette formule nous observons d'abord que toute fonction $x\mapsto Ce^{rx}$, pour $C\in\eR$ est une solution de l'équation différentielle (par construction). Ensuite nous utilisons la méthode de variation de la constante. On trouve rapidement que si une fonction de la forme $x\mapsto C(x)e^{rx}$ est une solution alors $C(x)$ est un polynôme de degré au plus 1, c'est-à-dire $C(x) = C_1 + C_2 x$ avec $C_1$ et $C_2$ dans $\eR$.
\end{description}

\subsection{Linéaires du second ordre à coefficients constants, non homogènes}

Nous ne présentons pas une méthode générale pour la résolution de ces équations. Comme dans le cas des équations différentielles linéaires du premier ordre non homogènes, la solution générale de \eqref{eq_lin_ordre_deux} est donnée par la somme d'une solution particulière et de la solution générale de l'équation homogène associée. La recherche d'une solution particulière est facilitée par le fait que les coefficients de \eqref{eq_lin_ordre_deux} sont supposés constants, c'est-à-dire que $a$, $b$ et $c$ sont des fonctions constantes. Il faut essayer de deviner la forme d'une solution particulière à partir de la forme du second membre de l'équation, la fonction $d$. Si $d$ est un polynôme il faut essayer avec un polynôme du même degré, si $d$ est une exponentielle, par exemple $d(x) = e^{5x}$, on pourra essayer avec un multiple de la m\^eme fonction exponentielle, dans l'exemple $f(x) = k e^{5x}$, avec $k$ à determiner. Si $d$ est une combinaison linéaire de sinus et cosinus, comme par exemple $12\cos(x) + 2\sin(x)$, on peut essayer avec $k_1\cos(x) + k_2\sin(x)$.

\begin{example}
  On considère l'équation différentielle
  \begin{equation}\label{exemple_non_hom}
    y'' + 12y' + 36 y = -192 e^{2x}, \quad x\in\eR.
  \end{equation}
  Son équation homogène associée est
\begin{equation}\label{exemple_hom_ass}
    y'' + 12y' + 36 y = 0,
  \end{equation}
dont le polynôme caractéristique est $r^2 + 12 r + 36$. Ce polynôme admet une racine double, qui est $-6$, par conséquent la solution générale de \eqref{exemple_hom_ass} est
\begin{equation*}
      \mathcal{Y}_h  = \left\{(C_1  +C_2x) e^{-6 x} \,:\, C_1,\, C_2 \in \eR, \: x\in \eR\right\}.
    \end{equation*}
Le membre de droite de \eqref{exemple_non_hom} est une fonction exponentielle, nous allons donc chercher une solution particulière de \eqref{exemple_non_hom} de la forme $f(x) = ke^{2x}$. Par substitution nous trouvons
\[
  ke^{2x}(4 + 12 \times 2 +36) = -192 e^{2x},
\]
ce qui veut dire que $k$ doit \^etre $-3$.

La solution générale de l'équation \eqref{exemple_non_hom} est donc
\begin{equation*}
      \mathcal{Y}  = \left\{(C_1  +C_2x) e^{-6 x} -3e^{2x} \,:\, C_1,\, C_2 \in \eR, \: x\in \eR\right\}.
    \end{equation*}
\end{example}

\begin{example}
  Nous allons résoudre l'équation
  \begin{equation}
    y'' + 12y' + 36 y = 12\cos(x) + 2\sin(x), \quad x\in\eR.
  \end{equation}

Cette équation a comme homogène associée l'équation \eqref{exemple_hom_ass}, comme dans l'exemple précédent. Il nous suffit donc de trouver une solution particulière de \eqref{exemple_non_hom}.

Nous pouvons essayer avec $f(x)= k_1\cos(x) + k_2\sin(x)$. Par substitution on trouve
\begin{equation*}
  \begin{aligned}
    -\left(k_1\cos(x) + k_2\sin(x)\right) & +12 \left(-k_1\sin(x) + k_2\cos(x)\right) + 36\left(k_1\cos(x) + k_2\sin(x)\right)\\
    &= 12\cos(x) + 2\sin(x)
  \end{aligned}
\end{equation*}

Cette équation doit \^etre satisfaite pour toute valeur de $x$, en particulier pour $x= 0$ et $x = \pi/2$. Cela revient à considère séparément les coefficients des fonctions sinus et cosinus. Il faut alors que $k_1$ et $k_2$ soient solutions du système
\begin{equation*}
  \begin{cases}
    -k_1 + 12 k_2 + 36 k_1& = 12, \\
    -k_2 - 12 k_1 + 36 k_2& = 2.
  \end{cases}
\end{equation*}
On trouve $k_1= 396/1369$ et $k_2 = 214/1369$, et la solution générale de notre équation est
\begin{equation*}
   \mathcal{Y}  = \left\{(C_1  +C_2x) e^{-6 x} +\frac{396}{1369}\cos(x) + \frac{214}{1369}\sin(x) \,:\, C_1,\, C_2 \in \eR, \: x\in \eR\right\}.
\end{equation*}
\end{example}

\begin{example}
   Nous allons résoudre l'équation
  \begin{equation}
    y'' + 12y' + 36 y = 10x^2+3, \quad x\in\eR.
  \end{equation}

Cette équation a comme homogène associée l'équation \eqref{exemple_hom_ass}, comme dans l'exemple précédent. Il nous suffit donc de trouver une solution particulière de \eqref{exemple_non_hom}.

Nous pouvons essayer avec $f(x)= k_1x^2+ k_2x + k_3$. Par substitution on trouve
\begin{equation*}
    \left(2k_1\right)  +12 \left(2k_1x+ k_2\right) + 36\left(k_1x^2+ k_2x + k_3\right)=  10x^2+3.
\end{equation*}

Pour trouver les bonnes valeurs des coefficients nous devons résoudre le système \begin{equation*}
  \begin{cases}
    36 k_1& = 10, \\
    24k_1 + 36 k_2& = 0,\\
    2k_1 + 12 k_2 + 36 k_3& = 3,
  \end{cases}
\end{equation*}
ce qui donne $k_1= 5/18$, $k_2 = -5/27$ et $k_3 = 7/54$. La solution générale de notre équation est
\begin{equation*}
   \mathcal{Y}  = \left\{(C_1  +C_2x) e^{-6 x} +\frac{5}{18}x^2 - \frac{5}{27}x + \frac{7}{54} \,:\, C_1,\, C_2 \in \eR, \: x\in \eR\right\}.
\end{equation*}
\end{example}

%+++++++++++++++++++++++++++++++++++++++++++++++++++++++++++++++++++++++++++++++++++++++++++++++++++++++++++++++++++++++++++
\section{Fonction de Green}
%+++++++++++++++++++++++++++++++++++++++++++++++++++++++++++++++++++++++++++++++++++++++++++++++++++++++++++++++++++++++++++

Soit l'équation différentielle
\begin{subequations}
    \begin{numcases}{}
        y''(x)=g(x)\\
        y(0)=y(1)=0
    \end{numcases}
\end{subequations}
pour \( x\in\mathopen] 0 , 1 \mathclose[\) et où \( g\) est continue sur \( \mathopen] 0 , 1 \mathclose[\).

Nous définissons la fonction de Green
\begin{equation}
    G(x,t)=\begin{cases}
        t(x-1)    &   \text{si }  0\leq t\leq x\leq 1  \\
        x(t-1)    &    \text{si }0\leq x\leq t\leq 1,
    \end{cases}
\end{equation}
et nous allons montrer que
\begin{equation}        \label{EQooCOFDooERUIhe}
    y(x)=\int_0^1G(x,t)g(t)dt
\end{equation}
est l'unique solution.

\begin{subproof}
    \item[Unicité]

        Si \( y_1\) et \( y_2\) sont des solutions, alors \( y_1''=y_2''\) et donc \( y_1(x)=y_2(x)+ax+b\). Les conditions aux bords donnent alors \( 0=y_1(0)=y_2(0)+b=b\). D'où \( b=0\). En imposant \( y_1(1)=0\) nous trouvons alors immédiatement \( a=0\), ce qui donne \( y_1=y_2\).

    \item[Existence]

    Il est vite vérifié qu'avec \eqref{EQooCOFDooERUIhe} nous avons \( y(0)=y(1)=0\) parce que \( G(0,t)=G(1,t)=0\) pour tout \( t\). Nous fixons une valeur pour \( x\in \mathopen] 0 , 1 \mathclose[\) et nous découpons l'intégrale :
        \begin{equation}
            y(x)=\int_0^xG(x,t)g(t)dt+\int_x^1G(x,t)g(t)dt.
        \end{equation}
        Pour calculer \( y'(x)\), il faut dériver à la fois à travers l'intégrale et dans la borne. Si vous connaissez une formule pour faire cela, c'est bien pour vous. Nous allons faire ça à la main et poser
        \begin{equation}
            I(x,y)=\int_0^yt(x-1)g(t)dt.
        \end{equation}
        La dérivation de \( I\) par rapport à \( x\) se fait en utilisant le théorème~\ref{PropDerrSSIntegraleDSD} :
        \begin{equation}
            \frac{ \partial I }{ \partial x }(x,y)=\int_0^ytg(t)dt.
        \end{equation}
        Pour la dérivation par rapport à \( y\), il s'agit du théorème fondamental de l'analyse, plus précisément le lien primitive et intégrale de la proposition~\ref{PropEZFRsMj} :
        \begin{equation}
            \frac{ \partial I }{ \partial y }(x,y)=y(x-1)g(y).
        \end{equation}
        Maintenant nous considérons la fonction \( \varphi_I(x)=I(x,x)\). Elle satisfait à
        \begin{equation}
            \varphi_I'(x)=\frac{ \partial I }{ \partial x }(x,x)+\frac{ \partial I }{ \partial y }(x,x)=\int_0^xtg(t)+x(x-1)g(x).
        \end{equation}
        Le même jeu avec \( J(x,y)=\int_y^1x(t-1)g(t)dt\) donne
        \begin{equation}
                \varphi_J'(x)=\int_0^xfg(t)dt+x(x-1)g(x).
        \end{equation}
        En remettant les bouts ensemble,
        \begin{equation}
            y(x)=\int_0^xtg(t)dt+\int_1^x(1-t)g(t)dt.
        \end{equation}
        Le calcul de la dérivée seconde donne alors
        \begin{equation}
            y''(x)=xg(x)+(1-x)g(x)=g(x).
        \end{equation}
\end{subproof}

Nous pouvons aussi, sur cette équation, estimer la variation de la solution en termes d'une variation de \( g\). Soit donc une fonction continue \( \delta_g\) sur \( \mathopen[ 0 , 1 \mathclose]\) et \( \tilde g=g+\delta_g\). Nous considérons l'équation différentielle
\begin{subequations}
    \begin{numcases}{}
        \tilde y''(x)=\tilde g(x)\\
        \tilde y(0)=\tilde y(1)=0.
    \end{numcases}
\end{subequations}
Par ce que nous venons de faire, l'unique solution est
\begin{equation}
    \tilde y(x)=\int_0^1G(x,t)\tilde g(t)dt=\int_0^1G(x,t)g(t)dt+\int_0^1G(x,t)\delta_g(t)dt=y(x)+\delta_y(x)
\end{equation}
où \( \delta_y\) est une fonction continue ainsi définie :
\begin{equation}
    \delta_y(x)=\int_0^1G(x,t)\delta_g(t)dt.
\end{equation}

Supposons que \( \| \delta_g \|_{\infty}=\epsilon\). Alors des majorations donnent
\begin{equation}        \label{EQooRJZPooCSlUGi}
    | \delta_y(x) |\leq \epsilon\int_0^1| G(x,t) |dt=\epsilon(1-x)\int_0^xtdt+\epsilon x\int_x^1(1-t)dt=\frac{ \epsilon }{2}x(1-x).
\end{equation}
Mais la fonction \( x\mapsto x(1-x)\) a son maximum en \( x=\frac{ 1 }{2}\), donc nous pouvons donner une majoration indépendante de \( x\):
\begin{equation}        \label{EQooTWQHooUPYRuc}
    \| \delta_y \|_{\infty}\leq \frac{1}{ 8 }\| \delta_g \|_{\infty}.
\end{equation}
Notons que la majoration \eqref{EQooTWQHooUPYRuc} en norme uniforme a l'air plus impressionnante, mais la majoration \eqref{EQooRJZPooCSlUGi} donnant une majoration séparée pour chaque \( x\) est en réalité plus précise.


\chapter{Équations aux dérivées partielles}
\input{179_edp}

\chapter{Numérique}
% This is part of Mes notes de mathématique
% Copyright (C) 2010-2013,2016-2017, 2019-2020
%   Laurent Claessens
% See the file LICENCE.txt for copying conditions.

D'autres lectures agréables dans \cite{GianlucaB}.

%+++++++++++++++++++++++++++++++++++++++++++++++++++++++++++++++++++++++++++++++++++++++++++++++++++++++++++++++++++++++++++
\section{Introduction}
%+++++++++++++++++++++++++++++++++++++++++++++++++++++++++++++++++++++++++++++++++++++++++++++++++++++++++++++++++++++++++++

À quels types de problèmes peut-on s'attendre lorsqu'on se lance dans du calcul numérique, et en particulier dans la résolution numérique d'équations (algébrique, différentielles ou aux dérivées partielles, etc) ?

Quelques réflexions en vrac sur ce sujet.

\begin{enumerate}
    \item
        Les erreurs de représentation de nombres : troncature et propogation de décalales (\emph{drift}),
    \item
        Erreur de compensation (\emph{cancellation}),
    \item
        Conditionnement, stabilité : les réponses peuvent fortement dépendre des paramètres,
    \item
        Si on utilise une méthode itérative, comment savoir à quel moment on s'arrête ? Calculer la différence \( | x_k-x_{k-1} |\) mène t-il à une erreur de cancellation ?
    \item
        Lors d'une implémentation, les matrices des systèmes à résoudre sont souvent très grandes et/ou très creuses. Cela pose la question de la manière de les enregistrer.
    \item
        Pour la parallélisation, il faut faire attention au fait que parfois créer un nouveau processus demande plus de ressources que le mini-calcul qu'on voulait faire. Donc il ne faut pas toujours paralléliser tout ce qui est théoriquement parallélisable.
    \item
        Le fait que certaines méthodes sont non-déterministes (Monté-Carlo) mène à des problèmes pour les tests unitaires des implémentations.
\end{enumerate}

%+++++++++++++++++++++++++++++++++++++++++++++++++++++++++++++++++++++++++++++++++++++++++++++++++++++++++++++++++++++++++++
\section{Représentations numériques}
%+++++++++++++++++++++++++++++++++++++++++++++++++++++++++++++++++++++++++++++++++++++++++++++++++++++++++++++++++++++++++++

Dans cette section, les séquences de chiffres écrites entre crochet sont à comprendre comme des séquences de chiffres qui représentent une quantité suivant un codage donné.

%---------------------------------------------------------------------------------------------------------------------------
\subsection{Entier relatif en complément à deux (binaire)}
%---------------------------------------------------------------------------------------------------------------------------

Si nous avons \( m\) bits pour coder un entier relatif, une idée serait de prendre le premier bit pour le signe (\( 0\) pour positif et \( 1\) pour négatif) et les autres pour la valeur absolue. Deux inconvénients :
\begin{enumerate}
    \item
        Il y a deux codages pour le zéro, donc gaspillage.
    \item
        L'algorithme pour faire la somme passe mal. Par exemple pour faire \( 1+(-1)\), le \( 1\) est codé comme \( [001]\) et le \( -1\) par \( [101]\) et la somme se ferait naïvement comme
        \begin{equation*}
            \begin{array}[]{ccc}
                0&0&1\\
                1&0&1\\
                \hline
                1&1&0\\
            \end{array}
        \end{equation*}
        Donc le résultat est \( [110]\) qui s'interprète comme \( -2\). Complètement faux.
\end{enumerate}
Une solution est d'utiliser le \defe{complément à deux}{complément!à deux}, qui est la façon usuelle de représenter des entiers signés.
\begin{description}
    \item[Les entiers positifs] se codent normalement, en laissant à zéro le premier bit (donc si nous disposons de \( m\) bits, nous codons sur \( m-1\) bits).
    \item[Les entiers négatifs] se codent en trois étapes.
        \begin{itemize}
            \item coder la valeur absolue
            \item inverser tous les bits (d'où le nom de «complément à deux» )
            \item soustraire \( 1\).
        \end{itemize}
\end{description}

\begin{example}
    Pour coder \( -1\) nous faisons
    \begin{itemize}
        \item Nous codons \( 1\) : \( [001]\)
        \item Nous inversons tous les bits : \( [110]\)
        \item Nous faisons \( -1\) : \( [101]\).
    \end{itemize}
\end{example}

Avec ce système, la somme passe bien : calculer \( 1+(-1)\) donne
    \begin{equation*}
        \begin{array}[]{ccc}
            0&0&1\\
            1&0&1\\
            \hline
            1&1&0\\
            \hline
        \end{array}
    \end{equation*}
La réponse est donc \( [110]\) qu'il faut interpréter via le complément à deux.
\begin{equation}
    110\stackrel{+1}{\longrightarrow}111\stackrel{\text{complément}}{\longrightarrow}000.
\end{equation}
Et ce dernier \( [000]\) s'interprète comme zéro.

\begin{definition}[Entier signé en complément à deux\cite{ooAPFIooUfhqqG}]
    La suite de bits \( [a_{m-1}\ldots a_0]\) s'interprète via la formule
    \begin{equation}        \label{EQooXFHKooHRXDmZ}
        -a_{m-1}2^{m-1}+\sum_{i=0}^{m-2}a_i2^i.
    \end{equation}
\end{definition}

Le premier bit donne effectivement le signe du nombre, mais l'interprétation d'un nombre n'est pas aussi simple que ce que l'on pourrait croire de prime abord.

\begin{example}[Entier signé en \( 8\) bits]
    Que pouvons nous faire avec \( 8\) bits ? Le plus grand nombre est codé par \( [01111111]\) qui vaut \( \sum_{k=0}^62^k=2^7-1=127\). (avez-vous utilisé la somme \eqref{EqASYTiCK} ?)

    Le plus petit nombre codable en \( 8\) bits n'est pas \( [11111111]\) mais bien \( [10000000]\) (cela est plus clair en regardant la formule \eqref{EQooXFHKooHRXDmZ} qu'en tentant de suivre la construction du complément à deux) qui signifie \( -2^7=-128\).

    Nous pouvons donc coder tous les nombres de \( -128\) à \( 127\).
\end{example}

Plus généralement un système qui codes des entiers signés en \( N\) bits utilisant le complément à deux peut coder de \( -(2^{N-1})\) à \( 2^{N-1}-1\).

\begin{normaltext}[Le dépassement]
    Que se passe-t-il lorsque nous commettons un dépassement ? Calculons sur \( 3\) bits la somme \( [011]+[001]\) qui revient à ajouter \( 1\) au nombre le plus grand :
    \begin{equation*}
        \begin{array}[]{ccc}
            0&1&1\\
            0&0&1\\
            \hline
            1&0&0\\
        \end{array}
    \end{equation*}
    qui signifie \( -2^2=-4\). Lors d'un dépassement, nous retombons automatiquement sur le plus petit.

    Ce phénomène est bien connu des personnes qui programment sans faire attention dans certains languages de programmation qui ne font pas attention à votre place.
\end{normaltext}


\begin{definition}[Représentation en virgule fixe]
	Soit $x$ un réel. On définit sa \defe{représentation en virgule fixe}{représentation!virgule fixe} par
	\begin{equation}
		x=\{[x_nx_{n-1}...x_0,x_{-1}...x_{-m}], b, s\}
	\end{equation}
	avec  $b\in\eN, b\geq2$, $s\in\{0,1\}$ et $x_j\in\eN,x_j<b$ suivant la formule
	\begin{equation}
		x=(-1)^{s}\sum_{j=-m}^nx_j.b^j.
	\end{equation}
\end{definition}

%---------------------------------------------------------------------------------------------------------------------------
\subsection{Représentation en virgule flottante}
%---------------------------------------------------------------------------------------------------------------------------

\begin{definition}[Représentation en virgule flottante]     \label{DEFooLYONooBNskty}
    La \defe{représentation en \href{https://docs.python.org/tutorial/floatingpoint.html}{virgule flottante} normalisée}{Représentation!virgule flottante normalisée} en base \( b\) d'un nombre est la donnée de
    \begin{enumerate}
        \item
            Un bit \( s\) pour le signe
        \item
            Un entier \emph{non signé} \( q\) de \( e\) chiffres pour l'exposant
        \item
            Une suite de chiffres \( [a_1\ldots a_m]\) pour la mantisse.
    \end{enumerate}
    Ces données s'interprètent via la formule
    \begin{equation}        \label{EQooAGWJooRuBbBn}
        \fl( s,q,[a_1,\ldots, a_m]  )=(-1)^s\sum_{j=1}^mb^ja_j\times b^{q-d}
    \end{equation}
    où \( d=b^{e-1}\) est le \defe{décalage}{décalage}.
\end{definition}
Une idée à retenir est que l'exposant est un entier non signé parce qu'il est plus simple d'introduire un décalage dans la formule \eqref{EQooAGWJooRuBbBn} que de compliquer l'écriture de l'exposant.

%---------------------------------------------------------------------------------------------------------------------------
\subsection{Simple précision, IEEE-754}
%---------------------------------------------------------------------------------------------------------------------------

En écriture binaire, la représentation en virgule flottante est un peu différente parce qu'il y a une idée supplémentaire; la simple précision que nous allons voir maintenant n'est donc pas un cas particulier de~\ref{DEFooLYONooBNskty} avec \( b=2\).

Nous commençons par une description informelle de la précision simple avant de donner la définition. La représentation en \defe{précision simple}{précision!simple} d'un nombre se fait sur \( 32\) bits répartis comme suit :
\begin{enumerate}
    \item
        \( 1\) bit pour le signe,
    \item
        \( 8\) bits pour l'exposant interprété comme nombre entier non signé
    \item
        \( 23\) bits pour la mantisse
\end{enumerate}
Soit le triple
\begin{equation}
    \big( s,q,[a_1,\ldots, a_{23}] \big)
\end{equation}
Dans le cas générique, l'idée est de donner \( 24\) bits pour la mantisse, mais en gardant en tête le fait que de toutes façons, le premier bit doit être \( 1\), sinon il suffirait de décaler, c'est-à-dire changer l'exposant. Par conséquent la mantisse ne reçoit que \( 23\) bits; il y a un «\( 1\)» sous-entendu en première position. Donc la mantisse \( [a_1,\ldots, a_{23}]\) est à lire comme le nombre
\begin{equation}
    1,a_1\ldots a_{23}=1+\sum_{j=1}^{23}a_j2^{-j}.
\end{equation}
\begin{example}
    La mantisse \( [011100\ldots 0]\) signifie \(1,0111=1+2^{-2}+2^{-3}+2^{-4}=1+\frac{1}{ 4 }+\frac{1}{ 8 }+\frac{1}{ 16 } \).
\end{example}
Cela pour justifier la formule
\begin{equation}
    \SimplePrec\big( s,q,[a_1,\ldots, a_{23}] \big)=(-1)^s\big( 1+\sum_{j=1}^{23}a_j2^{-j} \big)2^{q-127}.
\end{equation}
Notons :
\begin{enumerate}
    \item
        Le «\( 1+\)» dans la parenthèse correspond au \( 1\) implicite en première position de la mantisse.
    \item
        Il y a un décalage de \( 127\) dans l'exposant, parce que \( q\) est un entier non signé.
\end{enumerate}

Notons que cette règle du \( 1\) implicite dans la mantisse empêche d'écrire le nombre \( 0\), et ne permet pas d'écrire des nombres franchement petits parce que le \( 1\) implicite est en \emph{première} position dans la mantisse.

D'où l'idée de donner une règle particulière lorsque l'exposant vaut \( 0\). Lorsque l'exposant est \( q=0\), alors nous ne considérons pas de \( 1\) implicite dans la mantisse, et le décalage de l'exposant est \( -126\) au lieu de \( -127\). D'où la formule
\begin{equation}
    \SimplePrec(s,q=0,[a_1\ldots a_{23}])=(-1)^s2^{-216}\sum_{j=1}^{23}a_j2^{-j}.
\end{equation}
En particulier, si \( q=0\) et \( a=[0\ldots 0]\), nous avons le nombre zéro exact (il y a deux possibilités pour le code).

Enfin, nous avons des cas particuliers lorsque l'exposant est maximum, c'est-à-dire lorsque \( q=[1111\,1111]=2^8-1=255\). Dans ce cas, le nombre codé est soit \( +\infty\) soit \( \NaN\). Nous posons 
\begin{equation}
\SimplePrec(s,q=255,a=0)=+\infty
\end{equation}
et 
\begin{equation}
 \SimplePrec(s,q=255,a\neq 0)=NaN.
\end{equation}
 Il y a en réalité plusieurs valeurs différentes de \( \NaN\), mais nous n'entrons pas dans ces détails\cite{ooPOZNooQlGiUN}.

\begin{definition}[Représentation en simple précision (binaire)]        \label{DEFooEIOZooYLDVjs}
    La représentation en \defe{précision simple}{précision!simple} d'un nombre se fait sur \( 32\) bits répartis comme suit :
    \begin{enumerate}
        \item
            \( 1\) bit pour le signe,
        \item
            \( 8\) bits pour l'exposant interprété comme nombre entier non signé
        \item
            \( 23\) bits pour la mantisse
    \end{enumerate}

    Un nombre est représenté par un triple
    \begin{equation}
        \big( s,q,[a_1,\ldots, a_{23}] \big)
    \end{equation}

    Selon que l'exposant \( q-d\) soit égal à \( 0\), \( 2^8-1=255\) ou autre chose, les règles d'interprétation sont différentes. Il y a donc trois cas.
    \begin{description}
        \item[Exposant \( q\) générique\cite{ooMPTNooYbSwJS}]
           Si \( q\neq 0\) et \( q\neq 255\) alors le nombre est \defe{normalisée}{nombre!normalisée}. La règle de lecture est alors
           \begin{equation}        \label{EQooEFEKooIrUaKj}
                \SimplePrec\big( s,q,[a_1,\ldots, a_{23}] \big)=(-1)^s\big( 1+\sum_{j=1}^{23}a_j2^{-j} \big)2^{q-127}.
            \end{equation}
        \item[Exposant \( q\) égal à \( 0\)] Le nombre est dit \defe{dénormalisé}{nombre!dénormalisé} et la règle de lecture est
            \begin{equation}        \label{EQooRTBFooIplydi}
                \SimplePrec\big( s,q,[a_1,\ldots, a_{23}] \big)=(-1)^s2^{-126}\sum_{j=1}^{23}a_j2^{-j}.
            \end{equation}
        \item[Exposant \( q\) égal à \( 255\)]
            La règle de lecture est alors au cas pas cas ou à peu près.
            \begin{enumerate}
                \item
                    \( \SimplePrec(s,q=255,a=0)=+\infty\).
                \item
                    \( \SimplePrec(s,q=255,a\neq 0)=\NaN\).
            \end{enumerate}
    \end{description}
\end{definition}

Vous pouvez jouer avec la simple précision dans \cite{ooOSFYooHCgMRL}.

\begin{example}[Plus petit normalisé]
    Pour faire un nombre normalisé, il faut au minimum \( q=1\). En prenant \( a_j=0\) nous obtenons le plus petit nombre normalisé possible en simple précision. La formule \eqref{EQooEFEKooIrUaKj} donne
    \begin{equation}
        \SimplePrec(1,q=1,a=0)=2^{1-127}=2^{-126}\simeq 1.17549435082229\times 10^{-38}.
    \end{equation}
\end{example}

\begin{example}[Plus grand normalisé]
    L'exposant \( q\) ne peut pas être maximum, sous peine de tomber dans les règles spéciales de \( +\infty\) ou \( \NaN\). Donc \( q=[1111\,1110]=2^{8}-2=254\). En ce qui concerne la mantisse, il faut la prendre maximale, c'est-à-dire \( a_j=1\) pour tout \( j\). Nous avons alors le nombre
        \begin{subequations}
            \begin{align}
                \SimplePrec(1,q=254,a=[1\ldots 1])&=\big( 1+\sum_{j=1}^{23}2^{-j} \big)2^{254-127}=(1-\frac{1}{ 2^{24} })2^{128}\\
                &=3.40282346638528859811704183484516925440\times 10^{38}  \label{EQooFRPYooRnxiFP}
            \end{align}
        \end{subequations}
    où nous avons utilisé la somme \eqref{EqASYTiCK} (et Sage pour le dernier calcul).
\end{example}

Notons ceci avec Sage :
\lstinputlisting{tex/sage/sageSnip003.sage}

La précisions du nombre donné en \eqref{EQooFRPYooRnxiFP} aurait été embarrassante si le type avait été un nombre en simple précision. Précision technique : en Python, le type \info{int} n'a pas de limite supérieure à part la mémoire.

\begin{example}[Plus petit non nul dénormalisé]
    Pour être dénormalisé il faut \( q=0\) (ce qui est toutefois assez logique si nous voulons un petit nombre), et pour ne pas être nul, il faut une mantisse non nulle. Donc \( a=[0\ldots 01]\). La formule \eqref{EQooRTBFooIplydi} donne alors
    \begin{equation}
        \SimplePrec(s=0,q=0,a=[0\ldots 01])=2^{-126}2^{-23}=2^{-149}\simeq 1.40129846432482\times 10^{-45}.
    \end{equation}
\end{example}

\begin{example}[Plus grand dénormalisé]      \label{EXEMooRHENooGwumoA}
    Pour être dénormalisé il faut toujours \( q=0\), mais cette fois nous prenons la plus grande mantisse possible :
    \begin{equation}
        \SimplePrec(s=0,q=0,a=[1\ldots 1])=2^{-126}\sum_{j=1}^{23}2^{-j}=2^{-216}(1-2^{-23})= 1.17549421069244\times 10^{-38}
    \end{equation}
\end{example}

Notons ceci avec Sage :
\lstinputlisting{tex/sage/sageSnip004.sage}

Vu que \( 2^{-23}\simeq 1.2\times 10^{-7}\), approximer la parenthèse par \( 1\) donne une faute sur la septième décimale, ce qui est visible en simple précision.

%+++++++++++++++++++++++++++++++++++++++++++++++++++++++++++++++++++++++++++++++++++++++++++++++++++++++++++++++++++++++++++
\section{Problèmes pour écrire des nombres}
%+++++++++++++++++++++++++++++++++++++++++++++++++++++++++++++++++++++++++++++++++++++++++++++++++++++++++++++++++++++++++++

\begin{definition}
	L'\defe{erreur relative}{erreur relative} commise en remplaçant un nombre réel $x$ par une valeur approchée $\hat{x}$ est définie par
	\begin{equation}
		\epsilon_x:=\left|\frac{x-\hat{x}}{x}\right|.
	\end{equation}
\end{definition}

L'erreur relative n'est pas influencée par l'ordre de grandeur de \( x\). En effet, l'ordre de grandeur de \( \hat x\) est certainement la même que celle de \( x\), dans la majorité des cas sans problèmes. Du coup si \( x'=200x\) alors \( \hat{x'}\simeq 200\hat{x}\) et le \( 200\) se simplifie.

Le nombre de chiffres significatifs correct dans l'approximation est donné par \( -\log_{10}(\epsilon_x)\). La partie entière de ce nombre est le nombre de chiffres tout à fait exacts et la partie décimale donne une idée sur le fait que le chiffre suivant est plus ou moins bien.


\begin{remark}
	Si nous voulons donner \( x\in \eR\) à un ordinateur, nous sommes soumis à deux erreurs :
	\begin{enumerate}
		\item
			D'abord, vu que nous ne pouvons pas taper sur le clavier toutes les décimales de \( x\), nous faisons une \defe{erreur de troncature}{erreur!troncature}.
		\item
			L'ordinateur devant convertir cela en base deux, il commet une seconde erreur, dite \defe{erreur d'assignation}{erreur!assignation}.
	\end{enumerate}
\end{remark}

%---------------------------------------------------------------------------------------------------------------------------
\subsection{Troncature : la base}
%---------------------------------------------------------------------------------------------------------------------------

Supposons que nous voulions écrire le nombre (écrit ici en base \( 10\))
\begin{equation}
	0.4567894251
\end{equation}
de façon plus facile à lire, on peut demander de ne laisser que \( t\) chiffres significatifs. Disons \( t=3\).

\begin{description}
	\item[Technique de troncature] On garde \( 3\) chiffres significatifs : \( 0.456\). Facile.
	\item[Technique d'arrondi] Vu que le premier qu'on supprime est un \( 7\), le dernier qu'on garde est majoré de \( 1\) : on écrit \( 0.457\).
\end{description}

Que faire si le premier chiffre rejeté est un \( 5\) ? En première approximation, nous pouvons prendre la règle suivante : si le premier chiffre rejeté est un \( 5\), il faut augmenter de \( 1\) de dernier chiffre gardé parce qu'il y a presque certainement encore un chiffre non nul derrière.

\begin{remark}
	Les ordinateurs travaillent tous en mode d'arrondi.
\end{remark}

\begin{example}
    Si on doit entrer le nombre \( 0.38358546\) dans un ordinateur qui ne garde que \( 3\) chiffres significatifs, il faut taper \( 0.384\) au clavier (erreur classique dans les exercices).
\end{example}

%---------------------------------------------------------------------------------------------------------------------------
\subsection{Troncature : le drift}
%---------------------------------------------------------------------------------------------------------------------------

Soit une machine ne pouvant retenir que \( 3\) chiffres significatifs et effectuant les arrondis vers le haut lorsque le chiffre à éliminer est un \( 5\). Nous notons \( \oplus\) et \( \ominus\) les opérations d'addition et soustraction avec arrondis\cite{ooAGVZooTIcZZb}. Les égalités comprenant plus de trois chiffres significatifs sont des égalités au sens de la machine. Nous écrirons donc sans états d'âme :
\begin{equation}
    1\oplus0.555=1.555=1.56.
\end{equation}

Considérons la suite numérique
\begin{subequations}
    \begin{numcases}{}
        x_0=1.00\\
        x_n=(x_{n-1}\ominus y)\oplus y
    \end{numcases}
\end{subequations}
avec \( y=-0.555\).

Nous avons
\begin{equation}
    x_1=(1\oplus 0.555)\ominus 0.555=1.56\ominus 0.555=1.005=1.01
\end{equation}
et ensuite
\begin{equation}
    x_2=(1.01\oplus 0.555)\ominus 0.555=1.565\ominus 0.555=1.57\ominus 0.555=1.015=1.02.
\end{equation}
Et ainsi de suite. La suite est donc croissante alors que la définition nous donnerait envie d'avoir \( x_n=x_0\) pour tout \( n\).

\begin{remark}
    En réalité, cette suite se stabilise à \( x_n=10\) pour tout \( n\) à partir de \( n=845\). En effet,
    \begin{equation}
        (10\oplus 0.555)\ominus 0.555=10.555\ominus 0.555=10.6\ominus 0.555=10.045=10.
    \end{equation}
    Le fait est qu'à ce moment, l'erreur de troncature est assez loin dans les décimales pour que le premier chiffre négligé soit un ``0'' au lieu d'un ``5''.

    Notons toutefois que cette stabilité n'est pas là pour nous rassurer parce qu'elle n'en est pas moins complètement fausse.
\end{remark}

La règle de troncature adoptée dans Sage est d'arrondir au nombre pair le plus proche lorsque le premier nombre à négliger est un \( 5\). Donc \( 12.5\) s'arrondit à \( 12\) plutôt que \( 13\).

\begin{example}
	Soient les expressions (algébriquement égales) :
	\begin{enumerate}
		\item
			\(A= x(x+1)\)
		\item
			\(B= x^2+x\)
	\end{enumerate}
	Nous savons que
	\begin{equation}
		x=\fl(x)=10^{-30}
	\end{equation}
	et
	\begin{equation}
		1=\fl(1)
	\end{equation}
	parce que pour \( 1\) et \( 10^{-30}\), il n'y a pas d'erreurs d'assignation.

	En précision simple, \( 10^{-30}+1=1\) parce qu'en précision simple, il n'y a que \( 7\) ou \( 8\) chiffres significatifs\footnote{Erreur de « relation normale».}.

	Nous avons $A=10^{-30}$, mais \( x^2\) donne un \info{underflow} parce que \( 10^{-60}\) ne peut pas être représenté en précision simple. En pratique, beaucoup de logiciels en font \( 0\). Dans ce cas, en réalité \( B\) donne effectivement \( 10^{-30}\) après avoir fait \( x^2+x=0+x=10^{-30}\).
\end{example}

%---------------------------------------------------------------------------------------------------------------------------
\subsection{Quelques bonnes règles}
%---------------------------------------------------------------------------------------------------------------------------

\begin{enumerate}
	\item
		Si on a plusieurs nombres à additionner ou soustraire, il vaut mieux commencer par sommer ou soustraire ceux dont on sait qu'ils ont le même ordre de grandeur. Il n'y a donc pas tout à fait «associativité» des erreurs.
	\item
		Les opérations délicates sont l'addition et la soustraction. La multiplication et la division sont sans dangers, à part l'erreur de dépassement du maximum. Dans une multiplication, on perd au pire quelques chiffres significatifs, mais certainement les derniers, pas les premiers.
\end{enumerate}

%---------------------------------------------------------------------------------------------------------------------------
\subsection{Erreur de ``cancellation''}
%---------------------------------------------------------------------------------------------------------------------------

Lorsque deux nombres sont de même ordre de grandeur, avec plusieurs nombres significatifs identiques. La cancellation est le fait que, suite à la soustraction, tous les chiffres significatifs ou presque se sont simplifiés et qu'il ne reste plus que des chiffres non significatifs.

\begin{example}[\cite{ooIZQWooYJmQmW}]
    Sur une machine ne gardant que \( 4\) chiffres significatifs, faire
    \begin{equation}
        0.5678\times 10^6-0.5677\times 10^6 = 0.0001\times 10^6=0.1000\times 10^3.
    \end{equation}
    Le fait est que les trois derniers zéros ne sont pas significatifs, mais maintenant la machine nous fait croire qu'ils le sont.

    Une autre façon de voir ce problème est d'imaginer qu'il faille calculer la différence 
    \begin{equation}
        0.5678\,289798\times 10^6 - 0.5677\,3136907
    \end{equation}
    sur cette machine. Certes la machine nous autorise à avoir \( 4\) chiffres significatifs, donc au moment d'entrer les nombres nous perdons un beau paquet de chiffres. Mais au moment de faire la différence, nous perdons (presque) tout le reste. Donc là où nous pouvions espérer avoir \( 4\) chiffres significatifs de la différence, nous n'en avons que \( 1\). Les trois derniers zéros de la réponse (\( 0.1000\times 10^3\)) sont faux.
\end{example}

\begin{remark}  \label{REMooRQIJooNLdAZE}
    L'erreur de cancellation provoque des chiffres significatifs faux, mais ne provoque pas de faute dans l'\emph{ordre de grandeur} des réponses\quext{Est-ce bien vrai, cela ?}. Donc si nous voulons nous assurer que \( a\) et \( b\) sont égaux «à erreur numérique près», le test
    \begin{equation}
        | a-b |<\epsilon
    \end{equation}
    est valide, malgré l'erreur de cancellation qui ne manquera pas de se produire dans le calcul de la différence.
\end{remark}

\begin{example}
	Soit à résoudre l'équation \( ax^2+bx+c=0\) avec \( a,b,c\neq 0\) et \( b^2-4ac>0\). Solution :
	\begin{equation}
		x_{1,2}=\frac{ -b\pm\sqrt{b^2-4ac} }{ 2a }.
	\end{equation}

	Supposons que \( | 4ac |\ll b^2\) avec tout de même pas tellement petit qu'on se perd dans la précision. Bref, on suppose que seules quelques dernières décimales de \( b^2-4ac\) sont différentes de zéro.

	On a :
	\begin{subequations}
		\begin{align}
			\sqrt{b^2-4ac}&=\sqrt{\tilde b}= | \tilde b | \\
			x_1&=\frac{ -b-\sqrt{b^2-4ac} }{ 2a }\\
			x_2&=\frac{ -b+\sqrt{b^2-4ac} }{ 2a }
		\end{align}
	\end{subequations}
	Si \( b>0\), nous avons une erreur de cancellation dans \( x_2\) parce qu'on fait la différence entre deux nombres presque égaux. Donc \( x_2\) mal calculé. Par contre \( x_1\) est bien calculé.


	Si par contre \( b<0\), c'est le contraire.


	Avec \( a=10^{-3}\), \( b=0.8\), \( c=-1.2\times 10^{-5}\). À la main nous obtenons : \( x_1=-800\), \( x_2=1.5\times 10^{-5}\), et un ordinateur se tromperait \ldots


\lstinputlisting{tex/sage/sageSnip001.sage}

	Donc Sage ne tombe pas dans le piège.
\end{example}

Comment résoudre ce problème ? Ou, autre façon de poser la question : comment Sage a fait pour résoudre le problème ?

Utilisons les relations coefficients-racines :
\begin{subequations}
	\begin{align}
		x_1+x_2&=-b/a\\
		x_1x_2&=c/a
	\end{align}
\end{subequations}
La première lie les deux racines par des opérations de addition et soustractions, et donc n'est pas intéressantes. La seconde est bien. Si nous connaissons \( x_1\), nous calculons
\begin{equation}
	x_2=\frac{ c }{ ax_1 }.
\end{equation}

Quitte à redéfinir \( x_1\) et \( x_2\), la solution bien calculée est :
\begin{equation}
	x_1=\frac{ -b-\signe(b)\sqrt{b^2-4ac} }{ 2a }.
\end{equation}

\begin{example}
	Nous considérons :
	\begin{equation}
		f(x)=cos(x+\delta)-\cos(x).
	\end{equation}
	Cela a une erreur de cancellation lorsque \( | \delta |\ll | x |\). On élimine l'erreur de cancellation par
	\begin{equation}
		f(x)=-2\sin(\delta/2)\sin\left( x+\frac{ \delta }{ 2 } \right).
	\end{equation}

	\begin{probleme}
		Pourquoi la condition pour avoir l'erreur est \( \delta\ll x\) et non simplement \( \delta\ll 1\) ?
	\end{probleme}

\end{example}

\begin{example}
	Pour
	\begin{equation}
		f(x)=\sqrt{x+\delta}-\sqrt{x}.
	\end{equation}
	On fait la coup du binôme conjugué :
	\begin{equation}
		f(x)=\frac{ \delta }{ \sqrt{x+\delta}+\sqrt{x} }.
	\end{equation}
	Plus d'erreur de cancellation, vu qu'au dénominateur nous avons une somme de deux positifs.
\end{example}

Les erreurs de cancellation ne se résolvent pas en augmentant la précision des nombres donnés.

\begin{example}[Dans la vie réelle]
    La préparation de l'exemple~\ref{EXooJXIGooQtotMc} nous a porté à calculer la différence entre \( \exp(x)\) et \( f_{30}(x)\) où \( f_{30}\) est censée être une bonne approximation de l'exponentielle. Des erreurs de cancellation sont donc à craindre.

Et en effet, le code suivant produit un résultat non déterministe :
\lstinputlisting{tex/sage/sageSnip016.sage}

Voir la question ici :\\ \url{https://ask.sagemath.org/question/37946/undeterministic-numerical-approximation/}

\end{example}

%---------------------------------------------------------------------------------------------------------------------------
\subsection{Calcul d'une dérivée}
%---------------------------------------------------------------------------------------------------------------------------

Pour calculer la dérivée de \( f\) en \( a\), il est loisible d'utiliser la formule
\begin{equation}
    f'(a)=\lim_{h\to 0} \frac{ f(a+h)-f(a) }{ h }.
\end{equation}
Le numérateur est alors sujet à une erreur d'absorption dans le calcul de \( a+h\) et ensuite une erreur de cancellation dans le calcul de la différence.

En utilisant la formule
\begin{equation}
    f'(a)=\lim_{h\to 0} \frac{ f(a+h)-f(a-h) }{ 2h }
\end{equation}
nous pouvons espérer avoir une erreur de cancellation plus petite.


%---------------------------------------------------------------------------------------------------------------------------
\subsection{Erreur d'absorption}
%---------------------------------------------------------------------------------------------------------------------------

L'addition d'un nombre avec un nombre très différent peut faire perdre de l'information sur le plus petit. Par exemple avec \( 4\) chiffres significatifs,
\begin{equation}
    0.5678\oplus 0.0001237=0.5679
\end{equation}
où nous avons perdu presque toute l'information du petit nombre.

Une situation particulièrement ennuyeuse est celle où justement c'est le petit nombre qui nous intéresse parce que le grand est censé se simplifier :
\begin{equation}
    (0.0001327\oplus 0.5678)\ominus 0.5678=0.5679\ominus 0.5678=0.0001
\end{equation}
qui ne possède qu'un seul chiffre significatif correct alors que voyant le calcul, la réponse aurait pu être trouvée.

Moralité : si certains manipulations algébrique peuvent faire apparaitre des simplifications avant de passer le calcul à la machine, il est bon de les effectuer.

%+++++++++++++++++++++++++++++++++++++++++++++++++++++++++++++++++++++++++++++++++++++++++++++++++++++++++++++++++++++++++++
\section{Conditionnement et stabilité}
%+++++++++++++++++++++++++++++++++++++++++++++++++++++++++++++++++++++++++++++++++++++++++++++++++++++++++++++++++++++++++++

\begin{definition}      \label{DEFooYIFAooSJbMkC}
	Soit $F$ une fonction à valeurs réelles définie sur $X\times D$ où $X$ et $D$ sont des espaces vectoriels réels normés. Le problème de la recherche des solutions de
	\begin{equation}
		F(x,d)=0
	\end{equation}
	est dit \defe{stable}{stable} autour de \( d_0\in D\) si
	\begin{enumerate}
		\item
			la solution $x=x(d)$ existe et est unique pour tout $d$;
		\item \label{ItemProbStableB}
			Pour tout $\eta>0$, et pour tout $d_0$, il existe un nombre $K>0$ tel que $\| d-d_0\|<\eta$ entraine $\|x(d)-x(d_0)\|\;\leq\;K\;\|d-d_0\|$.
	\end{enumerate}
    La seconde condition est le fait que \( x\) soit Lipschitz\footnote{Définition~\ref{DEFooQHVEooDbYKmz}.} sur un voisinage de \( d_0\).
\end{definition}

\begin{example}[Stabilité de la différence]    \label{ExooXJONooTAYZVc}
    Prenons le problème qui consiste à calculer la différence entre deux nombres : \( x=a-b\). Cela se traduit par
    \begin{equation}
        \begin{aligned}
            F\colon \eR\times \eR^2&\to \eR \\
            x&\mapsto x-a+b.
        \end{aligned}
    \end{equation}
    Nous avons :
    \begin{subequations}
        \begin{align}
            \big| x(a,b)-x(a',b') \big|&=| a-b-a'+b' |\\
            &\leq| a-a' |+| b-b' |\\
            &=\|  (a,b)-(a',b')  \|_1
        \end{align}
    \end{subequations}
    où nous avons utilisé la norme \( \| . \|_1\) sur \( \eR^2\). Par la proposition~\ref{PropLJEJooMOWPNi} sur les équivalences de normes, le nombre \( K=\sqrt{2}\) fonctionne pour toute valeurs de \( \eta\).

    La problème de la différence est donc un problème stable.
\end{example}

\begin{example}[Stabilité de la multiplication]
    Si \( a\) est fixé, le problème de calculer \( ab\) (\( b\) est la donnée) est stable. En effet ce problème est donné par la fonction \( F(x,b)=x-ab\), dont la solution est \( x(b)=ab\). Nous avons donc
    \begin{equation}
        \big| x(b)-x(b') \big|=| ab-ab' |=| a | |b-b' |.
    \end{equation}
    La constante de Lipschitz de ce problème est donc \( | a |\).
\end{example}

\begin{definition}
    Le nombre
    \begin{equation}        \label{EqDefAABSOLU}
	    K_{abs}(d_0,\eta):=\sup_{d\text{ tel que }|d_0-d|<\eta}\frac{\| x(d)-x(d_0)\|_X}{\|d-d_0\|_D}
    \end{equation}
    est appelé le \defe{conditionnement absolu}{conditionnement!absolu} du problème autour de $d_0$.

	Soit $F(x,d)=0$ un problème stable de conditionnement absolu $K_{\text{abs}}(d,\eta)$.  Le conditionnement relatif est défini par
    \begin{equation}        \label{DEFEQooSXDBooYbvGrC}
		K_{\text{rel}}(d,\eta):=K_{\text{abs}}(d,\eta)\frac{\| d \|_D}{\|x(d)\|_X}.
	\end{equation}
	Le problème est dit \defe{bien conditionné}{bien!conditionné} près de $d$ si $K_{\text{rel}}(d,\eta)$ est petit.
\end{definition}

\begin{example}[Mauvais conditionnement de la différence]
    Reprenons le problème de la différence, mais en fixant \( a\). Nous avons donc \( x(b)=a-b\) et le conditionnement absolu est
    \begin{equation}
        \sup\frac{ | x(b)-x(b_0) | }{ | b-b_0 | }=1
    \end{equation}
    Le conditionnement relatif est :
    \begin{equation}
        K_{rel}(b_0,\eta)=\frac{ | b | }{ | a-b | }.
    \end{equation}
    Et donc le problème est mal conditionné autour de \( a\).

    Autrement dit, si \( a'\) est un nombre proche de \( a\), calculer la différence \( a-a'\) est un problème mal conditionné.
\end{example}

\begin{example}[Bon conditionnement de la multiplication]
    Pour le problème \( F(x,b)=x-ab\) nous avons
    \begin{equation}
        K_{abs}=\sup_{b'}\frac{ | ab-ab' | }{ | b-b' | }=| a |.
    \end{equation}
    Et aussi
    \begin{equation}
        K_{rel}=a\frac{ | b | }{ | ab | }=1.
    \end{equation}
    Le conditionnement relatif du problème de la multiplication est donc toujours \( 1\). Il est donc un toujours un problème bien conditionné.
\end{example}

Ne pas confondre :
\begin{description}
	\item[Le conditionnement] provient du problème lui-même.
	\item[La stabilité] provient de l'algorithme de résolution.
\end{description}

\begin{example}[Un problème mal conditionné]
	Le système
	\begin{subequations}
		\begin{numcases}{}
			2.1x +  3.5y = 8 \\
			4.19x + 7.0y = 15
		\end{numcases}
	\end{subequations}
	Solution : \( x=100\), \( y=  -57.714285\ldots \) (périodique)

	Perturbons : nous remplaçons \( 4.19\) par \( 4.192\). L'erreur relative est : \( 4.77\times 10^{-4}\).

	Solution : \( \bar x=125\), \( \bar y=-72.714285\ldots\), avec donc erreur relative de \( 0.26\). Autrement dit : l'erreur relative sur la solution est grande même avec une petite erreur relative sur la donnée.

	C'est un problème mal conditionné.

	Le fait est que c'est une intersection de deux droites presque parallèles. Donc effectivement une petite perturbation d'une des deux droites donne une grande perturbation du point d'intersection.

	Le fait est qu'un ordinateur effectue \emph{toujours} une perturbation, au moins de l'ordre \( 10^{-16}\) pour ne fut-ce que représenter les nombres. C'est-à-dire une perturbation sur les six nombres définissant le système. Il n'y a donc pas d'espoir d'obtenir un algorithme donnant une bonne réponse.
\end{example}

Un résultat pratique pour étudier le conditionnement d'un problème est le suivant.
\begin{corollary}       \label{CorConditionnementNormeNabla}
	Soit $x=x(d)$ un problème stable. Supposons $\eD$ de dimension finie, supposons que $U$ est ouvert dans $\eD$. Supposons encore $x\colon U\to \eR$ différentiable en $d_0$. Alors quand $\eta$ est petit, on a
	\begin{equation}
		K_{\text{abs}}^{\eta}(d_0)\sim \| \nabla x(d_0) \|.
	\end{equation}
\end{corollary}

\begin{lemma}   \label{LemITCxqyS}
	 Tout  problème de la forme $x=x(d)$ avec $d\in\eR$ et $x \in C^1(\eR)$ est stable.
\end{lemma}

\begin{proof}
	Il faut démontrer qu'une fonction $C^1$ sur $\eR$ vérifie automatiquement la condition~\ref{ItemProbStableB} de la définition de la stabilité. Pour cela, remarquons qu'une fonction $C^1$ possède une dérivée continue, et donc bornée sur tout compact\footnote{Un compact est un ensemble fermé et borné, typiquement un intervalle du type $[a,b]$.}

	Prenons $\eta>0$ et $d_0\in\eR$ et puis un $d$ tel que $| d-d_0 |<\eta$. Par le théorème des bornes atteintes, la fonction $x'$ est bornée sur l'intervalle $[d_0-\eta,d_0+\eta]$. Appelons $K$ un majorant de $x'$ sur cet intervalle. La fonction
	\begin{equation}
		f(d)=x(d_0)+K| d-d_0 |
	\end{equation}
	majore $x(d)$, et donc on a
	\begin{equation}
		\big| x(d)-x(d_0) \big|\leq K| d-d_0 |.
	\end{equation}

	Attention : vérifier si ce raisonnement est correct avec $d_0>d$, et adapter au besoin.
\end{proof}

\begin{example} \label{ExRZrOeoi}
	Un exemple de problème stable de la forme  $x=x(d)$ avec $d\in\eR$ et $x \in C^0(\eR)\setminus C^1(\eR)$.

	La fonction
	\begin{equation}
		x(d)=\begin{cases}
			0   &   \text{si }x\geq 0\\
			x   &   \text{si }x>0
		\end{cases}
	\end{equation}
	est continue, mais pas $C^1$ (non dérivable en $x=0$). La dérivée est partout bornée par $1$, et donc le problème est stable.

	Un autre exemple très classique serait de prendre $x(d)=| d |$. Dans ce cas, on peut prendre n'importe que $\eta$ et $K=1$. Le calcul est que
	\begin{subequations}
		\begin{align}
			| x(d)-x(d_0) |&<K| d-d_0 |\\
			\big| | d |-| d_0 | \big|&<| d-d_0 |.
		\end{align}
	\end{subequations}
	Cette dernière inéquation est correcte, comme on peut le voir en mettant au carré les deux membres.

\end{example}

\begin{example} \label{PIluknK}
	Un exemple de problème instable de la forme $x=x(d)$ avec $d\in\eR$ et $x \in C^0(\eR)$.

	Un exemple assez classique de fonction dont la dérivée n'est pas bornée sans pour autant que la fonction aie un comportement immoral\footnote{Penser à $x\mapsto x\sin(1/x)$.} est $x\mapsto\sqrt{x}$. Afin d'avoir une fonction définie sur $\eR$ tout entier, nous regardons la fonction
	\begin{equation}
		x(d)=\sqrt{|d|}.
	\end{equation}
	Si nous considérons maintenant $d_0=0$ et n'importe quel $\eta$, nous avons
	\begin{equation}
		\frac{ | x(d)-x(d_0) | }{ | d-d_0 | }=\frac{ \sqrt{d} }{ d }=\frac{1}{ \sqrt{d} }.
	\end{equation}
	Il n'est pas possible de trouver un $K$ qui majore ce rapport. Le problème est donc mal conditionné.

	Attention : dans ce calcul nous avons supposé $d>0$. Pensez à adapter au cas $d<0$.
\end{example}

\begin{example}[Problème bien conditionné avec algorithme instable]
	Soit à calculer
	\begin{equation}
		I_n=\frac{1}{ e }\int_0^1x^ne^xdx
	\end{equation}
	avec \( n\geq 0\). Par partie, nous obtenons :
	\begin{equation}
		I_n=1-nI_{n-1}.
	\end{equation}
	D'autre part, \( I_0=\frac{ e-1 }{ e }\), \( I_1=\frac{1}{ e }\). Puis par récurrence, c'est tout en main.

	Du côté de l'ordinateur, nous lui donnons forcément une approximation de \( I_1\), parce que nous lui donnons une approximation de \( e\). Soit l'erreur \( \epsilon_1\) sur \( I_1\).

	Sans démonstration :
	\begin{lemma}
		Nous avons \( \lim_{n\to \infty} I_n=0\).
	\end{lemma}
	Mais numériquement, il n'est pas possible de rester longtemps sous \( \epsilon_1\) parce que nous n'espérons pas avoir une erreur plus petite que ça. Donc à partir du moment où \( I_n<\epsilon_1\), les valeurs sont toutes complètement fausses. Cela est le mieux que l'on puisse espérer. Mais la réalité est pire.

	En réalité, en lançant le calcul sur un ordinateur, les valeurs sont même croissantes avec \( n\) à partir d'un certain moment.

	On peut étudier l'erreur et montrer que l'erreur est donnée par :
	\begin{equation}
		\epsilon_n=(-1)^{n-1}n!\epsilon_1.
	\end{equation}
	Mais comme la factorielle est tellement forte que c'est sans espoir d'aller loin en essayant très fort de donner une petite erreur sur \( \epsilon_1\).

\end{example}

Il existe heureusement un algorithme stable pour cette intégrale. La formule est :
\begin{equation}
	I_{n-1}=\frac{1}{ n }(1-I_n).
\end{equation}
Si nous savons un \( I_N\) avec \( N\) grand, cette formule donne les \( I_i\) avec \( i=N,N-1,\ldots, 2\). Posons donc \( I_N=a\in \eR\) n'importe comment. Donc \( \epsilon_N\) est grand. Mais il se trouve que l'erreur sur \( \epsilon_1\) est donnée par
\begin{equation}
	\epsilon_1=\frac{ (-1)^{N-1} }{ N! }\epsilon_N.
\end{equation}
Donc même en prenant vraiment n'importe quoi pour \( I_N\), nous obtenons de bonnes approximations pour \( I_i\) avec les petits \( i\). Même avec \( I_{20}=1000\) (qui est complètement faux), nous trouvons énormément de chiffres significatifs corrects pour \( I_1\).

%---------------------------------------------------------------------------------------------------------------------------
\subsection{Comment choisir et penser le \texorpdfstring{$ K$}{K} ?}
%---------------------------------------------------------------------------------------------------------------------------

La formule \eqref{EqDefAABSOLU} contient une formule qui ressemble étrangement à la dérivée. La stabilité d'un problème est très liée à la dérivée de $F$. La stabilité et la dérivée ne sont pas les mêmes choses, mais il n'est pas mauvais de penser au $K$ de la stabilité comme la dérivée. Ou plus précisément : le supremum de la dérivée.

Un fil conducteur du lemme~\ref{LemITCxqyS} et des exemples~\ref{ExRZrOeoi},~\ref{PIluknK} est que l'on a un $K$ qui fonctionne lorsque la dérivée est bornée sur l'intervalle $\mathopen] d_0-\eta , d_0+\eta \mathclose[$. Dans le cas où ce supremum existe, le prendre en guise de $K$ fonctionne souvent.

Il faut cependant parfois faire acte d'imagination. La fonction $x\mapsto| x |$ n'est pas dérivable en $0$. Il n'empêche que $K=1$ fait fonctionner la définition de la stabilité. Remarquez que $K=1$ est le supremum de la dérivée là où elle existe.

À partir du moment où c'est clair que le $K$ est le supremum de la dérivée, on comprend pourquoi c'est le gradient qui arrive dans le corolaire~\ref{CorConditionnementNormeNabla}. En effet, le gradient indique la direction de plus grande pente. C'est donc bien dans cette direction qu'il faut chercher la «plus grande dérivée».

\begin{proposition}
	Pour le problème stable $x=x(d)$ avec $x\in C^1(\eR^n,\eR)$, on a
	\begin{equation}
		K_{abs}(d)\sim\| dx_d \|
	\end{equation}
	où \( dx_d\) désigne la différentielle de $x$ en $d$ et la norme est la norme opérateur.
\end{proposition}

%+++++++++++++++++++++++++++++++++++++++++++++++++++++++++++++++++++++++++++++++++++++++++++++++++++++++++++++++++++++++++++
\section{Un peu de points fixes}
%+++++++++++++++++++++++++++++++++++++++++++++++++++++++++++++++++++++++++++++++++++++++++++++++++++++++++++++++++++++++++++
\label{SECooWUVTooMhmvaW}

%---------------------------------------------------------------------------------------------------------------------------
\subsection{Choix de la fonction à point fixe}
%---------------------------------------------------------------------------------------------------------------------------

Pour l'équation \( f(x)=0\), il existe une infinité de fonctions \( g\) pour lesquelles l'équation est équivalente à \( x=g(x)\).

Exemple : \( f(x)=x^2-2-\ln(x)\), nous pouvons faire
\begin{enumerate}
    \item
        \( x=x^2-2-\ln(x)+x\)
    \item
        Poser \( x^2=2+\ln(x)\) et donc
        \begin{subequations}
            \begin{align}
                x=-\sqrt{2+\ln(x)}\\
                x=\sqrt{2+\ln(x)}.
            \end{align}
        \end{subequations}
    \item
        Ou encore
        \begin{equation}
            x=\frac{ 2+\ln(x) }{ x }
        \end{equation}
        où nous savons déjà que \( x\neq 0\) parce que \( x=0\) n'est pas dans le domaine de \( f\).
    \item
        Ou par l'exponentielle :
        \begin{equation}
            x= e^{x^2-2}.
        \end{equation}
\end{enumerate}
Dans tous ces cas nous pouvons construire une suite \( (x_n)\) en posant un nombre arbitraire pour \( x_0\) et ensuite la récurrence
\begin{equation}
    x_{n+1}=g(x_n).
\end{equation}

Graphiquement, la solution de l'équation est l'intersection entre les courbes \( y=x\) et \( y=g(x)\). Un petit dessin pour montrer la convergence :

\begin{center}
   \input{auto/pictures_tex/Fig_UEGEooHEDIJVPn.pstricks}
\end{center}

Attention : cette méthode ne converge pas toujours. Parfois elle converge de façon monotone, et parfois pas. Le choix de la fonction \( g\) qui fait \( x=g(x)\) peut énormément changer la vitesse de convergence.

\begin{theorem}[Condition suffisante pour existence d'un point fixe]
    Une fonction continue \( f\colon \mathopen[ a , b \mathclose]\to \mathopen[ a , b \mathclose]\) admet au moins un point fixe dans \( \mathopen[ a , b \mathclose]\).
\end{theorem}

\begin{theorem}[Condition suffisante pour l'unicité]
    Soit \( f\) continue sur \( \mathopen[ a , b \mathclose]\) avec \( g(x)\in\mathopen[ a , b \mathclose]\) pour tout \( x\in\mathopen[ a , b \mathclose]\). Supposons qu'il existe \( 0<k<1\) tel que pour tout \( x\in\mathopen[ a , b \mathclose]\) nous ayons \( | g'(x) |\leq k\) alors
    \begin{enumerate}
        \item
            La fonction \( g\) possède un unique point fixe dans \( \mathopen[ a , b \mathclose]\).
        \item
            Pour tout \( x_0\in\mathopen[ a , b \mathclose]\), tous les termes de la suite \( x_{n+1}=g(x_n)\) sont dans \( \mathopen[ a , b \mathclose]\).
        \item
            Ladite suite \( (x_n)\) converge vers le point fixe.
    \end{enumerate}
\end{theorem}

\begin{theorem}
    Soit \( f\) continue sur \( \mathopen[ a , b \mathclose]\) avec \( g(x)\in\mathopen[ a , b \mathclose]\) pour tout \( x\in\mathopen[ a , b \mathclose]\). Supposons
    \begin{enumerate}
        \item
    qu'il existe \( 0<k<1\) tel que pour tout \( x\in\mathopen[ a , b \mathclose]\) nous ayons \( | g'(x) |\leq k\) et
\item
    \( g\) est \( p\) fois dérivable sur \( \mathopen[ a , b \mathclose]\).
\item
    \( g'(\alpha)=g''(\alpha)=\ldots=g^{(p-1)}(\alpha)\) et \( g^{(p)}(\alpha)\neq 0\) où \( \alpha\) est l'unique point fixe.
    \end{enumerate}
    Alors la suite \( (x_n)\) converge avec un ordre \( p\).
\end{theorem}



\begin{example}
    Nous reprenons
    \begin{equation}
        f(x)=x^2-2-\ln(x).
    \end{equation}
    Et nous voulons résoudre \( f(x)=0\). Graphiquement c'est l'intersection entre \( y=x^2-2\) et \( y=\ln(x)\). Il est vite tracé de savoir qu'il y a deux solutions  : \( \alpha_1\in\mathopen[ 0 , 1 \mathclose]\) et \( \alpha_2\in\mathopen[ \sqrt{2} , 2 \mathclose]\).

    Déjà un petit problème : l'intervalle \( \mathopen[ 0 , 1 \mathclose]\) ne va pas parce que \( f\) n'y est pas continue. Un petit raffinement d'analyse nous fournit \( \alpha_1\in\mathopen[ e^{-2} , 1 \mathclose]\).

    Nous avons au moins les fonctions de points fixes suivantes :
    \begin{subequations}
        \begin{align}
            g_1(x)=\sqrt{ 2+\ln(x) }\\
            g_2(x)=e^{x^2-2}.
        \end{align}
    \end{subequations}
    Pour la première, il y avait un \( \pm\) qui a été négligé parce que nous savons que les deux solutions cherchées sont positives.
    Travaillons avec la première. D'abord
    \begin{equation}
        g'_1(x)=\frac{ 1 }{ 2x\sqrt{ 2-\ln(x) } }.
    \end{equation}
    Nous avons \( \lim_{x\to e^{-2}} g'_2(x)=+\infty\). Il ne sera donc pas possible de trouver \( 0<k<1\) tel que \( | g'(x) |\leq k\). Tentons quand même la méthode :
    \begin{equation}
        x_0=0.5
    \end{equation}
    Il se fait que cela est plus proche de \( \alpha_1\) que de \( \alpha_2\). Mais en réalité la suite converge vers \( \alpha_2\).

    Passons à la seconde méthode.
    \begin{equation}
        g'_2(x)=2xe^{x^2-2}.
    \end{equation}
    Sur l'intervalle \( \mathopen[ e^{-2} , 1 \mathclose]\), \( g'_2\) est croissante et prend toutes ses valeurs dans \( \mathopen[ e^{-2} , 1 \mathclose]\). Nous pouvons prouver que
    \begin{equation}
        | g'_2(x) |\leq 2e^{-1}<1.
    \end{equation}
    Donc poser \( k=2e^{-1}\) fait fonctionner la proposition. Donc quel que soit le \( x_0\) pris dans cet intervalle, nous aurons une suite convergente vers un point fixe à l'intérieur de l'intervalle. C'est-à-dire convergente vers \( \alpha_1\).

    Cela est un exemple de problème pour lequel changer de fonction \( g\) change réellement la vie.
\end{example}

%---------------------------------------------------------------------------------------------------------------------------
\subsection{Convergence quadratique}
%---------------------------------------------------------------------------------------------------------------------------

\begin{definition}      \label{DEFooSUTRooAcXXjj}
    Une suite \( (x_n)\) a une convergence \defe{quadratique}{convergence!quadratique} vers \( \alpha\) si elle converge vers \( \alpha\) et s'il existe un \( C\) tel que pour tout \( n\) nous ayons
    \begin{equation}    \label{EQooMWBIooLGashp}
        \| x_{n+1}-\alpha \|\leq C\| x_n-\alpha \|^2.
    \end{equation}
\end{definition}
Il est bien entendu possible de parler de convergence quadratique si la relation \eqref{EQooMWBIooLGashp} a lieu seulement à partir d'un certain indice.

Le lemme suivant donne l'importance du choix de point de départ lorsqu'on utilise une méthode itérative dont la convergence est quadratique.
\begin{lemma}       \label{LEMooLQMAooICcmrn}
    Soit une suite \( x_n\to \alpha\) de convergence quadratique. Si \( \| x_0-\alpha \|\leq r\) alors
    \begin{equation}        \label{EQooVYRIooTxetPn}
        \| x_n-\alpha \|\leq \frac{1}{ C }(Cr)^{2^n}
    \end{equation}
\end{lemma}

\begin{proof}
    Nous pourrions directement prouver la formule \eqref{EQooVYRIooTxetPn} par récurrence, mais nous allons la reconstruire un peu. Nous cherchons
    \begin{equation}        \label{EQooZCZVooHqpkqs}
        \| x_n-\alpha \|\leq C^{k(n)}r^{2^n}.
    \end{equation}
    Nous avons les inégalités
    \begin{subequations}
        \begin{align}
            \| x_{n+1}-\alpha \|&\leq C\| x_n-\alpha \|^2\\
            &\leq CC^{2k(n)}r^{2^{n+1}}\\
            &=C^{2k(n)+1}r^{2^{n+1}}
        \end{align}
    \end{subequations}
    d'où nous voyons que la fonction \( k\) doit vérifier
    \begin{subequations}
        \begin{numcases}{}
            k(0)=0\\
            k(n+1)=2k(n)+1
        \end{numcases}
    \end{subequations}
    La première équation est l'hypothèse \( \| x_0-\alpha \|\leq r\) comparée à la formule \eqref{EQooZCZVooHqpkqs}. Il est vite vérifié que \( k(n)=2^n-1\). D'où le résultat.
\end{proof}

Si le point de départ est choisi de façon à avoir \( Cr<1\) alors nous avons là un très bon majorant parce qu'il s'agit d'un majorant convergeant très rapidement vers zéro. Si au contraire \( Cr>1\) alors ce majorant ne sert à rien.

\begin{normaltext}
    Le fait d'avoir une convergence quadratique signifie que le nombre décimales correctes double (environ) à chaque itération, dans n'importe quelle base. En effet supposons que \( x_n\) ait \( k\) décimales correctes; cela signifie que \( | x_n-\alpha |\sim 10^{-k}\). Donc
    \begin{equation}
        | x_{n+1}-\alpha |\lessapprox M 10^{-2k}.
    \end{equation}
    Cela est le double de décimales correctes de \( | x_n-\alpha |\), moins l'ordre de grandeur de \( M\).

    Pour la méthode de bisection, le nombre de décimales augmente de \( 1\) à chaque itération, mais seulement en base \( 2\). En base \( 10\), de façon générique\footnote{C'est-à-dire sauf coup de malchance ou coup de chance.} il faut entre \( 3\) et \( 4\) itérations pour avoir une décimale de plus.
\end{normaltext}

\begin{normaltext}[Condition d'arrêt\cite{ooGYJXooIWExXK}]       \label{NTooVXLXooXlAGEq}
    D'autre part, lorsqu'une méthode a une convergence quadratique, nous avons un test d'arrêt. Pour ce voir, nous avons la limite
    \begin{equation}
        \lim_{n\to \infty} \frac{ | x_{n+1}-\alpha | }{ | x_n-\alpha | }\leq\lim_{n\to \infty} \frac{ C| x_n-\alpha |^2 }{ | x_n-\alpha | }=0.
    \end{equation}
    Cette limite est alors également valable sans les valeurs absolues et si nous soustrayons \( x_n-\alpha\) au numérateur, la limite devient \( -1\) :
    \begin{equation}
        -1=\lim_{n\to \infty} \frac{ x_{n+1}-x_n }{ x_n-\alpha }.
    \end{equation}
    Ou encore
    \begin{equation}
        \lim_{n\to \infty} \frac{ x_n-x_{n+1} }{ x_n-\alpha }=1.
    \end{equation}
    Cela a pour conséquence que si \( n\) est grand,
    \begin{enumerate}
        \item
            \( x_{n+1}\) a le même ordre de grandeur que \( x_n-\alpha\).
        \item
            \( x_n-x_{n+1}\) et \( x_n-\alpha\) ont le même signe.
    \end{enumerate}
    Donc si nous voulons une approximation de \( \alpha\) avec une erreur \( \epsilon\), il suffit d'arrêter le calcul lorsque \( | x_{n+1}-x_n |\leq \epsilon\). Et ce faisant nous savons de plus si l'approximation est par excès ou par défaut.
\end{normaltext}

%---------------------------------------------------------------------------------------------------------------------------
\subsection{Convergence}
%---------------------------------------------------------------------------------------------------------------------------

\begin{proposition}[Convergence d'une méthode de point fixe\cite{ooGYJXooIWExXK}]      \label{PROPooRPHKooLnPCVJ}
    Soit \( g\colon \eR\to \eR\) de classe \( C^1\) et \( \alpha\) un point fixe attractif\footnote{Définition~\ref{DEFooTMZUooMoBDGC}.} de \( g\). Soit \( k\) tel que \( | g'(\alpha) |<k<1\) et \( \delta\) tel que \( \| g' \|_{B(\alpha,\delta)}<k\).

    Alors
    \begin{enumerate}
        \item       \label{ITEMooOQKMooTRSvUo}
            La fonction \( g\) est \( k\)-contractante\footnote{Définition~\ref{DEFooRSLCooAsWisu}} sur \( B(\alpha,\delta)\).
        \item       \label{ITEMooFTAQooPBsBcR}
            Nous avons \( g\big( B(\alpha,\delta) \big)\subset B(\alpha,\delta)\).
        \item       \label{ITEMooFSOAooKlcxih}
            Pour tout \( x_0\in B(\alpha,\delta)\) la suite \( x_{n+1}=g(x_n)\) converge vers \( \alpha\) et
            \begin{equation}
                | x_n-\alpha |\leq | x_0-\alpha |k^n.
            \end{equation}
    \end{enumerate}
    Si de plus \( g'(\alpha)=0\) et \( g\) est de classe \( C^2\) alors nous avons convergence quadratique (définition~\ref{DEFooSUTRooAcXXjj}).
\end{proposition}

\begin{proof}
    Vu que \( \alpha\) est un point fixe attractif de \( g\) nous pouvons considérer un \( k\) tel que \( | g'(\alpha) |<k<1\). Et comme \( g\) est de classe \( C^1\), la fonction \( g'\) est continue et donc bornée sur toute boule du type \( \overline{ B(\alpha,\delta) }\). Soit \( \delta\) le plus grand nombre tel que \( \| g' \|_{\overline{ B(\alpha,\delta) }}\leq k\). Nous notons \( I=\overline{ B(\alpha,\delta) } \) pour cette valeur de \( \delta\).

    Pour tout \( x\in I\) nous avons, en utilisant le théorème des accroissements finis~\ref{ThoAccFinis}\ref{ITEMooXRQKooDBFpdQ} :
    \begin{subequations}        \label{SUBEQooYXLHooSCnnRA}
        \begin{align}
            | g(x)-\alpha |&=| g(x)-g(\alpha) |\\
            &\leq\sup_{t\in I}| g'(t) | |x-\alpha |\\
            &\leq k| x-\alpha |\\
            &<\delta
        \end{align}
    \end{subequations}
    parce que \( k<1\) et \(| x-\alpha |\leq \delta\). Par conséquent \( g(x)\in B(\alpha,\delta)\). Cela prouve le point~\ref{ITEMooFTAQooPBsBcR}. Pour le point~\ref{ITEMooOQKMooTRSvUo}, soient \( x,y\in B(\alpha,\delta)\) et
    \begin{equation}
        | g(x)-g(y) |\leq \sup_{a\in I}| g'(a) | |x-y |\leq k| x-y |.
    \end{equation}
    Pout le point~\ref{ITEMooFSOAooKlcxih} nous avons \( | g(x_n)-\alpha |\leq k| x_n-\alpha |\), c'est-à-dire
    \begin{equation}
        | x_{n+1}-\alpha |\leq k| x_n-\alpha |.
    \end{equation}
    Le résultat annoncé s'obtient par récurrence sur \( n\).

    En ce qui concerne la convergence quadratique, c'est du Taylor (proposition~\ref{PropDevTaylorPol}). Développons \( g(x_n)\) autour de \( g(\alpha)\) :
    \begin{equation}
        g(x_n)=g(\alpha)+g'(\alpha)(x_n-\alpha)+\frac{ 1 }{2}(x_n-\alpha)^2\epsilon(x_n-\alpha)
    \end{equation}
    avec \( \lim_{t\to 0} \epsilon(t)=0\). En posant \( C=\frac{ 1 }{2}\sup_{t<\delta}| \epsilon(t) | \) nous avons $| g(x_n)-g(\alpha) |\leq C|x_n-\alpha  |^2$, c'est-à-dire
    \begin{equation}
        | x_{n+1}-\alpha |\leq C| x_n-\alpha |^2.
    \end{equation}
\end{proof}

Ce corolaire est une paraphrase de la proposition~\ref{PROPooRPHKooLnPCVJ}. Il en retient seulement les points intéressants en pratique.

\begin{corollary}     \label{CORooHKZCooEXRzcW}
    Soit \( \alpha\) une solution de l'équation \( x=g(x)\), avec \( g\) continue sur un voisinage de \( \alpha\) et dérivable dans l'intérieur. Nous supposons que
    \begin{equation}
        | g'(\alpha) |<1.
    \end{equation}
    Alors il existe un rayon \( \delta\) tel que si \( x_0\in B(\alpha,\delta)\), la suite \( (x_n)\) converge vers \( \alpha\).
\end{corollary}

Certes cette proposition demande moins d'hypothèses, mais en réalité, il ne donne pas de vrais moyens de choisir un point de départ \( x_0\). Avec les deux théorèmes précédents, nous pouvions prendre \( x_0\) n'importe où dans \( \mathopen[ a , b \mathclose]\). Le fait est que pour choisir \( x_0\) nous pouvons tracer et donner à la main un \( x_0\) proche de ce qui semble être \( \alpha\). Si ça ne converge pas, il faut donner un \(x_0\) plus proche. La proposition nous assure que si nous jouons bien à choisir \( x_0\) très proche, la suite finira par converger.

Notons que le corolaire~\ref{CORooHKZCooEXRzcW} a encore l'inconvénient de demander de calculer \( g'(\alpha)\) alors que \( \alpha\) est inconnu. La résolution de l'inéquation \( | g'(x) |<1\) nous donne un certain nombre d'intervalles dans \( \eR\).

Soient \( I_n\) les intervalles solutions de l'inéquation.  Si \( \alpha\in I_n\) alors la méthode converge. Sinon, c'est pas garantit. En tout cas nous ne devons pas savoir réellement \( \alpha\) pour appliquer le théorème. Il suffit de savoir que \( \alpha\) est dans un des \( I_n\).

%+++++++++++++++++++++++++++++++++++++++++++++++++++++++++++++++++++++++++++++++++++++++++++++++++++++++++++++++++++++++++++
\section{Méthode de Newton}
%+++++++++++++++++++++++++++++++++++++++++++++++++++++++++++++++++++++++++++++++++++++++++++++++++++++++++++++++++++++++++++
\label{SECooIKXNooACLljs}

L'objectif de la méthode de Newton est d'évaluer une racine \( \alpha\) de l'équation \( f(x)=0\) lorsque nous avons déjà une approximation \( x_0\) de la racine \( \alpha\).

C'est la méthode de Newton qui est à l'origine de la suite de la proposition \ref{PROPooSTQXooHlIGVf} donnant une suite dans \( \eQ\) qui converge vers \( \sqrt{ A} \).

\begin{definition}      \label{DEFooXSOQooAnWqKM}
    Le nombre \( \alpha\) est une \defe{racine simple}{racine!simple} de l'équation \( f(x)=0\) si \( f(\alpha)=0\) et \( f'(\alpha)\neq 0\). Le nombre \( \alpha\) est une \defe{racine multiple}{racine!multiple} d'ordre \( r\) de \( f(x)=0\) si\index{multiplicité!racine de \( f(x)=0\)}
    \begin{equation}
        f(\alpha)=f'(\alpha)=\ldots=f^{(r-1)(\alpha)}=0
    \end{equation}
    et \( f^{(r)}(\alpha)\neq 0\).
\end{definition}

\begin{example}
    La fonction \( x\mapsto x^3\) en \( x=0\) est une racine d'ordre \( 3\).
\end{example}

%---------------------------------------------------------------------------------------------------------------------------
\subsection{«Justification» par la formule par Taylor}
%---------------------------------------------------------------------------------------------------------------------------

    Soit une fonction \( f\) continue et dérivable sur \( \mathopen[ a , b \mathclose]\). Soit \( \alpha\) une racine de \( f\) et \( x_n\) une de ses approximations.  Nous notons l'erreur \( \theta\) et nous avons \( \alpha=x_n+\theta\). Du coup nous avons \( f(x_n+\theta)=f(\alpha)=0\).

    Écrivons la série de Taylor du théorème~\ref{ThoTaylor} autour de \( x_n\) : il existe une fonction \( \epsilon\colon \eR\to \eR\) telle que \( \lim_{t\to 0} \epsilon(t)=0\) telle que
    \begin{equation}        \label{EQooOPUBooYaznay}
        f(\alpha)=f(x_n+\theta)=f(x_n)+\theta f'(x_n)+\frac{ \theta^2 }{ 2 }\epsilon(\theta).
    \end{equation}
    Nous isolons le \( \theta\) du terme d'ordre \( 1\) en nous souvenant que le membre de gauche est nul :
    \begin{equation}
        \theta=-\frac{ f(x_n)-\theta^2\epsilon(\theta) }{ f'(x_n) }
    \end{equation}
    Vu que \( \alpha=x_n+\theta\), nous pouvons écrire
    \begin{equation}
        \alpha=x_n-\frac{ f(x_n)+\theta^2\epsilon(\theta) }{ f'(x_n) }.
    \end{equation}
    Il est donc raisonnable de poser
    \begin{equation}
        x_{n+1}=x_n-\frac{ f(x_n) }{ f'(x_n) }
    \end{equation}
    en espérant que cela soit une meilleure approximation de \( \alpha\) que \( x_n\).

    En tout cas l'erreur sur \( x_{n+1}\) est
    \begin{equation}
        \alpha-x_{n+1}=x_n+\theta-x_n+\frac{ f(x_n)+\theta^2\epsilon(\theta) }{ f'(x_n) }=\theta+\frac{ f(x_n)+\theta^2\epsilon(\theta) }{ f'(n_n) },
    \end{equation}
    qui ne doit pas être fondamentalement plus grand que \( \theta\) dès que \( \theta\) est petit, surtout que si \( x_n\) est une approximation de \( \alpha\), nous pouvons espérer que \( f(x_n)\) soit également petit. Là où les choses peuvent déraper en grand, c'est si \( f'(x_n)\) est petit.

Cette méthode de Newton ne converge pas toujours. Le pire est lorsque par malheur il y a une bosse pas loin de la racine. Alors il y a un risque de tomber sur \( f'(x_{n+1})=0\) ou en tout cas très proche de zéro. Dans ce cas le point \( x_{n+2}\) est envoyé très loin.

%---------------------------------------------------------------------------------------------------------------------------
\subsection{«Justification» par points fixes}
%---------------------------------------------------------------------------------------------------------------------------
\label{SUBSECooIBLNooTujslO}

Nous savons que pour résoudre \( f(x)=0\) par une méthode de point fixe, il y a de nombreux choix possibles de fonctions \( g\) telles que \( g(x)=x\) donne la même solution que \( f(x)=0\). Soit \( \alpha\) une solution de \( f(x)=0\) et cherchons une fonction \( g\) de la forme
\begin{equation}        \label{EQooYVFIooJXnJXa}
    g(x)=x-kf(x).
\end{equation}
Nous savons par la proposition~\ref{PROPooRPHKooLnPCVJ} que la fonction \( g\) donne une convergence quadratique lorsque \( g'(\alpha)=0\). Pour la forme \eqref{EQooYVFIooJXnJXa} nous avons \( g'(\alpha)=1-kf'(\alpha)\), ce qui nous donne l'idée de poser \( k=\frac{1}{ f'(\alpha) }\).

Le fait est que \( f'(\alpha)\) n'est pas connu, mais nous pouvons l'approximer par \( f'(x)\) lorsque \( x\) est proche de \( \alpha\). D'où l'idée de considérer la fonction
\begin{equation}
    g(x)=x-\frac{ f(x) }{ f'(x) },
\end{equation}
et donc la suite \( x_{n+1}=g(x_n)\) c'est-à-dire
\begin{equation}
    x_{n+1}=x-\frac{ f(x_n) }{ f'(x_n) }.
\end{equation}
Dès que \( x_n\) est proche de \( \alpha\), sous l'hypothèse (raisonnable par continuité) que \( f'(x_n)\) soit proche de \( f'(\alpha)\), la méthode devrait donner une convergence quadratique.

\begin{remark}
    Cette justification par points fixes n'est pas vraiment différente de celle par Taylor parce que Taylor est utilisé dans la preuve de la proposition~\ref{PROPooRPHKooLnPCVJ}.
\end{remark}

\begin{definition}[Méthode de Newton]
    La \defe{méthode de Newton}{Méthode!de Newton} pour la fonction \( f\) est la suite définie par récurrence
    \begin{equation}
        x_{n+1}=x_n-\frac{ f(x_n)  }{ f'(x_n) }.
    \end{equation}
    Cette définition ne précise pas la valeur de \( x_0\), ni de condition d'arrêt.
\end{definition}

%---------------------------------------------------------------------------------------------------------------------------
\subsection{Convergence de la méthode de Newton}
%---------------------------------------------------------------------------------------------------------------------------

\begin{theorem}[Convergence quadratique de la méthode de Newton\cite{ooGYJXooIWExXK}]       \label{THOooDOVSooWsAFkx}
    Soit \( f\) une fonction continue vérifiant \( f(\alpha)=0\) et \( f'(\alpha)\neq 0\). Nous considérons la fonction
    \begin{equation}
        g(x)=x-\frac{ f(x) }{ f'(x) }
    \end{equation}
    que nous supposons être de classe \( C^2\).

    Si \( C\) est une majoration de \( \| g'' \|\) sur un intervalle contenant \( \alpha\), alors en posant \( \delta=1/C\) nous avons
    \begin{enumerate}
        \item       \label{ITEMooVXSKooWCVWQc}
            La boule \( B(\alpha,\delta)\) est préservée par \( g\) : \( g\big( B(\alpha,\delta) \big)\subset B(\alpha,\delta)\).
        \item       \label{ITEMooZPSXooGgbfhG}
            Pour tout \( x_0\in B(\alpha,\delta)\) nous avons convergence quadratique vers \( \alpha\) de la suite définie par \( x_{n+1}=g(x_n)\).
        \item       \label{ITEMooZCXZooCjeWPl}
            Nous avons l'estimation
            \begin{equation}        \label{EQooFAIPooDpoNWK}
                | x_n-\alpha |\leq \frac{1}{ C }\big( C| x_0-\alpha | \big)^{2^n}
            \end{equation}
            où \( C\) est la constante de la définition de convergence quadratique.
    \end{enumerate}
\end{theorem}

\begin{proof}
    Nous commençons par calculer la dérivée de \( g\) :
    \begin{equation}
        g'(x)=-\frac{ f(x)f''(x) }{ f'(x)^2 },
    \end{equation}
    d'où nous déduisons que \( g'(\alpha)=0\). Ensuite nous utilisons abondamment la formule des accroissements finis (théorème~\ref{val_medio_2}) en commençant par
    \begin{equation}        \label{EQooZITHooEbGVKG}
        | g(t)-g(\alpha) |\leq \| g' \|_{\mathopen[ t , \alpha \mathclose]}| t-\alpha |
    \end{equation}
    où par \( \| f \|_A\) nous entendons la norme uniforme de \( f\) sur \( A\), c'est-à-dire \( \| f \|_A=\sup_{x\in A}\| f(x) \|\). Note : nous écrivons \( \mathopen[ t , \alpha \mathclose]\), mais ça pourrait être \( \mathopen[ \alpha , t \mathclose]\).

    Si \( x\in\mathopen[ t , \alpha \mathclose]\) alors
    \begin{subequations}
        \begin{align}
            | g'(x) |&=| g'(x)-g'(\alpha) |\\
            &\leq \| g'' \|_{\mathopen[ x , \alpha \mathclose]}| x-\alpha |\\
            &\leq \| g'' \|_{\mathopen[ x , \alpha \mathclose]}| t-\alpha |\\
            &\leq \| g'' \|_{\mathopen[ t , \alpha \mathclose]}| t-\alpha |.
        \end{align}
    \end{subequations}
    En particulier, \( \| g' \|_{\mathopen[ t , \alpha \mathclose]}\leq \| g'' \|_{\mathopen[ t , \alpha \mathclose]}| t-\alpha |\), et nous pouvons continuer les majorations \eqref{EQooZITHooEbGVKG} :
    \begin{equation}
        | g(t)-g(\alpha) |\leq \| g'' \|_{\mathopen[ t , \alpha \mathclose]}| t-\alpha |^2.
    \end{equation}

    La fonction \( g\) étant de classe \( C^2\), la dérivée seconde \( g''\) est bornée (nous supposons déjà travailler sur un compact contenant \( \alpha\)). Soit \( C\) une borne. Nous sommes en mesure de prouver le point~\ref{ITEMooVXSKooWCVWQc} avec \( \delta=1/C\). En effet si \( t\in B(\alpha,1/C)\) alors
    \begin{equation}
        | g(t)-\alpha |=| g(t)-g(\alpha) |\leq C| t-\alpha |^2\leq C\frac{1}{ C^2 }=\frac{1}{ C },
    \end{equation}
    ce qui prouve que \( g(t)\in B(\alpha,1/C)\).

    Le point~\ref{ITEMooZPSXooGgbfhG} se prouve de la même manière : si \( x_n\in B(\alpha,1/C)\) alors
    \begin{equation}
        | x_{n+1}-\alpha |=| g(x_n)-g(\alpha) |\leq C| x_n-\alpha |^2,
    \end{equation}
    ce qui est bien la convergence quadratique.

    La majoration du point~\ref{ITEMooZCXZooCjeWPl} s'obtient par récurrence sur \( n\). Pour \( n=0\), la relation \eqref{EQooFAIPooDpoNWK} devient \( | x_0-\alpha |\leq | x_0-\alpha |\) qui est vraie. Ensuite par la convergence quadratique et la récurrence,
    \begin{equation}
        | x_{n+1}-\alpha |\leq C| x_n-\alpha |^2\leq C\big[  \frac{1}{ C }(C| x_0-\alpha |)^{2^n}  \big]^2=\frac{1}{ C }\big[ M| x_0-\alpha | \big]^{2^{n+1}}.
    \end{equation}
\end{proof}

\begin{normaltext}
    Dans le cas pratiques, nous commençons souvent par résoudre l'équation \( f(x)=0\) par dichotomie. Au moment où nous sommes assez proche de la solution nous commençons Newton.

    La raison est que la dichotomie fonctionne toujours : nous allons toujours nous approcher de la solution. Si par contre le point de départ est mal choisit, la méthode de Newton peut envoyer n'importe où, y compris très loin de la solution.
\end{normaltext}

La proposition suivante nous indique que dans le cas d'une fonction convexe, le choix de point de départ de la méthode de Newton n'est pas tellement crucial parce que il sont tous bons. De plus la convergence se faisant de façon décroissante (si on part de la droite), nous savons que le résultat sera une approximation par excès de \( \alpha\).
\begin{proposition}[Newton dans le cas convexe]     \label{PROPooVTSAooAtSLeI}
    Soit \( f\) de classe \( C^2\) et une racine \( \alpha\) telle que \( f'(\alpha)>0\). Soit \( b>\alpha \) tel que \( f\) soit convexe sur \( \mathopen[ \alpha , b \mathclose]\).

    Alors pour tout \( x_0\in\mathopen[ \alpha , b \mathclose]\) la suite de la méthode de Newton est
    \begin{enumerate}
        \item
            décroissante
        \item
            reste dans \( \mathopen[ \alpha , b \mathclose]\)
        \item
            converge vers \( \alpha\).
    \end{enumerate}
\end{proposition}
\index{méthode!Newton!cas convexe}
\index{convexité!méthode de Newton}

\begin{proof}
    Nous savons par la proposition~\ref{PropYKwTDPX}\ref{ITEMooLLSIooFwkxtV} que la fonction \( f'\) est croissante, et par hypothèse \( f'(\alpha)>0\), donc sur \( \mathopen[ \alpha , b \mathclose]\) nous avons \( f'>0\). Par conséquent, nous avons aussi \( f>0\) sur \( \mathopen[ \alpha , b \mathclose]\).

    Le graphe de \( f\) est au dessus de la tangente de \( f\) en \( x=x_n\) (proposition~\ref{PROPooQPOSooDZlUAJ}). Si nous nommons \( t_x\) la fonction qui donne la tangente en \( x\) nous avons \( t_{x_n}(\alpha)<0\) parce que \( f(\alpha)=0\). Par conséquent
    \begin{equation}
        t_{x_n}(x)=0
    \end{equation}
    pour \( \alpha<x<x_n\). Cela prouve que \( x_{n+1}\in\mathopen[ \alpha , b \mathclose]\), et que \( (x_n)\) est une suite décroissante

    Étant donné que \( (x_n)\) est une suite décroissante dans le compact \( \mathopen[ \alpha , b \mathclose]\), elle est convergente. Notons \( \beta\) sa limite. Nous avons la relation de récurrence
    \begin{equation}
        x_{n+1}=x_n-\frac{ f(x_n) }{ f'(x_n) }.
    \end{equation}
    En passant à la limite \( n\to \infty\) nous avons l'équation
    \begin{equation}
        \beta=\beta-\frac{ f(\beta) }{ f'(\beta) }.
    \end{equation}
    Vu que \( f(x)>0\) sur \( \mathopen] \alpha , b \mathclose]\) nous avons automatiquement \( \beta=\alpha\).
\end{proof}

%---------------------------------------------------------------------------------------------------------------------------
\subsection{Formalisation de l'algorithme}
%---------------------------------------------------------------------------------------------------------------------------

La méthode de Newton consiste a exprimer la solution $x$ de $f(x)=0$ avec $f\in C^1(\eR)$ comme limite d'une suite $\{x_n\}_{n\in\eN}$ définie par récurrence par la formule
\begin{equation}
	x_{n+1}=x_n-\frac{f(x_n)}{f'(x_n)}.
\end{equation}
où $x_0$ est arbitraire.

Si on veut exprimer cela en termes d'algorithmes, nous disons que l'algorithme de Newton est donné par la suite de problèmes
\begin{equation}        \label{EqFPourNewtonUn}
	F_n(x_{n+1},x_n,f)=x_{n+1}-x_n+\frac{ f(x_n) }{ f'(x_n) }.
\end{equation}
La donnée du problème est la fonction $f$, et rien que elle.

Plus précisément, une fois que la fonction $f$ est donnée, il existe une infinité de problèmes : pour chaque $a\in \eR$ nous avons le problème
\begin{equation}
	G_a(x_n,f)=x-a+\frac{ f(a) }{ f'(a) }.
\end{equation}
La méthode de Newton consiste à sélectionner une partie de ces problèmes de la façon suivante :
\begin{subequations}
	\begin{numcases}{}
		F_0 = G_{x_0}\\
		F_n = G_{x_n}.
	\end{numcases}
\end{subequations}
Le problème $F_0$ fournit un nombre $x_1$ qui nous permet de sélectionner le problème $G_{x_1}$ qui va fournir le nombre $x_2$, etc.

Au moment de calculer le conditionnement de $F_n$, nous ne devons pas voir $x_{n-1}$ comme fonction de $x_0$ et de la donnée $f$. Il ne faut donc pas dériver à travers les $x_n$.

\begin{proposition}
    Si une racine est multiple, alors l'ordre de convergence de la méthode de Newton est \( 1\).
\end{proposition}

Voici un algorithme possible :

\lstinputlisting{tex/frido/codeSnip_2.py}

Commentaires :
\begin{enumerate}
    \item
        Notons que dans un langage vraiment numérique comme Matlab, il faut passer \( f'\) en argument.
    \item
        Dans le \info{while} il faudrait mettre \( x_{n+1}-x_n\) (en valeur absolue), mais cette différence est aussi utilisée pour calculer \( x_{n+1}\) donc on la calcule une seule fois.
    \item
        Il faudrait faire une vérification sur \( f(x_n)\neq 0\). Il n'y a pas tellement de choix que de changer le point initial.
\end{enumerate}

%---------------------------------------------------------------------------------------------------------------------------
\subsection{Caractéristiques}
%---------------------------------------------------------------------------------------------------------------------------

L'algorithme de Newton a les caractéristiques suivantes :
\begin{enumerate}

	\item
		Pour résoudre le problème numéro $n$, il faut avoir résolu le problème numéro $n-1$.
	\item
		Aucune des solutions $x_n$ aux problèmes intermédiaires n'est une solution au problème de départ (à moins d'un coup de chance).
	\item
		Étant donné que la donnée du problème $F_n$ est la fonction $f$ de départ, nous avons $d_m=d_n=d$ pour tout $m$ et $n$.
\end{enumerate}

\begin{theorem}     \label{THOooMACHooLofCVu}
    Soit \( f\) continue sur un voisinage de \( \alpha\), racine simple. Alors il existe un voisinage de \( \alpha\) de rayon \( \sigma\) tel que pour tout \( x_0\) dans ce voisinage, la méthode converge vers \( \alpha\) avec ordre de convergence \( p=2\).
\end{theorem}

Donc dès qu'on a continuité autour de la solution recherchée, il suffit de prendre \( x_0\) assez proche pour que tout se passe bien. Cela se fait par localisation des racines, par exemples en traçant la fonction avec un bon niveau de zoom. Le fait est qu'on cherche disons \( 3\) décimales à la main (travail sur ordinateur et graphique) et Newton donne les \( 20\) décimales suivantes à la vitesse de la lumière.

%---------------------------------------------------------------------------------------------------------------------------
\subsection{Exemple de la racine carrée}
%---------------------------------------------------------------------------------------------------------------------------

Nous allons nous lancer dans un exemple : le cas de la racine carrée. Soit à calculer une approximation numérique de \( \sqrt{ 2 }\). Il s'agit d'une racine de la fonction \( f(x)=x^2+2\). La fonction de la méthode de Newton associée est :
\begin{equation}
    g(x)=x-\frac{ f(x) }{ f'(x) }=\frac{ x^2-2 }{ 2x }.
\end{equation}
Cherchons un intervalle autour de \( \sqrt{ 2 }\) sur lequel nous avons convergence de la méthode de Newton. Cela s'obtient grace à la proposition~\ref{PROPooRPHKooLnPCVJ} qui nous informe qu'il suffit de trouver un intervalle autour de \( \sqrt{ 2 }\) sur lequel \( | g'(x) |\leq 1\).

Nous avons
\begin{equation}
    g'(x)=\frac{ x^2-2 }{ 2x^2 },
\end{equation}
et nous cherchons à résoudre \( | g'(x) |\leq 1\). D'abord \( g'(x)=1\) n'a aucune solutions alors que \( g'(\sqrt{ 2 })=0\). Donc nous avons \(  g'(x) \leq 1\) pour tout \( x\in \eR^+\). Par contre l'équation \( g'(x)=-1\) a des solutions : \( x=\pm\sqrt{ 2/3 }\).

Nous avons donc convergence de la méthode de Newton pour \( x_0\) dans un intervalle de la forme
\begin{equation}
    \mathopen[ \sqrt{ 2/3 } , \sqrt{ 2 }+\ldots \mathclose]
\end{equation}
où les les trois points représentent l'expression qu'il faut pour que ce soit symétrique autour de \( \sqrt{ 2 }\). La valeur précise n'a pas tellement d'importance parce, vu que nous sommes en train de chercher \( \sqrt{ 2 }\), il est peu probable que nous ayons déjà en main une bonne approximation de nombres du type \( \sqrt{ 2/3 }\).

\begin{proposition}
La méthode de Newton pour la fonction \( f(x)=x^2-2\) converge vers \( \sqrt{ 2 }\) pour toute valeur de départ dans \( \mathopen] 0 , +\infty \mathclose[\).
\end{proposition}

\begin{proof}
    La fonction \( f(x)=x^2-2\) est convexe et \( f'(\sqrt{ 2 })=2\sqrt{ 2 }>0\). Donc la méthode converge vers \( \sqrt{ 2 }\) pour tout \( x_0\geq \sqrt{ 2 }\) par la proposition~\ref{PROPooVTSAooAtSLeI}.

    Si par contre \( x_0\in\mathopen] 0 , \sqrt{ 2 } \mathclose[\) nous avons
        \begin{equation}
            x_1=\frac{ x_0^2+2 }{ 2x_0 }.
        \end{equation}
    En posant \( h(x)=(x^2+2)/2x\) et en résolvant \( h'(x)=0\) nous trouvons \( x=\sqrt{ 2 }\). Et là, \( h(\sqrt{ 2 })=\sqrt{ 2 }\). Donc \( h(x)\) est toujours plus grand que \( \sqrt{ 2 }\) pour tout \( x\in\mathopen] 0 , \sqrt{ 2 } \mathclose[\).

    En d'autres termes, si \( x_0\in\mathopen] 0 , \sqrt{ 2 } \mathclose[\) alors \( x_1\geq \sqrt{ 2 }\) et nous retombons dans le premier cas.
\end{proof}

%---------------------------------------------------------------------------------------------------------------------------
\subsection{Si multiplicité}
%---------------------------------------------------------------------------------------------------------------------------

Supposons que \( \alpha\) soit de multiplicité \( r\) (définition~\ref{DEFooXSOQooAnWqKM}).

Cela se remarque en voyant que la méthode de Newton demande plutôt \( 20\) itérations que \( 5\). Le problème que cela pose est que chaque itération, les évaluations provoquent des erreurs. Donc moins d'itérations, c'est mieux.

Nous pouvons modifier la formule avec
\begin{equation}
    x_{n+1}=x_n-r\frac{ f(x_n) }{ f'(x_n) }.
\end{equation}
Il est possible de prouver que cette suite est à nouveau à convergence quadratique.

Ou alors on pose \( F(x)=f^{(r-1)}(x)\) et \( \alpha\) est une racine simple pour \( F\). Donc faire Newton pour \( F\) est à nouveau quadratique, tout en donnant la même solution parce que \( F(\alpha)=0\) et \( F'(\alpha)\neq 0\).

La seconde façon est bien parce que le théorème de localisation fonctionne~\ref{THOooMACHooLofCVu}

Et si \( r\) n'est pas connu ?

Il est toujours possible de faire \( r=2\) puis \( r=3\) et caetera jusqu'au moment où l'on remarque que le nombre d'itérations baisse un grand coup.

Mais ça demande beaucoup de calculs.  Le mieux est de changer de méthode.

%---------------------------------------------------------------------------------------------------------------------------
\subsection{Et la dérivée ?}
%---------------------------------------------------------------------------------------------------------------------------

Un des problèmes de la méthode de Newton est que l'on doit pouvoir calculer la dérivée. Typiquement, il faut savoir \( f\) de façon analytique. Si cela n'est pas possible, nous pouvons changer de méthode et utiliser la méthode des sécantes décrite en~\ref{SECooIUEUooVcHAoc}.


%---------------------------------------------------------------------------------------------------------------------------
\subsection{Méthode de Newton : le cas général}
%---------------------------------------------------------------------------------------------------------------------------

\begin{lemma}       \label{LemXdObnV}
    Soient \( A\) et \( B\) deux matrices inversibles telles que la matrice \( (A+\epsilon B)\) soit inversible pour tout \( \epsilon\) assez petit. Alors il existe une matrice \( X(\epsilon)\) telle que
    \begin{equation}
        (A+\epsilon B)^{-1}=(A^{-1}+\epsilon X)
    \end{equation}
    et telle que \( \lim_{\epsilon\to 0}X(\epsilon)=-A^{-1} BA^{-1}\).
\end{lemma}

\begin{proof}
    Le candidat matrice \( X\) est relativement simple à trouver en écrivant
    \begin{equation}
        (A+\epsilon B)(A^{-1}+\epsilon X)=\mtu+\epsilon AX+\epsilon BA^{-1}+\epsilon^2BX.
    \end{equation}
    En imposant que cela soit \( \mtu\), nous trouvons
    \begin{equation}
        X(\epsilon)=-(A+\epsilon B)^{-1} BA^{-1}.
    \end{equation}
    La matrice \( X(\epsilon)\) étant un inverse à droite de \( (A+\epsilon B)\), son déterminant est non nul et \( X\) est inversible. Par conséquent elle est également inversible au sens usuel. Le calcul de la limite est direct :
    \begin{equation}
        \lim_{\epsilon\to 0}-(A+\epsilon B)^{-1} BA^{-1}=A^{-1} BA^{-1}
    \end{equation}
    parce que l'inverse est une fonction continue sur \( \eM(n,\eR)\).
\end{proof}

\begin{remark}
    Un calcul naïf nous permet de trouver le même résultat de façon plus heuristique. En effet un développement usuel (dans \( \eR\)) est
    \begin{equation}
        \frac{1}{ a+\epsilon b }=\frac{1}{ a }-\frac{ \epsilon b }{ a^2 }+\ldots
    \end{equation}
    Si nous récrivons cela avec des matrices, nous écrivons (attention : passage heuristique!) :
    \begin{equation}
        (A+\epsilon B)^{-1}=A^{-1}-\epsilon A^{-1} BA^{-1}+\ldots
    \end{equation}
    Notons le choix de généraliser \( b/a^2\) par \( a^{-1} ba^{-1}\). Dans les réels les deux écritures sont équivalentes, mais pas dans les matrices.

    Étudions si \( A^{-1}-\epsilon A^{-1}BA^{-1}\) est bien un inverse à \( \epsilon^2\) près de \( (A+\epsilon B)\) :
    \begin{equation}
        (A+\epsilon B)(A^{-1}+\epsilon A^{-1} BA^{-1})=1-\epsilon BA^{-1}+\epsilon BA^{-1}-\epsilon^2BA^{-1}BA^{-1}=1-\epsilon^2BA^{-1} BA^{-1}.
    \end{equation}
    Par conséquent, à des termes en \( \epsilon^2\) près la matrice \( A^{-1}-\epsilon A^{-1}BA^{-1}\) est bien un inverse de \( A+\epsilon B\).
\end{remark}

\begin{theorem}[Méthode de Newton\cite{ChambertNewton}]\label{ThoHGpGwXk}
    Soit \( f\colon \eR^n\to \eR^n\) une application de classe \( C^2\) et un point \( a\in \eR^n\) tel que \( f(a)=0\). Nous supposons que \( df_a\) est inversible.

    Alors il existe un voisinage \( V\) de \( a\) tel que pour tout \( x_0\in V\) la suite définie par récurrence
    \begin{equation}
        x_{n+1}=x_n-(df_a)^{-1}\big( f(x_n) \big)
    \end{equation}
    converge vers \( a\). De plus la vitesse est quadratique au sens où il existe \( C>1\) tel que
    \begin{equation}        \label{EqtkiDXt}
        \| x_n-a \|\leq C^{-1-2^n}.
    \end{equation}
\end{theorem}
\index{Newton!méthode}
\index{méthode!Newton}
\index{formule!Taylor!utilisation}
\index{convergence!rapidité}
\index{suite!définie par itération}

\begin{proof}
    Étant donné que \( df_a\) est inversible et que \( df\) est continue, l'application \( df_x\) est continue\footnote{Nous pouvons voir \( df\) comme l'application qui à \( x\) fait correspondre la matrice \( df_x\in\eM(n,\eR)\). Cette application étant continue et la non inversibilité d'une matrice étant donnée par l'annulation du déterminant, les matrices inversibles forment un ouvert dans l'ensemble des matrices.} pour tout \( x\) dans un voisinage de \( a\). Nous prenons \( r>0\) tel que \( df_x\) est inversible pour tout \( x\in B(a,r)\).

    Nous considérons la fonction
    \begin{equation}
        \begin{aligned}
                F\colon B(a,r)&\to \eR^n \\
                x&\mapsto x-(df_x)^{-1}\big( f(x) \big).
            \end{aligned}
        \end{equation}
        Cela est une application \( C^1\). La clef est de montrer que l'application de \( F\) à un point \( a+h\) rapproche de \( a\) pourvu que \( h\) soit assez petit. Nous avons la formule suivante :
        \begin{equation}        \label{EqyDLQeE}
            F(a+h)-F(a)=h-\big( df_{a+h} \big)^{-1}\big( f(a+h) \big).
        \end{equation}
        Nous allons maintenant utiliser un développement de Taylor par rapport à \( h\) en suivant la formule \eqref{EquQtpoN}. Nous avons
        \begin{equation}
            f(a+h)=f(a)+df_a(h)+\| h \|^2\xi(h)
        \end{equation}
        où \( \xi\colon \eR^n\to \eR^n\) est une fonction qui tend vers une constante lorsque \( h\to 0\). Nous avons aussi
        \begin{equation}
            df_{a+h}=df_a+\| h \|\tau(h)
        \end{equation}
        où \( \tau\colon \eR^n\to \eM(n,\eR)\) est une application qui tend vers une constante lorsque \( h\to 0\). En ce qui concerne l'inverse nous utilisons le lemme\footnote{Pour l'inversibilité de \( \| h \|\tau(h)\), notons que \( df_a\) est inversible et que par hypothèse la somme \( df_a+\| h \|\tau(h)\) est inversible.}~\ref{LemXdObnV} :
        \begin{equation}
            \big( df_a+\| h \|\tau(h) \big)^{-1}=(df_a)^{-1}+\| h \|A(h)
        \end{equation}
        où \( A\) est une autre matrice fonction de \(h\) qui tend vers une constante lorsque \( h\) tend vers zéro. En substituant le tout dans \eqref{EqyDLQeE} nous trouvons
        \begin{equation}
            F(a+h)-F(a)=\| h \|^2(df_a)^{-1}\xi(h)+\| h \|\big( A(h)\circ df_a \big)(h)+\| h \|^3A(h)\xi(h).
        \end{equation}
        En ce qui concerne la norme nous utilisons le fait que si \( T\) est un opérateur, \( \| Tx \|\leq \| T \|\| x \|\). Nous trouvons
        \begin{subequations}
            \begin{align}
                \| F(a+h)-F(a) \|&\leq \| h \|^2\| (df_a)^{-1} \|\| \xi(h) \|+\| h \|^2\| A(h)\circ df_a \|+\| h \|^3\| A(h) \|\| \xi(h) \|\\
                &=\| h \|^2\alpha(h)
            \end{align}
        \end{subequations}
    pour une certaine fonction \( \alpha\colon \eR^n\to \eR\) qui tend vers une constante lorsque \( h\to 0\).

    En posant \( C=\lim_{h\to 0}\alpha(h) \) nous avons la majoration
    \begin{equation}        \label{EqSYiuYF}
        \| F(x)-a \|\leq C\| x-a \|^2.
    \end{equation}
    Nous pouvons également supposer que \( C>1\). Afin de prouver la vitesse de convergence \eqref{EqtkiDXt}, nous allons encore redéfinir \( r\) en demandant \( r<1/C^2\). De cette manière nous avons
    \begin{equation}
        \| x_0-a \|\leq \frac{1}{ C^2 }
    \end{equation}
    et la récurrence sur \( n\) est :
    \begin{equation}
        \| x_{n+1}-a \|=\| F(x_n)-a \|\leq C\| x_n-a \|^2\leq C\big( C^{-1-2^n} \big)^2=C^{-1-2^{n+1}}.
    \end{equation}
    Note : ce dernier calcul est le lemme~\ref{LEMooLQMAooICcmrn} appliqué à \( r=(1/C^2)\).
\end{proof}

\begin{remark}
    La valeur de la constante \( C\) a été fixée par l'équation \eqref{EqSYiuYF}. Certes nous pouvons toujours choisir \( C\) plus grand affin d'augmenter la vitesse de convergence, mais le point de départ \( x_0\) devant être dans une boule de taille \( 1/C^2\) autour de \( a\), demander \( C \) plus grand revient à demander un point de départ plus précis.
\end{remark}

\input{148_numerique}
% This is part of Mes notes de mathématique
% Copyright (C) 2010-2019
%   Laurent Claessens
% See the file LICENCE.txt for copying conditions.

%+++++++++++++++++++++++++++++++++++++++++++++++++++++++++++++++++++++++++++++++++++++++++++++++++++++++++++++++++++++++++++
\section{Conditionnement d'une matrice}
%+++++++++++++++++++++++++++++++++++++++++++++++++++++++++++++++++++++++++++++++++++++++++++++++++++++++++++++++++++++++++++
\label{SECooQGLRooZQzzsA}

Soit le système d'équations linéaires \( Au=b\) avec la matrice inversible \( A\) ainsi que le système perturbé \( (A+\Delta A)u'=(b+\Delta b)\). Nous notons \( \Delta u=u'-u\) et nous voudrions pouvoir dire des choses de l'erreur relative \( \frac{ \| \Delta u \| }{ \| u \| }\).

\begin{example}[\cite{ooLMMRooUXhOdx}]
    Soit la matrice
    \begin{equation}
        A=\begin{pmatrix}
            10    &   7    \\
            7    &   5
        \end{pmatrix}
    \end{equation}
    et \( b=\begin{pmatrix}
        32    \\
        23
    \end{pmatrix}\). La solution de \( Au=b\) est \( u=\begin{pmatrix}
        -1    \\
        6
    \end{pmatrix}\). Si nous conservons la même matrice mais nous considérons \( b=\begin{pmatrix}
        32.1    \\
        22.9
    \end{pmatrix}\). La solution devient \( u'=\begin{pmatrix}
        0.2    \\
        4.3
    \end{pmatrix}\)

    En norme \( \| . \|_{\infty}\) nous avons\footnote{La proposition~\ref{PropLJEJooMOWPNi}\ref{ItemABSGooQODmLNiii} montre que si nous voulions des estimations en norme \( \| . \|_2\), il y aurait au maximum un facteur \( \sqrt{2}\) par-ci par là.}
    \begin{equation}
        \frac{ \| \Delta b \| }{ \| b \| }=\frac{ 0.1 }{ 32 }=0.003125
    \end{equation}
    et
    \begin{equation}
        \frac{ \| \Delta u \| }{ \| u \| }=\frac{ 1.7 }{ 6 }=0.28.
    \end{equation}
    Cela montre environ amplification d'un facteur \( 100\) entre l'erreur sur \( b\) et l'erreur sur la solution.
\end{example}

\begin{definition}      \label{DEFooBKQWooJuoCGX}
    Le \defe{conditionnement}{conditionnement!d'une matrice inversible} de la matrice inversible \( A\in \GL(n,\eC)\) est le nombre positif
    \begin{equation}
        \Cond(A)=\| A \|\| A^{-1} \|.
    \end{equation}
\end{definition}

Cette dénomination sera justifié par le corolaire~\ref{CORooXKPWooJVHVvh} parce qu'il est évident que le conditionnement d'une matrice est lié au conditionnement du problème de résolution d'un système linéaire.

\begin{remark}
    Le conditionnement dépend de la norme choisie, mais cette dependence est contrôlée par la proposition~\ref{PropLJEJooMOWPNi} qui nous indique que si le conditionnement d'une matrice est grand dans une norme, il sera grand dans une autre norme.

    D'autre part, lorsque nous écrirons \( \| A \|\) nous supposerons toujours que \( \| . \|\) est une norme d'algèbre\footnote{Définition~\ref{DefJWRWQue}.} et donc que nous avons toujours
    \begin{equation}
        \| AB \|\leq \| A \|\| B \|.
    \end{equation}
    De plus nous supposerons toujours avoir une norme subordonnée à une norme sur l'espace \( \eC^n\), de telle sorte à avoir
    \begin{equation}
        \| Au \|\leq \| A \|\| u \|
    \end{equation}
    pour tout \( u\in\eC^n\). Voir aussi le lemme~\ref{LEMooIBLEooLJczmu}.
\end{remark}

\begin{proposition}[\cite{ooLMMRooUXhOdx}]
    Si \( A\) est une matrice inversible et si \( \alpha\in \eC\) nous avons :
    \begin{enumerate}
        \item
            \( \Cond(A)\geq 1\)
        \item
            \( \Cond(A)=\Cond(A^{-1})\)
        \item
            \( \Cond(\alpha A)=\Cond(A)\).
    \end{enumerate}
    Si \( Q\in\gO(n)\) alors
    \begin{enumerate}
        \item
            Nous avons \( \Cond_2(Q)=1\) où \( \Cond_2\) est le conditionnement pour la norme \( \| . \|_2\).
        \item
            Nous avons aussi
            \begin{equation}
                \Cond_2(A)=\Cond_2(AQ)=\Cond_2(QA).
            \end{equation}
    \end{enumerate}
\end{proposition}

\begin{proof}
    Nous savons que \( \Cond(\mtu)=1\) et donc
    \begin{equation}
        1=\| \mtu \|\leq \| A \|\| A^{-1} \|
    \end{equation}
    parce que la norme utilisée est une norme matricielle.

    Les deux autres formules sont évidentes à partit du fait que la définition du conditionnement de \( A\) est symétrique entre \( A\) et \( A^{-1}\).

    En ce qui concerne les formules relatives à la matrice orthogonale \( Q\) nous savons par la proposition~\ref{PropKBCXooOuEZcS}\ref{ITEMooOWMBooHUatNb} qu'une matrice orthogonale est une bijection de l'ensemble \(  \{ x\in \eR^n\tq \| x \|=1 \}  \). Par conséquent
    \begin{equation}
        \| AQ \|=\sup_{x\tq \| x \|=1}\| AQx \|=\sup_{ Q^{-1}x\tq \| x \|=1  }\| AQQ^{-1}x \|=\| A \|.
    \end{equation}
    Donc \( \| AQ \|=\| A \|\). Les assertions s'ensuivent immédiatement en remarquant que \( Q^{-1}\) est également orthogonale.
\end{proof}

Soit une matrice inversible \( A\in \GL(n,\eC)\). La matrice \( A^*A\) est hermitienne\footnote{Définition~\ref{DEFooKEBHooWwCKRK}.} et le théorème~\ref{LEMooVCEOooIXnTpp} nous assure que ses valeurs propres sont réelles. Par la remarque~\ref{REMooMLBCooTuKFmz}, ses valeurs propres sont même positives.

\begin{proposition}[\cite{ooLMMRooUXhOdx}]      \label{PROPooNUAUooIbVgcN}
    Soit une matrice inversible \( A\in\GL(n,\eC)\), et \( \mu_1\leq\ldots\leq \mu_n\) les valeurs propres de \( A^*A\). Alors nous avons la formule
    \begin{equation}
        \Cond_2(A)=\sqrt{ \frac{ \mu_n }{ \mu_1 }}.
    \end{equation}
\end{proposition}

\begin{proof}
    Par le théorème~\ref{THOooNDQSooOUWQrK}, la norme de \( A\) est liée au au rayon spectral de \( A^*A\) par
    \begin{equation}
        \| A \|_2=\sqrt{ \rho(A^*A) }=\sqrt{ \mu_n }.
    \end{equation}
    Vu que le spectre de \( AA^*\) est le même que celui de \( A^*A\) (lemme~\ref{LEMooHUGEooVYhZdZ}) nous avons aussi
    \begin{equation}
        \| A^{-1} \|_2=\sqrt{ \rho\big( (A^{-1})^*A^{-1} \big) }=\sqrt{ \rho\big( (A^*A)^{-1} \big) }=\frac{1}{ \sqrt{ \mu_1 } }
    \end{equation}
    parce que la plus grande valeur propre de \( (A^*A)^{-1}\) est l'inverse de la plus petite de \( A^*A\).

    Ces deux calculs étant,
    \begin{equation}
        \Cond_2(A)=\| A \|_2\| A^{-1} \|_2=\sqrt{ \frac{ \mu_n }{ \mu_1 } }.
    \end{equation}
\end{proof}

\begin{probleme}
    À mon avis ce qui est dans la proposition~\ref{PropSOOooGoMOxG} est le conditionnement de la matrice ou sa racine carrée ou un truc du genre. Il faut voir le lien entre les valeurs propres de \( A\) et celles de \( AA^*\).
\end{probleme}

%---------------------------------------------------------------------------------------------------------------------------
\subsection{Perturbation du vecteur}
%---------------------------------------------------------------------------------------------------------------------------

\begin{proposition}[Système linéaire : perturbation du vecteur\cite{ooLMMRooUXhOdx}]        \label{PROPooGIXFooAhJkIs}
    Soit une matrice inversible \( A\) et les systèmes d'équations linéaires
    \begin{subequations}        \label{EQooYQIGooPXqWoX}
        \begin{align}
            Au=b\\
            Au'=b'.
        \end{align}
    \end{subequations}
    En notant \( \Delta u=u'-u\) et \( \Delta b=b'-b\) nous avons
    \begin{equation}        \label{EQooESXRooMYuvRa}
        \frac{ \| \Delta u \| }{ \| u \| }\leq \Cond(A)\frac{ \| \Delta b \| }{ \| b \| }.
    \end{equation}
\end{proposition}

\begin{proof}
    En soustrayant les équations \eqref{EQooYQIGooPXqWoX} nous avons \( \Delta b=A\Delta u\), et donc \( \Delta u=A^{-1} \Delta b\). D'une part nous avons alors
    \begin{equation}
        \| \Delta u \|\leq \| A^{-1} \|\| \Delta b \|.
    \end{equation}
    Et d'autre part, \( \| b \|\leq \| A \|\| u \|\), ce qui donne
    \begin{equation}
        \frac{ \| b \| }{ \| A \| }\leq \| u \|.
    \end{equation}
    En mettant les deux ensemble,
    \begin{equation}
        \frac{ \| \Delta u \| }{ \| u \| }\leq \frac{ \| A^{-1} \|\| \Delta b \| }{ \| b \| }\| A \|=\Cond(A)\frac{ \| \Delta b \| }{ \| b \| }.
    \end{equation}
\end{proof}

Le corolaire suivant justifie le nom «conditionnement» au conditionnement d'une matrice.
\begin{corollary}       \label{CORooXKPWooJVHVvh}
    Soit \( A\in \GL(n,\eC)\) fixée et le problème de résoudre \( Au=b\), c'est-à-dire la fonction
    \begin{equation}
        F(u,b)=Au-b.
    \end{equation}
    \begin{enumerate}
        \item
            Ce problème est stable pour toute valeur de \( b\).
        \item
            Nous avons une majoration pour le conditionnement relatif\footnote{Si vous doutez de la norme à prendre, lisez la remarque~\ref{REMooAIKIooJEBEqi}} :
            \begin{equation}        \label{EQooZHQJooTMKYfr}
                K_{rel}(\eta,b_0)\leq \Cond(A).
            \end{equation}
    \end{enumerate}
\end{corollary}

\begin{proof}
    \begin{subproof}
    \item[Stabilité]
        Vu que \( A\) est inversible, il existe une solution unique à tout système de la forme \( Au=b'\). De plus \( u(b)=A^{-1} b\), donc
        \begin{equation}
            \| u(b)-u(b_0) \|= \| A^{-1}(b-b_0) \|\leq \| A^{-1} \|\| b-b_0 \|,
        \end{equation}
        de telle sorte que la condition~\ref{DEFooYIFAooSJbMkC}\ref{ItemProbStableB} fonctionne avec \( K=\| A^{-1} \|\).
    \item[Conditionnement]
        En partant de la définition~\ref{DEFEQooSXDBooYbvGrC}, et en utilisant la majoration de la proposition~\ref{PROPooGIXFooAhJkIs} sous la forme
        \begin{equation}
            \| u(b)-u(b_0) \|\leq \Cond(A)\| u(b_0) \|\frac{ \| \Delta b \| }{ \| b_0 \| },
        \end{equation}
        nous obtenons :
        \begin{subequations}
            \begin{align}
                K_{rel}(b_0,\eta)&=K_{abs}(b_0,\eta)\frac{ \| b_0 \| }{ \| u(b_0) \| }\\
                &=\sup_{\| b-b_0 \|\leq \eta}\frac{ \| u(b)-u(b_0) \| }{ \| b-b_0 \| }\frac{ \| b_0 \| }{ \| u(b_0) \| }\\
                &\leq \sup_b\Cond(A)\frac{ \| u(b_0) \| }{ \| b_0 \| }\| \Delta b \|\frac{1}{ \| b-b_0 \| }\frac{ \| b_0 \| }{ \| u(b_0) \| }\\
                &=\Cond(A).
            \end{align}
        \end{subequations}
    \end{subproof}
\end{proof}

\begin{remark}      \label{REMooAIKIooJEBEqi}
    La notion de conditionnement relatif dépend aussi de la norme choisie. Dans la formule \eqref{EQooZHQJooTMKYfr} il faut prendre le conditionnement \( \Cond(A)\) pour la norme dans laquelle le \( K_{rel}\) est écrit. Encore une fois, toutes les normes étant équivalentes,  cette majoration est à constante près bonne pour toutes les normes. Si la dimension est très grande, cette constante peut par contre être grande.
\end{remark}

%---------------------------------------------------------------------------------------------------------------------------
\subsection{Perturbation de la matrice}
%---------------------------------------------------------------------------------------------------------------------------

\begin{proposition}[Système linéaire : perturbation de la matrice\cite{ooLMMRooUXhOdx}]
    Soient les systèmes linéaires
    \begin{subequations}
        \begin{align}
            Au=b\\
            A'u'=b
        \end{align}
    \end{subequations}
    avec \( A\) et \( A'\) inversibles. Nous notons \( \Delta A=A'-A\). Alors
    \begin{enumerate}
        \item       \label{ITEMooJMTKooSEBavB}
            \begin{equation}
                \frac{ \| \Delta u \| }{ \| u' \| }\leq \Cond(A)\frac{ \| \Delta A \| }{ \| A \| }
            \end{equation}
        \item
            \begin{equation}
                \frac{ \| \Delta u \| }{ \| u \| }\leq \Cond(A)\frac{ \| \Delta A \| }{ \| A \| }\big( 1+\alpha(\| \Delta A \|) \big)
            \end{equation}
            où \( \lim_{x\to 0} \alpha(x)=0\).
    \end{enumerate}
\end{proposition}

\begin{proof}
    D'abord nous avons
    \begin{subequations}
        \begin{align}
            0&=Au'-Au\\
            &=(A'-A)u'-Au'-Au\\
            &=\Delta Au'+A\Delta u.
        \end{align}
    \end{subequations}
    Par conséquent, \( \Delta u=-A^{-1}(\Delta A)u'\) et
    \begin{equation}        \label{EQooYYITooSSczEj}
        \| \Delta u \|\leq \| A^{-1} \|\| \Delta A \|\| u' \|.
    \end{equation}
    Donc
    \begin{equation}
        \frac{ \| \Delta u \| }{ \| u' \| }\leq   \| A^{-1} \|\| A \|\frac{ \| \Delta A \| }{ \| A \| }   =\Cond(A)\frac{ \| \Delta A \| }{ \| A \| }.
    \end{equation}
    Cela est~\ref{ITEMooJMTKooSEBavB}.

    Pour l'autre inégalité, nous avons \( A'=A+\Delta A\) et donc
    \begin{equation}
        \| A'^{-1} \|=\| (A+\Delta A)^{-1} \|
    \end{equation}
    Nous repartons alors de \eqref{EQooYYITooSSczEj} en changeant le rôle de \( A\) et \( A'\) (et donc aussi de \( u\) et \( u'\)). Ce changement étant, \( \| \Delta u \|\) et \( \| \Delta A \|\) ne changent pas. Nous avons :
    \begin{subequations}
        \begin{align}
            \frac{ \| \Delta u \| }{ \| u \| }&\leq \| A'^{-1} \|\| \Delta A \|\\
            &=\| (A+\Delta A)^{-1} \|\| \Delta A \|\frac{ \Cond(A) }{ \| A \|\| A^{-1} \| }\\
            &=\frac{ \| (A+\Delta A)^{-1} \| }{ \| A^{-1} \| }\frac{ \| \Delta A \| }{ \| A \| }\Cond(A).
        \end{align}
    \end{subequations}
    Il reste à voir que
    \begin{equation}
        \lim_{\| \Delta A \|\to 0} \frac{ \| (A+\Delta A)^{-1} \| }{ \| A^{-1} \| }=1,
    \end{equation}
    ou autrement dit que
    \begin{equation}        \label{EQooJURGooFvYiAs}
        \lim_{A\to A'} \frac{ \| A'^{-1} \| }{ \| A^{-1} \| }=1
    \end{equation}
    où la limite est celle dans \( \GL(n,\eC)\). Par définition de la topologie, la norme est continue (quelle qu'elle soit par l'équivalence de norme~\ref{ThoNormesEquiv}). Par le théorème~\ref{ThoCINVBTJ}, l'application \( A\mapsto A^{-1}\) est également continue et commute donc avec la limite. Nous avons donc
    \begin{equation}
        \lim_{A'\to A}\| A'^{-1} \|=\| (\lim_{A'\to A} A')^{-1} \|=\| A^{-1} \|.
    \end{equation}
    Donc la limite du quotient \eqref{EQooJURGooFvYiAs} est bien \( 1\).
\end{proof}

%+++++++++++++++++++++++++++++++++++++++++++++++++++++++++++++++++++++++++++++++++++++++++++++++++++++++++++++++++++++++++++
\section{Système linéaires (généralités)}
%+++++++++++++++++++++++++++++++++++++++++++++++++++++++++++++++++++++++++++++++++++++++++++++++++++++++++++++++++++++++++++

Soit un système d'équations linéaires \( Ax=b\) avec \( A\in \eM(n,\eR)\). Le problème est évidemment de savoir s'il existe une unique solution \( x\) et de la déterminer. Nous supposons l'existence et l'unicité. C'est-à-dire que les conditions équivalentes\footnote{L'équivalence est la proposition~\ref{PropYQNMooZjlYlA}\ref{ITEMooNZNLooODdXeH}.} sont vérifiées :
\begin{enumerate}
    \item
        \( A\) est inversible, c'est-à-dire qu'il existe une matrice notée \( A^{-1}\) telle que \( AA^{-1}=A^{-1}A=\mtu\).
    \item
        \( \det(A)\neq 0\).
\end{enumerate}
Note : si  nous avons un système pas carré du type \( Bx=v\) avec \( B\in \eM(n\times m)\) alors nous pouvons nous ramener à un système carré en écrivant
\begin{equation}
    B^tBx=B^tv.
\end{equation}
Mais attention : bien que \( B^tB\) soit symétrique et semi-définie positive, certaines valeurs propres peuvent être nulles.

\begin{normaltext}
    Deux choses générales en calcul numérique :
    \begin{enumerate}
        \item
            On ne calcule pas l'inverse d'une matrice.
        \item
            On ne calcule même pas son déterminant.
    \end{enumerate}
    Par conséquent nous ne faisons pas \( x=A^{-1}v\).

    Il faut garder en tête le fait que dans la pratique, la matrice \( A\) possède des millions de lignes et colonnes, si pas pire. Pour une matrice de taille de l'ordre du million, il y a \( 1000\) milliards d'entrées. Si on compte \( 32\) bits par nombre (précision simple, définition~\ref{DEFooEIOZooYLDVjs}), c'est-à-dire \( 4\) octets, il faut \( 4000\) giga-octets pour enregistrer la matrice. Même pour la mémoire actuellement disponible, ce n'est pas rien. Surtout que souvent, la précision simple n'est pas utilisée, mais la précision double, ce qui donne \( 8000\) giga pour enregistrer la matrice.

    Heureusement, dans la majorité des cas pratiques, les matrices géantes qui apparaissent sont pleines de zéros.

\end{normaltext}

\begin{definition}
    Une matrice est \defe{creuse}{matrice!creuse} si elle possède beaucoup de zéros. Une matrice non creuse est dite \defe{dense}{matrice!dense}.

    Notons que lorsqu'on parle de matrice comprenant beaucoup de «zéros», nous pensons à des éléments très petits, et non de vrai zéros.
\end{definition}
    Les matrices creuses ne sont pas mémorisées entièrement, mais plutôt comme un dictionnaire \( (i,j,v)\) qui donne la valeur \( v\) de \( A_{ij}\).

\begin{definition}
    Une matrice est de «grande dimension» si elle ne peut pas être mise en mémoire sur un ordinateur donné. Sur certains ordinateurs, ça commence à \( 5000\) inconnues. Mais sur des plus forts, on peut aller jusqu'au million ou le milliard.
\end{definition}

Si la matrice est de petite dimension, il est possible d'utiliser des méthodes dites «directes». Sinon, il faudra utiliser des méthodes itératives.

%---------------------------------------------------------------------------------------------------------------------------
\subsection{Les méthodes directes}
%---------------------------------------------------------------------------------------------------------------------------

Une méthode directe consiste à successivement transformer un système \( A^{(0)}x=b^{(0)}\) en de nouveaux systèmes \( A^{(i)}x=b^{(i)}\) dont la solution est identique jusqu'à obtenir un système \( A^{(n-1)}x=b^{(n-1)}\) qui est à résolution immédiate.

L'avantage d'une méthode directe est qu'elle fournit une réponse exacte, pour autant que les calculs intermédiaires soient bien faits (ce qui n'est pas le cas sur un ordinateur).

Une méthode directe fonctionne en général avec un nombre de pas fixés par la taille du système. Par exemple pour un système \( n\times n\), la méthode de Gauss demande exactement \( n\) pas, et il n'y a pas moyen de faire mieux. Or chaque pas demande de recalculer tous les éléments de la matrice. Encore une fois, si la matrice a une taille de l'ordre du milliard, cela fait \( 10^{18}\) éléments à recalculer un milliard de fois (sans compter les éléments du vecteur \( b\)). Infaisable.


Souvent une méthode directe passe par une factorisation \( A=BC\) avec \( B,C\in \eM(n\times n)\).

Quelques types de matrices dont la résolution est immédiate :
\begin{itemize}
    \item Matrice diagonale.
    \item Matrice orthogonale parce que si \( A\) est orthogonale alors \( Ax=v\) se résout par \( x=A^tv\) qui n'est pas particulièrement lourd à faire numériquement.
    \item Matrice triangulaire.
\end{itemize}

\begin{remark}
    Pour une matrice diagonale, le déterminant et l'inverse sont faciles. Mais également pour la triangulaire. Pour une matrice triangulaire, le déterminant est le produit des éléments diagonaux, et il se fait qu'il y a une algorithme facile pour calculer l'inverse.

    Donc en fait les matrices à résolution immédiates sont des matrices pour lesquelles l'inverse et le déterminant sont facile à calculer.
\end{remark}

%---------------------------------------------------------------------------------------------------------------------------
\subsection{Méthodes itératives}
%---------------------------------------------------------------------------------------------------------------------------

Si la matrice est trop grande, il n'est pas possible de faire des manipulations de matrices à chaque itération.

En général, les méthodes itératives ne convergent pas toujours. Mais lorsqu'une méthode converge, c'est une propriété de la matrice, et donc la convergence aura lieu pour tout vecteur de départ \( x_0\). Cela est très différent du cas des équations non linéaires type Newton pour lesquelles la convergence peut fortement dépendre du point de départ.

%+++++++++++++++++++++++++++++++++++++++++++++++++++++++++++++++++++++++++++++++++++++++++++++++++++++++++++++++++++++++++++
\section{Système linéaires (méthodes directes)}
%+++++++++++++++++++++++++++++++++++++++++++++++++++++++++++++++++++++++++++++++++++++++++++++++++++++++++++++++++++++++++++

Les matrices que nous sommes autorisés à inverser sont les matrices
\begin{itemize}
    \item orthogonales : l'inverse est la transposée
    \item diagonales : l'inverse est diagonale avec les inverses sur la diagonale
    \item triangulaires : nous en parlons maintenant.
\end{itemize}

%---------------------------------------------------------------------------------------------------------------------------
\subsection{Inversion de matrice triangulaire}
%---------------------------------------------------------------------------------------------------------------------------

Si \( T\) est une matrice triangulaire (mettons supérieure pour fixer les idées), il est possible d'en calculer l'inverse sans trop d'efforts. Notons \( B\) la matrice inverse que nous allons construire ligne par ligne. Vu que \( BT=\mtu\) nous avons
\begin{equation}
    \delta_{1j}=\sum_{k=1}^nB_{1k}T_{kj}=\sum_{k=1}^jB_{1k}T_{kl}
\end{equation}
parce que \( T_{kj}=0\) pour \( k>j\). Donc nous pouvons calculer les éléments \( B_{1j} \) un par un parce que chacun ne dépend que des précédents. Le même procédé fonctionne pour les autres lignes :
\begin{equation}
    \delta_{ij}=\sum_{k=1}^jB_{ij}T_{kj}.
\end{equation}
Et tu notes que le calcul peut être parallélisé : le calcul de la ligne numéro \( j\) ne dépend pas du résultat des autres lignes.

%---------------------------------------------------------------------------------------------------------------------------
\subsection{Transformation gaussienne}
%---------------------------------------------------------------------------------------------------------------------------

\begin{definition}[Transformation gaussienne\cite{ooPFJDooUQIwHZ}]
    Soit \( x\in \eR^n\) avec \( x_k\neq 0\). La \( k\)\ieme \defe{transformation gaussienne}{transformation!gaussienne} pour \( x\) est la matrice
    \begin{equation}
        M_k(x)=\mtu-T_k(x)
    \end{equation}
    où \( T_k(x)\) est la matrice unité à qui on a ajouté le vecteur
    \begin{equation}
        \tau_k(x)=
        \begin{pmatrix}
            0    \\
            \vdots    \\
            0    \\
            x_{k+1}/x_k    \\
            \vdots    \\
            x_n/x_k
        \end{pmatrix}
    \end{equation}
    à la \( k\)\ieme colonne.
\end{definition}
Autrement dit, la matrice \( M_k(x)\) est la matrice
\begin{equation}        \label{EQooMWXLooBDtsKS}
    M_k(x)=\begin{pmatrix}
        1    &       &       &       &       &   \\
        0    &    \ddots    &                 &       &       &   \\
        \vdots    &   0    &   1               &         &       &   \\
        \vdots    & \vdots      &   -x_{k+1}/x_k    &   1    &       &   \\
        \vdots    &   \vdots    &   \vdots    &   0    &   \ddots    &   \\
        0    &   0    &           -x_n/x_k   &  0   &      &   1
    \end{pmatrix}
\end{equation}
En coordonnées nous avons
\begin{equation}
    \big( M_k(x)\big)_{ij}=\delta_{ij}-\tau_k(x)_i\delta_{kj}.
\end{equation}

\begin{normaltext}
    Les matrices de transformation gaussienne sont des matrices triangulaires de diagonale unitaire (c'est-à-dire avec des \( 1\) sur la diagonale).
\end{normaltext}

\begin{lemma}
    Si \( x\in \eR^n\) alors nous avons
    \begin{equation}
        M_k(x)x=\begin{pmatrix}
            x_1    \\
            \vdots    \\
            x_k    \\
            0    \\
            \vdots    \\
            0
        \end{pmatrix}.
    \end{equation}
\end{lemma}

\begin{proof}
    Nous avons
    \begin{equation}
        \big( M_k(x)x \big)_i=\sum_lM_k(x)_{il}x_l=\sum_l\big( \delta_{il}-\tau_k(x)_i\delta_{kl} \big)x_l=x_i-\tau_k(x)_ix_k.
    \end{equation}
    Si \( i\leq k\) nous avons \( \tau_k(x)_i=0\) et donc \(  \big( M_k(x)x \big)_i=x_i   \). Si par contre \( i\geq k+1\) alors \( \tau_k(x)_i=\frac{ x_i }{ x_k }\) et alors \( \big( M_k(x)x \big)_i=0\).
\end{proof}

\begin{lemma}       \label{LEMooPFWWooUmMsVH}
    Si \( y\in \eR^n\) vérifie \( y_i=0\) pour \( i>k\) alors \( M_{k+1}(x)y=y\).
\end{lemma}

\begin{proof}
    C'est une simple vérification :
    \begin{equation}
        \big( M_{k+1}(x)y \big)_i=\sum_l\big( \delta_{il}-\tau_{k+1}(x)_i\delta_{k+1,l} \big)y_l=y_i-\tau_{k+1}(x)_iy_{k+1}.
    \end{equation}
    Mais comme \( y_{k+1}=0\) il nous reste automatiquement \( y_i\).
\end{proof}
Le sens de ce lemme est si un vecteur est déjà «gaussiannisé» au niveau \( k\), alors en lui appliquant une transformation gaussienne de niveau plus élevé que \( k\), il ne change pas. Ce fait est important parce qu'il assure que lorsque l'on avance dans le processus de Gauss, chaque étape ne détruit pas les précédentes.

Le lemme suivant nous indique que l'inverse d'une matrice de transformation gaussienne est facile à calculer\footnote{Elle rentre d'ailleurs dans la catégorie des matrices triangulaires dont nous avons déjà discuté l'inverse.}.

\begin{lemma}       \label{LEMooFHZDooZiKdbr}
    L'inverse de la transformation gaussienne
    \begin{equation}
        M_k(x)_{ij}=\delta_{ij}-\tau_k(x)_i\delta_{kj}.
    \end{equation}
    est la matrice donnée par
    \begin{equation}
        M_k(x)^{-1}_{ij}=\delta_{ij}+\tau_k(x)_i\delta_{kj}.
    \end{equation}
    Autrement dit, il suffit de changer le signe de la partie non diagonale.
\end{lemma}

\begin{proof}
    Il s'agit d'une simple vérification, utilisant le produit matriciel explicite, et en remarquant que \( \tau_k(x)_k=0\) pour tout \( k\).
\end{proof}

%---------------------------------------------------------------------------------------------------------------------------
\subsection{Méthode de Gauss pour résoudre des systèmes d'équations linéaires}
%---------------------------------------------------------------------------------------------------------------------------

Pour résoudre un système d'équations linéaires, on procède comme suit:
\begin{enumerate}
\item Écrire le système sous forme matricielle. \[\text{p.ex. } \begin{cases} 2x+3y &= 5 \\ x+2y &= 4 \end{cases} \Leftrightarrow \left(\begin{array}{cc|c} 2 & 3 & 5 \\ 1 & 2 & 4 \end{array}\right) \]
\item Se ramener à une matrice avec un maximum de $0$ dans la partie de gauche en utilisant les transformations admissibles:
\begin{enumerate}
\item Remplacer une ligne par elle-même + un multiple d'une autre;
\[\text{p.ex. } \left(\begin{array}{cc|c} 2 & 3 & 5 \\ 1 & 2 & 4 \end{array}\right)  \stackrel{L_1  - 2. L_2 \mapsto L_1'}{\Longrightarrow} \left(\begin{array}{cc|c} 0 & -1 & -3 \\ 1 & 2 & 4 \end{array}\right) \]
\item Remplacer une ligne par un multiple d'elle-même;
\[\text{p.ex. } \left(\begin{array}{cc|c} 0 & -1 & -3 \\ 1 & 2 & 4 \end{array}\right)  \stackrel{-L_1  \mapsto L_1'}{\Longrightarrow} \left(\begin{array}{cc|c} 0 & 1 & 3 \\ 1 & 2 & 4 \end{array}\right) \]
\item Permuter des lignes.
\[\text{p.ex. } \left(\begin{array}{cc|c} 0 & 1 & 3 \\ 1 & 0 & -2 \end{array}\right)  \stackrel{L_1  \mapsto L_2' \text{ et } L_2  \mapsto L_1'}{\Longrightarrow} \left(\begin{array}{cc|c} 1 & 0 & -2 \\ 0 & 1 & 3  \end{array}\right) \]
\end{enumerate}
\item Retransformer la matrice obtenue en système d'équations.
\[\text{p.ex. }  \left(\begin{array}{cc|c} 1 & 0 & -2 \\ 0 & 1 & 3  \end{array}\right) \Leftrightarrow \begin{cases} x &= -2 \\ y &= 3 \end{cases}  \]
\end{enumerate}

\textbf{Remarques :}
\begin{itemize}
\item Si on obtient une ligne de zéros, on peut l'enlever:
\[\text{p.ex. }  \left(\begin{array}{ccc|c} 3 & 4 & -2 & 2 \\ 4 & -1 & 3 & 0 \\ 0 & 0 & 0 & 0 \end{array}\right) \Leftrightarrow  \left(\begin{array}{ccc|c} 3 & 4 & -2 & 2 \\ 4 & -1 & 3 & 0 \end{array}\right) \]
\item Si on obtient une ligne de zéros suivie d'un nombre non-nul, le système d'équations n'a pas de solution:
\[\text{p.ex. }  \left(\begin{array}{ccc|c} 3 & 4 & -2 & 2 \\ 4 & -1 & 3 & 0 \\ 0 & 0 & 0 & 7 \end{array}\right) \Leftrightarrow  \begin{cases} \cdots \\ \cdots \\ 0x + 0y + 0z = 7 \end{cases} \Rightarrow \textbf{Impossible} \]
\item Si on moins d'équations que d'inconnues, alors il y a une infinité de solutions qui dépendent d'un ou plusieurs paramètres:
\[\text{p.ex. }  \left(\begin{array}{ccc|c} 1 & 0 & -2 & 2 \\ 0 & 1 & 3 & 0 \end{array}\right) \Leftrightarrow  \begin{cases} x - 2z = 2 \\ y + 3z = 0 \end{cases} \Leftrightarrow  \begin{cases} x = 2 + 2\lambda \\ y = -3\lambda \\ z = \lambda \end{cases} \]
\end{itemize}

%---------------------------------------------------------------------------------------------------------------------------
\subsection{Méthode de Gauss sans pivot (décomposition LU)}
%---------------------------------------------------------------------------------------------------------------------------

La méthode de Gauss est encore utilisée aujourd'hui dans les vrais problèmes.

La méthode de Gauss est souvent aussi appelée méthode «LU» qui va décomposer \( A=LU\) où \( L\) est triangulaire inférieure et \( U\) est triangulaire supérieure. La décomposition est même plus précise que cela : on demande que \( L\) ait seulement des \( 1\) sur la diagonale.

Si \( A\) est une matrice nous notons
\begin{equation}
    \Delta_k(A)= (A_{ij})_{1\leq i,j\leq k}
\end{equation}
la matrice tronquée dont nous ne gardons que le carré \( k\times k\) en haut à gauche.

\begin{lemma}       \label{LEMooXEJFooGiYoyb}
    Soit \( S\) une matrice triangulaire inférieure. Soient également \( A\) et \( B\) telles que \( B=SA\). Alors
    \begin{equation}
        \Delta_k(B)=\Delta_k(S)\Delta_k(A).
    \end{equation}
\end{lemma}

\begin{proof}
     En effet nous avons
     \begin{equation}       \label{EQooHBZZooHtjjsE}
         \Delta_k(B)_{ij}=\sum_{l=1}^nS_{il}A_{lj}.
     \end{equation}
     Dans la somme sur \( l\) il ne reste que les termes \( l\leq i\). Mais dans le calcul des éléments de matrice \( \Delta_k(B)_{ij}\), nous avons évidemment \( i,j\leq k\). Donc \( l\leq i\leq k\). Les seuls éléments de matrice de \( A\) qui sont utilisés dans la somme \eqref{EQooHBZZooHtjjsE} sont les éléments \( A_{lj}\) avec \( l,j\leq k\).

     Nous pouvons donc limiter la somme à \( l=k\) au lieu de \( l=n\) et écrire \( \Delta_k(A)_{lj}\) au lieu de \( A_{lj}\).

     Même chose en ce qui concerne \( S\). À partir du moment où \( l\) est limité à \( k\), les éléments \( S_{il}\) et \( \Delta_k(S)_{il}\) sont les mêmes.
\end{proof}

\begin{theorem}[Décomposition \( LU\)\cite{ooDANFooPSmBfd,MonCerveau}]       \label{THOooUXKJooYaPhiu}
    Soit une matrice \( A\) inversible telles que \( \det(\Delta_k(A))\neq 0\) pour tout \( k\). Alors il existe un unique couple de matrices \( (L,U)\) telles que
    \begin{itemize}
        \item \( U\) soit triangulaire supérieure
        \item \( L\) soit triangulaire inférieure, de diagonale unité
        \item \( A=LU\).
    \end{itemize}
    De plus pour tout \( k\leq n\) nous avons
    \begin{equation}
        \Delta_k(A)=\Delta_k(L)\Delta_k(U).
    \end{equation}
\end{theorem}

\begin{proof}
    Nous allons prouver par récurrence le fait suivant : pour tout \( 1\leq k\leq n-1\) il existe des matrices \( E_i\) ($i=1,\ldots, k$) telles que en posant
    \begin{equation}
        A_k=E_{k}\ldots E_1A,
    \end{equation}
    \begin{itemize}
        \item \( E_j\) est une transformation gaussienne pour la \( j\)\ieme colonne,
        \item pour tout \( j\leq k\), \( A_{ij}=0\) dès que \( i>j\). Autrement dit la matrice \( A_k\) est triangulaire supérieure jusqu'à y compris la \( (k+1)\)\ieme colonne (laquelle est quelconque). Exemple pour fixer les idées : pour une matrice \( A\in \eM(4\times 4) \), la matrice \( A_2\) doit avoir la forme
            \begin{equation}
                A_2=E_2E_1A=\begin{pmatrix}
                     *   &   *    &   *    &   *    \\
                     0   &   *    &   *    &   *    \\
                     0   &   0    &   \oasterisk    &   *    \\
                     0   &   0    &   *    &   *
                 \end{pmatrix}
            \end{equation}
            où les éléments notés \(*\) sont à priori non nuls,
        \item l'élément de matrice \( (A_k)_{ k+1,k+1  }  \) est non nul (celui entouré dans l'exemple).
    \end{itemize}

    La chose un peu triste dans cette démonstration est que l'initialisation va être très ressemblante au pas de récurrence.
    \begin{subproof}
        \item[Initialisation : \( k=1\)]

            Vu que \( \Delta_1(A)\) est inversible, l'élément \( A_{11}\) est non nul. Il existe donc une transformation gaussienne \( E_1\) telle que la première colonne de la matrice \( A_1=E_1A\) soit nul sauf la première composante. En particulier \( (A_1)_{21}=0\).

            Par le lemme~\ref{LEMooXEJFooGiYoyb}, nous avons \( \Delta_2(A_1)=\Delta_2(E_1)\Delta_2(A)\), donc\footnote{Le déterminant est multiplicatif, proposition~\ref{PropYQNMooZjlYlA}\ref{ItemUPLNooYZMRJy}.}
            \begin{equation}
                \det\big( \Delta_2(A_1) \big)=\det\big( \Delta_2(E_1) \big)\det\big( \Delta_2(A) \big).
            \end{equation}
            Étant donnée la forme \eqref{EQooMWXLooBDtsKS}, toutes les matrices du type \( \Delta_k(E_i)\) ont un déterminant unité, et par hypothèse \( \Delta_2(A)\) est inversible, donc de déterminant non nul. Par conséquent \( \det\big( \Delta_2(A_1) \big)\neq 0\). Mais comme ce déterminant est le produit des éléments diagonaux (c'est une matrice triangulaire), ces derniers ne sont pas nuls. Finalement, \( (A_1)_22\neq 0\).

        \item[Le pas de récurrence]

            Nous supposons avoir \( A_k=E_k\ldots E_1A\) avec \( (A_k)_{k+1,k+1}\neq 0\). Alors il existe une transformation gaussienne \( E_{k+1}\) de la \( (k+1)\)\ieme colonne telle que \( A_{k+1}=E_{k+1}A_k\) soit une matrice dont la \( (k+1)\)\ieme colonne n'ait que des zéros en dessous de la \( (k+1)\)\ieme position. Vu le lemme~\ref{LEMooPFWWooUmMsVH}, cette transformation n'affecte pas les colonnes précédentes.

            La matrice \( A_{k+1}\) est donc triangulaire supérieure jusqu'à la \( (k+2)\)\ieme colonne.

            Vu que le produit \( E_k\ldots E_1\) est une matrice triangulaire inférieure, le lemme~\ref{LEMooXEJFooGiYoyb} fonctionne encore et nous avons
            \begin{equation}
                \Delta_{k+1}(A_k)=\Delta_{k+1}(E_k\ldots E_1)\Delta_{k+1}(A).
            \end{equation}
            En ce qui concerne les déterminants, par hypothèse, nous avons \( \det\big( \Delta_{k+1}(A) \big)\neq 0\) ainsi que \( \det\big( \Delta_{k+1}(E_k\ldots E_1) \big)=1\). Donc
            \begin{equation}
                \det\big( \Delta_{k+1}(A_k) \big)\neq 0.
            \end{equation}
            Cette matrice étant triangulaire, ses éléments diagonaux sont non nuls et nous avons \( (A_k)_{k+1,k+1}\neq 0\).
    \end{subproof}

    En poussant la récurrence jusqu'au bout, la matrice
    \begin{equation}
        A_{n-1}=E_{n-1}\ldots E_nA
    \end{equation}
    est triangulaire supérieure.

    Nous posons alors \(   L=(E_{n-1}\ldots E_n)^{-1}  \) et \( U=A_{n-1}\). Cela prouve l'existence parce que
    \begin{equation}
        A=(E_{n-1}\ldots E_1)^{-1}A_{n_1}.
    \end{equation}
    Encore une fois, le lemme~\ref{LEMooXEJFooGiYoyb} nous donne
    \begin{equation}
        \Delta_k(A)=\Delta_k\Big( (E_{n_1}\ldots E_1)^{-1} \Big)\Delta_k(A_{n-1}),
    \end{equation}
    ou encore \( \Delta_k(A)=\Delta_k(L)\Delta_k(U)\).

    En ce qui concerne l'unicité, si \( A=L_1U_1=L_2U_2\) alors \( L_2^{-1}L_1=U_2U_1^{-1} \). Vu qu'à gauche nous avons une matrice triangulaire inférieure et que à droite nous avons une triangulaire inférieure, nous savons que les deux membres représentent une matrice diagonale. Mais à gauche, la diagonale est unitaire. Donc les deux membres représentent la matrice unité.
\end{proof}

\begin{normaltext}
    En pratique, pour résoudre \( Ax=b\), il faut seulement appliquer les transformations gaussiennes à la matrice élargie \( (A|b)\) pour finir sur un système du type
    \begin{equation}
        Ux=b'
    \end{equation}
    qui est immédiatement soluble. Autrement dit, en effectuant les annulations de colonnes, la matrice \( U\) est «gratuite».

    Il n'est pas indispensable de calculer la matrice \( L\) qui, elle, demande à chaque étape de se souvenir de la matrice \( E_i\) utilisée. S'il faut résoudre plusieurs systèmes \( Ax_i=b_i\), nous pouvons encore travailler avec la matrice encore plus élargie \( (A|b_1\ldots b_m)\).

    Si par contre nous ne connaissons pas à l'avance l'ensemble des vecteurs \( b\) avec lesquels il faudra résoudre le système, il est bon de calculer la décomposition \( A=LU \) in extenso, c'est-à-dire de garder une trace des matrices \( L\) et \( U\) séparément. Dans ce cas, résoudre \( Ax=b\) revient à résoudre \( Ly=b\), et ensuite \( Ux=y\). Ce sont deux systèmes de résolution directe parce que les matrices sont triangulaires.
\end{normaltext}

\begin{normaltext}
    Le fait que
    \begin{equation}
        \Delta_k(A)=\Delta_k(L)\Delta_k(U)
    \end{equation}
    nous dit que si après avoir calculé \( L\) et \( U \) nous remarquons que le système est un peu plus petit ou un peu plus grand que prévu, tout le travail n'est pas perdu. En particulier si le système est plus petit que prévu, l'adaptation de \( L\) et \( U\) est immédiate.
\end{normaltext}

Notons que \( U\) et \( L\) sont inversibles, et que \( \det(L)=1\). Donc \( \det(U)=\det(A)\).

\begin{example}
    Pour travailler la méthode de Gauss pour le système \( Ax=b\), nous introduisons la matrice un peu augmentée \( (A|b)\). Nous faisons un exemple. Soit à résoudre
    \begin{equation}
        \begin{pmatrix}
             2   &   1    &   3       \\
             4   &   3    &   10     \\
             -2   &   1    &   7  3
         \end{pmatrix}
         \begin{pmatrix}
             x   \\
             y   \\
             z
         \end{pmatrix}=\begin{pmatrix}
             11   \\
             28   \\
             3
         \end{pmatrix}.
    \end{equation}
    Nous introduisons la matrice augmentée
    \begin{equation}
        (A|b)^{(0)}=\begin{pmatrix}
             2   &   1    &   3    &   11    \\
             4   &   3    &   10    &   28    \\
             -2   &   1    &   7    &   3
         \end{pmatrix}.
    \end{equation}
    Le premier pas consiste à annuler tous les éléments sous la diagonale de la première colonne. Autrement dit, nous prenons le \( 2\) comme pivot. Nous introduisons les multiplicateurs \( l_{ij}= \frac{ A_{ij} }{ A_{i1} }\). La nouvelle matrice est :
    \begin{equation}
        (A|b)^{(1)}=\begin{pmatrix}
             2   &   1    &   3    &   11    \\
             0   &   1    &   4    &   6    \\
             0   &   2    &   10    &   14
         \end{pmatrix}
    \end{equation}
    où nous avons utilisé les multiplicateurs \( l_{21}=2\), \( l_{31}=-1\).

    Et la matrice suivante est :
    \begin{equation}
        (A|b)^{(2)}=\begin{pmatrix}
             2   &   1    &   3    &   11    \\
             0   &   1    &   4    &   6    \\
             0   &   0    &   2    &   2
         \end{pmatrix}
    \end{equation}
    où nous avons utilisé le multiplicateur \( l_{32}=2\).

    Cela est un système de résolution immédiate :
        \begin{subequations}
            \begin{numcases}{}
                2x+y+3z=11\\
                y+4z=6\\
                2z=2.
            \end{numcases}
        \end{subequations}
    La troisième donne \( z=1\). Ensuite \( y+4=6\), donc \( y=2\). Et la première donne : \( 2x+2+3=11\), c'est-à-dire \( 2x=6\), enfin : \( x=3\).

    Solution : \( (x,y,z)=(3,2,1)\).

    Nous notons surtout que dans \( (A|b)^{(2)}\) nous avons une matrice triangulaire supérieure. Où est la matrice triangulaire inférieure ? En réalité la matrice \( L\) est la matrice des multiplicateurs :
    \begin{equation}
        L=\begin{pmatrix}
            1    &   0    &   0    \\
            2    &   1    &   0    \\
            -1    &   2    &   1
        \end{pmatrix}.
    \end{equation}
\end{example}

Le problème de cette méthode est que faisant ainsi nous risquons d'avoir un zéro sur un des pivots. Par exemple tomber sur
\begin{equation}
    (A|b)=\begin{pmatrix}
         2   &   1    &   3    &   11    \\
         0   &   0    &   4    &   6    \\
         0   &   2    &   10    &   14
     \end{pmatrix}.
\end{equation}
Le zéro sur la deuxième ligne nous ennuie si nous voulons tout faire dans l'ordre. Mais notons qu'en échangeant les deux dernières lignes, tout va bien : le système donné par
\begin{equation}
    (A|b)=\begin{pmatrix}
         2   &   1    &   3    &   11    \\
         0   &   2    &   10    &   14\\
         0   &   0    &   4    &   6
     \end{pmatrix}
\end{equation}
fonctionne très bien. Et même tellement bien qu'il est de résolution immédiate, dans ce cas.

Un autre problème est que si un des pivots est \( 10^{-14}\), le multiplicateur sera de l'ordre \( 10^{14}\), qui est mal représenté en mémoire. Il est donc bon de prendre les pivots le plus grand possible. Si le pivot est le plus grand nombre en valeur absolue d'une colonne, alors les nombres \( x_{k+i}/x_k\) qui entrent dans la matrice de transformation gaussienne sont des nombres dans \( \mathopen[ -1 , 1 \mathclose]\) qui sont bien représentés en mémoire.

Tout cela nous incite à développer une méthode de Gauss qui permet de tenir une trace des permutations.

%---------------------------------------------------------------------------------------------------------------------------
\subsection{Matrice de permutation élémentaire}
%---------------------------------------------------------------------------------------------------------------------------

\begin{definition}
    Une \defe{matrice de permutation élémentaire}{matrice!permutation!élémentaire} est une matrice obtenue en permutant deux lignes de la matrice identité. Nous notons \( P_{ij}\) la matrice obtenue en inversant les lignes \( i\) et \( j\) de la matrice identité.
\end{definition}

\begin{example}
    \begin{equation}
        \begin{pmatrix}
            1    &   0    &   0    \\
            0    &   1    &   0    \\
            0    &   0    &   1
        \end{pmatrix}\to
        \begin{pmatrix}
            0    &   1    &   0    \\
            1    &   0    &   0    \\
            0    &   0    &   1
        \end{pmatrix}=P_{12}.
    \end{equation}
\end{example}

\begin{lemma}
    La matrice \( P_{ij}A\) est la matrice \( A\) avec ses lignes \( i\) et \( j\) inversées.
\end{lemma}

\begin{proof}
    Il suffit d'écrire
    \begin{equation}
        (P_{ij}A)_{kl}=\sum_m(P_{ij})_{km}A_{ml}
    \end{equation}
    et de faire trois cas selon que \( k=i\), \( k=j\) ou \( k\) différent de \( i\) et \( j\). Si \( k=i\) alors \( (P_{ij})_{im}=\delta_{mj}\) et si \( k\) est différent de \( i\) et \( j\) alors \( (P_{ij})_{mk}=\delta_{km}\) (troisième cas similaire au premier).
\end{proof}

Et la matrice \( AP_{12}\) est la \( A\) avec ses deux premières \emph{colonnes} échangées.

Avec ces notations, notre matrice \( (A|b)^{0'}\) est
\begin{equation}
    P_{12}(A|b)^{(0)}.
\end{equation}
Puis la matrice \( (A|b)^{(1')}\) est
\begin{equation}
    P_{23}(A|b)^{(1)}.
\end{equation}
Et la matrice \( P\) qui arrive dans \( PA=LU\) est la matrice \(P= P_{23}P_{21}\), qui est une matrice de permutations non élémentaire. Elle vaut :
\begin{equation}
    P=\begin{pmatrix}
        0    &   1    &   0    \\
        0    &   0    &   1    \\
        1    &   0    &   0
    \end{pmatrix}.
\end{equation}

\begin{lemma}[\cite{ooJZHZooLRqIsV}]        \label{LEMooYIYIooYhnaOt}
    Si \( i,j>k\) alors les matrices de permutation élémentaires ont la relation de «commutation» suivante avec les transformations gaussiennes :
    \begin{equation}
        M_k(x)P_{ij}=P_{ij}M_k\big(  P_{ij}(x) \big).
    \end{equation}
\end{lemma}

\begin{proof}
    Il suffit de calculer les éléments de matrice :
    \begin{equation}
        \big( P_{ij}M_k(x) \big)_{st}=(P_{ij})_{st}-\sum_m(P_{ij})_{sm}\tau_k(x)_{m}\delta_{kt},
    \end{equation}
    mais
    \begin{equation}
        \sum_m(P_{ij})_{sm}\tau_k(x)_{m}=\big( P_{ij}\tau_k(x) \big)_s=\tau_k\big( P_{ij}(x) \big)_s
    \end{equation}
    parce que \( i,j>k\) implique que dans \( P_{ij}\tau_k(x)\) nous inversons deux élément non nuls de \( \tau_k(x)\), tout en laissant le \( k\)\ieme élément. Le dénominateur ne change pas et il s'agit réellement d'une inversion de ligne. Donc
    \begin{equation}        \label{EQooIBVJooTOWCGT}
        \big( P_{ij}M_k(x) \big)_{st}=(P_{ij})_{st}-\tau_k\big( P_{ij}x \big)_s\delta_{kt}.
    \end{equation}
    De l'autre côté,
    \begin{equation}
        \big( M_k(y)P_{ij} \big)_{st}=(P_{ij})_{st}-\tau_k(y)_s(P_{ij})_{kt}.
    \end{equation}
    Mais comme \( i,j>k\) la \( k\)\ieme ligne de \( P_{ij}\) est la même que celle de la matrice unité, donc \( (P_{ij})_{kt}=\delta_{kt}\).
    \begin{equation}
        \big( M_k(y)P_{ij} \big)_{st}=(P_{ij})_{st}-\tau_k(y)_s\delta_{kt}.
    \end{equation}
    Cela correspond bien à \eqref{EQooIBVJooTOWCGT}.
\end{proof}

%+++++++++++++++++++++++++++++++++++++++++++++++++++++++++++++++++++++++++++++++++++++++++++++++++++++++++++++++++++++++++++
\section{Méthode de Gauss avec pivot partiel (décomposition PLU)}
%+++++++++++++++++++++++++++++++++++++++++++++++++++++++++++++++++++++++++++++++++++++++++++++++++++++++++++++++++++++++++++

%---------------------------------------------------------------------------------------------------------------------------
\subsection{L'idée}
%---------------------------------------------------------------------------------------------------------------------------

À chaque pas, nous faisons une permutation de ligne. Nous permutons à chaque pas la première ligne avec celle qui a le pivot le plus grand (en valeur absolue). Donc :
\begin{equation}
    (A|b)^{(0)}=\begin{pmatrix}
         2   &   1    &   3    &   11    \\
         4   &   3    &   10    &   28    \\
         -2   &   1    &   7    &   3
     \end{pmatrix}
\end{equation}
Nous commençons par déplacer des lignes :
\begin{equation}
    (A|b)^{(0')}=\begin{pmatrix}
         4   &   3    &   10    &   28    \\
         2   &   1    &   3    &   11    \\
         -2   &   1    &   7    &   3
     \end{pmatrix}.
\end{equation}
Les multiplicateurs sont \( l_{21}=1/2\) et \( l_{31}=-1/2\). Le fait est que les multiplicateurs ont toujours le plus grand dénominateur possible et nous avons alors toujours \( 0\leq | l_{ij} |\leq 1\), qui sont des nombres relativement petits, et bien représentés en mémoire.

Nous avons la nouvelle matrice
\begin{equation}
    (A|b)^{(1)}=\begin{pmatrix}
         4   &   3    &   10    &   28    \\
         0   &   -1/2    &   -2    &   -3    \\
         0   &   5/2    &   12    &   17
     \end{pmatrix}.
\end{equation}
Le pivot serait \( -1/2\). Nous cherchons un pivot plus grand en dessous de ce \( -1/2\) (et pas au dessus, sinon on casserait les zéros déjà trouvés).
Nous trouvons le \( 5/2\) qui est plus grand. Nous permutons donc les deux dernières lignes :
\begin{equation}
    (A|b)^{(1')}=\begin{pmatrix}
         4   &   3    &   10    &   28    \\
         0   &   5/2    &   12    &   17\\
         0   &   -1/2    &   -2    &   -3
     \end{pmatrix}
\end{equation}
où le pivot est maintenant \( l_{32}=-1/5\). La matrice suivante :
\begin{equation}
    (A|b)^{(1)}=\begin{pmatrix}
         4   &   3    &   10    &   28    \\
         0   &   5/2    &   12    &   17\\
         0   &   0   &   2/5    &   2/5
     \end{pmatrix}
\end{equation}

Dans ce cas, la matrice \( L\) n'est pas aussi simple à construire parce que nous avons permuté des choses. Dans ce cas, la matrice \( L\) est encore de la forme
\begin{equation}
    L=\begin{pmatrix}
        1    &   0    &   0    \\
        .    &   1    &   0    \\
        .    &   .    &   1
    \end{pmatrix}.
\end{equation}
Mais vu  que nous avons permuté les lignes \( 2\) et \( 3\) au deuxième pas, nous devons permuter \( l_{21}\) et \( l_{31}\) avant de remplir la matrice \( L\) avec les multiplicateurs :
\begin{equation}
    L=\begin{pmatrix}
        1    &   0    &   0    \\
        -1/2    &   1    &   0    \\
        1/2    &  -1/5    &   1
    \end{pmatrix}.
\end{equation}

Notons que ces \( L\) et \( U\) ne sont pas les mêmes que le \( L U\) obtenu sans pivot. Où est l'unicité ? Elle est que en fait maintenant nous n'avons pas \( A=LU\), mais
\begin{equation}
    PA=LU
\end{equation}
où \( P\) est une matrice de permutation.

%---------------------------------------------------------------------------------------------------------------------------
\subsection{Le théorème}
%---------------------------------------------------------------------------------------------------------------------------

\begin{proposition}[Méthode de Gauss avec pivot partiel\cite{ooJZHZooLRqIsV}]       \label{PROPooGCPAooDrlrGu}
    Soit une matrice inversible \( A\in\eM(n,\eC)\). Il existe
    \begin{itemize}
        \item une matrice de permutations \( P\)
        \item une matrice triangulaire inférieure de diagonale unitaire \( L\),
        \item une matrice triangulaire supérieure inversible \( U\)
    \end{itemize}
    telles que
    \begin{equation}
        PA=LU.
    \end{equation}
\end{proposition}
Notons que cette proposition ne demande que l'hypothèse d'inversibilité pour \( A\). Il n'y a pas d'hypothèses sur tous les mineurs comme c'était le cas avec Gauss sans pivot.

\begin{proof}
    Nous prouvons par récurrence qu'il existe des matrices \( Q_k\), \( E_1\), \ldots, \( E_k\) et \( A_k\) telles que
    \begin{equation}        \label{EQooOBXWooXxwXSe}
        Q_kA=E_1\ldots E_kA_k
    \end{equation}
    avec
    \begin{enumerate}
        \item
            \( Q_k\) est une matrice de permutation
        \item
            \( E_i\) est une transformation gaussienne sur la \( i\)\ieme colonne
        \item
            \( A_k\) est triangulaire supérieure jusqu'à la \( k\)\ieme colonne.
    \end{enumerate}

    Sachant que \( \det(Q_k)=\pm 1\), et que \( \det(E_i)=1\), le passage au déterminant dans \eqref{EQooOBXWooXxwXSe} nous donne \( \det(A_k)\neq 0\) et si nous notons \( \Omega_k(A)\) la matrice tronquée de \( A\), ne gardant que les entrées plus grandes que \( k\), nous avons
    \begin{equation}
        \det(A_k)=\prod_{i=1}^k(A_k)_ii\det\big( \Omega_{k+1}(A_k) \big).
    \end{equation}
    Donc  : \( (A_k)_{ii}\neq 0\) pour \( i\leq k\) et \( \det\big( \Omega_{k+1}(A_k) \big)\neq 0\).

    Pour fixer les idées, voici une image de \( k=2\) :
    \begin{equation}
    \input{auto/pictures_tex/Fig_FCUEooTpEPFoeQ.pstricks}
    \end{equation}

    Étant donné que \( \det\big( \Omega_{k+1}(A_k) \big)\neq 0\), parmi les nombres \( (A_k)_{i,k+1}\) (\( i\geq k+1\)), au moins un est non nul et nous posons \( r_{k+1}\) tel que \( | (A_k)_{r_{k+1},k+1} | \) soit maximum parmi ces éléments.

    Le nombre \( r_{k+1}\) est enregistré parce qu'il servira à écrire la matrice \( P\) plus tard. Les matrices \( E_i\) ne sont pas enregistrées, parce que nous verrons qu'elles vont encore changer. Seule la dernière sera enregistrée.

    La composante \( (k+1,k+1)\) de la matrice
    \begin{equation}
        P_{r_{k+1},k+1}A_k
    \end{equation}
    est non nulle et peut donc servir de pivot. Soit \( M_{k+1}\) la transformation gaussienne pour la \( (k+1)\)\ieme colonne de la matrice \( P_{r_{k+1},k+1}A_k\). La matrice
    \begin{equation}        \label{EQooCFIFooNDvPFE}
        A_{k+1}=M_{k+1}P_{r_{k+1},k+1}A_k
    \end{equation}
    est alors une matrice triangulaire supérieure jusqu'à la \( (k+1)\)\ieme colonne. En posant \( E_{k+1}=M_{k+1}^{-1}\) nous avons
    \begin{equation}
        P_{r_{k+1},k+1}E_{k+1}A_{k+1}=A_k,
    \end{equation}
    et nous nous sentons en droit de récrire l'équation de départ \eqref{EQooOBXWooXxwXSe} :
    \begin{equation}
        Q_kA=E_1\ldots E_kA_k=E_1\ldots E_kP_{r_{k+1},k+1}E_{k+1}A_{k+1}.
    \end{equation}
    Le lemme~\ref{LEMooYIYIooYhnaOt} nous permet de ramener la matrice \( P_{r_{k+1},k+1}\) en première position, quitte à modifier un peu (pas beaucoup) chacune des matrices \( E_i\) (\( i=1,\ldots, k\)). C'est pour cela que nous n'enregistrons pas les matrices \( E_i\). Nous avons donc
    \begin{equation}
        P_{r_{k+1},k+1}Q_kA=E'_1\ldots E'_kE_{k+1}A_{k+1}
    \end{equation}
    où
    \begin{itemize}
        \item Le produit \( P_{r_{k+1},k+1}Q_k\) est encore une matrice de permutation, et mieux : elle vaut
            \begin{equation}
                \prod_{i=1}^{k+1}P_{r_i,i}.
            \end{equation}
            Cela montre qu'il est suffisant d'enregistrer les nombres \( r_i\) pour reconstituer cette partie.
        \item
            La matrice \( E'_i\) est une transformation gaussienne pour la \( i\)\ieme colonne.
        \item
            La matrice $A_{k+1}$ est triangulaire supérieure jusqu'à la \( k+1\)\ieme colonne.
    \end{itemize}

    La récurrence est maintenant finie et nous pouvons écrire avec \( k=n\) :
    \begin{equation}        \label{EQooFUEUooHVPFwn}
        Q_nA=E_1\ldots E_nA_n
    \end{equation}
    où le produit \( E_1\ldots E_n\) est triangulaire inférieure et \( A_n\) est triangulaire supérieur.

    Maintenant nous enregistrons la matrice \( U=A_n\), le produit \( L=\prod_{i=1}^nE_n\) et les nombres \( r_i\) qui permettent de retrouver \( P\).
\end{proof}

Note : dans l'équation \eqref{EQooFUEUooHVPFwn} nous avons bien entendu massivement renommé les $E_i'$ en \( E_i\). En réalité la matrice \( E_1\) vient avec \( n\) primes sur la tête.

Dans les exemples~\ref{ExooNTECooXvTcoh}, \ref{EXooNVRNooJgQmQc} et~\ref{EXooNCRSooTfmPFr}, nous allons résoudre le système
\begin{equation}
    \begin{pmatrix}
        10^{-9}    &   1    \\
        1    &   1
    \end{pmatrix}\begin{pmatrix}
        x_1    \\
        x_2
    \end{pmatrix}=\begin{pmatrix}
        1    \\
        2
    \end{pmatrix}
\end{equation}
d'abord de façon exacte, et ensuite en supposant une machin ne tenant que \( 8\) chiffres significatifs en utilisant la méthode de Gauss avec ou sans pivot.

Commençons par voir comment se passe en pratique la décomposition \( PA=LU\) de Gauss avec pivot partiel.

\begin{example}     \label{EXooAZTDooTUXZJb}
    Décomposons la matrice
    \begin{equation}
        A=\begin{pmatrix}
            1    &   2    &   3    \\
            2    &   5    &   0    \\
            3    &   8    &   0
        \end{pmatrix}.
    \end{equation}
    Sur la première colonne, le plus grand nombre est \( 3\). Nous commençons par permuter la première et la troisième ligne en utilisant la matrice de permutation \( P_1=P_{3,1}\) et nous enregistrons \( r_1=3\). Nous avons alors la matrice
    \begin{equation}        \label{EQooJCCLooOZVajj}
        A_0'=\begin{pmatrix}
            3    &   8    &   0    \\
            2    &   5    &   0    \\
            1    &   3    &   3
        \end{pmatrix}.
    \end{equation}
    Pour trouver la matrice \( A_1\) nous suivons l'équation \eqref{EQooCFIFooNDvPFE}. Bien que le résultat net soit des combinaisons de lignes : \( L_2\to L_2-2L_1/3\) et \( L_3\to L_3-L_1/3\) (que nous pourrions savoir dès à présent), il est important de passer par la matrice gaussienne pour obtenir la matrice \( L_1\).

    La matrice de transformation gaussienne pour la première colonne de \eqref{EQooJCCLooOZVajj} est :
    \begin{equation}
        M_1=\begin{pmatrix}
            1    &   0    &   0    \\
            -2/3    &   1    &   0    \\
            -1/3    &   0    &   1
        \end{pmatrix}
    \end{equation}
    et \( L_1=M_1^{-1}\). Le lemme~\ref{LEMooFHZDooZiKdbr} nous dit comment calculer facilement cet inverse :
    \begin{equation}
        L_1=\begin{pmatrix}
            1    &   0    &   0    \\
            2/3    &   1    &   0    \\
            1/3    &   0    &   1
        \end{pmatrix}
    \end{equation}
    En suivant l'équation \eqref{EQooCFIFooNDvPFE} nous posons \( A_1=M_1A'_0\) :
    \begin{equation}        \label{EQooUBRPooRJbCYn}
        A_1=
        \begin{pmatrix}
            1    &   0    &   0    \\
            -2/3    &   1    &   0    \\
            -1/3    &   0    &   1
        \end{pmatrix}
        \begin{pmatrix}
            3    &   8    &   0    \\
            2    &   5    &   0    \\
            1    &   3    &   3
        \end{pmatrix}=
        \begin{pmatrix}
            3    &   8    &   0    \\
            0    &   -1/3    &   0    \\
            0    &   -2/3    &   3
        \end{pmatrix}
    \end{equation}
    et nous avons
    \begin{equation}
        Q_1A=L_1A_1
    \end{equation}
    où \( L_1\), \( A_1\) et \( r_1=3\) sont enregistrés. La matrice \( Q_1\) peut être retrouvée en sachant \( r_1\) parce que \( P\) est la matrice de permutation \( P_{r_1,1}\).

    Nous travaillons maintenant sur la deuxième colonne de \( A_1\). Le plus grand élément en valeur absolue (sur ou sous la diagonale) est \( -2/3\). Nous posons \( r_2=3\) et
    \begin{equation}
        A'_1=\begin{pmatrix}
            3    &   8    &   0    \\
            0    &   -2/3    &   3    \\
            0    &   -1/3    &   0
        \end{pmatrix}
    \end{equation}
    et la matrice gaussienne pour la deuxième colonne est
    \begin{equation}
        M_2=\begin{pmatrix}
            1    &   0    &   0    \\
            0    &   1    &   0    \\
            0    &   -1/2    &   1
        \end{pmatrix}
    \end{equation}
    Le \( -1/2\) provient du calcul \( -\big( (-1/3)/(-2/3) \big)\). L'inverse de cette matrice est facile :
    \begin{equation}
        L_2=\begin{pmatrix}
            1    &   0    &   0    \\
            0    &   1    &   0    \\
            0    &   1/2    &   1
        \end{pmatrix}
    \end{equation}
    et la matrice suivante à enregistrer est
    \begin{equation}
        A_2=M_2P_{3,2}A_1=M_2A'_1=\begin{pmatrix}
            3    &   8    &   0    \\
            0    &   -2/3    &   3    \\
            0    &   0    &   -3/2
        \end{pmatrix}.
    \end{equation}
    Notons toutefois que pour calculer cette matrice, seul le dernier élément demande un calcul. La première colonne ne change pas (par construction), la seconde gagne un zéro en dernière ligne (la matrice \( M_2\) sert à ça) et sur la dernière colonne, seule la dernière ligne est sujette à changement.

    Avec la matrice \( A_2\), la trigonalisation supérieure est faite. La décomposition n'est cependant pas terminée. Nous devons encore trouver la partie triangulaire inférieure. Nous en sommes à
    \begin{equation}
        Q_1A=L_1A_1=L_1P_{3,2}L_2A
    \end{equation}
    où \( Q_1\) est la première matrice de permutation.

    Utilisant le lemme~\ref{LEMooYIYIooYhnaOt}, il est facile de permuter \( L_{1}\) avec \( P_{3,2}\) :
    \begin{equation}
        L_1P_{3,2}=P_{3,2}
        \underbrace{
        \begin{pmatrix}
            1    &   0    &   0    \\
            1/3    &   1    &   0    \\
            2/3    &   0    &   1
        \end{pmatrix}}_{L'_1}
    \end{equation}
    Nous avons donc
    \begin{equation}
        P_{3,2}P_{3,1}A=L'_1L_2A
    \end{equation}
    Deux multiplications matricielles plus tard nous terminons :
    \begin{equation}
        PA=LU
    \end{equation}
    avec
    \begin{equation}
        \begin{aligned}[]
        P&=\begin{pmatrix}
            0    &   0    &   1    \\
            1    &   0    &   0    \\
            0    &   1    &   0
        \end{pmatrix},&
        L&=\begin{pmatrix}
            1    &   0    &   0    \\
            1/3    &   1    &   0    \\
            2/3    &   1/2    &   1
        \end{pmatrix},&
        U&=A_2=\begin{pmatrix}
            3    &   8    &   0    \\
            0    &   -2/3    &   3    \\
            0    &   0    &   -3/2
        \end{pmatrix}.
        \end{aligned}
    \end{equation}
\end{example}

Notons que Sage utilise la méthode de Gauss avec pivots :
\lstinputlisting{tex/sage/sageSnip006.sage}
Mais attention : Sage crée une décomposition \( A=PLU\) et non \( PA=LU\). D'où le fait que la matrice de permutations de Sage est l'inverse de celle donnée ici.

%---------------------------------------------------------------------------------------------------------------------------
\subsection{D'un point de vue algorithmique}
%---------------------------------------------------------------------------------------------------------------------------

\begin{probleme}
    Je ne suis pas certain de l'optimalité de ce que je raconte ici. Je décris simplement ce que j'ai fait pour écrire mon programme \href{https://github.com/LaurentClaessens/finitediff}{finitediff}.

    Si vous êtes expert en calcul numérique, n'hésitez pas à donner votre avis.
\end{probleme}

Nous décrivons à présent la décomposition \( A=PLU\) (du théorème~\ref{PROPooGCPAooDrlrGu}, avec le \( P \) à droite). En suivant l'exemple~\ref{EXooAZTDooTUXZJb} nous voyons assez bien comment créer les matrices \( U\) et \( P\) au fur et à mesure. La construction de \( L\) est peut-être moins évidente.

Écrivons un exemple très explicite pour
\begin{equation}
    A=\begin{pmatrix}
        2    &   1    &   3    \\
        4    &   3    &   10    \\
        -2    &   1    &   7
    \end{pmatrix}.
\end{equation}
Nous commençons par permuter des lignes pour avoir un grand pivot :
\begin{equation}
    P_{12}A=\begin{pmatrix}
        4    &   3    &   10    \\
        2    &   1    &   3    \\
        -2    &   1    &   7
    \end{pmatrix}.
\end{equation}
Et nous effectuons l'élimination avec la matrice
\begin{equation}
    M_1=\begin{pmatrix}
        1    &   0    &   0    \\
        -1/2    &   1    &   0    \\
        1/2    &   0    &   1
    \end{pmatrix}.
\end{equation}
Cela donne le premier résultat :
\begin{equation}        \label{EQooKTBLooHeOkgk}
    M_1P_{12}A=\begin{pmatrix}
        4    &   3    &   10    \\
        0    &   -1/2    &   -2    \\
        0    &   5/2    &   12
    \end{pmatrix}
\end{equation}
Nous continuons avec \( P_{23}\) pour avoir un nouveau grand pivot :
\begin{equation}
    P_{23}M_1P_{12}A=\begin{pmatrix}
        4    &   3    &   10    \\
        0    &   5/2    &   12\\
        0    &   -1/2    &   -2
    \end{pmatrix}.
\end{equation}
Nous utilisons la matrice
\begin{equation}
    M_2=\begin{pmatrix}
        1    &   0    &   0    \\
        0    &   1    &   0    \\
        0    &   1/5    &   1
    \end{pmatrix}
\end{equation}
et au final :
\begin{equation}
    M_2P_{23}M_1P_{12}A=\begin{pmatrix}
        4    &   3    &   10    \\
        0    &   5/2    &   12    \\
        0    &   0    &   2/5
    \end{pmatrix}=U.
\end{equation}
L'égalité obtenue est
\begin{equation}
    M_2P_{23}M_1P_{12}A=U.
\end{equation}
Pour avoir la décomposition \( PLU\) il faut écrire
\begin{equation}        \label{EQooWKUYooUBQYtc}
    A=P_{12}M_1^{-1}P_{23}M_2^{-1}U,
\end{equation}
et permuter \( P_{23}\) avec \( M_1^{-1}\), ce qui est facile par le lemme~\ref{LEMooYIYIooYhnaOt}.

\begin{remark}
    Nous ne devons permuter la matrice \( M_k\) avec une matrice de permutations qu'à partir de la deuxième étape. En effet l'équation \eqref{EQooKTBLooHeOkgk} revient à
    \begin{equation}
        A=M_1^{-1}P_{12}m_U
    \end{equation}
    qui est dans le bon ordre. Ce n'est qu'à partir de la seconde étape que des matrices de permutations apparaissent à droite des matrices gaussiennes.
\end{remark}

Cependant dans un cas \( 4\times 4\), cette méthode deviendrait fastidieuse parce que nous aurions encore des étapes à faire. En repartant de \eqref{EQooWKUYooUBQYtc}, mais avec \( m_U\) (la matrice pas encore tout à fait triangularisée) au lieu de \( U\), nous aurons, pour un certain \( k>3\) :
\begin{equation}
    M_3P_{3k}M_2P_{23}M_1P_{12}A=U,
\end{equation}
ce qui fait :
\begin{equation}
    A=P_{12}M_1^{-1}P_{23}M_2^{-1}P_{3k}M_3^{-1}U.
\end{equation}
Tous les \( P_{ij}\) peuvent être mis à gauche parce que leurs indices sont toujours strictement supérieurs à ceux des \( M_l\) placés devant eux. Mais c'est fastidieux.

Nous allons donc permuter à chaque étape pour ne retenir que l'important. Si à une certaine étape nous avons
\begin{equation}
    A=P_{1,r_1}\ldots P_{k,r_k}M_1^{-1}\ldots M_k^{-1} m_U
\end{equation}
avec
\begin{equation}
    m_U=P_{k+1,r_{k+1}}M_{k+1}^{-1}U
\end{equation}
alors nous allons directement permuter \( P_{k+1,r_{k+1}}\) avec tous les \( M_i^{-1}\). Si nous notons \( P_k\) la permutation (pas élémentaire) à l'étape \( k\) et \( L_k\) la matrice triangulaire inférieure à de l'étape \( k\),
\begin{equation}
    A=P_kL_km_U=P_kL_kP_{k+1,r_{k+1}}M_{k+1}^{-1}m_U'.
\end{equation}
Nous enregistrons alors \( P_{k+1}=P_kP_{k+1,r_{k+1}}\) et pour \( L_{k+1}\) nous partons de \( L_k\) et nous faisons deux opérations suivantes :
\begin{itemize}
    \item nous permutons, sur ses colonnes non triviales, les indices \( k+1\) et \( r_{k+1}\),
    \item nous multiplions par \( M_{k+1}^{-1}\), ce qui revient à simplement lui ajouter une colonne non triviale.
\end{itemize}
Notons que \( r_{k+1}\geq k+1\), de telle sorte que sur les colonnes non triviales (qui sont jusqu'au numéro \( k\)), la permutation des lignes \( k+1\) et \( r_{k+1}\) ne change pas l'aspect de la matrice : elle reste multi-gaussienne de dernière colonne \( k\).

%---------------------------------------------------------------------------------------------------------------------------
\subsection{Exemples}
%---------------------------------------------------------------------------------------------------------------------------

Nous nous lançons dans la résolution du système
\begin{equation}
    \begin{pmatrix}
        10^{-9}    &   1    \\
        1    &   1
    \end{pmatrix}\begin{pmatrix}
        x_1    \\
        x_2
    \end{pmatrix}=\begin{pmatrix}
        1    \\
        2
    \end{pmatrix}.
\end{equation}

\begin{example}     \label{ExooNTECooXvTcoh}
    Nous commençons de façon exacte, par la méthode de Gauss sans pivot. La première transformation gaussienne est
    \begin{equation}
        E_1=\begin{pmatrix}
            1    &   0    \\
            -10^9    &   1
        \end{pmatrix}
    \end{equation}
    et nous calculons
    \begin{equation}
        E_1A=\begin{pmatrix}
            1    &   0    \\
            -10^9    &   1
        \end{pmatrix}\begin{pmatrix}
            10^{-9}    &   1    \\
            1    &   1
        \end{pmatrix}=
        \begin{pmatrix}
            10^{-9}    &   1    \\
            0    &   1-10^{-9}
        \end{pmatrix}.
    \end{equation}
    Vu que cette dernière est triangulaire supérieure, nous avons fini la méthode de Gauss et \( U=E_1A\). En ce qui concerne la matrice \( L\), elle est donnée par \( L=E_1^{-1}\), c'est-à-dire
    \begin{equation}
        L=\begin{pmatrix}
            1    &   0    \\
            -10^9    &   1
        \end{pmatrix}^{-1}=
        \begin{pmatrix}
            1    &   0    \\
            10^9    &   1
        \end{pmatrix}.
    \end{equation}
    Au final nous avons la décomposition \( A=LU\) exacte suivante :
    \begin{equation}
        \begin{aligned}[]
            L&=\begin{pmatrix}
                1    &   0    \\
                10^9    &   1
            \end{pmatrix}&U&=\begin{pmatrix}
                10^{-9}    &   1    \\
                0    &   1-10^9
            \end{pmatrix}.
        \end{aligned}
    \end{equation}
    Résoudre le système \( Ax=b\) revient à résoudre \( LUx=b\) et donc résoudre successivement les systèmes
    \begin{subequations}
        \begin{numcases}{}
            Ly=b\\
            Ux=y.
        \end{numcases}
    \end{subequations}
    D'abord le système
    \begin{equation}
        \begin{pmatrix}
            1    &   0    \\
            10^9    &   1
        \end{pmatrix}\begin{pmatrix}
            y_1    \\
            y_2
        \end{pmatrix}=\begin{pmatrix}
            1    \\
            2
        \end{pmatrix}
    \end{equation}
    donne \( y_1=1\) et \( y_2=2-10^9\).

    Ensuite nous résolvons
    \begin{equation}
        \begin{pmatrix}
            10^{-9}    &   1    \\
            0    &   1-10^9
        \end{pmatrix}\begin{pmatrix}
            x_1    \\
            x_2
        \end{pmatrix}=\begin{pmatrix}
            1    \\
            2-10^9
        \end{pmatrix}.
    \end{equation}
    Cela donne
    \begin{equation}
        \begin{pmatrix}
            x_1    \\
            x_2
        \end{pmatrix}=\begin{pmatrix}
            -\frac{ 10^9 }{ 1-10^9 }    \\
            \frac{ 2-10^9 }{ 1-10^9 }
        \end{pmatrix}\simeq\begin{pmatrix}
            1    \\
            1
        \end{pmatrix}.
    \end{equation}

    C'est également le résultat que trouve Sage :
    \lstinputlisting{tex/sage/sageSnip007.sage}
\end{example}

\begin{example}[\cite{ooJZHZooLRqIsV}]     \label{EXooNVRNooJgQmQc}
    Nous recommençons tout le calcul avec une précision limitée à \( 8\) chiffres significatifs, sans pivot.

    Nous avons à nouveau la transformation gaussienne
    \begin{equation}
        E_1=\begin{pmatrix}
            1    &   0    \\
            -10^9    &   1
        \end{pmatrix},
    \end{equation}
    mais pour calculer \( U\) nous effectuons le produit matriciel
    \begin{equation}
        U=E_1A=\begin{pmatrix}
            1    &   0    \\
            -10^9    &   1
        \end{pmatrix}\begin{pmatrix}
            10^{-9}   &   1    \\
            1    &   1
        \end{pmatrix}=\begin{pmatrix}
            10^{-9}    &   1    \\
            0    &   *
        \end{pmatrix}.
    \end{equation}
    Nous détaillons à présent le calcul de l'élément noté \( *\). Le calcul de \( 10^9\ominus 1)\) donne
    \begin{equation}
        999999999=9.99999999\times 10^{8},
    \end{equation}
    mais la précision étant limitée à \( 8\) chiffres, un arrondit arrive. Étant donné que le premier chiffres supprimé est un \( 9\) nous retombons sur \( 10^9\), et donc notre machine à précision limitée donnera
    \begin{equation}
        U=\begin{pmatrix}
            10^{-9}    &   1    \\
            0    &   -10^{9}
        \end{pmatrix}.
    \end{equation}
    Ensuite le calcul de \( L=E_1^{-1}\) ne cause pas de problèmes :
    \begin{equation}
        L=\begin{pmatrix}
            1    &   0    \\
            -10^9    &   1
        \end{pmatrix}.
    \end{equation}
    Maintenant il s'agit de résoudre les systèmes \( Ly=b\) et \( Ux=y\). Du système
    \begin{equation}
        \begin{pmatrix}
            1    &   0    \\
            10^9    &   1
        \end{pmatrix}\begin{pmatrix}
            y_1    \\
            y_2
        \end{pmatrix}=\begin{pmatrix}
            1    \\
            2
        \end{pmatrix}
    \end{equation}
    nous tirons tout de suite \( y_1=1\) et ensuite \( 10^9+y_2=2\), c'est-à-dire \( y_2=2-10^9\), qui en précision limitée donne encore \( y_2=-10^9\). À résoudre maintenant :
    \begin{equation}
        \begin{pmatrix}
            10^{-9}    &   1    \\
            0    &   -10^9
        \end{pmatrix}\begin{pmatrix}
            x_1    \\
            x_2
        \end{pmatrix}=\begin{pmatrix}
            1    \\
            -10^9
        \end{pmatrix}.
    \end{equation}
    Cela donne immédiatement \( x_2=1\) et ensuite
    \begin{equation}
        10^{-9}x_1+1=1,
    \end{equation}
    donc \( x_1=0\). La solution trouvée est
    \begin{equation}        \label{EQooBGWEooVGSVoe}
        \begin{pmatrix}
            x_1    \\
            x_2
        \end{pmatrix}=\begin{pmatrix}
            0    \\
            1
        \end{pmatrix},
    \end{equation}
    qui est complètement faux au niveau de la première variable.
\end{example}

\begin{example}[\cite{ooJZHZooLRqIsV}]     \label{EXooNCRSooTfmPFr}
    Nous résolvons encore le même système en précision limitée, mais en utilisant cette fois la méthode de Gauss avec pivot partiel.

    Le plus grand élément de la première colonne est \( 1\); nous utilisons donc la permutation \( P_{1,2}\) :
    \begin{equation}
        P_{1,2}A=\begin{pmatrix}
            1    &   1    \\
            10^{-9}    &       1
        \end{pmatrix}.
    \end{equation}
    La matrice de transformation gaussienne pour la première colonne de cette matrice est
    \begin{equation}
        M_1=\begin{pmatrix}
            1    &   0    \\
            -10^{-9}    &  1
        \end{pmatrix}
    \end{equation}
    et nous posons
    \begin{equation}
        A_1=M_1P_{1,2}A=
        \begin{pmatrix}
            1    &   0    \\
            -10^{-9}    &   1
        \end{pmatrix}
        \begin{pmatrix}
            1    &   1    \\
            10^{-9}    &   1
        \end{pmatrix}
        =\begin{pmatrix}
            1    &   1    \\
            0    &   -10^{-9}+1
        \end{pmatrix}=
        \begin{pmatrix}
            1    &   1    \\
            0    &   1
        \end{pmatrix}=U
    \end{equation}
    où un arrondi a eu lieu pour \( -10^{-9}+1=1\). En inversant \( M_1\) nous avons
    \begin{equation}
        L_1=M_1^{-1}=\begin{pmatrix}
            1    &   0    \\
            10^{-9}    &   1
        \end{pmatrix}.
    \end{equation}
    La décomposition est
    \begin{equation}
        A=\underbrace{\begin{pmatrix}
            0    &   1    \\
            1    &   0
        \end{pmatrix}
    }_{P}
    \underbrace{
        \begin{pmatrix}
            1    &   0    \\
            10^{-9}    &   1
        \end{pmatrix}}_{L}
        \underbrace{
        \begin{pmatrix}
            1    &   1    \\
            0    &   1
        \end{pmatrix}}_{U}
    \end{equation}
    Le moment de résoudre est venu. Vu que \( PLUx=b\) nous devons résoudre les systèmes
    \begin{subequations}
        \begin{numcases}{}
            Pz=b\\
            Ly=z\\
            Ux=y.
        \end{numcases}
    \end{subequations}
    Pour \( z\) c'est facile :
    \begin{equation}
        z=\begin{pmatrix}
            2    \\
            1
        \end{pmatrix}.
    \end{equation}
    Pour $y$ il y a un arrondi :
    \begin{equation}
        \begin{pmatrix}
            1    &   0    \\
            10^{-9}    &   1
        \end{pmatrix}\begin{pmatrix}
            y_1    \\
            y_2
        \end{pmatrix}=\begin{pmatrix}
            2    \\
            1
        \end{pmatrix}.
    \end{equation}
    Tout de suite : \( y_1=2\) et ensuite \( 2\times 10^{-9}+y_2=1\), ce qui donne \( y_2=1\ominus 2\times 10^{-9}=1\). Donc
    \begin{equation}
        y=\begin{pmatrix}
            2    \\
            1
        \end{pmatrix}.
    \end{equation}
    Et enfin pour \( x\) c'est le système
    \begin{equation}
        \begin{pmatrix}
            1    &   1    \\
            0    &   1
        \end{pmatrix}\begin{pmatrix}
            x_1    \\
            x_2
        \end{pmatrix}=\begin{pmatrix}
            2    \\
            1
        \end{pmatrix}.
    \end{equation}
    Nous avons \( x_2=1\) et ensuite \( x_1+1=2\) c'est-à-dire \( x_1=1\). Au final la solution trouvée est
    \begin{equation}
        x=\begin{pmatrix}
            1    \\
            1
        \end{pmatrix}.
    \end{equation}
    Cette solution est considérablement meilleure que \eqref{EQooBGWEooVGSVoe}.
\end{example}

\begin{normaltext}
    L'utilisation du pivot non seulement assure le fait que la trigonalisation va bien se passer (on évite les zéros en pivot), mais aussi et surtout, en choisissant de prendre le plus grand pivot possible, nous obtenons une meilleur stabilité numérique.
\end{normaltext}

\input{182_numerique}

\chapter{Méthode des différences finies}
\input{193_numerique}

\chapter{Variables aléatoires et théorie des probabilités}
% This is part of Mes notes de mathématique
% Copyright (c) 2011-2019
%   Laurent Claessens
% See the file fdl-1.3.txt for copying conditions.

%+++++++++++++++++++++++++++++++++++++++++++++++++++++++++++++++++++++++++++++++++++++++++++++++++++++++++++++++++++++++++++
\section{Espace de probabilité}
%+++++++++++++++++++++++++++++++++++++++++++++++++++++++++++++++++++++++++++++++++++++++++++++++++++++++++++++++++++++++++++

\begin{definition}
    Une \defe{mesure de probabilité}{mesure!probabilité} sur un espace mesurable\footnote{Espace mesurable :~\ref{DefjRsGSy}, mesure positive :~\ref{DefBTsgznn}.} \( (\Omega,\tribA)\) est une mesure positive telle que \( P(\Omega)=1\). Dans ce cas, le triple \( (\Omega,\tribA,P)\) est un \defe{espace de probabilité}{espace!de probabilité}.
\end{definition}

Un point \( \omega\in\Omega\) est une \defe{observation}{observation}, une partie mesurable \( A\in\tribA\) est un \defe{événement}{evenement@événement}. L'ensemble \( A\cup B\) représente l'événement \( A\) ou \( B\) tandis que l'ensemble \( A\cap B\) représente l'événement \( A\) et \( B\).

Si les \( A_n\) sont des événements, nous avons défini en~\ref{ooEEQJooRMFzVR} limite supérieure et la limite inférieure de la suite \( A_n\) par
\begin{equation}
    \limsup_{n\to\infty}A_n=\bigcap_{n\geq 1}\bigcup_{k\geq n}A_k
\end{equation}
et
\begin{equation}
    \liminf_{n\to\infty}A_n=\bigcup_{n\geq 1}\bigcap_{k\geq n}A_k
\end{equation}
Si \( \omega\in\liminf A_n\), alors \( \omega\) réalise tous les \( A_n\) sauf un nombre fini.

Nous avons
\begin{equation}
    \limsup A_n=\{ \omega\in\Omega\tq \omega\in A_n\text{pour une infinité de } n \}.
\end{equation}

\begin{theorem}[Borel-Cantelli]\index{théorème!Borel-Cantelli}
    Si
    \begin{equation}
        \sum_{n=1}^{\infty}P(A_n)<\infty
    \end{equation}
    alors \( P(\limsup A_n)=0\).
\end{theorem}
%TODO : une conséquence de Borel-Cantelli a l'air d'être le théorème des nombres normaux,
% prouvé sur la page https://fr.wikipedia.org/wiki/Nombre_normal

\begin{proof}
    La condition \( \sum_{n\geq 1}P(A_n)<\infty\) signifie que la fonction
    \begin{equation}
        \varphi=\sum_{n\geq 1}\caract_{A_n}
    \end{equation}
    est \( P\)-intégrable. Par conséquent, elle est finie presque partout (au sens de \( P\)), c'est-à-dire
    \begin{equation}
        P(\varphi=\infty)=0.
    \end{equation}
    Les points \( \omega\) sur lesquels \( \varphi(\omega)=\infty\) sont ceux tels que
    \begin{equation}
        \sum_{n\geq 1}\caract_{A_n}(\omega)=\infty,
    \end{equation}
    c'est-à-dire les \( \omega\) qui appartiennent à une infinité d'ensembles \( A_n\), ou encore les \( \omega\in\limsup A_n\). Nous avons donc montré que
    \begin{equation}
        \{ \omega\tq \varphi(\omega)=\infty \}=\{ \omega\in\Omega\tq \omega\in A_n\text{pour une infinité de } n \}=\limsup A_n.
    \end{equation}
    Or l'hypothèse signifie que la probabilité du membre de gauche est nulle.
\end{proof}

\begin{corollary}
    Si \( \sum_{n=1}^{\infty}P(\complement A_n)<\infty\), alors presque surement tous les \( B_n\) sont réalisés à l'exception d'un nombre fini.
\end{corollary}

%+++++++++++++++++++++++++++++++++++++++++++++++++++++++++++++++++++++++++++++++++++++++++++++++++++++++++++++++++++++++++++
\section{Variables aléatoires}
%+++++++++++++++++++++++++++++++++++++++++++++++++++++++++++++++++++++++++++++++++++++++++++++++++++++++++++++++++++++++++++

\begin{definition}
    Une \defe{variable aléatoire}{variable aléatoire} est une application mesurable
    \begin{equation}
        X\colon (\Omega,\tribA)\to (\eR^d,\Borelien(\eR^d)).
    \end{equation}
\end{definition}
Nous convenons que \( \eR^1=\bar\eR\), c'est-à-dire que dans le cas où la variable aléatoire \( X\) est réelle, nous acceptons les valeurs \( \pm\infty\).

\begin{definition}
    Une variable aléatoire réelle \( X\) est \defe{absolument continue}{variable aléatoire!absolument continue} s'il existe une fonction positive et intégrable \( f\colon \eR\to \eR\) telle que pour tout intervalle \( I\subset\eR\),
    \begin{equation}
        P(X\in I)=\int_If(t)dt.
    \end{equation}
    Nous disons alors que \( f\) est la \defe{densité}{densité!d'une variable aléatoire} de \( X\).
\end{definition}
Cela ne devrait pas être sans rappeler la définition~\ref{DefAbsoluCont}.

%---------------------------------------------------------------------------------------------------------------------------
\subsection{Indépendance}
%---------------------------------------------------------------------------------------------------------------------------

La définition suivante vient de l'instructive motivation de \cite{CourgGudRennes}. La définition d'indépendance de deux événements se généralise à \( n\) événements de la façon suivante.
\begin{definition}      \label{DEFooVYCUooKWvReO}
    Nous disons que les événements \( A_1,\ldots,A_n\) sont \defe{indépendants}{indépendance!événements} si pour tout choix \( \{ i_1,\ldots,i_k \}\subset\{ 1,\ldots,n \}\) nous avons
    \begin{equation}
        P(A_{i_1}\cap\ldots\cap A_{i_k})=P(A_{i_1})\ldots P(A_{i_k}).
    \end{equation}
    Les sous tribus \( \tribA_1,\ldots,\tribA_n\) sont \defe{indépendantes}{indépendance!sous tribus} si pour tout choix \( A_i\in \tribA_i\), les événements \( A_i\) sont indépendants.
\end{definition}

\begin{example}
    Soit \( \Omega=\mathopen[ 0 , 1 \mathclose]\times \mathopen[ 0 , 1 \mathclose]\) muni de la mesure de Lebesgue. Soient \( A=\mathopen[ 0 , a \mathclose]\times \mathopen[ 0 , 1 \mathclose]\) et \( B=\mathopen[ 0 , 1 \mathclose]\times \mathopen[ 0 , b \mathclose]\). Nous avons \( P(A)=a\) et \( P(B)=b\) ainsi que \( P(A\cup B)=ab\).
\end{example}

\begin{lemma}       \label{LemTribIndepProdProb}
    Les tribus \( \tribA_1,\ldots,\tribA_n\) sont indépendantes si et seulement si
    \begin{equation}
        P(A_1\cap\ldots\cap A_n)=P(A_1)\ldots P(A_n)
    \end{equation}
    pour tout \( A_i\in\tribA_i\).
\end{lemma}

\begin{proof}
    L'implication dans le sens direct découle immédiatement des définitions.

    Nous supposons avoir un choix \( (A_i)_{i=1,\ldots,n}\) avec \( A_i\in\tribA_i\) et nous devons montrer que ces événements sont indépendants, c'est-à-dire que si \( J\subset\{ 1,\ldots,n \}\) alors les événements \( (A_j)_{j\in J}\) sont indépendants. Sans perte de généralité, nous pouvons supposer que si \( i\notin J\), \( A_i=\Omega\). Alors nous avons
    \begin{equation}
        P\big( \bigcap_{j\in J}A_j \big)=P\big( \bigcap_{i=1}^nA_i \big)=\prod_{i=1}^nP(A_i)=\prod_{j\in J}P(A_j)
    \end{equation}
    parce que \( P(A_i)=P(\Omega)=1\) lorsque \( i\) n'est pas dans \( J\).
\end{proof}

Si \( A\) est un événement, la \defe{tribu engendrée}{tribu!engendrée!par un événement} par \( A\) est
\begin{equation}
    \sigma(A)=\{ \emptyset,A,\complement A,\Omega \}.
\end{equation}

Soit \( X\colon \Omega\to \eR^d\) une variable aléatoire. Conformément à la définition~\ref{DefNOJWooLGKhmJ}, la \defe{tribu engendrée}{tribu!engendrée!par une variable aléatoire} est\index{engendré!tribu!par une variable aléatoire}
\begin{equation}
    \tribA_X=\{ X^{-1}(B)\tq B\in\Borelien(\eR^d) \}.
\end{equation}
Cela est la plus petite tribu sous tribu de \( \tribA\) pour laquelle \( X\) est mesurable. Elle sera aussi (le plus) souvent notée \( \sigma(X)\)\nomenclature[P]{\( \sigma(X)\)}{La tribu engendrée par la variable aléatoire \( X\)}.

\begin{definition}  \label{DefNJUkotc}
    Nous disons que les variables aléatoires \( X_k\colon \Omega\to \eR^d\) sont \defe{indépendantes}{indépendance!variables aléatoires} lorsque les tribus engendrées \( \tribA_{X_1},\ldots,\tribA_{X_n}\) le sont.
\end{definition}

\begin{remark}
     Il n'a de sens de dire que \( X_1\) et \( X_2\) sont indépendants que si \( X_1\) et \( X_2\) sont des applications dont l'espace de départ est identique.

     Si nous voulons modéliser le jet de deux pièce indépendantes, le mauvais choix est de faire \( \Omega=\{ 0,1 \}\), y mettre la mesure d'équiprobabilité, et de considérer les deux variables aléatoires
     \begin{equation}
         X_i(\omega)=\begin{cases}
             f   &   \text{si } \omega=0\\
             p   &    \text{si } \omega=1.
         \end{cases}
     \end{equation}
     Ces deux variables sont évidemment pas indépendantes. Il faut poser \( \Omega=\{ 0,1 \}\times \{ 0,1 \}\), y mettre la mesure d'équiprobabilité et poser
     \begin{equation}
         X_1(x,y)=\begin{cases}
             f   &   \text{si } x=0\\
             p   &    \text{si } x=1
         \end{cases},
     \end{equation}
     \begin{equation}
         X_2(x,y)=\begin{cases}
             f   &   \text{si } y=0\\
             p   &    \text{si } y=1
         \end{cases},
     \end{equation}
     Ces variables aléatoires sont indépendantes. Par exemple
     \begin{subequations}
         \begin{align}
             X_1^{-1}\{ p \}=\{ (1,0),(1,1) \}\\
             X_2^{-1}\{ p \}=\{ (0,1),(1,1) \}
         \end{align}
     \end{subequations}
     et on a bien
     \begin{equation}
         P\big( X_1^{-1}\{ p \}\cap X_2^{-1}\{ p \} \big)=P\{ (1,1) \}=\frac{1}{ 4 }
     \end{equation}
     ainsi que
     \begin{subequations}
         \begin{align}
             P\{ X_p^{-1}(p) \}=\frac{ 1 }{2}
         \end{align}
     \end{subequations}
     pour \( i=1\) et \( i=2\).
\end{remark}

\begin{proposition} \label{PropMLbfRTk}
    Soient \( (X_k\colon \Omega\to \eR^{d_k})\) des variables aléatoires indépendantes.
    \begin{enumerate}
        \item
            Si \( B_k\in \Borelien(\eR^{d_k})\). Alors
    \begin{equation}
        P(X_k\in B_k\forall k\leq n)=P(X_1\in B_1)\ldots P(X_n\in B_n).
    \end{equation}
\item\label{ItemHRjuTTii}
    Les événements \( \{   X_i\in B_i   \}\) sont indépendants.
\item\label{ItemHRjuTTiii}
    Les tribus engendrées par des \( X_i\) et d'autres sont indépendantes. Plus précisément, si \( I\) et \( J\) sont deux ensembles disjoints de \( \eN\) alors les tribus
    \begin{equation}
        \sigma(  \{ X_i,i\in I \}  )
    \end{equation}
    et
    \begin{equation}
        \sigma(  \{ X_i,i\in J \}  )
    \end{equation}
    sont indépendantes.
    \end{enumerate}

\end{proposition}

\begin{proof}
    Lorsque nous écrivons \( X_i\in B_i\), nous parlons de l'événement
    \begin{equation}
        (X_i\in B_i)=\{ \omega\in\Omega\tq X_i(\omega)\in B_i \}=X_i^{-1}(B_i)\in \tribA_{X_i}.
    \end{equation}
    Vu que par hypothèse les tribus \( (\tribA_i)\) sont indépendantes, le lemme~\ref{LemTribIndepProdProb} nous montre que
    \begin{equation}
        P\big( \bigcap_{i=1}^nX_i\in B_i \big)=\prod_iP(X_i\in B_i).
    \end{equation}
    Il reste à voir que l'ensemble \( X_i^{-1}(B_i)\) fait partie de la tribu \( \tribA\) de départ. Cela est la définition du fait que l'application \( X_i\) soit une variable aléatoire : elle doit être mesurable en tant qu'application
    \begin{equation}
        X_i\colon (\Omega,\tribA)\to (\eR^d,\Borelien(\eR^d)).
    \end{equation}

    Les affirmations~\ref{ItemHRjuTTii} et~\ref{ItemHRjuTTiii} ne sont que des façons alternatives d'exprimer la même chose.
\end{proof}

\begin{lemma}       \label{LemIndepEvenCompl}
    Les événements \( (A_i)_{i=0,\ldots,n}\) sont indépendants si et seulement si les événements que nous obtenons en remplaçant certains des \( A_i\) par \( \complement A_i\) le sont.
\end{lemma}

\begin{proof}
    Sans perte de généralité, nous pouvons nous contenter de prouver que les événements \( \complement A_0,A_1,\ldots,A_n\) sont indépendants sous l'hypothèse que les événements \( A_0,A_1,\ldots,A_n\) sont indépendants. Soit \( I\) un sous-ensemble de \( \{ 1,\ldots,n \}\). Nous avons
    \begin{subequations}
        \begin{align}
            P\big( \complement A_0\bigcap_{i\in I}A_i \big)&=P\big( \bigcap_{i\in I}A_i\setminus\bigcap_{i\in I}A_i\cap A_0 \big)\\
            &=P\big( \bigcap_{i\in I}A_i \big)-P\big( \bigcap_{i\in I}A_i\cap A_0 \big)\\
            &=P\big( \bigcap_{i\in I}A_i \big)\big( 1-P(\complement A_0) \big)\\
            &=P\big( \bigcap_{i\in I}A_i \big)P(\complement A_0).
        \end{align}
    \end{subequations}
\end{proof}

\begin{proposition}
    Les événements \( (A_i)_{i=1,\ldots,n}\) sont indépendants si et seulement si les variables aléatoires associées \( \mtu_{A_1},\ldots,\mtu_{A_n}\) le sont.
\end{proposition}

\begin{proof}
    La tribu engendrée par la variable aléatoire \( \mtu_{A_k}\) est
    \begin{equation}    \label{EqtribAAimtu}
        \tribA_{\mtu_{A_k}}=\{ \emptyset,A_k,\complement A_k,\Omega \}.
    \end{equation}
    En effet si \( 1\in B\), alors \( A_i\subset\mtu_{A_i}^{-1}(B)\), et si \( 0\in B\), alors \( \complement A_i\subset\mtu_{A_i}^{-1}(B)\). Les éléments \( 0\) et \( 1\) sont tous deux soit dans \( B\), soit hors de \( B\). Cela donne les \( 4\) possibilités énumérées dans \eqref{EqtribAAimtu}.

    Supposons que les événements \( (A_i)\) sont indépendants. Nous devons vérifier que les tribus le soient, c'est-à-dire que les événements \( A_i\) et \( \complement A_j\) sont indépendants. Cela est une conséquence du lemme~\ref{LemIndepEvenCompl}.
\end{proof}

\begin{theorem}[Doob\cite{ProbaDanielLi}]     \label{ThofrestemesurablesXYYX}
    Soit \( X\colon \Omega\to \eR^d\) une variable aléatoire. Une fonction \( Y\colon \Omega\to \eR^{p}\) est une variable aléatoire \( \tribA_X\)-mesurable si et seulement s'il existe une fonction borélienne \( f\colon \eR^d\to \eR^{p}\) telle que \( Y=f(X)\).
\end{theorem}

\begin{proposition}
    Soient des variables aléatoires \( X_k\colon \Omega\to \eR^{d_k}\) des variables aléatoires indépendantes et des fonctions boréliennes \( f_k\colon \eR^{d_k}\to \eR^{p_k}\). Alors les variables aléatoires \( f_k(X_k)\) sont indépendantes.
\end{proposition}

\begin{proof}
    Le théorème~\ref{ThofrestemesurablesXYYX} assure que les applications
    \begin{equation}
        f_k\circ X_k\colon \Omega\to \eR^{d_k}
    \end{equation}
    sont \( \tribA_{X_k}\)-mesurables. En particulier pour tout borélien \( B\subset\eR^{p_k}\), nous avons \( X^{-1}_k\circ f^{-1}_k(B)\in\tribA_{X_k}\). Nous avons donc
    \begin{equation}
        \sigma(f_k\circ X_k)\subset\sigma(X_k),
    \end{equation}
    et par conséquent les tribus \( \sigma(f_k\circ X_k)\) sont indépendantes étant donné que les tribus \( \sigma(X_k)\) le sont.
\end{proof}

\begin{lemma}[Lemme de regroupement]\index{lemme!regroupement}  \label{LemHOjqqw}
    Soit \( (\Omega,\tribA,P)\) un espace de probabilité et \( (\tribA)_{i\in I}\) une famille de tribus indépendantes dans \( \tribA\). Si \( (M_j)_{j\in J}\) est une partition de \( I\), alors les tribus
    \begin{equation}
        \tribB_j=\sigma\big( \bigcup_{i\in M_j}\tribA_i \big)
    \end{equation}
    sont indépendantes.

    Si les variables aléatoires \( \{ X_1,X_2,X_3,X_4,X_5 \}\) sont indépendantes, et si \( f\) et \( g\) sont des fonctions mesurables, alors les variables aléatoires \( f(X_2,x_3,X_5)\) et \( g(X_1,X_4)\) sont indépendantes.
\end{lemma}
Une preuve a l'air d'être donnée dans \cite{VincentBa}.
%TODO : lire cette preuve.

%---------------------------------------------------------------------------------------------------------------------------
\subsection{Lois conjointes et indépendance}
%---------------------------------------------------------------------------------------------------------------------------

\begin{definition}
    Deux événements \( A\) et \( B\) sont dits \defe{indépendants}{indépendance} si
    \begin{equation}
        P(A\cap B)=P(A)P(B).
    \end{equation}
\end{definition}
Si nous considérons \( n\) variables aléatoires réelles \( X_1,\ldots,X_n\colon\Omega\to\eR\), la loi du \( n\)-uplet \( X=(X_1,\ldots,X_n)\) est une variable aléatoire \( X\colon \Omega\to \eR^n\) appelée la \defe{loi conjointe}{loi!conjointe} des lois \( X_i\). Dans ce cas, les variables aléatoires \( X_i\) elles-mêmes sont dites lois \defe{marginales}{loi!marginale} de \( X\).

\begin{proposition}     \label{PropPXXXPXPXPX}
    Les variables aléatoires \( \{ X_i \}\) sont indépendantes si et seulement si
    \begin{equation}
        P_{(X_1,\ldots,X_n)}=P_{X_1}\otimes\ldots\otimes P_{X_n}.
    \end{equation}
\end{proposition}

\begin{definition}      \label{DefFonrepConj}
    Soient \( \{ X_i \}_{1\leq i\leq n}\) des variables aléatoires réelles (pas spécialement indépendantes). La \defe{densité conjointe}{densité!conjointe} de \( X_1\),\ldots,\( X_n\) est la fonction \( f\colon \eR^n\to \eR\) qui satisfait
    \begin{enumerate}
        \item
            \( f(x_1,\ldots,x_n)\geq 0\) pour tout \( (x_1,\ldots,x_n)\in\eR^n\),
        \item
            \( \int_{\eR^n}f=1\),
        \item       \label{ItemDefFonrepConjiii}
            pour tout \( A_i\subset\eR \) nous avons
            \begin{equation}
                P(\bigcap_{i=1}^n X_i\in A_i)=\int_{\prod_i A_i}f(x_1,\ldots,x_n)dx_1\ldots dx_n.
            \end{equation}
    \end{enumerate}
\end{definition}

\begin{proposition}     \label{PropDensiteConjIndep}
    Si les variables aléatoires \( X_1\),\ldots \( X_n\) sont indépendantes et ont des densités \( f_{X_1}\),\ldots,\( f_{X_n}\), alors la variable aléatoire conjointe \( X=(X_1,\ldots,X_n)\) a pour densité conjointe la fonction
    \begin{equation}
        f_X(x_1,\ldots,x_n)=f_{X_1}(x_1)\ldots f_{X_n}(x_n).
    \end{equation}
\end{proposition}

\begin{proof}
    En partant de la définition de l'indépendance et de la fonction de densité conjointe, ainsi qu'en utilisant le théorème de Fubini,
    \begin{equation}
        \begin{aligned}[]
            \int_{A_1\times \ldots\times A_n}f_X(x_1,\ldots,x_n)dx_1\ldots dx_n&=
            P(X_1\in A_1,\ldots,X_n\in A_n)\\
            &=P(X_1\in A_1)\ldots P(X_n\in A_n)\\
            &=\left( \int_{A_1}f_{X_1}(x_1)dx_1 \right)\ldots\left( \int_{A_n}f_{X_n}(x_n)dx_n \right)\\
            &=\int_{A_1\times\ldots\times A_n}f_{X_1}(x_1)\ldots f_{X_n}(x_n)dx_1\ldots dx_n.
        \end{aligned}
    \end{equation}
    La fonction \( (x_1,\ldots,x_n)\mapsto f_{X_1}(x_1)\ldots f_{X_n}(x_n)\) vérifie donc la condition~\ref{ItemDefFonrepConjiii} de la définition~\ref{DefFonrepConj}. La vérification des autres conditions est immédiate.
\end{proof}


La proposition suivante provient du fait que la mesure d'une loi conjointe est le produit des mesures lorsque les variables aléatoires sont indépendantes (proposition~\ref{PropPXXXPXPXPX}).
\begin{proposition}[\cite{ProbaDanielLi}]
    Si les variables aléatoires réelles \( X_1\),\ldots,\( X_n\) sont intégrables et indépendantes, alors leur produit est intégrable et l'espérance du produit est égal au produit des espérances :
    \begin{equation}
        E(X_1\cdots X_n)=E(X_1)\ldots E(X_n).
    \end{equation}
\end{proposition}

%---------------------------------------------------------------------------------------------------------------------------
\subsection{Somme et produit de variables aléatoires indépendantes}
%---------------------------------------------------------------------------------------------------------------------------
\label{subsecscnvommevariablsindep}


Soient \( X\) et \( Y\), deux variables aléatoires réelles indépendantes. Nous voudrions étudier la loi de la variable aléatoire \( S=X+Y\). Nous commençons par calculer la fonction de répartition en utilisant le résultat de la proposition~\ref{PropDensiteConjIndep} :
\begin{subequations}
    \begin{align}
        F_{X+Y}(z)=P(X+Y\leq z)&=\int_{x+y\leq z}f_{X,Y}(x,y)dx\,dy\\
        &=\int_{-\infty}^{\infty}dx\int_{-\infty}^{z-x}dyf_X(x)f_Y(y)\\
        &=\int_{\eR}\left( \int_{-\infty}^{z-x}f_Y(y)dy \right)f_X(x)dx\\
        &=\int_{\eR}F_Y(z-x)f_X(x)dx.
    \end{align}
\end{subequations}
Pour calculer la fonction de densité de \( S\), nous dérivons la fonction de répartition :
\begin{subequations}
    \begin{align}
        f_{X+Y}(z)&=\frac{ d F_{X+Y} }{ d z }(z)\\
        &=\int_{\eR}f_Y(z-x)f_X(x)dx,
    \end{align}
\end{subequations}
ce qui nous amène à dire que la densité de la somme est le produit de convolution\footnote{Définition~\ref{DEFooHHCMooHzfStu}.}\index{convolution} des densités :
\begin{equation}        \label{EqdensitesooemXYint}
    f_{X+Y}(x)=\int_{\eR}f_Y(x-t)f_X(t)dt,
\end{equation}
ou encore \( f_{X+Y}=f_X* f_Y\).

Notez que nous avons passé sous le silence la difficulté d'inverser la dérivée et l'intégrale. Un exemple sera donné au point~\ref{subsecPoissonetexpo}.

\begin{lemma}       \label{LemEXYEXEYprodindep}
    Soient \( X\) et \( Y\), deux variables aléatoires indépendantes. Alors
    \begin{equation}
        E(XY)=E(X)E(Y).
    \end{equation}
\end{lemma}

\begin{proof}
    Par indépendance et par proposition~\ref{PropDensiteConjIndep}, la fonction de densité conjointe de \( X\) et \( Y\) vaut \( f_{X,Y}=f_Xf_Y\). Par conséquent l'utilisation de Fubini sous la forme \eqref{EqTJEEsJW} entraine
    \begin{equation}
        E(XY)=\int_{\eR\times\eR}xyf_{X,Y}(x,y)dxdy=E(X)E(Y).
    \end{equation}
\end{proof}

Nous dirons tout un tas de chose sur l'indépendance et la variance en~\ref{subsecTTHohur}, mais pour l'instant nous allons mentionner et démontrer déjà ceci :
\begin{lemma}   \label{LemVarXpYsmindep}
    Soient \( X\) et \( Y\) deux variables aléatoires indépendantes et identiquement distribuées. Alors
    \begin{equation}
        \Var(X+Y)=\Var(X)+\Var(Y).
    \end{equation}
\end{lemma}

\begin{proof}
    Par définition, \( \Var(X+Y)=E\big( [X+Y-E(X)-E(Y)]^2 \big)\). En développant le carré et en utilisant le lemme~\ref{LemEXYEXEYprodindep},
    \begin{equation}
        \Var(X+Y)=E(X^2)-E(X)^2+E(Y^2)-E(Y)^2=\Var(X)+\Var(Y).
    \end{equation}
\end{proof}

\begin{example} \label{ExWLzkuWd}
    Deux variables aléatoires non indépendantes dont la covariance est nulle. Nous considérons la variable aléatoire
    \begin{equation}
        Z\colon \Omega\to \{ (1,0),(-1,0),(0,1),(0,-1) \}
    \end{equation}
    de loi uniforme. C'est-à-dire que \(  P\big( Z=z \big)=\frac{1}{ 4 }  \) pour tout \( z\). Ensuite nous considérons les variables aléatoires \( X=\pr_1\circ Z\) et \( Y=\pr_2\circ Z\). Toute personne étant capable de compter jusqu'à \( 4\) voit que
    \begin{subequations}
        \begin{align}
            P(X=1)&=P(X=-1)=\frac{1}{ 4 }\\
            P(X=0)&=\frac{ 1 }{2},
        \end{align}
    \end{subequations}
    et les mêmes probabilités pour \( Y\). De même \( E(X)=E(Y)=0\). Par conséquent
    \begin{equation}
        \Cov(X,Y)=E(XY)=0
    \end{equation}
    parce que pour tout \( \omega\in \Omega\) nous avons soit \( X(\omega)=0\) soit \( Y(\omega)=0\). Ces variables aléatoires \( X\) et \( Y\) ne sont donc pas corrélées.

    Mais elles ne sont pas indépendantes pour autant, comme nous allons le voir pas plus tard qu'immédiatement. Nous avons
    \begin{equation}
        P(X=0|Y=0)=\frac{ P(X=0,Y=0) }{ P(Y=0) }=0
    \end{equation}
    parce que \( X\) et \( Y\) ne peuvent pas être simultanément nulles, tandis que
    \begin{equation}
        P(X=0)P(Y=0)=\frac{1}{ 4 }.
    \end{equation}
\end{example}

%---------------------------------------------------------------------------------------------------------------------------
\subsection{Espérance}
%---------------------------------------------------------------------------------------------------------------------------

Nous dirons que la variable aléatoire \( X\) a un \defe{moment d'ordre \( p\)}{moment} si \( X\in L^p(\Omega,\tribA,P)\) (\( 1\leq p<\infty\)). Si \( X\) est \defe{intégrable}{variable aléatoire!intégrable} (c'est-à-dire si \( X\in L^1\)), alors nous définissons l'\defe{espérance}{espérance} de \( X\) par
\begin{equation}        \label{EqdCBLst}
    E(X)=\int_{\Omega}XdP\in\eR^d.
\end{equation}
Si \( E(X)=0\) nous disons que la variable aléatoire est \defe{centrée}{variable aléatoire!centrée}. La variable aléatoire \( X-E(X)\) est la variable aléatoire centrée associée à \( X\).

Le \defe{moment}{moment} d'ordre \( p\) de la variable aléatoire \( X\) est l'espérance
\begin{equation}
    m_n(X)=E(X^n).
\end{equation}

\begin{proposition} \label{PropZBnsCgh}
    Si \( X\) et \( Y\) sont deux variables aléatoires (pas spécialement indépendantes), nous avons
    \begin{equation}
        E(X+Y)=E(X)+E(Y).
    \end{equation}
\end{proposition}

Nous donnons la preuve dans le cas de variables aléatoires indépendantes. Le cas plus général de variable aléatoires non indépendantes peut être trouvé dans \cite{Marazzi}.
%TODO : ce serait pas mal de la faire.
\begin{proof}
    Nous avons le calcul suivant :
    \begin{subequations}
        \begin{align}
            E(X+Y)&=\int_{\eR}xf_{X+Y}(x)dx\\
            &=\int_{\eR}x\int_{\eR}f_Y(x-t)f_X(t)dtdx\\
            &=\int_{\eR}f_X(t)\underbrace{\int_{\eR}xf_Y(x-t)dx}_{=E(Y)+t}\,dt\\
            &=\int_{\eR}f_X(t)\big( E(Y)+t \big)dt\\
            &=E(Y)+\int_{\eR}tf_X(t)dt\\
            &=E(Y)+E(X)
        \end{align}
    \end{subequations}
    où nous avons utilisé la proposition~\ref{EqdensitesooemXYint} et le fait que l'intégrale sur \( \eR\) d'une densité vaut \( 1\).
\end{proof}

Une application de l'inégalité de Hölder (proposition~\ref{ProptYqspT}) est la suivante. Si \( X\) et \( Y\) sont des variables aléatoires intégrables alors
\begin{equation}
    E(XY)\leq E(X^2)^{1/2}E(Y^2)^{1/2}.
\end{equation}
En effet
\begin{equation}    \label{EqEXYleqXdYdNormHolder}
    E(XY)\leq \| XY \|_{L^1(\Omega)}\leq \| X \|_{L^2(\Omega)}\| Y \|_{L^2(\Omega)}.
\end{equation}
\index{inégalité!Hölder!utilisation}

%---------------------------------------------------------------------------------------------------------------------------
\subsection{Variance}
%---------------------------------------------------------------------------------------------------------------------------

Si \( X\in L^2(\Omega,\tribA,P)\) alors nous définissons la \defe{variance}{variance} de \( X\) par
\begin{equation}
    \Var(X)=E\big( [X-E(X)]^2 \big).
\end{equation}

\begin{proposition}     \label{PrropVarAlterfrom}
    La variance de la variable aléatoire \( X\) peut être exprimée par la formule
    \begin{equation}        \label{EqtWqMGB}
        \Var(X)=E(X^2)-[E(X)]^2
    \end{equation}
    où \( X^2=X\cdot X\) et \( E(X)^2=E(X)\cdot E(X)\) sont des produits scalaires dans \( \eR^d\).
\end{proposition}

\begin{proof}
    De façon explicite, nous avons
    \begin{equation}
        E\big( [X-E(X)]^2 \big)=\int_{\Omega}\big( X(\omega)-E(X) \big)\cdot\big( X(\omega)-E(X) \big)dP(\omega)
    \end{equation}
    où \( E(X)\in\eR^d\) est une constante. En développant le produit scalaire nous avons
    \begin{subequations}
        \begin{align}
            E\big( [X-E(X)]^2 \big)&=E\big( X^2-2X\cdot E(X)+E(X)^2 \big)\\
            &=E(X^2)-2E(X)^2+E(X)^2\\
            &=E(X^2)-E(X)^2.
        \end{align}
    \end{subequations}
\end{proof}


Nous définissons l'\defe{écart-type}{ecart-type@écart-type} de \( X\) par
\begin{equation}
    \sigma_X=\sqrt{\Var(X)}.
\end{equation}
En d'autres termes,
\begin{equation}
    \sigma_X=\| X-E(X) \|_{L^2}.
\end{equation}
On définit encore la \defe{moyenne quadratique}{moyenne!quadratique} de \( X\) par
\begin{equation}
    \| X \|_{L^2}=\big[ E(X^2) \big]^{1/2}.
\end{equation}

La variable aléatoire
\begin{equation}
    \bar V_n=\frac{1}{ n }\sum_i(X_i-\bar X_n)^2
\end{equation}
est la \defe{variance empirique}{variance!empirique} de l'échantillon \( (X_i)\).

\begin{lemma}       \label{LemEXYEXEYindep}\label{PropVarPropnnlin}
    Si \( X\) est une variable aléatoire,
    \begin{enumerate}
        \item
            $\Var(ax)=a^2\Var(X)$ pour tout \( a\in\eR\);
        \item
            Si de plus \( Y\) est une variable aléatoire indépendante de \( X\), alors $\Var(X+Y)=\Var(X)+\Var(Y)$.
    \end{enumerate}
\end{lemma}

\begin{proof}
    Nous avons
    \begin{subequations}
        \begin{align}
            \Var(X+Y)&=E(X^2+Y^2+2XY)-\big( E(X)+E(Y) \big)^2\\
            &=E(X^2)+E(Y^2)+2E(XY)-E(X)^2-E(Y)^2+2E(X)E(Y).
        \end{align}
    \end{subequations}
    Étant donné que \( X\) et \( Y\) sont indépendantes nous avons \( E(XY)=E(X)E(Y)\) par le lemme~\ref{LemEXYEXEYprodindep}.
\end{proof}

Si les \( X_1,\ldots,X_n\) sont des variables aléatoires on considère la \defe{moyenne empirique}{moyenne!empirique}
\begin{equation}
    \bar X_n=\frac{ X_1+\cdots+X_n }{ n }.
\end{equation}

%---------------------------------------------------------------------------------------------------------------------------
\subsection{Covariance}
%---------------------------------------------------------------------------------------------------------------------------

Soient \( X\) et \( Y\), deux variables aléatoires réelles. Leur \defe{covariance}{covariance} est définie par
\begin{equation}    \label{EqHUWtttN}
    \Cov(X,Y)=E\Big[ \big( X-E(X) \big)\big( Y-E(Y) \big) \Big]
\end{equation}
L'idée est que la covariance devient grande si \( X\) et \( Y\) s'écartent de leurs moyennes dans le même sens. Il existe une formule alternative :
\begin{equation}
    \Cov(X,Y)=E(XY)-E(X)E(Y)
\end{equation}

En ce qui concerne les dimensions plus hautes, si \( X\colon \Omega\to \eR^d\) est un vecteur aléatoire de carré intégrable, nous définissons
\begin{equation}    \label{EqZlvLWx}
    \Cov(X)=E\Big[ \big(  X-E(X) \big)\otimes\big( X-E(X)\big) \Big]
\end{equation}
où par \( a\otimes b\) nous entendons la matrice \( (a\otimes b)_{ij}=a_ib_j\). Cela peut aussi être noté \( a^tb\) si l'on fait bien attention à qui est un vecteur colonne et qui est un vecteur ligne.

\begin{proposition}     \label{PropoVarXpYCov}
    Si \( X\) et \( Y\) sont deux variables aléatoires non spécialement indépendantes, nous avons
    \begin{equation}
        \Var(X+Y)=\Var(X)+\Var(Y)+2\Cov(X,Y).
    \end{equation}
\end{proposition}

\begin{proof}
    Il s'agit d'un calcul en partant de
    \begin{equation}
        \begin{aligned}[]
            \Var(X+Y)&=E\big( (X+Y)^2 \big)-E(X+Y)^2\\
            &=E(X^2)+E(Y^2)+2E(XY)\\
            &\quad+\big( E(X)+E(Y) \big)^2-2E(X)^2-2E(X)E(Y)\\
            &\quad-2 E(Y)E(X)-2E(Y)^2.
        \end{aligned}
    \end{equation}
    À partir d'ici il s'agit de recombiner tous les termes pour former la formule annoncée.
\end{proof}

Plus généralement nous avons la formule
\begin{equation}
    \Var(\sum_i X_i)=\sum_i\Var(X_i)+2\sum_{1\leq i< j\leq n}\Cov(X_i,X_j).
\end{equation}

%---------------------------------------------------------------------------------------------------------------------------
\subsection{Probabilité conditionnelle : événements}
%---------------------------------------------------------------------------------------------------------------------------

\begin{propositionDef}      \label{DEFooGJVHooVbhVYv}
    Soit \( (\Omega,\tribA,P)\) un espace de probabilité et \( B\in\tribA\) avec $P(B)>0$. Alors avec la formule
    \begin{equation}    \label{EqProbCond}
        P_B(A)=\frac{ P(A\cap B) }{ P(B) },
    \end{equation}
    l'espace \( (\Omega,\tribA,P_B)\) est un espace probabilisé. Nous notons \( P_B(A)  \) le nombre \( P(A|B)\) et nous le nommons \defe{probabilité conditionnelle}{probabilité!conditionnelle} de \( A\) sachant \( B\).
\end{propositionDef}

\begin{proof}
    On vérifie que \( (\Omega,\tribA,P_B)\) est un espace de probabilité parce que \( P_B(\Omega)=1\) et
    \begin{equation}
        P_B(\bigcup_iA_i)=\sum_iP_B(A_i)
    \end{equation}
    si les \( A_i\) sont deux à deux disjoints.
\end{proof}

Une conséquence immédiate de \eqref{EqProbCond} est que si \( A\) et \( B\) sont des événements indépendants alors
\begin{equation}
    P(A|B)=\frac{ P(A\cap B) }{ P(B) }=P(A).
\end{equation}

La probabilité conditionnelle à \( B\) est quelque chose qui ne tient compte que de ce qui se passe dans \( B\). Si \( K\) est un événement tel que \( A\cap B=K\cap B\), alors
\begin{equation}    \label{EqOVHCWom}
    P(A|B)=P(K|B).
\end{equation}

\begin{theorem}     \label{ThoBayesEtAutres}
    Soient \( (B_n)_{n\geq 1}\) une partition finie de \( \Omega\) telle que \( P(B_i)>0\). Soit \( A\in\tribA\) tel que \( P(A)>0\).
    \begin{enumerate}
        \item
            Si \( A\), \( B\) et \( C\) sont des événements, alors
            \begin{equation}
                P(A\cap B|C)=P(A|B\cap C)P(B|C).
            \end{equation}
        \item
            Si \( P(B)>0\), alors \( P(A\cap B)=P(A|B)P(B)=P(B|A)P(A)\).
        \item On a la \defe{formule des probabilités totales}{formule!probabilité totales} :
            \begin{equation}
                P(A)=\sum_{i=1}^nP(A|B_i)P(B_i)=\sum_iP(A\cap B_i).
            \end{equation}
        \item
            On a la \defe{formule de Bayes}{formule!Bayes} :
            \begin{equation}
                P(B_k|A)=\frac{ P(A|B_k)P(B_k) }{ \sum_iP(A|B_i)P(B_i) }.
            \end{equation}
    \end{enumerate}
\end{theorem}

\begin{proof}
    \begin{enumerate}
        \item
            En développant le membre de droite,
            \begin{equation}
                \begin{aligned}[]
                    P(A\cap B|C)&=\frac{ P(A\cap B\cap C) }{ P(B\cap C) }\frac{ P(B\cap C) }{ P(C) }\\
                    &=P(A\cap B|C).
                \end{aligned}
            \end{equation}
        \item
            C'est la définition de \( P(A|B)\) et \( P(B|A)\).
        \item
            Vu que les \( B_i\) forment une partition, nous avons
            \begin{equation}
                P(A)=\sum_iP(A\cap B_i)=\sum_iP(A|B_i)P(B_i).
            \end{equation}
        \item
            En utilisant les deux premiers points, nous trouvons
            \begin{equation}
                \begin{aligned}[]
                    P(A|B_k)P(B_k)&=P(A\cap B_k)\\
                    &=P(B_k|A)P(A)\\
                    &=P(B_k|A)\sum_iP(A|B_i)P(B_i).
                \end{aligned}
            \end{equation}
    \end{enumerate}
\end{proof}

%---------------------------------------------------------------------------------------------------------------------------
\subsection{Espérance conditionnelle}
%---------------------------------------------------------------------------------------------------------------------------

\begin{theoremDef}[Définition de l'espérance conditionnelle\cite{ProbCOndutetz}]     \label{ThoMWfDPQ}
    Soit un espace de probabilité \( (\Omega,\tribA,P)\) et une variable aléatoire intégrable \( X\colon \Omega\to \eR\). Pour chaque sous tribu \( \tribF\) de \( \tribA\), il existe une (presque partout) unique variable aléatoire \( Y\colon \Omega\to \eR\) telle que
    \begin{enumerate}
        \item
            \( Y\) est \( \tribF\)-mesurable
        \item
            \( Y\) est \( P\)-intégrable
        \item
            pour tout \( B\in\tribF\),
            \begin{equation}        \label{EqBwBkgE}
                \int_{B}XdP=\int_B YdP.
            \end{equation}
    \end{enumerate}
    Cette variable aléatoire sera notée \( E(X|\tribF)\)\nomenclature[P]{\( E(X|\tribF)\)}{Espérance conditionnelle de \( X\) sachant \( \tribF\)} pour des raisons qui apparaîtront plus tard.
\end{theoremDef}
\index{espérance!conditionnelle}

\begin{proof}
        Remarquons que prendre \( Y=X\) ne fonctionne pas parce qu'en général si \( \mO\) est mesurable dans \( \eR\), alors \( X^{-1}(\mO)\) est dans la tribu \( \tribA\), mais n'est pas automatiquement dans la tribu \( \tribF\). Il faudra donc un peu plus travailler.
    \begin{subproof}
        \item[Unicité] Si \( Y_1\) et \( Y_2\) vérifient tous les deux les conditions, l'ensemble \( \{ Y_1<Y_2 \}\) est un élément de \( \tribF\) et nous avons
            \begin{equation}
                \int_{\{ Y_1<Y_2 \}}X=\int_{Y_1<Y_2}Y_1=\int_{Y_1<Y_2}Y_2.
            \end{equation}
            En particulier nous avons \( \int_{\{ Y_1<Y_2 \}}(Y_1-Y_2)=0\) et donc
            \begin{equation}
                (Y_1-Y_2)\mtu_{Y_1-Y_2}=0
            \end{equation}
            presque partout. Le corolaire~\ref{CorjLYiSm} montre alors que \( Y_1-Y_2\geq 0\) presque partout. De la même manière, l'ensemble \( \{ Y_2<Y_1 \}\) est dans \( \tribF\) et nous trouvons que \( Y_2-Y_1\geq 0\) presque partout. Par conséquent \( Y_1=Y_2\) presque partout.
        \item[Existence dans le cas de carré intégrable]

            Nous supposons que \( X\in L^2(\Omega,\tribA,P)\) et nous considérons \( K\), le sous-ensemble de \( L^2(\Omega,\tribA,P)\) des fonctions \( \tribF\)-mesurables. Le théorème des projections~\ref{ThoProjOrthuzcYkz} nous indique que
            \begin{equation}
                L^2(\Omega,\tribA,P)=K\oplus K^{\perp}
            \end{equation}
            par la décomposition \( X=\pr_{K}X+(X-\pr_KX)\). La variable aléatoire \( Y=\pr_KX\) a les propriétés d'être \( \tribF\)-mesurable et \( \langle Y-X, Z\rangle =0\) pour tout \( Z\in K\). Soit \( A\in\tribF\), si nous considérons \( Z=\mtu_A\), la dernière condition signifie que
            \begin{equation}
                \int_{\Omega}X\mtu_A=\int_{\Omega}Y\mtu_A,
            \end{equation}
            ou encore
            \begin{equation}
                \int_AY=\int_AX.
            \end{equation}
            La variable aléatoire \( Y=\pr_K(X)\) répond donc à la question lorsque \( X\in L^2(\Omega,\tribF,P)\).

        \item[Existence en général]

            Nous considérons maintenant que \( X\in L^1(\Omega,\tribA,P)\). Quitte à décomposer \( X\) en deux fonctions positives \( X_+\) et \( X_-\) telles que \( X=X_++X_-\), nous pouvons supposer que \( X\) est positive. Par hypothèse \( X\in L^1(\Omega,\tribA,P)\); pour chaque \( n\in\eN\) nous posons
            \begin{equation}
                X_n(\omega)=\min\{ X(\omega),n \}.
            \end{equation}
            Étant donné que la mesure \( P\) est une mesure de probabilité, les constantes sont intégrables et \( X_n\in L^2(\Omega,\tribA,P)\). De plus la suite \( (X_n)\) est croissante et
            \begin{equation}
                \lim_{n\to \infty} X_n(\omega)=X(\omega).
            \end{equation}

            Si nous notons encore \( K\) l'ensemble des variables aléatoires dans \( L^2(\Omega,\tribA,P)\) qui sont \( \tribF\)-mesurables, pour chaque \( n\) nous avons donc la variable aléatoire
            \begin{equation}
                Y_n=\pr_KX_n=E(X_n|\tribF)
            \end{equation}
            qui est \( \tribF\)-mesurable et telle que
            \begin{equation}
                \int_A X_n=\int_AY_n
            \end{equation}
            pour tout \( A\in\tribF\). Nous voudrions prouver que la variable aléatoire \( Y=\lim_nY_n\) existe et est la solution au problème, c'est-à-dire est \( E(X|\tribF)\).

            Commençons par prouver que \( Y_n\geq 0\) presque partout. Pour cela nous remarquons que l'ensemble \( \{ Y_n<0 \}\) est mesurable et
            \begin{equation}
                0\geq\int_{Y_n<0}Y_n=\int_{Y_n<0}X_n\geq 0.
            \end{equation}
            La première inégalité est évidente et la dernière est due au fait que \( X_n\) est positive. Par conséquent
            \begin{equation}
                \int_{Y_n<0}Y_n=0
            \end{equation}
            et le lemme~\ref{CorjLYiSm} conclut que \( P(Y_n<0)=0\).

            Soit \( Z\colon \Omega\to \eR\) une variable aléatoire positive dans \( L^2(\Omega,\tribA,P)\). Montrons que \( \pr_KZ\) est encore positive. Pour cela nous considérons l'ensemble \( A=\{ \pr_KZ<0 \}\) et les inégalités
            \begin{equation}
                0\leq \int_AZ=\int_A\pr_KZ\leq 0,
            \end{equation}
            ce qui montre que \( \int_A\pr_KZ=0\) et par conséquent que \( P\{ \pr_K(Z)<0 \}=0\). Cela nous montre que la projection depuis \( L^2\) conserve la positivité.

            Étant donné que \( X_{n-1}-X_n\geq 0\) nous avons aussi
            \begin{equation}
                Y_{n-1}-Y_{n}\geq 0
            \end{equation}
            La suite de fonctions
            \begin{equation}
                n\mapsto Y_n=E(X_n|\tribF)
            \end{equation}
            est croissante et vérifie le théorème de la convergence monotone :
            \begin{equation}
                    \int_A X=\lim_{n\to \infty} \int_A X_n =\lim_{n\to \infty} \int_A E(X_n|\tribF)=\int_A\lim_{n\to \infty } E(X_n|\tribF)=\int_A Y.
            \end{equation}
            Par conséquent \( E(X|\tribF)\) existe et
            \begin{equation}
                Y=\lim_{n\to \infty} E(X_n|\tribF)=E(X|\tribF).
            \end{equation}
    \end{subproof}
\end{proof}

\begin{normaltext}      \label{NORMooHPHOooUuJWHR}
    Vu la définition~\ref{ThoMWfDPQ} nous pourrions croire que la variable aléatoire \( E(X|\tribF)=X\) fait l'affaire. Il n'en est rien parce que la variable aléatoire \( X\) n'est pas spécialement \( \tribF\)-mesurable alors qu'il est requis que \( E(X|\tribF)\) le soit. Avec la tribu \( \tribF=\{ \emptyset,\Omega \}\), nous n'avons en général pas que \( X^{-1}(B)\in \tribF\) pour tout borélien \( B\).

    Par contre si \( \sigma(X)\) est la tribu engendrée par la variable aléatoire \( X\), alors \( E\big( X |\sigma(X) \big)=X\).
\end{normaltext}

\begin{definition}      \label{DefooKIHPooMhvirn}
    Soit \( Z\) une variable aléatoire. L'\defe{espérance conditionnelle}{espérance!conditionnelle} «\( X\) sachant \( Z\)» est la variable aléatoire
    \begin{equation}
        E(X|Z)=E(X|\sigma(Z))
    \end{equation}
    où \( \sigma(Z)\) est la tribu engendrée par \( Z\). Le membre de droite est une variable aléatoire définie en~\ref{ThoMWfDPQ}.
\end{definition}

\begin{definition}      \label{DEFooEYVCooCeyOXW}
    Soient \( A\in\tribA\) un événement et \( \tribF\) une sous-tribu de \( \tribA\). Nous définissons\index{espérance!conditionnelle!événement} \( P(A|\tribF)\) par
    \begin{equation}
        P(A|\tribF)=E(\mtu_{A}|\tribF).
    \end{equation}
    Notons que cela est une variable aléatoire et non un réel. Le membre de droite est l'espérance conditionnelle de la variable aléatoire \( \caract_A\) par rapport à \( \tribF\) définie en~\ref{ThoMWfDPQ}.

    Et l'espérance conditionnelle d'un événement par rapport à une variable aléatoire est :
    \begin{equation}
        E(A|X)=E\big( A|\sigma(A) \big).
    \end{equation}
\end{definition}


\begin{proposition}
    Soit une espace probabilisé \( (\Omega,\tribA,P)\) ainsi qu'une variable aléatoire \( X\) à valeurs dans \( \eR^d\), et un événement \( A\). Alors
    \begin{equation}
        E\big( P(A|X) \big)=P(A).
    \end{equation}
\end{proposition}

\begin{proof}
    Tout le point de la preuve est de remarquer que \( E(\mtu_A)=E(\mtu_A|X)\).

    \begin{subproof}
    \item[La formule \( E(\mtu_A)=E(\mtu_A|X)\)]

     La notation \( E(\mtu_A|X)\) est un raccourci pour écrire la variable aléatoire \( E\big(\mtu_A|\sigma(X)\big)\). Cette dernière est l'application \( \Omega\to \eR^d\) telle que
    \begin{equation}
        \int_B E\big( \mtu_A|\sigma(X) \big)=\int_B\mtu_A
    \end{equation}
    pour tout borélien \( B\) de \( \eR^d\) tout en étant \( \sigma(X)\)-mesurable. Comme expliqué en~\ref{NORMooHPHOooUuJWHR}, il est tentant de dire \( E\big( \mtu_A|\sigma(X) \big)=\mtu_A\), mais ce n'est pas le cas parce qu'il n'y a aucune raisons que \( \mtu_A\) soit une application \( \sigma(X)\)-mesurable. Au niveau des espérances, par contre, l'égalité tient :
    \begin{equation}
            E\big( E(\mtu_A|X) \big)=\int_{\Omega}E(\mtu_A|X)=\int_{\Omega}\mtu_A=E(\mtu_A)
    \end{equation}
    où nous avons utilisé le fait que \( \Omega\) lui-même soit \( \sigma(X)\)-mesurable.

        \item[La preuve]

            Nous avons alors
            \begin{equation}
                P(A)=E(\mtu_A)=E\big( E(\mtu_A|X) \big),
            \end{equation}
            alors que \( E(\mtu_A|X)=P(A|X)\). En mettant l'un dans l'autre :
            \begin{equation}
                P(A)=E\big( P(A|X) \big).
            \end{equation}
    \end{subproof}
\end{proof}

\begin{proposition}[Transitivité de l'espérance conditionnelle]     \label{PropRGcscXj}
    Si \( \tribB_2\subseteq\tribB_1\subset\tribA\) alors
    \begin{equation}
        E\Big( E(X|\tribB_1)|\tribB_2 \Big)=E(X|\tribB_2).
    \end{equation}
\end{proposition}

\begin{proof}
    Si \( B\in\tribB_2\), nous avons
    \begin{equation}
        \int_BE\big( E(X|\tribB_1)|\tribB_2 \big)dP=\int_B E(X|\tribB_1)dP=\int_BdP.
    \end{equation}
    La première égalité est la définition de l'espérance conditionnelle par rapport à \( \tribB_2\). La seconde égalité est celle de l'espérance conditionnelle par rapport à \( \tribB_1\) et le fait que \( B\in\tribB_2\subset\tribB_1\). Ce que nous avons prouvé est que
    \begin{equation}
        E\big( E(X|\tribB_1)|\tribB_2 \big)
    \end{equation}
    est une variable aléatoire \( \tribB_2\)-mesurable vérifiant la condition
    \begin{equation}
        \int_BE\big( E(X|\tribB_1)|\tribB_2 \big)=\int_BE(X|\tribB_2)
    \end{equation}
    pour tout \( B\in \tribB_2\). C'est donc \( E(X|\tribB_2)\) par la partie unicité du théorème~\ref{ThoMWfDPQ}.
\end{proof}

\begin{proposition}
    Soit \( (\Omega,\tribF,P)\) un espace de probabilité, soit \( \tribA\) une sous tribu de \( \tribF\) et \( X\), une variable aléatoire \( \tribF\)-mesurable et intégrable. Alors la variable aléatoire \( E(X|\tribA)\) du théorème~\ref{ThoMWfDPQ} est l'unique (presque partout) variable aléatoire à être \( \tribA\)-mesurable telle que nous ayons
    \begin{equation}
        E\big( E(X|\tribA)Y \big)=E(XY).
    \end{equation}
    pour toute variable aléatoire \( Y\) \( \tribA\)-mesurable.
\end{proposition}

\begin{proof}
    Supposons pour commencer que \( Y\) soit une fonction simple positive, alors \( Y=\sum_{i=1}^na_i\mtu_{E_i}\) et nous avons
    \begin{subequations}
        \begin{align}
            \int_{\Omega}E(X|Y)&=\sum_{i}a_i\int_{E_i}E(X|\tribA)\\
            &=\sum_ia_i\int_{E_i}X\\
            &=\int_{\Omega}XY.
        \end{align}
    \end{subequations}
    Maintenant si \( Y\) est mesurable et bornée, elle est limite croissante de fonctions étagées bornées (proposition~\ref{THOooXHIVooKUddLi}) et le résultat tient par la convergence monotone, théorème~\ref{ThoRRDooFUvEAN}.

    Si \( Y\) n'est pas positive, nous séparons \( Y=Y_+-Y_-\).

    Pour l'unicité, soit \( Z\) et \( Z'\) deux variables aléatoires telles que pour toute variable aléatoire \( Y\),
    \begin{equation}
        \int_{\Omega}ZY=\int_{\Omega}XY=\int_{\Omega}Z'Y.
    \end{equation}
    Si nous prenons \( Y=\mtu_{\{ Z\neq Z' \}}\), nous avons
    \begin{equation}
        0=\int_{\Omega}(Z-Z')\mtu_{Z\neq Z'}=\int_{Z\neq Z'}Z-Z',
    \end{equation}
    d'où le fait que \( P(Z\neq Z')=0\).
\end{proof}

Si \( X\) est une variable aléatoire dont la tribu engendrée est indépendante de la tribu \( \tribF\), nous voudrions que la connaissance de \( \tribF\) n'influence pas la connaissance de \( X\), c'est-à-dire que
\begin{equation}
    E(X|\tribF)=E(X).
\end{equation}
Ce que nous avons est même mieux. Nous avons le lemme suivant.
\begin{lemma}[\cite{ProbaDanielLi}]     \label{LemxUZFPV}
    Les tribus \( \tribF_1\) et \( \tribF_2\) sont indépendantes si et seulement si
    \begin{equation}
        E(U|\tribF_1)=E(U)
    \end{equation}
    pour toute variable aléatoire \( U\) étant \( \tribF_1\)-mesurable.
\end{lemma}
Ici, par \( E(U)\) nous entendons la variable aléatoire constante prenant la valeur numérique \( E(U)\) en tout point de \( \Omega\).

\begin{proof}
    Si \( \tribF_1\) et \( \tribF_2\) sont indépendantes, alors pour tout \( B\in\tribF_2\) nous avons
    \begin{subequations}    \label{EqGGqgxl}
            \begin{align}
                \int_B UdP&=E(U\mtu_B)\\
                &=E(U)E(\mtu_B)         \label{subeqBZWLNS}\\
                &=E(U)\int_{\Omega}\mtu_BdP\\
                &=\int_B E(U)dP.
            \end{align}
        \end{subequations}
    Justifications.
    \begin{itemize}
        \item L'intégrale \( \int_BUdP\) a un sens même si \( B\in\tribF_2\) alors que \( U\) est \( \tribF_1\)-mesurable. Le supremum \eqref{EqDefintYfdmu} définissant l'intégrale est tout de même bien défini, en particulier, l'ensemble sur lequel on prend le supremum est non vide.
        \item
            Pour \eqref{subeqBZWLNS}, la variable aléatoire \( U\) est \( \tribF_1\)-mesurable (donc la tribu engendrée par \( U\) est dans \( \tribF_1\)) alors que \( \mtu_B\) est \( \tribF_2\)-mesurable. Les tribus engendrées étant indépendantes, les variables aléatoires le sont et nous pouvons décomposer l'espérance.
    \end{itemize}
    Ce que montre le calcul \eqref{EqGGqgxl} est que \( E(U)\) est une variable aléatoire \( \tribF_2\)-mesurable (parce que constante) dont l'intégrale sur chaque élément de \( \tribF_2\) vaut l'intégrale de \( U\). Par la partie unicité du théorème~\ref{ThoMWfDPQ}, nous déduisons que \( E(U)=E(U|\tribF_2)\).
\end{proof}

\begin{corollary}   \label{CorakyvMp}
    Si \( X\) est une variable aléatoire et si \( \tribF\) est une tribu, alors
    \begin{equation}
        E\big( E(X|\tribF) \big)=E(X).
    \end{equation}
\end{corollary}

\begin{proof}
    Il suffit d'appliquer la définition \eqref{EqBwBkgE} à \( B=\Omega\) :
    \begin{equation}
            E\big( E(X|\tribF) \big)=\int_{\Omega}E(X|\tribF)(\omega)dP(\omega)=\int_{\Omega}X(\omega)dP(\omega)=E(X).
    \end{equation}
\end{proof}

\begin{example}
    Soient \( X_1\), \( X_2\) deux variables aléatoires à valeurs dans \( \{ 0,1 \}\) avec probabilité \( 1/2\) et indépendantes. Nous considérons \( S=X_1+X_2\). La situation est modélisée par l'espace
    \begin{equation}
        \Omega=\{ (0,0),(0,1),(1,0),(1,1) \}
    \end{equation}
    et les variables aléatoires
    \begin{subequations}
        \begin{align}
            X_i(\omega_1,\omega_2)&=\omega_{i}\\
            S(\omega_1,\omega_2)=\omega_1+\omega_2.
        \end{align}
    \end{subequations}
    Pour vérifier que de cette manière nous avons bien que \( X_1\) est indépendante de \( X_2\), nous commençons par voir les tribus associées. Un ouvert de \( \eR\) soit contient \( 0\) et \( 1\), soit contient un seul des deux soit n'en contient aucun des deux. En appliquant \( X_1^{-1}\) à chacune de ces quatre situations nous voyons que la tribu \( \sigma(X_1)\) est
    \begin{equation}
        \tribF_1=\sigma(X_1)=\big\{ \{ (0,0),(0,1) \},\{ (1,0),(1,1) \},\Omega,\emptyset \}.
    \end{equation}
    De la même façon nous avons
    \begin{equation}
        \tribF_2=\sigma(X_1)=\big\{ \{ (0,0),(1,0) \},\{ (0,1),(1,1) \},\Omega,\emptyset \}.
    \end{equation}
    Nous posons
    \begin{subequations}
        \begin{align}
            A_0&=\{ (0,0),(0,1) \}\\
            A_1&=\{ (1,0),(1,1) \}\\
            B_0&=\{ (0,0),(1,0) \}\\
            B_1&=\{ (0,1),(1,1) \}.
        \end{align}
    \end{subequations}
    Étant donné que \( A_i\cap B_j=(i,j)\), nous avons toujours que \( P(A_i\cap B_j)=\frac{1}{ 4 }=P(A_i)P(B_j)\). L'indépendance est donc assurée.

    Calculons l'espérance conditionnelle \( E(S|\tribF_1)\). Une fonction \( \tribF_1\)-mesurable doit être constante sur \( A_0\) et \( A_1\), donc l'espérance conditionnelle est une fonction constante sur \( A_0\) et \( A_1\) dont l'intégrale sur ces ensembles est égale à l'intégrale de \( S\). Nous avons en particulier
    \begin{equation}
        \int_{A_0}E(S|\tribF_1)=\int_{A_0}S,
    \end{equation}
    c'est-à-dire
    \begin{equation}
        E(S|\tribF_1)(0,0)+E(S|\tribF_1)(0,1)=S(0,0)+S(0,1)=1.
    \end{equation}
    Nous en concluons que \( E(S|\tribF_1)(0,0)=E(S|\tribF_1)(0,1)=\frac{ 1 }{2}\). Cela correspond à l'intuition que si on est au point \( (0,1)\) ou au point \( (0,0)\) en ne sachant que \( X_1\), nous ne savons que le premier zéro, et donc l'espérance de la somme est \( \frac{ 1 }{2}\).

    Un calcul très similaire montre que
    \begin{equation}
        E(S|\tribF_1)(1,0)=E(S|\tribF_1)(1,1)=\frac{ 3 }{2}.
    \end{equation}
    Cela correspond au fait qu'en ces points, nous ne savons que le fait que le premier tirage a donné \( 1\), et donc que l'espérance est \( \frac{ 3 }{2}\).

    Complétons ce tour d'horizon en mentionnant que la tribu engendrée par \( X_1\) et \( X_2\) est la tribu des parties de \( \Omega\), de telle façon que l'espérance conditionnelle de \( S\) sachant \( X_1\) et \( X_2\) est égale à \( S\).
\end{example}

\begin{proposition}[\cite{ProbaDanielLi}]   \label{PropRNBtfql}
    Soit \( (\Omega,\tribA,P)\) un espace probabilisé et \( X,Y\) deux variables aléatoires sur \( \Omega\) réelles. Soit \( \tribB\) une sous-tribu de \( \tribA\). Nous supposons que \( X\in L^1(\Omega,\tribA,P)\), que \( Z\in L^{\infty}(\Omega,\tribB,P)\) et que \( XZ\in L^1(\Omega,P)\). Alors
    \begin{equation}
        E(ZX|\tribB)=ZE(X|\tribB)
    \end{equation}
    presque surement.
\end{proposition}

\begin{proof}
    Nous commençons par prouver que
    \begin{equation}    \label{EqNDQWIea}
        \int_{\Omega}ZE(X|\tribB)=\int_{\Omega}ZX.
    \end{equation}
    Si \( Z=\mtu_B\) pour un ensemble \( B\in\tribB\), alors cette égalité est vraie par définition de l'espérance conditionnelle\footnote{Théorème~\ref{ThoMWfDPQ}.}. Donc cette égalité est correcte tant que \( Z\) est une variable aléatoire \( \tribB\)-mesurable et étagée. Nous considérons alors, grâce au lemme~\ref{LemYFoWqmS}, une suite \( Z_n\) de variables aléatoires étagées et \( \tribB\)-mesurables avec \( | Z_n |<Z\). Pour chaque \( n\) nous avons donc
    \begin{equation}    \label{EqNVpOSaH}
        \int_{\Omega}Z_nX=\int_{\Omega}ZE(X|\tribB).
    \end{equation}
    Notre idée est de passer à la limite. Vu que \( Z\) et \( Z_n\) sont bornées (et donc intégrables sur \( \Omega\)), pour chaque \( n\) nous avons \( | Z_nX |\leq M| X |\) où \( M\) majore \( Z\) et donc tous les \( Z_n\) de façon uniforme vis-à-vis de \( n\). Tout cela pour dire que le théorème de la convergence dominée fonctionne et que
    \begin{equation}
        \lim_{n\to \infty} \int_{\Omega}Z_nX=\int_{\Omega}ZX.
    \end{equation}
    D'autre part vu que \( X\in L^1\) et que \( \Omega\in\tribB\) nous avons l'égalité \( \int_{\Omega}E(X|\tribB)=\int_{\Omega}X\), ce qui prouve que \( | E(X|\tribB) | \) est intégrable. Cela nous permet d'utiliser encore la convergence dominée avec l'inégalité \( | Z_nE(X|\tribB) |\leq | E(X|\tribB) |\) pour écrire
    \begin{equation}
        \lim_{n\to \infty} \int_{\Omega}Z_nE(X|\tribB)=\int_{\Omega}ZE(X|\tribB).
    \end{equation}
    En passant à la limite des deux côtés de \eqref{EqNVpOSaH} nous avons donc
    \begin{equation}
        \int_{\Omega}ZE(X|\tribB)=\int_{\Omega}ZX.
    \end{equation}
    L'égalité \eqref{EqNDQWIea} est prouvée.

    Nous passons maintenant à la preuve de l'égalité demandée : \( E(EX|\tribB)=ZE(X|\tribB)\). Pour cela il faut montrer que pour tout \( B\in\tribB\) nous avons
    \begin{equation}
        \int_{B}ZE(X|\tribB)=\int_BZX.
    \end{equation}
    Cela n'est rien d'autre que l'égalité \eqref{EqNDQWIea} que nous venons de prouver avec \( Z\mtu_{B}\) au lieu de \( Z\).
\end{proof}

\begin{proposition}
    Soit une variable aléatoire réelle \( X\in L^1(\Omega,\tribA,P)\). Pour toute variable aléatoire \( Y\colon \Omega\to \eR^d\), il existe une fonction borélienne \( \tribA_Y\)-mesurable \( h\colon \eR^d\to \eR\) telle que
    \begin{equation}
        E(X|Y)=h\circ Y.
    \end{equation}
\end{proposition}

\begin{proof}
    Nous utilisons le résultat de Doob (théorème~\ref{ThofrestemesurablesXYYX}). Par définition \( E(X|Y)\) est une variable aléatoire réelle \( \tribA_Y\)-mesurable, et il existe une fonction borélienne \( h\colon \eR^d\to \eR\) telle que \( E(X|Y)=f\circ Y\).
\end{proof}

Cette fonction \( h\colon \eR^d\to \eR\) nous permet de définir\index{espérance!conditionnelle!variable aléatoire}
\begin{equation}
    E(X|Z=z)=h(z).
\end{equation}
Cela est l'espérance conditionnelle d'une variable aléatoire par rapport à une valeur donnée d'une autre variable aléatoire.

%---------------------------------------------------------------------------------------------------------------------------
\subsection{Probabilité conditionnelle : tribu}
%---------------------------------------------------------------------------------------------------------------------------

Soit un espace probabilisé \( (\Omega,\tribA,P)\).

\begin{lemma}       \label{LEMooXXTYooZCXiYr}
    Soit \( (B_i)_{i\in\eN}\) une partition de \( \Omega\) en éléments de \( \tribA\) deux à deux disjoints tels que \( P(B_i)\neq 0\). Soit \( \tribF\) la tribu engendrée par les \( B_i\). Une variable aléatoire réelle est \( \tribF\)-mesurable si et seulement si elle est constante sur chaque \( B_i\).
\end{lemma}

\begin{proposition}
        Soit \( (B_i)_{i\in\eN}\) une partition de \( \Omega\) en éléments de \( \tribA\) deux à deux disjoints tels que \( P(B_i)\neq 0\). Soit \( \tribF\) la tribu engendrée par les \( B_i\). Soit une variable aléatoire \( X\). Alors nous avons :
    \begin{equation}    \label{EqCibwoG}
        E(X|\tribF)=\sum_{i\in \eN}\left( \frac{1}{ P(B_i) }\int_{B_i}XdP \right)\mtu_{B_i}.
    \end{equation}
\end{proposition}

\begin{proof}
    Si \( X\) est une variable aléatoire, alors la variable aléatoire \( E(X|\tribF)\) définie en~\ref{ThoMWfDPQ} est une variable aléatoire \( \tribF\)-mesurable et elle est donc constante sur les ensembles \( B_i\) par le lemme~\ref{LEMooXXTYooZCXiYr} :
    \begin{equation}
        E(X|\tribF)=\sum_{i\in\eN}a_i\mtu_{B_i}.
    \end{equation}
    Étant donné que, par construction, \( B_i\) est \( \tribF\)-mesurable, nous avons
    \begin{equation}
            \int_{B_i}XdP=\int_{B_i}E(X|\tribF)
            =\sum_ja_j\int_{B_i}\mtu_{B_j}
            =\sum_ja_j\delta_{ij}P(B_j)
            =a_iP(B_i).
    \end{equation}
    Par conséquent
    \begin{equation}
        a_i=\frac{1}{ P(B_i) }\int_{B_i}XdP
    \end{equation}
    et
    \begin{equation}
        E(X|\tribF)=\sum_{i\in \eN}\left( \frac{1}{ P(B_i) }\int_{B_i}XdP \right)\mtu_{B_i},
    \end{equation}
    ce qu'il fallait.
\end{proof}

Notons que si \( B\in\tribA\) alors la tribu engendrée par \( B\) est aussi celle engendrée par la partition \( \{ B,\complement B \}\) de \( \Omega\). Cette circonstance nous permet d'aller plus loin.

\begin{proposition}
    Soit un espace probabilisé \( (\Omega,\tribA,P)\) et un événement \( B\in\tribA\) avec sa tribu engendrée \( \tribF=\sigma(B)\). Alors
    \begin{equation}
        E(\mtu_A|\tribF)=P(A|B)\mtu_B+P(A|\complement B)\mtu_{\complement B}.
    \end{equation}
\end{proposition}

\begin{proof}
    Nous allons particulariser la formule \eqref{EqCibwoG}. Si \( B\in \tribA\) nous considérons la partition \( \{ B,\complement B \}\) de \( \Omega\) et la tribu engendrée
    \begin{equation}
        \tribF=\{ \emptyset,B,\complement B,\Omega \}.
    \end{equation}
    La formule \eqref{EqCibwoG} devient
    \begin{equation}
        E(X|\tribF)=\left( \frac{1}{ P(B) }\int_BXdP \right)\mtu_B+\left( \frac{1}{ P(\complement B) }\int_{\complement B}XdP \right)\mtu_{\complement B}.
    \end{equation}
    Si nous considérons \( A\in\tribA\), nous écrivons cette égalité avec \( X=\mtu_A\) pour obtenir
    \begin{equation}
            E(\mtu_A|\tribF)=\frac{ P(A\cap B) }{ P(B) }\mtu_B+\frac{ P(A\cap\complement B) }{ P(\complement B) }\mtu_{\complement B} =P(A|B)\mtu_B+P(A|\complement B)\mtu_{\complement B}
    \end{equation}
    parce que nous avons reconnu la probabilité conditionnelle \( P(A|B)\) de la définition~\ref{DEFooGJVHooVbhVYv}.
\end{proof}

\begin{remark}
    Les nombres \(P\big( A |\sigma(B) \big)=P\big(\caract_A|\sigma(B)\big)\) n'est pas la probabilité conditionnelle de \( A\) sachant \( B\).
\end{remark}



Il nous reste à définir la probabilité conditionnelle d'un événement relativement à une variable aléatoire.

\begin{definition}      \label{DEFooFRLFooNvXuPK}
    Si la variable aléatoire \( X\) est à valeurs discrètes, nous disons que \( P(A|X)\) est la variable aléatoire de valeur
    \begin{equation}
        P(A|X)(\omega)=P(A|X=X(\omega)).
    \end{equation}
\end{definition}
Dans le cas d'une variable aléatoire à valeurs continues, cette définition ne fonctionne pas parce que la condition \( X=X(\omega)\) est souvent de probabilité nulle, tandis que c'est toujours une mauvaise idée de conditionner par rapport à un événement de probabilité nulle. C'est la base du \wikipedia{en}{Borel's_paradox}{paradoxe de Borel}. La bonne définition du conditionnement de l'événement \( A\) par rapport à la variable aléatoire $X$ est

\begin{definition}      \label{DEFooIUJMooBAVtMW}
    Si \( A\) est un événement et \( X\) une variable aléatoire à valeurs continues dans \( \eR\), nous définissons
    \begin{equation}
        P(A|X)=P(A|\sigma(X))=E\big( \mtu_A|\sigma(X) \big).
    \end{equation}
    La première égalité est une notation. La seconde est la définition.
\end{definition}
Cette définition s'appuie également sur la définition~\ref{ThoMWfDPQ}.

\begin{proposition}
    Si \( X\) est une variable aléatoire et si \( A\) est un événement, alors
    \begin{equation}
        E\big( P(A|X) \big)=P(A).
    \end{equation}
\end{proposition}

\begin{proof}
    Nous commençons par le cas discret, c'est-à-dire \( X\colon \Omega\to \eN\). Nous notons \( p_k=P(X=k)\). En décomposant l'intégrale sur \( \Omega\) par rapport à l'union disjointe
    \begin{equation}
        \Omega=\bigcup_{k\in \eN}A_k=\bigcup_{k\in \eN}\{ \omega\in\Omega \tq X(\omega)=k\},
    \end{equation}
    nous obtenons
    \begin{subequations}
        \begin{align}
            E\big( P(A|X) \big)&=\int_{\Omega}P(A|X)(\omega)dP(\omega)\\
            &=\sum_{k=0}^{\infty}\int_{A_k}P(A|X=X(\omega))dP(\omega)\\
            &=\sum_k\int_{A_k}\frac{ P(A\cap X=k) }{ P(X=k) }dP(\omega) & \text{dans } A_k\text{, } X(\omega)=k\\
            &=\sum_k\frac{1}{ p_k }P(A\cap X=k)\underbrace{\int_{A_k}1dP(\omega)}_{=P(A_k)=p_k}\\
            &=\sum_{k}P(A\cap X=k)\\
            &=P(A).
        \end{align}
    \end{subequations}
    Nous devons maintenant prouver la propriété dans le cas où \( X\) prend des valeurs continues. Pour cela il suffit d'appliquer le corolaire~\ref{CorakyvMp} :
    \begin{equation}
        E\big( E(\mtu_A|\sigma(A)) \big)=E(\mtu_A)=P(A).
    \end{equation}
\end{proof}

%---------------------------------------------------------------------------------------------------------------------------
\subsection{Variables de Rademacher indépendantes}
%---------------------------------------------------------------------------------------------------------------------------
\label{SUBSECooWOOGooVxflVZ}

Une variable aléatoire de Rademacher est une variable aléatoire qui prend les valeurs \( 1\) et $-1$ avec probabilité \( \frac{ 1 }{2}\). Nous pouvons en décrire une explicitement de la façon suivante. L'espace probabilité est à deux éléments : \( \Omega=\{ a,b \}\) avec la mesure \( P(\{ a \})=P(\{ b \})=\frac{ 1 }{2}\). La variable aléatoire est alors l'application \( X\colon \Omega\to \eR\) donnée par \( X(a)=1\) et \( X(b)=-1\).

Soient \( X\) et \( Y\) deux variables aléatoires de Rademacher indépendantes. Cela donne \( \Omega=\{ a,b \}^2\) et
\begin{equation}
    \begin{aligned}[]
        X(a,a)&=1&X(a,b)&=1&X(b,a)&=-1&X(b,b)=-1\\
        Y(a,a)&=1&Y(a,b)&=-1&Y(b,a)&=1&Y(b,b)=-1
    \end{aligned}
\end{equation}

\begin{remark}
    Si une variable aléatoire d'un certain type est donnée par une application \( X\colon \Omega\to \eR\), pour construire des variables aléatoires indépendantes identiquement distribuées, il faut considérer les variables aléatoires sur (au moins) le produit \( \Omega\times \Omega\) munie de la mesure produit.
\end{remark}

\begin{subproof}
    \item[Tribu du produit \( XY\)]

        Quelle est la tribu de la variable aléatoire produit \( XY\) ? Le produit \( XY\) peut prendre les valeurs \( 1\) et \( -1\). Nous avons
        \begin{equation}
            \begin{aligned}[]
                (XY)^{-1}(1)&=\{ (a,a),(b,b) \}
                (XY)^{-1}(-1)&=\{ (a,b),(b,a) \}
            \end{aligned}
        \end{equation}
        La tribu est donc
        \begin{equation}
            \sigma(XY)=\{  \Omega,\emptyset, A,B  \}
        \end{equation}
        avec \( A=\{ (a,a),(b,b) \}\) et \( B=\{ (a,b),(b,a) \}\).

    \item[Calcul de \( E(X|XY)\)]

        La définition de l'espérance à calculer est le théorème~\ref{ThoMWfDPQ}. Pour chaque élément \( B\) de \( \sigma(XY)\) nous avons besoin de \( \int_BX=\int_B E(X|XY)\). Nous notons \( V=E(X|XY)\) pour alléger la notation. Nous avons
        \begin{equation}
            4\int_AV=V(a,a)+V(b,b)
        \end{equation}
        et
        \begin{equation}
            4\int_AX=X(a,a)+X(1,1)=0.
        \end{equation}

        Pourquoi le facteur \( 4\) ? Parce que sur \( \Omega\) nous avons la mesure produit de celle que dont nous avions parlé sur \( \{ a,b \}\). C'est la mesure d'équiprobabilité et donc chaque singleton a mesure \( 1/4\). Pour plus de détails, il y a le théorème~\ref{ThoWWAjXzi}.

        Nous en déduisons \( V(a,a)+V(b,b)=0\). Mais pour tout \( t\in \eR\) nous avons \( V^{-1}(t)\in \sigma(XY)\) parce que la contrainte est que \( V\) soit \( XY\)-mesurable. En particulier
        \begin{equation}
            V^{-1}\big( V(a,a) \big)
        \end{equation}
        est un mesurable qui contient \( (a,a)\). C'est donc soit \( \Omega\), soit \( \{ (a,a),(b,b) \}\). Dans les deux cas nous avons \( V(a,a)=V(b,b)\) et nous en déduisons \( V(a,a)=V(b,b)=0\).

        En faisant de même avec \( \int_BV=V(a,b)+V(b,a)\) nous déduisons \( V(a,b)=V(b,a)=0\) et au final nous avons
        \begin{equation}
            E(X|XY)=0.
        \end{equation}
        Cette égalité signifie \( E(X|XY)(\omega)=0\) pour tout \( \omega\in \Omega\).

    \item[Calcul de \( E(X|X+Y)\)] Il ne faudrait pas croire que, seulement parce que \( X\) a une espérance nulle, nous trouverons une espérance nulle quel que soit le conditionnement. Juste pour le plaisir, nous calculons \( E(X|X+Y)\).

        La variable aléatoire \( X+Y\) peut prendre trois valeurs : \( -2\), \( 0\) et \( 2\). La tribu engendrée par \( X+Y\) doit en particulier contenir \( A=\{ (a,a) \}\), \( B=\{ (b,b) \}\) et \( C=\{ (a,b),(b,a) \}\).

        Nous notons \( V=E\big( X|\sigma(X+Y) \big)\). Vu que
        \begin{equation}
            \int_AV=\int_AV,
        \end{equation}
        nous avons \( V(a,a)=X(a,a)=1\). Même chose pour \( B\) qui donne \( V(b,b)=X(b,b)=-1\). En ce qui concerne l'intégrale sur \( C\) nous avons
        \begin{equation}        \label{EQooGWDYooTNxYAg}
            V(a,b)+V(b,a)=X(a,b)+X(b,a)=0.
        \end{equation}
        Par ailleurs l'ensemble \( V^{-1}\big( V(a,b) \big)\) est un ensemble mesurable qui doit au moins contenir \( (a,b)\). Vu la tribu que nous avons, cela doit également contenir \( (b,a)\), de telle sorte que \( V(a,b)=V(b,a)\). La relation \eqref{EQooGWDYooTNxYAg} nous permet alors de conclure que \( V(a,b)=V(b,a)=0\).

        Quoi qu'il en soit, l'espérance conditionnelle \( E(XY|X+Y)\) n'est pas nulle.

    \item[Calcul de \( E(XY|\sigma(XY))\)]. Celle-là, elle est facile par~\ref{NORMooHPHOooUuJWHR} : c'est \( XY\).

\end{subproof}

Nous aurions pu croire que si \( X\) et \( Y\) sont indépendantes, alors
\begin{equation}        \label{EQooHSVCooAcgRhw}
    E(XY|\tribA)=E(X|\tribA)E(Y|\tribA).
\end{equation}
L'exemple que nous venons de faire montre qu'il n'en est rien.

\begin{example}[\cite{ooPVYRooKqZQZd}]
    Un autre exemple, peut-être plus simple, pour contredire l'équation \eqref{EQooHSVCooAcgRhw}. Soient \( X\) et \( Y\) des expériences indépendantes de pile ou face non truquées. Les résultats sont représentés par \( 0\) et \( 1\). Nous notons $\tribA$  la tribu engendrée par l'événement «les résultats des deux lancers sont différents»; c'est-à-dire la tribu engendrée par l'événement $A={(1,0),(1,0)}$. La variable aléatoire \( X\) et la tribu \( \tribA\) sont indépendants (définition~\ref{DEFooVYCUooKWvReO}), donc, donc $E(X|\tribA)=E(X)=1/2$. Pareil pour Y. En revanche, le produit $XY$ est nul sur $A$ donc $E(XY|\tribA)$ aussi. Ça ne peut donc être égal à la constante $1/4=(1/2)^2$.
\end{example}

%---------------------------------------------------------------------------------------------------------------------------
\subsection{Un petit paradoxe}
%---------------------------------------------------------------------------------------------------------------------------
\label{subSecGXVYooTDdZaB}

Attention : ce qui est écrit ici est ma réflexion personnelle sur le sujet. Merci de me dire si je me trompe.

Soit une famille dont vous savez seulement qu'il y a exactement deux enfants. Trois situations :
\begin{enumerate}
    \item       \label{ITEMooNUPAooWCXwBE}
        Vous frappez, une fille ouvre la porte et dit «Bonjour, je suis l'aînée». Quelle est la probabilité que l'autre enfant soit une fille ?
    \item   \label{ITEMooBGIWooLnVCpm}
        Vous frappez, une fille ouvre la porte et dit «Bonjour». Quelle est la probabilité que l'autre enfant soit une fille ?
    \item       \label{ITEMooCBNSooMaIwwB}
        Vous demandez aux parents s'il y a au moins une fille, ils répondent «oui». Quelle est la probabilité que les deux enfants soient des filles ?
\end{enumerate}
Dans les trois cas l'intuition dit que la probabilité est \( 1/2\). Il semble que de plus la~\ref{ITEMooBGIWooLnVCpm} et la~\ref{ITEMooCBNSooMaIwwB} soient les mêmes parce que l'on sait qu'il y a une fille et on se demande quelle est la probabilité qu'il y ait deux filles.

Nous allons voir ça de plus près.

%///////////////////////////////////////////////////////////////////////////////////////////////////////////////////////////
\subsubsection{«Bonjour, je suis l'aînée»}
%///////////////////////////////////////////////////////////////////////////////////////////////////////////////////////////

\paragraph{Résolution}

Si nous notons \( X_0\) et \( X_1\) les variables aléatoires donnant le sexe des deux enfants, ce sont des variables aléatoires indépendantes et identiquement distribuées, avec \( P(X_i=f)=\frac{ 1 }{2}\). La formule \eqref{EqProbCond} de la probabilité conditionnelle ainsi que l'indépendance donnent :
\begin{equation}
    P(X_1=f|X_2=f)=\frac{ P(X_1=f,X_2=f) }{ P(X_2=f) }.
\end{equation}
Le numérateur vaut \( \frac{1}{ 4 }\) et le dénominateur vaut \( \frac{ 1 }{2}\); le résultat vaut \( \frac{ 1 }{2}\). Fin de l'histoire.

\paragraph{Simulation}

Voici un petit programme qui simule la situation. Il retourne clairement \( 1/2\).
\lstinputlisting{tex/sage/simul_famille_aine.py}

%///////////////////////////////////////////////////////////////////////////////////////////////////////////////////////////
\subsubsection{«Bonjour»}
%///////////////////////////////////////////////////////////////////////////////////////////////////////////////////////////

Nous frappons à la porte, une fille ouvre en disant «bonjour», sans préciser si elle est la première ou la seconde. Quelle est la probabilité que l'autre soit une fille ? Naïvement on croirait que la probabilité est également \( \frac{ 1 }{2}\). 

Un raisonnement moins naïf montre le contraire. 

Et nous allons voir qu'un raisonnement encore moins naïf montre que la probabilité est bien \( \frac{ 1 }{2}\).

\paragraph{Premier raisonnement (incorrect)}

Voici le raisonnement qui est, à mon avis, faux. Vu que l'enfant qui ouvre la porte est une fille, la famille a une des compositions suivantes : \( fg\), \( ff\) ou \( gf\). Le cas où une fille ouvre la porte \emph{et} que l'autre est également une fille est seulement le cas \( ff\) dont la probabilité est \( \frac{1}{ 3 }\). 

Pour justifier cela nous considérons le couple de variables aléatoires \( \left( X_1,X_2 \right)\) et le conditionnement \( A=\{ X_1=f \}\cup\{ X_2=f \}\) : évidemment \( P(A)=\frac{ 3 }{ 4 }\). Nous calculons facilement la loi du couple \( (X_1,X_2)\) conditionné à \( A\) :
\begin{equation}
    P(X_1=f,X_2=f|A)=\frac{ P(  \{ X_1=f,X_2=f \}\cap A  ) }{ P(A) }=\frac{ 1/4 }{ 3/4 }=\frac{1}{ 3 }.
\end{equation}
Donc sachant \( A\), la probabilité que la famille soit constituée de deux filles est \( \frac{1}{ 3 }\).

\paragraph{Comment faire mieux ?}

Ce calcul semble être correct, mais il ne l'est pas. Ce raisonnement fait l'hypothèse implicite que l'espace probabilisé décrivant la situation contient deux variables aléatoires \( X_1\) et \( X_2\) représentant les deux enfants. Or nous avons bien trois événements aléatoires dans l'histoires : le sexe des deux enfants et le \emph{choix} de l'enfant qui ouvre la porte.

Certes, nous pouvons penser que cette troisième variable aléatoire ne change rien. Oui oui, on peut le penser. Mais ici, on ne doit pas penser, on doit \emph{démontrer}. 

Nous allons donc rédiger un calcul complet, en introduisant toutes les variables aléatoires, et en décrivant correctement l'espace probabilisé \( \Omega\) et la mesure de probabilité \( P\).

Peut-être que ça ne changera rien. Ou peut-être pas. Mais au moins nous serons surs d'avoir résolu le problème correctement.

\paragraph{La vraie réponse}

Nous considérons les variables aléatoires \( X_0,X_1\colon \Omega_E\to \{ f,g \}\) avec probabilité \( \frac{ 1 }{2}\). De plus nous considérons une nouvelle variable aléatoire qui donne le numéro de l'enfant qui ouvre la porte :
\begin{equation}
    \sigma\colon \Omega_C\to \{ 1,2 \}.
\end{equation}

Notre espace de probabilité est donc l'ensemble \( \Omega=\{ f,g \}\times \{ f,g \}\times {0,1}\) sur lequel nous considérons la mesure d'équiprobabilité\footnote{C'est une hypothèse forte faisant appel d'un part ce que l'on sait de la reproduction humaine, et d'autre part ce que l'on sait de la sociologie de deux enfants qui entendent une sonnette.}.

Nous introduisons les variables aléatoires\footnote{Sur \( \Omega\), sur \( \{ f,g \}\) et sur \( \{ 0,1 \}\) nous mettons la tribu des parties. Vérifiez que \( X_1\), \( X_2\) et \( \sigma\) sont mesurables.}
\begin{equation}
    \begin{aligned}
        X_1\colon \Omega&\to \{ f,g \} \\
        (s_1,s_2,n)&\mapsto s_1 
    \end{aligned}
\end{equation}
et
\begin{equation}
    \begin{aligned}
        X_2\colon \Omega&\to \{ f,g \} \\
        (s_1,s_2,n)&\mapsto s_2 
    \end{aligned}
\end{equation}
et
\begin{equation}
    \begin{aligned}
        \sigma\colon \Omega&\to \{ 1,2 \} \\
        (s_1,s_2,n)&\mapsto n
    \end{aligned}
\end{equation}


Nous devons calculer
\begin{equation}
    P\big( X_{1-\sigma}=f|X_{\sigma}=f \big)=\frac{ P\big( X_{1-\sigma}=f,X_{\sigma}=f \big) }{ P(X_{\sigma}=f) }.
\end{equation}

Pour être explicite jusqu'au bout, nous énumérons tous les éléments de \( \Omega\) :
\begin{multicols}{4}
    \begin{enumerate}
        \item
            \( g,g,0\)
        \item
            \( g,g,1\)
        \item
            \( g,f,0\)
        \item
            \( g,f,1\)
        \item
            \( f,g,0\)
        \item
            \( f,g,1\)
        \item
            \( f,f,0\)
        \item
            \( f,f,1\).
    \end{enumerate}
\end{multicols}

Et tant qu'à être explicite, l'événement vulgairement noté \( \{ X_{\sigma}=f \}\) est la partie
\begin{subequations}
    \begin{align}
    \{ X_{\sigma}=f \}&=\{ \omega\in \Omega\tq X_{\sigma(\omega)}(\omega)=f \}\\
    &=\{ (s_1,s_2,n)\tq X_n(s_1,s_2,n)=f \}\\
    &=\{ (s_1,s_2,n)\tq s_n=f \}.
    \end{align}
\end{subequations}
Méditez la dernière égalité; elle n'est pas totalement indispensable au raisonnement, mais elle est cool.


Nous avons
\begin{equation}
    \{ X_{\sigma}=f \}=\{ (g,f,1),(f,g,0),(f,f,0),(f,f,1) \}.
\end{equation}
et
\begin{equation}
    \{ X_{1-\sigma}=f \}\cap\{ X_{\sigma}=f \}=\{ (f,f,0),(f,f,1) \}.
\end{equation}
Donc
\begin{equation}
    P\big( X_{1-\sigma}=f,X_{\sigma}=f \big)=\frac{ 2 }{ 8 }=\frac{1}{ 4 }
\end{equation}
et
\begin{equation}
    P(X_{\sigma}=f)=\frac{ 4 }{ 8 }=\frac{ 1 }{2}.
\end{equation}
Au final,
\begin{equation}
    P\big( X_{1-\sigma}=f|X_{\sigma}=f \big)=\frac{ 1/4 }{ 1/2 }=\frac{ 2 }{ 4 }=\frac{ 1 }{2}.
\end{equation}

\paragraph{Simulation}

Vous avez encore un doute ? Faites tourner la simulation suivante :
\lstinputlisting{tex/sage/simul_famille_simple.py}

Le faisant tourner, la réponse est sans appel : la fréquence observée est beaucoup plus proche de \( 0.5\) que de \( 0.33\) ou \( 0.66\).

%///////////////////////////////////////////////////////////////////////////////////////////////////////////////////////////
\subsubsection{Le parent qui répond aux questions}
%///////////////////////////////////////////////////////////////////////////////////////////////////////////////////////////

Nous avons une famille de deux enfants dont nous savons qu'au moins un des deux est une fille. Quelle est la probabilité que la famille contienne deux filles ? Cela est à priori la même question que celle où une fille ouvre la porte sans dire si elle est l'aînée ou non.

\paragraph{Simulation}

Commençons par la simulation :

\lstinputlisting{tex/sage/simul_famille_une_fille.py}

Et là, bing, la réponse est clairement plutôt \( 0.33\) que \( 0.5\).

\paragraph{Résolution}

Nous avons les variables aléatoires \( X_1\) et \( X_2\) qui valent \( 0\) ou \( 1\) suivant que l'enfant soit une fille ou un garçon; ce sont des variables aléatoires indépendantes et identiquement distribuées. Nous définissons la variable aléatoire somme
\begin{equation}
    S=X_1+X_2
\end{equation}
qui compte le nombre de filles. La question est de calculer
\begin{equation}
    P( S=2|S\geq 1 )=\frac{ P(S=2\cap S\geq 1) }{ P(S\geq 1) }=\frac{ P(S=2) }{ P(S\geq 1) }.
\end{equation}
L'événement \( S=2\) est réduit au singleton \( \{ ff \}\) et sa probabilité est \( \frac{1}{ 4 }\). Au contraire l'événement \( S\geq 1\) est l'ensemble \( \{ fg,gf,ff \}\) et sa probabilité est \( \frac{ 3 }{ 4 }\). Nous avons donc
\begin{equation}
    P( S=2|S\geq 1 )=\frac{ 1/4 }{ 3/4 }=\frac{1}{ 3 }.
\end{equation}
Et là, la réponse est \( 1/3\) et non \( 1/2\) comme d'aucuns auraient pu le croire.

\paragraph{Précision}
Notons que l'événement \( S\geq 1\) n'est pas le même que l'événement \( X_{\sigma}=f\). En effet
\begin{equation}
    S\geq 1=\{ (ff,1),(ff,2),(fg,1),(fg,2),(gf,1),(gf,2) \}
\end{equation}
tandis que
\begin{equation}
    \{ X_{\sigma}=f \}=\{ (ff,1),(ff,2),(fg,1),(gf,2) \}.
\end{equation}

%///////////////////////////////////////////////////////////////////////////////////////////////////////////////////////////
\subsubsection{Conclusion}
%///////////////////////////////////////////////////////////////////////////////////////////////////////////////////////////

L'internet regorge de sites discutant du paradoxe des deux enfants\footnote{Par exemple \cite{BIBooBXKDooOTEkjy}.}.

Beaucoup insistent sur le fait que non seulement certaines informations apparemment anodines sont importantes, mais en plus \emph{la façon} dont on obtient l'information est importante. Dans la situation «une fille ouvre», nous obtenons l'information «il y a au moins une fille» en en voyant une; dans la situation «la parent dit qu'il y a au moins une fille», nous obtenons l'information «il y a au moins une fille» de façon plus «pure».

Personnellement je ne souscris pas vraiment à cette façon de penser. Le fait est que la formule
\begin{equation}
    P(A|B)=\frac{ P(A\cap B) }{ P(B) }
\end{equation}
n'est pas seulement une formule dans laquelle il faut remplacer \( A\) par «la question» et \( B\) par «ce qu'on sait». Il faut également remplacer \( P\) par «la bonne» mesure de probabilité.

Il est important de construire le bon espace de probabilité, avec la bonne mesure. Et pour cela, il faut bien s'assurer d'introduire une variable aléatoire pour chaque événement aléatoire se produisant dans l'histoire.

%///////////////////////////////////////////////////////////////////////////////////////////////////////////////////////////
\subsubsection{À propos des simulations}
%///////////////////////////////////////////////////////////////////////////////////////////////////////////////////////////

Si vous lisez ces lignes avec l'intention de passer l'agrégation en utilisant Sage à l'épreuve de modélisation, vous devez être capable de refaire les trois simulations. Les bouts de code donnés ici sont écrits pour python3 alors que Sage utilise Python2. Je ne vous dit pas si ça change quelque chose.

Allez oui, je vous dit. Si vous changez dans \info{simul\_famille\_une\_fille.py} la première ligne pour utiliser python2 au lieu de python3, le résultat affiché sera \info{0} et non \info{0.333}. La raison est que dans Python2, l'opérateur \info{/} entre deux entiers est une {\bf division entière}. Autrement dit : le résultat \( 0.33\) est arrondi à zéro.

Solution : forcer python à interpréter le \info{/} comme une vraie division. Pour Sage, ça donne ceci comme début de programme :

\lstinputlisting{tex/frido/codeSnip_3.py}

Importez toujours \info{division} de \info{ \_\_future\_\_ }.

Ah oui, et dernière remarque : pour autant que je le sache, le jour de l'oral, vous n'aurez que Sage en mode notebook. Je ne sais pas si l'import fonctionne aussi bien.

Sinon vous pouvez forcer la division dans les \info{float} de la façon suivante : \info{a/float(b)}.

%---------------------------------------------------------------------------------------------------------------------------
\subsection{Inégalité de Jensen}
%---------------------------------------------------------------------------------------------------------------------------

\begin{proposition}[Inégalité de Jensen]    \label{PropABtKbBo}
    Soit \( g\) une fonction convexe\footnote{Définition~\ref{DefVQXRJQz}.} sur \( \eR\) et une variable aléatoire \( Y\in L^1(\Omega,\tribA,P)\) telle que \( g\circ Y\) soit également \( L^1\). Alors
    \begin{equation}
        g\big( E(Y|\tribF) \big)\leq E\big( (g\circ Y)|\tribF \big).
    \end{equation}
\end{proposition}
\index{inégalité!Jensen!espérance conditionnelle}

\begin{proof}
    La convexité de \( g\) et la proposition~\ref{PropPEJCgCH} nous donnent deux suites \( (a_n)\) et \( (b_n)\) dans \( \eR\) telles que pour tout \( x\in \eR\),
    \begin{equation}
        g(x)=\sup_{n\in \eN}(a_nx+b_n).
    \end{equation}
    Nous avons alors
    \begin{equation}    \label{EqVAvCziG}
        a_nE(Y|\tribF)\stackrel{p.s.}{=}E\big( a_nY+b_n|\tribF \big)\leq  E(g\circ Y|\tribF).
    \end{equation}
    L'inégalité est due au fait que \( g\circ Y\) est le supremum sur les \( n\) de \( a_nY+b_n\). Pour chaque \( n\), l'inégalité \eqref{EqVAvCziG} est fausse sur un ensemble de mesure nulle \( R_n\subset\Omega\). L'union
    \begin{equation}
        R=\bigcup_{n\in \eN}R_n
    \end{equation}
    est encore de mesure nulle. Sur \( \Omega\setminus R\), nous avons
    \begin{equation}
        a_nE(Y|\tribF)+b_n\leq E(g\circ Y|\tribF).
    \end{equation}
    Vu que cela est vrai presque partout et pour tout \( n\) nous passons a supremum et nous avons encore presque partout l'inégalité
    \begin{equation}
        \sup_{n\in\eN}\big( a_nE(Y|\tribF)+b_n \big)\leq E(g\circ Y|\tribF).
    \end{equation}
\end{proof}

Si nous ne nous intéressons pas à \( E(Y|\tribF)\) mais seulement à \( E(Y)\), alors une démonstration plus simple est donnée sur Wikipédia\cite{YMmJevi}.

\input{92_vars_al}
\input{93_vars_al}
\input{203_vars_al}

\chapter{Statistiques}
\input{94_statistiques}

\chapter{Chaînes de Markov à temps discret}
% This is part of Mes notes de mathématique
% Copyright (c) 2012-2013,2015,2018-2020
%   Laurent Claessens
% See the file fdl-1.3.txt for copying conditions.

\begin{quote}
    Mets tes deux pieds en canard, c'est la chaine de Markov qui se prépare.
\end{quote}

%+++++++++++++++++++++++++++++++++++++++++++++++++++++++++++++++++++++++++++++++++++++++++++++++++++++++++++++++++++++++++++
\section{Généralités}
%+++++++++++++++++++++++++++++++++++++++++++++++++++++++++++++++++++++++++++++++++++++++++++++++++++++++++++++++++++++++++++

Les chaines de Markov interviennent pour la description des systèmes dont l'évolution future ne dépend que de l'état présent.

\begin{definition}      \label{DEFooGDPFooWsvfRv}
    Soit \( E\) un ensemble au plus dénombrable\footnote{Une chaine de Markov sur un ensemble indénombrable demanderait plus de technique à cause du lemme \ref{LEMooQIMGooOUpZjk} qui fait que toutes les sommes sur des ensembles indénombrables sont infinies.} et \( (\Omega,\tribF,P)\) un espace de probabilité. Une \defe{chaine de Markov}{chaine!de Markov} à valeurs dans \( E\) est une famille \( (X_n)_{n\in\eN}\) de variables aléatoires telles que pour tout \( x_0,\ldots,x_{n+1}\in E\),
    \begin{equation}
        P(X_{n+1}=x_{n+1}|X_n=x_n,\ldots,X_0=x_0)=P(X_{n+1}=x_{n+1}|X_n=x_n).
    \end{equation}
\end{definition}
Pour une chaine de Markov, il n'est pas important de savoir l'historique pour prédire la futur : \( X_{n+1}\) est seulement déterminé par \( X_n\).

\begin{remark}
    Il existe une théorie des chaines de Markov à temps continu ou avec \( E\) non dénombrable, mais ce n'est pas au programme.
\end{remark}

\begin{normaltext}
    Vu que l'ensemble \( E\) des états est au plus dénombrable, nous rappelons très humblement au lecteur la proposition \ref{PropoWHdjw} qui nous permet de changer des sommes sur \( E\) en des sommes sur \( \eN\) sans nous soucier de l'ordre sur \( E\). Si \( f\) est une fonction sur \( E\), nous nous écrirons
    \begin{equation}
        \sum_{x\in E}f(x)=\sum_{k=0}^{\infty}f(x_k)
    \end{equation}
    sans citer \ref{PropoWHdjw} à chaque fois.
\end{normaltext}

\begin{definition}      \label{DEFooVVWUooKIBQDv}
    Si \( (X_n)\) est une chaine de Markov\footnote{Définition \ref{DEFooGDPFooWsvfRv}.}, nous notons
    \begin{equation}
        p_n(x,y)=P(X_{n+1}=y|X_n=x)
    \end{equation}
    la \defe{probabilité de transition}{transition!probabilité} de la chaine à l'instant \( n\). Si cette probabilité ne dépend pas de \( n\), nous disons que la chaine de Markov est \defe{homogène}{homogène!chaine de Markov}\index{chaine!de Markov!homogène}, et nous notons \( p(x,y)\) au lieu de \( p_n(x,y)\). 
\end{definition}

\begin{definition}[Matrice de transition]       \label{DEFooKQROooYvJvvl}
    Nous notons \( Q^{(n)}\) la matrice de transition\index{matrice de transition} qui est éventuellement infinie:
    \begin{equation}
        Q^{(n)}_{xy}=p_n(x,y).
    \end{equation}
    Si l'ensemble des états \( E\) est infini, ce n'est pas une matrice à proprement parler.
\end{definition}

Le lemme suivant est intuitivement rien d'autre que le fait que la somme des probabilités doit être \( 1\).
\begin{lemma}       \label{LEMooQNIWooQBMlge}
    Soit un processus de Markov \( (X_n)\) sur l'ensemble \( E\). Pour chaque \( x\in E\) pour pour tout \( n\in \eN\) nous avons
    \begin{equation}
        \sum_{y\in E}p_n(x,y)=1.
    \end{equation}
\end{lemma}

\begin{proof}
    Nous avons le calcul
    \begin{subequations}
        \begin{align}
            \sum_{y\in E}p_x(x,y)&=\sum_{y\in E}P(X_{n+1}=y|X_n=x)\\
            &=  P(\Omega|X_n=x)        \label{SUBEQooIRGEooHDOxOd}\\
            &=\frac{ P(\Omega\cap X_n=x) }{ P(X_n=x) }\\
            &=1\label{SUBEQooBRMTooQJiKfR}.
        \end{align}
    \end{subequations}
    Justifications :
    \begin{itemize}
        \item Pour \eqref{SUBEQooIRGEooHDOxOd}, c'est le lemme \ref{LEMooRDXRooQLMRGF} en observant que \( \Omega=\bigcup_{y\in E}\{ X_{n+1}=y \}\).
        \item Pour \eqref{SUBEQooBRMTooQJiKfR}, c'est \( \Omega\cap A=A\).
    \end{itemize}
\end{proof}

\begin{remark}
    Attention à ce que ce lemme \ref{LEMooQNIWooQBMlge} ne fonctionne que sur les colonnes de \( p_n\). En effet, la somme \( \sum_{x\in E}p(x,y)\) ne vaut pas spécialement \( 1\). Si les états \( x_1\) et \( x_2\) arrivent tous les deux en \( y\) de façon certaine, alors nous avons \( \sum_xp(x,y)\geq 2\). Il n'y a donc pas de limites aux sommes des lignes.
\end{remark}

\begin{lemmaDef}[Produit de matrices de transition]         \label{LEMooZIEPooXHGnvy}
    Soient deux processus de Markov \( (X_n)\) et \( (Y_n)\) sur le même ensemble \( E\). En notons \( p\) et \( q\) leurs matrices de transition\footnote{Qui n'est pas spécialement une matrice, voir la définition \ref{DEFooKQROooYvJvvl}.} alors
    \begin{enumerate}
        \item       \label{ITEMooWNWXooCKOYpE}
            La somme \( \sum_{x\in E}p(a,x)q(x,b)\) converge pour tout \( a,b\in E\).
        \item       \label{ITEMooEZIEooFEbwhj}
            Nous notons \( pq\) la «matrice de transition»
            \begin{equation}
                (pq)(a,b)=\sum_{x\in E}p(a,x)q(x,b).
            \end{equation}
        \item       \label{ITEMooKEFXooMLREkO}
            La matrice \( pq\) vérifie
            \begin{equation}
                \sum_{y\in E}(pq)(x,y)=1
            \end{equation}
            pour tout \( x\in E\).
    \end{enumerate}
\end{lemmaDef}

\begin{proof}
    Nous considérons la mesure de comptage sur \( E\) (définition \ref{DEFooILJRooByDzhs}). Cela nous permet d'écrire la somme comme une intégrale. 
    
    \begin{subproof}
        \item[Pour \ref{ITEMooWNWXooCKOYpE}]
            Nous devons donc démontrer la convergence de
            \begin{equation}        \label{EQooCNNCooGJRWyi}
                \sum_{x\in E}p(a,x)q(x,b)=\int_Ep(a,x)q(x,b)dm(x)
            \end{equation}
            où \( dm(x)\) n'est pas du tout la mesure de Lebesgue (qui n'aurait aucun sens), mais bien la mesure de comptage en \( x\).
            
            Nous pouvons majorer \( q(x,b)\) par \( 1\) (tous les nombres sont strictement plus grands que zéro) : pour chaque \( x\) nous avons
            \begin{equation}
                p(a,x)q(x,b)\leq p(a,x), 
            \end{equation}
            alors que $\int_Ep(a,x)dm(x)=1$ par le lemme \ref{LEMooQNIWooQBMlge}.

            Vu que la fonction \( x\mapsto p(a,x)q(x,b)\) est dominée par la fonction \( x\mapsto p(a,x)\) et que cette dernière est intégrable, le lemme \ref{LemPfHgal} conclu à l'intégrabilité de la première. Bref, l'intégrale \eqref{EQooCNNCooGJRWyi} existe et est finie.
        \item[Pour \ref{ITEMooEZIEooFEbwhj}]
            Il n'y a rien à prouver, c'est seulement une définition.
        \item[Pour \ref{ITEMooKEFXooMLREkO}]
            Nous devons calculer la valeur de
            \begin{equation}
                \sum_{b\in E}\big( \sum_{x\in E}p(a,x)q(x,b) \big)=\int_E\big( \int_Ep(a,x)q(x,b)dm(x) \big)dm(b).
            \end{equation}
            Pour cela nous allons utiliser le théorème de Fubini comme expliqué en \ref{NORMooKIRJooPvyPWQ}, et nous partons de la somme dans le sens inverse. D'abord nous prouvons que \( (x,b)\mapsto p(a,x)q(x,b)\) est dans \( L^1(E\times E)\) en étudiant les intégrales en chaine :
            \begin{subequations}
                \begin{align}
                    \int_E\Big( \int_E| p(a,x)q(x,b) |dm(b) \Big)dm(x)&=\int_E p(a,x)\Big( \underbrace{\int_Eq(x,b)dm(b)}_{=1} \Big)dm(x)\\
                    &=\int_Ep(a,x)dm(x)\\
                    &=1.
                \end{align}
            \end{subequations}
            Donc la fonction est \( L^1(E\times E)\) et nous pouvons fusionner et permuter les intégrales à volonté. Nous avons alors
            \begin{subequations}
                \begin{align}
                    1&=\int_E\Big( \int_E| p(a,x)q(x,b) |dm(b) \Big)dm(x)\\
                    &=\int_{E\times E}p(a,x)q(x,b)dm(x,b)\\
                    &=\int_E\Big( \int_Ep(a,x)q(x,b)dm(x) \Big)dm(b).
                \end{align}
            \end{subequations}
        \end{subproof}
\end{proof}

\begin{definition}      \label{DefGJEBooZvuIAV}
    Une matrice dont tous les éléments sont positifs ou nuls et donc la somme de toutes les lignes sont \( 1\) est une \defe{matrice stochastique}{matrice!stochastique}.
\end{definition}
Notons que l'ensemble des matrices stochastiques est un fermé dans l'ensemble des matrices.

\begin{lemma}
    Si \( U\) est une matrice stochastique\footnote{Définition~\ref{DefGJEBooZvuIAV}.}, alors il existe une chaine de Markov dont la matrice de transition est \( U\).
\end{lemma}


\begin{example}
    Nous considérons une fourmi qui se déplace dans un appartement à trois pièces \( A\), \( B\), \( C\). Supposons qu'à chaque minute, elle a une probabilité \( 1/3\) de rester dans la pièce et une probabilité \( 2/3\) de se déplacer. Le plan de l'appartement est
    \begin{equation}
        \xymatrix{%
        A \ar[r]      &  B\ar[r]&C
           }
    \end{equation}
    De la pièce \( A\) est est donc uniquement possible d'aller vers la pièce \( B\); de la \( B\) il est possible d'aller en \( A\) et en \( C\) et de la \( C\) il est uniquement possible d'aller en \( B\).

    La matrice de transition de cette chaine de Markov est
    \begin{equation}
        Q=\begin{pmatrix}
            1/3    &   2/3    &   0    \\
            1/3    &   1/3    &   1/3    \\
            0    &   2/3    &   1/3
        \end{pmatrix}
    \end{equation}
\end{example}

\begin{proposition}     \label{PROPooWAVPooHDVsER}
    Si \( N_t\) est un processus de Poisson, alors les variables aléatoires \( X_n=N_n\) forment une chaine de Markov.
\end{proposition}

%+++++++++++++++++++++++++++++++++++++++++++++++++++++++++++++++++++++++++++++++++++++++++++++++++++++++++++++++++++++++++++
\section{Chaînes de Markov sur un ensemble fini}
%+++++++++++++++++++++++++++++++++++++++++++++++++++++++++++++++++++++++++++++++++++++++++++++++++++++++++++++++++++++++++++

\begin{definition}
Une chaine de Markov est \defe{finie}{chaine!de Markov!finie} si l'ensemble \( E\) dans lequel elle prend ses valeurs est fini.
\end{definition}

\begin{proposition}[\cite{GMbugcT}]
    Si \( (X_n)\) est une chaine de Markov irréductible sur un ensemble fini, alors pour tout ensemble \( A\subset E\) nous avons
    \begin{equation}
        P(\tau_A<\infty)=\lim_{n\to \infty} P(\tau_A\leq n)=1
    \end{equation}
    où \( \tau_A=\min\{ k\tq X_k\in A \}\).
\end{proposition}

Les propositions à venir vont montrer que
\begin{enumerate}
    \item
        Toute matrice stochastique admet un état stationnaire, proposition~\ref{PropOJumFwe}.
    \item
        Si la chaine de Markov est irréductible, alors il y a unicité de l'état stationnaire, proposition~\ref{PropUMPpOHW}. Mais attention : cela ne veut pas encore dire que la chaine converge effectivement vers cet état.
    \item
        Si la chaine est irréductible et apériodique, alors il y a convergence en loi vers l'unique loi invariante, théorème~\ref{ThoQSuLZoz}.
\end{enumerate}
%TODO : il faut de preuves à tout ça; c'est dans le document cité.
%TODO : mettre les définitions utiles au dessus.

\begin{proposition}[\cite{MarkGuy}] \label{PropOJumFwe}
    Toute matrice stochastique admet un état stationnaire.
\end{proposition}

\begin{proposition}[\cite{MarkGuy}]     \label{PropUMPpOHW}
    Soit une chaine de Markov irréductible finie. Alors il existe une unique loi stationnaire \( \pi\) et de plus nous avons \( \pi_i>0\) pour tout état \( i\) de \( E\).
\end{proposition}

\begin{definition}
    Une chaine de Markov finie est \defe{régulière}{chaine!de Markov!régulière} s'il existe un \( n\in \eN\) tel que \( P^n\) a uniquement des éléments strictement positifs.
\end{definition}

\begin{theorem}[\cite{GMbugcT}]
    Soit \( P\) la matrice de transition d'une chaine de Markov régulière sur un ensemble \( E\) de cardinal \( N\). Alors il existe des nombres \( \pi_1,\ldots, \pi_N\) tels que
    \begin{enumerate}
        \item
            \( \pi_i>0\) pour tout \( i=1,\ldots, N\)
        \item
            \( \pi_1+\cdots +\pi_N=1\)
        \item
            \begin{equation}
                \lim_{n\to \infty} P^n=\Pi=\begin{pmatrix}
                     \pi_1   &   \pi_2    &   \ldots    &   \pi_N    \\
                     \vdots   &   \vdots    &       &   \vdots    \\
                     \pi_1   &   \pi_2    &   \ldots    &   \pi_N
                 \end{pmatrix}
            \end{equation}
    \end{enumerate}
    De plus le vecteur \( \pi=(\pi_1,\ldots, \pi_N)\) est l'unique solution de
    \begin{equation}
        \pi P=\pi.
    \end{equation}
\end{theorem}

\begin{proof}
    Si la chaine n'a qu'un seul état c'est évident parce que la probabilité de transition est toujours \( 1\); fin de l'histoire.

    \begin{subproof}
        \item[Hypothèse]

            Sinon nous supposons que \( P\) n'a que des éléments positifs, quitte à considérer \( P^n\) au lieu de \( P\). Nous notons \( d\) le plus petit élément de \( P\); il vérifie \( d\leq \frac{ 1 }{2}\) parce que la somme des élément d'une ligne de la matrice \( P\) doit être égale à \( 1\).

        \item[Les suites min et max]

            Soit \( x\) un vecteur quelconque (de composantes positives). Nous notons \( m_0=\min\{ x_i \}\) et \( M_0=\max\{ x_i \}\). Étant donné que les éléments du vecteur \( Px\) sont des moyennes pondérées des éléments de \( x\), si nous posons
            \begin{subequations}
                \begin{align}
                    m_k=\min\{ (P^kx)_i \}_{i=1,\ldots, N}\\
                    M_k=\max\{ (P^kx)_i \}_{i=1,\ldots, N},
                \end{align}
            \end{subequations}
            la suite \( (m_k)\) est croissante et la suite \( (M_k)\) est décroissante.

        \item[Stricte croissance et décroissance]

            Si \( M_{k+1}=M_k\), alors toutes les composantes de \( P^kx\) sont égales à \( M_k\) et le théorème est prouvé. Cela est encore une propriété de la moyenne. Même remarque pour la suite \( (m_k)\).

            Nous pouvons donc supposer que la suite \( (m_k)\) est strictement croissante et que la suite \( (M_k)\) est strictement décroissante. Elles sont toutes les deux bornées dans \( \mathopen[ m_0 , M_0 \mathclose]\). Le lemme~\ref{LemSuiteCrBorncv} nous donne la convergence.

        \item[Égalité des limites]

            Vu que les éléments de \( P^kx\) ne sont pas tous les mêmes et s'étalent de \( m_k\) à \( M_k\), pour majorer \( M_{k+1}\) nous mettons le plus petit coefficient possible (c'est-à-dire \( d\)) devant \( m_k\) et nous supposons que toutes les autres composantes sont \( M_k\); nous avons alors
            \begin{equation}
                M_{k+1}\leq dm_k+(1-d)M_k
            \end{equation}
            parce que tous les autres coefficients de la ligne contenant le \( d\) (dans \( P^k\)) sont plus petits ou égaux à \( 1-d\). De la même façon nous avons la minoration
            \begin{equation}
                m_{k+1}\geq dM_k+(1-d)m_k.
            \end{equation}
            En faisant la différence, et en nous souvenant que \( 0<1-2d<1\),
            \begin{equation}
                M_{k+1}-m_k\leq (1-2d)(M_k-m_k),
            \end{equation}
            ce qui signifie que
            \begin{equation}
                M_{k+1}-m_k\leq (1-2d)^k(M_0-m_0),
            \end{equation}
            et donc que les deux limites sont égales.

        \item[Conclusion pour la limite]

            Pour tout vecteur \( x\), la suite \( P^kx\) tend vers un vecteur dont toutes les composantes sont égales. En particulier pour le vecteur \( e_i\) de la base canonique,
            \begin{equation}
                P^ke_i\to\begin{pmatrix}
                    \pi_1    \\
                    \vdots    \\
                    \pi_1
                \end{pmatrix}.
            \end{equation}
            Mais \( P^ke_i\) est la \( i\)\ieme\ colonne de la matrice \( P^k\). Cela prouve la convergence annoncée \( P^k\to \Pi\).
    \end{subproof}

    Réglons rapidement le cas des deux autres allégations du théorème. D'abord les matrices \( P^k\) sont toutes des matrices stochastiques; et l'ensemble des matrices stochastiques est fermé, donc la convergence se fait à l'intérieur de l'ensemble des matrices stochastiques. Cela prouve que \( \pi_1+\ldots +\pi_N=1\).

    Ensuite la suite \( (m_k)\) étant strictement croissante et \( m_0\) étant égal à \( 0\) dans le cas de \( e_i\) nous avons toujours \( \pi_i>0\) (strictement).
\end{proof}

\begin{theorem}[\cite{MarkGuy}]     \label{ThoQSuLZoz}
    Si \( (X_n)\) est une chaine de Markov finie, irréductible et apériodique de loi stationnaire \( \pi\), alors
    \begin{enumerate}
        \item
            La suite de matrices stochastiques \( P^k\) converge vers la matrice
            \begin{equation}
                P^k\to\Pi=\begin{pmatrix}
                    \pi    \\
                    \vdots    \\
                    \pi
                \end{pmatrix}.
            \end{equation}
        \item
            Nous avons convergence des variables aléatoires au sens où
            \begin{equation}
                P(X_k=\mu P^k)\to \pi.
            \end{equation}
    \end{enumerate}
\end{theorem}

%+++++++++++++++++++++++++++++++++++++++++++++++++++++++++++++++++++++++++++++++++++++++++++++++++++++++++++++++++++++++++++
\section{Marche aléatoire sur \texorpdfstring{$\eZ$}{\( \eZ\)}}
%+++++++++++++++++++++++++++++++++++++++++++++++++++++++++++++++++++++++++++++++++++++++++++++++++++++++++++++++++++++++++++
\index{variable aléatoire!Bernoulli!marche aléatoire}

%TODO : dans le fichier markov.pdf de Nils Berglund, il y a des choses intéressantes, dont le fait que l'espérance du temps qu'il faut pour retourner en zéro est infinie.

Soit \( (Y_n)\) une suite de variables aléatoires indépendantes et identiquement distribuées valant \( -1\) avec une probabilité \( p\) et \( 1\) avec une probabilité \( (1-p)\). La loi est
\begin{equation}
    Y_n\sim p\delta_{-1}+(1-p)\delta_{1}.
\end{equation}
Nous considérons la variable aléatoire
\begin{equation}
    X_n=X_0+\sum_{i=1}^nY_i
\end{equation}
où \( X_0\) est une variable aléatoire indépendante des \( Y_i\) à valeurs dans \( \eZ\). Nous vérifions à présent que \( X_n\) est une chaine de Markov avec comme espace d'états \( E=\eZ\). Nous devons montrer que
\begin{equation}        \label{EqAVoirMarkovMAZ}
    P\big( X_{n+1}=x_{n+1}| X_n=x_n,\ldots,X_0=x_0 \big)=P\big( X_{n+1}=x_{n+1}|X_n=x_n \big).
\end{equation}
Pour ce faire nous allons exprimer tout cela en termes des \( Y_i\) au lieu des \( X_i\). D'abord étant donné que nous avons égalité des événements
\begin{equation}
    \{ X_{n+1}=x_{n+1} \}\cap\{ X_n=x_n,\ldots, X_0=x_0 \}= \{ Y_{n+1}=x_{n+1}-x_n \}\cap\{X_n=x_n,\ldots, X_0=x_0\},
\end{equation}
nous pouvons, en vertu du principe \eqref{EqOVHCWom}, remplacer \( X_{n+1}=x_{n+1}\) par \( Y_{n+1}=x_{n+1}-x_n\) dans le membre de gauche de \eqref{EqAVoirMarkovMAZ}. Nous avons donc déjà
\begin{equation}
    P\big( X_{n+1}=x_{n+1}| X_n=x_n,\ldots,X_0=x_0 \big)=P\big( \underbrace{Y_{n+1}=x_{n+1}-x_n}_{A}| \underbrace{X_n=x_n,\ldots,X_0=x_0}_{B} \big).
\end{equation}

L'événement \( B\) est égal à l'événement
\begin{equation}
    \{ X_0=x_0,Y_1=x_1-x_0,Y_2=x_2-x_1,\ldots,Y_n=x_n-x_{n-1} \},
\end{equation}
qui n'est autre que l'ensemble
\begin{equation}
    X_0^{-1}(x_0)\cap Y_1^{-1}(x_1-x_0)\cap\ldots\cap Y_{n}^{-1}(x_n-x_{n-1})
\end{equation}
qui est dans la tribu engendrée par les variables aléatoires \( X_0,(Y_i)_{i=1,\ldots,n}\). Le point délicat du raisonnement est de montrer que les événements \( A\) et \( B\) donnés par
\begin{subequations}
    \begin{align}
        A&=\{ Y_{n+1}=x_{n+1}-x_n \}\\
        B&=\{ X_0=x_0 \}\cap\bigcap_{i=1}^n \{ Y_i=x_i-x_{i-1}\}
    \end{align}
\end{subequations}
sont indépendants. Nous ne pouvons pas montrer directement que \( P(A\cap B)=P(A)P(B)\) parce que cela est la formule que nous voulons utiliser pour montrer que la chaine est de Markov. Nous passons donc par les tribus :
\begin{subequations}        \label{subesqqsABtribsYYXzY}
    \begin{align}
        A&\in\sigma(Y_{n+1})\\
        B&\in\sigma(X_0,Y_1,\ldots,Y_n).
    \end{align}
\end{subequations}
Nous utilisons maintenant l'hypothèse d'indépendance des variables aléatoires \( X_0\) et \( Y_i\) pour conclure que les deux tribus des équations \eqref{subesqqsABtribsYYXzY} sont indépendantes. Les événements \( A\) et \( B\) sont par conséquent indépendants.

L'événement \( A\) est indépendant de l'événement \( \{ X_n=x_n \}\). Nous avons donc successivement
\begin{subequations}
    \begin{align}
        P\big( X_{n+1}=x_{n+1}| X_n=x_n,\ldots,X_0=x_0 \big)&=P\big( Y_{n+1}=x_{n+1}-x_n| X_n=x_n,\ldots,X_0=x_0 \big)\\
        &=P\big( Y_{n+1}=x_{n+1}-x_n|  Y_i=x_i-x_{i-1},X_0=x_0\big)\\
        &=P(Y_{n+1}=x_{n+1}-x_n)        \label{EqEDQkUcm_c}\\
        &=P(Y_{n+1}=x_{x+1}-x_n|X_n=x_n)    \label{EqEDQkUcm_d}\\
        &=P(Y_{n+1}=x_{n+1}-X_n|X_n=x_n)        \label{EqEDQkUcm_e}\\
        &=P(X_{n+1}=x_{n+1}|X_n=x_n).
    \end{align}
\end{subequations}
Justifications :
\begin{itemize}
    \item \eqref{EqEDQkUcm_c} parce que les tribus $\sigma(Y_{n+1})$ et \( \sigma(Y_i,X_0)\) sont indépendantes.
    \item \eqref{EqEDQkUcm_d} Nous avons
        \begin{equation}
    \{ X_n=x_n \}\in\sigma(X_0,Y_1,\ldots, Y_n)
\end{equation}
tandis que
\begin{equation}
    \{ Y_{n+1}=x_{n+1}-x_n \}\in\sigma(Y_{n+1});
\end{equation}
ce sont donc deux événements issus de tribus indépendantes. Donc conditionner ou non l'événement \( Y_{n+1}=x_{n+1}-x_n\) à l'événement \( X_n=x_n\) ne change rien.
\item \eqref{EqEDQkUcm_e} est encore l'utilisation du fait que \( P(A|B)=P(K|B)\) dès que \( A\cap B=K\cap B\).

\end{itemize}

La chaine est par conséquent de Markov.

La matrice de transition de cette chaine de Markov est une matrice infinie «dans tous les sens» :
\begin{equation}
    p(x,y)=\begin{cases}
        p    &   \text{si }y=x-1\\
        (1-p)    &    \text{si }y=x+1\\
        0    &   \text{sinon}.
    \end{cases}
\end{equation}

\begin{remark}
    La plupart du temps lorsqu'il faut démontrer qu'une chaine est de Markov, il faut suivre la procédure que nous venons de suivre pour la marche aléatoire sur \( \eZ\).
    \begin{itemize}
        \item Écrire tout en fonction des incréments.
        \item Dire que les incréments conditionnés sont indépendants des incréments qui conditionnent (via les tribus engendrées).
        \item Écrire que la probabilité cherchée est égale à l'événement conditionné dans lequel on a juste remplacé l'incrément par sa valeur.
        \item Conditionner à nouveau par rapport au dernier incrément qui est indépendant.
        \item Changer la valeur du dernier incrément par la variable aléatoire.
    \end{itemize}
    Dans ce raisonnement nous utilisons deux fois le fait que \( P(A|B)=P(K|B)\) si \( A\cap B=K\cap B\).
\end{remark}

%---------------------------------------------------------------------------------------------------------------------------
\subsection{Chaînes de Markov homogènes}
%---------------------------------------------------------------------------------------------------------------------------

\begin{proposition}     \label{PROPooYIDWooAKTVvS}
    Voici quelques propriétés des chaines de Markov homogènes\footnote{Définition \ref{DEFooVVWUooKIBQDv}.}.
    \begin{enumerate}
        \item       \label{ITEMooSDDUooVRnpjv}
            La probabilité d'une trajectoire donnée est
            \begin{equation}
                P(X_n=x_n,X_{n-1}=x_{n-1},\ldots,X_0=x_0)=p(x_{n-1},x_n)\dots p(x_0,x_1)P(X_0=x_0)
            \end{equation}
            où les \( p(x,y)\) sont les probabilités de transitions introduits dans la définition \ref{DEFooVVWUooKIBQDv}.
        \item       \label{ITEMooJZNRooXFQTQc}
            La probabilité de transition «en \( n\) coups» est donnée par la puissance \( n\)\ieme\ de la matrice de transition :
            \begin{equation}
                P(X_n=x_n|X_0=x_0)=Q^n_{x_0,x_n}.
            \end{equation}
        \item       \label{ITEMooJUEMooWXEkBO}
            Si l'espace des états \( E\) est fini, l'espérance d'une fonction bornée\footnote{L'hypothèse de borne sur \( f\) n'est pas très chère parce que \( E\) est fini. Il suffit que \( f\) ne soit infinie en aucun point.} \( f\colon E\to \eR\) de l'état est donnée par
            \begin{equation}
                    E\big( f(X_{n+1})|X_n=x_n,\ldots,X_0=x_0 \big)=E\big( f(X_{n+1})|X_n=x_n \big)
                    =\sum_{y\in E}f(y)p(x_n,y)
            \end{equation}
            pour tout \( x_0,\ldots, x_n\in E\).
    \end{enumerate}
    Pour être précis, ce que nous notons «\( f(X_{n+1})\)» est la composée \( f\circ X_{n+1}\).
\end{proposition}

\begin{proof}
    En plusieurs étapes.
    \begin{subproof}
    \item[Pour \ref{ITEMooSDDUooVRnpjv}]
        Nous écrivons la formule \( P(A\cap B)=P(A|B)P(B)\) pour les événements \( A=\{ X_n=x_n \}\) et \( B=\bigcap_{i=0}^{n-1}\{ X_i=x_i \}\):
            \begin{equation}
                \begin{aligned}[]
                P(X_n=x_n,\ldots,&X_0=x_0)=\\
                    \qquad &P(X_n=x_n|X_{n-1}=x_{n-1},\ldots,X_0=x_0)P(X_{n-1}=x_{n-1},\ldots,X_0=x_0).
                \end{aligned}
            \end{equation}
            Par la propriété de Markov, le premier facteur est
            \begin{equation}
                P(X_n=x_n|X_{n-1}=x_{n-1})=p(x_{n-1},x_n).
            \end{equation}
            Le reste est une récurrence sur \( n\).

        \item[Pour \ref{ITEMooJZNRooXFQTQc}]
            Montrons avec \( n=2\). En utilisant les divers points du théorème~\ref{ThoBayesEtAutres}, nous avons
            \begin{subequations}
                \begin{align}
                    P(X_2=x_2|X_0=x_0)&=\sum_{y\in E}P(X_2=x_2,X_1=y|X_0=x_0)\\
                    &=\sum_{y\in E}P(X_2=x_2|X_1=y,X_0=x_0)P(X_1=y|X_0=x_0)\\
                    &=\sum_{y\in E}P(X_2=x_2|X_1=y)P(X_1=y|X_0=x_0)\\
                    &=\sum_{y\in E}p(x_2,y)p(y,x_0) \label{subEqyExdyyxz}\\
                    &=Q^2_{x_2,x_0}.
                \end{align}
            \end{subequations}
            Nous avons utilisé l'homogénéité de la chaine de Markov au moment d'écrire l'expression \eqref{subEqyExdyyxz}. En principe nous aurions dû écrire \( p_2(y,x_2)p_1(x_0,y)\).

        \item[Pour \ref{ITEMooJUEMooWXEkBO}]
            Vu que \( E\) est fini, \( f(E)\) est fini et nous notons \( \{a_k \}_{k=1,\ldots, N}\) l'ensemble des valeurs non nulles (dans \( \eR\)) atteintes par \( f\). Nous utilisons le lemme \ref{LEMooRTVBooCEeIxL} pour la variable aléatoire\footnote{La composée de fonctions mesurables est mesurable, proposition \ref{PROPooEFHKooARJBwW}. De plus \( f\) est mesurable parce que \( E\) étant dénombrable, nous y mettons la tribu de toutes les parties.} \( f\circ X_{n+1}\colon \Omega\to \{ 0,a_k \}_{k=1,\ldots,N}\) :
            \begin{equation}    \label{EQooLRZJooJGceeV}
                E\big( f(X_{n+1})|X_n=x_n,\ldots, X_0=x_0 \big)=\sum_{k=1}^{N}a_kP\big( f\circ X_{n+1}=a_k|X_n=x_n,\ldots, X_0=x_0 \big).
            \end{equation}
            L'événement \( f\circ X_{n+1}=a_k\) signifie \( X_{n+1}\in f^{-1}(a_k)\) ou encore
            \begin{equation}
                \{ \omega\in \Omega\tq X_{n+1}(\omega)\in f^{-1}(a_k) \}.
            \end{equation}
            Vu que \( E\) est fini, l'ensemble \( f^{-1}(a_k)\) est fini et nous écrivons
            \begin{equation}
                f^{-1}(a_k)=\{ y_{k,1},\ldots, y_{k,N_k} \}.
            \end{equation}
            Nous avons
            \begin{equation}
                \{ \omega\in \Omega\tq X_{n+1}(\omega)\in f^{-1}(a_k) \}=\bigcup_{i=1}^{N_k}\{ \omega\in \Omega\tq X_{n+1}(\omega)=y_{k,i} \}.
            \end{equation}
            Nous pouvons utiliser le lemme \ref{LEMooRDXRooQLMRGF} pour décomposer
            \begin{equation}
                P(f\circ X_{n+1}=a_k|\cdots)=\sum_{i=1}^{N_k}P(X_{n+1}=y_{k,i}|\cdots)
            \end{equation}
            et continuer \eqref{EQooLRZJooJGceeV} pas
            \begin{subequations}
                \begin{align}
                    E\big( f(X_{n+1})|X_n=x_n,\ldots, X_0=x_0 \big)&=\sum_{k=1}^Na_k\sum_{i=1}^{N_k}P(X_{n+1}=y_{k,i}|X_n=x_n,\ldots, X_0=x_0)\\
                    &=\sum_{k=1}^Na_k\sum_{i=1}^{N_k}P(X_{n+1}=y_{k,i}|X_n=x_n)     \label{SUBEQooOWSUooTlUXAs}\\
                    &=\sum_{k=1}^Na_kP\big( f\circ X_{n+1}=a_k|X_n=x_n \big)\\
                    &=E\big( f(X_{n+1})|X_n=x_n \big).
                \end{align}
            \end{subequations}
                Pour \eqref{SUBEQooOWSUooTlUXAs}, nous avons utilisé la propriété de Markov.
    \end{subproof}
\end{proof}

%---------------------------------------------------------------------------------------------------------------------------
\subsection{Graphe de transition}
%---------------------------------------------------------------------------------------------------------------------------

Le \defe{graphe de transition}{graphe!de transition (chaine de Markov)} d'une chaine de Markov est le graphe dont les sommets sont les éléments de l'espace des états de la chaine et dont les sommets sont reliés par des arrêtes pondérées par la probabilité de transition correspondante.

\begin{definition}
    Une chaine de Markov est \defe{irréductible}{irréductible!chaine de Markov}\index{chaine!de Markov!irréductible} si pour tout \( x,y\in E\), il existe \( n\) tel que \( p^n(x,y)>0\) où
    \begin{equation}
        p^n(x,y)=P(X_n=y|X_0=x).
    \end{equation}
    Le nombre \( n\) peut dépendre de \( x\) et \( y\).
\end{definition}

\begin{lemma}
    Une chaine de Markov homogène est irréductible si et seulement si son graphe de transition est connexe.
\end{lemma}

\begin{proof}
    Pour chaque couple \( (x,y)\in E^2\) nous avons
    \begin{equation}
        \begin{aligned}[]
            p^n(x,y)&=\sum_{z_i\in E}P(X_n=y,X_{n-1}=z_{n-1},\ldots,X_1=z_1,X_0=x)\\
            &=\sum_{z_i}p(z_{n-1},y)p(z_{n-2},z_{n-1})\ldots p(z_1,z_2)p(x,z_1).
        \end{aligned}
    \end{equation}
    La positivité d'un des termes de la somme signifie que le graphe est connexe tandis que la positivité de \( p^n(x,y)\) signifie que la chaine est irréductible.
\end{proof}

%---------------------------------------------------------------------------------------------------------------------------
\subsection{Chaîne de Markov définie par récurrence}
%---------------------------------------------------------------------------------------------------------------------------

%///////////////////////////////////////////////////////////////////////////////////////////////////////////////////////////
\subsubsection{Le cas général}
%///////////////////////////////////////////////////////////////////////////////////////////////////////////////////////////

\begin{proposition}     \label{PropqiMdHh}
    Soit \( X_0\) une variable aléatoire à valeurs dans $E$, un ensemble au plus dénombrable. Soit \( (Y_n)\) une suite de variables aléatoires réelles indépendantes et identiquement distribuées indépendantes de \( X_0\).

    Soit \( (X_n)\) la suite de variables aléatoires à valeurs dans \( E\) définie par récurrence selon la formule
    \begin{equation}
        X_{n+1}=G(X_n,Y_{n+1})
    \end{equation}
    où \( G\colon E\times \eR\to E\) est une fonction mesurable. Alors \( (X_n)\) est une chaine de Markov.
\end{proposition}

\begin{proof}
    Soient \( x_0,\ldots, x_{n+1}\) des éléments de \( E\). Nous devons calculer la valeur de
    \begin{equation}
        P(X_{n+1}=x_{n+1}|X_n=x_n,\ldots, X_0=x_0).
    \end{equation}
    Commençons par préciser les espaces sur lesquels nos variable aléatoires sont définies. Nous avons
    \begin{equation}
        X_0\colon \Omega_0\to E
    \end{equation}
    et
    \begin{equation}
        Y_i\colon \Omega\to \eR.
    \end{equation}
    La variable aléatoire \( X_1\) est donnée par
    \begin{equation}
        \begin{aligned}
            X_1\colon \Omega_0\times \Omega&\to E \\
            (\omega_0,\omega_1)&\mapsto G\big( X_0(\omega_0),Y_1(\omega_1) \big).
        \end{aligned}
    \end{equation}
    La variable aléatoire \( X_2\) est
    \begin{equation}
        \begin{aligned}
            X_2\colon \Omega_0\times \Omega^2&\to E \\
            (\omega_0,\omega_1,\omega_2)&\mapsto G\big( X_1(\omega_0,\omega_1),Y_2(\omega_2) \big)\\
                &\quad=G\Big( G\big( X_0(\omega_0),\omega_1 \big),Y_2(\omega_2) \Big)
        \end{aligned}
    \end{equation}
    et ainsi de suite.

    Considérons maintenant l'événement
    \begin{equation}
        \{ X_1=x_1,X_0=x_0 \}\subset \Omega_0\times \Omega.
    \end{equation}
    Il est donné explicitement par
    \begin{subequations}
        \begin{align}
            \{ X_1=x_1,X_0=x_0 \}&=\{ (\omega_0,\omega_1)\tq G\big( X_0(\omega_0),Y_1(\omega_1)=x_1,X_0(\omega_0)=x_0 \big) \}\\
            &=\{ (\omega_0,\omega_1)\tq G\big( x_0,Y_1(\omega_1)=x_1,X_0(\omega_0)=x_0 \big) \}\\
            &=\{ \omega_0\in \Omega_0\tq X_0(\omega_0)=x_0 \}\times \{ \omega_1\in \Omega\tq G\big( x_0,Y_1(\omega_1)=x_1 \big) \}.
        \end{align}
    \end{subequations}
    Le premier terme du produit cartésien est dans \( \sigma(X_0)\), tandis que le second est dans \( \sigma(Y_1)\). Étant donné la définition des tribus produit (définition~\ref{DefTribProfGfYTuR}) nous avons
    \begin{equation}
        \{ X_1=x_1,X_0=x_0 \}\in\sigma(X_0,Y_1).
    \end{equation}
    Ce raisonnement se généralise immédiatement et nous trouvons que
    \begin{equation}
        \{ X_n=x_n,\ldots, X_0=x_0 \}\in\sigma(X_0,Y_1,\ldots, Y_n).
    \end{equation}
    Nous sommes donc à calculer
    \begin{subequations}
        \begin{align}
        \diamondsuit&=P(X_{n+1}=x_{n+1}|X_n=X_n,\ldots, X_0=X_0)\\
        &=P\big( \underbrace{G(x_n,Y_{n+1})=x_{n+1}}_{\in\sigma(Y_{n+1})}|\underbrace{X_n=x_n,\ldots, X_0=x_0}_{\in\sigma X_0,Y_1,\ldots, Y_n} \big).
        \end{align}
    \end{subequations}
    Les tribus \( \sigma(Y_{n+1})\) et \( \sigma(X_0,Y_1,\ldots, Y_n)\) étant indépendantes nous avons
    \begin{subequations}
        \begin{align}
            \diamondsuit&=P\big( G(x_n,Y_{n+1})=x_{n+1} \big)\\
            &=P\big( G(x_n,Y_{n+1})=x_{n+1}|X_n=X_n \big)       \label{jdvyUK}\\
            &=P\big( G(X_n,Y_{n+1})=x_{n+1}|X_n=x_n \big)\\
            &=P(X_{n+1}=x_{n+1}|X_n=x_n).
        \end{align}
    \end{subequations}
    Pour \eqref{jdvyUK} nous avons utilisé le fait que \( \sigma(Y_{n+1})\) est indépendante de \( \sigma(X_n)\). Nous avons prouvé que la chaine était de Markov.
\end{proof}
Les probabilités de transition de la chaine de Markov définie dans la proposition~\ref{PropqiMdHh} sont
\begin{equation}
    P(X_1=y|X_0=x)=P\big( G(X_0,Y_1)=y|X_0=x_0 \big)=P\big( G(x_0,Y_1)=y \big).
\end{equation}

%///////////////////////////////////////////////////////////////////////////////////////////////////////////////////////////
\subsubsection{Exemple : la file de réparation de machines à laver}
%///////////////////////////////////////////////////////////////////////////////////////////////////////////////////////////

Nous considérons un magasin de réparation d'électroménager. Durant le jour \( n\), un nombre aléatoire \( Z_{n}\) de machines en panne arrivent au magasin. Une machine est réparée chaque jour (aucune si le magasin est vide). Nous supposons que les \( Z_n\) soient indépendantes et identiquement distribuées, et nous posons \( X_n\), le nombre de machines en magasin le jour \( n\).

La loi d'avancement de \( X_n\) est
\begin{equation}
    X_{n+1}=\begin{cases}
        X_n+Z_n-1    &   \text{si } X_n\neq 0\\
        Z_n    &    \text{si } X_n=0.
    \end{cases}
\end{equation}
Cela est une chaine de Markov en vertu de la proposition~\ref{PropqiMdHh}. Ici la fonction est
\begin{equation}
    G(x,y)=x+y-\mtu_{x\neq 0}.
\end{equation}
Les probabilités de transitions sont
\begin{equation}
    p(x,y)=\begin{cases}
        0    &   \text{si } x\leq y-2\\
        P(Z=0)    &    \text{si } x=y-1\\
        P(Z=k)&\text{si } x=y+k-1
    \end{cases}
\end{equation}
pour \( x\neq 0\).


\begin{example}

Soit \( (X_n)\) une chaine de Markov de matrice de transition
\begin{equation}
    P=\begin{pmatrix}
        0.2    &   0.5    &   0.3    \\
        0.1    &   0.1    &   0.8    \\
        0.5    &   0.2    &   0.3
    \end{pmatrix}.
\end{equation}
Calculer \( P(X_3=1|X_0=1)\) et \( P(X_7=0|X_4=0)\).

Déterminer, s'il en existe, une loi stationnaire vers laquelle converge la chaine.

    Nous avons 
    \begin{equation}
        P^3=\begin{pmatrix}
            0.344    &   0.251    &  0.405     \\
            0.283    &   0.307    &   0.41    \\
            0.287    &   0.248    &   0.465
        \end{pmatrix}.
    \end{equation}
    La probabilité d'aller de l'état \( 1\) à l'état \( 1\) en trois étapes est donc \( 0.307\). La chaine étant de Markov, sans mémoire, les probabilités entre les temps \( 4\) et \( 7\) sont les mêmes qu'entre \( 0\) et \( 3\). Nous avons alors
    \begin{equation}
        P(X_7=0|X_4=0)=0.344.
    \end{equation}
    
    La chaine est irréductible et n'a pas d'états absorbants.



\end{example}

%+++++++++++++++++++++++++++++++++++++++++++++++++++++++++++++++++++++++++++++++++++++++++++++++++++++++++++++++++++++++++++
\section{Classification des états}
%+++++++++++++++++++++++++++++++++++++++++++++++++++++++++++++++++++++++++++++++++++++++++++++++++++++++++++++++++++++++++++

Sauf mention expresse du contraire, nous considérons toujours une chaine de Markov homogène.

\begin{definition}
    Un état \( x\in E\) est \defe{absorbant}{absorbant} pour la chaine \( (X_n)\) si \( p(x,x)=1\).
\end{definition}
Il n'est pas spécialement impossible d'arriver sur un état absorbant, mais il est impossible d'en sortir.

Si \( x\in E\), nous notons
\begin{equation}
    T(x)=\inf\{ k\geq 1\tq X_k=x \},
\end{equation}
le \defe{premier temps d'atteinte}{premier temps d'atteinte} de l'état \( x\). Si \( X_0=x\), alors \( T(x)\) est le \defe{temps de retour}{temps de retour} en \( x\). Si \( p\in \eN\) nous notons
\begin{equation}
    T_p(x)=\inf\{ k\geq 1\tq X_{k+p}=x \}.
\end{equation}
C'est le temps mis pour atteindre \( x\) à partir de l'instant \( p\).

\begin{proposition}
    La loi de la variable aléatoire \( [T_p(x)|X_p=x]\) est la même que celle de la variable aléatoire \( [T(x)|X_0=x]\).
\end{proposition}

\begin{proof}
    Nous devons montrer que
    \begin{equation}
        P(T_p(x)=k|X_p=x)=P(T(x)=k|X_0=x).
    \end{equation}
    Cela est intuitivement évident du fait qu'une chaine de Markov soit un processus sans mémoire. Afin de prouver, nous allons sommer sur tous les états intermédiaires possibles :
    \begin{subequations}
        \begin{align}
            P&(T_p(x)=k|X_0=x)=P(X_{p+k}=x,X_{p+k-1}\neq x,\ldots,X_{p+1}\neq x|X_p=x)\\
            &=\sum_{z_i\neq x}P(X_{p+k}=x,X_{p+k-1}=z_{k-1},\ldots,X_{p+1}=z_1|X_p=x)\\
            &=\sum_{z_i}P(X_{p+k}=x,X_{p+k-i}=z_i|X_{p+1}=z_1,X_p=x)\underbrace{P(X_{p+1}=z_1|X_p=x)}_{=p(x,z_1)}\\
            &=\sum_{z_i}P(X_{p+k}=x,X_{p+k-i}=z_i|X_{p+2}=z_2,X_{p+1}=z_1,X_p=x)\\
            &\qquad\underbrace{P(X_{p+2}=z_2|X_{p+1}=z_1,X_p=x)}_{P(X_{p+2}=z_2|X_{p+1}=z_1)=p(z_1,z_2)}p(x,z_1)\\
            &=\ldots\\
            &=\sum_{z_i}p(x,z_1)p(z_1,z_2)\ldots p(z_{k-1},z_{k-1})p(z_{k-1},x).
        \end{align}
    \end{subequations}
    À ce point ci, nous avons éliminé toute référence à \( p\) grâce à l'homogénéité de la chaine. Nous pouvons refaire le calcul à l'envers pour reconstituer l'expression de départ sans le \( p\) :
    \begin{subequations}
        \begin{align}
         \sum_{z_i}p(x,z_1)p(z_1,z_2)\ldots &p(z_{k-1},z_{k-1})p(z_{k-1},x)\\
         &=P(x_k=x,X_{k-1}\neq x,\ldots,X_1\neq x|X_0=x)\\
         &=P(T(x)=k),
        \end{align}
    \end{subequations}
    ce qu'il fallait obtenir.
\end{proof}

\begin{definition}\label{DefWknULk}
    Un état \( x\) est \defe{récurrent}{récurrent!état}\index{état!récurrent} si \( P(T(x)=\infty|X_0=x)=0\), c'est-à-dire si la probabilité de ne jamais retourner en $x$ lorsqu'on y est passé est nulle. L'état \( x\) est \defe{transient}{transient!état} ou \defe{transitoire}{transitoire!état}\index{état!transitoire} dans le cas contraire.

    Si \( x\) est un état récurrent, et si \( E\big( T(x)|X_0=x \big)<\infty\), nous disons que \( x\) est \defe{récurrent positif}{récurrent!positif}\index{état!récurrent positif}. Si \( E\big( T(x)|X_0=x \big)=\infty\) alors nous disons que est \defe{récurrent nul}{récurrent!nul}.
\end{definition}

Nous introduisons une variable aléatoire qui compte le nombre de fois que la chaine de Markov passe par l'état \( x\) :
\begin{equation}    \label{EqDefNxmtuXkn}
    N_x=\sum_{k=0}^{\infty}\mtu_{\{ X_k=x\}}.
\end{equation}
C'est une variable aléatoire à valeurs dans \( \eN\cup\{ \infty \}\).

\begin{proposition} \label{PropEquivEPrecuequiv}
    Les deux propriétés suivantes sont équivalentes à dire que \( x\) est récurrent:
    \begin{enumerate}
        \item
            \( P(N_x<\infty|X_0=x)=0\)
        \item
            \( E(N_x|X_0=x)=\infty\).
    \end{enumerate}
    Les deux propriétés suivantes sont équivalentes à dire que \( x\) est transient :
    \begin{enumerate}
        \item   \label{ItemiMnGpD}
            \( P(N_x<\infty|X_0=x)=1\)
        \item
            \( E(N_x|X_0=x)<\infty\).
    \end{enumerate}
\end{proposition}

\begin{proof}
    En tant que événements, nous avons l'égalité
    \begin{equation}
        N_x<\infty=\bigcup_{n\in\eN}\{\underbrace{ X_n=x,X_{n+k}\neq x\forall k\geq 1}_{F_n} \}.
    \end{equation}
    Nous avons donc
    \begin{equation}    \label{Eqreprencalculstd}
        P(N_x<\infty|X_0=x)=\sum_{n=0}^{\infty}P(F_n|X_0=x),
    \end{equation}
    et
    \begin{subequations}
        \begin{align}
            P(F_n|X_0=x)&=P(X_{n+k}\neq x, \forall k\geq 1,X_n=x|X_0=x)\\
            &=P(X_{n+k}\neq x,k\geq 1|X_n=x,X_0=x)P(X_n=x|X_0=x)\\
            &=P(X_{n+k}\neq x,k\geq 1|X_n=x)P(X_n=x|X_0=x)  \label{subEqPFnXkneqxii}\\
            &=P(X_k\neq x,k\geq 1|X_0=x)P(X_n=x|X_0=x)  \label{subEqPFnXkneqxi}\\
            &=P(T(x)=\infty|X_0=x)P(X_n=x|X_0=x)    \label{subEqPFnXkneqxiii}
        \end{align}
    \end{subequations}
    Justifications :
    \begin{enumerate}
        \item
            Pour \eqref{subEqPFnXkneqxii}, nous utilisons le fait que la chaine soit «sans mémoire».
        \item
            Pour \eqref{subEqPFnXkneqxi}, nous utilisons le fait que la chaine soit homogène.
        \item
            Pour \eqref{subEqPFnXkneqxiii}, l'événement \( X_k\neq x\) pour tout \( k\geq 1\) est exactement l'événement \( T(x)=\infty\).
    \end{enumerate}
    En nous servant de la proposition~\ref{PropInversSumIntFub} (théorème de Fubini et mesure de comptage), nous permutons l'espérance et la somme dans l'expression
    \begin{subequations}        \label{EqPEEEntstq}
        \begin{align}
            \sum_{n=0}^{\infty}P(X_n=x|X_0=x)&=\sum_{n=0}^{\infty}E(\mtu_{\{ X_n=x \}}|X_0=x)\\
            &=E\big( \sum_{n=0}^{\infty}\mtu_{\{ X_n=x \}}|X_0=x \big)\\
            &=E(N_x|X_0=x).
        \end{align}
    \end{subequations}
    Voyons ce passage plus en détail. D'abord, en général nous avons
    \begin{equation}
            E(Y|X=x_0)=\int_{\{ X=x_0 \}}Y(\omega)dP(\omega)
            =\int_{\Omega}\mtu_{\{ X=x_0 \}}(\omega)Y(\omega)dP(\omega).
    \end{equation}
    Dans notre cas,
    \begin{equation}
        E\big( \mtu_{\{ X_n=x \}}|X_0=x \big)=\int_{\Omega}\mtu_{X_0=x}(\omega)\mtu_{\{ X_n=x \}}(\omega)dP(\omega).
    \end{equation}
    La fonction qui correspond à la proposiiton~\ref{PropInversSumIntFub} est
    \begin{equation}
        f(n,\omega)=f_n(\omega)=\delta_{X_0(\omega),x}\delta_{X_n(\omega),x},
    \end{equation}
    qui est bien une fonction positive et mesurable.

    Nous reprenons à présent le calcul \eqref{Eqreprencalculstd} en remplaçant les éléments par leurs valeurs que nous avons calculées :
    \begin{equation}    \label{EqPnxXzTxarn}
        P(N_x<\infty|X_0=x)=P\big(T(x)=\infty|X_0=x\big)E(N_x|X_0=x).
    \end{equation}
    Si \( x\) est récurrent, nous avons \( P\big( T(x)=\infty|X_0=x \big)=0\), mais la relation \eqref{EqPnxXzTxarn} ne permet pas de conclure que le membre de gauche est nul parce qu'il reste la possibilité que \( E(N_x|X_0=x)=\infty\). Nous devons donc faire un pas en arrière et écrire cette espérance comme la limite des sommes partielles :
    \begin{equation}
        P(N_x<\infty|X_0=x)=\lim_{N\to \infty} \sum_{n=0}^NP\big( T(x)=\infty|X_0=x \big)P(X_n=x|X_0=x)=0
    \end{equation}
    parce que tous les termes de la suite des sommes partielles sont nuls. Nous avons donc bien que \( P(N_x<\infty|X_0=x)=0\). Il s'ensuit immédiatement que \( E(N_x|X_0=x)=1\).

    Nous devons maintenant démontrer l'implication inverse. Supposons que \( P(N_x<\infty|X_0=x)=0\). Dans ce cas nous avons immédiatement \( P(N_x=\infty|X_0=x)=1\) et \( E(N_x|X_0=x)=\infty\). L'équation \eqref{EqPnxXzTxarn} nous indique alors que
    \begin{equation}
        P\big( T(x)=\infty|X_0=x \big)=0,
    \end{equation}
    c'est-à-dire que \( x\) est récurrent.
\end{proof}

%---------------------------------------------------------------------------------------------------------------------------
\subsection{Chaînes irréductibles}
%---------------------------------------------------------------------------------------------------------------------------

\begin{proposition}     \label{Proptoustanstousrecirrsi}
    Soit \( (X_n)\) une chaine de Markov irréductible.
    \begin{enumerate}
        \item
            Un état \( x\) est récurrent si et seulement si tous les états sont récurrents.
        \item
            Un état \( x\) est transient si et seulement si tous les états sont transients.
    \end{enumerate}
\end{proposition}

\begin{proof}
    Soient \( x\) et \( y\) des états de la chaine de Markov. Nous devons tester la valeur de \( P(X_n=y|X_0=y)\). Afin d'exploiter l'hypothèse d'irréductibilité, nous considérons \( r,s\in\eN\) tels que
    \begin{subequations}
        \begin{align}
            p^r(x,y)>0\\
            p^s(y,x)>0
        \end{align}
    \end{subequations}
    et nous calculons majorons en passant par quelques intermédiaires :
    \begin{subequations}
        \begin{align}
            P(X_{n+r+s}=y|X_0=y)&\geq P(X_{n+r+s}=y,X_{n+s}=x,X_s=x|X_0=y)\\
            &=P(X_{n+r+s}=y|X_{n+s}=x,X_s=x,X_0=y)\\
            &\qquad P(X_{n+s}=x|X_s=x,X_0=y)P(X_s=x|X_0=y)\nonumber.
        \end{align}
    \end{subequations}
    Les deux premiers facteurs se calculent en utilisant la propriété de Markov et l'homogénéité de la chaine. Pour le premier,
    \begin{equation}
        P(X_{n+s}=x|X_s=x,X_0=y)=P(X_{n+s}=x|X_s=x)=P(X_n=x|X_0=x).
    \end{equation}
    Nous avons donc
    \begin{equation}        \label{EqRmuXtG}
        \sum_{n\in\eN}P(X_{n+r+s}=y|X_0=y)\geq p^r(x,y)p^s(y,x)\sum_{n\in\eN}P(X_n=x|X_0=x).
    \end{equation}
    En réutilisant Fubini comme dans l'équation \eqref{EqPEEEntstq}, nous avons
    \begin{equation}
        \sum_{n\in \eN}P(X_{n+r+s}=y|X_0=y)\geq KE(N_x|X_0=x)
    \end{equation}
    où \( K\) est une constante strictement positive, par hypothèse d'irréductibilité de la chaine de Markov.

    Si \( x\) est un état récurrent, alors le membre de gauche est infini par la proposition \eqref{PropEquivEPrecuequiv} et donc
    \begin{equation}
        \sum_{n\in\eN}P(X_{n+r+s}=y|X_0=y)=\infty.
    \end{equation}
    Aux \( r+s\) premiers termes près (qui ne changent pas la somme), nous avons
    \begin{equation}
        \sum_{n\in\eN}P(X_n=y|X_0=y)=\infty,
    \end{equation}
    ce qui signifie que \( y\) est récurrent.
\end{proof}

Nous rappelons que \( T(x)\) est le temps que première atteinte de l'état \( x\). Nous notons\nomenclature[M]{\( \pi(x)\)}{lié au temps de retour}
\begin{equation}        \label{EqKyuLYk}
    \pi(x)=\frac{1}{ E\big( T(x)|X_0=x \big) }.
\end{equation}
Étant donné que \( T(x)\) est un entier positif ou nul nous avons \( E\big( T(x)|X_0=x \big)\in\mathopen[ 1 , \infty \mathclose]\) et donc \( \pi(x)\in\mathopen[ 0 , 1 \mathclose]\).

Si \( x\) est un état transient, alors \( T(x)=\infty\) lorsque \( X_0=x\) et donc \( E\big( T(x)|X_0=x \big)=0\) et \( \pi(x)=0\). Si \( x\) est récurrent par contre, \( P\big( T(x)<\infty|X_0=x \big)=1\) et il n'y a pas de garanties sur la valeur de \( E\big( T(x)|X_0=x \big)\).

\begin{corollary}       \label{CorLhpRsk}
    Un état récurrent est récurrent positif si et seulement si \( \pi(x)>0\). Un état récurrent est récurrent nul si et seulement si \( \pi(x)=0\).
\end{corollary}

\begin{proof}
    C'est la formule \eqref{EqKyuLYk}.
\end{proof}

\begin{proposition} \label{PropMrkIrreLoishLCkpjkptXk}
    Soit \( (X_n)\) est une chaine de Markov irréductible.
    \begin{enumerate}
        \item
            Si \( x\) est un état récurrent, alors \( T(X)<\infty\) presque surement.
        \item
            Nous avons une égalité entre les lois
            \begin{equation}    \label{PropMrkIrreLoishLCkpjkptXkItemii}
                \hL\big( X_{k+T(x)}|T(x)<\infty \big)=\hL(X_k|X_0=x).
            \end{equation}
    \end{enumerate}
\end{proposition}

%---------------------------------------------------------------------------------------------------------------------------
\subsection{Nombre de visites}
%---------------------------------------------------------------------------------------------------------------------------

La fonction
\begin{equation}
    \frac{1}{ n }\sum_{k=1}^n\mtu_{\{ X_k=x \}}
\end{equation}
est la \defe{fréquence empirique}{fréquence!empirique} de la chaine de Markov.

Soit \( x\) un état récurrent, c'est-à-dire que \( P\big( T(x)<\infty|X_0=x \big)=1\). Nous classons les visites de la façon suivante :
\begin{subequations}    \label{SubEqsDefTirectempsretou}
    \begin{align}
        T_1(x)&=T(x)=\inf\{ k\geq 1\tq X_k=x \}\\
        T_2(x)&=\inf\{ k\geq 1\tq X_{T_1(x)+k}=x \}\\
        &\vdots\\
        T_n(x)&=\inf\{ k\geq 1\tq X_{T_{n-1}(x)+k}=x \}
    \end{align}
\end{subequations}
La variable aléatoire \( T_i\) représente le temps entre la visite numéro \( i-1\) et la visite numéro \( i\) (si \( X_0\neq x\), sinon il faut décaler). Nous définissons l'instant la na visite numéro \( n\) :
\begin{equation}
    S_n=\sum_{k=1}^nT_k(x).
\end{equation}

\begin{lemma}
    Les variables aléatoires \( T_i\) sont indépendantes.
\end{lemma}

\begin{proof}
    Nous choisissons \( n\) des \( T_i\) et nous calculons la probabilité
    \begin{equation}
        \spadesuit=P(T_{i_1}=k_1,T_{i_2}=k_2,\ldots,T_{i_n}=k_n)
    \end{equation}
    où nous supposons \( i_1>i_2>\ldots>i_n\). Nous décomposons cette probabilité en sommant sur toutes les histoires de la chaine de Markov compatibles avec les nombres \( k_i\) donnés :
    \begin{equation}
        \spadesuit=\sum_{\substack{\{ z_j \}\\\text{compatibles}}}P(X_j=z_j,j=1,\ldots,N).
    \end{equation}
    Notons qu'ici, le numéro du dernier terme de la somme n'est pas certain parce que tous les \( T_i\) ne sont pas fixés. Nous l'avons noté \( N\), mais en réalité il est différent d'un terme à l'autre de la somme. Il est certain que \( z_N=x\) et \( z_{N-k_1}=x\) et si \( N-k_1<j<N\), alors \( z_j\neq x\). Cela est simplement le fait que nous demandions aux \( z_i\) de respecter les conditions données par les \( k_i\). Nous avons
    \begin{subequations}
        \begin{align}
            \spadesuit&=\sum_{\{ z_j \}} P(X_N=x,X_j=z_j,N-k_1<j<N|X_j=z_j,j\leq N-k_1)P(X_j=z_j,j<N-k_1)\\
            &=\sum_{\{ z_j \}} P(X_N=x,X_j=z_j,N-k_1<j<N|X_{N-k_1}=x)P(X_j=z_j,j<N-k_1)\\
        \end{align}
    \end{subequations}
    Le premier facteur est \( P(T_{i_1}=k_1)\) tandis que le second facteur est précisément \( P(T_{j}=k_j,j>1)\). Nous avons donc montré que
    \begin{equation}
        P(T_{i_1}=k_1,T_{i_2}=k_2,\ldots,T_{i_n}=k_n)=P(T_{i_1}=k_1)P(T_{j}=k_j,j>1),
    \end{equation}
    et donc les \( T_i\) sont indépendants.
\end{proof}

\begin{proposition}     \label{PropjOjDux}
    Si \( (X_n)\) est une chaine de Markov irréductible et si \( x\in E\) alors
    \begin{equation}        \label{EqPiEndempropropkeu}
        \pi(x)=\lim_{n\to \infty} \frac{1}{ n }\sum_{k=1}^n\mtu_{\{ X_k=x \}}
    \end{equation}
    presque surement.
\end{proposition}

\begin{proof}
    Étant donné que la chaine est irréductible, les états sont soit tous transient soit tous récurrents par la proposition~\ref{Proptoustanstousrecirrsi}. Nous commençons par considérer que \( x\) est transient.

    En comparant la définition \eqref{EqDefNxmtuXkn} de \( N_x\) et le membre de droite de \eqref{EqPiEndempropropkeu}, nous avons pour chaque \( n\) l'inégalité
    \begin{equation}    \label{EqkkueusnfracunENx}
        \frac{1}{ n }\sum_{k=1}^n\mtu_{\{ X_k=x \}}\leq\frac{1}{ n }E(N_x).
    \end{equation}
    Dans le cas d'un élément transient, nous avons \( \pi(x)=0\), donc il serait bon de montrer que \( E(N_x)<\infty\), de sorte que prendre la limite \( n\to\infty\) dans \eqref{EqkkueusnfracunENx} donne zéro.

    Nous décomposons le calcul en deux morceaux :
    \begin{equation}
        E(N_x)=E\big( N_x|T(x)=\infty \big)P\big( T(x)=\infty \big)+E\big( N_x|T(x)<\infty \big)P\big( T(x)<\infty \big).
    \end{equation}
    Le fait que le premier terme soit fini découle immédiatement du fait que \( T(x)=\infty\) implique \( X_k\neq x\) pour tout \( k\geq 1\). Dans ce cas l'espérance de \( N_x\) est évidemment finie.

    Pour le second terme nous avons
    \begin{subequations}
        \begin{align}
            E\big( N_x|T(x)<\infty \big)&=E\big( \sum_{k=0}^{\infty}\mtu_{\{ X_k=x \}}|T(x)<\infty \big)\\
            &=\sum_{k=1}^{\infty}E\big( \mtu_{\{ X_k=x \}}|T(x)<\infty \big).
        \end{align}
    \end{subequations}
    Pour inverser la somme et l'espérance, nous avons utilisé le théorème de théorème de Fubini-Tonelli qui est encore valable pour des fonctions qui prennent la valeur \( \infty\). Le fait d'inverser ne signifie pas que ni la somme ni l'intégrale soit finie. D'ailleurs c'est exactement ce que nous sommes en train de déterminer.

    Étant donné que nous voulons seulement savoir si cette somme est finie ou non, nous pouvons nous restreindre à la somme depuis \( k=1\) ou oublier le premier terme. D'autre par nous avons
    \begin{equation}
        \sum_{k=1}^{\infty}\mtu_{\{ X_k=x \}}=\sum_{j=0}^{\infty}\mtu_{\{ X_{j+T(x)}=x \}}
    \end{equation}
    parce que les \( T(x)\) premiers termes sont par définition nuls. Nous regardons donc
    \begin{subequations}
        \begin{align}
            \sum_{j=0}^{\infty}E\big( \mtu_{X_{j+T(x)}=x}|T(x)<\infty \big)&=\sum_{j}P\big( X_{j+T(x)}=x|T(x)<\infty \big)\\
            &=\sum_jP(X_j=x|X_0=x)  \label{subeqsumkPXjXzsezii}\\
            &=\sum_jE\big( \mtu_{\{ X_j=x \}}|X_0=x \big)\\
            &=E\big( \sum_j\mtu_{X_j=x}|X_0=x \big)\\
            &=E(N_x|X_0=x)\\
            &<\infty    &\text{parce que } x\text{ est transient.}
        \end{align}
    \end{subequations}
    L'équation \eqref{subeqsumkPXjXzsezii} provient de la proposition~\ref{PropMrkIrreLoishLCkpjkptXk} et plus précisément de l'égalité entre les lois \eqref{PropMrkIrreLoishLCkpjkptXkItemii}. Nous avons terminé la preuve dans le cas où \( x\) est transient.

    Nous passons maintenant au cas où \( x\) est récurrent, c'est-à-dire \( P(T(x)<\infty|X_0=x)=1\). Les variables aléatoires \( T_i\) définies en \eqref{SubEqsDefTirectempsretou} pour \( i\geq 2\) sont indépendantes et identiquement distribuées et
    \begin{equation}
        \hL\big( T_k(x) \big)\sim\hL\big( T(X)|X_0=x \big).
    \end{equation}
    La loi des grands nombres nous indique que
    \begin{subequations}        \label{EqlgnMarkdemked}
        \begin{align}
            \frac{ S_n }{ n }=\frac{ T_1(x) }{ n }+\frac{1}{ n }\sum_{k=2}^nT_k(x)\stackrel{p.s.}{\longrightarrow}& E\big( T_2(x) \big)\\
            &=E\big( T(x)|X_0=x \big).
        \end{align}
    \end{subequations}
    \begin{remark}
        La loi des grands nombres est encore vraie sans l'hypothèse de variables aléatoires dans \( L^1\) pourvu qu'elles soient positives. Alors dans la conclusion de la loi nous devons accepter la possibilité que l'espérance soit infinie.
    \end{remark}
    Nous posons pour \( m\in\eN\)
    \begin{equation}    \label{EqDennmsumjmtu}
        n(m)=\sum_{j=1}^m\mtu_{\{ X_j=x \}}
    \end{equation}
    qui est le nombre de visites de \( x\) avant l'instant \( m\). Nous avons évidemment \( n(m)\leq m\). Mais \( S_n\) est l'instant de la \( n\)ième visite, par conséquent \( S_{n(m)}\) est l'instant de la dernière visite avant le moment \( m\). Pour tout \( m\) nous avons les inégalités
    \begin{equation}
        S_{n(m)}\leq m<S_{n(m)+1}.
    \end{equation}
    Nous divisons par \( n(m)\) et nous effectuons la limite \( m\to\infty\):
    \begin{equation}    \label{EqdrembSnm}
        \frac{ S_n(m) }{ n(m) }\leq \frac{ m }{ n(m) }\leq\frac{ S_{n(m)}+1 }{ n(m) }
    \end{equation}
    En ce qui concerne la limite de \( n(m)\), nous utilisons la définition \eqref{EqDennmsumjmtu} :
    \begin{equation}
        n(m)\to\sum_{j=1}^{\infty}\mtu_{\{ X_j=x \}}=
    \end{equation}
heur\ldots
     \begin{equation}
         \lim_{m\to \infty}  n(m)=\lim_{m\to \infty} \sum_{n=1}^m\mtu_{\{ X_j=x \}}\stackrel{p.s.}{\longrightarrow}\infty
     \end{equation}
     par la proposition \eqref{PropEquivEPrecuequiv}. Plus précisément, la limite vaut \( N_x\) qui vaut presque surement \( \infty\) dans le cas où \( x\) est récurrent. Par ailleurs la loi des grands nombres \eqref{EqlgnMarkdemked} nous enseigne en particulier que
     \begin{equation}
         \frac{ S_{n(m)} }{ n(m) }\stackrel{p.s.}{\longrightarrow} E\big( T(x)|X_0=x \big).
     \end{equation}
     Le terme de droite dans \eqref{EqdrembSnm} se traite de façon usuelle :
     \begin{equation}
         \frac{ S_{n(m)+1} }{ n(m) }=\frac{ S_{n(m)+1} }{ n(m)+1 }\frac{ n(m)+1 }{ n(m) }.
     \end{equation}
     Le dernier facteur tend vers \( 1\) et le tout a pour limite \( E\big( T(x)|X_0=x \big)\). Par conséquent nous avons
     \begin{equation}
         \frac{ m }{ n(m) }\stackrel{p.s.}{\longrightarrow}E\big( T(x)|X_0=x \big)
     \end{equation}
     et
     \begin{equation}
         \frac{ n(m) }{ n }=\frac{1}{ m }\sum_{j=1}^m\mtu_{\{ X_j=x \}}\to\frac{1}{ E\big( T(x)|X_0=x \big) }=\pi(x).
     \end{equation}
\end{proof}

\begin{lemma}       \label{LembyftKs}
    Soit $(X_k)$ une chaine de Markov dont l'espace des états est noté $E$. Pour chaque $ x\in E$ nous notons
    \begin{equation}
        T(x)=\inf\{ k\geq 1\tq X_k=x \}
    \end{equation}
    et
    \begin{equation}
        T_p(x)=\inf\{ k\geq 1\tq X_{k+p}=x \}
    \end{equation}
    Alors nous avons
    \begin{equation}
        P(T_p(x)=k|X_p=y)=P(T(x)=k|X_0=y).
    \end{equation}
\end{lemma}

La proposition suivante nous permet de parler de chaine de Markov \defe{récurrence positive}{chaine!de Markov!récurrente positive}.
\begin{proposition}     \label{PropUyLCzp}
    Soit \( (x_n)\) une chaine de Markov irréductible.
    \begin{enumerate}
        \item
            Un état \( x\) est transient si et seulement si tous les états sont transients.
        \item
            Un état est récurrent positif  si et seulement si tous les états sont récurrents positifs.
    \end{enumerate}
\end{proposition}

\begin{proof}
    Nous rappelons (proposition~\ref{PropjOjDux}) que si la chaine est irréductible
    \begin{equation}        \label{EqZZMqsm}
        \pi(x)=\lim_{n\to \infty} \frac{1}{ n }\sum_{k=1}^n\mtu_{[X_k=x]}
    \end{equation}
    Notons aussi que
    \begin{equation}
        \sum_{k=1}^N\mtu_{X_k=x}=\begin{cases}
            0    &   \text{si } N<T(x)\\
            \sum_{k=0}^{N-T(x)}\mtu_{X_{k+T(x)}=x}    &    \text{si } N>T(x)
        \end{cases}
    \end{equation}
    où dans la seconde ligne nous avons effectué le changement de variable de sommation \( k'=k+T(x)\). Dans la limite \eqref{EqZZMqsm} nous sommes toujours dans le cas où \( N\) est assez grand. Nous pouvons donc écrire
    \begin{equation}
        \pi(x)=\lim_{N\to \infty} \frac{1}{ N }\sum_{k=0}^{N-T(x)}\mtu_{X_{k+T(x)}=x}.
    \end{equation}
    Nous pouvons aussi écrire
    \begin{equation}
        \frac{1}{ N-T(x) } \sum_{k=0}^{N-T(x)}\mtu_{X_{k+T(x)}=x}=\frac{ N }{ N-T(x) }\frac{1}{ N } \sum_{k=0}^{N-T(x)}\mtu_{X_{k+T(x)}=x}.
    \end{equation}
    Dans cette dernière égalité le membre de droite tend vers \( \pi(x)\) et nous avons
    \begin{equation}
        \lim_{N\to \infty} \frac{1}{ N-T(x) } \sum_{k=0}^{N-T(x)}\mtu_{X_{k+T(x)}=x}=\pi(x)
    \end{equation}
    ou encore
    \begin{equation}
        \lim_{N\to \infty} \frac{1}{ N} \sum_{k=0}^{N}\mtu_{X_{k+T(x)}=x}=\pi(x)
    \end{equation}
    Étant donné que $\pi(x)$ est une constante nous avons évidemment $E(\pi(x))=\pi(x)$. Nous pouvons cependant considérer les variables aléatoires
    \begin{equation}
        Z_n=\frac{1}{ n }\sum_{k=1}^n\mtu_{X_{k+T(x)}=x}
    \end{equation}
    et remarquer que $Z_n\stackrel{p.s.}{\longrightarrow} \pi(x)$ avec $0\leq Z_n\leq 1$. Le théorème de la convergence dominée (\ref{ThoConvDomLebVdhsTf}) nous permet d'inverser la limite et l'espérance et écrire
    \begin{subequations}
        \begin{align}
            \pi(x)&=\lim_{n\to \infty} \frac{1}{ n }\sum_{k=1}^nE\big( \mtu_{X_{k+T(x)}=x} \big)\\
            &=\lim_{n\to \infty} \frac{1}{ n }\sum_{k=1}^nP\big( X_{k+T(x)}=x \big).
        \end{align}
    \end{subequations}
    Par le lemme~\ref{LembyftKs} nous avons
    \begin{equation}
        P(X_{k+T(x)}=x)=P(X_k=k|X_0=x)
    \end{equation}
    et $\pi(x)$ prend la forme
    \begin{equation}        \label{EqurCteK}
        \pi(x)=\lim_{n\to \infty} \frac{1}{ n }\sum_{k=1}^nP(X_k=x|X_0=x).
    \end{equation}

    Soit maintenant un état \( x\) positif récurrent et \( y\), un autre état. Par définition~\ref{DefWknULk} et par corolaire~\ref{CorLhpRsk} nous avons \( \pi(x)>0\). Nous devons prouver que \( \pi(y)>0\).

    Étant donné que la chaine est irréductible il existe \( r\) et \( s\) tels que
    \begin{subequations}
        \begin{numcases}{}
            p^r(x,y)=P(X_r=y|X_0=x)>0\\
            p^s(x,y)=P(X_s=x|X_0=y)>0
        \end{numcases}
    \end{subequations}
    Nous reprenons l'équation \eqref{EqRmuXtG} multipliée par \( 1/N\) :
    \begin{equation}
        \frac{1}{ N }\sum_{n=1}^NP(X_{r+s+n=y|X_0=y})\geq \underbrace{p^r(x,y)p^s(y,x)}_{\geq 0}\underbrace{\frac{1}{ N }\sum_{n=1}^NP(X_n=x|X_0=x)}_{\to \pi(x)}
    \end{equation}
    et nous prenons la limite lorsque \( N\to\infty\). À \(r+s\) termes près, nous trouvons à gauche l'expression \eqref{EqurCteK} de \( \pi(y)\). Par conséquent
    \begin{equation}
        \pi(y)\geq\lim_{N\to \infty} \frac{1}{ n }\sum_{n=1}^NP(X_{r+s+n}=y|X_0=y)\geq \alpha\pi(x)
    \end{equation}
    où \( \alpha\) est une constante positive. Le nombre \( \pi(x)\) étant strictement positif par hypothèse nous avons montré que \( \pi(y)>0\), c'est-à-dire que \( y\) est récurrent positif.
\end{proof}

%+++++++++++++++++++++++++++++++++++++++++++++++++++++++++++++++++++++++++++++++++++++++++++++++++++++++++++++++++++++++++++
\section{Mesure invariante}
%+++++++++++++++++++++++++++++++++++++++++++++++++++++++++++++++++++++++++++++++++++++++++++++++++++++++++++++++++++++++++++

\begin{definition}
    Une mesure de probabilité \( \mu\) sur l'espace des états \( E\) d'une chaine de Markov est \defe{invariante}{invariante!mesure!pour une chaine de Markov} si pour tout \( x\in E\)
    \begin{equation}
        \mu(x)=\sum_{y\in E}p(y,x)\mu(y).
    \end{equation}
\end{definition}
\begin{remark}
    Une mesure invariante est une mesure de probabilité et nous noterons par abus $ \mu(x)$ pour $\mu(\{x\})$. Si \( A\subset E\) nous avons
    \begin{equation}
        \mu(A)=\sum_{x\in A}\mu(x).
    \end{equation}
\end{remark}

\begin{remark}  \label{RemwcRRFZ}
    Une loi invariante associée à une chaine de Markov est une loi associée à la matrice de transition de la chaine, mais pas à la loi de $X_0$. Par conséquent nous pouvons tester si \( \mu\) est une mesure invariante pour une certaine chaine de Markov $(X_k)$ en considérant la chaine $(Y_k)$ avec $Y_k=X_k$ pour $k>0$ et $Y_0$ arbitraire.
\end{remark}

L'adjectif \emph{invariant} provient du lemme suivant.
\begin{lemma}       \label{LemUVMwbM}
    Soit \( (X_n)\) une chaine de Markov telle que \( X_0\sim\mu\) où \( \mu\) est une mesure invariante sur l'espace des états. Alors \( X_k\sim \mu\) pour tout \( k\).
\end{lemma}

\begin{proof}
    Par hypothèse, \( P(X_0=x)=\mu(x)\). Ensuite nous avons
    \begin{subequations}
        \begin{align}
            P(X_1=y)&=\sum_{x\in E}P(X_1=y|X_0=x)P(X_0=x)\\
            &=\sum_xp(x,y)\mu(x)\\
            &=\mu(y).
        \end{align}
    \end{subequations}
    Par conséquent \( X_1\) suit également la loi \( \mu\). Par récurrence tous les états suivent cette même loi.
\end{proof}
Si les états d'une chaine de Markov ont comme loi une mesure invariante, alors nous disons que la chaine est \defe{stationnaire}{stationnaire!chaine de Markov}.

\begin{remark}\label{RemcOEylF}
    Pour une chaine de Markov stationnaire de loi invariante $\mu$ nous avons
    \begin{equation}
        \mu(x)=\sum_yp(y,x)\mu(y)
    \end{equation}
    et si l'ensemble \( E\) est fini cette équation signifie
    \begin{equation}
        \mu=Q\mu
    \end{equation}
    où \( Q\) est la matrice de transition de la chaine de Markov.
\end{remark}


\begin{theorem}[Théorème ergodique]
    Une chaine de Markov irréductible est positive récurrente si et seulement si elle accepte une mesure invariante. Cette mesure est invariante est alors unique et vérifie \( \mu=Q\mu\) où \( Q\) est la matrice de transition.
\end{theorem}

\begin{proof}
    Nous allons seulement prouver le théorème ergodique dans le cas où \( E\) est fini. Soit \( (X_n)\) une chaine de Markov récurrente positive; nous avons \( \pi(x)>0\) pour tout \( x\in E\). Nous allons montrer que \( \pi\) est une mesure invariante.

    Nous commençons par montrer que
    \begin{equation}
        \sum_{x\in E}\pi(x)=1.
    \end{equation}
    Pour cela nous reprenons la propriété de chaine irréductible pour écrire
    \begin{equation}
        \pi(x)=\lim_{N\to \infty}\frac{1}{ N }\sum_{k=1}^N\mtu_{X_k=x}
    \end{equation}
    Étant donné que \( E\) est fini nous pouvons sommer sur \( x\in E\) et permuter la somme avec la limite :
    \begin{equation}
        \sum_{x\in E}\pi(x)=\lim_{N\to \infty} \frac{1}{ N }\sum_{k=1}^N\underbrace{\sum_{x\in E}\mtu_{X_k=x}}_{=1}.
    \end{equation}
    Nous nous retrouvons donc avec \( \lim_{N\to \infty} \frac{1}{ N }N=1\). La fonction \( \pi\) définit donc bien une mesure de probabilité sur \( E\).

    Nous montrons à présent que cette mesure est invariante, c'est-à-dire que
    \begin{equation}
        \pi(x)=\sum_{y\in E}p(y,x)\pi(y).
    \end{equation}
    Pour cela nous utilisons encore le théorème de la convergence dominée pour permuter la limite et l'intégrale dans
    \begin{equation}        \label{EqcKxNcL}
        \pi(x)=E(\pi(x))=\lim_{N\to \infty} \frac{1}{ N }\sum_{k=1}^N\underbrace{E\big( \mtu_{X_k=x} \big)}_{P(X_k=x)}=\lim_{N\to \infty} \frac{1}{ N }\sum_{k=1}^NP(X_{k+1}=x).
    \end{equation}
    La dernière égalité découle du fait que en divisant par \( N\) et en faisant tendre \( N\) vers l'infini, le fait d'enlever un terme à la somme ne change pas la valeur de la limite. Nous pouvons substituer dans \eqref{EqcKxNcL} la valeur
    \begin{equation}
        P(X_{k+1}=x)=\sum_{y\in E}p(y,x)P(X_k=y).
    \end{equation}
    Nous avons alors
    \begin{subequations}
        \begin{align}
            \pi(x)&=\lim_{N\to \infty} \frac{1}{ N }\sum_{k=1}^N\sum_{y\in E}p(y,x)P(X_k=y)\\
            &=\sum_{y\in E}p(y,x)\lim_{N\to \infty} \frac{1}{ N }\sum_{k=1}^NP(X_k=y)\\
            &=\sum_{y\in E}p(y,x)\pi(y),
        \end{align}
    \end{subequations}
    ce qui signifie que \( \pi\) est une mesure invariante. Notons que nous avons encore utilisé le fait que \( E\) soit fini pour permuter avec la limite.

    Il nous reste à montrer l'unicité de la mesure invariante sur la chaine de Markov. Soit \( \mu\) une mesure invariante pour la chaine de Markov $(X_k)$. Comme indiqué dans la remarque~\ref{RemwcRRFZ} nous pouvons supposer que $X_0$ suit la loi \( \mu\). Par le lemme~\ref{LemUVMwbM} nous avons \( P(X_k=x)=\mu(x)\) pour tout \( k\). Par conséquent
    \begin{equation}
        \pi(x)=\lim_{N\to \infty} \frac{1}{ N }\sum_{k=1}^NP(X_k=x)=\mu(x).
    \end{equation}
\end{proof}

\begin{theorem}[loi des grands nombres pour les chaine de Markov]\index{loi!des grands nombres!pour les chaines de Markov}
    Soit \( (X_n)\) une chaine de Markov irréductible acceptant une mesure invariante. Soit \( f\colon E\to \eR\) une fonction dans \( L^1(E,\mu)\). Alors nous avons
    \begin{equation}
        \frac{1}{ N }\sum_{k=1}^Nf(X_k)\stackrel{p.s.}{\longrightarrow}\sum_{x\in E}f(x)\mu(x).
    \end{equation}
\end{theorem}
En ce qui concerne les notations, l'hypothèse \( f\in L^1(E,\mu)\) signifie
\begin{equation}
    \sum_{x\in E}| f(x) |\mu(x)=\int_E| f(x) |d\mu(x)<\infty.
\end{equation}

\begin{proof}
    Nous prouvons le théorème dans le cas où \( E\) est fini. Si nous écrivons
    \begin{equation}
        f(X_k)=\sum_{y\in E}f(y)\mtu_{X_k=y},
    \end{equation}
    alors
    \begin{equation}
        \frac{1}{ N }\sum_{k=1}^Nf(X_k)=\sum_{y\in E}f(y)\frac{1}{ N }\sum_{k=1}^N\mtu_{X_k=y}.
    \end{equation}
    Étant donné que \( E\) est fini nous pouvons permuter les sommes et prendre la limite \( N\to\infty\):
    \begin{equation}
        \lim_{N\to \infty} \frac{1}{ N }\sum_kf(X_k)=\sum_{y\in E}f(y)\lim_{N\to \infty} \frac{1}{ N }\sum_{k=1}^N\mtu_{X_k=y}=\sum_{y\in E}f(y)\pi(y).
    \end{equation}
% TODO : il me semble que je dois encore terminer cette preuve.
\end{proof}

%+++++++++++++++++++++++++++++++++++++++++++++++++++++++++++++++++++++++++++++++++++++++++++++++++++++++++++++++++++++++++++
\section{Convergence vers l'équilibre}
%+++++++++++++++++++++++++++++++++++++++++++++++++++++++++++++++++++++++++++++++++++++++++++++++++++++++++++++++++++++++++++

Nous voudrions savoir sous quelles conditions la variable aléatoire \( X_n\) converge en loi vers quelque chose lorsque \( n\to \infty\). Une telle loi limite doit dépendre de la loi initiale\footnote{Lorsque la loi limite ne dépend pas de la loi initiale, nous disons que la chaine de Markov est ergodique, nous y reviendrons.} comme le montre l'exemple de la chaine de Markov
\begin{equation}
\xymatrix{%
A\ar@(dl,ul)^1  & C\ar[l]_{1/2}\ar[r]^{1/2} &  B\ar@(dr,ur)_1
   }
\end{equation}
Si \( X_0=C\), alors la loi limite est
\begin{equation}
    \frac{ 1 }{2}(\delta_A+\delta_B).
\end{equation}
Si par contre \( X_0=B\), la loi limite est \(\delta_B\). Notons que la chaine de Markov proposée ici est irréductible.

Notons qu'il n'y a pas toujours de lois limite comme le montre l'exemple
\begin{equation}
    \xymatrix{%
        A\ar@<1ex>[r]^1  & B\ar@<1ex>[l]^1
       }
\end{equation}
avec \( X_0=A\). La loi en est
\begin{equation}
    X_k=\begin{cases}
        \delta_A    &   \text{si } k\text{ est pair}\\
        \delta_B    &    \text{si } k\text{ est impair}.
    \end{cases}
\end{equation}

\begin{lemma}
    Si nous avons une loi limite
    \begin{equation}    \label{EqmbQMAY}
        P(X_n=x)\to l(x),
    \end{equation}
    et que la chaine est irréductible, alors nous avons \( l=\pi\).
\end{lemma}

\begin{proof}
    D'après la proposition~\ref{PropjOjDux} nous avons
    \begin{equation}
        \frac{1}{ n }\sum_{k=1}^nP(X_k=x)\to \pi(x).
    \end{equation}
    Par le lemme~\ref{LemyGjMqM} sur la moyenne de Cesaro et l'hypothèse \eqref{EqmbQMAY}, nous avons aussi
    \begin{equation}
        \frac{1}{ n }\sum_{k=1}^nP(X_k=x)\to l(x).
    \end{equation}
    Du coup \( \pi(x)=l(x)\).
\end{proof}

\begin{lemma}[\cite{MarkGuy}]
    Si \( \pi\) est une loi stationnaire et si \( x\) est un étant transient, alors \( \pi(x)=0\).
\end{lemma}
Ce lemme (qui peut être prouvé rigoureusement) est principalement dû au fait que la chaine de Markov ne visite un état transitoire qu'un nombre fini de fois par la proposition~\ref{PropEquivEPrecuequiv}\ref{ItemiMnGpD}.

\begin{definition}  \label{DefCxvOaT}
    Un état \( x\in E\) est \defe{apériodique}{état!apériodique}\index{apériodique!état d'une chaine de Markov} si
    \begin{equation}
        \pgcd\{ n\geq 1\tq p^n(x,x)>0 \}=1.
    \end{equation}
\end{definition}
Mettons que tous les \( n\) tels que \( p^n(x,x)>0\) ont \( 2\) comme diviseur. L'état n'est alors pas apériodique, mais on voit que si \( X_0=x\), alors les états impairs ne peuvent pas être sur \( x\). Cela est une forme de périodicité.

Si un état est apériodique, il existe \( p\) et \( q\) premiers entre eux tels que \( p^p(x,x)\) et \( p^q(x,x)\) sont non nuls. En particulier pour tout \( n\in p\eN+q\eN\), \( P(X_n=x)\neq 0\). Par conséquent la proposition~\ref{PropLAbRSE} nous indique qu'à partir d'un certain moment tous les \( X_k\) pourraient être \( x\).

L'état \( C\) de la chaine de Markov suivante est apériodique :
\begin{equation}
    \xymatrix{%
    A\ar@<0.5ex>[rr]^1  && B\ar@<0.5ex>[ll]^{2/3}\ar[dl]^{1/3}\\
    &C\ar[ul]^1
       }
\end{equation}
En effet \( p^3(C,C)\neq 0\) par le chemin \( C\to A\to B\to C\) tandis que \( p^5(C,C)\neq 0\) également par le chemin \( C\to A\to B\to A\to B\to C\). Or \( \pgcd\{ 3,5 \}=1\).

\begin{proposition}[\cite{MarkGuy}]     \label{PropSaOysS}
    Soit \( (X_n)\), une chaine de Markov irréductible. Un état \( x\) est apériodique si et seulement s'il existe \( N\) tel que
    \begin{equation}
        p^k(x,x)=P(X_k=x|X_0=x)>0
    \end{equation}
    pour tout \( k\geq N\).
\end{proposition}

La proposition suivante va nous permettre de parler de \defe{chaine apériodique}{chaine!de Markov!apériodique}
\index{apériodique!chaine de Markov}.
\begin{proposition}
    Si une chaine de Markov est irréductible, alors un état est apériodique si et seulement si tous les états sont apériodiques.
\end{proposition}

\begin{proof}
    Soit \( x\) un état apériodique de la chaine de Markov \( (X_n)_{n\in \eN}\). En vertu de la proposition~\ref{PropSaOysS} il existe \( N_x\) tel que \( p^k(x,x)\neq 0\) pour tout \( k\geq N_x\). Soit \( y\in E\). Étant donné que la chaine est irréductible, il existe \( r\) et \( s\) tels que\( p^r(x,y)>0\) et \( p^s(y,x)>0\). Nous avons
    \begin{equation}
        p^{k+r+s}(y,y)=P(X_{k+r+s}=y|X_0=y)\geq p^s(x,y)P(X_k=x|X_0=x)p^r(y,x).
    \end{equation}
    Si \( k\) est assez grand, cette quantité est strictement positive. Donc il suffit de prendre \( N_y=N_x+r+s\) pour savoir que \( y\) est également apériodique.
\end{proof}

\begin{example}
Quelle est la différence entre une chaine irréductible et une chaine apériodique ? Une chaine est irréductible lorsque aucune sous-chaine ne peut piéger le système. Pour toute paire d'états \( x,y\in E\), il existe un \( n\) tel qu'il soit possible d'aller de \( x\) à \( y\) en \( n\) pas. Une chaine est apériodique lorsqu'après un temps suffisamment long, \emph{tous} les états soient possibles en même temps.

Un exemple de chaine irréductible non apériodique :
\begin{equation}
\xymatrix{%
    A \ar@/^/[r]^-{1}    &   B\ar@/^/[l]^{1}\\
   }
\end{equation}
Cette chaine est irréductible parce que le graphe est connexe, par contre il n'est pas apériodique parce que si \( X_0=A\) il n'est pas possible d'être dans l'état \( A\) après un nombre impair de pas.

Plus formellement, \( p^n(A,A)=1\) dès que \( n\) est pair; le PGCD de la définition~\ref{DefCxvOaT} n'est donc certainement pas \( 1\).
\end{example}

Si \( E\) est fini et si la chaine de Markov est irréductible, alors en posant \( N=\max_{x\in E}N(x)\), la matrice \( P^k\) a des éléments non nuls sur toute la diagonale pour tout \( k>N\). Ces éléments diagonaux ne sont autre que les \( p^k(x,x)\).

\begin{theorem}[Convergence en loi des chaine de Markov]
    Si \( (X_n)\) est
    \begin{enumerate}
        \item
            irréductible,
        \item
            récurrente positive,
        \item
            apériodique,
    \end{enumerate}
    alors \( X_n\) converge en loi vers l'unique probabilité invariante \( \pi\) vérifiant
    \begin{equation}
        \pi(x)=\sum_{u\in E}p(y,x)\pi(y)=\frac{1}{ E\big( T(x)|X_0=x \big) }.
    \end{equation}

    Cette convergence est indépendante de la loi de \( X_0\) et on a
    \begin{equation}
        P(X_n=x|X_0=y)\to_{n\to \infty} \pi(x).
    \end{equation}
\end{theorem}
\index{chaine!de Markov!convergence}

%+++++++++++++++++++++++++++++++++++++++++++++++++++++++++++++++++++++++++++++++++++++++++++++++++++++++++++++++++++++++++++
\section{Processus de Galton-Watson}
%+++++++++++++++++++++++++++++++++++++++++++++++++++++++++++++++++++++++++++++++++++++++++++++++++++++++++++++++++++++++++++
\label{SecBPmrPdtGalton}
\index{processus!Galton-Watson}
\index{suite de variables aléatoires de Bernoulli}

Nous considérons une maladie et notons \( Z_n\) le nombre de malades à l'instant \( n\). Nous posons \( Z_0=1\) et
\begin{equation}        \label{EqBvILKj}
    Z_{n+1}=\begin{cases}
        0    &   \text{si } Z_n=0\\
        \sum_{i=1}^{Z_n}\xi_i^{(n)}    &    \text{sinon}
    \end{cases}
\end{equation}
où \( \xi_i^{(n)}\) est le nombre de personnes contaminées par le malade \(i\) à l'instant \( n\). Nous supposons que ces variables aléatoires sont indépendantes et identiquement distribuées et admettent un moment d'ordre \( 1\).

L'équation de propagation~\ref{EqBvILKj} signifie que nous supposons qu'une personne malade à l'instant \( n\) n'est plus malade à l'instant \( n+1\). Par ailleurs les hypothèses d'indépendance signifient qu'à chaque instant, le nombre de personnes contaminées par le malade \( i\) est indépendant du nombre de personnes contaminées par le malade \( j\). De plus la façon dont la contamination se passe à l'instant \( n\) est indépendant de la façon dont la contamination se passe à l'instant \( m\). Ces hypothèses sont raisonnables tant que le nombre de personnes non contaminées est grand. À partir du moment où presque tout le monde est malade, l'approximation de Galton-Watson ne fonctionne plus.

Nous notons \( \xi\) la loi parente des \( \xi_i^{(n)}\). Ensuite nous considérons
\begin{subequations}
    \begin{align}
        G(s)=E(s^{\xi})\\
        m=E(\xi)\\
        G_n(s)=E(s^{Z_n}).
    \end{align}
\end{subequations}

    Par le théorème de transfert (proposition~\ref{PropintdPintdPXeR}) avec \( f(t)=s^t\). Ce que nous avons est
    \begin{equation}        \label{EqNRtXdC}
        G_n(s)=E\big( f(Z_n) \big)=\int_{\eR}s^xdP_{Z_n}(x)=\sum_{k=0}^{\infty}s^kP(Z_n=k)
    \end{equation}
    où l'intégrale s'est transformée en somme parce que la loi de \( Z_n\) est discrète : \( dP_{Z_n}\) est une somme de masses de Dirac. En particulier nous avons
    \begin{subequations}
        \begin{align}
            G_n(s)&=\sum_{k=0}^{\infty}s^kP(Z_n=k)\\
            G(0)&=P(Z_n=0)
        \end{align}
    \end{subequations}
    et
    \begin{equation}
        \eta=\lim_{n\to \infty} P(Z_n=0)=\lim_{n\to \infty} G_n(0).
    \end{equation}
    D'où l'intérêt d'étudier \( G_n\).

\begin{lemma}       \label{LemezrOiI}
    Pour tout \( n\in \eN^*\) et pour tout \( s\in\mathopen[ 0 , 1 \mathclose]\), nous avons
    \begin{equation}
        G_n(s)=\underbrace{G\circ G\circ\ldots\circ G(s)}_{ n\text{ fois}}.
    \end{equation}
\end{lemma}

\begin{proof}
    Pour \( n=1\), nous avons \( Z_1=\xi^{(1)}_1\) et donc
    \begin{equation}
        G_1(s)=E(s^{\xi})=G(s),
    \end{equation}
    comme il se doit.

    Si \( n\neq 1\) nous écrivons
    \begin{subequations}    \label{subEqsxhILKg}
        \begin{align}
            G_n(s)&=E(s^{Z_n})\\
            &=E\left( s^{\sum_{i=1}^{Z_{n-1}}\xi_i^{(n-1)}} \right)\\
            &=E\left( \sum_{k=0}^{\infty}\mtu_{\{ Z_{n-1}=k \}}s^{\sum_{i=1}^k\xi_i^{(n-1)}} \right).
        \end{align}
    \end{subequations}
    À ce niveau, nous voulons permuter la somme et l'espérance. Étant donné que le lemme est facile à vérifier pour \( s=1\), nous supposons \( s<1\). Du coup
    \begin{equation}
        s^{\sum_{i=1}^k\xi_i^{(n-1)}}<1
    \end{equation}
    et ce qui se trouve dans l'espérance est majoré par
    \begin{equation}
        \sum_{k=0}^{\infty}\mtu_{Z_{n-1}=k}=1.
    \end{equation}
    La fonction constante \( 1\) est intégrable sur \( \Omega\) (ici nous utilisons à fond le fait que l'espace \( \Omega\) soit un espace de probabilité) et nous pouvons utiliser le théorème de convergence dominée de Lebesgue~\ref{ThoockMHn} pour permuter la somme et l'intégrale. Nous continuons donc le calcul \eqref{subEqsxhILKg}:
    \begin{equation}
        G_n(s)=\sum_{k=0}^{\infty}E\left(  \mtu_{\{ Z_{n-1}=k \}}s^{\sum_{i=1}^k\xi_i^{(n-1)}}  \right).
    \end{equation}
    La tribu engendrée par la variable aléatoire \( \mtu_{\{ Z_{n-1}=k \}}\) est une fonction des variables aléatoires \( \xi_i^{(m)}\) avec \( m\leq n-2\) tandis que la variable aléatoire \( s^{\sum_{i=1}^k\xi_i^{(n-1)}}\) est une fonction des variables aléatoires \( \xi_{i}^{(n-1)}\). Par conséquent le lemme de regroupement~\ref{LemHOjqqw} nous dit que ces variables aléatoires sont indépendantes, donc
    \begin{equation}
        G_n(s)=\sum_{k=0}^{\infty}\underbrace{E\big( \mtu_{\{ Z_{n-1}=k \}} \big)}_{=P(Z_{n-1}=k)}E\big( s^{\sum_{i=1}^{k}\xi_i^{(n-1)}} \big).
    \end{equation}
    Nous avons utilisé le fait que l'espérance d'une fonction indicatrice est la probabilité de l'événement.

    En ce qui concerne la puissance de \( s\), les événements \( \xi_i^{n-1}\) sont indépendants et suivent tous la même loi \( \xi\), donc
    \begin{equation}
        s^{\sum_{i=1}^{k}\xi_i^{(n-1)}}=\prod_{i=1}^ks^{\xi_i^{(n-1)}}
    \end{equation}
    et
    \begin{equation}
        E\big( \prod_{i=1}^ks^{\xi} \big)=E(s^{\xi})^k=G(s)^k.
    \end{equation}
    En mettant tout bout à bout,
    \begin{equation}
        G_n(s)=\sum_{k=1}^{\infty}P(Z_{n-1}=k)G(s)^j=G_{n-1}\big( G(s) \big).
    \end{equation}
\end{proof}

\begin{theorem}  \label{ThoJZnAOA}
    La probabilité d'extinction \( \eta\) est donnée par
    \begin{equation}
        \eta=P\left(\bigcup_{n\geq 1}(Z_n=0)\right)=\lim_{n\to \infty} P(Z_n=0).
    \end{equation}
    Ce nombre est la plus petite solution positive de l'équation \( G(s)=s\).

    De plus la classification des cas est comme suit.
    \begin{enumerate}
        \item
            Si \( P(\xi=0)=0\) alors \( \eta=0\).
        \item
            Si \( P(\xi=0)\neq 0\) alors
            \begin{enumerate}
                \item
                    si \( m\leq 1\) alors \( \eta=1\),
                \item
                    si \( m>1\) alors \( \eta\in\mathopen] 0 , 1 \mathclose[\).
            \end{enumerate}
    \end{enumerate}
\end{theorem}
\index{point fixe}
\index{convolution}
\index{série!entière!processus de Markov}
Le cas \( m<1\) est dit \defe{sous-critique}{Galton-Watson!sous-critique}, le cas \( m=1\) est dit \defe{critique}{critique!Galton-Watson}. Le cas \( m>1\) est dit \defe{sur-critique}{Galton-Watson!sur-critique}.

\begin{proof}
    Commençons par prouver que \( G\) est une fonction continue. En utilisant la théorème de transfert comme pour l'équation \eqref{EqNRtXdC} nous trouvons que
    \begin{equation}    \label{EqQWTBfn}
        G(s)=E(s^{\xi})=\sum_{k=0}^{\infty}p_ks^k
    \end{equation}
    où nous avons noté \( p_k=P(\xi=k)\). Si \( r<1\), alors la suite \( p_kr^k\) est bornée, donc le critère d'Abel (\ref{LemmbWnFI}) nous indique que la série \eqref{EqQWTBfn} converge absolument et la théorie générale des séries entières conclut que la fonction \( G\) est en particulier dérivable terme à terme pour tout \( s\in\mathopen] -1 , 1 \mathclose[\).

    \begin{subproof}

        \item[Le probabilité d'extinction est un point fixe de \( G\)]

    En utilisant la continuité de \( G\) en \( 0\) nous passons à la limite dans \( G_{n+1}(0)=G\big( G_n(0) \big)\) et nous obtenons
    \begin{equation}
        \eta=G(\eta),
    \end{equation}
    ce qui signifie que la probabilité d'extinction est un point fixe de \( G\).

        \item[\( \eta\) est le plus petit point fixe de \( G\)]

    Nous démontrons maintenant que \( \eta\) est plus précisément le plus petit point fixe de \( G\) sur \( \mathopen[ 0 , 1 \mathclose]\). Nous allons effectuer cette partie en décomposant selon les valeurs de \( p_0\) et de \( p_1\).

    Au vu de l'écriture \eqref{EqQWTBfn}, si \( p_1=1\) alors \( G(s)=s\) pour tout \( s\in\mathopen[ 0 , 1 \mathclose]\). Mais dans ce cas nous savons par ailleurs que l'extinction est impossible.  Zéro est bien la plus petite solution de \( G(s)=s\).

    Supposons maintenant que \( p_1<1\) et \( p_0+p_1=1\). Alors \( G(s)=p_0+p_1s\) et \( s=1\) est l'unique solution. Mais vu que nous savons que \( \eta\) est solution, c'est que \( \eta=1\) et l'extinction est certaine.

    Nous passons au cas général : \( p_0+p_1<1\). D'abord nous remarquons que \( s=1\) est solution parce que
    \begin{equation}
        G(1)=p_0+p_1+\cdots=1.
    \end{equation}
    Remarquons aussi que dans ce cas \( s=0\) n'est plus solution.

    La fonction \( G\) est strictement convexe sur \( \mathopen[ 0 , 1 \mathclose]\) (parce que \( G''>0\)). Cela se voir en effectuant deux dérivations termes à termes (le rayon de convergence de la dérivée est le même que celui de la fonction). Cette stricte convexité entraine que l'équation \( G(s)=s\) a au maximum une autre solution que \( s=1\). Nous nommons \( s_0\) la plus petite solution dans \( \mathopen[ 0 , 1 \mathclose]\). Étant donné que \( G\) est croissante on a
    \begin{equation}
        G(0)\leq G(s_0)=s_0.
    \end{equation}
    En appliquant \( G\) à cette équation nous obtenons \( G\big( G(s_0) \big)\leq G(s_0)=s_0\) et en appliquant \( n\) fois,
    \begin{equation}
        G_n(0)\leq s_0.
    \end{equation}
    En passant à la limite, \( \eta\leq s_0\) mais \( \eta\) étant solution, nous avons \( \eta=s_0\). Nous avons donc prouvé que la probabilité d'extinction \( \eta\) est la plus petite solution de \( G(s)=s\).

\item[Classification des cas]

    Nous devons encore discuter les cas. Si \( P(\xi=0)=0\), alors \( p_0=0\) et \( G(0)=0\), ce qui signifie que \( s_0=\eta=0\) et l'extinction est impossible.

    Nous passons au cas \( p_0\neq 0\). Si \( p_0+p_1=1\), alors \( m=p_1<1\) et nous avions déjà vu que dans le cas \( p_0+p_1=1\), la probabilité d'extinction est \( \eta=1\).

    Il nous reste à traiter le cas \( p_0+p_1<1\). Encore une fois, la courbe \( G\) est strictement convexe sur \( \mathopen[ 0 , 1 \mathclose]\) et elle est en particulier plus grande que sa tangente en \( s=1\), c'est-à-dire
    \begin{equation}
        G(s)>G'(1)(s-1)+G(1).
    \end{equation}
    Nous savons que \( G(1)=1\). En ce qui concerne \( G'(1)\), nous dérivons encore terme à termes :
    \begin{equation}
        G'(s)=\sum_{k=1}^{\infty}kp_ks^{k-1},
    \end{equation}
    donc
    \begin{equation}
        G'(1)=\sum_{k=1}^{\infty}kp_k=E(\xi)=m.
    \end{equation}
    Ce que nous avons donc est
    \begin{equation}
        G(s)>1+m(s-1).
    \end{equation}
    Nous nous particularisons au cas sous-critique (\( m\leq 1\)). En nous rappelant que \( s-1<0\),
    \begin{equation}
        G(s)>1+(s-1)=s,
    \end{equation}
    donc \( s=1\) est la plus petite solution et effectivement nous avons déjà vu que \( \eta=1\) dans ce cas.

    Si \( m>1\), alors on a
    \begin{equation}
        G(s)>1+m(s-1).
    \end{equation}
    Mais dire \( m>1\) revient à dire \( G'(1)>1\) et donc dans un voisinage de \( s=1\) on a
    \begin{equation}
        \frac{ G(s)-G(1) }{ s-1 }>1,
    \end{equation}
    ce qui implique que
    \begin{equation}
        G(s)<s-1+G(1)=s.
    \end{equation}
    Nous avons donc \( G(s)<s\) dans un voisinage de \( 1\). Mais \( G(0)-0=p_0>0\), donc la fonction \( f(s)=G(s)-s\) est positive en \( 0\) et négative proche de \( s=1\). Le théorème de la valeur intermédiaire nous indique alors qu'il existe un \( s\in \mathopen] 0 , 1 \mathclose[\) tel que \( f(s)=0\), c'est-à-dire tel que \( G(s)=s\).
    \end{subproof}
\end{proof}


\chapter{Martingales}
\input{96_Martingales}

\chapter{Processus de Poisson}
\input{97_ProcessusPoisson}

\chapter{Langages}
\input{202_langages}

\chapter{Utilisation dans les autres sciences}
\input{Science}
\input{146_relativite}

\chapter{Exemples avec Sage}
% This is part of (almost) Everything I know in mathematics
% Copyright (C) 2009-2010,2016-2020
%   Laurent Claessens
% See the file fdl-1.3.txt for copying conditions.

Ce chapitre est un foure-tout de choses que l'on peut faire avec Sage.

%+++++++++++++++++++++++++++++++++++++++++++++++++++++++++++++++++++++++++++++++++++++++++++++++++++++++++++++++++++++++++++ 
\section{Graphiques}
%+++++++++++++++++++++++++++++++++++++++++++++++++++++++++++++++++++++++++++++++++++++++++++++++++++++++++++++++++++++++++++

Pour afficher le graphe d'une fonction, vous pouvez faire
\begin{verbatim}
+--------------------------------------------------------------------+
| SageMath version 8.1, Release Date: 2017-12-07                     | 
| Type "notebook()" for the browser-based notebook interface.        | 
| Type "help()" for help.                                            |
+--------------------------------------------------------------------+
sage: plot(cos(x),0,5)
Launched png viewer for Graphics object consisting of 1 graphics primitive
sage: f(x)=sin(x)
sage: f.plot(-pi,pi)
Launched png viewer for Graphics object consisting of 1 graphics primitive
\end{verbatim}
Un programme externe se lance automatiquement pour afficher le graphique que vous avez demandé.

Il se peut qu'aucun programme ne se lance et vous ayez, au lieu de \info{Launched png viewer for Graphics object \ldots} uniquement \info{Created graphics object \ldots}. Disons pour faire court que Sage a produit un \info{png} et qu'il ne sait pas quel programme externe utiliser pour l'afficher.

La solution est à l'adresse \url{http://doc.sagemath.org/html/en/reference/misc/sage/misc/viewer.html}
%--------------------------------------------------------------------------------------------------------------------------- 
\subsection{Autres}
%---------------------------------------------------------------------------------------------------------------------------

Dans le but d'automatiser certaines tâches, j'ai écrit ce module, nomé \info{outilsINGE.sage}, dans le cadre d'un cours de première année donné à des ingénieurs. Certaines des fonctions définies ici sont utilisée dans les exemples qui suivent.

\lstinputlisting{tex/sage/outilsINGE.sage}

\begin{example}     \label{ExBCRXooDVUdcf}
	Calculer la limite
			\begin{equation}
				\lim_{x\to\infty}\frac{ \sin(x)\cos(x) }{ x }
			\end{equation}


            \begin{verbatim}
			var('x')
			f(x)=sin(x)*cos(x)/x
			limit(f(x),x=oo)
            \end{verbatim}
	La première ligne déclare que la lettre \texttt{x} désignera une variable. Pour la troisième ligne, notez que l'infini est écrit par deux petits « o ».
\end{example}

\begin{example}     \label{ExCWDRooKxnjGL}
    Quelques limites et graphes avec Sage.

    \begin{enumerate}

		\item
			$\lim_{x\to 0} \frac{ \sin(\alpha x) }{ \sin(\beta x) }$.

			Pour effectuer cet exercice avec Sage, il faut taper les lignes suivantes~:


\begin{verbatim}
sage: var('x,a,b')
(x, a, b)
sage: f(x)=sin(a*x)/sin(b*x)
sage: limit( f(x),x=0  )
a/b
\end{verbatim}

			Noter qu'il faut déclarer les variables \texttt{x}, \texttt{a} et \texttt{b}.

		\item
			$\lim_{x\to \pm\infty} \frac{ \sqrt{x^2+1}-x }{ x-2 }$

            \begin{verbatim}
sage: f(x)=(sqrt(x**2+1))/(x-2)
sage: limit(f(x),x=oo)
1
sage: limit(f(x),x=-oo)
-1
            \end{verbatim}

			Noter la commande pour la racine carré~: \texttt{sqrt}. Étant donné que cette fonction diverge en $x=2$, si nous voulons la tracer, il faut procéder en deux fois :

            \begin{verbatim}
sage: plot(f,(-100,1.9))
Launched png viewer for Graphics object consisting of 1 graphics primitive
sage: plot(f,(2.1,100))
Launched png viewer for Graphics object consisting of 1 graphics primitive
            \end{verbatim}
			La première ligne trace de $-100$ à $1.9$ et la seconde de $2.1$ à $100$. Ces graphiques vous permettent déjà de voir les limites. Attention : ils ne sont pas des \emph{preuves} ! Mais ils sont de sérieux indices qui peuvent vous inspirer dans vos calculs.

	\end{enumerate}
\end{example}

\begin{example} \label{exJMGTooZcZYNy}


Calculer les dérivées partielles $\partial_xf$, $\partial_yf$, $\partial^2_xf$, $\partial^2_{xy}f$, $\partial^2_{yx}f$ et $\partial^2_yf$ des fonctions suivantes.
\begin{multicols}{2}
\begin{enumerate}
\item
$2x^3+3x^2y-2y^2$
\item
$\ln(xy^2)$
\item
$\tan(x/y)$
\item
$\frac{ xy^2 }{ x+y }$

\end{enumerate}
\end{multicols}


Le script Sage suivant (\verb+exoDV002.sage+) résout l'exercice :


\lstinputlisting{tex/sage/exoDV002.sage}

La sortie est :

\VerbatimInput[tabsize=3]{tex/sage/exoDV002.txt}

\end{example}

\begin{example}\label{exKGDIooVefujD}



Résoudre les systèmes suivants.
\begin{multicols}{2}
	\begin{enumerate}

			\item
                                        $
                                        \left\{
                                        \begin{array}{ll}
                                                        x_1 - 2x_2 + 3x_3 - 2x_4 = 0\\
                                        3x_1 - 7x_2 - 2x_3 + 4x_4 = 0\\
                                        4x_1 + 3x_2 + 5x_3 + 2x_4 = 0\\

                                        \end{array}
                                        \right.
                                        $



                                \item
                                        $
                                        \left\{
                                        \begin{array}{ll}
                                                        2x_1 + x_2 - 2x_3 + 3x_4 = 0\\
                                        3x_1 + 2x_2 - x_3 + 3x_4 = 4\\
                                        3x_1 + 3x_2 + 3x_3 - 3x_4 = 9\\

                                        \end{array}
                                        \right.
                                        $



                                \item
                                        $
                                        \left\{
                                        \begin{array}{ll}
                                                        x_1 + 2x_2 - 3x_3 = 0\\
                                        2x_1 + 5x_2 + 2x_3 = 0\\
                                        3x_1 - x_2 - 4x_3 = 0\\

                                        \end{array}
                                        \right.
                                        $



                                \item
                                        $
                                        \left\{
                                        \begin{array}{ll}
                                                        x_1 + 2x_2 - x_3 = 0\\
                                        2x_1 + 5x_2 + 2x_3 = 0\\
                                        x_1 + 4x_2 + 7x_3 = 0\\
                                        x_1 + 3x_2 + 3x_3 = 0\\

                                        \end{array}
                                        \right.
                                        $



                                \item
                                        $
                                        \left\{
                                        \begin{array}{ll}
                                                        x_1 + x_2 + x_3 + x_4 = 0\\
                                        x_1 + x_2 + x_3 - x_4 = 4\\
                                        x_1 + x_2 - x_3 + x_4 = -4\\
                                        x_1 - x_2 + x_3 + x_4 = 2\\

                                        \end{array}
                                        \right.
                                        $



                                \item
                                        $
                                        \left\{
                                        \begin{array}{ll}
                                                        x_1 + 3x_2 + 3x_3 = 1\\
                                        x_1 + 3x_2 + 4x_3 = 0\\
                                        x_1 + 4x_2 + 3x_3 = 3\\

                                        \end{array}
                                        \right.
                                        $



                                \item
                                        $
                                        \left\{
                                        \begin{array}{ll}
                                                        x_1 - 3x_2 + 2x_3 = -6\\
                                        -3x_1 + 3x_2 - x_3 = 17\\
                                        2x_1 - x_2 = 3\\

                                        \end{array}
                                        \right.
                                        $



                                \item
                                        $
                                        \left\{
                                        \begin{array}{ll}
                                                        x_1 - 2x_2 + 3x_3 - 2x_4 = 0\\
                                        3x_1 - 7x_2 - 2x_3 + 4x_4 = 0\\
                                        4x_1 + 3x_2 + 5x_3 + 2x_4 = 0\\

                                        \end{array}
                                        \right.
                                        $



                                \item
                                        $
                                        \left\{
                                        \begin{array}{ll}
                                                        2x_1 + x_2 - 2x_3 + 3x_4 = 0\\
                                        3x_1 + 2x_2 - x_3 + 3x_4 = 4\\
                                        3x_1 + 3x_2 + 3x_3 - 3x_4 = 9\\

                                        \end{array}
                                        \right.
                                        $



                                \item
                                        $
                                        \left\{
                                        \begin{array}{ll}
                                                        x_1 + 2x_2 - 3x_3 = 0\\
                                        2x_1 + 5x_2 + 2x_3 = 0\\
                                        3x_1 - x_2 - 4x_3 = 0\\

                                        \end{array}
                                        \right.
                                        $



                                \item
                                        $
                                        \left\{
                                        \begin{array}{ll}
                                                        x_1 + 2x_2 - x_3 = 0\\
                                        2x_1 + 5x_2 + 2x_3 = 0\\
                                        x_1 + 4x_2 + 7x_3 = 0\\
                                        x_1 + 3x_2 + 3x_3 = 0\\

                                        \end{array}
                                        \right.
                                        $



                                \item
                                        $
                                        \left\{
                                        \begin{array}{ll}
                                                        x_1 + x_2 + x_3 + x_4 = 0\\
                                        x_1 + x_2 + x_3 - x_4 = 4\\
                                        x_1 + x_2 - x_3 + x_4 = -4\\
                                        x_1 - x_2 + x_3 + x_4 = 2\\

                                        \end{array}
                                        \right.
                                        $



                                \item
                                        $
                                        \left\{
                                        \begin{array}{ll}
                                                        x_1 + 3x_2 + 3x_3 = 1\\
                                        x_1 + 3x_2 + 4x_3 = 0\\
                                        x_1 + 4x_2 + 3x_3 = 3\\

                                        \end{array}
                                        \right.
                                        $



                                \item
                                        $
                                        \left\{
                                        \begin{array}{ll}
                                                        x_1 - 3x_2 + 2x_3 = -6\\
                                        -3x_1 + 3x_2 - x_3 = 17\\
                                        2x_1 - x_2 = 3\\

                                        \end{array}
                                        \right.
                                        $

	\end{enumerate}


\end{multicols}


	Nous résolvons les systèmes en utilisant Sage avec le script suivant.

\lstinputlisting{tex/sage/exo11.sage}

Le résultat est le suivant :

\VerbatimInput[tabsize=3]{tex/sage/exo11.txt}

\end{example}

\begin{example}     \label{ExBGCEooPIQgGW}


	Pour chacun des systèmes suivants $A\cdot X=B$,
	\begin{enumerate}

		\item
			Résoudre le système par échelonnement,
		\item
			Calculer $A^{-1}$,
		\item
			Vérifier votre réponse en calculant $A^{-1}B$. Qu'êtes-vous censé obtenir ?

	\end{enumerate}

	Les énoncés sont
	\begin{enumerate}

		\item
			\begin{equation}
				\begin{aligned}[]
					A=\begin{pmatrix}
						2	&	1	&	-2	\\
						3	&	2	&	2	\\
						5	&	4	&	3
					\end{pmatrix},
					&&B=\begin{pmatrix}
						10	\\
						1	\\
						4
					\end{pmatrix}
				\end{aligned}
			\end{equation}
	\end{enumerate}


	Nous utilisons Sage pour fournir la réponse. Le code suivant résout le système et donne l'inverse de la matrice :

\lstinputlisting{tex/sage/exo13.sage}

La sortie est ici :

\VerbatimInput[tabsize=3]{tex/sage/exo13.txt}



\end{example}

\begin{example}     \label{exBNGVooIvKfTT}


	 Sachant que $(-1,0,1,0)$ est un vecteur propre de la matrice
	\begin{equation}
		A=\begin{pmatrix}
			 2	&	1	&	-1	&	1	\\
			 1	&	0	&	1	&	1	\\
			 -1	&	1	&	2	&	1	\\
			 1	&	1	&	1	&	0
		 \end{pmatrix}
	\end{equation}
	\begin{enumerate}

		\item
			Diagonaliser $A$ au moyen d'une matrice orthogonale
		\item
			Écrire la forme quadratique $X^tAX$ sous forme d'une somme pondérée de carrés.
	\end{enumerate}



	Calculons $Av$ afin de savoir la valeur propre associée au vecteur donné :
	\begin{equation}
		\begin{pmatrix}
			 2	&	1	&	-1	&	1	\\
			 1	&	0	&	1	&	1	\\
			 -1	&	1	&	2	&	1	\\
			 1	&	1	&	1	&	0
		 \end{pmatrix}
		 \begin{pmatrix}
			 -1	\\
			 0	\\
			 1	\\
			 0
		 \end{pmatrix}
		 =
		 \begin{pmatrix}
			 -3	\\
			 0	\\
			 3	\\
			 0
		 \end{pmatrix}.
	\end{equation}
	La valeur propre est donc $3$. Nous savons donc que $(\lambda-3)$ pourra être factorisé dans le polynôme caractéristique.

	Pour le reste de l'exercice c'est standard et c'est résolu de la façon suivante :

	\lstinputlisting{tex/sage/exo65.sage}

	qui retourne

	\VerbatimInput[tabsize=3]{tex/sage/exo65.txt}

\end{example}

\begin{example}     \label{exZHGRooTQpVpq}


	 Rechercher les extrémums des fonctions suivantes
	\begin{enumerate}

		\item
			$f(x,y)=2-\sqrt{x^2+y^2}$
		\item
			$f(x,y)=x^3+3xy^2-15x-12y$
		\item
			$f(x,y)=\frac{ x^3 }{ 3 }+\frac{ 4y^3 }{ 3 }-x^2-3x-4y-3$

	\end{enumerate}




	Les corrigés sont créés par le script Sage \verb+exo101.sage+

	\VerbatimInput[tabsize=3]{tex/sage/exo101.sage}

	Des réponses :

	\begin{enumerate}

		\item
			\VerbatimInput[tabsize=3]{tex/sage/exo101A.txt}

			Ici nous voyons que Sage a du mal à calculer la matrice hessienne en $(0,0)$. En effet, nous tombons sur une division par zéro. Pour résoudre l'exercice, il faut se rendre compte que la fonction $(x,y)\mapsto\sqrt{x^2+y^2}$ est toujours positive et est nulle seulement au point $(0,0)$. Donc $f$ est toujours plus petite ou égale à deux tandis que $f(0,0)=2$. Le point est donc un maximum global.
		\item
			\VerbatimInput[tabsize=3]{tex/sage/exo101B.txt}

		\item
			\VerbatimInput[tabsize=3]{tex/sage/exo101C.txt}

	\end{enumerate}

\end{example}

\begin{example}     \label{exHWIHooOAvaDQ}



	Déterminer les valeurs extrêmes et les points de selle des fonctions suivantes.
	\begin{multicols}{2}
		\begin{enumerate}
			\item	%1
				$f(x,y)=x^2+4x+y^2-2y$.
			\item	%8
				$f(x,y)= e^{x^2+xy}$.
			\item	%17
				$f(x,y)=e^x\sin(y)$.
		\end{enumerate}
	\end{multicols}


	Certains corrigés de cet exercice ont étés réalisés par Sage. Le script utilisé est \verb+exo103.sage+

	\lstinputlisting{tex/sage/exo103.sage}

	Des réponses :

	\begin{enumerate}
		\item	%1
			\VerbatimInput[tabsize=3]{tex/sage/exo103A.txt}
		\item	%8
			\VerbatimInput[tabsize=3]{tex/sage/exo103H.txt}
		\item	%17
			\VerbatimInput[tabsize=3]{tex/sage/exo103Q.txt}

			Ici, Sage n'est pas capable de résoudre les équations qui annulent le jacobien. Les équations à résoudre sont pourtant faciles :
			\begin{subequations}
				\begin{numcases}{}
					e^{x}\cos(y)=0\\
					e^{x}\sin(y)=0
				\end{numcases}
			\end{subequations}
			Étant donné que l'exponentielle ne s'annule jamais, il faudrait avoir en même temps $\cos(y)=0$ et $\sin(y)=0$, ce qui est impossible. La fonction n'a donc aucun extrémums local.

	\end{enumerate}

\end{example}


\begin{example}     \label{exEEHPooKDxLTJ}


	Considérons la fonction
	\begin{equation}
		f(x,y)=xy^2 e^{-(x^2+y^2)/4}.
	\end{equation}
	\begin{enumerate}

		\item
			Montrer qu'il y a une infinité de points critiques.
		\item
			Déterminer leur nature.

	\end{enumerate}

	Voici la fonction Sage qui fournit les informations :

	\lstinputlisting{tex/sage/exo104.sage}

	La sortie est

	\VerbatimInput[tabsize=3]{tex/sage/exo104.txt}

	Notez la présence de \verb+r1+ comme paramètres dans les solutions. Tous les points avec $y=0$ sont des points critiques. Cependant, Sage\footnote{ou, plus précisément, le programme que j'ai écrit avec Sage.} ne parvient pas à conclure la nature de ces points $(x,0)$.

	Notons que le nombre $f(x,y)$ a toujours le signe de $x$ parce que $y^2$ et l'exponentielle sont positives. Toujours ? En tout cas lorsque $x\neq 0$. Prenons un point $(a,0)$ avec $a>0$. Dans un voisinage de ce point, nous avons $f(x,y)>0$ parce que si $a>0$, alors $x>0$ dans un voisinage de $a$. Le point $(a,0)$ est un minimum local parce que $0=f(a,0)\leq f(x,y)$ pour tout $(x,y)$ dans un voisinage de $(a,0)$.

	De la même façon, les points $(a,0)$ avec $a<0$ sont des maximums locaux parce que dans un voisinage, la fonction est négative.

	Le point $(0,0)$ n'est ni maximum ni minimum local. C'est un point de selle.

\end{example}

\begin{example}     \label{exRNZKooUIOfPU}

    Dériver les fonctions suivantes.
	\begin{enumerate}
		\item
			$\sin\big( \ln(x) \big)$
		\item
			$\displaystyle \frac{\sin x}{x}$ ;
		\item
			$ e^{x^2}$
		\item
			$\cos(x)^{\sin(x)}$
	\end{enumerate}

Le programme suivant par Sage résout l'exercice:
\lstinputlisting{tex/sage/corrDerive_0002.sage}

Le résultat est :
\VerbatimInput[tabsize=3]{tex/sage/corrDerive_0002.txt}

\end{example}

\begin{example}     \label{exLFYFooNCXCJz}

	Donner une approximation de $\ln(1.0001)$.

	\begin{verbatim}
		----------------------------------------------------------------------
		| Sage Version 4.5.3, Release Date: 2010-09-04                       |
		| Type notebook() for the GUI, and license() for information.        |
		----------------------------------------------------------------------
		sage: numerical_approx(ln(1.0001))
		0.0000999950003332973
	\end{verbatim}

\end{example}


\chapter{Épilogue : la constante de Weiner}
\input{191_weiner}

% SCRIPT MARK -- DÉVELOPPEMENTS POSSIBLES
% -------------------------------------------------------------------------

\chapter{Développements possibles}
% This is part of Mes notes de mathématique
% Copyright (c) 2012-2017, 2019-2020
%   Laurent Claessens
% See the file fdl-1.3.txt for copying conditions.

{\bf Attention } : cette liste de développements est un document de travail pour ceux qui écrivent dans la Frido. Elle sera supprimée des versions commerciales et donc des versions disponibles pour l'agrégation.

\vspace{2cm}


Nous donnons ici quelques idées de développements associés aux leçons données dans le rapport du jury 2016\cite{ooJECQooJvIKEJ}. Parfois, il est bon d'ajouter quelques lemmes au développement proposé, s'il est trop court. Si l'un ou l'autre ne vous semble pas adapté à l'énoncé de la leçon, faites le moi savoir.

%+++++++++++++++++++++++++++++++++++++++++++++++++++++++++++++++++++++++++++++++++++++++++++++++++++++++++++++++++++++++++++
\section{Algèbre et géométrie}
%+++++++++++++++++++++++++++++++++++++++++++++++++++++++++++++++++++++++++++++++++++++++++++++++++++++++++++++++++++++++++++

\paragraph{Applications des nombres complexes à la géométrie.}
\begin{itemize}
    \item Action du groupe modulaire sur le demi-plan de Poincaré, théorème~\ref{ThoItqXCm}.
    \item Générateurs du groupe diédral, proposition~\ref{PropLDIPoZ}
    \item Le groupe circulaire, proposition~\ref{THOooKMKWooZPIDaK}.
\end{itemize}
%---------------------------------------------------------------------------------------------------------------------------------------------
\paragraph{Formes linéaires et dualité en dimension finie. Exemples et applications.}
%---------------------------------------------------------------------------------------------------------------------------------------------
\paragraph{Polynômes d'endomorphismes en dimension finie. Réduction d'un endomorphisme en dimension finie. Applications.}
%---------------------------------------------------------------------------------------------------------------------------------------------
\paragraph{Anneaux principaux. Applications.}
%---------------------------------------------------------------------------------------------------------------------------------------------
\paragraph{Caractères d'un groupe abélien fini et transformée de Fourier discrète. Applications}
%---------------------------------------------------------------------------------------------------------------------------------------------
\paragraph{Algèbre des polynômes à plusieurs indéterminées. Applications.}
%---------------------------------------------------------------------------------------------------------------------------------------------
\paragraph{Racines d'un polynôme. Fonctions symétriques élémentaires. Exemples et applications.}

Le rapport du jury mentionne les théorèmes de Gershgorin (\ref{THOooUJNFooHpvCCF} et~\ref{THOooTXAPooQqsBCj}) pour cette leçon, mais je ne sais pas si ils valent un développement.
% En tout cas, cette note est utile pour qu'il y ait quelque part un lien vers ces théorèmes.

%---------------------------------------------------------------------------------------------------------------------------------------------
\paragraph{Sous-groupes discrets de \( \eR^2\). Réseaux. Exemples}
\begin{itemize}
    \item Les groupes de pavage de \( \eR^2\), théorème \ref{THOooUPHQooYfeHAy}.
\end{itemize}
%---------------------------------------------------------------------------------------------------------------------------------------------
\paragraph{Isométries d'un espace affine euclidien de dimension finie. Applications en dimensions $2$ et $3$.}
\begin{itemize}
    \item Isométries du cube, section~\ref{SecPVCmkxM}.
\end{itemize}
%---------------------------------------------------------------------------------------------------------------------------------------------
\paragraph{Systèmes d'équations linéaires ; opérations élémentaires, aspects algorithmiques et conséquences théoriques.}
\begin{itemize}
    \item Algorithme des facteurs invariants~\ref{PropPDfCqee}.
    \item Méthode du gradient à pas optimal~\ref{PropSOOooGoMOxG}.
\end{itemize}
%---------------------------------------------------------------------------------------------------------------------------------------------
\paragraph{Exemples d’équations diophantiennes.}
\begin{itemize}
    \item Dans~\ref{subsecZVKNooXNjPSf}, nous résolvons \( ax+by=c\) en utilisant Bézout (théorème~\ref{ThoBuNjam}).
    \item L'exemple~\ref{ExmuQisZU} résout l'équation \( x^2+2=y^3\) en parlant de l'extension \( \eZ[i\sqrt{2}]\) et de stathme.
    \item Les propositions~\ref{PropXHMLooRnJKRi} et~\ref{propFKKKooFYQcxE} parlent de triplets pythagoriciens.
    \item Le dénombrement des solutions de l'équation \( \alpha_1 n_1+\ldots \alpha_pn_p=n\) utilise des séries entières et des décompositions de fractions en éléments simples, théorème~\ref{ThoDIDNooUrFFei}.
\end{itemize}
%---------------------------------------------------------------------------------------------------------------------------------------------
\paragraph{Groupe opérant sur un ensemble. Exemples et applications.}
\begin{itemize}
    \item Les groupes de pavage de \( \eR^2\), théorème \ref{THOooUPHQooYfeHAy}.
    \item Action du groupe modulaire sur le demi-plan de Poincaré, théorème~\ref{ThoItqXCm}.
    \item Polynômes semi-symétriques, proposition~\ref{PropUDqXax}.
    \item Lemme de Morse, lemme~\ref{LemNQAmCLo}.
    \item Générateurs du groupe diédral, proposition~\ref{PropLDIPoZ}.
    \item Sous-groupes compacts de \( \GL(n,\eR)\), lemme~\ref{LemOCtdiaE} ou proposition~\ref{PropQZkeHeG}.
    \item Théorème de Wedderburn~\ref{ThoMncIWA}.
    \item Isométries du cube, section~\ref{SecPVCmkxM}.
    \item Algorithme des facteurs invariants~\ref{PropPDfCqee}.
\end{itemize}
%---------------------------------------------------------------------------------------------------------------------------------------------
\paragraph{Exemples de sous-groupes distingués et de groupes quotients. Applications.}
\begin{itemize}
    \item Suites de décomposition et théorème de Jordan-Hölder~\ref{ThoLgxWIC}.
    \item Groupes d'ordre \( pq\), théorème~\ref{ThoLnTMBy}.
    \item Le groupe alterné est simple, théorème~\ref{ThoURfSUXP}.
    \item Théorème de Lie-Kolchin~\ref{ThoUWQBooCvutTO}.
\end{itemize}
%---------------------------------------------------------------------------------------------------------------------------------------------
\paragraph{Sous-groupes finis de \( \gO(2,\eR)\) et \( \gO(3,\eR)\). Applications}
\begin{itemize}
    \item Les groupes de pavage de \( \eR^2\), théorème \ref{THOooUPHQooYfeHAy}.
\end{itemize}
%---------------------------------------------------------------------------------------------------------------------------------------------
\paragraph{Groupes finis. Exemples et applications.}
\begin{itemize}
    \item RSA, section~\ref{SecEVaFYi}, plus l'exponentielle rapide, plus la recherche de couples de Bézout.
    \item Théorème de Wedderburn~\ref{ThoMncIWA}.
    \item Théorème de Sylow~\ref{ThoUkPDXf}. Tout le théorème, c'est un peu long. On peut se contenter de la partie qui dit que \( G\) contient un \( p\)-Sylow.
    \item Coloriage de roulette (\ref{pTqJLY}) et composition de colliers (\ref{siOQlG}).
    \item Suites de décomposition et théorème de Jordan-Hölder~\ref{ThoLgxWIC}.
    \item Théorème de Burnside sur les sous-groupes d'exposant fini de \( \GL(n,\eC)\), théorème~\ref{ThooJLTit}.
    \item \( (\eZ/p\eZ)^*\simeq \eZ/(p-1)\eZ\), corolaire~\ref{CorpRUndR}.
    \item Groupes d'ordre \( pq\), théorème~\ref{ThoLnTMBy}.
    \item Générateurs du groupe diédral, proposition~\ref{PropLDIPoZ}.
    \item Le groupe alterné est simple, théorème~\ref{ThoURfSUXP}.
    \item Isométries du cube, section~\ref{SecPVCmkxM}.
\end{itemize}
%---------------------------------------------------------------------------------------------------------------------------------------------
\paragraph{Groupe des permutations d'un ensemble fini. Applications.}
\begin{itemize}
    \item RSA, section~\ref{SecEVaFYi}, plus l'exponentielle rapide, plus la recherche de couples de Bézout.
    \item Coloriage de roulette (\ref{pTqJLY}) et composition de colliers (\ref{siOQlG}).
    \item Forme alternées de degré maximum, proposition~\ref{ProprbjihK}.
    \item Décomposition de Bruhat, théorème~\ref{ThoizlYJO}.
    \item Polynômes semi-symétriques, proposition~\ref{PropUDqXax}.
    \item Table des caractères du groupe diédral, section~\ref{SecWMzheKf}.
    \item Table des caractères du groupe symétrique \( S_4\), section~\ref{SecUMIgTmO}.
    \item Isométries du cube, section~\ref{SecPVCmkxM}.
    \item Le groupe alterné est simple, théorème~\ref{ThoURfSUXP}.
\end{itemize}
%---------------------------------------------------------------------------------------------------------------------------------------------
\paragraph{Groupe linéaire d’un espace vectoriel de dimension finie $E$ , sous-groupes de $\GL(E)$. Applications.}
%\index{groupe!linéaire}
\begin{itemize}
    \item Théorème de Burnside sur les sous-groupes d'exposant fini de \( \GL(n,\eC)\), théorème~\ref{ThooJLTit}.
    \item Décomposition de Bruhat, théorème~\ref{ThoizlYJO}.
    \item Le lemme au lemme de Morse, lemme~\ref{LemWLCvLXe}.
    \item Décomposition polaire~\ref{ThoLHebUAU}.
    \item Enveloppe convexe du groupe orthogonal~\ref{ThoVBzqUpy}.
    \item Sous-groupes compacts de \( \GL(n,\eR)\), lemme~\ref{LemOCtdiaE} ou proposition~\ref{PropQZkeHeG}.
    \item Théorème de Von Neumann~\ref{ThoOBriEoe}.
\end{itemize}
%---------------------------------------------------------------------------------------------------------------------------------------------
\paragraph{Représentations et caractères d'un groupe fini sur un \( \eC\)-espace vectoriel. Exemples}
\begin{itemize}
    \item Table des caractères du groupe diédral, section~\ref{SecWMzheKf}.
    \item Table des caractères du groupe symétrique \( S_4\), section~\ref{SecUMIgTmO}.
\end{itemize}
%---------------------------------------------------------------------------------------------------------------------------------------------
\paragraph{Exemples de parties génératrices d’un groupe. Applications.}
\begin{itemize}
    \item RSA, section~\ref{SecEVaFYi}. Assez indirect : la système RSA se base sur la formule \( \varphi(pq)=(p-1)(q-1)\), laquelle se base sur l'isomorphisme \( \eZ/p\eZ\times \eZ/q\eZ\simeq \eZ/pq\eZ\) et leurs générateurs.
    \item Action du groupe modulaire sur le demi-plan de Poincaré, théorème~\ref{ThoItqXCm}, parce que c'est avec lui qu'on montre les générateurs du groupe modulaire dans le corolaire~\ref{CorJQwgNp}.
    \item Générateurs du groupe diédral, proposition~\ref{PropLDIPoZ}
    \item Table des caractères du groupe diédral, section~\ref{SecWMzheKf}.
    \item Le groupe alterné est simple, théorème~\ref{ThoURfSUXP}.
    \item Les groupes de pavage de \( \eR^2\), théorème \ref{THOooUPHQooYfeHAy}.
\end{itemize}

%---------------------------------------------------------------------------------------------------------------------------------------------
\paragraph{Anneaux $\eZ/n\eZ$. Applications.}
\begin{itemize}
    \item RSA, section~\ref{SecEVaFYi}, plus l'exponentielle rapide, plus la recherche de couples de Bézout.
    \item Forme faible du théorème de Dirichlet (avec ses deux lemmes)~\ref{ThoxwTjcl}.
    \item \( (\eZ/p\eZ)^*\simeq \eZ/(p-1)\eZ\), corolaire~\ref{CorpRUndR}.
    \item Groupes d'ordre \( pq\), théorème~\ref{ThoLnTMBy}.
    \item Irréductibilité des polynômes cyclotomiques, proposition~\ref{PropoIeOVh}.
\end{itemize}

%---------------------------------------------------------------------------------------------------------------------------------------------
\paragraph{Nombres premiers. Applications.}
\begin{itemize}
    \item Structure des groupes d'ordre \( pq\), théorème~\ref{ThoLnTMBy}.
    \item Divergence de la somme des inverses des nombres premiers, théorème~\ref{ThonfVruT}.
    \item RSA, section~\ref{SecEVaFYi}, plus l'exponentielle rapide, plus la recherche de couples de Bézout.
    \item Forme faible du théorème de Dirichlet (avec ses deux lemmes)~\ref{ThoxwTjcl}.
    \item \( (\eZ/p\eZ)^*\simeq \eZ/(p-1)\eZ\), corolaire~\ref{CorpRUndR}, peut-être redondant avec les groupes d'ordre \( pq\).
    \item Irréductibilité des polynômes cyclotomiques, proposition~\ref{PropoIeOVh}.
    \item Théorème des deux carrés, théorème~\ref{ThospaAEI}.
\end{itemize}
%---------------------------------------------------------------------------------------------------------------------------------------------
\paragraph{Corps finis. Applications.}
\begin{itemize}
    \item Théorème de Chevalley-Warning~\ref{ThoLTcYKk}.
    \item Loi de réciprocité quadratique~\ref{ThoMiEiUm}.
    \item \( (\eZ/p\eZ)^*\simeq \eZ/(p-1)\eZ\), corolaire~\ref{CorpRUndR}.
    \item Polynômes irréductibles sur \( \eF_q\).
\end{itemize}

%---------------------------------------------------------------------------------------------------------------------------------------------
\paragraph{Groupe des nombres complexes de module 1. Sous-groupes des racines de l'unité. Applications.}
\begin{itemize}
    \item Théorème de Burnside sur les sous-groupes d'exposant fini de \( \GL(n,\eC)\), théorème~\ref{ThooJLTit}.
    \item Forme faible du théorème de Dirichlet (avec ses deux lemmes)~\ref{ThoxwTjcl} (parce qu'on parle de polynômes cyclotomiques qui sont basés sur les racines de l'unité).
    \item Action du groupe modulaire sur le demi-plan de Poincaré, théorème~\ref{ThoItqXCm}, parce qu'on y utilise un peu les propriétés des nombres du type \( | z |=1\).
    \item Générateurs du groupe diédral, proposition~\ref{PropLDIPoZ}.
    \item Irréductibilité des polynômes cyclotomiques, proposition~\ref{PropoIeOVh}.
    \item Théorème de Wedderburn~\ref{ThoMncIWA}.
\end{itemize}

%---------------------------------------------------------------------------------------------------------------------------------------------
\paragraph{Polynômes irréductibles à une indéterminée. Corps de rupture. Exemples et applications.}
\begin{itemize}
    \item Irréductibilité des polynômes cyclotomiques, proposition~\ref{PropoIeOVh}.
    \item Polynômes irréductibles sur \( \eF_q\).
\end{itemize}
%---------------------------------------------------------------------------------------------------------------------------------------------
\paragraph{Exemples d'actions de groupes sur les espaces de matrices.}
\begin{itemize}
    \item Action du groupe modulaire sur le demi-plan de Poincaré, théorème~\ref{ThoItqXCm}.
    \item Lemme de Morse, lemme~\ref{LemNQAmCLo}.
    \item Sous-groupes compacts de \( \GL(n,\eR)\), lemme~\ref{LemOCtdiaE} ou proposition~\ref{PropQZkeHeG}.
    \item Algorithme des facteurs invariants~\ref{PropPDfCqee}.
\end{itemize}
%---------------------------------------------------------------------------------------------------------------------------------------------
\paragraph{Dimension d'un espace vectoriel (on se limitera au cas de la dimension finie). Rang. Exemples et applications.}
%\index{rang}
\begin{itemize}
    \item Forme alternées de degré maximum, proposition~\ref{ProprbjihK}, parce que c'est ce théorème qui donne l'unicité du déterminant du fait que l'espace est de dimension un.
    \item Théorème de la dimension~\ref{ThonmnWKs}.
        %Ici on peut mettre le théorème de Sylvester.
    \item Extrema liés, théorème~\ref{ThoRGJosS}.
    \item Théorème~\ref{ThoeTMXla} sur la diagonalisation de matrices symétriques.
    \item Stabilité du rang par extension des scalaires, proposition~\ref{PROPooJFQDooZSsxMf}.
\end{itemize}
%---------------------------------------------------------------------------------------------------------------------------------------------
\paragraph{Déterminant. Exemples et applications.}
\begin{itemize}
    \item Forme alternées de degré maximum, proposition~\ref{ProprbjihK}, parce que c'est ce théorème qui donne l'unicité du déterminant du fait que l'espace est de dimension un.
    \item Théorème de Rothstein-Trager~\ref{ThoXJFatfu} parce que le résultant est est un.
    \item Ellipsoïde de John-Loewner, proposition~\ref{PropJYVooRMaPok}.
\end{itemize}
%---------------------------------------------------------------------------------------------------------------------------------------------
\paragraph{Polynômes d’endomorphisme en dimension finie. Réduction d’un endomorphisme en dimension finie. Applications.}
\begin{itemize}
    \item Racine carrée d'une matrice hermitienne positive, proposition~\ref{PropVZvCWn}.
    \item Théorème de Burnside sur les sous-groupes d'exposant fini de \( \GL(n,\eC)\), théorème~\ref{ThooJLTit}.
    \item Décomposition de Dunford, théorème~\ref{ThoRURcpW}.
    \item Algorithme des facteurs invariants~\ref{PropPDfCqee}.
\end{itemize}
%---------------------------------------------------------------------------------------------------------------------------------------------
\paragraph{Sous-espaces stables par un endomorphisme ou une famille d’endomorphismes d’un espace vectoriel de dimension finie. Applications.}
%\index{endomorphisme!sous-espace stable}
\begin{itemize}
    \item Équation de Hill \( y''+qy=0\), proposition~\ref{PropGJCZcjR}.
    \item Décomposition de Dunford, théorème~\ref{ThoRURcpW}.
    \item Théorème de Lie-Kolchin~\ref{ThoUWQBooCvutTO}.
\end{itemize}
%---------------------------------------------------------------------------------------------------------------------------------------------
\paragraph{Endomorphismes diagonalisables en dimension finie.}
\begin{itemize}
    \item Théorème de Burnside sur les sous-groupes d'exposant fini de \( \GL(n,\eC)\), théorème~\ref{ThooJLTit}.
    \item Racine carrée d'une matrice hermitienne positive, proposition~\ref{PropVZvCWn}, parce qu'un utilise le résultat de diagonalisation simultanée.
    \item Équation de Hill \( y''+qy=0\), proposition~\ref{PropGJCZcjR}.
    \item Décomposition de Dunford, théorème~\ref{ThoRURcpW}.
    \item Endomorphismes cycliques et commutant dans le cas diagonalisable, proposition~\ref{PropooQALUooTluDif}.
\end{itemize}
%---------------------------------------------------------------------------------------------------------------------------------------------
\paragraph{Exponentielle de matrices. Applications.}
\begin{itemize}
    \item Décomposition de Dunford, théorème~\ref{ThoRURcpW}.
    \item Théorème de Von Neumann~\ref{ThoOBriEoe}.
\end{itemize}
%---------------------------------------------------------------------------------------------------------------------------------------------
\paragraph{Endomorphismes trigonalisables. Endomorphismes nilpotents.}
\begin{itemize}
    \item Théorème de Burnside sur les sous-groupes d'exposant fini de \( \GL(n,\eC)\), théorème~\ref{ThooJLTit}.
    \item Décomposition de Dunford, théorème~\ref{ThoRURcpW}.
    \item Théorème de Lie-Kolchin~\ref{ThoUWQBooCvutTO}.
\end{itemize}
%---------------------------------------------------------------------------------------------------------------------------------------------
\paragraph{Matrices symétriques réelles, matrices hermitiennes.}
\begin{itemize}
    \item Racine carrée d'une matrice hermitienne positive, proposition~\ref{PropVZvCWn}.
    \item Le lemme au lemme de Morse, lemme~\ref{LemWLCvLXe}.
    \item Connexité des formes quadratiques de signature donnée, proposition~\ref{PropNPbnsMd}.
    \item Théorème~\ref{ThoeTMXla} sur la diagonalisation de matrices symétriques.
\end{itemize}
%---------------------------------------------------------------------------------------------------------------------------------------------
\paragraph{Formes quadratiques sur un espace vectoriel de dimension finie. Orthogonalité, isotropie. Applications.}
\begin{itemize}
    \item Le lemme au lemme de Morse, lemme~\ref{LemWLCvLXe}, voir le lemme de Morse lui-même~\ref{LemNQAmCLo}.
    \item Connexité des formes quadratiques de signature donnée, proposition~\ref{PropNPbnsMd}.
    \item Sous-groupes compacts de \( \GL(n,\eR)\), lemme~\ref{LemOCtdiaE} ou proposition~\ref{PropQZkeHeG}.
    \item Ellipsoïde de John-Loewner, proposition~\ref{PropJYVooRMaPok}.
\end{itemize}
%---------------------------------------------------------------------------------------------------------------------------------------------
\paragraph{Endomorphismes remarquables d’un espace vectoriel euclidien (de dimension finie).}
%\index{endomorphisme!décomposition!polaire}
\begin{itemize}
    \item Décomposition polaire~\ref{ThoLHebUAU}.
    \item Sous-groupes compacts de \( \GL(n,\eR)\), lemme~\ref{LemOCtdiaE} ou proposition~\ref{PropQZkeHeG}.
    \item Théorème~\ref{ThoeTMXla} sur la diagonalisation de matrices symétriques.
\end{itemize}
%---------------------------------------------------------------------------------------------------------------------------------------------
\paragraph{Angles : Définitions et utilisation en géométrie}
\begin{itemize}
    \item Les groupes de pavage de \( \eR^2\), théorème \ref{THOooUPHQooYfeHAy}.
\end{itemize}
%---------------------------------------------------------------------------------------------------------------------------------------------
\paragraph{Isométries d'un espace affine euclidien de dimension finie. Formes réduites. Applications}
\begin{itemize}
    \item Les groupes de pavage de \( \eR^2\), théorème \ref{THOooUPHQooYfeHAy}.
\end{itemize}
%---------------------------------------------------------------------------------------------------------------------------------------------
\paragraph{Applications affines}
\begin{itemize}
    \item Les groupes de pavage de \( \eR^2\), théorème \ref{THOooUPHQooYfeHAy}.
\end{itemize}
%---------------------------------------------------------------------------------------------------------------------------------------------
\paragraph{Barycentres dans un espace affine réel de dimension finie, convexité. Applications.}
\begin{itemize}
    \item Théorème de Carathéodory~\ref{ThoJLDjXLe}.
    \item Points extrémaux de la boule unité dans \( \aL(E)\), théorème~\ref{ThoBALmoQw}.
    \item Enveloppe convexe du groupe orthogonal~\ref{ThoVBzqUpy}.
\end{itemize}
%---------------------------------------------------------------------------------------------------------------------------------------------
\paragraph{Utilisation des groupes en géométrie.}
\begin{itemize}
    \item Coloriage de roulette (\ref{pTqJLY}) et composition de colliers (\ref{siOQlG}).
    \item Forme alternées de degré maximum, proposition~\ref{ProprbjihK}, parce que c'est ce théorème qui donne l'unicité du déterminant du fait que l'espace est de dimension un.
    \item Action du groupe modulaire sur le demi-plan de Poincaré, théorème~\ref{ThoItqXCm}.
    \item Générateurs du groupe diédral, proposition~\ref{PropLDIPoZ}
    \item Isométries du cube, section~\ref{SecPVCmkxM}.
    \item Les groupes de pavage de \( \eR^2\), théorème \ref{THOooUPHQooYfeHAy}.
\end{itemize}
%---------------------------------------------------------------------------------------------------------------------------------------------
\paragraph{Méthodes combinatoires, problèmes de dénombrement.}
\begin{itemize}
    \item Coloriage de roulette (\ref{pTqJLY}) et composition de colliers (\ref{siOQlG}).
    \item Nombres de Bell, théorème~\ref{ThoYFAzwSg}.
    \item Le dénombrement des solutions de l'équation \( \alpha_1 n_1+\ldots \alpha_pn_p=n\) utilise des séries entières et des décompositions de fractions en éléments simples, théorème~\ref{ThoDIDNooUrFFei}.
\end{itemize}
%---------------------------------------------------------------------------------------------------------------------------------------------
\paragraph{Formes quadratiques réelles. Coniques. Exemples et applications.}
\paragraph{Extensions de corps. Exemples et applications.}
\begin{itemize}
    \item Polynômes séparables, proposition~\ref{PropolyeZff}.
    \item Lien entre les racines (multiples) de \( P\) et \( P'\), proposition~\ref{PropolyeZff}.
    \item Théorème de l'élément primitif~\ref{ThoORxgBC}.
    \item À propos d'extensions de \( \eQ\), le lemme~\ref{LemSoXCQH}.
    \item Polynômes irréductibles sur \( \eF_q\).
    \item Polygones réguliers constructibles, théorème de Gauss-Wantzel,~\ref{ThoTWAooEsLjJu}.
\end{itemize}
%---------------------------------------------------------------------------------------------------------------------------------------------
%+++++++++++++++++++++++++++++++++++++++++++++++++++++++++++++++++++++++++++++++++++++++++++++++++++++++++++++++++++++++++++
\section{Analyse}
%+++++++++++++++++++++++++++++++++++++++++++++++++++++++++++++++++++++++++++++++++++++++++++++++++++++++++++++++++++++++++++

%---------------------------------------------------------------------------------------------------------------------------------------------
%---------------------------------------------------------------------------------------------------------------------------------------------
\paragraph{Fonctions holomorphes sur un ouvert de $\eC$. Exemples et applications.}
%---------------------------------------------------------------------------------------------------------------------------------------------
\paragraph{Problèmes d'interversion de limites et d'intégrales.}
%---------------------------------------------------------------------------------------------------------------------------------------------
\paragraph{Suites vectorielles et réelles définies par une relation de récurrence \( u_{n+1}=f(u_n)\). Exemples. Applications à la résolution approchée d'équations.}
%---------------------------------------------------------------------------------------------------------------------------------------------
\paragraph{Approximation d'une fonction par des polynômes et des polynômes trigonométriques. Exemples et applications.}
%---------------------------------------------------------------------------------------------------------------------------------------------
\paragraph{Espérance, variance et moments d'une variable aléatoire.}
%---------------------------------------------------------------------------------------------------------------------------------------------
\paragraph{Extremums : existence, caractérisation, recherche. Exemples et applications.}
%---------------------------------------------------------------------------------------------------------------------------------------------
\paragraph{Suites numériques. Convergence, valeurs d'adhérence. Exemples et applications.}
%---------------------------------------------------------------------------------------------------------------------------------------------
\paragraph{Exemples de développements asymptotiques de suites et de fonctions.}
%---------------------------------------------------------------------------------------------------------------------------------------------
\paragraph{Utilisation de la notion de convexité en analyse.}
\begin{itemize}
    \item Ellipsoïde de John-Loewner, proposition~\ref{PropJYVooRMaPok}.
    \item Peut-être la méthode de Newton, théorème~\ref{ThoHGpGwXk}, mais je ne sais pas très bien pourquoi.
\end{itemize}
%---------------------------------------------------------------------------------------------------------------------------------------------
\paragraph{Équations différentielles \( X'=f(t,X)\). Exemple d'étude des solutions en dimension \( 1\) et \( 2\).}
\begin{itemize}
    \item Théorème de Cauchy-Lipschitz global~\ref{THOooZIVRooPSWMxg}.
\end{itemize}
%---------------------------------------------------------------------------------------------------------------------------------------------
\paragraph{Exemples d'équations aux dérivées partielles linéaires}
%---------------------------------------------------------------------------------------------------------------------------------------------
\paragraph{Équations différentielles linéaires. Systèmes d’équations différentielles linéaires.\\ Exemples et applications.}
\begin{itemize}
    \item Équation de Hill \( y''+qy=0\), proposition~\ref{PropGJCZcjR}.
    \item Théorème de stabilité de Lyapunov~\ref{ThoBSEJooIcdHYp}.
    \item Le système proie-prédateur de Lotka-Volterra~\ref{ThoJHCLooHjeCvT}
    \item Théorème de Cauchy-Lipschitz global~\ref{THOooZIVRooPSWMxg}, si on parvient à réexprimer le théorème dans le cas linéaire.
\end{itemize}
%--------------------------------------------------------------------------------------------------------------------------------------------
\paragraph{Méthodes itératives en analyse numérique matricielle.}
%---------------------------------------------------------------------------------------------------------------------------------------------
\paragraph{Transformation de Fourier. Applications.}
%---------------------------------------------------------------------------------------------------------------------------------------------
\paragraph{Fonctions monotones. Fonctions convexes. Exemples et applications.}
\begin{itemize}
    \item La proposition~\ref{PropMYskGa} donne un résultat sur \( y''+qy=0\) à partir d'une hypothèse de croissance.
    \item L'inégalité de Jensen, proposition~\ref{PropABtKbBo}.
    \item Méthode de Newton, théorème~\ref{ThoHGpGwXk}, si on parvient à expliquer quelle est le lien entre la méthode de Newton et la convexité.
    \item Ellipsoïde de John-Loewner, proposition~\ref{PropJYVooRMaPok}.
\end{itemize}
%---------------------------------------------------------------------------------------------------------------------------------------------
\paragraph{Espaces de fonctions : exemples et applications.}
\begin{itemize}
    \item Théorème de Fischer-Riesz~\ref{ThoGVmqOro}.
    \item Espace de Sobolev \( H^1(I)\), théorème~\ref{ThoESIyxfU}.
    \item Théorème de Cauchy-Lipschitz~\ref{ThokUUlgU}.
    \item Dual de \( L^p\big( \mathopen[ 0 , 1 \mathclose] \big)\) pour \( 1<p<2\), proposition~\ref{PropOAVooYZSodR}.
\end{itemize}
%---------------------------------------------------------------------------------------------------------------------------------------------
\paragraph{Exemples de parties denses et applications.}
\begin{itemize}
    \item Prolongement de fonction définie sur une partie dense, théorème~\ref{ThoPVFQMi}
    \item Complétion d'un espace métrique, théorème~\ref{ThoKHTQJXZ}.
    \item Points extrémaux de la boule unité dans \( \aL(E)\), théorème~\ref{ThoBALmoQw}.
    \item Critère de Weyl, proposition~\ref{PropDMvPDc}.
    \item Densité des polynômes dans \( C^0\big( \mathopen[ 0 , 1 \mathclose] \big)\), théorème de Bernstein~\ref{ThoDJIvrty}.
    \item Enveloppe convexe du groupe orthogonal~\ref{ThoVBzqUpy}.
\end{itemize}
%---------------------------------------------------------------------------------------------------------------------------------------------
\paragraph{Utilisation de la notion de compacité.}
\begin{itemize}
    \item Le théorème de Weierstrass sur la limite uniforme de fonctions holomorphes, théorème~\ref{ThoArYtQO}.
    \item Suite telle que \( \lim_{k\to \infty} d(u_{k+1},u_k)=0\), théorème~\ref{PropLHWACDU}.
    \item Sous-groupes compacts de \( \GL(n,\eR)\), lemme~\ref{LemOCtdiaE} ou proposition~\ref{PropQZkeHeG}.
    \item Théorème de Montel~\ref{ThoXLyCzol}.
    \item Ellipsoïde de John-Loewner, proposition~\ref{PropJYVooRMaPok}.
\end{itemize}
%---------------------------------------------------------------------------------------------------------------------------------------------
\paragraph{Connexité. Exemples et applications.}
\begin{itemize}
    \item Théorème de Runge~\ref{ThoMvMCci}.
    \item Suite telle que \( \lim_{k\to \infty} d(u_{k+1},u_k)=0\), théorème~\ref{PropLHWACDU}.
    \item Théorème de Brouwer en dimension \( 2\) via l'homotopie~\ref{ThoLVViheK}.
    \item Théorème de Lie-Kolchin~\ref{ThoUWQBooCvutTO}.
\end{itemize}
%---------------------------------------------------------------------------------------------------------------------------------------------
\paragraph{Espaces complets. Exemples et applications.}
\begin{itemize}
    \item La proposition~\ref{PropWoywYG} qui donne des indications sur la notion de classes dans \( L^p\).
    \item Prolongement de fonction définie sur une partie dense, théorème~\ref{ThoPVFQMi}
    \item Complétion d'un espace métrique, théorème~\ref{ThoKHTQJXZ}.
    \item Théorème de Fischer-Riesz~\ref{ThoGVmqOro}.
    \item Théorème de Cauchy-Lipschitz global~\ref{THOooZIVRooPSWMxg}.
\end{itemize}
%---------------------------------------------------------------------------------------------------------------------------------------------
\paragraph{Prolongement de fonctions. Exemples et applications.}
\begin{itemize}
    \item Prolongement de fonction définie sur une partie dense, théorème~\ref{ThoPVFQMi}
    \item Lemme de Borel~\ref{LemRENlIEL}.
    \item Prolongement méromorphe de la fonction \( \Gamma\) d'Euler.
    \item Théorème de Tietze~\ref{ThoFFQooGvcLzJ}.
        % théorème de Hadamard
\end{itemize}
%---------------------------------------------------------------------------------------------------------------------------------------------
\paragraph{Espaces de Hilbert. Bases hilbertiennes. Exemples et applications.}
\begin{itemize}
    \item Espace de Sobolev \( H^1(I)\), théorème~\ref{ThoESIyxfU}.
    \item Inégalité isopérimétrique, théorème~\ref{ThoIXyctPo}.
    \item Dual de \( L^p\big( \mathopen[ 0 , 1 \mathclose] \big)\) pour \( 1<p<2\), proposition~\ref{PropOAVooYZSodR}.
        % Fonctions de Haar
\end{itemize}
%---------------------------------------------------------------------------------------------------------------------------------------------
\paragraph{Théorème d'inversion locale, théorème des fonctions implicites. Exemples et applications en analyse et en géométrie.}
%---------------------------------------------------------------------------------------------------------------------------------------------
\paragraph{Applications différentiables définies sur un ouvert de $\eR^n$ . Exemples et applications.}
\begin{itemize}
    \item Extrema liés, théorème~\ref{ThoRGJosS}.
    \item Théorème d'inversion locale, théorème~\ref{ThoXWpzqCn}.
    \item Lemme de Morse, lemme~\ref{LemNQAmCLo}.
\end{itemize}
%---------------------------------------------------------------------------------------------------------------------------------------------
\paragraph{Applications des formules de Taylor.}
\begin{itemize}
    \item Méthode de Newton, théorème~\ref{ThoHGpGwXk}
    \item Lemme de Morse, lemme~\ref{LemNQAmCLo}.
\end{itemize}
%---------------------------------------------------------------------------------------------------------------------------------------------
\paragraph{Continuité et dérivabilité des fonctions réelles d'une variable réelle. Exemples et applications.}
%---------------------------------------------------------------------------------------------------------------------------------------------
%\index{série!numérique}
\begin{itemize}
    \item Divergence de la somme des inverses des nombres premiers, théorème~\ref{ThonfVruT}.
    \item Formule sommatoire de Poisson, proposition~\ref{ProprPbkoQ}.
    \item Théorème taubérien de Hardy-Littlewood~\ref{ThoPdDxgP}.
    \item Nombres de Bell, théorème~\ref{ThoYFAzwSg}.
    \item Partitions d'un entier en parts fixes, proposition~\ref{PropWUFpuBR}.
    \item Théorème d'Abel angulaire~\ref{ThoTGjmeen}.
\end{itemize}
%---------------------------------------------------------------------------------------------------------------------------------------------
\paragraph{Espaces \( L^p\), \( 1\leq p\leq\infty\)}
\begin{itemize}
    \item La proposition~\ref{PropWoywYG} qui donne des indications sur la notion de classes dans \( L^p\).
    \item Théorème de Fischer-Riesz~\ref{ThoGVmqOro}.
    \item Espace de Sobolev \( H^1(I)\), théorème~\ref{ThoESIyxfU}.
    \item Dual de \( L^p\big( \mathopen[ 0 , 1 \mathclose] \big)\) pour \( 1<p<2\), proposition~\ref{PropOAVooYZSodR}.
\end{itemize}
%---------------------------------------------------------------------------------------------------------------------------------------------
\paragraph{Illustrer par des exemples quelques méthodes de calcul d'intégrales de fonctions d’une ou plusieurs variables.}
%---------------------------------------------------------------------------------------------------------------------------------------------
\paragraph{Fonctions définies par une intégrale dépendant d’un paramètre. Exemples et applications.}
\begin{itemize}
    \item Le théorème de Weierstrass sur la limite uniforme de fonctions holomorphes, théorème~\ref{ThoArYtQO}.
    \item Les théorèmes sur les fonctions définies par des intégrales, section~\ref{SecCHwnBDj}.
    \item Lemme de Morse, lemme~\ref{LemNQAmCLo}.
    \item Prolongement méromorphe de la fonction \( \Gamma\) d'Euler.
\end{itemize}
%---------------------------------------------------------------------------------------------------------------------------------------------
\paragraph{Suites et séries de fonctions. Exemples et contre-exemples.}
\begin{itemize}
    \item Formule sommatoire de Poisson, proposition~\ref{ProprPbkoQ}.
    \item Théorème taubérien de Hardy-Littlewood~\ref{ThoPdDxgP}.
    \item Le théorème de Weierstrass sur la limite uniforme de fonctions holomorphes, théorème~\ref{ThoArYtQO}.
    \item La proposition~\ref{PropWoywYG} qui donne des indications sur la notion de classes dans \( L^p\).
    \item Théorème de Montel~\ref{ThoXLyCzol}.
    \item Prolongement méromorphe de la fonction \( \Gamma\) d'Euler.
\end{itemize}
%---------------------------------------------------------------------------------------------------------------------------------------------
\paragraph{Convergence des séries entières, propriétés de la somme. Exemples et applications.}
\begin{itemize}
    \item Processus de Galton-Watson, théorème~\ref{ThoJZnAOA}.
    \item Formule sommatoire de Poisson, proposition~\ref{ProprPbkoQ}.
    \item Nombres de Bell, théorème~\ref{ThoYFAzwSg}.
    \item Théorème d'Abel angulaire~\ref{ThoTGjmeen}.
\end{itemize}
\paragraph{Séries de Fourier. Exemples et applications.}
\begin{itemize}
    \item Formule sommatoire de Poisson, proposition~\ref{ProprPbkoQ}.
    \item Inégalité isopérimétrique, théorème~\ref{ThoIXyctPo}.
    \item Fonction continue et périodique dont la série de Fourier ne converge pas, proposition~\ref{PropREkHdol}.
\end{itemize}
%---------------------------------------------------------------------------------------------------------------------------------------------
\paragraph{Espaces vectoriels normés, applications linéaires continues. Exemples.}
\begin{itemize}
    \item Théorème de Fischer-Riesz~\ref{ThoGVmqOro}.
    \item Théorème de Banach-Steinhaus~\ref{ThoPFBMHBN}.
    \item Dual de \( L^p\big( \mathopen[ 0 , 1 \mathclose] \big)\) pour \( 1<p<2\), proposition~\ref{PropOAVooYZSodR}.
\end{itemize}

% This is part of Mes notes de mathématique
% Copyright (c) 2012-2015,2017, 2019
%   Laurent Claessens
% See the file fdl-1.3.txt for copying conditions.

%+++++++++++++++++++++++++++++++++++++++++++++++++++++++++++++++++++++++++++++++++++++++++++++++++++++++++++++++++++++++++++
\section{Anciennes leçons}
%+++++++++++++++++++++++++++++++++++++++++++++++++++++++++++++++++++++++++++++++++++++++++++++++++++++++++++++++++++++++++++

\paragraph{Opérations élémentaires sur les lignes et les colonnes d’une matrice. Exemples et applications.}
%\index{matrice!lignes et colonnes}
\begin{itemize}
    \item Décomposition de Bruhat, théorème~\ref{ThoizlYJO}.
    \item Algorithme des facteurs invariants~\ref{PropPDfCqee}.
\end{itemize}

\paragraph{Exemples de décompositions remarquables dans le groupe linéaire. Applications}
\begin{itemize}
    \item Décomposition polaire~\ref{ThoLHebUAU}.
    \item Décomposition de Dunford, théorème~\ref{ThoRURcpW}.
\end{itemize}
%---------------------------------------------------------------------------------------------------------------------------------------------
\paragraph{Résultant. Applications.}
\begin{itemize}
    \item Théorème de Rothstein-Trager~\ref{ThoXJFatfu}.
    \item Théorème de Kronecker~\ref{ThoOWMNAVp}.
\end{itemize}
%---------------------------------------------------------------------------------------------------------------------------------------------
\paragraph{Matrices équivalentes. Matrices semblables. Applications.}
\begin{itemize}
    \item Racine carrée d'une matrice hermitienne positive, proposition~\ref{PropVZvCWn}.
    \item Sous-groupes compacts de \( \GL(n,\eR)\), lemme~\ref{LemOCtdiaE} ou proposition~\ref{PropQZkeHeG}.
        %Ici on peut mettre le théorème de Sylvester.
\end{itemize}
\paragraph{Exemples d'utilisation de la notion de dimension d'un espace vectoriel.}
%\index{espace!vectoriel!dimension}
\begin{itemize}
    \item Forme alternées de degré maximum, proposition~\ref{ProprbjihK}, parce que c'est ce théorème qui donne l'unicité du déterminant du fait que l'espace est de dimension un.
    \item Théorème de la dimension~\ref{ThonmnWKs}, bien que ce soit plutôt dans la définition de la dimension que dans l'utilisation.
    \item Théorème de Carathéodory~\ref{ThoJLDjXLe}.
\end{itemize}

\paragraph{Corps des fractions rationnelles à une indéterminée sur un corps commutatif. Applications.}
\begin{itemize}
    \item Théorème de Rothstein-Trager~\ref{ThoXJFatfu}.
    \item Partitions d'un entier en parts fixes, proposition~\ref{PropWUFpuBR}.
\end{itemize}

\paragraph{Anneau de séries formelles. Applications.}
\begin{itemize}
    \item Nombres de Bell, théorème~\ref{ThoYFAzwSg}.
    \item Partitions d'un entier en parts fixes, proposition~\ref{PropWUFpuBR}.
\end{itemize}
%---------------------------------------------------------------------------------------------------------------------------------------------
\paragraph{Formes linéaires et hyperplans en dimension finie. Exemples et applications.}
\begin{itemize}
    \item Extrema liés, théorème~\ref{ThoRGJosS}.
    \item Enveloppe convexe du groupe orthogonal~\ref{ThoVBzqUpy}.
    \item Une forme canonique pour les transvections et dilatations, théorème~\ref{ThoooAZKDooNDcznv}.
\end{itemize}
%---------------------------------------------------------------------------------------------------------------------------------------------
\paragraph{Algèbre des polynômes d'un endomorphisme en dimension finie. Applications.}
\begin{itemize}
    \item Racine carrée d'une matrice hermitienne positive, proposition~\ref{PropVZvCWn}.
\end{itemize}
\paragraph{Formes quadratiques réelles. Exemples et applications.}
\begin{itemize}
    \item Le lemme au lemme de Morse, lemme~\ref{LemWLCvLXe}.
    \item Connexité des formes quadratiques de signature donnée, proposition~\ref{PropNPbnsMd}.
    \item Sous-groupes compacts de \( \GL(n,\eR)\), lemme~\ref{LemOCtdiaE} ou proposition~\ref{PropQZkeHeG}.
    \item Ellipsoïde de John-Loewner, proposition~\ref{PropJYVooRMaPok}.
% On pourra mettre le théorème de Sylvester.
\end{itemize}
\paragraph{Exemples et représentations de groupes finis de petit cardinal}
\paragraph{Coniques. Applications.}
\paragraph{Polynômes d'endomorphisme en dimension finie. Applications à la réduction d'un endomorphisme en dimension finie.}
\begin{itemize}
    \item Endomorphismes cycliques et commutant dans le cas diagonalisable, proposition~\ref{PropooQALUooTluDif}.
\end{itemize}
\paragraph{Applications des nombres complexes à la géométrie. Homographies.}
\begin{itemize}
    \item Action du groupe modulaire sur le demi-plan de Poincaré, théorème~\ref{ThoItqXCm}. Parce que l'action est avec des homographies.
\end{itemize}

\paragraph{Anneaux principaux. Applications}
\begin{itemize}
    \item Polynôme minimal d'endomorphisme semi-simple, théorème~\ref{ThoFgsxCE}.
    \item Théorème de Bézout, corolaire~\ref{CorimHyXy}.
    \item Théorème des deux carrés, théorème~\ref{ThospaAEI}.
    \item Algorithme des facteurs invariants~\ref{PropPDfCqee}.
\end{itemize}
\paragraph{Représentations de groupes finis de petit cardinal.}
\begin{itemize}
    \item Table des caractères du groupe diédral, section~\ref{SecWMzheKf}.
    \item Table des caractères du groupe symétrique \( S_4\), section~\ref{SecUMIgTmO}.
\end{itemize}
\paragraph{Algèbre des polynômes à \( n\) indéterminées (\( n\geq 2\)). Polynômes symétriques. Applications.}
\begin{itemize}
    \item À propos d'extensions de \( \eQ\), le lemme~\ref{LemSoXCQH}.
    \item Polynômes semi-symétriques, proposition~\ref{PropUDqXax}.
    \item Théorème de Chevalley-Warning~\ref{ThoLTcYKk}.
    \item Théorème de Kronecker~\ref{ThoOWMNAVp}.
\end{itemize}

\paragraph{Racines d’un polynômes. Fonctions symétriques élémentaires. Localisation des racines dans les cas réel et complexe.}
\begin{itemize}
    \item À propos d'extensions de \( \eQ\), le lemme~\ref{LemSoXCQH}.
\end{itemize}

\paragraph{135 - Isométries d'un espace affine euclidien de dimension finie. Forme réduite. Applications en dimensions $2$ et $3$.}
\begin{itemize}
    \item Points extrémaux de la boule unité dans \( \aL(E)\), théorème~\ref{ThoBALmoQw}.
    \item Générateurs du groupe diédral, proposition~\ref{PropLDIPoZ}
    \item Isométries du cube, section~\ref{SecPVCmkxM}.
\end{itemize}

\paragraph{248 - Approximation des fonctions numériques par des fonctions polynomiales.\\ Exemples.}
\begin{itemize}
    \item Théorème taubérien de Hardy-Littlewood~\ref{ThoPdDxgP}.
    \item Théorème de Runge~\ref{ThoMvMCci}.
    \item Densité des polynômes dans \( C^0\big( \mathopen[ 0 , 1 \mathclose] \big)\), théorème de Bernstein~\ref{ThoDJIvrty}.
\end{itemize}
%---------------------------------------------------------------------------------------------------------------------------------------------
\paragraph{250 - Loi des grands nombres. Théorème central limite. Applications.}
\begin{itemize}
    \item Presque tous les nombres sont normaux, proposition~\ref{PropEEOXLae}.
    \item Estimation des grands écarts, théorème~\ref{ThoYYaBXkU}.
\end{itemize}
%---------------------------------------------------------------------------------------------------------------------------------------------
\paragraph{251 - Indépendance d’événements et de variables aléatoires. Exemples.}
\begin{itemize}
    \item Presque tous les nombres sont normaux, proposition~\ref{PropEEOXLae}.
    \item Estimation des grands écarts, théorème~\ref{ThoYYaBXkU}.
    \item Densité des polynômes dans \( C^0\big( \mathopen[ 0 , 1 \mathclose] \big)\), théorème de Bernstein~\ref{ThoDJIvrty}.
    \item Problème de la ruine du joueur, section~\ref{SecMSOjfgM}.
\end{itemize}
%---------------------------------------------------------------------------------------------------------------------------------------------
\paragraph{252 - Loi binomiale. Loi de Poisson. Applications.}
\begin{itemize}
        % Cette leçon est classée dans les non couvertes parce qu'il faudrait un développement sur la loi de Poisson.
    \item Estimation des grands écarts, théorème~\ref{ThoYYaBXkU}.
    \item Densité des polynômes dans \( C^0\big( \mathopen[ 0 , 1 \mathclose] \big)\), théorème de Bernstein~\ref{ThoDJIvrty}.
    \item Problème de la ruine du joueur, section~\ref{SecMSOjfgM}.
\end{itemize}
%---------------------------------------------------------------------------------------------------------------------------------------------
\paragraph{219 - Problèmes d'extrémums.}
\begin{itemize}
    \item Extrema liés, théorème~\ref{ThoRGJosS}.
    \item Lemme de Morse, lemme~\ref{LemNQAmCLo}.
    \item Ellipsoïde de John-Loewner, proposition~\ref{PropJYVooRMaPok}.
\end{itemize}
%---------------------------------------------------------------------------------------------------------------------------------------------
\paragraph{Équations différentielles $X' = f (t , X )$. Exemples d'étude des solutions en dimension $1$ et $2$.}
\begin{itemize}
    \item Théorème de Cauchy-Lipschitz~\ref{ThokUUlgU}.
    \item Théorème de stabilité de Lyapunov~\ref{ThoBSEJooIcdHYp}.
    \item Équation de Hill \( y''+qy=0\), proposition~\ref{PropGJCZcjR}.
    \item Le système proie-prédateur de Lotka-Volterra~\ref{ThoJHCLooHjeCvT}
\end{itemize}
%---------------------------------------------------------------------------------------------------------------------------------------------
\paragraph{223 - Convergence des suites numériques. Exemples et applications.}
\begin{itemize}
    \item Calcul d'intégrale par suite équirépartie~\ref{PropDMvPDc}.
    \item Théorème taubérien de Hardy-Littlewood~\ref{ThoPdDxgP}.
    \item Méthode de Newton, théorème~\ref{ThoHGpGwXk}
    \item Théorème d'Abel angulaire~\ref{ThoTGjmeen}.
\end{itemize}
%---------------------------------------------------------------------------------------------------------------------------------------------
\paragraph{224 - Comportement asymptotique de suites numériques. Rapidité de convergence. Exemples.}
\begin{itemize}
    \item Le dénombrement des solutions de l'équation \( \alpha_1 n_1+\ldots \alpha_pn_p=n\) utilise des séries entières et des décompositions de fractions en éléments simples, théorème~\ref{ThoDIDNooUrFFei}.
    \item Divergence de la somme des inverses des nombres premiers, théorème~\ref{ThonfVruT}.
    \item Formule sommatoire de Poisson, proposition~\ref{ProprPbkoQ}, grâce à l'exemple~\ref{ExDLjesf}.
    \item Méthode de Newton, théorème~\ref{ThoHGpGwXk}
    \item Estimation des grands écarts, théorème~\ref{ThoYYaBXkU}.
    \item Le dénombrement des solutions de l'équation \( \alpha_1 n_1+\ldots \alpha_pn_p=n\) utilise des séries entières et des décompositions de fractions en éléments simples, théorème~\ref{ThoDIDNooUrFFei}.
\end{itemize}
%---------------------------------------------------------------------------------------------------------------------------------------------
\paragraph{226 - Comportement d’une suite réelle ou vectorielle définie par une itération \( u_{n+1}=f(u_n)\). Exemples.}
\begin{itemize}
    \item Processus de Galton-Watson, section~\ref{SecBPmrPdtGalton}.
    \item Méthode de Newton, théorème~\ref{ThoHGpGwXk}
    \item Méthode du gradient à pas optimal~\ref{PropSOOooGoMOxG}.
\end{itemize}
%---------------------------------------------------------------------------------------------------------------------------------------------
\paragraph{245 - Fonctions holomorphes et méromorphes sur un ouvert de \( \eC\). Exemples et applications.}
\begin{itemize}
    \item Le théorème de Weierstrass sur la limite uniforme de fonctions holomorphes, théorème~\ref{ThoArYtQO}.
    \item Théorème de Montel~\ref{ThoXLyCzol}.
    \item Prolongement méromorphe de la fonction \( \Gamma\) d'Euler.
\end{itemize}
%---------------------------------------------------------------------------------------------------------------------------------------------
\paragraph{238 - Méthodes de calcul approché d'intégrales et de solutions d’équations différentielles.}
\begin{itemize}
    \item Calcul d'intégrale par suite équirépartie~\ref{PropDMvPDc}.
\end{itemize}
%---------------------------------------------------------------------------------------------------------------------------------------------
\paragraph{240 - Transformation de Fourier. Applications.}
\begin{itemize}
    \item Formule sommatoire de Poisson, proposition~\ref{ProprPbkoQ}.
    \item Équation de Schrödinger, théorème~\ref{ThoLDmNnBR}.
\end{itemize}
%---------------------------------------------------------------------------------------------------------------------------------------------
\paragraph{222 - Exemples d’équations différentielles. Solutions exactes ou approchées.}
\begin{itemize}
    \item Équation \( y''+qy=0\),~\ref{subsecSyTwyM}.
\end{itemize}
%---------------------------------------------------------------------------------------------------------------------------------------------
\paragraph{225 - Étude locale de surfaces. Exemples.}
\begin{itemize}
    \item Lemme de Morse, lemme~\ref{LemNQAmCLo}.
\end{itemize}
%---------------------------------------------------------------------------------------------------------------------------------------------
\paragraph{Exemples de problèmes d’interversion de limites.}
%\index{limite!inversion}
\begin{itemize}
    \item Théorème taubérien de Hardy-Littlewood~\ref{ThoPdDxgP} parce que l'énoncé revient à montrer la limite \( \lim_{x\to 1^-} \sum_{n\in \eN}a_nx^n=\sum_{n\in \eN}a_n\).
    \item Le théorème de Weierstrass sur la limite uniforme de fonctions holomorphes, théorème~\ref{ThoArYtQO}.
    \item La proposition~\ref{PropWoywYG} qui donne des indications sur la notion de classes dans \( L^p\). Ça utilise la convergence monotone pour  pour permuter une somme et une intégrale.
    \item Les théorèmes sur les fonctions définies par des intégrales, section~\ref{SecCHwnBDj}.
    \item Nombres de Bell, théorème~\ref{ThoYFAzwSg}.
\end{itemize}
\paragraph{254 - Espaces de Schwartz et distributions tempérées.}
\begin{itemize}
    \item Formule sommatoire de Poisson, proposition~\ref{ProprPbkoQ}.
    \item Équation de Schrödinger, théorème~\ref{ThoLDmNnBR}.
\end{itemize}

\paragraph{Théorème d’inversion locale, théorème des fonctions implicites. Exemples et applications.}
\begin{itemize}
    \item Extrema liés, théorème~\ref{ThoRGJosS}.
    \item Théorème d'inversion locale, théorème~\ref{ThoXWpzqCn}.
    \item Lemme de Morse, lemme~\ref{LemNQAmCLo}.
    \item Théorème de Von Neumann~\ref{ThoOBriEoe}.
\end{itemize}
\paragraph{Continuité et dérivabilité des fonctions réelles d'une variable réelle. Exemples et contre-exemples.}
\begin{itemize}
    \item Les théorèmes sur les fonctions définies par des intégrales, section~\ref{SecCHwnBDj}.
    \item Lemme de Borel~\ref{LemRENlIEL}.
\end{itemize}
\paragraph{232 - Méthodes d'approximation des solutions d’une équation $F(X)=0$. Exemples.}
\begin{itemize}
    \item Méthode de Newton, théorème~\ref{ThoHGpGwXk}
    \item Méthode du gradient à pas optimal~\ref{PropSOOooGoMOxG}.
\end{itemize}
\paragraph{Illustrer par des exemples quelques méthodes de calcul d'intégrales de fonctions d’une ou plusieurs variables réelles.}
\begin{itemize}
    \item Calcul d'intégrale par suite équirépartie~\ref{PropDMvPDc}.
    \item Théorème de Rothstein-Trager~\ref{ThoXJFatfu}.
\end{itemize}
\paragraph{256 - Transformation de Fourier dans \( \swS(\eR^d)\) et \( \swS'(\eR^d)\).}
\begin{itemize}
    \item Formule sommatoire de Poisson, proposition~\ref{ProprPbkoQ}.
    \item Équation de Schrödinger, théorème~\ref{ThoLDmNnBR}.
\end{itemize}
%---------------------------------------------------------------------------------------------------------------------------------------------
\paragraph{Espaces de Schwartz \( \swS(\eR^d)\) et distributions tempérées. Transformation de Fourier dans \( \swS(\eR^d)\) et \( \swS'(\eR^d)\)}
\begin{itemize}
    \item Formule sommatoire de Poisson, proposition~\ref{ProprPbkoQ}.
    \item Équation de Schrödinger, théorème~\ref{ThoLDmNnBR}.
\end{itemize}
%---------------------------------------------------------------------------------------------------------------------------------------------
\paragraph{Espaces de Schwartz. Distributions. Dérivation au sens des distributions.}
\begin{itemize}
    \item L'équation \( (x-x_0)^{\alpha}u=0\) pour \( u\in\swD'(\eR)\), théorème~\ref{ThoRDUXooQBlLNb}.
    \item Espace de Sobolev \( H^1(I)\), théorème~\ref{ThoESIyxfU}.
    \item Équation de Schrödinger, théorème~\ref{ThoLDmNnBR}.
\end{itemize}

\paragraph{Analyse numérique matricielle : résolution approchée de systèmes linéaires, recherche de vecteurs propres, exemples.}
\paragraph{Fonctions développables en série entière, fonctions analytiques. Exemples.}
\begin{itemize}
    \item Le dénombrement des solutions de l'équation \( \alpha_1 n_1+\ldots \alpha_pn_p=n\) utilise des séries entières et des décompositions de fractions en éléments simples, théorème~\ref{ThoDIDNooUrFFei}.
\end{itemize}
\paragraph{Fonction caractéristique et transformée de Laplace d'une variable aléatoire. Exemples et applications.}
\paragraph{Modes de convergence d’une suite de variables aléatoires. Exemples et applications.}
\paragraph{Variables aléatoires à densité. Exemples et applications.}
\paragraph{Variables aléatoires discrètes. Exemples et applications.}
\paragraph{Produit de convolution, transformation de Fourier. Applications.}
\paragraph{Étude métrique des courbes. Exemples.}
\paragraph{Théorèmes de point fixe. Exemples et applications.}
\begin{itemize}
    \item Processus de Galton-Watson, théorème~\ref{ThoJZnAOA}.
    \item Théorème d'inversion locale, théorème~\ref{ThoXWpzqCn}.
    \item Théorème de Brouwer en dimension \( 2\) via l'homotopie~\ref{ThoLVViheK}.
    \item Théorème de Picard~\ref{ThoEPVkCL} et l'inséparable théorème de Cauchy-Lipschitz~\ref{ThokUUlgU}
\end{itemize}
%---------------------------------------------------------------------------------------------------------------------------------------------
\begin{itemize}
    \item Inégalité isopérimétrique, théorème~\ref{ThoIXyctPo}.
    \item Théorème des quatre sommets, théorème~\ref{THOooFRBBooWKZcfY}.
\end{itemize}
%---------------------------------------------------------------------------------------------------------------------------------------------
\begin{itemize}
    \item Formule sommatoire de Poisson, proposition~\ref{ProprPbkoQ}.
\end{itemize}
%---------------------------------------------------------------------------------------------------------------------------------------------
\paragraph{Sous-variétés de \( \eR^n\). Exemples.}
\begin{itemize}
    \item Extrema liés, théorème~\ref{ThoRGJosS}.
    \item Théorème de Von Neumann~\ref{ThoOBriEoe}.
    \item Lemme de Morse, lemme~\ref{LemNQAmCLo}.
\end{itemize}
%---------------------------------------------------------------------------------------------------------------------------------------------
\paragraph{Suites et séries de fonctions intégrables. Exemples et applications.}
\begin{itemize}
    \item La proposition~\ref{PropWoywYG} qui donne des indications sur la notion de classes dans \( L^p\).
    \item Le théorème de Weierstrass sur la limite uniforme de fonctions holomorphes, théorème~\ref{ThoArYtQO}.
    \item Les théorèmes sur les fonctions définies par des intégrales, section~\ref{SecCHwnBDj}.
    \item Théorème de Fischer-Riesz~\ref{ThoGVmqOro}.
    \item Prolongement méromorphe de la fonction \( \Gamma\) d'Euler.
    \item Problème de la ruine du joueur, section~\ref{SecMSOjfgM}.
\end{itemize}
%---------------------------------------------------------------------------------------------------------------------------------------------
\paragraph{Utilisation en probabilités du produit de convolution et de la transformation de Fourier ou de Laplace.}
\begin{itemize}
    \item Processus de Galton-Watson, lemme~\ref{LemezrOiI} et théorème~\ref{ThoJZnAOA}.
    \item Fonction caractéristique~\ref{PropDerFnCaract}.
    \item Théorème central limite~\ref{ThoOWodAi}.
\end{itemize}
\paragraph{Suites de variables de Bernoulli indépendantes.}
\begin{itemize}
    \item Processus de Galton-Watson, section~\ref{SecBPmrPdtGalton}.
    \item Estimation des grands écarts, théorème~\ref{ThoYYaBXkU}.
    \item Densité des polynômes dans \( C^0\big( \mathopen[ 0 , 1 \mathclose] \big)\), théorème de Bernstein~\ref{ThoDJIvrty}.
    \item Problème de la ruine du joueur, section~\ref{SecMSOjfgM}.
\end{itemize}


% SCRIPT MARK -- RESEARCH PART

\emptyInputPath
\addInputPath{tex/research}

    \selectlanguage{english}
\part{Giulietta}

\input{questionsMazhe}

\chapter{Some old results}
% This is part of (almost) Everything I know in mathematics
% Copyright (c) 2013-2014, 2019-2020
%   Laurent Claessens
% See the file fdl-1.3.txt for copying conditions.

%+++++++++++++++++++++++++++++++++++++++++++++++++++++++++++++++++++++++++++++++++++++++++++++++++++++++++++++++++++++++++++ 
\section{Some old questions}
%+++++++++++++++++++++++++++++++++++++++++++++++++++++++++++++++++++++++++++++++++++++++++++++++++++++++++++++++++++++++++++


\newcommand{\refprob}[1]{\ref{#1}, page \pageref{#1}}

\section{Still some questions and problems}


\begin{enumerate}
\item~\refprob{ProbNablades} Où vient le $\nabla$ dans la définition de WKB ?
 \item~\refprob{ProbFibra} à voir que quand on a une courte suite exacte d'espaces qui s'injèctent les uns dans les autres, on a une longue suite exacte dans les espaces de cohomologie.
\item ~\refprob{ProbAdJthetaj} En réalité, c'est sans doute l'avant-dernière ligne de la preuve du théorème 3.1 dans Semisimple Symplectic Symmetric spaces. Ou bien (i.2) de la proposition 4.1
\item~\refprob{propCrtadeux} Encore le problème de $\theta$ comme automorphisme interne.

\item~\refprob{ProbWeylMoy} Une référence pour l'équivalence entre Moyal et Weyl
\item~\refprob{ProbEnonSLdef} Un énoncé exact du produit sur $\SL(2,\eR)$.
\item~\refprob{Probintertw} Prouver que $\mT_{\theta}$ entrelace réellement $\rho_{\nu}$ et $dL$.

    \item
        A friend said me that the condition \eqref{eq:symple_Lie} means that the form is closed in the sense of Chevalley and thus its left invariant prolongation is closed in the sense of de Rahm.
    \item

	L’espace symétrique $M$ est vu comme $G/H$ où $G$ est la composante connexe à l’identité de $Aff(M)$, c'est à dire  le plus grand groupe connexe de transformations affines (c'est le théorème \ref{ThoGplugdSymssgpAff}), où $Aff(M)=$groupe des transformations affines de $M$.

A-t-on forcement : composante connexe à l’identite de $Aff(M)  =$ larger connex group of $Aff(M)$ ?
(genre, pourrait-on avoir un ss-groupe connexe qui ne passe pas par l’identité, plus grand que tous ceux qui passent par l’identité ?)

On sait que toutes les symétries $s_x$ sont des transformations affines, mais le contraire n’est pas vrai ; en particulier, $Aff(M)$  peut contenir des elements qui ne sont pas des symétries.  Ceci pour savoir la definition du théorème ~\ref{ThoGplugdSymssgpAff}, était la même que celle du théorème 2.4 du papier de Bieliavsky, où il exprime $M=G/H$, avec cette fois $G =$ transvection group de $M$, qui est INCLUS dans $Aff(M)$.

Dans ton théorème~\ref{ThoGplugdSymssgpAff}, $G$ be the LARGER connex group of affine transformation, tandis que chez Bieliavsky (i), le transvection group $G$ is the SMALLEST subgroup of Aff(M) which is transitive etc.

\end{enumerate}


%+++++++++++++++++++++++++++++++++++++++++++++++++++++++++++++++++++++++++++++++++++++++++++++++++++++++++++++++++++++++++++ 
\section{Some statements waiting for proofs}
%+++++++++++++++++++++++++++++++++++++++++++++++++++++++++++++++++++++++++++++++++++++++++++++++++++++++++++++++++++++++++++

We provide here some results that are at undergrad level. Some of them are contained in the part ``Pour l'agrégation'' in French.

\begin{lemma}[Schur's lemma] \label{lem:Schur}
If $\dpt{\phi}{\lG}{\gl(V)}$ is irreducible, then the only endomorphism of $V$ which commutes with all $\phi(\lG)$ are multiples of identity.
\end{lemma}
%TODO: look the link with the Schur'lemma given in Agregation.

Now we consider $W=\{u\in V\tq\exists n\in\eN:A^nu=0\}$ and $v\in V$. There exists a $v'\in V$ such that $A^p(v)=A^{p+1}(v')$. Writing  $v= A(v')+(v-A(v'))$, we find
\begin{equation}\label{eq:VAW}
V\subset A(V)+W
\end{equation}
because $A^p(v)-A^{p+1}(v')=0$, $v-A(v')\in W$.

If we apply $A$ on this, we find $A(V)\subset A^2(V)+A(W)$. Reinserting it into the right hand side of \eqref{eq:VAW}, we find $V\subset A^2(V+W)$ and repeating $p$ times this process, we find $V=A^p(V)+W$ and the sum is direct because none of the elements of $A^p(V)$ is annihilated by $A$:
\begin{equation}\label{eq:ApoplusW}
V=A^p(V)\oplus W.
\end{equation}
%TODO: regarder si ce truc est dans la partie agrégative. Sinon l'ajouter.

\begin{theorem}[Cayley-Hamilton]\index{Cayley-Hamilton theorem} \label{ThoCayleyHamilton}
    A square matrix on \( \eR\) or \( \eC\) satisfies its own characteristic equation. If \( A\in\eM_n(\eK)\), we consider the polynomial \( p(\lambda)=\det(A-\lambda\mtu)\). Thus \( p(A)=0\).
\end{theorem}
For a proof see \wikipedia{en}{Cayley-Hamilton_theorem}{Wikipedia}, theorem~\ref{ThoCalYWLbJQ} and theorem~\ref{ThoHZTooWDjTYI}.

\begin{proposition}[\cite{SerreMatrices}]     \label{PropMtrDiagablaUnit}
    If \( M\) is a complex \( n\times n\) matrix, then there exists an unitary matrix \( U\) such that \( U^*MU\) is upper triangular.
\end{proposition}

\begin{definition}
    A \defe{positive defined}{positive!defined matrix} matrix is a matrix $B$ such that
    \begin{equation}
        \sum_{ij}B_{ij}\overline{ x_i }x_j
    \end{equation}
    is real and positive for every complex vector $x$.
\end{definition}

\begin{proposition}
    A positive defined matrix is Hermitian.
\end{proposition}

\begin{proof}
    We define the Hermitian matrices $M=(B+B^*)/2$ and $N=(B-B^*)/2i$, so $B=M+iN$ and
    \begin{equation}
        \bar x Bx=\bar x M x+i\bar x Nx.
    \end{equation}
    The matrices $M$ and $B$ being Hermitian, the numbers $\bar xMx$ and $\bar xNx$ are real. If $\bar xBx$ has to be real, we need $\bar xNx=0$ for every $x$. This shows that $N=0$, so that $B=N$.
\end{proof}

\begin{theorem}
Let $G$ be a Lie group and $H$ a subgroup (with no special other structures) of $G$. If $H$ is a closed subset of $G$ then there exists an unique analytic structure on $H$ such that $H$ is a topological Lie subgroup of $G$.
\label{Helgason2.3}
\end{theorem}
This comes from \cite{Helgason}, chapter 2, theorem 2.3.

\begin{lemma}
Let $G$ be a connected Lie group with Lie algebra $\mG$ and let $\varphi$ be an analytic homomorphism of $G$ into a Lie group $X$ with Lie algebra $\mathcal{X}$. Then

\begin{enumerate}
\item The kernel $\varphi^{-1}(e)$ is a topological Lie subgroup of $G$. Its Lie algebra is the kernel of $d\varphi_e$.
\item The image $\varphi(G)$ is a Lie subgroup of $\mathcal{X}$ with Lie algebra $d\varphi(\mG)\subset\mathcal{X}$.
\end{enumerate}
\label{Helgason5.1}
\end{lemma}
This comes from \cite{Helgason}, chapter 2, lemma 5.1.

\begin{lemma}
Let $G$ and $H$ be two Lie group, whose Lie algebra are $\mG$ and $\mH$. If $\dpt{\theta}{G}{H}$ is a surjective map, then we have $\mH\simeq\mG/Ker\,d\theta_e$.
\label{1203r1}
\end{lemma}

\begin{theorem} \label{1503t1}
Let us consider $\dpt{Ad}{SU(2)}{GL(3)}$, $Ad(U)X=UXU^{-1}$. We have the following properties:

\begin{enumerate}
\item $Ad$ is a linear homomorphism,
\item it takes his values in $\SO(3)$; then we can write $\dpt{Ad}{SU(2)}{\SO(3)}$,
\item it is surjective,
\item $Ker\,Ad=\eZ_2$,
\item all these properties show that \[\SO(3)=\frac{SU(2)}{\eZ_2}.\]
\end{enumerate}
\end{theorem}

\begin{definition}
If $(a_k)$ is a sequence in $\eR$, its \defe{upper limit}{upper limit} is the real number
\[
  \lim\sup_{n\to\infty}a_n=\lim_{l\to\infty}\sup\{a_k:k\geq l\}.
\]
\end{definition}

\begin{lemma}
If $\omega$ is a $k$-form (not specially a symplectic one), and $\nabla$ a torsion free connection, one has
\begin{equation}\label{eq:d_omega_nabla}
  (d\omega)(X_0,\ldots,X_k)=\sum_{i=0}^k (-1)^i(\nabla_{X_i}\omega)(X_0,\ldots,\hX_i,\ldots X_k).
\end{equation}
\end{lemma}

\begin{remark}
The link between $d$ and $\nabla$ comes from the fact that in the left hand side of \eqref{eq:d_omega_nabla} appears some commutators $[X_i,X_j]$, but since the connection is torsion-free,
\[
  [X_i,X_j]=\nabla_{X_i}X_j-\nabla_{X_j}X_i
\]
\end{remark}
The main consequence of this lemma is that $\nabla\omega=0$ implies $d\omega=0$.

\begin{proposition} \label{prop:fdefint}
    Consider a function $\dpt{f}{X\times E}{\overline{\eR}}$ and $z_0\in E$ such that
    \begin{itemize}
        \item for all $z\in E$, the function $x\to(x,z)$ is integrable,
        \item for (almost) all $x\in X$, the function $z\to f(x,z)$ is continuous at $z_0$,
        \item there exists a function $g\geq 0$ such that for all $z\in E$, $| f(x,z) |\leq g(x)$ almost everywhere in $X$.
    \end{itemize}
    Then the function $\dpt{h}{E}{\eR}$ defined by $h(z)=\int_Xf(x,z)$ is continuous at $z_0$.
\end{proposition}
This proposition is (up to ``almost'') the theorem~\ref{ThoKnuSNd}.

% \begin{definition}

% If $(a_k)$ is a sequence in $\eR$, its \defe{upper limit}{upper limit} is the real number
% \[
%   \lim\sup_{n\to\infty}a_n=\lim_{l\to\infty}\sup\{a_k:k\geq l\}.
% \]
% \end{definition}
%
% \begin{proposition}\label{prop:cv_lim_sup}
% If $(a_k)$ is a sequence in $\eR$ such that there exists a $a\in\eR$ for which for any $k\in\eN$,
% \[
% \lim\sup_{n\to\infty}a_n\leq a\leq a_k
% \]
% then $(a_k)$ admits a limit and $\lim_{n\to\infty}a_k=a$.
% \end{proposition}
%
% \begin{lemma}
% If $\omega$ is a $k$-form (not specially a symplectic one), and $\nabla$ a torsion free connection, one has
%
% \begin{equation}\label{eq:d_omega_nabla}
%   (d\omega)(X_0,\ldots,X_k)=\sum_{i=0}^k (-1)^i(\nabla_{X_i}\omega)(X_0,\ldots,\hX_i,\ldots X_k).
% \end{equation}
% \end{lemma}
%
% \begin{remark}
% The link between $d$ and $\nabla$ comes from the fact that in the left hand side of \eqref{eq:d_omega_nabla} appears some commutators $[X_i,X_j]$, but since the connection is torsion-free,
% \[
%   [X_i,X_j]=\nabla_{X_i}X_j-\nabla_{X_j}X_i
% \]
% \end{remark}
% The main consequence of this lemma is that $\nabla\omega=0$ implies $d\omega=0$.


\chapter{Categories}        \label{chap_category}
\input{categorie}

\chapter{Topology}              \label{chap_topology}
\input{147_topo}
\input{directedSets}

\chapter{Manifolds} \label{Chapitre_FB}
% This is part of (almost) Everything I know in mathematics
% Copyright (c) 2010-2017, 2019
%   Laurent Claessens
% See the file fdl-1.3.txt for copying conditions.

\section{Differentiable manifolds}
%+++++++++++++++++++++++++++++++++

Most of the results about differential geometry come from \cite{kobayashi, madore, Helgason, ms_book, dgbook}.

\subsection{Definition and examples}
%-----------------------------------

\begin{definition}
    A $n$-dimensional \defe{differentiable manifold}{differentiable!manifold}\index{manifold} is a set $M$ and a system of charts $\{(\mU_{\alpha},\varphi_{\alpha})\}_{\alpha\in I}$ where each set $\mU_{\alpha}$ is open in $\eR^n$ and the maps $\dpt{\varphi_{\alpha}}{\mU_{\alpha}}{M}$ are injective and satisfy the three following conditions:

    \begin{itemize}
    \item every $x\in M$ is contained in at least one set $\varphi_{\alpha}(\mU_{\alpha})$,
    \item for any two charts $\dpt{\varphi_{\alpha}}{\mU_{\alpha}}{M}$ and $\dpt{\varphi_{\beta}}{\mU_{\beta}}{M}$, the set
    \[
       \varphi_{\alpha}^{-1}( \varphi_{\alpha}(\mU_{\alpha})\cap\varphi_{\beta}(\mU_{\beta}) )
    \]
    is an open subset of $\mU_{\alpha}$,
    \item the map
    \[
      \dpt{  (\varphi_{\beta}^{-1}\circ\varphi_{\alpha})  }{   \varphi_{\alpha}^{-1}( \varphi_{\alpha}(\mU_{\alpha})\cap\varphi_{\beta}(\mU_{\beta})  )   }{\mU_{\beta}}
    \]
    is differentiable\footnote{In the sequel, by ``differentiable'' we always mean smooth. If this map is differentiable, $C^k$, analytic,\ldots then the manifold is said to be differentiable, $C^k$, analytic,\ldots} as map from $\eR^n$ to $\eR^n$.
    \end{itemize}
\end{definition}

Each time we say ``manifold``, we mean ``differentiable manifold``. We will only consider manifolds with Hausdorff topology (see later for the definition of a topology on a manifold). Any open set of $\eR^n$ is a differentiable manifold if we choose the identity map as chart system. Most of surfaces $z=f(x,y)$ in $\eR^3$ are manifolds, depending on certain regularity conditions on~$f$.

If $M_1$ and $M_2$ are two differentiable manifolds, a map $\dpt{f}{M_1}{M_2}$ is \defe{differentiable}{differentiable!map} if $f$ is continuous and for each two coordinate systems $\dpt{\varphi_1}{\mU_1}{M_1}$ and $\dpt{\varphi_2}{\mU_2}{M_2}$, the map $\varphi_2^{-1}\circ f\circ\varphi_1$ is differentiable on its domain. One can show that if $\dpt{f}{M_1}{M_2}$ and $\dpt{g}{M_2}{M_3}$ are differentiable, then $\dpt{g\circ f}{M_1}{M_3}$ is differentiable.

\subsubsection{Example: the sphere}\index{sphere}

The sphere $S^n$ is the set
\[
  S^n=\{  (x_1,\ldots, x_{n+1})\in\eR^{n+1}\tq \|x\|=1  \}
\]
for which we consider the following open set in $\eR^n$:
\[
   \mU=\{  (u_1,\ldots,u_n)\in\eR^n\tq\|u\|<1  \}
\]
and the charts $\dpt{\varphi_i}{\mU}{S}$, and $\dpt{\tilde{\varphi}_i}{\mU}{S}$
\begin{subequations}
\begin{align}
   \varphi_i(u_1,\ldots,u_n)&=(u_1,\ldots,u_{i-1}, \sqrt{  1-\|u\|^2  },u_i,\ldots,u_n )\\
   \tilde{\varphi}_i(u_1,\ldots,u_n)&=(u_1,\ldots,u_{i-1}, -\sqrt{  1-\|u\|^2  },u_i,\ldots,u_n ).
\end{align}
\end{subequations}
These map are clearly injective. To see that $\varphi(\mU)\cup\tilde{\varphi}(\mU)=S$, consider $(x_1,\ldots,x_{n+1})\in S$. Then at least one of the $x_i$ is non zero. Let us suppose $x_1\neq 0$, thus $x_1^2=1-(x_2^2+\cdots+x_{n+1}^2)$ and
\begin{equation}\label{eq:xupm}
   x_1=\pm\sqrt{1-(\ldots)}.
\end{equation}
If we put $u_i=x_{i+1}$, we have $x=\varphi(u)$ or $x=\tilde{\varphi}(u)$ following the sign in relation \eqref{eq:xupm}. The fact that $\varphi^{-1}\circ\tilde{\varphi}$ and $\tilde{\varphi}^{-1}\circ\varphi$ are differentiable is a ``first year in analysis exercise``.

\subsubsection{Example: projective space}

On $\eR^{n+1}\setminus\{o\}$, we consider the equivalence relation $v\sim\lambda w$ for all non zero $\lambda\in\eR$, and we put
\[
  \eR P^n=\left(\eR^{n+1}\setminus\{o\}\right)/\sim.
\]
This is the set of all the one dimensional subspaces of $\eR^{n+1}$. This is the real \defe{projective space}{projective!real space} of dimension $n$. We set $\mU=\eR^n$ and
\[
  \varphi_i(u_1,\ldots,u_n)=\Span\{ (u_1,\ldots,u_{i-1},1,u_i,\ldots,u_n) \}.
\]
One can see that this gives a manifold structure to $\eR P^n$. Moreover, the map
		\begin{equation}
		\begin{aligned}
			A \colon S^n &\to \eR P^n\
			v&\mapsto \Span v
		\end{aligned}
	\end{equation}
is differentiable.

Let us show how to identify $\eR\cup\{ \infty \}$ to $\eR P^1$, the set of directions in the plane $\eR^2$. Indeed consider any vertical line $l$ (which does contain the origin). A non vertical vector subspace of $\eR^2$ intersects $l$ in one and only one point, while the vertical vector subspace is associated with the infinite point.

\subsection{Topology}
%--------------------

\begin{propositionDef}
    Let \( M\) be a manifold. A subset $V\subset M$ is \defe{open}{topology!on manifold} if for every chart $\dpt{\varphi}{\mU}{M}$, the set $\varphi^{-1}(V\cap\varphi(\mU))$ is open in $\mU$. 

    The set of open sets in \( M\) is a topology.
\end{propositionDef}

\begin{proof}
    First we prove that the open system defines a topology. For this, remark that $\varphi_{\alpha}^{-1}$ is injective (if not, there should be some multivalued points). Then $\varphi^{-1}(A\cap B)=\varphi^{-1}(A)\cap\varphi^{-1}(B)$. If $V_1$ and $V_2$ are open in $M$, then
    \begin{equation}
        \varphi^{-1}(V_1\cap V_2\cap\varphi(\mU))=\varphi^{-1}(V_1\cap\varphi(\mU))\cap\varphi^{-1}(V_2\cap\varphi(\mU))
    \end{equation}
    which is open in $\eR^n$. The same property works for the unions.
\end{proof}

\begin{theorem}
    Let \( M\) be a manifold. Its topology has the following properties.
    \begin{enumerate}
        \item the charts maps are continuous,
        \item the sets $\varphi_{\alpha}(\mU_{\alpha})$ are open.
    \end{enumerate}
\end{theorem}

\begin{proof}
    We proof the continuity of $\dpt{\varphi}{\mU}{M}$; for an open set $V$ in $M$, we have to show that $\varphi^{-1}(V)$ is open in $\mU\subset\eR^n$. But the definition of the topology on $M$, is precisely the fact that $\varphi^{-1}(V\cap\varphi(\mU))$ is open.
\end{proof}

\begin{definition}
    If $M$ and $M$ are two analytic manifolds, a map $\dpt{\phi}{M}{N}$ is \defe{regular}{regular}\label{PgDefRegular} at $p\in M$ if it is analytic at $p$ and $\dpt{d\phi_p}{T_pM}{T_{\phi(p)}N}$ is injective.
\end{definition}


%+++++++++++++++++++++++++++++++++++++++++++++++++++++++++++++++++++++++++++++++++++++++++++++++++++++++++++++++++++++++++++
\section{Tangent and cotangent bundle}
%+++++++++++++++++++++++++++++++++++++++++++++++++++++++++++++++++++++++++++++++++++++++++++++++++++++++++++++++++++++++++++

\subsection{Tangent vector}
%--------------------------

As first attempt, we define a tangent vector of $M$ at the point $x\in M$ as the ``derivative'' of a path $\dpt{\gamma}{(-\epsilon,\epsilon)}{M}$ such that $\gamma(0)=x$. It is denoted by
\[
\gamma'(0)= \dsdd{\gamma(t)}{t}{0}.
\]
The question is to correctly define de derivative in the right hand side. Such a definition is achieved as follows. 

\begin{definition}      \label{DEFooJJVIooDUBwDJ}
    A \defe{tangent vector}{} to the manifold $M$ is a linear map $X\colon  C^{\infty}(M)\to \eR$ which can be written under the form
    \begin{equation}  \label{eq_deftgpath}
       Xf=(f\circ X)'(0)=\Dsdd{f(X(t))}{t}{0}
    \end{equation}
    for a certain path $X\colon \eR\to M$. Notice the abuse of notation between the tangent vector and the path which defines it.

    A more formal way to define a tangent vector is to say that it is an equivalent class of path in the sense that two path are equivalent if and only if they induced maps by \eqref{eq_deftgpath} are equals.
\end{definition}

\begin{remark}      \label{REMooJQFHooQuoZxt}
    The notation \( \gamma'(0)\) for the tangent vector to the curve \( \gamma\) has to be taken with caution. In particular, \( \gamma'(0)\) is not defined by the limit
    \begin{equation}        \label{EQooVMGFooFUCNEY}
        \lim_{\epsilon\to 0} \frac{ \gamma(\epsilon)-\gamma(0) }{ \epsilon }
    \end{equation}
    because when \( M\) is a manifold, there is in general no notion of difference between the points of \( M\), so that the difference \( \gamma(\epsilon)-\gamma(0)\) has no meaning.

    The only definition of \( \gamma'(0)\) is as differential operator.

    The manifold could, of course have some additional structure which allows to write the differential quotient \eqref{EQooVMGFooFUCNEY}. This is the case when \( M=\eR^n\) or when \( M\) is a matrix group. In these cases, the question of the link between \( \gamma'(0)\) and the ``true'' derivative of \( \gamma\) has to be studied.

    In that case we have the same notational problem with ``$df$''. Let \( f\colon M\to \eR\) where \( M\) is a manifold like \( \eR^n\). The symbol \( df_a\) with \( a\in M\) can be the differential of \( f\) as function \( M\to \eR\), so that \( df_a\) is a linear map from \( \eR^n\) to \( \eR\). But \( df_a\) can also be the linear map \( df_a\colon T_aM\to T_{f(a)}\eR\) where the spaces \( T_aM\) and \( T_{f(a)}\eR\) are made of differential operators.

    This is the point of the section \ref{SECooTSAJooNtjgMD}.
\end{remark}

Using the chain rule $d(g\circ f)(a)=dg(f(a))\circ df(a)$ for the differentiation in $\eR^n$, one sees that this equivalence notion doesn't depend on the choice of $\varphi$. In other words, if $\varphi$ and $\tilde{\varphi}$ are two charts for a neighbourhood of $x$, then $(\varphi^{-1} \circ\gamma)'(0)=(\varphi^{-1} \circ\sigma)'(0)$ if and only if $(\tilde{\varphi}^{-1} \circ\gamma)'(0)=(\tilde{\varphi}^{-1} \circ\sigma)'(0)$. The space of all tangent vectors at $x$ is denoted by $T_xM$. There exists a bijection $[\gamma]\leftrightarrow (\varphi^{-1}\circ\gamma)'(0)$ between $T_xM$ and $\eR^n$, so $T_xM$ is endowed with a vector space structure.

If $(\mU,\varphi)$ is a chart around $X(0)$, we can express $Xf$ using only well know objects by defining the function $\tilde f =f\circ\varphi$ and $\tX=\varphi^{-1}\circ X$
\[
  Xf=\Dsdd{ (\tilde f \circ\tX)(t) }{t}{0}=\left.\dsd{\tilde f }{x^{\alpha}}\right|_{x=\tX(0)}\left.\frac{d\tX^{\alpha}}{dt}\right|_{t=0}.
\]
In this sense, we write
\begin{equation}
  X=\frac{d\tX^{\alpha}}{dt} \dsd{}{x^{\alpha}}
\end{equation}
and we say that $\{\partial_1,\ldots,\partial_n\}$ is a basis of $T_xM$. As far as notations are concerned, from now a tangent vector is written as $X=X^{\alpha}\partial_{\alpha}$ where $X^{\alpha}$ is related to the path $\dpt{X}{\eR}{M}$ by $X^{\alpha}=d\tX^{\alpha}/dt$. We will no more mention the chart $\varphi$ and write
\[
  Xf=\Dsdd{f(X(t))}{t}{0}.
\]
Correctness of this short notation is because the equivalence relation is independent of the choice of chart. When we speak about a tangent vector to a given path $X(t)$ without specification, we think about $X'(0)$.

All this construction gives back the notion of tangent vector when $M\subset \eR^m$. In order to see it, think to a surface in $\eR^3$. A tangent vector is precisely given by a derivative of a path: if $\dpt{c}{\eR}{\eR^n}$ is a path in the surface, a tangent vector to this curve is given by
\[
   \lim_{t\to 0}\frac{c(t_0)-c(t_0+t)}{t}
\]
which is a well know limit of a difference in $\eR^n$.

\label{pg:vecto_vecto}Let us precise how does a tangent vector acts on maps others than $\eR$-valued functions. If $V$ is a vector space and $\dpt{f}{M}{V}$, we define
\[
   Xf=(Xf^i)e_i
\]
where $\{e_i\}$ is a basis of $V$ and the functions $\dpt{f^i}{M}{\eR}$, the decomposition of $f$ with respect to this basis. If we consider a map $\dpt{\varphi}{M}{N}$ between two manifolds, the natural definition is $Xf:=dfX$. More precisely, if we consider local coordinates $x^{\alpha}$ and a function $\dpt{f}{M}{\eR}$,
\begin{equation}\label{eq:dvp_phi}
   (d\varphi X)f=\Dsdd{  (f\circ\varphi\circ X)(t) }{t}{0}=\dsd{f}{x^{\alpha}}\dsd{\varphi^{\alpha}}{x\hbeta}\frac{dX\hbeta}{dt}.
\end{equation}
Now we are in a notational trouble: when we write $X=X^{\alpha}\partial_{\alpha}$, the ``$X^{\alpha}$``{} is the derivative of the ``$X^{\alpha}$``{} which appears in the path $X(t)=(X^1(t),\ldots,X^n(t))$ which gives $X$ by $X=X'(0)$. So equation \eqref{eq:dvp_phi} gives
\begin{equation}
   X(\varphi):=d\varphi X=X\hbeta(\partial_{\beta}\varphi^{\alpha})\partial_{\alpha}.
\end{equation}

\subsection{Differential of a map}
%------------------------------------------

Let $\dpt{f}{M_1}{M_2}$ be a differentiable map, $x\in M_1$ and $X\in T_xM_1$, i.e. $\dpt{X}{\eR}{M_1}$ with $X(0)=x$ and $X'(0)=X$. We can consider the path $Y=f\circ X$ in $M_2$. The tangent vector to this path is written $df_x X$.

\begin{proposition}
If $\dpt{f}{M_1}{M_2}$ is a differentiable map between two differentiable manifolds, the map
		\begin{equation}
		\begin{aligned}
			df_x \colon T_xM_1 &\to T_{f(x)}M_2\
			X'(0)&\mapsto (f\circ X)'(0)
		\end{aligned}
	\end{equation}
is linear.
\end{proposition}

\begin{proof}
We consider local coordinates $\dpt{x}{\eR^n}{M_1}$ and $\dpt{y}{\eR^m}{M_2}$. The maps $\dpt{f}{M_1}{M_2}$ and $\dpt{y^{-1}\circ f\circ x}{\eR^n}{\eR^m}$ will sometimes be denoted by the same symbol $f$. We have $(x^{-1}\circ X)(t)=(x_1(t),\ldots,x_n(t))$ and $(y^{-1}\circ Y)(t)=\big( y_1(x_1(t),\ldots,x_n(t),\ldots, y_m(x_1(t),\ldots,x_n(t)  \big)$, so that
\[
  Y'(0)=\left(   \sum_{i=1}^n \dsd{y_1}{x_i}x_i'(0),\ldots,\sum_{i=1}^n \dsd{y_m}{x_i}x_i'(0)   \right)\in\eR^m
\]
which can be written in a more matricial way under the form
\[
   Y'(0)=\left( \dsd{y_i}{x_j}x'_j(0) \right).
\]
So in the parametrisations $x$ and $y$, the map $df_x$ is given by the matrix $\partial y^i/\partial x_j$ which is well defined from the only given of $f$.
\end{proof}


Let $\dpt{x}{\mU}{M}$ and $\dpt{y}{\mV}{M}$ be two charts systems around $p\in M$. Consider the path $c(t)=x(0,\ldots,t,\ldots 0)$ where the $t$ is at the position $k$. Then, with respect to these coordinates,
\[
  c'(0)f=\Dsdd{ f(c(t))  }{t}{0}=\dsd{f}{x^i}\frac{dc^i}{dt}=\dsd{f}{x^k},
\]
so $c'(0)=\partial/\partial x^k$. Here, implicitly, we wrote $c^i=(x^i)^{-1}\circ c$ where $(x^i)^{-1}$ is the $i$th component of $x^{-1}$ seen as element of $\eR^n$. We can make the same computation with the system $y$. With these abuse of notation,
\begin{equation}
   \dsd{}{x^i}=\sum_j\dsd{y^j}{x^i}\dsd{}{y^j}
\end{equation}
as it can be seen by applying it on any function $\dpt{f}{M}{\eR}$. More precisely if $\dpt{x}{\mU}{M}$ and $\dpt{y}{\mU}{M}$ are two charts (let $\mU$ be the intersection of the domains of $x$ and $y$), let $\dpt{f}{M}{\eR}$ and $\ovf=f\circ x$, $\tilde f =f\circ y$. The action of the vector $\partial_{x^i}$ of the function $f$ is given by
\[
  \partial_{x^i}f=\dsd{\ovf}{x^i}
\]
where the right hand side is a real number that can be computed with usual analysis on $\eR^n$. This real \emph{defines} the left hand side. Now, $\ovf=\tilde f \circ y^{-1}\circ x$, so that
\[
   \dsd{\ovf}{x^i}=\dsd{ (\tilde f \circ y^{-1}\circ x) }{x^i}=\dsd{\tilde f }{y^j}\dsd{y^j}{x^i}
\]
where $\dsd{\tilde f }{y^j}$ is precisely what we write now by $\partial_{y^j}f$ and $\dsd{y^j}{x^i}$ must be understood as the derivative with respect to $x^i$ of the function $\dpt{(y^{-1}\circ x)}{\eR^n}{\eR^n}$.

Let $\dpt{f}{M}{N}$ and $\dpt{g}{N}{\eR}$; the definitions gives
\[
  (df_xX)g=\Dsdd{(g\circ f)(X(t))}{t}{0}
          =\dsd{g}{y^i}\dsd{f^i}{x^{\alpha}}\frac{dX^{\alpha}}{dt}.
\]
This shows that $\dsd{f^i}{x^{\alpha}}\frac{dX^{\alpha}}{dt}$ is $(df_xX)^i$.  But $dX^{\alpha}/dt$ is what we should call $X^{\alpha}$ in the decomposition $X=X^{\alpha}\partial_{\alpha}$ then the matrix of $df$ is given by $\dsd{f^i}{x^{\alpha}}$. So we find back the old notion of differential.

\begin{remark}
If $X\in T_xM$ and $f$ is a \emph{vector valued} function on $M$, then one can define $Xf$ by exactly the same expression. In this case,
\[
  df X=\Dsdd{f(v(t))}{t}{0}=Xf.
\]
\end{remark}

A map $\dpt{f}{M_1}{M_2}$ is an \defe{immersion}{immersion} at $p\in M_1$ if $\dpt{df_p}{T_pM_1}{T_{f(p)}M_2}$ is injective. It is a \defe{submersion}{submersion} if $df_p$ is surjective.

\subsection{Tangent and cotangent bundle}
%--------------------------------------

If $M$ is a $n$ dimensional manifold, as set the tangent bundle\index{tangent!space} is the \emph{disjoint} union of tangent spaces
\[
  TM=\bigcup_{x\in M}T_xM.
\]

\begin{theorem}
	The tangent bundle admits a $2n$ dimensional manifold structure for which the projection
	\begin{equation}
		\begin{aligned}
			\pi \colon TM &\to M\
			T_pM&\mapsto p
		\end{aligned}
	\end{equation}
	is a submersion.
\end{theorem}

The structure is easy to guess. If $\dpt{\varphi_{\alpha}}{\mU_{\alpha}}{M}$ is a coordinate system on $M$ (with $\mU_{\alpha}\subset\eR^n$), we define $\dpt{\psi_{\alpha}}{\mU_{\alpha}\times \eR^n}{TM}$ by
\[
  \psi( \underbrace{x_1,\ldots x_n}_{\in\mU_{\alpha}},\underbrace{a_1,\ldots a_n}_{\in\eR^n}  )
          =\sum_i a_i\left.\dsd{}{x_i}\right|_{\varphi(x_1,\ldots,x_n)}.
\]
The map $\psi_{\beta}^{-1}\circ\psi_{\beta}$ is differentiable because
\[
(\psi_{\beta}^{-1}\circ\psi_{\beta})(x,a)=( y(x),\sum_i a_i\left.\dsd{y_j}{x_i}\right|_{y(x)}  )
\]
which is a composition of differentiable maps. The set $TM$ endowed with this structure is called the \defe{tangent bundle}{tangent!bundle}.

%---------------------------------------------------------------------------------------------------------------------------
\subsection{Vector space structure on the tangent space}
%---------------------------------------------------------------------------------------------------------------------------

If \( X,Y\in T_pM\) are tangent vectors, one can define \( X+Y\) and \( \lambda X\) for every \( \lambda\in\eR\). The second one is easy:
\begin{equation}
    \lambda X=\Dsdd{ X(\lambda t) }{t}{0}.
\end{equation}
In order to define the sum of two vectors one has to consider a neighbourhood \( \mU\) of \( p\) in \( M\) and a chart \( \varphi\colon \mU\to \mO\) where \( \mO\) is an open set in \( \eR^n\). Then one consider a basis \( \{ e_i \}_{1\leq i\leq n}\) of \( \eR^n\) at the point \( \varphi(p)\). With these choices we define the ``basis'' path
\begin{equation}
    \gamma_i(t)=\varphi^{-1}(te_i)
\end{equation}
and we write
\begin{equation}
    \partial_i=\frac{ \partial  }{ \partial x_i }=\Dsdd{ \varphi^{-1}(te_i) }{t}{0}.
\end{equation}
The vectors \( \partial_i\) form a basis of \( T_pM\) in the sense of the following lemma.

\begin{lemma}       \label{LEMooXDESooHXzIJU}
    The action of a vector \( X\in T_pM\) on a function \( f\colon M\to \eR\) can be decomposed into
    \begin{equation}
        Xf=\sum_{i=1}^n X_i(\partial_if)
    \end{equation}
    with \( X_i\in\eR\)
\end{lemma}

\begin{proof}
    Let \( \varphi\colon M\to \eR^n\) be a chart of a neighbourhood of \( p\) with \( \varphi(p)=0\). We determine the value of \( X_i\) using the function
    \begin{equation}
        f_i(x)=\varphi(x)_i,
    \end{equation}
    that is the \( i\)th component of the point \( \varphi(x)\in\eR^n\). Then if we write \( \varphi\big( X(t) \big)=\sum_j a_j(t)e_j\) we have
    \begin{subequations}
        \begin{align}
            X(f_i)=\Dsdd{ f_i\big( X(t) \big) }{t}{0}=\Dsdd{ \big[ \sum_ja_j(t)e_j \big]_i }{t}{0}=\Dsdd{ ai_(t) }{t}{0}=a_i'(0).
        \end{align}
    \end{subequations}
    Notice that \( a_i(0)=0\) since \( X(0)=p\) and \( \varphi(p)=0\). The combination \( f\circ\varphi^{-1}\) is an usual function from \( \eR^n\) to \( \eR\), so that we can use the chain rule on it. The following computation thus make sense:
    \begin{subequations}
        \begin{align}
            Xf&=\Dsdd{ f\big( X(t) \big) }{t}{0}\\
            &=\Dsdd{ f\Big( \varphi^{-1}\varphi\big( X(t) \big) \Big) }{t}{0}\\
            &=\Dsdd{ (f\circ\varphi^{-1})\big( \sum_ja_j(t)e_j \big) }{t}{0}\\
            &=\sum_k \frac{ \partial (f\circ\varphi^{-1}) }{ \partial x_k }\big( \underbrace{\sum_ja_j(0)e_j}_{=\varphi(p)=0} \big)\underbrace{\frac{ d\big[ \sum_ja_j(t)e_j \big]_k  }{ dt }}_{=a'_k(0)}\\
            &=\sum_k a'_k(0)\frac{ \partial (f\circ\varphi^{-1}) }{ \partial x_k }(0).
        \end{align}
    \end{subequations}
    Now using the definition of a derivative of a function \( \eR^n\to \eR\) and of the ``basis'' tangent vector \( \partial_k\),
    \begin{subequations}
        \begin{align}
            \frac{ \partial (f\circ\varphi^{-1}) }{ \partial x_k }(0)&=\Dsdd{ (f\circ\varphi^{-1})(te_k) }{t}{0}\\
            &=\partial_k f
        \end{align}
    \end{subequations}
   At the end of the day we have
   \begin{equation}
       Xf=\sum_k a'_k(0)\partial_kf.
   \end{equation}
\end{proof}

This lemma allows us to define the sum in \( T_pM\) as\quext{This is not really true because we still have to prove that for every choice of \( X_i\), there exists a path \( \alpha\) such that \( \alpha'(0)=\sum_iX_i\partial_i\).}      %TODOooNHVGooLYbUkg
\begin{equation}
    \left( \sum_kX_k\partial_k \right)+\left( \sum_kY_k\partial_k \right)=\sum_k (X_k+Y_k)\partial_k
\end{equation}
when \( X_k\) and \( Y_k\) are reals.

The tangent space \( T_pM\) is thus a vector space.

\begin{proposition}     \label{PROPooOHLQooCNetuD}
    Let \( M\) be a differentiable manifold and \( x\in M\). The set of tangent vectors of \( M\) at \( x\) is a vector space denoted by \( T_xM\).
\end{proposition}
% when it is proved, remove the quext TODOooNHVGooLYbUkg.

\begin{proposition}     \label{PROPooWXNDooPeORjA}
    Let \( p\in M\) and \( \varphi\colon U\to M\) be a chart around \( p\) with \( U\) open in \( \eR^m\). If \( a\in U\) and \( p=\varphi(a)\), then  the set \( \{  d\varphi_a(e_i) \}\) is a basis of \( T_pM\).
\end{proposition}

\begin{lemma}       \label{LEMooVCSJooEuDZFz}
    Let \( M\) and \( N\) be smooth manifolds of dimension \( m\) and \( n\) with charts \( \varphi\colon U\to M\) and \( \psi\colon V\to N\) around \( p\in M\) and \( f(p)\in N\). We consider basis \( \{ e_i \}_{i=1,\ldots, m}\) of \( \eR^m\) and \( \{ e'_{\alpha} \}_{\alpha=1,\ldots, n}\) of \( \eR^n\).

    The matrix of \( df_p\colon T_pM\to T_{f(p)}N\) in the basis \( \{ d\varphi_{\varphi^{-1}(p)}(e_i) \}\) and \( \{ d\psi_{\psi^{-1}(f(p))}(e'_{\alpha}) \}\) is the same as the matrix of \( d_{\varphi^{-1}(p)}(\psi^{-1}\circ f\circ\varphi)\) as map from \( \eR^m\) to \( \eR^n\).
\end{lemma}

\begin{proof}
    Let subdivise.
    \begin{subproof}
        \item[Notations]
            As a preliminary remark, the fact that the proposed sets are basis is the proposition \ref{PROPooWXNDooPeORjA}. For the notations, we write
            \begin{subequations}
                \begin{align}
                    \frac{ \partial  }{ \partial x_i }&=d\varphi_{\varphi^{-1}(p)}(e_i),\\
                    \frac{ \partial  }{ \partial y_{\alpha} }&=d\psi_{\psi^{-1}\big( f(p) \big)}(e'_{\alpha}).
                \end{align}
            \end{subequations}
        \item[Component]
            Let \( v\in T_{f(p)}N\). We prove that
            \begin{equation}        \label{EQooISXNooJOzUmS}
                v_{\alpha}=\Big( (d\psi^{-1})_{f(p)}v \Big)_{\alpha}
            \end{equation}
            where in the left-hand side we are speaking of component with respect to the basis \( \{ \partial_{y_{\alpha}} \}\) while in the right-hand side, the ones with respect to the basis \( \{ e'_{\alpha} \}\).

            First we decompose \( v\):
            \begin{equation}
                v=\sum_{\alpha}v_{\alpha}\frac{ \partial  }{ \partial y_{\alpha} }=\sum_{\alpha}v_{\alpha}d\psi_{\psi^{-1}\big( f(p) \big)}e'_{\alpha},
            \end{equation}
            then we apply \( d\psi^{-1}_{f(p)}\) to that equation:
            \begin{equation}
                d\psi^{-1}_{f(p)}v=\sum_{\alpha}v_{\alpha}e'_{\alpha}.
            \end{equation}
            Taking the \( \alpha\)\th\ component on both side we have our result \eqref{EQooISXNooJOzUmS}.
        \item[Matrix]
            The matrix of a linear map is defined by the proposition \ref{PROPooGXDBooHfKRrv}. In our case,
            \begin{equation}
                    (df_p)_{\alpha i}=\left( df_p\big( \frac{ \partial  }{ \partial x_i } \big) \right)_{\alpha} =\Big( df_p\circ d\varphi_{\varphi^{-1}(p)}e_i \Big)_{\alpha}.
            \end{equation}
            Using the formula \eqref{EQooISXNooJOzUmS},
            \begin{subequations}
                \begin{align}
                    (df_p)_{\alpha i}&=\Big( df_p\circ d\varphi_{\varphi^{-1}(p)}e_i \Big)_{\alpha}\\
                    &=\big( (d\psi^{-1})_{f(p)}\circ df_p\circ d\varphi_{\varphi^{-1}(p)}e_i \big)_{\alpha}\\
                    &=\big( (d\psi^{-1})_{f(p)}\circ df_p\circ d\varphi_{\varphi^{-1}(p)} \big)_{\alpha i}\\
                \end{align}
            \end{subequations}
    \end{subproof}
\end{proof}

\subsection{Commutator of vector fields}

\begin{lemma}       \label{LEMooPSWEooVKLWMQ}
    If \( X\) is a smooth vector field on the manifold \( M\) and if \( f\) is a smooth function, then the formula
    \begin{equation}
        (Xf)(x)=X_x(f)
    \end{equation}
    defines a smooth function \( Xf\) on \( M\).
\end{lemma}

\begin{propositionDef}      \label{DEFooHOTOooRaPwyo}
    If $X$, $Y\in\cvec(M)$, one defines the \defe{commutator}{commutator of vector fields} $[X,Y]$ by
    \begin{equation}
      [X,Y]_xf=X_x(Yf)-Y_x(Xf).
    \end{equation}
    where \( Yf\) and \( Xf\) are defined by virtue of lemma \ref{LEMooPSWEooVKLWMQ}.
\end{propositionDef}

If $X=X^i\partial_i$ and $Y=Y^j\partial_j$, then
$XY(f)=X^i\partial_i(Y^j\partial_jf)
     =X^i\partial_i Y^j\partial_j f+X^iY^j\partial^2_{ij}f$.
From symmetry $\partial^2_{ij}f=\partial^2_{ji}f$, the difference $XYf-YXf$ is only $X^i\partial_iY^j-Y^i\partial_iX^j$, so that
\begin{equation}
  [X,Y]^i=XY^i-YX^i
\end{equation}
where $X^i$ and $Y^i$ are seen as functions from $M$ to $\eR$.

\begin{proposition}[\cite{BIBooWSHFooKoDjAs}]       \label{PROPooSWQSooSEfTuX}
    The set of all the smooth vector fields on a manifold is a Lie algebra.
\end{proposition}

\subsection{Some Leibnitz formulas}

\begin{lemma}[\cite{kobayashi}]
If $M$ and $N$ are two manifolds, we have a canonical isomorphism
\[
     T_{(p,q)}(M\times N)\simeq T_pM+T_qN.
\]
\label{lemLeibnitz}
\end{lemma}

\begin{proof}
A $Z\in T_{(p,q)}(M\times N)$ is the tangent vector to a curve $(x(t),y(y))$ in $M\times N$. We can consider $X\in T_pM$ given by $X=x'(0)$ and $Y\in T_qN$ given by $Y=y'(0)$. The isomorphism is the identification $(X,Y)\simeq Z$. Indeed, let us define $\oX\in T_{(p,q)}(M\times N)$, the tangent vector to the curve $(x(t),q)$, and $\oY\in T_{(p,q)}(M\times N)$, the tangent vector to the curve $(p,y(t))$. Then $Z=\oX+\oY$ because for any $\dpt{f}{M\times N}{\eR}$,
\begin{equation}
 Zf=\dsdd{f(x(t),y(t))}{t}{0}
   =\dsdd{f(x(t),y(0))}{t}{0}+\dsdd{f(x(0),y(t))}{t}{0}
   =\oX f+\oY f.
\end{equation}
\end{proof}

\begin{proposition}[Leibnitz formula] \label{Leibnitz}
Let us consider $M,N,V$, three manifold; a map $\dpt{\varphi}{M\times N}{V}$ and a vector $Z\in T_{(p,q)}(M\times N)$ which corresponds (lemma~\ref{lemLeibnitz}) to $(X,Y)\in T_pM+T_qN$.

If we define $\dpt{\varphi_1}{M}{V}$ and  $\dpt{\varphi_2}{N}{V}$ by $\varphi_1(p')=\varphi(p',q)$ and $\varphi_2(q')=\varphi(p,q')$, we have the \defe{Leibnitz formula}{Leibnitz formula}:
\begin{equation}
    d\varphi(Z)=d\varphi_1(X)+d\varphi_2(Y).
\end{equation}
\end{proposition}
\begin{proof}
 Since $Z=\oX+\oY$, we just have to remark that
\[
                  d\varphi(\oX)=\dsdd{\varphi(x(t),q)}{t}{0}=d\varphi_1(X),
\]
so $d\varphi(Z)=d\varphi(\oX+\oY)=d\varphi_1(X)+d\varphi_2(Y)$.
\end{proof}
One of the most important application of the Leibnitz rule is the corollary~\ref{cor_PrincLeib} on principal bundles.

\subsection{Cotangent bundle}

A form on a vector space $V$ is a linear map $\dpt{\alpha}{V}{\eR}$. The set of all forms on $V$ is denoted by $V^*$ and is called the \defe{dual space}{dual!of a vector space} of $V$. On each point of a manifold, one can consider the tangent bundle which is a vector space. Then one can consider, for each $x\in M$ the dual space $T^*_xM:=(T_xM)^*$ which is called the \defe{cotangent bundle}{cotangent bundle}. A $1$-\defe{differential form}{differential!form} on $M$ is a smooth map $\dpt{\omega}{M}{T^*M}$ such that $\omega_x:=\omega(x)\in T^*_xM$. So, for each $x\in M$, we have a $1$-form $\dpt{\omega_x}{T_xM}{\eR}$.

Here, the smoothness is the fact that for any smooth vector field $X\in\cvec(M)$, the map $x\to\omega_x(X_x)$ is smooth as function on $M$. One often considers vector-valued forms. This is exactly the same, but $\omega_xX_x$ belongs to a certain vector space instead of $\eR$. The set of $V$-valued $1$-forms on $M$ is denoted by $\Omega(M,V)$ \nomenclature{$\Omega(M,V)$}{$V$ valued $1$-forms} and simply $\Omega(M)$ if $V=\eR$
The cotangent space $T^*_pM$ of $M$ at $p$ is the dual space of $T_pM$, i.e. the vector space of all the (real valued) linear\footnote{When we say \emph{a form}, we will always mean \emph{a linear form}.} $1$-forms on $T_pM$. In the coordinate system $\dpt{x}{\mU}{M}$, we naturally use, on $T^*_pM$, the dual basis of the basis $\{\partial/\partial_{x^i},\ldots\partial/\partial_{x^i}\}$ of $T_pM$. This dual basis is denoted by $\{dx_1,\ldots,dx_n\}$, the definition being as usual:
\begin{equation}\label{eq:dx_v}
  dx_i(\partial^j)=\delta^j_i.
\end{equation}
The notation comes from the fact that equation \eqref{eq:dx_v} describes the action of the differential of the projection $\dpt{x_i}{\mU}{\eR}$ on the vector $\partial^j$.

If $(\mU_{\alpha},\varphi_{\alpha})$ is a chart of $M$, then the maps
		\begin{equation}
		\begin{aligned}
			\phi_{\alpha} \colon \mU_{\alpha}\times\eR^n &\to T^*M\
			(x,a)&\mapsto a^idx_i|_x
		\end{aligned}
	\end{equation}
give to $T^*M$ a $2n$ dimensional manifold structure such that the canonical projection $\dpt{\pi}{T^*M}{M}$ is an immersion.

When $V$ is a finite-dimensional vector space, we denote by $V^*$ its dual\footnote{The vector space of all the linear map $V\to \eR$.} and we often use the identifications $V\simeq V^*\simeq T_vV\simeq T_wV\simeq T^*_vV$ where $v$ and $w$ are any elements of $V$. Note however that there are no \emph{canonical} isomorphism between these spaces, unless we consider some basis.

%--------------------------------------------------------------------------------------------------------------------------- 
\subsection{Immersion, embedding}
%---------------------------------------------------------------------------------------------------------------------------

\begin{definition}[\cite{BIBooECJTooEfmLsr}]    \label{DEFooZEWNooMVOzWI}
    A smooth map \( f\colon M\to N\) between the manifolds \( M\) and \( N\) is an \defe{immersion}{immersion} if its differential \( df_p\colon T_pM\to T_{f(p)N}\) is injective for every \( p\in M\).
\end{definition}

\begin{definition}[Topological embedding\cite{BIBooTUOOooJZFtGe}]
    Let \( X\), \( Y\) be topological spaces. A map \( f\colon X\to Y\) is a \defe{topological embedding}{topological embedding} if
    \begin{enumerate}
        \item
            \( f\) is continuous
        \item
            \( f\) is injective
        \item
            \( f\colon X\to f(X)\) is an homeomorphism when \( f(X)\) is equipped with the induced topology from \( Y\).
    \end{enumerate}
\end{definition}

\begin{definition}[Embedding\cite{BIBooTUOOooJZFtGe}]
    Let \( M\) and \( N\) be smooth manifolds. A smooth function \( f\colon M\to N\) is an \defe{embedding}{embedding} if
    \begin{enumerate}
        \item
            \( f\) is an immersion,
        \item
            \( f\) is a topological embedding.
    \end{enumerate}
\end{definition}

\begin{definition}[Tensor algebra]      \label{DEFooHPQXooETvEyn}
    Let \( V\) be a vector space over \( \eC\). The \defe{tensor algebra}{tensor algebra} of \( V\) is the vector space
    \begin{equation}
        T(V)=\bigoplus_{n\geq 0}\left(\otimes^nV\right)=\eC\oplus V\oplus(V\otimes V)\oplus\ldots
    \end{equation}
\end{definition}


%+++++++++++++++++++++++++++++++++++++++++++++++++++++++++++++++++++++++++++++++++++++++++++++++++++++++++++++++++++++++++++ 
\section{Rank theorem}
%+++++++++++++++++++++++++++++++++++++++++++++++++++++++++++++++++++++++++++++++++++++++++++++++++++++++++++++++++++++++++++

We proof a generalization of the rank theorem \ref{ThoGkkffA}.

\begin{definition}
    Let \( M\) and \( N\) be smooth manifolds of dimension \( m\) and \( n\). Let a smooth map \( f\colon M\to N\). The \defe{rank}{rank} of \( f\) at \( p\in M\) is the rank of the linear map \( df_p\colon T_pM\to T_{f(p)N}\).
\end{definition}

\begin{lemma}
    Let a smooth map \( f\colon M\to N\) and \( p\in M\). Let \( \varphi\colon U\to M\) and \( \psi\colon V\to N\) be chats around \( p\) and \( f(p)\). Then
    \begin{equation}
        \rang(df_p)=\rang\big( f_{\varphi^{-1}(p)}(\psi^{-1}\circ f\circ \varphi) \big)
    \end{equation}
    where the rank on the right han side is the usual rank of a map \( \eR^m\to \eR^n\).
\end{lemma}

\begin{proof}
    By proposition \ref{PROPooEGNBooIffJXc} we can compute the rank of a linear map in whatever base. When a basis is chosen in \( \eR^m\) and \( \eR^n\) we know from lemma \ref{LEMooVCSJooEuDZFz} that the matrix of \( df_p\) is the same as the one of \(  f_{\varphi^{-1}(p)}(\psi^{-1}\circ f\circ \varphi) \). Since these two linear maps have the same matrix, they have the same rank.
\end{proof}

\begin{theorem}[Rank theorem\cite{BIBooVYIRooZyqygg}]       \label{THOooSWKVooTJQsXc}
    Let \( M\) and \( N\) be smooth manifolds of dimension \( m\) and \( n\). Let \( f\colon M\to N\) be a smooth map. Let \( p\in M\). We suppose that the rank of \( f\) is equal to \( k\) at every point \( x\) in a neighbourhood of \( p\). 

    There exists charts \( \varphi\colon U\to M\) around \( p\in M\) and \( \psi\colon V\to N\) around \( f(p)\in M\) such that 
    \begin{enumerate}
        \item
            \( \varphi(0)=p\),
        \item
            \( \psi(0)=f(p)\)
        \item
            the function \( f\) is more or less trivialized in the sense that
            \begin{equation}
                (\psi^{-1}\circ f\circ\varphi)(x_1,\ldots, x_m)=(x_1,\ldots, x_k,0,\ldots, 0)
            \end{equation}
            for every \( (x_1,\ldots, x_m)\in U\).
    \end{enumerate}
\end{theorem}

\begin{proof}
    We prove in two parts. Fist we consider the case in which \( M\) and \( N\) are open sets of \( \eR^m\) and \( \eR^n\). Then we will generalize to any smooth manifolds.
    \begin{subproof}
        \item[The case of \( \eR^m\) and \( \eR^n\)]
            Let \( W\) be open in \( \eR^m\), \( W'\) be open in \( \eR^n\). We consider a smooth map \( f\colon W\to W'\) such that \( f(0)=0\) and \( \rang(f)=k\) on \( W\).

            By hypothesis, the rank of \( df_0\) is \( k\), so that is one chooses good bases on \( \eR^m\) and \( \eR^n\) we can suppose that the matrix of \( df_0\) has a upper-left square \( k\times k\) with non-zero determinant. We write \( A\) that square matrix:
            \begin{equation}
                A_{ij}=\frac{ \partial f_i }{ \partial x_j }(0)
            \end{equation}
            with \( i,j\leq k\).

            \begin{subproof}
                \item[On the \( \eR^m\) side]

                    We consider the map
                    \begin{equation}
                        \begin{aligned}
                            \varphi\colon W\subset \eR^m&\to \eR^m \\
                            (x_1,\ldots, x_m)&\mapsto \big( f_1(x_1,\ldots, x_m),\ldots, f_k(x_1,\ldots, x_m),x_{k+1},\ldots, x_m \big). 
                        \end{aligned}
                    \end{equation}
                    We have \( \varphi(0)=0\) because \( f_i(0)=f(0)_i=0\). The matrix of the differential is
                    \begin{equation}
                        d\varphi_0=\begin{pmatrix}
                            A    &   *    \\ 
                            0    &   \mtu_{n-k}    
                        \end{pmatrix}
                    \end{equation}
                    where \( A\) is \( k\times k\) and \( *\) is some \( k\times (n-k)\) matrix. Thus we have \( \det(d\varphi_0)=\det(A)\neq 0\). From the inverse function theorem \ref{ThoXWpzqCn}, the map \( \varphi\) is a local diffeomorphism, more precisely there exists an open set \( W_1\subset W\subset \eR^m\) such that the restriction
                    \begin{equation}
                        \varphi\colon W_1\to W_1
                    \end{equation}
                    is a diffeomorphism. From now on we only consider \( \varphi\) as being that restriction.

                    The vector \( (y_1,\ldots, y_m)\) such that \( \varphi^{-1}(x_1,\ldots, x_m)=(y_1,\ldots, y_m)\) has the property that
                    \begin{equation}
                        \varphi(y_1,\ldots, y_m)=(x_1,\ldots, x_m),
                    \end{equation}
                    which means that\footnote{At this point, it is really important that \( f\) takes its values in \( \eR^n\), not in a general manifold: if \( (y)\) was in a manifold, the expression \( f_i(y)\) would not make sense.}
                    \begin{equation}
                        f_i(y_1,\ldots, y_m)=x_i
                    \end{equation}
                    when \( i=1,\ldots, k\) and
                    \begin{equation}
                        y_l=x_l
                    \end{equation}
                    when \( l=k+1,\ldots, m\).

                \item[On the middle side]

                    Thus we have
                    \begin{subequations}
                        \begin{align}       \label{EQooAQJGooLqlnXJ}
                            (f\circ \varphi^{-1})(x_1,\ldots, x_m)&=f(y_1,\ldots, y_m)\\
                            &=\big( x_1,\ldots, x_k,f_{k+1}(y_1,\ldots, y_m),\ldots, f_n(y_1,\ldots, y_m) \big)\\
                            &=\big( x_1,\ldots, x_k,\tilde f_{k+1}(x),\ldots, \tilde f_n(x)\big)
                        \end{align}
                    \end{subequations}
                    where \( \tilde f_i=f_i\circ \varphi^{-1}\colon W_1\to \eR\) are some smooth functions.

                    For every \( x\in W_1\) we have
                    \begin{equation}        \label{EQooEDJIooLyPslk}
                        f(f\circ \varphi^{-1})_x=\begin{pmatrix}
                            \mtu_{k\times k}    &   0    \\ 
                            *    &   d\tilde f_x    
                        \end{pmatrix}
                    \end{equation}
                    where \( d\tilde f_x\) is the matrix whose elements are \( \left( \frac{ \partial \tilde f_i }{ \partial x_s } \right)\) with \( i=k+1,\ldots, n\) and \( s=k+1,\ldots, m\). This is not a square matrix by the way. We have, by proposition \ref{PROPooBWZFooTxKavX},
                    \begin{equation}
                        d(f\circ\varphi^{-1})_x=df_{\varphi^{-1}(x)}\circ(d\varphi^{-1})_x
                    \end{equation}
                    while \( (d\varphi^{-1})_x\) is invertible. Thus
                    \begin{equation}
                        \rang\big( d(f\circ\varphi^{-1})_x \big)=\rang\big( df_{\varphi^{-1}(x)} \big)=k.
                    \end{equation}
                    So the rank of \( f\circ\varphi^{-1}\) is \( k\) all over \( W_1\). But the image of \( d(f\circ\varphi^{-1})_x\) is spanned by the columns of its differential given by \eqref{EQooEDJIooLyPslk}. The \( k \) columns spanned by the identity matrix are obviously linearly independent; these are thus a basis of the image. Since the vectors in the ``\( d\tilde f_x\)'' part are linearly independent of these \( k\) vectors, they must be vanishing:
                    \begin{equation}
                        \frac{ \partial \tilde f_i }{ \partial x_s }(x)=0
                    \end{equation}
                    for every \( x\in W_1\), \( i=k+1,\ldots, m\) and \( s=k+1,\ldots, n\).

                \item[On the \( \eR^n\) side]

                    We do not know if \( n\geq m\) or \( m\geq n\). If \( n\geq n\), we choose \( V_1\) such that the projection of \( V_1\) on its \( m\) first components is included in \( W_1\). If \( n<m\) we choose \( V_1\) such that the projection of \( W_1\) on its \( n\) first components is included in \( V_1\)\quext{This precision about the choice of \( V_1\) is not done in \cite{BIBooVYIRooZyqygg} and seems strange to me. Am I correct ? By the way, there could be a misprint in the definition of \( T\) in \cite{BIBooVYIRooZyqygg}: \( y\) must have \( n\) components, not \( m\).}.

                    With that choice of \( V_1\) in mind we can remember the functions \( \tilde f_i\colon W_1\to \eR\). If \( y\in V_1\), we define \( \tilde f(y)\) as \( \tilde f(x)\) with \( x\in W_1\) created from \( y\) either by adding zeroes or by projecting on \( \eR^m\). In both cases, the resulting \( y\) belongs to \( V_1\).

                    So now we consider the map
                    \begin{equation}        \label{EQooKEZOooSOTBlo}
                        \begin{aligned}
                            T\colon V_1&\to \eR^n \\
                            (y_1,\ldots, y_n)&\mapsto \big( y_1,\ldots, y_k,y_{k+1}+\tilde f_{k+1}(y),\ldots, y_n+\tilde f_n(y) \big). 
                        \end{aligned}
                    \end{equation}
                    If \( y\in V_1\), the differential is the matrix given by
                    \begin{equation}
                        (dT_y)_{ij}=\frac{ \partial T_i }{ \partial y_j }(y)
                    \end{equation}
                    where
                    \begin{itemize}
                        \item 
                    The upper-left \( k\times k\) corner is \( \mtu_{k\times k}\).
                \item
                    The upper-right \( k\times (n-k)\) corner (non square in general) is given by elements of the form
                    \begin{equation}
                        \frac{ \partial y_i }{ \partial y_{j} }
                    \end{equation}
                    with \( i\leq k\) and \( j>k\). So this is vanishing.
                \item
                    The lower-left (non square in general) corner is made of
                    \begin{equation}
                        \frac{ \partial (y_i+\tilde f_i(y)) }{ \partial y_j }=\frac{ \partial \tilde f_i(y) }{ \partial y_j }
                    \end{equation}
                    with \( i>k\) and \( j\leq k\). The elements in this pars are some numbers.
                \item
                    The lower-right square \( (n-k)\times (n-k)\) corner is made of
                    \begin{equation}
                        \frac{ \partial (y_i+\tilde f_i(y)) }{ \partial y_j }=\delta_{ij}+\frac{ \partial \tilde f_i }{ \partial y_j }
                    \end{equation}
                    with \( i>k\) and \( j\geq k\). For these elements we have \( \frac{ \partial \tilde f_i(y) }{ \partial y_j }=0\) and then the identity matrix.
                    \end{itemize}
                    With all that,
                    \begin{equation}
                        dT_y=\begin{pmatrix}
                            \mtu_{k\times k}    &   0    \\ 
                            *    &   \mtu_{n-k}    
                        \end{pmatrix}.
                    \end{equation}
                    Moreover \( T(0)=0\) because
                    \begin{equation}
                        \tilde f_i(0)=f_i\big( \varphi^{-1}(0) \big)=f_i(0)=0.
                    \end{equation}
                    We deduce that there exist an open set \( V\subset \eR^n\) included in \( V_1\) such that \( T\colon V\to T(V)\) is a diffeomorphism. We restrict \( V\) in such a way that \( T(V)\subset V_1\).

                \item[The final map]

                    Finally we consider the map
                    \begin{equation}
                        T^{-1}\circ f\circ \varphi^{-1}\colon W_1 \to V.
                    \end{equation}
                    If \( (x_1,\ldots, x_m)\in W_1\) from \eqref{EQooAQJGooLqlnXJ} we have
                    \begin{equation}
                        (f\circ \varphi^{-1})(x_1,\ldots, x_m)=\big( x_1,\ldots, x_k,\tilde f_{k+1}(x),\ldots, \tilde f_n(x) \big).
                    \end{equation}
                    Using the definition \eqref{EQooKEZOooSOTBlo} we see that
                    \begin{equation}
                        T(x_1,\ldots, x_k,0,\ldots, 0)=\big( x_1,\ldots, x_k,\tilde f_{k+1}(x),\ldots, \tilde f_n(x) \big)
                    \end{equation}
                    which proves that
                    \begin{equation}
                        T^{-1}\big( x_1,\ldots, ,x_k,\tilde f_{k+1}(x),\ldots, \tilde f_n(x) \big)=(x_1,\ldots, x_k,\,\ldots, 0).
                    \end{equation}
            \end{subproof}

        \item[The general case]

            Now we consider the manifolds \( M\) and \( N\) with the map \( f\colon M\to N\). Let \( p\in M\) and charts \( \varphi_0\colon U_0\to M\), \( \psi_0\colon V_0\to N\) where \( U_0\) is a neighbourhood of \( 0\) in \( \eR^m\) and \( V_0\) a neighbourhood of \( 0\) in \( \eR^n\). We suppose that \( \varphi_0(0)=p\) and \( \psi_0(0)=f(p)\).

            Now we consider the function \( \tilde f=\psi_0^{-1}\circ f\circ \varphi_0\) from \( U_0\) to \( V_0\) and we are left in the previous case.
    \end{subproof}
\end{proof}

%+++++++++++++++++++++++++++++++++++++++++++++++++++++++++++++++++++++++++++++++++++++++++++++++++++++++++++++++++++++++++++
\section{Exterior calculus}
%+++++++++++++++++++++++++++++++++++++++++++++++++++++++++++++++++++++++++++++++++++++++++++++++++++++++++++++++++++++++++++

%---------------------------------------------------------------------------------------------------------------------------
\subsection{The exterior algebra}
%---------------------------------------------------------------------------------------------------------------------------

\begin{definition}[Exterior product]
    If $V$ is a vector space, we denote by $\Lambda^kV^*$ the space of all the $k$-form on $V$. We define the \defe{exterior product}{product!exterior} $\dpt{\wedge}{\Lambda^kV^*\times\Lambda^lV^*}{\Lambda^{k+l}V^*}$ by
    \begin{equation}
      (\omega^k\wedge\eta^l)(v_1,\ldots,v_{k+l})
      =\us{k!l!}\sum_{\sigma\in S_{k+l}} (-1)^{\sigma}   \omega(v_{\sigma(1)},\ldots,v_{\sigma(k)})\eta(v_{\sigma(k+1)},v_{\sigma(k+1)})
    \end{equation}
\end{definition}
If $\{e_1,\ldots,e_n\}$ is a basis of $V$, the dual basis $\{\sigma^1,\ldots,\sigma^n\}$ of $V^*$ is defined by $\sigma^i(e_j)=\delta^i_j$.

If $I=\{1\leq i_1\leq\ldots i_k\leq n\}$, we write $\sigma^I=\sigma^{i_1}\wedge\ldots\sigma^{i_k}$ any $k$-form can be decomposed as
\[
  \omega=\sum_{I}\omega_I\sigma^I.
\]
The exterior algebra is provided with the \defe{interior product}{interior!product} denoted by $\iota$. It is defined by\label{pg_DefProdExt}
\begin{equation}
\begin{aligned}
 \iota(v_0)\colon\Lambda^kW&\to \Lambda^{k-1}W \\
(\iota(v_0)\omega)(v_1,\ldots,v_{k-1})& =\omega(v_0,v_1,\ldots,v_{k-1}).
\end{aligned}
\end{equation}

\begin{lemma}
    Let \( \sigma\) be an element of the symmetric group\footnote{Definition~\ref{DEFooJNPIooMuzIXd}.} of the set \( \{ a_1,\ldots, a_n \}\) where the \( a_i\) are integers. Then
    \begin{equation}
        (dx_{a_1}\wedge\ldots \wedge dx_{a_n})(e_{\sigma(a_1)},\ldots, e_{\sigma(a_n)})=(-1)^{\sigma}.
    \end{equation}
\end{lemma}

\begin{proof}
    We make it by induction over \( n\). With \( n=1\) the only permutation is the identity; the claim reduces to \( dx_{a_1}(e_{a_1})=1\). Let us try with \( n=2\). Up to renumbering we have
    \begin{equation}
        (dx_1\wedge dx_2)(e_1,e_2)=1
    \end{equation}
    and
    \begin{equation}
        (dx_1\wedge dx_2)(e_2,e_1)=-1.
    \end{equation}
    We pass to the induction. Let \( \sigma\in S_n\). We have
    \begin{subequations}
        \begin{align}
            (dx_{a_1}\wedge dx_{a_n})(&e_{\sigma(a_1)},\ldots, e_{\sigma(a_n)})=dx_{a_1}\wedge (dx_{a_n})(e_{\sigma(a_1)},\ldots, e_{\sigma(a_n)})\\
            &=\sum_{\phi\in S_n}(-1)^{\phi}\frac{1}{ (n-1)! }dx_{a_1}\big( e_{\phi\sigma(a_1)} \big)(dx_{a_2}\wedge\ldots\wedge dx_{a_n})(e_{\phi\sigma(a_2)},\ldots, e_{\phi\sigma(a_n)})\\
            &=\sum_{\phi\in S_n}(-1)^{\phi}\frac{1}{ (n-1)! }\delta_{a_1,\phi\sigma(a_1)}(-1)^{\phi\sigma}\\
            &=\sum_{\phi\in S_n}\delta_{a_1,\phi\sigma(a_1)}(-1)^{\sigma}\frac{1}{ (n-1)! }
        \end{align}
    \end{subequations}
    where we used the fact that the sign of a permutation provides a morphism between \( S_n\) and \( \{ -1,1 \}\) (proposition~\ref{ProphIuJrC}\ref{ITEMooBQKUooFTkvSu}). In the sum over \( S_n\), only the \( \phi\) that make \( \sigma(a_1)\to a_1\) remain; there are \( | S_{n-1} |=(n-1)!\) such elements. Thus the whole evaluates to \( (-1)^{\sigma}\).
\end{proof}

\begin{lemma}[\cite{MonCerveau}]    \label{LEMooICRXooFKPCRd}
    Let \( \tau_i\colon \eR^n\to \eR^{n-1}\) defined by
    \begin{equation}
        \tau_i(v)_k=\begin{cases}
            v_k    &   \text{if } k<i\\
            v_{k+1}    &    \text{if } k\geq i\text{.}
        \end{cases}
    \end{equation}
    Then we have
    \begin{equation}
        (dx_1\wedge\ldots\wedge\widehat{dx_i}\wedge\ldots\wedge dx_n)(v_1,\ldots, \widehat{v_i},\ldots, v_n)=
        \det\Big(  \tau_i(v_1),\ldots, \widehat{\tau_i(v_i)},\ldots, \tau_i(v_n)  \Big)
    \end{equation}
    where the hat denotes a non present term.
\end{lemma}

\begin{proof}
    We extend \( \tau_i\) to the dual : \( \tau_i\colon(\eR^n)^*  \to (\eR^{n-1})^*\) is defined by
    \begin{equation}
        \tau_i(dx_k)=\begin{cases}
            dy_k    &   \text{if } k<i\\
            dy_{k-1}    &    \text{if } k>i
        \end{cases}
    \end{equation}
    (not defined on \( dx_i\)). It is easy to check that, if \( k\neq i\),
    \begin{equation}
        \tau_i(dx_k)\tau_i(v)=dx_k(v).
    \end{equation}
    The value of  $(dx_1\wedge\ldots\wedge\widehat{dx_i}\wedge\ldots\wedge dx_n)(v_1,\ldots, \widehat{v_i},\ldots, v_n)$ is a polynomial in the variables \( dx_k(v_l)\) (with \( k\neq l\)). Since \( dx_k(v_l)=\tau\i(dx_k)\big( \tau_iv_l \big)\), the same polynomial will give the value of
    \begin{equation}
        (\tau_idx_1\wedge\ldots\wedge \widehat{\tau_idx_i}\wedge\ldots\wedge \tau_idx_n  )(\tau_i v_1,\ldots, \widehat{\tau_iv_i},\ldots, \tau_iv_n).
    \end{equation}
    Thus we have
    \begin{subequations}
        \begin{align}
            (dx_1\wedge\ldots\wedge\widehat{dx_i}&\wedge\ldots\wedge dx_n)(v_1,\ldots, \widehat{v_i},\ldots, v_n)\\
            &=(\tau_idx_1\wedge\ldots\wedge \widehat{\tau_idx_i}\wedge\ldots\wedge \tau_idx_n  )(\tau_i v_1,\ldots, \widehat{\tau_iv_i},\ldots, \tau_iv_n)\\
            &=(dy_1\wedge\ldots\wedge dy_{n-1})(\tau_iv_1,\ldots,\widehat{\tau_iv_i},\ldots, \tau_iv_n) \label{SUBEQooQGSKooSgfxJh}\\
            &=\det\big( \tau_iv_1,\ldots, \widehat{\tau_i v_i},\ldots, \tau_iv_n \big)
        \end{align}
    \end{subequations}
    The last equality is because \eqref{SUBEQooQGSKooSgfxJh} is is a \( (n-1)\)-form applied to \( n-1\) vectors in \( \eR^{n-1}\) and so is the determinant.
\end{proof}

%---------------------------------------------------------------------------------------------------------------------------
\subsection{Differential of \texorpdfstring{$k$}{k}-forms}
%---------------------------------------------------------------------------------------------------------------------------

The differential of a $k$-form is defined by the following theorem.

\begin{theorem}
Let $M$ be a differentiable manifold. Then for each $k\in \eN$, there exists an unique map
\[
  \dpt{d}{\Omega^k(M)}{\Omega^{k+1}(M)}
\]
such that

\begin{enumerate}
\item $d$ is linear,
\item for $k=0$, we find back the $\dpt{d}{\Cinf(M)}{\Omega^1(M)}$ previously defined,
\item if $f$ is a function and $\omega^k$ a $k$-form, then
\begin{equation}
d(f\omega^k)=df\wedge\omega^k+fd\omega^k,
\end{equation}


\item $d(\omega^k\wedge\eta^l)=d\omega^k\wedge\eta^l+(-1)^k\omega^k\wedge d\eta^l$,
\item $d\circ d=0$.
\end{enumerate}
\end{theorem}

An explicit expression for $d\omega^k$ is actually given by
\begin{equation}
   d\omega^k=\sum d\omega_I\wedge dx^I
\end{equation}
if $\omega^k=\sum\omega_I dx^I$.
An useful other way to write it is the following. If $\omega$ is a $k$-form and $X_1,\ldots,X_{p+1}$ some vector fields,
\begin{equation}\label{eq:formule_domega}
\begin{split}
  (k+1)d\omega(X_1,\ldots,X_{p+1})&=\sum_{i=1}^{p+1}(-1)^{i+1}X_i\omega(X_1,\ldots\hat{X}_i,\ldots,X_{p+1})\\
                                  &\quad+\sum_{i<j}(-1)^{i+j}\omega([X_i,X_j],X_1,\ldots,\hX_i,\hX_j,\ldots,X_{p+1}).
\end{split}
\end{equation}
Let us show it with $p=1$. Let $\omega=\omega_i dx^i$ and compute $d\omega(X,Y)=\partial_i\omega_j(dx^i\wedge dx^j)(X,Y)$. For this, we have to keep in mind that the $\partial_i$ acts only on $\omega_j$ while, in equation \eqref{eq:formule_domega}, a term $X\omega(Y)$ means --pointwise-- the action of $X$ on the function $\dpt{\omega(Y)}{M}{\eR}$. So we have to use Leibnitz formula:
\[
  (\partial_i\omega_j)X^iY^j=(X\omega_j)Y^j
                            =X(\omega_j Y^j)-\omega_j XY^j.
\]
On the other hand, we know that $[X,Y]^i=XY^i-YX^i$, so
\begin{equation}
   d\omega(X,Y)=X\omega(Y)-Y\omega(X)-\omega([X,Y]).
\end{equation}

\subsubsection{Hodge dual operator}
%/////////////////////////////
Let us take a manifold $M$ endowed with a metric $g$.  We can define a map $\dpt{r}{T^*_xM}{T_xM}$ by, for $\alpha\in T^*_xM$,
\[
   \scal{r(\alpha)}{v}=\alpha(v).
\]
for all $v\in T_xM$, where $\scal{\cdot}{\cdot}$ stands for the product given by the metric $g$. If we have $\alpha,\beta\in T^*_xM$, we can define
\[
   \scal{\alpha}{\beta}=\scal{r(\alpha)}{r(\beta)}.
\]
With this, we define an inner product on $\Lambda^p(T^*_xM)$:
\[
   \scal{\alpha_1\wedge\ldots\alpha_p}{\beta_1\wedge\ldots\beta_p}=\det_{ij}\scal{\alpha_i}{\beta_j}.
\]

\begin{definition}      \label{DEFooUOJQooSzKjNR}
    The \defe{Hodge operator}{Hodge operator} is $\dpt{\hodge}{\Lambda^p(T^*_xM)}{\Lambda^{n-p}(T^*_xM)}$ such that for any $\phi\in\Lambda^p(T^*_xM)$,
    \begin{equation}
       \phi\wedge(\hodge\psi)=\scal{\phi}{\psi}\Omega=\langle \phi,\psi \rangle\sqrt{|\det(g)|}dx^1\wedge\ldots\wedge dx^n.
    \end{equation}
\end{definition}

\begin{example} \label{EXooCIYIooFPMLMU}
    We consider \( \eR^n\) with the euclidian metric. If \( \sigma=dx_j\), then we expect \( \hodge\sigma\) to be \( sdx_1\wedge\ldots\wedge \widehat{dx_j}\wedge\ldots\wedge dx_n\) for a certain factor \( s\) to be fixed (something like \( (-1)^j\)).

    For every \( 1\)-form \( \phi\) we need \( \phi\wedge(\hodge \sigma)=\langle \phi, \sigma\rangle dx_1\wedge\ldots\wedge dx_n\). A basis of \( \Wedge^1(TM)\) is \( \{ dx_k \}_{k=1,\ldots, n}\), so we test on \( dx_k\).

    First we have
    \begin{equation}
        \langle dx_k, dx_j\rangle =\langle e_k, e_j\rangle =\delta_{kj}.
    \end{equation}
    Then
    \begin{equation}
        s\,dx_k\wedge dx_1\wedge\ldots\wedge \widehat{dx_j}\wedge\ldots\wedge dx_n=s\delta_{kj}(-1)^{j+1}dx_1\wedge\ldots\wedge dx_n.
    \end{equation}
    Thus we need \( s=(-1)^{j+1}\) and we have
    \begin{equation}
        \hodge dx_j=(-1)^{j+1}dx_1\wedge\ldots\wedge \widehat{dx_j}\wedge\ldots\wedge dx_n.
    \end{equation}
\end{example}

\subsubsection{Volume form and orientation}
%//////////////////////////////////////////

Let $M$ be a $n$ dimensional smooth manifold. A \defe{volume form}{volume!form} on $M$ is a nowhere vanishing $n$-form and the manifold itself is said to be \defe{orientable}{orientable manifold} if such a volume form exists. Two volume forms $\mu_1$ and $\mu_2$ are describe the same orientation if there exists a function $f>0$ such that\footnote{Recall that the space of $n$-forms is one-dimensional.} $\mu_1=f\mu_2$.

\begin{proposition}
There exists only two orientations on a connected orientable manifold.
\end{proposition}
\begin{probleme}
    Check if the statement of that proposition is correct. Find a reference.
\end{probleme}

One says that the \emph{ordered} basis $(v_1,\cdots,v_n)$ of $T_xM$ is \defe{positively oriented}{positive!orientation} with respect to the volume form $\mu$ is $\mu_x(v_1,\cdots,v_n)>0$.

\subsection{Musical isomorphism}\label{subsec_musique}\index{musical isomorphism}
%---------------------------------

In some literature, we find the symbols $v^{\flat}$ and $\alpha^{\sharp}$. What does it mean ? For $X\in\cvec(M)$ and $\omega\in\Omega^2(M)$, the \defe{flat}{flat} operation $v^{\flat}\in\Omega^1(M)$ is simply defined by the inner product:
\begin{equation}        \label{EQooBTWXooTqoNxa}
  v^{\flat}=i(v)\omega
\end{equation}
 In the same way, we define the \defe{sharp}{sharp} operation by taking a $1$-form $\alpha$ and defining $\alpha^{\sharp}$ by
\begin{equation}
   i(\alpha^{\sharp})\omega=\alpha.
\end{equation}
An immediate property is, for all $v\in\cvec(M)$, $v^{\flat\sharp}=v$, and for all $\alpha\in\Omega^1(M)$, $\alpha^{\sharp\flat}=\omega$.

\subsection{Pull-back and push-forward}

\begin{normaltext}
Let $\dpt{\varphi}{M}{N}$ be a smooth map, $\alpha$ a $k$-form on $N$, and $Y$ a vector field on $N$. Consider the map $\dpt{d\varphi}{T_xM}{T_{\varphi(x)}M}$. The aim is to extend it to a map from the tensor algebra\footnote{Definition \ref{DEFooHPQXooETvEyn}} of ${T_xM}$ to the one of $T_{\varphi(x)}M$.
\end{normaltext}

The \defe{pull-back}{pull-back!of a $k$-form} of $\varphi$ on a $k$-form $\alpha$ is the map
\[
 \dpt{\varphi^*}{\Omega^k(N)}{\Omega^k(M)}
\]
 defined by
\begin{equation}\label{306e1}
 (\varphi^*\alpha)_m(v_1,\ldots,v_k)
 =\alpha_{\varphi(m)}(d\varphi_mv_1,\ldots,d\varphi_mv_k)
\end{equation}
for all $m\in M$ and $v_i\in\cvec(M)$.

Note the particular case $k=0$. In this case, we take --instead of $\alpha$-- a function $\dpt{f}{N}{\eR}$ and the definition \eqref{306e1} gives $\dpt{\varphi^*f}{M}{\eR}$ by
\[
     \varphi^*f=f\circ\varphi.
\]

The \defe{push-forward}{push-forward!of a $k$-form} of $\varphi$ on a $k$-form is the map
\[
 \dpt{\varphi_*}{\Omega^k(M)}{\Omega^k(N)}
\]
defined by $\varphi_*=(\varphi^{-1})^*$. For $v\in T_nN$, we explicitly have:
\[
                   (\varphi_*\alpha)_n(v)=\alpha_{\varphi^{-1}(n)}(d\varphi_n^{-1} v).
\]

Let now $\dpt{\varphi}{M}{N}$ be a diffeomorphism. The \defe{pull-back}{pull-back!of a vector field} of $\varphi$ on a vector field is the map
\[
           \dpt{\varphi^*}{\cvec(N)}{\cvec(M)}
\]
defined by
\[
              (\varphi^*Y)(m)=[(d\varphi^{-1})_m\circ Y\circ\varphi](m),
\]
or
\[
 (\varphi^*Y)_{\varphi^{-1}(n)}=(d\varphi^{-1})_nY_n,
\]
for all $n\in N$ and $m\in M$. Notice that \[\dpt{(d\varphi^{-1})_n}{T_nN}{T_{\varphi^{-1}(n)}M},\] and that  $\varphi^{-1}(n)$ is well defined because $\varphi$ is an homeomorphism.

The \defe{push-forward}{push-forward!of a vector field} is, as before, defined by $\varphi_*=(\varphi^{-1})^*$. In order to show how to manipulate these notations, let us prove the following equation:
\[
   f_{*\xi}=(df)_{\xi}.
\]
For $\dpt{\varphi}{M}{N}$ and $Y$ in $\cvec(N)$, we just defined $\dpt{\varphi^*}{\cvec(N)}{\cvec(M)}$, by
\begin{eqnarray}
 \label{2112r1}(\varphi^*Y)_{\varphi^{-1}(n)}=(d\varphi^{-1})_nY_n.
\end{eqnarray}
Take $\dpt{f}{M}{N}$; we want to compute $f_*=(f^{-1})^*$ with $\dpt{(f^{-1})^*}{\cvec(M)}{\cvec(N)}$. Replacing the ``$^{-1}$``\ on the right places, the definition \eqref{2112r1} gives us
\[
 \Big[(f^{-1})^*X\Big]_{f(m)}=(df)_mX_m,
\]
if $X\in\cvec(M)$, and $m\in M$.

We can rewrite it without  any indices: the coherence of the spaces automatically impose the indices: $(f^{-1})^*X=(df)X$. It can also be rewritten as $(f^{-1})^*=df$, and thus $f_*=df$. From there to $f_{* \xi}=(df)_{\xi}$, it is straightforward.

%+++++++++++++++++++++++++++++++++++++++++++++++++++++++++++++++++++++++++++++++++++++++++++++++++++++++++++++++++++++++++++ 
\section{Submanifold}
%+++++++++++++++++++++++++++++++++++++++++++++++++++++++++++++++++++++++++++++++++++++++++++++++++++++++++++++++++++++++++++

\begin{definition}      \label{DEFooLQHWooMOTgzq}
    If $M$ is a differentiable manifold and $N$, a subset of $M$, we say that $N$ is a \defe{submanifold}{submanifold} of dimension $k$ if $\forall\,p\in N$, there exists a chart $\dpt{\varphi}{\mU}{M}$ around $p$ such that
    \begin{equation}
       \varphi^{-1}(\varphi(\mU)\cap N)=\eR^k\cap\mU:=\{(x_1,\ldots,x_k,0\ldots,0)\in\mU\}.
    \end{equation}
\end{definition}

\begin{lemma}
    If \( N\) is a submanifold of the manifold \( M\), then \( N\) is a manifold for its own.
\end{lemma}

\begin{definition}
    Let us consider $M$ and $N$, two differentiable manifolds, $\dpt{f}{M}{N}$ a $\Cinf$ map and $x\in M$. We say that $f$ is an \defe{immersion}{immersion} at $x$ if $\dpt{df_x}{T_xM}{T_{f(x)}N}$ is injective and that $f$ is a \defe{submersion}{submersion} if $df_x$ is surjective.
\end{definition}

\begin{proposition}
Let $M$ be a submanifold of the manifold $N$. If $p\in M$, then there  exists a coordinate system $\{x_1,\ldots,x_n\}$ on a neighbourhood of $p$ in $N$ such that $x_1(p)=\ldots=x_n(p)=0$ and such that the set
\[
   U=\{q\in V\tq x_j(q)=0\,\forall\, m+1\leq j\leq n\}
\]
gives a local chart of $M$ containing $p$.
\label{prop:var_coord}
\end{proposition}

\begin{proof}
No proof.
\end{proof}

The sense of this proposition is that one can put $p$ at the center of a coordinate system on $N$ such that $M$ is just a submanifold of $N$ parametrised by the fact that its last $m-n$ components are zero.

Now we can give a characterization for a submanifold: $N$ is a submanifold of $M$ when $N\subset M$ (as set) and the identity $\dpt{\iota}{N}{M}$ is regular.\label{pg:caract_subvar}

\begin{proposition}
The own topology of a submanifold is finer than the induced one from the manifold.
\label{prop:topo_sub_manif}
\index{topology!on submanifold}
\end{proposition}

\begin{proof}
Let $M$ be a manifold of dimension $n$ and $N$ a submanifold\footnote{In the whole proof, we should say ``there exists a sub-neighbourhood such that\ldots``} of dimension $k<n$. We consider $V$, an open subset of $N$ for the induced topology, so $V=N\cap\mO$ for a certain open subset $\mO$ of $M$. The aim is to show that $V$ is an open subset in the topology of $N$.

Let us define $\mP=\varphi^{-1}(\mO)$.  The charts of $N$ are the projection to $\eR^k$ of the ones of $M$. We have to consider $W=\varphi^{-1}(V)$, since $N$ is a submanifold, $\varphi^{-1}(\mO\cap N)=\eR^k\cap\mP$. It is clear that $W=\eR^k\cap\mP$ is an open subset of $\eR^k$ because it is the projection on the $k$ first coordinates of an open subset of $\eR^n$.

The subset $V$ of $N$ will be open in the sense of the own topology of $N$ if $\varphi'{}^{-1}(V\cap\varphi'(\mU'))$ is open in $\eR^k$ where $\varphi'$ is the restriction of $\varphi$ to his $k$ first coordinates: $\varphi'(a)=\varphi(a,0)$ and $\mU'$ is the projection of $\mU$.
\end{proof}


\begin{lemma}
Let $V,M$ be two manifolds and $\varphi\colon V\to M$, a differentiable map. We suppose that $\varphi(V)$ is contained in a submanifold $S$ of $M$. If $\dpt{\varphi}{V}{S}$ is continuous, then it is differentiable.
\label{lem:var_cont_diff}
\end{lemma}

\begin{remark}
The map $\varphi$ is certainly continuous as map from $V$ to $M$ (this is in the assumptions). But this don't imply that it is continuous for the topology on $S$ (which is the induced one from $M$). So the continuity of $\dpt{\varphi}{V}{S}$ is a true assumption.
\end{remark}

\begin{proof}
Let $p\in V$. By proposition~\ref{prop:var_coord}, we have  a coordinate system $\{x_1,\ldots,x_m\}$ valid on a neighbourhood $N$ of $\varphi(p)$ in $M$ such that the set
\[
  \{r\in N\tq x_j(r)=0\, \forall s<j\leq m  \}
\]
with the restriction of $(x_1,\ldots x_s)\in N_S$ form a local chart which contains $\varphi(p)$. From the continuity of $\varphi$, there exists a chart $(W,\psi)$ around $p$ such that $\varphi(W)\subset N_S$. The coordinates $x_j(\varphi(q))$ are differentiable functions of  the coordinates of $q$ in $W$. In particular, the coordinates $x_j(\varphi(q))$ for $1\leq j\leq s$ are differentiable and $\dpt{\varphi}{V}{S}$ is differentiable because its expression in a chart is differentiable.
\end{proof}

A consequence of this lemma: if $V$ and $S$ are submanifolds of $M$ with $V\subset S$, and if $S$ has the induced topology from $M$, then $V$ is a submanifold of $S$. Indeed, we can consider the inclusion $\dpt{\iota}{V}{S}$: it is differentiable from $V$ to $M$ and continuous from $V$ to $S$ then it is differentiable from $V$ to $S$ by the lemma. Thus $V=\iota^{-1}(S)$ is a submanifold of $S$ (this is a classical result of differential geometry).

\begin{proposition}
A submanifold is open if and only if it has the same dimension as the main manifold.
\label{prop:subvar_ouvert}
\end{proposition}

\begin{proof}
\subdem{Necessary condition}
We consider some charts $\dpt{\varphi_i}{U_i}{M}$ on some open subsets $U_i$ of $\eR^n$. If $N$ is open in $M$, then this can be written as
\[
  N=\bigcup_iU_i.
\]
If we choose the charts on $M$ in such a manner that $\dpt{\varphi_i}{U_i\cap \eR^k}{N}$ are charts of $N$, we must have $\varphi_i(U_i\cap\eR^k)=\varphi_i(U_i)$. Then it is clear that $k=n$ is necessary.
\subdem{Sufficient condition}
If $N$ has same dimension as $M$, the charts $\dpt{\varphi_i}{U_i}{M}$ are trivially restricted to $N$.
\end{proof}

The following result allow to extend a smooth function defined on a submanifold to an open set of the «larger» manifold. 
\begin{proposition}     \label{PROPooOTZQooIfboXV}
    Let \( N\) be a submanifold of \( M\) and \( f\in  C^{\infty}(N)\). Let \( p\in N\). There exists a neighbourhood \( W\) of \( p\) in \( M\) and a function \( \tilde f\in  C^{\infty}(W)\) such that
    \begin{equation}
        \tilde f(n)=f(n)
    \end{equation}
    for every \( n\in N\).
\end{proposition}

\begin{proof}
    Since \( N\) is a submanifold of \( M\), the definition \ref{DEFooLQHWooMOTgzq} provides a chart \( \varphi\colon U\to M\) around \( p\) such that 
    \begin{equation}
        \varphi^{-1}\big( \varphi(U)\cap N \big)=\{ (x_1,\ldots, x_n,0,\ldots, 0) \}.
    \end{equation}
    From the function \( f\colon N\to \eR\) we consider 
    \begin{equation}
        \begin{aligned}
            f_1\colon \varphi^{-1}\big( \varphi(U)\cap N \big)&\to \eR \\
            f_1&=f\circ\varphi
        \end{aligned}
    \end{equation}
    This is the function \( f\) seen trough the chart. The function \( f_1\) is only defined on the ``\( N\)'' part of the chart, but can be extended as
    \begin{equation}
        \begin{aligned}
            \tilde f_1\colon U&\to \eR \\
            (x_1,\ldots, x_m)&\mapsto f_1(x_1,\ldots, x_n,0,\ldots, 0), 
        \end{aligned}
    \end{equation}
    which is a good definition since \( (x_1,\ldots, x_n,0,\ldots, 0)\) is in \( \varphi^{-1}\big( \varphi(U)\cap N \big)\).

    Finally we write
    \begin{equation}
        \begin{aligned}
            \tilde f\colon \varphi(U)&\to \eR \\
            \tilde f&=\tilde f_1\circ\varphi^{-1}.
        \end{aligned}
    \end{equation}
    This is the extension we are searching for. Indeed it is defined on \( \varphi(U)\) which is an open set in \( M\) which contains \( p\) and if \( q\in N\cap\varphi(U)\) we have \( q=\varphi(x_1,\ldots, x_n,0,\ldots, 0)\) and then
    \begin{subequations}
        \begin{align}
            \tilde f(q)&=(\tilde f_\circ\varphi^{-1})\varphi(x_1,\ldots, x_n,0,\ldots, 0)\\
            &=\tilde f_1(x_1,\ldots, x_n,0,\ldots, 0)\\
            &=f_1(x_1,\ldots, x_n,0,\ldots, 0)\\
            &=(f\circ\varphi)(x_1,\ldots, x_n,0,\ldots, 0)\\
            &=f(q).
        \end{align}
    \end{subequations}
    Thus \( \tilde f=f\) on \( \varphi(U)\cap N\).
\end{proof}

%+++++++++++++++++++++++++++++++++++++++++++++++++++++++++++++++++++++++++++++++++++++++++++++++++++++++++++++++++++++++++++
\section{Partition of unity}
%+++++++++++++++++++++++++++++++++++++++++++++++++++++++++++++++++++++++++++++++++++++++++++++++++++++++++++++++++++++++++++


\begin{definition}[\cite{ooQCDSooCpqDvB}]       \label{DEFooKFXLooFRLaBG}
    Let \( X\) be a topological space. A \defe{partition of unity}{partition!of unity} of \( X\) is a family of continuous functions \( \{ \psi_j \}_{j\in J}\) such that
    \begin{enumerate}
        \item
            \( \psi_j\colon X\to \mathopen[ 0 , 1 \mathclose]\);
        \item
            for every \( x\in X\) there exists a neighbourhood of \( x\) in \( X\) in which only a finite number of the \( \psi_j\)'s is non zero;
        \item
            for every \( x\in X\) we have
            \begin{equation}
                \sum_{j\in J}\psi_j(x)=1.
            \end{equation}
    \end{enumerate}
\end{definition}

\begin{definition}[\cite{ooQCDSooCpqDvB}]
    Let \( X\) be a topological space and \( \{ U_i \}_{i\in I}\) be a locally finite covering of \( X\). A partition of unity is \defe{subordinate}{partition!of unity!subordinate} to that covering if it is indexed by \( I\) (\( J=I\) in the definition~\ref{DEFooKFXLooFRLaBG}) and such that \( \supp(\psi_i)\subset U_i\) for every \( i\in I\).
\end{definition}

\begin{theorem}[\cite{ooQCDSooCpqDvB}]      \label{THOooPCHDooITWKpC}
    Let \( \Omega\) be an open set in \( \eR^d\) and \( \{ U_i \}_{i\in I}\) be an open covering of \( \Omega\). There exists
    \begin{enumerate}
        \item
            a \(  C^{\infty}\) partition of unity \( \{ \psi_j \}_{j\in J}\) such that \( \supp(\psi_j)\) is compact in one of the \( U_i\);
        \item       \label{ITEMooFGMJooQPLqGY}
            a \(  C^{\infty}\) partition of unity \( \{ \alpha_i \}_{i\in I}\) subordinated to the covering, such that for every compact \( K\) only a finite number of these \( \psi_i\)'s is non zero.
    \end{enumerate}
\end{theorem}

\begin{remark}
    This theorem does not furnish a smooth compactly supported partition of unity subordinated to the given covering. Either you choose the partition to be compactly supported, either you choose them subordinated to the covering.
\end{remark}

\begin{corollary}  \label{CORooMSWPooCxvuhm}
    If \( \Omega\) is bounded and \( \{U_i \}_{i\in I}\) is an open covering of \( \Omega\), there exists a partition of unity subordinated to \( \{ U_i \}_{i\in I}\) such that each \( \psi_i\) belongs to \( \swD(U_i)\).
\end{corollary}

\begin{proof}
    If \( \Omega\) is bounded in \( \eR^d\) we can consider \( U'_i=U_i\cap \Omega\) and use the point~\ref{ITEMooFGMJooQPLqGY} of theorem~\ref{THOooPCHDooITWKpC} for the covering \( \{ U'_i \}_{i\in I}\). So we have a partition of unity subordinated to that covering with supports in the \( U'_i\)'s. Since the support is closed and the \( U_i\)'s are bounded, the supports are compact. The functions of this partition of unity are also subordinated to the original \( U_i\)'s.
\end{proof}

%+++++++++++++++++++++++++++++++++++++++++++++++++++++++++++++++++++++++++++++++++++++++++++++++++++++++++++++++++++++++++++
\section{Integration of a differential form}
%+++++++++++++++++++++++++++++++++++++++++++++++++++++++++++++++++++++++++++++++++++++++++++++++++++++++++++++++++++++++++++

%---------------------------------------------------------------------------------------------------------------------------
\subsection{Open set in \( \eR^n\)}
%---------------------------------------------------------------------------------------------------------------------------

Let \( U\) be an open set of \( \eR^n\). A differential form of degree \( n\) over \( U\) can always be written under the form
\begin{equation}
    \omega_x=f(x)dx_1\wedge\ldots\wedge dx_n;
\end{equation}
this is proposition~\ref{ProprbjihK}.

\begin{definition}      \label{DEFooEYRFooRQTmRF}
    The integral of \( \omega\) on \( U\) is
    \begin{equation}
        \int_{U}f\,dx_1\wedge\ldots\wedge dx_n=\int_Uf
    \end{equation}
    The second integral is the integral of a function on \( \eR^n\), that is definition~\ref{DefTVOooleEst} where the measure is the Lebesgue measure on \( \eR^n\).
\end{definition}

\begin{lemma}[Change of variable]       \label{LEMooNCYSooXtnCKq}
    Let \( f\colon V\to U\) be a diffeomorphism of open sets in \( \eR^n\) and \( \omega\) be a \( n\)-form on \( U\). Then we have
    \begin{equation}
        \int_U\omega=\int_{f^{-1}(U)}f^*\omega
    \end{equation}
    if \( \det(f)>0\). A sign change if \( \det(df)<0\).
\end{lemma}

\begin{proof}
    Let, for \( y\in U\), write the form \( \omega\) as \( \omega_y=h(y)dy_1\wedge\ldots\wedge dy_n\). Taking \( v_i\in \Gamma(TV)\) we have
    \begin{subequations}
        \begin{align}
            (f^*\omega)_x(v_1,\ldots, v_n)&=\omega_{f(x)}\big( df_xv_1,\ldots, df_xv_n \big)\\
            &=h\big( f(x) \big)\det\begin{pmatrix}
                df_xv_1    \\
                \vdots    \\
                df_xv_n
            \end{pmatrix}\\
            &=(h\circ f)(x)\det(df_x)\det\begin{pmatrix}
                v_1    \\
                \vdots    \\
                v_n
            \end{pmatrix}\\
            &=(h\circ f)(x)\det(df_x)(dx_1\wedge\ldots\wedge dx_n)(v_1,\ldots, v_n).
        \end{align}
    \end{subequations}
    Thus
    \begin{equation}
        f^*\omega= (h\circ f)\det(df)dx_1\wedge\ldots\wedge dx_n
    \end{equation}
    Using the usual change of variable theorem~\ref{THOooUMIWooZUtUSg}\ref{ITEMooAJGDooGHKnvj} (and taking a sign if \( \det(df)<0\) because there is an absolute value in around the jacobian in \eqref{EQooLYAWooTArAZR}) :
    \begin{equation}
        \int_{f^{-1}(U)}f^*\omega=\int_V(h\circ f)\det(df)=\int_{f(V)}h=\int_Uh=\int_U\omega.
    \end{equation}
\end{proof}

That is for integrating a differential form on an open set of \( \eR^n\). In order to integrate on a manifold we ``simply'' use a pull-back with a chart system. There will be three complications
\begin{itemize}
    \item If an atlas is made from more than one chart, what about the intersections ?
    \item Independence with respect to the choice of the chart.
    \item Integrating a vector field (that is not obviously a \( n\)-form).
\end{itemize}

%---------------------------------------------------------------------------------------------------------------------------
\subsection{One chart on a manifold}
%---------------------------------------------------------------------------------------------------------------------------

We suppose \( (M,g)\) to be a \( n\)-dimensional Riemannian manifold and \( S\) to be a \( (n-1)\)-dimensional submanifold. We suppose that both are inside only one chart
\begin{equation}
    \phi\colon U\subset \eR^n\to M
\end{equation}
and
\begin{equation}
    \varphi\colon A\subset \eR^{n-1}\to S.
\end{equation}
We also consider a differential form \( \omega\in \Wedge^n(T^*M)\) and \( \sigma\in\Wedge^{n-1}(T^*M)\). These are respectively \( n\) and \( n-1\) differential forms on \( M\).  We also consider \( v\), a vector field on \( M\) and \( \tau\), a \( 1\)-form on \(M\).

Let us see what is possible to integrate.

\begin{definition}[\cite{ooMLEZooCKxedX}]       \label{DEFooPDRCooPiBklC}
    Let \( \omega\) be a \( n\)-form defined on \( \phi(U)\) (vanishing everywhere else). Its integral is :
    \begin{equation}
        \int_{\phi(U)}\omega=\int_U\phi^*\omega.
    \end{equation}
    The last integral is an integral of type \( \int_{U}F(x_1,\ldots, x_n)dx_1\wedge\ldots \wedge dx_n\) on an open set in \( \eR^n\). That is definition~\ref{DEFooEYRFooRQTmRF}.
\end{definition}

This definition is nothing if it depend on the parametrisation. The following proposition show slightly more than the independence.
\begin{proposition}[\cite{MonCerveau,ooBTXRooUEBLMV}]       \label{PROPooNJCLooMqeeeX}
    Let be the charts \( \phi\colon U\to M\) and \( \psi\colon V\to N\) and a map \( f\colon M\to N\). The whole is supposed to be minimal :
    \begin{equation}
        f\big( \phi(U) \big)=\psi(V).
    \end{equation}
    Then we have the ``change of variable'' formula :
    \begin{equation}
        \int_{\phi(U)}\omega=\int_{\psi(V)}(f^{-1})^*\omega.
    \end{equation}
\end{proposition}

\begin{proof}
    By definition \( \int_{\phi(U)}\omega=\int_U\phi^*\omega\) and we have the diffeomorphism
    \begin{equation}
        \phi^{-1}\circ f^{-1}\circ \psi\colon V\to U,
    \end{equation}
    so that we can use the result of lemma~\ref{LEMooNCYSooXtnCKq} :
    \begin{equation}
        \int_U\phi^*\omega=\int_{(\phi^{-1}\circ f^{-1}\circ \psi)^{-1}(U)}  (\phi^{-1}\circ f^{-1}\circ \psi)^*\phi^*\omega=\int_{(\psi^{-1}\circ f\circ \phi )U}\psi^*(f^{-1})^*\omega=\int_{\psi^{-1}(N)}\psi^*(f^{-1})^*\omega.
    \end{equation}
    The last integral is the definition of an integral on \( N\) :
    \begin{equation}
        \int_{\psi^{-1}(N)}\psi^*(f^{-1})^*\omega=\int_N(f^{-1})^*\omega.
    \end{equation}
\end{proof}

Here is the lemma that shows the independence of definition~\ref{DEFooPDRCooPiBklC} with respect to the change of chart system.
\begin{lemma}
    Let \( \varphi\colon V\to M\) be a chart such that \( \varphi(V)\cap \varphi(U)=N\) is not empty. We define \( U'=\phi^{-1}(N)\) and \( V'=\varphi^{-1}(N)\). Then
    \begin{equation}        \label{EQooLSZMooPcyMWN}
        \int_{\phi(U')}\omega=\int_{\varphi(V')}\omega.
    \end{equation}
\end{lemma}
This lemma allows us to write \( \int_N\omega\) the common value of both sides of \eqref{EQooLSZMooPcyMWN}.

\begin{proof}
    Taking \( f=\id\) and two charts for the same open set in \( M\) in proposition~\ref{PROPooNJCLooMqeeeX} shows the result.
\end{proof}

%---------------------------------------------------------------------------------------------------------------------------
\subsection{On manifold that require a finite atlas}
%---------------------------------------------------------------------------------------------------------------------------

We restrict ourself to manifolds that accept a finite atlas.

\begin{definition}[\cite{ooBTXRooUEBLMV}]      \label{DEFooITDTooWwrPPr}
    If \( \omega\) is a \( n\)-form on \( M\) and if \( \{ f_{\alpha} \} \) is a partition of unity\footnote{See theorem~\ref{THOooPCHDooITWKpC}.} subordinate to the finite atlas \( \{ U_{\alpha} \}\) then
    \begin{equation}
        \int_M\omega=\sum_{\alpha}\int_{\phi_{\alpha}(U_{\alpha})}f_{\alpha}\omega.
    \end{equation}
\end{definition}

We show that this definition does not depend on the choice of the partition of unity.
\begin{lemma}[\cite{MonCerveau,ooBTXRooUEBLMV}] \label{LEMooCMIZooHhHaHV}
    The definition~\ref{DEFooITDTooWwrPPr} is independent of the choice of atlas and partition of unity.
\end{lemma}

\begin{proof}
    Let \(  \{ U_{\alpha},\phi_{\alpha},f_{\alpha} \}_{\alpha\in A}  \) and \( \{ V_i,\varphi_i,g_i \}_{i\in I}\) be two choices of atlas, charts and subordinate partition of unity. We have to show that
    \begin{equation}        \label{EQooPVQZooHvbioJ}
        \sum_{\alpha\in A}\int_{\phi_{\alpha}(U_{\alpha})}f_{\alpha}\omega=\sum_{i\in I}\int_{\varphi_i(V_i)}g_i\omega.
    \end{equation}
    Since \( \{ g_i \}\) is a partition of unity,
    \begin{equation}
            \spadesuit=\sum_{\alpha}\int_{\phi_{\alpha}(U_{\alpha})}f_{\alpha}\omega=\sum_{\alpha}\int_{\phi_{\alpha}(U_{\alpha})}\sum_ig_if_{\alpha}\omega.
    \end{equation}
    Since the atlas are finite, the sums are finite and can be permuted with the integral. Moreover the function \( g_if_{\alpha}\) is nonzero only on \( \phi_{\alpha}(U_{\alpha})\cap\varphi_i(V_i)\) so that the integral can be taken on \( \phi_{\alpha}(U_{\alpha})\), \( \phi_{\alpha}(U_{\alpha})\cap\varphi_i(V_i)\) or \( \varphi_i(V_i)\). We have
    \begin{subequations}
        \begin{align}
            \spadesuit=\sum_i\sum_{\alpha}\int_{\phi_{\alpha(U_{\alpha})}}g_if_{\alpha}\omega&=  \sum_i\sum_{\alpha}\int_{\phi_{\alpha(U_{\alpha})}\cap \varphi_i(V_i)}g_if_{\alpha}\omega\\
            &=  \sum_i\sum_{\alpha}\int_{\varphi_i(V_i)}g_if_{\alpha}\omega\\
            &= \sum_i\int_{\varphi_i(V_i)}g_i\sum_{\alpha}f_{\alpha}\omega\\
            &=\sum_i\int_{\varphi_i(V_i)}g_i\omega.
        \end{align}
    \end{subequations}
\end{proof}
The common values of both sides of \eqref{EQooPVQZooHvbioJ} is denoted by \( \int_M\omega\).

The following is not really a definition, but a particular case of~\ref{DEFooPDRCooPiBklC}. The integral of a \( n-1\)-form on a \( (n-1)\)-submanifold is
\begin{equation}        \label{EQooYPOGooRYOXQe}
    \int_S\sigma=\int_A\varphi^*\sigma.
\end{equation}
Once again the last integral is an integral of a \( n-1\)-form on an open set in \( \eR^{n-1}\).

\begin{definition}[\cite{ooMLEZooCKxedX}]       \label{DEFooAXFXooWiMLKP}
    The integral of a \( 1\)-form on a \( n-1\) dimensional submanifold is :
    \begin{equation}
        \int_S\tau=\int_S\hodge\tau
    \end{equation}
    where \( \hodge\) is the Hodge dual defined by~\ref{DEFooUOJQooSzKjNR}.
\end{definition}
The last integral is the integral of a \( (n-1)\)-form on a \( (n-1)\)-submanifold, given by \eqref{EQooYPOGooRYOXQe}.

\begin{definition}      \label{DEFooAXZGooJairMQ}
    The integral of a vector field on a \( (n-1)\)-submanifold is :
    \begin{equation}
        \int_Sv=\int_Sv^{\flat}
    \end{equation}
    where \( v^{\flat}\) is the \( 1\)-form defined by the musical isomorphism \eqref{EQooBTWXooTqoNxa}.
\end{definition}

The following proposition provides a much more explicit formula for the integral of a vector field.

\begin{proposition}     \label{PROPooETLZooAVsrwy}
    Let \( \varphi\colon A\subset \eR^{n-1}\to \eR^n\) be an hypersurface and \( X\) be a vector field in \( \eR^n\). Then
    \begin{subequations}
        \begin{align}
            \int_SX&=\int_A\det\big( X,\frac{ \partial \varphi }{ \partial y_1 },\ldots, \frac{ \partial \varphi }{ \partial y_{n-1} } \big)  \label{SUBEQooWJSPooImJjQN}\\
            &=\int_A X\cdot\det\begin{pmatrix}
                e_1    &   \ldots    &   e_n    \\
                &   \partial_{y_1}\varphi    &       \\
                &    \vdots   &       \\
                &   \partial_{y_{n-1}}\varphi    &
            \end{pmatrix}\\
            &=\int_A X\cdot n
        \end{align}
    \end{subequations}
    where \( \{ y_1,\ldots, y_{n-1} \}\) are the coordinates on \( A\) and \( n\) is the normal vector to the parametrization.
\end{proposition}
Note : thanks to lemma~\ref{LEMooCMIZooHhHaHV}, the value of \( n\) can depend on the choice of coordinates, but the integral will not depend.

\begin{proof}
    If \( X=\sum_{i=1}^nX_i\partial_i\), then \( X^{\flat}=\sum_{i}X_idx_i\) and its Hodge dual is
    \begin{equation}
        \sum_{i}(-1)^i dx_1\wedge\ldots\wedge\widehat{dx_i}\wedge\ldots\wedge dx_n
    \end{equation}
    where the hat denotes a factor that is not present. Using the definitions~\ref{DEFooAXZGooJairMQ},~\ref{DEFooAXFXooWiMLKP} and~\ref{DEFooPDRCooPiBklC} it remains to integrate
    \begin{equation}
        \int_A\sum_i(-1)^iX_i\varphi^*\big( dx_1\wedge\ldots\wedge\widehat{dx_i}\wedge\ldots\wedge dx_n \big).
    \end{equation}
    If \( u_1,\ldots, u_{n-1}\) are vectors on \( A\) (that is on \( T_xA\) where \( x\) is the integration variable) we have
    \begin{subequations}
        \begin{align}
            \varphi^*(dx_1\wedge\ldots\wedge \widehat{dx_i}\wedge\ldots\wedge dx_n)(u_1,\ldots, u_{n-1})&= (dx_1\wedge\ldots\wedge \widehat{dx_i}\wedge\ldots\wedge dx_n)(d\varphi u_1,\ldots, d\varphi u_{n-1})\\
            &=\det\big( \tau_id\varphi u_1,\ldots, \tau_id\varphi u_{n-1} \big)
        \end{align}
    \end{subequations}
    where we used the lemma~\ref{LEMooICRXooFKPCRd}.

    What lies in the integral is the \( (n-1)\) differential form
    \begin{subequations}        \label{EQooEVAPooSbRfaj}
        \begin{align}
           (u_1,\ldots, u_{n-1})\mapsto \sum_{i}(-1)^iX_i&\det\big(    \tau_id\varphi u_1,\ldots, \tau_id\varphi u_{n-1}  \big)\\
            &=\det\big( X,d\varphi u_1,\ldots, d\varphi u_{n-1} \big).
        \end{align}
    \end{subequations}
    Since this is a \( (n-1)\) differential form over \( \eR^{n-1}\), this has to be proportional to \( dy_1\wedge\ldots dy_{n-1}\). The proportionality factor is found by applying \eqref{EQooEVAPooSbRfaj} to the basis \( \{ e_1,\ldots, e_n \}\). Since \( d\varphi(e_i)=\frac{ \partial \varphi }{ \partial y_i }\) we have the proportionality factor
    \begin{equation}
        \det\left( X,\frac{ \partial \varphi }{ \partial y_1 },\ldots, \frac{ \partial \varphi }{ \partial y_n } \right)
    \end{equation}
    and the integral to be computed is
    \begin{equation}
        \int_A\det\left( X,\frac{ \partial \varphi }{ \partial y_1 },\ldots, \frac{ \partial \varphi }{ \partial y_n } \right)dy_1\wedge\ldots\wedge dy_{n-1}=\int_A\det\left( X,\frac{ \partial \varphi }{ \partial y_1 },\ldots, \frac{ \partial \varphi }{ \partial y_n } \right).
    \end{equation}
    The formula \eqref{SUBEQooWJSPooImJjQN} is proven. The two others are application of lemma~\ref{LEMooFRWKooVloCSM}.
\end{proof}

\begin{example}
    Let us make the example with \( n=3\). We have
    \begin{equation}
        \varphi^*(dx\wedge dy)(v_1,v_2)=(dx\wedge dy)(d\varphi v_1 , d\varphi v_2)=\det
        \begin{pmatrix}
            d\varphi(v_1)_x    &   d\varphi(v_2)_x    \\
            d\varphi(v_1)_y    &   d\varphi(v_2)_y
        \end{pmatrix},
    \end{equation}
    and then
    \begin{equation}
        \sum_i(-1)^iX_i \varphi^*(   \Wedge_{k\neq i}dx_k    )(v_1,v_2)=\sum_i(-1)^iX_i
        \begin{pmatrix}
            d\varphi(v_1)_x    &   d\varphi(v_2)_x    \\
            d\varphi(v_1)_y    &   d\varphi(v_2)_y
        \end{pmatrix}=
        \det\begin{pmatrix}
             X_1  &   d\varphi (v_1)_x    &   d\varphi(v_2)_x    \\
             X_2  &   d\varphi (v_1)_y    &   d\varphi(v_2)_y    \\
             X_3  &   d\varphi (v_1)_z    &   d\varphi(v_2)_z
        \end{pmatrix}
    \end{equation}
\end{example}

%---------------------------------------------------------------------------------------------------------------------------
\subsection{Integrating by part}
%---------------------------------------------------------------------------------------------------------------------------

\begin{proposition}[\cite{MonCerveau}]
    Let \( \Omega\) be an open set in an manifold \( M\) of dimension \( n\) and \( \varphi\colon A\to \eR^n\) be a parametrisation of the boundary \( \partial\Omega\) with tangent vector field \( n\) (defined on \( \partial\Omega\)). Let \( u,v\in  C^{\infty}(M)\). Then we have
    \begin{equation}        \label{EQooQSMNooKHwbqp}
        \int_{\partial \Omega}uv\,n_j=\int_{\Omega}\frac{ \partial u }{ \partial x_j }v+\int_{\Omega}u\frac{ \partial v }{ \partial x_j }.
    \end{equation}
\end{proposition}

\begin{proof}
    We use the Stokes formula (theorem~\ref{ThoATsPuzF}) on the \( (n-1)\)-form
    \begin{equation}
        \omega=uv\,dx_1\wedge\ldots\wedge\widehat{dx_j}\wedge\ldots\wedge dx_n,
    \end{equation}
    and we know from example~\ref{EXooCIYIooFPMLMU} that \( \omega=(-1)^{j+1}\hodge dx_j\). On the other hand,
    \begin{equation}
        d\omega=\sum_k\frac{ \partial (uv) }{ \partial x_k }dx_j\wedge dx_1\wedge\ldots\wedge\widehat{dx_j}\wedge\ldots\wedge dx_n=\frac{ \partial (uv) }{ \partial x_j }(-1)^{j+1}dx_1\wedge\ldots\wedge dx_n.
    \end{equation}
    We can use the Stokes formula :
    \begin{equation}
        \int_{\partial \Omega} uv dx_1\wedge\ldots\wedge\widehat{dx_j}\wedge\ldots\wedge dx_n=  (-1)^{j+1} \int_{\Omega}\frac{ \partial (uv) }{ \partial x_j }.
    \end{equation}
    The left-hand side can be transformed as
    \begin{equation}
        \int_{\partial\Omega}\hodge(dx_j)=\int_{\partial\Omega}uv\partial_j=\int_{\partial\Omega}uv\,n_j
    \end{equation}
    where we used the definition~\ref{DEFooAXFXooWiMLKP} and the proposition~\ref{PROPooETLZooAVsrwy}.

    The coefficients \( (-1)^{j+1}\) simplify and the derivation of product produce the result.
\end{proof}

\begin{example}     \label{EXooWLUVooNamnKG}
    If we integrate by part the function \( u\frac{ \partial^2 v }{ \partial x_j^2 }\) we have
    \begin{equation}
        \int_{\omega}u\frac{ \partial^2 }{ \partial x_j^2 }=-\int_{\Omega}\frac{ \partial u }{ \partial x_j }\frac{ \partial v }{ \partial x_j }+\int_{\partial \Omega}u\frac{ \partial v }{ \partial x_j }n_j.
    \end{equation}
    Summing over \( j\) we have the interesting formula
    \begin{equation}        \label{EQooJLDTooIMtxEX}
        \int_{\Omega}u\Delta v=-\int_{\Omega}\nabla u\cdot\nabla v+\int_{\partial \Omega}u\frac{ \partial v }{ \partial n }
    \end{equation}
    where \( \Delta v=\sum_j\frac{ \partial^2v }{ \partial x_j^2 }\) and \( \frac{ \partial v }{ \partial n }\) is a notation for \( \nabla v\cdot n\).
\end{example}

%+++++++++++++++++++++++++++++++++++++++++++++++++++++++++++++++++++++++++++++++++++++++++++++++++++++++++++++++++++++++++++
\section{Lie derivative}
%+++++++++++++++++++++++++++++++++++++++++++++++++++++++++++++++++++++++++++++++++++++++++++++++++++++++++++++++++++++++++++

Consider $X\in\cvec(M)$ and $\alpha\in\Omega^p(M)$. Let $\dpt{\varphi_t}{M}{M}$ be the flow of $X$. The \defe{Lie derivative}{Lie!derivative!of a $p$-form} of $\alpha$ is
\begin{equation}\label{liesurforme}
         \mL_X\alpha=\lim_{t\to 0}\us{t}[(\varphi^*_t\alpha)-\alpha]=\dsdd{\varphi^*_t\alpha}{t}{0}.
\end{equation}
More explicitly, for $x\in M$ and $v\in T_xM$,
\[
             (\mL_X\alpha)_x(v)=\lim_{t\to 0}\us{t}\left[(\varphi_t^*\alpha)_x(v)-\alpha_x(v)\right]
\]
In the definition of the \defe{Lie derivative}{Lie!derivative!of a vector field} for a vector field, we need an extra minus sign:
\begin{equation}		\label{EqDefLieDerivativeVect}
            (\mL_XY)_x=\dsdd{\varphi_{-t*}Y_{\varphi_t(x)}}{t}{0}.
\end{equation}
Why a minus sign ? Because $Y_{\varphi_t(x)}\in T_{\varphi_t(x)}M$, but $\dpt{(d\varphi_{-t})_a}{T_aM}{T_{\varphi_{-t}(a)}M}$ so that, if we want, $\varphi_{-t*}Y_{\varphi_t(x)}$ to be a vector at $x$, we can't use $\varphi_{t*}$.

These two definitions can be embedded in only one. Let $X\in\cvec(M)$ and $\varphi_t$ its integral curve\footnote{\textit{i.e.} for all $x\in M$, $\varphi_0(x)=x$ and $\dsdd{\varphi_{u+t}(x)}{t}{0}=X_{\varphi_u(x)}$.}\index{integral!curve}. We know that $\varphi_{t*}$ is an isomorphism $\dpt{\varphi_{t*}}{T_{\varphi^{-1}(x)}M}{T_xM}$. It can be extended to an isomorphism of the tensor algebras at $\varphi^{-1}(x)$ and $x$. We note it $\tilde{\varphi}_t$. For all tensor field $K$ on $M$, we define
\[
            (\mL_XK)_x=\lim_{t\to 0}[K_x-(\tilde{\varphi_t}K)_x].
\]

On a Riemannian manifold $(M,g)$, a vector field $X$ is a \defe{\href{http://en.wikipedia.org/wiki/Killing_vector_fields}{Killing vector field}}{killing!vector field} if $\mL_Xg=0$.



\begin{lemma}
Let $\dpt{f}{(-\epsilon,\epsilon)\times M}{\eR}$ be a differentiable map with $f(0,p)=0$ for all $p\in U$. Then there exists $\dpt{g}{(-\epsilon,\epsilon)\times M}{\eR}$, a differentiable map such that $f(t,p)=tg(t,p)$ and
\[
                g(0,q)=\left.\dsd{f(t,q)}{t}\right|_{t=0}.
\]
\end{lemma}
\begin{proof}
Take
\[
                g(t,q)=\int_0^1\dsd{f(ts,p)}{(ts)}ds,
\]
and use the change of variable $s\to ts$.
\end{proof}

\begin{lemma}
If $\varphi_t$ is the integral curve of $X$, for all function $\dpt{f}{M}{\eR}$, there exists a map $g$, $g_t(p)=g(t,p)$ such that
$f\circ\varphi_t=f+tg_t$ and $g_0=Xf$.
\end{lemma}

\begin{proof}
Consider $\overline{f}(t,p)=f(\varphi_t(p))-f(p)$, and apply the lemma:
\[
          f\circ\varphi_t=tg_t(p)+f(p).
\]
Thus we have
\[
      Xf=\lim_{t\to 0}\us{t}[f(\varphi_t(p))-f(p)]=\lim_{t\to 0}g_t(p)=g_0(p).
\]
\end{proof}

One of the main properties of the Lie derivative is the following:
\begin{theorem}		\label{ThoLieDerrComm}
Let $X$, $Y\in\cvec(M)$ and $\varphi_t$ be the integral curve of $X$. Then
\[
         [X,Y]_p=\lim_{t\to 0}\us{t}[Y-d\varphi_tY](\varphi_t(p)),
\]
or
\begin{equation}
          \mL_XY=[X,Y].
\end{equation}
where the commutator is given by the definition \ref{DEFooHOTOooRaPwyo}.
\end{theorem}
\begin{proof}
Take $\dpt{f}{M}{\eR}$ and the function given by the lemma: $\dpt{g_t}{M}{\eR}$ such that $f\circ \varphi_t=f+tg_t$ and $g_0=Xf$. Then put $p(t)=\varphi_t^{-1}(p)$. The rest of the proof is a computation:
\[
            (\varphi_{t*}Y)_pf=Y(f\circ\varphi_t)_{p(t)}=(Yf)_{p(t)}+t(Yg_t)_{p(t)},
\]
so
\begin{equation}
\begin{split}
  \lim_{t\to 0}\us{t}[Y_p-(\varphi_{t*}Y)_p]f&=\lim_{t\to 0}\us{t}[(Yf)_p-(Yf)_{p(t)}]-\lim_{t\to 0}(Yg_t)_{p(t)}\\
                                         &=X_p(Yf)-Y_pg_0\\
                                         &=[X,Y]_pf.
\end{split}
\end{equation}

\end{proof}

A second important property is
\begin{theorem}
For any function $f\colon M\to V$,
\[
           \mL_Xf=Xf.
\]
\end{theorem}

\begin{proof}
If $X(t)$ is the path which defines the vector $X$, it is obvious that at $t=0$, $X(t)$ is an integral curve to $X$, so that we can take $X(t)$ instead of $\varphi_t$ in \eqref{liesurforme}. Therefore we have:
\begin{equation}
    \mL_Xf=\dsdd{\varphi_t^*f}{t}{0}
          =Xf
\end{equation}
by definition of the action of a vector on a function.
\end{proof}

\input{204_diff_geom}

\chapter{Representations}
\input{207_representations}

\chapter{Lie algebras}
% This is part of (almost) Everything I know in mathematics and physics
% Copyright (c) 2013-2014, 2019
%   Laurent Claessens
% See the file fdl-1.3.txt for copying conditions.

\begin{definition}      \label{DEFooVBPKooGxlDBn}
    A \defe{Lie algebra}{Lie algebra} is a vector space \( \lG\) on \( \eK(=\eR,\eC)\) endowed with a bilinear operation \( (x,y)\mapsto [x,y]\) from \( \lG\times\lG\) with the properties
    \begin{enumerate}
        \item
            \( [x,y]=-[y,x]\)
        \item
            \( \big[ x,[y,z] \big]+\big[ y,[z,x] \big]+\big[ z,[x,y] \big]=0\).
    \end{enumerate}
    The second condition is the \defe{Jacobi identity}{Jacobi!identity}.
\end{definition}

\begin{definition}      \label{DEFooDUEUooZLhKdv}
    \defe{derivation}{derivation!of a Lie algebra} of $\lA$:
    \begin{equation}
      D[X,Y]=[DX,Y]+[X,DY]
    \end{equation}
    for every $X$, $Y\in\lA$. 
\end{definition}

\begin{lemma}       \label{LemadhomomadXadYadXY}
    The adjoint map is an homomorphism \( \ad\colon \lG\to \GL(\lG)\). In other terms for every \( X,Y\in\lG\) we have
    \begin{equation}
        \big[ \ad(X),\ad(Y) \big]=\ad\big( [X,Y] \big)
    \end{equation}
    as operators on \( \lG\). In particular the algebra acts on itself and \( \lG\) carries a representation of each of its subalgebra.
\end{lemma}

\begin{proof}
    Using the fact that \( \ad(X)\) is a derivation and Jacobi, for \( Z\in\lG\) we have
    \begin{subequations}
        \begin{align}
            \big[ \ad(X),\ad(Y) \big]Z&=\ad(X)\ad(Y)Z-\ad(Y)\ad(X)Z\\
            &=\big[ [X,Y],Z \big]+\big[ Y,[X,Z] \big]-\big[ [Y,X],Z \big]-\big[ X,[Y,Z] \big]\\
            &=\ad\big( [X,Y] \big)Z.
        \end{align}
    \end{subequations}
\end{proof}

\section{Adjoint representation}
%////////////////////////////////////

Let $G$ be a Lie group and $g\in G$ we consider the map $\AD(g)\colon G\to G$ given by $\AD(g)h=g hg^{-1}$. This is an analytic automorphism of $G$. We define:\nomenclature[D]{$\Ad$}{Adjoint representation}
\[
    \Ad(g)=d\AD(g)_e.
\]
Using equation $\varphi(\exp X)=\exp d\varphi_e(X)$ with $\varphi=I(g)$,
\begin{equation}\label{eq:sigma_X_sigma}
  g e^{X}g^{-1}=\exp[ \Ad(g)X ]
\end{equation}
for every $g\in G$ and $X\in\lG$. The map $g\to\Ad(g)$ is a homomorphism from $G$ to $\GL(\lG)$. This homomorphism is called the \defe{adjoint representation}{adjoint!representation!Lie group on its Lie algebra}\index{representation!adjoint} of $G$.

\begin{proposition}
The adjoint representation is analytic.
\end{proposition}

\begin{proof}
We have to prove that for any $X\in\lG$ and for any linear map $\dpt{\omega}{\lG}{\eR}$, the function $\omega(\Ad(g)X)$ is analytic at $g=e$. Indeed if we take as $\omega$ , the projection to the $i$th component and $X$ as the $j$th basis vector ($\lG$ seen as a vector space), and if we see the product $\Ad(g)X$ as a product matrix times vector, $(\Ad(g)X)_i$ is just $\Ad(g)_{ij}$. Then our supposition is the analyticity of $g\to\Ad(g)_{ij}$ at $g=e$. \quext{L'analicité de $\Ad$, elle vient par prolongement analytique depuis juste un point ?}

Now we prove it. Consider $f\in\Cinf(G)$, analytic at $g=e$ and such that $Yf=\omega(Y)$ for any $Y\in\lG$. Using equation \eqref{eq:sigma_X_sigma},
\begin{equation}
  \omega(\Ad(g)X)=(\Ad(g)X)f
                      =\Dsdd{ f(e^{t\Ad(g)X}) }{t}{0}
                      =\Dsdd{ f(g e^{tX}g^{-1}) }{t}{0},
\end{equation}
which is well analytic at $g=e$.
\end{proof}


\begin{proposition}
Let $G$ be a connected Lie group and $H$, an analytic subgroup of $G$. Then $H$ is a normal subgroup\index{normal!subgroup} of $G$ if and only if $\lH$ is an ideal in $\lG$.
\end{proposition}

\begin{proof}
We consider $X$, $Y\in\lG$. Formula $\exp tX\exp tY\exp-tY=\exp( tY+t^2[X,Y]+o(t^3) )$ and equation \eqref{eq:sigma_X_sigma} give
\[
   \exp\Big( \Ad(e^{tX})tY \Big)=\exp\Big(  tY+t^2[X,Y]+o(t^3)  \Big).
\]
Since it is true for any $X$, $Y\in\lG$, $\Ad(e^{tX})tY=tY+t^2[X,Y]$; thus
\begin{equation}
  \Ad(e^{tX})=\mtu+t[X,Y]+o(t^2).
\end{equation}
Since we know that $\dpt{d\Ad_e}{\lG}{\gl(\lG)}$ is a homomorphism ($\Ad$ is seen as a map $\dpt{\Ad}{G}{\GL(\lG)}$), taking the derivative of the last equation with respect to $t$ gives
\begin{equation}
  d\Ad_e(X)=\ad X.
\end{equation}
Then $\Ad(e^X)=e^{\ad X}$. Since is connected, an element of $G$ can be written as $\exp X$ for a certain $X\in\lG$\footnote{Because $G$ is generated by any neighbourhood of $e$ and there exists such a neighbourhood of $e$ which is diffeomorphic to a subset of $\lG$ by $\exp$.}. The purpose is to prove that $g\exp Xg^{-1}=\exp(\Ad(g)X)$ remains in $H$ for any $g\in G$ if and only if $\lH$ is an ideal in $\lG$. In other words, we want $\Ad(g)X\in\lH$ if and only if $\lH$ is an ideal. We can write $g=e^Y$ for a certain $Y\in\lG$. Thus
\[
  \Ad(g)X=\Ad(e^Y)X=e^{\ad Y}X.
\]
Using the expansion
\begin{equation}
e^{\ad Y}=\sum_k\us{k!}(\ad Y)^k,
\end{equation}
we have the thesis.
\end{proof}

%+++++++++++++++++++++++++++++++++++++++++++++++++++++++++++++++++++++++++++++++++++++++++++++++++++++++++++++++++++++++++++ 
\section{Representation of the complex algebra}
%+++++++++++++++++++++++++++++++++++++++++++++++++++++++++++++++++++++++++++++++++++++++++++++++++++++++++++++++++++++++++++

When \(\lG \) is a real Lie algebra, the corresponding complex Lie algebra is defined as
\begin{equation}
    \lG_{\eC}=\lG\otimes_{\eR}\eC.
\end{equation}
We have, as an example, in the lemma \ref{LEMooVEJZooUVNdmE} the equality $\su(2)_{\eC}=\gsl(2,\eC)$.

\begin{lemma}[\cite{MonCerveau}]        \label{LEMooIGAFooTSUsJR}
    Let \( \lG\) be a real Lie algebra and \( \rho\colon \lG\to \GL(V)\) be an irreducible representation. There exists an irreducible representation \( \rho'\colon \lG_{\eC}\to \GL(V)\) such that \( \rho=\rho'|_{\lG}\).
\end{lemma}

\begin{proof}
    We prove that the map
    \begin{equation}
        \begin{aligned}
            \rho'\colon \lG_{\eC}&\to \GL(V) \\
            X+iY&\mapsto \rho(X)+i\rho(Y) 
        \end{aligned}
    \end{equation}
    is the representation we are searching for.

    \begin{subproof}
        \item[This is a representation]
            Using the linearity, \( [X+iY,Z+iT]=[X,Z]-[Y,T]+i[X,T]+i[Y,Z]\). On the one hand we have
            \begin{equation}
                \rho'\big( [X+iY,Z+iT] \big)=\rho\big( [X,Z]-[Y,T] \big)+i\rho\big( [X,T]+[Y,Z] \big);
            \end{equation}
            while on the other hand,
            \begin{subequations}
                \begin{align}
                    \big[ \rho'(X+iY),\rho'(Z+iT) \big]&=\big[ \rho(X)+i\rho(Y),\rho(Z)+i\rho(T) \big]\\
                    &=\big[ \rho(X),\rho(Z) \big]+i\big[ \rho(X),\rho(T) \big]\\
                    &\qquad+i\big[ \rho(Y),\rho(Z) \big]-\big[ \rho(Y),\rho(T) \big]\\
                    &=\rho\big( [X,Z] \big)-\rho\big( [Y,T] \big)+i\rho\big( [X,T]+[Y,Z] \big).
                \end{align}
            \end{subequations}
            This proves that the map \( \rho'\) commutes with the Lie bracket, so that \( \rho'\) is a representation.
        \item[Irreducible]
            Let \( W\) be a subspace of \( V\) and suppose that \( W\) is invariant under \( \rho'\). This means that, for every \( X,Y\in \lG_{\eC}\), we have \( \rho'(X+iY)W\subset W\). In particular, with \( Y=0\) we have
            \begin{equation}
                \rho(X)W=\rho'(X)W\subset W.
            \end{equation}
            This shows that \( W\) is invariant under \( \rho\). Since \( \rho\) is irreducible, the subspace \( W\) must be \( \{ 0 \}\) or \( V\). Thus \( \rho'\) is irreducible.
    \end{subproof}
\end{proof}

%+++++++++++++++++++++++++++++++++++++++++++++++++++++++++++++++++++++++++++++++++++++++++++++++++++++++++++++++++++++++++++
\section{Jordan decomposition}
%+++++++++++++++++++++++++++++++++++++++++++++++++++++++++++++++++++++++++++++++++++++++++++++++++++++++++++++++++++++++++++
\index{Jordan decomposition}
\index{decomposition!Jordan}

If $V$ is a finite dimensional space, a subspace $W$ in $V$ is \defe{invariant}{invariant!vector subspace} under a subset $G\subset\Hom(V,V)$ if $sW\subset W$ for any $s\in G$. The space $V$ is \defe{irreducible}{irreducible!vector space} when $V$ and $\{0\}$ are the only two invariant subspaces. The set $G$ is \defe{semisimple}{semisimple} if any invariant subspace has an invariant complement. In this case, the vector space split into $V=\sum_iV_i$ with $V_i$ invariant and irreducible.

% Reformulate these two ``Jordan'' theorems.
\begin{theorem}[Jordan decomposition]\index{Jordan decomposition}\index{decomposition!Jordan}
Any element $A\in\Hom(V,V)$ is decomposable in one and only one way as $A=S+N$ with $S$ semisimple and $N$ nilpotent and $NS=SN$. Furthermore, $S$ and $N$ are polynomials in $A$. More precisely:

If $V$ is a complex vector space and $A\in\Hom(V,V)$ with $\lambda_1,\ldots,\lambda_r$ his eigenvalues, we pose
\[
V_i=\{ v\in V\tq (A-\lambda_i\mtu)^kv=0 \textrm{ for large enough $k$}\}.
\]
Then

\begin{enumerate}\label{tho:jordan}
\item $V=\sum_{i=1}^rV_i$,
\item each $V_i$ is invariant under $A$,
\item the semisimple part of $A$ is given by
\[
   S(\sum_{i=1}^rv_i)=\sum_{i=1}^r\lambda_iv_i,
\]
for $v_i\in V_i$,

\item the characteristic polynomial of $A$ is
\[
  \det(\lambda\mtu-A)=(\lambda-\lambda_1)^{d_1}\ldots(\lambda-\lambda_r)^{d_r}
\]
where $d_i=\dim V_i$ ($1\leq i\leq r$).
\end{enumerate}
\end{theorem}

Now we give a great theorem without proof.
\begin{theorem}[Jordan decomposition]
Let $V$ be a finite dimensional vector space and $x\in\End{V}$.

\begin{enumerate}
\item There exists one and only one choice of $x_s,x_n\in\End(V)$ such that $x=x_s+x_n$, $x_s$ is semisimple, $n_n$ is nilpotent and $[x_s,x_n]=0$.

\item There exists polynomials $p$ and $q$ without independent term such that $x_s=p(x)$, $x_n=q(x)$; in particular if $y\in\End{V}$ commutes with $x$, then it commutes with $x_s$ and $x_n$.

\item If $A\subset B\subset V$ are subspaces of $V$ and if $x(B)\subset A$, then $x_s(B)\subset A$ and $x_n(B)\subset A$.
\end{enumerate}
\label{prop:Jordan_decomp}
\end{theorem}

\begin{lemma}\label{lem:Jordan_ad}
    Let $x\in\End{V}$ with his Jordan decomposition $x=x_s+x_n$. Then the Jordan decomposition of $\ad x$ is
    \begin{equation}\label{eq:ad_x_xs_xn}
       \ad x=\ad x_s+\ad x_n.
    \end{equation}
\end{lemma}

\begin{proof}
We already know that $\ad x_s$ is semisimple and $\ad x_n$ is nilpotent. They commute because $[\ad x_s,\ad x_n]=\ad[x_s,x_n]=0$. Then the unicity part of Jordan theorem~\ref{prop:Jordan_decomp} makes \eqref{eq:ad_x_xs_xn} the Jordan decomposition of $\ad x$.
\end{proof}

\begin{lemma}\label{lem:M_nil}
Let $A\subset B$ be two subspace of $\gl(V)$ with $\dim V<\infty$. We pose
\[
   M=\{x\in\gl(V)\tq [x,B]\subset A\},
\]
and we suppose that $x\in M$ verify $\tr(x\circ y)=0$ for all $y\in M$. Then $x$ is nilpotent.
\end{lemma}

\begin{proof}
We use the Jordan decomposition $x=x_s+x_n$ and a basis in which $x_s$ takes the form $diag(a_1,\ldots,a_m)$; let $\{v_1,\ldots,v_m\}$ be this basis. We denotes by $E$ the vector space on $\eQ$ spanned by $\{a_1,\ldots,a_m\}$. We want to prove that $x_s=0$, i.e. $E=0$. Since $E$ has finite dimension, it is equivalent to prove that its dual is zero. In other words, we have to see that any linear map $\dpt{f}{E}{\eQ}$ is zero.

We consider $y\in\gl(V)$, an element whose matrix is $diag(f(a_1),\ldots,f(a_m))$ and $(E_{ij})$, the usual basis of $\gl(V)$. We know that
\begin{subequations}
\begin{align}
  (\ad x_s)E_{ij}&=(a_i-a_j)E_{ij},\\
  (\ad y)E_{ij}&=(f(a_i)-f(a_j))E_{ij}.
\end{align}
\end{subequations}
It is always possible to find a polynomial $r$ on $\eR$ without constant term such that $r(a_i-a_j)=f(a_i)-f(a_j)$. Note that this is well defined because of the linearity of $f$: if $a_i-a_j=a_k-a_l$, then $f(a_i)-f(a_j)=f(a_k)-f(a_l)$. Since $\ad x_s$ is diagonal, $r(\ad x_s)$ is the matrix with $r(\ad x_s)_{ii}$ on the diagonal and zero anywhere else. Then $r(\ad x_s)=\ad y$. By lemma~\ref{lem:Jordan_ad}, $\ad x_s$ is the semisimple part of $\ad x$, then $\ad y$ is  a polynomial without constant term with respect to $\ad x$ (second point of theorem~\ref{prop:Jordan_decomp}).

Since $(\ad y)B\subset A$, $y\in M$ and $\tr(xy)=0$. It is easy to convince ourself that the $s_n$ part of $x$ will not contribute to the trace because $x_n$ is strictly upper triangular and $y$ is diagonal. From the explicit forms of $x_s$ and $y$,
\[
  \tr(xy)=\sum_ia_if(a_i)=0.
\]
This is a $\eQ$-linear combination of element of $E$: we have to see it as $a_i$ being a basis vector and $f(a_i)$ a coefficient, so that we can apply $f$ on both sides to find $0=\sum_if(a_i)^2$. Then for all $i$, $f(a_i)=0$, so that $f=0$ because  the $a_i$ spans $E$.
\end{proof}

\begin{definition}[semisimple endomorphism]\index{semisimple!endomorphism}
    If $V$ is a finite dimensional vector space, we say that an element $u\in\End{V}$ is \defe{semisimple}{semisimple!endomorphism}\label{pg:def_semisimple} if every \( u\)-invariant subspace of \( V\) has a complementary \( u\)-invariant.
\end{definition}

\begin{proposition}[\cite{SIUaYwD}]
    If \( V\) a vector space over an algebraically closed field, an endomorphism is semisimple if and only if it is diagonalizable.
\end{proposition}
% TODO: give the proof; it is easy.

\label{pg:E_ij}Let $E_{kl}$ be the $(n+2)\times(n+2)$ matrix with a $1$ at position $(k,l)$ and $0$ anywhere else: $(E_{kl})_{ij}=\delta_{ki}\delta_{lj}$. An easy computation show that \nomenclature{$E_{ij}$}{Matrix full of zero's and $1$ at position $ij$}
\begin{equation}        \label{EqFormMulEmtr}       %\label{EqJsqnmunmtu}       Ce second label est certainement une erreur.
    E_{kl}E_{ab}=\delta_{la}E_{kb},
\end{equation}
and
\begin{equation}\label{comm_de_E}
    [E_{kl},E_{rs}]=\delta_{lr}E_{ks}-\delta_{sk}E_{rl}.
\end{equation}

%+++++++++++++++++++++++++++++++++++++++++++++++++++++++++++++++++++++++++++++++++++++++++++++++++++++++++++++++++++++++++++
\section{Killing form}
%+++++++++++++++++++++++++++++++++++++++++++++++++++++++++++++++++++++++++++++++++++++++++++++++++++++++++++++++++++++++++++

\begin{definition}
    The \defe{Killing form}{Killing!form} of $\mG$ is the symmetric bilinear form:
    \begin{equation}
                 B(X,Y)=Tr(\ad X\circ \ad Y).
    \end{equation}
\end{definition}

\begin{proposition}
    It is \defe{invariant}{invariant!form} in the sense of
    \begin{equation}                        \label{eq:Killing_invariant}
         B\big((\ad S)X,Y\big)=-B\big(X,(\ad S)Y\big),
    \end{equation}
    $\forall X$, $Y$, $S\in\mG$.
\end{proposition}

\begin{proposition} \label{PropAutomInvarB}
If $\dpt{\varphi}{\mG}{\mG}$ is an automorphism of $\mG$, then
\[
   B(\varphi(X),\varphi(Y))=B(X,Y).
\]
\label{prop:auto_2}
\end{proposition}

\begin{proof}
The fact that $\varphi$ is an automorphism of $\mG$ is written as $\varphi\circ\ad X=\ad(\varphi(X))\circ\varphi$, or
\[
  \ad(\varphi(X))=\varphi\circ\ad X\circ\varphi^{-1}.
\]
Then
\begin{equation}
\begin{split}
\tr(\ad(\varphi(X))\circ\ad(\varphi(Y)))&=\tr(\varphi\circ\ad X\circ\varphi^{-1}\circ\varphi\ad Y\circ\varphi^{-1})\\
                                &=\tr(\ad X\circ\ad Y).
\end{split}
\end{equation}
\end{proof}


\begin{remark}
The Killing $2$-form is a map $\dpt{B}{\mG\times\mG}{\eR}$. When we say that it is preserved by a map $\dpt{f}{G}{G}$, we mean that it is preserved by $df$: $B(df\cdot,df\cdot)=B(\cdot,\cdot)$.
\end{remark}

\begin{remark}
The Killing form is \emph{a priori} only defined on $\mG=T_eG$. For $A$, $B\in T_gG$, one naturally defines
\begin{equation}
  B_g(A,B)=B(dL_{g^{-1}}A,dL_{g^{-1}}B).
\end{equation}
This assures the left invariance of $B$. Now we prove the right invariance.
\end{remark}

An other important property of the Killing form is its bi-invariance.

\begin{lemma}
Let $\lG$ be a Lie algebra and $\lI$ an ideal in $\lG$. Let $\dpt{B}{\lG\times\lG}{\eR}$ be the Killing form on $\lG$ and $\dpt{B'}{\lI\times\lI}{\eR}$, the one of $\lI$. Then $B'=B|_{\lI\times\lI}$, i.e. the Killing form on $\lG$ descent to the ideal $\lI$.
\label{lem:Killing_descent_ideal}
\end{lemma}

\begin{proof}
If $W$ is a subspace of a (finite dimensional) vector space $V$ and $\dpt{\phi}{V}{W}$ and endomorphism, then $\tr\phi=\tr(\phi|_W$). Indeed, if $\{X_1,\ldots,X_n\}$ is a basis of $V$ such that $\{X_1,\ldots,X_r\}$ is a basis of $W$, the matrix element $\phi_{kk}$ is zero for $k>r$. Then
\[
  \tr\phi=\sum_{i=1}^{n}\phi_{ii}=\sum_{i=1}^r\phi_{ii}=\tr(\phi|_W).
\]

Now consider $X$, $Y\in\lI$; $(\ad X\circ\ad Y)$ is an endomorphism of $\lG$ which sends $\lG$ to $\lI$ (because $\lI$ is an ideal). Then
\[
B'(X,Y)=\tr\big( (\ad X\circ\ad Y)|_{\lI} \big)=\tr(\ad X\circ\ad Y)=B(X,Y).
\]
\end{proof}

%+++++++++++++++++++++++++++++++++++++++++++++++++++++++++++++++++++++++++++++++++++++++++++++++++++++++++++++++++++++++++++
\section{Solvable and nilpotent algebras}
%+++++++++++++++++++++++++++++++++++++++++++++++++++++++++++++++++++++++++++++++++++++++++++++++++++++++++++++++++++++++++++

If $\lG$ is a Lie algebra, the \defe{derived Lie algebra}{derived!Lie algebra}\index{Lie!algebra!derived} is
\[
   \dD\lG=\Span\{[X,Y]\tq X,Y\in\lG\}.
\]
We naturally define $\dD^0\lG=\lG$ and $\dD^n\lG=\dD(\dD^{n-1}\lG)$ this is the \defe{derived series}{derived!series}. Each $\dD^n\lG$ is an ideal in~ $\lG$. We also define the \defe{central decreasing sequence}{central!decreasing sequence} by $\lA^0=\lA$, $\lA^{p+1}=[\lA,\lA^p]$.

\begin{definition}
The Lie algebra $\lG$ is \defe{solvable}{solvable!Lie algebra}\index{Lie!algebra!solvable} if there exists a $n\geq 0$ such that $\dD^n\lG=\{0\}$. A Lie group is solvable when its Lie algebra is\index{Lie!group!solvable}\index{solvable!Lie group}.

    The Lie algebra  \( \lG\) is \defe{nilpotent}{nilpotent!Lie algebra} if \( \lG^n=0\) for some~\( n\). We say that \( \lG\) is \( \ad\)-nilpotent if \( \ad(X)\) is a nilpotent endomorphism of \( \lG\) for each \( X\in\lG\).
\end{definition}

Do not confuse \emph{nilpotent} and \emph{solvable} algebras. A nilpotent algebra is always solvable, while the algebra spanned by $\{ A,B \}$ with the relation $[A,B]=B$ is solvable but not nilpotent.

If $\lG\neq\{0\}$ is a solvable Lie algebra and if $n$ is the smallest natural such that $\dD^n\lG=\{0\}$, then $\dD^{n-1}\lG$ is a non zero abelian ideal in $\lG$. We conclude that a solvable Lie algebra is never semisimple (because the center of a semisimple Lie algebra is zero).

A Lie algebra is said to fulfil the \defe{chain condition}{chain!condition} if for every ideal $\lH\neq\{0\}$ in $\lG$, there exists an ideal $\lH_1$ in $\lH$ with codimension $1$.

\begin{lemma}
A Lie algebra is solvable if and only if it fulfils the chain condition.
\end{lemma}

\begin{proof}
\subdem{Necessary condition}
The Lie algebra $\lG$ is solvable (then $\dD\lG\neq\lG$) and $\lH$ is an ideal in $\lG$. We consider $\lH_1$, a subspace of codimension $1$ in $\lH$ which contains $\dD\lH$. It is clear that $\lH_1$ is an ideal in $\lH$ because $[H_1,H]\in\dD\lH\subset\lH_1$.
\subdem{Sufficient condition}
We have a sequence
\begin{equation}\label{eq:solvable_chaine}
   \{0\}=\lG_n\subset\lG_{n-1}\subset\ldots\subset\lG_0=\lG
\end{equation}
where $\lG_r$ is an ideal of codimension $1$ in $\lG_{r-1}$. Let $A$ be the unique vector in $\lG_{r-1}$ which don't belong to $\lG_r$.  When we write $[X,Y]$ with $X$, $Y\in\lG_{r-1}$, at least one of $X$ or $Y$ is not $A$ (else, it is zero) then at least one of the two is in $\lG_r$. But $\lG_r$ is an ideal; then $[X,Y]\in\lG_r$. Thus $\dD(\lG_{r-1})\subset\lG_r$ and
\[
\dD^n\lG=\dD^{n-1}\dD\lG\subset\dD^{n-1}\lG_1\subset\ldots\subset\lG_n=0.
\]
\end{proof}

\begin{theorem}[Lie theorem]\label{tho:Lie_Vu}\index{Lie!theorem}
    Consider $\lG$, a real (resp. complex) solvable Lie algebra and a real (resp. complex) vector space $V\neq\{0\}$. If $\dpt{\pi}{\lG}{\gl(V)}$ is a homomorphism, then there exists a non zero vector in $V$ which is eigenvector of all the elements of $\pi(\lG)$.
\end{theorem}

\begin{probleme}
    It is strange to be stated for real and complex Lie algebras. Following \cite{SamelsonNotesLieAlg}, this is only true for complex Lie algebras while there exists other versions for reals ones.
\end{probleme}

\begin{proof}
Let us do it by induction on the dimension of $\lG$. We begin with $\dim\lG=1$. In this case, $\pi$ is just a map $\dpt{\pi}{\lG}{\gl(V)}$ such that $\pi(aX)=a\pi(X)$. We have to find an eigenvector for the homomorphism $\dpt{\pi(X)}{V}{V}$. Such a vector exists  from the Jordan decomposition~\ref{tho:jordan}. Indeed, if there are no eigenvectors, there are no spaces $V_i$ and the decomposition $V=\sum V_i$ can't be true.

Now we consider a general solvable Lie algebra $\lG$ and we suppose that the theorem is true for any solvable Lie algebra with dimension less that $\dim\lG$. Since $\lG$ is solvable, there exists an ideal $\lH$ of codimension $1$ in $\lG$; then there exists a $e_0\neq 0\in V$ which is eigenvector of all the $\pi(H)$ with $H\in\lH$. So we have $\dpt{\lambda}{\lH}{\eR}$ naturally defined by
\[
  \pi(H)e_0=\lambda(H)e_0.
\]
Now we consider $X\in\lG\setminus\lH$ and $e_{-1}=0$, $e_p=\pi(X)^pe_0$ for $p=1,2,\ldots$ We will show that $\pi(H)e_p=\lambda(H)e_p\mod(e_0,\ldots,e_{p-1})$ for all $H\in\lH$ and $p\geq 0$. It is clear for $p=0$. Let us suppose that it is true for $p$. Then
\begin{equation}
\begin{split}
  \pi(H)e_{p+1}&=\pi(H)\pi(X)e_p\\
               &=\pi([H,X])e_p+\pi(X)\pi(H)e_p\\
           &=\lambda([H,X])e_p+\pi(X)\lambda(H)e_p\\
                      &\quad\mod(e_0,\ldots,e_{p-1},\pi(X)e_0,\ldots,\pi(X)e_{p-1}).
\end{split}
\end{equation}
But we can put $\pi([H,X])$ and $\pi(X)e_i$ into the modulus. Thus we have
\[
  \pi(H)e_{p+1}=\lambda(H)e_{p+1}\mod(e_0,\ldots,e_p).
\]

Now we consider the subspace of $V$ given by $W=\Span\{e_p\}_{p=1,\ldots}$. The algebra $\pi(\lH)$ leaves $W$ invariant and our induction hypothesis works on $(\pi(\lH),W)$; then one can find in $W$ a common eigenvector for all the $\pi(H)$. This vector is the one we were looking for.
\end{proof}

\begin{corollary}
Let $\lG$ be a solvable Lie group and $\pi$ a representation of $\lG$ on a finite dimensional vector space $V$. Then there exists a basis $\{e_1,\ldots,e_n\}$ of $V$ in which all the endomorphism $\pi(X)$, $X\in\lG$ are upper triangular matrices.
\label{cor:de_Lie_Vu}
\end{corollary}

\begin{proof}
Consider $e_1\neq 0\in V$, a common eigenvector of all the $\pi(X)$, $X\in\lG$. We consider $E_1=\Span\{e_1\}$. The representation $\pi$ induces a representation $\pi_1$ of $\lG$ on the space $V/E_1$. If $V/E_1\neq\{0\}$, we have a $e_2\in V$ such that $(e_2+E_1)\in V/E_1$ is an eigenvector of all the $\pi_(X)$.

In this manner, we build a basis $\{e_1,\ldots,e_n\}$ of $V$ such that $\pi(X)e_i=0\mod(e_1,\ldots,e_i)$ for all $X\in\lG$. In this basis, $\pi(X)$ has zeros under the diagonal.
\end{proof}

\begin{theorem}
Let $V$ be a real or complex vector space and $\lG$, a subalgebra of $\gl(V)$ made up with nilpotent elements. Then

\begin{enumerate}
\item $\lG$ is nilpotent;
\item $\exists v\neq 0$ in $V$ such that $\forall Z\in\lG$, $Zv=0$;
\item There exists a basis of $V$ in which the elements of $\lG$ are matrices with only zeros under the diagonal.
\end{enumerate}
\label{tho:trois_nil}
\end{theorem}

\begin{proof}
\subdem{First item} We consider a $Z\in\lG$ and we have to see that $\ad_{\lG}Z$ is a nilpotent endomorphism of $\lG$. Be careful on a point: an element $X$ of $\lG$ is nilpotent as endomorphism of $V$ while we want to prove that $\ad X$ is nilpotent as endomorphism of $\lG$. We denote by $L_Z$ and $R_Z$, the left and right multiplication; since we are in a matrix algebra, the bracket is given by the commutator: $\ad Z=L_Z-R_Z$. We have
\begin{equation}
(\ad Z)^p(X)=\sum_{i=0}^p(-1)^p \binom{p}{i}  Z^{p-i}XZ^i
\end{equation}
There exists a $k\in\eN$ such that $Z^k=0$. For this $k$, $(\ad Z)^{2k+1}$ is a sum of terms of the form $Z^{p-i}XZ^i$: either $p-i$ either $i$ is always bigger than $k$. But $\ad_{\lG}Z$ is the restriction of $\ad Z$ (which is defined on $\gl(V)$) to $\lG$. Then $\lG$ is nilpotent.

\subdem{Second item} Let $r=\dim\lG$. If $r=1$, we have only one $Z\in\lG$ and $Z^k=0$ for a certain (minimal) $k\in\eN$. We take $v$ such that $w=Z^{k-1}v\neq 0$ (this exists because $k$ is the minimal natural with $Z^k=0$). Then $Zw=0$.

Now we suppose that the claim is valid for any algebra with dimension less than $r$. Let $\lH$ be a strict subalgebra of $\lG$ with maximal dimension. If $H\in\lH$, $\ad_{\lG}H$ is a nilpotent endomorphism of $\lG$ which sends $\lH$ onto itself. Thus $\ad_{\lG}H$ induces a nilpotent endomorphism $H^*$ on the vector space $\lG/\lH$. We consider the set $\mA=\{H^*\tq H\in\lH\}$; this is a subalgebra of $\gl(\lG/\lH)$ made up with nilpotent elements which has dimension strictly less than $r$.

The induction assumption gives us a non zero $u\in \lG/\lH$ which is sent to $0$ by all $\mA$, i.e. $(\ad_{\lG}H)u=0$ in $\lG/\lH$. In other words, $u\in\lG\setminus\lH$ is such that $(\ad_{\lG}H)u\in\lH$.

The space $\lH+\eK X$ (here, $\eK$ denotes $\eR$ or $\eC$) of $\lG$ is a subalgebra of $\lG$. Indeed, with obvious notations,
\begin{equation}\label{eq:H_k_X}
[H+kX,H'+k'X]=[H,H']+\ad H(k'X)-\ad H'(kX)+kk'[X,X].
\end{equation}
The first term lies in $\lH$ because it is a subalgebra; the second and third therms belongs to $\lH$ by definition of $X$. The last term is zero. Since $\lH$ is maximal, $\lH+\eK X=\lG$. Then \eqref{eq:H_k_X} shows that $\lH$ is also an ideal. Now we consider
\[
  W=\{e\in V\tq\forall H\in\lH, He=0\}.
\]
Since $\dim\lH< r$, $W\neq\{0\}$ from our induction assumption. Furthermore, for $e\in W$, $HXe=[H,X]e+XHe=0$. Then $X\cdot W\subset W$. The restriction of $X$ to $W$ is nilpotent. Then there exists a $v\in W$ such that $Xv=0$. For him $Hv=0$ because $v\in W$ and $Xv=0$ by definition of $X$. Then $Gv=0$ for any $G\in\lH+\eK X=\lG$.

\subdem{Third item} Let $e_1$ be a non zero vector in $V$ such that $Ze_1=0$ for any $Z\in\lG$ (the existence comes from the second item). We consider $E_1=\Span e_1$. Any $Z\in\lG$ induces a nilpotent endomorphism $Z^*$ on the vector space $V/E_1$. If $V/E_1\neq\{0\}$, we take a $e_2\in V\setminus E_1$ such that $e_2+E_1\in V/E_1$ fulfils $Z^*(e_2+E_1)=0$ for all $Z\in\lG$. By going on so, we have $Ze_1=0$, $Ze_i=0\mod(e_1,\ldots,e_{i-1})$. In this basis, the matrix of $Z$ has zeros on and under the diagonal.
\end{proof}

\begin{corollary}
Let us consider $V$, a finite dimensional vector space on $\eK$ and $\lG$, a subalgebra of $\gl(V)$ made up with nilpotent elements. Then if $s\geq\dim V$ and $X_i\in\lG$, we have $X_1X_2\ldots X_s=0$.
\label{cor:nil_XXX}
\end{corollary}

\begin{proof}
We write the $X_i$'s in a basis where they have zeros on and under the diagonal. It is rather easy to see that each product push the non zero elements into the upper right corner.
\end{proof}

\begin{corollary}
A nilpotent algebra is solvable.
\end{corollary}

\begin{proof}
The algebra $\ad_{\lG}(\lG)$ is a subalgebra of $\gl(\lG)$ made up with nilpotent endomorphisms of $\lG$. The product of $s$ (see notations of previous corollary) such endomorphism is zero. In particular $\lG$ is solvable.
\end{proof}

We recall the definition of the central decreasing sequence: $\lA^0=\lA$, $\lA^{p+1}=[\lA,\lA^p]$.

\begin{corollary}
A Lie algebra $\lA$ is nilpotent if and only if $\lA^m=\{0\}$ for $m\geq\dim\lA$.
\label{cor:nil_Gn}
\end{corollary}

\begin{proof}
The direct sense is easy: we use corollary~\ref{cor:nil_XXX} with $\lG=\ad(\lA)$ ($\dim\lG=\dim\lA$). Since $\lG$ is nilpotent, for any $X_i\in\lG$ we have $X_1\ldots X_s$, so that $\lA^m=0$. The inverse sense is trivial.

\end{proof}

\begin{corollary}
A nilpotent Lie algebra $\lA\neq\{0\}$ has a non zero center
\end{corollary}

\begin{proof}
If $m$ is the smallest natural such that $\lA^m=0$, $\lA^{m-1}$ is in the center.
\end{proof}


\begin{lemma}
If $\lI$ and $\lJ$ are ideals in $\lG$, then we have a canonical isomorphism $\dpt{\psi}{(\lI+\lJ)/\lJ}{\lI/(\lI\cap\lJ)}$ given by
\[
  \psi([x])=\cloi
\]
if $x=i+j$ with $i\in\lI$ and $j\in\lJ$. Here classes with respect to $\lJ$ are denoted by $[.]$ and the one with respect to $(\lI\cap\lJ)$ by a bar.
\label{lem:pre_trois_resoluble}
\end{lemma}

\begin{proof}
We first have to see that $\psi$ is well defined. If $x'=i+j+j'$, $\psi([x])=\cloi$ because $j+j'\in\lJ$. If $x=i'+j'$ (an other decomposition for $x=i+j$), $\cloi=\cloj$, $j'-j=i-i'\in\lJ\cap\lI$. Then $\cloi=\overline{i'+j'-j}=\cloip$.

Now it is easy to see that $\psi$ is a homomorphism.
\end{proof}

\begin{proposition}
Let $\lG$ and $\lG'$ be Lie algebras.

\begin{enumerate}
\item If $\lG$ is solvable then any subalgebra is solvable and if $\dpt{\phi}{\lG}{\lG'}$ is a Lie algebra homomorphism, then $\phi(\lG)$ is solvable in $\lG'$.

\item  If $\lI$ is a solvable ideal in $\lG$ such that $\lG/\lI$ is solvable, then $\lG$ is solvable.
\item If $\lI$ and $\lJ$ are solvable ideals in $\lG$, then $\lI+\lJ$ is also a solvable ideal in $\lG$.
\end{enumerate}
\label{prop:trois_resoluble}
\end{proposition}

\begin{proof}
\subdem{First item}
If $\lH$ is a subalgebra of $\lG$, then $\dD^k\lH\subset\dD^k\lG$, so that $\lH$ is solvable. Now consider $\lH=\phi(\lG)\subset\lG'$. This is a subalgebra of $\lG'$ because $[h,h']=[\phi(g),\phi(g')]=\phi([g,g'])\in\lH$. It is clear that $\dD(\phi(\lG))\subset\phi(\dD(\lG))$ and
\begin{equation}
\dD^2(\phi(\lG))=\dD\big( \dD\phi(\lG) \big)
                \subset\dD(\phi\dD(\lG))
        \subset\phi\dD\dD(\lG)
        =\phi(\dD^2(\lG)).
\end{equation}
Repeating this argument, $\dD^k(\lH)\subset\phi(\dD^k\lG)$. So $\lH$ is also solvable. Note that $\phi([g,g'])=[\phi(g),\phi(g')]\subset\dD(\pi(\lG))$. Then
\begin{equation}
  \dD^k\pi(\lG)=\pi(\dD^k\lG).
\end{equation}

\subdem{Second item}
Let $n$ be the smallest integer such that $\dD^n(\lG/\lI)=0$; we look at the canonical homomorphism $\dpt{\pi}{\lG}{\lG/\lI}$. This satisfies $\dD^n(\pi(\lG))=\pi(\dD^n\lG)=0$. Then $\dD^n(\lG)\subset\lI$. If $\dD^m\lI=0$, then $\dD^{m+n}\lG=0$.

\subdem{Third item}
The space $\lI/(\lI\cap\lJ)$ is the image of $\lI$ by a homomorphism, then it is solvable and $(\lI+\lJ)/\lJ$ is also solvable. The second item makes $\lI+\lJ$ solvable.
\end{proof}

Now we consider $\lG$, any Lie algebra and $\lS$ a maximum solvable ideal i.e. it is included in none other solvable ideal. Let us consider $\lI$, an other solvable ideal in $\lG$. Then $\lI+\lS$ is a solvable ideal; since $\lS$ is maximal, $\lI+\lS=\lS$. Thus there exists an unique maximal solvable ideal which we call the \defe{radical}{radical!of a Lie algebra} of $\lG$. It will be often denoted by $\Rad\lG$. If $\beta$ is a symmetric bilinear form, his \defe{radical}{radical!of a quadratic form} is the set
\begin{equation}
  S=\{x\in\lG\tq\beta(x,y)=0\;\forall y\in\lG\}.
\end{equation}
The form $\beta$ is nondegenerate if and only if $S=\{0\}$.

\begin{proposition}
Let $\lG$ and $\lG'$ be Lie algebras.

\begin{enumerate}
\item If $\lG$ is nilpotent, then his subalgebras are nilpotent and if $\dpt{\phi}{\lG}{\lG'}$ is a Lie algebra homomorphism, then $\phi(\lG)$ is nilpotent.

\item If $\lG/\mZ(\lG)$ is nilpotent, then $\lG$ is nilpotent. For recall,
\[
   \mZ(\lG)=\{z\in\lG\tq [x,z]=0\;\forall x\in\lG\}.
\]

\item If $\lG$ is nilpotent, then $\mZ(\lG)\neq 0$.
\end{enumerate}
\label{prop:nil_homom_nil}
\end{proposition}

\begin{proof}
The proof of the first item is the same as the one of~\ref{prop:trois_resoluble}. Now if $(\lG/\mZ(\lG))^n=0$, then $\lG^n/\mZ(\lG)=0$; thus $\lG^n\subset\mZ(\lG)$, so that $\lG^{n+1}=[\lG,\mZ(\lG)]=0$. Finally, if $n$ is the smallest natural such that $\lG^n=0$, then $[\lG^{n-1},\lG]=0$ and $\lG^{n-1}\subset\mZ(\lG)$.
\end{proof}

The condition to be nilpotent can be reformulated by $\exists n\in\eN$ such that $\forall X_i$, $Y\in\lG$,
\[
   (\ad X_1\circ\ldots\circ\ad X_n)Y=0,
\]
in particular for any $X\in\lG$, there exists a $n\in\eN$ such that $(\ad X)^n=0$. An element for which such a $n$ exists is \defe{ad-nilpotent}{ad-nilpotent@$\ad$-nilpotent}. If $\lG$ is nilpotent, then all his elements are ad-nilpotent.

Some results without proof:

\begin{lemma}\label{lem:pre_Engel}
If $X\in\gl(V)$ is a nilpotent endomorphism, then $\ad X$ is nilpotent.
\end{lemma}

\begin{lemma}
If $x\in\gl(V)$ is semisimple, then $\ad(x)$ is also semisimple.
\end{lemma}

\begin{proof}
We choose a basis $\{v_1,\cdots,v_n\}$ of $V$ in which $x$ is diagonal with eigenvalues $a_1,\ldots,a_n$. For $\gl(V)$, we consider the basis $\{E_{ij}\}$ in which $E_{ij}$ is the matrix with a $1$ at position $(i,j)$ and zero anywhere else. This satisfies $[E_{kl},E_{rs}]=\delta_{lr}E_{ks}-\delta_{sk}E_{rl}$. We easily check that $E_{kl}(v_i)=\delta_{li}v_k$. Since we are in a matrix algebra, the adjoint action is the commutator: $(\ad x)E_{ij}=[x,E_{ij}]$; as we know that $x=a_kE_{kk}$,
\begin{equation}
 (\ad x)E_{ij}=a_k[E_{kk},E_{ij}]=(a_i-a_j)E_{ij}
\end{equation}
which proves that $\ad x$ has a diagonal matrix in the basis $\{E_{ij}\}$ of $\gl(V)$. Furthermore, we have an explicit expression for his matrix: the eigenvalues are $(a_i-a_j)$.
\end{proof}

\begin{remark}
The inverse implication is not true, as the unit matrix shows.
\end{remark}

\begin{theorem}[Engel,\cite{SamelsonNotesLieAlg}]\label{tho:Engel}
    A Lie algebra is nilpotent if and only if all his elements are ad-nilpotent.
\end{theorem}
\index{theorem!Engel}\index{Engel theorem}



\begin{proposition}\label{PropBDrongP}
    If an algebra $\lG\subset\gl(V)$ is made up with nilpotent endomorphisms of $V$, then $\lG$ is nilpotent as Lie algebra.
\end{proposition}
%TODO: a proof

\begin{proposition}[\cite{Sagle,SamelsonNotesLieAlg}] \label{PropKillingTraceDeuxn}
    On the Lie algebra \( \gl(\eR^n)\), the following formula holds:
    \begin{equation}
       B(X,Y)=2n\tr(XY).
    \end{equation}
\end{proposition}
%TODO: préciser que gl est l'algèbre de Lie des $n\times n$ matrices with vanishing trace.

\begin{proof}
    We consider a simple subalgebra $\lG$ of $\gl(V)$ for a certain vector space $V$ and a nondegenerate $\ad$-invariant symmetric $2$-form $f$. Then there exists a $S\in\GL(\lG)$ such that
    \begin{subequations}
    \begin{align}
      f(X,Y)&=B(SX,Y) \label{eq:S_un}  \\
      B(SX,Y)&=B(X,SY).  \label{eq:S_deux}
    \end{align}
    \end{subequations}
    If we consider a basis of $\lG$, we can write $f(X,Y)$ (and the Killing) in a matricial form\footnote{We systematically use the sum convention on the repeated subscript.} as
    \[
      f(X,Y)=f_{ij}X^iY^j,\qquad B(X,Y)=B_{ij}X^iY^j.
    \]
    Since $B$ is nondegenerate, we can define the matrix $(B^{ij})$ by $B^{ij}B_{jk}=\delta^i_k$. It is easy to see that the searched endomorphism of $\lG$ is given by $S^k_l=f_{kj}B^{jl}$.

    Using the invariance \eqref{eq:Killing_invariant} of the Killing form and \eqref{eq:S_deux}, we find
    \[
       B\big( (\ad X\circ S)Y,Z  \big)=-B\big( (S\circ\ad X)Z,Y  \big)
    \]
    for any $X$, $Y$, $Z\in\lG$. Now using \eqref{eq:S_un},
    \begin{equation}
     f\big(  (S^{-1}\circ\ad X\circ S)Y,Z  \big)=-f\big((\ad X) Z,Y\big)
                                               =f\big( (\ad Z) X,Y \big)
                           =f\big( Z, (\ad X)Y \big).
    \end{equation}
    Since $f$ is nondegenerate, we find $\ad X\circ S=S\circ\ad X$. It follows from Schurs'lemma that $S=\lambda I$. Note that $f(X,Y)=\lambda B(X,Y)$; this proves a certain unicity of the Killing form relatively to his invariance properties.

    Now we consider $f(X,Y)=\tr(XY)$. This is symmetric because of the cyclic invariance of the trace and this is $ad$-invariant because of the formula $\tr([a,b]c)=\tr(a[b,c])$ which holds for any matrices $a,b,c$.

    The next step to show that $f$ is nondegenerate; we define
    \[
      \lG\hperp=\{X\in\lG\tq f(X,Y)=0\,\forall Y\in\lG   \}.
    \]
    The simplicity of $\lG$ ($\lG$ has no proper ideals) makes $\lG$ equal to $0$ or $\lG$. Indeed consider $Z\in\lG\hperp$. For any $X$, $Y\in\lG$, we have
    \[
    0=f(Z,[X,Y])=f([Z,X],Y).
    \]
    Then $[Z,X]\in\lG\hperp$ and $\lG\hperp$ is an ideal. We will see that the reality is $\lG\hperp=0$ (cf. error~\ref{err:f_dege}). Let us suppose $\lG\hperp=\lG$ and consider the lemma~\ref{lem:M_nil} with $A=B=\lG$. We define
    \[
       M=\{ X\in\lG\tq [X,\lG]\subset\lG \}=\lG.
    \]
    If $X\in M$ satisfies $\tr(XY)=0$ for any $Y\in M$, then $X$ is nilpotent. Here, $X\in M$ is not a true condition because $M=\lG$. Since $\lG\hperp=\lG$, the trace condition is also trivial. Then $\lG$ is made up with nilpotent endomorphisms of $V$. Then lemma~\ref{lem:pre_Engel} makes all the $X\in\lG$ ad-nilpotent, so that $\lG$ is nilpotent by proposition~\ref{PropBDrongP}.

    By the third item of proposition~\ref{prop:nil_homom_nil}, $\mZ(\lG)\neq 0$ which contradicts the simplicity of $\lG$. Then $\lG\hperp=0$ and $f$ is nondegenerate. Finally,
    \begin{equation}
      B(X,Y)=\lambda\tr(X,Y)
    \end{equation}
    for a certain real number $\lambda$. With a certain amount of work, one can determine the exact value of $\lambda$ when $\lG$ is the Lie algebra of $n\times n$ matrices with vanishing trace.
\end{proof}
%TODO: finish the proof

%+++++++++++++++++++++++++++++++++++++++++++++++++++++++++++++++++++++++++++++++++++++++++++++++++++++++++++++++++++++++++++
\section{Flags and nilpotent Lie algebras}
%+++++++++++++++++++++++++++++++++++++++++++++++++++++++++++++++++++++++++++++++++++++++++++++++++++++++++++++++++++++++++++

Here we give a ``flag description'' of some previous results. In particular the chain \eqref{eq:solvable_chaine}. If $V$ is a vector space of dimension $n<\infty$, a \defe{flag}{flag} in $V$ is a chain of subspaces $0=V_0\subset V_1\subset\ldots\subset V_{n-1}\subset V_n=\lG$ with $\dim V_k=k$. If $x\in\End{V}$ fulfils $x(V_i)\subset V_i$, then we say that $x$ \defe{stabilise}{stabiliser of a flag} the flag.

\begin{theorem}
If $\lG$ is a subalgebra of $\gl(V)$ in which the elements are nilpotent endomorphisms and if $V\neq 0$, then there exists a $v\in V$, $v\neq 0$ such that $\lG v=0$.
\end{theorem}

\begin{proof}
This is the second item of theorem~\ref{tho:trois_nil}.
\end{proof}

\begin{corollary}
Under the same assumptions, there exists a flag $(V_i)$ stable under $\lG$ such that $\lG V_i\subset V_{i-1}$. In other words, there exists a basis of $V$ in which the matrices of $\lG$ are nilpotent; this basis is the one given by the flag.
\end{corollary}

\begin{proof}
Let $v\neq 0$ such that $\lG v=0$ which exists by the theorem and $V_1=\Span v$. We consider $W=V/V_1$; the action of $\lG$ on $W$ is also made up with nilpotent endomorphisms. Then we go on with $V_1$ and $W_1=W/V_2$,\dots
\end{proof}


\begin{lemma}
If $\lG$ is nilpotent and if $\lI$ is an non trivial ideal in $\lG$, then $\lI\cap\mZ(\lG)\neq 0$.
\end{lemma}

\begin{proof}
Since $\lI$ is an ideal, $\lG$ acts on $\lI$ with the adjoint representation. The restriction of an element $\ad X$ for $X\in\lG$ to $\lI$ is in fact a nilpotent element in $\gl(\lI)$. Then we have a $I\in\lI$ such that $\lG I=0$. Thus $I\in\lI\cap\mZ(\lG)$.
\end{proof}

\begin{theorem}
Let $\lG$ be a solvable Lie subalgebra of $\gl(V)$. If $V\neq 0$, then $V$ posses a common eigenvector for all the endomorphisms of $\lG$.
\label{tho:sol_ss_dem}
\end{theorem}

\begin{proof}
This is exactly the Lie theorem~\ref{tho:Lie_Vu}
\end{proof}

\begin{corollary}[Lie theorem]\index{Lie!theorem}\index{theorem!Lie}
Let $\lG$ be a solvable subalgebra of $\gl(V)$. Then $\lG$ stabilize a flag of $V$.
\label{tho:Lie_Vd}
\end{corollary}

\begin{proof}

This corollary is the corollary given in~\ref{cor:de_Lie_Vu}.

We consider $v_1$ the vector given by theorem~\ref{tho:sol_ss_dem}. Since it is eigenvector of all $\lG$, $\Span v_1$ is stabilised by $\lG$. Next we consider $v_2$ in the complementary which is also a common eigenvector,\ldots
\end{proof}

\begin{corollary}
If $\lG$ is a solvable Lie algebra, then there exists a chain of ideals in $\lG$
\[
  0=\lG_0\subset\lG_1\subset\ldots\subset\lG_n=\lG
\]
with $\dim\lG_k=k$.
\end{corollary}

\begin{proof}
If $\dpt{\phi}{\lG}{\gl(V)}$ is a finite-dimensional representation of $\lG$, then $\phi(\lG)$ is solvable by proposition~\ref{prop:nil_homom_nil}. Then $\phi(\lG)$ stabilises a flag of $V$. Now we take as $\phi$ the adjoint representation of $\lG$. A stable flag is the chain of ideals; indeed if $\lG_i$ is a part of the flag, then $\forall H\in\lG$ $\ad H\lG_i\subset\lG_i$ because the flag is invariant.
\end{proof}


\begin{corollary}
If $\lG$ is solvable then $X\in\dD\lG$ implies that $\ad_{\lG}X$ is nilpotent. In particular $\dD\lG$ is nilpotent.
\end{corollary}

\begin{proof}
We consider the ideals chain of previous corollary and an adapted basis: $\{X_1,\ldots,X_n\}$ is such that $\{X_1,\ldots,X_i\}$ spans $\lG_i$. In such a basis the matrices of $\ad(\lG)$ are upper triangular and it is easy to see that in this case, the matrices of $[\ad\lG,\ad\lG]$ are \emph{strictly} upper triangular: they have zeros on the diagonal. But $[\ad\lG,\ad\lG]=\ad_{\lG}[\lG,\lG]$. Then for $X\in\ad_{\lG}\dD\lG$, $\ad_{\lG}X$ is nilpotent. \emph{A fortiori}, $\ad_{\dD\lG}X$ is nilpotent and by the Engels'theorem~\ref{tho:Engel}, $\dD\lG$ is nilpotent.
\end{proof}

The following lemma is computationally useful because it says that if $X$ is a nilpotent element of a Lie algebra, then $g\cdot X$ is also nilpotent with (at most) the same order.

\begin{lemma}
  The following formula
\begin{equation}
\ad(g\cdot X)^nY=g\cdot \ad(X)^n(g^{-1}\cdot Y)
\end{equation}
holds for all $g\in G$ and $X$,$Y\in\lG$,
\label{lem:nil_Ad}
\end{lemma}

The proof is a simple induction on $n$.

%+++++++++++++++++++++++++++++++++++++++++++++++++++++++++++++++++++++++++++++++++++++++++++++++++++++++++++++++++++++++++++
\section{Semisimple Lie algebras}
%+++++++++++++++++++++++++++++++++++++++++++++++++++++++++++++++++++++++++++++++++++++++++++++++++++++++++++++++++++++++++++

A useful reference to go trough semisimple Lie algebras is \cite{Wisser}. Very few proofs, but the statements of all the useful results with explanations.

\begin{definition}
    A Lie algebra is \defe{semisimple}{semisimple!Lie algebra} if it has no proper abelian invariant Lie subalgebra. A Lie algebra is \defe{simple}{simple!Lie algebra} if it is not abelian and has no proper Lie subalgebra.
\end{definition}

In that definition, we say that a Lie subalgebra \( \lH\) is \defe{invariant}{invariant!Lie subalgebra} if \( \ad(\lG)\lH\subset\lH\).

There are a lot of equivalent characterisations. Here are some that are going to be proved (or not) in the next few pages. A Lie algebra is semisimple if an only if one of the following conditions is respected.
\begin{enumerate}
    \item
        The Killing form is nondegenerate.
    \item
        The radical of \( \lG\) is zero (theorem~\ref{ThoRadicalEquivSS}).
    \item
        There are no abelian proper invariant subalgebra.
\end{enumerate}

\begin{probleme}
    I think that in the following I took the degenerateness of Killing as definition.
\end{probleme}

The Killing form is a convenient way to define a Riemannian metric on a semisimple\footnote{In this case, $B$ is nondegenerate.} Lie group.

\begin{proposition}
Let $\lG$ be a semisimple Lie algebra, $\lA$ an ideal in $\lG$, and $\lA^{\perp}=\{X\in\lG\tq B(X,A)=0\forall A\in\lA\}$.
Then
\begin{enumerate}
\item $\lA^{\perp}$ is an ideal,
\item $\lG=\lA\oplus\lA^{\perp}$,
\item $\lA$ is semisimple,
\end{enumerate}
\label{prop:a_aperp}
\end{proposition}

\begin{proof}
\subdem{First item}
We have to show that for any $X\in\lG$ and $P\in\lA^{\perp}$, $[X,P]\in\lA^{\perp}$, or $\forall\, Y\in\lA$, $B(Y,[X,P])=0$. From invariance of $B$,
\[
  B(Y,[X,P])=B(P,[Y,X])=0.
\]
\subdem{Second item}
Since $B$ is nondegenerate, $\dim\lA+\dim\lA^{\perp}=\dim\lG$. Let us consider $Z\in\lG$ and $X$, $Y\in\lA\cap\lA^{\perp}$. We have $B(Z,[X,Y])=B([Z,X],Y)=0$. Then $[X,Y]=0$ because $B(Z,[X,Y])=0$ for any $Z$ and $B$ is nondegenerate. Thus $\lA\cap\lA^{\perp}$ is abelian. It is also an ideal because $\lA$ and $\lA^{\perp}$ are.

Now we consider $\lB$, a complementary of $\lA\cap\lA^{\perp}$ in $\lG$, $Z\in\lG$ and $T\in\lA\cap\lA^{\perp}$. The endomorphism $E=\ad T\circ\ad Z$ sends $\lA\cap\lA^{\perp}$ to $\{0\}$. Indeed consider $A\in\lA\cap\lA^{\perp}$; $(\ad Z) A\in\lA\cap\lA^{\perp}$ because it is an ideal, and then $(\ad T\circ\ad Z)A=0$ because it is abelian.

The endomorphism $E$ also sends $\lB$ to $\lA\cap\lA^{\perp}$ (it may not be surjective); then $\tr(\ad T\circ\ad Z)=0$ and $\lA\cap\lA^{\perp}=\{0\}$. Since $B$ is nondegenerate, $\dim\lA+\dim\lA^{\perp}=\dim\lG$. Then $\lA\oplus\lA^{\perp}=\lG$ is well a direct sum.
\subdem{Third item}
From lemma~\ref{lem:Killing_descent_ideal}, the Killing form of $\lG$ descent to the ideal $\lA$; then it is also nondegenerate and $\lA$ is also semisimple.
 \end{proof}

\begin{corollary}
A semisimple Lie algebra has center $\{0\}$.
\label{cor:ss_no_centre}
\end{corollary}

\begin{proof}
If $Z\in\ker\lG$, $\ad Z=0$. So $B(Z,X)=0$ for any $X\in\lG$. Since $B$ is nondegenerate, it implies $Z=0$.
\end{proof}


\begin{corollary}
If $\lG$ is a semisimple Lie algebra, it can be written as a direct sum
\[
   \lG=\lG_1\oplus\ldots\oplus\lG_r
\]
where the $\lG_i$ are simples ideals in $\lG$. Moreover each simple ideal in $\lG$ is a direct sum of some of them.
\label{cor:decomp_ideal}
\end{corollary}

\begin{proof}
If $\lG$ is simple, the statement is trivial. If it is not, we consider $\lA$, an ideal in $\lG$.
Proposition~\ref{prop:a_aperp} makes $\lG=\lA\oplus\lA^{\perp}$. Since $\lA$ and $\lA^{\perp}$ are semisimple, we can once again brake them in the same way. We do it until we are left with simple algebras.

For the second part, consider $\lB$ a simple ideal in $\lG$ which is not a sum of $\lG_i$. Then $[\lG_i,\lB]\subset\lG_i\cap\lB=\{0\}$. Then $\lB$ is in the center of $\lG$. This contradict corollary~\ref{cor:ss_no_centre}.
\end{proof}

\begin{proposition}
If $\lG$ is semisimple then
\[
   \ad(\lG)=\partial(\lG),
\]
i.e. any derivation is an inner automorphism:
\label{prop:ss_derr_int}
\end{proposition}

\begin{proof}
We saw at page \pageref{pg:ad_subset_der} that $\ad(\lG)\subset\partial(\lG)$ holds without assumptions of (semi)simplicity. Now we consider $D$, a derivation: $\forall X\in\lG$,
\[
   \ad(DX)=[D,\ad X].
\]
Then $\ad(\lG)$ is an ideal in $\partial(\lG)$ because the commutator of $\ad X$ with any element of $\partial(\lG)$ still belongs to $\ad(\lG)$. Let us denote by $\lA$ the orthogonal complement of $\ad(\lG)$ in $\partial(\lG)$ (for the Killing metric). The algebra $\ad(\lG)$ is semisimple because of it isomorphic to $\lG$. Since the Killing form on $\ad(\lG)$ is nondegenerate, $\lA\cap\ad(\lG)=\{0\}$. Finally $D\in\lA$ implies $[D,\ad X]\in\lA\cap\ad(\lG)=\{0\}$. Then $\ad(DX)=0$ for any $X\in\lG$, so that $D=0$. This shows that $\lA=\{0\}$, so that $\ad(\lG)=\partial(\lG)$.
\end{proof}

% This is part of (almost) Everything I know in mathematics and physics
% Copyright (c) 2013-2014,2017-2018, 2020
%   Laurent Claessens
% See the file fdl-1.3.txt for copying conditions.

\subsection{Cartan criterion}
%----------------------------

Let us recall a result: $\dD\lG=\lG^1$, $[\dD\lG,\dD\lG]\subset\lG^2$; then $\dD^k\lG\subset\lG^k$. Thus if $\lG$ is nilpotent, it is solvable. On the other hand, by the Engel theorem~\ref{tho:Engel}, $\dD\lG$ is nilpotent if and only if all the $\ad_{\dD\lG}x$ are nilpotent for $x\in\dD\lG$.


\begin{theorem}[Cartan criterion]
Let $\lG$ be a subalgebra of $\gl(V)$. We suppose that $\tr(xy)=0$ $\forall x\in\dD\lG, y\in\lG$. Then $\lG$ is solvable.
\end{theorem}

\begin{proof}
It is sufficient to prove that $\dD\lG$ is nilpotent indeed if we write $\dD^k\lG\subset\lG^k$ with $\dD\lG$ instead of $\lG$, $\dD^{k+1}\lG\subset(\dD\lG)^k$. If $\dD\lG$ is nilpotent, $(\dD\lG)^n=0$ and $\dD^{n+1}\lG=0$ so that $\lG$ is solvable.

Let us consider $x\in\dD\lG$. We have to prove that it is ad-nilpotent (see the Engel theorem~\ref{tho:Engel}). Let $A=\dD\lG$, $B=\lG$ and $M=\{x\in\gl(V)\tq [x\lG]\subset\dD\lG\}$. By definition of $\dD\lG$, $\lG\subset M$. The lemma~\ref{lem:M_nil} will conclude that $x\in\dD\lG$ is nilpotent if $\tr(xy)=0$ for any $y\in M$. Here we just have this equality for $y\in\lG$.

A typical generator of $\dD\lG$ is $[x,y]$ with $x$, $y\in\lG$. Take a $z\in M$; by the formula $\tr([x,y]z)=\tr(x[y,z])$, the trace that we have to check is
\begin{equation}
  \tr([x,y]z)=\tr(x[y,z])
             =\tr([y,z]x).
\end{equation}
But with $z\in M$, $[y,z]\in\dD\lG$, then $\tr([x,y]z)=\tr([y,z]x)=0$. Thus we are in the situation of the lemma.
\end{proof}


\begin{corollary}\label{cor:ad_g_sol}
A Lie algebra $\lG$ for which $\tr(\ad x\circ\ad y)=0$ for all $x\in\dD\lG$, $y\in\lG$ is solvable.
\end{corollary}

\begin{proof}
We consider $\lH=\ad\lG$; this is a subalgebra of $\gl(V)$ such that $a\in\dD\lH$ and $b\in\lH$ imply $\tr(ab)=0$. In order to see it, remark that $a\in\dD\lH$ can be written as $a=[\ad x,\ad y]=\ad[x,y]$ for certain $x$, $y\in\lG$. Then $\tr(ab)=\tr(\ad[x,y]\ad z)$ with $x$, $y$, $z\in\lG$; this is zero from the hypothesis. Then $\lH=\ad\lG$ is solvable.

It is also known that $\ker(\ad)=\mZ(\lG)$ is also solvable. Now we consider $\lM$ a complementary of $\mZ(\lG)$ in $\lG$: $\lG=\mZ\oplus\lM$. The Lie algebra $\ad(\lM)$ is solvable and the homomorphism $\dpt{\phi}{\ad\lM}{\lM}$ defined by $\phi(\ad x)=x$ is well defined. From the first item of the proposition~\ref{prop:trois_resoluble}, $\lM$ is solvable. With obvious notations, an element of $\dD\lM$ can be written as $[m,m']$ (because $\mZ(\lG)$ don't contribute to $\dD\lG$). Then $\dD\lG=\dD\lM$, so that $\lG$ is as much solvable than $\lM$.
\end{proof}


\begin{lemma}
The radical of a Lie algebra is non zero if and only if it has at least non zero abelian ideal.
\label{lem:ss_ideal}
\end{lemma}

\begin{proof}
The radical of $\lG$ is its unique maximal solvable ideal. An eventually non empty abelian ideal should be in the radical.

Let us now consider that the radical is non zero, and consider the derived series of $\Rad\lG$. Since $\Rad\lG$ is solvable, we can consider $n$, the minimal integer such that $\dD^n\Rad\lG=0$. Then $\dD^{n-1}\Rad\lG$ is a non zero abelian ideal.
\end{proof}


\begin{theorem}     \label{ThoRadicalEquivSS}
A Lie algebra is semisimple if and only if its radical is zero.
\end{theorem}

\begin{proof}
\subdem{Direct sense}
We suppose $\Rad\lG=0$ and we consider $S$, the radical of the Killing form:
\[
   S=\{X\in\lG\tq B(X,Y)=0\,\forall Y\in\lG\}.
\]
By definition, for any $X\in S$ and $Y\in\lG$, $\tr(\ad X\circ\ad Y)=0$. The Cartan criterion makes $\ad S$ solvable and the corollary~\ref{cor:ad_g_sol} makes $S$ solvable.

Now, the $ad$-invariance of the Killing form turns $S$ into an ideal, so that $S\subset\Rad(\lG)$ because any solvable ideal is contained in $\Rad\lG$. From the assumptions, $\Rad S=0$, then $S\subset\Rad\lG=0$. This shows that the Killing form is nondegenerate.

\subdem{Inverse sense}
We suppose $S=0$ and we will show that any abelian ideal of $\lG$ is in $S$. In this case, if $A$ is a solvable ideal with $\dD^nA=0$, then $\dD^{n-1}A$ is an abelian ideal, so that $\dD^{n-1}A=0$. By induction, $A=0$.

Let $I$ be an abelian ideal of $\lG$, $X\in I$ and $Y\in\lG$. Then $\ad X\circ\ad Y$ is nilpotent because for $Z\in\lG$,
\begin{equation}
  (\ad X\ad Y\ad X\ad Y)Z=(\ad X\ad Y)\underbrace{( [X,[Y,Z]] )}_{=X_1\in I}
                         =(\ad X)\underbrace{[Y,X_1]}_{=X_2\in I}
             =(\ad X)X_2
             =0.
\end{equation}
Then $0=\tr(\ad X\ad Y)=B(X,Y)$ and $X\in S$, so that $I\subset S=0$.

\end{proof}

\subsection{More about radical}
%-------------------------------

If $\lG$ is a Lie algebra whose radical is $\lR$, we say that a subalgebra $\lS$ of $\lG$ is a \defe{Levi subalgebra}{levi subalgebra} if $\lG=\lR\oplus\lS$.

Any Lie algebra posses a Levi subalgebra\quext{Reference needed.}.

\begin{lemma}
If $\lA$ is an ideal in a Lie algebra $\lG$, then
\[
  \Rad\lA=(\Rad\lR)\cap\lA.
\]
\label{lem:rad_ideal}
\end{lemma}

Before to begin the proof, let us recall that lemma~\ref{lem:pre_trois_resoluble} gives us an isomorphism $\dpt{\psi}{(\lA+\lB)/\lA}{\lB/(\lA\cap\lB)}$ when $\lA$ and $\lB$ are ideals in $\lG$.

\begin{proof}[Proof of the lemma]
If $\lR$ is the radical of $\lG$, then the radical of $\lG/\lR$ is zero, so that $\lR/\lR$ is semisimple. Let $\lA$ be an ideal in $\lG$, then $(\lA+\lR)/\lR$ is an ideal in the semisimple Lie algebra $\lG\lR$, so that it is also semisimple. From the isomorphism, $\lA/(\lA\cap\lR)$ is also semisimple and $\lA\cap\lR$ must contains the radical of $\lA$. Indeed if a solvable ideal of $\lA$ where not in $\lA\cap\lR$, then this should give rise to a non zero solvable ideal in $\lA/(\lA\cap\lR)$ although the latter is semisimple. Then $\lA\cap\lR=\Rad\lA$.
\end{proof}

\begin{proposition}
If $A$ is a compact group of automorphisms of the Lie algebra $\lG$, then there exists a Levi subalgebra of $\lG$ which is invariant under $A$.
\end{proposition}

\begin{proof}
Let $\lR$ be the radical of $\lG$; we will split our proof into two cases following $[\lR,\lR]=0$ or not.
\subdem{The radical is abelian}
In this first case we consider an induction with respect to the dimension of $\lG$. We consider $\olG=\lG/\lRlR$ and $\olR=\lR/\lRlR$: these are algebras with one less dimension that $\lG$ and $\lR$. We denote by $\dpt{\pi}{\lG}{\olG}$ the natural projection.

We begin to prove that $\olR$ is the radical of $\olG$. It is clear from the Lie algebra structure on a quotient that $\olR$ is an ideal because $\lR$ is. It is also clear that $\olR$ is solvable. We just have to see that $\olR$ is maximal in $\olG$. For this, suppose that $\olR\cup\oX$ is a solvable ideal in $\olG$. Then it is easy to see that $\lR\cup X$ is an ideal in $\lG$. Taking commutators in $\olR\cup\oX$, we always finish in $\overline{0}\in\olG$, i.e. in $\lRlR$. Taking again some commutators, we finish on $0\in\lG$ because $\lR$ is solvable. This contradict the maximality of $\lR$.

Since $A$ is made up of automorphisms, it leaves $\lR$ invariant, so that it also acts on $\olG$ as an automorphism group: $a\oX=\overline{aX}$ for $a\in A$ and $X\in\lG$. From the induction assumption, we can find a Levi subalgebra $\olS$ in $\olG$: $\olS\oplus\olR=\olG$. In this case, the radical of $\pi^{-1}(\olS)$ is $\lRlR$. Indeed in the one hand, $\olR\cap\olS=0$, so that $\pi^{-1}(\olR\cap\olS)=\lRlR$. In the other hand $\pi^{-1}(\olR\cap\olS)=\pi^{-1}(\olR)\cap\pi^{-1}(\olS)=\lR\cap\pi^{-1}(\olS)$. The lemma~\ref{lem:rad_ideal} conclude that $\Rad\pi^{-1}(\olS)=\lRlR$.

Now $A$ is a compact group of automorphism which leaves invariant $\pi^{-1}(\olS)$, so we have a Levi subalgebra $\lS$ of $\pi^{-1}(\olS)$ invariant under $A$. We will see that this is in fact a Levi subalgebra of the whole $\lG$, i.e. we have to prove that $\lS\oplus\lR=\lG$. From the definition of $\lS$,
\[
   \lS\oplus\lRlR=\pi^{-1}(\olS),
\]
and by definition of $\olS$,
\[
   \olS\oplus\frac{\lR}{\lRlR}=\olG.
\]
Then
\begin{equation}
  \lG=\pi^{-1}(\olS)\oplus\lR+\lRlR
     =\lS\oplus\lRlR\oplus\lR+\lRlR
     =\lS\oplus\lR.
\end{equation}

We can now pass to the second case: $\lRlR=0$.
\subdem{The radical is not abelian}
Let $\lS_0$ and $\lS$ be Levi subalgebras of $\lG$. For $X\in\lS_0$, we write
\[
   X=f(X)+X_{\lS}
\]
with respect to the decomposition $\lG=\lS\oplus\lR$. This defines a linear map $\dpt{f}{\lS_0}{\lR}$. For any $X$, $Y\in\lS_0$, $[X_{\lS},X_{\lS}]=[X,Y]-[X,f(y)]-[f(X),Y]$ because $\lR$ is abelian. Since\quext{C'est pas clair pourquoi on a \c a.}, $[X_{\lS},X_{\lS}]=[X,Y]_{\lS}$,
\begin{equation}\label{eq:f_presque_isom}
f([X,Y])=[X,f(Y)]-[f(X),Y].
\end{equation}
Now let us consider a map $\dpt{f}{\lS_0}{\lR}$ which satisfy this equation. Then the map $X\to X-f(X)$ is an isomorphism between $\lS_0$ and his image which is a Levi subalgebra of $\lG$. Indeed
\begin{equation}
\begin{split}
[X,Y]&\to[X,Y]-f([X,Y])\\
     &=[X,Y]-[X,f(Y)]-[f(X),Y]\\
     &=[X-f(X),Y-f(Y)].
\end{split}
\end{equation}
Now we consider $V$, the space of all the linear maps $\lS_0\to\lR$ which fulfil the condition \eqref{eq:f_presque_isom}. We have a bijection between $V$ and the Levi subalgebras of $\lG$: for any Levi subalgebra we associate the map $f\in V$ given by $X=f(X)+X_{\lS}$.

So our proof can be reduced to find a fixed point of $V$ under the action of $A$. In order to do that, we will see that $A$ is a group of \emph{affine} transformations on $V$. Consider a $\alpha\in A$ and $f_0$, $f_0^{\alpha}$, $f^{\alpha}$ be the elements of $V$ corresponding to $\lS_0$, $\lS$ and $\alpha(\lS)$. We take a $X\in\lS_0$ and we denote by $\oalpha(X)$ the $\lS_0$-component of $\alpha(X)$ with respect to the decomposition $\lG=\lR\oplus\lS_0$:
\[
   \alpha(X)=\oalpha(X)+\beta(X).
\]
This also defines $\dpt{\beta}{\lG}{\lR}$ and $-\beta(X)$ is the $\lR$-component of $\oalpha(X)$ with respect to $\lG=\lR\oplus\alpha(\lS_0)$. Since $f_0^{\alpha}$ just correspond to this decomposition, $f_0^{\alpha}(\oalpha(X))=-\beta(X)$, so that
\begin{equation}
\begin{split}
\oalpha(X)&=f_0^{\alpha}(\oalpha(X))+\alpha(X)\\
          &=f_0^{\alpha}(\oalpha(X))+\alpha(f(X))-\alpha(f(X))+\alpha(X).
\end{split}
\end{equation}
Since $X-f(X)\in\lS$, $\alpha(X)-\alpha(f(X))\in\alpha(\lS)$, then $f_0^{\alpha}(\oalpha(X))+\alpha(X)$ is the $\lR$-component of $\oalpha(X)$ with respect to $\lG=\lR\oplus\alpha(\lS)$. Then
\[
   f_0^{\alpha}(\oalpha(X))+\alpha(f(X))=f^{\alpha}(\oalpha(X))=f^{\alpha}(\oalpha(X)).
\]
Since $X$ was taken arbitrary, $f^{\alpha}=f^{\alpha}_0+\alpha\circ f\circ\oalpha^{-1}$. Then the map $V\to V$, $f\to f^{\alpha}$ is an affine transformation with translation equals to $f_0^{\alpha}$ and linear part being $f\to\alpha\circ f\circ\oalpha$.

A general result shows that a compact group of affine transformations on a vector space has a fixed point.

\end{proof}

An other result that will be used:
\begin{lemma}		\label{lem:Killing_ss_descent}
If $G$ is a semisimple Lie group and $H$ a semisimple subgroup of $G$, the restrictions on $H$ of the Killing form of $G$ is nondegenerate.
\end{lemma}
%TODO: a proof.

\subsection{Lorentz algebra}
%---------------------------

\begin{lemma}[\cite{Schomblond_em}]     \label{LemCommsopqAlg}
The matrices of $\so(p,q)$ satisfy the definition relation
\begin{equation}
    M^t\eta+\eta M=0,
\end{equation}
and if $M^{ab}$ is the ``rotation'' in the place of directions $a$ and $b$ (i.e. a trigonometric or an hyperbolic rotation following that $a$ and $b$ are of the same type or not), then the action on $\eR^{(p,q)}$ is given by $(x')^{\mu}=(M^{ab})^{\mu}_{\nu}x^{\nu}$ with
\begin{equation}
    (M^{ab})^{\mu}_{\nu}=\eta^{a\mu}\delta^b_{\nu}-\eta^{b\mu}\delta^a_{\nu}.
\end{equation}
The commutation relations are given by
\begin{equation}
    [M^{ab},M^{cd}]=-\eta^{ac}M^{bd}+\eta^{ad}M^{bc}+\eta^{bc}M^{ad}-\eta^{bd}M^{ac}.
\end{equation}
Notice that $M^{ab}=-M^{ba}$.
\end{lemma}

There is an other physical reason (which is in fact the same, but differently presented) justifying the study of the Clifford algebra. The quantum field theory need representation of the Lorentz algebra\footnote{When one think to real infinitesimal rotation matrices, the presence of $i$ seems not natural, but one redefines $J\to iJ$ for formalism reasons.}\index{lorentz!algebra}
\[
 [J^{\mu\nu},J^{\rho\sigma}]=i(\eta^{\nu\rho}J^{\mu\sigma}-\eta^{\mu\rho}J^{\nu\sigma}
 -\eta^{\nu\sigma}J^{\mu\rho}+\eta^{\mu\sigma}J^{\nu\rho}).
\]
Dirac had a trick to find such $J$ matrices from a representation of the Clifford algebra. If we have $n\times n$ matrices $\gamma_{\mu}$ such that
\[
    \gamma^{\mu}\gamma^{\nu}+\gamma^{\nu}\gamma^{\mu}=2\eta^{\mu\nu}\mtu_{n\times n},
\]
a $n$-dimensional representation of the Lorenz algebra is obtained by
\[
    S^{\mu\nu}=\frac{i}{4}\left[\gamma^{\mu},\gamma^{\nu}\right].
\]

By a simple redefinition $J=iM$, one obtains
\begin{equation}            \label{EqJJietaJcomm}
    [J,J]=i\eta J
\end{equation}
instead of $[M,M]=\eta M$, and the matrices $J$ are Hermitian. Here $\eta$ is the matrix 
\begin{equation}
\eta=diag(\underbrace{+,\ldots,+}_{p \text{times}},\underbrace{-,\ldots,-}_{\text{$q$ times}}). 
\end{equation}
As convention, we say that a direction corresponding to a \emph{positive} entry in the metric is a \emph{time} direction, while the spatial directions are negative. That corresponds to the convention of page \pageref{PgDefsGenre} to say that a \emph{time-like} vector has positive norm.

\section{Clifford algebra}
%++++++++++++++++++++++++++

\subsection{Definition and universal problem}
%------------------------------------------------------


\begin{theorem}\index{Clifford!algebra!Universal property of}
Let $E$ be an unital associative algebra and $\dpt{j}{V}{E}$ a linear map such that
\begin{equation}
    j(v)\cdot j(v)=q(v)1.        \label{102r1}
\end{equation}
Then we have an unique extension of $j$ to a homomorphism $\dpt{\tilde\jmath}{\Cliff(V,q)}{E}$. Moreover, $\Cliff(V,q)$ is the unique associative algebra which have this property for all such~$E$.
\[
\xymatrix{
    \Cliff(V,q) \ar@{^{(}->}[d]_{\displaystyle i} \ar[rd]^{\displaystyle\tilde{j}} &  \\
    V \ar[r]_{\displaystyle j} & D
  }
\]
\label{tho_Cliffunif}
\end{theorem}
This theorem can be seen as a definition of $\Cliff(V,q)$.

\begin{proof}
The proof shall belongs two parts: the first one will show how to extend $j$ and why it is unique, and the second one will prove the unicity of $\Cliff(V,q)$.

We begin by define the extension of $j$. First note that any linear map $\dpt{f}{V}{E}$ can be extended to an algebra homomorphism $\dpt{\overline{f}}{T(V)}{E}$ in only one way. Indeed, the homomorphism condition require that $\overline{f}(v\otimes w)=f(v)\cdot f(w)$.  The whole map $\overline{f}$ is then well defined by the data of $f$ alone.

As far as the map $j$ is concerned, we have the relation \eqref{102r1} which says that $\overline{j}(\mI)=0$. Indeed,
\begin{equation}
 \ovj(v\otimes v-q(v)\cdot(1))=\ovj(v)\cdot\ovj(v)-q(v)\ovj(1)
                              =j(v)\cdot j(v)-q(v)1
                              =0.
\end{equation}
Thus $\dpt{\ovj}{T(V)}{E}$ is a class map for $\mI$, and we can descent $\ovj$ from $T(V)$ to $\Cliff(V,q)$, We define
$\dpt{\tilde\jmath}{\Cliff(V,q)}{E}$ by
\begin{equation}
         \tilde\jmath[x]=\ovj(x)
\end{equation}
where $[x]$ is the class of $x$. That's for the existence part.

The unicity is clear: $f_1=f_2$ on $V$ implies that $\overline{f_1}=\overline{f_2}$ on $T(V)$. Thus $\tilde{f_1}=\tilde{f_2}$ on $\Cliff(V,q)$.

We turn now our attention to the unicity of $\Cliff(C,q)$. Let $D$ be an unital associative algebra such that
\begin{enumerate}
\item $V\subset D$,
\item For any unital associative algebra $E$ and for any $\dpt{f}{D}{E}$ such that $f(v)\cdot f(v)=-q(v)1$, there exists only one homomorphic map $\dpt{\tilde{f}}{D}{E}$ which extend $f$.
\end{enumerate}
We should find a homomorphic map $\dpt{\tilde{k}}{D}{\Cliff(V,q)}$. Let $i$ be the canonical injection $\dpt{i}{V}{D}$. Clearly, we have a homomorphism $V\rightarrow i(V)$. Now, as a space $E$, we can take $\Cliff(V,q)$; $i$ can be seen as a linear map $\dpt{i}{V}{\Cliff(V,q)}$ such that $i(v)\cdot i(v)=q(v)1$. The assumptions say that $i$ can be extended (in only one way) to a homomorphic map $\dpt{\tilde{i}}{D}{\Cliff(V,q)}$.

The Clifford algebra is thus unique up to a homomorphism.

\end{proof}

What we proved is the following: if for any $E$ and for any $\dpt{j}{V}{E}$ such that $j(v)\cdot j(v)=q(v)1$, there exist an unique $\dpt{\tilde{j}}{D}{E}$ which extend $j$, then $D=\Cliff(V,q)$ up to a homomorphism. One ays that $\Cliff(V,q)$ solve an \defe{universal problem}{universal!problem}.

%---------------------------------------------------------------------------------------------------------------------------
\subsection{Representations of the algebra \texorpdfstring{$\gsu(2)=\so(3)$}{su2so3}}
%---------------------------------------------------------------------------------------------------------------------------
\label{subsecPJmtqrG}

If one knows a load of theory, it is possible to determine the irreducible representations of \( \so(3)\) in a very short way; This will be done in example~\ref{ExHESKimc}. We are now going to determine the irreducible representations of \( \so(3)\) in a quite explicit way.

%///////////////////////////////////////////////////////////////////////////////////////////////////////////////////////////
\subsubsection{Ladder operators}
%///////////////////////////////////////////////////////////////////////////////////////////////////////////////////////////

The algebra $\gsu(2)$ is the real algebra generated by the matrices of the form
$
\begin{pmatrix}
\alpha  &\beta\\
-\beta^*&-\alpha
\end{pmatrix}
$ with $\alpha$, $\beta\in\eC$. A convenient basis is given by
\begin{align}       \label{EqGenssudeux}
u_1&=\frac{ 1 }{2}
\begin{pmatrix}
  i &   0   \\
  0 &   -i
\end{pmatrix},
&u_2&=
\frac{ 1 }{2}
\begin{pmatrix}
  0 &   1   \\
  -1    &   0
\end{pmatrix},
&u_3&=\frac{ 1 }{2}
\begin{pmatrix}
  0 &   i   \\
  i &   0
\end{pmatrix}.
\end{align}
That algebra satisfies the commutation relations
\begin{equation}
    [u_i,u_j]=\epsilon_{ijk}u_k.
\end{equation}
The trick to build finite dimensional representations\index{representation!of $\gsu(2)$} of that algebra is common (see \cite{MQSenechal} for example). The first step is to perform a change of basis $J_k=iu_k$ that brings the algebra under the form (see section~\ref{SubSecTheGroupSotrois} to understand why)
\begin{equation}        \label{EqAlgsuiepsijk}
    [J_i,J_j]=i\epsilon_{ijk}J_k.
\end{equation}
We are going to construct all the finite dimensional irreducible representations of the algebra \eqref{EqAlgsuiepsijk}. The key point of that new basis is that one can define the \defe{ladder operators}{ladder operators}
\begin{equation}
    J_{\pm}=J_1\pm iJ_2
\end{equation}
that have the property that
\begin{equation}
    [J_3,J_{\pm}]=\pm J_{\pm}.
\end{equation}
Notice that for every $i$, we have $(J_i)^*=J_i$, so that $(L^{\pm})^*=L^{\mp}$. An other important property is that, defining $J^2=J_1^2+J_2^2+J_3^2$, we have
\begin{equation}
    [J_i,J^2]=0,
\end{equation}
which show that $J^2$ is a Casimir operator, and is thus by Schur's lemma a multiple of identity. Notice that we are using an abuse of notation between $J_i$ as element of $\gsu(2)$ and $J_i$ as the operator that represent $J_i$. In the first case, products like $J_iJ_j$ make no sense\footnote{In fact, one has to understand these products as elements of the universal enveloping algebra. What we are building is a reprensentation of that algebra, which, obviously, restricts to a representation of the algebra. When we use the Schur's lemma, in fact we invoke it in $\mU\big(\so(3)\big)$}, but it makes sense as operator composition.

The subalgebra $\{ J^2,J_3 \}$ being abelian, we can simultaneously diagonalise $J^2$ and $J_3$. Let $| m,\sigma \rangle $ be an orthonormal basis of the eigenspace of $J_3$ associated with the eigenvalue $m$. The index $\sigma$ is for a possible degenerateness to be studied later.  We have
\[
    J_3| m,\sigma \rangle =m| m,\sigma \rangle .
\]
Using the commutation relations between $J_3$ and the ladder operators, we have
\begin{equation}        \label{EqJtroisJpmmplusun}
    J_3J_{\pm}| m,\sigma \rangle =\big( \pm J_{\pm}+J_{\pm}J_3 \big)| m,\sigma \rangle =(m\pm 1)J_{\pm}| m,\sigma \rangle .
\end{equation}
Thus $J_{\pm}| m,\sigma \rangle $ is an eigenvector of $J_3$ with the eigenvalue $m\pm 1$, which means that $J_{\pm}| m,\sigma \rangle $ is a linear combination of the vectors $| m\pm 1,\sigma \rangle $ with different values of $\sigma$. This is the reason of the name of the \emph{ladder} operators: they raise and lower the eigenvalue of $J_3$.

We can now prove that one has to drop the index $\sigma$ because eigenvalues of $J_3$ cannot be degenerated. For, compute
\begin{equation}        \label{JpJmJcarrerelation}
    J_+J_-=(J_1+iJ_2)(J_1-iJ_2)=J^2-J_3^2+i[J_2,J_1]=J^2-J_3^2+J_3,
\end{equation}
so that
\[
    J_+J_-| m,\sigma \rangle =(\alpha-m^2+m)| m,\sigma \rangle
\]
where $\alpha$ is defined by $J^2=\alpha\mtu$. That proves that the space generated by $| m,\sigma \rangle $ and the action of $J_3$, $J_+$ and $J_-$ is invariant under the representation, while one cannot obtain $\ket{m,\sigma'}$ by action of $J_{\pm}$ on $\ket{m,\sigma}$. Since we are looking for \emph{irreducible} representations, that space must actually be all the representation space. That rules out the possibility to have two different vectors $| m,\sigma_1 \rangle $ and $| m,\sigma_2 \rangle $.

The explicit matrix form of $J_{\pm}$ are:
\begin{align}
J_{+}&=
\begin{pmatrix}
0   &   0   &   0   &0  &\hdots\\
1   &   0   &   0   &0  &\hdots\\
0   &   1   &   0   &0  &\hdots\\
0   &   0   &   1   &0  &\hdots\\
\vdots  &   \vdots  &   \vdots  &\vdots &\ddots
\end{pmatrix},
&J_{-}&=
\begin{pmatrix}
0   &   1   &   0   &0  &\hdots\\
0   &   0   &   1   &0  &\hdots\\
0   &   0   &   0   &0  &\hdots\\
0   &   0   &   0   &1  &\hdots\\
\vdots  &   \vdots  &   \vdots  &\vdots &\ddots
\end{pmatrix},
\end{align}
Since we are searching for finite dimensional representations, there exists a maximal eigenvalue of $J_3$. Let us denote by $j$ that maximal eigenvalue and by $| j \rangle$ the corresponding eigenvector. The relation \eqref{EqJtroisJpmmplusun} shows that if $J_+| j \rangle\neq 0$, then $J_+| j \rangle$ is an eigenvector for $J_3$ with eigenvalue $j+1$, which contradicts maximality. Then we have $J_+| j \rangle=0$.

Since we know the action of $J_3$ and $J_+$ on $| j \rangle$, it is convenient to write $J^2$ in terms of these two operators. This is done in the same way as probing equation \eqref{JpJmJcarrerelation}:
\begin{equation}
    J^2=J_3^2+J_3+J_-J_+,
\end{equation}
so that
\begin{equation}        \label{EqJcarrejjplusun}
    J^2| j \rangle=j(j+1)| j \rangle.
\end{equation}
We know that $J^2=\alpha\mtu$ and that $\alpha$ is a characteristic of the representation. What equation \eqref{EqJcarrejjplusun} tells us is that the maximal eigenvalue of $J_3$ is related to $\alpha$ by $j(j+1)=\alpha$.

We are now able to determine the proportionality constant of relation $J_{\pm}| m \rangle\propto| m\pm 1 \rangle$. Since $(J_-)^*=J_+$, we have
\begin{equation}    \label{EqnormeJmoinm}
    \| J_-| m \rangle \|^2=\langle m| J_+J_- | m \rangle = j(j+1)-m^2+m.
\end{equation}
Then one has
\begin{subequations}
    \begin{align}
        J_-| m \rangle  &=\sqrt{j(j+1)-m(m-1)}| m-1 \rangle,    \label{EqJmoinsmanglemmointun}      \\
        J_+| m \rangle  &=\sqrt{j(j+1)-m(m+1)}| m+1 \rangle.
    \end{align}
\end{subequations}
As expected, $J_-| -j \rangle=0$ and $J_+| j \rangle=0$. Notice that we avoid the possibility $J_-| m \rangle=-\sqrt{\cdots}| m-1 \rangle$ by a simple redefinition $| m-1 \rangle\to -| m-1 \rangle$.

Equation \eqref{EqnormeJmoinm} shows that the norm of $| m \rangle$ becomes negative for $m<-j$ and $m>j+1$. We conclude that the minimal eigenvalue of $J_3$ is $-j$. Since $| j \rangle$ has to be reached from $| -j \rangle$ by action of $J_+$, the difference $j-(-j)$ must be an integer. Thus $j\in\eN/2$. The number $j$ is the \defe{spin}{spin!of representation of $\so(3)$} of the representation.

Let us give the explicit example with spin one half.
When $j=\frac{ 1 }{2}$, the vector space is generated by the vectors $| 1/2 \rangle$ and $| -1/2 \rangle$, and the operators are given by
\begin{align}
    J_3&=\frac{ 1 }{2}
\begin{pmatrix}
  1 &   0   \\
  0 &   -1
\end{pmatrix},
&J_-&=
\begin{pmatrix}
  0 &   0   \\
  1 &   0
\end{pmatrix},
&J_+&=
\begin{pmatrix}
  0 &   1   \\
  0 &   0
\end{pmatrix},
\end{align}
from which we deduce
\begin{align*}
J_1&=\frac{ 1 }{2}
\begin{pmatrix}
  0 &   1   \\
  1 &   0
\end{pmatrix},
&J_2&=
\begin{pmatrix}
  0 &   -i  \\
  i &   0
\end{pmatrix}.
\end{align*}
Notice that we have $J_i=\frac{ 1 }{2}\sigma_i$ with the \defe{Pauli matrices}{pauli matrices},
\begin{align}
\sigma_1&=
\begin{pmatrix}
  0 &   1   \\
  1 &   0
\end{pmatrix},
&\sigma_2&=
\begin{pmatrix}
  0 &   -i  \\
  i &   0
\end{pmatrix},
&\sigma_3&=
\begin{pmatrix}
  1 &   0   \\
  0 &   -1
\end{pmatrix}.
\end{align}
These matrices fulfil the relation
\begin{equation}
    \sigma_i\sigma_j=\delta_{ij}+i\epsilon_{ijk}\sigma_k.
\end{equation}

%---------------------------------------------------------------------------------------------------------------------------
\subsubsection{Weight vectors}  \label{subSubSecweightsotrois}
%---------------------------------------------------------------------------------------------------------------------------

The algebra $\so(3)$ does not contain abelian subalgebra of dimension bigger than one, so a Cartan subalgebra is generated by $J_3$. The unique (up to dilatation) element of $\hH^*$ is thus given by $\alpha(J_3)=1$. The relation $[J_z,J_{\pm}]$ provides the root spaces:
\begin{equation}
    \begin{aligned}
        \so(3)_1    &=\{ J_+ \}\\
        \so(3)_{-1} &=\{ J_- \},
    \end{aligned}
\end{equation}
thus $\lN^{\pm}$ is generated by $J_{\pm}$.


%+++++++++++++++++++++++++++++++++++++++++++++++++++++++++++++++++++++++++++++++++++++++++++++++++++++++++++++++++++++++++++
\section{Cartan subalgebras in complex Lie algebras}
%+++++++++++++++++++++++++++++++++++++++++++++++++++++++++++++++++++++++++++++++++++++++++++++++++++++++++++++++++++++++++++
\label{SecCartaninComplex}
About Cartan algebra, one can read \cite{Dragan,Berndt,Hochschild,SamelsonNotesLieAlg}.

In this section $\lG$ will always denotes a complex finite dimensional Lie algebra.

\begin{definition}\label{PgDefCentralisateur}
    When \( \lH\) is a subalgebra of \( \lG\), the \defe{centralizer}{centralizer} of \( \lH\) is the set\nomenclature[G]{$\mZ(\lH)$}{the centralizer of \( \lH\)}
    \begin{equation}
        \mZ(\lH)=\{x\in\lG\tq [x,\lH]\subset\lH\}.
    \end{equation}
    More generally if $\lG$ is a Lie algebra and if $\lA$, $\lB$ are two subset of $\lG$, the centraliser of $\lA$ in $\lB$ is
    \begin{equation}
        \mZ_{\lB}(\lA)=\{X\in\lB\tq [X,\lA]=0\}.
    \end{equation}
    If $\lA$ is a subalgebra of $\lG$, its \defe{normalizer}{normalizer} is
    \begin{equation}
        \lN_{\lA}=\{X\in\lG\tq [X,\lA]\subset\lA\}.
    \end{equation}
    One can check that $\lA$ is an ideal in $\lN_{\lA}$.
\end{definition}

\begin{definition}
A subalgebra $\lH$ of a Lie algebra $\lG$ is a \defe{Cartan subalgebra}{Cartan!subalgebra} if it is nilpotent and if it is its own centralizer: $[x,\lH]\subset\lH$ implies $x\in\lH$.
\end{definition}

Our first task is to show that every Lie algebra has a Cartan algebra.

\begin{lemma}[Primary decomposition theorem]
    Let \( V\) be a complex vector space and \( A\colon V\to V\) be linear map. Then we have the direct sum decomposition
    \begin{equation}        \label{EqPrimDecomTho}
        V=\bigoplus_{\lambda\in\eC}V_{\lambda}(A)
    \end{equation}
    where \( V_{\lambda}(A)=\{ v\tq (A-\lambda\mtu)^nv=0\text{ for some } n\in\eN \} \)
\end{lemma}
This is the result that restricts ourself to \emph{complex} Lie algebras when proving that Cartan subalgebras exist. Notice that the sum in \eqref{EqPrimDecomTho} is reduced to the eigenvalues of \( A\) since \( \lG_{\lambda}(A)=0\) when \( \lambda\) is not an eigenvalue. Indeed if \( \big( A-\lambda\mtu \big)^nY=0\) then \( (A-\lambda\mtu)^{n-1}Y\) is an eigenvector for \( A\) with eigenvalue \( \lambda\).

For any \( \lambda\in\eC\) and \( X\in\lG\) we consider the space
\begin{equation}
    \lG_{\lambda}(X)=\{ Y\in\lG\tq \big( \ad(X)-\lambda\mtu \big)^nY=0\text{ for some } n \}.
\end{equation}
The primary decomposition theorem implies the decomposition
\begin{equation}        \label{EqDecomplGpRimDecombijk}
    \lG=\bigoplus_{\lambda}\lG_{\lambda}(X)
\end{equation}
for each \( X\in\lG\).

A small useful formula: if \( u\) is a derivation of the Lie bracket and if \( [X,Y]\) is any bracket, then
\begin{equation}\label{EqWGujmeF}
    (u-\lambda\mtu)[X,Y]=\big[ (u-\lambda\mtu)X,Y \big]+[X,uY].
\end{equation}

\begin{lemma}   \label{LemVZzSnUW}
    For each \( X\in\lG\) and \( \lambda,\mu\in\eC\) we have
    \begin{equation}
        \big[ \lG_{\lambda}(X),\lG_{\mu}(X) \big]\subset\lG_{\lambda+\mu}(X).
    \end{equation}
\end{lemma}

\begin{proof}
    Let \( X_{\lambda}\in\lG_{\lambda}(X)\) and \( X_{\mu}\in\lG_{\mu}(X)\). Using the fact that \( \ad(X)\) is a derivation we have
    \begin{equation}
        \ad(X)[X_{\lambda},X_{\mu}]-(\lambda+\mu)[X_{\lambda},X_{\mu}]=\Big[ \big( \ad(X)-\mu\mtu \big)X_{\lambda},X_{\mu} \Big]+\Big[ X_{\lambda},\big( \ad(X)+\mu\mtu \big)X_{\mu} \Big]
    \end{equation}
    Let us show by induction the following equality for all \( n\):
    \begin{equation}    \label{EqPIzsRhb}
        \big( \ad(X)-(\lambda+\mu)\mtu \big)^n[X_{\lambda},X_{\mu}]=\sum_{i=0}^{\infty}\binom{ n }{ i }\Big[ \big( \ad(X)-\lambda\mtu \big)^iX_{\lambda},\big( \ad(X)-\mu\mtu \big)^{n-i}X_{\mu} \Big].
    \end{equation}
    In order to prove that, it is sufficient to apply \( \big( \ad(X)-(\lambda+\mu)\mtu \big)\) to that equality and use the fact that \( \ad(X)\) is a derivation of the Lie bracket. Then apply formula \eqref{EqWGujmeF}.

The expression \eqref{EqPIzsRhb} vanishes when \( n\) is large enough.
\end{proof}

We say that \( X\) is \defe{regular}{regular} if \( \dim\lG_0(X)\) is the smallest with respect to the others \( \dim\lG_0(Y)\).

The following proposition shows that every complex Lie algebra has a Cartan Lie subalgebra.
\begin{proposition}
    If $X$ is regular in \( \lG\) then the subalgebra \( \lG_0(X)\) is Cartan.
\end{proposition}

\begin{proof}
    Since \( X\in\lG_0(X)\) we have \( \ad(X)\lG_{\lambda}(X)\subset\lG_{\lambda}(X)\). Thus we see \( \ad(X)\) as a linear operator on \( \lG_{\lambda}(X)\). The operator \( \ad(X)|_{\lG_{\lambda}(X)}\) is nonsingular\footnote{it means that \( \ad(Y)\) is invertible.} when \( \lambda\neq 0\). Indeed all the eigenvalues of \( \ad(X)\) on \( \lG_{\lambda}(X)\) are equal to \( \lambda\) because
    \begin{equation}
        \big( \ad(X)-\mu\mtu \big)Y=0
    \end{equation}
    implies \( Y\in\lG_{\mu}(X)\). If \( Y\in\lG_{\lambda}(X)\) it only occurs when \( \mu=\lambda\) since the sum \eqref{EqPrimDecomTho} is direct.

    For each eigenvalue \( \lambda\) we have a neighborhood \( \mU_{\lambda}\) of $X$ in \( \lG_0(X)\) such that for all \( Y\in\mU_{\lambda}\), \( \ad(Y)\) is nonsingular on \( \lG_{\lambda}(X)\). We consider \( \mU=\bigcap_{\lambda}\mU_{\lambda}\) which is a non empty open set since the intersection is taken over the eigenvalues of \( \ad(X)\) that are in finite numbers.

    Let us prove that the restriction to \( \lG_0(X)\) of the linear operator \( \ad(Y)\) is nilpotent for each \( Y\in\mU\). First we have
    \begin{equation}        \label{EqLgzsubsetlzY}
        \lG_0(Y)\subseteq\lG_0(X)
    \end{equation}
    because by construction \( \ad(Y)\) cannot be nilpotent on the other spaces \( \lG_{\lambda}(X)\). But by hypothesis the element \( X\) is regular, thus the inclusion \eqref{EqLgzsubsetlzY} cannot be strict. Thus \( \lG_0(X)\subset\lG_0(Y)\) which means that \( \ad(Y)\) is nilpotent on \( \lG_0(X)\).

    Now the fact for \( \ad(Y)\) to be nilpotent means the vanishing of a polynomial determined by the coefficients of the matrix of \( \ad(Y)\). Since this polynomial vanishes on the open set \( \mU\), it vanishes identically, so that \( \ad(Y)\) is nilpotent on \( \lG_0(X)\). It results that \( \lG_0(X)\) is a \( \ad\)-nilpotent algebra and the Engel's theorem~\ref{tho:Engel} concludes that \( \lG_0(X)\) is nilpotent.

    We still have to prove that \( \lG_0(X)\) is its own centralizer. Since \( \lG_0(X)\) is a subalgebra we have the inclusion
    \begin{equation}
        \lG_0(X)\subseteq\mZ\big( \lG_0(X) \big).
    \end{equation}
    Let \( Z\in\mZ\big( \lG_0(X) \big)\). For each \( Y\in\lG_0(X)\) we have \( [Z,Y]\in\lG_0(X)\). In particular with \( Y=X\) we have \( \ad(X)Z\in\lG_0(X)\). Thus
    \begin{equation}
        \ad(X)^nZ=\ad(X)^{n-1}\underbrace{\ad(X)Z}_{\in\lG_0(X)}
    \end{equation}
    and there exists a \( n\) such that \( \ad(X)^{n-1}\ad(X)Z=0\).
\end{proof}

If \( \lG\) is a Lie algebra, the group of \defe{inner automorphism}{inner!automorphism} is the subgroup of \( \Aut(\lG)\) generated by the elements of the form \(  e^{\ad(X)}\) with \( X\in\lG\). This definition is motivated in the context of matrix groups by the fact that when \( g= e^{Y}\in G\) and \( X\in\lG\) we have
\begin{equation}
    gXg^{-1}= e^{\ad(Y)}X.
\end{equation}
\begin{example}
    If
    \begin{equation}
        \begin{aligned}[]
            g=\begin{pmatrix}
                \cos(t)    &   \sin(t)    &   0    \\
                -\sin(t)    &   \cos(t)    &   0    \\
                0    &   0    &   1
            \end{pmatrix},&&X=\begin{pmatrix}
                0    &   a    &   b    \\
                -a    &   0    &   0    \\
                -b    &   0    &   0
            \end{pmatrix},
        \end{aligned}
    \end{equation}
    then one checks that \( g= e^{Y}\) with
    \begin{equation}
        Y=\begin{pmatrix}
              0  &  t     &   0    \\
            -t    &   0    &   0    \\
            0    &   0    &   0
        \end{pmatrix}
    \end{equation}
    and
    \begin{equation}
        gXg^{-1}= e^{\ad(Y)}X=\begin{pmatrix}
            0    &   a    &   b\cos(t)    \\
            -a    &   0    &   -b\sin(t)    \\
            -b\cos(t)    &   b\sin(t)    &   0
        \end{pmatrix}.
    \end{equation}
\end{example}

\begin{theorem}
    The group of inner automorphisms of \( \lG\) acts transitively on the set of Cartan subalgebras.
\end{theorem}

For a proof, see \cite{SerreSSAlgebres}. In particular they have all the same dimension and the definition of the \defe{rank}{rank!of a complex Lie algebra} as the dimension of its Cartan algebra make sense. In \cite{SerreSSAlgebres} we have a more abstract definition of the rank, see page III-2.

\begin{proposition}     \label{PropCartanLzXtjs}
    If \( \lH\) is a Cartan subalgebra of the complex Lie algebra \( \lG\), there exists a regular element \( X\in\lG\) such that \( \lH=\lG_0(X)\).
\end{proposition}

For a proof, see \cite{SerreSSAlgebres}.

\begin{proposition}\label{prop:Cartan_max_nil}
A Cartan subalgebra is a maximal nilpotent subalgebra.
\end{proposition}

\begin{proof}
Let $\lH$ be a Cartan subalgebra of $\lG$ and $\lN$, a nilpotent algebra which contains $\lH$. Let $\{X_1,\ldots,X_n\}$ be a basis of $\lG$ chosen in such a way that the $p$ first vectors form a basis of $\lH$ while the $r$ first, a basis of $\lN$ ($r>p$ of course). As notational convention, the subscript $i,j$ are related to $\lH$ and $u,t$ to $\lN\ominus\lH$.

Let us first suppose $\dim\lN=\dim\lH+1$ and let $X_u$ be the basis vector of $\lN$ which is not in $\lH$. Since $\lH$ is Cartan, we can find $X_i\in\lH$ such that $Y=[X_u,X_i]\notin\lH$. Then $Y$ has a $X_u$-component and this contradict the fact that $\ad X_i$ is nilpotent.

The next case is $\lN=\lH\oplus X_u\oplus X_t$. In this case we can find a $X_i\in\lH$ such that $Y=[X_u,X_i]\notin\lH$. The fact to be nilpotent makes that $Y$ has no $X_u$-component, so that it has a $X_t$-component. Now it is clear that for any $X_j\in\lH$, $[Y,X_j]$ still has no $X_u$-component (because $(\ad X_i\circ\ad X_j)$ has to be nilpotent), but has also no $X_t$-component. Then for any $X\in\lH$, $[Y,X]\in\lH$ with $Y\notin\lH$. There is a contradiction.

Now the step to the general case is easy: if $\dim\lN=\dim\lH+m$, we  consider $X_1,\ldots,X_m\in\lH$ and $A=(\ad X_1\circ\ad X_m)X_u$. This is not in $\lH$ although $[A,X]\in\lH$ for any $X\in\lH$.
\end{proof}


\begin{proposition}
    If \( \lG\) is a semisimple Lie algebra, a subalgebra \( \lH\) is Cartan if and only if the two following conditions are satisfied:
    \begin{enumerate}
        \item
            \( \lH\) is a maximal abelian subalgebra
        \item
            the endomorphism \( \ad(H)\) is diagonalizable for every \( H\in\lH\).
    \end{enumerate}
\end{proposition}

%+++++++++++++++++++++++++++++++++++++++++++++++++++++++++++++++++++++++++++++++++++++++++++++++++++++++++++++++++++++++++++
\section{Root spaces in semisimple complex Lie algebras}
%+++++++++++++++++++++++++++++++++++++++++++++++++++++++++++++++++++++++++++++++++++++++++++++++++++++++++++++++++++++++++++
\label{SecRootcomplexss}
In this section we particularize ourself to complex semisimple Lie algebras. A very good reference about complex semisimple algebras including the reconstruction \emph{via} the Cartan matrix and Chevalley-Weyl basis is \cite{SerreSSAlgebres}.

%---------------------------------------------------------------------------------------------------------------------------
\subsection{Introduction and notations}
%---------------------------------------------------------------------------------------------------------------------------

Real and complex Lia algebras deserve quite different treatment with root space. We review here the main steps in both cases, emphasising the differences. We restrict ourself to semisimple Lie algebras. See \cite{Wisser}.

%///////////////////////////////////////////////////////////////////////////////////////////////////////////////////////////
\subsubsection{Complex Lie algebras}
%///////////////////////////////////////////////////////////////////////////////////////////////////////////////////////////

If \( \lG\) is a complex semisimple Lie algebra, we choose a Cartan subalgebra \( \lH\) and the root spaces are given by
\begin{equation}
    \lG_{\alpha}=\{ X\in\lG\tq [H,X]=\alpha(H)X \forall H\in\lH \}.
\end{equation}
The dimension of \( \lH\) is the rank of \( \lG\). Then the root space decomposition reads
\begin{equation}
    \lG=\lH\oplus\bigoplus_{\alpha\in\Phi}\lG_{\alpha}
\end{equation}
where \( \Phi\) is the set of roots.

%///////////////////////////////////////////////////////////////////////////////////////////////////////////////////////////
\subsubsection{Real Lie algebras}
%///////////////////////////////////////////////////////////////////////////////////////////////////////////////////////////

If \( \lG\) is a real semisimple Lie algebra we consider a Cartan involution and the Cartan decomposition \( \lG=\lK\oplus\lP\). Then we choose a maximally abelian subalgebra \( \lA\) in \( \lP\) and we define
\begin{equation}
    \lG_{\lambda}=\{ X\in\lG\tq [J,X]=\alpha(J)X \forall J\in\lA \}.
\end{equation}
The rank of \( \lG\) is the dimension of \( \lA\). The root space decomposition then reads
\begin{equation}
    \lG=\lG_0\oplus\bigoplus_{\lambda\in\Sigma}\lG_{\lambda}
\end{equation}
where \( \Sigma\) is the set of \( \lambda\in\lA^*\) such that \( \lambda\neq 0\) and \( \lG_{\lambda}\neq 0\).

%///////////////////////////////////////////////////////////////////////////////////////////////////////////////////////////
\subsubsection{Notations}
%///////////////////////////////////////////////////////////////////////////////////////////////////////////////////////////
\label{SubsecNotationRootsDel}

We summarize the notations that will be used later. Let \( \lH\) be a Cartan algebra in the complex semisimple Lie algebra \( \lG\). An element \( \alpha\in\lH^*\) is a root if the space
\begin{equation}
    \lG_{\alpha}=\{ X\in\lG\tq \ad(H)X=\alpha(H)x,\forall H\in\lH \}
\end{equation}
is non empty.

\begin{enumerate}
    \item
        \( \Phi\) is the set of all the roots. We consider an ordering notion on \( \Phi\) and \( \Phi^+=\Pi\) is the set of positive roots.
    \item
        An element in \( \Phi^+\) is simple if it cannot be written as the sum of two positive roots.
    \item
        \( \Delta\) is the set of simple roots\footnote{The symbol \( \Delta\) has not a fixed signification in the literature. As example, in \cite{Cornwell} the symbol \( \Delta\) is the set of roots while in \cite{SternLieAlgebra} it denotes the set of simple roots.}. The simple roots are denoted by \( \{ \alpha_1,\ldots,\alpha_l \}\).
\end{enumerate}

%---------------------------------------------------------------------------------------------------------------------------
\subsection{Root spaces}
%---------------------------------------------------------------------------------------------------------------------------

We are considering a complex semisimple Lie algebra \( \lG\) with a Cartan subalgebra \( \lH\).

\begin{definition}      \label{DefRootSpace}
    For each \( \alpha\in\lH^*\) we define
    \begin{equation}            \label{eq:lG_alpha_nss}
        \lG_{\alpha}=\{  x\in\lG\tq\forall h\in\lH, \big(\ad h-\alpha(h)\big)^nx=0\text{ for some }n\in\eN    \}.
    \end{equation}
    If \( \lG_{\alpha}\) is not reduced to \( 0\), we say that \( \alpha\) is a \defe{root}{root} and \( \lG_{\alpha}\) is a \defe{root space}{root!space}.
\end{definition}
Corollary~\ref{CorCoolWrlGbalpha} will provide an easier formula for the root spaces when the algebra \( \lG\) is complex and semisimple.
\begin{theorem}     \label{TholGCartalphaplusbeta}
Let $\lG$ be a complex Lie algebra with Cartan subalgebra $\lH$. If $\alpha,\beta\in\lG^*$ then
\begin{enumerate}
    \item   \label{ItemTholGCartalphaplusbetai}
    $[\lG_{\alpha},\lG_{\beta}]=\lG_{\alpha+\beta}$,
\item $\lG_0=\lH$.
\end{enumerate}
\label{prop:deux_racine}
\end{theorem}

\begin{proof}
For $z\in\lH$ and $x$, $y\in\lG$ we have
\begin{equation}
\big( \ad z-(\alpha+\beta)(z) \big)[x,y]=[ (\ad z-\alpha(z))x,y ]+[ x,(\ad z-\beta(z))y ].
\end{equation}
Using the same induction as in the proof of lemma~\ref{LemVZzSnUW} we show that
\begin{equation}
\big( \ad z-(\alpha+\beta)(z) \big)^n[x,y]=\sum_k \binom{k}{n}
\left[
(\ad z-\alpha(z))^k(x),(\ad z-\beta(z))^{n-k}(y)
\right].
\end{equation}
This formula shows that $[\lG_{\alpha},\lG_{\beta}]\subset\lG_{\alpha+\beta}$. Indeed let $x\in\lG_{\alpha}$, $y\in\lG_{\beta}$ and $n$ be large enough,
\begin{equation}
    \left( \ad z-(\alpha+\beta)(z) \right)^n[x,y]=0.
\end{equation}

    Now we turn our attention to the second part. Let us apply the Lie theorem~\ref{tho:Lie_Vu} to the action of \( \lG\) on the quotient \( \lG_0/\lH\). There exists \( [X_0]\in\lG_0/\lH\) such that \( h[X_0]=\lambda(h)[X_0]\) where the bracket stand for the class. Since \( \lH\) is nilpotent on \( \lG_0\) we have \( \lambda=0\) identically. Looking outside the class, the existence of a non vanishing \( [X_0]\in\lG/\lH\) such that \( h[X_0]=0\) means that there exists \( X_0\in\lG_0\setminus\lH\) such that \( [h,X_0]\in\lH\) for every \( h\in\lH\). This contradicts the fact that \( \lH\) is its own centralizer.
\end{proof}

\begin{proposition}
    The complex Lie algebra decomposes into the root spaces as
    \begin{equation}
        \lG=\bigoplus_{\alpha\in\lH^*}\lG_{\alpha}.
    \end{equation}
\end{proposition}

\begin{proof}
    Let \( H\in\lH\). We consider the primary decomposition \eqref{EqDecomplGpRimDecombijk} with respect to the operator \( \ad(H)\):
    \begin{equation}
        \lG=\bigoplus_{\lambda}\lG_{\lambda}(H).
    \end{equation}
    If \( H'\in\lH\) the operator \( \ad(H')\) acts the space \( \lG_{\lambda}(H)\) because \( H'\in\lG_0(H)\) so that
    \begin{equation}
        [H',\lG_{\lambda}(H)]\subset\lG_{\lambda}(H).
    \end{equation}
    Thus we can write the primary decomposition of \( \lG_{\lambda}(H)\) with respect to the operator \( \ad(H')\) knowing that
    \begin{equation}
        \big( \lG_{\lambda}(H) \big)_{\mu}(H')=\{ X\in\lG_{\lambda}(H)\tq\big( \ad(H')-\mu \big)^nX=0 \}=\lG_{\lambda}(H)\cap\lG_{\mu}(H').
    \end{equation}
    What we get is the decomposition
    \begin{equation}
        \lG=\bigoplus_{\lambda}\bigoplus_{\mu}\lG_{\lambda}(H)\cap\lG_{\mu}(H').
    \end{equation}
    We continue the decomposition with \( H'',H''',\ldots\) until each \( \ad(H)\) with \( H\in\lH\) has only one eigenvalue on each of the summand of the decomposition
    \begin{equation}
        \lG=\bigoplus_{\lambda_1,\ldots,\lambda_l}\lG_{\lambda_1}(H_1)\cap\ldots\cap\lG_{\lambda_l}(H_l).
    \end{equation}
    For each \( l\)-uple \( (\lambda_1,\ldots,\lambda_l)\), the eigenvalue of \( H_i\) on \( \lG_{\lambda_1}\cap\ldots\cap\lG_{\lambda_l}\) is \( \lambda_i\). Thus we can see \( \lambda\) as a \( 1\)-form on \( \lH\) and write
    \begin{equation}        \label{EqdirectumlGRoots}
        \lG=\bigoplus_{\lambda}\lG_{\lambda}
    \end{equation}
    with
    \begin{equation}
        \lG_{\lambda}=\{ X\in\lG\tq\big( \ad(H)-\lambda(H) \big)^nX=0 \}.
    \end{equation}
\end{proof}

\begin{corollary}\label{cor:Bxy_zero}
    If $X_{\alpha}\in\lG_{\alpha}$ and $X_{\beta}\in\lG_{\beta}$ with $\alpha+\beta\neq 0$, then $B(X_{\alpha},X_{\beta})=0$.
\end{corollary}

\begin{proof}
    From the second point of proposition~\ref{prop:deux_racine}, we have $\dpt{\ad X_{\alpha}\circ\ad X_{\beta}}{\lG_{\mu}}{\lG_{\mu+\alpha+\beta}}$. If $\alpha+\beta\neq 0$, the fact that the sum \eqref{EqdirectumlGRoots} is direct makes the trace of $\ad X_{\alpha}\circ\ad X_{\beta}$ zero.
\end{proof}


Since \( \lG\) is semisimple, the restriction of the Killing form on \( \lH\) is nondegenerate\footnote{Because the Killing form is zero on each space \( \lG_{\alpha}\) with \( \alpha\neq 0\).}. Thus we can introduce, for each linear function $\phi\colon \lH\to \eC$, the unique element $t_{\phi}\in\lH$ such that
\begin{equation}
    \phi(h)=B(t_{\phi},h)
\end{equation}
for every $h\in\lH$. \nomenclature[G]{\( t_{\alpha}\)}{a basis of \( \lH\)} This element is nothing else that the dual \( \phi^*\) with respect to the Killing form. Indeed
\begin{equation}
    t_{\phi}^*(h)=B(t_{\phi},h)=\phi(h),
\end{equation}
so that \( t_{\phi}^*=\phi\). Incidentally, this proves that when \( \phi\) runs over a basis of \( \lH^*\), the vector \( t_{\phi}\) runs over a basis of \( \lH\). The space $\lH^*$ is endowed with an inner product defined by\nomenclature{$(\alpha,\beta)$}{Inner product on the dual $\lH^*$ of a Cartan algebra}\nomenclature[G]{\( (\alpha,\beta)\)}{inner product on the dual \( \lH^*\).}
\begin{equation}        \label{EqDefInnprHestrar}
    (\alpha,\beta) = B(t_{\alpha},t_{\beta})=\beta(t_{\alpha})=\alpha(t_{\beta}).
\end{equation}

\begin{lemma}       \label{LemXYBXYtalpha}\label{Propoxalphaymoinaalpha}
    If \( X\in\lG_{\alpha}\) and \( Y\in\lG_{-\alpha}\), then
    \begin{equation}
        [X,Y]=B(X,Y)t_{\alpha}.
    \end{equation}
\end{lemma}

\begin{proof}
    By theorem~\ref{TholGCartalphaplusbeta}\ref{ItemTholGCartalphaplusbetai}, $[X,Y]\in\lG_0=\lH$. Now we consider $h\in\lH$ and the invariance formula \eqref{eq:Killing_invariant}. We find:
    \begin{equation}
        B\big( h,[X,Y] \big)=-B\big( [X,h],Y \big)=\alpha(h)B(X,Y)=B(h,t_{\alpha})B(X,Y)=B\big(h,B(X,Y)t_{\alpha}\big).
    \end{equation}
    The lemma is proven since it is true for any $h\in\lH$ and $B$ is nondegenerate on $\lH$.
\end{proof}

The elements \( t_{\alpha}\) allow to introduce an inner product on \( \lH^*\) and hence on the roots by defining
\begin{equation}
    (\alpha,\beta)=B(t_{\alpha},t_{\beta}).
\end{equation}

\begin{lemma}       \label{Leminnerabequaaggb}
    If \( \alpha\) and \( \beta\) are roots we have the formula
    \begin{equation}
        (\alpha,\beta)=\sum_{\gamma\in\Phi}(\dim\lG_{\gamma})(\alpha,\gamma)(\beta,\gamma).
    \end{equation}
\end{lemma}

\begin{proof}
    We consider for \( \lG\) a basis in which all the elements are part of one of the root spaces and we look at the endomorphism \( \ad(t_{\alpha})\) of \( \lG\). This is diagonal and has zeros on the entries corresponding to \( \lH\). The other entries on the diagonal are of the form \( \gamma(t_{\alpha})\). Thus
    \begin{equation}
        B(t_{\alpha},t_{\beta})=\sum_{\gamma\in\Phi}(\dim\lG_{\gamma})\gamma(t_{\alpha})\gamma(t_{\beta}).
    \end{equation}
    Thus we have \( (\alpha,\beta)=B(t_{\alpha},t_{\beta})=\sum_{\gamma\in\Phi}(\dim\lG_{\gamma})(\alpha,\gamma)(\beta,\gamma)\).
\end{proof}

\begin{proposition}[\cite{Cornwell}]     \label{PropScalrooTsQ}
    Let \( \alpha\) and \( \beta\) be roots. We have
    \begin{enumerate}
        \item
            \( (\alpha,\beta)\in\eQ\),
        \item
            \( (\alpha,\alpha)\geq 0\).
    \end{enumerate}
\end{proposition}

\begin{proof}
    Let \( \alpha,\beta\in\Phi\) and consider the space
    \begin{equation}
        V=\bigoplus_{m\in\eZ}\lG_{\beta+m\alpha}.
    \end{equation}
    If \( X_{\alpha}\in\lG_{\alpha}\) and \( X_{-\alpha}\in\lG_{-\alpha}\) with \( [X_{\alpha},X_{-\alpha}]=t_{\alpha}\) we have, for all \( v\in V\),
    \begin{subequations}
        \begin{align}
            [X_{\alpha},v]&\in V\\
            [X_{-\alpha},v]&\in V\\
            [t_{\alpha},v]&\in V.
        \end{align}
    \end{subequations}
    Thus we can consider the restrictions to \( V\) of the operators \( \ad(X_{\alpha})\), \( \ad(X_{-\alpha})\) and \( \ad(t_{\alpha})\). Since \( \ad\) is an homomorphism we have, as operator on \( V\),
    \begin{equation}
        \ad(t_{\alpha})=\big[ \ad(X_{\alpha}),\ad(X_{-\alpha}) \big],
    \end{equation}
    and then \( \tr\big( \ad(t_{\alpha})|_V \big)=0\).

    Let us compute that trace on the basis \( \{ v_k^{(i)} \}\) where \( v_k^{(i)}\in\lG_{\beta+k\alpha}\). Since
    \begin{equation}
        \ad(t_{\alpha})v_k^{(i)}=(\beta+k\alpha)(t_{\alpha})v_k^{(i)}
    \end{equation}
    we have
    \begin{subequations}
        \begin{align}
            0&=\tr\big( \ad(t_{\alpha})|_V \big)\\
            &=\sum_{k\in\eZ}\dim\lG_{\beta+k\alpha}(\beta+k\alpha)(t_{\alpha})\\
            &=\sum_{k\in\eZ}\dim_{\beta+k\alpha}\big( (\alpha,\beta)+(\alpha,\alpha) \big)
        \end{align}
    \end{subequations}
    and
    \begin{equation}        \label{EqunderbAabmaaB}
        \underbrace{\left( \sum_{k\in\eZ}\dim\lG_{\beta+k\alpha} \right)}_{A\in\eN}(\alpha,\beta)=-(\alpha,\alpha)\underbrace{\left( \sum_{k\in\eZ}k\dim\lG_{\beta+k\alpha} \right)}_{B\in\eZ}.
    \end{equation}
    If \( (\alpha,\alpha)=0\) then we have \( (\beta,\alpha)=0\) for every \( \beta\in\Phi\), hence \( B(t_{\alpha},t_{\beta})=0\) which contradicts non degeneracy of the Killing form. We conclude that \( (\alpha,\alpha)\neq 0\). By the formula of lemma~\ref{Leminnerabequaaggb} we get
    \begin{equation}
        (\alpha,\alpha)=\sum_{\beta\in\Phi}\dim\lG_{\beta}(\alpha,\beta)^2.
    \end{equation}
    Replacing in that formula the value of \( (\alpha,\beta)\) taken from formula \eqref{EqunderbAabmaaB} we found
    \begin{equation}
        (\alpha,\alpha)=\sum_{\beta\in\Phi}\dim\lG_{\beta}\frac{ B^2 }{ A^2 }(\alpha,\alpha)^2
    \end{equation}
    and then \( (\alpha,\alpha)\in\eQ^+\). The fact that \( (\alpha,\beta)\) is rational follows.

    Notice that the sign of \( B\) is not guaranteed because it's not sure because we do not know whether there are more positive or negative terms in the sum of the right hand side of \eqref{EqunderbAabmaaB}.
\end{proof}

\begin{proposition}
    Let \( \alpha\) be a root of the complex semisimple Lie algebra \( \lG\). Then
    \begin{enumerate}
        \item
            \( \dim\lG_{\alpha}=1\),
        \item
            the only integer multiple of \( \alpha\) to be roots are \( \pm\alpha\).
    \end{enumerate}
\end{proposition}

\begin{proof}
    Let \( X_{\alpha}\in\lG_{\alpha}\) and consider the vector space
    \begin{equation}
        V=\eC t_{\alpha}\oplus\eC X_{\alpha}\oplus\bigoplus_{m<0}\lG_{m\alpha}.
    \end{equation}
    Let \( y\in\lG_{-\alpha}\) be chosen in such a way that \( [X_{\alpha},y]=t_{\alpha}\); by lemma~\ref{LemXYBXYtalpha} this is only a matter of normalization. The space \( V\) is invariant under \( \ad(X_{\alpha})\) and \( \ad(y)\). Indeed
    \begin{enumerate}
        \item
            \( \ad(X_{\alpha})t_{\alpha}=-\alpha(t_{\alpha})X_{\alpha}\in\eC X_{\alpha}\);
        \item
            \( \ad(X_{\alpha})X_{\alpha}=0\);
        \item
            \( \ad(X_{\alpha})\lG_{m\alpha}\subset\lG_{(m+1)\alpha}\); if \( m<-1\), \( (m+1)<0\), while if \( m=-1\) we know that the commutator \( [X_{\alpha},\lG_{-\alpha}]\) is included in \( \eC t_{\alpha}\in V\);
        \item
            \( \ad(y)t_{\alpha}\in\lG_{-\alpha}\)
        \item
            \( \ad(y)X_{\alpha}=-t_{\alpha}\) by definition;
        \item
            \( \ad(y)\lG_{m\alpha}\subset\lG_{(m-1)\alpha}\).
    \end{enumerate}
    Since \( \ad\colon \lG\to \GL(\lG)\) is an homomorphism (lemma~\ref{LemadhomomadXadYadXY}) we have
    \begin{equation}
        \big[ \ad(X_{\alpha}),\ad(y) \big]=\ad(t_{\alpha})
    \end{equation}
    and then \( \tr\big( \ad(t_{\alpha}) \big)=0\) because the trace of a commutator is zero\footnote{From the cyclic invariance of the trace.}. Since \( V\) is an invariant subspace, the trace of \( \ad(t_{\alpha})\) restricted to \( V\) is also vanishing. Let us compute that trace on the basis \( \{ X_{\alpha},t_{\alpha},X^i_{m\alpha} \}_{m<0}\) where \( i\) takes as many values as the dimension of \( \lG_{m\alpha}\).

    We have \( \ad(t_{\alpha})X_{-\alpha}=-\alpha(t_{\alpha})X_{-\alpha}\), \( \ad(t_{\alpha})t_{\alpha}=0\) and \( \ad(t_{\alpha})X^i_{m\alpha}=m\alpha(t_{\alpha})X_{m\alpha}\), thus the trace is
    \begin{equation}    \label{Eqzequalmulsumnotinfty}
        0=\alpha(t_{\alpha})\Big( -1+\sum_{m=1}^{\infty}m\dim\lG_{m_{\alpha}} \Big).
    \end{equation}
    Notice that the sum is in fact finite since the dimension of \( \lG\) is finite. We know that \( \alpha(t_{\alpha})=B(t_{\alpha},t_{\alpha})\neq 0\), so that equation \eqref{Eqzequalmulsumnotinfty} is only possible with \( \dim\lG_{\alpha}=1\) and \( \dim\lG_{m\alpha}=0\) for \( m\neq 1\).
\end{proof}

A very similar proof can be found in \cite{Cornwell}, page 827.


\begin{corollary}       \label{CorCoolWrlGbalpha}
    In the case of semisimple complex Lie algebra,

    \begin{enumerate}
        \item
            the root spaces are given by
            \begin{equation}        \label{EqExpWeightSemiSimple}
                \lG_{\alpha}=\{ X\in\lG\tq\forall h\in\lH, [h,X]=\alpha(h)X \};
            \end{equation}
        \item
            for every $ x_{\alpha}\in\lG_{\alpha}$, and for every $h\in\lH$, we have
            \begin{equation}
                [h, x_{\alpha}]=\alpha(h) x_{\alpha}.
            \end{equation}
    \end{enumerate}
\end{corollary}

\begin{proof}
    Let \( X\in\lG_{\alpha}\), we have
    \begin{equation}
        \big( \ad(h)-\alpha(h) \big)^nX=0,
    \end{equation}
    so
    \begin{equation}    \label{EqadhalphahnnmuvX}
        \big( \ad(h)-\alpha(h) \big) \underbrace{\big( \ad(h)-\alpha(h) \big)^{n-1} X}_v=0.
    \end{equation}
    In particular the vector \( v= \big( \ad(h)-\alpha(h) \big)^{n-1} X\) belongs to \( \lG_{\alpha}\). Since the latter space has dimension one, the vector \( v\) is a multiple of \( X\) and consequently equation \eqref{EqadhalphahnnmuvX} shows that
    \begin{equation}
        \big( \ad(h)-\alpha(h) \big)v=\big( \ad(h)-\alpha(h) \big)X=0.
    \end{equation}

    The second point is only an other way to write the same.
\end{proof}

\begin{lemma}       \label{LemHzesialphaHz}
    If \( H\) is an element of \( \lH\) with \( \alpha(H)=0\) for every root, then \( H=0\)
\end{lemma}

\begin{proof}
    Consider the decompositions (not unique) \( H=\sum_{\alpha\in\Phi}a_{\alpha} t_{\alpha}\) and \( H'=\sum_{\beta\in\phi}a'_{\beta}t_{\beta}\). Then
    \begin{subequations}
        \begin{align}
            B(H,H')&=\sum_{\alpha,\beta}a_{\alpha}a_{\beta}'B(t_{\alpha},t_{\beta})\\
            &=\sum_{\alpha,\beta}a'_{\beta}\beta(a_{\alpha},t_{\alpha})\\
            &=\sum_{\beta}a'_{\beta}\beta(H)\\
            &=0.
        \end{align}
    \end{subequations}
    Such an element is thus Killing-orthogonal to the whole space \( \lH\) but we already know the \( \lH\) is orthogonal to each space \( \lG_{\alpha}\) (\( \alpha\neq 0\)). By non degeneracy of the Killing form we must have \( H=0\).
\end{proof}

\begin{proposition}
    The set of roots of a complex semisimple Lie algebra spans the dual space \( \lH^*\).
\end{proposition}

\begin{proof}
    Consider a basis \( \{ H_i \}\) of \( \lH\) with \( \{ H_0,\ldots,H_m \}=\Span(\Phi)\) and \( \{ H_{m+1},\ldots,H_r \}\) be outside of \( \Span\Phi\). A root reads \( \alpha=\sum_{k=0}^ma_kH_k^*\).
    Thus \( \alpha(H_{m+1})=0\), which implies that \( H_{m+1}=0\) by lemma~\ref{LemHzesialphaHz}.
\end{proof}

\begin{corollary}
    A Cartan algebra \( \lH\) of a complex semisimple Lie algebra is abelian.
\end{corollary}

\begin{proof}
    Let \( H',H''\in\lH\) and consider \( H=[H',H'']\), a root \( \alpha\) and \( X_{\alpha}\in\lG_{\alpha}\). On the one hand we have
    \begin{equation}
        \big[ [H',H''],X_{\alpha} \big]=-\alpha(H')[X_{\alpha},H']+\alpha(H')[X_{\alpha},H'']=0
    \end{equation}
    and on the other hand we have \( \big[ [H',H''],X_{\alpha} \big]=[H,X_{\alpha}]=\alpha(H)X_{\alpha}\). We deduce that \( \alpha(H)=0\) for every root and then that \( H=0\) by lemma~\ref{LemHzesialphaHz}.
\end{proof}

We denote by \( \Phi\) the set of roots. These are the elements \( \lambda\in\lH^*\) such that \( \lG_{\lambda}\) is non trivial. We suppose to have chosen a positivity notion on \( \lH^*\), so that we can speak of \( \Phi^+\), the set of \defe{positive roots}{positive root}.

A positive root is \defe{simple}{simple!root} is it cannot be written as the sum of two positive roots.

%---------------------------------------------------------------------------------------------------------------------------
                    \subsection{Generators}
%---------------------------------------------------------------------------------------------------------------------------

We are going to build the Chevalley basis of the complex semisimple Lie algebra \( \lG\). That will essentially be a choice of a basis vector in each of the root spaces. We are following the notations summarized in point~\ref{SubsecNotationRootsDel}.


Now, for each root $\alpha$, we pick $e_{\alpha}\in\lG_{\alpha}$. We will see that, up to renormalization, we can set the in nice commutation relations.

\begin{lemma}       \label{LemBalpahbetaef}
    If \( \alpha\) and \( \beta\) are roots such that \( \alpha+\beta\neq 0\), then
    \begin{equation}
        B(e_{\alpha},e_{\beta})=0.
    \end{equation}
    If \( f_{\alpha}\in\lG_{-\alpha}\) we also have \( B(e_{\alpha},f_{\alpha})\neq 0\).
\end{lemma}

\begin{proof}
    By definition \( B(e_{\alpha},e_{\beta})=\tr\big( \ad(e_{\alpha})\circ\ad(e_{\beta}) \big)\). If we apply \( \ad(e_{\alpha})\circ\ad(e_{\beta})\) to an element of \( e_{\gamma}\) (including \( \lG_0=\lH\)), we get an element of \( \lG_{\alpha+\beta+\gamma}\). Thus the trace defining the Killing form is zero and \( B(e_{\alpha},e_{\beta})=0\) when \( \alpha+\beta=0\).

    Since the Killing form is nondegenerate, we conclude that \( B(e_{\alpha},e_{-\alpha})\neq 0 \).
\end{proof}

\begin{corollary}       \label{CorrExistInverseRoot}
    Let \( \lG\) be a semisimple complex Lie algebra and \( \lH\) be a Cartan subalgebra of \( \lG\). Let \( \alpha\) be a root of \( \lG\) and \( \lH_{\alpha}=[\lG_{\alpha},\lG_{-\alpha}]\). There exist an unique \( H_{\alpha}\in\lH_{\alpha}\) such that \( \alpha(H_{\alpha})=2\).
\end{corollary}

\begin{proof}
    We have \( [e_{\alpha},f_{\alpha}]=B(e_{\alpha},f_{\alpha})t_{\alpha}\) and the lemma~\ref{LemBalpahbetaef} shows that the Killing form is non zero. Multiplying by a suitable number provides the result.
\end{proof}
The element \( H_{\alpha}\in\lH\) defined in this lemma is the \defe{inverse root}{inverse!root} of \( \alpha\).

\begin{lemma}       \label{Lemalphaakbetaknimport}
    Let \( \{ \beta_1,\ldots,\beta_l \}\) be a choice of elements in \( \lH^*\) such that the set \( \{ t_{\beta_1},\ldots,t_{\beta_l} \}\) is a basis of \( \lH\). Thus the roots can be decomposed as
    \begin{equation}
        \alpha=\sum_{k=1}^la_k\beta_k
    \end{equation}
    with \( a_k\in\eQ\).
\end{lemma}

\begin{proof}

    Let \( \alpha=\sum_{k=1}^la_k\beta_k\). We know that the vectors \( t_{\beta_i}\) form a basis of \( \lH\), so we have the decomposition \( t_{\alpha}=\sum_ka_kt_{\beta_k}\). Indeed
    \begin{equation}
        B\big( h,\sum_k a_kt_{\beta_k} \big)=\sum_ka_k B(h,t_{\beta_k})=\sum_ka_k\beta_k(h)=\alpha(h).
    \end{equation}
    For each \( k=1,2,\ldots,l\) we have
    \begin{equation}
        (\alpha_k, \alpha) =\sum_{j=1}^la_k(\alpha_k, \alpha_j).
    \end{equation}
    This is a system of linear equations for the \( l\) variables \( a_k\). Since the coefficients \( (\alpha_k,\alpha)\) and \( (\alpha_k,\alpha_j)\) are rational by proposition~\ref{PropScalrooTsQ}, the solutions are rational too.
\end{proof}

\begin{remark}
    The lemma~\ref{Lemalphaakbetaknimport} deals with a quite general basis of \( \lH\). We will see in the proposition~\ref{ThoposrootnjajnZ} that in the case of the basis of simple roots, the coefficients \( a_k\) are integers, either all positive or all negative.
\end{remark}

%---------------------------------------------------------------------------------------------------------------------------
\subsection{Subalgebra \texorpdfstring{$ \gsl(2)_i$}{SL2R} }
%---------------------------------------------------------------------------------------------------------------------------
\label{SubSecCopiedeSLdansGi}

For each nonzero root \( \alpha\in\lH^*\), we choose \( e_{\alpha}\in\lG_{\alpha}\) and \( f_{\alpha}\in\lG_{-\alpha}\) in such a way to have
\begin{equation}
    B(e_{\alpha},f_{\alpha})=\frac{ 2 }{ B(t_{\alpha},t_{\alpha}) },
\end{equation}
and then we pose
\begin{equation}
    h_{\alpha}=\frac{ 2 }{ B(t_{\alpha},t_{\alpha}) }t_{\alpha}.
\end{equation}
Notice that these choices are possible because the Killing form is non degenerated on \( \lH\).

% This proposition about gsl(2,R) was at position 198631779.
\begin{proposition}[\cite{SternLieAlgebra}] \label{PropWEzZYzC}
    For each root, the set $\{ e_{\alpha},f_{\alpha},h_{\alpha} \}$ generates an algebra isomorphic to $\gsl(2,\eR)$, i.e. they satisfy
    \begin{subequations}
        \begin{align}
            [h_{\alpha},e_{\alpha}]&=2e_{\alpha}\\
            [h_{\alpha},f_{\alpha}]&=-2f_{\alpha}\\
            [e_{\alpha},f_{\alpha}]&=h_{\alpha}\\
        \end{align}
    \end{subequations}
\end{proposition}

\begin{proof}
    Since \( \alpha(t_{\alpha})=B(t_{\alpha},t_{\alpha})\) we have \( \alpha(h_{\alpha})=2\). Now the computations are quite direct. The first is
    \begin{equation}
        [h_{\alpha},e_{\alpha}]=\alpha(h_{\alpha})e_{\alpha}=2e_{\alpha}.
    \end{equation}
    For the second,
    \begin{equation}
        [h_{\alpha},f_{\alpha}]=-\alpha(h_{\alpha})f_{\alpha}=-2f_{\alpha}.
    \end{equation}
    For the third, we know that \( [e_{\alpha},f_{\alpha}]\in\lH\); thus \( B\big( X,[e_{\alpha},f_{\alpha}] \big)=0\) whenever \( X\in\lG_{\lambda}\) with \( \lambda\neq 0\). Let \( h\in\lH\). Using the invariance of the Killing form,
    \begin{equation}
        B\big( h,[e_{\alpha},f_{\alpha}] \big)=B\big( [h,e_{\alpha}],f_{\alpha} \big)=\alpha(h)B(e_{\alpha},f_{\alpha})=B(t_{\alpha},t_{\alpha})B(e_{\alpha},f_{\alpha})=B\big( B(e_{\alpha},f_{\alpha})t_{\alpha},h \big).
    \end{equation}
    Thus
    \begin{equation}
        [e_{\alpha},f_{\alpha}]=B(e_{\alpha},f_{\alpha})f_{\alpha}=h_{\alpha}.
    \end{equation}
\end{proof}
Remark that we used the non degeneracy of the Killing form in a crucial way. The copy of \( \gsl(2,\eR)\) formed by \( \{ e_{\alpha},f_{\alpha},h_{\alpha} \}\) is denoted by $\gsl(2,\eR)_{\alpha}$.

\begin{proposition}
    In the universal enveloping algebra,
    \begin{equation}        \label{Eqhjfikplusun}
        [h_j,f_i^{k+1}]=-(k+1)\alpha_i(h_j)f_i^{k+1}
    \end{equation}
    as generalisation of the previous one.
\end{proposition}

\begin{proof}
    We use an induction over $k$. Since $\ad(h_j)$ is a derivation in $\mU(\lG)$, the induction hypothesis and the definition relation $[h,f_i]=-\alpha_i(h)f_i$ with $h=h_i$, we have
    \begin{equation}
        \begin{split}
            \ad(h_j)f^{k+1}_i   &=\big( \ad(h_j)f_i^k \big)f_i+f_i^k\ad(h_j)f_i.\\
                        &=-k\alpha+i(h_j)f_i^kf_i-\alpha_i(h_j)f^{k+1}_i\\
                        &=-(k+1)\alpha_i(h_j)f_i^{k+1}.
        \end{split}
    \end{equation}
\end{proof}

Now the Lie algebra \( \lG\) can be seen as a \( \gsl(2,\eR)\)-module. As an example, for each choice of \( \beta\in\Phi\), the algebra \( \gsl(2)_{\alpha}\) acts on the vector space
\begin{equation}
    V=\bigoplus_{k\in\eZ}\lG_{\beta+k\alpha}.
\end{equation}
The vector space \( \lG\) carries thus several representations of \( \gsl(2)\); this fact will be used in a crucial way during the proof of proposition~\ref{Proppqasbabaa}.

% This is part of (almost) Everything I know in mathematics and physics
% Copyright (c) 2013, 2019-2020
%   Laurent Claessens
% See the file fdl-1.3.txt for copying conditions.

%---------------------------------------------------------------------------------------------------------------------------
\subsection{Chevalley basis}
%---------------------------------------------------------------------------------------------------------------------------

The Chevalley basis corresponds to an other choice of normalization of the element \( e_{\alpha}\), \( h_{\alpha}\). If we set
\begin{subequations}
    \begin{numcases}{}
        H_{\alpha}=K_{\alpha}t_{\alpha}\\
        E_{\alpha}=N_{\alpha}e_{\alpha}
    \end{numcases}
\end{subequations}
with
\begin{equation}
    \begin{aligned}[]
        K_{\alpha}&=\frac{ 2 }{ (\alpha,\alpha) }\\
        N_{\alpha}&=\sqrt{\frac{ 2 }{ B(e_{\alpha},e_{-\alpha})(\alpha,\alpha) }},
    \end{aligned}
\end{equation}
then we have the \defe{Chevalley relations}{Chevalley!basis}:
\begin{subequations}        \label{EqsChevalleuRels}
    \begin{align}
        [E_{\alpha},E_{-\alpha}]&=H_{\alpha}\\
        [H_{\alpha},E_{\beta}]&=\frac{ 2(\alpha,\beta) }{ (\alpha,\alpha) }E_{\beta}\\
        [H_{\alpha},H_{\beta}]&=0.
    \end{align}
\end{subequations}
The last relation is nothing else than the fact that the Cartan subalgebra \( \lH\) is abelian. Notice that we don't give relations between \( E_{\alpha}\) and \( E_{\beta}\). Of course \( [E_{\alpha},E_{\beta}]\sim E_{\alpha+\beta}\) but the spaces \( \lG_{\alpha}\) and \( \lG_{\beta}\) being Killing orthogonal, the Killing does not provides a natural relative normalisation between \( E_{\alpha}\) and \( E_{\beta}\).

\begin{definition}  \label{DefORftFjP}
    If \( \{ \alpha_i \}_{i=1,\ldots,l}\) is the set of simple roots, we consider the notation \( X_i^+=E_{\alpha_i}\), \( X^-_i=E_{-\alpha_i}\), \( H_i=H_{\alpha_i}\) and we introduce the \defe{Cartan matrix}{Cartan!matrix}
    \begin{equation}
        A_{ij}=\frac{ 2(\alpha_i,\alpha_j) }{ (\alpha_i,\alpha_i) }.
    \end{equation}
\end{definition}
Reduced to the simple roots the relations \eqref{EqsChevalleuRels} become
\begin{equation}        \label{EqChevalleySimple}
    \begin{aligned}[]
        [X^+_i,X^-_j]&=\delta_{ij}H_i\\
        [H_i,X^{\pm}_j]&=\pm A_{ij}X^{\pm}_j\\
        [H_i,H_j]&=0.
    \end{aligned}
\end{equation}
The first relation comes from the fact that \( \alpha_i-\alpha_j\) is not a root when \( \alpha_i\) and \( \alpha_j\) are simple roots.

\begin{remark}
    The idea behind the Chevalley relations  is that the algebra \( \lG\) is generated by the elements \( X^{\pm}_i\), \( H_i\) and the commutation relations \eqref{EqChevalleySimple}. Even if these elements do not form a basis (while the elements \( E_{\alpha}\), \( H_{\alpha}\) do), one can define a function on \( \lG\) by giving its values on \( X^{\pm}_i\) and \( H_i\) providing one has a canonical way to extend it on commutators.

    The definition~\ref{EqDefCobrackStandard} of standard cobracket works in this way.
\end{remark}

\begin{example}
    The basis \( \{ H,E,F \}\) given by
    \begin{equation}
    H=\begin{pmatrix}
    1 & 0 \\
    0 & -1
    \end{pmatrix}
    ,\quad
      E=\begin{pmatrix}
    0 & 1 \\
    0 & 0
    \end{pmatrix}
    ,\quad
     F=\begin{pmatrix}
    0 & 0 \\
    1 & 0
    \end{pmatrix},
    \quad
    T=\begin{pmatrix}
    0&1\\
    -1&0
    \end{pmatrix}
    \end{equation}
    satisfy the relations
    \begin{subequations}    \label{subEqsSBhuAWx}
        \begin{align}
            [H,E]&=2E\\
            [H,F]&=-2F\\
            [E,F]&=H.
        \end{align}
    \end{subequations}
    which are nothing else than the relations \eqref{EqChevalleySimple}.

    The only positive root is \( \alpha(H)=2\). The Cartan matrix\footnote{Definition~\ref{DefORftFjP}.} reduces to one number:
    \begin{equation}
        A=A_{11}=\frac{ 2(\alpha,\alpha) }{ (\alpha,\alpha) }=2.
    \end{equation}

    The Killing form, in the basis \( \{ H,E,F \}\) is given by
    \begin{equation}
        B=\begin{pmatrix}
            8    &   0    &   0    \\
            0    &   0    &   4    \\
            0    &   4    &   0
        \end{pmatrix}
    \end{equation}
    and the element \( t_{\alpha}\) is then
    \begin{equation}
        t_{\alpha}=\frac{1}{ 4 }H.
    \end{equation}
    The inner product on \( \lH^*\) is then
    \begin{equation}        \label{Eqinnerhstarsldc}
        (\alpha,\alpha)=B(t_{\alpha},t_{\alpha})=\frac{ 1 }{2}.
    \end{equation}
\end{example}

\begin{remark}
    Notice that these relations do not give the value of
    \begin{equation}
        [E_{\alpha},E_{\beta}]=N_{\alpha,\beta}E_{\alpha+\beta}
    \end{equation}
    when \( \alpha+\beta\) is a root.
\end{remark}

\begin{probleme}
    It has to be possible to compute \( N_{\alpha,\beta}\), but I do not know how. The answer is given in equation \eqref{EqChevalleyBasis} but I don't know where I got them. Maybe there are some hints in \cite{Cornwell} (Il faut ajouter Cornwell à la bibliographie et enlever le problème~\ref{ProbAvecCorwell}).
\end{probleme}

\begin{probleme}
    It seems that \( A_{ij}\) is the larger integer \( k\) such that \( \alpha_i+k\alpha_j\) is a root. This is the justification of the other Serre's relations that read
    \begin{equation}
        \ad^{1-A_{ij}}(X^{\pm}_i)X^{\pm}_j=0.
    \end{equation}
    That relation has to be written with the Chevalley's ones.
\end{probleme}

One can choose the coefficients in a more scientific way\cite{SerreSSAlgebres}. Let \( \alpha\) be a positive root, let \( H_{\alpha}\) be the inverse root of \( \alpha\) and \( e_{\alpha}\in\lG_{\alpha}\). We have
\begin{equation}
    [e_{\alpha},e_{\beta}]=\begin{cases}
        N_{\alpha,\beta}e_{\alpha+\beta}    &   \text{if } \alpha+\beta\text{ is a root}\\
        0    &    \text{if } \alpha+\beta\text{ is not a root}.
    \end{cases}
\end{equation}
We are going to find a multiple \( E_{\alpha}\) of \( e_{\alpha}\) in such a way to have in the same time
\begin{subequations}
    \begin{numcases}{}
        [E_{\alpha},E_{-\alpha}]=H_{\alpha}\\
        N_{\alpha,\beta}=-N_{-\alpha,-\beta}.
    \end{numcases}
\end{subequations}

Let \( \sigma\) be an involutive automorphism of \( \lG\) such that \( \sigma|_{\lH}:-\id\). First we have \( \sigma(\lG_{\alpha})=\lG_{-\alpha}\) because
\begin{equation}
    [h,\sigma(e_{\alpha})]=\sigma[\sigma(h),e_{\alpha}]=-\sigma\alpha(h)e_{\alpha}=-\alpha(h)\sigma(e_{\alpha})
\end{equation}
for every \( h\in\lH\) and \( e_{\alpha}\in\lG_{\alpha}\). From corollary~\ref{CorrExistInverseRoot} there exist a number \( a_{\alpha}\) such that
\begin{equation}
    [e_{\alpha},\sigma(e_{\alpha})]=a_{\alpha} H_{\alpha}.
\end{equation}
We pose
\begin{subequations}
    \begin{numcases}{}
        E_{\alpha}=\frac{1}{ \sqrt{-a_{\alpha}} }e_{\alpha}\\
        E_{-\alpha}=-\sigma(E_{\alpha}).
    \end{numcases}
\end{subequations}
With that choice we immediately have \( [E_{\alpha},E_{-\alpha}]=H_{\alpha}\). We also have \( N_{\alpha,\beta}=-N_{-\alpha,-\beta}\); in order to see it, consider
\begin{equation}
    [\sigma E_{\alpha},\sigma E_{\beta}]=\sigma[E_{\alpha},E_{\beta}]=N_{\alpha,\beta}\sigma(E_{\alpha+\beta})=-N_{\alpha,\beta}E_{-\alpha-\beta}.
\end{equation}
But the same is also equal to
\begin{equation}
    [-E_{-\alpha},-E_{-\beta}]=[E_{-\alpha},E_{-\beta}]=N_{-\alpha,-\beta}E_{-\alpha,-\beta}.
\end{equation}

\begin{proposition}
    With these choices we have
    \begin{equation}
        N_{\alpha,\beta}=\pm(p+1)
    \end{equation}
    where \( p\) is the largest integer \( j\) such that \( \beta-j\alpha\) is a root.
\end{proposition}

\begin{probleme}
    I don't know a proof of that, but \cite{SerreSSAlgebres} gives a reference.
\end{probleme}

From proposition~\ref{Propoxalphaymoinaalpha} we know that \( t_{\alpha}\in\lH_{\alpha}\), so that \( H_{\alpha}\) is a multiple of \( H_{\alpha}\). The proportionality factor is easy to fix since
\begin{equation}        \label{EqRealsHalpatalphaNorsmd}
    \begin{aligned}[]
        \alpha(H_{\alpha})&=2 &\text{definition of }H_{\alpha}\\
        \alpha(t_{\alpha})&=(\alpha,\alpha) &\text{definition \eqref{EqDefInnprHestrar}}.
    \end{aligned}
\end{equation}
Thus \( H_{\alpha}=\frac{ 2 }{ (\alpha,\alpha) }t_{\alpha}\) and
\begin{equation}
    [H_{\alpha},E_{\beta}]=\beta(H_{\alpha})E_{\beta}=\frac{ 2 }{ (\alpha,\alpha) }\beta(t_{\alpha})=\frac{ 2(\alpha,\beta) }{ (\alpha,\beta) }
\end{equation}
again by the definition \eqref{EqDefInnprHestrar}.

%---------------------------------------------------------------------------------------------------------------------------
\subsection{Coefficients in the Cartan matrix}
%---------------------------------------------------------------------------------------------------------------------------

In this section we search to give the form of the coefficients in the Cartan matrix. We will show that the values of \( (\alpha,\beta)\) are quite restricted.

\begin{remark}
    The notations are not standard. Here the symbol \( \Delta\) denotes the set of \emph{simple} roots while the set of all roots is denoted by \( \Phi\). In the book \cite{Cornwell}, the symbol \( \Delta\) is the set of all roots. This makes quite a difference!
\end{remark}

\begin{definition}
    If \( \alpha\) and \( \beta\) are roots of the complex semisimple Lie algebra \( \lG\), then the \defe{\( \alpha\)-string}{string!of roots} containing \( \beta\) is the set of roots of the form \( \alpha+k\beta\) with \( k\in\eZ\).
\end{definition}

Among other things, the following proposition shows that a string has no gap.
\begin{proposition}     \label{Proppqasbabaa}
    Let \( \alpha,\beta\in\Phi\). Then there exits integers \( p,q\) such that \( \{ \beta+k\alpha \}_{-p\leq k\leq q} \) is the \( \alpha\)-string containing \( \beta\). The numbers \( p\) and \( q\) satisfy
    \begin{equation}        \label{Eq2qbaaapmq}
        p-q=\frac{ 2(\alpha,\beta) }{ (\alpha,\alpha) }
    \end{equation}
    and the form
    \begin{equation}
        \beta-\frac{ 2(\beta,\alpha) }{ (\alpha,\alpha) }\alpha
    \end{equation}
    is a nonzero root.
\end{proposition}

\begin{proof}
    We consider the vector space
    \begin{equation}
        V=\bigoplus_{k\in\eZ}\lG_{\beta+k\alpha}
    \end{equation}
    and the Lie algebra \( \gsl(2)_{\alpha}=\langle e_{\alpha},f_{\alpha},h_{\alpha}\rangle\) defined in subsection~\ref{SubSecCopiedeSLdansGi}. The latter acts on \( V\). Simple computation using the fact that \( \beta(h_{\alpha})=2(\alpha,\beta)/(\alpha,\alpha)\) shows that
    \begin{equation}
        [\frac{ 1 }{2}h_{\alpha},e_{\beta+k\alpha}]=\left( \frac{ (\alpha,\beta) }{ (\alpha,\alpha) }+k \right)e_{\beta+k\alpha}.
    \end{equation}
    Thus the matrix of \( \ad(\frac{ 1 }{2}h_{\alpha})\) is diagonal and has no multiplicity in its eigenvalues. We deduce that the representation if irreducible. From general theory of irreducible representations of \( \gsl(2)\) we know that there exists an half-integer number \( j\) such that the diagonal entries of \( \ad(\frac{ 1 }{2}h_{\alpha})\) take \emph{all} the values from \( -j\) to \( j\) by integer steps. Thus the \( \alpha\)-string containing \( \beta\) has the form \( \{ \beta+k\alpha \}_{-p\leq k\leq q}\) where \( p\) and \( q\) satisfy
    \begin{subequations}
        \begin{numcases}{}
            \frac{ (\alpha,\beta) }{ (\alpha,\alpha) }-p=-j\\
            \frac{ (\alpha,\beta) }{ (\alpha,\alpha) }+q=j.
        \end{numcases}
    \end{subequations}
    Summing we get
    \begin{equation}
        p-q=\frac{ 2(\alpha,\beta) }{ (\alpha,\alpha) }.
    \end{equation}
    If \( \lambda\) is an eigenvalue of \( \ad(\frac{ 1 }{2}h_{\alpha})\), then \( -\lambda\) is also an eigenvalue (this is still from the irreducible representation theory of \( \gsl(2)\)). The number \( (\alpha,\beta)/(\alpha,\beta)\) is obviously an eigenvalue (with \( k=0\)), thus the string contains a \( k\) such that
    \begin{equation}
        \frac{ (\alpha,\beta) }{ (\alpha,\alpha) }+k=-\frac{ (\alpha,\beta) }{ (\alpha,\alpha) }.
    \end{equation}
    The solution is \( k=-2(\alpha,\beta)/(\alpha,\alpha)\) and we deduce that
    \begin{equation}
        \beta-2\frac{ (\alpha,\beta) }{ (\alpha,\alpha) }\alpha
    \end{equation}
    is a root of \( \lG\).
\end{proof}

\begin{proposition}
    Let \( \alpha,\beta\) be two roots. Then we have
    \begin{equation}
        \frac{2(\alpha,\beta)}{(\alpha,\alpha)}=0,\pm 1,\pm 2,\pm 3.
    \end{equation}
\end{proposition}

\begin{proof}
    First, equation \eqref{Eq2qbaaapmq} shows that \( \frac{2(\alpha,\beta)}{(\alpha,\alpha)}\) is integer. If \( \alpha=\pm\beta\), the result is \( 2\). If \( \alpha\neq\pm\beta\), the vectors \( t_{\alpha}\) and \( t_{\beta}\) are linearly independent and the Schwarz inequality shows
    \begin{equation}        \label{EqprofmqxqCqrtmai}
        (\alpha,\beta)^2=| B(t_{\alpha},t_{\beta}) |< B(t_{\alpha},t_{\alpha})B(t_{\beta},t_{\beta})=(\alpha,\alpha)(\beta,\beta).
    \end{equation}
    Thus
    \begin{equation}        \label{EqprofmqxqCqrtmaii}
        \left| \frac{2(\alpha,\beta)}{(\alpha,\alpha)} \right| \left| \frac{2(\alpha,\beta)}{(\beta,\beta)} \right| <\frac{ 4| (\alpha,\beta)(\alpha,\beta) | }{ (\alpha,\beta)^2 }=4.
    \end{equation}
    Consequently the number \( | 2(\alpha,\beta)/(\alpha,\alpha) |\) being integer can only take the values \( 0\), \( 1\), \( 2\) and \( 3\). Notice that the inequality in \eqref{EqprofmqxqCqrtmai} and \eqref{EqprofmqxqCqrtmaii} are strict since \( \alpha_i\) is not collinear to \( \alpha_j\).
\end{proof}

\subsection{Simple roots}
%------------------------
As seen before, $\Phi$ admits an order inherited from $\lHeR^*$. A root $\alpha>0$ is \defe{simple}{simple!root} if it cannot be written as a sum of two positive roots.

\begin{theorem}      \label{ThoposrootnjajnZ}
    Let \( \{ \alpha_1,\ldots,\alpha_l \}\) be the set of simple roots. Then every root \( \beta\in\Phi\) can be decomposed into
    \begin{equation}
        \beta=\sum_{i=1}^ln_i\alpha_i
    \end{equation}
    where non vanishing the numbers \( n_i\in\eZ\) are either all positive or all negative.
\end{theorem}

\begin{proof}
    Let \( \beta\) be positive. If it is not simple, the one can decompose it into two positive roots:
    \begin{equation}
        \beta=\gamma+\delta
    \end{equation}
    with \( \gamma,\delta>0\). If \( \gamma\) and/or \( \delta\) are not simple, they can be decomposed further. This process has to be finite, indeed if the process is not finite, the decomposition of at least one positive root has to contains itself (because there are finitely many of them) while it is impossible to have \( \gamma=\gamma+\alpha\) with \( \alpha>0\).
\end{proof}

Two corollaries: a root is either positive or negative (this is part of the definition of positivity) and when a root is positive, its decomposition into simple roots has only positive coefficients.


%---------------------------------------------------------------------------------------------------------------------------
\subsection{Weyl group}
%---------------------------------------------------------------------------------------------------------------------------
References about Weyl group: \cite{Knapp_reprez}. See also \cite{Cornwell}, page 530.

If \( \alpha\) is a root of \( \lG\) we define the \defe{symmetry}{symmetry!of a root} of \( \alpha\) as
\begin{equation}
    \begin{aligned}
        s_{\alpha}\colon \lH^*&\to \lH^* \\
        \beta&\mapsto \beta-\beta(H_{\alpha})\alpha
    \end{aligned}
\end{equation}
where \( H_{\alpha}\in\lH\) is the inverse root of \( \alpha\). Since \( \alpha(H_{\alpha})=2\) we have \( s_{\alpha}(\alpha)=-\alpha\). The group generated by the symmetries and the identity is the \defe{Weyl group}{Weyl!group}.

From what is said around equation \eqref{EqRealsHalpatalphaNorsmd} and the definition \( (\alpha,\beta)=\alpha(t_{\beta})\), we have
\begin{equation}
    s_{\alpha}(\beta)=\beta-\frac{ 2(\alpha,\beta) }{ (\alpha,\alpha) }\alpha.
\end{equation}

We know from proposition~\ref{Proppqasbabaa} that \( s_{\alpha}(\beta)\) is a root while there are only finitely many roots; thus the Weyl group is finite since there are only a finite number of maps from a finite set to itself.

The symmetries associated to roots are involutive:
\begin{equation}
    s_{\alpha}^2=\id.
\end{equation}
Indeed
\begin{equation}
    \begin{aligned}[]
        s^2_{\alpha}(\beta)&=s_{\alpha}\big( \beta-\beta(H_{\alpha})\alpha \big)\\
        &=\beta-\beta(H_{\alpha})-\big( \beta-\beta(H_{\alpha})\alpha \big)(H_{\alpha})\alpha\\
        &=\beta
    \end{aligned}
\end{equation}
if we take into account \( \alpha(H_{\alpha})=2\).

Relative to the symmetry \( s_{\alpha_i}\) we have the symmetry \( s_i\) on \( \lH\) defined by
\begin{equation}        \label{EqSymsiReltosalphai}
    s_i(h)=h-\alpha_i(h)H_i
\end{equation}
where \( h\in\lH\) and \( H_i\) is the inverse root of \( \alpha_i\).

\begin{remark}
    The simple roots \( \alpha_i\) {\bf are not} orthogonal.
\end{remark}

Let $\Delta$ be a reduced abstract root system on a real finite dimensional vector space $V$. The group $W$ generated by the $s_{\alpha}:\alpha\in\Delta$ is the \defe{Weyl group}{Weyl!group}\index{group!Weyl}.

\begin{proposition}     \label{PropWeylIsomalphai}
    The elements \( s_{\alpha_i}\) are isometries of \( \lH^*\), i.e.
    \begin{equation}
        \big( s_{\alpha_i}(\alpha),s_{\alpha_i}(\beta) \big)=(\alpha,\beta).
    \end{equation}
\end{proposition}

\begin{proof}
    For the sake of shortness, let us write
    \begin{equation}
        n_{i,\alpha}=\frac{ 2(\alpha_i,\alpha) }{ (\alpha_i,\alpha_i) }.
    \end{equation}
    We have \( t_{s_{\alpha_i}(\alpha)}=t_{\alpha}-n_{i,\alpha}t_{\alpha_i}\). Thus
    \begin{equation}
        B\big( t_{s_{\alpha_i}(\alpha)}, t_{s_{\alpha_i}(\beta)} \big)=B(t_{\alpha}-n_{i,\alpha}t_{\alpha_i},t_{\beta}-n_{i,\beta}t_{\alpha_i})
    \end{equation}
    distributing and taking into account the fact all the relations like \( B(t_{\alpha},t_{\alpha_i})=(\alpha,\alpha_i)\), the right hand side reduces to \( B(t_{\alpha},t_{\beta})=(\alpha,\beta)\).
\end{proof}

When \( \Phi\) is the root system, one can chose many different notions of positivity; each of them bring to different simple systems. It turns out that the action of the Weyl group on a simple system produces the simple system of an other choice of positivity on \( \Phi\).

\begin{lemma}       \label{LemalphajsPhipinjsasbab}
    If \( s_{\alpha_i}\alpha=s_{\alpha_i}\beta\), then \( \alpha=\beta\).
\end{lemma}

\begin{proof}
    The hypothesis \( s_{\alpha_j}(\alpha-\beta)=0\) provides
    \begin{equation}
        0=\alpha-\beta-\frac{ 2(\alpha-\beta,\alpha_j) }{ (\alpha_j,\alpha_j) }\alpha_j
    \end{equation}
    so that \( \alpha=\beta+z\alpha_j\) for some \( z\in\eC\). Thus we have
    \begin{equation}
        s_{\alpha_j}(\alpha)=s_{\alpha_j}(\beta)+zs_{\alpha_j}(\alpha_j)=s_{\alpha_j}(\alpha)-z\alpha_j.
    \end{equation}
    Thus \( z=0\) and \( \alpha=\beta\).
\end{proof}

\begin{proposition}     \label{PropsalphaisurjPhipmaj}
    Let \( \alpha_i\) a simple root. The set \( \Phi^+\setminus\{ \alpha_i \}\) is stable under \( s_{\alpha_i}\), i.e.
    \begin{equation}
        s_{\alpha_i}\big( \Phi^+\setminus\{ \alpha_i \} \big)=\Phi^+\setminus\{ \alpha_i \}.
    \end{equation}
\end{proposition}

\begin{proof}
    Let \( \lambda\in\Phi^+\) be a positive root. By theorem~\ref{ThoposrootnjajnZ} we have
    \begin{equation}        \label{Eqllamsumajajajpos}
        \lambda=\sum_ja_j\alpha_j
    \end{equation}
    with \( a_j\geq 0\). Since \( \lambda\neq\alpha_i\) we have \( a_j>0\) for some \( j\neq i\). Indeed the only multiple of \( \alpha_i\) to be a root are \( 0\) and \( \pm\alpha_i\). Since \( \lambda\in\Phi^+\) and \( \lambda\neq \alpha_i\), none of these three solutions are taken into consideration.

    Let's apply \( s_{\alpha_i}\) on both sides of \eqref{Eqllamsumajajajpos}:
    \begin{equation}        \label{eqalialphaillamfracalphai}
        \begin{aligned}[]
            s_{\alpha_i}(\lambda)&=s_{\alpha_i}\big( \sum_ja_j\alpha_j \big)\\
            &=\sum_{j\neq i}a_js_{\alpha_i}(\alpha_j)+a_i\underbrace{s_{\alpha_i}(\alpha_i)}_{-\alpha_i}\\
            &=\sum_{j\neq i}a_j\alpha_j-\sum_{j\neq i}a_j\frac{ 2(\alpha_i,\alpha_j) }{ (\alpha_i,\alpha_i) }\alpha_i-a_i\alpha_i
        \end{aligned}
    \end{equation}
    Since a root is either positive or negative, the coefficients are either \emph{all} positive or \emph{all} negative. Since all the coefficients (apart for the one of \( \alpha_i\)) are the same as the ones of \( \lambda\), the root \eqref{eqalialphaillamfracalphai} is positive.

    We still have to prove that \( s_{\alpha_i}(\lambda)\neq \alpha_i\). Indeed if \( s_{\alpha_i}(\lambda)=\alpha_i\) we have
    \begin{equation}
        \lambda=s_{\alpha_i}s_{\alpha_i}(\lambda)=s_{\alpha_i}(\alpha_i)=-\alpha_i,
    \end{equation}
    which contradicts the positivity of \( \lambda\).

    Up to now we proved that \( s_{\alpha_i}\big( \Phi^+\setminus\{ \alpha_i \} \big)\subset\Phi^+\setminus\{ \alpha_i \}\). If \( \lambda\in\Phi^+\setminus\{ \alpha_i \}\), then
    \begin{equation}
        \sigma=s_{\alpha_i}(\lambda)\in s_{\alpha_i}\big( \Phi^+\setminus\{ \alpha_i \} \big)\subset\Phi^+\setminus\{ \alpha_i \}
    \end{equation}
    and \( s_{\alpha_i}(\sigma)=\lambda\), so that \( \lambda\) is the image by \( s_{\alpha_i}\) of \( \sigma\in\Phi^+\setminus\{ \alpha_i \}\).
\end{proof}

\begin{theorem}      \label{ThosajBijSurpPpsmaj}
    The map \( s_{\alpha_j}\colon \Phi^+\setminus\{ \alpha_j \}\to \Phi^+\setminus\{ \alpha_j \}\) is bijective.
\end{theorem}

\begin{proof}
    Surjectivity is proposition~\ref{PropsalphaisurjPhipmaj} while injectivity is lemma~\ref{LemalphajsPhipinjsasbab}.
\end{proof}

\begin{lemma}[\cite{Cornwell}, page 533]
    We consider the half sum of the positive roots:
    \begin{equation}
        \delta=\frac{ 1 }{2}\sum_{\alpha\in\Phi^+}\alpha.
    \end{equation}
    We have
    \begin{enumerate}
        \item
            If \( \alpha_j\) is a simple root, \( s_{\alpha_j}\delta=\delta-\alpha_j\).
        \item
            If \( \alpha_j\) is a simple root, \( (\delta,\alpha_j)=\frac{ 1 }{2}(\alpha_j,\alpha_j)\).
    \end{enumerate}
\end{lemma}

\begin{proof}
    We compute \( s_{\alpha_j}\delta\) dividing the sum into two parts:
    \begin{subequations}
        \begin{align}
            s_{\alpha_j}\delta&=\frac{ 1 }{2}\sum_{\substack{\alpha\in\Phi^+\\\alpha\neq\alpha_j}}s_{\alpha_j}(\alpha)+\frac{ 1 }{2}s_{\alpha_j}(\alpha_j)\\
            &=\frac{ 1 }{2}\sum_{\substack{\alpha\in\Phi^+\\\alpha\neq\alpha_j}}\alpha-\frac{ 1 }{2}\alpha_j.
        \end{align}
    \end{subequations}
    The second inequality is from the fact that \( s_{\alpha_j}\) is bijective on \( \Phi^+\setminus\{ \alpha_j \}\) by theorem~\ref{ThosajBijSurpPpsmaj}. Adding a subtracting \( \frac{ \alpha_j }{2}\) we get
    \begin{equation}
        s_{\alpha_j}\delta=\frac{ 1 }{2}\sum_{\alpha\in\Phi^+}\alpha-\frac{ \alpha_j }{2}-\frac{ \alpha_j }{2}=\delta-\alpha_j
    \end{equation}

    Using the proposition~\ref{PropWeylIsomalphai}, we have
    \begin{equation}
        (\delta,\alpha_j)=(s_{\alpha_j}\delta,s_{\alpha_j}\alpha_j)=(\delta-\alpha_j,-\alpha_j)=-(\delta,\alpha_j)+(\alpha_j,\alpha_j),
    \end{equation}
    consequently, \( 2(\delta,\alpha_j)=(\alpha_j,\alpha_j)\) and the result follows.
\end{proof}

\subsection{Abstract root system}
%--------------------------------

The material about abstract root system  mainly comes from \cite{Knapp_reprez}.

\begin{definition}      \label{DefAbsRootSystSerre}
    An \defe{abstract root system}{abstract!root system}\index{root!abstract} in a finite dimensional vector space $V$ endowed with an is a subset $\Phi$ of $V$ such that
    \begin{itemize}
        \item \( \Phi\) is finite and $\Span\Phi=V$,
        \item
            For every \( \alpha\in \Phi\), there is a symmetry \( s_{\alpha}\) of vector \( \alpha\) leaving \( \Phi\) stable.
        \item
            For every \( \alpha,\beta\in\Phi\), the vector \( s_{\alpha}(\beta)-\beta\) is an integer multiple of \( \alpha\).
    \end{itemize}
\end{definition}
The abstract system is \defe{reduced}{reduced abstract root system} when $\alpha\in\Phi$ implies $2\alpha\notin\Phi$. It is \defe{irreducible}{irreducible!abstract root system} is $\Phi$ doesn't admits non trivial decomposition as $\Phi=\Phi'\cup\Phi''$ with $(\alpha,\beta)=0$ for any $\alpha\in\Phi'$ and $\beta\in\Phi''$. We use the notation $\Phi:=\Phi\cup\{0\}$.


The following is a consequence of all we did up to now.
\begin{theorem}
    The root system of a complex semisimple Lie algebra is a reduced abstract root system.
\end{theorem}

The \defe{Weyl group}{Weyl group!abstract setting} of \( \Phi\) is the subgroup of \( \GL(V)\) generated by the transformations \( s_{\alpha}\) with \( \alpha\in\Phi\).

%///////////////////////////////////////////////////////////////////////////////////////////////////////////////////////////
\subsubsection{Link with other definitions}
%///////////////////////////////////////////////////////////////////////////////////////////////////////////////////////////

The definition~\ref{DefAbsRootSystSerre} is not the ``usual'' one (in \cite{Wisser}, page 14 for example). We show now that we retrieve the usual features of an abstract.

\begin{lemma}
    An abstract root system admits a bilinear positive symmetric non degenerate form which is invariant under its Weyl group.
\end{lemma}

\begin{proof}
    If \( (.,.)_1\) is a bilinear positive non degenerate symmetric form on the vector space \( V\), the form
    \begin{equation}
        (\alpha,\beta)=\sum_{w\in W}(w\alpha,w\beta)_1
    \end{equation}
    is invariant under the Weyl group. This construction is possible since the Weyl group is finite.
\end{proof}

\begin{definition}
    Let \( V\) be a vector space and \( v\in V\) a non vanishing vector. A symmetry of vector \( v\) is an automorphism \( s\colon V\to V\) such that
    \begin{enumerate}
        \item
            \( s(v)=-v\);
        \item
            the set \( H=\{ w\in V\tq \alpha(w)=w \}\) is an hyperplane in \( V\).
    \end{enumerate}
\end{definition}
A symmetry of vector \( v\) induces the decomposition \( V=H\oplus\eR v\). The symmetries are of order \( 2\): \( s^2=\id\).

\begin{lemma}
    let \( v\) be a nonzero vector of \( V\) and \( A\) be a finite part of \( V\) such that \( \Span(A)=V\). Then there exists at most one symmetry of vector \( v\) leaving \( A\) invariant.
\end{lemma}

\begin{proof}
    Let \( s\) and \( s'\) be two such symmetries and consider \( u=ss'\). We immediately have \( u(A)=A\) and \( u(v)=v\). Let us prove that \( u\) induce the identity on the quotient \( V/\eR v\). A general vector in \( V\) can be written (in a non unique way) under the form
    \begin{equation}
        h+h'+v
    \end{equation}
    with \( h\in H\) and \( h'\in H'\). Let \( h=h'_1+\beta v\) be the decomposition of \( h\) in \( H'\oplus \eR v\) and \( h'=h_1+\gamma v\) be the decomposition of \( h'\) with respect to the direct sum \( V=H\oplus\eR v\).  Then we have
    \begin{subequations}
        \begin{align}
            ss'(h+h'+\alpha v)&=ss'\big( (h'_1+\beta v)+h'+\alpha v \big)\\
            &=s\big( (h'_1-\beta v)+h'+\alpha v \big)\\
            &=s(h-2\beta v+h_1+\gamma v+\alpha v)\\
            &=h+2\beta v+h_1-\gamma v+\alpha v\\
            &=h+h'+(\alpha-2\gamma+2\beta)v.
        \end{align}
    \end{subequations}
    Thus at the level of the quotient, $u$ leaves invariant \( h+h'\).

    It is not guaranteed that \( u\) is the identity, but the eigenvalues of \( u\) are \( 1\). For each \( x_i\in A\), there exists \( n_i\in\eN\) such that \( u^{n_i}x_i=x\). If \( n\) is a common multiple of all the \( n_i\) (these are finitely many), we have \( u^n(x)=x\) for every \( x\in A\). Since \( A\) generates \( V\), we have \( u^n=\id\) and then \( u\) is diagonalizable.

    We already mentioned the fact that the eigenvalues of \( u\) are \( 1\). Since \( u\) is diagonalizable, it is the identity and \( s=s'\).
\end{proof}

The invariant form give to \( V\) a structure of euclidian vector space for which the elements of the Weyl group are orthogonal matrices. Thus the symmetries read
\begin{equation}    \label{EqSymparnnusul}
    s_{\alpha}(x)=x-2\frac{ (x,\alpha) }{ (\alpha,\alpha) }\alpha.
\end{equation}
This is the only transformation which makes \( s_{\alpha}(\alpha)=-\alpha\) in the same time as being implemented by an orthogonal matrix. The symmetry \( s_{\alpha}\) is nothing else than the orthogonal symmetry with respect to the hyperplane orthogonal to \( \alpha\).

The expression \eqref{EqSymparnnusul} has the consequence that
\begin{equation}
    s_{\alpha}(\beta)-\beta=-\frac{ (\beta,\alpha) }{ (\alpha,\alpha) }\alpha.
\end{equation}
By the definition of an abstract root system, the latter has to be an integer multiple of \( \alpha\), so
\begin{equation}
    \frac{ 2(\beta,\alpha) }{ (\alpha,\alpha) }\in\eZ.
\end{equation}

\begin{definition}
    Two abstract root systems \( \Phi\) on \( V\) and \( \Phi'\) on \( V'\) are \defe{isomorphic}{isomorphism!of abstract root system} is there exists an isomorphism of vector space \( \psi\colon V\to V'\) such that \( \psi(\Phi)=\Phi'\) and
    \begin{equation}
        2\frac{ (\alpha,\beta) }{ (\alpha,\alpha) }=2\frac{ \big( \psi(\alpha),\psi(\beta) \big) }{ \big( \psi(\alpha),\psi(\alpha) \big) }
    \end{equation}
    for every \( \alpha,\beta\in \Phi\).
\end{definition}

%///////////////////////////////////////////////////////////////////////////////////////////////////////////////////////////
\subsubsection{Basis of abstract root system}
%///////////////////////////////////////////////////////////////////////////////////////////////////////////////////////////
The part about basis of abstract root system comes from \cite{SerreSSAlgebres}.

\begin{definition}      \label{DefbasisabsRoot}
    Let \( \Phi\) be an abstract root system. A part \( S\subset \Phi\) is a \defe{basis}{basis!of an abstract root system} of \( \Phi\) if
    \begin{enumerate}
        \item
            \( S\) is a basis of \( V\) as vector space;
        \item
            every \( \beta\in\Phi\) can be written under the form
            \begin{equation}
                \beta=\sum_{\alpha\in S}m_{\alpha}\alpha
            \end{equation}
            where \( m_{\alpha}\) are all integers of the same sign.
    \end{enumerate}
\end{definition}
The set \( \Delta\) of simple roots of the root system of a complex semisimple Lie algebra is a basis.

We are going to build a basis of an abstract root system. Let \( h\in V^*\) be such that \( \alpha(h)\neq 0\) for every  \( \alpha\in\Phi\) and define
\begin{equation}
    \Phi_h^+=\{ \alpha\in\Phi\tq \alpha(h)>0 \}.
\end{equation}
We have \( \Phi=\Phi^+_h\cup -\Phi_h^+\). We say that an element \( \alpha\in\Phi^+_h\) is \defe{decomposable}{decomposable!in an abstract root system} if there exist \( \beta,\gamma\in\Phi_h^+\) such that \( \alpha=\beta+\gamma\). We write \( S_h\) the set of undecomposable elements in \( \Phi^+_h\).

\begin{lemma}       \label{LemShPhihpCBLSh}
    Any element in \( \Phi^+_h\) is a linear combination with positive coefficients of elements of \( S_h\).
\end{lemma}

\begin{probleme}
    It seems to me that Serre's book\cite{SerreSSAlgebres} has a misprint here. At page V-11 he writes:
    \begin{quote}
        Tout élément de \( R^+_t\) est combinaison linéaire, à coefficients entiers \( \geq 0\) des éléments de S.
    \end{quote}
    Shouldn't he have written \( S_t\).
\end{probleme}

\begin{proof}
    Let \( I\) be the set of \( \alpha\in\Phi^+_h\) that cannot be written under such a decomposition. We choose \( \alpha\in I\) such that \( \alpha(h)\) is minimal. If \( \alpha\) is undecomposable, then \( \alpha\in S_h\) and the condition \( \alpha\in I\) is contradicted. Thus \( \alpha\) is decomposable. Let \( \beta,\gamma\in\Phi^+_h\) be such that \( \alpha=\beta+\gamma\). Since \( \alpha(h)\) is minimal,
    \begin{equation}
        \begin{aligned}[]
            \beta(h)&\leq \alpha(h)\\
            \gamma(h)&\leq \alpha(h).
        \end{aligned}
    \end{equation}
    Thus we have \( \beta(h)=\alpha(h)-\gamma(h)<0\) which contradicts \( \beta\in\Phi^+\). We conclude that \( I\) is empty.
\end{proof}

%///////////////////////////////////////////////////////////////////////////////////////////////////////////////////////////
\subsubsection{Properties}
%///////////////////////////////////////////////////////////////////////////////////////////////////////////////////////////

The main properties of an abstract root system are given in the  following proposition.
\begin{proposition}     \label{PropPropAbstrRootviiiikl}
If $\Phi$ is an abstract root system in a vector space $V$, one has the following properties:

\begin{enumerate}
\item\label{enubi} If $\alpha\in\Phi$ then $-\alpha\in\Phi$.

\item\label{enubii} If $\alpha\in\Phi$, the multiples of $\alpha$ which could also be in $\Phi$ are either $\pm\alpha$, or $\pm\alpha$ and $\pm 2\alpha$ or $\pm\alpha$ and $\pm\frac{1}{2}\alpha$.

\item\label{enubiii} If $\alpha\beta\in\Phi$ then $\frZ{\alpha}{\beta}$ can take the nonzero values $\pm 1$, $\pm 2$, $\pm 3$ or $\pm 4$. The case $\pm 4$ can only arise if $\beta=\pm 2\alpha$.

\item\label{enubiv} If $\alpha,\beta\in\Phi$ are not proportional each other and if $|\alpha|\leq|\beta|$, then $\frZ{\beta}{\alpha}$ equals $0$ or $\pm 1$.

\item\label{enubv} If $\alpha,\beta\in\Phi$ and $(\alpha,\beta)>0$, then $\alpha-\beta\in\Phi$ and if $(\alpha,\beta)<0$, the $\alpha+\beta\in\Phi$.

\item\label{enubvi} If $\alpha,\beta\in\Phi$ and neither $\alpha+\beta$ neither $\alpha-\beta$ belongs to $\Phi$, then $(\alpha,\beta)=0$.

\item\label{enubvii} If $\alpha\in\Phi$ and $\beta\in\Phi$, the $n\in\eZ$ such that $\beta+n\alpha\in\Phi$ fulfils $-p\leq n\leq q$ for certain $p,q\geq 0$. Moreover there are no gap,
\[
   p-q=\frZ{\alpha}{\beta},
\]
and there are at most four roots in the set $\{\beta+n\alpha\}_{-p\leq n\leq q}$.

\item\label{enubviii} If $\Phi$ is reduced,

\begin{enumerate}
\item\label{enubviiia} If $\alpha\in\Phi$, the only multiples of $\alpha$ to lies in $\Phi$ are $\pm\alpha$,
\item\label{enubviiib} If $\alpha\in\Phi$ and $\beta\in\Phi$, then $\frZ{\alpha}{\beta}$can be equal to $0$, $\pm 1$, $\pm 2$ or $\pm 3$.
\end{enumerate}
\end{enumerate} \label{prop:Cartan_matr}
\end{proposition}
The proof will not use the fact that $\Phi$ spans $V$.

\begin{proof}
\ref{enubi} $s_{\alpha}\alpha=-\alpha$.

\ref{enubii} If $\beta=c\alpha$ with $|c|<1$, then
\[
\frZ{\alpha}{\beta}=2c
\]
must belongs to $\eZ$, then $c=0,\pm\frac{1}{2}$. If $|c|>1$, we use exactly the same with $\alpha=\us{c}\beta$, so that $\us{c}=0;\pm\frac{1}{2}$. Now if $2\alpha$ is a root, it is clear that $\frac{1}{2}\alpha$ can't be.

If $\Phi$ is reduced, the fact that $\frac{1}{2}\alpha\in\Phi$ implies that $\alpha\notin\Phi$, so that $\pm\frac{1}{2}\alpha$ is excluded if $\alpha\in\Phi$, under the same assumption, $2\alpha$ is also excluded. This proves~\ref{enubviiia}.

\ref{enubiii} The Schwartz inequality $|(\alpha,\beta)|\leq|\alpha||\beta|$ gives
\[
\left|   \frZ{\alpha}{\beta}\frZ{\beta}{\alpha}     \right|\leq 4.
\]
The equality only holds for $\beta=c\alpha$. In this case, we just saw that $\frZ{\alpha}{\beta}=2c$ with $c=2$ at most. If the equality is strict, then $\frZ{\alpha}{\beta}$ and $\frZ{\beta}{\alpha}$ are two integers whose product is $\leq 3$. The possibilities are $0$, $\pm 1$, $\pm 2$, $\pm 3$.

\ref{enubiv} If $|\alpha|\leq|\beta|$, then the following integer inequality holds:
\[
\left|\frZ{\alpha}{\beta}   \right|\leq\left|\frZ{\beta}{\alpha}   \right|.
\]
Since the product of the two  is $\leq 3$, the smallest is $0$ or $1$.

\ref{enubv}
If $\beta=c\alpha$, then $c=\pm\frac{1}{2},\pm 2,\pm 1$. All the cases are easy. If $(\alpha,\beta)>0$, then $c>0$ and $\alpha-\beta= \alpha-\frac{1}{2}\alpha=\frac{1}{2}\alpha$ or $\alpha-\beta=\alpha-2\alpha=-\alpha$.

Then we can suppose that $\alpha$ and $\beta$ are not proportional each other. We consider $\alpha,\beta\in\Phi$ and $(\alpha,\beta)>0$ (the other case is proved in much the same way). We just saw in~\ref{enubiv} that $\frZ{\beta}{\alpha}$ could be equals to $0$ or $\pm 1$, then the fact that $(\alpha,\beta)>0$ imposes $\frZ{\beta}{\alpha}=1$, so that $s_{\beta}(\alpha)=\alpha-\beta$.

If $|\beta|\leq|\alpha|$, we use
\begin{equation}
s_{\alpha}(\beta)=\beta-\frZ{\beta}{\alpha}\alpha
               =\beta-\alpha,
\end{equation}

\ref{enubvi} is an immediate consequence of the previous point.

\ref{enubvii} Let $-p$ and $q$ be the smallest and the largest values of $n$ such that $\beta+n\alpha \in\Phi$. They exist because $\Phi$ is a finite set. Suppose that there is a gap between $r$ and $s$ ($r<s-1$), i.e. $\beta+r\alpha\in\Phi$, $\beta+s\alpha\in\Phi$, but $\beta+(r+1)\alpha,\beta+(s-1)\alpha\notin\Phi$.

By the point~\ref{enubv}, $(\beta+r\alpha,\alpha)\geq 0$ and $(\beta+s\alpha,\alpha)\leq 0$. Making the difference between these two inequalities,
\[
   (r-s)|\alpha|^2\geq 0,
\]
then $r\geq s$, which contradict the definition of $r$ and $s$. So there is no gap. Now let us compute
\begin{equation}
\begin{split}
   s_{\alpha}(\beta+n\alpha)&=\beta+n\alpha-\frZ{\alpha}{\beta+n\alpha}\alpha\\
                          &=\beta+n\alpha-\left(    \frZ{\alpha}{\beta}+2n    \right)\alpha\\
              &=\beta-n\alpha-\frZ{\alpha}{\beta}\alpha\in\Phi.
\end{split}
\end{equation}
Then for any $n$ in $-p\leq n\leq q$,
\[
   -q\leq n+\frZ{\alpha}{\beta}\leq p.
\]
With $n=q$, the second inequality gives $\frZ{\alpha}{\beta}\leq p-q$ while the first one with $n=-p$ gives  $p-q\leq\frZ{\alpha}{\beta}$.

The last point is to check the length of the string of root. We can suppose $q=0$ (i.e to look the string of $\beta-q\alpha$ instead of the one of $\alpha$; of course this is the same), then the length is $p+1$ and
\[
   p=\frZ{\alpha}{\beta}.
\]
If $\alpha$ and $\beta$ are not proportional, the point~\ref{enubiii} makes it equals at most to $3$. If they are proportional, then the possibilities are $\alpha=\pm\beta,\pm\frac{1}{2}\beta,\pm 2\beta$. The string $\beta+n\alpha$ with $\alpha=\beta$ is at most $\{\beta,2\beta\}$, if $\alpha=\frac{1}{2}\beta$, this is just $\{\beta\}$ and if $\alpha=2\beta$, this is $\{\beta,-\beta\}$.

The proof is complete.
\end{proof}

\begin{lemma}       \label{LemShabShablesz}
    If \( \alpha,\beta\in S_h\), then \( (\alpha,\beta)\leq 0\).
\end{lemma}

\begin{proof}
    If \( (\alpha,\beta)\geq 0\), then proposition~\ref{PropPropAbstrRootviiiikl}\ref{enubv} shows that \( \gamma=\alpha-\beta\) is a root. There are two possibilities: \( \gamma\in\pm\Phi^+_h\). If \( \gamma\in\Phi^+_h\), then \( \alpha=\gamma+\beta\) is decomposable; contradiction. If \( \gamma\in -\Phi^+_h\), then \( \beta=\alpha-\gamma\) is decomposable; contradiction.
\end{proof}

\begin{lemma}[Lemme 4 page V-12]        \label{LemIndepAhVstar}
    Let \( h\in V^*\) and \( A\subset V\) be a subset satisfying
    \begin{enumerate}
        \item
            \( \alpha(h)>0\) for every \( \alpha\in A\);
        \item
            \( (\alpha,\beta)\leq 0\) for every \( \alpha,\beta\in A\).
    \end{enumerate}
    Then the elements in \( A\) are linearly independent.
\end{lemma}

\begin{proof}
    Let us consider a vanishing linear combination of elements in \( A\):
    \begin{equation}        \label{EqNullCombinsumAmAuAd}
        \sum_{\alpha\in A}m_{\alpha}\alpha=0.
    \end{equation}
    We can sort the terms following that \( m_{\alpha}\) is positive or negative and cut the sum in two parts:
    \begin{equation}
        \sum_{\beta\in A_1}y_{\beta}\beta=\sum_{\gamma\in A_2}z_{\gamma}\gamma
    \end{equation}
    with \( y_{\beta},z_{\gamma}\geq 0\) and where \( A_1\) and \( A_2\) are disjoint subsets of \( A\). Let us consider \( \lambda=\sum_{\beta\in A_1}y_{\beta}\beta\) and compute
    \begin{equation}        \label{EqllamllamnprofAunAdeuxsom}
        (\lambda,\lambda)=\sum_{\substack{\beta\in A_1\\\gamma\in A_2}}y_{\beta}z_{\gamma}(\beta,\gamma).
    \end{equation}
    By hypothesis \( (\beta,\gamma)\) is lower than zero and by construction the product \( y_{\beta},z_{\gamma}\) is positive. Thus the right hand side of equation \eqref{EqllamllamnprofAunAdeuxsom} is negative. We conclude that \( \lambda=0\). Thus
    \begin{equation}
        0=\lambda(h)=\sum_{\beta\in A_1}y_{\beta}\beta(h).
    \end{equation}
    Since all the terms in the sum are larger than zero we have \( y_{\beta}=0\). In the same way we get \( z_{\gamma}=0\). The vanishing linear combination \eqref{EqNullCombinsumAmAuAd} is then trivial and the elements of \( A\) are linearly independent.
\end{proof}

\begin{proposition}\label{PropSestShsi}
    The elements of \( S_h\) form a basis of \( \Phi\) in the sense of definition~\ref{DefbasisabsRoot}. Conversely, if \( S\) is a basis of \( \Phi\) and if \( h\in V^*\) is such that \( \alpha(h)>0\) for every \( \alpha\in S\),we have \( S=S_h\).
\end{proposition}

\begin{proof}
    The set \( S_h\) satisfies the conditions of lemma~\ref{LemIndepAhVstar} since by definition \( \alpha(h)>0\) for every \( \alpha\in S_h\) and by lemma~\ref{LemShabShablesz} the inner products are all negative. Thus \( S_h\) is a free set. It is generating by lemma~\ref{LemShPhihpCBLSh}. Again by lemma~\ref{LemShPhihpCBLSh}, every element in \( \Phi\) can be written as sum of elements of \( S_h\) with all coefficients of the same sign. Here we use the fact that \( v\) is positive if and only if \( -v\) is negative and that every vector is either positive or negative.

    For the second part, let \( S\) be a basis and \( h\in V^*\) such that \( \alpha(h)>0\) for all \( \alpha\in S\). Let
    \begin{equation}
        \Phi^+=\{ \sum_{\alpha\in S}m_{\alpha}\alpha \text{ with } m_{\alpha}\in\eN \}.
    \end{equation}
    We have \( \Phi^+\subset\Phi_h^+\) and \( -\Phi^+\subset -\Phi_h^+\). Since \( \Phi=\Phi^+\cup-\Phi^+\) we also have \( \Phi^+=\Phi_h^+\). Since elements of \( S\) are indecomposable in \( \Phi^+\), they are indecomposable in \( \Phi^+_h\) and we have \( S\subset S_h\).

    The sets \( S\) and \( S_h\) have the same number of elements because they both are basis of \( V\), thus \( S=S_h\).
\end{proof}

\begin{lemma}\label{LemwShShpahwahp}
    If \( h\) and \( h'\) are elements of \( V^*\) related by \( \alpha(h)=(w\alpha)h'\), then \( w(S_h)=S_{h'}\) (if these space can be defined).
\end{lemma}

\begin{proof}
    Let \( \alpha\in S_h\). The element \( w(\alpha)\) belongs to \( \Phi_{h'}^+\) because
    \begin{equation}
        w(\alpha)h'=\alpha(h)>0
    \end{equation}
    because \( \alpha\in\Phi_h^+\). We still have to check that \( w(\alpha)\) is undecomposable in \( \Phi_{h'}^+\). If \( w(\alpha)=\beta+\gamma\) with \( \beta,\gamma\in\Phi_{h'}^+\), we have \( \alpha=w^{-1}\beta+w^{-1}\gamma\). From the link between \( h\) and \( h'\) we have
    \begin{equation}
        (w^{-1}\beta)(h)=(ww^{-1}\beta)h'=\beta(h')>0.
    \end{equation}
    Thus \( w^{-1}\beta\in \Phi_h^+\) which is a contradiction because we supposed that \( \alpha\) is undecomposable.
\end{proof}

\begin{lemma}\label{Lemswwsbwemucirc}
    If \( \alpha,\beta\in\Phi\) and if \( w\in W_S\), then \( s_{w(\beta)}=w\circ s_{\beta}\circ w^{-1}\).
\end{lemma}

\begin{proof}
    Using the fact that the symmetries are isometries of the inner product,
    \begin{equation}
        s_{w(\beta)}(\alpha)=\alpha-\frac{ \big( w(\beta),\alpha \big) }{ \big( w(\beta),w(\beta) \big) }w(\beta)=\alpha-\frac{ (\beta),w^{-1}\alpha }{ (\beta,\beta) }w\beta.
    \end{equation}
    Applying that to \( w(\alpha)\) instead of \( \alpha\) and applying \( w^{-1}\), we have
    \begin{subequations}
        \begin{align}
            w^{-1}s_{w(\beta)}\big(w(\alpha)\big)&=w^{-1}\left( w\alpha-\frac{ (\beta,w^{-1}w\alpha) }{ (\beta,\beta) }w\beta \right)\\
            &=\alpha-\frac{ (\beta,\alpha) }{ (\beta,\beta) }w^{-1}w\beta\\
            &=s_{\beta}(\alpha).
        \end{align}
    \end{subequations}
\end{proof}

\begin{theorem}[\cite{SerreSSAlgebres}]     \label{ThoWeylGenere}
    Let \( W\) be the Weyl group of the abstract root system \( \Phi\). Let \( S\) a basis of \( \Phi\) and \( W_S\) the subgroup of \( W\) generated by \( s_{\alpha}\) with \(\alpha\in S\). Then
    \begin{enumerate}
        \item   \label{ItemThoWeylGenerei}
            for every \( h\in V^*\), there exists \( w\in W_S\) such that \( (w\alpha)(h)\geq 0\) for every \( \alpha\in S\).
        \item   \label{ItemThoWeylGenereii}
            If \( S'\) is a basis of \( \Phi\), the there exists a \( w\in W_S\) such that \( w(S')=S\).
        \item\label{ItemThoWeylGenereiii}
            For every \( \beta\in\Phi\) there exists \( w\in W_S\) such that \( w(\beta)\in S\).
        \item\label{ItemThoWeylGenereiv}
            The group \( W\) is generated by the symmetries \( s_{\alpha}\) with \( \alpha\in S\).
    \end{enumerate}
\end{theorem}

\begin{proof}
    For item~\ref{ItemThoWeylGenerei}, consider \( h\in V^*\) and \( \delta=\frac{ 1 }{2}\sum_{\gamma\in S}\gamma\). Let \( w\in W_S\) be such that \( w(\delta)h\) is the largest possible\footnote{We can consider that \( w\) because \( W\) is finite.}. If \( \alpha\in S\) we have
    \begin{equation}
        w(\delta)h\geq ws_{\alpha}(\delta)h=w(\delta)h-w(\alpha)h,
    \end{equation}
    so that \( w(\alpha)h\geq 0\) for every \( \alpha\in S\). This proves our first assertion.

    We pass to point~\ref{ItemThoWeylGenereii}. Let \( h'\in V^*\) be such that \( \alpha'(h')> 0\) for every \( \alpha'\in S'\). By the first item there exists \( w\in W_S\) such that
    \begin{equation}
        (w\alpha)(h')\geq 0
    \end{equation}
    for every \( \alpha\in S\). In fact we even have \( w\alpha h'>0\) for every \( \alpha\in S\). Indeed \( w\alpha\) can be decomposed as \( \sum_{\alpha'\in S'}m_{\alpha'}\alpha'\) where all the \( m_{\alpha'}\) have the same sign. In this case
    \begin{equation}        \label{Eqwapsummapaphp}
        (w\alpha)h'=\sum_{\alpha'}m_{\alpha'}\alpha'(h')\neq 0
    \end{equation}
    because each of \( \alpha'(h')\) is strictly positive while all the terms of the sum have the same sign. This means, by the way, that \( S'=S_{h'}\) following the proposition~\ref{PropSestShsi}.

    We define \( h\in V^*\) by the relation
    \begin{equation}
        \alpha(h)=(w\alpha)(h').
    \end{equation}
    By what we said in equation \eqref{Eqwapsummapaphp} we have \( \alpha(h)>0\) for every \( \alpha\in S\), so that we have \( S=S_h\). Finally by lemma~\ref{LemwShShpahwahp}, \( w(S_h)=S_{h'}\).

    We prove now the point~\ref{ItemThoWeylGenereiii}. For \( \gamma\in\Phi\) we consider the hyperplane
    \begin{equation}
        L_{\gamma}=\{ h\in V^*\tq \gamma(h)=0 \}.
    \end{equation}
    Consider a particular \( \beta\in\Phi\) the hyperplanes \( L_{\gamma}\) with \( \gamma\neq\pm\beta\) do not coincide with \( L_{\beta}\) and there is only finitely many of them, so there exists a \( h_0\in L_{\beta}\) such that \( h_0\) do not belong to any \( L_{\gamma}\) for any \( \gamma\neq \pm\beta\).
    In particular we have \( \beta(h_0)=0\) and \( \gamma(h_0)\neq 0\) for every \( \gamma\in\Phi\), \( \gamma\neq\pm\beta\). If we choose \( \epsilon\) small enough, there exists \( h\) near from \( h_0\) such that
    \begin{subequations}
        \begin{numcases}{}
            \beta(h)=\epsilon>0\\
            | \gamma(h) |>\epsilon&if $\gamma\neq \pm\beta$.
        \end{numcases}
    \end{subequations}
    Let \( S_{h}\) be the basis associated with this \( h\). We have \( \beta\in S_h\). Indeed first \( \beta(h)=\epsilon>0\) and if \( \beta=\gamma+\rho\), we would have
    \begin{equation}
        \gamma(h)=\beta(h)-\rho(h)=\epsilon-\rho(h)<0,
    \end{equation}
    so that \( \beta\) is undecomposable in \( \Phi_h^+\). Now from point~\ref{ItemThoWeylGenereii} there exists \( w\in W_S\) such that \( w(S_h)=S\). In particular \( w(\beta)\in S\).

    We turn our attention to the item~\ref{ItemThoWeylGenereiv}. We are going to prove that \( W=W_S\). Since \( W\) is generated by the symmetries \( s_{\beta}\) (\( \beta\in\Phi\)), it is sufficient to prove that \( W_S\) generates the symmetries \( s_{\beta}\).

    Let \( \beta\in\Phi\) and consider the element \( w\in W_S\) such that \( \alpha=w(\beta)\in S\). From lemma~\ref{Lemswwsbwemucirc} we have
    \begin{equation}
        s_{\alpha}=s_{w(\beta)}=w\circ s_{\beta}\circ w^{-1},
    \end{equation}
    so that
    \begin{equation}
        s_{\beta}=w^{-1}\circ s_{\alpha}\circ w\in W_S.
    \end{equation}
\end{proof}

What this theorem says in the case of complex semisimple Lie algebras is that if \( \{ \alpha_1,\ldots,\alpha_l \}\) is the set of simple roots, the symmetries \( s_{\alpha_i}\) generate the Weyl group. Now, since any root can be mapped on a simple one using the Weyl group, any root can be recovered from a simple one acting with the Weyl group that is generated by the simple ones.

Thus one can determine all the roots from the data of the simple ones by computing \( s_{\alpha_i}\alpha_j\) and then acting again with the \( s_{\alpha_i}\) on the results and again and again. This is the fundamental reason from which the root system can be recovered for the Cartan matrix.


When we have the Cartan matrix \( A \) of a semisimple complex Lie algebra, the first point is to find the norm of the roots by finding the diagonal matrix \( D\). We have \( (\alpha_i,\alpha_i)=D_{ii}\). For the other products we write
\begin{equation}
    A_{ij}=\frac{ 2(\alpha_i,\alpha_j) }{ D_{ii} },
\end{equation}
thus
\begin{equation}
    (\alpha_i,\alpha_j)=\frac{ D_{ii}A_{ij} }{ 2 }.
\end{equation}

%---------------------------------------------------------------------------------------------------------------------------
\subsection{Abstract Cartan matrix}
%---------------------------------------------------------------------------------------------------------------------------

The following proposition summarize the properties of the of the Cartan matrix.
\begin{definition}      \label{DeabstrCartanmatr}
    A matrix \( (A_{ij})_{1\leq i,j\leq l}\) satisfying the following conditions is an \defe{abstract Cartan matrix}{Cartan!matrix!abstract}\index{abstract!Cartan matrix}
    \begin{enumerate}
        \item
            \( A_{ij}\in\eZ\),
        \item
            \( A_{ii}=2\),
        \item   \label{ItempoprCartaniii}
            \( A_{ij}\leq 0\) if \( i\neq j\),
        \item
            \( A_{ij}=0\) if and only if \( A_{ji}=0\),
        \item\label{ItempoprCartanv}
            there exists a diagonal matrix \( D\) with positive coefficients such that \( DAD^{-1}\) is symmetric and positive defined.
    \end{enumerate}
\end{definition}
The classification of abstract Cartan matrix will be performed in subsection~\ref{SubsecDynkindiam}. The data of an abstract Cartan matrix defines an abstract root system. For a proof, see \cite{CartanRootProject}.

\begin{proposition}
    The Cartan matrix of a semisimple complex Lie algebra is an abstract Cartan matrix.
\end{proposition}

\begin{proof}
    The first two points are already done. For the point~\ref{ItempoprCartaniii}, note that the sign of \( (\alpha,\beta)\) is not sure when \( \alpha\) is any root. However here we are speaking of simple roots. Let us consider the root
    \begin{equation}
        \lambda=\alpha_i-\frac{ 2(\alpha_i,\alpha_j) }{ (\alpha_i,\alpha_i) }\alpha_i
    \end{equation}
    Since it is a root, proposition~\ref{ThoposrootnjajnZ} says that the coefficients in the decomposition in simple roots have to be all integer and of the same sign. Thus the combination \( (\alpha_i,\alpha_j)/(\alpha_i,\alpha_i)\) has to be negative.

    The point~\ref{ItempoprCartanv} is also non trivial. Consider the diagonal matrix \( D=\diag\big( (\alpha_i,\alpha_i) \big)_{i=1,\ldots,l}\). We have
    \begin{subequations}
        \begin{align}
            (DAD^{-1})_{ij}&=\sum_{kl}D_{ik}A_{kl}(D^{-1})_{lj}\\
            &=\frac{ 2(\alpha_i,\alpha_j) }{ (\alpha_i,\alpha_i)^{1/2}(\alpha_j,\alpha_j)^{1/2} }.
        \end{align}
    \end{subequations}
    This is a symmetric matrix. In order to proof that this is positive defined, we are going to provide a matrix \( B\) such that \( DAD^{-1}=BB^t\). Let \( \{ \lambda_i \}\) be an orthonormal basis of \( \lH^*\) and consider the matrix \( b\) given by the decomposition of the simple roots in this basis:
    \begin{equation}
        \alpha_i=\sum_j b_{ij}\lambda_j.
    \end{equation}
    In particular we have \( (\alpha_i,\alpha_j)=\sum_kb_{ik}b_{jk}\). Then we consider the matrix
    \begin{equation}
        B_{ij}=\frac{ b_{ij} }{ (\alpha_i,\alpha_i)^{1/2} }
    \end{equation}
    which is non degenerate since the \( \alpha_i\) are simple and thus a re linearly independent. Small computation shows that
    \begin{subequations}
        \begin{align}
            (BB^t)_{ij}&=\sum_k\frac{ b_{ik} }{ (\alpha_i,\alpha_i)^{1/2} }\frac{ b_{jk} }{ (\alpha_j,\alpha_j)^{1/2} }\\
            &=\frac{ (\alpha_i,\alpha_j) }{ (\alpha_i,\alpha_i)^{1/2}(\alpha_j,\alpha_j)^{1/2} }\\
            &=(DAD^{-1})_{ij}.
        \end{align}
    \end{subequations}
    But \( BB^t\) is positive defined, then \( DAD^{-1}\) is.
\end{proof}


%---------------------------------------------------------------------------------------------------------------------------
\subsection{Dynkin diagrams}
%---------------------------------------------------------------------------------------------------------------------------
\label{SubsecDynkindiam}

The sources for Dynkin diagrams is \cite{SternLieAlgebra,Wisser}.

We are going to associate to each abstract Cartan matrix, a diagram that will uniquely correspond to an abstract root system. In other words what we are going to do is to classify the matrix satisfying the conditions of definition~\ref{DeabstrCartanmatr}.

If \( A\) is an abstract Cartan matrix we build the \defe{Dynkin diagram}{Dynkin diagram} of \( A\) with the following rules.
\begin{enumerate}
    \item
        We put \( l\) vertices (one for each root)
    \item
        The vertex \( i\) and \( j\) are joined with \( A_{ij}A_{ji}\) lines.
\end{enumerate}
A step by step construction is available in \cite{Wisser}.

In the following we are considering an abstract Cartan matrix \( A\) and its associated abstract root system \( \{ \alpha_i \}\).

\begin{lemma}   \label{LesmabsCartDynk}
    A abstract Cartan matrix with its abstract root system and its Dynkin diagram have the following properties.
    \begin{enumerate}
        \item\label{ItemLesmabsCartDynki}
            If one remove the \( i\)th line an column of an abstract Cartan matrix, one still has an abstract Cartan matrix. In other words, each subdiagram of a Dynkin diagrm is a Dynkin diagram.
        \item\label{ItemLesmabsCartDynkii}
            Two vertices are linked by \emph{at most} three lines.
        \item\label{ItemLesmabsCartDynkiii}
            Each Dynkin diagram has more vertices than linked pairs.
        \item\label{ItemLesmabsCartDynkiv}
            A Dynkin diagram has no loop.
        \item\label{ItemLesmabsCartDynkv}
            A vertex in a Dynkin diagram has at most three lines attached (including multiplicities). Note: this is a generalization of point~\ref{ItemLesmabsCartDynkii}.
        \item\label{ItemLesmabsCartDynkvi}
            Two root linked by a simple edge have equal \defe{weight}{weight!in a Dynkin diagram}, that is \( (\alpha_i,\alpha_i)=(\alpha_j,\alpha_j)\).
        \item\label{ItemLesmabsCartDynkvii}
            If the two roots \( \alpha_i\), \( \alpha_i\) are connected by a simple edge, we can collapse them, removing the connecting edge and conserving all the other edges.

    \end{enumerate}
\end{lemma}

\begin{proof}
    For point~\ref{ItemLesmabsCartDynkii} we have
    \begin{equation}
        A_{ij}A_{ji}=4\frac{ (\alpha_i,\alpha_j) }{ (\alpha_i,\alpha_i) }\frac{ (\alpha_j,\alpha_i) }{ (\alpha_j,\alpha_j) }<4
    \end{equation}
    by Cauchy-Schwarz inequality. We insist on the fact that the inequality is strict since \( \alpha_i\) and \( \alpha_j\) are not collinear: they are simple roots.

    For point~\ref{ItemLesmabsCartDynkiii} consider the form
    \begin{equation}
        \gamma=\sum_{i=1}^l\alpha_i(\alpha_i,\alpha_i)^{1/2}.
    \end{equation}
    Since the simple roots are linearly independent, this sum is nonzero and we have \( 0<(\gamma,\gamma)\). We have
    \begin{equation}
        \begin{aligned}[]
            0<(\gamma,\gamma)&=\sum_{ij}\frac{ (\alpha_i,\alpha_j) }{ \sqrt{(\alpha_i,\alpha_i)(\alpha_j,\alpha_j)} }\\
            &=2\sum_{i<j} \frac{ (\alpha_i,\alpha_j) }{ \sqrt{(\alpha_i,\alpha_i)(\alpha_j,\alpha_j)} }+\text{number of nodes}\\
            &=-\sum_{i<j}(A_{ij}A_{ji})^{1/2}+\text{number of nodes}.
        \end{aligned}
    \end{equation}
    Since for each linked pair \( (i,j)\) we have a term \( A_{ij}A_{ji}\geq 1\), we have 
    \begin{equation}
    -\sum_{i<j}(A_{ij}A_{ji})^{1/2}\leq \text{number of pairs}
    \end{equation}
    and the positivity of the sum shows that
    \begin{equation}
        \text{number of nodes}>\sum_{ij}A_{ij}A_{ji}\geq\text{number of pairs}.
    \end{equation}

    For item~\ref{ItemLesmabsCartDynkiv}, suppose that a loop is given by the roots \( \alpha_1,\ldots,\alpha_n\). Since any sub-Dynkin diagram is a Dynkin diagram (from point~\ref{ItemLesmabsCartDynki}), we can consider only the loop. This is a diagram with \( n\) vertices and \( n\) pairs, which contradicts point~\ref{ItemLesmabsCartDynkiii}.

    We pass to item~\ref{ItemLesmabsCartDynkv}. Let \( \alpha_0\) be a root linked to \( n\) simple lines, \( m\) double lines and \( p\) triple lines. For notational convenience, we write \( v_i=\alpha_i/(\alpha_i,\alpha_i)\), \( \{ v_i \}_{1\leq i\leq n}\) is the set of ``simply'' linked roots to \( \alpha_0\), \( \{ v'_i \}_{1\leq i\leq m}\) the set of ``doubly'' linked and \( \{ v''_i \}_{1\leq i\leq p}\) the set of ``triply'' ones. Consider the vector
    \begin{equation}
        \gamma=v_0+\sum_{i=1}^nf_iv_i+\sum_{i=1}^mg_iv'_i+\sum_{i=1}^ph_iv''_i
    \end{equation}
    where \( f_i\), \( g_i\) and \( h_i\) are constant to be determined. In order to compute the norm of \( \gamma\), notice that since there are no loops, no lines join \( v_i\), \( v'_i\) and \( v''_i\) together, so we have \( (v_i,v'_j)=(v_i,v''_j)=(v'_i,v''_j)=0\) and from the number of lines, \( (v_0,v_i)=-1/2\), \( (v_0,v'_i)=-1/\sqrt{2}\) and \( (v_0,v''_i)=-\sqrt{3}/2\). Thus we have
    \begin{equation}
        (\gamma,\gamma)=1+\sum_{i=1}^m(f_i^2-f_i)+\sum_{i=1}^m(g_i^2-\sqrt{2}g_i)+\sum_{i=1}^p(h_i^2-\sqrt{3}h_i).
    \end{equation}
    The minimum is realised with \( f_i=1/2\), \( g_i=\sqrt{2}/2\) and \( h_i=\sqrt{3}/2\) and for these values we have
    \begin{equation}
        (\gamma,\gamma)=1-\frac{ n+2m+3p }{ 4 }.
    \end{equation}
    Since the inner product has to be positive we must have \( n+2m+3p<4\), the is the number of lines issued from \( \alpha_0\) has to be lower or equal to \( 3\).

    In order to proof~\ref{ItemLesmabsCartDynkvi}, remark that if \( \alpha_i\) and \( \alpha_j\) are connected by a simple edge, then \( A_{ij}A_{ji}=1\), which is only possible with \( A_{ij}=A_{ji}=-1\). In particular we have \( 2(\alpha_i,\alpha_j)/(\alpha_i,\alpha_i)=2(\alpha_j,\alpha_i)/(\alpha_j,\alpha_j)\), which proves that \( (\alpha_i,\alpha_i)=(\alpha_j,\alpha_j)\).

    Proof of item~\ref{ItemLesmabsCartDynkvii}. Since the two roots have same weight, the item~\ref{ItemLesmabsCartDynkvi} says that up to permutation the Cartan matrix has a block \( 2\times 2\) looking like
    \begin{equation}
        \begin{pmatrix}
            2    &   -1    \\
            -1    &   2
        \end{pmatrix}.
    \end{equation}
    The proposed move consist to replace that block with the \( 1\times 1\) matrix \( (2)\). As an example,
    \begin{equation}
        \begin{pmatrix}
             2   &   -1    &   0    &   0    \\
             -1   &   2    &   -1    &   -1    \\
             0   &   -1    &   2    &   0    \\
             0   &   -1    &   0    &   2
         \end{pmatrix}\mapsto
         \begin{pmatrix}
             2   &   -1    &   -1    \\
             -1   &   2    &   0    \\
             -1   &   0    &   2
         \end{pmatrix}.
    \end{equation}
    It is clear that the obtained matrix is still an abstract Cartan matrix.
\end{proof}

From these properties we can deduce much constrains on the Dynkin diagrams. First, the only diagram containing a triple edge is
\begin{equation}
    \xymatrix{%
    \alpha_1 \ar@3{-}[r]        &   \alpha_2
       }
\end{equation}

Let pass to the diagrams with only simple and double edges. If there is a double, there cannot be a triple point: the following is impossible
\begin{equation}
    \xymatrix{%
         &                          &           &               &       \alpha_5\\
        \alpha_1 \ar@2{-}[r]   &    \alpha_2 \ar@{-}[r] & \alpha_3\ar@{-}[r]&  \alpha_4 \ar@{-}[ru]\ar@{-}[rd]\\
        &&&&\alpha_6
       }
\end{equation}
since collapsing the roots \( \alpha_2\), \( \alpha_3\) and \( \alpha_4\) should create a point with four edges. Thus a diagram with a double edge is only possible inside a straight chain. Let us study the diagram
\begin{equation}        \label{EqdiaguduuuDy}
    \xymatrix{%
    \alpha_1 \ar@1{-}[r]&\alpha_2  \ar@2{-}[r]   &    \alpha_3 \ar@{-}[r] & \alpha_4\ar@{-}[r]&  \alpha_5
       }
\end{equation}
Once again we denote \( v_i=\alpha_i/| \alpha_i |\) and we consider the (non vanishing) vector
\begin{equation}
    \gamma=v_1+bv_2+cv_3+dv_4+ev_5
\end{equation}
whose norm is given by
\begin{equation}
    (\gamma,\gamma)=1+b^2+c^2+d^2+e^2-b-\sqrt{2}bc=cd=de.
\end{equation}
Equating all the partial derivative to zero provides the point
\begin{equation}
    \begin{aligned}[]
        b&=2&c&=\frac{ 3 }{ \sqrt{2} }&d&=\sqrt{2}&e&=\frac{1}{ \sqrt{2} }.
    \end{aligned}
\end{equation}
One check that with these values \( (\gamma,\gamma)=0\) which is impossible. The diagram \eqref{EqdiaguduuuDy} is thus impossible. By the collapsing principle, all the diagrams of the form
\begin{equation}
    \xymatrix{%
    \alpha_1 \ar@1{-}[r]&\alpha_2  \ar@2{-}[r]   &    \alpha_3 \ar@{-}[r] & \alpha_4\ar@{-}[r]& \ldots \ar@{-}[r]&  \alpha_l
       }
\end{equation}
are impossible. The only possible diagrams with double edge are thus
\begin{subequations}
    \begin{align}
    \xymatrix{%
    \alpha_1 \ar@1{-}[r]&\alpha_2  \ar@2{-}[r]   &    \alpha_3 \ar@{-}[r] & \alpha_4
       }\\
    \xymatrix{%
    \alpha_1 \ar@2{-}[r]&\alpha_2  \ar@1{-}[r]   & \ldots \ar@{-}[r] & \alpha_l
    }    \label{subEqDynkdspds}\\
    \xymatrix{%
    \alpha_1 \ar@{-}[r]&\alpha_2  \ar@1{-}[r]   & \ldots \ar@{-}[r] & \alpha_{l-1}\ar@2{-}[r]&\alpha_l.
    }    \label{subEqdunksspd}
    \end{align}
\end{subequations}
The diagrams \eqref{subEqDynkdspds} and \eqref{subEqdunksspd} are the same. They however do not completely determine the abstract Cartan matrix because the diagram \eqref{subEqdunksspd} induces an asymmetry between \( \alpha_1\) and \( \alpha_2\). The so written Dynkin diagram cannot distinguish between the matrices
\begin{equation}
    \begin{aligned}[]
        \begin{pmatrix}
            2    &   -2    &   0    \\
            -1    &   2    &   -1    \\
            0    &   -1    &   2
        \end{pmatrix}&&
        \text{and}&&
        \begin{pmatrix}
            2    &   -1    &   0    \\
            -2    &   2    &   -1    \\
            0    &   -1    &   2
        \end{pmatrix}&
    \end{aligned}
\end{equation}
Thus we split the diagram \eqref{subEqdunksspd} into
\begin{subequations}        \label{suBeqdfynkabGP}
    \begin{align}
    \xymatrix{%
    \alpha_1 \ar@{-}[r]&\alpha_2  \ar@1{-}[r]   & \ldots \ar@{-}[r] & \alpha_{l-1}\ar@2{->}[r]&\alpha_l.
    }   \\
    \xymatrix{%
    \alpha_1 \ar@{-}[r]&\alpha_2  \ar@1{-}[r]   & \ldots \ar@{-}[r] & \alpha_{l-1}\ar@2{<-}[r]&\alpha_l.
    }
    \end{align}
\end{subequations}
In which the arrow points to the biggest root. The first one means that \( | \alpha_1 |=\ldots=| \alpha_{l-1} |=1\), \( \alpha_{l}=2\) while the second diagram means \( | \alpha_1 |=\ldots=| \alpha_{l-2} |=| \alpha_l |=1\), \( \alpha_{l-1}=2\).

We'll have to come back on this point later in subsection~\ref{subsecRecbyhanfd}. Notice that this is the only diagram on which that problem occurs.

We are left to study the diagrams with only single edge. The following diagram is the simplest possible one:
\begin{equation}
    \xymatrix{%
    \alpha_1 \ar@{-}[r]&\alpha_2  \ar@1{-}[r]   & \ldots \ar@{-}[r] & \alpha_{l}.
       }
\end{equation}
We have to know under what conditions one can have a triple point. We already know that there can be only one triple point.

If a diagram has a triple point, then one of the branch is of length \( 1\). Indeed if not we would have the following diagram:
\begin{equation}
    \xymatrix{%
    &                          &                       &       \alpha_2\ar@{-}[r]&\alpha_5\\
        \alpha_7 \ar@{-}[r]   &    \alpha_4 \ar@{-}[r] & \alpha_1 \ar@{-}[ru]\ar@{-}[rd]\\
        &&&\alpha_6\ar@{-}[r]&\alpha_6
       }
\end{equation}
Looking at the vector \( \gamma=3v_1+2(v_2+v_3+v_4)+v_5+v_6+v_7\) provides \( (\gamma,\gamma)=-3\) which is impossible. Thus the diagrams with a branch are straight chains with one unique triple point which has a branch of length one. The question is: where can happen that branch? The diagram
\begin{equation}
    \xymatrix{%
    \alpha_1 \ar@{-}[r]&\alpha_2  \ar@1{-}[r]&\alpha_3  \ar@1{-}[r]&\alpha_4  \ar@1{-}[r]\ar@{-}[d]&\alpha_5  \ar@1{-}[r]&\alpha_6  \ar@1{-}[r]&\alpha_7\\
    &&&\alpha_8
       }
\end{equation}
cannot happen since the corresponding vector \( v_1+2v_2+3v_3+4v_4+3v_5+2v_6+v_7+2v_8\) has norm zero. Thus on a triple point, one branch has one branch of length \( 1\) and at least one other to be of length \( 1\) or \( 2\). It turns out that all the diagrams of the form
\begin{equation}
    \xymatrix{%
    &                          &                       &       \alpha_{l-1}\\
    \alpha_1 \ar@{-}[r]   &    \ldots \ar@{-}[r] & \alpha_{l-2} \ar@{-}[ru]\ar@{-}[rd]\\
        &&&\alpha_l
       }
\end{equation}
are possible. We are thus left with diagrams with a triple point with a branch of length \( 1\) and a branch of length \( 2\):
\begin{equation}
    \xymatrix{%
    \alpha_1 \ar@{-}[r]&\alpha_2  \ar@1{-}[r]&\alpha_3  \ar@1{-}[r]\ar@{-}[d]&\alpha_5  \ar@1{-}[r]&\ldots  \ar@1{-}[r]&\alpha_l \\
    &&\alpha_4
       }
\end{equation}
The diagram with a branch of length \( 5\)
\begin{equation}
    \xymatrix{%
    \alpha_1 \ar@{-}[r]&\alpha_2  \ar@1{-}[r]&\alpha_3  \ar@1{-}[r]\ar@{-}[d]&\alpha_5  \ar@1{-}[r]   &\alpha_6  \ar@1{-}[r]   &\alpha_7  \ar@1{-}[r]    &\alpha_8 \ar@1{-}[r]&\alpha_9  \\
    &&\alpha_4
       }
\end{equation}
does not exist. We achieve the proof of that fact using for example this code for \href{http://www.sagemath.org}{sage}:
\begin{verbatim}
----------------------------------------------------------------------
| Sage Version 4.7.1, Release Date: 2011-08-11                       |
| Type notebook() for the GUI, and license() for information.        |
----------------------------------------------------------------------
sage: a=[var('a'+str(i-1)) for i in range(1,11)]
sage: l=9
sage: a[1]=1
sage: squares = sum( [a[i]**2 for i in range(1,l+1)] )     # The sum goes to l

# The sum goes up to l-2
sage: lines = sum(  [a[i]*a[i+1] for i in range(1,l-1)  ]   )+a[3]*a[9]  
sage: f=symbolic_expression(squares - lines)
sage: X = solve( [f.diff(a[i])==0 for i in range(2,l+1)],
                [ a[i] for i in range(2,l+1)  ]  )
sage: print X[0]
[a2 == 2, a3 == 3, a4 == (5/2), a5 == 2, 
a6 == (3/2), a7 == 1, a8 == (1/2), 
a9 == (3/2)]
sage: f(*tuple(  [  X[0][i].rhs() for i in range(0,l-1)]) )
0
\end{verbatim}
This proves that the vector \( v_1+2v_2+3a_3+\frac{ 5 }{2}v_4+2v_5+\frac{ 3 }{2}v_6+v_7+\frac{ 1 }{2}v_8+\frac{ 3 }{2}v_9\) has vanishing norm, which is impossible.

\begin{probleme}
    This code raises a deprecation warning that I'm not able to solve.
\end{probleme}

We are finally left with the diagrams with one triple point with one branch of length \( 1\), one branch of length \( 2\) and the third branch with length \( 1\), \( 2\), \( 3\) or \( 4\):
\begin{subequations}
    \begin{align}
        &\xymatrix{%
        \alpha_1 \ar@{-}[r]&\alpha_2  \ar@1{-}[r]&\alpha_3  \ar@1{-}[r]\ar@{-}[d]&\alpha_4   \\
        &&\alpha_5
           }\\
        &\xymatrix{%
        \alpha_1 \ar@{-}[r]&\alpha_2  \ar@1{-}[r]&\alpha_3  \ar@1{-}[r]\ar@{-}[d]&\alpha_4  \ar@1{-}[r]& \alpha_5\\
        &&\alpha_6
        }\\
        &\xymatrix{%
        \alpha_1 \ar@{-}[r]&\alpha_2  \ar@1{-}[r]&\alpha_3  \ar@1{-}[r]\ar@{-}[d]&\alpha_4  \ar@1{-}[r]&\alpha_5  \ar@1{-}[r]& \alpha_6\\
        &&\alpha_7
        }\\
        &\xymatrix{%
        \alpha_1 \ar@{-}[r]&\alpha_2  \ar@1{-}[r]&\alpha_3  \ar@1{-}[r]\ar@{-}[d]&\alpha_4  \ar@1{-}[r]&\alpha_5  \ar@1{-}[r]& \alpha_6  \ar@1{-}[r]&\alpha_7\\
        &&\alpha_8
        }
    \end{align}
\end{subequations}

In order to list all the possible complex semisimple Lie algebra, we have to check each of the left Dynkin diagrams if they give rise to an abstract Cartan matrix.

\lstinputlisting{tex/research/calculs.py}

%---------------------------------------------------------------------------------------------------------------------------
\subsection{Example of reconstruction by hand}
%---------------------------------------------------------------------------------------------------------------------------
\label{subsecRecbyhanfd}

We turn now our attention on the difference between the two diagrams \eqref{suBeqdfynkabGP}. The Cartan matrix of the diagram $
        \xymatrix{%
        \alpha_1 \ar@{-}[r]&\alpha_2 \ar@2{->}[r]&\alpha_3
        }   $ is given by
        \begin{equation}
        A=\begin{pmatrix}
            2    &   -1    &   0    \\
            -1    &   2    &   -2    \\
            0    &   -1    &   2
       \end{pmatrix}.
        \end{equation}
The diagonal matrix \( D\) of definition~\ref{DeabstrCartanmatr} is
\begin{equation}
    D=\begin{pmatrix}
        1    &       &       \\
            &   1    &       \\
            &       &   2
    \end{pmatrix}
\end{equation}
and the length of the roots are \( \| \alpha_1 \|=\| \alpha_2 \|=1\) and \( | \alpha_3 |=2\). Let us compute the angles between the roots. In order to compute \( (\alpha_1,\alpha_2)\) we look at \( A_{12}\):
\begin{equation}
    A_{12}=-1=2\frac{ (\alpha_1,\alpha_2) }{ (\alpha_1,\alpha_1) },
\end{equation}
and the same computation with \( A_{23}\) provides
\begin{subequations}
    \begin{align}
        (\alpha_1,\alpha_2)&=-\frac{ 1 }{2}\\
        (\alpha_2,\alpha_3)&=-1
    \end{align}
\end{subequations}
We compute all the roots using the theorem~\ref{ThoWeylGenere} which basically says that acting with the ``simple'' Weyl group \( W_S\) on the simple roots generates all the roots. On the first strike we have
\begin{equation}
    \begin{aligned}[]
        s_1(\alpha_2)&=\alpha_2+\alpha_1&s_2(\alpha_1)&=\alpha_1+\alpha_2&s_3(\alpha_1)&=\alpha_1\\
        s_1(\alpha_3)&=\alpha\alpha_3 &s_2(\alpha_3)&=\alpha_3+2\alpha_2&s_3(\alpha_2)&=\alpha_2+\alpha_3.
    \end{aligned}
\end{equation}
We discovered the roots \( \alpha_2+\alpha_1\), \( \alpha_3+2\alpha_2\) and \( \alpha_2+\alpha_3\). Acting again on these roots by \( s_{\alpha_1}\), \( s_{\alpha_2}\) and \( s_{\alpha_3}\) the only new results are
\begin{equation}
    \begin{aligned}[]
        s_1(\alpha_3+\alpha_2)&=\alpha_1+\alpha_2+\alpha_3\\
        s_1(\alpha_3+2\alpha_2)&=2\alpha_1+2\alpha_2+\alpha_3.
    \end{aligned}
\end{equation}
Acting again we find only one new root:
\begin{equation}
    s_{\alpha_2}(\alpha_1+\alpha_2+\alpha_3)=\alpha_1+2\alpha_2+\alpha_3.
\end{equation}
We check that acting once again with the three simple roots on this last one does not brings new roots. Thus we have \( 9\) positive roots. Adding the negative ones, we are left with \( 18\) root spaces of dimension one. The Cartan algebra has dimension \( 3\), so the algebra we are looking at has dimension \( 21\).

Now take a look at the similar Dynkin diagram and its Cartan matrix:
\begin{subequations}
    \begin{align}
        \xymatrix{%
        \alpha_1 \ar@{-}[r]&\alpha_2 \ar@2{<-}[r]&\alpha_3
        }   &
        &A&=\begin{pmatrix}
            2    &   -1    &   0    \\
            -1    &   2    &   -1    \\
            0    &   -2    &   2
       \end{pmatrix}
    \end{align}
\end{subequations}
The inner products are
\begin{equation}
    \begin{aligned}[]
        | \alpha_1 |=|\alpha_3|=1, | \alpha_2 |=2 \\
        (\alpha_1,\alpha_2)=-1/\sqrt{2},(\alpha_2,\alpha_3)=-1
    \end{aligned}
\end{equation}
and the roots are
\begin{subequations}
    \begin{align}
        \alpha_1\\
        \alpha_2\\
        \alpha_3\\
        \alpha_1+\alpha_2\\
        \alpha_2+\alpha_3\\
        \alpha_2+2\alpha_3\\
        \alpha_1+\alpha_2+\alpha_3\\
        \alpha_1+\alpha_2+2\alpha_3\\
        \alpha_1+2\alpha_2+2\alpha_3.
    \end{align}
\end{subequations}
We see that the inner products are already not the same. Notice that the roots are really different: it is not simply a renaming \( \alpha_2\leftrightarrow \alpha_3\).

Thus the two Dynkin diagrams \eqref{subEqdunksspd} are describing two different Lie algebras.


%---------------------------------------------------------------------------------------------------------------------------
\subsection{Reconstruction}
%---------------------------------------------------------------------------------------------------------------------------

The construction theorem is the following.
\begin{theorem}
    Let \( R\) be an abstract root system in a complex vector space \( V^*\) and \( \{ \alpha_1,\ldots,\alpha_n \}\) be a basis of \( R\). We denote by \( H_i\in V\) the \defe{inverse root}{inverse!root}\index{root!inverse} of \( \alpha_i\)(i.e. \( \alpha(H_{\alpha})=2\)). We define the Cartan matrix
    \begin{equation}
        A_{ij}=\alpha_j(H_i).
    \end{equation}
    Let \( \lG\) be the Lie algebra defined by the \( 3n\) generators \( X_i,Y_i,H_i\) and the relations
    \begin{subequations}
        \begin{align}
            [H_i,H_j]&=0\\
            [X_i,Y_j]&=\delta_{ij}H_i\\
            [H_i,X_j]&=A_{ij}X_j\\
            [H_i,Y_j]&=-A_{ij}Y_j
        \end{align}
    \end{subequations}
    and, for \( i\neq j\),
    \begin{subequations}
        \begin{align}
            \ad(X_i)^{-A_{ij}+1}(X_j)&=0        \label{EqSerrea}\\
            \ad(Y_i)^{-A_{ij}+1}(Y_j)&=0.
        \end{align}
    \end{subequations}
    Then \( \lG\) is a semisimple Lie algebra in which a Cartan subalgebra is generated by \( H_1,\ldots,H_n \) and its root system is \( R\).
\end{theorem}
A complete proof can be found in \cite{SerreSSAlgebres} at page VI-19. We are going to give some ideas.

We consider \( \lG\), the Lie algebra generated by the elements \( H_i\), \( X_i\) and \( Y_i\). We denote by \( \lH\) the abelian Lie algebra generated by the elements \( H_i\).
\begin{lemma}       \label{LemadXiNilpotent}
    The endomorphism \( \ad(X_i)\) and \( \ad(Y_i)\) are nilpotent.
\end{lemma}

\begin{proof}
    Let \( V_i\) the subspace of \( \lG\) of elements \( z\) such that \( \ad(X_i)^kz=0\) for some \( k\in\eN\). The space \( V_i\) is a Lie subalgebra of \( \lG\) because
    \begin{equation}
        \ad(X_i)[z,z']=-[z,\ad(X_i)z']+[z',\ad(X_i)z].
    \end{equation}
    Acting with \( \ad(X_i)^n\) we get terms of the form \( [\ad(X_i)^kz,\ad(X_i)^lz']\) with \( k+l=n\). If \( n\) is large enough, all the terms vanish.

    From the relation \eqref{EqSerrea} we see that \( X_j\in V_i\) for every \( j\). Since \( [X_i,H_j]\) is proportional to \( X_i\) we also have \( H_j\in V_i\) and then \( Y_j\in V_i\) because \( [X_i,Y_j]=\delta_{ij}H_i\in V_i\). Thus the Lie algebra \( V_i\) contains all the Chevalley generators and then \( V_i=\lG\).
\end{proof}

For \( \lambda\in\lH^*\) we define
\begin{equation}
    \lG_{\lambda}=\{ z\in\lG\tq\ad(h)z=\lambda(h)z\forall h\in\lH \}.
\end{equation}

Then one prove that \( \dim\lG_{\alpha_i}=1\) and \( \dim\lG_{m\alpha_i}=0\) if \( m\neq \pm 1,0\). This corresponds to the fact that we have a reduced root system, which is always the case in complex semisimple Lie algebras\footnote{However, at this point we have not proved yet that \( \lG\) is semisimple and has that root system.}. We denote by \( \Phi\) the subset of \( \lambda\in\lH^*\) such that \( \lG_{\lambda}\neq 0\).

It turns out that we have the direct sum decomposition
\begin{equation}
    \lG=\lH\oplus\bigoplus_{\alpha\in\Phi}\lG_{\alpha}.
\end{equation}

One of the key ingredients in this building is the following lemma.
\begin{lemma}
    If \( \lambda\) and \( \mu\) are related by an element of the Weyl group, then \( \dim\lG_{\lambda}=\lG_{\mu}\).
\end{lemma}


\begin{proof}
    Lemma~\ref{LemadXiNilpotent} allows us to introduce the automorphism
    \begin{equation}
        \theta_i= e^{\ad(X_i)} e^{-\ad(Y_i)} e^{\ad(X_i)}
    \end{equation}
    of \( \lG\). We see that the restriction of \( \theta_i\) to \( \lH\) is the symmetry associated to \( \alpha_i\) (see \eqref{EqSymsiReltosalphai}). Indeed the first exponential reduces to
    \begin{equation}
        e^{\ad(X_i)}H_k=H_k-A_{ki}X_i
    \end{equation}
    where \( A_{ki}=\alpha_i(H_k)\). The second exponential gives
    \begin{equation}
        \begin{aligned}[]
            e^{\ad(-Y_i)}(H_k-A_{ki}X_i)&=H_k-A_{ki}X_i+(-A_{ki}Y_i-A_{ki}H_i)+\frac{ 1 }{2}(2A_{ki}Y_i)\\
            &=H_k-A_{ki}H_i-A_{ki}X_i.
        \end{aligned}
    \end{equation}
    Notice the simplification of \( A_{ki}Y_i\). The third exponential then provides the result (after some simplifications):
    \begin{equation}
        e^{\ad(X_i)}(H_k-A_{ki}H_i-A_{ki}X_i)=H_k-A_{ki}H_i=H_k-\alpha_i(H_k)H_i.
    \end{equation}
    We proved that \( \theta_i(H_k)=s_I(H_k)\).  We deduce that \( \theta_ie_{\alpha}\in\lG_{s_{\alpha_i}(\alpha)}\) whenever \( e_{\alpha}\in\lG_{\alpha}\). Since \( \theta_i\) is an automorphism of \( \lG\) we have
    \begin{equation}
        [H_k,\theta_ie_{\alpha}]=\theta_i[\theta_i^{-1}H_k,e_{\alpha}].
    \end{equation}
    Since \( \theta_i\) reduces to the involutive automorphism \( s_i\) on \( \lH\) we have \( \theta_i^{-1}H_k=\theta_iH_k=s_i(H_k)\). Then we have
    \begin{equation}
        [H_k,\theta_ie_{\alpha}]=\theta_i[s_i(H_k),e_{\alpha}]=\theta_i\alpha\big( s_i(H_k) \big)e_{\alpha}.
    \end{equation}
    The eigenvalue of \( \theta_ie_{\alpha}\) for \( \ad(H_k)\) is thus \( \alpha\big( s_i(H_k) \big)\). Using the definition and \( A_{ki}=\alpha_i(H_k)\) we have
    \begin{equation}
        \begin{aligned}[]
            \alpha\big( s_i(H_k) \big)&=\alpha(H_k)-\alpha_i(H_k)\alpha(H_i)\\
            &=\big( \alpha-\alpha(H_i)\alpha_i \big)H_k\\
            &=s_{\alpha_i}(\alpha)H_k.
        \end{aligned}
    \end{equation}
    At the end we got
    \begin{equation}
        [H_k,\theta_ie_{\alpha}]=s_{\alpha_i}(H_k)\theta_ie_{\alpha}
    \end{equation}
    and then \( \theta_ie_{\alpha}\in\lG_{s_{\alpha_i}(\alpha)}\). Thus the automorphism \( \theta_i\) transforms \( \lG_{\lambda}\) into \( \lG_{\mu}\) when \( \mu=s_i(\lambda)\) and
    \begin{equation}
        \dim\lG_{\lambda}=\dim\lG_{s_i(\lambda)}.
    \end{equation}
\end{proof}
From here we prove that \( \dim\lG_{\alpha}=1\) for every root \( \alpha\)\footnote{\cite{SerreSSAlgebres} page VI-23. Be careful: this is not the statement of page VI-2.}.

Now if \( \alpha+\beta=\gamma+\mu\), the elements \( [E_{\alpha},E_{\beta}]\) and \( [E_{\gamma},E_{\mu}]\) are proportional since they belong to the one-dimensional space \( \lG_{\alpha+\beta}\).


\begin{remark}      \label{RemChevDefmapCommXH}
    A linear map \( \phi\colon \lG\to V\) from \( \lG\) to a vector space \( V\) can be defined on the generators \( X_i\), \( Y_i\) and \( H_i\) among with a formula giving \( \phi([X,Y])\) in terms of \( \phi(X)\) and \( \phi(Y)\).
\end{remark}

\begin{probleme}
    This remark could be made more precise. I'm thinking to the proposition~\ref{PropStandardBialgStruct} giving the standard bialgebra structure on a Lie algebra.
\end{probleme}

The classification of complex semisimple Lie algebras is the following:
\begin{enumerate}
    \item
        \( A_l\) avec \( l=1,2,\ldots \).
        \begin{itemize}
            \item \( A_l=\gsl(l+1,\eC)\)
            \item \( \dim(A_l)=l(l+2)\)
            \item $
        \xymatrix{%
        \alpha_1 \ar@{-}[r]&\alpha_2  \ar@1{-}[r]&\ldots  \ar@1{-}[r]&\alpha_l
           }$
        \end{itemize}
       \item
           \( B_l\) avec \( l=2,3,\ldots\).
           \begin{itemize}
               \item \( B_l=\go(2l+1,\eC)\)
               \item \( \dim(B_l)=l(2l+1)\)
               \item $
    \xymatrix{%
    \alpha_1 \ar@2{->}[r]&\alpha_2  \ar@1{-}[r]   & \ldots \ar@{-}[r] & \alpha_{l-1}\ar@{-}[r]&\alpha_l.
    }   $
           \end{itemize}
       \item
           \( C_l\) avec \( l=3,4,\ldots\)
           \begin{itemize}
               \item \( C_l=\gsp(l,\eC)\)
               \item \( \dim(C_l)=l(2l+1)\)
               \item $
    \xymatrix{%
    \alpha_1 \ar@{->}[r]&\alpha_2  \ar@1{-}[r]   & \ldots \ar@{-}[r] & \alpha_{l-1}\ar@2{->}[r]&\alpha_l.
    }   $
           \end{itemize}
       \item \( D_l\) avec \( l=4,5\ldots\)
           \begin{itemize}
       \item \( D_l=\go(2l, \eC)\)
       \item \( \dim(D_l)=l(2l-1)\)
       \item $
    \xymatrix{%
    &                       &                       &                          &\alpha_{l-1}\\
    \alpha_1 \ar@{-}[r]&\alpha_2  \ar@1{-}[r]   & \ldots \ar@{-}[r] & \alpha_{l-2}\ar@{-}[dr]\ar[ur]\\
    &               &                               &                              &\alpha_l
    }  $
           \end{itemize}
       \item
        \( E_6\)
        \begin{itemize}
            \item \( E_6\)
            \item \( \dim(E_6)=78\)
            \item $ 
    \xymatrix{%
    &                   &   \alpha_{6}\\
    \alpha_1 \ar@{-}[r]&\alpha_2  \ar@1{-}[r]   & \alpha_3 \ar@{-}[r]\ar@{-}[u] & \alpha_4\ar@{-}[r] &\alpha_5
    }  $
        \end{itemize}
    \item \( E_7\)
        \begin{itemize}
            \item \( E_7\)
            \item \( \dim(E_7)=133\)
            \item $ 
    \xymatrix{%
    &                  & &   \alpha_{7}\\
    \alpha_1 \ar@{-}[r]&\alpha_2  \ar@1{-}[r]&\alpha_3\ar@{-}[r]   & \alpha_4 \ar@{-}[r]\ar@{-}[u] & \alpha_5\ar@{-}[r] &\alpha_6
    }  $
        \end{itemize}
    \item \( E_8\)
        \begin{itemize}
            \item
                \( E_8\)
            \item \( \dim(E_8)=248\)
            \item$
    \xymatrix{%
    &                &  & &   \alpha_{8}\\
    \alpha_1 \ar@{-}[r]&\alpha_2  \ar@1{-}[r]&\alpha_3\ar@{-}[r]&\alpha_4\ar@{-}[r]  & \alpha_5 \ar@{-}[r]\ar@{-}[u] & \alpha_6\ar@{-}[r] &\alpha_7
    }  $  
        \end{itemize}
\item \( F_4\)
    \begin{itemize}
        \item \( F_4\)
        \item \( \dim(F_4)=52\)
        \item
    $\xymatrix{%
    \alpha_1 \ar@{-}[r]&\alpha_2  \ar@2{->}[r]   & \alpha_3 \ar@{-}[r] & \alpha_4
    }   $
    \end{itemize}
\item \( G_2\)
    \begin{itemize}
        \item \( G_2\)
        \item \( \dim(G_2)=14\)
        \item$
    \xymatrix{%
    \alpha_1 \ar@3{-}[r]&\alpha_2
    }$
    \end{itemize}
\end{enumerate}

%---------------------------------------------------------------------------------------------------------------------------
                    \subsection{Cartan-Weyl basis}
%---------------------------------------------------------------------------------------------------------------------------

Let us study the eigenvalue equation
\begin{equation}        \label{EqvalpradAprho}
    \ad(A)X=\rho X.
\end{equation}
The number of solutions with $\rho=0$ depends on the choice of $A\in\lG$.

\begin{lemma}
    If $A$ is chosen in such a way that $\ad(A)X=0$ has a maximal number of solutions, then the number of solutions is equal to the rank\index{rank of a Lie algebra} of $\lG$ and the eigenvalue $\alpha=0$ is the only degenerated one in equation \eqref{EqvalpradAprho}.
\end{lemma}

We suppose $A$ to be chosen in order to fulfill the lemma. Thus we have linearly independent vectors $H_i$ ($i=1,\ldots l$) such that
\begin{equation}
    [A,H_i]=0
\end{equation}
where $l$ is the rank of $\lG$. Since $[A,A]=0$, the vector $A$ is a combination $A=\lambda^iH_i$. Since $\ad(A)$ is diagonalisable, one can find vectors $E_{\alpha}$ with
\begin{equation}
    [A,E_{\alpha}]=\alpha E_{\alpha},
\end{equation}
and such that $\{ H_i,E_{\alpha} \}$ is a basis of $\lG$. Using the fact that $\ad(A)$ is a derivation, we find
\begin{equation}
    [A,[H_i,E_{\alpha}]]=\alpha[H_i,E_{\alpha}],
\end{equation}
The eigenvalue $\alpha=0$ being the only one to be degenerated, one concludes that $[H_i,E_{\alpha}]$ is a multiple of $E_{\alpha}$:
\begin{equation}
    [H_i,E_{\alpha}]=\alpha_i E_{\alpha}.
\end{equation}
Replacing $A=\lambda^iH_i$, we have
\begin{equation}
    \alpha E_{\alpha}=[\lambda^iH_i,E_{\alpha}]=\lambda^i\alpha_iE_{\alpha},
\end{equation}
thus $\alpha=\lambda^i\alpha_i$ (with a summation over $i=1,\ldots,l$).

Before to go further, notice that the space spanned by $\{ H_i \}_{i=1,\ldots,l}$ is a maximal abelian subalgebra of $\lG$, so that it is a Cartan subalgebra that we,  naturally denote by $\lH^*$. Thus, what we are doing here is the usual root space construction. In order to stick the notations, let us associate the form $\sigma_{\alpha}\in\lH^*$ defined by $\sigma_{\alpha}(H_i)=\alpha_i$. In that case,
\begin{equation}
    \sigma_{\alpha}(A)=\sigma_{\alpha}(\lambda^iH_i)=\lambda^i\alpha_i=\alpha
\end{equation}
and we have
\begin{equation}
    [A,E_{\alpha}]=\sigma_{\alpha}(A)E_{\alpha}.
\end{equation}
On the other hand, we have $[H_i,E_{\alpha}]=\alpha_iE_{\alpha}=\sigma_{\alpha}(H_i)E_{\alpha}$, so that the eigenvalue $\alpha$ is identified to the root $\alpha$, and we have $E_{\alpha}\in\lG_{\alpha}$.

Let us now express the vectors $t_{\alpha}$ in the basis of the $H_i$. The definition property is $B(t_{\alpha},H_i)=\alpha(H_i)=\alpha_i$. If $t_{\alpha}=(t_{\alpha})^iH_i$, we have
\begin{equation}
    \alpha_i=B(t_{\alpha},H_i)=B_{kl}(t_{\alpha})^k\underbrace{(H_i)^l}_{=\delta^l_i}=B_{ki}(t_{\alpha})^k.
\end{equation}
If $(B^{ij})$ are the matrix elements of $B^{-1}$, we have
\begin{equation}
    (l_{\alpha})^l=\alpha_iB^{il}=\alpha^l
\end{equation}
where $\alpha^l$ is defined by the second equality. Using proposition~\ref{Propoxalphaymoinaalpha}, we have
\begin{equation}
    [E_{\alpha},E_{-\alpha}]=B(E_{\alpha},E_{-\alpha})\alpha^lH_l.
\end{equation}
Thus one can renormalise $E_{\alpha}$ in such a way to have
\begin{equation}
    \begin{aligned}[]
        [H_i,H_j]       &=0,\\
        [E_{\alpha},E_{-\alpha}]    &=\alpha^iH_i\\
        [H_i,E_{\alpha}]    &=\alpha_iE_{\alpha}=\alpha(H_i)E_{\alpha}\\
        [E_{\alpha},E_{\beta}]  &=N_{\alpha\beta}E_{\alpha+\beta}
    \end{aligned}
\end{equation}
where the constant $N_{\alpha\beta}$ are still undetermined. A basis $\{ H_i,E_{\alpha} \}$ of $\lG$ which fulfill these requirements is a basis of \defe{Cartan-Weyl}{Cartan-Weyl basis}.

%---------------------------------------------------------------------------------------------------------------------------
                    \subsection{Cartan matrix}
%---------------------------------------------------------------------------------------------------------------------------

We follow \cite{Wybourne}. We denote by $\Pi$ the system of simple roots of $\lG$. All the positive roots have the form
\begin{equation}
    \sum_{\alpha\in\Pi}k_{\alpha}\alpha
\end{equation}
with $k_{\alpha}\in\eN$.

\begin{theorem}
    Let $\alpha$ and $\beta$ be simple roots Thus
    \begin{enumerate}
        \item
            $\alpha-\beta$ is not a simple root
        \item
            we have
            \begin{equation}        \label{EqabSuraaStrictNEf}
            \frac{ 2(\alpha,\beta) }{ (\alpha,\alpha) }=-p
        \end{equation}
        where $p$ is a strictly positive integer.
    \end{enumerate}
\end{theorem}

\begin{proof}[Partial proof]
    We are going to prove that $\frac{ 2(\alpha,\beta) }{ (\alpha,\alpha) }$ is an integer. Let $\alpha$ and $\gamma$ be non vanishing roots such that $\alpha+\gamma$ is not a root, and define
\begin{equation}
    E'_{\gamma-j\alpha}=\ad(E_{-\alpha})^kE_{\gamma}\in\lG_{\gamma-k\alpha}.
\end{equation}
Since there are a finite number of roots, there exists a minimal positive integer $g$ such that $\ad(E_{-\alpha})^{g+1}E_{\gamma}=0$. We define the constants $\mu_k$ (which depend on $\gamma$ and $\alpha$) by
\begin{equation}
    [E_{\alpha},E'_{\gamma-k\alpha}]=\mu_kE'_{\gamma-(k-1)\alpha}.
\end{equation}
Using the definition of $E'_{\gamma-k\alpha}$ and Jacobi, one founds
\begin{equation}
    \mu_kE'_{\gamma-(k-1)\alpha}=\big[E'_{\alpha},[E_{-\alpha},E'_{\gamma-(k-1)\alpha}]\big]=\alpha^i[H_i,E'_{\gamma-(k-1)\alpha}]+\mu_{k-1}E'_{\gamma-(k-a)\alpha},
\end{equation}
so that $\mu_k=\alpha^i\gamma_i-(k-1)\alpha^i\alpha_i+\mu_{k-1}$, and we have the induction formula
\begin{equation}
    \mu_k=(\alpha,\gamma)-(k-1)(\alpha,\alpha)+\mu_{k-1}
\end{equation}
for $k\geq 2$. If we define $\mu_0=0$, that relation is even true for $k=1$. The sum for $k=1$ to $k=j$ is easy to compute and we get
\begin{equation}
    \mu_j=j(\alpha,\gamma)-\frac{ j(j-1) }{ 2 }(\alpha,\alpha).
\end{equation}
Since $\mu_{g+1}=0$, we have
\begin{equation}        \label{Eqalphagammapargdeux}
    (\alpha,\gamma)=g(\alpha,\alpha)/2,
\end{equation}
and thus
\begin{equation}
    \mu_j=\frac{ j(g-j+1)(\alpha,\alpha) }{ 2 }.
\end{equation}
Let $\beta$ be any root and look at the string $\beta+j\alpha$. There exists a maximal $j\geq 0$ for which $\beta+j\alpha$ is a root while $\beta+(j+1)\alpha$ is not a root. Now we consider $\gamma=\beta+j\alpha$ with that maximal $j$. Putting $\gamma=\alpha+j\beta$ in \eqref{Eqalphagammapargdeux}, one finds
\begin{equation}
    (\alpha,\beta)=\frac{ (g-2j)(\alpha,\alpha) }{ 2 },
\end{equation}
and finally,
\begin{equation}
    \frac{ 2(\alpha,\beta) }{ (\alpha,\alpha) }=g-2j,
\end{equation}
which is obviously an integer.


\end{proof}

From the inner product on $\lH^*$, we deduce a notion of \defe{angle}{angle between roots}:
\begin{equation}
    \cos(\theta_{\alpha,\beta})=\frac{ (\alpha,\beta) }{ \sqrt{(\alpha,\alpha)(\beta,\beta)} }.
\end{equation}
The \defe{length}{length of a root} of the root $\alpha$ is the number $\sqrt{(\alpha,\alpha)}$.

\begin{lemma}
    If $\alpha$ and $\beta$ are roots, then
    \begin{equation}
        \frac{ 2(\alpha,\beta) }{ (\alpha,\alpha) }\in\eZ,
    \end{equation}
    and
    \begin{equation}
        \beta-\frac{ 2(\alpha,\beta) }{ (\alpha,\alpha) }
    \end{equation}
    is a root too.

    If $\alpha$ and $\beta$ are non vanishing, then the $\alpha$-string which contains $\beta$ contains at most $4$ roots. Finally, the ratio
    \begin{equation}
        \frac{ 2(\alpha,\beta) }{ (\alpha,\beta) }
    \end{equation}
    takes only the values $0$, $\pm 1$, $\pm 2$ or $\pm 3$.
\end{lemma}

Let $\Pi=\{ \alpha_1,\ldots,\alpha_l \}$ be a system of simple roots. The \defe{Cartan matrix}{Cartan!matrix} is the $l\times l$ matrix with entries
\begin{equation}        \label{EqDefMatriceCartan}
    A_{ij}=\frac{ 2(\alpha_i,\alpha_j) }{ (\alpha_i,\alpha_i) }.
\end{equation}
Notice that, in the literacy, one find also the convention $A_{ij}=2(\alpha_i,\alpha_j)/(\alpha_j,\alpha_j)$, as in \cite{rncahn}, for example.

\begin{lemma}       \label{LemRatdjaijdjaji}
    There exist positive rational numbers \( d_i\) such that
    \begin{equation}        \label{EqdiAijdjAji}
        d_i A_{ij}=d_jA_{ji}
    \end{equation}
    where \( A\) is the Cartan matrix.
\end{lemma}

\begin{proof}
    The numbers are given by
    \begin{equation}
        d_i=\frac{ (\alpha_i,\alpha_i) }{ (\alpha_1,\alpha_1) }.
    \end{equation}
    The relations \eqref{EqdiAijdjAji} are easy to check using the definition \eqref{EqDefMatriceCartan}. The fact that \( d_i\) is a strictly positive rational number comes from \eqref{EqabSuraaStrictNEf}.
\end{proof}

\begin{probleme}
    I think that there is a property saying (something like) that \( A_{ij}\) is the larger integer \( k\) such that \( \alpha_i+k\alpha_j\) is  a root.
\end{probleme}

\input{102_mazhe}
\input{103_mazhe}
\input{104_helgaLie}

\chapter{Lie groups}
\input{143_Lie_gp_and_subgp}

\chapter{Lie group and Lie algebra}
% This is part of Giulietta
% Copyright (c) 2013-2015, 2019-2020
%   Laurent Claessens
% See the file fdl-1.3.txt for copying conditions.

Here are the results which relate Lie groups and Lie algebras.

%+++++++++++++++++++++++++++++++++++++++++++++++++++++++++++++++++++++++++++++++++++++++++++++++++++++++++++++++++++++++++++ 
\section{Lie algebra of a Lie group}
%+++++++++++++++++++++++++++++++++++++++++++++++++++++++++++++++++++++++++++++++++++++++++++++++++++++++++++++++++++++++++++

\begin{propositionDef}      \label{DEFooKDCPooZOJsMD}
    If \( G\) is a Lie group, its tangent space on at the identity is a Lie algebra. 
    
    This is the \defe{Lie algebra}{Lie algebra of a Lie group} is is tangent space at identity. From a notational point of view, this is written
    \begin{equation}
        \lG=T_eG.
    \end{equation}
\end{propositionDef}

\begin{proof}
    We know from proposition \ref{PROPooOHLQooCNetuD} that \( T_eG\) is a vector space. We have to define a Lie bracket on it. For that we use the left-invariant vector field. Let \( X\in T_eG\) and \( g\in M\) we define
    \begin{equation}
        X^L_g=dL_gX
    \end{equation}
    where \( L_g\colon G\to G\) is the left translation: \( L_g(h)=gh\). If \( X,Y\in T_eG\) we define
    \begin{equation}
        [X,Y]=[X^L,Y^L]_e
    \end{equation}
    where the bracket on the right hand side is the commutator of vector field defined in \ref{DEFooHOTOooRaPwyo}. It defines a Lie algebra structure by the proposition \ref{PROPooSWQSooSEfTuX}.
\end{proof}

In order to make the notations clear, let us write the formula explicitly. If \( X,Y\in T_eG\) are given by \( X=\alpha'(0)\) and \( Y=\beta'(0)\) we have
\begin{subequations}        \label{SUBEQSooHKWMooQbeStl}
    \begin{align}
        (XY)f&=X(Y(f))\\
        &=\Dsdd{ (Yf)\big( \alpha(t) \big) }{t}{0}\\
        &=\Dsdd{ Y_{\alpha(t)}(f) }{t}{0}\\
        &=\Dsdd{ Y^L_{\alpha(t)}(f) }{t}{0}\\
        &=\DDsdd{ f\big( \alpha(t)\beta(u) \big) }{t}{0}{s}{0}.
    \end{align}
\end{subequations}

Now a great theorem without proof:
\begin{theorem} \label{tho:loc_isom}
Two Lie groups are locally isomorphic if and only if their Lie algebras are isomorphic.
\end{theorem}

\begin{theorem}		\label{ThoSubGpSubAlg}		\label{tho:gp_alg}
If $G$ is a Lie group, then
\begin{enumerate}
\item\label{ThoSubGpSubAlgi} if $\lH$ is the Lie algebra of a Lie subgroup $H$ of $G$, then it is a subalgebra of $\lG$,
\item Any subalgebra of $\lG$ is the Lie algebra of one and only one connected Lie subgroup of $G$.
\end{enumerate}

\begin{probleme}
À mon avis, il faut dire ``connexe et simplement connexe'', et non juste ``connexe''.
\end{probleme}

\end{theorem}
\begin{proof}

\subdem{First item}
Let $\dpt{i}{H}{G}$ be the identity map; it is a homomorphism from $H$ to $G$, thus $di_e$ is a homomorphism from $\lH$ to $\lG$. Conclusion: $\lH$ is a subalgebra of $\lG$.

\subdem{Characterization for $\lH$}
Before to go on with the second point, we derive an important characterization of $\lH$:
\begin{equation}\label{eq:path_alg}
\lH=\{X\in\lG:\text{the map } t\to\exp tX\text{ is a path in $H$}\}.
\end{equation}
For that, consider $\dpt{\exp_H}{\lH}{H}$ and $\dpt{\exp_G}{\lG}{G}$; from unicity of the exponential, for any $X\in\lH$, $\exp_HX=\exp_GX$, so that one can simply write ``$\exp$''\ instead of ``$\exp_h$''\ or ``$\exp_G$''.

Now, if $X\in\lH$, the map $t\to\exp tX$ is a curve in $H$. But it is not immediately clear that such a curve in $H$ is automatically build from a vector in $\lH$ rather than in $\lG$.  More precisely, consider a $X\in\lG$ such that $t\to\exp tX$ is a path (continuous curve) in H. By lemma~\ref{lem:var_cont_diff}, the map $t\to\exp tX$ is differentiable and thus by derivation, $X\in\lH$.
The characterisation \eqref{eq:path_alg} is proved.

Thus $\lH$ is a Lie subalgebra of $\lG$.

\subdem{Second item}
For the second part, we consider $\lH$ any subalgebra of $\lG$ and $H$, the smallest subgroup of $G$ which contains $\exp\lH$. We also consider a basis $\{X_1,\ldots,X_n\}$ of $\lG$ such that $\{X_{r+1},\ldots,X_n\}$ is a basis of $\lH$.

By corollary~\ref{cor:/24}, the set of linear combinations of elements of the form $X(M)$ with $M=(0,\ldots,0,m_{r+1},\ldots,m_r)$ form a subalgebra of $U(\lG)$. If $X=x_1X_1+\cdots+x_nX_n$, we define $|X|=(x_1^2+\cdots+x_n^2)^{1/2}$ ($x_i\in\eR$).

Let us consider a $\delta>0$ such that $\exp$ is a diffeomorphism (normal neighbourhood) from $B_{\delta}=\{X\in\lG:|X|<\delta\}$ to a neighbourhood $N_e$ of $e\in G$ and such that $\forall x,y,xy\in N_e$,
\begin{equation}\label{eq:coord_xy}
   (xy)_k=\sum_{M,N}C^{[k]}_{MN}x^My^N
\end{equation}
holds\footnote{The validity of this second condition is assured during the proof of theorem~\ref{tho:loc_isom} which is not given here.}. We note $V=\exp(\lH\cap B_{\delta})\subset N_e$. The map
\[
   \exp(x_{r+1}X_{r+1}+\cdots+x_nX_n)\to(x_{r+1},\ldots,x_n)
\]
is a coordinate system on $V$ for which $V$ is a connected manifold. But $\lH\cap B_{\delta}$ is a submanifold of $B_{\delta}$, then $V$ is a submanifold of $N_e$ and consequently of~$G$.

Let $x$, $y\in V$ such that $xy\in N_e$ (this exist: $x=y=e$); the canonical coordinates of $xy$ are given by \eqref{eq:coord_xy}. Since $x_k=y_k=0$ for $1\leq k\leq r$, $(xy)_k=0$ for the same $k$ because for $(xy)_k$ to be non zero, one need $m_1=\ldots=m_r=n_1=\ldots=n_r=0$ -- otherwise, $x^M$ or $y^N$ is zero. Now we looks at $C^{[k]}_{MN}$ for such a $k$ (say $k=1$ to fix ideas): $[k]=(\delta_{11},\ldots,\delta_{1k})=(1,0,\ldots,0)$ and by definition of the $C$'s,
\[
   X(M)X(N)=\sum_PC_{MN}^PX(P).
\]
But we had seen that the set of the $X(A)$ with $A=(0,\ldots,0,a_{r+1},\ldots,a_n)$ form a subalgebra of $U(\lG)$. Then, only terms with $P=(0,\ldots,0,p_{r+1},\ldots,p_n)$ are present in the sum; in particular, $C_{MN}^{[k]}=0$ for $k=1,\ldots,r$. Thus $VV\cap N_e\subset V$.

The next step is to consider $\mV$, the set of all the subset of $H$ whose contains a neighbourhood of $e$ in $V$. We can check that this fulfils the six axioms of a topological group\index{topological!group}:

\begin{enumerate}
\item The intersection of two elements of $\mV$ is in $\mV$;
\item the intersection of all the elements of $\mV$ is $\{e\}$;
\item any subset of $H$ which contains a set of $\mV$ is in $\mV$;
\item If $\mU\in\mV$, there exists a $\mU_1\in\mV$ such that $\mU_1\mU_1\subset\mU$ because $VV\cap N_e\subset V$;
\item if $\mU\in\mV$, then $\mU^{-1}\in\mV$ because the inverse map is differentiable and transforms a neighbourhood of $e$ into a neighbourhood of $e$;
\item if $\mU\in\mV$ and $h\in H$, then $h\mU h^{-1}\in\mV$.
\end{enumerate}

To see this last item, we denote by $\log$ the inverse map of $\dpt{\exp}{B_{\delta}}{N_e}$. By definition of $V$, it sends $V$ on $\lH\cap B_{\delta}$. If $X\in\lG$, there exists one and only one $X'\in\lG$ such that $he^{tX}h^{-1}=e^{tX'}$ for any $t\in\eR$. Indeed we know that $he^{X}h^{-1}=e^{\Ad_hX}$, then $X'$ must satisfy $e^{tX'}=e^{\Ad_htX}$. If it is true for any $t$, then, by derivation, $X'=\Ad_hX$.

The map $X\to X'$ is an automorphism of $\lG$ which sent $\lH$ on itself. So one can find a $\delta_1$ with $0<\delta_1<\delta$ such that
\[
   h\exp({B_{\delta_1}\cap\lH})h^{-1}\subset V.
\]
Indeed, $he^{\lH} h^{-1}\subset\lH$, so that taking $\delta_1<\delta$, we get the strict inclusion. We can choose $\delta_1$ even smaller to satisfy $he^{B_{\delta_1}}h^{-1}\subset N_e$. Since the map $X\to\log(he^{X}h^{-1})$ from $B_{\delta_1\cap\lH}$ to $B_{\delta}\cap\lH$ is regular, the image of $B_{\delta_1}\cap\lH$ is a neighbourhood of $0$ in $\lH$. Thus $he^{B_{\delta_1}\cap\lH}h^{-1}$ is a neighbourhood of $e$ in $V$. Finally, $h\mU h^{-1}\in\mV$ and the last axiom of a topological group is checked.

This is important because there exists a topology on $H$ such that $H$ becomes a topological group and $\mV$ is a family of neighbourhood of $e$ in $H$. In particular, $V$ is a neighbourhood of $e$ in $H$.

For any $z\in G$, we define the map $\dpt{\phi_z}{zN_e}{B_{\delta}}$ by
\begin{equation}
  \phi_z(ze^{x_1X_1+\cdots+x_nX_n})=(x_1,\ldots,x_n),
\end{equation}
and we denote by $\varphi_z$ the restriction of $\phi_z$ to $zV$. If $z\in H$, then $\varphi_z$ sends the neighbourhood $zV$ of $z$ in $H$ to the open set $B_{\delta}\cap\lH$ in $\eR^{n-r}$. Indeed, an element of $zV$ is a $ze^Z$ with $Z\in\lH\cap B_{\delta}$ which is sent by $\varphi_z$ to an element of $\lH\cap B_{\delta}$. (we just have to identify $x_1X_1+\cdots+x_nX_n$ with $(x_1,\ldots,x_n)$).

Moreover, if $z_1,z_2\in H$, the map $\varphi_{z_1}\circ\varphi_{z_2}^{-1}$ is the restriction to an open subset of $\lH$ of $\phi_{z_1}\circ\phi_{z_2}$. Then $\varphi_{z_1}\circ\varphi_{z_2}^{-1}$ is differentiable. Conclusion: $(H,\varphi_z: z\in H)$ is a differentiable manifold.

Recall that the definition of $\lH$ was to be a subalgebra of $\lG$; therefore $V=e^{\lH\cap B_{\delta}}$ is a submanifold of $G$. But the left translations are diffeomorphism of $H$ and $H$ is the smallest subgroup of $G$ containing $e^{\lH}$. Thus $H$ is a manifold on which the multiplication is diffeomorphic and consequently, $H$ is a Lie subgroup of $G$.

Rest to prove that the Lie algebra of $H$ is $\lH$ and the unicity part of the theorem.

We know that $\dim H=\dim\lH$ and moreover for $i>r$, the map $t\to\exp tX_i$ is a curve in $H$. Now, the fact that $\lH$ is the set of $X\in\lG$ such that $t\to\exp tX$ is a path in $H$ show that $X_i\in\lH$. Then the Lie algebra of $H$ is $\lH$ and $H$ is a connected group because it is generated by $\exp\lH$ which is a connected neighbourhood of $e$ in $H$.

We turn our attention to the unicity part. Let $H_1$ be a connected Lie subgroup of $G$ such that $T_eH_1=\lH$. Since $\exp_{\lH}X=\exp_{\lH_1}X$, $H=H_1$ as set. But $\exp$ is a differentiable diffeomorphism from a neighbourhood of $0$ in $\lH$ to a neighbourhood of $e$ in $H$ and $H_1$, so as Lie groups, $H$ and $H_1$ are the same.

Let us consider an element $X\in\lG$ such that $\exp tX\in H$ for every $t\in\eR$, and the map $\dpt{\varphi}{\eR}{G}$, $\varphi(t)=\exp tX$. This is continuous, then there exists a connected neighbourhood $\mU$ of $0$ in $\eR$ such that $\varphi(\mU)\subset V$. Then $\varphi(\mU)\subset H\cap V$ and the connectedness of $\varphi(\mU)$ makes $\varphi(\mU)\subset\exp\mU_h$. But $\exp\mU_h$ is an arbitrary small neighbourhood of $e$ in $H$; the conclusion is that $\varphi$ is a continuous map from $\eR$ into $H$. Indeed, we had chosen $X$ such that $\exp tX\in H$.

Moreover, we know that
\[
  e^{(t_0+\epsilon)X}=e^{t_0X}e^{\epsilon X},
\]
but $\exp \epsilon X$ can be as close to $e$ as we want (this proves the continuity at $t_0$). Then $\varphi$ is a path in $H$.

In definitive, we had shown that $\exp tX\in H$ implies that $t\to\exp tX$ is a path. Now equation \eqref{eq:path_alg} gives the result.

\end{proof}

\begin{corollary}
Let $G$ be a Lie group and $H_1$, $H_2$, two subgroups both having a finite number of connected components (each for his own topology). If $H_1=H_2$ as sets, then $H_1=H_2$ as Lie groups.
\end{corollary}

\begin{proof}
The proposition shows that $H_1$ and $H_2$ have same Lie algebra. But any Lie subalgebra of $\lG$ is the Lie algebra of exactly one connected subgroup of $G$ (theorem~\ref{tho:gp_alg}). Then as Lie groups, ${H_1}_0={H_2}_0$. Since $H_1$ and $H_2$ are topological groups, the equality of they topology on one connected component gives the equality everywhere (because translations are differentiable).
\end{proof}

\begin{lemma}
Let $\lG$ admit a direct sum decomposition (as vector space) $\lG=\lM\oplus\lN$. Then there exists open and bounded neighbourhoods $\mU_m$ and $\mU_n$ of $0$ in $\lM$ and $\lN$ such that the map
		\begin{equation}
		\begin{aligned}
			\phi \colon \mU_m\times\mU_n &\to G\
			(A,B)&\mapsto e^Ae^B
		\end{aligned}
	\end{equation}
is a diffeomorphism between $\mU_m\times\mU_n$ and an open neighbourhood of $e$ in $G$.
 \label{lem:decomp}
\end{lemma}


\begin{proof}
Let $\{X_1,\ldots,X_n\}$ be a basis of $\lG$ such that $X_i\in\lM$ for $1\leq i\leq r$ and $X_j\in\lN$ for $r<j\leq n$. We consider $\{t_1,\ldots,t_n\}$, the canonical coordinates of $\exp(x_1X_1+\cdots+x_rX_r)\exp(x_{r+1}X_{r+1}+\cdots+x_nX_n)$ in this coordinate system. By properties of the exponential, the function $\varphi_j$ defined by $t_j=\varphi_j(x_1,\ldots,x_n)$ is differentiable at $(0,\ldots,0)$. If $x_i=\delta_{ij}s$, then $t_i=\delta_{ij}s$ and the Jacobian of
\[
   \dsd{(\varphi_1,\ldots,\varphi_n)}{(x_1,\ldots,x_n)}
\]
is $1$ for $x_1=\ldots=x_n=0$. Thus $d\varphi_e$ is a diffeomorphism and so $\varphi$ is a locally diffeomorphic.
\end{proof}

\begin{theorem}
Let $G$ be a Lie group whose Lie algebra is $\lG$ and $H$, a closed subgroup (not specially a \emph{Lie} subgroup) of $G$. Then there exists one and only one analytic structure on $H$ for which $H$ is a topological Lie subgroup of $G$.
\label{tho:diff_sur_ferme}
\end{theorem}

\begin{remark}
A \textit{topological} Lie subgroup\index{topological!Lie subgroup} is stronger that a common Lie subgroup because it needs to be a topological subgroup: it must carry \emph{exactly} the induced topology. In our definition of a Lie group, this feature doesn't appears.
\end{remark}

\begin{proof}
   Let $\lH$ be the subspace of $\lG$ defined by
\begin{equation}\label{eq:lH_de_G}
  \lH=\{X\in\lG\tq \forall t\in\eR,\, e^{tX}\in H\}.
\end{equation}
We begin to show that $\lH$ is a subalgebra of $\lG$; i.e. to show that $t(X+Y)\in\lH$ and $t^2[X,Y]\in\lH$ if $X$, $Y\in\lH$. Remark that $X\in\lH$ and $s\in\eR$ implies $sX\in\lH$. Consider now $X$, $Y\in\lH$ and the classical formula:
\begin{subequations}        \label{SUBEQSooASPNooZOpKRt}
\begin{align}
\left(  \exp(\frac{t}{n}X)\exp(\frac{t}{n}Y)  \right )^n
                       =\exp( t(X+Y)+\frac{t^2}{2n}[X,Y]+o(\frac{1}{n^2}) ),\\
\left(  \exp(-\frac{t}{n}X)\exp(-\frac{t}{n}Y)\exp(\frac{t}{n}X)\exp(\frac{t}{n}Y)   \right)^{n^2}
                       =\exp\left( t^2[X,Y]+o(\frac{1}{n})\right).
\end{align}
\end{subequations}
The left hand side of these equations are in $H$ for any $n$; but, since $H$ is closed, it keeps in $H$ when $n\to\infty$. The right hand side, at the limit, is just $\exp(t(X+Y))$ and $\exp(t^2[X,Y])$, which keeps in $H$ for any $t$. Thus $X+Y$ and $[X,Y]$ belong to $\lH$. The space $\lH$ is thus a Lie subalgebra of $\lG$.

Let $H^*$ be the connected Lie subgroup of $G$ whose Lie algebra is $\lH$ (existence and unicity from~\ref{tho:gp_alg}). From the proof of theorem~\ref{tho:gp_alg}, we know that $H^*$ is the smallest subgroup of $G$ containing $\exp\lH$, then it is made up from products and inverses of elements of the type $e^X$ with $X\in\lH$, and thus is is included in $H$ by definition of $\lH$. So, $H^*\subset H$.

We will show that if we put on $H^*$ the induced topology from $G$ and if $H_0$ denotes the identity component of $H$, then $H^*=H_0$ as topological groups. For this, we first have to show the equality as set and then prove that if $N$ is a neighbourhood of $e$ in $H^*$, then it is a neighbourhood of $e$ in $H_0$. In facts, the equality as set can be derives from this second fact. Indeed, since $H_0$ is a connected topological group, it is generated by any neighbourhood of $e$, so if one can show that any neighbourhood $N$ of $e$ in $H^*$ is a neighbourhood of $e$ in $H$, then $H^*$ is a neighbourhood of $e$ in $H_0$ and then $H_0$ should be generated by $H^*$, so that $H_0\subset H^*$ (as set). Moreover, the most general element of $H^*$ is product and inverse of $e^X$ with $X\in\lH$ and $e^X$ is connected to $e$ by the path $e^{tX}$ ($\dpt{t}{1}{0}$). Then $H^*\subset H_0$, and $H^*=H_0$ as set. Immediately, $H^*=H_0$ as topological groups from our assertion about neighbourhoods of $e$. Let us now prove it.

We consider a neighbourhood $N$ of $e$ in $H^*$ and suppose that this is not a neighbourhood of $e$ in $H$. Thus there exists a sequence $c_k\in H\setminus N$ such that $c_k\to e$ in the sense of the topology on $G$. Indeed, a neighbourhood of $e$ in the sense of $H$ must contains at least a point which is not in $N$ because if we have an open set of $H$ around $e$ included in $N$, then $N$ is a neighbourhood of $e$ for $H$. So we consider a suitable sequence of such open sets around $e$ and one element not in $N$ in each of them. There is the $c_k$'s\quext{Je crois qu'on utilise l'axiome du choix.}.

Using lemma~\ref{lem:decomp} with a decomposition $\lG=\lH\oplus\lM$ (i.e. $\lM$: a complementary for $\lH$ for $\lG$), one can find sequences $A_k\in\mU_m$ and $B_k\in\mU_n$ such that
\[
   c_k=e^{A_k}e^{B_k}.
\]
Here, $\mU_m$ is an open neighbourhood of $0$ in $\lM$ and $\mU_h$, an open neighbourhood of $0$ in $\lH$.

As $e^{B_k}\in N$ and $c_k\in H\setminus N$, $A_k\neq 0$ and $\lim A_k=\lim B_k=0$ (because $(A,B)\to e^Ae^B$ is a diffeomorphism and $e^0e^0=e$ -- and also because all is continuous and thus has a good behaviour with respect to the limit). The set $\mU_m$ is open and bounded --this is a part of the lemma. Then there exist a sequence of positive reals numbers $r_k\in$ such that $r_kA_k\in\mU_m$ and $(r_k+1)A_k\notin\mU_m$. We know that $\mU_m$ is a bounded open subset of the vector space $\lM$, then the whole sequences $r_kA_k$ and $(r_k+1)A_k$ are in a compact domain of $\lM$. Then --by eventually considering subsequences-- there are no problems to consider limits of these sequences in $\lM$: $r_kA_k\to A\in\lM$ (not necessary in $\mU_m$). Since $A_k\to 0$, the point $A$ is the common limit of $r_kA_k\in\mU_m$ and of $(r_k+1)A_k\notin\mU_m$. Thus $A$ is in the boundary of $\mU_m$; in particular, $A\neq 0$.

On the other hand, consider two integers $p,q$ with $q>0$. One can find sequences $s_k,t_k\in\eN$ and $0\leq t_k<q$ such that $pr_k=qs_k+t_k$. It is clear that
\begin{equation}
  \lim_{k\to\infty}\frac{t_k}{q}A_k=0,
\end{equation}
thus
\[
   \exp \frac{p}{q}A=\lim \exp\frac{pr_k}{a}A_k=\lim (\exp A_k)^{s_k},
\]
which belongs to $H$. By continuity, $\exp tA\in H$ for any $t\in\eR$ and finally $A\in\lH$; this contradict $A\neq 0$ so that $A\in\lM$ (because by definition, $A\in\lM$ and the sum $\lG=\lH\oplus\lM$ is direct).

By its definition, $H^*$ has an analytic structure of Lie subgroup of $G$; but we had just proved that the induced topology from $G$ is the one of $H_0$ which by definition is a submanifold of $G$. So the set $H_0=H^*$ becomes a submanifold of $G$ whose topology is compatible with the analytic structure: thus it is a Lie subgroup of $G$. From analyticity, this structure is extended to the whole $H$.

\begin{probleme}
Est-ce bien vrai, tout \c ca ? En particulier, je n'utilise pas que $H_0$ est ouvert dans $H$ (ce qui est un tho de topo classique : je ne vois pas pourquoi Helgason fait tout un cin\'ema --que je ne comprends pas-- dessus). En prenant $N=H^*$, on a juste d\'emontr\'e que $H_0$ est un voisinage de $e$ dans $H$, mais ça, on le savait bien avant.
\end{probleme}

The unicity part comes from the corollary~\ref{cor:top_subgroup}.
\end{proof}


With the notations and the structure of theorem~\ref{tho:diff_sur_ferme}, the subgroup $H$ is discrete if and only if $\lH=\{0\}$. Indeed, recall the definition \eqref{eq:lH_de_G}:
\[
  \lH=\{X\in\lG: \forall t\in\eR, e^{tX}\in H\},
\]
and the fact that there exists a neighbourhood of $e$ in $H$ on which the exponential map is a diffeomorphism.

\begin{remark}
This fact should not be placed after the following lemma. In fact, we use here just the existence of normal neighbourhood (which is a common result) while the following lemma gives much more than normal neighbourhood.
\end{remark}

The lemma (without proof):

\begin{lemma}
 Let $G$ be a Lie group and $H$, a Lie subgroup of $G$ ($\lG$ and $\lH$ are the corresponding Lie algebras). If $H$ is a topological subspace of $G$, then there exists an open neighbourhood $V$ of $0$ in $\mG$ such that
 \begin{enumerate}
 \item $\exp$ is a diffeomorphism between $V$ and an open neighbourhood of $e$ in~$G$,
 \item $\exp(V\cap\lH)=(\exp V)\cap H$.
 \end{enumerate}
\label{lem:sugroup_normal}
\end{lemma}

\begin{definition}
A \defe{differentiable subgroup}{differentiable!subgroup} is a connected Lie subgroup.
\end{definition}

\begin{corollary}
Let $G$ be a Lie group, and $K$, $H$ two differentiable subgroups of $G$. We suppose $K\subset H$. Then $K$ is a differentiable subgroup of the Lie group $H$.
\end{corollary}

\begin{proof}
The Lie algebras of $K$ and $H$ are respectively denoted by $\lK$ and $\lH$. We denote by $K^*$ the differentiable subgroup of $H$ which has $\lK$ as Lie algebra. The differentiable subgroups $K$ and $K^*$ have same Lie algebra, and then coincide as Lie groups.
\end{proof}

\label{pg:ex_topo_Lie}
Consider the group $T=S^1\times S^1$ and the continuous map $\dpt{\gamma}{\eR}{T}$ given by
\[
  \gamma(t)=(e^{it},e^{i\alpha t})
\]
with a certain irrational $\alpha$ in such a manner that $\gamma$ is injective and $\Gamma=\gamma(\eR)$ is dense in $T$.

The subset $\Gamma$ is not closed because his complementary in $T$ is not open: any neighbourhood of element $p\in T$ which don't lie in $\Gamma$ contains some elements of $\Gamma$. We will show that the inclusion map $\dpt{\iota}{\Gamma}{T}$ is continuous. An open subset of $T$ is somethings like
\[
  \mO=(e^{iU},e^{iV})
\]
where $U,V$ are open subsets of $\eR$. It is clear that
\[
   \iota^{-1}(\mO)=\{ \gamma(t)\tq t\in U+2k\pi,\alpha t\in V+2m\pi \},
\]
but the set of elements $t$ of $\eR$ which satisfies it is clearly open. Then $\Gamma$ has at least the induced topology from $T$ (as shown in proposition~\ref{prop:topo_sub_manif}). In fact, the own topology of $\Gamma$ is \emph{more} than the induced: the open subsets of $\Gamma$ whose are just some small segments clearly doesn't appear in the induced topology. Thus the present case is an example (and not a counter-example) of theorem~\ref{tho:H_ferme}.

This example show the importance of the condition for a topological subspace to have \emph{exactly} the induced topology. If not, any Lie subgroup were a topological Lie subgroup because a submanifold has at least the induced topology. We will go further with this example after the proof.

\section{Cosets}
%------------------

We consider $G$, a Lie group and $H$, a closed subgroup. Then from theorem~\ref{tho:diff_sur_ferme},  there exists an unique analytic structure on $H$ for which $H$ is a topological Lie subgroup of $G$. We naturally consider this structure on $H$. We also consider $\lG$ and $\lH$, the Lie algebras of $G$ and $H$, and $\lM$ be a subspace of $\lG$ such that $\lG=\lM\oplus\lH$.

Now we will study the structure of the coset space $G/H$ on which we put the topology such that $\pi$ is continuous and open; this is the \defe{natural topology}{natural topology}\index{topology!natural on $G/H$}\label{pg:natur_topo}. As notations, we define $p_0=\pi(e)$ and $\dpt{\psi}{\lM}{G}$, the restriction to $\lM$ of the exponential.

\begin{lemma}
The dimension of $G/H$ is $\dim (G/H)=\dim G-\dim H$.
 \label{lem:dim_G_H}
\end{lemma}

\begin{proof}
We decompose the Lie algebra $\lG$ as $\lG=\lH\oplus\lM$, and we will see that there exists a real vector space isomorphism $\dpt{\psi}{T_{[e]}(G/H)}{\lM}$ given by
\begin{equation}
   \psi(X)=\Dsdd{ e^{m(t)} }{t}{0}
\end{equation}
if $X(t)=[g(t)]$ with $g(t)=e^{m(t)}e^{h(t)}$ where $m(t)\in\lM$ and $h(t)\in\lH$ (the existence of such a decomposition in reasonably small neighbourhood of $e$ is given by lemma~\ref{lem:decomp}). The fact that $\psi$ is surjective is clear. The injectivity is also easy: $\psi(X)=0$ implies that $\exp m(t)$ is a constant. Thus
\[
X=\Dsdd{ [cst\, e^{h(t)}] }{t}{0}=\Dsdd{[cst]}{t}{0}=0.
\]

\end{proof}

\begin{lemma} \label{lem:vois_U}
There exists a neighbourhood $U$ of $0$ in $\lM$ such that
 \begin{enumerate}
 \item $\psi$ is homeomorphic on $U$,
 \item $\pi$ sends homeomorphically $\psi(U)$ on a neighbourhood of $p_0$ in $G/H$.
 \end{enumerate}
\end{lemma}

\begin{proof}
By lemma~\ref{lem:decomp}, we consider bounded, open and connected neighbourhoods $\mU_m$ and $\mU_h$ of $0$ in $\lM$ and $\lH$ such that
\[
  \dpt{\phi}{(A,B)}{e^Ae^B}
\]
is a diffeomorphism from $\mU_m\times\mU_h$ to an open neighbourhood of $e$ in $G$. Since $H$ has the induced topology from $G$, we can find a neighbourhood $V$ of $e$ in $G$ such that $V\cap H=\exp\mU_h$.

Now we take $U$, a compact neighbourhood of $0$ in $\mU_m$ such that
\begin{equation}\label{eq:UUV}
  e^{-U}e^{U}\subset V.
\end{equation}
So, $\psi$ is an homeomorphism from $U$ to $\psi(U)$. Indeed for $X\in U$, $\psi(X)=e^X=\phi(X,0)$ and $\phi$ is diffeomorphic.

On the other hand, $\pi$ is bijective on $\psi(U)$. In order to see that it is injective, let us consider $X$, $Y\in U$ such that $\pi(e^{X})=\pi(e^{Y})$. Then $\exp X$ and $\exp Y$ are in the same class with respect to $H$: $\exp X\in[\exp Y]$. Then $\exp(-X)\exp Y\in H$, and reversing the role\angl of $X$ and $Y$, $\exp(-Y)\exp X\in H$. Since $X',X''\in U$ and \eqref{eq:UUV},
\[
  e^{-Y}e^{X}\in V\cap H.
\]
Then there exists a $Z$ in $\mU_h$ such that $\exp X=\exp Y\exp Z$, but $U$ is a subset of $\mU_m$ (so that $(A,B)\to e^Ae^B$ is diffeomorphic), then $X=Y$ and $Z=0$.

Since $\pi$ is bijective on $\psi(U)$, it is homeomorphic because the topology is build in order for $\pi$ to be open and continuous.

On a third hand, $U\times\mU_h$ is a neighbourhood of $(0,0)$ in $\mU_m\times\mU_h$, so that $e^Ue^{\mU_h}$ is a neighbourhood of $e$ in $G$. Since $\pi$ is open, $\pi(\exp U\exp \mU_h)=\pi(\psi(U))$ is a neighbourhood of $p_0$ in $G/H$.
\end{proof}


Let $N_0$ be the interior of $\pi(\psi(U))$ and $\{X_1,\ldots, X_r\}$ a basis of $\lM$. If $g\in G$, we looks at the map
\[
  \pi(g\cdot e^{x_1X_1+\cdots+x_rX_r})\to(x_1,\ldots,x_r).
\]
This is an homeomorphism from $g\cdot N_0$ to an open subset of $\eR^r$ because $\pi$ is homeomorphic from $U$. With this chart, $G/H$ is an analytic manifold \nomenclature{$G/H$}{as analytic manifold} and moreover if $x\in G$, the map
\begin{equation}\label{eq:tau_x_y}
  \dpt{\tau(x)}{[y]}{[xy]}
\end{equation}
is an analytic diffeomorphism of $G/H$. Let us prove it. If we consider $[x]\in G/H$, we can write $x=gm$ for a certain $m\in\psi(U)$. Hence the chart around $[x]$ will be around $[gm]=[ge^{x_1X_1+\cdots+x_rX_r}]$ (in other word, we can find an open set around $[x]$ on which can be parametrised so). We can forget the $g$ because the action is a diffeomorphism. Then we looks at the chart $\dpt{\varphi}{G/H}{\eR^r}$, $\varphi[e^{x_1X_1+\cdots+x_rX_r}]=(x_1,\ldots,x_r)$. The map \eqref{eq:tau_x_y} makes 
\begin{equation}
    (y_1,\ldots,y_r)\to( CBH_1(x_1,\ldots,x_r,y_1,\ldots y_r),\ldots, CBH_r(x_1,\ldots,x_r,y_1,\ldots y_r)).
\end{equation}
But $CBH$ is a diffeomorphism.

\begin{theorem}[\cite{Helgason}]\label{Helgason4.2}\label{tho:struc_anal}
Let $G$ be a Lie group, $H$ a closed subgroup of $G$ and $G/H$ with the natural topology. Then $G/H$ has an unique analytic structure with the property that $G$ is a Lie transformation group of $G/H$.
\end{theorem}

\begin{proof}
We denote by $\UU$ the interior of the $U$ given by the lemma~\ref{lem:vois_U}, and $B=\psi(\UU)\subset G$. Since $\dpt{\phi}{(A,B)}{\exp A\exp B}$ is a diffeomorphism, $\psi(\UU)=\phi(U,0)$ is a submanifold of $G$. We consider the following diagram:
\[
\xymatrix{
    G\times B  \ar[d]_{\displaystyle I\times\pi}\ar[r]^{\displaystyle\Phi}    &
                                                                     G\ar[d]^{\displaystyle\pi}\\
    G\times N_0 &                                                             G/H
  }\]
with, for $g\in G$ and $x\in B$,
\[
I\times\pi\colon (g,x)\mapsto (g,[x])
\]
and
\[
\Phi\colon (g,x)\mapsto gx.
\]

\noindent The classes $[x]$ are taken with respect to $H$. The map $\dpt{\mu}{G\times N_0}{G/H}$, $\mu(g,[x])=[gx]$ can be written under the form
\[
   \mu=\pi\circ\Phi\circ(I\times\pi)^{-1}
\]
which is analytic\footnote{Notice that the inverse of $I\times\pi$ exists because $\pi$ is homeomorphic on the spaces considered here.}. So $G$ is a Lie transformation group on $G/H$.

% Faut encore taper l'unicité
\end{proof}


\begin{lemma}[Category theorem] \label{lem:categ}
If a locally compact space $M$ can be written as a countable union
\begin{equation}\label{eq:M_union}
   M=\bigcup_{n=1}^{\infty}M_n
\end{equation}
where each $M_i$ is closed in $M$, then at least one of them contains an open subset of $M$.
\end{lemma}

\begin{proof}
We suppose that none of the $M_i$ contains an open subset of $M$. Let $U_1$ be an open whose closure is compact, $a_1\in U_1\setminus M_1$ and a neighbourhood $U_2$ of $a_1$ such that $\overline{U}_2\subset U_1$ and $\overline{U_2}\cap M_1=\varnothing$. Let $a_2\in U_2\setminus M_2$ and a neighbourhood $U_3$ of $a_2$ such that $\overline{U_3}\subset U_2$ and $\overline{U_3}\cap M_2=\varnothing$\ldots and so on. The existence of the $a_i$ comes from the fact that $U_j$ is open, so that it is contained in no one of the $M_k$.

The decreasing sequence $\overline{U}_1,\overline{U}_2 ,\ldots$ is made up from non empty compacts sets. Then $\bigcap_{i=1}^{\infty}U_i\neq\emptyset$ and the elements of this intersection are in none of the $M_i$; this contradict \eqref{eq:M_union}.
\end{proof}


\begin{theorem} \label{tho:homeo_action}
Let $G$ be a locally compact group with a countable basis. Suppose that it is a transitive, locally compact and Hausdorff topological group of transformation on $M$. Consider $p\in M$ and $H=\{g\in G\tq g\cdot p=p\}$. Then
\begin{enumerate}
\item $H$ is closed,
\item the map $[g]\to g\cdot p$ is homeomorphic between  $G/H$ and $M$.
\end{enumerate}
\end{theorem}

\begin{proof}
By definition of a group action, the map $\dpt{\varphi}{G}{M}$, $\varphi(g)=g\cdot p$ is continuous. Then $H=\varphi^{-1}(p)$ is closed in $G$.

As usual, the topology considered on $G/H$ is a topology which makes the canonical projection $\dpt{\pi}{G}{G/H}$ continuous and open. Now we study the map $\dpt{\psi}{G/H}{M}$, $\psi([g])=g\cdot p$ which is well defined because $H$ fixes $p$ by definition. It is clearly injective, and it is surjective because the action is transitive.

Now remark that $\psi=\varphi\circ\pi^{-1}$. Since $\pi$ is continuous and open, and $\varphi$ is continuous, it just remains to be proved that $\varphi$ is open in order for $\psi$ to be continuous and open. In order to do it, consider $V$, an open subset of $G$, $g\in V$ and a compact neighbourhood $U$ of $e$ in $G$ such that $U=U^{-1}$ and $gU^2\subset V$. If $U$ is small and $u$, $v\in U$ close to $e$, then $guv$ can keep in $V$, so that such a $U$ exists.

We can find a sequence $(g_n)$ in $G$ such that $G=\bigcup_ng_nU$; the transitivity of $G$ on $M$ implies that
\[
  M=\bigcup_ng_nU\cdot p.
\]
Each term in this union is compact, and therefore closed in $M$. By lemma~\ref{lem:categ}, one of the $g_nU\cdot p$ contains an open subset of $M$. Since the action ``$g\cdot$''\ is continuous, $U\cdot p$ also contains an open subset in $M$. The conclusion is that one can find a $u\cdot p$ in the interior of $M$, and $p$ is then an interior point of $u^{-1} U\cdot p\subset U^2\cdot p$. Then $g\cdot p$ in in the interior of $V\cdot p$ and $\varphi$ is therefore open.
\end{proof}

\begin{proposition}
Let $G$ be a transitive transformation Lie group on a $\Cinf$ manifold $M$. Consider $p_0\in M$ and $H$, the stabilizer of $p_{0}$:
\[
  H=\{ g\in G\tq g\cdot p_0=p_0 \}.
\]
Let
\begin{equation}
\begin{aligned}
 \alpha\colon G/H&\to M \\
[g]&\mapsto g\cdot p_{0}.
\end{aligned}
\end{equation}
We have:
\begin{enumerate}
\item The stabilizer $H$ is closed in $G$.
\item If $\alpha$ is homeomorphic, then it is diffeomorphic (if $G/H$ has the analytic structure of theorem~\ref{tho:struc_anal}).
\item If $\alpha$ is homeomorphic and if $M$ is connected, then $G_0$, the identity component of $G$, is transitive on $M$.
\end{enumerate}
\label{propHelgason4.3}
\end{proposition}

This comes from \cite{Helgason}, chapter 2, proposition 4.3. The interest of this theorem is the fact that one only has to check the continuity of $\alpha$ and $\alpha^{-1}$ in order to have a diffeomorphism $M\simeq G/H$.

\begin{proof}
\subdem{The group $H$ is closed in $G$}
We consider the map $\dpt{\varphi}{G}{M}$, $\varphi(g)=g\cdot p_0$. This is continuous; therefore $\varphi^{-1}(p_0)$ is closed. Remark that we are in the situation of theorem~\ref{tho:homeo_action}

\subdem{First item}
We will use lemma~\ref{lem:vois_U}. Se denotes by $\lH$, the Lie algebra of $H$ and we consider a $\lM$ such that $\lG=\lM\oplus\lH$; the lemma~\ref{lem:vois_U} assure us that we have a neighbourhood $U$ of $0$ in $\lM$ on which $\psi$ is homeomorphic and such that $\pi$ sends homeomorphically $\psi(U)$ to a neighbourhood of $p_0$ in $G/H$. We define $\UU$, the interior of $U$, $B=\psi(\UU)$ and $N_0$, the interior of $\pi(\psi(U))$.

The set $B$ is a submanifold of $G$, diffeomorphic to $N_0$ by $\pi$ because everything is continuous and then everything respect the interiors.

\begin{probleme}
C'est n'importe quoi comme justification. C'est lié au problème~\ref{prob:diffeo_2}.
\label{prob:diffeo_1}
\end{probleme}

Consider $\dpt{\iota}{B}{G}$, the identity and $\dpt{\beta}{G}{M}$, $\beta(g)=g\cdot p_0$. The restriction $\alpha_{N_0}$ of $\alpha$ to $N_0$ is an homeomorphism (this is a part of the assumptions) from $N_0$ to an open subset of $M$: $N_0$ is open (this is an interior), then its image by an homeomorphism is open.

Now we can see that $\alpha_{N_0}$ is differentiable. The reason is that it can be written as $\alpha_{N_0}=\dpt{\beta\circ\iota\circ\pi^{-1}}{N_0}{M}$. The construction makes $\pi$ a diffeomorphism and $\beta$ a diffeomorphism when $G$ is a Lie group of transformations (as it is the case here); $\iota$ is clear. Now we have to see that the whole $\alpha$ is also differentiable, and then we will have to prove the same for $\alpha^{-1}$.

By definition, $\alpha([g])=g\cdot p_0$ (the classes $[g]$ is taken with respect to $H$). Consider $[n]\in H$; for any $g\in G$, one can write $[g]=[gn^{-1} n]$. Then
\begin{equation}
  \alpha([g])=\alpha([gn^{-1} n])
             =gn^{-1} n\cdot p_0
	     =gn^{-1}\alpha([n])
	     =gn^{-1}\cdot\alpha_{N_0}([n]),
\end{equation}
but the last dot denotes a differentiable action, and $\alpha_{N_0}$ is differentiable. Thus $\alpha$ is differentiable.

In order for $\alpha$ to be a diffeomorphism, we still have to prove that $\alpha^{-1}$ is differentiable., we begin to show that the Jacobian of $\beta$ at $g=e$ has rank $r_{\beta}=\dim M$. We looks at $\dpt{d\beta_e}{\lG}{T_{p_0}M}$, and consider $X\in\ker (d\beta_e)$. For $f\in\Cinf(M)$, we compute
\begin{equation}
  0=(d\beta_e X)f=X(f\circ\beta)=\Dsdd{ f(e^{tX}\cdot p_0) }{t}{0}.
\end{equation}
Let $s\in\eR$, and we write this equation for $f^*$ instead of $f$, which $f^*$ defined by $f^*(q)=f(e^{sX}\cdot q)$ for each $q\in M$:
\begin{equation}
  0=\Dsdd{f^*(e^{tX}\cdot p_0)}{t}{0}
      =\Dsdd{ f(e^{(s+t)X}\cdot p_0) }{t}{0}
      =\Dsdd{ f(e^{tX}\cdot p_0) }{t}{s}.
\end{equation}
Thus $f(e^{sX}\cdot p_0)$ is a constant with respect to $s$. Since $f$ is arbitrary, $e^{sX}\cdot p_0=p_0$ for any $s$. So $X\in\lH$ because $\exp sX\in H$ for any $s$. Then $\ker d\beta_e\subset \lH$.

On the other hand, $\lH\subset\ker d\beta_e$ is clear, then
\[
   \ker d\beta_e=\lH
\]
and $r_{\beta}=\dim\lG-\dim\lH$.

Since $\alpha$ is an homeomorphism, the dimension of the origin and the target space are the same: $\dim G/H=\dim M$. On the other hand, lemma~\ref{lem:dim_G_H} gives $\dim G/H=\dim\lG-\dim\lH$, so that $r_{\beta}=\dim M$.

Now we prove that $\alpha^{-1}$ is differentiable. Remark that $\beta(g)=g\cdot p_0$ and $\alpha([g])=g\cdot p_0=\beta(g)$ is a good definition for $\alpha$ because the class are taken with respect to the stabilizer of $p_0$. Since $r_{\beta}=\dim M$, the map $\beta$ is locally a diffeomorphism from a neighbourhood of $e$ to a neighbourhood of $p_0$.

If $p=g\cdot p_0$, $\alpha^{-1}(p)=[g]$ because $[k]\in\alpha^{-1}(o)$ if $\alpha([k])=p$, i.e. $k\cdot p_0=p$. But $k=gr$ for a certain $r\in G$. It is clear that $p=k\cdot p_0=gr\cdot p_0$. In particular, $g\cdot(r\cdot p_0)$. We know that in general $g\cdot p=g\cdot q$ implies $p=q$; here it gives us $r\in H$, so that $k\in [g]$.

We consider a $n\in G$ such that $n\cdot$ and $n^{-1}\cdot$ are diffeomorphic. We can make the following manipulation:
\begin{equation}
   \alpha^{-1}(p)=[g]
                =[gnn^{-1}]
		=\pi(gn)\alpha^{-1}(n^{-1}\cdot p_0).
\end{equation}
Under this form, $\alpha^{-1}$ is diffeomorphic.


\subdem{Second item}
If $\alpha$ is an homeomorphism, then $\beta$ is open. Let us denote by $G_0$ the identity component of $G$. There exists a subset $\{x\bgamma\tq\gamma\in I\}$ of $G$ such that
\[
    G=\bigcup_{\gamma\in I}G_0x\bgamma.
\]
This comes from the fact that the components are all some left translations of the identity component (this is true for any Lie group). Each orbit $G_0x\bgamma\cdot p_0$ is open in $M$ and two orbits are either disjoint either equals. Since $M$ is connected, all these orbits must coincide; thus each orbit contains the whole $M$. In particular, the orbit $G_0\cdot p_0=M$: $G_0$ is transitive on $M$.

\end{proof}

\begin{probleme}
Il parra\^it que \c ca donne l'unicit\'e pour~\ref{tho:struc_anal}.
\end{probleme}

%+++++++++++++++++++++++++++++++++++++++++++++++++++++++++++++++++++++++++++++++++++++++++++++++++++++++++++++++++++++++++++ 
\section{Matrix Lie group and its algebra}
%+++++++++++++++++++++++++++++++++++++++++++++++++++++++++++++++++++++++++++++++++++++++++++++++++++++++++++++++++++++++++++
\label{SECooTSAJooNtjgMD}

In this section we deal with Lie groups made from matrices, that is subgroups of \( \GL(n, \eC)\) (typically \( \SO(n)\) or \( \SU(n)\)) and their Lie algebra. We will denote the identity either by \( e\) or by \( \mtu\).

\begin{normaltext}      \label{NORMooHZGKooJEiamo}
    It is time to reread the remark \ref{REMooJQFHooQuoZxt}. In this section, when \( \gamma\) is a path in the matrix group \( G\), we denote by \( \gamma'(0)\) the ``usual'' derivative of \( \gamma\): that is the component-wise derivative; not the differential operator.

    We denote by \( D_{\gamma}\) the differential operator
    \begin{equation}
        \begin{aligned}
            D_{\gamma}\colon  C^{\infty}(G)&\to \eR \\
            f&\mapsto \Dsdd{ f\big( \gamma(t) \big) }{t}{0}. 
        \end{aligned}
    \end{equation}

    We aim to study the link between \( D_{\gamma}\) and \( \gamma'(0)\).

    From the Lie group of matrix \( G\) we can build (at least) two Lie algebras\footnote{Definition \ref{DEFooVBPKooGxlDBn}.}:
    \begin{itemize}
        \item The usual Lie algebra of the group: \( T_eG\) with the definition \ref{DEFooKDCPooZOJsMD}. As set, this is
            \begin{equation}
                T_eG=\{ D_{\gamma}\st \gamma(0)=e \}
            \end{equation}
            with the implicit that \( \gamma\) is a smooth path in \( G\).
        \item 
            The set of ``usual'' derivatives of the paths in \( G\):
            \begin{equation}
                G'=\{ \gamma'(0)\tq \gamma(0)=e \}.
            \end{equation}
            This is a set of matrices on which we can use the bracket \( [X,Y]=XY-YX\) (matrix product). We will see the following facts.
            \begin{itemize}
                \item 
                    The set \( G'\) is a Lie algebra in proposition \ref{PROPooUKITooLnEKZW},
                \item
                    The Lie algebras \( G'\) and \( T_eG\) are isomorphic as Lie algebras in theorem \ref{THOooWQGMooHyjRtx} for the case \( G=\GL(n,\eC)\)
                \item
                    When \( H\) is a Lie subgroup of \( \GL(n,\eC)\), the Lie algebras \( H'\) and \( T_eH\) are isomorphic as Lie algebras in proposition \ref{PROPooSQHLooGQAykc} for the Lie subgroups of \( \GL(n,\eC)\).
            \end{itemize}
    \end{itemize}
\end{normaltext}

\begin{lemma}[\cite{MonCerveau}]
    Let \( G\) be a matrix Lie group, et \( g\in G\) and \( X\in G'\). Then \( gXg^{-1}\in G'\).
\end{lemma}

\begin{proof}
    Let \( x\colon \eR\to G\) be a smooth path such that \( X=x'(0)\). Then we the derivative of the path given by the matrix product
    \begin{equation}
        t\mapsto gx(t)g^{-1}
    \end{equation}
    is \( gXg^{-1}\).
\end{proof}

\begin{lemma}[\cite{MonCerveau}]        \label{LEMooHQUYooSoiKbI}
    Let \( G\) be a matrix Lie group. Then \( G'\) is a vector space on \( \eR\).
\end{lemma}

\begin{proof}
    Let \( X,Y\in G'\) be the derivatives of the paths \( x\) and \( y\). If we set \( \varphi_1(t)=x(t)y(t)\) we have
    \begin{equation}
        \varphi_1'(0)=x'(0)y(0)+x(0)y'(0).
    \end{equation}
    Since \( x(0)=y(0)=e\) we have \( \varphi'(0)=X+Y\), so that \( X+Y\in G'\).

    For the product by a scalar, let the path \( \varphi_2(t)=x(\lambda t)\). The component-wise derivative
    \begin{equation}
        \varphi_2'(0)=\lambda x'(0)=\lambda X,
    \end{equation}
    so that \( \lambda X\in G'\).
\end{proof}

\begin{proposition}     \label{PROPooUKITooLnEKZW}
    Let \( G\) be a matrix Lie group. The vector space \( G'\) is a Lie algebra for the matrix commutator.
\end{proposition}

\begin{proof}
    We already know that \( G'\) is a real vector space by lemma \ref{LEMooHQUYooSoiKbI}. The fact that \( (X,Y)\mapsto XY-YX\) satisfies the axioms of a Lie algebra is easy to check. The only point is to show that if \( X,Y\in G'\), then \( [X,Y]=XY-YX\in G'\).

    Let
    \begin{equation}        \label{EQooJDTLooGWsDiq}
        \varphi(t)=x(t)Yx(-t).
    \end{equation}
    This is for sure a path in the full matrix vector space, and this is derivable because \( x\) is derivable while the matrix product is linear. So the derivative \( \varphi'(0)\) is still a matrix. The question is: why \( \varphi'(0)\in G'\) ?

    By lemma \ref{LEMooHQUYooSoiKbI}, for each \( t\) we have
    \begin{equation}
        \frac{ \varphi(t)-\varphi(0) }{ t }\in G'.
    \end{equation}
    Now, \( G'\) is a vector subspace of \( \eM(n,\eC)\) which is finite dimensional; is is thus closed and the limit belongs to \( G'\).

    Is is now a simple computation to show that \( \varphi'(0)=[X,Y]\).
\end{proof}

\begin{normaltext}
The following theorem is a Giulietta's masterpiece in the following sense:
\begin{itemize}
    \item It is fundamental because the Lie algebra isomorphism between \( T_eGL(n,\eR)\) and the matrices is used everywhere one says «The Lie algebra of $\SO(3)$ is the set of skew-symmetric traceless matrices».
    \item
        Either I'm idiot, either I never seen that theorem even stated (let alone being proved)\footnote{There is in fact a third possibility:  this theorem is a classic one but cannot be found \emph{on internet}.}.
    \item
        I think that the fundamental misunderstanding\footnote{Once again, either I'm idiot either everybody is wrong but me\ldots well \ldots} is that in the context of Lie groups, people \emph{define} \( [X,Y]\) as being \( \ad(X)Y\) while \( \ad\) is defined as the ``second differential'' of \( \AD(g)h=ghg^{-1}\). In that case, obviously we get \( [X,Y]=XY-YX\) with the matrix product. This way fails to make the link with the commutator of vector fields as defined by \ref{DEFooHOTOooRaPwyo}.
    \item
        So you must read this proof with much care and write me if you see any mistake or unclear point.
\end{itemize}
\end{normaltext}
So here it is with the notations explained in \ref{NORMooHZGKooJEiamo}.
    

\begin{theorem}     \label{THOooWQGMooHyjRtx}
    Let \( G=\GL(n,\eC)\) be the group of invertible matrices. The map
    \begin{equation}
        \begin{aligned}
            \phi\colon G'&\to T_eG \\
            \gamma'(0)&\mapsto D_{\gamma} 
        \end{aligned}
    \end{equation}
    is 
    \begin{enumerate}
        \item
            well defined,
        \item
            bijective,
        \item
            linear,
        \item
            a Lie algebra isomorphism.
    \end{enumerate}
\end{theorem}

\begin{proof}
    Several points to be proved.
    \begin{subproof}
        \item[\( \phi\) is well defined]
            Let \( \alpha\) and \( \beta\) be paths in \( G\) such that \( \alpha'(0)=\beta'(0)\) and let \( f\colon G\to \eR\) be a smooth function. We have to prove that \( D_{\alpha}(f)=D_{\beta}(f)\).

            We consider a chart \( \varphi\colon \mU\to \mO\) where \( \mU\) is a neighbourhood of \( 0\) in \( \eR^m\) and \( \mO\) is a neighbourhood of \( e\) in \( \GL(n,\eC)\). We suppose that \( \varphi(0)=e\). We set \( \tilde f=f\circ \varphi\), \( \tilde \alpha=\varphi^{-1}\circ \alpha\) and \( \tilde \beta=\varphi^{-1}\circ\beta\). We have
            \begin{subequations}
                \begin{align}
                    D_{\alpha}(f)&=\Dsdd{ f\big( \alpha(t) \big) }{t}{0}\\
                    &=\Dsdd{ \tilde f\big( \tilde \alpha(t) \big) }{t}{0}\\
                    &=\sum_{i=1}^m\frac{ \partial \tilde f }{ \partial x_i }\big( \tilde \alpha(0) \big)\tilde \alpha_i(0).
                \end{align}
            \end{subequations}
            Since \( \tilde \alpha(0)=\tilde \beta(0)\) we still have to prove that \( \tilde \alpha_i'(0)=\tilde \beta_i'(0)\). As you remember, \( \tilde \alpha\) is a map from \( \eR\) to \( \eR^m\), so that the following derivative is quite usual:
            \begin{subequations}
                \begin{align}
                    \tilde \alpha'(0)&=\Dsdd{ (\varphi^{-1}\circ \alpha)(t) }{t}{0}\\
                    &=d\varphi^{-1}_{\alpha(0)}\big( \alpha'(0) \big)\\
                    &=d\varphi^{-1}_{\beta(0)}\big( \beta'(0) \big).
                \end{align}
            \end{subequations}
            Thus the map \( \phi\) is well defined.
        \item[\( \phi\) is linear]
            This is from the linearity of the derivation.
        \item[\( \phi\) is injective]
            If \( \phi(\alpha')=\phi(\beta')\), then \( D_{\alpha}(f)=D_{\beta}(f)\) for every function \( f\). In that case,
            \begin{equation}
                \sum_{i=1}^m\frac{ \partial \tilde f }{ \partial x_i }(e)\tilde \alpha_i'(0)=\sum_{i=1}^m\frac{ \partial \tilde f }{ \partial x_i }(e)\tilde \beta_i'(0).
            \end{equation}
            That equation must be satisfied for every function. Taking the projection on the components, we get \( \tilde \alpha_i'(0)=\tilde b_i'(0)\), which means \( \alpha'(0)=\beta'(0)\) because \( \varphi^{-1}\) is bijective.
        \item[\( \phi\) is surjective]
            Every element of \( T_eG\) is of the form \( D_{\alpha}\) for some path \( \alpha\), so \( \phi\) is surjective.
        \item[\( \phi\) is a Lie algebra isomorphism]
            Let \( X,Y\in G'\) being the derivative of the paths \( \alpha\) and \( \beta\). We have to prove that
            \begin{equation}
                [\phi(X),\phi(Y)]=\phi[X,Y].
            \end{equation}
            If \( t\) is small enough, the paths
            \begin{subequations}
                \begin{align}
                    \alpha(t)=\mtu+tX\\
                    \beta(t)=\mtu+tY\\
                \end{align}
            \end{subequations}
            are good ones because \( \det(\mtu)\neq 0\), so that the determinant of \( \mtu+tX\) remains different from zero when \( t\) is small, whatever \( X\) is. So \( \alpha\) and \( \beta\) are paths in \( \GL(n,\eC)\). Using the general definition in differential geometry,
            \begin{subequations}        \label{SUBEQSooCYRDooFOdLrn}
                \begin{align}
                    [\phi(X),\phi(Y)]f&=[\phi(X)^L,\phi(Y)^L]_ef\\
                    &=\phi(X)^L_e\big( \phi(Y)^L(f) \big)-\phi(Y)^L_e\big( \phi(X)^L(f) \big) \label{SUBEQooOPUAooZYsZlX}.
                \end{align}
            \end{subequations}
            We focus on the first term:
            \begin{subequations}        \label{SUBEQooTUNFooFkDmuP}
                \begin{align}
                    \phi(X)^L\big( \phi(Y)^L(f) \big)&=\Dsdd{ \phi(Y)^L_{\phi(X)^L_e(t)}(f) }{t}{0}\\
                    &=\DDsdd{ f\big( (\mtu+tX)(\mtu+sY) \big) }{t}{0}{s}{0}\\
                    &=\DDsdd{ f(\mtu+tX+sY+tsXY) }{t}{0}{s}{0}\\
                    &=\Dsdd{ df_{\mtu+tX}\big( (\mtu+tX)Y \big) }{t}{0} \label{SUBEQooLHPBooTnXiZd}\\
                    &=\Dsdd{ df_{\mtu+tX}(Y) }{t}{0}+\Dsdd{ df_{\mtu+tX}(tXY) }{t}{0}   \label{SUBEQooMXJJooBFTLsM}
                \end{align}
            \end{subequations}
            where we have used the linearity of \( df_{\mtu+tX}\) and where \( XY\) stands for the matrix product. In the expression \eqref{SUBEQooLHPBooTnXiZd}, the symbol \( df\) stands for the differential of \( f\) as function from \( \eM(n,\eC)\) (as vector space), not for the differential of \( f\) on \( G\) as manifold. This is why we are allowed to put an expression as the matrix \( Y\) as argument of \( df_{\mtu+tX}\) while \( Y\) is not an element of \( T_{\mtu+tX}G\).

            The expression \eqref{SUBEQooMXJJooBFTLsM} is still made of two terms. The second one is
            \begin{equation}
                \Dsdd{ df_{\mtu+tX}(tXY) }{t}{0}=\Dsdd{ tdf_{\mtu+tX}(XY) }{t}{0}=df_{\mtu}(XY)
            \end{equation}
            where we used the Leibnitz rule\footnote{In general, notice that \( \Dsdd{ tf(t) }{t}{0}=f(0)\)}.

            The first term in \eqref{SUBEQooMXJJooBFTLsM} is computed as
            \begin{equation}
                    \Dsdd{ df_{\mtu+tX}(Y) }{t}{0}=\DDsdd{ f(\mtu+tX+sY) }{t}{0}{s}{0}.
            \end{equation}
            We set 
            \begin{equation}
                \begin{aligned}
                    \gamma\colon \eR^2&\to G \\
                    (t,s)&\mapsto \mtu+tX+sY, 
                \end{aligned}
            \end{equation}
            so that
            \begin{subequations}
                \begin{align}
                    \Dsdd{ df_{\mtu+tX}(Y) }{t}{0}&=\DDsdd{ f(\mtu+tX+sY) }{t}{0}{s}{0}\\
                    &=\DDsdd{ (\tilde f\circ\varphi^{-1}\circ\gamma)(t,s) }{t}{0}{s}{0}\\
                    &=\DDsdd{ g(t,s) }{t}{0}{s}{0}
                \end{align}
            \end{subequations}
            where the function \( g=\tilde f\circ\varphi^{-1}\circ \gamma\) is a smooth function from \( \eR^2\) to \( \eR\).        

            The expression \eqref{SUBEQooTUNFooFkDmuP} is now
            \begin{equation}
                \phi(X)^L\big( \phi(Y)^L(f) \big)=\DDsdd{ g(t,s) }{t}{0}{s}{0}+df_{\mtu}(XY).
            \end{equation}
            The commutator we have to compute, with the same computations is
            \begin{equation}
                [\phi(X),\phi(Y)]f=\DDsdd{ g(t,s) }{t}{0}{s}{0}+df_{\mtu}(XY)-\DDsdd{ g(s,t) }{t}{0}{s}{0}-df_{\mtu}(YX).
            \end{equation}
            The function \( g\) being \(  C^{\infty}\), the derivative commute and the corresponding termes annihilate each other and we are left with
            \begin{equation}
                [\phi(X),\phi(Y)]f=df_{\mtu}(XY)-df_{\mtu}(YX)=df_{\mtu}(XY-YX)
            \end{equation}
            where we used the linearity of the differential.

            In the other sense,
            \begin{equation}
                \phi[X,Y]f=\Dsdd{ f(\mtu+tXY-tYX) }{t}{0}=df_{\mtu}\big( [X,Y] \big)
            \end{equation}
            where, once again, \( df\) stands for the ``usual'' differential.
    \end{subproof}
\end{proof}

Ok. This is proved for \( G=\GL(n,\eC)\), the full matrix group. What about subgroups ? Here is the result.

\begin{proposition}[\cite{MonCerveau}]      \label{PROPooSQHLooGQAykc}
    Let \( H\) be a Lie subgroup\footnote{Thanks to the Cartan theorem \ref{THOooDEJHooVKJYBL}, there are plenty of them.} of \( \GL(n,\eC)\). With the same notations as above, the map
    \begin{equation}
        \begin{aligned}
            \phi\colon H'&\to T_eH \\
            \gamma'(0)&\mapsto D_{\gamma} 
        \end{aligned}
    \end{equation}
    is a Lie algebra isomorphism.
\end{proposition}

\begin{proof}
    We have to prove that
    \begin{equation}        \label{EQooRLBBooYgHhtH}
        \phi[X,Y]f=[\phi(X),\phi(Y)]f
    \end{equation}
    for every \( X,Y\in H'\) and \( f\in  C^{\infty}(H)\). For that, we will see the left and right hand sides of \eqref{EQooRLBBooYgHhtH} in \( G=\GL(n,\eC)\), and use the already proved result, theorem \ref{THOooWQGMooHyjRtx}.

    If \( X,Y\in H'\) we know from proposition \ref{PROPooUKITooLnEKZW} that \( [X,Y]\in H'\). Thus there exists a path \( \gamma\colon \eR\to H\) such that \( [X,Y]=\gamma'(0)\). We consider the extension\footnote{The proposition \ref{PROPooOTZQooIfboXV} can be used since \( H\) is a submanifold of \( G\) by \ref{PROPooFXZJooCOFXZX}.} \( \tilde f\colon W\to \eR\) of \( f\) such that \( \tilde f=f\) on \( H\) and \( W\) is an open set around \( e\) in \( \GL(n,\eC)\). For the sake of making things complicated we also define \( \tilde \gamma=\iota\circ \gamma\) where \( \iota\colon H\to \GL(n,\eC)\) is the inclusion. With all that we have
    \begin{equation}
        \phi[X,Y]f=\Dsdd{ f\big( \gamma(t) \big) }{t}{0}=\Dsdd{ \tilde f\big( \tilde \gamma(t) \big) }{t}{0}=\clubsuit.
    \end{equation}
    At this point, notice that \( [X,Y]\in \GL(n,\eC)'\) and \( [X,Y]=\tilde \gamma'(0)\), so that if we consider the map \( \tilde \phi\colon \GL(n,\eC)\to T_e\GL(n,\eC)\) we also have
    \begin{equation}
        \clubsuit=\Dsdd{ \tilde f\big( \tilde \gamma(t) \big) }{t}{0}=\tilde \phi[X,Y]\tilde f=\big[ \tilde \phi(X),\tilde \phi(Y) \big]\tilde f
    \end{equation}
    where we used the result \ref{THOooWQGMooHyjRtx} on \( \GL(n,\eC)\).

    We still have to prove that \( \tilde \phi(X)\tilde \phi(Y)\tilde f=\phi(X)\phi(Y)f\). Using, among others the formula \ref{SUBEQSooHKWMooQbeStl} adapted to \( \tilde \phi(X)\) instead of \( X\):
    \begin{subequations}
        \begin{align}
            \tilde \phi(X)\tilde \phi(Y)\tilde f&=\Dsdd{ \big( \tilde \phi(Y)^L\tilde f \big)\big( \alpha(t) \big) }{t}{0}\\
            &=\Dsdd{ \tilde \phi(Y)^L_{\alpha(t)}\tilde f }{t}{0}\\
            &=\DDsdd{ \tilde f\big( \alpha(t)\beta(u) \big) }{t}{0}{s}{0}.
        \end{align}
    \end{subequations}
    At this point, notice that \( \alpha(t)\) and \( \beta(u)\) are elements in \( H\) which is a group, so \( \tilde f\big( \alpha(t)\beta(u) \big)=f\big( \alpha(t)\beta(u) \big)\). Thus
    \begin{subequations}
        \begin{align}
            \tilde \phi(X)\tilde \phi(Y)\tilde f&=\DDsdd{ \tilde f\big( \alpha(t)\beta(u) \big) }{t}{0}{s}{0}\\
            &=\DDsdd{ f\big( \alpha(t)\beta(u) \big) }{t}{0}{s}{0}\\
            &=\phi(X)\phi(y)f.
        \end{align}
    \end{subequations}
\end{proof}

\begin{lemma}[\cite{MonCerveau}]
    Let \( G\) be a Lie group of matrices and \( X\in T_eG\) such that 
    \begin{equation}
        df_e(X)=0
    \end{equation}
    for every smooth function \( f\colon G\to \eR\). Then \( X=0\).
\end{lemma}

\begin{proof}
    We consider the functions \( \pr_{ij}\colon G\to \eR\) defined by \( \pr_{ij}(A)=A_{ij}\). If \( g\colon \eR\to G\) is a path, for every \( t\) we have \( \pr_{ij}g(t)=g(t)_{ij}\) and then
    \begin{equation}
        \Dsdd{ \pr_{ij}g(t) }{t}{0}=g'(0)_{ij}.
    \end{equation}
    Then we build
    \begin{equation}
        \begin{aligned}
            f\colon G&\to \eR \\
            A&\mapsto \pr_{11}(A)\pr_{ij}(A). 
        \end{aligned}
    \end{equation}
    If \( g\colon \eR\to G\) is a path such that \( g(0)=e\) and \( g'(0)=X\), then we have
    \begin{subequations}
        \begin{align}
            \Dsdd{ f\big( g(t) \big) }{t}{0}&=\Dsdd{ \pr_{11}\big( g(t) \big)\pr_{ij}\big( g(t) \big) }{t}{0}\\
            &=\pr_{11}g(0)\Dsdd{ \pr_{ij}g(t) }{t}{0}+\Dsdd{ \pr_{11}g(t) }{t}{0}\pr_{ij}g(0)\\
            &=X_{ij}+\delta_{ij}X_{11}\\
            &=X_{ij}+\delta_{ij}X_{11}.
        \end{align}
    \end{subequations}
    We know that this is zero for every choice of \( ij\):
    \begin{equation}
        X_{ij}+\delta_{ij}X_{11}=0
    \end{equation}
    In particular with \( i=j=1\) we have \( 2X_{11}=0\), so that \( X_{11}=0\). Then we are left with \( X_{ij}=0\) for every \( ij\).
\end{proof}

\section{Adjoint group, inner automorphisms}\label{sec:adj_gp}
%--------------------------

Let $\lA$ be a \emph{real} Lie algebra. We denote by $GL(\lA)$\nomenclature[G]{$GL(\lA)$}{The group of nonsingular endomorphisms of $\lA$} the group of all the nonsingular endomorphisms of $\lA$: the linear and nondegenerate operators on $\lA$ as vector space. An element $\sigma\in\GL(\lA)$ does not specially fulfils somethings like $\sigma[X,Y]=[\sigma X,\sigma Y]$. The Lie algebra $\gl(\lA)$\nomenclature[G]{$\protect\gl(\lA)$}{space of endomorphisms with usual bracket} is the vector space of the endomorphisms (without non degeneracy condition) endowed with the usual bracket $(\ad A)B=[A,B]=A\circ B-B\circ A$. The map $X\to\ad X$ is a homomorphism from $\lA$ to the subalgebra $\ad(\lA)$ of $\gl(\lA)$.

The group $\Int(\lA)$\nomenclature[G]{$\Int(\lA)$}{Adjoint group of $\lA$} is the analytic Lie subgroup of $\GL(\lA)$ whose Lie algebra is $\ad(\lA)$ by theorem~\ref{tho:gp_alg}. This is the \defe{adjoint group}{adjoint!group}\index{group!adjoint} of $\lA$.

\begin{proposition}
The group $\Aut(\lA)$\nomenclature[G]{$\Aut\lA$}{Group of automorphisms of $\lA$} of all the automorphisms of $\lA$ is a closed subgroup of $\GL(\lA)$.
\end{proposition}

\begin{proof}
The property which distinguish the elements in $\Aut(\lA)$ from the ``commons'' elements of $\GL(\lA)$ is the preserving of structure: $\varphi[A,B]=[\varphi A,\varphi B]$. These are equalities, and we know that a subset of a manifold which is given by some equalities is closed.
\end{proof}

Now, theorem~\ref{tho:diff_sur_ferme} provides us an unique analytic structure on $\Aut(\lA)$ in which it is a topological Lie subgroup of $\GL(\lA)$. From now we only consider this structure. We denote by $\partial(\lA)$\nomenclature[G]{$\partial\lA$}{The Lie algebra of $\Aut(\lA)$} the Lie algebra of $\Aut(\lA)$: this is the set of the endomorphisms $D$ of $\lA$ such that $\forall t\in\eR$, $e^{tD}\in\Aut(\lA)$. By differencing the equality
\begin{equation}\label{eq:exp_der}
  e^{tD}[X,Y]=[e^{tD}X,e^{tD}Y]
\end{equation}
with respect to $t$, we see\footnote{As usual, if we consider a basis of $\lA$ as vector space, the expression in the right hand side of \[[e^{tD}X,e^{tD}Y]=\ad(e^{tD}X)e^{tD}X\] can be seen as a product matrix times vector, so that Leibnitz works.} that $D$ is a derivation\footnote{Definition \ref{DEFooDUEUooZLhKdv}.} of \( \lA\)

Conversely, consider $D$, any derivation of $\lA$; by induction,
\begin{equation}
   D^k[X,Y]=\sum_{i+j=k}\frac{k!}{i!j!}[D^iX,D^jY]
\end{equation}
where by convention, $D^0$ is the identity in $\lA$. This relation shows that $D$ fulfils condition \eqref{eq:exp_der}, so that any derivation of $\lA$ lies in $\partial(\lA)$. Then
\[
  \partial(\lA)=\{\text{derivations of }\lA\}.
\]
The Jacobi identities show that
\[
\ad(\lA)\subset\partial(\lA).    \label{pg:ad_subset_der}
\]
From this, we deduce\footnote{See error~\ref{err:Intt_Aut}}: 
\begin{equation}\label{eq:int_sub_aut}
  \Int(\lA)\subset\Aut(\lA).
\end{equation}
Indeed the group $\Int(\lA)$ being connected, it is generated\footnote{See proposition~\ref{PropUssGpGenere}} by any neighbourhood of $e$; note that $\Aut(\lA)$ has not specially this property. We take a neighbourhood of $e$ in $\Int(\lA)$ under the form  $\exp V$  where $V$ is a sufficiently small neighbourhood of $0$ in $\ad(\lA)$ to be a neighbourhood of $0$ in $\partial(\lA)$ on which $\exp$ is a diffeomorphism. In this case, $\exp V\subset\Aut(\lA)$ and then $\Int(\lA)\subset\Aut(\lA)$.

Elements of $\ad(\lA)$ are the \defe{inner derivations}{derivation!inner} while the ones of $\Int(\lA)$ are the \defe{inner automorphisms.}{inner!automorphism}

Let $\mO$ be an open subset of $\Aut(\lA)$; for a certain open subset $U$ of $\GL(\lA)$, $\mO=U\cap\Aut(\lA)$. Then
\begin{equation}
  \iota^{-1}(\mO)=\mO\cap\Int(\lA)
           =U\cap\Aut(\lA)\cap\Int(\lA)
       =U\cap\Int(\lA).
\end{equation}

The subset $U\cap\Int(\lA)$ is open in $\Int(\lA)$ for the topology because $\Int(\lA)$ is a Lie\quext{Is it true??} subgroup of $\GL(\lA)$ and thus has at least the induced topology. This proves that the inclusion map $\dpt{\iota}{\Int(\lA)}{\Aut(\lA)}$ is continuous.

The lemma \ref{lem:var_cont_diff} and the consequence below makes $\Int(\lA)$ a Lie subgroup of $\Aut(\lA)$. Indeed $\Int(\lA)$ and $\Aut(\lA)$ are both submanifolds of $\GL(\lA)$ which satisfy \eqref{eq:int_sub_aut}. 


By definition, $\Aut(\lA)$ has the induced topology from $\GL(\lA)$. Then $\Int(\lA)$ is a submanifold of $\Aut(\lA)$. 
This is also a subgroup and a topological group : $\Int(\lA)$ is not a topological subgroup of $\Aut(\lA)$. Then $\Int(\lA)$ is a Lie subgroup of $\Aut(\lA)$. Schematically, links between $\Int\lG$, $\ad\lG$, $\Aut\lG$ and $\partial\lG$ are
\begin{subequations}\label{eq:schem_ad_int}
\begin{align}
  \Int\lG&\longleftarrow\ad\lG\\
  \Aut\lG&\longrightarrow\partial\lG.
\end{align}
\end{subequations}
Remark that the sense of the arrows is important. By definition $\partial\lG$ is the Lie algebra of $\Aut\lG$, then there exist some algebras $\lG$ and $\lG'$ with $\Aut\lG\neq\Aut\lG'$ but with $\partial\lG=\partial\lG'$, because the equality of two Lie algebras doesn't implies the equality of the groups. The case of $\Int\lG$ and $\ad\lG$ is very different: the group is defined from the algebra, so that $\ad\lG=\ad\lG'$ implies $\Int\lG=\Int\lG'$ and $\Int\lG=\Int\lG'$ if and only if $\ad\lG=\ad\lG'$.

A result about the group of inner automorphism which will be useful later:

\begin{lemma}\label{lem:Int_g_gR}
If $\lG$ is a complex semisimple Lie algebra, then $\Int\lG=\Int\lG\heR$.
\end{lemma}

\begin{proof}
If $\{X_i\}$ is a basis of $\lG$, then $\{X_j,iX_j\}$ is a basis of $\lG\heR$. We define $\dpt{\psi}{\ad\lG}{\ad\lG\heR}$ by
\[
   \psi(\ad(a^jX_j))=\ad(a^jX_j).
\]
It is clearly surjective. On the other hand, if $\ad(a^jX_j)\ad(b^kX_k)$ as elements of $\ad\lG\heR$, then they are equals as elements of $\ad\lG$. The discussion following equations \eqref{eq:schem_ad_int} finishes the proof.
\end{proof}

\begin{corollary}
Any two real compact form of a complex semisimple Lie algebra are conjugate by an inner automorphism.
\end{corollary}

\begin{proof}
    We know that any real form of $\lG$ induces an involution (the conjugation) and that if the real form is compact, the involution is Cartan on $\lG\heR$. Let $\lU_0$ and $\lU_1$ be two compact real forms of $\lG$ and $\tau_0$, $\tau_1$ the associated involutions of $\lG$ (which are Cartan involutions of $\lG\heR$). For a suitable $\varphi\in\Int\lG\heR$,
    \[
       \tau_0=\varphi\tau_1\varphi^{-1}.
    \]
    The fact that $\Int\lG=\Int\lG\heR$ (lemma~\ref{lem:Int_g_gR}) finishes the proof.
\end{proof}

\begin{proposition}
 The group $\Int(\lA)$ is a normal subgroup of $\Aut(\lA)$.
\end{proposition}

\begin{proof}
Let us consider a $s\in\Aut(\lA)$. The map $\dpt{\sigma_s}{\Aut(\lA)}{\Aut(\lA)}$, $\sigma_s(g)=sgs^{-1}$ is an automorphism of $\Aut(\lA)$. Indeed, consider $g$, $h\in\AutA$; direct computations show that $\sigma_s(gh)=\sigma_s(g)\sigma_s(h)$ and $[\sigma_s(g),\sigma_s(h)]=\sigma_s([g,h])$. From this, $(d\sigma_s)_e$ is an automorphism of $\partial(\lA)$, the Lie algebra of $\AutA$. For any $D\in\partial(\lA)$ we have
\begin{equation}\label{eq:ad_s_2}
 (d\sigma_s)_eD=\Dsdd{ sD(t)s^{-1} }{t}{0}
             =sDs^{-1}.
\end{equation}
Since $s$ is an automorphism of $\lA$ and $\ad(\lA)$, a subalgebra of $\gl(\lA)$,
\begin{equation}\label{eq:ad_s_1}
  s\ad Xs^{-1}=\ad(sX)
\end{equation}
for any $X\in\lA$, $s\in\Aut(\lA)$. Since $\ad(\lA)\subset\partial(\lA)$, we can write \eqref{eq:ad_s_2} with $D=\ad X$ and put it in \eqref{eq:ad_s_1}:
\[
   (d\sigma)_e\ad X=s\ad Xs^{-1}=\ad(s\cdot X).
\]
We know from general theory of linear operators on vector spaces that if $A,B$ are endomorphism of a vector space and if $A^{-1}$ exists, then $Ae^BA^{-1}=e^{ABA^{-1}}$. We write it with $A=s$ and $B=\ad X$:
\[
  \sigma_s\cdot e^{\ad X}=se^{\ad X}s^{-1}=e^{s\ad Xs^{-1}}=e^{\ad(s\cdot X)},
\]
sot that
\begin{equation}\label{eq:sigma_aut_s}
  \sigma_s\cdot e^{\ad X}=e^{\ad(s X)}.
\end{equation}

Ont the other hand, we know that $\IntA$ is connected, so it is generated by elements of the form $e^{\ad X}$ for $X\in\lA$. Then $\IntA$ is a normal subgroup of $\AutA$; the automorphism $s$ of $\lA$ induces the isomorphism $g\to sgs^{-1}$ in $\IntA$ because of equation \eqref{eq:sigma_aut_s}.
\end{proof}

More generally, if $s$ is an isomorphism from a Lie algebra $\lA$ to a Lie algebra $\lB$, then the map $g\to sgs^{-1}$ is an isomorphism between $\AutA$ and $\AutB$ which sends $\IntA$ to $\IntB$. Indeed, consider an isomorphism $\dpt{s}{\lA}{\lB}$ and $g\in\AutA$. If $g\in\IntA$, we have to see that $sgs^{-1}\in\IntB$. By definition, $\IntA$ is the analytic subgroup of $\GL(\lA)$ which has $\ad(\lA)$ as Lie algebra. We have $g=e^{\ad A}$, then $sgs^{-1}=e^{\ad(sA)}$ which lies well in $\IntB$.

\section{Fundamental vector field}\label{sec:fond_vec}
%++++++++++++++++++++++++++++++++++++

\begin{definition}
    If $\yG$ is the Lie algebra\footnote{Lie algebra of a Lie group, definition \ref{DEFooKDCPooZOJsMD}.} of a Lie group $G$ acting on a manifold $M$ (the action of $g$ on $x$ being denoted by $x\cdot g$), the \defe{fundamental vector field}{fundamental!vector field} associated with $A\in\yG$ is given by
    \begin{equation}			\label{EqDefChmpFond}
       A^*_x=\Dsdd{ x\cdot e^{-tA} }{t}{0}.
    \end{equation}
\end{definition}

If the action of $G$ is transitive, the fundamental vectors at point $x\in M$ form a basis of $T_xM$. More precisely, we have the

\begin{lemma}
For any $v\in T_xM$, there exists a $A\in\yG$ such that $v=A^*_x$, in other terms
\[
  \Span\{ A^*_{x}\tq A\in\yG \}=T_{x}M.
\]
\label{LemFundSpansTan}
\end{lemma}

\begin{proof}
The vector $v$ is given by a path $v(t)$ in $M$. Since the action is transitive, one can write $v(t)=x\cdot c(t)$ for a certain path $c$ in $G$ which fulfills $c(0)=e$. We have to show that $v$ depends only on $c'(0)\in\yG$. We consider
\begin{equation}  \label{eq_def_RGM}
\begin{aligned}
 R\colon G\times M&\to M \\
R(g,x)&= x\cdot g,
\end{aligned}
\end{equation}
so
\begin{equation}\label{eq:v_Rc}
   v=\Dsdd{ R(c(t),x) }{t}{0}=dR_{(e,x)}\big[  (d_tc(t),x)+(c(0),x)   \big].
\end{equation}

\end{proof}

\begin{lemma}\label{lem:As_Bs_A_B}
If $A$, $B\in\yG$ are such that $A^*=B^*$, and if the action is effective, then $A=B$.
\end{lemma}

\begin{proof}
 We consider once again the map \eqref{eq_def_RGM} and we look at
\[
  v=\Dsdd{ R(c(t),x) }{t}{0}
   =(dR)_{(e,x)}\Dsdd{ (c(t),x) }{t}{0},
\]
keeping in mind that $c(t)=e^{-tA}$. In order to treat this expression, we define
\begin{subequations}
\begin{align}
  \dpt{R_1}{G}{M},\quad  R_1(h)&=R(h,x),\\
  \dpt{R_2}{M}{M},\quad  R_2(y)&=R(g,y).
\end{align}
\end{subequations}
So
\[
  v=dR_1(X)+dR_2(0)=dR_1c'(0)
\]
and the assumption $A^*_x=B^*_x$ becomes $dR_1 A=dR_1 B$. This makes, for small enough $t$, 
\begin{equation}
    R_1(e^{tA}e^{-tB})=x\cdot e^{tA}e^{-tB}=x; 
\end{equation}
if the action is effective, it imposes $A=B$.
\end{proof}

\begin{lemma}
If we consider the action of a matrix group, $R_g$ acts on the fundamental field by
\[
  dR_g(A^*_{\xi})=\big( \Ad(g^{-1})A \big)^*_{\xi\cdot g}.
\]
\label{lem:dRgAstar}
\end{lemma}

\begin{proof}
Just notice that $e^{-t\Ad(g^{-1})A}=\AD_{g^{-1}}(e^{-tA})=g^{-1} e^{-tA}g$, thus
\begin{equation}
  \big( \Ad(g^{-1})A \big)^*_{\xi\cdot g}=\Dsdd{ \xi\cdot ge^{-t\Ad(g^{-1})A} }{t}{0}=dR_g(A^*_{\xi}).
\end{equation}
\end{proof}

%+++++++++++++++++++++++++++++++++++++++++++++++++++++++++++++++++++++++++++++++++++++++++++++++++++++++++++++++++++++++++++
\section{Exponential map}
%+++++++++++++++++++++++++++++++++++++++++++++++++++++++++++++++++++++++++++++++++++++++++++++++++++++++++++++++++++++++++++

\begin{definition}
    A \defe{topological group}{topological!group} is a group $G$ equipped with a topological structure such that the maps $(x,y)\in G^2\to xy\in G$ and $x\in G\to x^{-1}\in G$ are continuous.
\end{definition}

\begin{remark}\label{rem:ouvert}
From the existence of an unique inverse for any element of $G$, the multiplication and the inversion are also open maps.
\end{remark}

\begin{definition}
    A \defe{Lie group}{Lie!group} is a group $G$ which is in the same times an manifold such that the group operations (multiplication and inverse) are smooth.

    A Lie group is \defe{analytic}{analytic Lie group} if the manifold is analytic and the group operations are analytic.
\end{definition}

\subsection{Invariant vector fields}\index{invariant!vector field}
%-----------------------------------

\begin{definition}[\cite{BIBooUGWHooPbodCu}]
    If $G$ is a Lie group, a vector field $X\in\Gamma^{\infty}(TG)$ is \defe{left invariant}{left invariant!vector field} if
    \begin{equation}
        (dL_g) X= X,
    \end{equation}
    which means that for every \( g,h\in G\),
    \begin{equation}
        (dL_h)_gX_g=X_{hg}.
    \end{equation}
    In the same way, the vector field \( Y\) is \defe{right invariant}{right!invariant!vector field} if
    \begin{equation}
        (dR_g)Y=Y.
    \end{equation}
\end{definition}

When \( X\in T_eG\), we define the associated left-invariant vector field \( X^L\) by
\begin{equation}        \label{DEFooYPUIooAzcdjP}
    X^L_g=(dL_g)_eX.
\end{equation}

\begin{theorem}[\cite{BIBooUGWHooPbodCu}]
	The map \( \varphi\colon X\mapsto X^L\) where \( X^L_g=(dL_g)_eX\) is a bijection from \( T_eG\) to the set of left-invariant vector fields.
\end{theorem}

\begin{proof}
    Two parts.
    \begin{subproof}
        \item[Surjective]
            Let \( X\) be a left-invariant vector field. We have \( X=(X_e)^L\) because
            \begin{equation}
                (X_e)^L_g=(dL_g)X_e=X_g.
            \end{equation}
            The first equality is the definition of the left-invariant associated vector field (equation \eqref{DEFooYPUIooAzcdjP} applied to \( X_e\)) and the second equality is the fact that \( X\) is left-invariant. Thus \( X\) is the left-invariant vector field associated with \( X_e\).
        \item[Injective]
            Let \( X,Y\in T_eG\) be such that \( X^L=Y^L\). In particular \( X^L_e=Y^L_e\), which means \( X=Y\).
    \end{subproof}
\end{proof}

\begin{proposition}[\cite{BIBooUGWHooPbodCu, MonCerveau}]
    Let \( G\) be a Lie group. The map
    \begin{equation}
        \begin{aligned}
            \varphi\colon G\times \lG&\to TG \\
            (g,X)&\mapsto X^L_g 
        \end{aligned}
    \end{equation}
    is a bijection.

    Moreover for each \( g\in G\), the map
    \begin{equation}
        \begin{aligned}
            \varphi_g\colon \lG&\to T_gG \\
           X&\mapsto X^L_g 
        \end{aligned}
    \end{equation}
    is a vector space isomorphism.
\end{proposition}

\begin{proof}
    Several points.
    \begin{subproof}
        \item[\( \varphi\) is surjective]
            Let \( X\in TG\); there is some \( g\in G\) such that \( X\in T_gG\). Since \( X=(dL_g)_e(dL_{g^{-1}})_gX\) we have
            \begin{equation}
                X=(dL_{g^{-1}}X)^L_g=\varphi(g,dL_{g^{-1}}X).
            \end{equation}
        \item[\( \varphi\) is injective]
            If \( \varphi(g,X)=\varphi(h,Y)\), we have \( X_g^L=Y^L_h\), so that \( g=h\). The equality  \( X_g^L=Y_g^L\) means \( (dL_g)_eX=(dL_g)_eY\). Applying \( (dL_{g^{-1}})_g\) on both sides we get \( X=Y\).
        \item[\( \varphi_g\) is bijective]
            These are the same verifications.
        \item[\( \varphi_g\) is linear]
            The map \( \varphi_g\) is nothing else than \( (dL_g)_e\), so it is linear.
    \end{subproof}
\end{proof}

%--------------------------------------------------------------------------------------------------------------------------- 
\subsection{Flow and exponential}
%---------------------------------------------------------------------------------------------------------------------------

\begin{proposition} \label{PROPooUXFQooIwimav}
    Let \( \Phi\) be the flow of the left-invariant vector field \( X\). We have
    \begin{equation}
        \Phi(t,g)=g\Phi(t,e).
    \end{equation}
\end{proposition}

\begin{proposition}     \label{PROPooZHBOooGTLXsi}
    Let \( G\) be a Lie group, \( \lG\) its Lie algebra\footnote{Definition \ref{DEFooKDCPooZOJsMD}.} and \( X\in\lG\). 
    \begin{enumerate}
        \item
            There exists a unique \(  C^{\infty}\) group homomorphism \( h_X\colon (\eR,+)\to G\) such that \( \Dsdd{ h_x(t) }{t}{0}=X\).
        \item
            The path \( h_X\) is the maximal integral curve of \( X^L\) and \( X^R\) for the initial condition \( h_X(0)=e\).
        \item
            The flows of \( X^L\) and \( X^R\) are defined on \( \eR\).
    \end{enumerate}
\end{proposition}

\begin{definition}
    If \( G\) is a Lie group with algebra \( \lG\), we define the \defe{exponential}{exponential from a Lie algebra} is the map
    \begin{equation}
        \begin{aligned}
            \exp\colon \lG&\to G \\
            X&\mapsto h_X(1) 
        \end{aligned}
    \end{equation}
    where \( h_X\) is the homomorphism defined by the proposition \ref{PROPooZHBOooGTLXsi}. We often write \(  e^{X} \) for \( \exp(X)\).
\end{definition}


The following proposition is a generalization of \ref{PROPooKDKDooCUpGzE}.
\begin{proposition}     \label{PROPooNRVJooEDCpOI}
    If \( X\in \lG\) and \( s,t\in \eR\) we have
    \begin{equation}
        e^{sX} e^{tX}= e^{(s+t)X}.
    \end{equation}
\end{proposition}

\begin{lemma}       \label{LEMooLMTZooCvunSl}
    Let \( G\) be a Lie group and \( X\in G\). We have
    \begin{equation}        \label{EQooNBENooPXLENs}
        X^R_g=\Dsdd{  e^{tX}g }{t}{0}
    \end{equation}
    and
    \begin{equation}
        X^L_g=\Dsdd{  ge^{tX} }{t}{0}
    \end{equation}
\end{lemma}

\begin{normaltext}      \label{NORMooSATDooIhwXXr}
    We will often write the relation \eqref{EQooNBENooPXLENs} under the form
    \begin{equation}
        X^R_g(t)= e^{tX}g.
    \end{equation}
    This is a way to implies that \( t\mapsto  e^{tX}g\) is a path for the vector \( X^R_g\). It is a common abuse of notation to write the vector and a path representing the vector with the same symbol.
\end{normaltext}

\begin{proposition}     \label{PROPooYFZZooLUOuOj}
    Let \( G\) be a Lie group. There exists a neighbourhood \( U\) of \( 0\) in \( \lG\) and a neighbourhood \( V\) of \( e\) in \( G\) such that
    \begin{equation}
        \exp\colon U\to V
    \end{equation}
    is a \(  C^{\infty}\) diffeomorphism\footnote{\( \exp\) is \(  C^{\infty}\), invertible and he inverse is \(  C^{\infty}\) as well.}.
\end{proposition}

\begin{proposition}     \label{PROPooAICDooQcmPZB}
    Let \( G\) be an analytic Lie group. There exists a neighbourhood \( U\) of \( 0\) in \( \lG\) and a neighbourhood \( V\) of \( e\) in \( G\) such that
    \begin{equation}
        \exp\colon U\to V
    \end{equation}
    is an analytic diffeomorphism\footnote{\( \exp\) is analytic, invertible and he inverse is analytic too.}.
\end{proposition}


%--------------------------------------------------------------------------------------------------------------------------- 
\subsection{Invariant vector and derivation}
%---------------------------------------------------------------------------------------------------------------------------

You may want to know how the exponential can be used to write some formulas linking left-invariant vector field and derivation of functions. Here you are.

\begin{normaltext}
    Let \( X\in \lG\), \( g\in G\) and \( u\in \eR\). Let \( f\colon G\to \eR\) be a smooth function. Using the abuse of notation described in \ref{NORMooSATDooIhwXXr} and the proposition \ref{PROPooNRVJooEDCpOI},
    \begin{subequations}
        \begin{align}
            (X^Lf)(g e^{uX})&=\Dsdd{ f\big( X^L_{g e^{uX}}(t) \big) }{t}{0}\\
            &=\Dsdd{ f\big( g e^{uX} e^{tX} \big) }{t}{0}\\
            &=\Dsdd{ f\big( g e^{(t+u)X)} \big)}{t}{0}\\
            &=\Dsdd{ f(g e^{tX}) }{t}{u}.
        \end{align}
    \end{subequations}
    The formula
    \begin{equation}
        (X^Lf)(g e^{uX})=\Dsdd{ f(g e^{tX}) }{t}{u}
    \end{equation}
    means that \( X^L\) derives \( f\) in the direction of the path \(  e^{tX}\) at right.
\end{normaltext}

\begin{normaltext}
    By the way, we recall that, if \( f\) is a function and \( X\) a vector field, \( (Xf)\) is a new function, given by
    \begin{equation}
        (Xf)(a)=X_a(f).
    \end{equation}
    In that sense we can write combinations like \( XYf\) or \( (X^2+X)f\) where \( X\) and \( Y\) are vector fields.
\end{normaltext}

\begin{proposition}[\cite{BIBooPBAMooNcYhCM}]       \label{PROPooKSIDooVIFkiM}
    Let \( G\) be a Lie group with Lie algebra \( \lG\). We consider \( X,Y\in \lG\) and a smooth function \( f\colon G\to \eR\). We have\quext{My source \cite{BIBooPBAMooNcYhCM} seems to write \( (X^R)^n(Y^R)^m\) instead of \( (X^R)^n(Y^L)^m\). Let me know where I'm wrong.}
    \begin{equation}
        \big( (X^R)^n(Y^L)^mf \big)( e^{sX} e^{tY})=\frac{ d^n }{ du^n }\frac{ d^m }{ dv^m }\Big( f( e^{uX} e^{vY}) \Big)_{\substack{u=s\\v=t}}.
    \end{equation}
\end{proposition}

\begin{proof}
    We have to do a proof by induction on \( (n,m)\). We start with \( (n,m)=(0,0)\) and we prove the steps \( (n,m)\to (n+1,m)\) and \( (n,m)\to (n,m+1)\).

    \begin{subproof}
        \item[\( (0,0)\)]
            With \( (n,m)=(0,0)\) we are okay.
        \item[\( (n+1,m)\)]
            We have
            \begin{equation}
                \Big( (X^R)^{n+1}(Y^L)^mf \Big)( e^{sX} e^{tY})=\big( (X^R)(X^R)^n(Y^L)^mf \big)( e^{sX} e^{tY}).
            \end{equation}
            We will apply the induction hypothesis on the function \( (X^R)^n(Y^L)^mf\), but in a first time we just apply the vector field \( X^R\) to the function \( (X^R)^n(Y^L)^m\) and we evaluate at \(  e^{sX} e^{tY}\). Here is a couple of computations:
            \begin{subequations}
                \begin{align}
                    \Big( (X^R)(X^R)^n(Y^L)^mf \Big)( e^{sX} e^{tY})&=\Dsdd{  \Big( (X^R)^n(Y^L)^mf \Big)\big( X^R_{ e^{sX} e^{tY}}(u) \big)  }{u}{0}\\
                    &=\Dsdd{  \Big( (X^R)^n(Y^L)^mf \Big)(  e^{uX} e^{sX} e^{tY} )  }{u}{0}\\
                    &=\Dsdd{  \Big( (X^R)^n(Y^L)^mf \Big)( e^{uX} e^{tY})  }{u}{s}.
                \end{align}
            \end{subequations}
            At this point we use the induction hypothesis:
            \begin{subequations}
                \begin{align}
                    \Dsdd{  \Big( (X^R)^n(Y^L)^mf \Big)( e^{uX} e^{tY})  }{u}{s}&=\frac{ d }{ du }\left( \frac{ d^n }{ dw^n }\frac{ d^m }{ dv^m }\big( f( e^{wX} e^{vY}) \big)_{\substack{w=u\\v=t}}  \right)_{u=s}\\
                    &=\frac{ d^{n+1} }{ dw^{n+1} }\frac{ d^m }{ dv^m }\left( f( e^{wX} e^{vY}) \right)_{\substack{w=s\\v=t}}.
                \end{align}
            \end{subequations}
        \item[\( (n,m+1)\)]
            Same kind of computations.
    \end{subproof}
\end{proof}

%--------------------------------------------------------------------------------------------------------------------------- 
\subsection{Analytic Lie group, Taylor formula}
%---------------------------------------------------------------------------------------------------------------------------

In this subsection we study the analytic functions over an analytic Lie group.

\begin{lemma}[\cite{BIBooPBAMooNcYhCM}]     \label{LEMooPILVooHQbtAH}
    Let \( G\) be an analytic Lie group. We consider an analytic function \( f\colon G\to \eR\), an element \( X\in \lG\), a basis \( \{ X_i \}\) of \( \lG\) and \( g\in G\). There exists an absolutely converging power series \( P\) such that
    \begin{equation}
        f(g e^{x_1X_1+\ldots +x_nX_n})=P(x_1,\ldots, x_n).
    \end{equation}
\end{lemma}

\begin{proof}
    First we make the proof for \( g=e\).

    We consider a basis \( \{ e_i \}\) of \( \lG\). Let \( U\) be a neighbourhood of \( 0\) in \( \lG\) and \( V\) a neighbourhood of \( e\) in \( G\) such that \( \exp\colon U\to V\) is an analytic diffeomorphism\footnote{By proposition \ref{PROPooAICDooQcmPZB}.}.

    We consider \( U'\), the open set in \( \eR^n\) which correspond to \( U\) via the basis \( \{ e_i \}\). The map
    \begin{equation}
        \begin{aligned}
            \varphi\colon U'&\to V \\
            (x_1,\ldots, x_n)&\mapsto \exp(x_1e_1+\ldots+x_ne_n)
        \end{aligned}
    \end{equation}
    is analytic chart of \( V\).

    The fact that \( f\) is analytic means that the composition of \( f\) with the charts are analytic. In our case, the map \( \tilde f =f\circ\varphi\) is analytic from \( U'\subset \eR^n\) to \( \eR\). Thus there exists an absolutely converging power series \( P\) such that
    \begin{equation}
        \tilde f(x_1,\ldots, x_n)=P(x_1,\ldots, x_n).
    \end{equation}
    We conclude:
    \begin{equation}
        f\big( \exp(x_1e_1+\ldots +x_ne_n) \big)=f\big( \varphi(x_1,\ldots, x_n) \big)=P(x_1,\ldots, x_n).
    \end{equation}
    
    If \( g\) is not \( e\), we consider the neighbourhood \( gV\) and the map
    \begin{equation}
        \begin{aligned}
            \varphi\colon U&\to gV \\
            (x_1,\ldots, x_n)&\mapsto g\exp(x_1e_1+\ldots +x_ne_n)
        \end{aligned}
    \end{equation}
    is a chart, so that
    \begin{equation}
        f(g e^{x_1e_1+\ldots +x_ne_n})=\tilde f(x_1,\ldots, x_n)
    \end{equation}
    which is a power series.
\end{proof}

\begin{proposition}[Taylor formula\cite{BIBooPBAMooNcYhCM}]     \label{PROPooIYWQooZJtKiu}
    Let \( G\) be an analytic Lie group. We suppose that \( f\colon G\to \eR\) is an analytic functions. For \( g\in G\) and \( X\in \lG\) we have
    \begin{equation}
        f(g e^{X})=\sum_{k=0}^{\infty}\frac{1}{ n! }\big( (X^R)^nf \big)(g).
    \end{equation}
\end{proposition}

\begin{proof}
    We know from proposition \ref{LEMooPILVooHQbtAH} that \( f(g e^{X})=P(x_1,\ldots, x_n)\) for some power series \( P\). We consider a neighbourhood \( U\) of \( 0\) in \( \lG\) and \( V\) of \( g\) in \( G\) such that
    \begin{equation}
        \begin{aligned}
            \varphi\colon U&\to V \\
            X&\mapsto  ge^{X} 
        \end{aligned}
    \end{equation}
is an analytic diffeomorphism (i.e. an analytic chart for \( G\) around \( g\)). Let \( X\in U\) and \( \delta\) such that \( tX\in U\) for all \( t\in \mathopen] -\delta , \delta \mathclose[\). Notice that \( \delta>1\). Now, \( X\) being fixed, the value of \( P(tx_1,\ldots, tx_n)\) is an absolutely convergent power series of \( t\). We have
    \begin{equation}
        f(g e^{tX})=P(tx_1,\ldots, tx_n)=\sum_{k=0}^{\infty}\frac{ a_m }{ m! }t^m
    \end{equation}
    for some constants \( a_m\in \eR\).

    But considering the function
    \begin{equation}
        \begin{aligned}
            r\colon \eR&\to \eR \\
            t&\mapsto f(g e^{tX}), 
        \end{aligned}
    \end{equation}
    there is an unicity of its power series expansion; thus \( a_m\) is the \( m\)-th derivative of \( r\) at \( t=0\).

    But we also know from proposition \ref{PROPooKSIDooVIFkiM} that
    \begin{equation}
        \big( (X^L)^mf \big)(g e^{tX})=\frac{ d^m }{ du^m }\big( f(g e^{uX}) \big)_{u=t};
    \end{equation}
    taking that at \( t=0\) we have
    \begin{equation}
        a_m=\big( (X^L)^mf \big)(g)
    \end{equation}
    and the Taylor formula
    \begin{equation}
        f(g e^{tX})=\sum_{k=0}^{\infty}\frac{1}{ k! }\frac{ d^k }{ du^k }\big( f(g e^{uX}) \big)_{u=0}t^m.
    \end{equation}
    Finally taking \( t=1\) (recall that \( \delta>1\), so it is valid):
    \begin{equation}
        f(g e^{X})=\sum_{k=0}^{\infty}\frac{1}{ k! }\frac{1}{ k! }\big( (X^L)^kf \big)(g).
    \end{equation}
\end{proof}

\begin{lemma}       \label{LEMooWKFIooRHsrFX}
    Let \( G\) be an analytic Lie group with algebra \( \lG\). We consider a basis \( \{ e_i \}_{i=1,\ldots, n}\) of \( \lG\) and the functions
    \begin{equation}
        \begin{aligned}
            f_i\colon U&\to \eR \\
            \exp(x_1e_1+\ldots+x_ne_n)&\mapsto x_i 
        \end{aligned}
    \end{equation}
    defined on a normal neighbourhood \( U\) of \( e\).
    
    If \( X,Y\in \lG\) satisfy
    \begin{equation}
        Xf_i=Yf_i
    \end{equation}
    for every \( i\), then \( X=Y\).
\end{lemma}

\begin{proof}
    If \( X=\sum_kX_ke_k\) we have
    \begin{equation}
        X(f_i)=\Dsdd{ f_i( e^{tX}) }{t}{0}=\Dsdd{ f_i\big(  e^{t\sum_kX_ke_k} \big) }{t}{0}=\Dsdd{ tX_i }{t}{0}=X_i.
    \end{equation}
\end{proof}

\begin{lemma}[\cite{BIBooPBAMooNcYhCM}]
    Let \( G\) be an analytic Lie group with Lie algebra \( \lG\). For \( X,Y\in \lG\) we have:
    \begin{enumerate}
        \item       \label{ITEMooHVOIooKDrUSw}
            \( \exp(tX)\exp(tY)=\exp\big( t(x+Y)+\frac{ t^2 }{2}[X,Y]+t^2\alpha(t) \big)\),
        \item       \label{ITEMooWIQIooHphJcP}
            \( \exp\big( t(X+Y) \big)=\exp(tX)\exp(tY)\exp(t\alpha(t))\)
        \item       \label{ITEMooVMDCooExpIrp}
            \( \exp(-tX)\exp(-tY)\exp(tX)\exp(tY)=\exp\big( t^2[X,Y]+t^3\alpha(t) \big)\).
    \end{enumerate}
    In both formulas, \( \alpha\) is a function \( \alpha\colon \eR\to \lG\) satisfying \( \lim_{t\to 0} \alpha(t)=0\).
\end{lemma}

\begin{proof}
    Several steps.
    \begin{subproof}
        \item[A good function]
        
            Let \( \{ e_i \}_{i=1,\ldots, n}\) be a basis of \( \lG\). We consider a neighbourhood \( U\) of \( 0\) in \( \lG\) and \( V\) of \( e\) in \( G\) such that \( \exp\colon U\to V\) is an analytic diffeomorphism. On that \( U\) we consider the function
            \begin{equation}
                \begin{aligned}
                    f\colon U&\to \eR \\
                    \exp(x_1e_1+\ldots +x_ne_n)&\mapsto x_i 
                \end{aligned}
            \end{equation}
            for some fixed \( i\). This function is analytic and satisfies \( f(e)=0\). 
        \item[Some Taylor expansions] 
            Using proposition \ref{PROPooKSIDooVIFkiM} we have
            \begin{equation}
                \big( (X^R)^n(X^L)^mf \big)( e^{sX} e^{tY})=\frac{ d^n }{ du^n }\frac{ d^m }{ dv^m }\big( f( e^{uX} e^{vY}) \big)_{\substack{u=s\\v=t}}.
            \end{equation}
            Considering the function \( q(s,t)=f( e^{sX} e^{tY})\), we have the Taylor expansion
            \begin{equation}        \label{EQooNBOIooRxlZmP}
                f( e^{sX} e^{tY})=q(s,t)=\sum_{m,n\geq 0}\frac{ s^n }{ n! }\frac{ t^m }{ m! }\big( (X^R)^n(Y^L)^mf \big)(e)=\sum_{m,n\geq 0}\frac{ s^n }{ n! }\frac{ t^m }{ m! }\big( X^nY^mf \big)(e).
            \end{equation}
            Here the second equality is due to the fact that \( (X^Lf)(e)=(X^Rf)(e)=X(f)\).

        \item[The function \( Z\)]

            On the other hand, when \( t\) is small enough, the element \(  e^{tX} e^{tY}\) belongs to a normal neighbourhood of \( e\), so that there exists an element \( Z(t)\in \lG\) satisfying
            \begin{equation}
                e^{tX} e^{tY}= e^{Z(t)}.
            \end{equation}
            The element \( Z(t)\) is given by
            \begin{equation}
                Z(t)=\exp^{-1}\big(  e^{tX} e^{tY} \big).
            \end{equation}
            Since the exponential is an analytic diffeomorphism\footnote{Proposition \ref{PROPooAICDooQcmPZB}.} (the inverse is analytic), \( Z\) is an analytic function around \( t=0\). Thus there exists a function \( \alpha\colon \eR\to \lG\) such that
            \begin{equation}        \label{EQooRPGGooXtZzFy}
                Z(t)=tZ_1+t^2Z_2+t^2\alpha(t)
            \end{equation}
            and \( \lim_{t\to 0} \alpha(t)=0\). Notice that \( Z(0)=0\), which explain the absence of constant term in \eqref{EQooRPGGooXtZzFy}.

        \item[A formula for \( f\big(  e^{Z(t)} \big)\)]

            We pose \( Z_1=\sum_ka_{1k}e_k\), \( Z_2=\sum_ka_{2k}e_k\) and \( \alpha(t)=\sum_k\sigma_k(t)e_k\), so that
            \begin{equation}
                Z(t)=\sum_k\big( ta_{1k}+t^2a_{2k}+t^2\alpha_k(t) \big)e_k.
            \end{equation}
            Applying \( f\) we have
            \begin{equation}
                f\big(  e^{Z(t)} \big)=ta_{1i}+t^2a_{2i}+t^2\alpha_i(t)=f\big(  e^{tZ_1+t^2Z_2} \big)+t^2\alpha_i(t).
            \end{equation}
            
        \item[Some more Taylor expansions]

            We use the Taylor expansion of proposition \ref{PROPooIYWQooZJtKiu} with \( g=e\) and \( X=Z(t)\):
            \begin{equation}        \label{EQooSFKOooDAavVy}
                f( e^{Z(t)})=\sum_k\frac{1}{ k! }\big( [tZ^L_1+t^2Z_2^L]^kf \big)(e)+t^2\alpha_i(t).
            \end{equation}
            Once again we can drop the \( L\) exponent since \( (X^Lf)(e)=X(f)\). We collect out of \eqref{EQooSFKOooDAavVy} the terms with \( t\) and \( t^2\):
            \begin{equation}        \label{EQooEYUSooTDntym}
                f( e^{tX} e^{tY})=f( e^{Z(t)})=tZ_1(f)+t^2 Z_2 +\frac{ t^2 }{2}Z_1^2 +t^2\beta(t)
            \end{equation}
            with \( \lim_{t\to 0} \beta(t)=0\).

        \item[Comparison]

            The formulas \eqref{EQooNBOIooRxlZmP} with \( s=t\) and \eqref{EQooEYUSooTDntym} are Taylor expansions of the same quantity. They are equal; we copy them here:
            \begin{equation}
                \sum_{m,n}\frac{ t^{m+n} }{ m!n! }\big( (X^R)^n(Y^L)^mf \big)(e)=tZ_1(f)+t^2 Z_2 +\frac{ t^2 }{2}Z_1^2 +t^2\beta(t)
            \end{equation}
            On the left hand side, the terms with \( t\) and \( t^2\) are obtained when \( (n,m)\) is among the possibilities $(0,1)$, $(1,0)$, $(2,0)$, $(0,2)$, and $(1,1)$. Collecting we have on the left
            \begin{equation}
                (X+Y)f+XYf+\frac{ 1 }{2}X^2f+\frac{ 1 }{2}Y^2f
            \end{equation}
            where we used the fact that \( \big( (X^R)^2f \big)(e)=X(Xf)=X^2f\).

            Using lemma \ref{LEMooWKFIooRHsrFX} we have \( Z_1f=(X+Y)f\), so that \( Z_1=X+Y\) and then
            \begin{equation}
                \frac{ 1 }{2}[X,Y]=Z_2.
            \end{equation}
    \end{subproof}
    At this point we proved that
    \begin{equation}
        e^{tX} e^{tY}= e^{t(X+Y)+\frac{ t^2 }{2}[X,Y]+t^2\alpha(t)}.
    \end{equation}
    This is \ref{ITEMooHVOIooKDrUSw}.

    For point \ref{ITEMooWIQIooHphJcP}, we are searching for a function \( \beta\) such that 
    \begin{equation}
        e^{tX} e^{tY} e^{t\beta(t)}= e^{t(X+Y)}.
    \end{equation}
    We replace in the left-hand side the value of \(  e^{tX} e^{tY}\) given by the point \ref{ITEMooHVOIooKDrUSw} (this is the reason why we write \( \beta\) instead of \( \alpha\)) and we isolate \(  e^{t\beta(t)}\):
    \begin{equation}        \label{EQooLTMBooVIChyC}
        e^{t\beta(t)}= e^{t(X+Y)} e^{-t(X+Y)-t^2[X,Y]/2-t^2\alpha(t)}.
    \end{equation}
    So now our aim is to show that the right-hand side of \eqref{EQooLTMBooVIChyC} can be written as only one exponential with an argument of the form \( t\beta(t)\) satisfying \( \beta(t)\to 0\). For that, we use \ref{ITEMooHVOIooKDrUSw} once again with \( X+Y\) instead of \( X\) and \( -(X+Y)-t[X,Y]/2-t\alpha(t)\) instead of \( Y\). What we get is
    \begin{subequations}
        \begin{align}
            e^{t\beta(t)}&=\exp\big( t(-t[X,Y]/3-t\alpha(t))+\frac{ t^2 }{2}\big[ X+Y,-(X+Y)-t[X,Y]/2-t\alpha(t) \big] \big)\\
            &=\exp\big( -\frac{ t^2 }{2}[X,Y]  -t^2\alpha(t)-\frac{ t^3 }{ 4 }\big[ X+Y,[X,Y] \big]-\frac{ t^3 }{ 2 }\alpha(t)  \big).
        \end{align}
    \end{subequations}
    We are done with \ref{ITEMooWIQIooHphJcP}.
    

et

    \ref{ITEMooVMDCooExpIrp}

\end{proof}


%--------------------------------------------------------------------------------------------------------------------------- 
\subsection{Other stuff}
%---------------------------------------------------------------------------------------------------------------------------

The concept of normal neighbourhood will be widely used for the study of the relations between a Lie group and its algebra. Let $M$ be a differentiable manifold. If $V$ is a neighbourhood of zero in $T_pM$ on which the exponential $\dpt{\exp_p}{T_pM}{M}$ is a diffeomorphism, then $\exp_pV$ is  \defe{normal neighbourhood}{normal!neighbourhood} of $p$.

\begin{lemma}
Let $\lG$ be a Lie algebra and $A$, a linear operator on $\lG$ (see as a common vector space) such that $\forall t\in\eR$, the map $e^{tA}$ is an automorphism of $\lG$. Then $A$ is a derivation of $\lG$.
\label{lem:autom_derr}
\end{lemma}

\begin{proof}
Let us consider $X$, $Y\in\lG$;  the assumption is
\[
  e^{tA}[X,Y]=[e^{tA}X,e^{tA}Y].
\]
Since $e^{tA}$ is a linear map, it has a ``good behavior''\ with the derivations:
\[
\Dsddc{e^{tA}[X,Y]}{t}{0}=\Dsddc{e^{tA}}{t}{0}[X,Y]=A[X,Y].
\]
Using on the other hand the linearity of $\ad$, we can see
\[
  (\ad(e^{tA}X))(e^{tA}Y)
\]
as a product ``matrix times vector''. Then
\begin{equation}
\begin{split}
  \Dsddc{[e^{tA}X,e^{tA}Y]}{t}{0}&=\Dsddc{(\ad e^{tA}X)Y}{t}{0}+\Dsddc{(\ad X)(e^{tA}Y)}{t}{0}\\
                                 &=(\ad AX)Y+(\ad X)(AY).
\end{split}
\end{equation}
Finally, $A[X,Y]=[AX,Y]+[X,AY]$.

\end{proof}

As notational convention, if $G$ and $H$ are Lie groups, their Lie algebra are denoted by $\lG$ and $\lH$.

\begin{lemma}		\label{LemAlgEtGroupesGenere}
	Let $\lG$ be a Lie algebra ans $\lS$ be a subset of $\lG$. The algebra of the group generated by $ e^{\lS}$ is the algebra generated by $\lS$.
\end{lemma}

Invariant vector fields are also often used in order to transport a structure from the identity of a Lie group to the whole group by $A_g(X_g)=A_e(dL_{g^{-1}}X_g)$ where $A_e$ is some structure and $X_g$, a vector at $g$.


%+++++++++++++++++++++++++++++++++++++++++++++++++++++++++++++++++++++++++++++++++++++++++++++++++++++++++++++++++++++++++++ 
\section{Exponential from the Lie algebra to the Lie group}
%+++++++++++++++++++++++++++++++++++++++++++++++++++++++++++++++++++++++++++++++++++++++++++++++++++++++++++++++++++++++++++

\begin{lemma}[\cite{Lie}]		\label{lemsur5d}
    Let $G$, $H$ be two Lie groups with algebras\footnote{Lie algebra of a Lie group, definition \ref{DEFooKDCPooZOJsMD}.} $\mG$ and $\mH$. Let $\dpt{\phi}{G}{H}$ be a homomorphism differentiable at $e$, the unit in $G$. Then for all $X\in\mG$, the following formula holds:
	\[
		\phi(\exp X)=\exp(d\phi_eX).
	\]
\end{lemma}

\begin{corollary}\label{Ad_e}
An useful formula:
\[
   \Ad(e^X)=e^{\ad X}.
\]
\end{corollary}

\begin{corollary}
Another useful corollary of lemma~\ref{lemsur5d} is the particular case $\phi=\AD(e^X)$:
\[
   e^Xe^Ye^{-X}=e^{Ad(e^Y)X}.
\]
\label{cor:eXeYe-X}
\end{corollary}

\begin{proposition}
	Let $G$ be a connected Lie group.
	\begin{enumerate}

		\item
			All the left invariant vector fields are complete. That means that the map $X\mapsto  e^{X}$ is defined for every $X\in \mG$.
		\item
			The map $\exp\colon \mG\to G$ is a local diffeomorphism in a neighbourhood of $0$ in $\mG$.
	\end{enumerate}
\end{proposition}

\begin{proof}
	\begin{enumerate}

		\item
			The flow is a one parameter subgroup. Thus if $ e^{tX}$ is defined for $t\in[0,a]$, by composition, $ e^{2a}$ is defined. So $ e^{tX}$ is defined for every value of $t$ in $\eR$.
		\item
			Let us consider the manifold $G\times \mG$ and the vector field $\Xi$ defined by
			\begin{equation}
				\Xi_{(g,X)}=\tilde X_g\oplus 0\in T_g(G)\oplus T_X\mG\simeq T_{(g,X)}(G\times \mG).
			\end{equation}
			The flow of that vector field is given by
			\begin{equation}
				\Phi_t(g,X)=\big( g\exp(tX),X \big).
			\end{equation}
			In particular, $\Xi$ is a complete vector field, and we consider the global diffeomorphism
			\begin{equation}
				\begin{aligned}
					\Phi_1\colon G\times \mG&\to G\times \mG \\
					(g,X)&\mapsto \big( g\exp(X),X \big).
				\end{aligned}
			\end{equation}
			On the point $(e,X)$ we have $\Phi_1(e,X)=(\exp(X),X)$. Thus the exponential is the projection on the first component of $\Phi_1(e,X)$ and we can write
			\begin{equation}
				\exp(X)=\pr_1\circ\Phi_1(e,X).
			\end{equation}
			It is a smooth function since both the projection and $\Phi_1$ are smooth.

			Now, the differential $(d\exp)_0$ is the identity on $\mG$, so that the theorem of inverse function makes $\exp$ a local diffeomorphism.
	\end{enumerate}
\end{proof}


\begin{theorem}
For any $p\in M$, there exist a $\delta>0$ and a neighbourhood $W$ of $p$ in $M$ such that for every $q\in W$, we have

\begin{itemize}
\item $\exp_q$ is a diffeomorphism on $B\bdelta(0)\subset T_qM$,
\item $\exp_q B\bdelta(0)$ contains $W$
\end{itemize}
\end{theorem}
This theorem says that everywhere on a differentiable manifold, one can find a neighbourhood which is a normal neighbourhood of each of its points. Such a neighbourhood is said a \emph{totally} normal neighbourhood.

\begin{lemma}
In a Lie group, $e$ is an isolated fixed point for the inversion.
\end{lemma}

\begin{proof}
One can use an exponential map in a neighbourhood of $e$. In this neighbourhood, an element $g$ can be written as $g=e^X$ for a certain $X\in\lG$. The equality $g=g^{-1}$ gives (because the exponential is a diffeomorphism) $X=-X$, so that $X=0$ and $g=e$.
\end{proof}

%--------------------------------------------------------------------------------------------------------------------------- 
\subsection{Properties using the exponential}
%---------------------------------------------------------------------------------------------------------------------------

\begin{theorem}
Let $G$ and $H$ be two Lie groups and $\dpt{\varphi}{G}{H}$ a continuous homomorphism. Then $\varphi$ is analytic.
\end{theorem}

\begin{proof}
The Lie algebra of the product manifold $G\times H$ as $\lG\times\lH$ is given in~\ref{lemLeibnitz}. We define
\begin{equation}
  K=\{(g,\varphi(g)):g\in G\}\subset G\times H.
\end{equation}
It is clear that $K$ is closed in $G\times H$ because $G$ is closed and $\varphi$ is continuous.
By theorem~\ref{tho:diff_sur_ferme}, there exists an unique differentiable structure on $G\times H$ such that $K$ is a topological Lie subgroup of $G\times H$ (i.e.: Lie subgroup + induced topology). The Lie algebra of $K$ is
\begin{equation}
  \lK=\{(X,Y)\in\lG\times\lH:\forall t\in\eR, (e^{tX},e^{tY})\in K\}.
\end{equation}
Let $N_0$ be an open neighbourhood of $0$ in $\lH$ such that $\exp$ is diffeomorphic between $N_0$ and an open neighbourhood $N_e$ of $e$ in $H$. We define $M_0$ and $M_e$ in the same way, for $G$ instead of $H$. We can suppose $\varphi(M_e)\subset N_e$: if it is not, we consider a smaller $M_e$: the openness of $N_e$ and the continuity of $\varphi$ make it coherent.

The lemma~\ref{lem:sugroup_normal} allow us to consider $M_0$ and $N_0$ small enough to say that
\[
   \dpt{\exp}{(M_0\times N_0)\cap\lK}{(M_e\times N_e)\cap K}
\]
is diffeomorphic. Now, we are going to show that for any $X\in\lG$, there exists an unique $Y\in\lH$ such that $(X,Y)\in\lK$. The unicity is easy: consider $(X,Y_1),(X,Y_2)\in\lK$; then $(0,Y_1-Y_2)\in\lK$ (because a Lie algebra is a vector space). Then the definition of $\lK$ makes for any $t\in\eR$, $(e,\exp{t(Y_1-Y_2)})\in K$. Consequently, $\exp t(Y_1-Y_2)=\varphi(e)=e$ and then $Y_1-Y_2=0$.

In order to show the existence, let us consider a $r>0$ such that $X_r=(1/r)X$ keeps in $M_0$. This exists because the sequence $X_r\to 0$ (then it comes $M_0$ from a certain $r$). From the definitions, $\exp$ is diffeomorphic between $M_0$ and $M_e$, then $\exp X_r\in M_e$ and $\varphi(\exp X_r)\in N_e$ because $\varphi(M_e)\subset N_e$.

From this, there exists an unique $Y_r\in N_0$ such that $\exp Y_r=\varphi(\exp X_r)$ and an unique $Z_r\in(M_0\times N_0)\cap\lK$ satisfying  $\exp Z_r=(\exp X_r,\exp Y_r)$. But $\exp$ is bijective from $M_0\times N_0$, so that $Z_r=(X_r,Y_r)$ and we can choose $Y=rY_r$ as a $Y\in\lH$ such that $(X,Y)\in\lK$ (it is not really a choice: the unicity was previously shown). We denotes by $\dpt{\psi}{\lG}{\lH}$ the map which gives the unique $Y\in\lH$ associated with $X\in\lG$ such that $(X,Y)\in\lK$. This is a homomorphism between $\lG$ and $\lH$.

By definition, $(X,\psi(X))\in\lK$, i.e. $(\exp tX,\exp t\psi(X))\in K$ or
\begin{equation}
  \varphi(\exp tX)=\exp t\psi(X).
\end{equation}
Let us now consider a basis $\{X_1,\ldots,X_n\}$ of $\lG$. Since $\varphi$ is a homomorphism,
\begin{equation}\label{eq:coord_vp_exp}
   \varphi\big((\exp t_1X_1)(\exp t_2X_2)\ldots(\exp t_nX_n)\big)
     =\big(\exp t_1\psi(X_1)\big)\ldots\big( \exp t_n\psi(X_n) \big)
\end{equation}
Now, we apply lemma~\ref{lem:decomp} on the decomposition of $\lG$ into the $n$ subspace spanned by the $n$ vector basis (this is $n$ applications of the lemma), the map
\[
  (\exp t_1X_1)\ldots(\exp t_nX_n)\to (t_1,\ldots,t_n)
\]
is a coordinate system around $e$ in $G$. In this case, the relation \eqref{eq:coord_vp_exp} shows that $\varphi$ is differentiable at $e$. Then it is differentiable anywhere in $G$.
\end{proof}


\begin{proposition}
Let $G$ be a Lie group and $H$, a Lie subgroup of $G$ ($\lG$ and $\lH$ are the corresponding Lie algebras). We suppose that $H$ has at most a countable number of connected components. Then
\begin{equation}
  \lH=\{ X\in\lG:\forall t\in\eR,e^{tX}\in H \}
\end{equation}
\end{proposition}

\begin{proof}
We will once again use the lemma ~\ref{lem:decomp} with $\lN=\lH$ and $\lM$, a complementary vector space of $\lH$ in $\lG$. We define
\[
   V=\exp\mU_m\exp\mU_h
\]
where $\mU_m$ and $\mU_h$ are the sets given by the lemma. We consider on $V$ the induced topology from $G$. If we define
\[
   \mA=\{A\in\mU_m:e^{A}\in H\},
\]
we have
\begin{equation}\label{eq:union_A}
   H\cap V=\bigcup_{A\in\mA}e^{A}e^{\mU_h}.
\end{equation}
First, the definition of $V$ makes clear that the elements of the form $\exp A\exp\mU_h$ are in $V$. They are also in $H$ because $\exp A\in H$ (definition of $\mA$) and $\exp\mU_h$ still by definition. In order to see the inverse inclusion, let us consider a $h\in H\cap V$. We know that
\begin{equation}\label{eq:AB_to_exp}
(A,B)\to\exp A\exp B
\end{equation}
is a diffeomorphism between $\mU_m\times\mU_h$ and a neighbourhood of $e$ in $G$ which we called $V$. Thus any element of $V$ (\emph{a fortiori} in $V\cap H$) can be written as $\exp A\exp B$ with $A\in\mU_m$ and $B\in\mU_h$. Then $h=e^Ae^B$ for some $A\in\mU_m$, $B\in\mU_h$. Since $H$ is a group and $e^B\in H$, in order the product to belongs to $H$, $e^A$ must lies in $H$: $A\in\mA$.

\begin{remark}\label{rem:union_disj}
Note that since \eqref{eq:AB_to_exp} is diffeomorphic, the union in right hand side of \eqref{eq:union_A} is disjoint. Each member of this union is a neighbourhood in $H$ because it is a set $h\exp\mU_h$ where $\exp\mU_h$ is a neighbourhood of $e$ in $H$.
\end{remark}

Now we consider the map $\dpt{\pi}{V}{\mU_m}$,
\[
  \pi(e^{X}e^Y)=X
\]
if $X\in\mU_m$ and $Y\in\mU_h$. This is a continuous map which sends $H\cap V$ into $\mA$. The identity component of $H\cap V$ (in the sense of topology of $V$) is sent to a countable subset of $\mU_m$. Indeed by remark~\ref{rem:union_disj}, identity component of $H\cap V$ is only one of the terms in the union \eqref{eq:union_A}, namely $A=0$. But we know that $\pi^{-1}(o)=\exp\mU_h$, thus $\exp\mU_h$ is the identity component of $H\cap V$ for the topology of $V$.
\end{proof}


\begin{theorem}
Let $G$ be a Lie group and $H$, a Lie subgroup of $G$.
\begin{enumerate}
\item If $H$ is a topological Lie subgroup of $G$, then it is closed in $G$,
\item If $H$ has at most a countable number of connected components and is closed in $G$, then $H$ is a topological subgroup of $G$.
\end{enumerate}
\label{tho:H_ferme}
\end{theorem}

\begin{proof}
\subdem{First point} It is sufficient to prove that if a sequence $h_n\in H$ converges (in $G$) to $g\in G$, then $g\in H$ (this is almost the definition of a closed subset). We consider $V$, a neighbourhood of $0$ in $\lG$ such that

\begin{itemize}
\item $\exp$ is diffeomorphic between $V$ and an open neighbourhood  of $e$ in $G$,
\item $\exp(V\cap \lH)=(\exp V)\cap H$.
\end{itemize}
This exists by the lemma~\ref{lem:sugroup_normal}; we can suppose that $V$ is bounded. Consider $\mU$, an open neighbourhood of $0$ in $\lG$ contained in $V$ such that $\exp-\mU\exp\mU\subset\exp V$.

Since $h_n\to g$, there exists a $N\in\eN$ such that $n\geq N$ implies $h_n\in g\exp\mU$ (i.e $h_n$ is the product of $g$ by an element rather close to $e$; since the multiplication is differentiable, the notion of ``not so far''\ is good to express the convergence notion). From now we only consider such elements in the sequence. So, $h_N^{-1} h_n\in(\exp V)\cap H$ ($n\geq N$) because
\[
   h_N^{-1} h_n\in\exp-\mU g^{-1} g\exp\mU\subset\exp V.
\]
(note that $H$ is a group, then $h_i^{-1}\in H$) From the second point of the definition of $V$, there exists a $X_n\in V\cap\lH$ such that $h^{-1}_N h_n=\exp X_n$ for any $n\geq N$.

Since $V$ in bounded, there exists a subsequence out of $(X_i)$ (which is also called $X_i$) converging to a certain $Z\in\lG$. But $\lH$ is closed in $\lG$ because it is a vector subspace (we are in a finite dimensional case), then $Z\in\lH$ and thus the sequence $(h_i)$ converges to $h_N\exp Z$; therefore $g\in H$.

\subdem{Second point} The subgroup $H$ is closed in $G$ and has a countable number of connected component. Since $H$ is closed, theorem~\ref{tho:diff_sur_ferme} it has an analytics structure for which it is a topological Lie subgroup of $G$. We denotes by $H'$ this Lie group.

The identity map $\dpt{I}{H}{H'}$ is continuous\quext{pourquoi ?} (see error~\ref{err:gross}). Thus any connected component of $H$ is contained in a connected component of $H'$, the it has only a countable number of connected components. By corollary~\ref{cor:top_subgroup}, $H=H'$ as Lie group.

\end{proof}

Now we take back our example with $G=S^1\times S^1$, $H=\gamma(\eR)$. In this case, the theorem doesn't works. Let us see why as deep as possible. We have $\lG=\eR\oplus\eR=\eR^2$ and $\lH=\eR$, a one-dimensional vector subspace of $\lG$. ($\lH$ is a ``direction ''\ in $\lG$) First, we build the neighbourhood $V$ of $0$ in $\lG$. It is standard to require that $\exp$ is diffeomorphic between $V$ and an open around $(1,1)\in S^1\times S^1$. It also must satisfy $e^{V\cap\lH}=e^V\cap H$. This second requirement is impossible.

Intuitively. We can see $V\subset\lG$ as a little disk tangent to  the torus. The exponential map deposits it on the torus, as well that $e^V$ covers a little area on $G$. Then $e^V\cap H$ is one of these amazing open subset of $\Gamma$ which are dense in a certain domain of $G$.

On the other hand, $V\cap\lH$ is just a little vector in $\lH$; the exponential deposits it on a small line in $G$. This is not the same at all. Then lemma~\ref{lem:sugroup_normal} fails in our case. Let us review the proof of this lemma until we find a problem.

Let $W_0\subset\lG$  be a neighbourhood of $0$ which is in bijection with an open around $e$ in $G$. We consider $N_0$, an open subset of $H$ such that $N_0\subset W_0$ and $N_0$ is in bijection with $N_e$, a neighbourhood of $e$ in $G$. Until here, no problems. But now the proof says that there exists an open $U_e$ in $G$ such that $N_e=U_e\cap H$. This is false in our case. Indeed, $N_e=e^{N_0}$ is just a segment in $G$ while any subset of $G$ of the form $U_e\cap H$ is an ``amazing''\ open.

So we see that deeply, the obstruction for a Lie subgroup to be a topological Lie subgroup resides in the fact that the topology of a submanifold is \emph{more} than the induced topology, so that we can't automatically find the open $U_e$ in $G$.

%+++++++++++++++++++++++++++++++++++++++++++++++++++++++++++++++++++++++++++++++++++++++++++++++++++++++++++++++++++++++++++ 
\section{Connected components}
%+++++++++++++++++++++++++++++++++++++++++++++++++++++++++++++++++++++++++++++++++++++++++++++++++++++++++++++++++++++++++++

\begin{lemma}\label{lem:vp_G_X}
    Let $G$ be a connected Lie group with Lie algebra $\lG$. If $\dpt{\varphi}{G}{X}$ is an analytic homomorphism ($X$ is a Lie group with Lie algebra $\lX$), then

    \begin{enumerate}
    \item The kernel $\varphi^{-1}(e)$ is a topological Lie subgroup of $G$; his algebra is the kernel of $d\varphi_e$.
    \item The image $\varphi(G)$ is a Lie subgroup of $X$ whose Lie algebra is $d\varphi(\lG)\subset\lX$.
    \item The quotient group $G/\varphi^{-1}(e)$ with his canonical analytic structure is a Lie group. The map $g\varphi^{-1}(e)\mapsto\varphi(g)$ is an analytic isomorphism $G/\varphi^{-1}(e)\to\varphi(G)$. In particular the map $\dpt{\varphi}{G}{\varphi(G)}$ is analytic.
    \end{enumerate}
\end{lemma}

\begin{proof}
\subdem{First item} We know that a subgroup $H$ closed in $G$ admits an unique analytic structure such that $H$ becomes a topological Lie subgroup of $G$. This is the case of $\varphi^{-1}(e)$. We know that $Z\in\lG$ belongs to the Lie algebra of $\varphi^{-1}(e)$ if and only if $\varphi(\exp tZ)=e$ for any $t\in\eR$. But $\varphi(\exp tZ)=\exp(td\varphi(Z))=e$ if and only if $d\varphi(Z)=0$.

\subdem{Second item}
Consider $X_1$, the analytic subgroup of $X$ whose Lie algebra is $d\varphi(\lG)$. The group $\varphi(G)$ is generated by the elements of the form $\varphi(\exp Z)$ for $Z\in\lG$. The group $X_1$ is generated by the $\exp(d\varphi Z)$. Because of lemma~\ref{lemsur5d}, these two are the same. Then $\varphi(G)=X_1$ and their Lie algebras are the same.

\subdem{Third item}
We consider $H$, a closed normal subgroup of $G$; this is a topological subgroup and the quotient $G/H$ has an unique analytic structure such that the map $G\times G/H\to G/H$, $(g,[x])\to [gx]$ is analytic. We consider a decomposition $\lG=\lH\oplus\lM$ and we looks at the restriction $\dpt{\psi}{\lM}{G}$ of the exponential. Then there exists a neighbourhood $U$ of $0$ in $\lM$ which is homomorphically send by  $\psi$ into an open neighbourhood of $e$ in $G$ and such that $\dpt{\pi}{G}{G/H}$ sends homomorphically $\psi(U)$ to a neighbourhood  of $p_0\in G/H$ (cf. lemma~\ref{lem:vois_U}).

We consider $\UU$, the interior of $U$ and $B=\psi(\UU)$. The following diagram is commutative:
\begin{equation}
 \xymatrix{
    G\times G/H  \ar[rr]^{\displaystyle\Phi}\ar[dr]_{\displaystyle \pi\times I} &&  G/H\\
     &     G/H\times G/H\ar[ur] _{\displaystyle\alpha}
  }
\end{equation}
with $\Phi(g,[x])=[g^{-1} x]$, $(\pi\times I)(g,[x])=([g],[x])$ and $\alpha([g],[x])=[g^{-1} x]$. Indeed,
\[
   \alpha\circ(\pi\times I)(g,[x])=\alpha([g],[x])=[g^{-1} x].
\]
In order to see that $\alpha$ is well defined, remark that if $[h]=[g]$ and $[y]=[x]$ $[g^{-1} x]=[h^{-1} y]$ because $H$ is a normal subgroup of $G$.

Now, we consider $g_0,x_0\in G$ and the restriction of $(\pi\times I)$ to $(g_0B)\times(G/H)$. Since $\pi$ is homeomorphic on $\psi(U)$ and $B=\psi(\UU)$, on $g_0B$, $\pi$ is a diffeomorphism (because the multiplication is diffeomorphic as well)

\begin{probleme}\label{prob:diffeo_2}
    Why is the \( \pi\) a diffeomorphism? I understand why it is qn homeomorphism, but no more.
\end{probleme}

This diffeomorphism maps to a neighbourhood $N$ of $([g_0],[x_0])$ in $G/H\times G/H$. From the commutativity, we know that $\alpha=\Phi\circ(\pi\times I)^{-1}$, so that $\alpha$ is analytic. Consequently, $G/H$ is a Lie group. On $N$, $\alpha$ is analytic, then $\alpha(N)$ is analytic.

All this is for a closed normal subgroup $H$ of $G$. Now we consider $H=\varphi^{-1}(e)$ and $\lH$, the Lie algebra of $H$. From the first item, we know that the Lie algebra of $H$ is the kernel of $d\varphi$: $\lH=d\varphi^{-1}(0)$ which is an ideal in $\lG$.

From the second point, the Lie algebra of $G/H$ is $d\pi(\lG)$ which is isomorphic to $\lG/\lH$; the bijection is $\gamma(d\pi(X))=[X]\in\lG/\lH$. In order to prove the injectivity, let us consider $\gamma(A)=\gamma(B)$; $A=d\pi(X)$, $B=d\pi(Y)$. The condition is $[X]=[Y]$; thus it is clear that $d\pi(X)=d\pi(Y)$

Let us consider on the other hand the map $Z+\lH\to d\varphi(Z)$ for $Z\in\lG$\footnote{Note that $\lG$ and $\lH$ are not groups; by $[X]$, we mean $[X]=\{ X+h\tq h\in\lH \}$.}. In other words, the map is $[Z]\to d\varphi(Z)$. This is an isomorphism $\lG/\lH\to d\varphi(\lG)$, which gives a local isomorphism between $G/H$ and $\varphi(G)$. This local isomorphism is $[g]\to\varphi(g)$ for $g$ in a certain neighbourhood of $e$ in $G$.

Since $[g]\to\varphi(g)$ has a differential which is an isomorphism, this is analytic at $e$. Then it is analytic everywhere.

\end{proof}


\begin{corollary}
If $G$ is a connected Lie group and if $Z$ is the center of $G$, then
\begin{enumerate}
\item $\Ad_G$ is an analytic homomorphism from $G$ to $\Int(G)$, with kernel $Z$,
\item the map $[g]\to\Ad_G(g)$ is an analytic isomorphism from $G/Z$ to $\Int(\lG)$ (the class $[g]$ is taken with respect to $Z$).
\end{enumerate}
\label{cor:Ad_homom}
\end{corollary}


\begin{proof}
\subdem{First item}
A connected Lie group is generated by a neighbourhood of identity, and any element of a suitable such neighbourhood can be written as the exponential of an element in the Lie algebra. So $\Int(\lG)$ is generated by elements of the form $\exp(\ad X)=\Ad(\exp X)$; this shows that $\Int(\lG)\subset\Ad(G)$. In order to find the kernel, we have to  see $\Ad_G^{-1}(e)$ by the formula
\[
   e^{\Ad(g)X}=g e^Xg^{-1}.
\]
We have to find the $g\in G$ such that $\forall X\in\lG$, $\Ad_G(g)X=X$. We taking the exponential of the two sides and using \eqref{eq:sigma_X_sigma},
\begin{equation}
  g e^Xg^{-1}=e^X.
\end{equation}
Then $g$ must commute with any $e^X\in G$: in other words, $g$ is in the kernel of $G$.

\subdem{Second item}
This is contained in lemma~\ref{lem:vp_G_X}. Indeed $G$ is connected and we had just proved that $\dpt{\Ad_G}{G}{\Int(\lG)}$ with kernel $Z$; the third item of lemma~\ref{lem:vp_G_X} makes $G/Z$ a Lie group and the map $[g]\to\Ad_G(g)$ an analytic isomorphism from $G/Z$ to $\Ad_G(G)=\Int(\lG)$.
\end{proof}


\begin{lemma}
Let $G_1$ and $G_2$ be two locally isomorphic connected Lie groups with trivial center (i.e. $\lG_1=\lG_2=\lG$ and $Z(G_i)=\{ e \}$). In this case, we have $G_1=G_2=\Int(\lG)$ where $\Int\lG$ stands for the group of internal automorphism of $\lG$.
\end{lemma}

\begin{proof}
We denote by $G_0$ the group $\Int\lG$. The adjoint actions $\Ad_i\colon G_i\to G_0$ are both surjective because of corollary~\ref{cor:Ad_homom}. Let us give an alternative proof for injectivity. Let $Z_i=\ker(\Ad_i)=\{ g\in G_i\tq\Ad(g)X=X,\,\forall X\in\lG \}$. Since $G_i$ is connected, it is generated by any neighbourhood of the identity in the sense of proposition~\ref{PropUssGpGenere}; let $V_0$ be such a neighbourhood. Taking eventually a subset we can suppose that $V_0$ is a normal coordinate system. So we have
\[
  g\exp_{G_i}(X)g^{-1}=\exp_{g_i}(X)
\]
for every $X\in V_0$. Using proposition~\ref{PropUssGpGenere} we deduce that $gxg^{-1}=x$ for every $x\in G_i$, thus $g\in Z(G_i)$. That proves that $\ker(\Ad_i)\subset Z(G_i)$. The assumption of triviality of $Z(G_i)$ concludes injectivity of $\Ad_i$.
\end{proof}

\begin{corollary}
Let $\lG$ be a real Lie algebra with center $\{0\}$. Then the center of $\Int(\lG)$ is only composed of the identity.
\end{corollary}

\begin{proof}
We note $G'=\Int(\lG)$ and $Z$ his center; $\ad$ is the adjoint representation of $\lG$ and $\Ad'$, $\ad'$, the ones of $G'$ and $\ad(\lG)$ respectively. We consider the map $\dpt{\theta}{G'/Z}{\Int(\ad(\lG))}$, $\theta([g])=\Ad'(g)$. By the second item of the corollary~\ref{cor:Ad_homom}, $[g]\to\Ad_{G'}(g)$ is an analytic homomorphism from $G'$ to $\Int(\lG')$ where $\lG'$ is the Lie algebra of $G'$; this is $\ad(\lG)$. So $\dpt{\theta}{G'/Z}{\Int(\lG')}$ is isomorphic.

Now we consider the map $\dpt{s}{\lG}{\ad(\lG)}$, $s(X)=\ad(X)$; this is an isomorphism. We also consider $\dpt{S}{G'}{ \GL(\ad(\lG))}$, $S(g)=s\circ g\circ s^{-1}$. The Lie algebra of $S(G')$ is $\ad(\lG')=\ad\big(\ad(\lG)\big)$. Then $S(G')$ is the subset of $\GL(\ad\lG)$ whose Lie algebra is $\ad\big(\ad\lG\big)$, i.e. exactly $\Int(\ad\lG)$. So $S$ is an isomorphism $\dpt{S}{G'}{\Int(\ad\lG)}$. From all this,
\begin{equation}
   S(e^{\ad X})=s\circ e^{\ad X}\circ s^{-1}
               =e^{\ad'(\ad X)}
           =\Ad'(e^{\ad X}).
\end{equation}
With this equality, $\dpt{S^{-1}\circ\theta}{G'/Z}{G'}$ is an isomorphism which sends $[g]$ on $g$ for any $g\in Z$. Then $Z$ can't contains anything else than the identity.
\end{proof}

If we relax the assumptions of the trivial center, we have a counter-example with $\lG=\eR^3$ and the commutations relation
\[
   [X_1,X_2]=X_3,\quad [X_1,X_3]=[X_2,X_3]=0.
\]
The group $\Int(\lG)$ is abelian; then his center is the whole group, although $\lG$ is not abelian.

Note that two groups which have the same Lie algebra are not necessarily isomorphic. For example the sphere $S^2$ and $\eR^2$ both have $\eR^2$ as Lie algebra. But two groups with same Lie algebra are locally the same. More precisely, we have the following lemma.

\begin{lemma}
If $G$ is a Lie group and $H$, a topological subgroup of $G$ with the same Lie algebra ($\lH=\lG$), then there exists a common neighbourhood $A$ of $e$ of $G$ and $G$ on which the products in $G$ and $H$ are the same.
\end{lemma}

\begin{proof}
The exponential is a diffeomorphism between $U\subset\lG$ and $V\subset G$ and between $U'\subset\lH$ and $W\subset H$ (obvious notations). We consider an open $\mO\subset\lH$ such that $\mO\subset U\subset U'$. The exponential is diffeomorphic from $\mO$ to a certain open $A$ in $G$ and $H$. Since $H$ is a subgroup of $G$, the product $e^Xe^Y$ of elements in $A$ is the same for $H$ and $G$. (cf error~\ref{err:gp_meme_alg})
\end{proof}

Under the same assumptions, we can say that $H$ contains at least the whole $G_0$ because it is generated by any neighbourhood of the identity. Since $H$ is a subgroup, the products keep in $H$.

For a semisimple Lie group, the Lie algebras $\partial(\lG)$ and $\ad(\lG)$ are the same. Then $\Int(\lG)$ contains at least the identity component of $\Aut(\lG)$. Since $\Int(\lG)$ is connected, for a semisimple group, it is the identity component of $\Aut(\lG)$.

%+++++++++++++++++++++++++++++++++++++++++++++++++++++++++++++++++++++++++++++++++++++++++++++++++++++++++++++++++++++++++++ 
\section{Compact Lie algebra}
%+++++++++++++++++++++++++++++++++++++++++++++++++++++++++++++++++++++++++++++++++++++++++++++++++++++++++++++++++++++++++++
\label{pg:compact_Lie}

We consider $\lG$, a real Lie algebra and $\lH$, a subalgebra of $\lG$. Let $K^*$ be the analytic subgroup of $\Int(\lG)$ which corresponds to the subalgebra $\ad_{\lG}(\lH)$ of $\ad_{\lG}(\lG)$.

\begin{definition}      \label{DEFooROMGooTLicyL}
We say that $\lH$ is \defe{compactly embedded}{compactly embedded} in $\lG$ if $K^*$ is compact. A Lie algebra is \defe{compact}{compact!Lie algebra}\index{Lie!algebra!compact} when it is compactly embedded in itself.
\end{definition}

The analytic subgroup of $\Int(\lG)$ which corresponds to $\ad_{\lG}(\lG)$, by definition, is $\Int(\lG)$. Then the compactness of $\lG$ is the one of $\Int(\lG)$.

\begin{remark}
The compactness notion on a Lie group is defined from the topological structure of the Lie group seen as a manifold. It is all but trivial that the compactness on a Lie group is related to the compactness on its Lie algebra; the proposition~\ref{prop:alg_grp_compact} will however make the two notions related in the natural way.
\end{remark}

\begin{remark}
The topology on $K^*$ is not necessary the same as the induced one from $\Int(\lG)$ and $\Int(\lG)$ has also not necessary the induced topology from $\GL(\lG)$. However the next proposition will show that the compactness notion is well the one induced from $\GL(\lG)$.
\end{remark}

\begin{proposition}
We consider $\tK$, the same set and group as $K^*$, but with the induced topology from $\GL(\lG)$. Then $\tK$ is compact if and only if $K^*$ is compact.
\end{proposition}

Note however that $K^*$ and $\tK$ are not automatically the same as manifold.

\begin{proof}
\subdem{$K^*$ compact implies $\tK$ compact}
The identity map $\dpt{\iota}{K^*}{\GL(\lG)}$ is analytic, and then is continuous because $\Int(\lG)$ is by definition an analytic subgroup of $\GL(\lG)$ and $K^*$ an analytic subgroup of $\Int(\lG)$. If we have a covering of $\tK$ with open set $\mO_i\cap\tK$ of $\tK$ ($\mO_i$ is open in $\GL(\lG)$), the continuity of $\iota$ make the finite subcovering of $K^*$ good for $\tK$.
\subdem{$\tK$ compact implies $K^*$ compact}
If $\tK$ is compact, then it is closed in $\GL(\lG)$. As set, $K^*$ is closed in $\GL(\lG)$ and by definition it is connected. Then by the theorem~\ref{tho:H_ferme}, $K^*$ is a topological subgroup of $\GL(\lG)$. Consequently, $K^*$ and $\tK$ are homeomorphic and they have same topology.
\end{proof}

\begin{lemma}[\cite{MonCerveau}]
    If $G$ is a compact group in $\GL(n,\eR)$, then there exists a $G$-invariant quadratic form on $\eR^n$\quext{Is it true ? I've not even found a precise statement of this claim.}.
\end{lemma}

\begin{proposition}     \label{ProplGcompactKillNeg}
Let $\lG$ be a real Lie algebra.

\begin{enumerate}
\item If $\lG$ is semisimple, then $\lG$ is compact if and only if  the Killing form is strictly negative definite.
\item If it is compact then it is a direct sum
\begin{equation}
   \lG=\mZ\oplus [\lG,\lG]
\end{equation}
where $\mZ$ is the center of $\lG$ and the ideal $[\lG,\lG]$ is compact and semisimple.
\end{enumerate}
\label{prop:compact_Killing}
\end{proposition}

\begin{proof}
\subdem{If the Killing form is nondegenerate}
We consider $\lG$, a Lie algebra whose Killing form is strictly negative definite. Up to some dilatations (and a sign), this is the euclidian metric. Then $O(B)$, the group of linear transformations which leave $B$ unchanged is compact in the topology of $\GL(\lG)$: this is almost the rotations. From equation \eqref{eq:Aut_Iso}, $\Aut(\lG)\subset O(B)$. With this, $\Aut(\lG)$ is closed in a compact, then it is compact. Then $\Int(\lG)$ is closed in $\Aut(\lG)$ --here is the assumption of semi-simplicity-- and $\Int(\lG)$ is compact.
\subdem{If $\lG$ is compact}
Since $\lG$ is compact, $\Int(\lG)$ is compact in the topology of $\Aut(\lG)$; then there exists an $\Int(\lG)$-invariant quadratic form $Q$. In a suitable basis $\{X_1,\ldots,X_n\}$ of $\lG$, we can write this form as
\[
   Q(X)=\sum x_i^2
\]
for $X=\sum x_iX_i$. In this basis the elements of $\Int(\lG)$ are orthogonal matrices and the matrices of $\ad(\lG)$ are skew-symmetric matrices (the Lie algebra of orthogonal matrices). Let us consider a $X\in\lG$ and denote by $a_{ij}(X)$ the matrix of $\ad(X)$. We have
\begin{equation}
\begin{split}
  B(X,X)=\tr(\ad X\circ\ad X)
        =\sum_i\sum_ja_{ij}(X)a_{ji}(X)
    =-\sum_{ij}a_{ij}(X)^2\leq 0.
\end{split}
\end{equation}
Then the Killing form is negative definite\footnote{Here we use ``negative definite''\ and ``\emph{strictly} negative definite''; in some literature, the terminology is slightly different and one says ``\emph{semi} negative definite''\ and ``negative definite''.}. On the other hand, $B(X,X)=0$ implies $\ad(X)=0$ and $X\in\mZ(\lG)$. Thus $\lG^{\perp}\subset\mZ$. If $\lG$ is semisimple, this center is zero; this conclude the first item of the proposition.

Now $\mZ$ is an ideal and corollary~\ref{cor:decomp_ideal} decomposes $\lG$ as
\begin{equation}
  \lG=\mZ\oplus\lG'.
\end{equation}
Let us suppose that the restriction of $B$ to $\lG'\times\lG'$ is actually the Killing form on $\lG'$ (we will prove it below). Then the Killing form on $\lG'$ is strictly negative definite; then $\lG'$ is compact.

Now we prove that the Killing form on $\lG$ descent to the Killing form on~$\lG'$. Remark that $\mZ$ is invariant under all the automorphism. Indeed consider $Z\in\mZ$, i.e.  $[X,Z]=0$. If $\sigma$ is an automorphism,
\[
   [X,\sigma Z]=\sigma[\sigma^{-1} X,Z]=0.
\]
Here the difference between $\Int(\lG)$ and $\Aut(\lG)$ is the fact that $\Int(\lG)$ is compact; then we can construct a $\Int(\lG)$-invariant quadratic form $Q$, but not a $\Aut(\lG)$-invariant one. We consider an orthogonal complement (with respect to $Q$) $\lG'$ of $\mZ$:
\begin{equation}
   \lG=\lG'\oplus_{\perp}\mZ.
\end{equation}
The algebra $\lG'$ is also invariant because for any $Z\in\mZ$,
\[
Q(Z,\sigma X)=Q(\sigma^{-1}(Z),X)=0.
\]
It is also clear that $\mZ$ is invariant under $\ad\lG$ because $(\ad X)Z=0$. Finally $\lG'$ is invariant as well under $\ad(\lG)$. Indeed $a\in\ad(\lG)$ can be written as $a=a'(0)$ for a path $a(t)\in\Int(\lG)$. We identify $\lG$ and his tangent space (as vector spaces),
\[
  aX=\Dsdd{ a(t)X }{t}{0}.
\]
If $X\in\lG'$, $a(t)X\in\lG'$ for any $t$ because $\lG'$ is invariant under $\Int(\lG)$\footnote{As physical interpretation, if something is invariant under a group of transformations, it is invariant under the infinitesimal transformations as well.}. Thus $a(t)X$ is a path in $\lG'$ and his derivative is a vector in $\lG'$.

All this make $\lG'$ an ideal in $\lG$; then the Killing form descent by lemma~\ref{lem:Killing_descent_ideal}. Now if $X\in\lG$, we have
\begin{equation}
  B(X,X)=\tr(\ad X\circ\ad X)
        =\sum_{ij}a_{ij}(X)a_{ji}(X)
    =-\sum_{ij}a_{ij}(X)^2;
\end{equation}
then $B(X,X)\leq 0$ and the equality holds if and only if $\ad X=0$ i.e. if and only if $X\in\mZ$. Thus $B$ is strictly negative definite on $\lG'$.

Up to now we have proved that $\lG'$ is semisimple (because $B$ is nondegenerate) and compact (because $B$ is strictly negative definite).

It remains to be proved that $\lG'=[\lG,\lG]=\dD(\lG)$. From corollary~\ref{cor:decomp_ideal}, $\dD\lG$ has a complementary $\lA$ which is also an ideal: $\lG=\dD\lG+\lA$. Then $[\lG,\lA]\subset\dD\lG$ and $[\lG,\lA]\subset\lA\cap\dD\lG:\{0\}$. Then $\lA\subset\mZ$, so that
\begin{align}\label{eq:G_Z_B}
   \lG=\mZ+\dD\lG&&\text{(non direct sum)}.
\end{align}
Now we have to prove that the sum is actually direct. The ideal $\mZ$ has a complementary ideal $\lB$: $\lG=\mZ\oplus\lB$ and
\[
   \dD\lG=[\lG,\lG]\subset\underbrace{[\lG,\mZ]}_{=0}+[\lG,\lB]\subset\lB.
\]
Then $\dD\lG\subset\lB$ which implies that $\dD\lG\cap\mZ=\{0\}$ because the sum $\lG=\mZ\oplus\lB$ is direct. Then the sum \eqref{eq:G_Z_B} is direct.

\end{proof}

\begin{proposition}
A real Lie algebra $\lG$ is compact if and only if one can find a compact Lie group $G$ which Lie algebra is isomorphic to $\lG$.
\label{prop:alg_grp_compact}
\end{proposition}

\begin{proof}
\subdem{Direct sense} Since $\lG$ is compact, $\lG=\mZ\oplus\dD\lG$ with $\dD\lG=\lG'$ compact and semisimple; in particular, the center of $\lG'$ is $\{0\}$. Since $\mZ$ is compact and abelian, it is isomorphic to the torus $S^1\times\ldots\times S^1$. Since $\lG'$ is compact, $\Int(\lG')$ is compact, but the Lie algebra if $\Int(\lG')$ is --by definition--  $\ad(\lG')$. The center of a semisimple Lie algebra is zero; then $\ad X'=0$ implies $X=0$ (for $X\in\lG'$). Then $\ad$ is an isomorphism between $\lG'$ and $\ad\lG'$.

All this shows that --up to isomorphism-- $\mZ$ and $[\lG,\lG]$ are Lie algebras of compact groups. We know from lemma~\ref{lemLeibnitz} that the Lie algebra of $G\times H$ is $\lG\oplus\lH$. Thus, here, $\lG$ is the Lie algebra of the compact group $S^1\times\ldots\times S^1\times\Int(\lG)$.
\subdem{Reverse sense}
We consider a compact group $G$ and we have to see the its Lie algebra $\lG$ is compact. If $G$ is connected, $\Ad_G$ is an analytic homomorphism from $G$ to $\Int(\lG)$. If $G$ is not connected, the Lie algebra of $G$ is $T_eG_0$ ($G_0$ is the identity component of $G$) where $G_0$ is connected and compact because closed in a compact.
\end{proof}

\begin{proposition}
Let $\lG$ be a real Lie algebra and $\mZ$, the center of $\lG$. We consider $\lK$, a compactly embedded in $\lG$. If $\lK\cap\mZ=\{0\}$ then the Killing form of $\lG$ is strictly negative definite on $\lK$.
\label{prop:K_Z_Killing}
\end{proposition}

\begin{proof}
Let $B$ be the Killing form on $\lG$ and $K$ the analytic subgroup of $\Int(\lG)$ whose Lie algebra is $\ad_{\lG}(\lK)$. By assumption, $K$ is a compact Lie subgroup of $\GL(\lG)$. Then there exists a quadratic form on $\lG$ invariant under $K$, and a basis in which the endomorphisms $\ad_{\lG}(T)$ for $T\in\lK$ are skew-symmetric because the matrices of $K$ are orthogonal. If the matrix of $\ad T$ is $(a_{ij})$, then
\begin{equation}
   B(T,T)=\sum_{ij}a_{ij}(T)a_{ji}(T)
         =-\sum_{ij}a_{ij}^2(T)\leq 0,
\end{equation}
and the equality hold only if $\ad T=0$ i.e. if $T\in\mZ$. From the assumptions, $\lK\cap\mZ=\{0\}$; then $B(T,T)=0$ if and only if $T=0$.
\end{proof}

%+++++++++++++++++++++++++++++++++++++++++++++++++++++++++++++++++++++++++++++++++++++++++++++++++++++++++++++++++++++++++++ 
\section{Representations from the Lie algebra to the Lie group}
%+++++++++++++++++++++++++++++++++++++++++++++++++++++++++++++++++++++++++++++++++++++++++++++++++++++++++++++++++++++++++++

\begin{proposition}[\cite{BIBooYTTJooYpPYLT}]       \label{PROPooXCGMooKlJlwp}
    Let \( G\) be a Lie group and \( \lG\) be its Lie algebra. Let \( (\rho, V)\) be a smooth representation of \( G\). We consider the map
    \begin{equation}
        \begin{aligned}
            s\colon \lG&\to \End(V) \\
            s(X)v&=\Dsdd{ \rho( e^{tX})v }{t}{0}.
        \end{aligned}
    \end{equation}
    We have the equality
    \begin{equation}
        \rho( e^{tX})= e^{ts(X)}
    \end{equation}
    as operators on \( V\).
\end{proposition}

\begin{proof}
    Let \( X\in \lG\). We define \( M(t)=\rho( e^{tX})\) and \( N(t)= e^{ts(X)}\). These are maps from \( \eR\) to \( \End(V)\); the proposition \ref{PROPooSDNNooQtHkhA} helps to derive them.

    For \( N\) we immediately have
    \begin{equation}
        N'(t)=s(X) e^{ts(X)}.
    \end{equation}
    For \( M\), we have few more prudence. We fix \( \epsilon>0\) and we write (thanks to proposition \ref{PROPooKDKDooCUpGzE})
    \begin{equation}
        M(t+\epsilon)=\rho( e^{(t+\epsilon)X})=\rho( e^{tX} e^{\epsilon X})= \rho( e^{tX}) e^{\epsilon X}.
    \end{equation}
    Now for the differential quotient,
    \begin{equation}
        \frac{ M(t+\epsilon)-M(t) }{ \epsilon }=\rho( e^{tX})\frac{ \rho( e^{\epsilon X})-\id }{ \epsilon }.
    \end{equation}
    If we compute the limit \( \epsilon\to 0\), the second factor goes, by definition to \( s(X)\). So
    \begin{equation}
        M'(t)=\rho( e^{tX})s(X)=s(X)M(t).
    \end{equation}
    So \( M\) and \( N\) satisfy the same differential equation
    \begin{subequations}
        \begin{numcases}{}
            y'=s(X)y(t)\\
            y(0)=\mtu
        \end{numcases}
    \end{subequations}
    for the function \( y\colon \eR\to \End(V)\). This is a work for Cauchy-Lipschitz, theorem \ref{THOooZIVRooPSWMxg}. So we define \( f(t,m)=s(X)m\) and we show that this is Lipschitz with respect to \( m\) :
    \begin{subequations}
        \begin{align}
            \| f(t_0,m_0)-f(t,m) \|=\| s(X)(m_0-m) \|\leq \| s(X) \|\| m_0-m \|.
        \end{align}
    \end{subequations}
    So the function \( f\) is Lipschitz with respect to \( m\) with a Lipschitz constant bounded by \( \| s(X) \|\).

    The unicity part of Cauchy-Lipschitz shows that \( M(t)=N(t)\) for every \( t\) for which the expressions make sense.
\end{proof}

\begin{theorem}[\cite{BIBooYTTJooYpPYLT}]       \label{THOooLVSNooOpzYgO}
    Let \( G\) be a Lie group and \( \lG\) be its Lie algebra. If \( (\rho, V)\) is a representation of \( G\), the map
    \begin{equation}
        \begin{aligned}
            s\colon \lG&\to \End(V) \\
            s(X)v&=\Dsdd{ \rho( e^{tX})v }{t}{0} 
        \end{aligned}
    \end{equation}
    is a representation of \( \lG\) on \( V\).
\end{theorem}

\section{Real Lie algebras}
%++++++++++++++++++++++++++

\subsection{Real and complex vector spaces}
%//////////////////////////////////////////////

If $V$ is a real vector space, the \defe{complexification}{complexification!of a vector space} of $V$ is the vector space\nomenclature{$V\heC$}{Complexification of $V$}
\[
  V\heC:=V\otimes\beR\eC.
\]
If $\{v_i\}$ is a basis of $V$ on $\eR$, then $\{v_i\otimes 1\}$ is a basis of $V\heC$ on $C$. Then
\[
   \dim_{\eR}V=\dim_{\eC}V\heC.
\]

Let $W$ be a complex vector space. If one restrains the scalars to $\eR$, we find a real vector space denoted by $W\heR$\nomenclature{$W\heR$}{Restriction of a complex vector spaces to $\eR$}. If $\{w_j\}$ is a basis of $W$, then $\{w_j,iw_j\}$ is a basis of $W\heR$ and
\[
  \dim\beC W=\frac{1}{2}\dim\beR W\heR.
\]
Note that $(V\heC)\heR=V\oplus iV$.

A real vector space $V$ is a \defe{real form}{real!form!of complex vector space} of a complex vector space $W$ if $W\heR=V\oplus iV$. If $V$ is a real form of $W$, the map $\dpt{\varphi}{V\heC}{V\heC}$ given by the identity on $V$ and the multiplication by $-1$ on $iV$ is the \defe{conjugation}{conjugation} of $V\heC$ with respect of the real form $V$.

\subsection{Real and complex Lie algebras}
%/////////////////////////////////////////

For notational convenience, if not otherwise mentioned, $\lG$ will denote a complex Lie algebra and $\lF$ a real one. If $\lF$ is a real Lie algebra and $\lF\heC=\lF\otimes\eC$, its complexification (as vector space), we endow $\lF\heC$ with a Lie algebra structure by defining
\[
  [ (X\otimes a),(Y\otimes b)  ]=[X,Y]\otimes ab.
\]
This is a bilinear extension of the Lie algebra bracket of $\lF$. It is rather easy to see that $[\lF,\lF]\heC=[\lF\heC,\lF\heC]$.

Now we turn our attention to the Killing form. Let $\lF$ be a real Lie algebra with a Killing form $B\blF$. A basis of $\lF$ is also a basis of $\lF\heC$. Then the matrix $B_{ij}=\tr( \ad X_i\circ\ad X_j )$ of the Killing form is the same for $\lF\heC$ than for $\lF$. In conclusion:
\[
   B_{\lF\heC}|_{\lF\times\lF}=B\blF.
\]

Let us study the inverse process: $\lG$ is a complex Lie algebra and $\lG\heR$ is the real Lie algebra obtained from $\lG$ by restriction of the scalars. If $\mB=\{v_j\}$ is a basis of $\lG$, $\mB'=\{v_j,iv_j\}$ is a one of $\lG\heR$. For a certain $X\in\lG$ we denote by $(c_{kl})$ the matrix of $\ad_{\lG}X$. Now we study the matrix of $\ad_{\lG\heR}X$ in the basis $\mB'$ by computing
\begin{equation}
(\ad_{\lG}X)v_i=c_{ik}v_k
               =\big[ \real(c_{ik})+i\imag(c_{ik}) \big]v_k
           =a_{ik}v_k+b_{ik}(iv_k)
\end{equation}
if $a=\real c$ and $b=\imag c$. Then the columns of $\ad_{\lG\heR}$ which correspond to the $v_i\in\mB'$'s are given by
\[
\ad_{\lG\heR}X=\begin{pmatrix}
                 a&\cdot\\
         b&\cdot
               \end{pmatrix}
\]
where the dots denote some entries to be find now:
\begin{equation}
(\ad_{\lG}X)(iv_i)=i\big(  a_{ik}v_k+b_{ik}(iv_k)  \big)\\
                  =a_{ik}(iv_k)-b_{ik}v_k,
\end{equation}
so that the complete matrix of $\ad_{\lG\heR}X$ in the basis $\mB'$ is given by
\[
\ad_{\lG\heR}X=\begin{pmatrix}
                 a&-b\\
         b&a
               \end{pmatrix}.
\]
So,
\[
\ad_{\lG\heR}X\circ\ad_{\lG\heR}X'=\begin{pmatrix}
                 aa'-bb'&\cdot\\
         \cdot&aa'-bb'
               \end{pmatrix}.
\]
Then $B(X,X')=2\tr(aa'-bb')$ while
\begin{equation}
  B(X,Y)=\tr\big(  (a+ib)(a'+ib')  \big)\\
        =\tr(aa'-bb')+i\tr(ab'+ba').
\end{equation}
Thus we have
\begin{equation}
     B_{\lG\heR}=2\real B_{\lG},
\end{equation}
so that $\lG\heR$ is semisimple if and only if $\lG$ is semisimple.

\subsection{Split real form}
%//////////////////////////////

Let $\lG$ be a complex semisimple Lie algebra, $\lH$ a Cartan subalgebra, $\Phi$ the set roots, $\Delta$ the set of non zero roots and $B$, the Killing form. From property \eqref{eq:enuaiv} and the fact that $c(-\alpha,-\beta)=c(\alpha,\beta)$, we find $c(\alpha,\beta)^2=\frac{1}{2}\lbha(1+\lbba)|\alpha|^2$,
 so that $c(\alpha,\beta)^2\geq 0$ which gives $c(\alpha,\beta)\in\eR$. We can define

\[
   \lGeR=\lH_0\bigoplus_{\alpha\in\Phi}\eR x_{\alpha}.
\]
Remark that $\lG_{\alpha}$ has dimension one with respect to $\eC$, not $\eR$; then $\eR x_{\alpha}\neq\lG_{\alpha}$, but $\eC x_{\alpha}=\lG_{\alpha}$ and $\lG_{\alpha}=\eR x_{\alpha}\oplus i\eR x_{\alpha}$. Since it is clear that $\bigoplus_{\alpha\in\Delta}( \eR x_{\alpha}\oplus i\eR x_{\alpha} )=\bigoplus_{\alpha\in\Delta}\lG_{\alpha}$, the proposition~\ref{prop:lHeR} gives
\begin{equation}
  \lG=\lGeR\oplus i\lGeR.
\end{equation}
Any real form of $\lG$ which contains the $\lHeR$ of a certain Cartan subalgebra $\lH$ of $\lG$ is said a \defe{split real form}{split!real form}. The construction shows that any complex semisimple Lie algebra admits a split real form.

\subsection{Compact real form}
%///////////////////////////////

\begin{definition}
    A \defe{compact real form}{compact!real form}\index{compact!Lie algebra} of a complex Lie algebra is a real form which is compact as Lie algebra\footnote{Definition \ref{DEFooROMGooTLicyL}.}. 
\end{definition}

\begin{theorem}
Any complex semisimple Lie algebra contains a compact real form.
\end{theorem}

\begin{proof}
Let $\lH$ be a Cartan algebra of the complex semisimple Lie algebra $\lG$ and $ x_{\alpha}$, some root vectors. We consider the space
\begin{equation}
 \lU_0=\underbrace{\sum_{\alpha\in\Phi}\eR ih_{\alpha}}_A+\underbrace{\sum_{\alpha\in\Phi}\eR( x_{\alpha}-\xbma)}_B+\underbrace{\sum_{\alpha\in\Phi}\eR i( x_{\alpha}+\xbma)}_C.
\end{equation}
Since $\lU_0\oplus i\lU_0$ contains all the $\eC h_{\alpha}$, $\lH\subset\lU_0\oplus i\lU_0$; it is also rather clear that $\lU_0$ is a real form of $\lG$ (as vector space), for example, $i\eR( x_{\alpha}-\xbma)+\eR i( x_{\alpha}+\xbma)=\eR i x_{\alpha}$. Now we have to check that $\lU_0$ is a real form of $\lG$ as Lie algebra, i.e. that $\lU_0$ is closed for the Lie bracket. This is a lot of computations:
\[
\begin{split}
[i h_{\alpha},i\hbb]               &=0,\\
[i h_{\alpha},( x_{\alpha}-\xbma)]        &=i(\alpha( h_{\alpha}) x_{\alpha}-(-\alpha)( h_{\alpha})\xbma)\\
                            &=i\alpha( h_{\alpha})( x_{\alpha}+\xbma)\in C,\\
[i h_{\alpha},i( x_{\alpha}+\xbma)]       &=-\alpha( h_{\alpha})( x_{\alpha}-\xbma)\in B,\\
[( x_{\alpha}-\xbma),(\xbb-\xbmb)] &=c(\alpha,\beta)( x_{\alpha+\beta}-x_{-(\alpha+\beta)} )\in B\\
                            &\quad -c(\alpha,\beta)(x_{\alpha-\beta}-x_{\beta-\alpha})\in B,\\
[( x_{\alpha}-\xbma),i(\xbb+\xbmb)]&=ic(\alpha,\beta)(x_{\alpha+\beta}+x_{-(\alpha+\beta)})\in C\\
                            &\quad +ic(\alpha,-\beta)(x_{\alpha-\beta)}+x_{-\alpha+\beta})\in C\\
[i h_{\alpha},(\xbb-\xbmb)]     &=i\beta( h_{\alpha})(\xbb-\xbmb)\in C\\
[i h_{\alpha},i(\xbb+\xbmb)]       &=-\beta( h_{\alpha})(\xbb-\xbmb)\in B\\
[i( x_{\alpha}+\xbma),i(\xbb+\xbmb)]&=-c(\alpha,\beta)(x_{\alpha+\beta}-x_{-(\alpha+\beta)})\\
                             &\quad -c'(\alpha,-\beta)(x_{\alpha-\beta}-x_{-\alpha+\beta}).
\end{split}
\]

From proposition~\ref{prop:compact_Killing}, it just remains to prove that the Killing form of $\lU_0$ is strictly negative definite. We know that $B_{\lG}(\lG_{\alpha},\lG_{\beta})=0$ if $\alpha,\beta\in\Phi$ and $\alpha+\beta\neq 0$; then $A\perp B$ and $A\perp C$. It is a lot of computation to compute the Killing form; we know that $B$ is strictly positive definite on $\sum_{\alpha\in\Delta}\eR h_{\alpha}$ (and then strictly negative definite on $A$) a part this, the non zero elements are (recall that if $\alpha\neq 0$, $B( x_{\alpha}, x_{\alpha})=0$ from corollary~\ref{cor:Bxy_zero})
\[
\begin{split}
  B( ( x_{\alpha}-\xbma),( x_{\alpha}-\xbma) )&=-2B( x_{\alpha},\xbma)=-2\\
  B(i( x_{\alpha}+\xbma),i( x_{\alpha},\xbma))&=-2.
\end{split}
\]

What we have in the matrix of $B_{\lG}|_{\lU_0\times\lU_0}$ is a negative definite block (corresponding to $A$), $-2$ on the rest of the diagonal and zero anywhere else. Then it is well negative definite and $\lU_0$ is a compact real from of $\lG$.
\end{proof}

\subsection{Involutions}
%//////////////////////////

Let $\lG$ be a (real or complex) Lie algebra. An automorphism $\dpt{\sigma}{\lG}{\lG}$ which is not the identity such that $\sigma^2$ is the identity is a \defe{involution}{involutive!automorphism}. An involution $\dpt{\theta}{\lF}{\lF}$ of a \emph{real} semisimple Lie algebra $\lF$ such that the quadratic form $B_{\theta}$ defined by
\[
   B_{\theta}(X,Y):=-B(X,\theta Y)
\]
is positive definite is a \defe{Cartan involution}{Cartan!involution}.

\begin{proposition}
Let $\lG$ be a complex semisimple Lie algebra, $\lU_0$ a compact real form and $\tau$, the conjugation of $\lG$ with respect to $\lU_0$. Then $\tau$ is a Cartan involution of $\lG\heR$.
\label{prop:conj_invol}
\end{proposition}

\begin{proof}
From the assumptions, $\lG=\lU_0\oplus i\lU_0$, $\tau_{\lU_0}=id$ and $\tau_{i\lU_0}=-id$; then it is clear that $\tau_{\lG\heR}^2=id|_{\lG\heR}$. If $Z\in\lG$, we can decompose into $Z=X+iY$ with $X$, $Y\in\lU_0$. For $Z\neq 0$, we have
\begin{equation}
    B_{\lG}(Z,\tau Z)=B_{\lG}(X+iY,X-iY)
                     =B_{\lG}(X,X)+B_{\lG}(Y,Y)
             =B_{\lU_0}(X,X)+B_{\lU_0}(Y,Y)<0
\end{equation}
because $B$ restricts itself to $\lU_0$ which is compact. Then
\begin{equation}
  (B_{\lG\heR})_{\tau}(Z,Z')=B_{\lG\heR}(Z,\tau Z)
                            =-2\real B_{\lG}(Z,\tau Z')
\end{equation}
is positive definite because $(B_{\lG})_{\tau}$ is negative definite. Thus $\tau$ is a Cartan involution of $\lG\heR$.
\end{proof}

\begin{lemma}
If $\varphi$ and $\psi$ are involutions of a vector space $V$ (we denote by $V_{\psi^+}$ and $V_{\psi^-}$ the subspaces of $V$ for the eigenvalues $1$ and $-1$ of $\psi$ and similarly for $\varphi$), then
\[
[\varphi,\psi]=0\quad\text{iff}\quad \left\{   \begin{aligned}
                                                   V_{\varphi^+}&=(V_{\varphi^+}\cap V_{\psi^+})\oplus(V_{\varphi^+}\cap V_{\psi^-})\\
                           V_{\varphi^-}&=(V_{\varphi^-}\cap V_{\psi^+})\oplus(V_{\varphi^-}\cap V_{\psi^-}),
                                           \end{aligned}
                      \right.
\]
i.e. if and only if the decomposition of $V$ with respect to $\varphi$ is ``compatible''{} with the one with respect to $\psi$.
\label{lem:invol_compat}
\end{lemma}

\begin{proof}
\subdem{Direct sense}
Let us first see that $\varphi$ leaves the decomposition $V=V_{\psi^+}\oplus V_{\psi^-}$ invariant. If $x=x_{\psi^+}+x_{\psi^-}$,
\[
   \varphi(x_{\psi^+})=(\varphi\circ\psi)(x_{\psi^+})=(\psi\circ\varphi)(x_{\psi^+}).
\]
Then $\varphi(x_{\psi^+})\in V_{\psi^+}$, and the matrix of $\varphi$ is block-diagonal with respect to the decomposition given by $\psi$. Thus $V_{\psi^+}$ and $V_{\psi^-}$ split separately into two parts with respect to $\varphi$.

\subdem{Inverse sense}
If $x\in V$, we can write $x=x_{++}+x_{+-}+x_{-+}+x_{--}$ where the first index refers to $\psi$ while the second one refers to $\psi$; for example, $x_{+-}\in V_{\psi^+}\cap V_{\varphi^-}$. The following computation is easy:
\begin{equation}
\begin{split}
(\varphi\circ\psi)(x)&=\varphi(x_{++}+x_{+-}-x_{-+}-x_{--})\\
                 &=x_{++}-x_{+-}-x_{-+}+x_{--}\\
         &=\psi(x_{++}-x_{+-}-x_{-+}-x_{--})\\
         &=(\psi\circ\varphi)(x).
\end{split}
\end{equation}
\end{proof}

\begin{corollary}\label{cor:Cartan_conj_inner}
    Any two Cartan involutions of a real semisimple Lie algebra are conjugate by an inner automorphism. \index{inner!automorphism}
\end{corollary}

\begin{proof}
Let $\sigma$ and $\sigma'$ be two Cartan involutions of $\lF$. We can find a $\varphi\in\inf\lF$ such that $[\varphi\sigma\varphi^{-1},\sigma']=0$. Thus it is sufficient to prove that any two Cartan involutions which commute are equals. So let us consider $\theta$ and $\theta'$, two Cartan involutions such that $[\theta,\theta']=0$. By lemma~\ref{lem:invol_compat}, we know that the decompositions into $+1$ and $-1$  eigenspaces with respect to $\theta$ and $\theta'$ are compatibles. If we consider $X\in\lF$ such that $\theta X=X$ and $\theta' X=-1$ (it is always possible if $\theta\neq\theta'$), we have
\[
\begin{split}
  0<B_{\theta}(X,X)=-B(X,\theta X)=-B(X,X)\\
  0<B_{\theta'}(X,X)=-B(X,\theta' X)=B(X,X)
\end{split}
\]
which is impossible.
\end{proof}

\begin{theorem}
Let $\lF$ be a real semisimple Lie algebra, $\theta$ a Cartan involution on $\lF$ and $\sigma$, another involution (not specially Cartan). Then there exists a $\varphi\in\Int\lF$ such that $[\varphi\theta\varphi^{-1},\sigma]=0$
\label{tho:sigma_theta_un}
\end{theorem}

\begin{proof}
If $\theta$ is a Cartan involution, then $B_{\theta}$ is a scalar product on $\lF$. Let $\omega=\sigma\theta$. By using $\sigma^2=\theta^2=1$, $\theta=\theta^{-1}$ and the invariance property~\ref{prop:auto_2} of the Killing form,
\begin{equation}
B(\omega X,\theta Y)=B(X,\omega^{-1}\theta Y)
                    =B(X,\theta\sigma\theta Y)
            =B(X,\theta\omega Y).
\end{equation}
Then $B_{\theta}(\omega X,Y)=B_{\theta}(X,\omega Y)$. This is a general property of scalar product that in this case, the matrix of $\omega$ is symmetric while the one of $\omega^2$ is positive definite. If we consider the classical scalar product whose matrix is $(\delta_{ij})$, the property is written as $A_{ij}v_jw_j=v_iA_{ij}w_j$ (with sum over $i$ and $j$); this implies the symmetry of $A$. To see that $A^2$ is positive definite, we compute (using the symmetry):
\[
   A_{ij}A_{jk}v_iv_k=v_iA_{ij}v_kA_{kj}=\sum_j(v_iA_{ij})^2>0.
\]
The next step is to see that there is an unique linear transformation $\dpt{A}{\lF}{\lF}$ such that $\omega^2=e^A$, and that for any $t\in\eR$, the transformation $e^{tA}$ is an automorphism of $\lF$.

We choose an orthonormal (with respect to the inner product $B_{\theta}$) basis $\{X_1,\ldots,X_n\}$  of $\lF$ in which $\omega$ is diagonal. In this basis, $\omega^2$ is also diagonal and has positive real numbers on the diagonal; then the existence and unicity of $A$ is clear. Now we take some notations:
\begin{subequations}
\begin{align}
  \omega(X_i)&=\lambda_iX_i\\
  \omega^2(X_i)&=e^{a_i}X_i,
\end{align}
\end{subequations}
(no sum at all) where the $a_i$ are the diagonals elements of $A$. The structure constants are as usual defined by
\begin{equation}
   [X_i,X_j]=c_{ij}^kX_k.
\end{equation}
Since $\sigma$ and $\theta$ are automorphisms, $\omega^2$ is also one. Then
\[
\omega^2[X_i,X_j]=c_{ij}^k\omega^2(X_k)=c_{ij}^ke^{a_k}X_k
\]
can also be computed as
\[
   \omega^2[X_i,X_j]=[\omega^2X_i,\omega^2X_j]=e^{a_i}e^{a_j}c_{ij}^kX_k,
\]
so that $c_{ij}^ke^{a_k}=c_{ij}^ke^{a_i}e^{a_j}$, and then $\forall t\in\eR$,
\[
   c_{ij}^ke^{ta_k}=c_{ij}^ke^{ta_i}e^{ta_j},
\]
which proves that $e^{tA}$ is an automorphism of $\lF$. By lemma~\ref{lem:autom_derr}, $A$ is thus a derivation of $\lF$. The semi-simplicity makes $\partial\lF=\ad\lF$, then $A\in\ad\lF$ and $e^{tA}\in\Int\lF$ because it clearly belongs to the identity component of $\Aut\lF$.

Now we can finish de proof by some computations. Remark that $\omega=e^{A/2}$ and $[e^{tA},\omega]=0$ because it can be seen as a common matrix commutator. Since $\omega^{-1}=\theta\sigma$, we have $\theta\omega^{-1}\theta=\sigma\theta$, or $\theta\omega^2\theta=\omega^2$ and
\begin{equation}\label{eq:eAth}
   e^{A}\theta=\theta e^{-A}.
\end{equation}
From this, one can deduce that $e^{tA}\theta=\theta e^{-tA}$. Indeed, as matricial identity, equation \eqref{eq:eAth} reads
\[
    (e^{A}\theta)_{ik}=(e^{A})_{ij}\theta_{jk}
                      =e^{a_i}\theta_{ik}
              =e^{-a_k}\theta_{ik}.
\]
Then for any $ik$ such that $\theta_{ik}\neq 0$, we find $e^{a_i}=e^{-a_k}$ and then also $e^{ta_i}=e^{-ta_k}$. Thus $(e^{tA}\theta)_{ik}=(e^{tA})_ij\theta_{jk}=e^{ta_i}\theta_{ik}=\theta_{ik}e^{-ta_k}=(\theta e^{-tA})_{ik}$. So we have
\begin{equation}
  e^{tA}\theta=\theta e^{-tA}.
\end{equation}
Now we consider $\varphi=e^{A/4}\in\Int\lF$ and $\theta_1=\varphi\theta\varphi^{-1}$. We find $\theta_1\sigma=e^{A/2}\omega^{-1}$ and $\sigma\theta^{-1}=e^{-A/2}\omega$. Since $\omega^2=A$, we have $e^{A/2}=e^{-A/2}\omega^2$ and thus $\theta_1\sigma=\sigma\theta_1$.

\end{proof}

\begin{corollary}
    Every real Lie algebra has a Cartan involution.
\end{corollary}

\begin{proof}
Let $\lF$ be a real Lie algebra and $\lG$ be his complexification: $\lG=\lF\heC$. Let $\lU_0$ be a compact real form of $\lG$ and $\tau$ the induced involution (the conjugation) on $\lG$. By the proposition~\ref{prop:conj_invol}, we know that $\tau$ is  a Cartan involution of $\lG\heR$. We also consider $\sigma$, the involution of $\lG$ with respect to the real form $\lF$. It is in particular an involution on the real Lie algebra $\lF$. Then one can find a $\varphi\in\Int\lG\heR$ such that $[\varphi\tau\varphi^{-1},\sigma]=0$ on $\lG\heR$. Let $\lU_1=\varphi\lU_0$ and $X\in\lU_1$. We can write $X=\varphi Y$ for a certain $Y\in\lU_0$. Then
\[
   \varphi\tau\varphi^{-1} X=\varphi\tau Y=\varphi Y=X,
\]
so that $\varphi\tau\varphi^{-1}=id|_{\lU_1}$. Note that $\lU_1$ is also a real compact form of $\lG$ because the Killing form is not affected by $\varphi$. Let $\tau_1$ be the involution of $\lG$ induced by $\lU_1$. We have
\[
   \tau_1|_{\lU_1}=\varphi\tau\varphi^{-1}_{\lU_1}=\id|_{\lU_1}.
\]
Since $\varphi$ is $\eC$-linear, we have in fact $\tau_1=\varphi\tau\varphi^{-1}$. Now we forget $\lU_0$ and we consider the compact real form $\lU_1$ with his involution $\tau_1$ of $\lG$ which satisfy $[\tau_1,\sigma]=0$ on $\lG\heR$ This relation holds also on $i\lG\heR$, then
\[
   [\tau_1,\sigma]=0
\]
on $\lG=\lF\heC$. Let $X\in\lF$, i.e. $\sigma X=X$; it automatically fulfils
\[
  \sigma\tau_1 X=\tau_1\sigma X=\tau_1 X,
\]
so that $\tau_1$ restrains to an involution on $\lF$ (because $\tau_1\lF\subset\lF$). Let $\theta=\tau_1|_{\lF}$. For $X$, $Y\in\lF$, we have
\begin{equation}
B_{\theta}(X,Y)=-B_{\lF}(X, \theta Y)
             =-B_{\lF}(X,\tau Y)
         =\frac{1}{2}(B_{\lG\heR})_{\tau_1}(X,Y),
\end{equation}
which shows that $\theta$ is a Cartan involution. The half factor on the last line comes from the fact that $\lG\heR=(\lF\heC)\heR=\lF\oplus i\lF$.

\end{proof}

\subsection{Cartan decomposition}
%-------------------------------

Examples of Cartan and Iwasawa decomposition are given in sections~\ref{SecToolSL},~\ref{SubSecIwaSOunn},\ref{subsecIwasawa_un},~\ref{SecSympleGp} and~\ref{SecIwasldeuxC}. An example of how it works to prove isomorphism of Lie algebras is provided in subsection~\ref{sssIsomsoslplussl}.

Let $\lF$ be a real semisimple Lie algebra. A vector space decomposition $\lF=\lK\oplus\lP$ is a \defe{Cartan decomposition}{Cartan!decomposition} if the Killing form is negative definite on $\lK$ and positive definite on $\lP$ and the following commutators hold:
\begin{equation}\label{eq:comm_Cartan}
   [\lK,\lK]\subseteq\lK,\quad[\lK,\lP]\subseteq\lP,\quad[\lP,\lP]\subseteq\lK.
\end{equation}
If $X\in\lK$ and $Y\in\lP$, we have $(\ad X\circ\ad Y)\lK\subseteq\lP$ and $(\ad X\circ\ad Y)\lP\subseteq\lK$, therefore $B_{\lF}(X,Y)=0$.

Let $\dpt{\theta}{\lF}{\lF}$ be a Cartan involution, $\lK$ its $+1$ eigenspace and $\lP$ his $-1$ one. It is easy to see that the relations \eqref{eq:comm_Cartan} are satisfied for the decomposition  $\lF=\lK\oplus\lP$. For example, for $X,X'\in\lK$, using the fact that $\theta$ is an automorphism,
\[
   [X,X']=[\theta X,\theta X']=\theta[X,X'],
\]
which proves that $[\lK,\lK]\subseteq\lK$. Since $\theta$ is a Cartan involution, $B_{\theta}$ is positive definite. For $X\in\lK$,
\[
  B(X,X)=B(X,\theta X)=-B_{\theta}(X,X)
\]
proves that $B$ is negative definite on $\lK$; in the same way we find that $B$ is also positive definite on $\lP$. Then the Cartan involution gives rise to a Cartan decomposition. We are going to prove that any Cartan decomposition defines a Cartan involution.

Let us now do the converse. Let $\lF=\lK\oplus\lP$ be a Cartan decomposition of the real semisimple Lie algebra $\lF$. We define $\theta=\id|_{\lK}\oplus(-\id)|_{\lP}$. If $X,X'\in\lK$, the definition of a Cartan algebra makes $[X,X']\in\lK$ and so
\[
  \theta[X,X']=[X,X']=[\theta X,\theta X'],
\]
and so on, we prove that $\theta$ is an automorphism of $\mF$. It remains to prove that $B_{\theta}$ is positive definite. If $X\in\lK$,
\[
   B_{\theta}(X,X)=-B(X,\theta X)=-B(X,X).
\]
Then $B_{\theta}$ is positive definite on $\lK$ because on this space, $B$ is negative definite by definition of a Cartan involution. The same trick shows that $B_{\theta}$ is also positive definite on $\lP$. We had seen that $\lP$ and $\lK$ where $B_{\theta}$-orthogonal spaces. Thus $B_{\theta}$ is positive definite and $\theta$ is a Cartan involution.

Let $\lF=\lK\oplus\lP$ be a Cartan decomposition. Then it is quite easy to see that $\lK\oplus i\lP$ is a compact real form of $\lG=(\lFeC)$.

\begin{proposition}
Let $\lL$ and $\lQ$ be the $+1$ and $-1$ eigenspaces of an involution $\sigma$. Then $\sigma$ is a Cartan involution if and only if $\lL\oplus i\lQ$ is  a compact real form of $\lFeC$.
\end{proposition}

\begin{proof}
First remark that $\lL\oplus i\lQ$ is always a real form of $\lFeC$. The direct sense is yet done. Then we suppose that $B_{\lFeC}$ is negative definite on $\lL\oplus i\lQ$ and we have to show that $\lL\oplus\lQ$ is a Cartan decomposition of $\lF$. The condition about the brackets on $\lL$ and $\lQ$ is clear from their definitions. If $X\in\lL$, $B(X,X)<0$ because $B$ is negative definite on $\lL$. If $Y\in\lQ$, $B(Y,Y)=-B(iY,iY)>0$ because $B$ is negative definite on $i\lQ$.
\end{proof}

\section{Root spaces in the real case}
%----------------------------------------

Let $\lF$ be a real semisimple Lie algebra with a Cartan involution $\theta$ and the corresponding Cartan decomposition $\lF=\lK\oplus\lP$. We consider $B$, a ``Killing like''{} form, i.e. $B$ is a symmetric nondegenerate invariant bilinear form on $\lF$ such that $B(X,Y)=B(\theta X,\theta Y)$ and $B_{\theta}:=-B(X,\theta X)$ is positive definite. Then $B$ is negative definite on the compact real form $\lK\oplus i\lP$. Indeed if $Y\in\lP$,
\begin{equation}
  B(iY,iY)=-B(\theta Y,\theta Y)
          =B(Y,\theta Y)
      =-B_{\theta}(Y,Y)<0.
\end{equation}
The case with $X\in\lK$ is similar. It is easy to see that $B_{\theta}$ is in fact a scalar product on $\lF$, so that we can define the orthogonality and the adjoint from $B_{\theta}$. If $\dpt{A}{\lF}{\lF}$ is an operator on $\lF$, his adjoint is the operator $A^*$ given by the formula
\[
   B_{\theta}(A X,Y)=B_{\theta}(X,A^*Y)
\]
for all $X$, $Y\in\lF$.

\begin{proposition}
With this definition, when $X\in\lF$, the adjoint operator of $\ad X$ is given by means of the Cartan involution:
\[
(\ad X)^*= \ad(\theta X).
\]
\end{proposition}

\begin{proof}
This is a simple computation
\begin{equation}
B_{\theta}\big(  (\ad\theta X)Y,Z \big)=-B\big(  Y,[\theta X,\theta Y]  \big)
                                     =-B_{\theta}(Y,[X,Z])
                     =-B_{\theta}\big( (\ad X)^*Y,Z \big).
\end{equation}
\end{proof}

Let $\lA$ be a maximal abelian subalgebra of $\lP$ (the existence comes from the finiteness of the dimensions). If $H\in\lA$, the operator $\ad H$ is self-adjoint because
\begin{equation}
(\ad H)^*X=(-\ad\theta H)X
          =[H,X]
      =(\ad H)X,
\end{equation}
where we used the fact that $H\in\lP$.  For $\lambda\in\lA^*$, we define the space
\begin{equation}
  \lF_{\lambda}=\{ X\in\lF\tq\forall H\in\lA,\, (\ad H)X=\lambda(H)X\}.
\end{equation}
If $\lF_{\lambda}\neq 0$ and $\lambda\neq 0$, we say that $\lambda$ is a \defe{restricted root}{restricted root (real case)}\index{root!restricted (real case)} of $\lF$. We denote by $\Sigma$ the set of restricted roots of $\lF$. We may sometimes write $\Sigma_{\lF}$ if the Lie algebra is ambiguous.

The main properties of the real root spaces are given in the following proposition.

\begin{proposition}     \label{PropPropRacincesReelles}
The set $\Sigma$ of the restricted roots of a real semisimple Lie algebra $\lF$ has the following properties:
\label{prop:enuc}
\begin{enumerate}
\item\label{enuci} $\lF=\lF_0\bigoplus_{\lambda\in\Sigma}\lF_{\lambda}$,
\item\label{enucii} $[\lF_{\lambda},\lF_{\mu}]\subseteq\lF_{\lambda+\mu}$,
\item\label{enuciii} $\theta\lF_{\lambda}=\lF_{-\lambda}$,
\item\label{enuciv} $\lambda\in\Sigma$ implies $-\lambda\in\Sigma$,
\item\label{enucv} $\lF_0=\lA\oplus\lM$ where $\lM=\mZ_{\lK}(\lA)$ and $\lA\perp\lM$.
\end{enumerate}
\end{proposition}

\begin{proof}
Proof of~\ref{enuci}. The operators $\ad H$ with $H\in\lA$ form an abelian algebra of self-adjoint operators, then they are simultaneously diagonalisable. Let $\{X_i\}$ be a basis which realize this diagonalisation, and $\lF_i=\Span X_i$, so that $\lF=\oplus_i\lF_i$. We have $(\ad H)\lF_i=\lF_i$ and then $(\ad H)X_i=\lambda_i(H)X_i$ for a certain $\lambda_i\in\lA^*$. This shows that $\lF_i\subseteq\lF_{\lambda_i}$.\quext{pourquoi ça n'implique pas que $\dim\lF_{\lambda_i}=1$? Réponse par Philippe: tu as oublié les valeurs propres nulles  dans ta base ce qui doit entrainer quelques modifs dans ton texte(par  ex.  $adH f_i = f_i$ pas toujours ) }

Proof of~\ref{enucii}. Let $H\in\lA$, $X\in\lF_{\lambda}$ and $Y\in\lF_{\mu}$. We have
\begin{equation}
   (\ad H)[X,Y]=[[H,X],Y]+[X,[H,Y]]
               =\big(  \lambda(H)+\mu(H) \big) [X,Y].
\end{equation}

Proof of~\ref{enuciii}. Using the fact that $\theta H=-H$ because $H\in\lP$,
\begin{equation}
  (\ad H)\theta X=\theta[\theta H,X]
                 =-\theta\lambda(H)X
         =-\lambda(H)(\theta X).
\end{equation}

Proof of~\ref{enuciv}. It is a consequence of~\ref{enuciii} because if $\lF_{\lambda}\neq 0$, then $\theta\lF_{_{\lambda}}\neq 0$.

Proof of~\ref{enucv}. By~\ref{enuciii}, $\theta\lF_0=\lF_0$, then $\lF_0=(\lK\cap\lF_0)\oplus(\lP\cap\lF_0)$. If $X\in\lF_0$, then it commutes with all the elements of $\lA$ and by the maximality property of $\lA$, provided that $X\in\lP$, it also must belongs to $\lA$. This fact makes $\lA=\lP\cap\lF_0$. Now,
\[
  \lM=\mZ_{\lK}(\lA)=\{X\in\lK\tq [X,\lA]=0\}=\lK\cap\lF_0.
\]
All this gives $\lF_0=\mZ_{\lK}(\lA)\oplus\lA$.
\end{proof}

We choose a positivity notion on $\lA^*$, we consider $\Sigma^+$, the set of restricted positive roots and we define\nomenclature{$\lN$}{Restricted roots}
\[
  \lN=\bigoplus_{\lambda\in\Sigma^+}\lF_{\lambda}.
\]

From finiteness of the dimension, there are only a finitely many forms $\lambda\in\lA^*$ such that $\lF_{\lambda}\neq 0$. Then, taking, more and more commutators in $\lN$, the formula $[\lF_{\lambda},\lF_{\mu}]\subseteq\lF_{\lambda+\mu}$ shows that the result finish to fall into a $\lF_{\mu}=0$. On the other hand, since $\lA\subset\lF_0$, we have $[\lA,\lN]=\lN$. If $a_1,a_2\in\lA$ and $n_1,n_2\in\lN$,
\begin{equation}
   [a_1+n_1,a_2+n_2]=\underbrace{[a_1,a_2]}_{=0}+\underbrace{[a_1,n_2]}_{\in\lN}
                      \quad+\underbrace{[n_1,a_2]}_{\in\lN}+\underbrace{[n_1,n_2]}_{\in\lN},
\end{equation}
then $[\lA\oplus\lN,\lA\oplus\lN]=\lN$. This proves the three following important properties:

\begin{enumerate}
\item $\lN$ is nilpotent.
\item $\lA$ is abelian.
\item $\lA\oplus\lN$ is a solvable Lie subalgebra of $\lF$.
\end{enumerate}

\subsection{Iwasawa decomposition}
%----------------------------------

\begin{theorem}
Let $\lF$ be a real semisimple Lie algebra and $\lK$, $\lA$, $\lN$ as before. Then we have the following direct sum:
\begin{equation}
   \lF=\lK\oplus\lA\oplus\lN.
\end{equation}
\end{theorem}

This is the \defe{Iwasawa decomposition}{Iwasawa!decomposition}\index{decomposition!Iwasawa} for the real semisimple Lie algebra $\lF$.

\begin{proof}
We yet know the direct sum $\lF=\lF_0\bigoplus_{\lambda\in\Sigma}\lF_{\lambda}$. Roughly speaking, in $\lN$ we have only vectors of $\Sigma^+$, in $\theta\lN$, only of $\Sigma^-$ and in $\lA$, only in ``zero''. Then the sum $\lA\oplus\lN\oplus\theta\lN$ is direct.

Now we prove that the sum $\lK+\lA+\lN$ is also direct. It is clear that $\lA\cap\lN=0$ because $\lA\subseteq\lF_0$. Let $X\in\lK\cap(\lA\oplus\lN)$. Then $\theta X=X$. But $\theta X\in\lA\oplus\theta\lN$. Thus $X\in\lA\oplus\lN\cap\lA\oplus\lN$ which implies $X\in\lA$. All this makes $X\in\lP\oplus\lK$ and $X=0$.

Now we prove that $\lK\oplus\lA\oplus\lN=\lF$. An arbitrary $X\in\lF$ can be written as
\[
   X=H+X_0+\sum_{\lambda\in\Sigma}X_{\lambda}
\]
where $H\in\lA$, $X_0\in\lM$ and $X_{\lambda}\in\lF_{\lambda}$. Now there are just some manipulations\ldots
\begin{equation}
  \sum_{\lambda\in\Sigma}X_{\lambda}=\sum_{\lambda\in\Sigma^+}(X_{-\lambda}+X_{\lambda})
                                  =\sum_{\lambda\in\Sigma^+}(X_{-\lambda}+\theta X_{-\lambda})
                  +\sum_{\lambda\in\Sigma^+}(X_{\lambda}+\theta X_{-\lambda}),
\end{equation}
but $\theta(X_{-\lambda}+\theta X_{-\lambda})=X_{-\lambda}+\theta X_{-\lambda}$, then $X_{-\lambda}+X_{-\lambda}\in\lK$. Moreover, $X_{\lambda}, \theta X_{-\lambda}\in\lF_{\lambda}$, then $X_{\lambda}-\theta X_{-\lambda}\in\lF_{\lambda}\subseteq\lN$. Then
\begin{equation}
  X=X_0+\sum_{\lambda\in\Sigma^+}(X_{-\lambda}+\theta X_{-\lambda})+H+\sum_{\lambda\in\Sigma^+}(X_{\lambda}-\theta X_{-\lambda})
\end{equation}
where the two first term belong to $\lK$, $H\in\lA$ and the last term belongs to $\lN$.
\end{proof}

\begin{lemma}
There exists a basis $\{X_i\}$ of $\lF$ in which

\begin{enumerate}
\item\label{enudi} The matrices of $\ad\lK$ are symmetric,
\item\label{enudii} The matrices of $\ad\lA$ are diagonal and real,
\item\label{enudiii} The matrices of $\ad\lN$ are upper triangular with zeros on the diagonal.
\end{enumerate}
\end{lemma}

\begin{proof}
We have the orthogonal decomposition $\lF=\lF_0\bigoplus_{\lambda\in\Sigma}\lF_{\lambda}$ given by proposition~\ref{prop:enuc}. Let $\{X_i\}$ be an orthogonal basis of $\lF$ compatible with this decomposition and in such an order that $i<j$ implies $\lambda_i\geq\lambda_j$. From the orthogonality of the basis it follows that the matrix of $B_{\theta}$ is diagonal. Thus the adjoint is the transposition.

\ref{enudi} If $X\in\lK$, $(\ad X)^t=(\ad X)^*=-\ad\theta X=-\ad X$.

\ref{enudii} Each $X_i$ is a restricted root; then $(\ad H)X_i=\lambda_i(H)X_i$, then the diagonal of $\ad H$ is made of $\lambda_i(H)$ whose are real.

\ref{enudiii} If $Y_i\in\lF_{\lambda_i}$ with $\lambda_i\in\Sigma^+$, $(\ad Y_i)X_j$ has only components in $\lF_{\lambda_i+\lambda_j}$ with $\lambda_i+\lambda_j>\lambda_j$ because $\lambda_i\in\Sigma^+$.
\end{proof}


\begin{lemma}
Let $\lH$ be a subalgebra of the real semisimple Lie algebra $\lF$. Then $\lH$ is a Cartan subalgebra if and only if $\lHeC$ is Cartan in $\lFeC$.
\end{lemma}

\begin{proof}
\subdem{Direct sense} If $\lH$ is nilpotent in $\lF$, it is cleat that $\lHeC$ is nilpotent in $\lFeC$. We have to prove that $[x,\lHeC]\subseteq\lHeC$ implies $x\in\lHeC$. As set, $\lFeC=\mF\oplus i\lF$  (but not as vector space!), then we can write $x=a+ib$ with $a$, $b\in\lF$. The assumption makes that for any $h\in\lH$, there exists $h',h''\in\lH$ such that
\[
   [a+ib,h]=h+ih''.
\]
This equation can be decomposed in $\lF$-part and $i\lF$-part: for any $h\in\lH$, there exists a $h'\in\lH$ such that $[a,h]=h'$,  and for any $h\in\lH$, there exists a $h''\in\lH$ such that $[b,h]=h''$. Thus $a$, $b\in\lH$ because $\lH$ is Cartan in $\lF$.

\subdem{Inverse sense} The assumption is that $[x,\lHeC]\subset\lHeC$ implies $x\in\lHeC$. In particular consider a $x\in\lH$ such that $[x,\lH]\subset\lH$. Then $x\in\lHeC$ because $[x,\lHeC]\subset\lHeC$. But $\lHeC\cap\lF=\lH$.
\end{proof}

In the complex case, the Cartan subalgebras all have same dimensions because they are maximal abelian.

\section{The group \texorpdfstring{$SL(2,\eR)$}{SL2R} and its algebra}  \label{SecToolSL}
%++++++++++++++++++++++++++++++++++++++++++++++++++++++++++++++++++++++

The study of \( \SL(2,\eR)\) and \( \gsl(2,\eC)\) is required before to go further in the general study because of proposition~\ref{PropScalrooTsQ} that will reduce the study of genera Lie algebras into combinations of \( \gsl(2,\eC)\) algebras.

\subsection{Iwasawa decomposition}
%----------------------------------
\index{Iwasawa!decomposition!of $SL(2,\eR)$}

Let $G=\SL(2,\eR)$ the group of $2\times 2$ matrices with unit determinant. The Lie algebra $\lG=\gsl(2,\eR)$ is the algebra of matrices with vanishing trace:
\begin{equation}
 \lG =  \{ X\in\End(\eR^2)\tq \tr(X) = 0\}
=\left\{ \begin{pmatrix}
x & y \\
z & -x
\end{pmatrix}\textrm{ with }x,y,z\in\eR  \right\}.
\end{equation}
The following elements will be intensively used:
\begin{equation}    \label{EqsXPdnZlG}
H=\begin{pmatrix}
1 & 0 \\
0 & -1
\end{pmatrix}
,\quad
  E=\begin{pmatrix}
0 & 1 \\
0 & 0
\end{pmatrix}
,\quad
 F=\begin{pmatrix}
0 & 0 \\
1 & 0
\end{pmatrix},
\quad
T=\begin{pmatrix}
0&1\\
-1&0
\end{pmatrix}
\end{equation}
where $T=E-F$ has been introduced for later convenience. The commutators are
\begin{subequations}\label{EqTableSLdR}
\begin{align}
  [H,E]&=2E &[T,H]&=-2T  \\
  [H,F]&=-2F    &[T,E]&=H   \\
  [E,F]&=H  &[T,F]&=H.
\end{align}
\end{subequations}
The exponentials can be easily computed and the result is
\begin{align}               \label{EqExpMatrsSLdeuxR}
 e^{tH}=
\begin{pmatrix}
   e^{t}    &   0   \\
  0 &    e^{-t}
\end{pmatrix},
&&
 e^{tE}=
\begin{pmatrix}
  1 &   t   \\
  0 &   1
\end{pmatrix},
&&
 e^{tF}=
\begin{pmatrix}
  1 &   0   \\
  t &   1
\end{pmatrix}.
\end{align}
Notice that the sets $\{ H,E,F \}$, $\{ H,E,F \}$ and $\{ H,E+F,T \}$ are basis. A Cartan involution is given by $\theta(X)=-X^t$, and the corresponding Cartan decomposition is
\begin{align}
   \lK&=\Span\{ T \},
&\lP&=\Span\{ H,E+F \}.
\end{align}
Indeed, we are in a matrix algebra, then $\tr(XY)$ is proportional to $\tr(\ad X\circ \ad Y)$.
 In order to see that $\theta$ is a Cartan involution, we have to prove that $B|_{\lK\times\lK}$ is negative definite and $B|_{\lP\times\lP}$ positive. It is true because for $X\in\lK$,
\[
    \tr(\ad X\circ \ad X)=\tr(XX)=\tr\begin{pmatrix}
-x^2 & 0 \\
0 & -x^2
\end{pmatrix}<0,
\]
and for $Y\in\lP$,
\[
    \tr(YY)=\tr\begin{pmatrix}
x & y \\
y & -x
\end{pmatrix}\begin{pmatrix}
x & y \\
y & -x
\end{pmatrix} =\tr\begin{pmatrix}
x^2+y^2 & 0 \\
0 & x^2+y^2
\end{pmatrix} >0.
\]

Up to some choices, the Iwasawa decomposition\label{pg_iwasldr} of the group $\SL(2,\eR)$ is given by the exponentiation of $\lA$, $\lN$ and~$\lK$
\begin{equation}
\begin{aligned}
  \lA&=\Span\{ H \}
&\lN&=\Span\{ E \}
&\lK&=\Span\{T\},
\end{aligned}
\end{equation}
so that
\begin{equation}\label{eq:expo_ANK}
A=\begin{pmatrix}
e^a & 0 \\
0 & e^{-a}
\end{pmatrix}\quad
N=\begin{pmatrix}
1 & l \\
0 & 1
\end{pmatrix}\quad
K=\begin{pmatrix}
\cos k & \sin k \\
-\sin k & \cos k
\end{pmatrix}.
\end{equation}

A common parametrization of $AN$ by $\eR^2$ is provided by
\begin{equation}   \label{EqParmalSL}
(a,l)=
\begin{pmatrix}
  e^a&le^a\\
  0  &e^{-a}
\end{pmatrix}.
\end{equation}
One immediately has the following formula for the left action of $AN$ on itself:
\[
  L_{(a,l)}(a',l')=\begin{pmatrix}
e^{a+a'} & e^{a+a'}l'+e^{a-a'}l \\
0 & e^{-a-a'}
\end{pmatrix}=(a+a',l'+e^{-2a'}l).
\]
In this setting, the inverse is given by $(a,l)^{-1}=(-a,-l e^{2a})$.

Let's give some formulas in $\SL(2,\eR)$. Using corollary~\ref{Ad_e} and exponentiating commutation relations,
\begin{subequations}  \label{eq_eaHsldr}
\begin{align}
\Ad(e^{aH})E&=e^{2a}E,\\
\Ad(e^{aH})F&=e^{-2a}F,\\
\Ad( e^{aH})T&= e^{2a}E- e^{-2u}E+ e^{-2u}T\\
\Ad(e^{tE})H&=H-2tE,                            \label{eq_AdetE}\\
\Ad( e^{tE})T&=-tH+(t^2+1)E-E+T\\
\Ad( e^{tT})H&=\cos(2t)H+\sin(2t)(2E-T)\\
\Ad( e^{tT})E&=\frac{ 1 }{2}\Big( \sin(2t)H+\cos(2t)(2E-T)+T \Big)
\end{align}
\end{subequations}
where $z$ belongs to the center: $z=\pm\mtu$.

%---------------------------------------------------------------------------------------------------------------------------
                    \subsection{A companion: \texorpdfstring{$A\bar N$}{AN}}
%---------------------------------------------------------------------------------------------------------------------------

We can consider the Iwasawa decomposition which is $\theta$-conjugated to the $AN$ that we just saw. That decomposition is generated by
\begin{equation}
    \bar\lN=\begin{pmatrix}
    0   &   0   \\
    1   &   0
\end{pmatrix}.
\end{equation}
The exponentiation produces
\begin{equation}
    \bar N=\begin{pmatrix}
    1   &   0   \\
    t   &   1
\end{pmatrix},
\end{equation}
and the Iwasawa group is given by
\begin{equation}        \label{EqGeneANbarSLdeuxR}
    A\bar N=\begin{pmatrix}
    e^a &   0   \\
    l e^{-a}    &    e^{-a}
\end{pmatrix}.
\end{equation}

\subsection{Killing form}
%------------------------

In the basis $\{ H,E,T \}$, the adjoint operators are given by
\[
\ad H=\begin{pmatrix}
 0 & 0 &-2 \\
 0 & 0 &0 \\
 0 & 2 &2
\end{pmatrix},
\ad E=\begin{pmatrix}
 0 & 0 &0 \\
 0 & 0 &-1\\
-2 & 0&0
\end{pmatrix},\textrm{ and }
  \ad T=\begin{pmatrix}
 2 & 0 &0 \\
0 & 1 &0\\
-2 & 0&0
\end{pmatrix}.
\]
so that the Killing form can be computed directly from definition $B(X,Y)=\tr(\ad X,\ad Y)$. The result is
\begin{subequations}
\begin{align}
B(T,H)&=0  & B(H,H)&=8\\
B(T,E)&=-4 & B(E,E)&=0\\
B(H,E)&=0  &  B(T,T)&=-4.
\end{align}
\end{subequations}
Expressed in the basis $\{H,E,F\}$, the matrix of the Killing form reads
\begin{equation}
B=
\begin{pmatrix}
8&&\\
&&4\\
&4&
\end{pmatrix}
\end{equation}
while, in the basis  $\{H,E+F,T\}$, we find
\begin{equation}   \label{EqBHEFTsldR}
B=
\begin{pmatrix}
8\\
&8\\
&&-8
\end{pmatrix}.
\end{equation}
The latter is the reason of the name of the vector $T$: the sign of its norm is different, so that $T$ is candidate to be a time-like direction.

\subsection{Abstract root space setting}
%---------------------------------------

Looking on the table \eqref{EqTableSLdR} from an abstract point of view, we see that $E$ and $F$ are eigenvectors of $\ad(H)$ with eigenvalues $2$ and $-2$. So $\lA=\lG_0=\eR H$; $\lG_2=\eR E$; and $\lG_{-2}=\eR F$. Using a more abstract notation, the table of $\SL(2,\eR)$ becomes
\begin{subequations}  \label{subeq_rootSLR}
\begin{align}
  [A_{0},A_{2}]&=2A_{2}\\
    [A_{0},A_{-2}]&=-2A_{-2}\\
    [A_{2},A_{-2}]&=A_{0}.
\end{align}
\end{subequations}

\subsection{Isomorphism}
%-----------------------

As pointed out in the chapter II, \S6 of \cite{Knapp_reprez}, the map (seen as a conjugation in $\SL(2,\eC)$)
\begin{equation}
    \begin{aligned}
        \psi\colon \SU(1,1)&\to \SL(2,\eR) \\
        U&\mapsto AUA^{-1}
    \end{aligned}
\end{equation}
with $A=\begin{pmatrix}
1&i\\i&1
\end{pmatrix}$ is an isomorphism between $\SL(2,\eR)$ and $\SU(1,1)$.

%---------------------------------------------------------------------------------------------------------------------------
                    \section{The complex algebra \texorpdfstring{$\protect\gsl(2,\eC)$}{sl2C} and its representations}
%---------------------------------------------------------------------------------------------------------------------------
\label{SecsldeuxCandrepres}

The book \cite{Kassel} contains the representations of \( \gsl(2,\eC)\).

The algebra $\gsl(2,\eC)$ is the complex algebra of complex $2\times 2$ matrices with vanishing trace. As generating matrices, one can take the elements $u_i$ of \eqref{EqGenssudeux} and complete them by
\begin{align*}
v_1&=\frac{ 1 }{2}
\begin{pmatrix}
  -1    &   0   \\
  0 &   1
\end{pmatrix},
&v_2&=\frac{ 1 }{2}
\begin{pmatrix}
  0 &   i   \\
  -i    &   0
\end{pmatrix},
&v_3&=\frac{ 1 }{2}
\begin{pmatrix}
  0 &   -1  \\
  -1    &   0
\end{pmatrix}
\end{align*}
which satisfy the commutation relations
\begin{subequations}
\begin{align}
    [v_i,v_j]&=-\epsilon_{ikj}u_k\\
    [v_i,u_j]&=\epsilon_{ikj}v_k.
\end{align}
\end{subequations}

\begin{remark}
    This is not the algebra \( \gsl(2,\eC)\) used in physics. The latter is the \emph{four}-dimensional \emph{real} algebra of trace vanishing \( 2\times 2\) complex matrices. There is one more generator and the representation theory is different. Moreover the physics works with the \emph{group} instead of the \emph{algebra}.
\end{remark}

The change of basis
\begin{align}
    x_j&=\frac{ 1 }{2}(u_j+iv_j),   &y_j&=\frac{ 1 }{2}(u_j-iv_j)
\end{align}
provides the simplification
\begin{align}
[x_i,x_j]&=\epsilon_{ijk}x_k    &[y_i,y_j]&=\epsilon_{ijk}y_k   &[x_i,y_j]&=0,
\end{align}
so that, as algebras, we have the isomorphism
\begin{equation}
    \gsl(2,\eC)=\gsu(2)\oplus\gsu(2).
\end{equation}
Thus the representation theory of $\gsl(2,\eC)$ is determined by the one of $\gsu(2)$.


\index{representation!of $\gsl(2,\eC))$}
Consider the space $\mP_m$ of homogeneous polynomials of degree $m$ in two variables with complex coefficients[\cite{GpAlgLie_Faraut}]. The dimension of $\mP_m$ is $m+1$ and we have the following representation of $\SL(2,\eC)$ thereon:
\begin{equation}
    \big( \pi_m(g)f \big)(u,v)=f\big(
g
\begin{pmatrix}
u\\v
\end{pmatrix}
 \big)
=
f(au+bv,cu+dv)
\end{equation}
if $g=\begin{pmatrix}
  a &   b   \\
  c &   d
\end{pmatrix}$. We are going to determine the corresponding representation $\rho_m$ of the Lie algebra $\gsl(2,\eC)$ as algebra over complex numbers.

A basis of $\gsl(2,\eC)$ over $\eC$ is given by the matrices $\{ H,E,F \}$ given in equation \eqref{EqsXPdnZlG} and are subject to the commutation relations \eqref{subEqsSBhuAWx}.




Using the exponentiation \eqref{EqExpMatrsSLdeuxR}, we find
\[
    \big( \pi_m( e^{tH})f \big)(u,v)=f( e^{t}u, e^{-t}v),
\]
so that
\begin{equation}
    \big( \rho_m(H)f \big)(u,v)=u\frac{ \partial f }{ \partial u }-v\frac{ \partial f }{ \partial v }.
\end{equation}
In the same way, we find
\begin{equation}
    \big( \rho_m(E)f \big)(u,v)=\Dsdd{ \big( \pi_m( e^{tE})f \big)(u,v) }{t}{0}=v\frac{ \partial f }{ \partial u },
\end{equation}
and
\begin{equation}
     \big( \rho_m(F)f \big)(u,v)=u \frac{ \partial f }{ \partial v }.
\end{equation}
A natural basis of $\mP_m$ is given by the monomials $f_j(u,v)=u^jv^{m-j}$ with $j=0,\ldots,m$. The representation $\rho_m$ on this basis reads
\begin{equation}        \label{EqReprezgsldeuxC}
\begin{split}
    \rho_m(H)f_j&=(2j-m)f_j\\
    \rho_m(E)f_j&=(m-j)f_{j+1}\\
    \rho_m(F)f_j&=jf_{j-1}.
\end{split}
\end{equation}

\begin{proposition}     \label{ProprhomirredsldeuxC}
The representation $\rho_m$ is irreducible.
\end{proposition}

\begin{proof}
Let $W\neq\{ 0 \}$ be an invariant subspace of $\mP_m$. If $p\in W$, from invariance, $\rho_m(H)(p)\in W$. If $p$ is a linear combination of $\{ f_j \}_{j\in I}$ ($I\subseteq \{ 0,\ldots m \}$), then $\rho_m(H)p$ is still a linear combination $q$ of elements in the same set. Thus there exists a linear combination of $p$ and $\rho_m(H)p$ which is a linear combination of $\{ f_j \}_{j\in J}$ with $J\subset I$ (strict inclusion). Using the same trick with $q$ and $\rho_m(H)q$, we still reduce the number of basis elements. Proceeding in the same way at most $m$ times, we find that one of the $f_j$ belongs to $W$. From there, acting with $\rho_m(E)$ and $\rho_m(F)$, one generates the whole $\mP_m$. That proves that $W=\mP_m$ and thus that $\rho_m$ is irreducible.
\end{proof}

\begin{theorem}[\cite{GpAlgLie_Faraut}]
Every $\eC$-linear irreducible finite dimensional representation of $\gsl(2,\eC)$ is equivalent to one of the $\rho_m$.
\end{theorem}

These results will be proved also in the quantum case in theorems~\ref{ThoVfintemofdsldcun} and~\ref{ThoVfintemofdslddeux}.

%+++++++++++++++++++++++++++++++++++++++++++++++++++++++++++++++++++++++++++++++++++++++++++++++++++++++++++++++++++++++++++
                    \section{The group \texorpdfstring{$\SO(3)$}{SO3} and its Lie algebra}
%++++++++++++++++++++++++++++++++++++++++++++++++++++++++++++++++++++++++++++++++++++++++++++++++++++++++++++++++++++++++++
\label{SubSecTheGroupSotrois}

\begin{proposition}[\cite{WormerAngular}]
An element of $\SO(3)$ has exactly one eigenvector with eigenvalue $1$. That vector is the \defe{rotation axis}{axis!of rotation in $\SO(3)$}.
\end{proposition}

The generator of rotation around the axis $n$ (unit vector) is given by the matrix
\begin{equation}
\begin{pmatrix}
  0 &   -n_3    &   n_2\\
  n_3   &   0   &   -n_1\\
 -n_2   &   n_1 & 0
\end{pmatrix}.
\end{equation}
That form results form the requirement that $Nr=n\times r$. If we denote by $R(n,\theta)$ the operator of rotation in $\eR^3$ by an angle $\theta$ around the axis $n$, one shows that
\begin{equation}
    R(b,\theta)=\mtu+\sin(\theta) N+\big(1-\cos(\theta)\big)N^2.
\end{equation}

%///////////////////////////////////////////////////////////////////////////////////////////////////////////////////////////
                    \subsection{Rotations of functions}
%///////////////////////////////////////////////////////////////////////////////////////////////////////////////////////////

Consider any function $f\colon \eR^3\to \eC$; we define the \defe{rotation operator}{rotation!on functions} $U(n,\theta)$\nomenclature{$U(n,\theta)$}{Rotation operator on functions} by
\begin{equation}        \label{EqRotFunSOtrois}
    \big( U(n\theta)f \big)(r)=f\big( R(n,\theta)^{-1}r \big).
\end{equation}
These operators form a group, and we have in particular that
\[
    U(n,\theta_1)U(n,\theta_2)=U(n,\theta_1+\theta_2).
\]
We are interested in \emph{infinitesimal} rotations, that is rotations of angle $d\theta$ for which $(d\theta)^2\ll d\theta$, or in other words, we are interested in a development of equation \eqref{EqRotFunSOtrois} restricted to linear terms in $\theta$. What one obtains is
\begin{equation}
    \big( U(n,d\theta)f \big)(r)=\big( (1-i d\theta\, n\cdot l)f\big)(r)
\end{equation}
where the operator $l$ is defined by
\begin{equation}
    l=-ir\times\nabla.
\end{equation}
Its components $l_i=-i\epsilon_{ijk}r_j\partial_k $ satisfy commutation relations
\begin{equation}    \label{EqAldllepsl}
    [l_i,l_j]=i\epsilon_{ijk}l_k.
\end{equation}
The operator $n\cdot l$ is refereed as the \defe{generator of infinitesimal rotations}{generator!of infinitesimal rotations}. One can derive an expression of $U(n,\theta)$ in terms of $n\cdot l$ by the following:
\[
    U(n,\theta+d\theta)f=U(n,\theta)U(n,d\theta)f=U(n,\theta)(1-id\theta\, n\cdot l)f,
\]
so that we have the differential equation
\begin{equation}
    \frac{ dU }{ d\theta }(n,\theta)=-iU(n,\theta)n\cdot l
\end{equation}
with the initial condition $U(n,0)=1$. The solution is
\begin{equation}
    U(n,\theta)= e^{-i\theta\, n\cdot l}.
\end{equation}

%+++++++++++++++++++++++++++++++++++++++++++++++++++++++++++++++++++++++++++++++++++++++++++++++++++++++++++++++++++++++++++
\section{Verma module}
%+++++++++++++++++++++++++++++++++++++++++++++++++++++++++++++++++++++++++++++++++++++++++++++++++++++++++++++++++++++++++++

When $\lG$ is a semisimple Lie algebra, we have the usual decomposition\cite{VermaPiercey}
\begin{equation}
    \lG=\lN^-\oplus\lH\oplus\lN^+,
\end{equation}
where each of the three components are Lie algebras. In particular, the universal enveloping algebra $\mU(\lN^-)$ makes sense. Let $\mu\in\lH^*$. We build a representation $\pi_{\mu}$ of $\lG$ on $V_{\mu}=\mU(\lN^-)$ in the following way
\begin{itemize}
\item If $Y_{\alpha}\in\lN^-$, we define
\begin{subequations}
    \begin{align}
        \pi_{\mu}(Y_{\alpha})1  &=Y_{\alpha}\\
        \pi_{\mu}(Y_{\alpha_1}\ldots Y_{\alpha_n})&=Y_{\alpha}Y_{\alpha_1}\ldots Y_{\alpha_n},
    \end{align}
\end{subequations}
\item if $H\in\lH$, we define
\begin{subequations}
    \begin{align}
        \pi_{\mu}(H)1   &=\mu(H)\\
        \pi_{\mu}(Y_{\alpha_1}\ldots Y_{\alpha_k})  &= \big( \mu(H)-\sum_{j=1}^k\alpha_j(H) \big)Y_{\alpha_1}\ldots Y_{\alpha_k},
    \end{align}
\end{subequations}
\item and if $X_{\alpha}\in\lN^+$, we define
\begin{subequations}
    \begin{align}
        \pi_{\mu}(X_{\alpha})1  &=0\\
        \pi_{\mu}(X_{\alpha})Y_{\alpha_1}\ldots Y_{\alpha_k}    &=Y_{\alpha_1}\big( \pi_{\mu}(X_{\alpha})Y_{\alpha_2}\ldots Y_{\alpha_k} \big)\\
                                    &\quad  -\delta_{\alpha,\alpha_1}\sum_{j=1}^k\alpha_j(H_{\alpha})Y_{\alpha_1}\ldots Y_{\alpha_k}.
    \end{align}
\end{subequations}
\end{itemize}
In the last one, we do an inductive definition.
\begin{lemma}
The couple $(\pi_{\mu},V_{\mu})$ is a representation of $\lG$ on $V_{\mu}$.
\end{lemma}
\begin{proof}
    No proof.
\end{proof}
That representation is one \defe{Verma module}{Verma module} for $\lG$. If the algebra $\lG$ is an algebra over the field $\eK$, the field $\eK$ itself is part of $\mU(\lN)^-$, so that the scalars are vectors of the representation. In that context, the multiplicative unit $1\in \eK$ is denoted by $v_0$.

\begin{theorem}
The representation $(\pi_{\mu},V_{\mu})$ of the semisimple Lie algebra $\lG$ is a cyclic module of highest weight, with highest weight $\mu$ and where $v_0$ is a vector of weight $\mu$.
\end{theorem}
\begin{proof}
    No proof.
\end{proof}
The Verma module is, \emph{a priori}, infinite dimensional and non irreducible, thus one has to perform quotients of the Verma module in order to build finite dimensional irreducible representations.


%+++++++++++++++++++++++++++++++++++++++++++++++++++++++++++++++++++++++++++++++++++++++++++++++++++++++++++++++++++++++++++
                    \section{Cyclic modules and representations}
%+++++++++++++++++++++++++++++++++++++++++++++++++++++++++++++++++++++++++++++++++++++++++++++++++++++++++++++++++++++++++++

An example over $\so(3)$ is given in subsection~\ref{subSubSecweightsotrois}. The case of $\so(5)$ is treated in subsection~\ref{SubSecsocinq}. Let $\lG$ be a semisimple Lie algebra with a Cartan subalgebra $\lH$ and a basis $\Delta$ for its roots $\Phi=\Phi^+\cup\Phi^-$. Let $W$ be a finite dimensional $\lG$-module.

\begin{lemma}
If $\lG$ is a nilpotent complex algebra and if $\gamma$ is a weight, then there exists a $v$ in $V_{\gamma}$ such that $c\cdot v=\gamma(x)v$ for every $x\in\lG$.
\end{lemma}
This is the proposition~\ref{prop:trois_poids}. Notice that a Cartan algebra is nilpotent, thus one has at least one vector of $W$ which is a common eigenvector of every elements of $\lH$, in other words, $\exists\mu\in\lH^*$ and $\exists w\in W$ such that
\begin{equation}
    hw=\mu(h)w
\end{equation}
for every $h\in\lH$, and $w\neq 0$. If $w$ is such and if $x\in\lG_{\alpha}$, we have
\begin{equation}
    (hx)\cdot w=[h,x]\cdot w+(xh)\cdot w=\alpha(h)x\cdot w+x\mu(h)w=(\alpha+\mu)(h)x\cdot w.
\end{equation}
If we define
\begin{equation}
    S=\{ w\in W\tq\exists\mu\in\lH^*\tq hw=\mu(h)w \},
\end{equation}
this is not a vector space, but the vector space $\Span S$ generated by $S$ is invariant under $\lG$ because $S$ itself is invariant under all the $\lG_{\alpha}$ with $\alpha\in\lG^*$.

On the other hand, we suppose that $\lG$ and $W$ are finite dimensional, so that their dual are isomorphic. Since a Cartan subalgebra is chosen, we have the decomposition
\begin{equation}
    \lG=\lH\oplus_{\alpha\in\lH^*}\lG_{\alpha}
\end{equation}
where $\lG_{\alpha}=\{ x\in\lG\tq [h,x]=\alpha(h)x\,\forall g\in\lH \}$. When $\alpha\in\lH^*$, the two following spaces are independent of the choice of the Cartan subalgebra $\lH$:
\begin{equation}
    \begin{aligned}
        W_{\alpha}  &=\{ v\in W \tq hv=\alpha(h)v\,\forall h\in\lH \}\\
        \lG_{\alpha}    &=\{ x\in\lG    \tq [h,x]=\alpha(h)x\,\forall h\in\lH \}.
    \end{aligned}
\end{equation}
If $v_{\alpha}\in W_{\alpha}$ and $x_{\beta}\in\lG_{\beta}$, we have
\begin{equation}
    h(x_{\beta}v_{\alpha})=\big( [h,x_{\beta}]+x_{\beta}h \big)v_{\alpha}=\big( \beta(h)+\alpha(h) \big)x_{\beta} v_{\alpha},
\end{equation}
so $x_{\beta}v_{\alpha}\in W_{\alpha+\beta}$. Thus $x_{\beta}$ is a map
\begin{equation}
    x_{\alpha}\colon W_{\alpha}\to W_{\alpha+\beta}.
\end{equation}
Since $W$ is finite dimensional, there exists a maximal $\alpha$ such that $W_{\alpha}\neq0$. We name it $\lambda$. For every $\beta\in\Phi^+$, we have $W_{\lambda+\beta}=\{ 0 \}$. In particular, if $v_{\lambda}\in W_{\lambda}$,
\begin{equation}
    x_{\alpha}x_{\lambda}=0
\end{equation}
for every $\alpha\in\Phi^+$, and, of course,
\begin{equation}
    hv_{\lambda}=\lambda(h)v_{\lambda}.
\end{equation}
On the other hand, for every vector $v\in W$, and for $v_{\lambda}$ in particular, the space $\mU(\lG)v$ is invariant, so
\begin{equation}
    W=\mU(\lG)v_{\lambda}
\end{equation}
by irreducibility. One say that $W$ is the \defe{cyclic module}{cyclic!module} generated by $v_{\lambda}$.


%---------------------------------------------------------------------------------------------------------------------------
                    \subsection{Choice of basis}
%---------------------------------------------------------------------------------------------------------------------------



\begin{theorem}     \label{ThoBaseUGxxmono}
    Let $\lG$ be a Lia algebra on a field of characteristic zero. If $\{ x_i \}$ is an ordered basis of $\lG$, then
    \begin{equation}
        \{ x_{i_1}\cdots x_{i_n}\tq i_1\leq\ldots\leq i_n \}
    \end{equation}
    is a basis for the universal enveloping algebra $\mU(\lG)$ of $\lG$.
\end{theorem}
One can find a proof in \cite{DirkEnvFiniteDimNilLieAlg}.

%---------------------------------------------------------------------------------------------------------------------------
                    \subsection{Roots and highest weight vectors}
%---------------------------------------------------------------------------------------------------------------------------

\begin{proposition}     \label{PropoIrrrgenffflamble}
An irreducible cyclic module is generated by the elements of the form $f_1^{i_1}\cdots f_m^{i_m}v_{\lambda}$.
\end{proposition}

\begin{proof}
    From theorem~\ref{ThoBaseUGxxmono}, the monomials of the form
    \begin{equation}
        (f_1^{i_1}\cdots f_m^{i_m})\cdot (h_1^{j_1}\cdots h_l^{j_l})\cdot (e_1^{k_1}\cdots e_m^{k_m})
    \end{equation}
    form a basis of $\mU(\lG)$. When one act with such an element on $v_{\lambda}$, the $e_i$ kill it, while the $h_i$ do not act (a part of changing the norm). Thus, in fact, the module $W$ is generated by the only elements $f_1^{i_1}\cdots f_m^{i_m}v_{\lambda}$
\end{proof}
In very short, one can write
\begin{equation}        \label{EqWnmoinvlambldarootmodul}
    W=(\lN^-)^nv_{\lambda}.
\end{equation}
Since $f_kv_{\alpha}\in\lG_{\alpha-\alpha_k}$, we have
\begin{equation}        \label{Eqfmlaphamoinsmouns}
    f_1^{i_1}\cdot f_m^{i_m}v_{\lambda}\in\lG_{\lambda-(i_m\alpha_m-\ldots i_1\alpha_1)}.
\end{equation}
The set of roots is ordered by
\begin{equation}
    \begin{aligned}
        \mu_1&\prec\mu_2    &   \text{iff}  &&  \mu_2-\mu_1&=\sum_i k_i\alpha_i
    \end{aligned}
\end{equation}
with $\alpha_i>0$ and with $k_i\in\eN$. Equation \eqref{Eqfmlaphamoinsmouns} means that
\begin{equation}
    \mu\prec\lambda
\end{equation}
for every weight $\mu$ of $W$.

\begin{definition}
Let $\lG$ be a finite dimensional Lia algebra. A \defe{cyclic module of highest weight}{module!highest weight} for $\lG$ is a module (not specially of finite dimension) in which there exists a vector $v_+$ such that $x_+v_+=0$ for every $x_+\in\lN^+$ and $hv_+=\lambda(h)v_+$ for every $h\in\lH$.
\end{definition}

\begin{proposition}
Every submodule of a cyclic highest weight module is a direct sum of weight spaces.
\end{proposition}
\begin{proof}
    No proof.
\end{proof}

From the relation $x_+v_+=0$, we know that all the weight spaces satisfy $V_{\mu}$ satisfy $\mu\prec\lambda$, and, since a module is the sum of all its submodules,
\begin{equation}        \label{EqVsumValpha}
    V=\bigoplus V_{\mu}.
\end{equation}
Notice that if $v_+$ is in a submodule, then that submodule is the whole $V$, thus the sum of two proper submodules is a proper submodule. We conclude that $V$ has an unique maximal submodule, and has thus an unique irreducible quotient.

%---------------------------------------------------------------------------------------------------------------------------
\subsection{Dominant weight}
%---------------------------------------------------------------------------------------------------------------------------
\label{SubSecDomiunSei}

We know that every representation is defined by a highest weight. The following proposition shows that every root cannot be a highest weight of an irreducible representation.

\begin{proposition}[\cite{Anupam}]
    The highest weight of an irreducible representation of a simple complex Lie algebra is an integral dominant weight.
\end{proposition}

\begin{proof}
    Let \( \alpha_i\) be a simple root and consider the corresponding copy of \( \gsl(2,\eC)\) generated by \( \{ e_i,f_i,h_i \}\) (see proposition~\ref{PropWEzZYzC}). The following part of \( L(\Lambda)\) is a \( \gsl(2,\eC)_i\)-module:
    \begin{equation}
        V(\alpha_i)=\bigoplus_{n\in\eZ}V_{\Lambda+n\alpha_i}=V_{\Lambda}\oplus V_{\Lambda-\alpha_i}\oplus V_{\Lambda-2\alpha_i}\oplus\ldots\oplus V_{\Lambda-r\alpha_i}
    \end{equation}
    for some positive integer \( r\). Notice that the sum over \( n\in\eZ\) does not contain terms with \( n<0\) because \( \Lambda\) being an highest weight, \( V_{\Lambda+k\alpha_i}=\emptyset\) when \( k>0\). We know that in a \( \gsl(2,\eC)\)-module the eigenvalues of \( h\) run from \( -m\) to \( m\) (see equations \eqref{EqReprezgsldeuxC} for example). Thus here
    \begin{equation}
        \Lambda(h_i)=-(\Lambda-r\alpha_i)(h_i).
    \end{equation}
    By construction \( \alpha_i(h_i)=2\), so \( \Lambda(h_i)=r\) and the proof is finished.
\end{proof}

\begin{proposition}
    If \( \Lambda\) is the highest weight of the representation \( L(\Lambda)\) of the complex simple Lie algebra \( \lG\) and if \( w_0\) is the longest elements of the Weyl group, then \( w_0\Lambda\) is the lowest weight.
\end{proposition}

\begin{proof}
    First remember that whenever \( \lambda\) is a weight of a representation and \( w\) is an element of the Weyl group, the root \( w\lambda\) is a weight\quext{To be proved.}; in particular \( w_0\Lambda\) is a weight of \( L(\Lambda)\).   Let \( v\in L(\Lambda)_{w_0\Lambda}\); we want to show that \( X_i^-v=0\).

    If \( X_i^-v\neq 0\), then \( w_0\Lambda-\alpha_i\) is a weight and \( w_0\big( w_0\Lambda-\alpha_i \big)=\Lambda-w_0\alpha_i\) is a weight too. Here we used the fact that \( w_0^2=\id\).
\end{proof}

\begin{probleme}
    Still to be shown:
    \begin{enumerate}
        \item
            \( w\lambda\) is a weight
        \item
            \( w_0^2=\id\)
    \end{enumerate}
\end{probleme}

%---------------------------------------------------------------------------------------------------------------------------
                    \subsection{Verma modules}
%---------------------------------------------------------------------------------------------------------------------------

Let us consider
\begin{equation}
    \lB=\lH\oplus\lN^+,
\end{equation}
and take $\alpha\in\lH^*$. Now, we define $\eC_{\alpha}$ as the vector space $\eC$ (one dimensional, generated by $z_+\in\eC$) equipped with the following action of $\lB$:
\begin{equation}
    \big( h+\sum_{\mu\prec 0}x_{\mu} \big)z_+=\alpha(h)z_+.
\end{equation}
The vector space $\eC_{\alpha}$ becomes a left $\mU(\lB)$-module. On the other hand, $\mU(\lG)$ is a free right $\mU(\lB)$-module because $\mU(\lB)\cup\mU(\lG)\subseteq\mU(\lG)$. As $\mU(\lB)$-module, a basis of $\mU(\lG)$ is given by $\lN^-$, i.e. by $\{ f_1^{i_1}\cdots f_m^{i_l} \}$. The \defe{Verma module}{Verma module} is the cyclic module
\begin{equation}
    \Verm(\alpha)=\mU(\lG)\otimes_{\mU(\lB)}\eC_{\alpha}
\end{equation}
which has a highest weight vector $v_{\lambda}=1\otimes z_+$. The tensor product over $\mU(\lB)$ beans that, when $X\in\mU(\lG)$, then
\begin{equation}
    \big( h+\sum_{\mu}x_{\mu} \big)X\otimes_{\mU(\lB)}zz_+=X\otimes\big( h+\sum_{\mu}x_{\mu} \big)zz_+=X\otimes_{\mU(\lG)}z\alpha(h)z_+=\alpha(h)X\otimes_{\mU(\lB)}zz_+.
\end{equation}
The Verma module is generated by $1\otimes z_+$ and the fact that
\begin{equation}
    zX(1\otimes z_+)=X\otimes zz_+.
\end{equation}

\begin{proposition}
Two irreducible cyclic modules with same highest weight are isomorphic.
\end{proposition}

\begin{proof}
Let $V$ and $W$ be two highest weight cyclic modules with highest weight $\lambda$ and highest weight vectors $v_{\lambda}$ and $w_{\lambda}$. In the module $V\oplus W$, the vector $v_{\lambda}\oplus w_{\lambda}$ is a highest weight vector of weight $\lambda$. Let us consider the module
\begin{equation}
    Z=\mU(\lG)(v_{\lambda}\oplus w_{\lambda}).
\end{equation}
That module is a highest weight cyclic module. The projections onto $V=Z/W$ and $W=Z/V$ are non vanishing surjective homomorphisms, so $V$ and $W$ are irreducible quotients of $Z$. But we saw bellow equation \eqref{EqVsumValpha} that $Z$ can only accept one irreducible quotient. Thus $V$ and $W$ are isomorphic.
\end{proof}
We denote by $\Irr_{\lG}(\lambda)$\nomenclature{$\Irr_{\lG}(\lG)$}{the unique cyclic highest weight $\lG$-module with highest weight $\lambda$.} the unique cyclic highest weight $\lG$-module with highest weight $\lambda$.



\chapter{A lot of algebra}
\input{110_algebre}
\input{111_algebre}
% This is part of Giulietta
% Copyright (c) 2020
%   Laurent Claessens
% See the file fdl-1.3.txt for copying conditions.

\section{Cyclic cohomology}
%+++++++++++++++++++++++++
\label{SecCyclicHomol}

Let $\cA$ be an algebra over $\eC$. We consider the complex $(C_{\lambda}^n,b)$ defined by
\begin{itemize}
\item $C_{\lambda}^n$ is the set of $(n+1)$-linear functionals $\varphi$ on $\cA$ such that
\begin{equation}
  \varphi(a^1,\cdots,a^{n},a^0)=(-1)^n\varphi(a^0,\cdots,a^{n}),
\end{equation}
\item the \defe{Hochschild coboundary}{Hochschild!coboundary} $b$ is defined by
\begin{equation}
\begin{split}
  (b\varphi)(a^0,\cdots,a^{n+1})=&\sum_{j=0}^{n}(-1)^j\varphi(a^0,\cdots,a^{j}a^{j+1},\cdots,a^{n+1})\\
				&+(-1)^{n+1}\varphi(a^{n+1}a^0,\cdots,a^{n}).
\end{split}
\end{equation}
We denote by $C^0_{\lambda}$ the set of linear functionals.

\end{itemize}
We denote by $HC^*(\cA)$ the cohomology of this complex.

\begin{definition}		\label{DefCycleCoh}
	A $n$-dimensional \defe{cycle}{cycle} is a triple $(\Omega,d,\int)$ with
	\begin{itemize}
		\item $\Omega=\sum_{j=0}^{n}\Omega^j$ is a graded algebra over $\eC$,
		\item $d$ is  graduate derivative of degree $1$,
		\item $\int\colon \Omega^n\to \eC$ is a \defe{closed trace}{closed!trace}, that is $\int d\omega=0$.
	\end{itemize}
	When $\cA$ is an algebra over $\eC$, we say that a cycle \emph{over $\cA$} is a cycle endowed with a homomorphism $\rho\colon \cA\to \Omega^0$.
\end{definition}

An important subset of $C^n_{\lambda}$ is the set
\begin{equation}
Z^n_{\lambda}(\cA)=\{ \varphi\tq b\varphi=0 \}.
\end{equation}
 When $n=0$, the condition $b\varphi=0$ becomes $(b\varphi)(a^0,a^1)=\varphi(a^0a^1)-\varphi(a^1a^0)=0$, in such a way that the elements of $Z^0_{\lambda}$ are exactly the traces on $\cA$.

If $(\Omega,d,\int)$ is a cycle and $\rho\colon \cA\to \Omega^0$ a homomorphism, then the \defe{character}{character!of a homomorphism} of $\rho$ is the functional $\tau$ defined by
\begin{equation}
\tau(a^0,\cdots,a^{n})=\int \rho(a^0)d\big( \rho(a^1) \big)\cdots d\big( \rho(a^{n}) \big).
\end{equation}


\subsection{Example: de Rham homology}
%--------------------------------------

Let $M$ be a differentiable manifold and $C$, a closed de Rham $q$-current on $M$ with $q\leq\dim M$. If $\omega$ is a form on $M$, we denote by $C(\omega)$ the evaluation of $C$ on $\omega$. We consider $\Omega^{i}= C^{\infty}\big( M,\Wedge^i T^*M \big)$, the space of differential forms of degree $i$. It provides a graded differential algebra $(\Omega,d)$ on which we can put the trace $\int\colon \Omega^q\to \eC$,
\[
  \int \omega=C(\omega).
\]

\subsection{Hochschild cohomology}
%---------------------------------

Let $\cA$ be an algebra over $\eC$ and $\cA_{\cun}=\cA\oplus\eC\cun$ be the algebra obtained by adduction of an unity to $\cA$. We define
\[
  \Omega^1(\cA)=\cA_{\cun}\times_{\eC}\cA
\]
and an $\cA$-bimodule structure by
\[
  x\Big( (a+\lambda\cun)\times b \Big)y:=(xa+\lambda x)\times by-(xa+\lambda xb)\times y
\]
with $a$, $b$, $x$, $y\in\cA$ and $\lambda\in\eC$. Then we consider
\begin{equation}
\begin{aligned}
 d\colon \cA&\to \Omega^1(\cA) \\
da&= 1\otimes a
\end{aligned}
\end{equation}
which can be checked to be a derivation.

\begin{proposition}
Let $\modE$ be a $\cA$-bimodule and $\delta\colon \cA\to \modE$, a derivation. Then there exists a bimodule morphism $\rho\colon \Omega^1(\cA)\to \modE$ such that $\delta=\rho\circ d$.
\end{proposition}

\begin{proof}
No proof.
\end{proof}

This proposition says that $\big( \Omega^1(\cA),d \big)$ is an universal derivation in a $\cA$-bimodule. From $\Omega^1(\cA)$, we define $\Omega^0(\cA)=\cA$.
\[
  \Omega^n(\cA)=\Omega^1(\cA)\otimes_{\cA}\cdots\otimes_{\cA}\Omega^1(\cA),
\]
and the differential extends to an unique graded derivation of $\Omega^*(\cA)$ such that $d^2=0$. Note the isomorphism
\begin{equation}
\begin{aligned}
 j\colon \cA_{\cun}\otimes\cA^{\otimes n}&\to\Omega^n(\cA)  \\
(a^0+\lambda\cun)\otimes a^1\otimes\cdots\otimes a^{n}&\mapsto a^0da^1\cdots da^{n}+\lambda da^1\cdots da^{n}
\end{aligned}
\end{equation}
for each $a^{j}\in\cA$ and $\lambda\in\eC$.

\begin{lemma}
The cohomology of the complex $\big( \Omega^*(\cA),d \big)$ is zero in any dimension, including zero.
\end{lemma}
\begin{proof}
No proof.
\end{proof}

A product in $\Omega^*(\cA)$ is defied as usual by juxtaposition and rearrangement using the fact that $d$ is a derivation, see for example the proof of proposition~\ref{prop_modMununique}:
\[
\begin{split}
(a^0da^1\cdots da^{n})(a^{n+1}da^{n+2}\cdots da^{n})&=\sum_{j=1}^{n}(-1)^{n-j}a^0da^1\cdots d(a^{j}a^{j+1})\cdots da^{n}da^{n+1}\cdots da^m\\
					&\quad+(-1)^na^0a^1 da^2\cdots da^{m}.
\end{split}
\]
It is nothing else than define the product in such a way that $\Omega^*(\cA)$ is a right $\cA$-module and $d$ a derivation.


\begin{proposition}
Let $\tau$ be a $(n+1)$-linear functional on $\cA$. The three following properties, where $a^0,\cdots a^n$ are some elements of $\cA$, are equivalent
\begin{enumerate}
\item There exists a cycle $(\Omega,d,\int)$ of dimension $n$ and a homomorphism $\rho\colon \cA\to \Omega^0$ such that
\[
  \tau(a^0,\cdots,a^{n})=\int \rho(a^0)d\big( \rho(a^1) \big)d\big( \rho(a^{n}) \big)
\]
for all $a^0,\cdots,a^{n}\in\cA$. In other words, $\varphi$ is the character of a cycle.
\item There exists a closed graded trace $\hat{\tau}$ of dimension $n$ on $\Omega^*(\cA)$ such that
\[
  \tau(a^0,\cdots,a^{n})=\hat{\tau}(a^0da^1,\cdots,da^{n})
\]
\item The functional $\tau$ satisfies $b\tau=0$ and $\tau\circ\gamma=\epsilon(\gamma)\tau$ where $\gamma$ is any cyclic permutation of $\{ 0,1,\cdots,n \}$ and $\epsilon(\gamma)$ is its parity, or more explicitly,
\[
	\tau(a^1,\cdots,a^{n},a^0)=(-1)^n\tau(a^0,\cdots,a^{n})
\]
and
\[
   \sum_{i=0}^{n}(-1)^n\tau(a^0,\cdots a^ia^{i+1},\cdots,a^{n+1})+(-1)^{n+1}\tau(a^{n+1}a^0,\cdots,a^{n})=0.
\]
The most compact way to express this condition is just $\tau\in Z^n_{\lambda}$.
\end{enumerate}
\end{proposition}

\subsection{Hochschild groups of cohomology}
%--------------------------------------------

\begin{definition}
    Let $\cA^e=\cA\otimes\cA^0$, the tensor product of $\cA$ with its opposite algebra. Let $\modM$ be a module over $\cA$. The \defe{Hochschild cohomology}{Hochschild!cohomology} is defined as follows. 

    \begin{itemize}
        \item 
    Let $C^n(\cA,\modM)$ is the set of $n$-linear maps from $\cA$ to $\modM$. 
\item A cochain is an element of $C^*(\cA,\modM)$ and the differential is given by
    \begin{equation}
    \begin{split}
    (bT)(a^1,\cdots,a^{n+1})&=a^1T(a^2,\cdots,a^{n+1})+\sum_{i=1}^{n}(-1)^iT(a^1,\cdots,a^ia^{i+1},\cdots,a^{n+1})\\
                    &\quad +(-1)^{n+1}T(a^1,\cdots,a^{n})a^{n+1}.
    \end{split}
    \end{equation}
\item
    The Hochschild cohomology of $\cA$ with coefficients in $\modM$ is finally defined as the cohomology of the complex $\big( C^*(\cA,\modM),b \big)$
    \end{itemize}
\end{definition}

Let us study a particular case. The space $\cA^*$ of functionals on $\cA$ becomes a $\cA$-bimodule when one defines $(a\varphi b)(c)=\varphi(abc)$ for each $a$, $b$, $c\in\cA$.

Let $T\in C^n(\cA,\cA^*)$; this is a $n$-linear functional from $\cA$ to $\cA^*$ which can be seen as a $(n+1)$-linear function $\tau_T$ on $\cA$ by
\[
  \tau_T(a^0,a^1,\cdots,a^{n})=T(a^1,\cdots,a^{n})(a^0).
\]
If one defines $b$ acting on $\tau$ by
\begin{align}
b(\tau)(a^0,\cdots,a^{n+1})&=\sum_{i=0}^{n}(-1)^i\tau(a^0,\cdots,a^ia^{i+1},\cdots,a^{n/1})\\
			&\quad+(-1)^{n+1}\tau(a^{n+1}a^0,\cdots,a^{n}),
\end{align}
we have $\tau_{bT}=b\tau_T$.

Let $A\colon C^n(\cA,\cA^*)\to C^n(\cA,\cA^*)$, the linear map defined by
\[
  (A\varphi)=\sum_{\gamma\in\Gamma}\epsilon(\gamma)(\varphi\circ\gamma)
\]
where $\Gamma$ is the group of cyclic permutations of $\{ 0,1,\cdots,n \}$, and $\epsilon(\gamma)$, the parity of $\gamma$. The image of $A$ is $C_{\lambda}^n(\cA)$.

\begin{lemma}
If one defines $b'\colon C^n(\cA,\cA^*)\to C^{n+1}(\cA,\cA^*)$ by
\begin{equation}
  (b'\varphi)(a^0,\cdots,a^{n+1})=\sum_{j=0}^{n}(-1)^j\varphi(a^0,\cdots,a^ja^{j+1},\cdots,a^{n}),
\end{equation}
we have $b\circ A=A\circ b'$.
\end{lemma}
\begin{proof}
No proof.
\end{proof}

As corollary,
\begin{corollary}
The complex $\big( C_{\lambda}^n(\cA),b \big)$ is a subcomplex of the Hochschild complex.
\end{corollary}
\begin{proof}
No proof.
\end{proof}

\subsection{Homomorphisms}
%--------------------------

Let $\cA$ and $\cB$ be two algebras. From a homomorphism $\rho\colon \cA\to \cB$, one induces a homomorphism of complex $\rho^*\colon C_{\lambda}^n(\cB)\to C_{\lambda}^n(\cA)$ with the definition
\[
  (\rho^*\varphi)(a^0,\cdots,a^{n})=\varphi\big( \rho(a^0),\cdots,\rho(a^{n}) \big).
\]
In order for $\rho^*$ to pass to quotient and define a homomorphism $\rho^*\colon HC^n(\cB)\to HV^n(\cA)$, we need that, for each $\varphi$, a certain $\eta$ fulfills $\rho^*b\varphi=b'\eta$.

\begin{proposition}
Let $u\in\cA\invtible$, and $\theta\colon \cA\to \cA$ defined by $\theta(x)=uxu^{-1}$. Then the induced map $\theta^*\colon HC^*(\cA)\to HC^*(\cA)$ is the identity.
\end{proposition}
\begin{proof}
No proof.
\end{proof}
The so defined map $\theta$ is called the \defe{inner automorphism}{inner!automorphism} associated with $u$.

\begin{lemma}
Let $\cA$ be an unital algebra for which there exists a homomorphism $\rho$ and an invertible element $X$ in $M_2(\cA)$ such that
\[
  X\begin{pmatrix}
a&0\\
0&\rho(a)
\end{pmatrix}
X^{-1}=
\begin{pmatrix}
0&0\\
0&\rho(a)
\end{pmatrix}
\]
for all $a\in\cA$. Then $HC^n(\cA)=0$ for all $n$.
\end{lemma}
\begin{proof}
No proof.
\end{proof}

We say that the cycle $(\Omega,d,\int)$ \defe{vanishes}{vanishing cycle} if the algebra $\Omega^0$ fulfills the hypothesis of the latter lemma.


\subsection{The cup product}
%---------------------------


In general, the space $\Omega^*(\cA\otimes\cB)$ is different to the space $\Omega^*(\cA)\otimes\Omega^*(\cB)$, but the first if a left $\cA$-module and a right $\cB$-module and $\Omega^*(\cA\otimes\cB)$ is the universal algebra with this property. So we have a homomorphism $\pi\colon \Omega^*(\cA\otimes\cB)\to \Omega^*(\cA)\otimes\Omega^*(\cB)$ such that $\pi\circ d_{\cA\otimes\cB}=(d_{\cA}\otimes d_{\cB})\circ\pi$.

Let now $\varphi$ be a $(n+1)$-linear functional on $\cA$. One define the linear functional $\hat{\varphi}$ on $\Omega^n(\cA)$ by the formula
\begin{equation}
 \hat{\varphi}\circ j\big( (a^0+\lambda\cun)\otimes a^1\otimes\cdots\otimes a^{n} \big)=\varphi(a^0,a^1,\cdots,a^{n}).
\end{equation}
This definition can be of course be adapted to $\cB$. Now we consider $\varphi\in C^n(\cA,\cA^*)$ and $\psi\in C^m(\cB,\cB^*)$ and the corresponding hat functions $\hat{\varphi}\colon \Omega^n(\cA)\to \cA^*$ and $\hat{\psi}\colon \Omega^m(\cB)\to \cB^*$. We can also consider the tensor product $\hat{\varphi}\otimes\hat{\psi}$. The homomorphism $\pi\colon \Omega^*(\cA\otimes\cB)\to \Omega^*(\cA)\otimes\Omega^*(\cB)$ allows us to consider the composition map $(\hat{\varphi}\otimes \hat{\psi})\circ \pi$ which has to be the hat function of a certain multilinear functional on $\cA\otimes\cB$. The latter is denoted by $\varphi  \cuppr \psi$:
\begin{equation}
  \widehat{\varphi\cuppr\psi}=(\hat{\varphi}\otimes\hat{\psi})\circ\pi.
\end{equation}
The map $\varphi\cuppr\psi$ and is called the \defe{cup product}{cup product} of $\varphi$ and $\psi$.

An element of $\Omega^l(\cA\otimes\cB)$ is of the form $\omega_1\otimes\omega_2\otimes\cdots\otimes\omega_l$ with $\omega_i\in\Omega^1(\cA\otimes\cB)$. But an element of the latter is of the form
\[
  \big( (a\otimes b)+\lambda\cun \big) \otimes(a'\otimes b')
\]
where $a$, $a'\in\cA$ and $b$, $b'\in\cB$. Notice that taking $b=b'=1$, one can identify the result to an element of $\Omega^1(\cA)$. Hence an element of $\Omega^n(\cA)\otimes\Omega^m(\cB)$ can be seen as a very special element of $\Omega^{n+m}(\cA\otimes\cB)$. From this point of view, we can see $\pi$ as a map
\[
  \pi\colon \Omega^{n+m}(\cA\otimes\cB)\to \Omega^n(\cA)\otimes\Omega^m(\cB).
\]

\begin{proposition}
The cup product enjoys the following main properties.
\begin{enumerate}
\item The map $\varphi\otimes\psi\mapsto \varphi\cuppr\psi$ is a homomorphism
\[
  HC^n(\cA)\otimes HC^m(\cB)\to HC^{n+m}(\cA\otimes\cB).
\]
\item The character of the tensor product of two cycles is the cup product of their character.
\end{enumerate}

\end{proposition}


\chapter{Fiber bundle}
% This is part of (almost) Everything I know in mathematics
% Copyright (c) 2013-2014, 2020
%   Laurent Claessens
% See the file fdl-1.3.txt for copying conditions.

\section{Vector bundle}
%++++++++++++++++++++++

Let $M$ be a smooth manifold. A \defe{$V$-vector bundle}{vector!bundle}\index{bundle!vector} of rank $r$ on $M$ is a smooth manifold $F$ and a smooth projection $\dpt{p}{F}{M}$ such that

\begin{itemize}
\item for any $x\in M$, the fiber $F_x:=p^{-1}(x)$ is a vector space of dimension $r$ on the same field that $V$ (let's say $\eK=\eR$ or $\eC$).
\item for any $x\in M$, there exists an open neighbourhood $\mU$ of $x$ and a ``chart diffeomorphism``{} $\dpt{\phi}{p^{-1}(\mU)}{\mU\times V}$ such that for any $l\in p^{-1}(y)$,
   \begin{itemize}
      \item $\phi(l)=(y,\phi_y(l))$
      \item $\dpt{\phi_y}{E_y}{V}$ is a vector space isomorphism.
   \end{itemize}
\end{itemize}

The pair $(\mU,\phi)$ is a \emph{local trivialization}; $M$ is the \emph{base space}; $F$, the \emph{total space}, $p$ the \emph{projection} and $r$, the \emph{rank} of the bundle. The denominations of total and base spaces will also be used in the same way for principal bundles.

We will sometimes use charts diffeomorphism $\dpt{\phi}{\mU\times V}{p^{-1}(\mU)}$ instead of $\dpt{\phi}{p^{-1}(\mU)}{\mU\times V}$. Since they are diffeomorphism, this difference don't affect anything.

\subsection{Transition functions}
%--------------------------------

The trivializations will be denoted by Greek indices: $\mU_{\alpha}$, $\phi_{\alpha}$,\ldots The symbol $\mU_{\alpha}{}_{\beta}$ naturally denotes $\mU_{\alpha}\cap\mU_{\beta}$. If we consider two local trivializations $(\mU_{\alpha},\phi_{\alpha})$ and $(\mU_{\beta},\phi_{\beta})$, we have to look at $\dpt{\phi_{\alpha}\circ\phi_{\beta}^{-1}}{\mU_{\alpha}{}_{\beta}\times\eK^r}{\mU_{\alpha}{}_{\beta}\times\eK^r}$. We define the \defe{transition functions}{transition function} $\dpt{g_{\alpha}{}_{\beta}}{\mU_{\alpha}{}_{\beta}}{GL(r,\eK)}$ by
\begin{equation}
\phi_{\alpha}\circ\phi_{\beta}^{-1}(x,v)=(x,g\bab(x)v).
\end{equation}
These functions take their values in $GL(r,\eK)$ because $\dpt{\phi_y}{E_y}{V}$ is a vector space isomorphism. Since $(\phi_{\alpha}\circ\phi_{\beta})^{-1}=\phi_{\beta}\circ\phi_{\alpha}^{-1}$, it is clear that $g\bab(x)=g_{\alpha\beta}(x)^{-1}$.

If $x\in\mU_{\alpha\beta\gamma}=\mU_{\alpha}\cap\mU_{\beta}\cap\mU\bgamma$, we have $\phi_{\alpha}\circ\phi\bgamma^{-1}(x,v)=(x,g_{\alpha\gamma}(x)v)$, but also $\phi_{\alpha}\circ\phi\bgamma^{-1}=\phi_{\alpha}\circ\phi_{\beta}^{-1}\phi_{\beta}\circ\phi\bgamma^{-1}$, then
\begin{equation}
  (x,g_{\alpha\gamma}(x)v)=(\phi_{\alpha}\circ\phi_{\beta}^{-1})(x,g_{\beta\gamma}(x)v)
               =(x,g\bab(x)_{\beta\gamma}(x)v).
\end{equation}
Thus $g_{\alpha\gamma}(x)=g\bab(x)g_{\beta\gamma}(x)$. So, as linear maps, we have
\begin{equation}\label{eq:g_compat}
  g\bab\circ g_{\alpha\gamma}\circ g_{\gamma\alpha}=\mtu.
\end{equation}

\subsection{Inverse construction}\label{subsec:inv_g}
%-------------------------------------------

Let us consider a manifold $M$, an open covering $\{\mU_{\alpha}:\alpha\in I\}$ and some functions $\dpt{g\bab}{\mU\bab}{GL(r,\eK)}$ which fulfill relations \eqref{eq:g_compat}. We will build a vector bundle $E\stackrel{p}{\longrightarrow}M$ whose transition functions are the $g_{\alpha\beta}$'s. Let $\tilde{E}$ be the disjoint union
\[
  \tilde{E}=\bigsqcup_{\alpha\in I}\mU_{\alpha}\times\eK^r,
\]
i.e. triples of the form $(x,v,\alpha)\in M\times\eK^r\times I$ with the condition that $x\in\mU_{\alpha}$. We define an equivalence relation on $\tilde{E}$ by $(x,v,\alpha)\sim(y,w,\beta)$ if and only if $x=y$ and $w=g\bab(x)v$. Next, we define $E=\tilde{E}/\sim$ and $\dpt{\omega}{\tilde{E}}{E}$, the canonical projection. The projection $\dpt{p}{E}{M}$ is naturally defined by $p([x,v,\alpha])=x$. The chart diffeomorphism is $\dpt{\varphi_{\alpha}}{\mU_{\alpha}\times\eK^r}{p^{-1}(\mU_{\alpha})}$,
\[
  \varphi_{\alpha}(x,v)=\omega(x,v,\alpha).
\]
Now we have to prove that $E$ endowed with the $\varphi_{\alpha}$'s is a vector bundle.

First we prove that $\varphi_{\alpha}$ is surjective. For this we remark that a general element in $p^{-1}(\mU_{\alpha})$ can be written under the form $\omega(x,v,\alpha)$ with $x\in\mU\bab$. But
\begin{equation}
\begin{split}
  \varphi_{\alpha}(x,g\bab(x)w)&=\omega(x,g\bab(x)w,\alpha)\\
                         &=\omega(x,g_{\alpha\beta}(x)g\bab(x)w,\beta)\\
			 &=\omega(x,w\beta),
\end{split}
\end{equation}
then $\varphi_{\alpha}$ is surjective. Now we suppose $\varphi_{\alpha}(x,v)=\varphi_{\alpha}(y,w)$. Then $\omega(x,v,\alpha)=\omega(y,w,\alpha)$ and $x=y$, $w=g_{\alpha\alpha}v$ which immediately gives $v=w$. Then $\varphi_{\alpha}$ is injective.

Finally, we have
\begin{equation}
  (\varphi\alpha\circ\varphi_{\beta}^{-1})(\omega(x,v,\alpha))=\varphi_{\alpha}(x,g_{\alpha\beta}(x)v)
                                                   =\omega(x,g_{\alpha\beta}(x)v,\alpha),
\end{equation}
which proves that the maps $g$ are the transition functions of the vector bundle $E$.

\subsection{Equivalence of vector bundle}
%----------------------------------------

Let $E\stackrel{p}{\longrightarrow}M$ and $F\stackrel{p'}{\longrightarrow}M$ be two vector bundles on $M$. They are \defe{equivalent}{equivalence!of vector bundle} if there exists a smooth diffeomorphism $\dpt{f}{E}{F}$ such that

\begin{itemize}
\item $p'\circ f=p$,
\item $\dpt{f|_{E_x}}{E_x}{F_x}$ is a vector space isomorphism.
\end{itemize}

Let $E$ and $F$ be two equivalent vector bundles, $\{\mU_{\alpha}\tq \alpha\in I\}$, an open covering which trivialize $E$ and $F$ in the same time and $\phi^E_{\alpha}$, $\phi^F_{\alpha}$ the corresponding trivializations. A map $\dpt{f}{E}{F}$ reads ``in the trivialization''\ as $\dpt{\phi^F_{\alpha}\circ f|_{p^{-1}(\mu_{\alpha})}\circ\phi^E{}^{-1}_{\alpha}}{\mU_{\alpha}\times\eK^r}{\mU_{\alpha}\times\eK^r}$ and defines a map $\dpt{\lambda_{\alpha}}{\mU_{\alpha}}{GL(r,\eK)}$ by
\begin{equation}
(\phi^F_{\alpha}\circ f|_{p^{-1}(\mu_{\alpha})}\circ\phi^E{}^{-1}_{\alpha})(x,v)=(x,\lambda_{\alpha}(x)v).
\end{equation}
If we denote by $g^E$ the transition functions for $E$ (and $g^F$ for $F$),
\[
 \phi^F_{\alpha}\circ\phi^F_{\beta}{}^{-1}= (\phi^F_{\alpha}\circ f\circ\phi_{\alpha}^E{}^{-1})\circ
                                        (\phi^E_{\alpha}\circ\phi^E_{\beta}{}^{-1})\circ
					(\phi^E_{\beta}\circ f^{-1}\circ\phi_{\beta}^E{}^{-1}),
\]
so that
\begin{equation}\label{eq:g_l_g_l}
  g\bab^F(x)=\lambda_{\alpha}(x) g^E\bab(x)\lambda(x)^{-1}.
\end{equation}

Once again we have an inverse construction. We consider a vector bundle $E$ on $M$ with transition functions $g^E$ and some maps $\dpt{\lambda_{\alpha}}{\mU_{\alpha}}{GL(r,\eK)}$; then we define $g^F\bab(x)$ by equation \eqref{eq:g_l_g_l}.

From subsection~\ref{subsec:inv_g}, one can construct a vector bundle $F$ on $M$ whose transition functions are these $g^F$. With the trivializations $\phi^F$ of $F$, one can define $\dpt{f}{E}{F}$ by
\[
(\phi^F_{\alpha}\circ f\circ\phi^E_{\alpha}{}^{-1})(x,v)=(x,\lambda_{\alpha}(x)v).
\]

When a basis space $B$ is given, we denote by $\Vect(B)$ the set of isomorphism classes of vector bundles over $B$. In the complex case, we denote it by $\Vect_{\eC}(B)$.

\begin{proposition}
Any vector bundle over $\eR^n$ is trivial.
\end{proposition}

\begin{proof}
Let $\dpt{p}{F}{M}$ be a vector bundle on $M=\eR^n$ and $\{\mU_{\alpha}\}$ be covering of $\eR^n$ by local trivializations. Now consider a partition of unity\index{partition of unity} related to the covering $\mU_{\alpha}$: a set of functions $\dpt{f_{\alpha}}{M}{\eR}$ such that
\begin{itemize}
\item $f_{\alpha}>0$,
\item $\forall x\in M$, one can find a neighbourhood of $x$ in which only a \emph{finite} number of $f_{\alpha}$ is non zero,
\item $\forall x\in M$, $\sum_{\alpha} f_{\alpha}(x)=1$.
\item $f_{\alpha}=0$ outside of $\mU_{\alpha}$.
\end{itemize}
Using that partition of unity, we build the trivialization function $\dpt{f}{F}{\eR^n\times V}$ by $f(l)=(x,\sum_{\alpha} f_{\alpha}(x)\phi_{\alpha x}(l))$.
\end{proof}

The following two propositions have some importance in K-theory.
\begin{proposition}		\label{PropEoplusEprimetriv}
Let $\pi\colon E\to B$ be a complex vector bundle over a basis compact, Hausdorff, connected basis $B$. Then there exists a vector bundle $E'$ such that $E\oplus E'$ is trivial.
\end{proposition}

\begin{proposition}		\label{PropmapfEEsun}
Let $f\colon A\to B$ be a map between the topological spaces $A$ and $B$, and consider a vector bundle $\pi\colon E\to B$. Then there exists one and only one vector bundle $\pi'\colon E'\to A$ and a map $f'\colon E'\to E$ such that $f'|_{E'_x}\colon E'_x\to E_{f(x)}$ is an isomorphism. The vector bundle $E'$ is unique up to isomorphism.
\end{proposition}
Proofs can be found in \cite{VB_and_K}. Let us denote by $f^*(E)$ the function given by proposition~\ref{PropmapfEEsun}. It satisfies the following properties
\begin{equation}		\label{EqPropfstarEVect}
\begin{split}
	(fg)^*(E)		&=g^*\big( f^*(E) \big)\\
	\id^*(E)		&=E\\
	f^*(E_1\oplus E_2)	&=f^*(E_1)\oplus f^*(E_2)\\
	f^*(E_1\otimes E_2)	&=f^*(E_1)\otimes f^*(E_2).
\end{split}
\end{equation}



\subsection{Sections of vector bundle}
%-------------------------------------

A \defe{section}{section!of vector bundle} of the vector bundle $p\colon E\to M$ is a smooth map $\dpt{s}{M}{E}$ such that $p\circ s=\id|_M$. The set of all the sections is denoted by $\Gamma^{\infty}(M)$ or simply $\Gamma(E)$.\nomenclature{$\Gamma(E)$}{Space of sections of the vector bundle $E$}

If $(\mU_{\alpha},\phi_{\alpha})$ is a local trivialization, one can describe the section $s$ by a function $\dpt{s_{\alpha}}{\mU_{\alpha}}{V}$ defined by $\phi_{\alpha}(s(x))=(x,s_{\alpha}(x))$, or equivalently by
\[
s(x)=\phi_{\alpha}^{-1}(x,s_{\alpha}(x)).
\]
As usual when we define such a local quantity, we have to ask ourself how are related $s_{\alpha}$ and $s_{\beta}$ on $\mU_{\alpha}\cap\mU_{\beta}$. The best is $s_{\alpha}=s_{\beta}$, but most of the time it is not. Here, we compute
\[
  \phi_{\beta}\circ\phi_{\alpha}^{-1}\circ\phi_{\alpha}(s(s))=(x,g_{\alpha\beta}(x)s_{\alpha}(x)),
\]
which is obviously also equal to $(x,s_{\beta}(x))$. Then
\begin{equation}\label{eq:tr_sec}
s_{\beta}(x)=g_{\alpha\beta}(x)s_{\alpha}(x)
\end{equation}
without summation.

\section{Vector valued differential forms}	\label{SecVectValFiffFor}
%+++++++++++++++++++++++++++++++++++++++++++

Let $E$ be a vector bundle over $M$. A \defe{$E$-valued $p$-form}{vector-valued differential form}\index{differential!form!vector-valued} is a section
\[
  e\in\Gamma\big( E\otimes\Wedge^pT^*M \big).
\]
We denote by $\Omega(M,E)=\Gamma\big( E\otimes\Wedge^pT^*M \big)$\nomenclature[D]{$\Omega(M,E)$}{the set of $E$-valued differential forms} the set of $E$-valued differential forms. An element of $\Omega^1(M,E)=\Gamma\big( E\otimes\Wedge T^*M\big)$ always reads  $\sum_is_i\otimes\omega_i$ for some sections $s_i$ and usual differential forms $\omega_i$.

A form of $\Omega^p(M,E)$ can be seen as a fiber morphism $\underbrace{TM\otimes\cdots\otimes TM}_{p\text{ times}}\to E$ by associating
\[
  s\otimes\omega(X_1,\cdots,X_p)=s(x)\omega(X_1,\cdots,X_p)\in E_x
\]
to the element $(s\otimes \omega)\in\Omega^p(M,E)$. There exists a wedge product between vector-valued forms. If $e\in\Omega^p(M,E_1)$ and $f\in\Omega^q(M,E_2)$, then we define $e\wedge f\in\Omega^{p+q}(M,E_1\otimes E_2)$ by
\begin{equation}	\label{EqDefwedgevecor}
(e\wedge f)(v_1,\cdots,v_{p+q})=\frac{1}{ p!q! }\sum_{\pi\in S_{p+q}}(-1)^{\pi} e(v_{\pi(1)},\cdots v_{\pi(p)})\otimes f(v_{\pi(p+1)},\cdots,v_{\pi(p+q)})\in E_1\otimes E_2.
\end{equation}
where $(-1)^{\pi}$ stands for the sign of the permutation $\pi$. For example when $e$, $f\in \Omega^1(M,E)$, we have
\[
  (e\wedge f)(X,Y)=e(X)\otimes f(Y)-e(Y)\otimes f(X)\in E\otimes E.
\]

When $M$ is a differentiable manifold, the \defe{fundamental $1$-form}{fundamental!$1$-form} is the element $\theta\in\Omega(M,TM)$ such that
\[
  \iota(X)\theta=X
\]
for every $X\in \Gamma(TM)$.

\subsection{A digression:  \texorpdfstring{$T_Y\yG$}{TYG} and \texorpdfstring{$\yG$}{G}}\label{subsec:digress}
%+++++++++++++++++++++++++++++++++++++++++++

We define two product: $G\times\yG\to TG$ and $\yG\times\yG\to\yG$. If $g\in G$ and $X\in\yG$, we put
\begin{subequations}
\begin{equation} \label{eq_gXdefa}
   gX=\dsdd{ge^{tX}}{t}{0},
\end{equation}
and if $X$, $Y\in\yG$,
\begin{equation}\label{eq:yGyGb}
   XY=\DDsdd{e^{tX}e^{uY}}{t}{0}{u}{0}.
\end{equation}
\end{subequations}
We naturally define the product of a $\yG$-valued $1$-form $A$ by an element $g\in G$ by $(gA)v=gA(v)$.

 Note that $gX$ does not belong to $\yG$ but to $T_{g}G$. Fortunately, in the expressions which we will meet, there will  always be a $g^{-1}$ to save the situation.

Let us give some precisions about derivatives as \eqref{eq:yGyGb}. We consider the expression
\[
  \frac{d}{du}\left( \left.\frac{d}{dt} c_u(t)\right|_{t=0}\right)_{u=0},
\]
which will be more simply written as:
\begin{equation}\label{eq:2307e1}
\DDsdd{ c_u(t) }{u}{0}{t}{0}
\end{equation}
with $c_u(t)\in G$ for all $u,t$; $c_u(0)=e$ for all $u$ and $c_0'(0)=Y\in\yG$ where the prime stands for the derivative with respect of $t$. So $\dsdd{c_u(t)}{t}{0}\in\yG$ for each $u$ and \eqref{eq:2307e1} belongs to $T_Y\yG$. But we know that $\yG$ is a vector space, then $T_Y\yG\simeq\yG$, the isomorphism being given by the following idea: if $\{\partial_i\}$ is a basis of $\yG$ and $\{e_i\}$ the corresponding basis of $T_Y\yG$, we define the action of $A^ie_i\in T_Y\yG$ on $\dpt{f}{G}{\eR}$ by $(A^ie_i)f:=A^i\partial_if$.

\begin{lemma}
Seen as an equality in $\yG$, for $\dpt{f}{G}{\eR}$ we have:
\begin{equation}
   \DDsdd{c_u(t)}{u}{0}{t}{0}f=\DDsdd{f(c_u(t))}{u}{0}{t}{0}.
\end{equation}
\end{lemma}

\begin{proof}
Let us consider $X_u=X_u^i\partial_i=c_u'(0)$ and $X_0=Y$. We naturally have
\begin{align}
   X_uf&=\dsdd{f(c_u(t))}{t}{0},&\text{ and } &&\dsdd{X_u}{u}{0}\in T_Y\yG.
\end{align}
Now, we consider a function $\dpt{h}{\yG}{\eR}$ and compute:
\[
  \Dsdd{X_u}{u}{0}h=\Dsdd{h(X_u)}{u}{0}
                   =\dsdd{ h(\Dsdd{c_u(t)}{t}{0}) }{u}{0}.
\]
If $\{\partial_i\}$ is a basis of $\yG$ and $\{e_i\}$, the corresponding one of $T_Y\yG$, thus
\begin{equation}
                   \Dsdd{X_u}{u}{0}h=\left.\dsd{h}{e_i}\right|_Y\DDsdd{c^i_u(t)}{u}{0}{t}{0}.
\end{equation}
So, we can write
\[
   \Dsdd{X_u}{u}{0}=\DDsdd{c^i_u(t)}{u}{0}{t}{0}\left.\dsd{}{e_i}\right|_Y,
\]
and, as element of $\yG$, we consider
\[
  \Dsdd{X_u}{u}{0}=\DDsdd{ c^i_u(t) }{u}{0}{t}{0}\partial_i|_e.
\]
Now, we can compute the action of $\dsdd{X_u}{u}{0}$ on a function $\dpt{f}{G}{\eR}$ as
\begin{equation}
\begin{split}
\Dsdd{X_u}{u}{0}f&=\DDsdd{c^i_u(t)}{u}{0}{t}{0}\left.\dsd{f}{x^i}\right|_e\\
                 &=\Dsdd{ \left.\dsd{f}{x^i}\right|_e\dsdd{c^i_u(t)}{t}{0}  }{u}{0}\\
		 &=\Dsdd{ \dsdd{f(c_u(t))}{t}{0} }{u}{0}.
\end{split}
\end{equation}
\begin{probleme}
Je ne sais pas pourquoi tout d'un coup la dernière équation était commentée, et donc la phrase n'était pas finie.
\end{probleme}

\end{proof}

Let us now see a great consequence of the definition \eqref{eq:yGyGb}
\begin{proposition} \label{prop:XY_YX}
The formula
\begin{equation}
   XY-YX=[X,Y].
\end{equation}
links the formal product inside the Lie algebra and the Lie bracket.
\end{proposition}

\begin{proof}
From this, we can precise our definition of $XY$ when $X$, $Y\in\yG$. The product $XY$ acts on $\dpt{f}{G}{\eR}$ by
\[
  (XY)f=\DDsdd{f(e^{tX}e^{uY})}{t}{0}{u}{0}.
\]
We can get a more geometric interpretation of this. We know how to build a left invariant vector field $\tilde Y$ from any $Y\in\yG$: for each $g\in G$ we just have to define
\[
  \tY_g(f)=\Dsdd{f(gY(s))}{s}{0}.
\]
%
First remark: $\tY_g$ is precisely the object that previously named ``$gY$''. In order to construct the basis blocks of the formula $XY-YX=[X,Y]$, we turn our attention to $\tX_e\tY$. It is clear that $\tY(f)$ is a function from $G$ to $\eR$, so we can apply $\tX_e$ on it. If $X_t$ is a path which gives the vector $\tX_e$ (for example: $X_t=e^{tX}$), we have
\begin{equation}
  \tX_e(\tY(f))=\Dsdd{\tY(f)_{X_t}}{t}{0}\\
               =\DDsdd{f(X_tY(u))}{u}{0}{t}{0}\\
	       =\DDsdd{f(e^{tX}e^{uY})}{u}{0}{t}{0}.
\end{equation}
Thus we have: $XY=\tX_e\tY$, but it is clear that $[\tX,\tY]_e=\tX_e\tY-\tY_e\tX$. The claim reads now: $[\tX,\tY]_e=[X,Y]$. We can actually take it as de \emph{definition} of $[X,Y]$. It is done in \cite{Helgason}. The link with the definition in terms of successive derivations of $\AD_g(x)=gxg^{-1}$ is not trivial but it can be done.
\end{proof}



Now, we can give a powerful definition of the wedge for two $\yG$-valued $1$-forms. If $A$, $B\in\Omega^1(M,\yG)$ and $v$, $w\in\cvec(M)$, we define
\begin{equation}
  (A\wedge B)(v,w)=A(v)B(w)-A(w)B(v).
\end{equation}
For $A^2$, we find back the usual definition:
\[
  (A\wedge A)(v,w)=A(v)A(w)-A(w)A(v)=[A(v),A(w)].
\]
%
One can see that for any section $\dpt{\salpha}{\mU_{\alpha}}{P}$, we have
\begin{equation}\label{eq:1907r2}
   \salpha^*(A\wedge A)=(\salpha^*A)\wedge(\salpha^*A).
\end{equation}


\begin{lemma}
If $A$ and $B$ are two $\yG$-valued $1$-forms, one can make  ``simplifications'' as
\begin{equation}
 (Ag)\wedge(g^{-1} B)=A\wedge B.
\end{equation}
\label{lem:simplif}
\end{lemma}

\begin{proof}
We just have to prove that for $A$, $B\in\yG$, $(Ag)(g^{-1} B)=AB$ with definitions \eqref{eq_gXdefa} and \eqref{eq:yGyGb}. Remark that $Ag=\Dsdd{e^{sA}g}{s}{0}$, so
\[
  e^{tAg}=\exp(t\dsdd{e^{sA}g}{s}{0})=\exp(\dsdd{e^{stA}g}{s}{0})=e^{tA}g.
\]
Therefore
\[
  (Ag)(g^{-1} B)=\DDsdd{  e^{tAg}e^{ug^{-1} B}  }{t}{0}{u}{0}=\DDsdd{  e^{tA}gg^{-1} e^{uB}  }{t}{0}{u}{0}=AB.
\]
\end{proof}

\begin{lemma}
\begin{equation}
    F_{\beta}=dA_{\beta}+A_{\beta}^2.
\end{equation}
\end{lemma}

\begin{proof}
This is  a direct consequence of \eqref{eq:1907r2} and $[\sbeta^*,d]=0$.
\end{proof}

\section{Lie algebra valued differential forms}	\label{SecLiaAlgformval}
%+++++++++++++++++++++++++++++++++++++++++++++++++

An important particular case of vector valued forms is given by Lie algebra valued forms. That case appears for example in the connection theory over principal bundle\footnote{So in Maxwell and other gauge field theories.}. If $\omega$ and $\eta$ are elements of $\Omega^1(M,\mG)$ for some Lie algebra $\mG$, we define
\[
		(\omega\wedge\eta)(X,Y)=\omega(X)\otimes\eta(Y)-\omega(Y)\otimes\eta(X).
\]
Combining with the Lie bracket, we define\nomenclature[D]{$[\omega\wedge\eta]$}{Combination of the wedge product and the Lie bracket in the case of Lie algebra-valued forms}
\begin{equation}	\label{EqDefomegawedgebeta}
[\omega\wedge\eta](X,Y):=[\omega(X),\eta(Y)]-[\omega(Y),\eta(X)].
\end{equation}
Using the proposition~\ref{prop:XY_YX}, we often implicitly transforms the tensor product into a product \eqref{eq:yGyGb} and put
\begin{equation}	\label{EqAbuswesgeomom}
  (\omega\wedge\omega)(X,Y)=[\omega(X),\omega(Y)].
\end{equation}
Let us point out the fact that that kind of formula only holds for a ``wedge square'', but not for a general product $\omega\wedge\eta$. Remark that for $\omega\in\Omega^1(M,\mG)$ and $\beta\in\Omega^2(M,\mG)$, a simple computation of definition \eqref{EqDefwedgevecor} yields
\begin{equation}	\label{EqomwedgebetaXYZ}
(\omega\wedge\beta)(X,Y,Z)=\omega(X)\otimes\beta(Y,Z)-\omega(Y)\otimes\beta(X,Z)+\omega(Z)\otimes\beta(X,Y),
\end{equation}
so that, using the same trick as for equation \eqref{EqAbuswesgeomom}, we find
\[
  (\omega\wedge\beta-\beta\wedge\omega)(X,Y,Z)=[\omega(X),\beta(Y,Z)]-[\omega(Y),\beta(X,Z)]+[\omega(Z),\beta(X,Y)].
\]
But that expression is exactly what we find by exchanging the tensor product by Lie bracket in expression \eqref{EqomwedgebetaXYZ}. So we define
\begin{equation}	\label{EqDefCrochwedgedeux}
[\omega\wedge\beta]=\omega\wedge\beta-\beta\wedge\omega
\end{equation}
when $\omega\in\Omega^1(M,\mG)$ and $\beta\in\Omega^2(M,\mG)$. The reader should remark that this is what one would expect from generalisation of definition \eqref{EqDefomegawedgebeta}.
\section{Principal bundle}
%+++++++++++++++++++++++++

Let $M$ be a manifold and $G$, a Lie group whose unit is denoted by $e$. A $G$-\defe{principal bundle}{principal!bundle}\index{bundle!principal} on $M$ is a smooth manifold $P$, a smooth map $\dpt{\pi}{P}{M}$ and a right action of $G$ on $P$ denoted by $\xi\cdot g$ with $g\in G$ and $\xi\in P$ such that

\begin{itemize}
\item $\pi(\xi\cdot g)=\pi(\xi)$,
\item $\forall \xi\in\pi^{-1}(x)$, $\pi^{-1}(x)=\{\xi\cdot g\tq g\in G\}\simeq G$,
\item $\forall x\in M$, there exists a neighbourhood $\mU_{\alpha}$ of $x$ in $M$, a diffeomorphism $\dpt{\phi_{\alpha}}{\pi^{-1}(\mU_{\alpha})}{\mU_{\alpha}\times G}$ and a diffeomorphism $\dpt{\phi_{\alpha x}}{P}{G}$ such that

\begin{itemize}
\item $\phi_{\alpha}(\xi)=(x,\phi_{\alpha x}(\xi))$,
\item $\phi_{\alpha x}(\xi\cdot g)=\phi_{\alpha x}(\xi)\cdot g$.
\end{itemize}
\end{itemize}
The group $G$ is often called the \defe{structure group}{structure!group}. We suppose that the action is effective. We will sometimes use the notation $P(G,M)$ to precise that $P$ is a principal bundle over $M$ with structure group $G$.

\begin{lemma}\label{lem:phixh}
The map $\phi_{\alpha}^{-1}$ fulfills
\[
  \phi\alpha^{-1}(x,h)\cdot g=\phi_{\alpha}^{-1}(x,hg).
\]
\end{lemma}

\begin{proof}

From the definition of a principal bundle, any $\xi\in P$ can be written under the form $\xi=\phi_{\alpha}^{-1}(x,\phi_{\alpha x}(\xi))$ with $\phi_x$ satisfying $\phi_x(\xi\cdot h)=\phi_x(\xi)h$ for a certain function $\dpt{\phi_x}{P}{G}$.  We consider in particular $\xi=\phi_{\alpha}^{-1}(x,h)\cdot g$. Then $\xi\cdot g^{-1}=\phi_{\alpha}^{-1}(x,h)$. But $\xi\cdot g^{-1}=\phi_{\alpha}^{-1}(x,\phi_{\alpha x}(\xi)g^{-1})$, then $h=\phi{\alpha_x}(\xi)g^{-1}$ and $\phi_{\alpha x}(\xi)=hg$. So we have
\[
\xi=\phi_{\alpha}^{-1}(x,h)\cdot g=\phi_{\alpha}^{-1}(x,\phi_{\alpha x}(\xi))=\phi_{\alpha}^{-1}(x,hg).
\]

\end{proof}

Let
\[
   R=\{ (x,y)\in P\times P\tq x=y\cdot g\,\textrm{ for a certain $g\in G$}  \}.
\]

\begin{proposition}
The function $\dpt{u}{R}{G}$ defined by the condition
\[
  p\cdot u(p,q)=q.
\]
is differentiable.
\end{proposition}

\begin{proof}
Let $\mU$ be an open subset of $M$ and $\dpt{\sigma}{\mU}{P}$, a section. We consider a differentiable map $\dpt{\rho}{\pi^{-1}(\mU)}{G}$ such that $\rho(\xi\cdot g)=\rho(\xi)\cdot g$ and $\rho(\sigma(x))=e$. Such a map is given by
\[
   \rho(\xi)=\phi_x(\sigma(x))^{-1}\phi_x(\xi)
\]
where $x=\pi(\xi)$. We naturally define $R_{\mU}=R\cap( \pi^{-1}(\mU)\times\pi^{-1}(\mU) )$ and we pick $(\xi,\eta)\in R_{\mU}$. Let $s\in G$ be the one such that $\xi\cdot s=\eta$, so that $\rho(\xi)\cdot s=\rho(\eta)$. Then the restriction of $u$ to $R_{\mU}$ is given by $u(\xi,\eta)=\rho(\xi)^{-1}\rho(\eta)$ which makes $u|_{\mU}$ differentiable. Since this reasoning can be made on every chart open $\mU$, $u$ is differentiable everywhere on $P$.
\end{proof}

The following is a corollary of Leibnitz rule, proposition~\ref{Leibnitz}.
\begin{corollary}  \label{cor_PrincLeib}
If $P$ is a $G$-principal bundle and $v$, $a$ are curves in $P$ and $G$ respectively, we can consider the curve $u(t)=v(t)a(t)$. We have:
\[
         \dsdd{u(t)}{t}{0}=\dsdd{v(t)a(0)}{t}{0}+\dsdd{v(0)a(t)}{t}{0}.
\]
\end{corollary}
 The proof is direct. This result is often written as
\begin{equation}
               \dot{u}_t=\dot{v}_ta_t+v_t\dot{a}_t.
\end{equation}
A main application is
\begin{equation}\label{eq:rdotht}
  \Dsdd{ r\cdot h(t) }{t}{0}=\Dsdd{r\cdot e^{th'(0)}}{t}{0}.
\end{equation}

\subsection{Transition functions}
%--------------------------------


Let $(\mU_{\alpha},\phi_{\alpha})$ be a local trivialization of $P$. This induces transition functions $\dpt{g\bab}{\mU_{\alpha}\cap\mU_{\beta}}{G}$ defined by
\begin{equation}
	\begin{aligned}
		\phi_{\alpha}\circ\phi_{\beta}^{-1}\colon \mU_{\alpha}\cap\mU_{\beta}\times G&\to \mU_{\alpha}\cap\mU_{\beta}\times G \\
		(x,a)&\mapsto (x,g\bab(x)a).\label{eq:transi_princ}
	\end{aligned}
\end{equation}
Clearly, $g_{\alpha\alpha}=e$ and $g\bab g_{\alpha\beta}=e$ on $\mU_{\alpha}\cap\mU_{\beta}$. Then the triviality
\[
  \phi_{\alpha}\circ\phi_{\beta}^{-1}\circ\phi_{\beta}\circ\phi\bgamma^{-1}\circ\phi\bgamma\circ\phi_{\alpha}^{-1}=\id
\]
implies the compatibility conditions
\begin{equation}
g\bab g_{\beta\gamma} g_{\gamma\alpha}=e
\end{equation}
on $\mU_{\alpha}\cap\mU_{\beta}\cap\mU\bgamma$.

There is an inverse construction. Let $\{\mU_{\alpha}\tq\alpha\in I\}$ be an open covering of $M$ and $\dpt{g\bab}{\mU_{\alpha}\cap\mU_{\beta}}{G}$ a family of functions such that $g_{\alpha\alpha}=e$, $g\bab g_{\alpha\beta}=e$ on $\mU_{\alpha}\cap\mU_{\beta}$ and $g\bab g_{\beta\gamma}g_{\gamma\alpha}=e$ on $\mU_{\alpha}\cap\mU_{\beta}\cap\mU\bgamma$. Then the following construction gives a $G$-principal bundle whose transition functions are the $g\bab$'s.

\begin{itemize}
\item $\tilde{P}=\bigsqcup_{\alpha\in I}\mU_{\alpha}\times G$  (disjoint union),
\item if $(x,a)\in\mU_{\alpha}\times G$ and $(y,b)\in\mU_{\beta}\times G$, then $(x,a)\sim(y,b)$ if and only if $x=y$ and $b=g\bab(x)a$,
\item $\dpt{\pi}{\tilde{P}}{M}$ is defined by $\pi[(x,a)]=x$ where $[(x,a)]$ is the class of $(x,a)$ for~$\sim$,
\item the action is defined by $[(x,a)]\cdot g=[(x,ag)]$.
\end{itemize}

\begin{theorem}
Let $G$ be a Lie group; $M$, a differentiable manifold; $\{\mU_{\alpha}\}_{\alpha\in I}$, an open covering of $M$ and some functions $\dpt{\varphi\bab}{\mU_{\alpha}\cap\mU_{\beta}}{G}$ such that $\varphi\bab(x)=\varphi_{\alpha\gamma}(x)\varphi_{\gamma\beta}(x)$. Then there exists a principal bundle $P$ whose transition functions are the $\varphi_{\alpha}$'s for the covering $\{\mU_{\alpha}\}_{\alpha\in I}$.
\end{theorem}

\begin{proof}
We consider the topological space
\begin{equation}
    E=\bigcup_{\alpha\in I}(G\times\mU_{\alpha}\times I)
\end{equation}
where we put the discrete topology on $I$. Each $G\times\mU_{\alpha}\times\{\alpha\}$ is a manifold. Thus $E$ has a structure of differentiable manifold induced from the one of $G\times M$. We consider on $E$ the equivalence relation given by the following subset of $E\times E$:
\[
   R=\left\{\big(  (g,x,\alpha),(h,y,\beta)\big)\in E\times E\tq y=x\text{ and }  h=\varphi_{\alpha\beta}(x)g \right\}.
\]
We will show that $P=E/R$ has a structure of principal bundle. We begin by defining an action of $G$ on $P$ by
\[
  [ (g,x,\alpha)\cdot h ]=[ (gh,x,\alpha) ].
\]
In order to see that this definition is correct, let us consider $[g',x,\beta]=[g,x,\alpha]$. From the definition of the equivalence class, $g'=\varphi_{\alpha\beta}(x)g$. Then $[(g',x,\beta)]\cdot h=[(\varphi_{\alpha\beta}(g)gh,x,\beta)]$, and the form of $R$ shows that this is well $[(gh,x,\alpha)]$. Since the map $(g,h)\to gh$ is differentiable on $G$, the so defined action is a differentiable action of $G$ on $P$ and $G$ is a transformation group on $P$\quext{Faut voir comment ça correspond à la définition de l'autre texte.}.

If $[(g,x,\alpha)]=[(gh,x,\alpha)]$, then $gh=\varphi_{\alpha\alpha}g=g$ and $h=e$. So the action is effective.

Now we consider the quotient $P/G$. A typical element is
\[
   \overline{ (s,x,i) }=\{ [s,x,i]\cdot g\tq g\in G \}.
\]
The projection $\dpt{\pi}{P}{M}$, $[(s,x,\alpha)]\to x$ is well defined and we can consider $\dpt{\varphi}{P/G}{M}$, $\varphi\overline{(s,x,\alpha)}=x$. It provides a bijection between $P/G$ and $M$. So we can identify $P/G$ and $M$. Now we are going to show that $P$ endowed with the projection $\dpt{\pi}{P}{X}$ is a principal bundle.

We consider the map
		\begin{equation}
		\begin{aligned}
			h_{\alpha} \colon G\times\mU_{\alpha} &\to P\
			(g,x)&\mapsto \omega(g,x,\alpha)
		\end{aligned}
	\end{equation}
%
where $\dpt{\omega}{E}{P=E/R}$ is the canonical projection. Since
\[
  (\pi\circ h_{\alpha})(g,x)=(\pi\circ\omega)(g,x,\alpha)=\pi[(g,x,\alpha)]=x,
\]
the map $h_{\alpha}$ actually is $\dpt{h_{\alpha}}{G\times\mU_{\alpha}}{\pi^{-1}(\mU_{\alpha})}$. In order to see that $h_{\alpha}$ is surjective on $\pi^{-1}(\mU_{\alpha})$, let us take a general element of $\pi^{-1}(\mU_{\alpha})$ under the form $\omega(g,x,\beta)$ with $x\in\mU_{\alpha}\cap\mU_{\beta}$. Then $(g,x,\beta)\in[ (\varphi\bab(x)g,x,\alpha) ]$ and therefore $\omega(g,x,\beta)=h_{\alpha}(\varphi\bab(x)g,x)$. For the injectivity, remark that $\omega(g,x,\beta)=\omega(h,y,\alpha)$ implies $x=y$ and $h=\varphi_{\beta\beta}(x)g=g$. In particular, $h_{\alpha}(g,x)=h_{\alpha}(h,y)$ implies $x=y$ and $g=h$.

Now we will prove that the inverse of $h_{\alpha}$ is continuous. For this we consider an open set $\Omega\subset G\times\mU_{\alpha}$ and we have to show that $h_{\alpha}(\Omega)$ is open in $\pi^{-1}(\mU_{\alpha})$.

We recall the \defe{quotient topology}{topology!quotient}: if $A$ is a topological space with an equivalence relation $\sim$ and the canonical projection $\dpt{\varphi}{A}{A/\sim}$, then $V\subset A/\sim$ is open if and only if $\varphi^{-1}(V)\subset A$ is open. So in our case, we have to check the openness of $V=\omega^{-1}( h_{\alpha}(\Omega) )$ in $E$. We consider the open covering
\[
  \{ G\times\mU_{\alpha}\times\{\alpha\} \}_{\alpha\in I}
\]
of $E$ and we will show that the intersection of $V$ with any of these open set is open. We have to show that
$\omega^{-1}\big(  h_{\alpha}(\Omega)\cap (G\times\mU_{\alpha}\times\{\beta\})  \big)$ is open for any $\beta\in I$. For this, we define a map $\dpt{\alpha}{G\times(\mU_{\alpha}\cap\mU_{\beta})\times\{\beta\}}{G\times\mU_{\alpha}}$ by
\begin{equation}
  \alpha_{\beta}(g,x,\beta)=(\varphi\bab(x)g,x)
\end{equation}
which is continuous. The set $(h_{\alpha}\circ\alpha_{\beta})^{-1}(h_{\alpha}(\Omega))=\alpha^{-1}_{\beta}(\Omega)$ is open because $h_{\alpha}\circ\alpha_{\beta}$ is the restriction of $\omega$ to $G\times (\mU_{\alpha}\cap\mU_{\beta})\times\{\beta\}$. Then $h_{\alpha}$ is an homeomorphism from $G\times\mU_{\alpha}$ tp $\pi^{-1}(\mU_{\alpha})$. Since it is build from differentiable functions, it is moreover a diffeomorphism.

So we have a chart system $\{ (h_{\alpha},\mU_{\alpha}) \}_{\alpha\in I}$ where $h_{\alpha}$ fulfils the ``good'' properties with respect to $\pi$. It remains to be proved that the $\varphi\bab$'s are the transition functions and that $\pi^{-1}(\pi(\xi))=\xi\cdot G$ for every $\xi\in P$. We begin by the latter. For $\xi=[(g,x,\alpha)]$, $\pi(\xi)=x$ and we have to study the set
\[
  \pi^{-1}(x)=\{ [(h,x,\beta)]\tq h\in G,\beta\in I \}.
\]
Clearly, $[(h,x,\beta)]\cdot G\subset\pi^{-1}(x)$. The fact that there is nothing else than $[(h,x,\beta)]\cdot G$ in $\pi^{-1}(x)$ is seen by
\[
  [h,x,\beta]=[\varphi\bab(x)g,x,\alpha]\in[(h,x,\alpha)]\cdot G.
\]

In order to check the change of charts, let us consider $g'=h_{\beta,x}^{-1}\circ h_{\alpha,x}(g)$ where
\begin{equation}
  h_{\alpha,x}(g)=h_{\alpha}(g,x)=\omega(g,x,\alpha).
\end{equation}
The fact that $h_{\beta}(g',x)=g_{\alpha}(g,x)$ concludes the proof. To see this fact, remark that $h_{\beta,x}(h^{-1}_{\beta,x}\circ h_{\alpha,x}(g))=h_{\alpha,x}(g)$, so that $h_{\alpha}(g',x)=h_{\alpha}(g,x)$ implies $\omega(g',x,\beta)=\omega(g,x,\alpha)$ which proves that $g'=\varphi\bab(g)$.
\end{proof}

The \defe{trivial bundle}{trivial!principal bundle} is simply $P=M\times G$ and $\pi(x,g)=x$ with the action $(x,a)\cdot g=(x,ag)$.

\subsection{Morphisms and such\texorpdfstring{\ldots}{...}}
%----------------------------------

An \defe{homomorphism}{homomorphism!of principal bundle} between $P(G,M)$ and $P'(G',M')$ is a differentiable map $\dpt{h}{P}{P'}$ such that $\forall \xi\in P,g\in G$,
\begin{equation}\label{eq:def_princ_homo}
   h(\xi\cdot g)=h(\xi)\cdot h_G(g)
\end{equation}
where $\dpt{h_G}{G}{G'}$ is a Lie group homomorphism. From the definition, $h$ maps a fiber to only one fiber, but it is not specially surjective on any fiber. So $h$ induces a homomorphism $\dpt{h_M}{M}{M'}$ such that $\pi'\circ h=h_M\circ\pi$.

An \defe{isomorphism}{isomorphism!of principal bundle} is a homomorphism $\dpt{g}{P(G,M)}{P'(G',M')}$ such that

\begin{itemize}
\item $h_P$ is a diffeomorphism $P\to P'$,
\item $h_G$ is a Lie group homomorphism $G\to G'$, and
\item $h_M$ is a diffeomorphism $M\to M'$.
\end{itemize}

A principal bundle is \defe{trivial}{trivial!principal bundle} if one can find an isomorphism $\dpt{h}{G\times M}{P}$ such that $\pi\circ h=\id\circ\pr_2$, i.e. the following diagram commutes:
\begin{equation}\label{diag:princ_triv}
\xymatrix{ G\times M \ar[d]_{\pr_2} \ar[r]^h & P\ar[d]^{\pi} \\M \ar[r] _{\id}&M}
\end{equation}
We say that $P$ is \defe{locally trivial}{locally!trivial!principal bundle} if for every $x\in M$, there exists an open neighbourhood $\mU$ in $M$ such that $\pi^{-1}(\mU)$ endowed with the induced structure of principal bundle is trivial.

\subsection{Frame bundle: first}\label{pg:frame_bundle}
%--------------------------------

In the ideas, the building of a vector bundle is just to put a vector space on each point of the base manifold. A principal bundle is to put something on which a group acts on each point. If you have a vector bundle on a manifold, you can consider, on each point $x\in M$, the set of all the basis of the fiber $E_x$ over $x$. The group $GL(r,\eK)$ naturally acts on this set which becomes a candidate to be a $GL(r,\eK)$-principal bundle.

More formally, we consider a vector bundle $\dptvb{F}{p}{M}$, and for each $x$, the set of the basis of the vector space $F_x=p^{-1}(x)$. We define
\[
  P=\bigcup_{x\in M}(\textrm{basis of $F_x$}).
\]
We naturally consider the projection $\dpt{\pi}{P}{M}$, $\pi(b_x)=x$ if $b_x$ is a basis of $F_x$.

Let $\dpt{\phi^F_{\alpha}}{p^{-1}(\mU_{\alpha})}{\mU_{\alpha}\times\eK^r}$ be a local trivialization of $F$, and $\{\overline{e}_1,\ldots,\overline{e}_r\}$, the canonical basis of $\eK^r$. We naturally define
\[
  \ovS_{\alpha i}(x)=\phi^F_{\alpha}{}^{-1}(x,\overline{e}_i).
\]
The set $\{\ovS_{\alpha 1}(x),\ldots,\ovS_{\alpha r}(x)\}$ is a ``reference''{} basis of $F_x$ with respect to the trivialization $\phi_{\alpha}$. If we choose another basis $\{\ovv_1,\ldots\ovv_r\}$ of $F_x$, we can find a matrix $A\in GL(r,\eK)$ such that $\ovv_k=A^l_k\ovS_{\alpha l}(x)$. This gives a bijection
\begin{equation}
	\begin{aligned}
		\phi_{\alpha}^P\colon \pi^{-1}(\mU_{\alpha})&\to \mU_{\alpha}\times GL(r,\eK) \\
		(\ovv_1,\ldots,\ovv_r)&\mapsto (x,A).
	\end{aligned}
\end{equation}
One can give to $P$ a $GL(r,\eK)$-principal bundle structure such that the $\phi_{\alpha}^P$ are diffeomorphism.

Let $(\mU_{\alpha},\phi_{\alpha}^F)$ be a local trivialization of  $F$ and $\dpt{g\bab^F}{\mU_{\alpha} \cap\mU_{\beta}}{GL(r,\eK)}$. In this case, $(\mU_{\alpha},\phi_{\alpha}^P)$ is a trivialization of $P$ whose transition function is $g\bab^P=g\bab^F$. Indeed
\[
  \phi_{\alpha}^P\circ\phi_{\beta}^P{}^{-1}(x,A)=\phi_{\alpha}^P(\{\ovv_1,\ldots,\ovv_r\})
\]
where $\ovv_s=(\phi^F_{\beta})^{-1}(x,A^l_s\overline{e}_l)$. In order to see it, recall that $\ovv_s=A^l_s\ovS_{\alpha l}(x)$ and that $\phi_{\alpha}^F{}^{-1}(x,\overline{e}_s)=\ovS_{\alpha s}(x)$. Then
\[
   \ovv_s=(\phi_{\beta}^F)^{-1}(x,A^l_s\overline{e}_l)=A^l_s\ovS_{\alpha s}(x).
\]
On the other hand, from the definition of $\phi_{\beta}^P$, the basis $(\phi_{\beta}^P)^{-1}(x,A)$ is the one obtained by applying $A$ on $S$. With all this,
\begin{equation}
\begin{split}
  \phi_{\alpha}^P\circ(\phi_{\beta}^P)^{-1}(x,A)&=\phi_{\alpha}^P\{  (\phi_{\beta}^F)^{-1}(x,A^l_s\overline{e}_l)\}_{s=1,\ldots r}\\
                 &=\phi_{\alpha}^P\{(\phi_{\alpha}^F)^{-1}\circ(\phi_{\alpha}^E\circ\phi_{\beta}^F{}^{-1})(x,A^l_s\overline{e}_l)  \}_{s=1,\ldots r}\\
		 &=\phi_{\alpha}^P\{  (\phi^E_{\alpha})^{-1}(x,g\bab^F(x)^s_iA^l_s\overline{e}_l   ) \}_{i=1,\ldots r}\\
		 &=(x,g^F\bab(x)A).
\end{split}
\end{equation}
The last product $g^F\bab(x)A$ is a matricial product.

\subsection{Frame bundle: second} \label{subsec_frbundle}
%------------------------------

\subsubsection{Basis}\label{subsubsecframebundle}
%////////////////////////////////////////////////

If $M$ is a $m$-dimensional manifold, a \defe{frame}{basis!of $T_xM$} of $T_xM$ is an isomorphism $\dpt{b}{\eR^m}{T_xM}$. In our purpose, we will always deal with (pseudo)Riemannian manifold. So, the tangents spaces $T_xM$ comes with a metric, and we ask a frame to be isometric. In other words, we ask $b$ to be an isometry from $(\eR^m,\cdot)$ to $(T_xM,g_x)$, where the dot denotes the (pseudo)euclidian product on $\eR^m$. Such a frame is given by a base point $x$ of $M$ and a matrix $S$ in $\SO(g_x)$:
\begin{equation}
                 \label{r1504e1}b(v)=(Sv)^i(\partial_i)_x,
\end{equation}
if the vector $v$ is written as $v=v^i\oui$ in the canonical orthogonal frame $\{\oui\}$ of $\eR^m$ and $\SO(g_x)$ is the set of the $m\times m$ matrix $A$ such that $A^tg_xA=g_x$.

This frame intuitively corresponds to the basis of $T_xM$ (see as a ``true''\ vector space) that we would have written by $\{Se_i\}_x$ if $e_i=\dsd{}{x^i}$. In order to follow this idea, we will effectively denote by $\baz{S}{x}$ the map $\dpt{b}{\eR^m}{T_xM}$ given by \eqref{r1504e1}.

We will often write the frame $b$ as $\baz{b}{x}$, making no differences in notation between the $b$ of $\SO(M)$ and the $b$ of $\SO(g_x)$ which implement it.

\begin{remark}
One has to distinguish a \emph{frame} and a \emph{basis}: a basis is only a free and generating set while a frame can be interpreted as an ordered basis.
\end{remark}


\subsubsection{Construction}
%////////////////////////////

We just saw how to build a frame bundle over a manifold. One can get another expression of the frame bundle when we express a basis of $T_xM$ by means of an isomorphism between $\eR^n$ and $T_xM$. If $M$ is a $n$-dimensional manifold, a \defe{frame}{frame} at $x$ is an ordered basis
\[
   b=(\overline{ b }_1,\ldots,\overline{ b }_n)
\]
of $T_xM$. It is clear that any frame defines an isomorphism (linear bijective map)
\begin{equation}
\begin{aligned}
	\tilde{b}\colon\eR^n&\to T_xM \\
    e_i&\mapsto \overline{ e_i }
\end{aligned}
\end{equation}
where $\{e_i\}$ is the canonical basis of $\eR^n$. It is also clear that any isomorphism gives rise to a frame. Then we see a frame of $M$ at $x$ as an isomorphism $\dpt{\tilde{b}}{\eR^n}{T_xM}$. Let $B(M)_x$ be the of all the frames of $M$ at $x$; we define
\[
   B(M)=\bigcup_{x\in M}B(M)_x.
\]
For all $b\in B(M)_x$, we define $p_B(b)=x$ and the action $B(M)\times GL(n,\eR)\to B(M)$ by $b\cdot g=(\overline{ b }'_1,\ldots,\overline{ b }'_n)$ where
\begin{equation}
  \overline{ b }'_j=\overline{ b }_i\bghd{g}{i}{j}.
\end{equation}
It is easy to see that $\dpt{\widetilde{b\cdot g}=\tilde{b}\circ g}{\eR^n}{T_xM}$. So we can give to
\begin{equation}
\xymatrix{
    GL(n,\eR)  \ar@{~>}[r] & B(M) \ar[d]^{p_B}\\& M
  }
\end{equation}
a structure of principal bundle\footnote{Much more details and proofs are given in \cite{Naber}.}. If $(\mU_{\alpha},\varphi_{\alpha})$ is a local coordinate chart on $M$, we define
		\begin{equation}
		\begin{aligned}
			\tilde{\varphi} \colon p_B^{-1}(\mU_{\alpha}) &\to \varphi_{\alpha}(\mU_{\alpha})\times GL(n,\eR)\
			b&\mapsto (\varphi_{\alpha}(x),A(b))
		\end{aligned}
	\end{equation}
where $A(b)\in GL(n,\eR)$ is defined by the condition $\overline{ b }_j=\bghd{A}{j}{i}\partial_i|_x$. The matrix $A(b)$ is the one which transforms the canonical basis (in the trivialization $\varphi_{\alpha}$) into $b\in B(M)_x$. That's for the principal bundle structure.

The manifold structure of $B(M)$ is given by $\dpt{\Phi_{\alpha}}{p_B^{-1}(\mU_{\alpha})}{\mU_{\alpha}\times GL(\eR)}$,
\begin{equation}
\begin{split}
  \Phi(b)&=(\varphi_{\alpha}^{-1}\times \id|_{GL(n,\eR)})\circ\tilde{\varphi}(b)\\
         &=(x,A(b))\\
         &=(p_B(b),A(b)).
\end{split}
\end{equation}
It fulfils $A(b\cdot g)=A(b)\cdot g$. A section $\dpt{s}{\mU_{\alpha}}{B(M)}$ is sometimes called a \defe{moving frame}{moving frame}\index{frame!moving} over $\mU_{\alpha}$.

Frame bundle over $\eR^2$ is given as example in page \pageref{Pg_exempleRdeux}
%
% \subsection{Space-time}\label{subsec:space_time}
% %----------------------
%
%
% We say that the basis $\{\gb_0,\ldots,\gb_3\}$ of $T_xM$ is oriented, time oriented and orthonormal if $g(\gb_i,\gb_j)=\eta_{ij}$ (pointwise) and if $\gb_0$ is time-like and future directed. All this is devoted to build the frame bundle
%
%
% \[
% \xymatrix{
%     \Lpf  \ar@{~>}[r] & L(M) \ar[d]^{p_L} \\
%     &M.
%   }
% \]
%
% From discussion in \autoref{subsec:sym_nature}, we know that, in physics, the relevant group is \emph{not} the Lorentz group $\Lpf$, but $\SLdc$ . So we want to build a $\SLdc$-principal bundle which ``fit''{} as deeply as possible the bundle $L(M)$. It is done with a \defe{spin structure}{spin!structure} which is a principal bundle
%
% \[
% \xymatrix{
%     \SLdc  \ar@{~>}[r] & S(M) \ar[d]^{p_S} \\
%     &M.
%   }
% \]
% with a map $\dpt{\lambda}{S(M)}{L(M)}$ such that $p_L\circ\lambda=p_S$ and $\lambda(\xi\cdot g)=\lambda(\xi)\cdot\mSpin(g)$. See \autoref{sec:spin_str} and \autoref{sec:incl_Lorentz}.

\subsection{Sections of principal bundle}
%----------------------------------------

A \defe{section}{section!of principal bundle} of a $G$-principal bundle is a smooth map $\dpt{s}{M}{P}$ such that $s(x)\in\pi^{-1}(x)$ for any $x\in M$. A trivialization $\phi_{\alpha}^P$ $P$ on $\mU_{\alpha}$ defines a section of $P$ over $\mU_{\alpha}$ by
\[
   \sigma_{\alpha}(x)=(\phi_{\alpha}^P)^{-1}(x,e)
\]
where $e$ is the neutral of the group. In the inverse sense, we have the following:

\begin{proposition}
If $\dpt{\sigma_{\alpha}}{\mU_{\alpha}}{P}$ is local section of $P$ over $\mU_{\alpha}\subset M$, then the definition $\phi_{\alpha}^P(\xi)=(x,a)$ if $\xi=\sigma_{\alpha}(x)\cdot a$ is a local trivialization.
\end{proposition}

\begin{proof}
The function $\phi_{\alpha}^P$ is well defined because $\xi\in\pi^{-1}(\mU_{\alpha})$ implies the existence of a $x\in\mU_{\alpha}$ such that $\xi\in\pi^{-1}(x)=\{\xi\cdot g\}\simeq G$. For this $x$, there exists a $g\in G$ such that $\xi=\sigma_{\alpha}(x)\cdot g$.

Now we prove that the couple $(x,a)$ is unique in the sense that $s_{\alpha}(x)\cdot a=\sigma_{\alpha}(y)\cdot b$ implies $(x,a)=(y,b)$. The left hand side belongs to $\pi^{-1}(x)$ while the right one belongs to $\pi^{-1}(y)$. Then $x=y$. The condition $\pi^{-1}(x)\simeq G$ imposes the unicity of the $g$ making $\xi=\eta\cdot g$ for each couple, $\xi,\eta\in\pi^{-1}(x)$.
\end{proof}

If $\sigma$ and $\sigma'$ are two sections of the same principal bundle $P$, then there exists a differentiable map $\dpt{f}{M}{G}$ such that $\sigma'(x)=\sigma(x)\cdot f(x)$. So all the sections can be deduced from only one and multiplication by such a $f$.

\begin{theorem}
If $\dpt{\pi}{P(G,M)}{M}$ is a principal bundle, then the four following propositions are equivalent:
\begin{enumerate}
\item\label{enuymai} $P$ is trivial,
\item\label{enuymaii} $P$ has a global section,
\item\label{enuymaiii} there exists a differentiable map $\dpt{\gamma}{P}{G}$ such that $\gamma(\xi\cdot g)=g^{-1}\gamma(\xi)$ for all $\xi\in P$ and $g\in G$,
\item\label{enuymaiv} there exists a differentiable map $\dpt{\rho}{P}{G}$ such that $\rho(\xi\cdot g)=\rho(\xi)g$.
\end{enumerate}
\label{ThoYPrincBTrivSect}
\end{theorem}

\begin{proof}
\subdem{\ref{enuymai}$\Rightarrow$~\ref{enuymaii}}
The diagram \eqref{diag:princ_triv} commutes and
		\begin{equation}
		\begin{aligned}
			\tau \colon M &\to G\times M\
			x&\mapsto (e,x)
		\end{aligned}
	\end{equation}
is a local section of $G\times M$. From it we build the following global section of $P$:
		\begin{equation}
		\begin{aligned}
			\sigma \colon M &\to P\
			x&\mapsto h(e,x).
		\end{aligned}
	\end{equation}
 This is injective because $\pi\circ h=\pr_2$ and differentiable because this is a composition of $\xdp{x}{(e,x)}$ and $\xdp{(g,x)}{h(g,x)}$.
\subdem{\ref{enuymaii}$\Rightarrow$~\ref{enuymai}}
The principal bundle $P$ admits a global section $\dpt{\sigma}{M}{P}$. From it, we can build the differentiable map
		\begin{equation}
		\begin{aligned}
			h \colon G\times M &\to P\
			(g,x)&\mapsto \sigma(x)\cdot g
		\end{aligned}
	\end{equation}
which satisfies $h(gh,x)=h(g,x)\cdot h$ and $\pi\circ(g,x)=x$. First we show that $h$ is a fiber homomorphism and an isomorphism between $P$ and $G\times M$ so that $P$ is trivial. For this remark that
\[
  g(gh,x)=g(g,x)\cdot h=\sigma(x)\cdot gh,
\]
hence equation \eqref{eq:def_princ_homo} reduces to $h( (g,x)\cdot h )=h(g,x)\cdot h_G(h)$ which is true with $h_G=\id$. Moreover $\dpt{h}{G\times M}{P}$ is bijective because $\sigma(\pi(\xi))$ belongs to the fiber of $\xi\in P$, therefore there is one and only one $\gamma(\xi)=u(\xi,\sigma(\pi(\xi)))$ such that $\xi\cdot\gamma(\xi)=(\sigma\circ\pi)\xi$. The inverse map is
		\begin{equation}
		\begin{aligned}
			\theta \colon P &\to G\times M\
			\xi&\mapsto (\gamma(\xi),\pi(\xi))
		\end{aligned}
	\end{equation}
which is differentiable because $\gamma$ and $\pi$ are. So far we see that $h$ and $h^{-1}$ are differentiable. Then $h$ is an isomorphism between $P$ and $G\times M$.

\subdem{\ref{enuymaii}$\Rightarrow$~\ref{enuymaiii}}
Let $\sigma$ be the global section and define
		\begin{equation}
		\begin{aligned}
			\gamma \colon P &\to G\
			\xi&\mapsto u(\xi,(\sigma\circ\pi)\xi)
		\end{aligned}
	\end{equation}
where $\dpt{u}{R}{G}$ is the map defined by the condition $\xi\cdot(\xi,\eta)=\eta$. The map $\gamma$ is differentiable and we have to prove that $\gamma(\xi\cdot g)=g^{-1}\gamma(\xi)$. Since $\xi\cdot \gamma(\xi)=\sigma\circ\pi(\xi)$,
\[
  \gamma(\xi\cdot g)=u(\xi\cdot g,(\sigma\circ\pi)(\xi\cdot g))=u(\xi\cdot g,(\sigma\circ\pi)(\xi)).
\]
But $(\xi\cdot g)(g^{-1}\gamma(\xi))=\xi\cdot\gamma(\xi)=x$. So $\gamma(\xi\cdot g)=u(\xi\cdot g,x)$. Thus $\gamma(\xi\cdot g)=g^{-1}\gamma(\xi)$.

\subdem{\ref{enuymaiii}$\Rightarrow$~\ref{enuymaii}}
The given map $\gamma$ fulfils $\xi\cdot g\gamma(\xi\cdot g)=\xi\cdot(\xi)$, so
		\begin{equation}
		\begin{aligned}
			\varphi \colon P &\to P\
			\xi&\mapsto \xi\cdot(\xi)
		\end{aligned}
	\end{equation}
is just function of the class of $\xi$, thus we have a section $\dpt{\sigma'}{P/G}{P}$, but we know that $P/G$ and $M$ are isomorphic.

\subdem{\ref{enuymaiii}$\Rightarrow$~\ref{enuymaiv}}
Let us define $\dpt{\rho}{P}{G}$ by $\rho=J\circ\gamma$ with $J(g)=g^{-1}$, thus $\rho(\xi)=\gamma(\xi)^{-1}$ and
\[
  \rho(\xi\cdot g)=\gamma(\xi\cdot g)^{-1}=(g^{-1}\gamma(\xi))^{-1}=\gamma(\xi)^{-1} g=\rho(\xi)g.
\]
\subdem{\ref{enuymaiv}$\Rightarrow$~\ref{enuymaiii}} The proof is just the same with $\rho=J\circ\rho$.
\end{proof}

\begin{definition}
A section $\psi\in\Gamma(P,TP)$ is \defe{$G$-equivariant}{equivariant!vector field on principal bundle} when
\[
  d\tau_{g}\psi(\xi)=\psi(\xi\cdot g).
\]
\label{DefEqVectPrinc}
\end{definition}
Be careful: this \emph{does not} define equivariant sections of the principal bundle.
\subsection{Equivalence of principal bundle}
%-------------------------------------------

Two principal bundles $\dpt{\pi}{P}{M}$ and $\dpt{\pi'}{P'}{M}$ are \defe{equivalent}{equivalence!of principal bundle} if there exists a diffeomorphism $\dpt{\varphi}{P}{P'}$ such that

\begin{itemize}
\item  $\pi'\circ\varphi=\pi$
\item $\varphi(\xi\cdot g)=\varphi(\xi)\cdot g$.
\end{itemize}

If $\{\mU_{\alpha}\}_{\alpha\in I}$ is an open covering of $M$ on which we have trivializations $\phi_{\alpha}$ of $P$ and $\psi_{\alpha}$ of $P'$, the diffeomorphism $\varphi$ induces some functions $\dpt{\lambda}{\mU_{\alpha}}{G}$ by setting
\[
   (\phi_{\alpha}\circ\varphi^{-1}\circ\psi_{\alpha}^{-1})(x,a)=(x,\lambda_{\alpha}(x)a).
\]
This definition works because from the definitions of principal bundle and equivalence, one sees that $(\phi_{\alpha}\circ\varphi^{-1}\circ\psi_{\alpha}^{-1})(x,\cdot)=(x,\cdot)$.

\subsubsection{Transition functions}
%///////////////////////////////////

We have some transition functions for $P$ and $P'$ given by equations
\[
 \begin{aligned}
   (\phi_{\alpha}\circ\phi_{\beta}^{-1})(x,g)&=(x,g\bab(x)g)\\
   (\psi_{\alpha}\circ\psi_{\beta}^{-1})(x,g)&=(x,g'\bab(x)g).
 \end{aligned}
\]
Now, we want to know what is $g'\bab$ in function of $g\bab$. First remark that $(\psi_{\alpha}\circ\varphi\circ\phi_{\alpha}^{-1})(x,a)=(x,\lambda_{\alpha}(x)^{-1})a$, and next, compute
\begin{equation}
\begin{split}
  (x,g\bab(x)a)a&=(\psi_{\alpha}\circ\varphi\circ\phi_{\beta}^{-1}\circ\phi_{\beta}\circ\varphi^{-1}\circ\psi_{\beta}^{-1})(x,a)\\
                &=(\psi_{\alpha}\circ\varphi\circ\phi_{\beta}^{-1})(x,\lambda_{\beta}(x)a)\\
		&=(\psi_{\alpha}\circ\varphi\circ\phi_{\alpha}^{-1}\circ\phi_{\alpha}\circ\phi_{\beta}^{-1})(x,\lambda_{\beta}(x)a)\\
		&=(x,\lambda_{\alpha}(x)^{-1} g\bab(x)\lambda_{\beta}(x)a).
\end{split}
\end{equation}
Then
\begin{equation}
    g\bab(x)=\lambda_{\alpha}(x)^{-1} g\bab(x)\lambda_{\beta}.
\end{equation}
One can show that if two principal bundle have transition functions whose fulfill this condition, they are equivalent. A $G$-principal bundle is \defe{trivial}{trivial!principal bundle} if it is equivalent to the one given by $\dpt{\pi_1}{M\times G}{M}$.

\subsection{Reduction of the structural group}
%---------------------------------------------

We say that a principal bundle $P(G,M)$ is \defe{reducible}{reducible!principal bundle} when there exists a principal bundle $P'(H,M)$ such that

\begin{itemize}
\item $H$ is a subgroup of $G$,
\item there exists an homeomorphism $\dpt{h}{P'}{P}$ such that $\dpt{h_G}{H}{G}$ is an injective homomorphism.
\end{itemize}

In this case we say that $G$ is reducible to $H$ and that $P'$ is a reduced principal bundle.

\begin{theorem}
If $P$ is a principal bundle over $M$, the structural group $G$ is reducible to the Lie subgroup $H$ if and only if there exists an open covering $\{ \mU_i \}_{i\in I}$ of $M$ and transition functions $\varphi_{ij}$ taking their values in $H$.
\end{theorem}
\begin{proof}
No proof.
\end{proof}
The following comes from \cite{Dieu4}. Let us consider the principal bundle
\begin{equation}
\xymatrix{%
   G \ar@{~>}[r]		&	P\ar[d]^{\pi_P}\\
   				&	M
 }
\end{equation}
and $H$, a closed subgroup of $G$. We denote by $j\colon H\to G$ the inclusion map. The principal bundle
\begin{equation}
\xymatrix{%
   H \ar@{~>}[r]		&	Q\ar[d]^{\pi_Q}\\
   				&	M
 }
\end{equation}
is a \defe{reduction}{reduction of a principal bundle} of $P$ to the group $H$ if there exists a map $u\colon Q\to P$ such that $\pi_P\circ u=\pi_Q$ and $u(\xi\cdot h)=u(\xi)\cdot j(h)$. In this case, $u$ is an embedding\quext{plongement} of $Q$ in $P$ and the image is a closed submanifold of $P$.

Let $M$ be a $n$-dimensional manifold and $B(M)$ be its frame bundle. This is a $\GL(n,\eR)$-principal bundle. If $G$ is a closed subgroup\footnote{Typically $\SO(p,q)$ or $\SO_0(p,q)$.} of $\GL(n,\eR)$, a \defe{$G$-structure}{$G$-structure} is a reduction of $B(M)$ to $G$.

%---------------------------------------------------------------------------------------------------------------------------
					\subsection{Density}
%---------------------------------------------------------------------------------------------------------------------------

A \defe{density}{density} on a $d$-dimensional manifold $M$ is a section of the principal bundle whose fiber $P_x$ over $x\in M$ is the space of homogeneous non vanishing maps
\begin{equation}
\rho\colon \Wedge^dT_xM\to \eR^*_+
\end{equation}
such that $\rho(\lambda v)=| \lambda |\rho(v)$ for every $\lambda\in\eR$ and $v\in\Wedge^d T_xM$.


\section{Associated bundle}  \index{bundle!associated}
%++++++++++++++++++++++++++

Let $\dpt{\pi}{P}{M}$ be a $G$-principal bundle and $\dpt{\rho}{G}{GL(V)}$, a representation of $G$ on a vector space $V$ (on $\eK=\eR$ or $\eC$) of dimension $r$.

The associated bundle $\dptvb{E=P\times_{\rho} V}{p}{M}$ is defined as following. On $P\times V$, we consider the equivalence relation
\[
   (\xi,v)\sim(\xi\cdot g,\rho(g^{-1})v)
\]
for $g\in G$, $\xi\in P$ and $v\in V$. Then we define

\begin{itemize}
\item $E=P\times_{\rho} V:=P\times V/\sim$,
\item $p[(\xi,v)]=\pi(\xi)$
\end{itemize}
where $[(\xi,v)]$ is the class of $(\xi,v)$ in $P\times V$.

If $\phi^P_{\alpha}(\xi)=(\pi(\xi),a(\xi))$ is a  trivialization of $P$ on $\mU_{\alpha}$, then
\begin{equation}\label{eq:triv_P_E}
\phi^E[(\xi,v)]=(\pi(\xi),\rho(a)v)
\end{equation}
is a trivialization of $E$.

In order to see that it is a good definition, let us consider $(\eta,w)\sim(\xi,v)$. It immediately gives the existence of a $g\in G$ such that $\eta=\xi\cdot g$ and $w=\rho(g^{-1})v$. Then $\phi^E[ (\xi\cdot g,\rho(g^{-1})v) ]=( \pi(\xi\cdot g),\rho(b)\rho(g^{-1})v )$.  From the definition of $\phi^E$, the vector $b$ is given by $\phi^P(\xi\cdot g)=( \pi(\xi\cdot g),b )$, and the definition of a principal bundle gives $b=\phi_{\pi(\xi)}(\xi\cdot g)=\phi_{\pi(\xi)}(\xi)\cdot g=ag$. The fact that $\rho$ is a homomorphism makes $\rho(ag)\rho(g^{-1})=\rho(a)v$ and $\phi^E$ is well defined.

Let $G$ be a Lie group, $\rho$ a representation of $G$ on $V$ and $M$, a manifold. We consider $\dptvb{P=M\times G}{\pr_1}{M}$, the trivial $G$-principal bundle on $M$. Then $\dptvb{E=P\times_{\rho}V}{p}{M}$ is \defe{trivial}{trivial!principal bundle}, i.e. we can build a $\dpt{\varphi}{P\times_{\rho} V}{M\times V}$ such that $\pr_1\circ\varphi=p$. It is rather easy: we define
\[
  \varphi\big[ \big((x,g),v\big) \big]=(x,\rho(g)v).
\]
It is easy to see that $(\pr_1\circ\varphi)[ (x,g),v ]=x$ and $p[(x,g),v]=\pr_1(x,g)=x$.

\subsection{Transition functions}
%--------------------------------

\begin{proposition}
Let $(\mU_{\alpha},\phi_{\alpha}^P)$ be a trivialization of $\dptvb{P}{\pi}{M}$ whose transition functions are $\dpt{g\bab}{\mU_{\alpha}\cap\mU_{\beta}}{G}$. Then $(\mU_{\alpha},\phi^E_{\alpha})$ given by \eqref{eq:triv_P_E} is a local trivialization of $\dptvb{E}{p}{M}$ whose transition functions $\dpt{g\bab^E}{\mU_{\alpha}\cap\mU_{\beta}}{GL(\dim V,\eK)}$ are given by
\[
   g\bab^E(x)=\rho(g\bab^P(x)).
\]

\end{proposition}

\begin{proof}
If we write $a:=\phi^E_{_{\beta} x}(\pi^{-1}(x))$, we have $\phi_{\beta}^P(\pi^{-1}(x))=(x,a)$ and $\phi_{\alpha}^E\circ(\phi_{\beta}^E)^{-1}(x,v)=\phi^E_{\alpha}[ (\pi^{-1}(x),\rho(a)^{-1} v) ]$. So,
\begin{equation}
\begin{split}
\phi_{\alpha}^E[ (\pi^{-1}(x),\rho(a)^{-1} v) ]&=\Big(x,
                            \rho\big(\phi_{\alpha x}(\pi^{-1}(x))\big)\rho\big(\phi_{\beta x}(\pi^{-1}(x))\big)^{-1} v   \Big)\\
			&=\Big(   x,\rho\big(\phi_{\alpha x}( \pi^{-1}(x) )\phi_{\beta x}( \pi^{-1}(x) )        \big)    \Big).
\end{split}
\end{equation}
Then
\begin{equation}
  g\bab^E=\rho\Big(  \phi_{\alpha x}( \pi^{-1}(x) )\phi_{\beta x}\big( \pi^{-1}(x) \big)  \Big)\\
         =\rho(g^P\bab(x)).
\end{equation}

\end{proof}

\subsection{Sections on associated bundle}  \label{sec_fnequiv}
%-----------------------------------------

\subsubsection{Equivariant functions}

We consider a bundle $\dptvb{E=P\times_{\rho} V}{p}{M}$ associated with the principal bundle $\dptvb{P}{\pi}{M}$ and a section $\dpt{\psi}{M}{E}$.
\[
 \xymatrix{ P \ar[rd]_{\displaystyle\pi^P} &&
 E=P\times_{\rho} V \ar[ld]^{\displaystyle\pi^E} \\ & M }
\]
A \defe{section}{section!of an associated bundle} of $E$ is a map $\dpt{\psi}{M}{E}$ such that $\pi^E\circ\psi=id_M$. We define the function $\dpt{\hpsi}{P}{V}$ by
\begin{equation}\label{eq:equiv_psi}
   \psi(\pi(\xi))=[\xi,\hpsi(\xi)].
\end{equation}
Let us see the condition under which this equation well defines $\hpsi$. First, remark that a $\psi$ defined by this equation is a section because $p[\xi,v]=\pi(\xi)$, so that $(p\circ\psi)(\pi(\xi))=\pi(\xi)$. Now, consider a $\eta$ such that $\pi(\eta)=\pi(\xi)$. Then there exists a $g\in G$ for which $\eta\cdot g=\xi$. For any $g$ and for this one in particular,
\[
  \psi(\pi(\eta))=[\eta,\hpsi(\eta)]=[\eta\cdot g,\rho(g^{-1})\hpsi(\eta)].
\]
Then equation \eqref{eq:equiv_psi} defines $\hpsi$ from $\psi$ if and only if
\begin{equation}\label{eq:equiv_psi_b}
  \hpsi(\xi\cdot g)=\rho(g^{-1})\hpsi(\xi).
\end{equation}
This condition is called the \defe{equivariance}{equivariant} of $\hpsi$. Reciprocally, any equivariant function $\hpsi$ defines a section of $E=P\times_{\rho} V$.

If $\eta=\xi\cdot g=\chi\cdot k$, one define a sum
\begin{equation}\label{eq:def:som_E}
  [\xi,v]+[\chi,w]=[\eta,\rho(g)v+\rho(k)w].
\end{equation}
If $\dpt{\psi,\eta}{M}{E}$ are two sections defined by the equivariant functions $\dpt{\hpsi,\hat\eta}{P}{V}$, then the section $\psi+\eta$ is defined by the equivariant function $\hat\psi+\hat \eta$.

\subsubsection{For the endomorphism of sections of \texorpdfstring{$E$}{E}}\label{equivendo}
%////////////////////////////////////////////////////

Let us now make a step backward, and take $A$ in $\End{\Gamma(E)}$. We will now see that $A$ defines (and is defined by) an equivariant function $\dpt{\hat A}{P}{\End{V}}$. Let $\dpt{\psi}{M}{E}$ be in $\Gamma(E)$. If $\psi(x)=[\xi,v]$, we define the new section $A\psi$ by
\[
         (A\psi)(x)=[\xi,\hat A(\xi)v]=[\xi,\hat A(\xi)\hat\psi(\xi)].
\]
In order for $A\psi$ to be well defined, the function $\hat A$ must satisfy
\begin{equation}
     \hat A(\xi\cdot g)=\rho(g^{-1})\hat A(\xi)\rho(g)                 \label{equivA}
\end{equation}
for all $g$ in $G$.

\subsubsection{Local expressions}
%////////////////////////////////

We consider a local trivialization $\dpt{\phi^P_{\alpha}}{\pi^{-1}(\mU_{\alpha})}{\mU_{\alpha}\times G}$ of $P$ on $\mU_{\alpha}$ and the corresponding section $\dpt{\sigma_{\alpha}}{\mU_{\alpha}}{P}$ given by
\[
\sigma_{\alpha}(x)=(\phi^P_{\alpha})^{-1}(x,e).
\]
We saw at page \pageref{eq:triv_P_E} that a trivialization of $P$ gives a trivialization of the associated bundle $E=P\times_{\rho} V$; the definition is
\begin{equation}
  \phi^E_{\alpha}[(\xi,v)]=( \pi(\xi),\rho(a)v )
\end{equation}
if $\phi_{\alpha}^P(\xi)=(\pi(\xi),a)$. With $\xi=\sigma_{\alpha}(x)$, we find
\begin{equation}
   \phi^E_{\alpha}[(\sigma_{\alpha}(x),v)]=(  \pi(\sigma_{\alpha}(x)),\rho(a)v  )
                                 =(x,v).
\end{equation}

The section $\psi$ can also be seen with respect to the ``reference''{} sections $\sigma_{\alpha}$ by means of the definition
\begin{equation}\label{eq:def:psisa}
  \psi(x)=[\sigma_{\alpha}(x),\psisa(x)]
\end{equation}
for a function $\dpt{\psisa}{M}{V}$.

\begin{lemma}
Let $\dpt{\psi}{M}{E}$ be a section and $\dpt{\hpsi}{P}{V}$, the corresponding equivariant function. Then
\[
   \psisa(x)=\hpsi(\sigma_{\alpha}(x)).
\]
\end{lemma}

\begin{proof}
By definition, $\psi(x)=\psi(\pi(\xi))=[\xi,\hpsi(\xi)]$.  Thus if we consider in particular $\xi=\sigma_{\alpha}(x)$,
\begin{equation}
  \phi^E_{\alpha}(\psi(x))=\phi^E_{\alpha}[\xi,\hpsi(\xi)]
                        =\phi^E_{\alpha}[s_{\alpha}(x),\hpsi(\sigma_{\alpha}(x))]
                        =(x,\hpsi(\sigma_{\alpha}(x))).
\end{equation}

\end{proof}

Let us anticipate. A \defe{spinor}{spinor} is a section of an associated bundle $E=P\times_{\rho} V$ where $P$ is a Lorentz-principal bundle, $V=\eC^2$ and $\rho$ is the spinor representation of Lorentz on $\eC^2$. So a spinor $\dpt{\psi}{M}{E}$ is \emph{locally} described by a function $\dpt{\psi\bsa}{M}{\eC^2}$. The latter is the one that we are used to handle in physics. In this picture, the transformation law of $\psi$ under a Lorentz transformation comes naturally.

Let $\{e_i\}$ be a basis of V; we consider some ``reference'' sections $\gamai$ of the associated bundle $E=P\times_{\rho} V$ defined by
\begin{equation}\label{eq:def:gamai}
\gamai(x)=[\phi_{\alpha}^{-1}(x,e),e_i].
\end{equation}
A general section $\dpt{\psi}{M}{E}$ is defined by an equivariant function $\dpt{\hpsi}{P}{V}$ which can be written as $\hpsi(\xi)=a^i(\xi)e_i$. If $\eta=\phi_{\alpha}^{-1}(x,e)$ and $\xi=\eta\cdot g(\xi)$,
\begin{equation}
  \psi(x)=[\xi,a^ie_i]
         =a^i[\eta,\rho(g)e_i]
	 =a^i(\xi)\bghd{\rho(g(\xi))}{i}{j}[\eta,e_j]
	 =c^j(\xi)\gamaj(x).
\end{equation}
Since the left hand side of this equation just depends on $x$, the functions $c^j$ must actually not depend on the choice of $\xi\in\pi^{-1}(x)$. So we have $\dpt{c^j}{M}{\eR}$. Indeed, if we choose $\chi\in\pi^{-1}(x)$,
\[
  \psi(x)=c^j(\xi)\gamaj(x)
         \stackrel{!}{=}[\xi,a^{i}(\chi)e_i]
	 =\ldots
	 =c^j(\chi)\gamaj(x),
\]
so that $c^j(\xi)=c^j(\chi)$. So any section $\dpt{\psi}{M}{E}$ can be decomposed (over the open set $\mU_{\alpha}$) as
\begin{equation}
  \psi(x)=s_{\alpha}^i(x)\gamai(x).
\end{equation}


\subsection{Associated and vector bundle}
%----------------------------------------

\subsubsection{General construction}
%///////////////////////////////////

We are going to see that a vector bundle is an associated bundle. For this, we consider a vector bundle $\dpt{p}{F}{M}$ with a fiber $F_x=V$ of dimension $m$. Let $G=GL(V)$, $P$ be the trivial principal bundle $P=M\times G$ and $\rho$ be the definition representation of $G$ on $V$. We set $E=P\times_{\rho} V$. Our aim is to put a vector bundle structure on $E$ which is equivalent to the one of $F$. The bijection $\dpt{b}{F}{E}$ will clearly be
\begin{equation}
   b(\phi^{-1}(x,v))=[(x,e),v].
\end{equation}
We define the projection $\dpt{q}{E}{M}$ by
\[
   q[(x,g),w]=x
\]
and we have to show that  $q^{-1}(x)=\{  [(x,g),w]\tq g\in G \textrm{ and } w\in V  \}$ is a vector space isomorphic to $V$. The following definitions define a vector space structure:
\begin{itemize}
\item multiplication by a scalar: $\lambda[(x,g),v]=[(x,g),\lambda v]$,
\item addition: $[(x,g),v]+[(x,h),w]=[(x,e),\rho(g)v+\rho(h)w]$.
\end{itemize}
As local trivialization map, we consider
\begin{equation}
\begin{aligned}
 \chi\colon q^{-1}(\mU)&\to \mU\times V \\
    [(x,g),v]  &\mapsto(x,\rho(g)v).
\end{aligned}
\end{equation}
With this structure, the bijection $b$ is an equivalence because $b|_{F_x}$ is a vector space isomorphism and $q\circ b=p$.

\subsection{Equivariant functions for a vector field}	\label{equivvec}
%----------------------------------------------------

In order to define in the same way an equivariant function for a vector field $X\in\cvec(M)$, we need to see $TM$ as an associated bundle.

\begin{proposition}
If $M$ is a $n$ dimensional manifold, we have the following isomorphism:
\[
     \SO(M)\times_{\rhoM}\eR^m\simeq TM
\]
where $\dpt{\rhoM}{\SO(m)\times\eR^m}{\eR^m}$ is defined by $\rhoM(A)v=Av$.
\end{proposition}
\begin{proof}
Recall that an element $b\in \SO(M)_x$ is a map $\dpt{b}{\eR^m}{T_xM}$. The isomorphism is no difficult. It is $\dpt{\psi}{\SO(M)\times_{\rhoM}\eR^m}{TM}$ defined by
\[\psi[b,v]=b(v).\]
It prove no difficult to see that $\psi$ is well defined, injective and surjective.
\end{proof}

Now, let us consider $X\in\cvec(M)$. We can see it as an element of $\Gamma(\SO(M)\times_{\rhoM}\eR^m)$, and define an equivariant function $\dpt{\hX}{\SO(M)}{\eR^m}$.

Let us make it more explicit. A vector field $Y\in\cvec(M)$ is, for each $x$ in $M$, the data of a tangent vector $Y_x\in T_xM$. Hence the formula $b(v)=Y_x$ defines an element $[b,v]$ in $\SO(M)\times_{\rhoM}\eR^m$, and $Y$ defines a section $\tilde{Y}(x)=[b(x),v(x)]$ of $\SO(M)\times_{\rhoM}\eR^m$. The associated equivariant function is given by $\hY(b)=v$ if $b(v)=Y_x$. In other words, the equivariant function $\dpt{\hY}{\SO(M)}{\eR^m}$ associated with the vector field $Y\in\cvec(M)$ is given by
\begin{equation}\label{r1404e1}
  \hY(b)=b^{-1}(Y_x),
\end{equation}
 where $x=\pi(b)$.


\subsection{Gauge transformations}
%---------------------------------

A \defe{gauge transformation}{gauge!transformation!of principal bundle} of the $G$-principal bundle $\dpt{\pi}{P}{M}$ is a diffeomorphism $\dpt{\varphi}{P}{P}$ such that

\begin{itemize}
\item $\pi\circ\varphi=\pi$,
\item  $\varphi(\xi\cdot g)=\varphi(\xi)\cdot g$.
\end{itemize}

When we consider some local sections on $\dpt{\sigma_{\alpha}}{\mU_{\alpha}}{P}$, we can describe a gauge transformation with a function $\dpt{\tilde{\varphi}_{\alpha}}{M}{G}$ by requiring
\[
   \varphi(\sigma_{\alpha}(x))=\sigma_{\alpha}(x)\cdot\tilde{\varphi}_{\alpha}(x).
\]
This formula defines $\varphi$ from $\tilde{\varphi}$ as well as $\tilde{\varphi}$ from $\varphi$.

The group of gauge transformations\index{gauge!transformation!of section of associated bundle} has a natural action on the space of sections given by
\begin{subequations}
   \begin{align}
   (\varphi\cdot\psi)(x)&=[\varphi(\xi),v].
\intertext{if $\psi(x)=[\xi,v]=[\xi,\hat{\psi}(\xi)]$. This law can also be seen on the equivariant function $\hpsi$ which defines $\psi$. The rule is}
   \widehat{\varphi\cdot\psi}(\xi)&=\hpsi(\varphi^{-1}(\xi)).
   \end{align}
\end{subequations}
Indeed, in the same way as before we find $(\varphi\cdot\psi)(x)=[\xi,\widehat{\varphi\cdot\psi}(x)]\stackrel{!}{=}[\varphi(\xi),v]=[\varphi(\xi),\hpsi(\xi)]$. Taking $\xi\to\varphi^{-1}(\xi)$ as representative, $(\varphi\cdot\psi)(x)=[\xi,\hpsi\circ\varphi^{-1}(\xi)]$.

% This is part of (almost) Everything I know in mathematics
% Copyright (c) 2013-2018, 2020
%   Laurent Claessens
% See the file fdl-1.3.txt for copying conditions.


\section{Adjoint bundle}
%+++++++++++++++++++++++

Let $\pi\colon P\to M$ be a $G$-principal bundle. The \defe{adjoint bundle}{adjoint!bundle} is the associated bundle $\Ad(P)=P\times_{\Ad}\mG$. An element of that bundle is an equivalent class given by\nomenclature[D]{$\Ad(P)$}{Adjoint bundle of the principal bundle $P$}
\[
  [\xi,X]=[\xi\cdot g,\Ad(g^{-1})X]
\]
for every $g\in G$. Here $\xi\in P$ and $X\in\mG$.

\section{Connection on vector bundle: local description}\label{sec:conn_vect}
%+++++++++++++++++++++++++++++++++++++++++++++++++++++++

\begin{definition}      \label{DEFooIESVooGNQHzl}
    A \defe{connection}{connection!on vector bundle} on the vector bundle $p\colon E\to M$ is a bilinear map
    \begin{equation}
        \begin{aligned}
                \nabla \colon \cvec(M)\times\Gamma(E) &\to \Gamma(E)\\
                (X,s)&\mapsto \nabla_Xs
        \end{aligned}
    \end{equation}
     such that
     \begin{itemize}
     \item $\nabla_{fX}s=f\nabla_Xs$,
     \item $\nabla_X(fs)=(X\cdot f)s+f\nabla_Xs$
     \end{itemize}
    for all $X\in\cvec(M)$, $f\in\Cinf(M)$ and $s\in\Gamma(E)$. The operation $\nabla$ is often called a \defe{covariant derivative}{covariant!derivative!on vector bundle}.
\end{definition}

An easy example is given on the trivial bundle $E=\pr_1\colon M\times\eC\to M$. For this bundle, $\Gamma(E)=\Cinf(M,\eC)$ and the common derivation is a covariant derivation: $\nabla_Xs=(ds)X$.

\begin{proposition}     \label{PROPooWZYOooEVhgFt}
The value of $(\nabla_Xs)(x)$ depends only on $X_x$ and $s$ on a neighbourhood of $x\in M$.
\end{proposition}

\begin{proof}
Let $X$, $Y\in\cvec(M)$ such that $Y_z=f(z)X_z$  with $f(x)=1$ and $f(z)\neq 1$ everywhere else. Then
\[
  (\nabla_Ys)(x)-(\nabla_Xs)(x)=(f(x)-1)(\nabla_Xs)(x)=0.
\]
Since it is true for any function, the linearity makes that it cannot depend on $X_z$ with $z\neq x$. If we consider now two sections $s$ and $s'$ which are equals on a neighbourhood of $x$, we can write $s'=fs$ for a certain function $f$ which is $1$ on the neighbourhood. Then
\[
  (\nabla_Xs')(x)-(\nabla_Xs)(x)=(f(x)-1)(\nabla_Xs)(x)+(Xf)s(x)
\]
which zero because on a neighbourhood of $x$, $f$ is the constant $1$.
\end{proof}

This proposition shows that it makes sense to consider only local descriptions of connections.  Let $\{e_1,\ldots,e_r\}$ be a basis of $V$ and consider the local sections $\dpt{\ovS_{\alpha i}}{\mU_{\alpha}}{E}$,
\[
  \ovS_{\alpha i}(x)=\phi_{\alpha}^{-1}(x,e_i).
\]
A local section $\dpt{s_{\alpha}}{\mU_{\alpha}}{V}$ can be decomposed as $s_{\alpha}(x)=s_{\alpha}^i(x)e_i$ with respect to this basis (up to an isomorphism between the different $V$ at each point). Then on $\mU_{\alpha}$,
\begin{equation}
  s_{\alpha}^i\ovS_{\alpha i}(x)=s_{\alpha}^i(x)\phi_{\alpha}^{-1}(x,e_i)
                              =\phi_{\alpha}^{-1}(x,s_{\alpha}^ie_i)
			      =\phi_{\alpha}^{-1}(x,s_{\alpha}(x))
			      =s(x).
\end{equation}
The first equality is the definition of the product $\eR\times F\to F$.

So any $s\in\Gamma(E)$ can be (locally!) written under the form\footnote{be careful on the fact that the ``coefficient'' $s_{\alpha}^i$ depends on $x$: the right way to express this equation is $s(x)=s^i_{\alpha}(x)\ovS_{\alpha i}(x)$.} $s=s_{\alpha}^i\ovS_{\alpha i}$; in particular $\nabla_X(\ovS_{\alpha i})$ can. We define the coefficients $\theta$ by\nomenclature[D]{$(\theta_{\alpha})_i^j$}{Matrix associated with a connection}
\begin{equation}
 \nabla_X(\ovS_{\alpha i})=(\theta_{\alpha})^j_i(X)\ovS_{\alpha j}.
\end{equation}
where, for each $i$ and $j$, $(\theta_{\alpha})^j_i$ is a $1$-form on $\mU_{\alpha}$. We can consider $\theta_{\alpha}$ as a matrix-valued $1$-form on $\mU_{\alpha}$.

\begin{proposition}	\label{PropFormnabXthe}\label{prop:namba_theta_u}   %TODOooIXPEooITWiYc: supprimer un label
The formula
\begin{equation}\label{eq:nab_theta}
   (\nabla_Xs)_{\alpha}=Xs_{\alpha}+\theta_{\alpha}(X)s_{\alpha}
\end{equation}
gives a local description of the connection.
\end{proposition}

\begin{proof}
For any $s\in\Gamma(E)$, we have
\begin{equation}
\nabla_Xs=\nabla_X\big(  \sum_j s^j_{\alpha}\ovS_{\alpha j}  \big)
         =\sum_j\Big(  (Xs_{\alpha}^j)\ovS_{\alpha j} + s^j_{\alpha}\nabla_X\ovS_{\alpha j}     \Big)
	 =\sum_i\Big[   (Xs_{\alpha}^i)+s_{\alpha}^j(\theta_{\alpha})_j^i(X)  \Big]\ovS_{\alpha i}.
\end{equation}
\end{proof}

\subsection{Connection and transition functions}
%//////////////////////////////////////////////////

A connection determines some local matrix-valued $1$-forms $\theta_{\alpha}$ on the trivialization $\mU_{\alpha}$. Two natural questions raise. The first is the converse: does a matrix-valued $1$-form defines a connection? The second is to know  what is $\theta_{\alpha}$ in function of $\theta_{\beta}$ on $\mU_{\alpha}\cap\mU_{\beta}$? The answer to the latter is  given by the following proposition:

\begin{proposition}
The $1$-form $\theta_{\alpha}$ relative to the trivialization $(\mU_{\alpha},\phi_{\alpha})$ is related to the $1$-form $\theta_{\beta}$ relative to the trivialization $(\mU_{\beta},\phi_{\beta})$ by
\begin{equation}\label{eq:theta_g}
  \theta_{\beta}=g\bab^{-1} dg\bab+g\bab^{-1}\theta_{\alpha} g\bab.
\end{equation}
\end{proposition}

\begin{proof}
We can use equation \eqref{eq:tr_sec} pointwise on $(\nabla_X s)_{\alpha}$:
\begin{equation}
\begin{split}
(\nabla_X s)_{\alpha}&=g\bab(\nabla_Xs)_{\beta}\\
                  &=g\bab\big(   Xs_{\beta}+\theta_{\beta}(X)s_{\beta}   \big) \\
                  &=g\bab\big(   X(g_{\alpha\beta} s_{\alpha})+\theta_{\beta}(X)g_{\alpha\beta} s_{\alpha}   \big).
\end{split}
\end{equation}
We have to compare it with equation \eqref{eq:nab_theta}. Note that $g\bab$ and $\theta_{\alpha}(X)$ are matrices, then one cannot do
\[
   g\bab\theta_{\beta}(X)g_{\alpha\beta} =g\bab g_{\alpha\beta}\theta_{\beta}(X)=\theta_{\beta}(X)
\]
by using $g\bab g_{\alpha\beta}=\mtu$.  Taking carefully subscripts into account, one sees that the correct form is $(g\bab)^i_j\theta_{\beta}(X)^j_k(g_{\alpha\beta})^k_l$. Applying Leibnitz formula ($X(fg)=f(Xg)+(Xf)g$), and making the simplification $g\bab g_{\alpha\beta}=\mtu$ in the first term, we find
\[
  \theta_{\alpha}(X)s_{\alpha}=g\bab(Xg_{\alpha\beta})s_{\alpha}+g_{\alpha\beta}^{-1}\theta_{\beta}(X)g_{\alpha\beta} s_{\alpha}.
\]
The claim follows from the fact that $Xg_{\alpha\beta}=dg_{\alpha\beta}(X)$.
\end{proof}


\begin{normaltext}
    The equation \eqref{eq:theta_g} are related to the equations \eqref{trans_A} or \eqref{tr_de_A} or any physical equation of gauge transformation for the bosons.
\end{normaltext}

\begin{normaltext}
    Notice that formula \eqref{eq:theta_g} shows in particular that $\theta_{\alpha}$ takes its values in the Lie algebra $\gl(V)$, see for example subsection~\ref{SubSecgmudg}.
\end{normaltext}

The inverse is given in the
\begin{proposition}	\label{Propformconnve}
If we choose a family of $\gl(V)$-valued $1$-forms $\theta_{\alpha}$ on $\mU_{\alpha}$ satisfying \eqref{eq:theta_g},then the formula
\[
  (\nabla_Xs)_{\alpha}=Xs_{\alpha}+\theta_{\alpha}(X)s_{\alpha}
\]
defines a connection on $E$.\label{prop:thet_conn_F}
\end{proposition}

\begin{proof}
Note that $\theta$ is $\Cinf(M)$-linear, thus
\begin{equation}
  (\nabla_{fX}s)_{\alpha}=(fX)s_{\alpha}+\theta_{\alpha}(fX)s_{\alpha}
                        =f[ Xs_{\alpha}+\theta_{\alpha}(X)s_{\alpha} ]
			=f(\nabla_Xs)_{\alpha}.
\end{equation}
In expressions such that $\theta_{\alpha}(X)(fs_{\alpha})$, the product is a matrix times vector product between $\theta_{\alpha}(X)$ and $s_{\alpha}$; the position of the $f$ is not important. So we can check the second condition:
\begin{equation}
\begin{split}
(\nabla_X(fs))_{\alpha}&=X(fs_{\alpha})+\theta_{\alpha}(X)(fs_{\alpha}) \\
                     &=X(f)s_{\alpha}+f(Xs_{\alpha})+f\theta_{\alpha}(X)s_{\alpha}\\
		     &=df(X)s_{\alpha}+f(\nabla_Xs)_{\alpha}.
\end{split}
\end{equation}
This concludes the proof.
\end{proof}


\subsection{Torsion and curvature}
%----------------------------------

The map $\dpt{T^{\nabla}}{\cvec(X)\times\cvec(X)}{\cvec(X)}$ defined by
\begin{equation}
     T^{\nabla}(X,Y)=\nabla_XY-\nabla_YX-[X,Y]\label{deftorsion}
\end{equation}
is the \defe{torsion}{torsion!of a connection} of the connection $\nabla$. When $T^{\nabla}(X,Y)=0$ for every $X$ and $Y$ in $\cvec(X)$, we say that $\nabla$ is a \defe{torsion free}{torsion!free, connection} connection. Let $X$, $Y$ be in $\cvec(M)$, and consider the map $\dpt{R(X,Y)}{\Gamma(E)}{\Gamma(E)}$ defined by
		\begin{equation}
		\begin{aligned}
			R(X,Y) \colon \Gamma(E) &\to \Gamma(E)\\
			s&\mapsto \nabla_X\nabla_Ys-\nabla_Y\nabla_Xs-\nabla_{[X,Y]}s.
		\end{aligned}
	\end{equation}

For each $x\in M$, $R$ can be seen as a bilinear map $\dpt{R}{T_xM\times T_xM}{\End(E_x)}$. It is called the \defe{curvature}{curvature} of the connection $\nabla$. For every $f\in C^{\infty}(M)$, it satisfies
\[
 R(fX,Y)s=fR(X,Y)s=R(X,Y)fs.
\]

In a trivialization $(\mU_{\alpha},\phi_{\alpha})$, we have $(\nabla_Xs)_{\alpha}=Xs_{\alpha}+\theta_{\alpha}(X)s_{\alpha}$. Looking in the expression of $(R(X,Y)s)_{\alpha}$, the terms coming from the $Xs_{\alpha}$ part of covariant derivative make
\[
  XYs_{\alpha}-YXs_{\alpha}-[X,Y]s_{\alpha}=0.
\]
The other terms are no more than matricial product. Hence the formula
\begin{equation}
  (R(X,Y)s)_{\alpha}=\Omega_{\alpha}(X,Y)s_{\alpha}
\end{equation}
 defines a $2$-form $\Omega_{\alpha}$ which takes values in $GL(r,\eK)$. We can find an expression for $\Omega$ in terms of $\theta$:
\[
  \Omega_{\alpha}(X,Y)=X\theta_{\alpha}(Y)-Y\theta_{\alpha}(X)-\theta_{\alpha}([X,Y])+\theta_{\alpha}(X)\theta_{\alpha}(Y)-\theta_{\alpha}(Y)\theta_{\alpha}(X);
\]
it is written as
\begin{equation}\label{eq:Omega_ttheta}
\Omega_{\alpha}=d\theta_{\alpha}+\theta_{\alpha}\wedge\theta_{\alpha}=d\theta_{\alpha}+\frac{1}{2}[\theta_{\alpha},\theta_{\alpha}]
\end{equation}
which is a notational shortcut for
\begin{equation}		\label{EaCurvdVVsq}
  \Omega_{\alpha}(X,Y)=d\theta_{\alpha}(X,Y)+[\theta_{\alpha}(X),\theta_{\alpha}(Y)].
\end{equation}
These equations are called \defe{structure equations}{structure!equations}. Pointwise, the second term is a matrix commutator; be careful on the fact that, when we will speak about principal bundle, the forms $\theta$'s will take their values in a Lie algebra. On $\mU_{\alpha}\cap\mU_{\beta}$, we have
\[
  \Omega_{\beta}(X,Y)=g\bab^{-1}\Omega_{\alpha}(X,Y)g\bab.
\]
The curvature and the connection fulfill the \defe{Bianchi identities}{Bianchi identities}:

\begin{lemma}
  \[
     d\Omega_{\alpha}+[\theta_{\alpha}\wedge\Omega_{\alpha}]=0.
  \]
\end{lemma}

\begin{proof}
For each matricial entry, $\theta_{\alpha}$ is a $1$-form on $\mU_{\alpha}$, then $\theta_{\alpha}(X)$ is a function which to $x\in M$ assign $\theta_{\alpha}(x)(X_x)\in\eR$. So we can apply $d$ and Leibnitz on the product $\theta_{\alpha}(X)\theta_{\alpha}(Y)$.
\[
 d\big(  \theta_{\alpha}(X)\theta_{\alpha}(Y)  \big)=\theta_{\alpha}(X)d\theta_{\alpha}(Y)+d\theta_{\alpha}(X)\theta_{\alpha}(Y).
\]
Differentiating equation \eqref{eq:Omega_ttheta}, $d\Omega_{\alpha}=d\theta_{\alpha}\wedge\theta_{\alpha}-\theta_{\alpha}\wedge d\theta_{\alpha}$.
\end{proof}

%+++++++++++++++++++++++++++++++++++++++++++++++++++++++++++++++++++++++++++++++++++++++++++++++++++++++++++++++++++++++++++
\section{Connexion on vector bundle: algebraic view}
%+++++++++++++++++++++++++++++++++++++++++++++++++++++++++++++++++++++++++++++++++++++++++++++++++++++++++++++++++++++++++++

We already defined a connection of the vector bundle \( \pi\colon E\to M\) in definition~\ref{DEFooIESVooGNQHzl}. As we know from section~\ref{SUBSECooAASYooVHZEhz}, this induces a map
\begin{equation}        \label{EQooBRLHooJgzIyT}
    \nabla\colon \Gamma(E)\to \Gamma(E)\otimes \Omega^1(M)
\end{equation}
Note: there is a non trivial identification \( \Gamma(TM)^*=\Gamma(T^*M)\). This \( \nabla\) is bilinear and satisfy the Leibnitz rule
\begin{equation}
\nabla(\sigma f)=(\nabla\sigma)f+\sigma\otimes df
\end{equation}
for any section $\sigma\colon M\to E$ and function $f\colon M\to \eC$. If $\{ \sigma_i \}$ is a local basis of $E$, one can write $\sigma=\sigma_if^i$ and one defines the \defe{Christoffel symbols}{Christoffel symbol} $\Gamma_{i\mu}^{j}$ in this basis by
\begin{equation}
\nabla \sigma=\nabla (\sigma_if^i)
		=(\nabla \sigma_i)f^i+\sigma_i\otimes f(f^i)
		=f^i\Gamma_{i\mu}^{j}\sigma_j\otimes dx^{\mu}+\sigma_i\otimes d(f^i).
\end{equation}
The notations $d\sigma=\sigma_i\otimes d(f^i)$ and $\Gamma\sigma=f^i\Gamma_{i\mu}^{j}\sigma_j\otimes dx^{\mu}$ lead us to the compact usual form
\[
  \nabla\sigma=(d+\Gamma)\sigma.
\]

When $E=TM$ over a (pseudo)Riemannian manifold $M$, we know the Levi-Civita\index{connection!Levi-Civita} connection which is compatible with the metric:
\begin{equation}\label{eq_230605r1}
  g(\nabla X,Y)+g(X,\nabla Y)=d\big( g(X,Y) \big).
\end{equation}
One can see $g$ as acting on $\big( \cvec(M)\otimes\Omega^1(M) \big)\times\cvec(M)$ with
\[
 g\big( r^i_{\nu}\partial_i\otimes dx^{\nu},t^j\partial_j \big):=r^i_{\nu}j^jg(\partial_i,\partial_j)dx^{\nu},
\]
which at each point is a form. From condition \eqref{eq_230605r1}, we see $\nabla$ as a Levi-Civita connection on the bundle $E=T^*M$ which values in
\[
 \Gamma^{\infty}(T^*M)\otimes\Omega^1(M)\simeq\Omega^1(M)\otimes\Omega^1(M).
\]
This is defined as follows. A $1$-form $\omega$ can always be written under the for $\omega=X^{\flat}:=g(X,.)$ for a certain $X\in\cvec(M)$. Then  \eqref{eq_230605r1} gives
\[
  (\nabla X)^{\flat}Y+\omega(\nabla Y)=d(\omega Y),
\]
and we put $\nabla\omega=(\nabla X)^{\flat}$, i.e
\begin{equation}
  (\nabla\omega)Y=d(\omega Y)-\omega(\nabla Y)
\end{equation}
for all $Y\in\cvec(M)$. When $\omega=dx^i$ and $Y=\partial_j$, we find
\begin{equation}
(\nabla dx^i)\partial_j=d(dx^i\partial_j)-dx^i(\nabla\partial_j)
		=d(\delta^i_j)-\Gamma_{jk}^{l}\partial_l\otimes dx^k
		=-\Gamma_{jk}^{l}\delta_l^i\otimes dx^k
		=-\Gamma_{jk}^{i}\,dx^k.
\end{equation}
So we get the local formula
\begin{equation}
\nabla dx^i=-\Gamma_{jk}^{i}\,dx^j\otimes dx^k.
\end{equation}
If the form writes locally $\omega=dx^if_i$,
\begin{equation}
  \nabla\omega=\nabla(dx^i)f_i+dx^i\otimes df_i
		=-f_i\Gamma_{jk}^{i}\,dx^j\otimes dx^k+d\omega
		=(d-\tilde\Gamma)\omega
\end{equation}
where we taken the notations $d\omega=dx^i\otimes df_i$ and $\tilde\Gamma\omega=f_i\Gamma_{jk}^{i}dx^j\otimes dx^k$.

%---------------------------------------------------------------------------------------------------------------------------
\subsection{Exterior derivative}

If $E$ is a $m$-dimensional vector bundle over $M$ and $s\colon M\to E$ is a section, we say that a \defe{exterior derivative}{exterior!derivative} is a map $D\colon \Gamma(E)\to \Gamma(E\otimes \Omega^1M)$ such that for every $f\in C^{\infty}(M)$ we have
\[
  D(fs)=s\otimes df+f(Ds).
\]
An exterior derivative can be extended to $D\colon \Gamma(E\otimes\Omega^kM)\to \Gamma(E\otimes\Omega^{k+1}M)$ imposing the condition
\begin{equation}		\label{EqExtExtDerrk}
D(\omega\wedge\alpha)=(D\omega)\wedge\alpha+(-1)^k\omega\wedge d\alpha
\end{equation}
for every $\omega\in\Gamma(E\otimes\Omega^kM)$ and $\alpha\in\Gamma(E\otimes\Omega^lM)$ . The result is an element of $\Gamma(E\otimes\Omega^{k+l+1}M)$.

Coordinatewise expressions are obtained when one choose a specific section $(e_i)$ of the frame bundle of $E$. In that case for each $i$, the derivative $e_i$ is an element of $\Gamma(E\otimes\Omega^1M)$ and we define $\omega_i^j\in\Omega^1(M)$\nomenclature{$\omega_i^j$}{Connection form} by
\begin{equation}
De_i=\sum_{j=1}^ke_j\otimes \omega_i^j.
\end{equation}
For each $i$ and $j$, we have an element $\omega_i^j\in\Omega^1(M)$, so that we say that $\omega\in\Omega^1(M,\gl(m))$. Now a section can be expressed as $s=s^ie_i$ where $s^i$ are functions, so we have
\begin{align}
  D(s)=D(s^ie_i)	&=e_i\otimes ds^i+s^iD(e_i)=e_i\otimes ds^i+s^ie_j\otimes \omega_i^j=e_i\otimes ds^i+e_i\otimes s^j\omega_j^i.
\end{align}
Expressed in component, we find $D(s)^i=ds^i+s^j\omega_j^i$, so that we often write
\begin{equation}
D=d+\omega.
\end{equation}
When a section $e$ is given, we write $s=s^i(e)e_i$, indicating the dependence of the functions $s^i$ in the choice of the frame $e$:
\[
  D(s)=e_i\otimes ds^i(e)+e_i\otimes s^j(e)\omega(e)_j^i.
\]
When we apply both sides to a vector $X\in\Gamma(TM)$, we find
\begin{equation}
D_X(s)=e_i\otimes\Big( X(s^i)+s^j\omega^i_j(X) \Big).
\end{equation}

By convention we say that, when $f\in C^{\infty}(M)$, is a function, $D_X$ reduces to the action of the vector field $X$:
\begin{equation}
  D_X(f)=X(f).
\end{equation}


%----------------------------------------------------------------------------------------------------------------------------
\subsubsection{Covariant exterior derivative}

An important exterior derivative is the covariant exterior derivative. If the vector bundle $E$ is endowed by a covariant derivative $\nabla$, we define the corresponding \defe{covariant exterior derivative}{covariant!derivative!exterior } by the following:
{
\renewcommand{\theenumi}{\arabic{enumi}.}
\begin{enumerate}
\item for a section $s\colon M\to E$ (i.e. a $0$-form) we define
\begin{equation}
   (d_{\nabla}s)(X)=\nabla_Xs,
\end{equation}
\item and on the $1$-form $\sum_i(s_i\otimes\omega_i \big)\in\Gamma(E\otimes T^*M)$,
\begin{equation}
d_{\nabla}\big( \sum_is_i\otimes\omega_i \big)=\sum_i(d_{\nabla}s_i)\wedge\omega_i+\sum_is_i\otimes d\omega_i.
\end{equation}
\end{enumerate}
}		% Fin de la mise en nombre arabes pour la liste énumérée
The latter relation is the condition \eqref{EqExtExtDerrk} with $k=0$.

%---------------------------------------------------------------------------------------------------------------------------
\subsection{Divergence, gradient and Laplacian (general)}
%---------------------------------------------------------------------------------------------------------------------------

When we have a connection (definition~\ref{DEFooIESVooGNQHzl}) we define the divergence of a vector field. Recall from equation \eqref{EQooBRLHooJgzIyT} that when \( x\in\Gamma(TM)\) we can see \( \nabla X\) as an element of \( \Gamma(TM)\otimes \Omega^1(M)\).
\begin{definition}      \label{DEFooTTSFooDdgiKg}
    The \defe{divergence}{divergence} of a vector field $X\in \Gamma(TM)$, is the function $\nabla\cdot X\in C^{\infty}(M)$\nomenclature[F]{$\nabla\cdot X$}{divergence of the vector field $X$} defined by
    \begin{equation}
      (\nabla\cdot X)(x)=\tr(\nabla X)
    \end{equation}
    where the trace stands for the contraction of the tensor \( \nabla X\) with itself.
\end{definition}

On an euclidian space, we can describe the divergence in an other way: by proposition~\ref{PROPooWZYOooEVhgFt}, the vector \( (\nabla_XY)_x\in T_xM \) does depend only on \( X_x\). Thus in fact, when \( v\in T_xM\), the value of \( \nabla_vX\) is the value of \( (\nabla_VX)_{x}\) where \( V\) is any vector field such that \( V_x=v\). Thus we are okay speaking about the map
\begin{equation}
    \begin{aligned}
         T_xM&\to T_xM \\
        v&\mapsto \nabla_vX.
    \end{aligned}
\end{equation}
This operation being linear, we can take the trace. Thus the definition could be
\begin{equation}    \label{EQooEGYQooJAvdTO}
    \nabla\cdot X=\tr(v\mapsto \nabla_vX).
\end{equation}
The point using the contraction notion is that the operation of tracing is more clear: from equation \eqref{EQooEGYQooJAvdTO} one does not see why the result should depend on the metric (and in particular why does it depend on the \emph{inverse} of the metric tensor in a matricial form.).

%---------------------------------------------------------------------------------------------------------------------------
\subsection{Divergence, gradient and Laplacian (Riemannian case)}
%---------------------------------------------------------------------------------------------------------------------------

If we have a (pseudo)Riemannian manifold,we define the gradient of a function.
\begin{definition}
    We define the \defe{gradient}{gradient} of a function $f\in C^{\infty}(M)$, denoted by $\nabla f$\nomenclature[F]{$\nabla f$}{Gradient of the function $f$} as the vector field such that
    \begin{equation}        \label{EQooECZSooYfQFYm}
        g(\nabla f,X)=X(f).
    \end{equation}
\end{definition}

Moreover on a Riemannian manifold, we have the Levi-Civita connection (see~\ref{subsection_levi}), and then it is possible to particularize the definition~\ref{DEFooTTSFooDdgiKg} to a well defined connection.

The \defe{Laplacian}{laplace operator} of the function $f$ is the function $\Delta f$\nomenclature[F]{$\Delta f$}{Laplace operator} given by
\begin{equation}
\Delta f=\nabla\cdot(\nabla f).
\end{equation}

%---------------------------------------------------------------------------------------------------------------------------
\subsection{Divergence, gradient and Laplacian (coordinatewise)}
%---------------------------------------------------------------------------------------------------------------------------

\begin{proposition}
    Let \( \{ e_i \}\) be a field of base for the riemannian manifold \( (M,g)\), and a function \( f\in C^{\infty}(M)\). Then the gradient of \( f\) is given by the formula
    \begin{equation}
        \nabla f=\sum_{im}(g^{-1})_{im}\partial_ife_m.
    \end{equation}
    The important point is that \( g^{-1}\) appears instead of \( g\).
\end{proposition}

\begin{proof}
    In coordinates, this is given by the \emph{inverse} of the metric tensor. Indeed equation \eqref{EQooECZSooYfQFYm} expands to (unwritten dependence to \( x\))
    \begin{equation}
        \sum_{kl}g_{kl}(\nabla f)_kX_l=X(f)=\sum_lX_l\partial_lf.
    \end{equation}
    Since that has to be true for every vector field \( X\), we can ``simplify'' \( X_l\):
    \begin{equation}
        \sum_kg_{kl}(\nabla f)_k=\partial_lf.
    \end{equation}
    Multiplying by \( (g^{-1})_{lj}\) and making the sum over \( l\):
    \begin{equation}
        \sum_{kl}g_{kl}(g^{-1})_{lj}(\nabla f)_k=\sum_l(g^{-1})_{lj}\partial_lf,
    \end{equation}
    and then
    \begin{equation}
        (\nabla f)_j=\sum_l(g^{-1})_{lj}\partial_lf.
    \end{equation}
\end{proof}

\begin{proposition} \label{PROPooLIJTooKFTwPY}
    Let \( (M,g)\) be a (pseudo)Riemannian manifold with constant \( g\). Then
    \begin{equation}
        \nabla\cdot Y=\sum_{ij}(g^{-1})_{ij}\partial_iY_j.
    \end{equation}
\end{proposition}

\begin{proof}
    The fact that \( g\) is constant implies that the Christoffel symbols are vanishing and the Levi-Civita connection is given by
    \begin{equation}
        \nabla_XY=X(Y)=\sum_{kl}X_l\partial(Y_k)\partial_k.
    \end{equation}
    As explained around equation~\ref{EQooBRLHooJgzIyT} we see \( \nabla Y\in\Gamma(TM)\otimes \Omega^1(M)\) as
    \begin{equation}
        \nabla Y=\sum_{kl}(\partial_l)(Y_k)\partial_k\otimes dx_l.
    \end{equation}
    Using formula \eqref{EQooDODKooOxCzZP} for the contraction,
    \begin{equation}
        \tr(\nabla Y)=\sum_{kl}(g^{-1})_{kl}\partial_l(Y_k).
    \end{equation}
    This is the divergence of \( Y\).
\end{proof}

%---------------------------------------------------------------------------------------------------------------------------
\subsection{Soldering form and torsion}
%---------------------------------------------------------------------------------------------------------------------------

Let us particularize to the case where $E$ has the same dimension as the manifold. In that case, we can introduce a \defe{soldering form}{soldering form}, that is an element $\theta\in \Omega^1(M,E)$ such that for every $x\in M$ the map $\theta_x\colon T_xM\to E_x$ is a vector space isomorphism.
When a soldering form $\theta$ is given, the \defe{torsion}{torsion!of exterior derivative} is the exterior derivative $D$ is
\begin{equation}
	T=D\theta.
\end{equation}
Using a local frame $e$, we have forms $\theta^i(e)\in\Omega^1(M)$ such that
\[
  \theta(X)=\theta^i(X)e_i.
\]
We see $\theta$ as an element of $\Gamma(E\otimes \Omega^1(M))$ by identifying $\theta=e_i\otimes\theta^i$. Thus we have
\[
D\theta=D(e_i\otimes\theta^i)	=De_i\wedge\theta^i(e)+e_i\wedge d\theta^i(e)
				=(e_j\otimes^j_i)\wedge\theta^i(e)+e_i\wedge d\theta^i(e),
\]
or in coordinates:
\begin{equation}
  (D\theta)^i=\omega_j^i\wedge \theta^j(e)+d\theta^i(e).
\end{equation}
Notice that it provides the formula
\begin{equation}
T=d_{\omega}\theta
\end{equation}
for the torsion as exterior covariant derivative of the connection form.

%---------------------------------------------------------------------------------------------------------------------------
\subsection{Example: Levi-Civita}
%---------------------------------------------------------------------------------------------------------------------------

We consider the vector bundle $E=TM$ and the local basis $e_i=\partial_i$. An exterior derivative in this case is a map $D\colon \Gamma(TM)\to \Gamma\Big( TM\otimes\Omega^1M \Big)$. In that particular case, we denote by $\nabla_XY$ the vector field $D(Y)X$, and it is computed by first writing $D(X)_x=\sum_iZ_x^i\otimes\omega_x^i$ with $Z^i\in\Gamma(TM)$ and $\omega^i\in\Omega^1(M)$. The we have
\begin{equation}
D(X)_xY_x=\omega+x^i(Y_x)Z_x^i.
\end{equation}
A good choice of soldering form is $\theta_x=\id$ for every $x\in M$, or $\theta(X)=X$. In coordinates, that soldering form is given by $\theta^i(\partial_j)=\delta^i_j$. The \defe{Christoffel symbols}{Christoffel symbol} are defined by
\begin{equation}
\nabla_{\partial_i}\partial_j=\Gamma_{ij}^k\partial_k,
\end{equation}
and the covariant derivative reads
\begin{equation}		\label{EqCovDerGamChr}
\nabla_XY	= \nabla_{X^i\partial_i}(Y^j\partial_j)
		= X^i\Big( (\partial_iY^j)\partial_j+Y^j\nabla_{\partial_i}\partial_j \Big)
		= \Big( X(Y^k)+X^iY^j\Gamma_{ij}^k \Big) \partial_k.
\end{equation}

We can determine the Christoffel symbols in function of the connection form using the fact that on the one hand, $\nabla_{\partial_i}\partial_j=\Gamma_{ij}^k\partial_k$, and on the other hand,
\[
  \nabla_{\partial_i}\partial_j=D(\partial_j)(\partial_i)=\partial_k\otimes\omega_j^k(\omega_i),
\]
so that
\begin{equation}
	\Gamma_{ij}^k=\omega_j^k(\partial_i)
\end{equation}
Now we can get the same result as equation \eqref{EqCovDerGamChr} using the exterior derivative formalism. First we have $DY=\partial_i\otimes dY^i+\partial_i\otimes X^j\omega_j^i$, so that
\[
  (DY)X=\partial_i\otimes dY^i(X)=\partial_i\otimes X^j\omega_j^i(X^k\partial_k),
\]
in which we use the relation $\omega_j^i(X^k\partial_k)=X^k\omega_j^i(\partial_k)=X^k\Gamma_{jk}^i$ to get
\[
  (DY)X=\big( X(Y^i)+X^jX^k\Gamma^i_{jk} \big)\partial_i.
\]
Notice that the anti-symmetric part of $\Gamma$ with respect to its two lower indices does not influence the covariant derivative. Let us compute the torsion in terms of $\Gamma$. For that remark that $d\theta^i=0$ because
\[
  (d\theta^i)(X,Y)=X\theta^i(Y)-Y\theta^i(X)-\theta^i\big( [X,Y] \big)=X(Y^i)-Y(X^i)-[X,Y]^i=0.
\]
Thus we have
\begin{align*}
(D\theta)(\partial_k\otimes\partial_l)	&=\big( (D\partial_i)\partial_k \big)\theta^i(\partial_l)-\big( (D\partial_i)\partial_l \big)\theta^i(\partial_k)\\
					&=\delta_l^i\Gamma_{ik}^j\partial_j-\delta_k^i\Gamma_{il}^j\partial_j\\
					&=(\Gamma_{lk}^j-\Gamma^j_{kl})\partial_j.
\end{align*}
The connection $\nabla$ is moreover compatible with the metric because
\[
  \nabla_Z\big( g(X,Y) \big)=Z\big( \eta(eX,eY) \big)=\eta\big( \underbrace{D_Z(eX)}_{=e(\nabla_ZX)},eY \big)+\eta\big( eX,D_Z(eY) \big)=g(\nabla_ZX,Y)+g(X,\nabla_ZY).
\]




\section{Connection on principal bundle}  %\label{subsec_defconnprinc}
%------------------------------------------

\subsection{First definition:  \texorpdfstring{$1$}{1}-form}
%------------------------------

\label{pg_connpriic}
We consider a $G$-principal bundle
\[
\xymatrix{%
   G \ar@{~>}[r]		&	P\ar[d]^{\pi}\\
   				&	  M
}
\]
and $\yG$, the Lie algebra of $G$.

\begin{definition}
A \defe{connection}{connection!on principal bundle} on $P$  is a $1$-form $\omega\in\Omega(P,\yG)$ which fulfills

\begin{itemize}\label{pg:def:conne}
\item $\omega_{\xi}(A^*_{\xi})=A$,
\item $(R_g^*\omega)_{\xi}(\Sigma)=\Ad(g^{-1})(\omega_{\xi}(\Sigma))$,
\end{itemize}
for all $A\in\yG$, $g\in G$, $\xi\in P$ and $\Sigma\in T_{\xi} P$
\label{defconnform}
\end{definition}
Here, $R_g$ is the right action: $R_g\xi=\xi\cdot g$ and $A^*$ stands for the \defe{fundamental field}{fundamental!vector field} associated with $A$ for the action of $G$ on $P$:
\begin{equation} \label{defastar}
   A^*_{\xi}=\Dsdd{ \xi\cdot e^{-tA} }{t}{0},
\end{equation}
For each $\xi\in P$, we have $\dpt{\omega_{\xi}}{T_{\xi} P}{\yG}$. See section~\ref{sec:fond_vec}.

If $\alpha$ is a connection $1$-form on $P$, we say that $\Sigma$ is an \defe{horizontal}{horizontal} vector field if $\alpha_{\xi}(\Sigma)=0$ for all $\xi\in P$. If $X_x\in T_xM$ and $\xi\in\pi^{-1}(x)$, there exists an unique\footnote{See \cite{kobayashi}, chapter II, proposition 1.2.} $\Sigma$ in $T_{\xi} P$ which is horizontal and such that $\pi_*(\Sigma)=X_x$. This $\Sigma$ is called the \defe{horizontal lift}{horizontal!lift} of $X_x$. We can also pointwise construct the horizontal lift of a vector field. The one of $X$ is often denoted by $\overline{X}$; it is an element of $\cvec(P)$.

\subsection[Horizontal space]{Second definition: horizontal space}
%-------------------------------------------------------------------

For each $\xi\in P$, we define the \defe{vertical space}{vertical space} $V_{\xi} P$ as the subspace of $T_{\xi} P$ whose vectors are tangent to the fibers: each $v\in V_{\xi} P$ fulfills $d\pi v=0$. Any such vector is given by a path contained in the fiber of $\xi$. So, $v\in V_{\xi} P$ if and only if there exists a path $g(t)\in G$ such that $v=\Dsdd{\xi\cdot g(t)}{t}{0}$.

A \defe{connection}{connection!on principal bundle} $\Gamma$ is a choice, for each $\xi\in P$, of an \defe{horizontal space}{horizontal!space} $H_{\xi} P$ such that

\begin{itemize}
\item $T_{\xi} P=V_{\xi} P\oplus H_{\xi} P$,
\item $H_{\xi\cdot g}=(dR_g)_{\xi} H_{\xi}$,
\item $H_{\xi} P$ depends on $\xi$ under a differentiable way.
\end{itemize}
The second condition means that the distribution $\xi\to H_{\xi}$ is invariant under $G$. Thanks to the first one, for each $X\in T_{\xi} P$, there exists only one choice of $Y\in H_{\xi} P$ and $Z\in V_{\xi} P$ such that $X=Y+Z$. These are denoted by $vX$ and $hX$ and are naturally named \emph{horizontal} and \emph{vertical components} of $X$. The third condition means that if $X$ is a differentiable vector field on $P$, then $vX$ and $hX$ are also differentiable vector fields. We will often write $V_{\xi}$ and $H_{\xi}$ instead of $V_{\xi} P$ and $V_{\xi} P$.

The word \emph{connection} probably comes from the fact that the horizontal space gives a way to jump from a fiber to the next one.
When we consider a connection $\Gamma$, we can define a $\yG$-valued connection $1$-form by
\[
   \omega(X)^*_{\xi}=vX_{\xi}.
\]
The existence is explained in section~\ref{sec:fond_vec}. It is clear that $\omega(X)=0$ if and only if $X$ is horizontal. The theorem which connects the two definitions is the following.

\begin{theorem}
If $\Gamma$ is a connection on a $G$-principal bundle, and $\omega$ is its $1$-form, then

\begin{enumerate}
\item\label{enuyai} for any $A\in\yG$, we have $\omega(A^*)=A$,
\item\label{enuyaii} $(R_g)^*\omega=\Ad(g^{-1})\omega$, i.e. for any $X\in T_{\xi} P$, $g\in G$ and $\xi\in M$,
\[
    \omega((dR_g)_{\xi} X)=\Ad(g^{-1})\omega_{\xi}(X)
\]
\end{enumerate}
Conversely, if one has a $\yG$-valued $1$-form on $P$ which fulfills these two requirement, then one has one and only one connection on $P$ whose associated $1$-form is $\omega$.

\end{theorem}

\begin{proof}
\ref{enuyai} The definition of $\omega$ is $\omega(X)^*_{\xi}=vX$. Then $\omega(A^*)^*_{\xi}=vA^*_{\xi}=A^*_{\xi}$ because $A^*$ is vertical. From lemma~\ref{lem:As_Bs_A_B}, $\omega(A^*)=A$.

\ref{enuyaii} Let $X\in\cvec(P)$. If $X$ is horizontal, the definition of a connection makes $dR_d X$ also horizontal, then the claim becomes $0=0$ which is true. If $X$ is vertical, there exists a $A\in\yG$ such that $X=A^*$ and a lemma shows that $dR_gX$ is then the fundamental field of $\Ad(g^{-1})A$. Using the properties of a connection,
\begin{equation}
  (R^*_g\omega)_{\xi}(X)=\omega_{\xi\cdot g}(dR_g X)=\Ad(g^{-1})A=\Ad(g^{-1})\omega_{\xi}(X).
\end{equation}

Now we turn our attention to the inverse sense: we consider a $1$-form which fulfills the two conditions and we define
\begin{equation}
   H_{\xi}=\{X\in T_{\xi} P\tq \omega(X)=0\}.
\end{equation}
We are going to show that this prescription is a connection. First consider a $X\in V_{\xi}$, then $X=A^*$ and $\omega(X)=A$. So $H_{\xi}\cap V_{\xi}=0$. Now we consider $X\in T_{\xi} P$ and we decompose it as
\[
   X=A^*+(X-A^*)
\]
where $A^*$ is the vertical component of $X$. If $\omega(dR_g X)=0$ for all $g\in G$, then $\omega(X)=0$, then a vector $X\in H_{\xi}$ fulfills at most $\dim G$ independent constraints $\omega(dR_g X)=0$ and $\dim H_{\xi}$ is at least $\dim P-\dim G$. On the other hand, $\dim V_{\xi}=\dim G$; then
\[
  \dim V_{\xi}+\dim H_{\xi}\geq\dim G+\dim P-\dim G.
\]
Then the equality must holds and $V_{\xi}\oplus H_{\xi}=T_{\xi} P$.

We have now to prove that $\omega$ is the connection form of $H_{\xi}$, i.e. that $\omega(X)$ is the unique $A\in\yG$ such that $A^*_{\xi}$ is the vertical component of $X$. Indeed if $X\in T_{\xi} P$, it can be decomposed as into $A^*\in V_{\xi}$ and $Y\in H_{\xi}$ and
\[
   \omega(X)=\omega(A^*+Y)=\omega(A^*)=A.
\]

It remains to be proved that the horizontal space $H_{\xi}$ of any connection $\Gamma$ is related to the corresponding $1$-form $\omega$ by $H_{\xi}=\{X\in T_{\xi} P\tq\omega_{\xi}(X)=0\}$. From the connection $\Gamma$, the $1$-form is defined by the requirement that $\omega(X)^*_{\xi}=vX_{\xi}$. For $X\in H_{\xi}$, it is clear that $vX=0$, so that $\omega(X)^*=0$. This implies $\omega(X)=0$ because we suppose that the action of $G$ is effective.

\end{proof}

The projection $\dpt{\pi}{P}{M}$ induces a linear map $\dpt{d\pi}{T_{\xi} P}{T_xM}$. We will see that, when a connection is given, it is an isomorphism between $H_{\xi}$ and $T_xM$ (if $x=\pi(\xi)$). The \defe{horizontal lift}{horizontal!lift}\index{lift!horizontal} of $X\in\cvec(M)$ is the unique horizontal vector field (i.e. it is pointwise horizontal) such that $d\pi(\ovX_{\xi})=X_{\pi(\xi)}$. The proposition which allows this definition is the following.

\begin{proposition}
For a given connection on the $G$-principal bundle $P$ and a vector field $X$ on $M$, there exists an unique horizontal lift of $X$. Moreover, for any $g\in G$, the horizontal lift is invariant under $dR_g$.

The inverse implication is also true: any horizontal field on $P$ which is invariant under $dR_g$ for all $g$ is the horizontal lift of a vector field on $M$.
\end{proposition}

This proposition comes from \cite{kobayashi}, chapter II, proposition 1.2.

\begin{proof}
We consider the restriction $\dpt{d\pi}{H_{\xi}}{T_{\pi(\xi)}M}$. It is injective because $d\pi(X-Y)$ vanishes only when $X-Y$ is vertical or zero. Then it is zero. It is cleat that $\dpt{d\pi}{T_{\xi} P}{T_{\pi(\xi)}M}$ is surjective. But $d\pi X=0$ if $X$ is vertical, then $d\pi$ is surjective from only $H_{\xi}$.

So we have existence and unicity of an horizontal lift. Now we turn our attention to the invariance. The vector $dR_g\ovX_{\xi}$ is a vector at $\xi\cdot g$. From the definition of a connection, $dR_g H_{\xi}=H_{\xi\cdot g}$, then $dR_g\ovX_{\xi}$ is the unique horizontal vector at $\xi\cdot g$ which is sent to $X_x$ by $d\pi$. Thus it is $\ovX_{\xi\cdot g}$.

For the inverse sense, we consider $\ovX$, an horizontal invariant vector field on $P$. If $x\in M$, we choose $\xi\in\pi^{-1}(x)$ and we define $X_x=d\pi(\ovX_{\xi})$. This construction is independent of the choice of $\xi$ because for $\xi'=\xi\cdot g$, we have
\[
   d\pi(\ovX_{\xi'})=\pi(dR_g\ovX_{\xi})=\pi(\ovX_{\xi}).
\]
\end{proof}

An other way to see the invariance is the following formula:
\[
   \ovX_{\xi\cdot g}=(dR_g)_{\xi} \ovX_{\xi}.
\]
By definition, $\ovX_{\xi\cdot g}$ is the unique vector of $T_{\xi\cdot g}P$ which fulfils $d\pi\ovX_{\xi\cdot g}=X_x$ if $\xi\pi^{-1}(x)$, so the following computation proves the formula:
\begin{equation}
  (d\pi)_{\xi\cdot g}((dR_g)_{\xi}\ovX_{\xi})=d(\pi\circ R_g)_{\xi}\ovX_{\xi}
                                         =d\pi_{\xi}\ovX_{\xi}
					 =X_x.
\end{equation}

\subsection{Curvature}
%////////////////////////

The curvature of a vector or associated bundle satisfies $\Omega_{\alpha}=d\theta_{\alpha}+\theta_{\alpha}\wedge\theta_{\alpha}$. So we naturally define the \defe{curvature}{curvature!on principal bundle} of the connection $\omega$ on a principal bundle as the $\yG$-valued $2$-form
\begin{equation}
  \Omega=d\omega+\omega\wedge\omega.
\end{equation}
When we consider a local section\label{PgLocSecCurv} $\dpt{\sigma_{\alpha}}{\mU_{\alpha}}{P}$ on $\mU_{\alpha}\subset M$, we can express the curvature with a $2$-form on $M$ instead of $P$ by the formula \label{pg:curv_princ}
\[
 F\bsa=\sigma_{\alpha}^*\Omega,
\]
or, more explicitly, by $F\bsa_x(X,Y)=\Omega_{\sigma_{\alpha}(x)}(d\sigma_{\alpha} X,d\sigma_{\alpha} Y))$. Note that if $\yG$ is abelian, $\Omega=d\omega$ and $d\Omega=0$.

\section{Exterior covariant derivative and Bianchi identity}
%--------------------------------------------------------------

Let $\omega\in\Omega^1(P,\mG)$ be a connection $1$-form on the $G$-principal bundle $P$. Using the operation $[.\wedge .]$ defined in section~\ref{SecLiaAlgformval}, we define the \defe{exterior covariant derivative}{exterior!covariant derivative} by\nomenclature{$d_{\omega}$}{Exterior covariant derivative associated with the connection form $\omega$}  %\index{exterior covariant derivative}
\begin{align}
d_{\omega}\alpha&=d\alpha+\frac{ 1 }{2}[\omega\wedge\alpha]&\text{when }\alpha\in\Omega^1(P,\mG),\\
d_{\omega}\beta&=d\beta+[\omega\wedge\beta]&\text{when }\beta\in\Omega^2(P,\mG),
\end{align}

The \defe{curvature}{curvature!form} is the $2$-form defined by
\begin{equation}
\Omega=d_{\omega}\omega=d\omega+\omega\wedge\omega
\end{equation}
where $d_{\omega}$ is the exterior covariant derivative associated with the connection form $\omega$, and the wedge has to be understood as in equation \eqref{EqAbuswesgeomom}.

\begin{proposition}
The curvature form satisfies the identity
\begin{equation}
d_{\omega}\Omega=0
\end{equation}
which is the Bianchi identity\index{Bianchi identities}
\end{proposition}

\begin{proof}
taking the differential of $\Omega=d\omega+\omega\wedge\omega$, we find
\[
  d\Omega=d^2\omega+d\omega\wedge\omega-\omega\wedge d\omega
\]
in which $d^2\omega=0$ and we replace $d\omega$ by $\Omega-\omega\wedge\omega$, so that
\[
  d\Omega=\Omega\wedge\omega-\omega\wedge\Omega,
\]
which becomes the Bianchi identity using the definition of $d_{\omega}$ and the notation \eqref{EqDefCrochwedgedeux}.
\end{proof}
Remark that the Bianchi identity reads $d_{\omega}^2\omega=0$, but that in general $d_{\omega}$ does not square to zero.

\section{Covariant derivative on associated bundle}
%-----------------------------------------------------

Now we consider a general $G$-principal bundle $\dpt{\pi}{P}{M}$ and an associated bundle $E=P\times_{\rho} V$. We define a product $\eR\times E\to E$ by
\begin{equation}\label{eq:def:REE}
  \lambda[\xi,v]=[\xi,\lambda v].
\end{equation}
It is clear that the equivariant function $\widehat{\lambda \psi}$ defines the section $\lambda\psi$. A \defe{covariant derivative}{covariant!derivative!on associated bundle} is a map
		\begin{equation}
		\begin{aligned}
			\nabla \colon \cvec(M)\times \Gamma(M,E) &\to \Gamma(M,E)\\
			(X,\psi)&\mapsto \nabla_X\psi
		\end{aligned}
	\end{equation}
such that
\begin{subequations}
\begin{align}
\nabla_{fX}\psi&=f\nabla_X\psi,   \\
\nabla_X(f\psi)&=(X\cdot f)\psi+f\nabla_X\psi                                      \label{eq:def:der_covii}
\end{align}
\end{subequations}
where products have to be understood by formula \eqref{eq:def:REE}.

\begin{theorem}
A connection on a principal bundle gives rise to a covariant derivative on any associated bundle by the formula
\begin{equation}
  \widehat{\nabla^E_X\psi}(\xi)=\ovX_{\xi}(\hpsi)
\end{equation}
where $\dpt{\hpsi}{P}{V}$ is the function associated with the section $\dpt{\psi}{M}{E}$.
\label{tho_dercovassoequiv}
\end{theorem}
We have to prove that it is a good definition: the function $\widehat{\nabla^E_X\psi}$ must define a section $\dpt{\nabla_X^R\psi}{M}{E}$ and the association $\psi\to\nabla^E_X\psi$ must be a covariant derivative.

With the discussion of page \pageref{pg:vecto_vecto} about the application of a tangent vector on a map between manifolds, we have $(d\varphi X)f=X(f\circ\varphi)$. By using this equality in the case of $\ovX$ with $\hpsi$ and $R_g$, we find $(dR_g\ovX)(\hpsi)=\ovX(\hpsi\circ R_g)$ and thus
\[
   \ovX_{\xi\cdot g}(\hpsi)=\ovX_{\xi}(dR_g\hpsi).
\]
We prove the theorem step by step.

\begin{proposition}
The function $\widehat{\nabla_X^E\psi}$ defines a section of $P$.
\end{proposition}

\begin{proof}
We have to see that $\widehat{\nabla_X^E\psi}$ is an equivariant function. The equivariance of $\hpsi$ gives $\hpsi\circ R_g=\rho(g^{-1})\hpsi$, thus
\begin{equation}
\widehat{ \nabla_X^E\psi }(\xi\cdot g)=\ovX_{\xi\cdot g}(\hpsi)\\
                                      =\big( (dR_g)_{\xi}\ovX_{\xi}\big)(\hpsi)\\
				      =\ovX_{\xi}(\hpsi\circ R_g)\\
				      =\ovX_{\xi}( \rho(g^{-1})\hpsi )\\
				      =\rho(g^{-1})\ovX_{\xi}(\hpsi).
\end{equation}
The last equality comes from the fact that the product $\rho(g^{-1})\hpsi$ is a linear product ``matrix times vector''{} and that $\ovX_{\xi}$ is linear.
\end{proof}

\begin{theorem}
The definition
\[
   \widehat{\nabla^E_X\psi}(\xi)=\ovX_{\xi}(\hpsi)
\]
defines a covariant derivative.
\label{tho_nablaE}
\end{theorem}

\begin{proof}
We have to check the two conditions given on page \pageref{sec:conn_vect}.

\subdem{First condition}
By definition, $\widehat{\nabla_{fX}^E\psi}(\xi)=\overline{fX_{\xi}}(\hpsi)$. Now we prove that
\begin{equation}\label{eq:fXhpsi}
  \overline{fX_{\xi}}(\hpsi)=(f\circ\pi)(\xi)\ovX_{\xi}(\hpsi).
\end{equation}
 This formula is coherent because $\ovX_{\xi}(\hpsi)\in V$ and $(f\circ\pi)(\xi)\in\eR$. By definition of the horizontal lift, $\overline{fX}_{\xi}$ is the unique vector such that

 \begin{itemize}
 \item $d\pi_{\xi}(\overline{fX}_{\xi})=(fX)_x=f(x)d\pi\ovX_{\xi}=(f\circ\pi)(\xi)d\pi\ovX_{\xi}$,
 \item $\omega_{\xi}(\overline{fX}_{\xi})=0$.
 \end{itemize}
We check that $(f\circ\pi)(\xi)\ovX_{\xi}$ also fulfills these two conditions because $d\pi$ and $\omega$ are $\Cinf(P)$-linear. Equation \eqref{eq:fXhpsi} immediately gives
\begin{equation}
\widehat{\nabla_{fX}^E\psi}(\xi)=(f\circ\pi)(\xi)\widehat{\nabla_X^E\psi}(\xi).
\end{equation}
Now we show that $\widehat{ f\nabla_X^E\psi }$ is the same. The section $\dpt{f\nabla_X^E\psi}{M}{E}$ is given by  $(f\nabla_X^E\psi)(x)=f(x)(\nabla_X^E\psi)(x)$, and by definition of the associated equivariant function,
\[
  f(x)(\nabla_X^E\psi)(x)=[ \xi,f(x)\widehat{\nabla_X^E\psi}(\xi) ].
\]
Then
\begin{equation}
  \widehat{f\nabla_X^E\psi}(\xi)=f(x)\widehat{\nabla_X^E\psi}(\xi)=(f\circ\pi)(\xi)\widehat{\nabla_X^E\psi}(\xi).
\end{equation}
All this shows that
$  \nabla_{fX}^E\psi=f\nabla_X^E\psi$.
\subdem{Second condition}
This is a computation using the Leibnitz rule:
\begin{equation}
\begin{split}
  \widehat{\nabla_X^E(f\psi)}(\xi)&=\ovX_{\xi}( \widehat{f\psi} )
                                  	\stackrel{(a)}{=}\ovX_{\xi}((\pi\circ f)\hpsi)\\
				  &\stackrel{(b)}{=}\ovX_{\xi}(\pi^*f)\hpsi(\xi)+(\pi^*f)(\xi)\ovX_{\xi}\hpsi
				  =d(f\circ\pi)_{\xi}\ovX_{\xi}\hpsi(\xi)+f\widehat{\nabla_X^E\psi}(x)\\
				  &=df_{\pi(\xi)}d\pi_{\xi}\ovX_{\xi}\hpsi(\xi)+f\widehat{\nabla_X^E\psi}(x)
				  =X_x(f)\hpsi(\xi)+f\widehat{\nabla_X^E\psi}(x)\\
				  &=\widehat{(Xf)\psi}(\xi)+\widehat{f\nabla_X^E\psi}(\xi)
\end{split}
\end{equation}
where (a) is because $\widehat{f\psi}=\pi^*f\hat{\psi}$, and (b) is an application of the Leibnitz rule.
\end{proof}




\begin{theorem}
Using the local coordinates related to the sections $\dpt{\sigma_{\alpha}}{\mU_{\alpha}}{P}$, the covariant derivatives reads:
\begin{equation}\label{eq:nabla_coord}
(\nabla_X\psi)\bsa(x)=X_x\psi\bsa-\rho_*(\sigma_{\alpha}^*\omega_x(X))\psi\bsa(x)
\end{equation}
where $\dpt{\rho_*}{\yG}{\End(V)}$ is defined by
\begin{equation}  \label{eq:def_rho_s}
  \rho_*(A)=\Dsdd{\rho(e^{tA})}{t}{0}
\end{equation}

\end{theorem}

\begin{proof}
The problem reduces to the search of $\ovX$ because
\[
   (\nabla_X\psi)\bsa(x)=\widehat{\nabla_X\psi}(\sigma_{\alpha}(x))=\ovX_{\sigma_{\alpha}(x)}(\hpsi).
\]
We claim that $\ovX_{\sigma_{\alpha}(x)}=d\sigma_{\alpha} X_x-\omega(d\sigma_{\alpha} X_x)^*$. We have to check that $d\pi\ovX=X$ and $\omega(\ovX)=0$. The latter comes easily from the fact that $\omega(A^*)=A$. For the first one, remark that $s_{\alpha}$ is a section, then $d(\pi\circ s_{\alpha})=\id$, and $d\pi(ds_{\alpha} X_x)=X_x$, while
\begin{equation}
  d\pi(A^*\bxi)=d\pi\Dsdd{\xi\cdot e^{-tA}}{t}{0}
               =\Dsdd{\pi(\xi)}{t}{0}
               =0.
\end{equation}
Since the horizontal lift is unique, we deduce
\begin{equation}
  (\nabla_X\psi)\bsa(x)=\big(  d\sigma_{\alpha} X_x-\omega(d\sigma_{\alpha} X_x)^*  \big)\hpsi.
\end{equation}
From the definition of a fundamental vector field,
\begin{equation}
\begin{aligned}
    \omega(d\sigma_{\alpha} X_x)^*_{\sigma_{\alpha}(x)}\hpsi
           &=\Dsdd{\hpsi\big(\sigma_{\alpha}(x)\cdot e^{-t\omega(d\sigma_{\alpha} X_x)}  \big) }{t}{0}\\
           &=\Dsdd{\rho(e^{t\omega(d\sigma_{\alpha} X_x)})\hpsi(\sigma_{\alpha}(x))}{t}{0}&&\text{from \eqref{eq:equiv_psi_b}}\\
           &=(d\rho)_e(\omega\circ d\sigma_{\alpha})X_x(\hpsi\circ\sigma_{\alpha})(x)\\
           &=\rho_*\big( (\sigma_{\alpha}^*\omega)(X_x) \big)\psi\bsa(x)    &&\text{by \eqref{eq:def_rho_s}}
\end{aligned}
\end{equation}

\end{proof}

We can express the covariant derivative by means of some maps $\dpt{\theta_{\alpha}}{\cvec(M)\times M}{\End(V)}$ given by
 \begin{equation}
\nabla_X\gamai=\bghd{\theta_{\alpha}(X)}{i}{j}\gamaj.
 \end{equation}
where the $\gamai$'s were given in equation \eqref{eq:def:gamai}. By the definition \eqref{eq:def:der_covii},
\[
\begin{split}
  (\nabla_X\psi)(x)&=(X\cdot s^i_{\alpha})_x\gamai(x)+s^i_{\alpha}(x)(\nabla_X\gamai)(x)\\
                   &=(X\cdot s^i_{\alpha})_x\gamai(x)+s^i_{\alpha}(x)\bghd{\theta_{\alpha}(X)}{i}{j}\gamaj(x).
\end{split}
\]
On the othre hand with the notations of equation \eqref{eq:def:psisa}, $\gamsai=e_i$ and $X_x\gamsai=0$. Then equation \eqref{eq:nabla_coord} gives $\theta_{\alpha}(X)=\rho_*(\sigma_{\alpha}^*\omega_x(X))$, or
\begin{equation}
\theta_{\alpha}=\rho_*(\sigma^*_{\alpha}\omega_x).
\end{equation}

\subsection{Curvature on associated bundle}

From the definition \eqref{eq:def:som_E}, it makes sense to define the curvature $2$-form by
\[
  R(X,Y)\psi=\nabla_X\nabla_Y\psi-\nabla_Y\nabla_X\psi-\nabla_{[X,Y]}\psi.
\]
It is also clear that $\psisa$ defines a section of the trivial vector bundle $F=M\times V$ by $x\to (x,\psisa(x))$, so one can define  $\dpt{\Omega_{\alpha}(X,Y)}{\Gamma(M,E)}{\Gamma(M,E)}$ by
\[
  \big(  R(X,Y)\psi  \big)\bsa=\Omega_{\alpha}(X,Y)\psisa
\]
and take back all the work around Bianchi because of the relation \eqref{eq:nabla_coord} which can be written as $(\nabla_X\psi)\bsa(x)=X_x\psisa+\theta_{\alpha}(X)\psisa(x)$ and which is the same as in proposition~\ref{prop:thet_conn_F}.

\subsection{Connection on frame bundle}\index{frame!bundle}
%//////////////////////////////////////

The frame bundle was defined at page \pageref{pg:frame_bundle}. Let $\dptvb{F}{p}{M}$ be a $\eK$-vector bundle with some local trivialization $(\mU_{\alpha},\phi^E_{\alpha})$ and the corresponding transition functions $\dpt{g\bab}{\mU_{\alpha}\cap\mU_{\beta}}{GL(r,\eK)}$. We consider $\dpt{\pi}{P}{M}$, the frame bundle of $F$; it is a $GL(r,\eK)$-principal bundle. Let $\nabla$ be a covariant derivative on $F$ and $\theta_{\alpha}$, the associated matrices $1$-form. The frame bundle is
\[
  P=\bigcup_{x\in M}(\text{frame of }F_x).
\]
A connection is a $\yG$-valued $1$-form; in our case it is a map
\[
  \dpt{\omega^{\alpha}\bxi}{T\bxi\big(\pi^{-1}(\mU_{\alpha})\big)}{\gl(r,\eK)}.
\]
We define our connection by, for $g\in GL(r,\eK)$, $x\in \mU_{\alpha}$, $X_x\in T_xM$ and $A\in \gl(r,\eK)$,
\begin{equation}\label{eq:def_omega_frame}
  \omega_{S_{\alpha}(x)\cdot g}^{\alpha}
         \big(   {R_g}_*s_{\alpha}(x)_* X_x + A^*_{S_{\alpha}(x)\cdot g}  \big):=A+\Ad(g^{-1})\theta_{\alpha}(X_x).
\end{equation}
where $\dpt{S_{\alpha}}{\mU_{\alpha}}{P}$ is the section defined by the trivialization $\phi^P_{\alpha}$:
\[
   S_{\alpha}(x)=\{ \ovv_{\alpha}={\phi^E_{\alpha}}^{-1}(x,e_i) \}_{i=1,\ldots,r}.
\]
Since $\theta_{\alpha}(X_x)\in\End(\eK^r)\subset\gl(r,\eK)$, the second term of \eqref{eq:def_omega_frame} makes sense. This formula is a good definition of $\omega$ because of the following lemma:

\begin{lemma}
If $\xi=S_{\alpha}(x)\cdot g$ and $\Sigma\in T\bxi P$, there exists a choice of $A\in\yG$, and $X_x\in T_xM$ such that
\begin{equation}\label{eq:geneSigma}
  \Sigma={R_g}_*{s_{\alpha}(X)}_* X_x+A^*_{S_{\alpha}(x)\cdot g}.
\end{equation}
\end{lemma}

\begin{proof}
If $\xi\in P$ is a basis of $E$ at $y$, there exists only one choice of $x\in M$ and $g\in G$ such that $\xi=S_{\alpha}(x)\cdot g$.

Let us consider a general path $\dpt{c}{\eR}{P}$ under the form $c(t)=s_{\alpha}(x(t))\cdot g(t)$ where $x$ and $g$ are path in $M$ and $GL(r,\eK)$. The frame $c(t)$ is the one of $F_{x(t)}$ obtained by the transformation $g(t)$ from $s_{\alpha}(x(t))$. It is a set of $r$ vectors, and each of them can be written as a combination of the vectors of $s_{\alpha}(x(t))$, so we write
\begin{equation}
  c^i(t)=s_{\alpha}^j(x(t))g_j^i(t)
\end{equation}
where $s_{\alpha}^j(x(t))\in F_{x(t)}$ and $g_j^i(t)\in\eK$. We compute $\Sigma=c'(0)$ by using the Leibnitz rule and we denote $x'(0)=X_x$, $x(0)=x$ and $g^i_j(0)=g^i_j$ (the matrix of $g$):
\begin{equation}
\begin{split}
  \Sigma^i&=\Dsdd{  s^j_{\alpha}(x(t))  }{t}{0}g^i_j+s^j_{\alpha}(x)\Dsdd{g^i_j(t)}{t}{0}\\
          &=(ds_{\alpha}^j)_xX_xg^i_j+{g^i_j}'(0)s^j_{\alpha}(x).
\end{split}
\end{equation}
Going to more compact matrix form, it gives
\[
  \Sigma=(ds_{\alpha})_xX_x\cdot g+s_{\alpha}(x)g'(0).
\]
The second term, $s_{\alpha}^j(x)g'^i_j(0)$, is a general vector tangent to a fiber. So it can be written as a fundamental field $A^*\bxi$.

\end{proof}

\begin{lemma}
On $\mU_{\alpha}\cap\mU_{\beta}$, the form fulfills $\omega^{\alpha}=\omega\hbeta$.
\end{lemma}

\begin{proof}
Let $\dpt{\gamma}{\eR}{M}$ be a path whose derivative is $X_x$. Then
\begin{equation}
\begin{split}
   (R_g)_*s_{\alpha}(x)_*X_x&=\Dsdd{s_{\alpha}(\gamma_t)\cdot g}{t}{0}
                          =\Dsdd{  s_{\beta}(\gamma_t)g_{\alpha\beta}(\gamma_t)\cdot g  }{t}{0}\\
                          &=\Dsdd{ s_{\alpha}(\gamma_t)g_{\alpha\beta}(x)\cdot g }{t}{0}
                          +\Dsdd{ s_{\beta}(x)\cdot g_{\alpha\beta}(\gamma_t)\cdot g }{t}{0}.
\end{split}
\end{equation}
What is in the derivative of the first term is $R_{g_{\alpha\beta}(x)g}(s_{\beta}(\gamma_t))$. Taking the derivative, we find the expected ${R_{g_{\alpha\beta}(x)g}}_*{s_{\beta}}_*X_x$.

For the second term, we note $r:=s_{\beta}(x)\cdot g_{\alpha\beta}(g)g$, and we have to compute the following, using equation \eqref{eq:rdotht},
\begin{equation}
\begin{aligned}
 \Dsdd{  r\cdot\Ad_{g^{-1}}( g_{\alpha\beta}^{-1}(x)&g_{\alpha\beta}(\gamma_t) ) }{t}{0}\\
                  &=\Dsdd{ r\cdot\exp t\big(    (d\AD_{g^{-1}})_e( g_{\alpha\beta}^{-1}(x)(dg_{\alpha\beta})_xX_x )   \big)}{t}{0}\\
                  &=\Dsdd{r\cdot\exp t\big( \Ad_{g^{-1}}g_{\alpha\beta}^{-1}(x)dg_{\alpha\beta}}{t}{0}\\
                  &=\left(  \Ad_{g^{-1}}g^{-1}_{\alpha\beta}(x)dg_{\alpha\beta} X_x    \right)^*_r.
\end{aligned}
\end{equation}
Using this, we can perform the computation:
\begin{equation}
\begin{aligned}
\omega\hbeta_{S_{\alpha}(x)\cdot g}\big(  {R_g}_*{s_{\alpha}(x)}_*X_x+A^*_{S_{\alpha}(x)\cdot g}  \big)
                          &=\omega\hbeta_{S_{\beta}(x)\cdot g_{\alpha\beta}(x)g}\Big(  {R_{g_{\alpha\beta}}(x)g}_*{s_{\beta}(x)}_*X_x \\
                              &\qquad + (  \Ad_{g^{-1}}g^{-1}_{\alpha\beta}(x)dg_{\alpha\beta} X_x  )^*_r +A^* \Big)\\
                          &=\Ad_{(g_{\alpha\beta}(x)g)^{-1}}\theta_{\beta}(X_x)\\
			&\quad+\Ad_{g^{-1}}g_{\alpha\beta}^{-1}(x)dg_{\alpha\beta}(X_x)+A\\
                          &=\Ad_{g^{-1}}\big(  (g_{\alpha\beta}^{-1}\theta_{\beta} g_{\alpha\beta}+g_{\alpha\beta}^{-1} dg_{\alpha\beta})(X_x)  \big)+A\\
                          &=\omega^{\alpha}_{S_{\alpha}(x)g}\big( {R_g}_*{s_{\alpha}(x)}_*X_x+A^A_{S_{\alpha}(x)\cdot g} \big).
\end{aligned}
\end{equation}

\end{proof}

\begin{proposition}
The $\omega$ defined by formula \eqref{eq:def_omega_frame} is a connection $1$-form.
\label{prop_omconfrom}
\end{proposition}

\begin{proof}
The first condition, $\omega(A^*\bxi)=A$, is immediate from the definition. The lemma~\ref{lem:dRgAstar} gives the second condition in the case $\Sigma=A^*\bxi$. It remains to be checked that $\omega(dR_g\Sigma)=\Ad(g^{-1})\omega(\Sigma)$ in the case $\Sigma=dR_hds_{\alpha} X_x$. This is obtained using the fact that $\Ad$ is a homomorphism.
\end{proof}

%---------------------------------------------------------------------------------------------------------------------------
subsection{Levi-Civita connection}\label{subsection_levi}
%---------------------------------------------------------------------------------------------------------------------------

Let $(M,g)$ be a Riemannian manifold. We look at a connection $1$-form $\alpha\in\Omega^1(\SO(M),so(\eR^m))$ on $\SO(M)$, and we define a covariant derivative $\dpt{\nabla^{\alpha}}{\cvec(M)\times T(M)}{T(M)}$, where $T(M)$ is the tensor bundle on $M$ by (cf. theorem \eqref{tho_nablaE})
\begin{eqnarray}\label{r2804e1}
 \widehat{\nabla^{\alpha}_X s}=\overline{X}\hat{s},
\end{eqnarray}
for any $s\in T(M)$.  Our purpose now is to prove that an automatic property of this connection is $\nabla^{\alpha} g=0$. The unique such connection which is torsion-free is the \defe{Levi-Civita}{Levi-Civita connection} one.

The metric $g$ is a section of the tensor bundle $T^*M\otimes T^*M$. So we have, in order to find $\hg$ and to use equation \eqref{r2804e1}, to see $T^*M\otimes T^*M$ as an associated bundle. As done in~\ref{equivvec}, we see that
\[
 T^*M\otimes T^*M\simeq \SO(M)\times_{\rho}(V^*\otimes V^*),
\]
with the following definitions:
\begin{itemize}
\item The isomorphism is given by $\psi[b,\alpha\otimes\beta](X\otimes Y)=\alpha(b^{-1} X)\beta(b^{-1} Y)$,
\item $\rho(A)\alpha=\alpha\circ A$,
\item $b\cdot A=b\circ A$.
\end{itemize}
Here, $V=\eR^m$; $\dpt{b}{V}{T_xM}$; $\alpha,\beta\in V^*$; $X$, $Y\in T_xM$ and $A\in \SO(m)$ is seen as $\dpt{A}{V}{V}$.

The following shows that $\psi$ is well defined:
\begin{equation}
\begin{split}
 \psi[b\cdot A,\rho(A^{-1})\alpha\otimes\beta](X\otimes Y)&=(\alpha\circ A)
                            (A^{-1}\circ b^{-1} X)(\beta\circ A)(A^{-1}\circ b^{-1} Y)\\
                                           &=\psi[b,\alpha\otimes\beta](X\otimes Y)
\end{split}
\end{equation}

\begin{proposition}
The function $\hg$ is given by
\[
 \hg(b)(v\otimes w)=g_x(b(v)\otimes b(w))=v\cdot w.
\]
\end{proposition}

\begin{proof}
The second equality is just the fact that $\dpt{b}{(\eR^m,\cdot)}{(T_xM,g_x)}$ is isometric. On the other hand, if $\hg(b)=\alpha\otimes\beta$, we have:
\begin{equation}
\begin{split}
 g_x(X\otimes Y)&=\psi[b,\alpha\otimes\beta](X\otimes Y)
                =\alpha(b^{-1} X)\beta(b^{-1} Y)\\
                &=\alpha\otimes\beta(b^{-1} X\otimes b^{-1} Y)
                =\hg(b)(b^{-1} X\otimes b^{-1} Y).
\end{split}
\end{equation}

Since $b$ is bijective, $X$ and $Y$ can be written as $bv$ and $bw$ respectively for some $v$, $w\in V$, so that
\[g_x(bv\otimes bw)=\hg(b)v\otimes w.\]
\end{proof}

It is now easy to see that $\oX\hg=0$. As $\hg$ takes its values in $V^*\otimes V^*$, $\oX\hg$ belongs to this space and can be applied on $v\otimes w\in V\otimes V$. Let $\oX(t)$ be a path in $\SO(M)$ which defines $\oX$; if $\oX\in T_b\SO(M)$, $\oX(0)=b$. We have
\begin{equation}
 \oX\hg(v\otimes w)	=\dsdd{\hg(\oX(t))v\otimes w}{t}{0}
			=\dsdd{v\cdot w}{t}{0},
\end{equation}
which is obviously zero.

\subsection{Holonomy}
%--------------------

Let the principal bundle
\begin{equation}
\xymatrix{%
   G \ar@{~>}[r]		&	P\ar[d]^{\pi}\\
   				&	M
 }
\end{equation}
  and $\omega$ a connection on $G$. Let $\gamma\colon [0,1]\to M$, a closed curve piecewise smooth; $\gamma(0)=\gamma(1)=x$. For each $p\in\pi^{-1}(x)$, there exists one and only one horizontal lift $\tilde\gamma\colon [0,1]\to P$ such that $\tilde\gamma(0)=p$. There exists of course an element $g\in G$ such that $\tilde\gamma(1)=p\cdot g$.

We define the following equivariance relation on $P$: we say that $p\sim q$ if and only if $p$ and $q$ can be joined by a piecewise smooth path. The \defe{holonomy group}{holonomy group} at the point $p$ is
\[
  \Hol_p(\omega)=\{ g\in G\tq p\sim p\cdot g \}.
\]
\subsection{Connection and gauge transformation}
%-----------------------------------------------


\begin{proposition}
If $\omega$ is a connection on a $G$-principal bundle and $\varphi$, a gauge transformation, the form $\beta=\varphi^*\omega$ is a connection $1$-form too.
\label{prop:vp_conn}
\end{proposition}

\begin{proof}
It is rather easy to see that $\varphi_*A^*\bxi=A^*_{\varphi(x)}$:
\[
  \varphi_*A^*\bxi=\Dsdd{ \varphi(\xi e^{-tA})  }{t}{0}=\Dsdd{ \varphi(\xi)e^{-tA}  }{t}{0}=A^*_{\varphi(x)}.
\]
The same kind of reasoning leads to $\varphi_*{R_g}_*={R_g}_*\varphi_*$. From here, it is easy to see that
\[
  (\varphi^*\omega)\bxi(A^*\bxi)=\omega_{\varphi(\xi)}(\varphi_*A^*\bxi)=A,
\]
and
\[
\big( R^*_g(\varphi^*\omega)\bxi \big)(\Sigma)
                =(R_g^*\omega)_{\varphi(\xi)}(\varphi_*\Omega)=\Ad(g^{-1})\big( (\varphi^*\omega)\bxi(\Sigma) \big).
\]
\end{proof}
So, the ``gauge transformed'' of a connection is still a connection. It is hopeful in order to define gauge invariants objects (Lagrangian) from connections (electromagnetic fields).

\subsubsection{Local description}

Let $\dpt{\pi}{P}{M}$ be a $G$-principal bundle given with some trivializations $\dpt{\phi^P_{\alpha}}{\pi^{-1}(\mU_{\alpha})}{\mU_{\alpha}\times G}$ over $\mU_{\alpha}\subset M$ and $\dpt{s_{\alpha}}{\mU_{\alpha}}{\pi^{-1}(\mU_{\alpha})}$, a section. In front of that, we consider an associated bundle $\dpt{p}{E=P\times_{\rho} V}{M}$ with a trivialization $\dpt{\phi^E_{\alpha}}{E}{\mU_{\alpha}\times V}$. One can choose a section $s_{\alpha}$ compatible with the trivialization in the sense that $\phi^P_{\alpha}(s_{\alpha}(x)\cdot g)=(x,g)$; the same can be done with $E$ by choosing $\phi^E_{\alpha}([s_{\alpha}(x),v])=(x,v)$.

A section\index{section!local description} $\dpt{\psi}{M}{E}$ is described by a function $\dpt{\psi_{\alpha}}{\mU_{\alpha}}{V}$ defined by $\phi^E_{\alpha}(\psi(x))=(x,\psi_{\alpha}(x))$.  In the inverse sense, $\psi$ is defined (on $\mU_{\alpha}$) from $\psi_{\alpha}$ by
$\psi(x)=[s_{\alpha}(x),\psi_{\alpha}(x)]$.
In the same way, a gauge transformation\index{gauge!transformation!local description} $\dpt{\varphi}{P}{P}$ is described by functions $\dpt{\tilde{\varphi}_{\alpha}}{\mU_{\alpha}}{G}$,
\begin{equation}
  \varphi(s_{\alpha}(x))=s_{\alpha}(x)\cdot\tilde{\varphi}_{\alpha}(x).
\end{equation}
The function $\tilde{\varphi}_{\alpha}$ also fulfils
\begin{equation}
  (\phi_{\alpha}^P\circ\varphi\circ{\phi_{\alpha}^P}^{-1})(x,g)=(x,\tilde{\varphi}(x)\cdot g)
\end{equation}
because
\begin{equation}
\begin{split}
  (\phi_{\alpha}^P\circ\varphi\circ{\phi_{\alpha}^P}^{-1})(x,g)&=(\phi_{\alpha}^P\circ\varphi)(s_{\alpha}(x)\cdot g)\\
                                                      &=\phi_{\alpha}^P( \varphi(s_{\alpha}(x))\cdot g )\\
                                                      &=\phi_{\alpha}^P( s_{\alpha}(c)\cdot\tilde{\varphi}_{\alpha}(x)g)\\
                                                      &=(x,\tilde{\varphi}_{\alpha}(x)g).
\end{split}
\end{equation}

We know that a connection on $P$ is given by its $1$-form $\omega$. Moreover we have the following:
\begin{proposition}
A connection on $P$ is completely determined on $\pi^{-1}(\mU_{\alpha})$ from the data of the $\yG$-valued $1$-form $\sigma_{\alpha}^*\omega$ on $\mU_{\alpha}$.
\end{proposition}

\begin{proof}
We consider a $1$-form $\omega$ which fulfils the two conditions of page \pageref{pg:def:conne}. Our purpose is to find back $\omega\bxi(\Sigma)$, $\forall\xi\in P,\Sigma\in T\bxi P$ from the data of $\sigma_{\alpha}^*\omega$ alone. For any $\xi$, there exists a $g$ such that $\xi=\sigma_{\alpha}(x)\cdot g$. We have
\begin{equation}
  \Ad_{g^{-1}}(\omega_{\sigma_{\alpha}(x)\Sigma})=(R^*_g\omega)_{\sigma_{\alpha}(x)}(\Sigma)
             =\omega_{\sigma_{\alpha}(x)\cdot g}\big( (dR_g)_{\sigma_{\alpha}(x)}\Sigma \big).
\end{equation}
If we know $s_{\alpha}^*\omega$, then we know $\omega\big(  (ds_{\alpha})_xv  \big)$ for any $v\in T_xM$. So
\[
   \omega_{\sigma_{\alpha}(x)\cdot g}\big( (dR_g)_{\sigma_{\alpha}(x)}\Sigma\big)
\]
is given from $\sigma_{\alpha}^*\omega$ for every $\Sigma$ of the form $\Sigma=(d\sigma_{\alpha})_xv$. From the form \eqref{eq:geneSigma} of a vector in $T\bxi P$, it just remains to express $\omega_{\sigma_{\alpha}(x)\cdot g}(A^*_{\sigma_{\alpha}(x)\cdot g})$ in terms of $s_{\alpha}^*$. The definition of a connection makes that it is simply $A$.

\end{proof}

\subsubsection{Covariant derivative}

If we have a connection on $P$, we can define a covariant derivative on the associated bundle $E$ by
\[
  (\nabla_X\psi)\bsa(x)=X_x(\psi_{\alpha})+\rho_*( s_{\alpha}^*\omega_x(X) )\psi\bsa(x),
\]
the matricial $1$-form being given by $\theta_{\alpha}=\rho_*\sigma^*_{\alpha}\omega$. The gauge transformation $\varphi$ acts on the connection $\omega$ by defining $\omega^{\varphi}:=\varphi^*\omega$.

\begin{proposition}
If $\beta=\varphi^*\omega$, then
\[
   s^*_{\alpha}(\beta)=\Ad_{\tilde{\varphi}_{\alpha}(x)^{-1}}s^*_{\alpha}(\omega)+\tilde{\varphi}_{\alpha}(x)^{-1} d\tilde{\varphi}_{\alpha}.
\]
\end{proposition}

\begin{proof}
Let $\dpt{\gamma}{\eR}{M}$ be a path such that $\gamma(0)=x$ and $\gamma'(0)=X_x$. We have to compute the following:
\begin{equation}\label{eq:ppu}
  (s_{\alpha}^*\beta)(X_x)=(s_{\alpha}^*\varphi^*\omega)(X_x)=\omega_{(\varphi\circ s_{\alpha})(x)}\big(  (\varphi\circ s_{\alpha})_*X_x  \big).
\end{equation}
What lies in the derivative is:
\begin{equation}
\begin{split}
  (\varphi\circ s_{\alpha})_*(X_x)&=\Dsdd{ (\varphi\circ s_{\alpha}\circ\gamma)(t) }{t}{0}\\
                            &=\Dsdd{  s_{\alpha}(\gamma(t))\cdot\tilde{\varphi}_{\alpha}(\gamma(t))  }{t}{0}\\
                            &=\Dsdd{ s_{\alpha}(\gamma(t))\cdot\tilde{\varphi}_{\alpha}(\gamma(0)) }{t}{0}
                             +\Dsdd{ s_{\alpha}(\gamma(0))\cdot\tilde{\varphi}_{\alpha}(\gamma(t)) }{t}{0}\\
                            &={R_{\tilde{\varphi}_{\alpha}(x)}}_*{s_{\alpha}}_*X_x
                             +\Dsdd{  s_{\alpha}(x)\cdot\tilde{\varphi}_{\alpha}(x)e^{ t\tilde{\varphi}_{\alpha}(x)^{-1}(d\tilde{\varphi}_{\alpha})_x\gamma'(0)}}{t}{0}.
\end{split}
\end{equation}
A justification of the remplacement $\tilde{\varphi}_{\alpha}(\gamma(t))\to \tilde{\varphi}_{\alpha}(x)e^{t\tilde{\varphi}_{\alpha}(x)^{-1}(d\tilde{\varphi}_{\alpha})_x\gamma'(0)}$ is given in the corresponding proof at page \pageref{pg:justif_s}.
If we put this expression into equation \eqref{eq:ppu}, the first term becomes
\[
\begin{split}
   \omega_{(\varphi\circ s_{\alpha})(x)}\big(  {{R_{\tilde{\varphi}_{\alpha}(x)}}_*{s_{\alpha}}_*X_x}   \big)
           &=(R^*_{\tilde{\varphi}_{\alpha}(x)}\omega)_{s_{\alpha}(x)}({s_{\alpha}}_*X_x)\\
           &=\Ad_{\tilde{\varphi}_{\alpha}(x)^{-1}} \big(\omega_{s_{\alpha}(x)} ( {s_{\alpha}}_*X_x ) \big)\\
           &=\Ad_{\tilde{\varphi}_{\alpha}(x)^{-1}}  (s_{\alpha}^*\omega)(X_x).
\end{split}
\]
The second term is the case of a connection applied to a fundamental vector field.

\end{proof}

\section{Product of principal bundle}\label{sec:produit_bundle}
%++++++++++++++++++++++++++++++++++++

In this section, we build a $G_1\times G_2$-principal bundle from the data of a $G_1$ and a $G_2$-principal bundle. The physical motivation is clear: as far as electromagnetism is concerned, particles are sections of $U(1)$-principal bundle while the relativistic invariance must be expressed by means of a $\SLdc$-associated bundle. So the physical fields must be sections of something as the product of the two bundles. See subsection~\ref{subsec:incl_Lorentz}.

\subsection{Putting together principal bundle}
%------------------------------------

Let us consider two principal bundle over the same base space
\[
\xymatrix{
    G_1  \ar@{~>}[r] & P_1 \ar[r]^{p_1}& M,}
\]
and
\[
\xymatrix{
    G_2  \ar@{~>}[r] & P_2 \ar[r]^{p_2}& M.
  }
\]
First we define the set
\begin{equation}
  P_1\circ P_2=\{   (\xi_1,\xi_2)\in P_1\times P_2\tq p_1(\xi_1)=p_2(\xi_2)    \}
\end{equation}
which will be the total space of our new bundle. The projection $\dpt{p}{P_1\circ P_2}{M}$ is naturally defined by
\[
  p(\xi_1,\xi_2)=p_1(\xi_1)=p_2(\xi_2),
\]
while the right action of $G_1\times G_2$ on $P_1\circ P_2$ is given by
\[
  (\xi_1,\xi_2)\cdot(g_1,g_2)=(\xi_1\cdot g_1,\xi_2\cdot g_2)
\]
With all these definitions,
\[
\xymatrix{
    G_1\times G_2  \ar@{~>}[r] & P_1\circ P_2 \ar[d]^{p}\\& M\\
  }
\]
is a $G_1\times G_2$-principal bundle over $M$. We define the natural projections
		\begin{equation}
		\begin{aligned}
			\pi_i \colon P_1\times P_2 &\to P_i\
			(\xi_1, \xi_2)&\mapsto \xi_i,
		\end{aligned}
	\end{equation}
%
and if $e_i$ denotes the identity element of $G_i$, we can identify $G_1$ to $G_1\times \{e_2\}$ and $G_2$ to $G_2\times \{e_1\}$; in the same way, $\yG_1=\yG_1\times\{0\}\subset\yG_1\times\yG_2$. So we get the following principal bundles:
\[
\xymatrix{
    G_2  \ar@{~>}[r] & P_1\circ P_2 \ar[r]^{\pi_1}& P_1\\
    G_1  \ar@{~>}[r] & P_1\circ P_2 \ar[r]^{\pi_2}& P_2.
  }
\]
It is clear that the following diagram commutes:
\[
\xymatrix{
    P_1  \ar[rd]_{P_1} & P_1\circ P_2\ar[r]^{\pi_2} \ar[l]_{\pi_1}\ar[d]_p& P_2 \ar[ld]^{p_2}\\
    &M
  }
\]

\subsection{Connections}
%-----------------------

Let $\omega_i$ be a connection on the bundle $\dpt{p_i}{P_i}{M}$. Using the identifications, $\pi_1^*\omega_1$ is a connection on $\dpt{\pi_2}{P_1\circ P_2}{P_2}$ (the same is true for $1\leftrightarrow 2$), and $\pi_1^*\omega_1\oplus\pi_2^*\omega_2$ is a connection on $\dpt{p}{P_1\circ P_2}{M}$. Let us prove the first claim.

Let $A\in\yG_1$. We first have to prove that $\pi_1^*\omega_1(A^*)=A$. For this, remark that $A=(A,0)\in\yG_1\oplus\yG_2$ and
\begin{equation}
   A^*\bxi=\Dsdd{\xi\cdot e^{-t(A,0)}}{t}{0}
          =\Dsdd{(\xi_1,\xi_2)\cdot(e^{-tA},e_2)}{t}{0}
          =\Dsdd{(\xi_1\cdot e^{-tA},\xi_2)}{t}{0},
\end{equation}
so $d\pi_1A^*=\Dsdd{\pi_1(\ldots)}{t}{0}=\omega_1(A^*)=A$. Let now $\Sigma\in T_{(\xi_1,\xi_2)}(P_1\circ P_2)$ be given by the path $(\xi_1(t),\xi_2(t))$. In this case we have
\begin{equation}
\begin{split}
   \big(  R^*_{(g,e_2)\pi_1^*\omega_1}  \big)_{(\xi_1,\xi_2)}\Sigma&=
      (\pi_1^*\omega_1)(dR_{(g,e_2)}\Sigma)\\
     &=\omega_1( \Dsdd{\xi_1(t)\cdot g}{t}{0}  )\\
     &=\omega_1( dR_g\Dsdd{\xi_1(t)}{t}{0} )\\
     &=\Ad(g^{-1})\pi_1^*\omega_1( \Dsdd{( \xi_1(t),\xi_2(t) )}{t}{0} )\\
     &=\Ad(g^{-1})\pi_1^*\omega_1\Sigma.
\end{split}
\end{equation}

\subsection{Representations}
%----------------------------

Let $V$ be a vector space and $\dpt{\rho_i}{G_i}{GL(V)}$ be some representations such that
\begin{equation}\label{eq:cond_reprez}
   [\rho_1(g_1),\rho_2(g_2)]=0
\end{equation}
for all $g_1\in G_1$ and $g_2\in G_2$ (in the sense of commutators of matrices). In this case, one can define the representation $\dpt{\rho_1\times\rho_2}{G_1\times G_2}{GL(V)}$ by
\begin{equation}
   (\rho_1\times\rho_2)(g_1,g_2)=\rho_1(g_1)\circ\rho_2(g_2)=\rho_2(g_2)\circ\rho_1(g_1).
\end{equation}
The relation \eqref{eq:cond_reprez} is needed in order for $\rho_1\times\rho_2$ to be a representation, as one can check by writing down explicitly the requirement
\[
  (\rho_1\times \rho_2)\big(  (g_1,g_2)(g'_1,g'_2)  \big)=(\rho_1\times \rho_2)(g_1g'_1,g_2g'_2)
\]

\section{Connections}
%++++++++++++++++++++

\subsection{Gauge potentials}
%----------------------------

Let us consider a \defe{section}{section} $\salpha$ of $P$ over $\mU_{\alpha}$. It is a map $\dpt{\salpha}{\mU_{\alpha}}{P}$ such that $\pi\circ\salpha=\id$. A \defe{connection}{connection} on $P$ is a $1$-form $\dpt{\omega}{T_pP}{\yG}\in\Omega^1(P)$ which satisfies the following two conditions:
\begin{subequations}
\begin{align}
   \omega_p(Y^*_p)&=Y,   \label{conn_1}\\
   \omega(dR_g\xi)&=g^{-1}\omega(\xi)g.\label{conn_2}
\end{align}
\end{subequations}
The \defe{gauge potential}{gauge!principal potential} of $\omega$ with respect of the local section\label{PgLocSecConn} $\salpha$ is  the $1$-form on $\mU_{\alpha}$ given by
\begin{equation}
          A_{\alpha}(x)(v)=(\salpha^*\omega)_x(v).
\end{equation}
We will not always explicitly write the dependence of $A_{\alpha}$ in $x$.\nomenclature{$A_{\alpha}$}{Gauge potentials} Now we consider another section $\dpt{\sbeta}{\mU_{\beta}}{P}$ which is related on $\mU_{\alpha}\cap\mU_{\beta}$ to $\salpha$ by $\sbeta(x)=\salpha(x)\cdot g_{\alpha\beta}(x)$ for a well defined map $\dpt{g_{\alpha\beta}}{\mU_{\alpha}\cap\mU_{\beta}}{G}$.

\begin{proposition}
The gauge potentials $A_{\alpha}$ and $A_{\beta}$ are related by
\begin{equation}\label{trans_A}
                A_{\beta}=g^{-1} A_{\alpha} g-g^{-1} dg.
\end{equation}
\label{prop:trans_A}
\end{proposition}

\begin{proof}
By definition, for $v\in T_x\mU_{\alpha}$,
\[
   A_{\beta}(v)=(\sbeta^*\omega)_x(v)=
         \omega_{\salpha(x)\cdot g_{\alpha\beta}(x)}\big((d\sbeta)_x(v)\big).
\]
We begin by computing $d\sbeta(v)$. Let us take a path $v(t)$ in $\mU_{\alpha}$ such that $v(0)=x$ and $v'(0)=v$. We have:
\begin{equation}\label{eq:1407r1}
\begin{split}
   (d\sbeta)_x(v)&=\dsdd{\sbeta(v(t))}{t}{0}\\
                 &=\dsdd{\salpha(v(t))\cdot\gab(v(t))}{t}{0}\\
		 &=\Dsdd{\salpha(v(t))\cdot\gab(x)}{t}{0}
		    +\Dsdd{\salpha(x)\cdot\gab(v(t))}{t}{0}\\
		 &=dR_{\gab(x)}d\salpha(v)+\Dsdd{\salpha(x)\cdot\gab(x)e^{-ts}}{t}{0}\\
		 &=dR_{\gab(x)}d\salpha(v)+s^*_{\salpha(x)\cdot\gab(x)}
\end{split}
\end{equation}
where $s$ is defined by the requirement\label{pg:justif_s} that $\gab(x)^{-1}\gab(v(t))$ can be replaced in the derivative by $e^{-ts}$, so that we can replace $\gab(v(t))$ by $\gab(x)e^{-ts}$. As far as the derivatives are concerned, $e^{-ts}=\gab(x)^{-1}\gab(v(t))$, then
\[
     s=-\dsdd{\gab(x)^{-1}\gab(v(t))}{t}{0}=-\gab(x)^{-1} d\gab(v),
\]
this equality being a notation. Now, properties \eqref{conn_1} and \eqref{conn_2} make that
\[
   A_{\beta}(v)=\gab(x)^{-1}\omega_{\salpha(x)}(d\salpha(v))\gab(x)+s.
\]
The thesis is just the same, with ``reduced'' notations (see section~\ref{subsec:digress}).
.
\end{proof}
An explicit form for this transformation law is:
\begin{equation}
    A_{\beta}(v)=\Dsdd{g^{-1} e^{tA_{\alpha}(v)}g}{t}{0}-\Dsdd{g^{-1}\gab(v(t))}{t}{0},
\end{equation}
where $g:=\gab(x)$.

\subsection{Covariant derivative}
%-------------------------------

When we have a connection on a principal bundle, we can define a covariant derivative\index{covariant!derivative} on any associated bundle. Let us quickly review it. An associated bundle is the semi-product $E=P\times_{\rho} V$ where $V$ is a vector space on which acts the representation $\rho$ of $G$. We denote the canonical projection by $\dpt{\pi_p}{E}{M}$. The classes are taken with respect to the equivalence relation $(p,v)\sim(p\cdot g,\rho(g^{-1})v)$.

A \defe{section}{section!of associated bundle} of $E$ is a map $\dpt{\psi}{M}{E}$ such that $\pi\circ\psi=\id$. We denote by $\Gamma(E)$ the set of all the sections of $E$. A section of $E$ defines (and is defined by) an equivariant function\index{equivariant!function} $\dpt{\hpsi}{P}{V}$ such that
\begin{subequations}
\begin{align}
  \psi(\pi(\xi))&=[\xi,\hpsi(\xi)],\\
  \hpsi(\xi\cdot g)&=\rho(g^{-1})\hpsi(\xi).
\end{align}
\end{subequations}
For a section $\psi\in\Gamma(E)$, we define $\dpt{\psi\bsa}{\mU_{\alpha}}{V}$ by
 \[
 \psi\bsa(x)=\hpsi(\sigma(x)).
 \]
We saw in \eqref{eq:nabla_coord} that a covariant derivative on $E$ is given by
\begin{equation}\label{3008e1}
  (D_X\psi)\bsa(x)=X_x\psi\bsa-\rho_*\Big((\salpha^*\omega)_x(X_x)\Big)\psi\bsa(x).
\end{equation}
Since $(d\psi)(X)=X(\psi)$, we can rewrite this formula in a simpler manner by forgetting the index $\alpha$ and the mention of $X$:
\[
    D\psi=d\psi-(\rho_*A_{\alpha})\psi.
\]
If we note $(\rho_*A_{\alpha})\psi$ by $A_{\alpha}\psi$, we have
\begin{equation}
        D\psi=d\psi-A\psi.
\end{equation}
One has to understand that equation as a ``notational trick''\ for \eqref{3008e1}. By the way, remark that $(\rho_*A_{\alpha})$ is the only ``reasonable'' way for $A$ to act on $\psi$.

%---------------------------------------------------------------------------------------------------------------------------
\subsection{Gauge transformation}
%---------------------------------------------------------------------------------------------------------------------------

A \defe{gauge transformation}{gauge!transformation} of a $G$-principal bundle is a diffeomorphism $\dpt{\varphi}{P}{P}$ which satisfies
\begin{subequations}
\begin{align}
   \pi\circ\varphi&=\pi,\\
   \varphi(\xi\cdot g)&=\varphi(\xi)\cdot g.
\end{align}
\end{subequations}
In local coordinates, it can be expressed in terms of a function $\dpt{\tilde{\varphi}_{\alpha}}{\mU_{\alpha}}{G}$ by the requirement that
\begin{equation}\label{def_vpt}
    \varphi(\salpha(x))=\salpha(x)\cdot\tilde{\varphi}_{\alpha}(x).
\end{equation}

We have shown in proposition~\ref{prop:vp_conn} that, if $\omega$ is a connection $1$-form on $P$, the form $\varphi\cdot\omega:=\varphi^*\omega$ is still a connection $1$-form on $P$. Of course, with the same section $\salpha$ than before, we can define a gauge potential $(\varphi\cdot A)_{\alpha}$ for this new connection. We will see how it is related to $A_{\alpha}$. The reader can guess the result (it will be the same as proposition~\ref{prop:trans_A}). We show it.

\begin{proposition}		\label{Proptr_de_A}
\begin{equation}\label{tr_de_A}
     (\varphi\cdot A)=\tilde{\varphi}^{-1} A\tilde{\varphi}-\tilde{\varphi}^{-1} d\tilde{\varphi}.
\end{equation}
\end{proposition}
\begin{proof}
Let us consider $x\in\mU_{\alpha}$, and $v\in T_x\mU_{\alpha}$, the vector which is tangent to the curve $v(t)\in\mU_{\alpha}$. We compute
\[
    \salpha^*(\varphi^*\omega)_x(v)=\omega_{(\varphi\circ\salpha)(x)}((d\varphi\circ d\salpha)(v)),
\]
but equation \eqref{def_vpt} makes
\begin{equation}
\begin{split}
   (d\varphi\circ d\salpha)(v)&=\dsdd{\varphi(\salpha(v(t)))}{t}{0}\\
                          &=\dsdd{\salpha(v(t))\cdot\tilde{\varphi}_{\alpha}(v(t))}{t}{0}.
\end{split}
\end{equation}
Now, we are in the same situation as in equation \eqref{eq:1407r1}.
\end{proof}

If $\dpt{\psi}{M}{E}$ is a section of $E$, the gauge transformation $\dpt{\varphi}{P}{P}$ acts on $\psi$ by
\begin{equation}
   \widehat{\varphi\cdot\psi}(\xi)=\hpsi(\varphi^{-1}(\xi)).
\end{equation}
On the other hand, $\varphi$ acts on the covariant derivative (and the potential):
$\varphi\cdot D$ is the covariant derivative  of the connection $\varphi\cdot\omega$. Of course, we define
\begin{equation}
    (\varphi\cdot D)\psi=d\psi-(\varphi\cdot A)\psi.
\end{equation}

\begin{lemma}
If $\dpt{\varphi}{P}{P}$ is a gauge transformation, then
\begin{enumerate}
\item $\varphi^{-1}$ is also a gauge transformation and
             $(\widetilde{\varphi^{-1}})_{\alpha}(x)=\tilde{\varphi}_{\alpha}(x)^{-1}$, \label{lem:i}
\item $(\varphi\cdot\psi)\bsa(x)=\rho(\tilde{\varphi}_x^{-1})\psi\bsa(x)$.\label{lem:ii}
\end{enumerate}
\label{lem:prop_gauge}
\end{lemma}

\begin{proof}
The first part is clear while the second is a computation:
\begin{equation}
    (\varphi\cdot\psi)\bsa=\widehat{\varphi\cdot\psi}(\salpha(x))
                         =\hpsi(\varphi^{-1}(\salpha(x)))
			 =\hpsi(\salpha(x)\cdot\tilde{\varphi}_{\alpha}(x)^{-1})
			 =\rho(\tilde{\varphi}_{\alpha}(x))\psi\bsa(x).
\end{equation}
\end{proof}

Now, we will proof the main theorem: the one which explains why the covariant derivative is ``covariant''.

\begin{theorem}\label{th:covariance}
The covariant derivative $D$ fulfils a ``covariant'' transformation rule under gauge transformations:
\begin{equation}\label{eq:covariance_math}
      (\varphi\cdot D)(\varphi^{-1}\cdot \psi)=\varphi^{-1}(D\psi).
\end{equation}
\end{theorem}

\begin{proof}[Proof of theorem~\ref{th:covariance}]
First, we look at $(\varphi\cdot A)\psi_{\alpha}$. Using all the notational tricks used to give a sens to $A\psi$, we write:
\[
   [(\varphi\cdot A)_X\psi]\bsa(x)=(\varphi\cdot A)_X\psi\bsa(x)=\rho_*(\varphi\cdot A(X))\psi\bsa(x).
\]
But we know that $\varphi\cdot A=\tilde{\varphi}^{-1} A\tilde{\varphi}-\tilde{\varphi}^{-1} d\tilde{\varphi}$, then
\begin{equation}\label{eq:en_deux}
\begin{split}
  (\varphi\cdot A)_X\psi\bsa(x)
  &=\rho_*(\tilde{\varphi}^{-1} A(X)\tilde{\varphi})\psi\bsa(x)\\
  &\quad-\rho_*(\tilde{\varphi}^{-1} d\tilde{\varphi}(X))\psi\bsa(x)\\
  &=\Dsdd{ \rho(\tilde{\varphi}^{-1} e^{tA(X)}\tilde{\varphi})\psi\bsa(x)}{t}{0}\\
 &\quad-\Dsdd{ \rho(\tilde{\varphi}^{-1}\tilde{\varphi}(X_t))\psi\bsa(x) }{t}{0}
\end{split}
\end{equation}
Now, we have to write this equation with $\varphi^{-1}\cdot\psi$ instead of $\psi$. Using lemma~\ref{lem:prop_gauge}, we find:
\begin{equation}
\begin{split}
   (\varphi\cdot A)_X(\varphi^{-1}\cdot\psi)\bsa(x)
   &=\Dsdd{ \rho(\tilde{\varphi}^{-1} e^{tA(X)}\tilde{\varphi}\tilde{\varphi}^{-1})\psi\bsa(x)}{t}{0}\\
   &\quad-\Dsdd{ \rho(\tilde{\varphi}^{-1}\tilde{\varphi}(X_t)\tilde{\varphi}^{-1})\psi\bsa(x) }{t}{0}
\end{split}
\end{equation}
After simplification, the first term is a term of the thesis: $\tilde{\varphi}(x)^{-1}(A\psi)_{\alpha}(x)$ and we let the second one as it is. Now, we turn our attention to the second term of \eqref{eq:covariance_math}; the same argument gives:
\begin{equation}
\begin{split}
  d(\varphi^{-1}\psi\bsa)_xX
  &=\Dsdd{(\varphi^{-1}\cdot\psi)\bsa(X_t)}{t}{0}\\
  &=\Dsdd{\rho(\tilde{\varphi}(X_t)^{-1})\psi\bsa(X_t)}{t}{0}\\
  &=\Dsdd{\rho(\tilde{\varphi}(X_t)^{-1})\psi\bsa(x)}{t}{0}
  +\Dsdd{\rho(\tilde{\varphi}^{-1})\psi\bsa(X_t)}{t}{0}.
\end{split}
\end{equation}
The second term is $\tilde{\varphi}^{-1} d\psi_{\alpha}(X)$. In definitive, we need to prove that the two exceeding terms cancel each other:
\begin{equation}\label{eq:le_zero}
  \Dsdd{\rho(\tilde{\varphi}^{-1}\tilde{\varphi}(X_t)\tilde{\varphi}^{-1})\psi\bsa(x)}{t}{0}
  +\Dsdd{\rho(\tilde{\varphi}(X_t)^{-1})\psi\bsa(x)}{t}{0}
\end{equation}
must be zero.

One can find a $g(t)\in G$ such that $\tilde{\varphi}(X_t)=\tilde{\varphi} g(t)$, $g(0)=e$. Then, what we have in the $\rho$ of these two terms is respectively $g(t)\tilde{\varphi}^{-1}$ and $g(t)^{-1}\tilde{\varphi}^{-1}$. As far as the derivative are concerned, $g(t)$ can be written as $e^{tZ}$ for a certain $Z\in\yG$. So we see that $g(t)^{-1}=e^{-tZ}$ and the derivative will come with the right sign to makes the sum zero.
\end{proof}

\begin{remark}
Let us use more intuitive notations: we write \eqref{tr_de_A} under the form $A'=g^{-1} Ag-g^{-1} dg$. If we have two sections  $\psi$ and $\psi'$, they are necessarily related by a gauge transformation: $\psi'=g^{-1}\psi$. Then, the theorem tells us that the equation $D\psi=d\psi-A\psi$ becomes $D'\psi'=g^{-1} D\psi$ ``under a gauge transformation''. This is: $D\psi$ transforms under a gauge transformation as $d\psi$ transforms under a constant linear transformation. This is the reason why $D$ is a \emph{covariant} derivative. The physicist way to write \eqref{eq:covariance_math} is
\begin{equation}\label{eq:covariance_phys}
    D'\psi'=g^{-1} D\psi
\end{equation}
\label{rem:intuitif}
\end{remark}

\begin{remark}
If we naively make the computation with the notations of remark~\ref{rem:intuitif}, we replace $\psi'=g^{-1}\psi$ and $A'=g^{-1} Ag-g^{-1} dg$ in
\[
  D'\psi'=d\psi'-A'\psi',
\]
using some intuitive ``Leibnitz formulas'', we find:
$D'\psi'=dg^{-1}\psi+g^{-1} d\psi+g^{-1} A\psi+g^{-1} dg g^{-1}\psi$. It is exactly $g^{-1} d\psi+g^{-1} A\psi$ with two additional terms: $dg^{-1}\psi$ and $g^{-1} dg g^{-1}\psi$. One sees that these are precisely the two terms of the expression \eqref{eq:le_zero}. We will give a sens to this ``naive''\ computation in section~\ref{subsec:digress}.
\end{remark}

%---------------------------------------------------------------------------------------------------------------------------
\subsection{Curvature}
%---------------------------------------------------------------------------------------------------------------------------

From the $\yG$-valued connection $1$-form $\omega$ on $P$, we may define its \defe{curvature $2$-form}{curvature}:
\begin{equation}
     \Omega=d\omega+\omega\wedge\omega.
\end{equation}
As before, we can see $\Omega$ as a $2$-form on $M$ instead of $P$. For this, we just need some sections $\dpt{\salpha}{\mU_{\alpha}}{P}$ and define
\begin{equation}
        F_{\alpha}=\salpha^*\Omega.
\end{equation}
This $F$ is called the \defe{Yang-Mills field strength}{Yang-Mills!field strength}. The question is now to see how does it transform under a change of chart? What is $F_{\beta}=\sbeta^*\Omega$ in terms of~$F_{\alpha}$?

First, note that we can't try to find a relation like $d(g\omega)=dg\wedge\omega+g\,d\omega$. Pose $A_x=g(x)\omega_x$:
\[
  A_x(v)=\dsdd{g(x)e^{t\omega_x(v)}}{t}{0}.
\]
Using
\[
   (d\alpha)(v,w)=v(\alpha(w))-w(\alpha(v))-\alpha([v,w]),
\]
we are led to write
\begin{equation}
     w(A(v))=d(A(v))w
            =\dsdd{A_{w_u}(v)}{u}{0}
	    =\dsdd{ \Dsdd{g(w_u)e^{t\omega_{w_u}(v)}}{t}{0} }{u}{0}.
\end{equation}
But at $t=u=0$, the expression in the bracket is $g(x)$, and not $e$. Then the whole expression is not an element of $\yG$. In other words, the problem is that for $\dpt{g}{M}{G}$, we have $\dpt{dg_x}{T_xM}{T_{g(x)}G\neq T_eG}$.

Now, remark that in our matter, the problem will not arise because in the expressions $A_{\beta}=g^{-1} A_{\alpha} g+g^{-1} dg$, each term has a $g$ and a $g^{-1}$.

\begin{lemma}
\begin{equation}
   d(g^{-1})_x(v)=-g(x)^{-1} dg(v)g(x)^{-1}.
\end{equation}
\label{lem:dgemu}
\end{lemma}

\begin{proof}
Let $v_t$ be a path which defines the vector $v$, and define $Y\in\yG$ such that as far as the derivative are concerned, we have $g(v_t)=g(x)e^{tY}$. Then,
\[
      g(g^{-1})(v)=\Dsdd{g(v_t)^{-1}}{t}{0}=\Dsdd{e^{-tY}g(x)^{-1}}{t}{0}.
\]
But on the other hand,
\[
  g^{-1} dg(v)g^{-1}=\Dsdd{g(x)^{-1} g(v_t)g(x)^{-1}}{t}{0}=\Dsdd{e^{tY}g(x)^{-1}}{t}{0},
\]
thus $d(g^{-1})_x(v)=-g(x)^{-1} dg(v)g(x)^{-1}$, as we want.
\end{proof}

\begin{theorem}
\begin{equation}
     F_{\beta}=g^{-1} F_{\alpha} g.
\end{equation}
\label{tho:trans_F}
\end{theorem}

\begin{proof}[Naive proof]
Let us accept $F_{\beta}=dA_{\beta}+A_{\beta}\wedge A_{\beta}$. With proposition~\ref{prop:trans_A}, we can perform a simple computation with all the intuitive ``Leibnitz rules'':
\[
   dA_{\beta}=-g^{-1} dg\, g^{-1}\wedge A_{\alpha} g+g^{-1} dA_{\alpha} g+g^{-1} A_{\alpha}\wedge dg-g^{-1} dg\,g^{-1}\wedge dg,
\]
and
\[
  A_{\beta}\wedge A_{\beta}=g^{-1} A_{\alpha} g\wedge g^{-1} A_{\alpha} g+g^{-1} A_{\alpha} g\wedge g^{-1} dg+g^{-1} dg\wedge g^{-1} A_{\alpha} g+g^{-1} dg\wedge g^{-1} dg.
\]
The sum is obviously the announced result.
\end{proof}
This proof seems too beautiful to be false\footnote{More precisely, it is as beautiful as we want it to be true.}. Here is the full proof.

\begin{proof}[Ultimate proof of theorem~\ref{tho:trans_F}]
First we compute $d(g^{-1} A_{\alpha} g)$. In order to do this, remark that the $1$-form $g^{-1} A_{\alpha} g$ is explicitly given on $v\in\cvec(M)$ by
\[
   (g^{-1} A_{\alpha} g)(v)_x=\Dsdd{g(x)^{-1} e^{tA(v)_x}g(x)}{t}{0}.
\]
For all $x\in M$, this expression is an element of $\yG$; then we can say that $(g^{-1} A_{\alpha} g)(v)$ is a map $\dpt{(g^{-1} A_{\alpha} g)(v)}{M}{\yG}$. So it is unambiguous to write $w((g^{-1} A_{\alpha} g)(v))\in\yG$ for $w\in T_xM$.

We will use the formula
\[
   d(g^{-1} A_{\alpha} g)(v,w)=v(g^{-1} A_{\alpha} g)(w)-w(g^{-1} A_{\alpha} g)(v)-(g^{-1} A_{\alpha} g)([v,w]).
\]
As $w((g^{-1} A_{\alpha} g)(v))=d((g^{-1} A_{\alpha} g)(v))w$, we have
\begin{equation}
\begin{split}
    w((g^{-1} A_{\alpha} g)(v))&=\dsdd{(g^{-1} A_{\alpha} g)(v)_{w_u}}{u}{0}\\
                &=\dsdd{ \Dsdd{g(w_u)^{-1} e^{tA(v)_{w_u}}g(w_u)}{t}{0}  }{u}{0}\\
		&=\dsdd{  \Dsdd{g(w_u)^{-1}}{u}{0} e^{tA(v)_x}g(x)  }{t}{0}\\
		&\quad+\dsdd{ g(x)^{-1} \Dsdd{e^{tA(v)_{w_u}}}{u}{0} g(x)  }{t}{0}\\
		&\quad+\dsdd{  g(x)^{-1} e^{tA(v)_x} \Dsdd{g(w_u)}{u}{0}  }{t}{0}\\
		&=d(g^{-1})(w)A(v)_xg(x)\\
		&\quad+g(x)^{-1} w_x(A(v))g(x)\\
		&\quad+g(x)^{-1} A(v)_x dg(w).
\end{split}
\end{equation}
On the other hand, one easily finds that
\[
     (g^{-1} A_{\alpha} g)([v,w])=g(x)^{-1} A([v,w])g(x).
\]
 Using lemma~\ref{lem:dgemu}, we have
\begin{equation}
\begin{split}
   d(g^{-1} A_{\alpha} g)_x(v,w)&=-g(x)^{-1} dg(v)g(x)^{-1} A(w)_xg(x)+g(x)^{-1} v(A(w))g(x)\\&\quad+g(x)^{-1} A(w)_xdg(v)_x\\
                   &\quad+g(x)^{-1} dg(w)_xg(x)^{-1} A(v)_xg(x)-g(x)^{-1} w(A(v))g(x)\\&\quad-g(x)^{-1} A(v)_xdg(w)\\
		   &\quad-g(x)^{-1} A([v,w])g(x).
\end{split}
\end{equation}
We can regroup the terms two by two in order to form $dA_{\alpha}$ and some wedge; with simpler notations, we can write:
\begin{equation}\label{eq:dA_1}
  d(g^{-1} A_{\alpha} g)=-(g^{-1} dg\,g)\wedge(A_{\alpha} g)-(g^{-1} A)\wedge dg+(g^{-1} dA g).
\end{equation}
We compute $d(g^{-1} dg)$ in the same way; the result is
\[
   (g^{-1} dg)(v)_x=\Dsdd{g(x)^{-1} g(v_t)}{t}{0}\in\yG.
\]
For $v$, $w\in\cvec(M)$, we have:
\begin{equation}
\begin{split}
   w\big((g^{-1} dg)(v)\big)&=\dsdd{ (g^{-1} dg)(v)_{w_u} }{u}{0}\\
                   &=\DDsdd{  g(w_u)^{-1} g(v_{w_u}(t))  }{u}{0}{t}{0}\\
		   &=\DDsdd{  g(w_u)^{-1} g(v_t)  }{t}{0}{u}{0}
		      +\DDsdd{  g(x)^{-1} g(w_u(t))  }{t}{0}{u}{0}\\
		   &=d(g^{-1})(w)dg(v)+\Dsdd{ g(x)^{-1} dg(v_{w_u}) }{u}{0}\\
		   &=-g^{-1} dg(w)g^{-1} dg(v)+g(x)^{-1} w(dg(v))
\end{split}
\end{equation}
where $w_u$ is a path such that $w'_0=w_x$ and $v_{w_u}(t)$ is, with respect of $t$, a path which gives the vector $v_{w_u}$. On the another hand, we have
\[
   (g^{-1} dg)([v,w])=g^{-1} dg([v,w]).
\]

Remark that the term $g(x)^{-1} w(dg(v))$ of  $w((g^{-1} dg)(v))$ together with the same with $v\leftrightarrow w$ and $(g^{-1} dg)([v,w])$ which comes from  $(g^{-1} dg)([v,w])$ will give $g(x)^{-1}(d^2g)(v,w)=0$ when we will compute $d(g^{-1} dg)$.
Finally,
\begin{equation}\label{eq:dA_2}
   d(g^{-1} dg)=-(g^{-1} dg\,g^{-1}\wedge dg).
\end{equation}
The equations \eqref{eq:dA_1} and \eqref{eq:dA_2} allow us to write:
\begin{equation}\label{eq:dA}
\begin{split}
    (dA_{\beta})&=d(g^{-1} A_{\alpha} g)+d(g^{-1} dg)\\
              &=-(g^{-1} dg\,g^{-1}) \wedge(A_{\alpha} g)-(g^{-1} A_{\alpha})\wedge dg\\
	      &\quad+(g^{-1} dA_{\alpha} g)-(g^{-1} dg\,g^{-1})\wedge dg.
\end{split}
\end{equation}
Notice that the term $(g^{-1} dA_{\alpha} g)$ corresponds to the first one in $F_{\beta}=g^{-1}(dA_{\beta}+A_{\beta}\wedge A_{\beta})g$.

For anyone who had understood the whole computations up to here, it is clear that
\begin{equation}
\begin{split}
     [A_{\beta}(v),A_{\beta}(w)]&=\DDsdd{ e^{tA_{\beta}(v)}e^{tA_{\beta}(w)} }{t}{0}{u}{0}\\
                            &\quad-\DDsdd{ e^{tA_{\beta}(w)}e^{tA_{\beta}(v)} }{t}{0}{u}{0}\,,
\end{split}
\end{equation}
so that
\begin{equation}
\begin{split}
  A_{\beta}\wedge A_{\beta}&=g^{-1} A_{\alpha} g\wedge g^{-1} A_{\alpha} g
                         +g^{-1} A_{\alpha} g\wedge g^{-1} dg\\
		       &\quad+g^{-1} dg\wedge g^{-1} A_{\alpha} g
		       +g^{-1} dg\wedge g^{-1} dg.
\end{split}
\end{equation}
Lemma~\ref{lem:simplif} allows us to write it under the form
\begin{equation}\label{eq:AA}
\begin{split}
  A_{\beta}\wedge A_{\beta}&=g^{-1} A_{\alpha} g\wedge g^{-1} A_{\alpha} g
                         +g^{-1} A_{\alpha} g\wedge g^{-1} dg\\
		       &\quad+g^{-1} dg\wedge g^{-1} A_{\alpha} g
		       +g^{-1} dg\wedge g^{-1} dg.
\end{split}
\end{equation}
Here the term $(g^{-1} A_{\alpha}\wedge A_{\alpha} g)$ corresponds to the second one in $F_{\beta}=g^{-1}(dA_{\beta}+A_{\beta}\wedge A_{\beta})g$. The sum of \eqref{eq:dA} and \eqref{eq:AA} is
\[
    F_{\beta}=g^{-1} F_{\alpha} g.
\]
\end{proof}

%+++++++++++++++++++++++++++++++++++++++++++++++++++++++++++++++++++++++++++++++++++++++++++++++++++++++++++++++++++++++++++
\section{Hodge decomposition theorem and harmonic forms}
%+++++++++++++++++++++++++++++++++++++++++++++++++++++++++++++++++++++++++++++++++++++++++++++++++++++++++++++++++++++++++++

Among other sources for Hodge decomposition and harmonic forms, we have \cite{JohnsonHodge,CohoHarBound,UndergradDeRham}. Some parts of the wikipedia article \wikipedia{en}{Hodge_dual}{Hodge\_dual} are interesting.

Let $E$ be an oriented Euclidian space of dimension $m=2n$. We define the operation $*$ by
\begin{equation}		\label{EqGradWedge}
	\begin{aligned}
		*\colon \Wedge E&\to \Wedge E \\
		e_{i_1}\wedge e_{i_2}\wedge\cdots\wedge e_{i_k}&\mapsto e_{i_{k+1}}\wedge\cdots\wedge e_{i_m}
	\end{aligned}
\end{equation}
when $\{ e_i \}$ is an oriented basis of $E$ and $\{ i_k \}$ is an even permutation of $\{ 1,2,\cdots,m \}$. If it is impossible to build an even permutation, then we add a minus sign. We have $**\omega=(-1)^{p(m-p)}\omega$ belongs to $\omega\in\Wedge^pE$.

\begin{example}
	If we consider the space $\eR^4$ with the coordinates $(x,y,z,t)$,
	\begin{equation}
		*(dx\wedge dz\wedge dt)=dy
	\end{equation}
	because $(x,z,t,y)$ is an even permutation of $(x,y,z,t)$. Now, $*dy=dz\wedge dx\wedge dt=-(dx\wedge dz\wedge dt)$ because $(y,z,x,t)$ is an even permutation of $(x,y,z,t)$.

	More generally, if we have a differential $p$-form $\omega$ on a $m$ dimensional space, we have
	\begin{equation}
		*(e_{\sigma(1)}\wedge e_{\sigma(2)}\wedge\ldots\wedge e_{\sigma(p)})=e_{\sigma(p+1)}\wedge \ldots\wedge e_{\sigma(m)}
	\end{equation}
	In order to compute $*(e_{\sigma(p+1)}\wedge \ldots\wedge e_{\sigma(m)})$, we need a permutation of $\big( \sigma(1),\ldots, \sigma(m)\big)$ which \emph{begins} by $\sigma(p+1)\ldots\sigma(m)$. This reduces to permute the $m-p$ elements $\sigma(p+1),\ldots,\sigma(m)$ with the $p$ first elements. Thus we have
	\begin{equation}
		**\omega=(-1)^{p(m-p)}\omega.
	\end{equation}
\end{example}

Let $V$ be a compact, oriented manifold. Each of the spaces of sections $ C^{\infty}\big( V, \Wedge^k_{\eC}(T^*V)\big)$ is endowed with a $2$-form
\begin{equation}		\label{EqProdWedgeHOfge}
	\langle \omega_1, \omega_2\rangle =\int_V\omega_1\wedge *\omega_2.
\end{equation}

\begin{lemma}
	The \defe{codifferential}{codifferential} $\delta$ defined by
	\begin{equation}
		\begin{aligned}[]
			\delta	\colon \Wedge^k_{\eC}(T^*V)&\to \Wedge_{\eC}^{k-1}(T^*V)\\
			\beta				&\mapsto (-1)^{mk+m+1}*d*\beta
		\end{aligned}
	\end{equation}
	is a formal adjoint of $d$ for the product \eqref{EqProdWedgeHOfge}.
\end{lemma}

\begin{proof}
	If $\beta\in\Wedge_{\eC}^k(T^*V)$ and $\alpha\in\Wedge_{\eC}^{k-1}(T^*V)$, we have
	\begin{equation}
		\begin{aligned}[]
			*\delta\beta&=(-1)^{mk+m+1}* *\big( d*\beta \big)\\
			&=(-1)^{mk+m+1}(-1)^{(m-k+1)(m-m+k-1)}d*\beta\\
			&=(-1)^kd*\beta.
		\end{aligned}
	\end{equation}
	Using that formula we find
	\begin{equation}
		\begin{aligned}[]
			\langle d\alpha, \beta\rangle -\langle \alpha, \delta\beta\rangle &=\int_V d\alpha\wedge *\beta-\alpha\wedge *\delta\beta\\
			&=\int_Vd\alpha\wedge *\beta-(-1)^k\alpha\wedge d*\beta\\
			&=\int_Vd\alpha\wedge *\beta+(-1)^{k+1}\alpha\wedge d*\beta\\
			&=\int_Vd(\alpha\wedge *\beta)\\
			&=\int_{\partial V}\alpha\wedge *\beta\\
			&=0.
		\end{aligned}
	\end{equation}
	This proves that $\langle d\alpha, \beta\rangle =\langle \alpha, \delta\beta\rangle$.
	\begin{probleme}
		I do not understand why the integral in the boundary is zero.
	\end{probleme}
\end{proof}

Now we define the \defe{Laplace-Beltrami operator}{Laplace-Beltrami operator}\index{operator!Laplace-Beltrami} by\nomenclature[D]{$\Delta$}{Laplace-Beltrami operator}
\begin{equation}
	\Delta=\delta d+d\delta
\end{equation}
and the space of \defe{harmonic forms}{harmonic form}
\begin{equation}
	H^k=\{ \omega\in\Omega^k\tq\Delta\omega=0 \}.
\end{equation}

\begin{lemma}
	If $M$ is a closed manifold, a $k$-form is harmonic if and only if $d\omega=\delta\omega=0$.
\end{lemma}

\begin{proof}
	No proof.
\end{proof}

\begin{theorem}[Hodge decomposition theorem]\index{theorem!Hodge decomposition}
	For every integer $0\leq k\leq m$, the space $H^p$ is finite dimensional and $\Omega^k(M)$ has the orthogonal decomposition
	\begin{equation}
		\Omega^k(M)=H^k\oplus\Delta\big( \Omega^k(M) \big),
	\end{equation}
	i.e. the space splits into the kernel of $\Delta$ and its image.
\end{theorem}

\begin{theorem}
	Let $M$ be a compact orientable manifold of dimension $m$. Any exterior differential $k$-form can be written as a unique sum of an exact form, a coexact form and an harmonic form:
	\begin{equation}
		\omega=d\alpha+\delta\beta+\gamma.
	\end{equation}
	with $\omega\in\Omega^k(M)$, $\alpha\in\Omega^{k-1}(M)$, $\beta\in\Omega^{k+1}(M)$ and $\gamma\in\Omega^k(M)$ harmonic.
\end{theorem}

The operator $\Delta$ commutes with the differential $d$ and we have $d\Delta=\Delta d= d\delta d$ since
\begin{equation}
	d\Delta \omega=dd\delta\omega+d\delta d\omega=d\delta d\omega,
\end{equation}
because $d^2=0$, while
\begin{equation}
	\Delta d\omega=d\delta d\omega+\Delta d d\omega=d\delta d\omega.
\end{equation}

\begin{lemma}
	On a close manifold, $\Delta\omega=0$ if and only if $\delta\omega=d\omega=0$.
\end{lemma}

In the case of a closed manifold, a form is harmonic if and only if is belongs to the kernel of $d+\delta$. Moreover, a form in the image of $d+\delta$ is orthogonal to the harmonic forms:
\begin{equation}
	\langle d\alpha^{k-1}+\delta\beta^{k+1}, \gamma\rangle =0
\end{equation}
whenever $\gamma$ is harmonic on a closed manifold.


\chapter{Examples of groups and representations}        \label{ChapThoComsGroupes}
\input{107_iwasawa}
% This is part of (almost) Everything I know in mathematics and physics
% Copyright (c) 2013-2014, 2019
%   Laurent Claessens
% See the file fdl-1.3.txt for copying conditions.

\section{Explicit choices in \texorpdfstring{$\so(2,l-1)$}{so2l-1}}
%\label{app_calc}
%+++++++++++++++++++++++++++++++++++++++++++

In the case of $AdS_4$ the matrices are $5\times 5$. We will write them down, but the general form are entirely similar. Our choice of Iwasawa decomposition is
\begin{subequations}
\begin{align}
\sN&=\{W_i,V_j,M,L\}\\
\sA&=\{ J_1, J_2\}.
\end{align}
\end{subequations}


The basis of $\sodn$ in which we want to decompose all our elements is the root space one:
\begin{equation}
\sG=\Span\{J_1,q_1,X,Y,V,W,M,N,F,L\}
\end{equation}
note in particular that $\sG_{(0,0)}=\Span\{J_1,q_1\}$ and $W,J_1\in\sH$.
\begin{equation}
\frac{1}{2}(W-Y)=
\begin{pmatrix}
&0\\
0&0&0&0&1\\
&0\\
&0\\
&1
\end{pmatrix},
\qquad
\frac{1}{2}(V+X)=
\begin{pmatrix}
&&&&0\\
&&&&0\\
&&&&1\\
&&&&0\\
0&0&-1&0&0
\end{pmatrix},
\end{equation}



\begin{equation}
\frac{1}{2}(W+Y)=
\begin{pmatrix}
&&&&0\\
&&&&0\\
&&&&0\\
&&&&1\\
0&0&0&-1&0
\end{pmatrix},
\qquad
q_3=\frac{1}{2}(V-X)=
\begin{pmatrix}
0&0&0&0&1\\
0\\
0\\
0\\
1
\end{pmatrix}.
\end{equation}


\subsection{Decompositions and commutators for \texorpdfstring{$\sQ$}{Q}}
%-------------------------------------------------------------------------

First, the root space decomposition of the basis $\{q_i\}$ of $\sQ$:
\begin{subequations}
\begin{align}
	q_0	&=\frac{1}{ 4 }(M+N+L+F)				&	q_2	&=\frac{1}{ 4 }(N+F-M-L)\\
		&=\frac{1}{ 4 }(X_{11}+X_{1,-1}+X_{-1,1}+X_{-1,-1})	&		&=\frac{1}{ 4 }(X_{-1,1}+X_{-1,-1}-X_{11}-X_{1,-1})\\
	q_1	&=q_1=J_2						&	q_3	&=\frac{1}{2}(V-X)\\
		&							&		&=\frac{ 1 }{2}(X_{01}-X_{0,-1}).
\end{align}
\end{subequations}
The commutators:
\begin{subequations}
\begin{align}
[q_0,q_1]&=\us{4}(L+F-M-N) &[q_1,q_2]&=\us{4}(L+N-F-M)\\\
&=\frac{1}{ 4 }(X_{1,-1}+X_{-1,-1}-X_{11}-X_{-1,1})	&&=\frac{1}{ 4 }(X_{1,-1}+X_{-1,1}-X_{-1,1}-X_{11})\\
[q_0,q_2]&=-J_1            &[q_1,q_3]&=\frac{1}{2}(V+X)\\
[q_0,q_3]&=\frac{1}{2}(Y-W)      &[q_2,q_3]&=\frac{1}{2}(W+Y)
\end{align}
\end{subequations}

\subsection{Commutators between root spaces and \texorpdfstring{$\sQ$}{Q}}

\begin{subequations}
\begin{align}
[q_0,J_1]	&=\frac{1}{4}(N+F-M-L)					&[q_1,J_1]&=0	&[q_2,J_1]&=q_0\\
		&=\frac{1}{ 4 }(X_{-1,1}+X_{-1,-1}-X_{11}-X_{1,-1})					\\
		&=q_2\\
[q_0,q_1]	&=\frac{1}{4}(L+F-M-N)&       				&    &[q_2,q_1]&=\us{4}(F+M-L-N)\\
		&=\frac{ 1 }{ 4 }(X_{1,-1}+X_{-1,-1}-X_{11}-X_{-1,1})\\
[q_0,X]  &=\frac{1}{2}(W-Y)          &[q_1,X]  &=-X &[q_2,X]&=-\frac{1}{2}(W+Y)\\
[q_0,Y]  &=\frac{1}{2}(X-V)          &[q_1,Y]  &=0  &[q_2,Y]&=\frac{1}{2}(V-X)\\
[q_0,V]  &=\frac{1}{2}(Y-W)          &[q_1,V]  &=V  &[q_2,V]&=\frac{1}{2}(W+Y)\\
[q_0,W]  &=\frac{1}{2}(V-X)          &[q_1,W]  &=0  &[q_2,W]&=\frac{1}{2}(V+X)\\
[q_0,M]  &=q_1+J_1             &[q_1,M]  &=M  &[q_2,M]&=q_1+J_1\\
[q_0,N]  &=q_1-J_1             &[q_1,N]  &=N  &[q_2,N]&=-q_1+J_1\\
[q_0,L]  &=-q_1+J_1            &[q_1,L]  &=-L &[q_2,L]&=-q_1+J_1\\
[q_0,F]  &=-q_1-J_1            &[q_1,F]  &=-F &[q_2,F]&=q_1+J_1
\end{align}
\end{subequations}

\begin{subequations}
\begin{align}
 [q_3,J_1]&=0           &[q_3,M]&=W   &[q_3,V]&=-q_1\\
 [q_3,q_1]&=-\frac{1}{2}(V+X) &[q_3,N]&=-Y  &[q_3,W]&=\frac{1}{2}(M+L)\\
          &             &[q_3,L]&=W   &[q_3,X]&=-q_1\\
          &             &[q_3,F]&=-Y  &[q_3,Y]&=-\frac{1}{2}(N+F)
\end{align}
\end{subequations}

\subsection{Commutators in the root spaces}

\begin{subequations}
\begin{align}
[J_1,q_1]&=0\\
[J_1,X]&=0&[q_1,X]&=-X\\
[J_1,Y]&=-Y&[q_1,Y]&=0&[X,Y]&=F\\
[J_1,V]&=0&[q_1,V]&=V&[X,V]&=2q_1\\
[J_1,W]&=W&[q_1,W]&=0&[X,W]&=-L\\
[J_1,M]&=M&[q_1,M]&=M&[X,M]&=-2W\\
[J_1,N]&=-N&[q_1,N]&=N&[X,N]&=2Y\\
[J_1,F]&=-F&[q_1,F]&=-F&[X,F]&=0\\
[J_1,L]&=L&[q_1,L]&=-L&[X,L]&=0
\end{align}
\end{subequations}

\begin{subequations}
\begin{align}
[V,W]&=M\\
[V,M]&=0&[W,M]&=0\\
[V,N]&=0&[W,N]&=-2V&[M,N]&=0\\
[V,F]&=-2Y&[W,F]&=2X&[M,F]&=-4q_1-4J_1\\
[V,L]&=2W&[W,L]&=0&[M,L]&=0
\end{align}
\end{subequations}


\begin{subequations}
\begin{align}
[N,F]&=0\\
[N,L]&=-4q_1+4J_1&[F,L]&=0
\end{align}
\end{subequations}

\subsection{Killing form}
%++++++++++++++++++++
The adopted definition is $B(x,y)=\tr(\ad x\circ\ad y)$ with no one half or such coefficient.
\begin{equation}
\begin{aligned}
B(J_1,q_1)&=0	&B(V,X)&=-12\\
B(J_1,J_1)&=6	&B(N,L)&=-24\\
B(W,Y)&=-12	&B(M,F)&=-24
\end{aligned}
\end{equation}
Some easy computations show that for $g\in \SO(2)$,
\[
\begin{split}
dL_gq_0&=
\begin{pmatrix}
-\sin u&\cos u\\
-\cos u&\sin u
\end{pmatrix},
\quad
dL_g q_1=
\begin{pmatrix}
0&0&\cos u\\
0&&-\sin u\\
1
\end{pmatrix}\\
dL_g H_1&=
\begin{pmatrix}
0&0&\sin u\\
0&&\cos u\\
0&1
\end{pmatrix}\\
dR_g J_1&=
\begin{pmatrix}
0\\
0&0&0&1\\
0\\
-\sin u&\cos u
\end{pmatrix},
\quad
dR_g J_2=
\begin{pmatrix}
0&0&1\\
0\\
\cos u&\sin u
\end{pmatrix}
\end{split}
\]
So
\begin{subequations}
\begin{align}
dR_g J_1&=-\sin u\, dL_g q_2+\cos u\, dL_g H_2\\
dR_g J_2&=\sin u\, dL_g H_1+\cos u\, dL_g q_1.
\end{align}
\end{subequations}
and
\begin{subequations}
\begin{align}
  B_{[g]}(J_1^*,J_1^*)&=6\sin^2 u\\
B_{[g]}(J_2^*,J_2^*)&=6\cos^2 u.
\end{align}
\end{subequations}


\section{Iwasawa decomposition for \texorpdfstring{$\gsl(2,\eC)$}{sl2C}}		\label{SecIwasldeuxC}
%++++++++++++++++++++++++++++++++++++++++++++++++++++++++++++++++++++++++
\index{Iwasawa!decomposition!of $\SL(2,\eC)$}

Matrices of $\gsl(2,\eC)$ are acting on $\eC^2$ as
\[
\begin{split}
  \begin{pmatrix}
\alpha&\beta\\\gamma&-\alpha
\end{pmatrix}&
\begin{pmatrix}
a+bi\\c+di
\end{pmatrix}\\
&=
\begin{pmatrix}
(\alpha_1a-\alpha_2b+\beta_1c-\beta_2d)+i(\alpha_2a+\alpha_1b+\beta_2c+\beta_1d)\\
(\gamma_1a-\gamma_2b-\alpha_1c+\alpha_2d)+i(\gamma_2a+\gamma_1b-\alpha_2c-\alpha_1d)
\end{pmatrix}
\end{split}
\]
if $\alpha=\alpha_1+i\alpha_2$.  Our aim is to embed $\SL(2,\eC)$ in $\SP(2,\eR)$ (see sections~\ref{SecSympleGp} and~\ref{SecDirADs}), so that we want a four dimensional realization of $\gsl(2,\eC)$. It is easy to rewrite the previous action under the form of $\begin{pmatrix}
\alpha&\beta\\\gamma&-\alpha
\end{pmatrix}$ acting of the vertical four component vector $(a,b,c,d)$. The result is that a general matrix of $\gsl(2,\eC)$ reads
\begin{equation}		\label{EqGenslMatr}
\gsl(2,\eC)\leadsto
\begin{pmatrix}
\boxed{
\begin{array}{cc}
\alpha_1&-\alpha_2\\
\alpha_2&\alpha_1
\end{array}
}&
\begin{array}{cc}
\beta_1&-\beta_2\\
\beta_2&\beta_1
\end{array}\\
\begin{array}{cc}
\gamma_1&-\gamma_2\\
\gamma_2&\gamma_1
\end{array}&
\boxed{
\begin{array}{cc}
-\alpha_1&\alpha_2\\
-\alpha_2&-\alpha_1
\end{array}
}
\end{pmatrix}.
\end{equation}
The boxes are drawn for visual convenience.  Using the Cartan involution $\theta(X)=-X^t$, we find the following Cartan decomposition:
\begin{equation}
\begin{split}
\iK_{\gsl(2,\eC)}&\leadsto
\begin{pmatrix}
\boxed{
\begin{array}{cc}
0&-\alpha_2\\
\alpha_2&0
\end{array}
}&
\begin{array}{cc}
\beta_1&-\beta_2\\
\beta_2&\beta_1
\end{array}\\
\begin{array}{cc}
-\beta_1&-\beta_2\\
\beta_2&-\beta_1
\end{array}&
\boxed{
\begin{array}{cc}
0&\alpha_2\\
-\alpha_2&0
\end{array}
}
\end{pmatrix},\\
\iP_{\gsl(2,\eC)}&\leadsto
\begin{pmatrix}
\boxed{
\begin{array}{cc}
\alpha_1&0\\
0&\alpha_1
\end{array}
}&
\begin{array}{cc}
\beta_1&-\beta_2\\
\beta_2&\beta_1
\end{array}\\
\begin{array}{cc}
-\beta_1&-\beta_2\\
\beta_2&-\beta_1
\end{array}&
\boxed{
\begin{array}{cc}
0&\alpha_2\\
-\alpha_2&0
\end{array}
}
\end{pmatrix}.
\end{split}
\end{equation}
We have $\dim\iP_{\gsl(2,\eC)}=3$ and $\dim\iP_{\gsl(2,\eC)}=3$. A maximal abelian subalgebra of $\iP_{\gsl(2,\eC)}$ is the one dimensional algebra generated by
\[
  A_1=
\begin{pmatrix}
1\\&1\\&&-1\\&&&-1
\end{pmatrix}.
\]
The corresponding root spaces are
\begin{itemize}
\item $\gsl(2,\eC)_0$:
\[
  I_1=
\begin{pmatrix}
1\\&1\\&&-1\\&&&-1
\end{pmatrix},\quad
I_2=
\begin{pmatrix}
0&-1\\
1&0\\
&&0&1\\
&&-1&0
\end{pmatrix}
\]
\item $\gsl(2,\eC)_2$:
\[
  D_1=\begin{pmatrix}
&&1&0\\
&&0&1\\
0&0\\
0&0
\end{pmatrix},\quad
D_2=
\begin{pmatrix}
&&0&-1\\
&&1&0\\
0&0&\\
0&0&
\end{pmatrix}
\]
\item $\gsl(2,\eC)_{-2}$
\[
  C_1=\begin{pmatrix}
&&0&0\\
&&0&0\\
1&0\\
0&1
\end{pmatrix},\quad
C_2=\begin{pmatrix}
&&0&0\\
&&0&0\\
0&-1\\
1&0
\end{pmatrix}.
\]
\end{itemize}
It is natural to choose $\gsl(2,\eC)_2$ as positive root space system. In this case, $\iN_{\gsl(2,\eC)}=\{ D_1,D_2 \}$, $\iA_{\gsl(2,\eC)}=\{ I_1 \}$ and the table of $\iA\oplus\iN$ is
\begin{align}
[I_1,D_1]&=2D_1&		[D_1,D_2]&=0\\
[I_1,D_2]&=2D_2&
\end{align}

    The full table is
\begin{align}
[I_1,D_1]&=2D_1&	[I_2,D_1]&=2D_2&	[D_1,D_2]&=0\\
[I_1,D_2]&=2D_2&	[I_2,D_2]&=-2D_1&	[D_1,C_1]&=I_1\\
[I_1,C_1]&=-2C_1&	[I_2,C_1]&=-2C_2&	[D_1,C_2]&=I_2\\
[I_1,C_2]&=-2C_2&	[I_2,C_2]&=2C_1&	[D_2,C_1]&=I_2\\
	&	&		&     &		[D_2,C_2]&=-I_1.
\end{align}
\section{Symplectic group}		\label{SecSympleGp}
%+++++++++++++++++++++++++

\subsection{Iwasawa decomposition}
%-----------------------------
\index{Iwasawa!decomposition!of $\SP(2,\eR)$}

A simple computation shows that $4\times 4$ matrices subject to $A^t\Omega+\Omega A=0$ are given by
\[
  \begin{pmatrix}
A&B\\
C&-A^t
\end{pmatrix}
\]
where $A$ is any $2\times 2$ matrix while $B$ and $C$ are symmetric matrices. Looking at general form \eqref{EqGenslMatr}, we see that the operation to invert the two last column and then to invert the two last lines provides a homomorphism $\phi\colon \gsl(2,\eC)\to \gsp(2,\eR)$. The aim is now to build an Iwasawa decomposition of $\gsp(2,\eR)$ which ``contains'' the one of $\gsl(2,\eC)$.

Using the Cartan involution $\theta(X)=-X^t$, we find the Cartan decomposition
\begin{align}
\iK_{\gsp(2,\eR)}&\leadsto
\begin{pmatrix}
A&S\\-S&A
\end{pmatrix},
&\iP_{\gsp(2,\eR)}&\leadsto
\begin{pmatrix}
S&S'\\S'&-S
\end{pmatrix}
\end{align}
where $S$ and $S'$ are any symmetric matrices while $A$ is a skew-symmetric one. We have $\dim\iK_{\gsp(2,\eR)}=4$ and $\dim\iP_{\gsp(2,\eR)}=6$. It turns out that $\phi(\iK_{\gsl(2,\eC)})\subset\iK_{\gsp(2,\eR)}$ and $\phi(\iP_{\gsl(2,\eC)})\subset \iP_{\gsp(2,\eR)}$. A maximal abelian subalgebra of $\iP_{\gsp(2,\eR)}$ is spanned by the matrices $A'_1$ and $A'_2$ listed below and the corresponding root spaces are:
\begin{itemize}
\item $\gsp(2,\eR)_{(0,0)}$:
\[
  A'_1=
\begin{pmatrix}
1&0\\
0&1\\
&&-1&0\\
&&0&-1
\end{pmatrix},
\quad
A'_2=
\begin{pmatrix}
0&1\\
1&0\\
&&0&-1\\
&&-1&0
\end{pmatrix}
\]
\item $\gsp(2,\eR)_{(0,2)}$:
\[
 X'= \begin{pmatrix}
1&-1&\\
1&-1&\\
&&-1&-1\\
&&1&-1
\end{pmatrix}
\]
\item $\gsp(2,\eR)_{(0,-2)}$:
\[
 V'= \begin{pmatrix}
1&1\\
-1&-1\\
&&-1&1\\
&&-1&1
\end{pmatrix}
\]
\item $\gsp(2,\eR)_{(2,0)}$:
\[
 W'= \begin{pmatrix}
&&1&0\\
&&0&-1\\
0&0\\0&0
\end{pmatrix}
\]
\item $\gsp(2,\eR)_{(2,2)}$:
\[
  L'=
\begin{pmatrix}
&&1&1\\
&&1&1\\
0&0\\0&0
\end{pmatrix}
\]
\item $\gsp(2,\eR)_{(2,-2)}$:
\[
  M'=
\begin{pmatrix}
&&1&-1\\
&&-1&1\\
0&0\\0&0
\end{pmatrix}
\]
\item $\gsp(2,\eR)_{(-2,0)}$
\[
Y'=
\begin{pmatrix}
&&0&0\\&&0&0\\
1&0\\0&-1
\end{pmatrix}
\]
\item $\gsp(2,\eR)_{(-2,2)}$:
\[
  N'=\begin{pmatrix}
&&0&0\\&&0&0\\
1&-1\\
-1&1
\end{pmatrix}
\]
\item $\gsp(2,\eR)_{(-2,-2)}$:
\[
  F'=\begin{pmatrix}
&&0&0\\
&&0&0\\
1&1\\
1&1
\end{pmatrix}
\]
\end{itemize}
It is important to notice how do the root spaces of $\gsl(2,\eC)$ embed:
\begin{align}
\phi(I_1)&=A'_1	&\phi(I_2)&=\frac{ V'-X' }{ 2 }\\
\phi(D_1)&=\frac{ L'-M' }{2}	&\phi(D_2)&=-W'\\
\phi(C_1)&=\frac{ F'-N' }{2}	&\phi(C_2)&=Y'.
\end{align}
So $\iN_{\gsp(2,\eR)}$ must at least contain the elements $L'$, $M'$ and $W'$. We complete the notion of positivity by $V'$. The Iwasawa algebra reads
\[
\begin{split}
\iA_{\gsp(2,\eR)}&=\{ B_1,B_2 \}\\
\iN_{\gsp(2,\eR)}&=\{ L',M',W',V' \}
\end{split}
\]
with
\begin{align*}
[L',V']&=-4W'	&[W',V']&=-2M'\\
[B_1',L']&=2L'	&[B'_2,M']&=2M'\\
[B_1',W']&=W'	&[B_2',W']&=W'\\
[B_1',V']&=-V'	&[B_2',V']&=V'
\end{align*}
where $B'_1=\frac{ 1 }{2}(A'_1+A'_2)$ and $B_2=\frac{ 1 }{2}(A_1'-A_2')$. The generators of $\iK_{\gsp(2,\eR)}$ are
\begin{align*}
K'_t&=
\begin{pmatrix}
&&1&0\\
&&0&1\\
-1&0\\
0&-1
\end{pmatrix}
	&K'_1&=
\begin{pmatrix}
0&1\\-1&0\\
&&0&1\\
&&-1&0
\end{pmatrix}\\
K'_2&=
\begin{pmatrix}
&&0&1\\&&1&0\\0&-1\\-1&0
\end{pmatrix}
	&K'_3&=
\begin{pmatrix}
&&1&0\\
&&0&-1\\
-1&0\\
0&1
\end{pmatrix}.
\end{align*}
Notice that $[K'_t,K'_i]=0$ for $i=1$, $2$, $3$.


\subsection{Isomorphism}		\label{SubSecIsosp}
%-----------------------

The following provides an isomorphism $\psi\colon \so(2,3)\to \gsp(2,\eR)$:
\begin{align*}
\psi(H_i)&=B'_i		&\psi(u)&=K'_t\\
\psi(W)&=W'		&\psi(R_1)&=\frac{ 1 }{2}K'_1\\
\psi(M)&=M'		&\psi(R_2)&=\frac{ 1 }{2}K'_2\\
\psi(L)&=L'		&\psi(R_3)&=\frac{ 1 }{2}K'_3\\
\psi(V)&=\frac{ 1 }{2}V'
\end{align*}
where the $R_i$'s are the generators of the $\so(3)$ part of $\sK_{\so(2,3)}$ satisfying the relations $[R_i,R_j]=\epsilon_{ijk}R_k$. It is now easy to check that the image of the embedding $\phi\colon \gsl(2,\eC) \to \gsp(2,\eR)$ is exactly $\so(1,3)$, so that
\begin{equation}
\psi^{-1}\circ\phi\colon \gsl(2,\eC)\to \sH
\end{equation}
is an isomorphism which realises $\sH$ as subalgebra of $\gsp(2,\eR)$. This circumstance will be useful in defining a spin structure on $AdS_4$.

One can prove that the kernel of the adjoint representation of $\SP(2,\eR)$ on its Lie algebra is $\pm\mtu$, in other words, $\Ad(a)=\id$ if and only if $a=\pm\mtu$. We define a bijective map $h\colon \SO(2,3)\to \SP(2,\eR)/\eZ_2$ by the requirement that
\begin{equation}		\label{Eqdefhspsl}
  \psi\big( \Ad(g)X \big)=\Ad\big( h(g) \big)\psi(X)
\end{equation}
for every $X\in\so(2,3)$. The following is true for all $\psi(X)$:
\[
\begin{split}
\Ad\big(h(gg'\big)) \psi(X)&=\psi\Big( \Ad(g)\big( \Ad(g')X \big) \Big)\\
			&=\Ad\big( h(g) \big)\psi\big( \Ad(g')X \big)\\
			&=\Ad\big( h(g)h(g') \big)\psi(X),
\end{split}
\]
 the map $h$ is therefore a homomorphism. If an element $a\in \SP(2,\eR)$ reads $a= e^{X_A} e^{X_N} e^{X_K}$ in the Iwasawa decomposition, the property $\Ad(a)\psi(X)=\psi\big( \Ad(g)X \big)$ holds for the element\label{PgSolhpsiSP} $g= e^{\psi^{-1}X_A} e^{\psi^{-1}X_N} e^{\psi^{-1}X_K}$ of $\SO(2,3)$. This shows that $h$ is surjective.

\subsection{Reductive structure on the symplectic group}		\label{SubSecRedspT}
%-------------------------------------------------------

A lot of structure of $\so(2,3)$, such as the reductive homogeneous space decomposition as $\sQ\oplus\sH$, can be immediately transported from $\so(2,3)$ to $\gsp(2,\eR)$. Indeed, let $\mT=\psi(\sQ)$ and $\mI=\phi\big( \gsl(2,\eC) \big)$. We have the direct sum decomposition
\[
\gsp(2,\eR)=\mT\oplus\mI.
\]
 Let $X\in\mT\cap\mI$, then $\psi^{-1}X$ belongs to $\sQ\cap\sH$ which only contains $0$. The fact that $\psi$ is an isomorphism yields that $X=0$. Since $\psi$ preserves linear independence, a simple dimension counting shows that the sum actually spans the whole space.

Putting $g=h^{-1}(a)$ in the definition \eqref{Eqdefhspsl} of $h$, we find
\[
  \psi\left( \Ad\big( h^{-1}(a) \big)X \right)=\Ad(a)\psi(X).
\]
Considering a path $a(t)$ with $a(0)=e$, we differentiate this expression with respect to $t$ at $t=0$ we find
\[
  \ad(dh^{-1}\dot a)X=d\psi^{-1}\big( \ad(\dot a)\psi(X) \big)=\ad(d\psi^{-1}\dot a)(d\psi^{-1}\psi X),
\]
but $d\psi=\psi$ because $\psi$ is linear, hence $[dh^{-1}\dot a,X]=[\psi^{-1}\dot a,X]$ for all $X\in \so(2,3)$ and $\dot a\in \gsp(2,\eR)$. We deduce that $(dh^{-1})_e=\psi^{-1}$. We define
\begin{align*}
	\theta_{\gsp}&=\id|_{\iK_{\gsp}}\oplus(-\id)|_{\iP_{\gsp}}\\
	\sigma_{\gsp}&=\id|_{\mT}\oplus(-\id)|_{\mI}.
\end{align*}
We can check that $\psi^{-1}\circ\theta_{\gsp}\circ\psi=\theta$ and $\psi^{-1}\circ\theta_{\gsp}\circ\psi=\theta$. Then it is clear that
\[
  [\sigma_{\gsp},\theta_{\gsp}]=0
\]
using the corresponding vanishing commutator in $\so(2,3)$. We denote $\mT_a=dL_a\mT$ and the fact that $dp= d\pi\circ dh^{-1}= d\pi\circ \psi^{-1}$ shows that $dp(\mT_a)$ is a basis of $T_{p(a)}(G/H)$. So we consider the basis $t_i=\psi(q_i)$ of $\mT$ and the corresponding left invariant vector fields $\tilde t_i(a)=dL_at_i$.


%+++++++++++++++++++++++++++++++++++++++++++++++++++++++++++++++++++++++++++++++++++++++++++++++++++++++++++++++++++++++++++
\section{Heisenberg group and algebra}
%+++++++++++++++++++++++++++++++++++++++++++++++++++++++++++++++++++++++++++++++++++++++++++++++++++++++++++++++++++++++++++


Let $V$ be a symplectic vector space with the symplectic form $\Omega$. The \hypertarget{HyperHeisenberg}{Heisenberg algebra} build on $V$ is the vector space
\begin{equation}
	\pH(V,\Omega)=V\oplus \eR E
\end{equation}
endowed with the bracket defined by
\begin{enumerate}

	\item
		$[\pH(V,\Omega),E]=0$,
	\item
		$[v,w]=\Omega(v,w)E$ for every $v,w\in V$.

\end{enumerate}
The first conditions makes $E$ central in $\pH$.

The Heisenberg group is, as set, the same as the algebra: $H=V\oplus\eR E$ with the product
\begin{equation}		\label{EqProduitHeisenbergGp}
	g_1\cdot g_2=g_1+g_2+\frac{ 1 }{2}[g_1,g_2]
\end{equation}
where the bracket is the one in the Lie algebra. Direct computations show that this product is associative, the neutral is $(0,0)$ and that the inverse is given by
\begin{equation}
	g^{-1}=-g.
\end{equation}
We are now going to prove that the Lie algebra of that group actually is $\pH(V,\Omega)$.

%---------------------------------------------------------------------------------------------------------------------------
\subsection{The exponential mapping}
%---------------------------------------------------------------------------------------------------------------------------

Let us build the exponential map between the Heisenberg algebra and its group. Let $(x,\tau)\in T_eH$ and consider $g(s)= e^{s(x,\tau)}=\big( v(s),h(s) \big)$.  This map is subject to the following three relations:
\begin{enumerate}

	\item
		$g(s)g(t)=g(s+t)$,
	\item
		$g(0)=0$,
	\item
		$g'(01)=(x,t)$.

\end{enumerate}
Taking the derivative of the first one with respect to $s$ and taking into account $v'(0)=x$ and $h'(0)=\tau$, we find
\begin{subequations}
	\begin{align}
		\Dsdd{g(s)+g(t)+\frac{ 1 }{2}\big[ g(s),g(t) \big]  }{s}{0}&=\Dsdd{ \big( v(s+t),h(s+t) \big) }{s}{0}\\
		\Dsdd{ v(s)+v(t),h(s)+h(t)+\frac{ 1 }{2}\Omega\big( v(s),v(t) \big) }{s}{0}	&=\big( v'(t),h'(t) \big)\\
		\Big( x,\tau+\frac{ 1 }{2}\Omega\big( x,v(t) \big) \Big)&=\big( v'(t),h'(t) \big)
	\end{align}
\end{subequations}
We deduce that $v'(t)=x$ and $h'(t)=\tau+\frac{ 1 }{2}\Omega\big( x,v(t) \big)$, so that $v(t)=tx$ and $h(t)=t\tau$. The exponential mapping is thus given by the identity:
\begin{equation}
	\exp(x,\tau)=(x,\tau).
\end{equation}

In order to prove that the law \eqref{EqProduitHeisenbergGp} accepts the Heisenberg algebra as Lie algebra, we need to compute the adjoint action.
\begin{equation}
	\begin{aligned}[]
		\Ad( e^{t(x,\tau)})(x',\tau')&=\Dsdd{ \AD( e^{t(x,\tau)}) e^{s(x',\tau')} }{s}{0}\\
		&=\Dsdd{ (tx,t\tau)(sx',s\tau')(-tx,-t\tau) }{s}{0}\\
		&=\Dsdd{ \big(tx+sx',t\tau+s\tau'+\frac{ ts }{2}\Omega(x,x')\big)(-tx,t\tau) }{s}{0}\\
		&=\big( x',\tau'+t\Omega(x,x') \big).
	\end{aligned}
\end{equation}
Now, the Lie algebra bracket is given by
\begin{equation}
	\begin{aligned}[]
		\big[ (x,\tau),(x',\tau') \big]&=\Dsdd{ \Ad( e^{t(x,\tau)})(x',\tau') }{t}{0}\\
			&=\big( 0,\Omega(x,x') \big)\\
			&=\Omega(x,x')E,
	\end{aligned}
\end{equation}
which is the bracket of $\pH(V,\Omega)$.

%+++++++++++++++++++++++++++++++++++++++++++++++++++++++++++++++++++++++++++++++++++++++++++++++++++++++++++++++++++++++++++ 
\section{Hermitian conjugate, unitary operators}
%+++++++++++++++++++++++++++++++++++++++++++++++++++++++++++++++++++++++++++++++++++++++++++++++++++++++++++++++++++++++++++

\begin{lemma}
    Let \( \hH\) be an Hilbert space. Let \( A\) be a linear continuous operator on \( \hH\). The map
    \begin{equation}
        \begin{aligned}
            S_A\colon \hH'&\to \hH' \\
            \alpha&\mapsto \alpha\circ A
        \end{aligned}
    \end{equation}
    is well defined and continuous.
\end{lemma}

\begin{proof}
    We have to prove two elements: firstly \( \alpha\circ A\) must be continuous, so that \( S_A\) takes its values in \( \hH'\), and secondly we want \( S_A\) itself to be continuous.

    The fact that \( \alpha\circ A\) is continuous is simply the fact that \( \alpha\) and \( A\) are continuous.

    Now we compute the norm of \( S_A\); first we have
    \begin{equation}
        \| S_A \|=\sup_{\alpha\in \hH'}\frac{ \| S_A(\alpha) \| }{ \| \alpha \| }.
    \end{equation}
    Then we compute
    \begin{equation}
        \| S_A(\alpha) \|=\sup_{\| u \|=1}\| S_A(\alpha)u \|=\sup_{\| u \|=1}\| \alpha(Au) \|\leq \| \alpha \|\| A \|\| u \|=\| \alpha \|\| A \|.
    \end{equation}
    Then we have
    \begin{equation}
        \| S_A \|=\sup_{\alpha\in \hH'}\frac{ \| S_A(\alpha) \| }{ \| \alpha \| }\leq \sup_{\alpha\in \hH'}\frac{ \| \alpha \|\| A \| }{ \| \alpha \| }=\| A \|<\infty.
    \end{equation}
    The proposition \ref{PROPooQZYVooYJVlBd} shows that \( S_A \) is continuous because it is bounded.
\end{proof}

We will use the map 
\begin{equation}
    \begin{aligned}
        \Phi\colon \hH&\to \hH' \\
        y&\mapsto \Phi_y 
    \end{aligned}
\end{equation}
where \( \Phi_y\) is defined by \( \Phi_y(x)=\langle x, y\rangle \). From the Riesz representation theorem \ref{ThoQgTovL} we know that \( \Phi\) is a bijective isometry. For the sake of notational convenience we will write \( \Phi(u)\) for \( \Phi_u\).

Notice the following formula:
\begin{equation}        \label{EQooHWQPooNeYokT}
    \langle u, \Phi^{-1}(\alpha)\rangle =\alpha(u)
\end{equation}
for every \( u\in \hH\) and \( \alpha\in\hH'\).

\begin{propositionDef}[\cite{MonCerveau}]
    Let \( \hH\) be an Hilbert space over \( \eC\). We consider a continuous \( A\colon \hH\to \hH\). There exists an unique linear operator \( B\colon \hH\to \hH\) such that
    \begin{equation}
        \langle Au, v\rangle =\langle u, Bv\rangle 
    \end{equation}
    for every \( u,v\in \hH\).

    The so-defined operator \( B\) is the \defe{hermitian conjugate of \( A\)}{hermitian conjugate} and is denoted \( A^{\dag}\). In other words, \( A^{\dag}\) is defined by the equality
    \begin{equation}        \label{EQooPTUWooPCbNxA}
        \langle Au, v\rangle =\langle u, A^{\dag}v\rangle 
    \end{equation}
    for every \( u,v\in\hH\).
\end{propositionDef}

\begin{proof}
    In two parts.
    \begin{subproof}
        \item[Unicity]
            We have to prove that, \( v\) being given, there exists an unique \( w\) such that
            \begin{equation}        \label{EQooVBJXooOtdmlQ}
                \langle Au, v\rangle =\langle u,w, \rangle 
            \end{equation}
            for every \( u\). Be clear: the same \( w\) must works for every \( u\). The condition \eqref{EQooVBJXooOtdmlQ} can be written as
            \begin{subequations}
                \begin{align}
                    \Phi(v)Au&=\Phi(w)u\\
                    S_A\big( \Phi(v) \big)u&=\Phi(w)u\\
                    S_A\big( \Phi(v) \big)&=\Phi(w),
                \end{align}
            \end{subequations}
            and finally
            \begin{equation}
                    w=\Phi^{-1}\Big( S_A\big( \Phi(v) \big) \Big).      \label{EQooPRVGooUfIELg}
            \end{equation}
            This proves the unicity: \( Bv\) must be given by the expression \eqref{EQooPRVGooUfIELg}.
        \item[Existence]
            We check that the formula
            \begin{equation}        \label{EQooOIROooXUjCWL}
                A^{\dag}=\Phi^{-1}\circ S_A\circ \Phi.
            \end{equation}
            satisfy the properties. Using the formula \eqref{EQooHWQPooNeYokT} we have:
            \begin{subequations}
                \begin{align}
                    \langle u, (\Phi^{-1}\circ S_A\circ \Phi)(v)\rangle &=(S_A\circ\Phi)(v)u\\
                    &=\Phi(v)Au\\
                    &=\langle Au, v\rangle ,
                \end{align}
            \end{subequations}
            so that we see that the operator given by \eqref{EQooOIROooXUjCWL} makes the work.
    \end{subproof}
\end{proof}

\begin{definition}[Unitary, hermitian]      \label{DEFooOKGXooFCzCHu}
    An operator \( A\colon \hH\to \hH\) is \defe{unitary}{unitary operator} if it satisfies
    \begin{equation}
        A^{\dag}A=AA^{\dag}=\id.
    \end{equation}
    An operator \( A\colon \hH\to \hH\) is \defe{hermitian}{hermitian operator} if it satisfies
    \begin{equation}
        A^{\dag}=A.
    \end{equation}
    This was already the definition \ref{DEFooKEBHooWwCKRK}.
\end{definition}

\begin{lemma}
    An unitary operator is an isometry: it preserves the hermitian product on an Hilbert space.
\end{lemma}

\begin{proof}
    Let \( u,v\in \hH\) and \( A\) be an unitary operator on the Hilbert space \( \hH\). We have
    \begin{equation}
        \langle Au, Au\rangle =\langle A^{\dag}Au, v\rangle =\langle u,v, \rangle .
    \end{equation}
\end{proof}

\begin{lemma}[\cite{MonCerveau}]        \label{LEMooJYGRooPTMZwY}
    The hermitian conjugation satisfies:
    \begin{enumerate}
        \item
            \( \langle A^{\dag}u, v\rangle =\langle u, Av\rangle \) for every \( u,v\in \hH\)
        \item
            \( (A^{\dag})^{\dag}=A\).
        \item
            \( (AB)^{\dag}=B^{\dag}A^{\dag}\).
    \end{enumerate}
\end{lemma}

\begin{proof}
    We have
    \begin{equation}
        \langle A^{\dag}u, v\rangle =\overline{ \langle v, A^{\dag}u\rangle  }=\overline{ \langle Av, u\rangle  }=\langle u, Av\rangle .
    \end{equation}
    This proves the first point.

    For the second point, \( (A^{\dag})^{\dag}\) is the unique operator satisfying
    \begin{equation}
        \langle A^{\dag}u, v\rangle =\langle u, (A^{\dag})^{\dag}v\rangle 
    \end{equation}
    for every \( u,v\in \hH\). Using the first point,
    \begin{equation}
        \langle u, (A^{\dag})^{\dag} v\rangle =\langle u, Av\rangle .
    \end{equation}
    This shows that \( \Phi\big( (A^{\dag})^{\dag}v \big)=\Phi(Av)\). Since \( \Phi\) is bijective, \( (A^{\dag})^{\dag}v=Av\) for every \( v\in \hH\).
\end{proof}

%+++++++++++++++++++++++++++++++++++++++++++++++++++++++++++++++++++++++++++++++++++++++++++++++++++++++++++++++++++++++++++ 
\section{The group \texorpdfstring{$ \SU(n)$}{SUn} of special unitary operators}
%+++++++++++++++++++++++++++++++++++++++++++++++++++++++++++++++++++++++++++++++++++++++++++++++++++++++++++++++++++++++++++

%--------------------------------------------------------------------------------------------------------------------------- 
\subsection{Some settings}
%---------------------------------------------------------------------------------------------------------------------------

We particularize ourself to the finite dimensional Hilbert space \( \hH=\eC^n\). The vector space \( \eC^n\) has a canonical basis; so we use it, and the linear maps \( \eC^n\to \eC^n\) are identified with their matrices in that very basis. Using the conventions described around \eqref{EQooOMSCooGsSBIA} we have 
\begin{equation}
    A(e_j)=\sum_iA(e_j)_ie_i=\sum_iA_{ij}e_i
\end{equation}
where \( \{ e_i \}_{i=1,\ldots, n}\) is the canonical basis of \( \eC^n\).

When \( u\in \eC^n\), we denote by \( u_k\in \eC\) the \( k\)\ieme\ component of \( u\) with respect to the canonical basis. The vector space \( \eC^n\) has an hermitian product\footnote{Definition \ref{DefMZQxmQ}.} given by
\begin{equation}
    \langle u, v\rangle =\sum_{k=1}^nu_k\bar v_k.
\end{equation}

\begin{lemma}       \label{LEMooKEUZooUjQVmp}
    Let \( A\) be a linear continuous operator on the Hilbert space \( \hH\). We have
    \begin{equation}
        \det(A^{\dag})=\overline{ \det(A) }
    \end{equation}
    where the bar stands for the complex conjugate.

    If \( A\) is unitary, then \( \det(A)\in S^1\) where \( S^1\) is the set of elements of norm \( 1\) in \( \eC\).
\end{lemma}

\begin{proof}
    We use for the determinant the formula given by lemma \ref{LEMooEZFIooXyYybe}:
    \begin{equation}
        \det(A^{\dag})=\sum_{\sigma\in S_n}\epsilon(\sigma)\prod_{i=1}^n\langle e_{\sigma(i)}, A^{\dag}e_i\rangle =  \sum_{\sigma\in S_n}\epsilon(\sigma)\prod_{i=1}^n \langle Ae_{\sigma(i)}, e_i\rangle = \sum_{\sigma\in S_n}\epsilon(\sigma)\prod_{i=1}^n\overline{ e_i,Ae_{\sigma(i)} }.
    \end{equation}
    At this point we re-index the product and we make a change of variable $\sigma\to \sigma^{-1}$ for the sum (as in the proof for the transposed matrix, lemma \ref{LEMooCEQYooYAbctZ}).

    If the operator is unitary,
    \begin{equation}
        1=\det(AA^{\dag})=\det(A)\det(A^{\dag})=| \det(A) |^2.
    \end{equation}
    Thus \( | \det(A) |=1\).
\end{proof}


\begin{lemma}
    As for the matrices,
    \begin{equation}
        (U^{\dag})_{ij}=\overline{ U_{ji} }.
    \end{equation}
\end{lemma}

\begin{proof}
    The matrix convention are summarized around the equation \eqref{EQooDSKBooQkgtWv}. We write the definition \eqref{EQooPTUWooPCbNxA} for the basis vectors:
    \begin{equation}
        \langle Ue_i, e_j\rangle =\langle e_i, U^{\dag}e_j\rangle .
    \end{equation}
    This means \( U_{ji}=\overline{ \langle U^{\dag}e_j, e_i\rangle  }=\overline{ U^{\dag}_{ij} }\).
\end{proof}

\begin{definition}[\cite{BIBooUXTFooXTeMOn}]        \label{DEFooVIQUooQbnYMu}
    The group of the unitary operators with determinant \( 1\) is \( \SU(n)\). More explicitly,
    \begin{equation}
        \SU(n)=\{U\in \GL(n,\eC)\tq U^{\dag}U=\id,\det(U)=1\}.
    \end{equation}
\end{definition}

\begin{proposition}     \label{PROPooAKZEooEfpxPp}
    The Lie algebra \( \su(n)\) is given by
    \begin{equation}
        \su(n)=\{X\in\gl(2,\eC)\tq X^{\dag}=-X,\tr(X)=0\}.
    \end{equation}
\end{proposition}

\begin{proposition}     \label{PROPooYXPRooBgikdE}
    A unitary endomorphism of \( \eC^n\) is diagonalizable by a unitary operator.
\end{proposition}

\begin{proof}
    A unitary operator \( U\) is normal (definition \ref{DefWQNooKEeJzv}), so that the spectral theorem \ref{ThogammwA} provides an unitary matrice \( V\) such that \( V^{\dag}UV\) is diagonal.
\end{proof}

\begin{proposition}     \label{PROPooZBJSooEIguXR}
    A special unitary matrix is the exponential of a skew-hermitian matrix with vanishing trace. 
\end{proposition}

\begin{proof}
    Let \( U\in \SU(n)\). We prove that here exists an hermitian operator \( H\) with \( U= e^{iH}\) and \( \tr(H)=0\). Then \( iH\) is the requested skew-hermitian operator.

    By mean of the proposition \ref{PROPooYXPRooBgikdE}, we diagonalize it with the unitary operator \( S\). We have 
    \begin{equation}
        U=SDS^{\dag},
    \end{equation}
    and
    \begin{equation}
        \det(U)=1,
    \end{equation}
    and
    \begin{equation}
        D=\begin{pmatrix}
            e^{i\varphi_1}    &       &       \\
                &   \ddots    &       \\
                &       &    e^{i\varphi_n}
        \end{pmatrix}.
    \end{equation}
    Since \( U\in\SU(n)\) we have
    \begin{equation}
        1=\det(U)=\det(S)\det(D)\det(S^{\dag})=| \det(S) |^2\det(D)=\det(D).
    \end{equation}
    Thus \( \det(D)=1\) and we deduce
    \begin{equation}
        \sum_{i=1}^n\varphi_i=2k\pi
    \end{equation}
    for some \( k\in \eZ\). We consider the two following  operators:
    \begin{equation}
        \Delta=\begin{pmatrix}
            \varphi_1    &       &       \\
                &   \ddots    &       \\
                &       &   \varphi_n
        \end{pmatrix},
    \end{equation}
    and \( H=S\Delta S^{\dag}\). Since \( \Delta^{\dag}=\Delta\), we have \( H^{\dag}=H\), so that \( H\) is hermitian. We also have \( U= e^{iH}\). Indeed, since \( (S\Delta S^{\dag})^k=S\Delta^kS^{\dag}\) we have
    \begin{equation}
        e^{iH}=\sum_{k=0}^{\infty}\frac{ (iH)^k }{ k! }=\sum_k\frac{ i^k }{ k! }(S\Delta S^{\dag})^k=\sum_k\frac{ i^k }{ k! }S\Delta^kS^{\dag}=S e^{i\Delta}S^{\dag}=SDS^{\dag}=U.
    \end{equation}
    We also have \( \tr(H)=2k\pi\).
    
    This means that \( H\) is almost the operator we are searching for. It is easy to modify \( H\) in order to get our answer. We set
    \begin{equation}
        \Delta'=\begin{pmatrix}
            \varphi_1    &       &       \\
                &   \ddots   &       \\
                &       &   \varphi_n-2k\pi
        \end{pmatrix}.
    \end{equation}
    This matrix satisfies \(   e^{i\Delta'}= e^{i\Delta}\) and \( \tr(\Delta')=0\). If we set \( H'=S\Delta'S^{\dag}\) we still have \( \tr(H')=\tr(\Delta')=0\) because of the cyclic invariance of the trace. And finally the operator \( H'\) satisfies
    \begin{equation}
        e^{iH'}=S e^{i\Delta'}S^{\dag}= e^{iH}=U.
    \end{equation}
\end{proof}

%--------------------------------------------------------------------------------------------------------------------------- 
\subsection{The center of \texorpdfstring{$ \SU(n)$}{SUn} }
%---------------------------------------------------------------------------------------------------------------------------

\begin{proposition}     \label{PROPooLMGHooKrKpsa}
    The center\footnote{The set of elements which commute with all the elements, see definition \ref{defGroupeCentre}.} of the group \( \SU(n)\), \( \mZ\big( \SU(n) \big)\), is the subgroup of the element of the form
    \begin{equation}
        e^{i\varphi}\id.
    \end{equation}
    In particular, the center of \( \SU(2)\) is \( \{ \id,-\id \}\).
\end{proposition}

\begin{proof}
    Let \( Z\in\mZ\big( \SU(n) \big)\). It can be diagonalized by an unitary matrix \( S\):
    \begin{equation}
        SZS^{\dag}=D.
    \end{equation}
    The operator \( S\) belongs to \( \gU(n)\) while \( Z\) only commutes with the elements of \( \SU(n)\). Thus we cannot immediately deduce \( Z=D\).  However, the operator \( S_0 = S/\det(S)\) diagonalizes \( Z\) as well as \( S\): \( S_0ZS_0^{\dag}=D\). But since \( S_0\) is in \( \SU(n)\) we can deduce \( Z=D\).

    Long story short, the elements of \( \mZ\big( \SU(n) \big)\) are diagonal. Let \( Ze_i=\lambda_ie_i\). We consider the operator \( A\) of \( \SU(n)\) given by the formula
    \begin{subequations}
        \begin{align}
            Ae_1&=-e_2\\
            Ae_2&=e_1\\
            Ae_k&=e_k\qquad\text{otherwise.}
        \end{align}
    \end{subequations}
    This is the element of \( \SU(n)\) which permutes \( e_1\) with \( e_2\) (with a sign for the sake of the determinant). Since \( ZA=AZ\), we have
    \begin{equation}
        AZe_1=\lambda_1Ae_1=-\lambda_1e_1
    \end{equation}
    and 
    \begin{equation}
        ZAe_1=Z(-)e_2=-\lambda_2e_2.
    \end{equation}
    So we have \( \lambda_1=\lambda_2\). The same being true not only for \( 1\) and \( 2\) but for the other ones, we know that the numbers \( \lambda_i\) are all equal.

    So far, \( Z=\lambda\id\). The value of \( \lambda\) is not arbitrary: we must impose \( Z\in\SU(n)\). The determinant condition provides
    \begin{equation}        \label{EQooZZPJooCeKPDD}
        1=\det(Z)=\lambda^n
    \end{equation}
    and the unitary condition imposes 
    \begin{equation}
        1=\langle e_i, e_i\rangle =\langle Ze_i, Ze_i\rangle =| \lambda |^2,
    \end{equation}
    so that \( | \lambda |^2=1\). This conditions imposes \( \lambda= e^{i\varphi}\) for some \( \varphi\in \eR\) while the condition \eqref{EQooZZPJooCeKPDD} furnishes
    \begin{equation}
        e^{in\varphi}=1.
    \end{equation}
    This shows the existence of \( k\in \eZ\) such that \( in\varphi=2ki\pi\) and finally
    \begin{equation}
        \varphi=\frac{ 2k\pi }{ n }.
    \end{equation}
    
    In the case \( n=2\) we have \( \varphi=k\pi\) and \(  e^{i\varphi}=\pm 1\).
\end{proof}


%+++++++++++++++++++++++++++++++++++++++++++++++++++++++++++++++++++++++++++++++++++++++++++++++++++++++++++++++++++++++++++ 
\section{Representations of \texorpdfstring{$ U(1)$}{U(1)}}
%+++++++++++++++++++++++++++++++++++++++++++++++++++++++++++++++++++++++++++++++++++++++++++++++++++++++++++++++++++++++++++

\begin{proposition}
    Let \( V\) be a complex vector space of dimension \( 1\). Let \( \xi\neq 0\in V\). If \( \langle ., .\rangle \) is an hermitian product on \( V\), there exists \( m\in \eR\) such that
    \begin{equation}
        \langle z_1\xi, z_2\xi\rangle =mz_1\bar z_2.
    \end{equation}
\end{proposition}

\begin{proof}
    Every element of \( V\) can be written under the form \( z\xi\) for some \( z\in \eC\). The properties of an hermitian product say
    \begin{equation}
        \langle z_1\xi, z_2\xi\rangle =z_1\bar z_2\langle \xi, \xi\rangle 
    \end{equation}
    and \( \langle \xi, \xi\rangle \in \eR\).
\end{proof}

\begin{normaltext}
    What we write \( S^1\) and \( \gU(1)\) are the same thing. When we write \( S^1\) we have in mind the geometrical object (with a measure) made of the complex numbers of norm \( 1\); when we write \( \gU(1)\) we have in mind the group structure. But it's the same.

    However the generalizations \( S^2\) and \( \gU(2)\) are not the same.
\end{normaltext}

\begin{normaltext}[\cite{MonCerveau}]
    We are searching now for the irreducible representations of \( U(1)\). More precisely we will determine the irreducible continuous representations of \( U(1)\). Here the fact to be «continuous» means that \( \rho\colon U(1)\to \GL(V)\) is continuous; in particular, \( V\) has to be a topological vector space.

    This is not a restriction because the theorem \ref{THOooFFJGooCekFQc} shows that every irreducible representation of \( U(1)\) has dimension \( 1\).
\end{normaltext}

\begin{proposition}[Irreducible representations of \( U(1)\)]       \label{PROPooLWWEooUmqbRA}
    Let \( m\in \eZ\). We consider
    \begin{equation}        \label{EQooXPXKooJasMyY}
        \begin{aligned}
            T_m\colon U(1)&\to \GL(\eC) \\
            T_m(g)z&=g^mz.
        \end{aligned}
    \end{equation}
    \begin{enumerate}
        \item
            The formula \eqref{EQooXPXKooJasMyY} defines a representation of \( U(1)\).
        \item
            The representation \( T_m\) is irreducible.
        \item
            The representation \( T_m\) is continuous.
        \item
            If \( m\neq l\), then the representations \( T_m\) and \( T_l\) are not equivalent.
        \item       \label{ITEMooUPVQooQddQOJ}
            Every continuous\footnote{For the norm of proposition \ref{PROPooNTCFooEcwZwt} on \( V\) and the corresponding operator norm one on \( \GL(V)\), definition \ref{DefNFYUooBZCPTr}.} irreducible representation of \( U(1)\) is equivalent to one of them.
    \end{enumerate}
\end{proposition}

\begin{proof}
    Several points.
    \begin{subproof}
        \item[It is a representation]
            Since \( U(1)\) is abelian, \( (g_1g_2)^m=g_1^mg_2^m\).
        \item[Irreducible]
            The representation \( T_m\) is irreducible because the vector space is \( \eC\) which has dimension \( 1\).
        \item[Continuous]
            Let \( g_k\stackrel{U(1)}{\longrightarrow}g\). We have
            \begin{equation}
                \| T_m(g_k)-T_m(g) \|=\sup_{| z |1=1}| g_k^m-g^mz |=| g_k^m-g^m |\to 0.
            \end{equation}
            This shows that \( T_m\) is continuous.
        \item[Non equivalence]
            Let \( \psi\colon \eC\to \eC\) be a linear map such that \( T_m(g)\circ\psi=\psi\circ T_l(g)\) for every \( g\in U(1)\). This implies
            \begin{equation}
                g^m\psi(z)=\psi(g^lz),
            \end{equation}
            hence
            \begin{equation}
                g^m\psi(z)=g^l\psi(z).
            \end{equation}
            Taking \( z\) such that \( \psi(z)\neq 0\) we have \( g^m=g^l\) for every \( g\in U(1)\), or \( g^{l-m}=1\). Writing \( g= e^{ix}\) we have
            \begin{equation}
                e^{i(l-m)x=1}
            \end{equation}
            for every \( x\in \eR\). We conclude \( l-m=2k \pi\). Since \( l-m\in \eZ\) the only solution is \( l-m=0\).
            
    \end{subproof}
    At this point, it ``remains'' to prove the point \ref{ITEMooUPVQooQddQOJ}. Let \( (\rho, V)\) be a continuous irreducible representation of \( U(1)\). 
    
    \begin{subproof}
        \item[The function \( \lambda\)]
            Since \( U(1)\) is abelian, \( \dim(V)=1\) (theorem \ref{THOooFFJGooCekFQc}). So there exist a function \( \lambda\colon U(1)\to \eC\) such that 
            \begin{equation}
                \rho(g)=\lambda(g)\id.
            \end{equation}

        \item[\( \lambda\) is continuous]

            The spaces \( V\), \( \GL(V)\) and \( \eC\) are metric, thanks to the restriction we imposed on the topology of \( V\). Let \( g_k\to g\) in \( U(1)\). Since \( \rho\) is continuous we have \( \rho(g_k)\stackrel{\GL(V)}{\longrightarrow}\rho(g)\). From the definition of the operator norm, that implies, for each \( v\in V\) that
            \begin{equation}
                \rho(g_k)\stackrel{V}{\longrightarrow}\rho(g)v,
            \end{equation}
            which means
            \begin{equation}        \label{EQooTPAXooAPSgxP}
                \lambda(g_k)v\stackrel{V}{\longrightarrow}\lambda(g)v.
            \end{equation}
            Using the definition of the topology on \( V\),
            \begin{equation}
                \| \lambda(g_k)v-\lambda(g)v \|=\|\big( \lambda(g_k)-\lambda(g) \big)v \|=\| \lambda(g_k)-\lambda(g) \|\| v \|.
            \end{equation}
            The convergence \eqref{EQooTPAXooAPSgxP} means
            \begin{equation}
                | \lambda(g_k)-\lambda(g) |\| v \|\stackrel{\eR}{\longrightarrow}0,
            \end{equation}
            which implies the convergence \( \lambda(g_k)\to \lambda(g)\), hence the continuity of \( \lambda\).

        \item[\( \lambda\) takes values in \( S^1\)]

            We show that \( |\lambda(g)|=1\) pour tout \( g\in U(1)\). An element of \( U(1)\) reads \( g= e^{2\pi i x}\) with \( x\in \eR\)\footnote{Proposition \ref{PROPooZEFEooEKMOPT}.}.

            If \( x\in \eZ\) we have \( g=1\), so that \( \lambda(g)=1\). If \( x=1/n\) (\( n\in \eZ\)) we have \( g^n=1\), but
            \begin{equation}
                \lambda(g^n)=\lambda(g)^n,
            \end{equation}
            so that \( | \lambda(g) |^n=1\) which proves that \( | \lambda(g) |=1\).

            From here we know that \( |\lambda( e^{2\pi i q})|=1\) for every \( q\in \eQ\).

            Since \( \lambda\) is continuous, the function \( x\mapsto | \lambda( e^{2\pi i x}) |\) is continuous. A continuous function whose value is \( 1\) over \( \eQ\) is constant.

        \item[Functional equation]
            For \( x,y\in \eR\) we have \( \rho\big(  e^{ix} e^{iy} \big)=\rho( e^{ix})\rho( e^{iy})\), but also \( \rho( e^{ix} e^{iy})=\rho( e^{i(x+y)})\). We introduce the notation
            \begin{equation}
                \alpha=\lambda\circ\varphi
            \end{equation}
            where \( \varphi\colon \eR\to U(1)\) is \( \varphi(x)= e^{ix}\). The function
            \begin{equation}
                \begin{aligned}
                    \alpha\colon \eR&\to S^1 \\
                    x&\mapsto \lambda( e^{ix}) 
                \end{aligned}
            \end{equation}
            satisfies
            \begin{subequations}
                \begin{numcases}{}
                    \alpha(0)=1\\
                    \alpha(x+y)=\alpha(x)\alpha(y).
                \end{numcases}
            \end{subequations}
            The proposition \ref{PROPooVJLYooOzfWCd} shows that there exists \( m\in \eR\) such that
            \begin{equation}
                \alpha(x)= e^{imx}.
            \end{equation}
            
            Since \( \alpha(2\pi)=\alpha(0)=1\) we have \(  e^{2\pi im}=1\) and we conclude that \( m\in \eZ\).

        \item[The value of \( \lambda\)]

            We have defined \( \alpha=\lambda\circ \varphi\). Since \( \varphi\) is not a bijection, we cannot write \( \lambda=\alpha\circ \varphi^{-1}\). However for every \( x\in \eR\) we have
            \begin{equation}
                \lambda( e^{ix})=\alpha(x)= e^{imx}=( e^{ix})^m.
            \end{equation}
            So
            \begin{equation}
                \lambda(g)=g^m.
            \end{equation}
            
        \item[Equivalence]

            We prove that \( \rho\) is equivalent to the representations \( T_m\). Let \( \{ v \}\) be a basis of \( V\); we consider the linear map
            \begin{equation}
                \begin{aligned}
                    \psi\colon V&\to \eC \\
                    sv&\mapsto s. 
                \end{aligned}
            \end{equation}
            We have
            \begin{equation}
                \big( T_m(g)\circ \psi \big)(sv)=T_m(g)s=g^ms
            \end{equation}
            while
            \begin{equation}
                \big( \psi\circ\rho(g) \big)(sv)=\psi\big( g^msv \big)=g^ms\psi(v)=g^ms.
            \end{equation}
            So we have
            \begin{equation}
                T_m(g)\circ \psi=\psi\circ\rho(g)
            \end{equation}
            which proves that \( T_m\) and \( \rho\) are equivalent.
    \end{subproof}
\end{proof}

% This is part of (almost) Everything I know in mathematics and physics
% Copyright (c) 2013-2014, 2019
%   Laurent Claessens
% See the file fdl-1.3.txt for copying conditions.

\section{The group \texorpdfstring{$SU(2)$}{SU2}}
%--------------------------------------------------

The group \( \SU(2)\) is already defined in \ref{DEFooVIQUooQbnYMu}.

\begin{proposition}[\cite{BIBooUXTFooXTeMOn,BIBooMXBZooFCLGYe}]       \label{PROPooZMPLooUFyAPW}
    The matrices of \( \SU(2)\) are
    \begin{equation}
        \SU(2)=\{ \begin{pmatrix}
        \alpha    &   -\bar \beta    \\ 
    \beta    &   \bar \alpha    
\end{pmatrix}\tq \alpha,\beta\in \eC,| \alpha |^2+| \beta |^2=1\}.
    \end{equation}
\end{proposition}

\begin{proof}
    We initiate with a matrix \( U=\begin{pmatrix}
        \alpha    &   \beta    \\ 
        \gamma    &   \delta    
    \end{pmatrix}\in \eM(2,\eC)\). Then we impose the conditions. The unitary property gives:
    \begin{equation}
    UU^{\dag}=
    \begin{pmatrix}
    \alpha & \beta \\
    \gamma & \delta
    \end{pmatrix}
    \begin{pmatrix}
    \oalpha & \ogamma \\
    \obeta & \odelta
    \end{pmatrix}
    =
    \begin{pmatrix}
    \alpha\oalpha+\beta\obeta & \alpha\bar \gamma+\beta\bar\delta \\
    \gamma\bar \alpha+\delta\bar\beta & \gamma\ogamma+\delta\odelta
    \end{pmatrix}
    \stackrel{!}{=}
    \begin{pmatrix}
    1  & 0 \\
    0 & 1
    \end{pmatrix}.
    \end{equation}
    Among with the determinant conditions, we have the system
    \begin{subequations}        \label{SUBEQSooGUDNooOoxdSO}
        \begin{numcases}{}
            \alpha\delta-\gamma\beta=1\\
            | \alpha |^2+| \beta |^1=1\\
            | \gamma |^2+| \delta |^2=1\\
            \alpha\bar \gamma+\beta\bar\delta=0.        \label{SUBEQooSPRRooWjAUNi}
        \end{numcases}
    \end{subequations}
    We multiply \eqref{SUBEQooSPRRooWjAUNi} by \( \gamma\), and we substitute \( \gamma\bar \gamma=1-| \delta |^2\)  and \( \gamma\beta=\alpha\delta-1\). What we get is
    \begin{equation}
        \alpha(1-| \delta |^2)+(\alpha\delta-1)\bar \delta=0.
    \end{equation}
    If you develop the products, you see some simplifications and you remain with \( \delta=\bar \alpha\).

    Now we substitute \( \delta=\bar \alpha\) in \eqref{SUBEQooSPRRooWjAUNi} again. We obtain
    \begin{equation}        \label{EQooNOVWooYSTXqJ}
        \alpha(\bar \gamma+\beta)=0.
    \end{equation}
    There are two possibilities: \( \alpha=0\) or \( \alpha\neq 0\).
    \begin{subproof}
    \item[If \( \alpha\neq 0\)]
        In that case the equality \eqref{EQooNOVWooYSTXqJ} produces \( \gamma=-\bar\beta\) and the result is proved.
    \item[If \( \alpha=0\)]
        The system \eqref{SUBEQSooGUDNooOoxdSO} reduces to
        \begin{subequations}
            \begin{numcases}{}
                \gamma\beta=-1 \label{SUBEQooVCDOooTGejzi}\\
                | \beta |^2=1\\
                | \gamma |^2=1.     \label{SUBEQooWJJMooItqDsi}
            \end{numcases}
        \end{subequations}
        There exist \( \theta\in \eR\) such that \( \beta= e^{i\theta}\). The equation \eqref{SUBEQooVCDOooTGejzi} then shows that \( \gamma=- e^{-i\theta}\). The condition \eqref{SUBEQooWJJMooItqDsi} is automatically satisfied. At the end we have the matrix
        \begin{equation}
            U=\begin{pmatrix}
                0    &   \beta    \\ 
                -\bar\beta    &   0    
            \end{pmatrix}.
        \end{equation}
    \end{subproof}
\end{proof}

%--------------------------------------------------------------------------------------------------------------------------- 
\subsection{\texorpdfstring{$ \SU(2)$}{SU(2)} as compact group}
%---------------------------------------------------------------------------------------------------------------------------

\begin{proposition}     \label{PROPooGLPQooKOfrjl}
    The group \( \SU(2)\) is compact.
\end{proposition}

\begin{proof}
    The proposition \ref{PROPooZMPLooUFyAPW} show that $\SU(2)$ is contained in a bounded subset of $\eR^8$, and it is clear that $\SU(2)$ is closed in $\eR^8$ because it is defined by equalities.
\end{proof}

%--------------------------------------------------------------------------------------------------------------------------- 
\subsection{Pauli matrices}
%---------------------------------------------------------------------------------------------------------------------------

We denote by \( V\) the set of the hermitian traceless matrices. These are the elements \( u\in \eM(2,\eC)\) such that
\begin{subequations}
    \begin{align}
        u^{\dag}&=u\\
        \trace(u)&=0.
    \end{align}
\end{subequations}
This is not the Lie algebra \( \su(2)\) because the elements of \( \su(2)\) are anti-hermitian\footnote{Proposition \ref{PROPooSERWooFtxBgV}.}. This set \( V\) is a vector space over \( \eR\), but not over the field \( \eC\) because if \( u\in V\), then \( (iu)^{\dag}=-iu^{\dag}=-iu\neq iu\).

\begin{definition}      \label{DEFooRNTDooTVkPtB}
    The \defe{Pauli matrices}{Pauli matrices} are the following three:
    \begin{equation}
        \begin{aligned}[]
            \sigma_1=\begin{pmatrix}
                0    &   1    \\ 
                1    &   0    
            \end{pmatrix},&&
            \sigma_2=\begin{pmatrix}
                0    &   -i    \\ 
                i    &   0    
            \end{pmatrix},&&
            \sigma_3=\begin{pmatrix}
                1    &   0    \\ 
                0    &   -1    
            \end{pmatrix}.
        \end{aligned}
    \end{equation}
\end{definition}

For the sake of notations, we write \( \sigma\) the vector of \( V^3\) given by \( \sigma=(\sigma_1, \sigma_2, \sigma_3)\). This allows us to write combinations like
\begin{equation}        \label{EQooXNRGooZaRQoZ}
    a\cdot \sigma=a_1\sigma_1+a_2\sigma_2+a_3\sigma_3
\end{equation}
when \( a\in \eR^3\). It must be noticed however that the notation «\( a\cdot \sigma\)» is not a scalar product. In particular, the formula \eqref{EQooXNRGooZaRQoZ} depends on the chosen basis \( \{ \sigma_i \}\) of \( V\) and \( \{ e_i \}\) on \( \eR^3\).

\begin{lemma}       \label{LEMooZNCQooLgoReX}
    The Pauli matrice form a basis\footnote{Definition \ref{DEFooNGDSooEDAwTh}.} of the real vector space \( V\) of hermitian traceless matrices.
\end{lemma}

\begin{proof}
    An element of \( V\) is a matrice of the form
    \begin{equation}
        u=\begin{pmatrix}
            a    &   b    \\ 
            c    &   d    
        \end{pmatrix}
    \end{equation}
    with \( a,b,c,d\in \eC\). The condition \( u=u^{\dag}\) imposes the relations \( a=\bar a\), \( d=\bar d\) and \( c=\bar b\), so that
    \begin{equation}
        u=\begin{pmatrix}
            x    &   z    \\ 
            \bar z    &   y    
        \end{pmatrix}.
    \end{equation}
    The trace condition imposes \( x=-y\). Finally a general element of \( V\) has the form
    \begin{equation}
        u=\begin{pmatrix}
            x    &   z    \\ 
            \bar z    &   -x    
        \end{pmatrix}
    \end{equation}
    with \( x\in \eR\) and \( z\in \eC\). We have:
    \begin{equation}
        u=-\imag(z)\sigma_1+\real(z)\sigma_2+x\sigma_3.
    \end{equation}
    This proves that \( \{ \sigma_i \}_{i=1,2,3}\) spans \( V\). 
    
    We still have to prove that \( \{ \sigma_i \}_{i=1,2,3}\) is free. For that, consider \( a,b,c\in \eR\) such that \( a\sigma_1+b\sigma_2+c\sigma_3=0\):
    \begin{equation}
        \begin{pmatrix}
            c    &   a-bi    \\ 
            a+bi    &   -c    
        \end{pmatrix}=\begin{pmatrix}
            0    &   0    \\ 
            0    &   0    
        \end{pmatrix}.
    \end{equation}
    We immediately deduce \( c=0\), \( a+bi=0\) and \( a-bi=0\). Thus \( a=b=c=0\).
\end{proof}

Notice that the hermitian matrices do not form a vector space over \( \eC\) because, if \( X\) is hermitian,
\begin{equation}
    (\lambda X)^{\dag}=\bar \lambda X^{\dag}=\bar \lambda X\neq \lambda X.
\end{equation}

\begin{lemma}
    The matrices \( i\sigma_k\) are unitary.
\end{lemma}

\begin{proof}
    A simple computation show that \( \sigma_k^2=\mtu\) and \( \sigma_k^{\dag}=\sigma_k\), so that 
    \begin{equation}
        (i\sigma_k)(i\sigma_k)^{\dag}=(i\sigma_k)(-i\sigma_k)=\sigma_k^2=\mtu.
    \end{equation}
\end{proof}

%///////////////////////////////////////////////////////////////////////////////////////////////////////////////////////////
\subsubsection{Some relations}
%///////////////////////////////////////////////////////////////////////////////////////////////////////////////////////////

\begin{lemma}       \label{LEMooIBJMooTYnooZ}
    For every \( i,j=1,2,3\) we have the formula
    \begin{equation}
        \sigma_i\sigma_j=\delta_{ij}\mtu+i\sum_m\epsilon_{ijm}\sigma_m.
    \end{equation}
    In particular \( \sigma_i^2=\mtu\).
\end{lemma}

\begin{proof}
    Explicit matricial computation.
\end{proof}

\begin{lemma}       \label{LEMooJRWXooMkzRnk}
    We have the commutator\footnote{We are adult here; I believe you will not confuse the \( i\) of the index and the \( i\) of the imaginary numbers.}
    \begin{equation}
        [\sigma_i,\sigma_j]=2i\sum_k\epsilon_{ijk}\sigma_k.
    \end{equation}
\end{lemma}

\begin{proof}
    This is a computation using the lemma \ref{LEMooIBJMooTYnooZ}:
    \begin{equation}
        [\sigma_i,\sigma_j]=\delta_{ij}\mtu+\sum_ki\epsilon_{ijk}\sigma_k-\delta_{ji}\mtu-\sum_{k}i\epsilon_{jik}\sigma_k=2i\sum_k\epsilon_{ijk}\sigma_k
    \end{equation}
    where we used the fact that \( \epsilon_{jik}=-\epsilon_{ijk}\).
\end{proof}

\begin{lemma}       \label{LEMooLNCSooPHsVut}
    If \( a,b\in \eR^3\) we have
    \begin{equation}
        (a\cdot \sigma)(b\cdot \sigma)=(a\cdot b)\mtu+i(a\times b)\cdot \sigma.
    \end{equation}
\end{lemma}

\begin{proof}
    We use lemme \ref{LEMooIBJMooTYnooZ}:
    \begin{subequations}
        \begin{align}
            (a\cdot b)(b\cdot \sigma)&=\big( \sum_i a_i\sigma_i \big)(\sum_jb_j\sigma_j)\\
            &=\sum_{ij}a_ib_j\sigma\sigma_j\\
            &=\sum_{ij}a_ib_j(\delta_{ij}\mtu+\sum_mi\epsilon_{ijm}\sigma_m)
        \end{align}
    \end{subequations}
    Then we use the property \( \sum_{kl}a_kb_l\epsilon_{klm}=(a\times b)_m\).
\end{proof}

%///////////////////////////////////////////////////////////////////////////////////////////////////////////////////////////
\subsubsection{Isomorphism with \( \eR^3\)}
%///////////////////////////////////////////////////////////////////////////////////////////////////////////////////////////

The following lemme does not aims to provide a norm on \( V\). The norm on \( V\) is already the operator norm:
\begin{equation}
    \|X  \|=\sup_{| z |=1}| Xz |
\end{equation}
where \( z\in \eC\) and \( X\in V\). This is not something new.

In the same perspective, the elements of \( \SU(2)\) are normed to \( 1\) because, if \( U\in \SU(2)\),
\begin{equation}
    \| U \|_{\SU(2)}=\sup_{| z |=1}| Uz |=\sup\langle Uz, Uz\rangle =\sup_{| z=1 |}\langle U^{\dag}Uz, z\rangle =\sup_{| z |=1} \langle z, z\rangle=1.
\end{equation}

\begin{lemma}       \label{LEMooRFBTooIRDbEq}
    The map
    \begin{equation}
        \begin{aligned}
            \phi\colon \eR^3&\to V \\
            a&\mapsto a\cdot \sigma 
        \end{aligned}
    \end{equation}
    is a vector space isomorphism and satisfies
    \begin{equation}
        \det\big( \phi(x) \big)=-\| x \|^2.
    \end{equation}
\end{lemma}
 
\begin{proof}
    Some immediate facts:
    \begin{itemize}
        \item \( \phi\) is linear,
        \item \( \phi\) is bijective because \( \{ \sigma_i \}_{i=1,2,3}\) is a basis (lemme \ref{LEMooZNCQooLgoReX}).
        \item Thus \( \phi\) is a vector space isomorphism.
    \end{itemize}
    The formula \( \det\big( \phi(x) \big)=-\| x \|^2\) is a computation:
    \begin{subequations}
        \begin{align}
        \det\big( \phi(x) \big)&=\det\begin{pmatrix}
            x_3    &   x_1-ix_2    \\ 
            x_1+ix_2    &   -x_3
        \end{pmatrix}\\
        &=-x_3^2-(x_1-ix_2)(x_1+ix_2)\\
        &=-(x_1^2+x_2^2+x_3^2)\\
        &=-\| x \|^2.
        \end{align}
    \end{subequations}
\end{proof}

Some more properties about the Pauli matrices and the map \( \phi\).
\begin{proposition}[\cite{ooJQZGooElGniq}]
    Let \( x,y\in \eR^3\). We have
    \begin{enumerate}
        \item       \label{ITEMooDDRNooGZASBN}
            $[\sigma_i,\sigma_j]=2i\sum_k\epsilon_{ijk}\sigma_k$.
        \item       \label{ITEMooXORKooXFwQhR}
            $\phi(x\times y)=\frac{1}{ 2i }[\phi(x),\phi(y)]$.
        \item       \label{ITEMooREMBooLPVnxz}
            \( \tr\big( \phi(x)\phi(y) \big)=2x\cdot y\).
    \end{enumerate}
\end{proposition}

\begin{proof}
    Several points.

    \begin{subproof}
        \item[Formula \ref{ITEMooDDRNooGZASBN}]
            We use the product formula of lemma \ref{LEMooIBJMooTYnooZ}:
            \begin{equation}
                [\sigma_i,\sigma_j]=\sigma_i\sigma_j-\sigma_j\sigma_i=\delta_{ij}\id+i\sum_k\epsilon_{ijk}\sigma_k-\delta_{ij}\id-i\sum_k\epsilon_{jik}\sigma_k.
            \end{equation}
            Using the fact that \( \epsilon_{ijk}=-\epsilon_{jik}\) we get the result.
        \item[Formula \ref{ITEMooXORKooXFwQhR}]
            We have
            \begin{equation}
                [\phi(x),\phi(y)]=[x\cdot \sigma,y\cdot \sigma]=\sum_{ij}x_iy_j[\sigma_i,\sigma_j].
            \end{equation}
            Substituting the first result and using the formula \( \sum_{ij}x_iy_j\epsilon_{ijk}=(x\times y)_k\)\footnote{Definition \ref{DEFooTNTNooRjhuJZ}.} we get
            \begin{equation}
                [\phi(x),\phi(y)] = \sum_{ijk}x_iy_j2i\epsilon_{ijk}\sigma_k=2i\sum_k(x\times y)_k\sigma_k=2i(x\times y)\cdot \sigma=2i\phi(x\times y).
            \end{equation}
        \item[Formula \ref{ITEMooREMBooLPVnxz}]
            Using formula of lemma \ref{LEMooLNCSooPHsVut},
            \begin{subequations}
                \begin{align}
                    \tr\big( \phi(x)\phi(y) \big)&=\tr\big( (x\cdot \sigma)(y\cdot \sigma) \big)\\
                    &=\sum_i\big[ (x\cdot \sigma)(y\cdot \sigma) \big]_{ii}\\
                    &=\tr\big( (x\cdot y)\mtu_2+i(x\times y)\cdot \sigma \big)      \label{SUBEQooRJKRooNrLhhV}\\
                    &=(x\cdot y)\tr(\mtu_2)\\
                    &=2(x\cdot y).
                \end{align}
            \end{subequations}
            We used the fact that the Pauli matrice have vanishing trace, so that the second term in \eqref{SUBEQooRJKRooNrLhhV} is zero.
    \end{subproof}
\end{proof}

%///////////////////////////////////////////////////////////////////////////////////////////////////////////////////////////
\subsubsection{Path connection}
%///////////////////////////////////////////////////////////////////////////////////////////////////////////////////////////

\begin{proposition}     \label{PROPooLEKXooSXPhRX}
    The Lie groups \( \SO(3)\) and \( \SU(2)\) are path connected.
\end{proposition}

%///////////////////////////////////////////////////////////////////////////////////////////////////////////////////////////
\subsubsection{One representation}
%///////////////////////////////////////////////////////////////////////////////////////////////////////////////////////////

We still consider \( V\), the real vector space of hermitian matrices with vanishing trace. Thanks to the lemma \ref{LEMooRFBTooIRDbEq} we define the following norm on \( V\):
\begin{equation}
    \| X \|=\| \phi^{-1}(X) \|_{\eR^3}=-\det(X).
\end{equation}

For the sake of notational convenience in the proof of the next proposition, we introduce the maps \( L\) and \( R\).
\begin{lemma}       \label{LEMooQVYXooQFNaGc}
    The maps
    \begin{equation}
        \begin{aligned}
            L\colon \GL(V)&\to \End\big( \End(V) \big) \\
            L(A)X&=AX
        \end{aligned}
    \end{equation}
    and
    \begin{equation}
        \begin{aligned}
            R\colon \GL(V)&\to \End\big( \End(V) \big) \\
            R(A)X&=XA
        \end{aligned}
    \end{equation}
    are continuous.
\end{lemma}

\begin{proof}
    We prove our statement for \( L\). Let \( A_k\stackrel{\End(V)}{\longrightarrow}A\). We want to prove that
    \begin{equation}
        \| L(A_k)-L(A) \|_{\End\big( \End(V) \big)}\to 0.
    \end{equation}
    Using the definition of the operator norm\footnote{Definition \ref{DefNFYUooBZCPTr}.}, and the fact that it is an algebra norm (lemme \ref{LEMooFITMooBBBWGI}),
    \begin{subequations}
        \begin{align}
            \| L(A_k)-L(A) \|&=\sup_{X\in \End(V)}\frac{ \| L(A_k)X-L(A)X \| }{ \| X \| }\\
            &=\sup_{X\in \End(V)}\frac{ \| (A_k-A)X \|_{\End(V)} }{ \| X \| }\\
            &\leq \sup_{X\in\End(V)}\| A_k-A \|\\
            &\to 0.
        \end{align}
    \end{subequations}
    Thus \( L\) is continuous by proposition \ref{PROPooJYOOooZWocoq}.
\end{proof}

\begin{proposition}     \label{PROPooRQUZooAoZzwx}
    We still consider \( V\), the real vector space of hermitian matrices\footnote{Definition \ref{DEFooOKGXooFCzCHu}.} with vanishing trace. Let
    \begin{equation}
        \begin{aligned}
            \rho\colon \SU(2)&\to \End(V) \\
            \rho(U)X&=UXU^{\dag}.
        \end{aligned}
    \end{equation}
    Then
    \begin{enumerate}
        \item
            The map \( \rho\) is well defined: \( \rho(U)X\in V\) for every \( U\in \SU(2)\) and \( X\in V\).
        \item
            The map \( \rho\) is a representation of \( \SU(2)\) on \( V\) by isometries,
        \item       \label{ITEMooBZUQooNXNVfs}
            for each \( U\) the map \( \rho(U)\colon V\to V\) is continuous,
        \item       \label{ITEMooGHZYooQuabWb}
            the map \( \rho\colon \SU(2)\to \End\big( \End(V)\big) \) is continuous.
    \end{enumerate}
\end{proposition}

\begin{proof}
    Let us prove that \( \rho(U)X\in V\). First, using the properties of lemma \ref{LEMooJYGRooPTMZwY},
    \begin{equation}
        (UXU^{\dag})^{\dag}=(U^{\dag})^{\dag}X^{\dag}U^{\dag}=UXU^{\dag}.
    \end{equation}
    Then, with the cyclic invariance of the trace (lemma \ref{LEMooUXDRooWZbMVN}),
    \begin{equation}
        \tr(UXU^{\dag})=\tr(U^{\dag}UX)=\tr(X)=0.
    \end{equation}
    So \( UXU^{-1}=UXU^{\dag}\in V\).

    The fact that \( \rho(U)\) is linear is a small computation. It is a representation because
    \begin{equation}
        \rho(U_1)\rho(U_2)X=\rho(U_1)U_2XU_2^{\dag}=U_1U_2XU_2^{\dag}U_1^{\dag}=\rho(U_1U_2)X.
    \end{equation}
    
    For the isometry part, the determinant being multiplicative (proposition \ref{PROPooHQNPooIfPEDH}),
    \begin{equation}
        \| UXU^{\dag} \|=-\det(UXU^{\dag})=-\det(U)\det(X)\det(U^{\dag})=-| \det(U) |\det(X).
    \end{equation}
    Since \( U\in\SU(2)\) we have \( | \det(U) |=1\) and then \( \| \rho(U)X \|=\| X \|\).

    The last point to check is the continuity of \( \rho\colon \SU(2)\to \End(V)\). With the notations of lemma \ref{LEMooQVYXooQFNaGc} we have \( \rho(U)=L(U)\circ R(U^{\dag})\) while \( L(U)\) and \( R(U)\) are continuous\footnote{They are matrix multiplication.}. This is point \ref{ITEMooBZUQooNXNVfs} of continuity.

    The point \ref{ITEMooGHZYooQuabWb} of the continuity statement is more subtle. It is done in proposition \ref{PROPooJGNFooEwtNmJ}.
\end{proof}

%--------------------------------------------------------------------------------------------------------------------------- 
\subsection{Link with \texorpdfstring{$ \SO(3)$}{SO(3)}}
%---------------------------------------------------------------------------------------------------------------------------

\begin{proposition}[\cite{BIBooYTTJooYpPYLT}]     \label{PROPooGEHAooPCReoU}
    The map
    \begin{equation}        \label{EQooSOZTooTIkONx}
        \begin{aligned}
            f\colon \SU(2)&\to \SO(3) \\
            U&\mapsto \phi^{-1}\circ \rho(U)\circ\phi 
        \end{aligned}
    \end{equation}
    where
    \begin{equation}
        \begin{aligned}
            \phi\colon \eR^3&\to V \\
            x&\mapsto x\cdot \sigma 
        \end{aligned}
    \end{equation}
    is
    \begin{enumerate}
        \item
            continuous,
        \item
            a group homomorphism,
        \item
            surjective,
        \item
            \( \ker(f)=\{ \pm \mtu \}\).
    \end{enumerate}
\end{proposition}

\begin{proof}
    Several points.
    \begin{subproof}
        \item[Continuous]
            We know from proposition \ref{PROPooRQUZooAoZzwx} that \( U\mapsto \rho(U)\) is continuous. The inequality
            \begin{equation}
                \| \phi^{-1}\circ\rho(U_k)\circ \phi-\phi^{-1}\circ\rho(U)\circ\phi \|\leq \| \phi^{-1} \|\| \phi(U_k)-\rho(U) \|\| \phi \|
            \end{equation}
            shows that \( f\) is continuous too.
        \item[Group homomorphism]
            Let \( U_1,U_2\in \SU(2)\) and \( x\in \eR^3\). We have
            \begin{subequations}
                \begin{align}
                    f(U_1U_2)x&=\phi^{-1}\big( U_1U_2\phi(x)U_2^{-1}U_1^{-1} \big)\\
                    &=\phi^{-1}\Big( \rho(U_1)\big( U_2\phi(x)U_2^{-1} \big) \Big)\\
                    &=\phi^{-1}\big( \rho(U_1)\circ\rho(U_2)\phi(x) \big)\\
                    &=\big( \phi^{-1}\circ\rho(U_1)\circ\rho(U_2)\circ\phi \big)(x).
                \end{align}
            \end{subequations}
            We can suppress the dependency on \( x\) and continue:
            \begin{subequations}
                \begin{align}
                    f(U_1U_2)&=\phi^{-1}\circ\rho(U_1)\circ\rho(U_2)\circ\phi\\
                    &=\phi^{-1}\circ\rho(U_1)\circ\phi\circ\phi^{-1}\circ\rho(U_2)\circ\phi\\
                    &=f(U_1)\circ f(U_2).
                \end{align}
            \end{subequations}
            Since \( \rho(\id)=\id\) we also have \( f(\id)=\id\). Thus \( f\) is a group homomorphism.
        \item[Surjective]
            Elements of \( \SO(3)\) are compositions of two reflexions by corollary \ref{CORooJCURooSRzSFb}. A generic element of \( \SO(3)\) has the form \(ST \) where \( S\) and \( T\) are reflexions. They have the form
            \begin{equation}
                \begin{aligned}
                    S\colon \eR^3&\to \eR^2 \\
                    x&\mapsto x-2(x\cdot n_1)n_1 
                \end{aligned}
            \end{equation}
            and
            \begin{equation}
                \begin{aligned}
                    T\colon \eR^3&\to \eR^3 \\
                    x&\mapsto x-2(x\cdot n_2)n_2 
                \end{aligned}
            \end{equation}
            with \( \| n_1 \|=\| n_2 \|=1\).
            
            Let \( M= \phi(n_1) =n_1\cdot \sigma\) and \( Q=\phi(n_2)=n_2\cdot \sigma\). We will prove that \( MQ\in \SU(2)\) and \( f(MQ)=S\circ T\).

            \begin{subproof}
                \item[\( M^2=\mtu\)]
                    We have
                    \begin{equation}
                        M^2=(n_1\cdot n_1)\mtu_2+i(n_1\times n_1)\cdot \sigma=\mtu_2.
                    \end{equation}
                \item[\( \det(M)=-1\)]
                    We know from lemma \ref{LEMooRFBTooIRDbEq} that \( \det(M)=\det\big( \phi(n_1) \big)=-\| n_1 \|^2=-1\).
                \item[\( MQ\in \SU(2)\)]
                    First, \( \det(MQ)=\det(M)\det(Q)=-1\). Second, since \( M\) and \( Q\) are hermitian, \( (MQ)^{\dag}=Q^{\dag}M^{\dag}=QM\) and then
                    \begin{equation}
                        (MQ)^{\dag}(MQ)=QMMQ=\mtu_2
                    \end{equation}
                    because \( M^2=Q^2=1\).
                \item[\( \phi(Sx)=-M\phi(x)M\)]
                    Let \( x\in \eR^3\). We have
                    \begin{subequations}        \label{EQooSHKEooAhOxfH}
                        \begin{align}
                        \phi(Sx)&=\phi\big( x-2(x\cdot n_1)n_1 \big)\\
                        &=\phi(x)-2(x\cdot n_1)\phi(n_1)\\
                        &=\phi(x)-2(x\cdot n_1)M.
                        \end{align}
                    \end{subequations}
                    Using the formula \( (a\cdot \sigma)(b\cdot \sigma)=(a\cdot b)\mtu+i(a\times b)\cdot \sigma\) we have
                    \begin{equation}
                        \phi(x)M=\phi(x)\phi(n_1)=(x\cdot n_1)\mtu+i(x\times n_1)\cdot \sigma
                    \end{equation}
                    and
                    \begin{equation}
                        M\phi(x)=(n_1\cdot \sigma)(x\cdot \sigma)=(n_1\cdot x)\mtu+i(n_1\times x)\cdot \sigma,
                    \end{equation}
                    si that
                    \begin{equation}
                        \phi(x)M+M\phi(x)=2(x\cdot n_1)\mtu.
                    \end{equation}
                    Multiplying that by \( M\) and using \( M^2=\mtu\) we deduce
                    \begin{equation}
                        \phi(x)=2(x\cdot n_1)M-M\phi(x)M.
                    \end{equation}
                    Now we substitute that into \eqref{EQooSHKEooAhOxfH} in order to see that
                    \begin{equation}
                        \phi(Sx)=-M\phi(x)M.
                    \end{equation}
                \item[Conclusion (surjective)]
                    We can now compute the action of \( f(MQ)\) on \( x\in \eR^3\):
                    \begin{subequations}
                        \begin{align}
                            f(MQ)x&=\big( \phi^{-1}\circ\rho(MQ)\circ\phi \big)x\\
                            &=\phi^{-1}\big( MQ\phi(x)QM \big)\\
                            &=\phi^{-1}\big( M\phi(Tx)M \big)\\
                            &=\phi^{-1}\big( \phi(STx) \big)\\
                            &=STx.
                        \end{align}
                    \end{subequations}
                    So we have \( f(MQ)=ST\) and \( f\) is surjective.
            \end{subproof}
        \item[Kernel]
            Let \( U\in \SU(2)\) be such that \( f(U)=\mtu_3\in \SO(3)\). For every \( x\in \eR^3\) we have \( x=f(U)x\) while
            \begin{equation}
                f(U)x=\phi^{-1}\big( U\phi(x)U^{-1} \big).
            \end{equation}
            We conclude that \( U\phi(x)U^{-1}=\phi(x)\) for every \( x\in \eR^3\). Since \( f\) is surjective on the vector space \( V\) of hermitian matrices with vanishing trace, we have
            \begin{equation}
                UXU^{\dag}=X
            \end{equation}
            for every \( X\in V\). In particular \( UX=XU\). Since the matricial product is continuous, we can commute \( U\) and the infinite sum and get
            \begin{equation}
                U\sum_{k=0}^{\infty}\frac{ (iX)^k }{ k! }=\lim_{N\to \infty} \sum_{k=0}^N\frac{ U(iX)^k }{ k! }=\lim_{N\to \infty} \sum_{k=0}^N\frac{ (iX)^kU }{ k! }= e^{iX}U.
            \end{equation}
            So we have \( [U, e^{iX}]=0\) for every \( X\in V\). Since proposition \ref{PROPooZBJSooEIguXR} says that every element of \( \SU(2)\) is the exponential of an element in \( V\) the element \( U\) is in the center of \( \SU(2)\). The center of \( \SU(2)\) is \( \{ \pm\id \}\) by the proposition \ref{PROPooLMGHooKrKpsa}.

            Until now we have \( \ker(f)\subset \{ \id,-\id \}\). It is a simple verification to check that \( \{ \id,-\id \}\) are in the kernel of \( f\). We conclude that \( \ker(f)=\{ \pm\mtu_2 \}\).
    \end{subproof}
\end{proof}

\begin{proposition}     \label{PROPooDKPTooBnLflt}
    We have the group isomorphism
    \begin{equation}
        \SO(3)=\frac{ \SU(2) }{ \eZ_2 }.
    \end{equation}
\end{proposition}

\begin{proof}
    We use the first isomorphism theorem \ref{ThoPremierthoisomo} with \( \theta\) being the map \( f\colon \SU(2)\to \SO(3)\) defined the proposition \ref{PROPooGEHAooPCReoU}. It says that
    \begin{equation}
        \frac{ \SU(2) }{ \ker(f) }=\Image(f).
    \end{equation}
    The known properties of \( f\) are that \( \ker(f)=\eZ_2\) and \( \Image(f)=\SO(3)\). This is the expected result.
\end{proof}

\begin{lemma}       \label{LEMooSYGUooVWxGYX}
    The images of the unitary matrices \( i\sigma_k\) by \( f\) are
    \begin{equation}
        \begin{aligned}[]
            f(i\sigma_1)=\begin{pmatrix}
                1    &       &       \\
                    &   -1    &       \\
                    &       &   -1
            \end{pmatrix},
            f(i\sigma_2)=\begin{pmatrix}
                -1    &       &       \\
                    &   1    &       \\
                    &       &   -1
            \end{pmatrix},
            f(i\sigma_3)=\begin{pmatrix}
               -1     &       &       \\
                    &   -1    &       \\
                    &       &   1
            \end{pmatrix}.
        \end{aligned}
    \end{equation}
\end{lemma}

\begin{proof}
    We know that 
    \begin{equation}
        \phi(x)=\begin{pmatrix}
            x_3    &   x_1-ix_2    \\ 
            x_1+ix_2    &   -x_3    
        \end{pmatrix}.
    \end{equation}
    We have
    \begin{equation}
        (i\sigma_1)\phi(x)(i\sigma_1)^{\dag}=\sigma_1\phi(x)\sigma_1=\begin{pmatrix}
            -x_3    &   x_1+ix_2    \\ 
            x_1-ix_2    &   x_3    
        \end{pmatrix}=\phi\begin{pmatrix}
            x_1    \\ 
            -x_2    \\ 
            -x_3    
        \end{pmatrix}.
    \end{equation}
    This shows that
    \begin{equation}
        f(i\sigma_1)=\begin{pmatrix}
            x_1    \\ 
            -x_2    \\ 
            -x_3    
        \end{pmatrix},
    \end{equation}
    so that the matrix of \( i\sigma_1\) is
    \begin{equation}
        f(i\sigma_1)=\begin{pmatrix}
            1    &       &       \\
                &   -1    &       \\
                &       &   -1
        \end{pmatrix}.
    \end{equation}
    The same kind of computations provide the result.
\end{proof}

\begin{proposition}
    Let \( f\colon \SU(2)\to \SO(3)\) be the map of the proposition \ref{PROPooGEHAooPCReoU}:
    \begin{equation}
        \begin{aligned}
            f\colon \SU(2)&\to \SO(3) \\
            f(U)&=\phi^{-1}\circ\rho(U)\circ \phi.
        \end{aligned}
    \end{equation}
    There exist no group homomorphism \( g\colon \SO(3)\to \SU(2)\) such that \( f\circ g=\id\).
\end{proposition}

\begin{proof}
    Let \( g\) be such an homomorphism and let's derive a contradiction. Since \( g\) is an homomorphism it satisfies \( g(\mtu_3)=\mtu_2\). Let 
    \begin{equation}
        T_x=\begin{pmatrix}
            1    &   0    &   0    \\
            0    &   -1    &   0    \\
            0    &   0    &   -1
        \end{pmatrix}\in \SO(3).
    \end{equation}
    The map \( f\) is surjective, so there exist \( U\in \SU(2)\) such that \( f(U)=T_x\). From lemma \ref{LEMooSYGUooVWxGYX} we have
    \begin{equation}
        f(i\sigma_1)=f(-i\sigma_1)=T_x.
    \end{equation}
    Thus \( g(T_x)=i\sigma_1\) or \( g(T_x)=-i\sigma_1\). In both cases we have a contradiction. Indeed, since \( T_x^2=\mtu\) and \( g(T_x^2)=g(T_x)^2\) we must have \( g(T_x)^2=\mtu\) while
    \begin{equation}
        (i\sigma_1)^2=(-i\sigma_1)^2=\mtu.
    \end{equation}
\end{proof}

\begin{lemma}       \label{LEMooRCSSooTvAaJY}
    Let \( \alpha_0\in \SO(3)\) and \( U\in \SU(2)\) such that \( f(U)=\alpha_0\). There exists a neighborhood \(\mO\) of \( \alpha_0\) in \( \SU(2)\) such that
    \begin{equation}
        f^{-1}(\mO)=V_1\cup V_2
    \end{equation}
    where \( V_1\) is a neighborhood of \( U\), \( V_2\) is a neighborhood of \( -U_1\) and \( V_1\cap V_2=\emptyset\).
\end{lemma}

\begin{proof}
    We know that \( f(U)=f(-U)=\alpha_0\). Let \( W_1\) be a neighborhood of \( U\) and \( W_2\) be a neighborhood of \( -U\) such that \( W_1\cap W_2=\emptyset\).

    The part \( -W_2\) is a neighborhood of \( U\). we consider \( V_1\), a neighborhood of \( U\) contained in \( W_1\cap -W_2\). Then we set \( V_2=-V_1\). This is a neighborhood of \(-U\) contained in \( W_2\).

    Thus we have \( V_1\cap V_2=\emptyset\) and \( f(V_1)=f(V_2)\) is a neighborhood of \( \alpha_0\). It remains to define \( \mO=f(V_1)\).
\end{proof}

\begin{proposition}       \label{PROPooHQENooUsQeiZ}
    Let \( U\in \SU(2)\) be such that 
    \begin{equation}
        UX=XU
    \end{equation}
    for every \( X\in V\). Then \( U\in\{\mtu,-\mtu\}\).
\end{proposition}

\begin{proof}
    The proof is a simple computation. Let \( a,b,c,d\in \eC\) such that \( U=\begin{pmatrix}
        a    &   b    \\ 
        c    &   d    
    \end{pmatrix}\). We have
    \begin{equation}
        U\sigma_1=\begin{pmatrix}
            a    &   b    \\ 
            c    &   d    
        \end{pmatrix}\begin{pmatrix}
            0    &   1    \\ 
            1    &   0    
        \end{pmatrix}=\begin{pmatrix}
            b    &   a    \\ 
            d    &   c    
        \end{pmatrix}
    \end{equation}
    while
    \begin{equation}
        \sigma_1U=\begin{pmatrix}
            c    &   d    \\ 
            a    &   b    
        \end{pmatrix}.
    \end{equation}
    We deduce \( b=c\) and \( a=d\) and \( U=\begin{pmatrix}
        a    &   b    \\ 
        b    &   a    
    \end{pmatrix}\). Taking that into account, the same work with \( \sigma_2\) provides
    \begin{equation}
        U\sigma_2=\begin{pmatrix}
            bi    &   -ia    \\ 
            ia    &   -ib    
        \end{pmatrix}
    \end{equation}
    and 
    \begin{equation}
        \sigma_2U=\begin{pmatrix}
            -bi    &   -ia    \\ 
            ia    &   bi    
        \end{pmatrix},
    \end{equation}
    so that \( b=0\). Now \( U=\begin{pmatrix}
        a    &   0    \\ 
        0    &   a    
    \end{pmatrix}\) for some \( a\in \eC\).

    The constrain \( U\sigma_3=\sigma_3U\) does not provide new informations.

    Since \( U\in \SU(2)\) we have \( \det(U)=1\) which implies \( a^2=1\), which in turn means \( a=\pm1\).
\end{proof}

\begin{lemma}[\cite{MonCerveau}]        \label{LEMooBHVBooEPbWwZ}
    Let \( U_k\in \SU(2)\) such that 
    \begin{enumerate}
        \item
            \( \rho_{U_k}\stackrel{\End(V)}{\longrightarrow}\id_V\)
        \item
            there exists a neighborhood of \( -\mtu\) in \( \SU(2)\) which contains no element \( U_k\).
    \end{enumerate}
    Then we have \( U_k\stackrel{\SU(2)}{\longrightarrow}\mtu\).
\end{lemma}

\begin{proof}
    Let \( A_k\) be a converging subsequence of \( U_k\) (we will see later that it exists) with \( A_k\stackrel{\SU(2)}{\longrightarrow}A\). For each \( X\in V\) we have \( A_kXA_k^{-1}\stackrel{\SU(2)}{\longrightarrow}X\), so that
    \begin{subequations}
        \begin{align}
            \| A_kX-XA_k \|&=\| A_kXA_k^{-1}A_k-XA_k \|\\
            &\leq\| A_kXA_k^{-1}-X \|\| A_k \|\\
            &\leq \| A_kXA_k^{-1}-X \|M\to 0
        \end{align}
    \end{subequations}
    where \( M\) is some constant majoration of \( \| A_k \|\). Thus we have \( A_kX-XA_k\to 0\) which means 
    \begin{equation}
        XA=AX.
    \end{equation}
    If it is true for every \( X\), we conclude that \( A=\pm\mtu\) (proposition \ref{PROPooHQENooUsQeiZ}). Since there is a neighborhood of \( -\mtu\) in which there are no elements \( U_k\), we cannot have \( A_k\to -\mtu\), so we have \( A=\mtu\).

    Now \( U_k\) is a sequence in the compact \( \SU(2)\) (proposition \ref{PROPooGLPQooKOfrjl}), so that every subsequence has a converging subsequence\footnote{Bolzano-Weierstrass \ref{THOooZJWLooAtGMxD}.}. We are in the case of the lemma \ref{LEMooSJKMooKSiEGq} and we conclude \( U_k\stackrel{\SU(2)}{\longrightarrow}\mtu\).
\end{proof}

\begin{lemma}[\cite{MonCerveau}]        \label{LEMooMNWSooAjmBQK}
    Let \( U\in \SU(2)\). We consider the linear map\footnote{See proposition \ref{PROPooRQUZooAoZzwx}.}
    \begin{equation}
        \begin{aligned}
            \rho_U\colon V&\to V \\
            v&\mapsto UvU^{-1}. 
        \end{aligned}
    \end{equation}
    Then \( \| \rho_U \|\leq 1\).
\end{lemma}

\begin{proof}
    By definition, the norme of \( \rho_U\colon V\to V\) is
    \begin{equation}
        \| \rho_U \|=\sup_{\| v \|=1}\| \rho_Uv \|=\sup_{\| v \|=1}\| UvU^{-1} \|.
    \end{equation}
    In the last expression, the norms are in \( \End(\eC^2)\) because \( U\) and \( v\) are both \( 2\times 2\) complex matrices. The operator norm is an algebra norm\footnote{Lemma \ref{LEMooFITMooBBBWGI}.}, so that
    \begin{equation}
        \| UvU^{-1} \|\leq \| U \|\| U^{-1} \|\| v \|=\| v \|
    \end{equation}
    because the éléments of \( \SU(2)\) are normed to \( 1\). Thus
    \begin{equation}
        \| \rho_1 \|\leq \sup_{\| v \|=1}\| v \|=1.
    \end{equation}
\end{proof}

\begin{lemma}       \label{LEMooHPQQooIGwljm}
    Let \( U_k\in \SU(2)\) and \( U\in \SU(2)\) such that 
    \begin{enumerate}
        \item
            \( \rho_{U_k}\stackrel{\End(V)}{\longrightarrow}\rho_U\)
        \item
            there exists a neighborhood of \( -A\) in \( \SU(2)\) which contains no element \( U_k\).
    \end{enumerate}
    Then we have \( U_k\stackrel{\SU(2)}{\longrightarrow}U\).
\end{lemma}

\begin{proof}
    Several steps.
        \begin{subproof}
        \item[\( \rho_{U_k^{-1}U}\to\id\)]
                We start by proving that \( \rho_{U_k^{-1} U}\to \id\). For each \( v\in V\) we have
                \begin{equation}
                    \| \rho_{U_k^{-1} U}v-v \|=\| \rho_{U_k^{-1}}\big( \rho_Uv-\rho_{U_k}v \big) \|\leq \| \rho_{U_k^{-1}}\| \rho_Uv-\rho_{U_k}v \| \|
                \end{equation}
                Thus we have
                \begin{subequations}
                    \begin{align}
                        \| \rho_{U_k^{-1}U}-\id \|&=\sup_{\| v \|=1}\| \rho_{U_k^{-1}U}v-v \|\\
                        &\leq \| \rho_{U_k^{-1}} \|\sup_{\| v \|=1}\| \rho_Uv-\rho_{U_k}v \|\\
                        &=\| \rho_{U_k} \|\| \rho_u-\rho_{U_k} \|\\
                        &\leq \| \rho_U-\rho_{U_k} \|.
                    \end{align}
                \end{subequations}
                We used lemme \ref{LEMooMNWSooAjmBQK}. In conclusion,
                \begin{equation}
                    \| \rho_{U_k^{-1}U}-\id \|\leq \| \rho_U-\rho_{U_k} \|\to 0.
                \end{equation}
            \item[No neighborhood of \( -\mtu\)]   
                We prove that there exists a neighborhood of \( -\mtu\) which contains no elements of the sequence \( U_k^{-1}U\). Suppose that each neighborhood of \( -\mtu\) contains one of the \( U_k^{-1}U\). At this point we have a subsequence \( (B_k)\) of \( (U_k)\) such that 
                \begin{equation}
                    B_k^{-1}U\to-\mtu. 
                \end{equation}
                Since the multiplication and the inverse are continuous operations\footnote{The group \( \SU(2)\) is a Lie group, proposition \ref{PROPooWMKGooKftzGv}.} we also have 
                \begin{equation}
                    B_k^{-1}\to-U^{-1}
                \end{equation}
                and
                \begin{equation}
                    B_k\to -U.
                \end{equation}
                This prove that for every neighborhood of \( -U\) we have a \( B_k\) and then a \( U_k\), which is a contradiction with the hypothesis.
            \item[Conclusion] 
                The sequence \( U_k^{-1}U\) satisfies the lemma \ref{LEMooBHVBooEPbWwZ} and we conclude \( U_k^{-1}U\to \mtu\). Thus \( U_k\to U\).
        \end{subproof}
\end{proof}

\begin{proposition}[\cite{BIBooYTTJooYpPYLT,MonCerveau}]        \label{PROPooHCVZooMOSzTm}
    Let \( \alpha_0\in \SO(3)\) and \( \mO\) be a neighborhood of \( \alpha_0\) such that \( f^{-1}(\mO)=V_1\cup V_2\) with \( V_1\cap V_2=\emptyset\)\footnote{Such choice is possible by the lemma \ref{LEMooRCSSooTvAaJY}.}.

    The map
    \begin{equation}
        \begin{aligned}
            \varphi\colon \mO&\to\SU(2) \\
            \alpha&\mapsto f^{-1}(\alpha)\cap V_1 
        \end{aligned}
    \end{equation}
    is continuous.
\end{proposition}

\begin{proof}
    Let \( \sigma_k\stackrel{\SO(3)}{\longrightarrow} \alpha\) with \( \alpha_k\in \mO\). We have to prove that \( \varphi(\alpha_k)\stackrel{\SU(2)}{\longrightarrow}\varphi(\alpha)\).
    \begin{subproof}
        \item[General setting]
            First we suppose that \( \alpha_k\) converges to the identity. For each \( k\) we have
            \begin{equation}
                f\big( \varphi(\alpha_k) \big)=\alpha_k,
            \end{equation}
            with the map \eqref{EQooSOZTooTIkONx}. That means, for each \( k\):
            \begin{equation}
                \phi^{-1}\circ\rho_{\varphi(\alpha_k)}\circ \phi=\alpha_k,
            \end{equation}
            or
            \begin{equation}
                \rho_{\varphi(\alpha_k)}=\varphi\circ \alpha_k\circ\phi^{-1}
            \end{equation}
            as operator on \( V\).
        \item[Norm convergence]
            We have the following computation:
            \begin{subequations}
                \begin{align}
                    \| \rho_{\varphi(\alpha_k)}-\rho_{\varphi(\alpha)} \|_{\End(V)}&=\| \phi\circ\alpha_k\circ\phi^{-1}-\phi\circ\alpha\circ\phi^{-1} \|\\
                    &=\| \phi\circ(\alpha_k-\alpha)\circ\phi^{-1} \|\\
                    &\leq \| \phi \|\| \alpha_k-\alpha \|\| \phi^{-1} \|\to0.
                \end{align}
            \end{subequations}
            This shows that
            \begin{equation}        \label{EQooCEFUooTCoczi}
                \rho_{\varphi(\alpha_k)}\to \rho_{\varphi(\alpha)}.
            \end{equation}

        \item[Conclusion]

            The sequence \( U_k=\varphi(\alpha_k)\) and the element \( U=\varphi(\alpha)\) satisfy the lemma \ref{LEMooHPQQooIGwljm}, so that \( \varphi(\alpha_k)\to \varphi(\alpha)\).
    \end{subproof}
\end{proof}

\begin{proposition}
    The map \( f\colon \SU(2)\to \SO(3)\) is a representation of \( \SU(2)\) on \( \eR^3\), but is not faithful\footnote{Définition \ref{DEFooAFSAooGDSDBb}.}.
\end{proposition}

\begin{proof}
    The function \( f\) is written as
    \begin{equation}
        f(U)=\phi^{-1}\circ\rho_U\circ\phi.
    \end{equation}
    On the other hand, we have
    \begin{equation}
        \rho_{U_1U_2}v=U_1U_2v(U_1U_2)^{-1}=U_1U_2vU_2^{-1}U_1^{-1}=(\rho_{U_1}\circ\rho_{U_2})v.
    \end{equation}
    Thus
    \begin{equation}
        f(U_1U_2)=\phi^{-1}\circ\rho_{U_1}\circ\rho_{U_2}\circ\phi=\underbrace{\phi^{-1}\rho_{U_1}\phi}_{f(U_1)}\underbrace{phi^{-1}\rho_{U_2}\phi}_{f(U_2)}=f(U_1)\circ f(U_2).
    \end{equation}
    Thus \( f\) is a representation.

    It is not faithful because \( f(\mtu)=f(-\mtu)=\id\).
\end{proof}

%--------------------------------------------------------------------------------------------------------------------------- 
\subsection{The Lie algebras \texorpdfstring{\( \su(2)\)}{su(2)} and \texorpdfstring{$ \so(3)$}{so(n)}}
%---------------------------------------------------------------------------------------------------------------------------

\begin{proposition}         \label{PROPooSERWooFtxBgV}
    The Lie algebra \( \su(n)\) of \( \SU(n)\) is the space of traceless anti-hermitian matrices\footnote{See the definitions \ref{DEFooKDCPooZOJsMD} and \ref{DEFooJJVIooDUBwDJ}.}.
\end{proposition}

\begin{proof}
    Let consider $G=\SU(n)$; the elements are complexes $n\times n$ matrices $U$ such that $UU^{\dag}=\mtu$ and $\det U=1$. An element of the Lie algebra is given by a path $\dpt{u}{\eR}{G}$ in the group with $u(0)=\mtu$. Since \( \SU(n)\) is a Lie subgroup of \( \GL(n,\eC)\)\footnote{Proposition \ref{PROPooWMKGooKftzGv}.}, by the proposition \ref{PROPooSQHLooGQAykc}, it is sufficient to compute the usual derivative of such a path.
    Since for all $t$, $u(t)u(t)^{\dag}=\mtu$,
    \begin{subequations}
        \begin{align}
      0&=\Dsdd{u(t)u(t)^{\dag}}{t}{0}\\
       &=u(0)\Dsdd{u(t)^{\dag}}{t}{0}+\Dsdd{u(t)}{t}{0}u(0)^{\dag}\\
       &=[d_tu(t)]^{\dag}+[d_tu(t)].
        \end{align}
    \end{subequations}
    So a general element of the Lie algebra $\su(n)$ is an anti-hermitian matrix.

    An element of \( \SU(n)\) has also a determinant equal to \( 1\). What condition does it implies on the elements of the Lie algebra? 
    
    Let \( X\) be an element of \( \su(2)\). For each \( t\), the element \(  e^{tX}\) is part of the Lie group and satisfy \( \det( e^{tX})=1\). Using the formula\footnote{Corollary \ref{CORooOKKSooHrsYOs}.}
    \begin{equation}
        \Dsdd{ \det( e^{tX}) }{t}{0}=\tr(X)
    \end{equation}
    we deduce \( \tr(X)=0\).
    
    An other way to prove the same result is to consider a path in \( \SU(n)\) and derive; let's do it. If \( g(t)\) is a path in \( \SU(n)\) with \( g(0)=\mtu\). For each \( t\) we have \( \det\big( g(t) \big)=1\).
    
    Using the formula expression the determinant with the minors,
    \begin{equation}
        \det\begin{pmatrix}
            g_{11}(t)    &   g_{12}(t)    &   \ldots    \\
            f_{21}(t)    &   g_{22}(t)    &   \ldots    \\
            \vdots    &   \vdots    &   \ddots
        \end{pmatrix}=g_{11}(t)M_{11}(t)+g_{12}(t)M_{12}(t)+\ldots=1
    \end{equation}
    where \( M_{ij}\) is the minor of \( g\). If we derive the left hand side we get
    \begin{equation}
        g'_{11}(0)M_{11}(0)+g_{11}(0)M'_{11}(0)+g'_{12}(0)M_{12}(0)+g_{12}(0)M'_{12}(0)+\ldots
    \end{equation}
    where the numbers \( g'_{ij}(0)\) are the matrix entries of the tangent matrix, that is the matrix elements of a general element in \( \gsu(n)\). Since \( g(0)=\mtu\) we have \( M_{11}(0)=1\), \( g_{11}(0)=1\), \( M_{12}(0)=0\) and \( g_{12}(0)=0\). Thus we have
    \begin{equation}
        (\det g)'(0)=X_{11}+M'_{11}(0)
    \end{equation}
    where \( X=g'(0)\). By induction we found that the trace of \( X\) appears. Thus the elements of \( \gsu(n)\) have vanishing trace.
\end{proof}

\begin{normaltext}
    The space \( V\) spanned by the matrices \( \sigma_i\) is not \( \su(2)\).
\end{normaltext}

\begin{proposition}     \label{PROPooDNNEooMOdrkq}
    The Lie algebra \( \so(n)\) is the vector space of antisymmetric matrices.
\end{proposition}

\begin{proof}
    As said in the proposition \ref{PROPooSQHLooGQAykc}, the Lie algebra \( \so(n)\) can be seen as \( \SO(n)'\), the Lie algebra of the matrices obtained by componentwise derivate paths in \( \SO(n)\). 
    
    \begin{subproof}
        \item[Inclusion in one sense]
            So let be a path \( g\colon \eR\to \SO(n)\). For each \( t\) we have the equality
            \begin{equation}
                g(t)g(t)^t=\mtu.
            \end{equation}
            We differentiate that equation with respect to \( t\) at \( t=0\) taking into account \( g(0)=\mtu\):
            \begin{equation}
                g'(0)+g'(0)^t=0.
            \end{equation}
            This shows that the matrices of the Lie algebra \( \so(n)\) are skew-symmetric.
        \item[Inclusion in the other sense]
            Now we prove that every skew-symmetric matrix is of the form \( \gamma'(0)\) for some path \( \gamma\colon \eR\to \SO(n) \). Let \( X\) be a skew-symmetric matrix. We consider the path
            \begin{equation}
                \gamma(t)= e^{tX}
            \end{equation}
            defined in the proposition \ref{PropPEDSooAvSXmY}. We have to prove that \( \gamma(t)\in \SO(3)\) for every \( t\) (at least in a neighborhood of \( t=0\)) and that \( \gamma(0)=\mtu\).
            \begin{subproof}
                \item[\( \gamma(0)=\mtu\)]
                    This is the proposition \ref{PROPooFLHPooRhLiZE}\ref{ITEMooCVALooEfLQCyI}.
                \item[\( \gamma(t)\) is orhogonal]
                    By proposition \ref{PROPooFLHPooRhLiZE}\ref{ITEMooEOSMooQWjcjA} we know that 
                    \begin{equation}
                        ( e^{tX})^t= e^{tX^t}.
                    \end{equation}
                    Since \( X\) is skew-symmetric we also have \( [X,X^t]=0\), because
                    \begin{equation}
                        (XX^t)_{ij}=\sum_kX_{ik}X^t_{kj}=\sum_kX_{ik}X_{jk}
                    \end{equation}
                    while
                    \begin{equation}
                        (X^tX)_{ij}=\sum_kX^t_{ik}X_{kj}=\sum_kX_{ki}X_{kj}=\sum_kX_{ik}X_{jk}.
                    \end{equation}
                    The last equality accounts the fact that \( X\) is skew-symmetric. Since \( X\) and \( X^t\) commute we can use the theorem \ref{THOooXCPEooYGyLOp}:
                    \begin{equation}
                        e^{tX}( e^{tX})^t= e^{tX} e^{tX^t}= e^{t(X+X^t)}= e^{0}=\mtu.
                    \end{equation}
                \item[\( \gamma(t)\) is special] We prove that \( \det\big( \gamma(t) \big)=1\). The proposition \ref{PROPooZUHOooQBwfZq} provides
                    \begin{equation}
                        \det( e^{tX})= e^{\tr(tX)}= e^{0}=1.
                    \end{equation}
                    By the way, these equalities are equalities in \( \eR\), not equalities on \( \GL(n,\eR)\).
                \item[Pause]
                    We finished to prove that \( \gamma(t)\in \SO(n)\) for every \( t\in \eR\). We still have to prove that \( \gamma'(0)=X\).
                \item[\( \gamma'(0)=X\)]
                    This is from proposition \ref{PROPooSDNNooQtHkhA} :
                    \begin{equation}
                        \Dsdd{  e^{tX} }{t}{0}=X.
                    \end{equation}
            \end{subproof}
    \end{subproof}
\end{proof}

\begin{normaltext}
    Notice that antisymmetric matrices are automatically with vanishing trace.    
\end{normaltext}

\begin{proposition}     \label{PROPooHOOLooOrcquD}
    Two Lie algebras.
    \begin{enumerate}
        \item       \label{ITEMooFSTMooGSjovL}
    The Lie algebra of \( \gU(n)\) is the set \( \gu(n)\) of anti-hermitian matrices.
\item       \label{ITEMooYEFMooRmGmlF}
    The Lie algebra of \( \SU(n)\) is the set \( \su(n)\) of anti-hermitian matrices with vanishing trace\footnote{Just to be clear: as set this is the skew-hermitian matrices. As vector space, this is a real vector space. The fact to be skew-hermitian is not preserved by a multiplication by \( i\).}.
\item           \label{ITEMooXXTRooQZzCfs}
    A basis of \( \su(2)\) is given by the matrices \( t_k=-i\sigma_k\) where \( \sigma_k\) are the Pauli matrices\footnote{Definition \ref{DEFooRNTDooTVkPtB}.}, that is
    \begin{equation}
        \begin{aligned}[]
            t_1=i\sigma_1=\begin{pmatrix}
                0    &   i    \\ 
                i    &   0    
            \end{pmatrix}&&t_2=i\sigma_2=\begin{pmatrix}
                0    &   1    \\ 
                -1    &   0    
            \end{pmatrix}&&t_3=i\sigma_3=\begin{pmatrix}
                i    &   0    \\ 
                0    &   -i    
            \end{pmatrix}.
        \end{aligned}
    \end{equation}
\item
    The commutation relations in \( \su(2)\) are
    \begin{equation}        \label{EQooFJIDooRtQGjA}
        [t_i,t_j]=2\sum_k\epsilon_{ijk}t_k.
    \end{equation}
    \end{enumerate}
\end{proposition}

\begin{proof}
    The point \ref{ITEMooFSTMooGSjovL} is the same kind of proof that the one of proposition \ref{PROPooDNNEooMOdrkq}; the only difference is that one starts with \( g(t)g(t)^{\dag}=\mtu\). Then one use \(  e^{X^{\dag}}=( e^{X})^{\dag}\).

    The point \ref{ITEMooYEFMooRmGmlF}, is already proved in the proposition \ref{PROPooSERWooFtxBgV}.

    For the point \ref{ITEMooXXTRooQZzCfs}, an explicit computation shows that the matrices \( t_k\) belong to \( \su(2)\) and are linearly independent. Now there are two ways to proceed.
    
    One way is to prove that \( \su(2)\) has dimension \( 3\). For that, you can write down an explicit manifold structure on \( \SU(2)\) and show that it is a manifold of dimension \( 3\). Then the Lie algebra has the same dimension.

    An other way is to make it by hand. We consider a matrix \( \begin{pmatrix}
        a    &   c    \\ 
        d    &   b    
    \end{pmatrix}\in \eM(\eC,2)\). The fact to be traceless imposes \( a=-b\). Then we have
    \begin{equation}
        \begin{pmatrix}
            a    &   c    \\ 
            d    &   -a    
        \end{pmatrix}^{\dag}=\begin{pmatrix}
            \bar a    &   \bar d    \\ 
            \bar c    &   -\bar a    
        \end{pmatrix}.
    \end{equation}
    The condition to be skew-hermitian is
    \begin{equation}
        \begin{pmatrix}
            a    &   c    \\ 
            d    &   -a    
        \end{pmatrix}=-\begin{pmatrix}
            \bar a    &   \bar d    \\ 
            \bar c    &   -\bar a    
        \end{pmatrix}.
    \end{equation}
    That provide the constrains that \( a\) is purely imaginary and that, if \( c=x+iy\), then \( d=-x+iy\). Thus a generic matrix in \( \su(2)\) is given by
    \begin{equation}
        \begin{pmatrix}
            \lambda i    &   x+iy    \\ 
            -x+iy    &   -\lambda i    
        \end{pmatrix}=\lambda\begin{pmatrix}
            i    &   0    \\ 
            0    &   -i    
        \end{pmatrix}+x\begin{pmatrix}
            0    &   1    \\ 
            -1    &   0    
        \end{pmatrix}+y\begin{pmatrix}
            0    &   i    \\ 
            i    &   0    
        \end{pmatrix}=x(i\sigma_2)+y(i\sigma_1)+\lambda(i\sigma_3).
    \end{equation}
    with \( x,y,\lambda\in \eR\). That shows that \(  \{ i\sigma_k \} \) is a basis of \( \su(2)\).  In the same time this is a proof a proof that \( \su(2)\) has dimension \( 3\).

    The lemma \ref{LEMooJRWXooMkzRnk} provides the commutators for the Pauli matrices. We can adapt them for our basis of \( \su(2)\):
    \begin{equation}
        [t_i,t_j]=[i\sigma_i , i\sigma_j ]=-[\sigma_i,\sigma_j]=-2i\sum_k\epsilon_{ijk}\sigma_k=2\sum_k\epsilon_{ijk}t_k.
    \end{equation}
\end{proof}

\begin{proposition}
    The Lie algebras \( \su(2)\) and \( \so(3)\) are isomorphic.
\end{proposition}

\begin{proof}
    The propositions \ref{PROPooDNNEooMOdrkq} and \ref{PROPooHOOLooOrcquD} provide a description of \( \su(2)\) and \( \so(3)\). The easiest way to prove the isomorphism is to show an explicit isomorphism. A basis of \( \so(3)\) is
    \begin{equation}
        \begin{aligned}[]
            O_1&=\begin{pmatrix}
                0    &   0    &   0    \\
                0    &   0    &   1    \\
                0    &   -1    &   0
            \end{pmatrix},&O_2&=\begin{pmatrix}
                0    &   0    &   1    \\
                0    &   0    &   0    \\
                -1    &   0    &   0
            \end{pmatrix},&O_3&=\begin{pmatrix}
                0    &   1    &   0    \\
                -1    &   0    &   0    \\
                0    &   0    &   0
            \end{pmatrix}.
        \end{aligned}
    \end{equation}
    These matrices satisfy
    \begin{equation}        \label{EQooWJMUooOtFAkW}
        [O_i,O_j]=\sum_k\epsilon_{ijk}O_k.
    \end{equation}
    A basis of \( \su(2)\) is given by \( t_k=i\sigma_k\). Our isomorphism is
    \begin{equation}
        \begin{aligned}
            \varphi\colon \so(3)&\to \su(2) \\
            O_i&\mapsto -\frac{ i\sigma_k }{ 2 }. 
        \end{aligned}
    \end{equation}
    The fact that \( \varphi\) is a bijection derives from the fact that it maps a basis on a basis. We have to check that \( \varphi\) is a morphism, that is
    \begin{equation}
        \big[ \varphi(O_i),\varphi(O_j) \big]=\varphi\big( [O_i,O_j] \big).
    \end{equation}
    This is done by virtue of the commutators \eqref{EQooWJMUooOtFAkW} and of lemma \ref{LEMooJRWXooMkzRnk}.
\end{proof}

%--------------------------------------------------------------------------------------------------------------------------- 
\subsection{Irreducible representations of \texorpdfstring{$\gsl(2,\eC)$}{sl(2,C)}}
%---------------------------------------------------------------------------------------------------------------------------

We are not here to joke or to be funny. We are here to make quantum fields theory. So we need (among maaaaaany other things) the irreducible representations of the groups \( \SL(2,\eC)\) and \( \SU(2)\). Here is a good news: at the Lie algebra level, these two are more or less related by the lemma \ref{LEMooIGAFooTSUsJR}.

\begin{lemma}[\cite{BIBooUXTFooXTeMOn}]     \label{LEMooVEJZooUVNdmE}
    As sets,
    \begin{equation}
        \su(2)_{\eC}=\su(2)\otimes_{\eR}\eC=\gsl(2,\eC)=\{ \begin{pmatrix}
        \alpha    &   \beta    \\ 
    \gamma    &   -\alpha    
\end{pmatrix}\tq \alpha,\beta,\gamma\in \eC\}.
    \end{equation}
    The first equality is a definition for the notation \( \su(2)_{\eC}\).
\end{lemma}

\begin{proof}
    A basis of \( \su(2)\) is \( \{ t_1,t_2,t_3 \}\); so \( \su(2)_{\eC}=\eC t_1\oplus \eC t_2\oplus \eC t_3\), that is
    \begin{subequations}
        \begin{align}
            \su(2)_\eC&=\{ \begin{pmatrix}
            0    &   ix    \\ 
        ix    &   0    
    \end{pmatrix}+\begin{pmatrix}
    0    &   y    \\ 
-y    &   0    
\end{pmatrix}+\begin{pmatrix}
iz    &   0    \\ 
0    &   -iz    
\end{pmatrix}\tq x,y,z\in \eC\}\\
&=\{\begin{pmatrix}
iz    &   ix+y    \\ 
ix-y    &   -iz    
\end{pmatrix}\rq x,y,z\in \eC\}\\
&=\{\begin{pmatrix}
\alpha    &   \beta    \\ 
\gamma    &   -\alpha    
\end{pmatrix}\tq \alpha,\beta,\gamma\in \eC\}.
        \end{align}
    \end{subequations}
    In order to determine the Lie algebra \( \gsl(2,\eC)\) of \( \SL(2,\eC)\) we use the proposition \ref{PROPooSQHLooGQAykc} to allow ourself to work at the matrix level. Let \( g\) be a smooth path in \( \SL(2,\eC)\) such that \( g(0)=\mtu\). A generic element of \( \gsl(2,\eC)\) has the form \( g'(0)\). We have
    \begin{equation}
        g(t)=\begin{pmatrix}
            \alpha(t)    &   \beta(t)    \\ 
            \gamma(t)    &   \delta(t)    
        \end{pmatrix}
    \end{equation}
    with 
    \begin{equation}        \label{EQooMNXMooVkbfDg}
        \alpha(t)\delta(t)-\gamma(t)\beta(t)=1
    \end{equation}
    for every \( t\).  Moreover \( g(0)=\mtu\) implies \( \alpha(0)=1\), \( \beta(0)=0\), \( \gamma(0)=0\) and \( \delta(0)=1\). Now we differentiate \eqref{EQooMNXMooVkbfDg} with respect to \( t\) at \( t=0\) :
    \begin{equation}
        \alpha'(0)\delta(0)+\alpha(0)\delta'(0)-\gamma'(0)\beta(0)-\gamma(0)\beta'(0)=0
    \end{equation}
    which reduces to \( \alpha'(0)+\delta'(0)=0\). An element of \( \gsl(2,\eC)\) is thus of the form \( \begin{pmatrix}
        a    &   b    \\ 
        c    &   -a    
    \end{pmatrix}\) with \( a,b,c\in \eC\).
\end{proof}

\begin{normaltext}
    Lemma \ref{LEMooVEJZooUVNdmE} speaks about the \emph{sets} of \( \su(2)_{\eC}\) and \( \gsl(2,\eC)\). The Lie bracket, in both cases, is the matrix commutator. As a vector space, one can consider on \( \gsl(2,\eC)\) a vector space structure on \( \eC\) or on \( \eR\). If you want \( \{ t_1, t_2, t_3 \}\) to be a basis, you have to consider the complex linear combinations. If you really want real linear combinations, you need a larger basis.
\end{normaltext}

We consider the following basis for \( \gsl(2,\eC)\):
\begin{subequations}        \label{EQSooORIBooAsgdDp}
    \begin{align}
        h_3&=\begin{pmatrix}
            1/2    &   0    \\ 
            0    &   -1/2    
        \end{pmatrix}\\
        h_+&=\begin{pmatrix}
            0    &   \sqrt{ 2 }/2    \\ 
            0    &   0    
        \end{pmatrix}\\
        h_-&=\begin{pmatrix}
            0    &   0    \\ 
            \sqrt{ 2 }/2    &   0    
        \end{pmatrix}
    \end{align}
\end{subequations}
They satisfy the commutation relations
\begin{subequations}        \label{SUBEQSooXMMVooKtnRXW}
    \begin{align}
        [h_3,h_+]&=h_+\\
        [h_3,h_-]&=-h_-\\
        [h_+,h_-]&=h_3.
    \end{align}
\end{subequations}

\begin{lemma}[\cite{BIBooUXTFooXTeMOn}]     \label{LEMooDGUYooPUkDNr}
    Let \( (\rho, V)\) be a representation of \( \gsl(2,\eC)\). If \( V_{\lambda}\) is the eigenspace of the eigenvalue \( \lambda\in \eC\) for \( \rho(h_3)\), then
    \begin{subequations}
        \begin{align}
            \rho(h_+)V_{\lambda}&\subset V_{\lambda+1}\\
            \rho(h_-)V_{\lambda}&\subset V_{\lambda-1}
        \end{align}
    \end{subequations}
\end{lemma}

\begin{proof}
    Let \( w\in V_{\lambda}\). We test the eigenvalue of \( \rho(h_3)\) on \( \rho(H_+)w\):
    \begin{subequations}
        \begin{align}
            \phi(h_3)\rho(h_+)w&=\big( [\rho(h_3),\rho(h_+)]+\rho(h_+)\rho(h_3) \big)\\
            &=\rho(h_+)w+\lambda\rho(h_+)w\\
            &=(\lambda+1)\rho(h_+)w,
        \end{align}
    \end{subequations}
    so that \( \phi(h_+)w\in V_{\lambda+1}\).

    The computation is the same for the other one.
\end{proof}

\begin{lemma}           \label{LEMooWXDYooUyijnm}
    Let \( (V,\rho)\) be a finite dimensional representation of \( \gsl(2,\eC)\) over the complex vector space \( V\). There exists \( \lambda_0\in \eC\) such that \( V_{\lambda_0}\neq \{ 0 \}\) and \( \rho(h_+)V_{\lambda_0}=\{ 0 \}\).
\end{lemma}

\begin{proof}
    A vector \( w\in V\) belong to \( V_{\lambda}\) if \( \rho(h_3)w=\lambda w\), which meas that \( \big( \rho(h_3)-\lambda\id \big)w=0\). The equation \( \det\big( \rho(h_3)-\lambda\id \big)\) has (at least) one solution \( \lambda\in \eC\). So there exists \( \lambda\in \eC\) such that \( V_{\lambda}\neq \{ 0 \}\). 
    
    Let \( \lambda\) be such a number and a non vanishing vector \( w\in V_{\lambda}\).

    Now the sequence of elements \( w_k= \rho(h_+)w   \) satisfy \( w_k\in V_{\lambda+k}\). Since \( V\) is finite dimensional only a finite number of the \( V_{\lambda+k}\) are different to \( \{ 0 \}\). The space \( V_{\lambda}\) on the other hand contains \( w\neq 0\). Let \( k_0\) be the lowest natural such that \( V_{\lambda+k_0}=\{ 0 \}\). What we have is \( V_{\lambda+k_0-1}\neq \{ 0 \}\) and \( V_{\lambda_0+k_0}=\{ 0 \}\).

    The proposition is done with \( \lambda_0=\lambda+k_0\).
\end{proof}

\begin{proposition}[\cite{BIBooUXTFooXTeMOn}]      \label{PROPooZCAOooHHGxQk}
    Let \( (\rho, V)\) be a finite dimensional complex representation of \( \gsl(2,\eC)\). Let \( \lambda_0\) be such that \( V_{\lambda_0}\neq \{ 0 \}\) and \( \rho(h_+)V_{\lambda_0}=\{ 0 \}\). Let \( w_0\in V_{\lambda_0}\) and 
    \begin{equation}
        w_k=\rho(h_-)^kw_0.
    \end{equation}
    Then
    \begin{enumerate}
        \item       \label{ITEMooBPPFooKdGyqO}
            \( w_k\in V_{\lambda_0-k}\)
        \item       \label{ITEMooHNULooHoTgEa}
                    \( \rho(h_+)w_k=\frac{ 1 }{2}k(2\lambda_0+1-k)w_{k-1}\)
                \item       \label{ITEMooHDAPooClASpy}
                    There exists \( n\in \eN\) such that \( w_n\neq 0\) and \( w_{n+1}=0\).
                \item       \label{ITEMooJBZFooGqallS}
                    \( \lambda_0=n/2\).
    \end{enumerate}
\end{proposition}

\begin{proof}
    Notice that the existence of \( \lambda_0\) such that \( V_{\lambda_0}\neq \{ 0 \}\) and \( \rho(h_+)V_{\lambda_0}\neq\{ 0 \}\) is provided by lemma \ref{LEMooWXDYooUyijnm}.
    \begin{subproof}
    \item[For \ref{ITEMooBPPFooKdGyqO}]
        By recursion, using lemma \ref{LEMooDGUYooPUkDNr}.
    \item[For \ref{ITEMooHNULooHoTgEa}]
        We'll have a recursion. Just to be clear here are two facts that are not yet proved:
        \begin{itemize}
            \item The spaces \( V_{\lambda_0+k}\) are one-dimensional.
            \item \( \rho(h_+)w_{k+1}\) is a multiple of \( w_k\).
        \end{itemize}
        We will now prove by recursion that the first fact is true. The second one is, in general, false. We will see later that it is true when the representation is irreducible.

        Ok. So let's begin our work. For \( k=0\) we already have
        \begin{equation}
            \rho(h_+)w_0=0.
        \end{equation}
        Let work out the case of \( k=1\).
        \begin{subequations}
            \begin{align}
                \rho(h_+)w_1&=\rho(h_+)\rho(h_-)w_0\\
                &=\big( \underbrace{[\rho(h_+),\rho(h_-)]}_{=\rho(h_3)}+\rho(h_-)\rho(h_+) \big)w_0\\
                &=\rho(h_3)w_0\\
                &=\lambda_0w_0.
            \end{align}
        \end{subequations}
        
        For the recursion, suppose that \( \rho(h_+)w_k=f(k)w_{k-1}\) for some function \( f\colon \eN\to \eC\). Then we compute \( \rho(h_+)w_{k+1}\):
        \begin{subequations}
            \begin{align}
                \rho(h_+)w_{k+1}=\rho(h_+)\rho(h_-)w_k&=\big( \underbrace{[\rho(h_+),\rho(h_-)]}_{=\rho(h_3)}+\rho(h_-)\rho(h_+) \big)w_k\\
                &=\lambda_0w_k+f(k)\rho(h_-)w_{k-1}\\
                &=\big( \lambda_0+f(k) \big)w_k.
            \end{align}
        \end{subequations}
        This shows that \( \rho(h_+)w_{k+1}\) is a multiple of \( w_k\) and that the proportionality factor \( f(k)\) satisfy
        \begin{subequations}        \label{SUBEQSooHGQNooRjMCap}
            \begin{numcases}{}
                f(k+1)=f(k)+\lambda_0-k\\
                f(1)=\lambda_0.
            \end{numcases}
        \end{subequations}
        The function \( f\) is defined by recursion and you see that at each step \( k\), we substrat \( k\) and add \( \lambda_0\). The guess is
        \begin{equation}
            f(k)=k\lambda_0-\frac{ k(k-1) }{ 2 }.
        \end{equation}
        Check that this satisfy \eqref{SUBEQSooHGQNooRjMCap}.
        
    \item[For \ref{ITEMooHDAPooClASpy}]
        The sequence of elements \( w_k\in V_{\lambda_0-k}\) has to finish on \( 0\) because the space \( V\) is finite dimensional.
    \item[For \ref{ITEMooJBZFooGqallS}]
        Let \( n\in \eN\) such that \( w_n\neq 0\) and \( w_{n+1}=0\). This means \( f(k+1)=0\). Solving
        \begin{equation}
            (n+1)\big( \lambda_0+\frac{ 1-(n+1) }{ 2 } \big)=0
        \end{equation}
        we get \( \lambda_0=n/2\).
    \end{subproof}
\end{proof}

This is quite an achievement because we proved not only that \( \rho(h_3)\) has a real eigenvalue, but that it has an eigenvalue in \( \eN/2\).

\begin{proposition}     \label{PROPooDAIQooPZVjju}
    Let \( \lambda_0\) and \( w_0\) be as before. We suppose that the representation is irreducible. Then
    \begin{equation}
        V=\Span\{  \rho(h_-)^kw_0 \}_{k=0,\ldots, 2\lambda_0}
    \end{equation}

    The eigenspaces of \( \rho(h_3)\) are one-dimensional.
\end{proposition}

\begin{proof}
    Let \(  W=\Span\{  \rho(h_-)^kw_0 \}_{k=0,\ldots, 2\lambda_0}\). This space is invariant under \( \rho\) because of the definitions and the proposition \ref{PROPooZCAOooHHGxQk}:
    \begin{equation}
        \begin{aligned}[]
            \rho(h_3)w_k&=(\lambda_0-k)w_k\\
            \rho(h_+)w_k&=\frac{ 1 }{2}k(2\lambda_0+1-k)w_{k-1}\\
            \rho(h_-)w_k&=w_{k+1}
        \end{aligned}
    \end{equation}
    with the convention that \( w_{k+1}\) could be \( 0\).   

    Since \( W\) is a non trivial invariant subspace, it has to be \( V\). So \( W=V\).
\end{proof}

\begin{normaltext}
In the proposition \ref{PROPooZCAOooHHGxQk} and \ref{PROPooDAIQooPZVjju}, the vectors \( w_k\) are more or less enumerated in the reverse order: the larger \( k\) is, the lower is the eigenvalue. That leads to missleading formula like \( \rho(h_-)v_k=v_{K+1}\). In the following theorem, we make it in the correct order and one has to think \( v_m\) as being \( w_{\lambda_0-m}\).

Notice that up to now, the results we have collected are «if a representation of \( \gsl(2,\eC)\) exists». The next theorem \ref{THOooSRQYooXQDZpT} will show that a representation exists.
\end{normaltext}

Here is the theorem which provides every irredicible finite-dimensional representations of the Lie algebra \( \gsl(2,\eC)=\su(2)_{\eC}\).
\begin{theorem}     \label{THOooSRQYooXQDZpT}
    Let \( j\in \eN/2\). Let \( V_j\) be a complex vector space of dimension \( 2j+1\); we label a basis of \( V_j\) in the following way: \( \{ v_m \}_{m=j,j-1,\ldots, -j}\).

    We define the map \( \rho_j\colon \gsl(2,\eC)\to \End(V_j)\) by
    \begin{subequations}
        \begin{align}
            \rho_j(h_3)v_m&=mv_m,\\
            \rho_j(h_+)v_m&=\begin{cases}
                0    &   \text{if } m=j\\
                \frac{ 1 }{2}(j-m)(j+m+1)v_{m+1}    &    \text{otherwise },
            \end{cases}\\
            \rho_j(h_-)v_m&=\begin{cases}
                v_{m-1}    &   \text{if } m\neq -j\\
                0    &    \text{if } m=-j.
            \end{cases}
        \end{align}
    \end{subequations}
    Two statements.
    \begin{enumerate}
        \item
            The map \( \rho_j\) is a representation of \( \gsl(2,\eC)\).
        \item
            Every finite dimensional complex irreducible representation is isomorphic to \( \rho_j\) for some \( j\in \eN/2\).
    \end{enumerate}
\end{theorem}

\begin{proof}
    For the first item we have to check the algebra of \( \gsu(2)\) given by \eqref{SUBEQSooXMMVooKtnRXW}. There are three computations.

    We begin to check \( [\rho_j(h_3), \rho_j(h_+)]=\rho_j(h_+)\). The bracket in the left-hand side is the commutator of operators in \( \End(V)\). We have:
    \begin{subequations}
        \begin{align}
            \big( \rho_j(h_3)\rho_j(h_+)-\rho(h_+)\rho(h_3) \big)v_m&=\rho_j(h_3)\frac{ 1 }{2}(j-m)(j+m+1)v_{m+1}-m\rho_j(h_+)v_m\\
            &=(m+1)\frac{ 1 }{2}(j-m)(j+m+1)v_{m+1}\\
                &\quad-m\frac{ 1 }{2}(j-m)(j+m+1)v_{m+1}\\
            &=\frac{ 1 }{2}(j-m)(j+m+1)v_{m+1}\\
            &=\rho_j(h_+)v_m.
        \end{align}
    \end{subequations}
    The two other ones are checks with the same kind of computations.

    For the second item, we consider an irreducible finite-dimensional representations \( (\rho,V)\) of \( \gsl(2,\eC)\). Combining the propositions  \ref{PROPooZCAOooHHGxQk} and \ref{PROPooDAIQooPZVjju} we have:
    \begin{itemize}
        \item A vector \( w_0\in V\) such that \( \rho(h_+)w_0=0\).
        \item Letting \( n=2\lambda_0\) we have \( n\in \eN\),
        \item we let \( j=\lambda_0\) (pure notational purpose),
        \item \( V=\Span\{ w_k=\rho(h_-)^kw_0 \}\),
        \item  \( w_k\in V_{\lambda_0-k}\) where \( V_{\lambda}\) is the eigenspace of \( \rho(h_3)\) for the eigenvalue \( \lambda\),
        \item \( w_k\neq 0\) if and only if \( k=0,\ldots, n\), so \( \dim(V)=2n+1\)
        \item \( \rho(h_3)w_k=(\lambda_0-k)w_k\)
        \item \( \rho(h_+)w_k=\frac{ 1 }{2}k(2\lambda_0+1-k)w_{k-1}\)
    \end{itemize}
    We choose \( j=\lambda_0\in \eN/2\). Do you believe that the map
    \begin{equation}
        \begin{aligned}
            \phi\colon V &\to V_j \\
            w_k&\mapsto v_{j-k} 
        \end{aligned}
    \end{equation}
    provides an equivalence of representations between \( \rho_j\) and \( \rho\) ? No ? Ok. We check that for every \( X\in \gsl(2,\eC)\) we have
    \begin{equation}
        \phi\circ\rho(X)=\rho_j(X)\circ \phi.
    \end{equation}
    For \( X=h_3\) we have
    \begin{equation}
        \phi\circ\rho(h_3)w_k=\phi\big( (\lambda_0-k)w_k \big)=(\lambda_0-k)v_{j-k}=(j-k)v_{j-k}
    \end{equation}
    while
    \begin{equation}
        \rho_j(h_3)\phi(w_k)=\rho_j(h_3)v_{j-k}=(j-k)v_{j-k}.
    \end{equation}
    Ok for the first one. Next: \( X=h_+\). We have
    \begin{equation}
        \phi\circ\rho(h_+)w_k=\phi\big( \frac{ 1 }{2}k(2j+1-k)w_{k-1} \big)=\frac{ 1 }{2}k(2j+1-k)v_{j-k+1}
    \end{equation}
    while
    \begin{equation}
        \rho_j\phi(w_k)=\rho_j(h_+)v_{j-k}=\frac{ 1 }{2}\big( j-(j-k) \big)\big( j+(j-k)+1 \big)v_{j-k+1}=\frac{ 1 }{2}k(2j-k+1)v_{j-k+1}.
    \end{equation}
    Ok again. And last one: \( X=h_-\); we have
    \begin{equation}
        \phi\circ\rho(h_-)w_k=\phi(w_{k+1})=v_{j-k-1}
    \end{equation}
    while
    \begin{equation}
        \rho_j(h_-)\phi(w_k)=\rho_j(h_-)v_{j-k}=v_{j-k-1}.
    \end{equation}
    Done\footnote{Now that we reached the end, I recognize that I did not belive neither until the last check.}.
\end{proof}

\begin{normaltext}
    The representations \( \rho_j\) of the theorem \ref{THOooSRQYooXQDZpT} are not yet hermitian for two reasons.
    \begin{itemize}
        \item The representations we expect to be hermitian are the ones of \( \su(2)\). The basis \( \{ h_2,h_{+}, h_{-} \}\) of \( \su(2)_{\eC}\) defined by \eqref{EQSooORIBooAsgdDp} is made of elements which do not belong to \( \su(2)\).
        \item We did not defined a scalar product on \( V_j\); thus the notion of «hermitian» makes no sene.
    \end{itemize}
\end{normaltext}

%--------------------------------------------------------------------------------------------------------------------------- 
\subsection{Representations of \texorpdfstring{$ \su(2)$}{su(2)}}
%---------------------------------------------------------------------------------------------------------------------------

We know every representations of \( \su(2)_{\eC}\) by the theorem \ref{THOooSRQYooXQDZpT}. Let \( \rho\colon \su(2)\to \End(V)\) be an irreducible representation of \( \su(2)\). By lemma \ref{LEMooIGAFooTSUsJR}, there exists an irreducible representation \( \rho'\colon \su(2)_{\eC}\to \End(V)\) such that \( \rho=\rho'|_{\su(2)}\).

Thus there exists a \( j\) such that \( \rho(X)=\rho_j(X)\).

%///////////////////////////////////////////////////////////////////////////////////////////////////////////////////////////
					\subsection{Haar measure on \texorpdfstring{$\SU(2)$}{SU2}}
%///////////////////////////////////////////////////////////////////////////////////////////////////////////////////////////

The quaternion\index{quaternion} field $\eH$ can be embed in $\eM_2(\eC)$ as a genera element reads
\begin{equation}
	q=
\begin{pmatrix}
  \alpha	&	\beta	\\
  -\bar\beta	&	\bar\alpha
\end{pmatrix}
\end{equation}
with $\alpha$, $\beta\in\eC$. Under that isomorphism, we have
\[
	| q |^2=| \alpha |^2+| \beta |^2=\det q.
\]
Thus we have the identification
\begin{equation}
	\SU(2)=\{ q\in\eH\tq | q |=1 \}.
\end{equation}
We can act on $\eH$ by $\SU(2)\times \SU(2)$ by
\begin{equation}
	(u,v)\cdot q=uqv^{-1}
\end{equation}
for every $(u,v)\in \SU(2)\times\SU(2)$ and $q\in\eH$. That action defines an homomorphism from $\SU(2)\times\SU(2)$ onto $O(4)$.

\begin{proposition}
The previously defined homomorphism
\[
	\phi\colon \SU(2)\times\SU(2)\to O(4).
\]
is surjective over $\SO(4)$ (which is the identity component of $O(4)$) and, moreover, the kernel is $\big\{  (e,e),(-e,-e) \big\}$.
\end{proposition}

\begin{proof}
The group $\SU(2)\times \SU(2)$ being connected, its image can only be included in $\SO(4)$. Let us first determine the kernel of $\phi$. If $(u,v)\in\ker\phi$, we have $uqv^{-1}=q$ for every $q\in\eH$. In particular, with $q=1$, we find $u=v$. Then the relation $uqu^{-1}=q$ means that $u$ belongs to the center of $\eH$, which is $\eR$. We conclude that $u=\pm 1$. That proves that $\ker\phi=\big\{  (e,e),(-e,-e) \big\}$.

The differential $(d\phi)_{(e,e)}$ is an homomorphism
\[
	d\phi\colon \gsu(2)\oplus\gsu(2)\to \so(4).
\]
Let $(S,T)\in\gsu(2)\oplus\gsu(2)$, we have
\[
	d\phi(S,T)q=\Dsdd{ \phi( e^{t(S,T)})q }{t}{0}=\Dsdd{ \phi( e^{tS}, e^{tT})q }{t}{0}=\Dsdd{  e^{tS}q e^{-tT} }{t}{0}=Sq-qT,
\]
on which one sees that $d\phi$ is injective. Moreover we have $\dim\big( \gsu(2)\oplus\gsu(2) \big)=6=\dim\so(4)$. An injective map between vector space of same dimension being an isomorphism, the image of $\phi$ contains a neighborhood of identity in $\SO(4)$. From connectedness of $\SO(4)$, that neighborhood generates the whole group (see proposition~\ref{PropUssGpGenere}), so that $\phi$ is in fact surjective.
\end{proof}

Since the map $\phi\colon \SU(2)\times \SU(2)\to \SO(4)$ is a surjective homomorphism with a discrete kernel, we have an isomorphism at the algebra level:
\[
	\so(4)\simeq \gsu(2)\oplus\gsu(2).
\]

\subsection{Building some representations for \texorpdfstring{$\SU(2)$}{SU2}}
%/////////////////////////////////////////////////////////////////////////////////

Since $\SU(2)$ acts on $\eC^2$, we can build a representation of $\SU(2)$ on functions on $\eC^2$. We define $\dpt{T}{SU(2)}{\End\big(\Cinf(\eC^2)\big)}$ by
\[
  (T(U)f)(\xi)=f(U^{-1}\xi),
\]
if $f\in\Cinf\big(\eC^2\big)$, $\xi\in\eC^2$ and $U\in SU(2)$.

Let $V_j$ be the space of the homogeneous polynomials of degree $j$ on $\eC^2$; a basis of this space is given by the $\phi_{pq}$, $p+q=2j$ defined by
\begin{equation}
   \phi_{pq}(\xi)=\xi_1^p\xi_2^q
\end{equation}
($\xi=\xi_1+i\xi_2$). If $j$ is fixed, we will often write $\phi_m$ instead of $\phi_{pq}$. The signification is $p=j+m$, $q=j-m$, and $m$ takes its values in $-j,\ldots,j$. Note that $p-q=2m$. It is clear that if $A$ is any invertible $2\times 2$ matrix , and $f\in V_j$, then
\[
   \rho(A)f:=f(A^{-1} \cdot)
\]
 is still an element of $V_j$. This representation $\rho$ is defined on the whole $\Cinf(\eC^2)$. We will descent it to $V_j$ later.
Now, we fix $j$ and a $m$ between $-j$ and $j$.

Consider the diagonal matrix
\[   U_{-\theta}=\begin{pmatrix}
e^{-i\theta} & 0 \\
0 & e^{i\theta}
\end{pmatrix} \in\SU(2).
\]
One has
\begin{equation}
  \left(\rho(U_{-\theta})\phi_{pq}\right)(\xi)=\phi_{pq}
\begin{pmatrix}
   e^{i\theta}\xi_1 \\
    e^{-i\theta}\xi_2
\end{pmatrix}
                                               = e^{pi\theta} e^{-qi\theta}\xi_1^p\xi_2^q
					       =e^{2mi\theta}\phi_{pq}(\xi).
\end{equation}
First conclusion: the $\phi$'s are eigenvectors of $\rho(U_{-\theta})$ because
\[
   \rho(U_{-\theta})\phi_m=e^{2mi\theta}\phi_m.
\]
Second, the trace of $\rho(U_{-\theta})$ is
\begin{equation}
   \chi_j(\theta)=\sum_{m=-s}^{s}e^{2mi\theta}.
\end{equation}
By the way, the $\chi_j$ are the characters of the representation $\rho$.

From considerations about the Haar\quextproj{} invariant measure on $\SU(2)$, one knows that the good notion product between functions is:
\begin{equation}
(f_1,f_2)_{\SU(2)}=\frac{2}{\pi}\int_0^{\pi}f_1(\theta)\overline{ f_2(\theta) }\sin^2\theta\,d\theta,
\end{equation}
so that $(\chi_j,\chi_j)=1$. This and the fact that $\SU(2)$ is compact make the theorem of Peter-Weyl (cf. \cite{Sternberg}) applicable, thus the restrictions of $\rho$ to the $V_j$'s are irreducible and moreover, these provide \emph{all} the irreducible representations.

\subsection{Special case: \texorpdfstring{$j=\frac{1}{2}$}{j=1/2}}
%//////////////////////////////////////////////////////////////////////

Consider a matrix $A\in\SU(2)$:
\begin{equation}
A=\begin{pmatrix}
\oalpha & -\beta \\
\obeta & \alpha
\end{pmatrix},\qquad
A^{-1}=\begin{pmatrix}
\alpha & \beta \\
-\obeta & \oalpha
\end{pmatrix}.
\end{equation}
A basis of $V_{\frac{1}{2}}$ is given by $\phi_{10}$ and $\phi_{01}$. Let us see how $\rho(A)$ acts on. Since
\[
A^{-1}\begin{pmatrix}
\xi_1 \\
\xi_2
\end{pmatrix}=
\begin{pmatrix}
\alpha\xi_1+\beta\xi_2 \\
-\obeta\xi_1+\oalpha\xi_2
\end{pmatrix},
\]
we find
\begin{equation}
\begin{split}
  (\rho(A)\phi_{10})(\xi)&=\alpha\xi_1+\beta\xi_2=(\alpha\phi_{10}+\beta\phi_{01})(\xi)\\
  (\rho(A)\phi_{01})(\xi)&=-\obeta\xi_1+\oalpha\xi_2=(-\obeta\phi_{10}+\oalpha\phi_{01})(\xi).
\end{split}
\end{equation}
Thus in the basis $\{\phi_{10},\phi_{01}\}$, the matrix of $\rho(A)$ is given by
\begin{equation}
\rho(A)=\begin{pmatrix}
\alpha & -\obeta \\
\beta & \oalpha
\end{pmatrix}=\overline{A}.
\end{equation}

Up to here, we were looking at the representation $\rho$ of $\SU(2)$ on the whole set of functions on $\eC^2$, and more precisely, its restriction to $V_j$. We could define the representation $\rho_{\frac{1}{2}}$ as $\rho_{\frac{1}{2}}=\rho|_{V_{\frac{1}{2}}}$, but we will not do it. Our definition is
\begin{equation}
  \rho_{\frac{1}{2}}(A)=\rho(\overline{A})|_{V_{\frac{1}{2}}}.
\end{equation}
Note that
\[
\begin{pmatrix}
0 & -1 \\
1 & 0
\end{pmatrix}
A
\begin{pmatrix}
0 & 1 \\
-1 & 0
\end{pmatrix}=\overline{A},
\]
thus the representation $A\to\rho(\overline{A})|_{V_{\frac{1}{2}}}$ is equivalent to $A\to\rho(A)|_{V_{\frac{1}{2}}}$. This equivalence can also be seen because these two representations have the same characters\quextproj.

The basis $\phi_{pq}$ is orthogonal; we will build an orthonormal one: $e_m(\xi)$ is the vector whose coordinates are
\begin{equation}
e_m^j(\xi)=\frac{ \xi_1^{j+m}\xi_2^{j-m} }{\sqrt{ (j+m)!(j-m)! }}
\end{equation}
for $m=-j,-j+1,\ldots,j$. The metric to take in order to define $(e_m,e_n)$ is the unique one on $V_j$ which is $\SU(2)$-invariant.

The Newton's formula for the binomial yields: 
\begin{equation}
\begin{split}
  \sum_{m=-s}^s e_m(\xi)\overline{e_m(\eta)}
        &=\sum\frac{ \xi_1^{j+m}\xi_2^{j-m}\oeta_1^{j+m}\oeta_2^{j-m} }{(j+m)!(j-m)!}\\
	&=\us{(2j)!}(\xi_1\oeta_1+\xi_2\oeta_2)^2\\
	&=\us{(2j)!}\scal{\xi}{\eta}^{2j}.
\end{split}
\end{equation}
But we know that $A\in\SU(2)$ preserves the scalar product: $\scal{A\xi}{A\eta}=\scal{\xi}{\eta}$. Therefore:
\begin{equation}\label{eq:produit_e_m}
\sum (\rho_j(A)e_m)(\xi)\overline{ (\rho_j(A)e_m)(\eta) }=\sum e_m(\xi)\overline{e_m(\eta)}.
\end{equation}
Now, instead of considering the matrices $\rho_j(A)$ on $V_j$ for the basis $\phi_m$, we looks at the ones with respect to the basis $e_m$:
\begin{equation}
\rho_j(A)e_m=r(A)^k_me_k;
\end{equation}
in others words, we looks at the representation $A\to r(A)$. The equations \eqref{eq:produit_e_m} makes
\[
  \sum_{m=-j}^j\left(
                      r(A)^l_me_l(\xi)\overline{ r(A)^k_me_k(\eta)   }
		        -\delta^l_me_l(\xi)\delta^k_me_k(\eta)
                \right)=0.
\]
Since the functions
\begin{equation}
\begin{aligned}
 e_k\otimes\overline{e_l}\colon \eC^2\times\eC^2 &\to \eC \\
(\xi,\eta) &\mapsto  e_k(\xi)\overline{e_l(\eta)}
\end{aligned}
\end{equation}
 are linearly independent, one gets $\sum_m r(A)^k_l\overline{r(A)^l_m}=\delta^{kl}$, or
\begin{equation}
r(A)r(A)^*=\mtu,
\end{equation}
the conclusion is that in this basis, the matrices $\rho_j(A)$ are unitary.

\subsection{Clebsch-Gordan}
%//////////////////////////////////////////////////////////////////////

From the knowledge of the characters of $\rho_j$, one can decompose the product $\rho_s\otimes\rho_r$ into irreducible representations. For example,
\[
   V_{\frac{1}{2}}\otimes V_{\frac{1}{2}}=V_0\oplus V_1.
\]
More generally,
\begin{equation}
  V_s\otimes V_r=V_{|r-s|} \oplus V_{|r-s|+1}\oplus\ldots\oplus V_{r+s}.
\end{equation}
For this reason, the representation $\rho_j$ is sometimes called the \defe{spin $j$}{spin!representation!of $\SU(2)$} representation of $\SU(2)$.

%--------------------------------------------------------------------------------------------------------------------------- 
\subsection{Representations of \texorpdfstring{$\SO(3)$}{SO3}}
%---------------------------------------------------------------------------------------------------------------------------

The group $\SO(3)$ is strongly linked with $SU(2)$ by the following property:
\begin{equation}
   \SO(3)=\frac{SU(2)}{\eZ_2}.
\end{equation}
proved in proposition \ref{PROPooDKPTooBnLflt}.

\begin{lemma}\label{lem:SO_3}
    A representation $\rho_j$ of $SU(2)$ is a representation of $\SO(3)$ if and only if $\rho_j(X)=\id$ for any $X$ in the kernel of the homomorphism $SU(2)\to \SO(3)$, namely: $\rho_j(\pm\mtu)=\id$.
\end{lemma}

\begin{proof}
    We consider $\dpt{\rho_j}{SU(2)}{\End{V_j}}$. By proposition \ref{PROPooDKPTooBnLflt} we have \( SO(3)=\SU(2)/\eZ_2\) and there exists a group homomorphism\footnote{Defined and studied in proposition \ref{PROPooGEHAooPCReoU}.} $\dpt{\psi}{SU(2)}{\SO(3)}$ such that $\psi(\mtu)=\psi(-\mtu)=\mtu$, which is an important equation because it ensures us that the rest of the expressions are well defined with respect to the class representative.

    If $\rho_j(-\mtu)=\mtu$, we define $\dpt{d_j}{\SO(3)}{\End{V}}$ by $d_j([x])=\rho_j(x)$ (check that this is well defined). With this,
    \[
      d_j([x])d_j([y])=\rho_j(x)\rho_j(y)=\rho_j(xy)=d_j([xy]).
    \]

    Now let us suppose that $d_j([x])=\rho_j(x)$ is a representation. Thus
    \[
      \rho_j(x)=d_j([x])=d_j([-x])=\rho_j(-x)=\rho_j(-\mtu)\rho_j(x),
    \]
    so $\rho_j(-\mtu)=\id_{V_j}$.
\end{proof}

Moreover, any representation of $\SO(3)$ comes from a representation $\tilde\rho$ of $SU(2)$ by setting $\tilde\rho(-\mtu)=\id$ and $\tilde\rho(x)=\rho([x])$.

Now, we research the representations of $SU(2)$ for which the matrix $-\mtu$ is represented by the identity operator. These will be representations of $\SO(3)$. The spin $j$ representations of $SU(2)$ is given by
\begin{equation}
   \rho_j(X)\phi_{pq}(\xi)=\phi_{pq}(X^{-1}\xi).
\end{equation}
With $X=-\mtu$, this gives: $\phi_{pq}(-\xi)=(-1)^{p+q}\phi_{pq}(\xi)$. If we want it to be equal to $\phi_{pq}(\xi)$, we need $p+q=2j$ even. This is true if and only if $j\in\eN$.

\begin{normaltext}      \label{NORMooHWAYooPlSDOp}
    The conclusion is that the irreducible representations of $\SO(3)$ are the integer spin irreducible representations of $\SU(2)$. Note that the non relativistic mechanics has $\SO(3)$ as group of space symmetry. Thus there are no hope to find any half integer spin in a non relativistic theory.
\end{normaltext}

%+++++++++++++++++++++++++++++++++++++++++++++++++++++++++++++++++++++++++++++++++++++++++++++++++++++++++++++++++++++++++++ 
\section{Lorentz group}
%+++++++++++++++++++++++++++++++++++++++++++++++++++++++++++++++++++++++++++++++++++++++++++++++++++++++++++++++++++++++++++

\begin{definition}
    We consider the vector space \( \eR^4\) with its usual scalar product \( \langle ., .\rangle \) which is positive defined. Using the matrix
    \begin{equation}
        \eta=\begin{pmatrix}
             1   &       &       &       \\
                &   -1    &       &       \\
                &       &   -1    &       \\ 
                &       &       &   -1     
         \end{pmatrix},
    \end{equation}
    we introduce the \defe{Minkowskian product}{Minkowsky product}
    \begin{equation}    \label{EQooQAXNooXhGUQV}
        x\cdot y=\langle \eta x, y\rangle .
    \end{equation}
    This product is not positive defined and is often called «pseudo-scalar product».
\end{definition}

\begin{lemma}       \label{LEMooICEYooNcjJjD}
    A map \( \Lambda\colon \eR^4\to \eR^4\) such that
    \begin{equation}
        \Lambda x\cdot \Lambda y=x\cdot y
    \end{equation}
    for every \( x,y\in  \eR^4\) is linear.
\end{lemma}

\begin{proof}
    The bilinear form
    \begin{equation}
        \begin{aligned}
            b\colon \eR^4\times \eR^4&\to \eR \\
            x,y&\mapsto x\cdot y 
        \end{aligned}
    \end{equation}
    is non degenerated. An element of \( \gO(3,1)\) satisfy \( b(\Lambda x,\Lambda y)=b(x,y)\). Thus the theorem \ref{ThoDsFErq} says that \( \Lambda\) must be linear.
\end{proof}

\begin{lemmaDef}
    The set of maps \( \Lambda\colon \eR^4\to \eR^4\) such that
    \begin{equation}        \label{EQooLPXWooNgrAXz}
        \Lambda x\cdot \Lambda y=x\cdot y
    \end{equation}
    for every \( x,y\in  \eR^4\) is a group.

    This group is named the \defe{Lorentz group}{Lorentz group} and is denoted by \( \gO(3,1)\) or \( L\).
\end{lemmaDef}

\begin{proof}
    From lemma \ref{LEMooICEYooNcjJjD} we know that the elements of \( \gO(3,1)\) are linear operators.
    \begin{enumerate}
        \item
            The identify is part of \( \gO(3,1)\).
        \item
            The product is associative.
        \item
            The only tricky part is to prove that if \( \Lambda\in \gO(3,1)\), then \( \Lambda\) is invertible and \( \Lambda^{-1}\in \gO(3,1)\).
    \end{enumerate}
    Let \( z\neq 0\in \eR^4\) being such that \( \Lambda z=0\), then there exists \( y\in \eR^4\) such that \( z\cdot y\neq 0\) while obviously \( \Lambda z\cdot \Lambda y=0\). Thus every element in \(  \gO(3,1)\) is invertible.

    Let \( \Lambda\in \gO(3,1)\), \( x,y\in \eR^4\). Using the fact that \( \Lambda\in \gO(3,1)\) we have
    \begin{equation}
        x\cdot y= \Lambda(\Lambda^{-1} x)\cdot\Lambda(\Lambda^{-1}y)=\Lambda^{-1}x\cdot \Lambda^{-1}y,
    \end{equation}
    so that \( \Lambda^{-1}\in \gO(3,1)\).
\end{proof}

%--------------------------------------------------------------------------------------------------------------------------- 
\subsection{Adjoint map}
%---------------------------------------------------------------------------------------------------------------------------

We have a notational issue here. We already defined the adjoint map \( A^*\) by
\begin{equation}
    \langle Ax, y\rangle =\langle x, A^*y\rangle 
\end{equation}
for the usual scalar product. Since the main product we consider now in \( \eR^4\), is the Minkowskian one, we will define \( \Lambda^*\) by \( \Lambda x\cdot y=x\cdot \Lambda^*y\) and leave the notation \( A^t\) for the adjoint with respect to the usual scalar product.

\begin{propositionDef}
    Let \( A\in \aL(\eR^4)\). There exists a unique operator \( B\in \aL(\eR^4)\) such that
    \begin{equation}
        Ax\cdot y=x\cdot By
    \end{equation}
    for every \( x,y\in\eR^4\).

    This operator is called \defe{adjoint}{adjoint in Minkowsky space}, is written \( A^*\) and is given by
    \begin{equation}        \label{EQooPFPGooXiGcXs}
        A^*=\eta A^t\eta.
    \end{equation}
\end{propositionDef}

\begin{proof}
    For the existence, we just have to check that \( B=\eta A^t\eta\) works. Using the fact that \( \eta^t=\eta\) and \( \eta^2=\mtu\),
    \begin{subequations}
        \begin{align}
            x\cdot \eta A^t\eta y&=\langle \eta x, \eta A^t\eta y\rangle \\
            &=\langle x, A^t\eta y\rangle \\
            &=\langle Ax, \eta y\rangle \\
            &=\langle \eta Ax, y\rangle \\
            &=Ax\cdot y.
        \end{align}
    \end{subequations}
    For the unicity, we suppose $Ax\cdot y=x\cdot By$ for every \( x,y\in \eR^4\). We have
    \begin{equation}
        x\cdot By=Ax\cdot y=x\cdot \eta A^t\eta y
    \end{equation}
    Since the product is non degenerate, this implies \( B=\eta A^t\eta\).
\end{proof}

\begin{lemma}       \label{LEMooVRWJooPsDRwU}
    The adjoint operator satisfy \( (\Lambda^*)^*=\Lambda\).
\end{lemma}

\begin{proof}
    We use the formula \eqref{EQooPFPGooXiGcXs}: $(\Lambda^*)^*=\eta (\Lambda^*)^t\eta=\eta(\eta \Lambda^t\eta)^t\eta=\eta\eta^t\Lambda\eta^t\eta=\Lambda$.
\end{proof}

\begin{lemma}[\cite{MonCerveau}]       \label{LEMooDLWDooWCXlWq}
    If \( \Lambda\in \aL(\eR^4,\eR^4)\), the following are equivalent:
            \begin{enumerate}
                \item \( \Lambda\in\gO(3,1)\),      \label{ITEMooWHGKooFPfujT}  
                \item \( \Lambda^*\in \gO(3,1)\),   \label{ITEMooNISDooMajEMS}
                \item \( \Lambda^*\Lambda=\mtu\),       \label{ITEMooNLZGooUINRiP}
                \item \( \Lambda\Lambda^*=\mtu\),       \label{ITEMooFFRVooOwLmnz}
                \item \( \Lambda^t\eta\Lambda=\eta\),       \label{ITEMooOYTDooCWImBJ}
                \item \( \Lambda\eta\Lambda^t=\eta\).       \label{ITEMooAEEYooDiJuEi}
            \end{enumerate}
\end{lemma}

\begin{proof}
    We prove the equivalences.
    \begin{subproof}
        \item[\ref{ITEMooWHGKooFPfujT} implies \ref{ITEMooNLZGooUINRiP}]
            Since \( \Lambda\in \gO(3,1)\) we have $x\cdot y=\Lambda x\cdot \Lambda y=x\cdot \Lambda^*\Lambda y$ for every \( x,y\in \eR^4\). This implies \( \Lambda^*\Lambda=\mtu\).
        \item[\ref{ITEMooNLZGooUINRiP} implies \ref{ITEMooWHGKooFPfujT}]
            Since \( \Lambda^*\Lambda=\mtu\) and the fact that \( (\Lambda^*)^*\) (lemma \ref{LEMooVRWJooPsDRwU}) we have
            \begin{equation}
                x\cdot y=\Lambda^*\Lambda x\cdot y=\Lambda x\cdot \Lambda y.
            \end{equation}
            This shows that \( \Lambda\in \gO(3,1)\).
        \item[\ref{ITEMooNLZGooUINRiP} if and only if \ref{ITEMooFFRVooOwLmnz}]
            Le corolaire \ref{CORooNFJLooJtzFwN} nous dit que \( AB=\mtu\) if and only if \( BA=\mtu\).
        \item[\ref{ITEMooNISDooMajEMS} if and only if \ref{ITEMooFFRVooOwLmnz}]
            Same proof as \ref{ITEMooWHGKooFPfujT} if and only if \ref{ITEMooNLZGooUINRiP}.
    \end{subproof}
    At this point, we have the equivalence between \ref{ITEMooWHGKooFPfujT}, \ref{ITEMooNISDooMajEMS}, \ref{ITEMooNLZGooUINRiP} and \ref{ITEMooFFRVooOwLmnz}.
    \begin{subproof}
        \item[\ref{ITEMooNLZGooUINRiP} implies \ref{ITEMooOYTDooCWImBJ}]
            We plug the expression \eqref{EQooPFPGooXiGcXs} of the adjoint in the equation \( \Lambda^*\Lambda=\mtu\):
            \begin{equation}
                \mtu=\Lambda^*\Lambda=\eta\Lambda^t\eta\Lambda.
            \end{equation}
            Multiplying by \( \eta\) on the left, we get the result.
        \item[\ref{ITEMooOYTDooCWImBJ} implies \ref{ITEMooNLZGooUINRiP}]
            We write \( \Lambda^t\eta\Lambda=\eta\) and we multiply by \( \eta\).
        \item[\ref{ITEMooFFRVooOwLmnz} if and only if \ref{ITEMooAEEYooDiJuEi}]
            It is the same as \ref{ITEMooOYTDooCWImBJ} if and only if \ref{ITEMooNLZGooUINRiP}.
    \end{subproof}
\end{proof}

%--------------------------------------------------------------------------------------------------------------------------- 
\subsection{Structure}
%---------------------------------------------------------------------------------------------------------------------------

\begin{lemma}
    Elements \( \Lambda\in\gO(3,1)\) satisfy \( \det(\Lambda)=\pm1\).
\end{lemma}

\begin{proof}
    Taking the determinant on both sides of \( \Lambda^*\eta\Lambda=\eta\) (lemma \ref{LEMooDLWDooWCXlWq}), we get \( \det(\Lambda^*)\det(\Lambda)=1\). Since the operator \( \Lambda\) is real, the proposition \ref{PROPooSHZMooGwdfBd} says that \( \det(\Lambda^*)=\det(\Lambda)^*=\det(\Lambda)\). Thus \( \det(\Lambda)^2=1\). The result follows.
\end{proof}

\begin{lemma}       \label{LEMooHRNXooJOgfpy}
    Let \( \Lambda\in \gO(3,1)\). We have
    \begin{subequations}
        \begin{align}
            \Lambda_{00}^2-\sum_{k=0}^3(\Lambda_{k0})^2&=1      \label{SUBEQooFLECooUFvwOy}\\
            \Lambda_{00}^2-\sum_{k=0}^3(\Lambda_{0k})^2&=1  \label{SUBEQooBLTOooPUTztZ}
        \end{align}
    \end{subequations}
    In particular, \( \Lambda_{00}^2\geq 1\).
\end{lemma}

\begin{proof}
    Just write the \( 00\) component of the equation \( \Lambda^t\eta\Lambda=\eta\) (lemma \ref{LEMooDLWDooWCXlWq}):
    \begin{subequations}
        \begin{align}
            1&=\sum_{kl}\Lambda^t_{0k}\underbrace{\eta_{kl}}_{\eta_{kk}\delta_{kl}}\Lambda_{l0}\\
            &=\sum_k\Lambda_{0k}^t\eta_{kk}\Lambda_{k0}\\
            &=\Lambda_{00}^t\Lambda_{00}-\sum_{k=1}^3\Lambda^t_{0k}\Lambda_{k0}\\
            &=\Lambda_{00}^2+\sum_{k=1}^3\Lambda_{k0}^2.
        \end{align}
    \end{subequations}
    This is \eqref{SUBEQooFLECooUFvwOy}; the same computation from \( \Lambda\eta\Lambda^t=\eta\) provides \eqref{SUBEQooBLTOooPUTztZ}.
\end{proof}

\begin{lemma}       \label{LEMooEKXWooLEMBIj}
    The set
    \begin{equation}
        L^{\uparrow}=\gO(3,1)^{\uparrow}=\{ \Lambda\in\gO(3,1)\tq \Lambda_{00}\geq 1 \}
    \end{equation}
    is a subgroup of \( \gO(3,1)\).

    Elements of \( \gO(3,1)^{\uparrow}\) are said \defe{orthochronous}{orthochronous}. We will not use the notation \( L^{\uparrow}\).
\end{lemma}

\begin{proof}
    Let \( A,B\in L^{\uparrow}\). We have
    \begin{equation}        \label{EQooBUMZooFGCDEa}
        (AB)_{00}=\sum_{k=0}^3A_{0k}B_{k0}=A_{00}B_{00}+\sum_{k=1}^3A_{0k}B_{k0}.
    \end{equation}
    Using lemma \ref{LEMooHRNXooJOgfpy}, we write
    \begin{equation}        \label{EQooBFWTooZNIdlP}
        A_{00}^2=1+\sum_{k=1}^3A_{k0}^2
    \end{equation}
    and
    \begin{equation}
        B_{00}^2=1+\sum_{k=1}^3B_{0k}^2.
    \end{equation}
    Since \( A_{00}\geq 1\), taking the square root of \eqref{EQooBFWTooZNIdlP} does not require additional caution:
    \begin{equation}
        A_{00}>\sqrt{ \sum_{k=1}^3A_{k0}^2 }.
    \end{equation}
    The same holds for \( B\). Using these (strict) inequalities in \eqref{EQooBUMZooFGCDEa} we have
    \begin{subequations}
        \begin{align}
            (AB)_{00}>\sqrt{ \sum_{k=1}^3A_{k0}^2 }\sqrt{ \sum_{k=1}^3B_{0k}^2 }+\sum_{k=1}^3A_{0k}B_{k0}.
        \end{align}
    \end{subequations}
    If we set \( a=\sum_{k=1}^3A_{k0}e_k\) and \( b=\sum_{k=1}^3B_{0k}e_k\) (here \( e_i\in \eR^3\)), we have
    \begin{equation}
        (AB)_{00}>\| a \|\| b \|+\langle a, b\rangle \\
        \geq \| a \|\| b \|-| \langle a, b\rangle  |\\
        \geq 0
    \end{equation}
    because of the Cauchy-Schwarz identity, theorem \ref{ThoAYfEHG}. The strict inequality \( (AB)_{00}>0\) implies the inequality \( (AB)_{00}\geq 1\) because \( AB\in \gO(3,1)\) (see lemma \ref{LEMooHRN}).
\end{proof}

\begin{lemma}       \label{LEMooLJMMooOXCyOl}
    About adjoint.
    \begin{enumerate}
        \item
            If \( \Lambda\in \gO(3,1)\), then \( \Lambda^*\in \gO(3,1)\).
        \item
            If \( \Lambda\in \SO(3,1)\), then \( \Lambda^*\in \SO(3,1)\).
        \item
            If \( \Lambda\in \gO(3,1)^{\uparrow}\), then \( \Lambda^*\in \gO(3,1)^{\uparrow}\).
    \end{enumerate}
    In particular, if \( \Lambda\in \SO(3,1)^{\uparrow}\), then \( \Lambda^*\in \SO(3,1)^{\uparrow}\).
\end{lemma}

\begin{proof}
    Just compute \( \Lambda^*_{00}\) and \( \det(\Lambda^*)\) with the formula \eqref{EQooPFPGooXiGcXs}.
\end{proof}

\begin{definition}          \label{DEFooVQLPooWyINoc}
    A \defe{boost}{boost} in the direction \( x\) is an element \( \Lambda\in \SO(3,1)^{\uparrow}\) such that \( \Lambda(e_2)=e_2\) and \( \Lambda(e_3)=e_2\). In other words, this is a transformations which only involves the components \( t\) and \( x\).

    The boost in the directions \( y\) and \( z\) are defined in a similar way.

    A \defe{spatial rotation}{spatial rotation} is an element \( \Lambda\in \gO(3,1)\) such that \( \Lambda(e_0)=e_0\).
\end{definition}

\begin{lemma}
    An operator \( \Lambda\colon \eR^4\to \eR^4\) is a boost if and only if there exists \( \gamma\in \eR\) such that the matrix of \( \Lambda\) has the form 
    \begin{equation}
        \Lambda=\begin{pmatrix}
             \cosh(\gamma)   &   \sinh(\gamma)    &   0    &   0    \\
             \sinh(\gamma)   &   \cosh(\gamma)    &   0    &   0    \\
             0   &   0    &   1    &   0    \\ 
             0   &   0    &   0    &   1     
         \end{pmatrix}.
    \end{equation}
\end{lemma}

\begin{proof}
    It is easy to see that the proposed matrix is a boost. The only difficult part is the direct sense. We suppose that \( \Lambda\) is a boost. The conditions \( \Lambda e_2=e_2\) and \( \Lambda e_3=e_3\) imply that the matrix of \( \Lambda\) has the form
    \begin{equation}
        \Lambda=\begin{pmatrix}
             .   &   .    &   0    &   0    \\
             .   &   .    &   0    &   0    \\
             .   &   .    &   1    &   0    \\ 
             .   &   .    &   0    &   1     
         \end{pmatrix}
    \end{equation}
    where the dots are to be determined. Since \( \Lambda\in\gO(3,1)\) we have
    \begin{equation}
        0=e_0\cdot e_2=\Lambda e_0\cdot \Lambda e_2=\Lambda e_0\cdot e_2=-\Lambda_{20}.
    \end{equation}
    The same shows that \( \Lambda_{20}=\Lambda_{21}=\Lambda_{30}=\Lambda_{31}=0\). The matrix of \( \Lambda\) is block diagonal:
    \begin{equation}
        \Lambda=\begin{pmatrix}
             a   &   c    &   0    &   0    \\
             b   &   d    &   0    &   0    \\
             0   &   0    &   1    &   0    \\ 
             0   &   0    &   0    &   1     
         \end{pmatrix}
    \end{equation}
    where \( a\), \( b\), \( c\) and \( d\) are still to be determined. Here are the constrains.

    First the operator \( \Lambda\) preserve the product, so that \( \Lambda e_0\cdot \Lambda e_0=1\), \( \Lambda e_1\cdot \Lambda e_1=-1\), and \( \Lambda e_0\cdot \Lambda e_1=0\). These conditions are translated into
    \begin{subequations}
        \begin{align}
            a^2-c^2&=1  \label{SUBEQooWEJSooPWfmNS}\\
            d^2-b^2&=1     \label{SUBEQooKLZFooCovszD}\\
            ab-cd&=0.      \label{SUBEQooUZQWooUxUCSe}
        \end{align}
    \end{subequations}
    Second, the operator \( \Lambda\) has determinant equals to \( 1\):
    \begin{equation}        \label{EQooJAEKooTCZaIG}
        ad-bc=1,
    \end{equation}
    and finally the element \( \Lambda\) is orthochronous: \( \Lambda_{00}\geq 0\), so that
    \begin{equation}        \label{EQooQCMYooPhHeas}
        a\geq 0.
    \end{equation}

    The proposition \ref{PROPooWEHGooOBqSHY} about hyperbolic functions along with the conditions \eqref{SUBEQooWEJSooPWfmNS} and \ref{SUBEQooKLZFooCovszD} show that there exist \( x\in \eR\), \( y\in \eR\), \( \sigma\in\{ \pm1 \}\) and \( \epsilon\in\{ \pm 1 \}\) such that 
    \begin{subequations}
        \begin{align}
            a&=\sigma\cosh(x)\\
            b&=\sinh(y)\\
            c&=\sinh(x)\\
            d&=\epsilon\cosh(y).
        \end{align}
    \end{subequations}

    \begin{subproof}
        \item[\( \sigma=1\)]
    The condition \eqref{EQooQCMYooPhHeas} show that \( \sigma=1\) because the hyperbolic cosine is always strictly positive.

        \item[\( \epsilon=1\)]
            The determinant condition \eqref{EQooJAEKooTCZaIG} provides
            \begin{equation}
                \epsilon\cosh(x)\cosh(y)-\sinh(x)\sinh(y)=1.
            \end{equation}
            If \( \epsilon=-1\) we are left with
            \begin{equation}
                1=-(\cosh(x)\cosh(y)+\sinh(x)\sinh(y))
            \end{equation}
            Using the formula of proposition \ref{PROPooUNHHooIksdoJ}\ref{ITEMooOJRFooUCUaDl} we get $1=-\cosh(x+y)$ which is impossible because the hyperbolic cosine is always positive.
        \item[\( x=y\)]
            The orhogonality condition \eqref{SUBEQooUZQWooUxUCSe} implies
            \begin{equation}
                0=\cosh(x)\sinh(y)-\sinh(x)\cosh(y)=-\sinh(x-y).
            \end{equation}
            Since the hyperbolic sine is bijective (proposition \ref{PROPooQLNYooIIOdvm}) we deduce \( x-y=0\) and then \( x=y\).
    \end{subproof}
\end{proof}

We define the projection from \( \eR^4\) to \( \eR^3\) as
\begin{equation}
    \pr(x)=(x_1,x_2,x_3)\in \eR^3
\end{equation}
and, if \( b\in \eR^3\) we write
\begin{equation}
    \bar b=(0,b_1,b_2,b_3).
\end{equation}
So if \( b\in \eR^3\) we have
\begin{equation}        \label{EQooOIBWooAvxfYz}
    \bar b\cdot x=-\langle b, \pr(x)\rangle .
\end{equation}


\begin{proposition}[Standard decomposition\cite{BIBooYTTJooYpPYLT}]
    Every operator \( \Lambda\in \SO(3,1)^{\uparrow}\) can be decomposed as
    \begin{equation}
        \Lambda=RLS
    \end{equation}
    where \( R\) and \( S\) are spatial rotations and \( L\) is a boost in the \( x\) direction\footnote{Definition \ref{DEFooVQLPooWyINoc}.}.
\end{proposition}

\begin{proof}
    We initiate with \( a=\pr(\Lambda e_0)\). If \( a\neq 0\) we define \( b_1=\frac{ a }{ \| a \| }\in \eR^3\) and we consider \( b_2\) and \( b_3\) in \( \eR^3\) such that \( \{ b_1, b_2, b_3 \}\) is an orthonormal basis of \( \eR^3\) with the same orientation\footnote{Definition \ref{DEFooNVRHooEBHUSu}.} as the canonical basis..

    \begin{subproof}
        \item[The first spatial rotation]
            Now we define the spatial rotation \( R\colon \eR^4\to \eR^4\) by
            \begin{subequations}
                \begin{numcases}{}
                    Re_0=e_0\\
                    Re_i=\bar b_i.
                \end{numcases}
            \end{subequations}
            Since the basis \( \{ \bar b_i \}\) is positive-oriented, the determinant of \( R\) is positive\footnote{Proposition \ref{PROPooNBAXooKNUrnk}.} and since \( Re_0=e_0\), we have \( R_{00}=1\), so that \( R\in\SO(3,1)^{\uparrow}\).
            
        \item[One property]
            We prove that \( \Lambda^*Re_i\cdot e_0-0\) for \( i=2,3\). For that:
            \begin{subequations}
                \begin{align}
                    \Lambda^*Re_i\cdot e_0&=\Lambda^*\bar b_i\cdot e_0\\
                    &=\bar b_i\cdot \Lambda e_0\\
                    &=-\langle b_i, \pr(\Lambda e_0)\rangle \\
                    &=-\langle b_i, \| a \|  b_1\rangle \\
                    &=0.
                \end{align}
            \end{subequations}
            We used the relation \eqref{EQooOIBWooAvxfYz} and the fact that \( \{ b_i \} \) is an orthonormal basis of \( \eR^3\).
        \item[A new basis]
            We define the following vectors:
            \begin{subequations}
                \begin{align}
                    f_0&=e_0\\
                    f_2&=\Lambda^*Re_2\\
                    f_3&=\Lambda^*Re_3.
                \end{align}
            \end{subequations}
            We check that these vectors are orthonormal. First:
            \begin{subequations}
                \begin{align}
                    f_0\cdot f_2&=e_0\cdot f_2\\
                    &=e_0\cdot \Lambda^*Re_2\\
                    &= Re_2\cdot \Lambda e_0\\
                    &=\bar b_2\cdot \Lambda e_0\\
                    &=-\langle b_2, \pr(\Lambda e_0)\rangle \\
                    &=0.
                \end{align}
            \end{subequations}
            We get \( f_0\cdot f_3=0\) in the same way. Second:
            \begin{equation}
                f_2\cdot f_3=\Lambda^*Re_2\cdot \Lambda^*Re_3=e_2\cdot e_3=0.
            \end{equation}
            Now we fix \( f_1\) in such a way that \( \{ f_0,f_1,f_2,f_3 \}\) is a pseudo-orthonormal basis of \( (\eR^4,\cdot)\). Up to redefinition \( f_1\to -f_1\) we also suppose that \( \{ f_1,f_2,f_3 \}\) is a basis of \( \eR^3\) with the same orientation as the canonical basis.
        \item[The second spatial rotation]
            We define the spatial rotation \( S\colon \eR^4\to \eR^4\) by
            \begin{equation}
                Sf_i=e_i
            \end{equation}
            for \( i=0,1,2,3\).

            Due to our choice of orientation for \( f_1\) we have \( S\in \SO(3,1)^{\uparrow}\).
        \item[Boost]
            We show that \( S\Lambda^*R\) is a boost in the \( x\) direction. We have
            \begin{equation}
                S\Lambda^*Re_2=Sf_2=e_2
            \end{equation}
            and the same for \( e_3\): \( S\Lambda^*Re_3=e_3\).

            We made some choices such that \( S\) and \( R\) belong to \( \SO(3,1)^{\uparrow}\). Moreover by hypothesis \( \Lambda\in \SO(3,1)^{\uparrow}\) and the lemma \ref{LEMooLJMMooOXCyOl} implies that \( \Lambda^*\in \SO(3,1)^{\uparrow}\). The whole shows that \( S\Lambda^*R\) is a boost.

            Finally, \( \Lambda^*=S^{-1}LR^{-1}\) and taking onto account the fact that the adjoint is the inverse (lemma \ref{LEMooDLWDooWCXlWq}),
            \begin{equation}        \label{EQooSJNLooWZztHU}
                \Lambda=RL^{-1}S.
            \end{equation}
            The operator \( L^{-1}\) is a boost because \( L\) is a boost.
    \end{subproof}
    The decomposition \eqref{EQooSJNLooWZztHU} is the requested one.
\end{proof}

% This is part of (almost) Everything I know in mathematics and physics
% Copyright (c) 2013-2014, 2019
%   Laurent Claessens
% See the file fdl-1.3.txt for copying conditions.

%+++++++++++++++++++++++++++++++++++++++++++++++++++++++++++++++++++++++++++++++++++++++++++++++++++++++++++++++++++++++++++ 
\section{Representations of \texorpdfstring{$ \SL(2,\eR)$}{SL(2,R)} and \texorpdfstring{$ \SU(2)$}{SU(2)}}
%+++++++++++++++++++++++++++++++++++++++++++++++++++++++++++++++++++++++++++++++++++++++++++++++++++++++++++++++++++++++++++

The representation $\pi_m$ of $\SL(2,\eC)$ restricts to $\SL(2,\eR)$.

\begin{lemma}
The representation $\pi_m$ of $\SL(2,\eR)$ is irreducible.
\end{lemma}

\begin{proof}
If $W$ is an invariant space under $\pi_m\big( \SL(2,\eR) \big)$, then is is invariant under the derived representation $\rho_m\big( \gsl(e,\eR) \big)$. The proof of proposition~\ref{ProprhomirredsldeuxC} still holds here, so that $W=\mP_m$.
\end{proof}

\begin{theorem}
Let $\pi$ be an irreducible representation of $G=\SL(2,\eR)$ or $\SU(2)$ in a complex finite dimensional vector space $V$. Then $\pi$ is equivalent to one of the $\pi_m$.
\end{theorem}

\begin{proof}
Let $\lG$ be the Lie algebra of $G$. One important property shared by $\SL(2,\eR)$ and $\SU(2)$ is that $G=\exp(\lG)$. It is clear that the representation $d\pi$ on $\gsl(2,\eR)$ extends $\eC$-linearly to a representation $\rho$ of $\gsl(2,\eC)$. Looking on the basis \eqref{EqGenssudeux}, one sees that in fact the same is true for $\gsu(2)$ which $\eC$-linearly extends to $\gsl(2,\eC)$.

Let us prove that $\rho$ is irreducible. Let $W\neq\{ 0 \}$ be a subspace of $V$ invariant under $\rho(\lG)$. Then $W$ is invariant under $ e^{\rho(X)}=\pi( e^{X})$ for every $X\in\lG$. Since $\exp(\lG)=G$, the space $W$ is in fact invariant under $\pi(G)$, and is therefore equal to $V$.

Since $\rho$ is irreducible, we have $\rho=\rho_m$ for a certain $m$. Thus there exists an intertwining operator $A\colon V\to \mP_m$ such that
\[
	A\rho(X)=\rho_m(X)A
\]
for every $X\in\lG$. By linearity, for every $N\in\eN$, we have $A\rho\big( \sum_{k=1}^n X^k/k! \big)=\rho_m\big( \sum_{k=1}^n X^k/k! \big)A$, and at the limit, we have
\begin{equation}
	A e^{\rho(X)}= e^{\rho_m(X)A}.
\end{equation}
From that we deduce that $A\pi( e^{X})=\pi_m( e^{X})A$ which means that
\[
	A\pi(g)=\pi_m(g)A.
\]
That shows that $A$ intertwines $\pi$ and $\pi_m$, so that $\pi$ is equivalent to $\pi_m$.
\end{proof}
%+++++++++++++++++++++++++++++++++++++++++++++++++++++++++++++++++++++++++++++++++++++++++++++++++++++++++++++++++++++++++++
\section{Representations of \texorpdfstring{$\so(2,d-1)$}{so2d}}
%+++++++++++++++++++++++++++++++++++++++++++++++++++++++++++++++++++++++++++++++++++++++++++++++++++++++++++++++++++++++++++

Here we deal with the representations of \( \so(2,d-1)\). For singleton theory as field theory, see the section~\ref{SecUKPhZVd} (you are welcome if you can fill that section which is for the moment almost empty).

%---------------------------------------------------------------------------------------------------------------------------
\subsection{Verma module}
%---------------------------------------------------------------------------------------------------------------------------

One can find text about these representations in \cite{Ferrata,Dolan_son,SingString}, while we will mainly follow the developments of \cite{SingletonCompposites,HowMassless,Teschner}. We are going to study representations of the algebra $\so(2,d-1)$ which fulfills the commutation relations described in lemma~\ref{LemCommsopqAlg}:
\begin{equation}		\label{EqCommsodeuxdmoinsun}
	[M_{ab},M_{cd}]=-i\eta_{ac}M_{bd}+i\eta_{ad}M_{bc}+i\eta_{bc}M_{ad}-i\eta_{bd}M_{ac}.
\end{equation}
Notice that $M_{ab}=-M_{ba}$. As convention, the indices $a$, $b$,\ldots run over $\{ 0,0',1,2,\ldots d-1 \}$ while $r$, $s$, \ldots run over $\{ 1,2,\ldots,d-1 \}$. As on page \pageref{PgDefsGenre}, we choose the convention
\begin{equation}
	\eta =
	\begin{pmatrix}
		\mtu_{2\times 2}\\
		&-\mtu_{(d-1)\times (d-1)}
	\end{pmatrix}.
\end{equation}
Notice that this convention numerically holds for the matrix $\eta_{st}$ as well as for its inverse $\eta^{st}$.

The algebra separates into two parts: the compact and the non compact part. The maximal compact subalgebra is $\so(2)\oplus\so(d-1)$ which is generated by $E=M_{00'}$ and $J_{rs}=M_{rs}$. The non compact generators are $M_{0'r}$ and $M_{0r}$ that we rearrange into ladder operators
\begin{equation}
	L^{\pm}_r=M_{0r}\mp iM_{0'r}.
\end{equation}
Using commutation relations \eqref{EqCommsodeuxdmoinsun}, one computes the commutators in the new basis. For example
\[
	[E,L^{\pm}_r]=[M_{0'0},M_{0r}]\mp i[E_{0'0},M_{0'r}]=\pm M_{0r}-iM_{0'r}=\pm L^{\pm}_r.
\]
The table of $\so(2,d-1)$ in this basis is
\begin{subequations}		\label{SubEqsCommssodeuxd}
	\begin{align}
		[E,L^{\pm}_r]&=\pm L^{\pm}_r\\
		[J_{rs},L_t^{\pm}]&=-i(\delta_{rt}L_s^{\pm}-\delta_{st}L_r^{\pm})\\
		[L_r^{-},L_s^+]&=2(iJ_{rs}+\delta_{rs}E)\\
		[J_{rs},J_{tu}]&=-i\delta_{ac}M_{bd}+i\delta_{ad}M_{bc}+i\delta_{bc}M_{ad}-i\delta_{bd}M_{ac}.
	\end{align}
\end{subequations}
The unitary properties are $(M_{rs})^{\dag}=M_{rs}$, $E^{\dag}=E$ and $(L^{\pm}_r)^{\dag}=L_r^{\mp}$. From these commutators, we deduce the following rules that will be always used
\begin{subequations}
	\begin{align}
		L^-_rL^+_s&=L^+_sL^-_r+2(iJ_{rs}+\delta_{rs}E)\\
		J_{rs}L^+_t&=L^+_tJ_{rs}-i(\delta_{rt}L^+_s-\delta_{st}L^+_r)\\
		EL^+_r&=L^+_rE+L^+_r.
	\end{align}
\end{subequations}

The Cartan algebra of $\so(d)$ is given by the elements $A_p=M_{2p-1,2p}$ with $p=1,\ldots, r$ for $\so(2r)$ and $\so(2r+1)$.

The unitary irreducible representations of $\so(2,n)$ have the form $\mD(e_0,\bar\jmath)$. It is given by a basis vector $| e_0,\bar\jmath \rangle$ on which $E$ and $J_{rs}$ act by their respective representations (of $\so(n)$ and $\so(2)$). The \defe{energy}{energy!in the representations of $\so(2,d-1)$} of the vector $\ket{e,\overline{ m }}$ is its eigenvalue for the operator $E$, namely $e$:
\begin{equation}
	E\ket{e,\overline{ m }}=e\ket{e,\overline{ m }}.
\end{equation}
Using the commutators \eqref{SubEqsCommssodeuxd}, we find $L_r^{\pm}=(E\pm 1)L_r^{\pm}$, so that
\begin{equation}
	E L^{\pm}_r\ket{e,\overline{ m }} =(e\pm 1)\ket{e,\overline{ m }}.
\end{equation}
We see that the ladder operator $L_r^+$ raises the value of the energy of one unit, while the operator $L_r^-$ lower the energy of one unit. The vector $\ket{e_0,\bar\jmath}$ is the \defe{vacuum vector}{vacuum!vector}, it has the lowest energy in the sense that $L^{-}_r| e_0,\jmath \rangle=0$. A \defe{scalar representation}{scalar!representation} is a representation with $\bar\jmath=0$. They are, logically, denoted by $\mD(e_0)$ and its vacuum is $| e_0 \rangle$ which satisfies
\begin{align}		\label{Eqaldefketezerovac}
	J_{rs}| e_0 \rangle & = 0	& (E-e_0)| e_0 \rangle&=0	&L_r^{-}| e_0 \rangle&=0.
\end{align}
Then one build the generalised Verma module
\begin{equation}	\label{EqmVVermaldots}
	\mV(e_0,0)\equiv \big\{   L_{r_1}^+\ldots L_{r_n}^+| e_0 \rangle   \big\}_{n=0}^{\infty}.
\end{equation}
Notice that the Verma module is not automatically irreducible. We will soon build irreducible representations by taking quotient of the Verma module by its singular module.

In order to compute the norm of $L_s^+L_r^+\ket{e_0}$, we compute $L^-_rL^-_sL^+_sL^+_r\ket{e_0} =4\big(E+E^2+\delta_{rs}E^2+(J_{rs})^2\big)\ket{e_0}$. In order to get that result, we moved all the $L^-$ on the right using the commutation relation, and we taken into account the simplifications induced by the definition relations \eqref{Eqaldefketezerovac}. Now, using the relation $J_{rs}\ket{e_0}=0$, we have
\begin{equation}
	4\big(E+E^2+\delta_{rs}E^2\big)\ket{e_0}.
\end{equation} We also have
\begin{equation}
	(E-e_0)L^+_sL^+_s\ket{e_0}=0.
\end{equation}

\begin{proposition}
The vectors $L^+_{r_1}\ldots L^+_{r_k}\ket{e_0}$ and $L^+_{t_1}\ldots L^+_{t_l}\ket{e_0}$ are orthogonal if $k\neq l$.
\end{proposition}
That proposition says that different layers are orthogonal\quext{À justifier en analysant qui est exactement $\lH$ et les racines simples, mais ça me semble ok.}

\begin{proof}
We proceed by induction. We suppose that the result is proved for $k,l\geq n$, and we prove that
\begin{equation}
	L^-_{t_1}\ldots L^-_{t_{n+1}}L^+_{r_1}\ldots L^+_{r_n}\ket{e_0}=0.
\end{equation}
First, remark that, using the commutation relations and the fact that $J_{rs}\ket{e_0}=0$ and $E\ket{e_0}=e_0\ket{e_0}$, the vectors
\begin{subequations}		\label{SubEqsJELLket}
	\begin{align}
		J_{st}L^+_{r_1}\ldots L^+_{r_k}\ket{e_0}\\
		EL^+_{r_1}\ldots L^+_{r_k}\ket{e_0}
	\end{align}
\end{subequations}
are combinations of vectors of the form $L^+_{a_1}\ldots L^+_{a_k}\ket{e_0}$. Now, we have
\begin{equation}
	L^-_{t_1}\ldots L^-_{t_{n+1}}L^+_{r_1}\ldots L^+_{r_n}\ket{e_0}= L^-_{t_1}\ldots L^-_{t_n}\big( L^+_{r_1}L^-_{t_{n+1}}+2i(J_{t_{n+1},r_1} + \delta_{t_{n+1},r_1}E ) \big)L^+_{r_2}\ldots L^+_{r_n}\ket{e_0}
\end{equation}
which decomposes in three terms. The first one is
\begin{equation}
	L^-_{t_1}\ldots L^-_{t_n}L^+_{r_1}L^-_{t_{n+1}}L^+_{r_2}\ldots L^+_{r_n}\ket{e_0},
\end{equation}
and according to equations \eqref{SubEqsJELLket}, the two other terms reduce to zero. Continuing that way, the operator $L^-_{t_{n+1}}$ advance of one position at each step and finishes to kill himself on $\ket{e_0}$.
\end{proof}

%---------------------------------------------------------------------------------------------------------------------------
					\subsection{Singular module}
%---------------------------------------------------------------------------------------------------------------------------

Let us compute the norm of the general vector $L^+_{r_1}\ldots L^+_{r_k}\ket{s_0,s}$. We have
\begin{equation}
	\begin{aligned}[]
		L^-_{r_k}\ldots L^-_{r_1}L^+_{r_1}\ldots L^+_{r_k}\ket{e_0,s}
				&= L^-_{r_k}\ldots L^-_{r_2} (L^+_{r_1}L^-_{r_2}+2E) L^+_{r_2}\ldots L^+_{r_k}\ket{e_0,s}\\
				&= L^-_{r_k}\ldots L^-_{r_2}L^+_{r_1}L^-_{r_2}L^+_{r_2}\ldots L^+_{r_k}\ket{e_0,s}\\
				&\quad +2(e_0+k-1)L^-_{r_k}\ldots L^-_{r_2} L^+_{r_2}\ldots L^+_{r_k}\ket{e_0,s}
	\end{aligned}
\end{equation}
Using again and again the commutation relations, we eliminate all the operators $L^+_r$ and we obtain a sum of terms of the form $(e_0+k-l)$. It will, obviously be positive for large enough $e_0$. Thus, unitarity of the representation is enforced for large values of $e_0$, and there exists a lower bound $E_0(s)$ such that negative norm states appears when $e_0<E_0(s)$. If $e_0=E_0(s)$, then these vectors have a vanishing norm.

Let us consider a limit representation: $e_0=E_0$; there are vectors of vanishing norm, but no vectors with negative norm. In that case, if $v\cdot v=0$, then $v\cdot w=0$ for every other vector $w$. Indeed, if $v\cdot w\neq 0$, we have
\begin{equation}
	(v-w)\cdot(v-w)=w\cdot w - v\cdot w-w\cdot v,
\end{equation}
which holds for every positive multiple $\lambda v$ and $\mu w$. Choosing a big $\lambda$ and a small $\mu$, the norm of $\lambda v-\mu w$ becomes negative. What we proved is
\begin{lemma}
	If the energy $e_0$ of a representation saturates the unitary condition, then a vector with vanishing norm is orthogonal to every other vectors. Moreover, the vectors with vanishing norm form an invariant subspace.
\end{lemma}
The second part is the fact that, if $A\in\lG$, and $\| \ket{\psi} \|=0$  then $\| A\ket{\psi} \|=0$, because it is the scalar product of $\ket{\psi}$ with the vector $A^{\dag}A\ket{\psi}$. The submodule made of vectors of zero norm is the \defe{singular submodule}{singular!submodule}, and is denoted by $\mS(e_0,s)$.

\begin{proposition}		\label{PropSinModRedSSIADesNuls}
A module is reducible if and only if it possesses a vector $\ket{v}$ (different from $\ket{e_0,s}$) such that $L^-_r\ket{v}=0$ for every $r$. Such a vector is said to be \defe{null}{null vector}.
\end{proposition}

\begin{proof}
Since the energy is bounded from bellow, applying several times the lowering operators $L^-_r$ on any vector ends up on zero. Thus, any submodule contains a vector $\ket{v}$ such that $L^-\ket{v}=0$ for every $r$. If that vector is not $\ket{e_0,s}$, then the submodule is a proper submodule.

If $\ket{v}\neq\ket{e_0,s}$, then it is of the form $L^+_{\bar r}\ket{e_0,s}$ and its norm is given by
\begin{equation}
	\| L^+_{\bar r}\ket{e_0,s} \|=\bra{e_0,s} L^-_{\bar r}L^+_{\bar r}\ket{e_0,s}=0
\end{equation}
because $L^-_{\bar r}L^+_{\bar r}\ket{e_0,s}=0$ by assumption.
\end{proof}
From the Verma module \eqref{EqmVVermaldots}, we thus extract the irreducible representation taking the quotient by the singular module:
\begin{equation}
	\mH(e_0) = \mV(e_0)/\mS(e_0).
\end{equation}

Most of time, we have only one extra vacuum, let $\ket{e'_0,s'}$, and in this case, the whole singular module is generated by vectors of the form
\begin{equation}
	L^+_{r_1}\ldots L^+_{r_k}\ket{e'_0,s'}.
\end{equation}
Let $\ket{e_0,s}$ be the vacuum with $s=(s_1,s_2,0,\ldots,0)$, corresponding to the Young diagram
\begin{equation}
   \input{auto/pictures_tex/Fig_AIFsOQO.pstricks}
%	\input{image_Young_sssSing.pstricks}
\end{equation}
where the first line has $s_1$ boxes and the second one has $s_2$ boxes. Thanks to theorem~\ref{ThoOpqrepreTens}, it can be realized with the tensor
\begin{equation}
	v_{a_1,\ldots a_{s_2}b_1\ldots s_1}(e_0)
\end{equation}
which is separately symmetric in the indices $a$ and $b$, in the same time as being antisymmetric in the couples $a_i$, $b_i$ when $i\leq s_2$, for example,
\begin{equation}
	v_{a_1\ldots a_{s_2},b_1\ldots b_{s_1}} = -v_{b_1 a_1\ldots a_{s_2},b_2\ldots b_{s_1}}.
\end{equation}
In particular, if we symmetrise $v$ on $s_1+1$ indices, we always found zero. Moreover, all the traces vanishes. If $\eta$ is the metric of $O(D-1)$, we have for example
\begin{equation}
	\eta^{b_1b_2}v_{a_1\ldots a_{s_2},b_1,\ldots b_{s_2}}=0.
\end{equation}

The vectors of the first level are the ones of the form	$L^+_r\ket{e_0,s}$. As far as notations are concerned, we have
\begin{equation}
	L^+_rv_{a_1\ldots a_{s_2},b_1\ldots b_{s_2}}=(L^+_rv)_{a_1\ldots a_{s_2},b_1\ldots b_{s_2}}.
\end{equation}
Remark that the operators $\{ L_r^+ \}_{r=1,\ldots,D-1}$ carry a representation of $o(D-1)$, namely the vector representation. Thus, the states of the first level form the representation given by the tensor product of $(s_1,s_2,0,\ldots)$ and the vector representation. In order to see the irreducible components of that representation, we have to know what are the symmetry properties that we can give to the indices
\begin{equation}
	r,a_1,\ldots,a_{s_2},b_1,\ldots,b_{s_1}.
\end{equation}
There are three possibilities: we can contract the $r$ with one of the $a_i$ (by symmetry, all of these contractions are equivalent), or with one of the $b_i$, or add one box in the Young diagram. The latter possibility splits into three cases: the diagram $(s_1,s_2,0,\ldots)$ can be transformed in $(s_1+1,s_2,0,\ldots)$, $(s_1,s_2+1,0,\ldots)$ or $(s_1,s_2,1,0,\ldots)$. So we have $5$ irreducible component in the $o(D-1)$ representation carried by the level one.

The question that naturally arises is to know if one of these have a singular vacuum. In other words, if $\Pi_{\beta}$ are the projections to the irreducible components, do we have
\begin{equation}
	L^-_{t}\Pi_{\alpha}\big( L^+_rv_{\bar a,\bar b}(e_0) \big)=0
\end{equation}
for a certain $e_0$?

Notice that the contraction with the last $b_i$'s is not the same as the one with the firsts ones because of the symmetry properties with respect to the $a_i$'s. The first representation with cell cut is given by
\begin{equation}
	v^1_{\bar a,b_1\ldots b_{s_1-1}}	=\eta^{rt}L^+_r\big\{ v_{\bar a,b_1 \ldots b_{s_1-1}t}(e_0)
							+\frac{ s_2 }{ s_1-s_2+1 } v_{ca_1\ldots a_{s_2-1},b_1\ldots b_{s_1-1}a_{s_2}(e_0)}\big\},
\end{equation}
while the second representation with cell cut is easier:
\begin{equation}
	v^2_{a_1\ldots a_{s_{2}-1},\bar b}=\eta^{rt}L^+_rv_{ta_1\ldots a_{s_2-1},\bar b}(e_0).
\end{equation}
Now, the sport is to compute $L^-_qv^1_{\bar a,b_1\ldots b_{s_1-1}}$ and $L^-_q v^2_{a_1\ldots a_{s_{2}-1},\bar b}$.

\begin{probleme}
Il y a du calcul non terminé, ici.
\end{probleme}

%---------------------------------------------------------------------------------------------------------------------------
					\subsection{The quotient for the scalar singleton}
%---------------------------------------------------------------------------------------------------------------------------

The value of the energy which saturates the unitary condition is $1$ when $s=\frac{1}{ 2 }$ and $\frac{ 1 }{2}$ when $s=0$. That is the reason why we consider the two special representations
\begin{equation}
	\begin{aligned}[]
		\rDi&=\mD(1,\frac{ 1 }{2})		&& \rRac&=\mD(\frac{ 1 }{2},0).
	\end{aligned}
\end{equation}
We are now interested in the scalar case, the $Rac$.

We know that, when $\epsilon_0$ is the value of the energy which saturates the unitary condition $e_0\geq \frac{ d-3 }{ 2 }$ (in the scalar case, then
\begin{enumerate}
\item the vectors $L^+_sL^+_s\ket{\epsilon_0}$ are singular vectors,
\item the vectors $L^+_{r_1}\cdots L^+_{r_n}L^+_sL^+_s\ket{\epsilon_0}$ is orthogonal to all other states, it is a null vector.
\end{enumerate}
On the other hand, we know from proposition~\ref{PropSinModRedSSIADesNuls} that a module is reducible if and only if it has a vector $\ket{v}\neq\ket{e_0,s}$ such that $L^-_r\ket{v}=0$ for every $r$. Thus one constructs irreducible representations by taking the quotient of the Verma module by the singular module.

What is the dimension of the scalar singleton? We have to count how many different vectors we have in the Verma module $\mV(e_0,0)\equiv \big\{   L_{r_1}^+\ldots L_{r_n}^+| e_0 \rangle   \big\}_{n=0}^{\infty}$, and which \emph{are not} build over $L^+_sL^+_s\ket{e_0}$. In the case of $\SO(2,3)$, we have the generators $L^+_1$, $L^+_2$ and $L^+_3$ (which are commuting), so the only vectors that are left after removing the singular modules are the seven following ones: $L^+_1\ket{e_0}$, $L^+_2\ket{e_0}$,$L^+_3\ket{e_0}$, $L^+_1L^+_2\ket{e_0}$, $L^+_1L^+_3\ket{e_0}$,$L^+_2L^+_3\ket{e_0}$, and $L^+_1L^+_2L^+_3\ket{e_0}$. The scalar singleton representation is thus $7$ dimensional.


\chapter{Lie groups of transformations}
\input{109_helga}

\chapter{Classical mechanics}
\input{Mecanique}

\chapter{Hilbert spaces}
% This is part of (almost) Everything I know in mathematics
% Copyright (c) 2014, 2016, 2020
%   Laurent Claessens
% See the file fdl-1.3.txt for copying conditions.

References for Hilbert spaces are \cite{Wassermann,Landsman}.

Do you know what is normed, complete and yellow? Answer in the footnote\footnote{A bananach space!}.

\section{Basis and orthonormal systems}
%+++++++++++++++++++++++++++++++++++++++++

A \defe{sesquilinear map}{sesquilinear} map on a complex vector space $V$ is a map $(.,.)\colon V\times V\to \eC$ such that
\[
\begin{split}
(x+y,x'+y')&=(x,x')+(x,y')+(y,x')+(y,y'),\\
(\lambda x,\mu y)&=\bar\lambda\mu(x,y).
\end{split}
\]

\begin{definition}		\label{DefBanchHilbertpre}
	\begin{enumerate}
		\item
			A \defe{Banach space}{Banach!space} is a complete and normed vector space.
		\item
			A \defe{pre-Hilbert}{pre-Hilbert} is a complex vector space with an inner product
		\item
			An \defe{Hilbert space}{Hilbert space} is a complex Banach space whose norm is induced from an inner product. Equivalently, it is a pre-Hilbert space in which the topology is complete.
	\end{enumerate}
\end{definition}

From a pre-Hilbert space, one can construct an Hilbert space by \defe{completion}{completion}. The completion of a pre-Hilbert space $H_0$ is the set of all the Cauchy sequences in $H_0$. It turns out that this set is an Hilbert space. Points in $H_0$ are identified with Cauchy sequences that converge in $H_0$.

\begin{remark}
	In some literature\cite{AlgOpGirard}, a pre-Hilbert space is defined as a complex vector space endowed with a sesquilinear positive form. That is a sesquilinear form such that $\langle x, x\rangle \geq 0$. In this case the map $x\mapsto\langle x, x\rangle ^{1/2}$ is only a seminorm: there could be elements with vanishing norm.

	If $H_0$ is such a space, before to take its completion, we have to take its \defe{separation}{separation}. The separation is as follows. Let $I=\{ x\in H_0\tq \langle x, x\rangle =0 \}$. The quotient space $H/I$ is then a pre-Hilbert in the sense of definition~\ref{DefBanchHilbertpre}.
\end{remark}

A subset $\mS$ of a pre-Hilbert $\mP$ is \defe{total}{total subset in a pre-Hilbert} if $0$ is the only element in $\mP$ to be orthogonal to each element of $\mS$, in other words: $\scalh{z}{s}=0$ for any $s\in\mP$ implies $z=0$.

\begin{proposition}		\label{PropCconvminiv}
If $C\subset\hH$ is a closed convex subset of the Hilbert space $\hH$ and if $v\in\hH$, there exists one and only one $c_C\in C$ such that
\[
  \| v-v_C \|=\min_{w\in C}\| v-w \|,
\]
i.e. $v_C$ minimizes the distance between $v$ and $C$.
\end{proposition}
\begin{proof}
No proof.
\end{proof}

\begin{proposition}
In the same setting that proposition~\ref{PropCconvminiv}, with the assumption that $C$ is a vector subspace of $\hH$, we have $v-v_C\in C^{\perp}$.
\end{proposition}

\begin{proof}
Let $v_C$ be the minimizer given by proposition~\ref{PropCconvminiv}; by definition for every $w\in C$, the distance between $v_C+tw$ and $v$ is bigger than the one between $v_C$ and $v$. In particular, the derivative of $\| v_c+tw-v \|^2$ with respect to $t$ vanishes on $t=0$. A small computation provides
\[
  \real\big( \langle v-v_C, w\rangle  \big)=0
\]
for every $w$. Doing the same with $iw$, we find that the imaginary part of $\langle v-v_C, w\rangle $ vanishes too, so that the proposition is proved.
\end{proof}

We conclude that when $C$ is a convex closed vector subspace of $\hH$, the latter accepts the decomposition $\hH=C\oplus C^{\perp}$.

A sequence $(x_n)$ in a Hilbert space $\pH$ is an  \defe{orthonormal basis}{basis!of Hilbert space} if
\begin{itemize}
\item $\scalh{x_i}{x_j}=\delta_{ij}$,
\item the sequence $(n_n)$ is total.
\end{itemize}

An Hilbert space $\pH$ is \defe{separable}{separable!Hilbert space} if it posses a total sequence. The link between this and the topological definition of \emph{separable} is not completely easy. A first step is done in lemma~\ref{lem:sep_metric}.

\begin{theorem}
An Hilbert space is separable if and only if it posses an orthonormal basis.
\end{theorem}

A classical but powerful theorem about orthonormal basis:

\begin{theorem}
Let $\pH$ be an infinite dimensional Hilbert space and a sequence $(x_n)$ in $\pH$. Then the following propositions are equivalent:

\begin{enumerate}
\item $(x_n)$ is an orthonormal basis,
\item $\sum_{k=1}^{\infty}|\scalh{x}{x_k}|^2=\|x\|^2$ for every $x\in\pH$,
\item $\sum_{k=1}^{\infty}\scalh{x_k}{x}|x_k=x$ for every $x\in\pH$.
\end{enumerate}
\end{theorem}

We will sometimes denote by $|\psi\rangle$ the vector $\psi$ and $\langle\phi|$ the form $|\psi\rangle\to\scal{ \phi }{ \psi }$. This notation is mainly used in the physics literature.

Let $\{ v_{\alpha} \}$ be a maximal orthogonal set in $\hH$, then each $v\in\hH$ can be written under the form
\begin{equation}		\label{Eqvsumvalphavmaxorth}
	v=\sum_{\alpha}\langle v_{\alpha},v\rangle v_{\alpha},
\end{equation}
and the norm is given by
\begin{equation}
\| v \|^2=\sum_{\alpha}| v_{\alpha},v |^2.
\end{equation}
Notice that the latter sum is absolutely convergent, while the first one is not. The sum in the right hand side of \eqref{Eqvsumvalphavmaxorth} is \defe{unconditionally convergent}{unconditional convergence}. One says that a sum $\sum_{\alpha\in A} X_{\alpha}=X$ unconditionally in a Banach space if $\forall\epsilon>0$, there exists a finite subset $F$ of $A$ such that for every finite subset $F'$ containing $F$,
\[
  \| \sum_{\alpha\in F'}(X_{\alpha}-X)\| \leq \epsilon.
\]
That notion of convergence is the good one in Hilbert space where one does not always have absolute convergence.

\begin{theorem}[Riesz-Fisher]		\label{ThoRiesz}\index{Riesz-Fisher theorem}
If $\varphi\colon \hH\to \eC$ is a continuous linear functional on $\hH$, there exists a vector $w\in\hH$ such that
\[
  \varphi(v)=\langle v, w\rangle
\]
for every $v\in\hH$.
\end{theorem}

\begin{proof}
First, remark that, because of continuity, $\ker\varphi$ is a closed subspace of $\hH$, so that $\hH=\ker(\varphi)\oplus\ker(\varphi)^{\perp}$. Let $z$ be any non vanishing element of $\ker(\varphi)^{\perp}$. In that case, the map $v\mapsto\langle z, v\rangle $ has the same kernel as $\varphi$. But we know that two linear maps with the same kernel are related by a simple multiplication by a constant scalar. A rescaling of $z$ by that scalar provides the answer of the theorem.

In order to be complete, notice that the kernel of $v\mapsto\langle z, v\rangle $ has codimension one in $\hH$ because the image has dimension one.
\end{proof}

A more complete version of that theorem is~\ref{ThoQgTovL}.

\section{Operators on Hilbert spaces}
%++++++++++++++++++++++++++++++++++++
About spectral theory: \cite{AndrewGreen}.

\begin{definition}	\label{DefVecteurTrace}
	A vector $v\in\hH$ is a \defe{trace vector}{trace!vector} if the functional
	\begin{equation}
		T\mapsto\langle v, Tv\rangle
	\end{equation}
	is a trace (that is $\omega(T^*T)=\omega(TT^*)$).
\end{definition}

\begin{definition}
    An operator \(T\colon X\to Y\) between two Banach spaces is \defe{\href{http://en.wikipedia.org/wiki/Closed_operator}{closed}}{closed!operator} if for every sequence \( (x_n)\in D(T)\) such that \(x_n\to x\in X\) and \(Tx_n\to y\in Y\) we have \(x\in D(a)\) and \(Ax=y\).
\end{definition}

\begin{proposition}     \label{PropoOpFermableLim}
    An operator admits a closure if and only of for every pair of sequences \( (x_n)\) and \( (y_n)\) in \(D(T)\) with \(\lim x_n=\lim y_n=x\) and such that \(Tx_n\) and \(Ty_n\) converge we have \(\lim Tx_n=\lim Ty_n\).
\end{proposition}

In this case the \defe{closure}{closure} of \(T\) is defined by \( Tx=\lim Tx_n \).

%---------------------------------------------------------------------------------------------------------------------------
\subsection{Adjoint, unitary and projection operator}
%---------------------------------------------------------------------------------------------------------------------------

Let a bounded operator \( T\colon \hH_1\to \hH_2\). We define the adjoint operator \(T^*\colon \hH_2\to \hH_1\) in the following way. Let \( v_2\in \hH_2\); we define
\begin{equation}
    \begin{aligned}
        \phi\colon \hH_1 &\to \eC \\
        v_1&\mapsto \langle Tv_1, v_2\rangle.
    \end{aligned}
\end{equation}
This is an element of \( \hH_1'\), so that the Riesz's theorem~\ref{ThoRiesz} produces an elemen t\( y\in \hH_1\) such that
\begin{equation}
    \phi(v_1)=\langle y, v_1\rangle
\end{equation}
for every \( v_1\in \hH_1\). We define \( T^*v_2\) to be that element.

\begin{definition}      \label{DEFooERIYooIIRLuy}
    In short, \( T^*\) is defined by the formula
    \begin{equation}
    \langle Tv_1, v_2\rangle =\langle v_1, T^*v_2\rangle
    \end{equation}
    for every \( v_1\in \hH_1\) and \( v_2\in\hH_2\).
\end{definition}

\begin{lemma}			\label{LemTTzepoT}
If $TT^*=0$, then $T=0$.
\end{lemma}

\begin{proof}
The assumption makes that for every $v\in\hH$,
\begin{equation}
0=\langle A^*Av, v\rangle =\langle Av, Av\rangle =\| Av \|^2.
\end{equation}
 That proves that $Av=0$ for every $v$, or that $A=0$.
\end{proof}

An operator such that $T^*T=\mtu$ is an \defe{isometry}{isometry!in Hilbert space}, but is not always invertible. An invertible isometry is an \defe{unitary operator}{unitary!operator} and fulfills $U^*U=\mtu=UU^*$. A \defe{projection}{projection!in Hilbert space} is an operator $P$ such that $P=P^*$ and $P^2=P$.

A \defe{partial isometry}{partial!isometry} is a linear map $\dpt{ W }{ V_1 }{ V_2 }$ between two vector spaces such that there exists a closed subspace $K_1\subset V_1$ with $\scal{ W\psi }{ W\phi }_2=\scal{ \psi }{ \phi }_1$ for all $\psi,\phi\in K_1$ and $W=0$ on $K_1^{\perp}$. The most immediate property is that $W$ is unitary between $K_1$ and $WK_1$.

\begin{lemma}		\label{LemPartIsomCstar}
The element $A\in\cA$ is a partial isometry between $\cA$ and itself if and only if $A^*A$ is a projection.
\end{lemma}

\begin{proof}
If we pose $V_1=V_2=\cA$ in the definition of a partial isometry, the fact for $p$ to be a projection is the existence of $K_1\subset\cA$ such that $\scal{ pA }{ pB }=\scal{ A }{ B }$ for every $A$, $B\in K_1$. For such a $K_1$, we have $p^*p=\id|_{K_1}$ and $p|_{K_1^{\perp}}=0$.
\end{proof}

\subsection{Topology on space of continuous endomorphism} \label{subsec_topomL}
%--------------------------------------------------------

Let $\pH$ be a Hilbert space and $\mL$ the space of continuous endomorphism on $\pH$. The \defe{uniform topology}{topology!uniform on $\protect\mL(\protect\pH)$} is the one of the norm $T\to\| T \|$; the \defe{strong topology}{topology!strong on $\mL(\pH)$} is given by semi-norms $T\to\| T\xi \|$ (one semi-norm for each $\xi\in\pH$); the \defe{weak topology}{topology!weak on $\protect\mL(\protect\pH)$} is given by semi-norms $T\to | \scal{ T\xi }{ \eta } |$.

For a sequence $A_n\in\mL(\pH)$, we write $A_n\to 0$ in the sense of \defe{strong convergence}{convergence!strong in $\protect\mL(\protect\pH)$} in $\mL$ if for all neighbourhood $V$ of $0$, there exists a $N\in\eN$ such that $A_n\in V$ for all $n\geq N$. A neighbourhood of $0$ is of the form
\[
  V=\{ T\in\mL(\pH)\tq s_{\xi}(T)<\epsilon \}
\]
with $s_{\xi}$, the strong semi-norm defined by $\xi$: $s_{\xi}(T)=\| T\xi \|$. So we have $A_n\to 0$ in the sense of strong topology in $\mL(\pH)$ if and only if for all $\xi\in\pH$, $\| A_n\xi \|\to 0$.

\subsection{Compact operators}
%-----------------------------

\begin{definition}
The \defe{adjoint}{adjoint!operator} $A^*$ of the operator $A$ on a Hilbert space is defined by the property $\scal{A\psi}{\phi}=\scal{\psi}{A^*\phi}$. It defines an involution on $\oB(\cB)$.  An element $x$ in an involutive algebra $\cA$ is \defe{hermitian}{hermitian} if $x^*=x$. In the case of $\cA=\cB(\hH)$ --the Banach space of the bounded operators on a Hilbert space $\hH$--- we say \defe{self-adjoint}{self-adjoint operator}.
\end{definition}

As usual notations, $\hH$ denotes a Hilbert space, $\opK$ the space of compact operators and $\opB$ the one of bounded operators. Let us recall some properties of compact operators.  An operator is compact when it can be norm approximated by operators of finite rank, more precisely, we define the \defe{characteristic values}{characteristic!value} of the operator $T$ as
\begin{equation}	\label{Defmuncaharacinfn}
  \mu_n(T)=\inf\{ \| T-R \| \textrm{ where $R$ is an operator of range $\leq n$} \}
\end{equation}

\begin{lemma}
The characteristic values $\mu_n(T)$ are the eigenvalues of the operator $| T |=(T^*T)^{1/2}$\nomenclature[F]{$|T|$}{Absolute value of an operator} classified in decreasing order with multiplicity.
\end{lemma}



We define $\sigma_n(t)=\sum_{k=0}^n\mu_k(T)$\nomenclature[F]{$\sigma_n(T)$}{The sum of the $n$ first characteristic values of the operator $T$}. In the case of an infinitesimal of order $1$, one has to expect a divergence
\[
  \sigma_n(T)=O(\ln n).
\]
Following the lemma, we have $\mu_0(T)\geq\mu_1(T)\geq\cdots$. One says that the operator $T$ is \defe{compact}{compact!operator} if $\lim_{n\to\infty}\mu_n(T)=0$.

\begin{lemma}		\label{LemAstAcomAcomp}
	If $A$ is an operator on $\hH$ such that $AA^*$ is compact, then $A$ is compact.
\end{lemma}

\begin{proof}
	No proof.
\end{proof}


\begin{proposition}
Let $T$ be a compact operator on a Hilbert space $\hH$. Then
\begin{enumerate}
\item The spectrum $\sigma(T)$ is discrete and has no limit point other than eventually zero,
\item any non zero element in $\sigma(T)$ is eigenvalue of finite multiplicity.
\end{enumerate}
\end{proposition}
Let us point out that a compact operator has no specially any eigenvalues.

\begin{proposition}
Let $T$ be a compact and self-adjoint operator on the Hilbert space $\hH$. There exists a complete orthonormal basis $\{ \phi_n \}_{n\in\eN}$ of $\hH$ such that $T\phi_n=\lambda_n\phi_n$ and $\lambda_n\to0$ when $n\to\infty$.
\end{proposition}

\begin{proposition}
Let $T$ be a compact operator on $\hH$. Then we have a (norm) uniform convergent expansion
\[
  T=\sum_{n\geq 0}\mu_n(T)\psi_n\langle \phi_i, .\rangle .
\]
where $0\leq\mu_{j+1}(T)\leq\mu_j(T)$ and $\{ \psi_n \}_{n\in\eN}$ is orthogonal to $\{ \phi_n \}_{n\in\eN}$.
\end{proposition}

This proposition allows us to decompose $T$ as
\[
  T=U| T |
\]
where $| T |=\sqrt{T^*T}$. Hence the $\mu_n(T)$ are eigenvalues of $| T |$ and
\[
  \lim_{n\to\infty}\mu_n(T)=0
\]
because $| T |$ is compact and self-adjoint. The $\phi_n$ are the corresponding eigenvectors and
\[
  \psi_n=U\phi_n.
\]
 In this setting, we say that the $\mu_n(T)$ are the \defe{characteristic values}{characteristic!value} of $T$. We have $\mu_0(T)=\| T \|$.

Remark\label{pg_char_inv_U} that the characteristic values $\mu_n(T)$ are invariant under $T\to UT$ when $U$ is unitary. Indeed if $\mu$ is eigenvalue of $T^*T$ with eigenvector $\psi$, then $U^*\psi$ is eigenvector of $U^*TU$ with the same eigenvalue $\mu$.

\begin{proposition}
The operator $T$ is compact if and only if for all $\epsilon>0$, there exists a finite dimensional subspace $E\subset \hH$ such that
\[
  \| T \|_{E^{\perp}}\leq\epsilon.
\]
 \label{prop_comp_ini}
\end{proposition}

\begin{lemma}		\label{LemAmtuBcompaBcm}
	Let $A$ be a compact operator. If $B$ is an invertible operator such that $(A+\mtu)B$ is compact, then $B$ is compact.
\end{lemma}

\begin{probleme}
	The proof is mine; without guarantee.
\end{probleme}

\begin{proof}
	Suppose that $B$ is not compact. There exists a $\sigma>0$ such that for every finite dimensional subspace $G$, we have
	\begin{equation}
		\sup_{\substack{x\in G^{\perp}\\\| x \|=1}}\| Bx \|>\sigma.
	\end{equation}
	Let $\epsilon>0$. Since $A$ is compact, there exists a finite dimensional subspace $F$ such that $\| A \|_{F^{\perp}}<\epsilon$. For $x\in F^{\perp}$, have
	\begin{equation}
		\big\| (A+\mtu)x \big\|\geq\Big| \| Ax \|-\| x \| \Big|=1-\| Ax \|\geq 1-\epsilon,
	\end{equation}
	so that
	\begin{equation}
		\inf_{\substack{x\in F^{\perp}\\\| x \|=1}}\| (A+\mtu)x \|\geq 1-\epsilon.
	\end{equation}
	In the same way from compactness of $(A+\mtu)B$, there exists a finite dimensional subspace $E$ such that
	\begin{equation}
		\sup_{\substack{x\in E^{\perp}\\\| x \|=1}}\| (A+\mtu)Bx \|\leq \epsilon.
	\end{equation}

	Using these properties,
	\begin{equation}
		\begin{aligned}[]
			\epsilon\geq\sup_{\substack{x\in E^{\perp}\\\| x \|=1}}\| (A+\mtu)x \|&\geq\sup_{\substack{x\in E^{\perp}\\ Bx\in F^{\perp}\\\| x \|=1}}\| (A+\mtu)x \|\\
			&=\sup_{\substack{y\in F^{\perp}\\B^{-1}y\in E^{\perp}\\\sigma<\| y \|<\| B_{E^{\perp}} \|}}\| (A+\mtu)y \|\\
			&\geq\inf_{\substack{y\in F^{\perp}\\B^{-1}y\in E^{\perp}\\\sigma<\| y \|<\| B_{E^{\perp}} \|}}\| (A+\mtu)y \|\\
			&\geq\inf_{\substack{y\in F^{\perp}\\\sigma<\| y \|<\| B_{E^{\perp}} \|}}\| (A+\mtu)y \|\\
			&\geq\inf_{\substack{y\in F^{\perp}\\\| y \|=\sigma}}\| (A+\mtu)y \|\\
			&\geq \sigma(1-\epsilon).
		\end{aligned}
	\end{equation}
	It is now sufficient to choose $\epsilon$ is such a way that $\frac{ \epsilon }{ 1-\epsilon }<\sigma$ in order to get a contradiction.
\end{proof}

\begin{theorem}[Spectral theorem]\index{spectral!theorem!compact operators}\index{theorem!spectral!compact operators}
If $T$ is a compact self-adjoint operator on the Hilbert space $\hH$, then there exists an orthogonal basis of $\hH$ of eigenvectors of $T$. The eigenvalues are moreover real and the sequence converges to zero.
\end{theorem}
\begin{proof}
No proof.
\end{proof}
\begin{proposition}
Two other characterisations of compact operators:
\begin{enumerate}
\item the operator $T$ is compact if and only if it is the limit (for the operator norm) of finite rank operators,
\item if $T$ is an operator over $L^2(X)$ and if $T$ can be written under the form
\[
  (Tf)(x)=\int_X k(x,y)f(y)dy,
\]
where $k$ is a square summable function on $X\times X$, then $T$ is compact. Such an operator is said to be \defe{Hilbert-Schmidt}{Hilbert-Schmidt!operator}, and every compact operators over $L^2(X)$ are \emph{not} of that form.
\end{enumerate}
\end{proposition}

Let $T$ be a bounded operator on $\hH$. The \defe{singular values}{singular!value of an operator} of $T$ are defined by
\begin{equation}
\mu_j(T)=\inf_{\dim(V)=j}\sup_{v\perp V}\frac{ \| Tv \| }{ \| v \| }.
\end{equation}
The first singular value gives the operator norm: $\mu_0(T)=\| T \|$.

\begin{proposition}
The operator $T$ is compact if and only if $\lim_{j\to\infty}\mu_j(T)=0$.
\end{proposition}

\begin{lemma}		\label{Lemmulamequ}
If $T$ is a positive\footnote{Notice that, for a positive operator, $\langle Tv, v\rangle \geq 0$, so that $T$ is self-adjoint too.} compact operator and if $(\lambda_j)$ is the sequence of eigenvalues sorted in decreasing order with multiplicity, then
\begin{equation}
\mu_j(T)=\lambda_j(T)
\end{equation}
to the condition that the eigenvalues are numbered from $\lambda_0$ instead of $\lambda_1$.
\end{lemma}

\begin{lemma}	\label{LemIneqscmpborn}
If $T_1$ and $T_2$ are two compact operators and if $S$ is a bounded operator, then
\begin{align*}
\mu_j(T_1+T_2)&\leq \mu_j(T_1)+\mu_j(T_2)\leq\mu_{2j}(T_1+T_2)\\
\mu_j(ST)&\leq\| S \|\mu_j(T)\\
\mu_j(TS)&\leq\| S \|\mu_j(T).
\end{align*}

\end{lemma}
\begin{proof}
No proof.
\end{proof}

Let $\hH$ be an Hilbert space and $\oB(\hH)$ be the set of bounded operators over $\hH$. The \defe{trace class}{trace!class operator} operator ideal is
\begin{equation}
	\oL^1(\hH)=\{ T\in\oB(\hH)\tq \sum\mu_j(T)<\infty \}.
\end{equation}
Such an operator is always compact and the inequalities of lemma~\ref{LemIneqscmpborn} assure that $\oL^1$ is an ideal.

An interesting property is that $\oL^1$ is not norm-closed, actually its norm closure is the full $\oB(\hH)$.

From definition of singular values, if $\{ v_1,\cdots,v_n \}$ is any orthogonal set in $\hH$, then
\begin{equation}	\label{Eqineqstrdav}
  \sum_{j=0}^N| \langle v_j, Tv_j\rangle  |\leq \sum_{j=0}^{N}\mu_j(T).
\end{equation}
So we can give the definition of a trace. If $T\in\oL^1$, the \defe{trace}{trace!of an operator} is given by
\begin{equation}
\tr(T)=\sum_{j=1}^{\infty}\langle v_j, Tv_j\rangle
\end{equation}
where $\{ v_j \}$ is any orthonormal basis of $\hH$. The relation \eqref{Eqineqstrdav} makes the sum absolutely convergent and the independent on the choice of basis, as can see by replacing $v_j$ by $A_{ji}v_i$ and using the fact that $(A^t)_{lj}A_{jk}=\delta_{lk}$.

An interesting property of the trace is
\begin{equation}
\tr(ST)=\tr(TS)
\end{equation}
whenever $S\in\oB(\hH)$ and $T\in\oL^1(\hH)$.

\subsection{Hilbert-Schmidt operators}\index{Hilbert-Schmidt!operator}
%--------------------------------------------------------

If $S$ and $T$ are Hilbert-Schmidt operators, one can show that $ST$ is a trace class operator. An operator $T$ over $L^2(X)$ is Hilbert-Schmidt if and only if $\sum\mu_j(T)^2<\infty$.

\begin{lemma}
Let $M$ be a closed manifold endowed with a smooth measure. If $k\in C^{\infty}(M)$, then the operator defined by
\[
  (Tf)(x)=\int_Mk(x,y)f(y)dy
\]
is a trace class operator and we have
\begin{equation}
\tr(T)=\int_Mk(x,x)dx.
\end{equation}

\end{lemma}


\subsection{The Schatten-von Neumann ideal}
%-------------------------------------------

The \defe{Schatten-von Neumann ideal}{Schatten-von Neumann ideal} is the set
\[
  \oL^p(\hH)=\{ \textrm{compact operator } T\tq \sum_{n=0}^{\infty}\mu_n(T)^p<\infty \}
\]
Interesting properties of this set (including the fact that it is an ideal) are proven by virtue of \defe{Hölder inequality}{hölder inequality}: when $p$ and $q$ are reals such that $\frac{1}{ p }+\frac{1}{ q }=1$, we have
\begin{equation}
\sum_{k}| u_k | |v_k |\leq \Big( \sum_{k}| u_k |^p \Big)^{1/p}\Big( \sum_{k}| v_k |^{q} \Big)^{1/q}.
\end{equation}
There also exists an integral version:
\begin{equation}
\int | fg |\leq \Big( \int | f |^p \Big)^{1/p}\Big( \int| g |^q \Big)^{1/q}.
\end{equation}

\begin{probleme}
	We should find a precise statement with precise hypothesis for that inequality.
\end{probleme}

One can prove that when $\sum_{j=1}^{k}\frac{1}{ p_j }=\frac{1}{ q }$, we have
\begin{equation}    \label{EqPropLLLsvn}
\oL^{p_1}\oL^{p_2}\ldots\oL^{p_k}\subset \oL^q.
\end{equation}

\begin{lemma}
The space $\oL^q(\hH)$ is a left ideal.
\end{lemma}

\begin{proof}
Let $T\in\oL^q(\hH)$: $\sum_n\mu_n(T)^q<\infty$.  If $a$ is an other linear operator on $\hH$, $\mu_n(aT)=\inf\{ \| aT-R \|\,;\Rank(R)\leq n \}=\inf\{ \| aT-aR \|;\Rank(R)\leq n \}$ because $\Rank(aR)$ and $\Rank(a^{-1}R)$ are always lower than $\Rank(T)$. Thus $\mu_n\leq\inf\{ \| a \|\| T-R \|;\Rank(R)\leq n \}=\| a \|\mu_n(T)$.
\end{proof}

The \defe{trace}{trace!of an operator} of an operator is defined on $T\in\oL(\hH)$ by
\begin{equation}
\tr(T)=\sum_{n}\langle T\xi_n,\,\xi_n\rangle
\end{equation}
where $\xi_n$ runs over an orthonormal basis. One can prove that $\tr(T)$ does not depend on the choice of this basis.

\begin{proposition}
When $T$ is compact and positive, on has
\[
  \tr(T)=\sum_{n}\mu_n(T).
\]

\end{proposition}
\begin{proof}
No proof.
\end{proof}

I found the following definitions in \cite{OlafPostDissertation}.
\begin{definition}
	A sesquilinear\index{sesquilinear!form} form $q$ on an Hilbert space $\hH$ is \defe{symmetric}{symmetric!sesquilinear form} if $q(u,v)=\overline{ q(v,u) }$ or, equivalently, if $q(u,u)\in\eR$ for every $u,v\in\Domain(q)$. It is \defe{positive}{positive!sesquilinear form} if $q(u)=q(u,u)\geq 0$ for every $u\in\Domain(q)$.
\end{definition}
The space $\Domain(q)$ is endowed by the inner product
\begin{equation}		\label{EqInnerProdqDomainsq}
	\langle u, v\rangle_q=\langle u, v\rangle_{\hH}+q(u,v).
\end{equation}
Most of time, when we speak about topology on $\Domain(q)$, we are speaking about the topology of that norm. The form $q$ is \defe{closed}{closed!sesquilinear form} is $\Domain(q)$ is complete (for the topology of the inner product \eqref{EqInnerProdqDomainsq}). In that case, $\Domain(q)$ is itself an Hilbert space.

\begin{definition}		\label{DefFormCoreDomq}
	A set $D\subset\Domain(q)$ which is $q$-norm-dense in $\Domain(q)$ is a \defe{form core}{form core} for $q$.
\end{definition}

\begin{definition}
    Consider the algebra of bounded operators on an Hilbert space \( \hH\). Let \( \Spec(T)\) be the spectrum of \( T\).

    The \defe{point spectrum}{spectrum!point}\index{point!spectrum} of \( T\), \( \Spec_P(T)\), is the set of eigenvalues. This is the set of \( \lambda\in\eC\) such that \( T-\lambda\mtu\) is not injective.

    The \defe{continuous spectrum}{spectrum!continuous}\index{continuous!spectrum}, \( \Spec_C(T)\), is the subset of its spectrum given by the values \( \lambda\) such that \( T-\lambda\mtu\) is injective but for which the image of \( T-\lambda\mtu\) is a dense proper subspace of \( \hH\).

    The \defe{residual spectrum}{spectrum!residual}\index{residual spectrum}, $\Spec_R(T)$, is the part of the spectrum that remains. So \( \lambda\in\Spec_R(T)\) if \( T-\lambda\mtu\) is injective and the closure \( \overline{ \Image(T-\lambda\mtu) }\) is a proper subspace of \( \hH\).
\end{definition}

%---------------------------------------------------------------------------------------------------------------------------
\subsection{Normal operators on Hilbert space}
%---------------------------------------------------------------------------------------------------------------------------

Many properties of normal operators can be found in \cite{AndrewGreen}.

\begin{definition}  \label{DefFQFKZbB}
    An operator \( T\) on an Hilbert space is said to be \defe{normal}{normal!operator} if \( T^*T=TT^*\)
\end{definition}
In the setting of \( C^*\)-algebras we will define the same kind of normal element, see definition~\ref{DefElemNormal}.

\begin{proposition}
    If \( T\) is a bounded normal operator then \( T-\lambda\mtu\) is a bounded normal operator for every \( \lambda\in\eC\).
\end{proposition}

\begin{proof}
    An immediate computation shows that \( (T-\lambda\mtu)(T^*-\bar\lambda\mtu)=(T^*-\bar\lambda\mtu)(T-\lambda\mtu)\), so \( T-\lambda\mtu\) is normal. In order to see that \( T-\lambda\mtu\) is bounded,
    \begin{subequations}
        \begin{align}
            \| T-\lambda\mtu \|&=\sup_{h\in\hH}\frac{ \| Th-\lambda h \| }{ \| h \| }\\
            &\leq\sup_{h\in\hH}\frac{ \| Th \|+| \lambda|\| h \| }{ \| h \| }\\
            &\leq\sup_{h\in\hH}\frac{ \| Th \| }{ \| h \| }+| \lambda |\\
            &=\| T \|+| \lambda |.
        \end{align}
    \end{subequations}
    The last line is bounded because \( T\) is bounded.
\end{proof}

\begin{proposition}     \label{PropoCartactNormal}
    An operator \( T\) is normal if and only if \( \| Tx \|=\| T^*x \|\) for every \( x\in\hH\).
\end{proposition}

\begin{proposition}
    If \( T\) is diagonalizable by an unitary operator, then it is normal.
\end{proposition}

\begin{proof}
    Let \( U\) be unitary and \( D\) be diagonal such that
    \begin{equation}
        UTU^*=D.
    \end{equation}
    Then we have \( (UTU^*)(UTU^*)^*=DD^*\) and thus
    \begin{equation}
        UTT^*U^*=DD^*.
    \end{equation}
    In the same time,
    \begin{equation}
        DD^*=D^*D=UT^*TU^*,
    \end{equation}
    so we have \( UTT^*U^*=UT^*TU^*\) and then \( TT^*=T^*T\) because \( U\) and \( U^*\) are invertible.
\end{proof}

\begin{proposition}
    If \( T\) is normal and bounded, the residual spectrum \( \Spec_R(T)\) is empty.
\end{proposition}

\begin{proof}
    See \cite{AndrewGreen} at page 20.
\end{proof}

\begin{proposition}
    For a normal operator we have
    \begin{equation}
        \| T \|=r(T)
    \end{equation}
    where \( r(T)\) is the spectral radius of \( T\).
\end{proposition}

\begin{proof}
    A proof is available \wikipedia{fr}{Endomorphisme_normal}{on wikipedia}.
\end{proof}

\begin{proposition}
    If \( T\) is a normal operator and if \( \lambda\) is an eigenvalue of \( T\), then \( \bar\lambda\) is an eigenvalue of \( T^*\) for the same eigenvector.
\end{proposition}

\begin{proof}
    If \( v\) is eigenvector of \( T\) for the eigenvalue \( \lambda\), using proposition~\ref{PropoCartactNormal} on the normal operator \( T-\lambda\mtu\) we have
    \begin{equation}
        0=\| (T-\lambda\mtu)v \|=\| (T-\lambda\mtu)^*v \|.
    \end{equation}
    Thus \( (T^*-\bar\lambda \mtu)v=0\) and \( v\) is eigenvector of \( T^*\) for the eigenvalue \( \bar\lambda\).
\end{proof}

\section{Spectral theory on Banach algebras}		\label{Sec_SpecBanach}
%++++++++++++++++++++++++++++++++++++++++++++

\begin{definition}      \label{DefInvolutionALge}
    An \defe{involution}{involution!on algebra} on an algebra $\cA$ is a $\eR$-linear map $*\colon \cA\to \cA$ which fulfils
    \begin{subequations}
    \begin{align}
      A^{**}&=A,\\
        (AB)^*&=B^*A^*\\
        (\lambda A)^*&=\overline{\lambda }A^*
    \end{align}
    \end{subequations}
    where $\lambda$ is any complex number. An algebra endowed with an involution is a $*$-algebra.
\end{definition}

\begin{remark}
    In the setting of Hopf algebras, we do not require \( A^{**}=A\). In that we follow \cite{SoibelmanI}; see subsection~\ref{subsecHopfInvolution}.
\end{remark}

\begin{definition}
The \defe{operator norm}{norm of an operator} of a linear map $\dpt{A}{\cB}{\cB}$ on a Banach space is
\[
   \|A\|=\sup_{\|v\|=1}\|Av\|\in\eR.
\]
The operator $A$ is \defe{bounded}{bounded!operator} if its norm is finite. \label{def:normappl}
\end{definition}

A classical result is
\begin{proposition}
A linear operator on a Banach space is bounded if and only if it is continuous.\label{prop:cont_born}
\end{proposition}

\begin{proposition}
The space $\oB(\cB)$\nomenclature[F]{$\cB$}{Space of bounded operators on a Banach space} of bounded operators on the Banach space $\cB$ endowed with the norm operator is a Banach space.
\end{proposition}

The norm of a functional is defined by
\[
   \|\rho\|:=\sup\{ |\rho(v)|:v\in\cB,\|v\|=1  \}
\]
It is the smallest $c$ that can be used in the definition of the continuity.

We denote by $\cB^*$ the dual of the Banach space $\cB$. It is the set of all the functionals on $\cB$ and it is itself a Banach space.

\begin{theorem}[Hahn-Banach]\index{Hahn-Banach theorem} \label{tho:hahnBanach}
Any functional defined on a linear subspace $\cB_0$ of $\cB$ can be extended to a functional of same norm defined on the whole $\cB$. In particular, if $\rho(v)=0$ for all $\rho\in\cB^*$, then $v=0$.
\end{theorem}



The set of self-adjoint operators of a $C^*$-algebra $\cA$ is written as
\[
  \cA_{\eR}=\{A\in\cA\tq A^*=A\}.
\]
As notation, we denotes by $G(\cA)$ the set of invertible elements of $\cA$:
\[
   G(\cA):=\{A\in\cA|A^{-1}\text{ exists}\}
\]

\begin{lemma}
A Banach $*$-algebra with $\|A\|^2\leq\|A^*A\|$ for all $A$ is a $C^*$-algebra.\label{lem:STARAlC}
\end{lemma}

\begin{proof}
The definition is that a $C^*$-algebra is a Banach $*$-algebra such that $\|A^*A\|=\|A\|^2$, then we have to show that  if $A$ belongs to a Banach $*$-algebra, then $\|A\|^2\leq\|A^*A\|$ implies $\|A\|^2=\|A^*A\|$. In a Banach algebra, $\|A\|^2\leq\|A^*A\|\leq\|A^*\|\|A\|$ so that $\|A\|\leq\|A^*\|$. The same with $A^*$ instead of $A$ gives the inverse inequality. Then $\|A\|=\|A^*\|$.
\end{proof}

\begin{definition}
An \defe{unit}{unit!in a Banach algebra} $\cA$ is an element $\cun$ such that $\|\cun\|=1$ and $\cun A=A\cun=A$ for all $A\in\cA$. A Banach algebra which contains an unit is \defe{unital}{}. When $z\in\eC$, we often write $z$ instead of $z\cun$.
\end{definition}

Note that in a $C^*$-algebra, the definition of an unit don't impose the norm because definition of a $C^*$-algebra applied to $A=\cun$ automatically gives $\|\cun\|=1$.

Let $\cA$ be a Banach algebra without unit. We define $\cA_{\cun}:=\cA\oplus\eC$ with the notation $A+\lambda\cun$ for $(A,\cun)$. We enforce an algebra structure on this set following the natural way:
\[
  (A+\lambda\cun)(B+\mu\cun):=AB+\lambda B+\mu A+\lambda\mu\cun,
\]
in other terms, $1\in\eC$ is assimilated to the $\cun$ of $\cA_{\cun}$. The norm on $\cA_{\cun}$ is defined by
\[
   \|(A+\lambda\cun)(B+\mu\cun)\|\leq\|(A+\lambda\cun)\|\|(B+\mu\cun)|
\]
So $\cA_{\cun}$ becomes an unital Banach algebra. We have the following:

\begin{proposition}
For each Banach algebra without unit, there exists an unital Banach algebra $\cA_{\cun}$ and an isometric morphism $\cA\to\cA_{\cun}$ such that $\frac{\cA_{\cun}}{\cA}\simeq\eC$.
\end{proposition}

Note that such an unitization of a Banach algebra is not unique see page \pageref{pg:unit_nonunic} and \cite{Landsman} page 16 and proposition 2.4.6.


%+++++++++++++++++++++++++++++++++++++++++++++++++++++++++++++++++++++++++++++++++++++++++++++++++++++++++++++++++++++++++++
\section{Spectral theorem and some consequences}\label{pg_spectralthe}
%++++++++++++++++++++++++++++++++++++++++++++++++++++++++++++++++++++++++

%---------------------------------------------------------------------------------------------------------------------------
\subsection{Spectrum}
%---------------------------------------------------------------------------------------------------------------------------

It is well know that a norm on a set gives rise to a topology. A normed vector space which is complete in the norm topology a \defe{Banach space}{Banach!space}. A \defe{Banach algebra}{Banach!algebra} is a Banach space equipped with an algebra structure such that
\begin{equation} \label{eq:normBanach}
  \|AB\|\leq\|A\|B\|
\end{equation}
for all $A,B$ in the algebra. \label{def_banach}

\begin{definition}			\label{def:fonctionelle}
A \defe{functional}{functional} on a Banach space $\cA$ is a continuous linear map $\dpt{\rho}{\cA}{\eC}$.
\end{definition}
We say that $\rho$ is \defe{continuous}{continuous!functional on Banach space} when there exists $c\in\eR$ such that $\forall v\in\cA$,  $|\rho(v)|\leq c\|v\|$.  We recall that, for linear maps, continuity is equivalent to boundedness. In fact we have the following \cite{Ops_Hilb_space}.
\begin{proposition}
	For a map $\rho\colon \hH_1\to \hH_2$ between two Hilbert space, the following are equivalent:
	\begin{enumerate}
	\item $\rho$ is continuous at $0$,
	\item $\rho$ is continuous,
	\item $\rho$ is bounded.
	\end{enumerate}
\end{proposition}

\begin{definition}
	Let $\cA$ be an unital Banach algebra. The \defe{resolvent}{resolvent}\nomenclature[F]{$\rho(A)$}{Resolvent of $A$} of an element $A\in \cA$ is
	\begin{equation}
	  \rho(A)=\{ z\in\eC\tq(A-z\mtu)^{-1}\text{exists as two-sided inverse}  \}.
	\end{equation}
	The \defe{spectrum}{spectrum}\nomenclature[F]{$\sigma(A)$}{Spectrum of $A$} $\sigma(A)$, or $\Spec(A)$\nomenclature[F]{$\Spec(A)$}{Spectrum of $A$}, is the complement of $\rho(A)$ in $\eC$:
	\begin{equation}
        \Spec(A)=\{ \lambda\in\eC\tq(A-\lambda) \text{ is not invertible in the algebra} \}.
	\end{equation}
\end{definition}
The spectrum of $A$ is sometimes written $\sigma(A)$.


Notice that in an algebra of polynomials, the spectrum of an element is is almost always $\eC$. The \defe{spectral radius}{spectral!radius}\nomenclature[F]{$r(A)$}{Spectral radius of $A$} of $A\in\cA$ is
\begin{equation}
   r(A)=\sup_{z\in\Spec(A)}|z|.
\end{equation} \label{def:spectre}
In the finite dimensional case, the spectrum is equal to the set of eigenvalues.

\begin{remark}
When $\cA$ has no unit, the spectrum and the resolvent are defined by considering the unitization $\cA_{\mtu}:=\cA\oplus\eC$ instead of $\cA$.
\end{remark}

%---------------------------------------------------------------------------------------------------------------------------
\subsection{Note about operator algebras}
%---------------------------------------------------------------------------------------------------------------------------

This subsection comes \wikipedia{en}{Decomposition_of_spectrum_(functional_analysis)}{from wikipedia}. Read it for more informations.

In this subsection we consider an algebra of operators on \( H\). We say that \( \lambda\) is an \defe{eigenvalue}{eigenvalue} of \( A\) if there exists \( h\in H\) such that
\begin{equation}
    Ah=\lambda h.
\end{equation}
This implies that \( \lambda\) belongs to the spectrum of \( A\). The contrary is in general not true. If for instance we consider the algebra of \emph{bounded} operators on \( H\), it can happen that \( A-\lambda\mtu\) is invertible but with non bounded inverse. In this case, \( \lambda\) belongs to the spectrum (because \( A-\lambda\mtu\) is not invertible \emph{in the algebra}) but is not an eigenvalue.

The set of eigenvalues is called the \defe{pure point spectrum}{spectrum!pure point}. This is in general a part of the spectrum.

An operator $A\in\oB(\hH)$ is \defe{positive}{positive!operator} if $\langle Av, v\rangle \geq 0$ for every $v\in \hH$. Notice that $A^*A$ is positive for every operator $A$ because $\langle A^*Av, v\rangle =\langle Av, Av\rangle =\| Av \|^2\geq 0$. We will see in theorem~\ref{ThoElsPositifsBBstar} that, in the case of unital $C^*$-algebra, all the positive elements are of that form.

%---------------------------------------------------------------------------------------------------------------------------
\subsection{Spectrum: next steps}
%---------------------------------------------------------------------------------------------------------------------------

\begin{lemma}
The spectrum of a self-adjoint operator is a compact subset of $\eR^+$.
\end{lemma}

\begin{lemma}[\cite{Landsman}]\label{lem:cv_Ak}
Let $\cA$ be an unital Banach algebra, and $A\in\cA$. Then the formula
\begin{equation}
   \lim_{N\to\infty}\sum_{k=0}^{N}A^k = (\mtu-A)^{-1}
\end{equation}
holds if $\|A\|< 1$,. As a consequence, $(A-z\mtu)^{-1}$ exists when $|z|>\|A\|$.
\end{lemma}

\begin{proof}
	Since $\cA$ is complete, it is sufficient for convergence to prove the fact that the sequence of partial sums is Cauchy. Suppose $n>m$ and compute:
	\[
	  \| \sum_{k=0}^nA^k-\sum_{k=0}^mA^k  \|=\| \sum_{k=m+1}^nA^k \|\leq\sum_{k=m+1}^n\|A^k\|
	     \leq\sum_{k=m+1}^n\|A\|^k.
	\]
	The last inequality comes from the fact that $\|AB\|\leq\|A\|\|B\|$. From the theory of the geometric series, we know that the last sum converges to zero when $n,m\to\infty$ because $\|A\|<1$.
	Then
	$   \sum_{k=0}^{\infty}A^k\in\cA$.
	Remains to check that this is the inverse of $(\mtu-A)$:
	\[
	   \sum_{k=0}^nA^k(\mtu-A)=\sum_{k=0}^n(A^k-A^{k+1})=\mtu-A^{n+1},
	\]
	so
	\[
	   \|\mtu-\sum_{k=0}^nA^k(\mtu-A)   \|=\|A^{n+1}\|\leq\|A\|^{n+1}.
	\]
	Since $\|A\|< 1$, the limit $n\to\infty$ of the right hand side is zero. The conclusion is that $\sum_{k=0}^{\infty}A^k$ is a left inverse of $(\mtu-A)$. The same shows that is is also a right inverse. This proves the first statement. The second is immediate by working with $A/z$ instead of $A$.

\end{proof}

\begin{lemma}
The set of invertible elements
\[
   G(\cA):=\{A\in\cA : A^{-1}\,\text{exists}\}
\]
is open in $\cA$\nomenclature{$G(\cA)$}{Set of invertible elements in $\cA$}
\end{lemma} \label{lem:G_ouvert}

\begin{proof}
	Let us consider a $A\in G(\cA)$ and $B\in\cA$ such that $\|B\|<\|A^{-1}\|^{-1}$. By definition~\ref{def_banach}, we have $\|A^{-1} B\|\leq\|A^{-1}\|\|B\|<1$. Thus $A+B=A(\mtu+A^{-1} B)$ has an inverse because $A$ and $(\mtu+A^{-1} B)$ have both an inverse ($\|A^{-1} B\|<1$ and lemma~\ref{lem:cv_Ak}).

	Thus, when $A$ has an inverse, the element $A+B$ also has a one when $\|B\|$ is not too big. In other words: any $C\in\cA$ such that $\|A-C\|<\epsilon$ is in $G(\cA)$ when $\epsilon\leq\|A^{-1}\|^{-1}$.
\end{proof}

Now we can prove a great and fundamental theorem.

\begin{theorem}			\label{ThoSpecBanach}
    For any Banach algebra $\cA$ and any element $A\in\cA$, the spectrum $\sigma(A)$ is
    \begin{enumerate}
        \item       \label{ItemThoSpecBanachi}
            a subset of $\{z\in\eC:|z|\leq\|A\|\}$,
        \item
            compact,
        \item
            non empty
        \end{enumerate}
\end{theorem} \label{tho:prop_sigma}

\begin{proof}
	If $\cA$ has no unit, we can add one and then one can suppose $\cA$ to be unital.

	The first point is contained in lemma~\ref{lem:cv_Ak}. Thus it is bounded and, in order to prove that it is compact, we only need to prove that it is closed.

	We consider the map $\dpt{f}{\eC}{\cA}$ , $f(z)=z\mtu-A$. Clearly, $\|f(z+\delta)-f(z)\|=\delta$, then $f$ is continuous\footnote{We consider the topology induced by the metric.}. Since $G(\cA)$ is open by the lemma~\ref{lem:G_ouvert}, continuity makes $f^{-1}(G(\cA))$ open in $\eC$. But this set is exactly the set of $z\in\eC$ such that $z-A$ has an inverse: $\rho(A)$. The complement $\sigma(A)$ is thus closed and therefore compact.

	Now we show that $\sigma(A)$ is non-empty. For this, we begin by defining $\dpt{g}{\rho(A)}{\cA}$,
	$g(z):=(z-A)^{-1}$ (which is well defined by definition of $ \rho(A)$). Now, pick a $z_0\in\rho(A)$ and a $z\in\eC$ such that
	\[
	   |z-z_0|<\| (A-z_0)^{-1} \|^{-1}.
	\]
	Since $\rho(A)$ is open, we can choose $z\in\rho(A)$. Note that $(z_0-A)^{-1}$ is a two-sided inverse because $z_0\in\rho(A)$. Since $\| (z_0-z)(z_0-A)^{-1} \|=|z_0-z|\| (z_0-A)^{-1} \|<1$,  lemma~\ref{lem:cv_Ak} assures the convergence and makes sense to the following computation:
	\begin{equation}
	\begin{split}
	   \frac{1}{z_0-A}\sum_{k=0}^{\infty}(\frac{z_0-z}{z_0-A})^k
	      &=   (z_0-A)^{-1}\left(   [(z_0-A)-(z_0-z)](z_0-A)^{-1}
	       \right)^{-1}\\
	      &=(z-A) ^{-1}\\
	      &=g(z).
	\end{split}
	\end{equation}
	Then $g(z)=\sum_{k=0}^{\infty}(z_0-z)^k(z_0-A)^{k-1}$ is a norm-converging power series with respect to $z$. Now, assume $z\neq 0$. We have $g(z)=z^{-1}(\mtu-A/z)^{-1}$, then
	\begin{equation}\label{eq:cv_de_g}
	   \|g(z)\|=|z|^{-1}\| (\mtu-A/z)^{-1} \|\to 0
	\end{equation}
	when $z\to\infty$.

	Let us now consider a functional $\rho\in\cA^*$. Recall that, by definition, it is bounded, and define
	$\dpt{g_{\rho}}{\eC}{\eC}$,
	$g_{\rho}(z):=\rho(g(z))$. The limit \eqref{eq:cv_de_g} makes
	\[
	\lim_{z\to\infty}|g_{\rho}(z)|\leq\lim_{z\to\infty}c\|g(z)\|=0
	\]
	where $c$ is the constant of definition~\ref{def:fonctionelle}. Then
	\begin{equation}\label{eq:limite_g_rho}
	  \lim_{z\to\infty}g_{\rho}(z)=0.
	\end{equation}
	Because $\lim_{z\to\infty}| \rho_g(z) |\leq\lim_{z\to\infty}\| g(z) \|=0$.

	Suppose that $\sigma(A)=\varnothing$, or $\rho(A)=\eC$; the function $g$ is then defined on the whole $\eC$. More precisely, $g_{\rho}$ is an analytic complex function whose vanishes at infinity, then Liouville's theorem makes it constant\quext{As far as I remember, holomorphic equals analytic.}. Due to equation \eqref{eq:limite_g_rho}, for every functional $\rho$,
	\[
	   \rho\big(g(z)\big)=0\quad\forall z\in\eC.
	\]
	This yields $g(z)=0$ for any $z$ in $\eC$, but it is not possible. Thus we are leads to $\rho(A)\neq\eC$ and thus $\sigma(A)\neq\varnothing$.
\end{proof}

\begin{theorem}
The spectrum of any element in a Banach algebra is
\begin{enumerate}
\item non empty,
\item compact.
\end{enumerate}
\end{theorem}

\begin{proof}
	Let $\cA$ be a Banach algebra and $A\in\cA$, and define $r_{\lambda}=(A-\lambda)^{-1}$. Formally, we have
	\[
	  r_{\lambda}=-\lambda^{-1}\left( 1-\frac{ A }{ \lambda } \right)^{-1}=-\lambda^{-1}\left( 1+\frac{ A }{ \lambda }+\frac{ A^2 }{ \lambda^2 }+\ldots \right).
	\]
	We have to study in which case does that expansion make sense, and check that it actually is an inverse of $(A-\lambda)$.

	When $| \lambda |>\| A \|$, that series is absolutely convergent thanks to the condition \eqref{eq:normBanach}, and one can check that it is an inverse. We conclude that, when $| \lambda |>\| A \|$, the expression $(A-\lambda)$ is invertible.

	Now, the algebraic identity $a^{-1}-b^{-1}=a^{-1}(b-a)b^{-1}$ reads $r_{\lambda}-r_{\mu}=r_{\lambda}(\lambda-\mu)r_{\mu}$, from which we deduce an expression for $r_{\lambda}$ in terms of $r_{\lambda}$ and $r_{\mu}$:
	\[
	  r_{\lambda}=r_{\mu}+r_{\lambda}(\lambda-\mu)r_{\mu}.
	\]
	If one substitutes that expression into itself, we find $r_{\lambda}=r_{\mu}+r_{\mu}(\lambda-\mu)r_{\mu}+r_{\lambda}(\lambda-\mu)r_{\mu}(\lambda-\mu)r_{\mu}$. Making the same again and again provides the following expansion:
	\begin{equation}		\label{EqDevrllambu}
		r_{\lambda}=r_{\mu}+(\lambda-\mu)r_{\mu}^2+(\lambda-\mu)^2r_{\mu}^3+\ldots
	\end{equation}
	So if $r_{\mu}$ exists (i.e. $\mu\notin\Spec(A)$), we want to define $r_{\lambda}$ by that formula. Now, if $| \lambda-\mu |\| r_{\mu} \|<1$, then $\lambda\notin\Spec(A)$ because formula \eqref{EqDevrllambu} actually works. That proves that the set $\eC\setminus\Spec(A)$ is open, and then that $\Spec(A)$ is closed. Since it is bounded too, we conclude that, being a part of $\eC^2$, the set $\Spec(A)$ is compact.

	Let us now prove that the map $\lambda\mapsto r_{\lambda}$ is continuous. Indeed, consider $r_{\lambda}-r_{\mu}j$, and compute the difference using the series \eqref{EqDevrllambu}. One sees that it converges to $0$ when $\lambda\to\mu$. We know that $(A-\lambda)$ is invertible when $| \lambda |>\| A \|$, so that it makes sense to compute the limit of $\| r_{\lambda} \|$ when $\lambda$ goes to infinity. The limit of $\| r_{\lambda} \|$ when $\lambda\to 0$ is $0$. The map $\lambda\mapsto r_{\lambda}$ is a differentiable map over $\eC$ and must then be constant, so that it must be zero everywhere, because its limit is zero.

	We deduce that $\Spec(A)\neq\emptyset$.

	\begin{probleme}
		Some questions to be elucidated:
	\begin{itemize}
	\item In order to prove that $\| r_{\lambda} \|\to 0$ when $\lambda\to\infty$, one has to say that the norm of $r_{\lambda}$ is the invert of the one of $(A-\lambda)$,  and that the latter goes to infinity when $\lambda$ goes to infinity.
	\item The lase deduction is not clear at all, but it is done in greater details in theorem~\ref{ThoSpecBanach}. To be merged.
	\end{itemize}

	\end{probleme}
\end{proof}



\begin{corollary}[Theorem of Gelfand-Mazur]
If all elements (except zero) of an unital Banach algebra $\cA$ are invertible, thus $\cA\simeq\eC$ as Banach algebras.
\label{cor:GelfandMazur}
\end{corollary}

\begin{proof}
	We just saw that $\sigma(A)\neq\emptyset$, thus for all non-zero element $A\in\cA$, there exists a $z_A\in\eC$ such that $A-z_A\cun$ is not invertible. From assumption, $A-z_A\cun=0$, so that $A\to z_A$ is the expected isomorphism. Since $\|A\|=\|z_A\cun\|=|z_A|$, this isomorphism is isometric.
\end{proof}

As a corollary, we have that
\begin{equation}
r(A)\leq\|A\|.
\end{equation}

The following version of the spectral theorem is known as the \emph{continuous functional calculus}.
\begin{theorem}[First version of the spectral theorem]\index{spectral!theorem!selfadjoint operators}			\label{ThoSpectralTho}
Let $\hH$ be an Hilbert space, and $T\in\oB(\hH)$. If $T$ is self-adjoint, there is an algebra isomorphism
\[
  C^*(T,\mtu)\stackrel{\simeq}{\longrightarrow}C\big( \Spec(T) \big)
\]
which maps $T$ to the identity function. That isomorphism is unique and continuous from the norm topology to the topology of uniform convergence. If we denote by $f(T)$ the image of $f\in C\big( \Spec(T) \big)$ by the isomorphism, we have the following properties
\begin{enumerate}
\item $f(1)=\id$,
\item $f(T)^*=\overline{ f }(T)$,
\item $\| f(T) \|=\| f \|=\sup\{ | f(x) |,\,x\in\Spec(T) \}$,
\item\label{ItemSpecffSpecThoSpectral} $\Spec\big( f(T) \big)=f\big( \Spec(T) \big)$,
\item $f(T)\geq 0$ if and only if $f\geq 0$.
\end{enumerate}
\end{theorem}

The map $f\mapsto f(T)$ is the \defe{continuous functional calculus}{continuous!functional calculus!selfadjoint operators}\index{functional calculus!continuous}.

\begin{proof}

	We recall that, by theorem~\ref{ThoSpecBanach}, the set $\Spec(T)$ is non empty and compact.

	For the second claim, first suppose that $\dim\hH$ is finite. Then there exists an orthonormal basis $\{ v_1,\ldots,v_n \}$ such that $Tv_j=\lambda_jv_j$, and $\Spec(T)=\{ \lambda_1,\ldots,\lambda_n \}$. Define $f(T)v_j=f(\lambda_j)v_j$. Properties of that map are easy to prove.

	In the infinite dimensional case, the proof is again to build an orthonormal basis of eigenvectors of $T$. We refer to \cite{Wassermann,Landsman} for a proof.
\end{proof}

\begin{corollary}
Every positive operator has an unique positive square root\index{square root of an operator}.
\end{corollary}

\begin{proof}
	Let $A\geq 0$ and $B\in C^*(A,\mtu)$ be the element associated with the square root function on $C\big(\Spec(A))$ by the spectral theorem. Then $B^2=A$ because the correspondence is a morphism and $A$ correspond to the identity. The fact that $B$ is positive comes from the fact that $B=\left( \sqrt[4]{A} \right)^2$.
\end{proof}

Taking the square root of the positive operator $T^*T$, we define the \defe{absolute value}{absolute value of an operator} of the operator $T$\nomenclature[F]{$| T |$}{Absolute value of an operator} by
\begin{equation}		\label{EqAbsValT}
	| T |=\sqrt{T^*T}.
\end{equation}

\begin{lemma}		\label{LemkerTkersqrtT}
We have
\[
 \ker(T)=\ker\big( \sqrt{T^*T} \big),
\]
for every $T\in\oB(\hH)$.
\end{lemma}

\begin{proof}
	Taking the adjoint term by term in the development of $f(t)=\sqrt{t}$, we see that
	\[
	  \Big( (T^*T)^{1/2} \Big)^*=(T^*T)^{1/2},
	\]
	then a vector $v\in\ker(T)$ satisfies
	\[
	  0=\langle Tv, Tv\rangle =\langle v, T^*Tv\rangle =\langle v,\sqrt{T^*T}\sqrt{T^*T}v \rangle =\langle \sqrt{T^*T}v, \sqrt{T^*T}v\rangle,
	\]
	which means that $v\in\ker\big( \sqrt{T^*T} \big)$.
\end{proof}



\begin{lemma}
Let $p$ be a polynomial on $\eC$, if we define
\[
   p(\sigma(A)):=\{ p(z):z\in\sigma(1) \},
\]
then we have $p(\sigma(A))=\sigma(p(A))$. \label{lem:sigma_poly}
\end{lemma}

\begin{proof}
	Let us consider $z,\alpha\in\eC$ and write the factorisations:
	\begin{subequations}
	\begin{align}
	  p(z)-\alpha&=c\prod_{i=1}^{n}(z-\beta_i(\alpha))\label{eq:prod_1}\\
	  p(A)-\alpha\mtu&=c\prod_{i=1}^{n}(A-\beta_i(\alpha)\mtu)\label{eq:prod_2},
	\end{align}
	\end{subequations}
	where --of course-- $c$ and $\beta_i$ are determined by $p$ and $\alpha$.

	Now, particularise to $\alpha\in\rho(p(A))$: $p(A)-\alpha\mtu$ is invertible and thus each of $A-\beta_i(\alpha)\mtu$ is too. In other words (taking the complement) $\alpha\in\sigma(p(A))$ implies that at least one out of $A-\beta_i(\alpha)\mtu$ is not invertible: $\beta_i(\alpha)\in\sigma(A)$ for one of the $i$. On the other hand, by definition $p(\beta_i)-\alpha=0$, then $\alpha\in p(\sigma(A))$. All in all, we have shown that $\sigma(p(A))\subset p(\sigma(A))$.

	For the inverse inclusion, take $\alpha\in p(\sigma(A))$, i.e. $\alpha=p(z)$ for some $z\in\sigma(A)$. In this case the product \eqref{eq:prod_1} is zero and thus $z$ is one of the $\beta_i(z)$. For this $i$, $\beta_i(\alpha)\in\sigma(A)$, then $A-\beta_i(\alpha)$ is non-invertible. By \eqref{eq:prod_2}, $p(A)-\alpha\mtu$ is also non-invertible, and thus $\alpha\in\sigma(p(A))$.

\end{proof}

\begin{lemma}[Polar decomposition]		\label{LemPolarHilbert}
	Every operator $A\in \oB(\hH)$ has a \defe{polar decomposition}{polar!decomposition!operator on Hilbert space} $A=U| A |$ where $| A |=\sqrt{A^*A}$ and $U$ is a partial isometry with the same kernel as $A$.
\end{lemma}

See \cite{Landsman} for a proof. A version of this decomposition in von~Neumann algebras is given in proposition~\ref{PropPolarvNA}. The operator $\sqrt{A^*A}$ is defined by means of the spectral theorem~\ref{ThoSpectralTho}, see equation \eqref{EqAbsValT}.

The partial isometry $U$ is sometimes called the \defe{sign}{sign of an operator} of $A$ because when it is selfadjoint, it is the operator associated with the sign function by the continuous functional calculus. Indeed, let
\begin{equation}
	\varphi_A\colon C\big( \Spec(A) \big)\to C^*(A,\mtu)
\end{equation}
be the isomorphism. We denote by $A'$ the restriction of $A$ to the space $\ker(A)^{\perp}$ and $\Sign$ the sign function. This is a continuous function on $\Spec(A')$, thus we can speak about $\varphi_{A'}(\Sign)$. The identity on $\Spec(A')\subset\eR$ can be expressed as
\begin{equation}
	\id(x)=\Sign(x)| x |=\Sign(x)\sqrt{x^*x},
\end{equation}
so that we have $\varphi_{A'}(\id)=\varphi_{A'}(\Sign)\circ \sqrt{(A')^*A'}$. Thus we have the decomposition
\begin{equation}		\label{EqPolarSSKerSign}
	A'=\Sign(A')\circ| A' |.
\end{equation}
The operator $\Sign(A')$ is a partial isometry of $\ker(A)^{\perp}$ because
\begin{equation}
	\Sign(A')\Sign(A')^*=\varphi_{A'}(\Sign)\varphi_{A'}(\Sign)=\varphi_{A'}(1)=\id.
\end{equation}

\begin{proposition}			\label{prop:cv_lim_sup}
If $(a_k)$ is a sequence in $\eR$ such that there exists a $a\in\eR$ for which for any $k\in\eN$,
\[
\lim\sup_{n\to\infty}a_n\leq a\leq a_k
\]
then $(a_k)$ admits a limit and $\lim_{n\to\infty}a_k=a$.
\end{proposition}

\begin{proposition}
Let $\cA$ be an unital Banach algebra, the spectral radius is given by the formula
\begin{equation}
   r(A)=\lim_{n\to\infty}\| A^n \|^{1/n}.
\end{equation} \label{prop:An_usn}
for every $A\in\cA$.
\end{proposition}

\begin{proof}
	Lemma~\ref{lem:cv_Ak} says that when $|z|>\|A\|$, the operator $(A-z\cA)^{-1}$ exists. Let us once again consider the function $g$. From the lemma,
	\begin{equation}
	\frac{1}{z}\sum_{k=0}^{\infty}\left(\frac{A}{z}\right)^k=\frac{1}{z}\left(\mtu-\frac{A}{z}\right)^{-1}
						  =(z-A)^{-1}\\
						  =g(z)
	\end{equation}

	On the other hand, for any $z\in\rho(A)$, one can find an element $z_0\in\rho(A)$ such that
	\[
	   g(z)=\sum_{k=0}^{\infty}(z_0-z)^k(z_0-A)^{k-1}
	\]
	converges. If $|z|>r(A)$, then $z\in\rho(A)$ and then this series converges. But in the interior of the convergence disk, a power series converges uniformly. Then
	\[
	  g(z)=\frac{1}{z}\sum_{k=0}^{\infty}\left(\frac{A}{z}\right)^k
	\]
	uniformly converges with respect to $z$ when $|z|>r(A)$.

	On the other hand, classical analysis makes this series norm-convergent only if $\|A\|^n/\|z\|^n$ from a certain large $n$. Since $\|A^n\|\leq\|A\|^n$, one can say:

	$\forall |z|>r(A)$, $\exists N$  such that $n>N$ implies
	\[
	   \frac{\|A^n\|}{|z|^n}<1.
	\]
	Of course, the choice of $N$ relies on $z$. A consequence of this:
	\[
	   \lim\sup_{n\to\infty}\frac{\|A^n\|}{|z|^n}<1
	\]
	Replacing, $<$ by $\leq$, one can replace $|z|$ by $r(A)$; this yields
	\begin{equation}
	   \lim\sup_{n\to\infty} \|A^n\|^{1/n}\leq r(A).
	\end{equation}
	Now, we show that for any $n$, $r(A)\leq\|A^n\|^{1/n}$, so that proposition~\ref{prop:cv_lim_sup} concludes our proof.

	Since $\sigma(A)$ is closed (theorem~\ref{tho:prop_sigma}), there exists an element $\alpha\in\sigma(A)$ such that $|\alpha|=r(A)$. Lemma~\ref{lem:sigma_poly} makes $\alpha^n\in\sigma(A^n)$, thus $|\alpha^n|\leq\|A^n\|$ and finally
	\begin{equation}
	  r(A)=\alpha\leq\|A^n\|^{1/n}.
	\end{equation}
	Now we are in the situation of a real sequence $(a_k)$ such that $\limsup_{n\to\infty} a_n\leq a \leq a_k$ for all $a_k$\footnote{We recall the definition of supremum limit\index{supremum!limit}:
	\[
	  \limsup_{n\to\infty} a_n=\lim_{n\to\infty}\sup\{ a_k\tq k\geq n \}.
	\]
	}.		% Fin de la note infrapaginale.
	Let us show that in this situation,
	\begin{equation} \label{eq_limsupnana}
	\limsup_{n\to\infty} a_n=a.
	\end{equation}
	 First suppose that $\limsup_{n\to\infty}a_k=b<a$.  In this case, $\forall \epsilon>0$, there exists a $N$ such that
	$  | \sup\{ a_k\tq k\geq N \}-b |<\epsilon$;
	in other words, $\sup\{ a_k\tq k\geq N \}\in B(b,\epsilon)$. The for all $\delta>0$, there are some $a_k$ with $B(b,\epsilon+\delta)$. If we choose $\epsilon$ and $\delta$ suitably small, it gives some $a_k<a$. We conclude that
	\[
	  \lim_{l\to\infty}\{ a_k\tq k\geq l \}=a\leq a_n
	\]
	for all $n$. Let $\epsilon>0$ and assume that there exists a $n>N$ with $a_n$ outside $B(a,\epsilon)$, i.e. suppose that $\lim_{n\to\infty}a_n\neq a$ or doesn't exist. So for all $l$, $\sup\{ a_k\tq k\geq l \}>a+\epsilon$. This is a contradiction with equation \eqref{eq_limsupnana}.

	Now we have to prove that $r(A)\leq \| A^n \|^{1/n}$ for all $n$. Since $\sigma(A)$ is closed, there exists a $\alpha\in\sigma(A)$ such that $| \alpha |=r(A)$ because the supremum of a bounded closed set is reached\angl. From continuous calculus,  $\alpha^n\in\sigma(A^n)$ and therefore $| \alpha^n |\leq\| A^n \|$ because $r(1)\leq \| A^n \|$. We conclude that
	\[
	  r(A)=\alpha\leq\| A^n \|^{1/n}.
	\]
\end{proof}

%+++++++++++++++++++++++++++++++++++++++++++++++++++++++++++++++++++++++++++++++++++++++++++++++++++++++++++++++++++++++++++
\section{Operators with compact resolvent}
%+++++++++++++++++++++++++++++++++++++++++++++++++++++++++++++++++++++++++++++++++++++++++++++++++++++++++++++++++++++++++++

The following come from \cite{Whittaker} and a remark after the definition of a K-cycle in \cite{itoNCG_Varilly}. Since the Dirac operator of a spectral triple has compact resolvent, we need to know some theory about operators with compact resolvent\index{compact!resolvent}.

\begin{lemma}		\label{LemResComKerFin}
	Let $T$ be an operator with compact resolvent and $\lambda\in\eC\setminus\Spec(T)$. Then the kernel of $T$ is an eigenspace of $(T-\lambda\mtu)^{-1}$ for the eigenvalue $\lambda$.
\end{lemma}

\begin{proof}
	First notice that $\lambda\neq 0$; if not, the kernel would be empty because we choose $\lambda$ among the values that are \emph{not} eigenvalues of $T$. If $x\in\ker(T)$, then $x=-\frac{1}{ \lambda }(T-\lambda\mtu)x$. Applying $(T-\lambda\mtu)^{-1}$ on both sides,
	\begin{equation}
		(T-\lambda\mtu)^{-1}x=-\frac{1}{ \lambda }x,
	\end{equation}
	as claimed.
\end{proof}

Since the eigenspaces of a compact operator are finite dimensional, we have in particular
\begin{corollary}		\label{CorRezcomkerfin}
	If $T$ is an operator with compact resolvent, then the kernel of $T$ is finite dimensional.
\end{corollary}

\begin{lemma}		\label{LemResLcmpResLLcmp}
	Let $D\colon \hH\to \hH$ be an operator and consider $R_{\lambda}=(D-\lambda\mtu)^{-1}$ for each $\lambda\in\eC\setminus\Spec(D)$ (i.e. $\lambda$ is in the resolvent of $D$). Then
	\begin{enumerate}
		\item
			If $\lambda_1,\lambda_2\in\eC\setminus\Spec(D)$, we have
			\begin{equation}
				R_{\lambda_1}-R_{\lambda_2}=(\lambda_1-\lambda_2)R_{\lambda_1}R_{\lambda_2};
			\end{equation}
		\item
			The operator $R_{\lambda_1}$ is compact if and only if $R_{\lambda_2}$ is compact. In other words, all the resolvent are compact if only one is compact.
	\end{enumerate}
\end{lemma}

\begin{proof}
	We have
	\begin{equation}
		\lambda_1-\lambda_2=(D-\lambda_2)-(D-\lambda_1)=(D-\lambda_1)\Big( (D-\lambda_1)^{-1}-(D-\lambda_2)^{-1} \Big)(D-\lambda_2),
	\end{equation}
	so that
	\begin{equation}
		(D-\lambda_1)^{-1}-(D-\lambda_2)^{-1}=(\lambda_2-\lambda_1)(D-\lambda_1)^{-1}(D-\lambda_2)^{-1}.
	\end{equation}
	That proves the first claim. In order to prove the second claim, suppose that $R_{\lambda_1}$ is compact and write
	\begin{equation}
		R_{\lambda_1}=\big( (\lambda_1-\lambda_2)R_{\lambda_1}+\mtu \big)R_{\lambda_2}.
	\end{equation}
	We are thus in the case of lemma~\ref{LemAmtuBcompaBcm} which makes $R_{\lambda_2}$ compact.
\end{proof}

\begin{proposition}
	Let $D$ be a selfadjoint operator on the Hilbert space $\hH$. The operator $(D^2-\mtu)^{-1}$ is compact on $\hH$ if and only if $(D-\lambda\mtu)^{-1}$ is compact for some $\lambda\notin\Spec(D)$.
\end{proposition}

\begin{proof}
	The operator $(D^2+\mtu)^{-1}$ can be written under the form
	\begin{equation}		\label{EqDecDdeuxplusuncmp}
		(D^2+\mtu)^{-1}=\Big( (D+i\mtu)(D-i\mtu) \Big)^{-1}=(D-i\mtu)^{-1}(D+i\mtu)^{-1}.
	\end{equation}
	Since $D$ is selfadjoint, the latter expression is of the form $AA^*$. If $(D^2+\mtu)^{-1}$ is compact, this shows that $(D\pm i\mtu)^{-1}$ are compacts by lemma~\ref{LemAstAcomAcomp} while the values $\pm i$ are not in the spectra of $D$ which is real.

	If $(D-\lambda\mtu)^{-1}$ is compact for some value $\lambda\notin\Spec(D)$, lemma~\ref{LemResLcmpResLLcmp} shows that $(D\pm i\mtu)^{-1}$ are compacts because $\pm i$ are outside the spectrum of $D$. Now the decomposition~\ref{EqDecDdeuxplusuncmp} shows that $(D^2+\mtu)$ is compact.
\end{proof}

%+++++++++++++++++++++++++++++++++++++++++++++++++++++++++++++++++++++++++++++++++++++++++++++++++++++++++++++++++++++++++++
					\section{Strong, weak and other topologies}
%+++++++++++++++++++++++++++++++++++++++++++++++++++++++++++++++++++++++++++++++++++++++++++++++++++++++++++++++++++++++++++

We are dealing with separable Hilbert spaces. More details in \cite{JonesVN}. The \defe{weak topology}{weak topology} on $\oB(\hH)$ is the one associated with the following convergence of nets. One says that $T_{\alpha}\to T$ if and only if for every $v$ and $w$ in $\hH$, we have
\begin{equation}
 \langle v, T_{\alpha}w\rangle \to \langle v, Tw\rangle .
\end{equation}

One of the main feature of the weak topology is the \defe{Banach-Alaoglu}{Banach-Alaoglu theorem} theorem.
\begin{theorem}[\href{http://en.wikipedia.org/wiki/Banach-Alaoglu_theorem}{Banach-Alaoglu}]		\label{ThoBanachAlaoglu}
If $\hH$ is a Hilbert space, then the unit ball in $\oB(\hH)$ is weakly compact.
\end{theorem}


The \defe{strong topology}{strong topology} on $\oB(\hH)$ is the topology generated by the open sets
\begin{equation}
\mU(S,v,\epsilon)=\{ T\in\oB(\hH)\tq \| Tv-Sv \|\leq\epsilon \}
\end{equation}
for all $S\in\oB(\hH)$, $v\in\hH$ and $\epsilon>0$. The associated convergence notion is the one of the pointwise convergence: $T_{\alpha}\to T$ if and only if
\begin{equation}		\label{EqDEflimforte}
\| T_{\alpha}v-Tv \|\to 0
\end{equation}
 for all $v\in\hH$.

The strong topology has more closed and open sets than the weak one. One difference between weak and strong topology is that the weak one is compatible with the involution while the strong is not. More precisely, \label{PgStarWeakRespecte}
\begin{align}
	T_{\alpha}\stackrel{w}{\to}T\,\Rightarrow\,(T_{\alpha})^*\stackrel{w}{\to}T^*
\end{align}
while the same is not true for convergence in the strong topology.

%---------------------------------------------------------------------------------------------------------------------------
					\subsection{Ultraweak topology}		\label{subSecUltraWtopol}
%---------------------------------------------------------------------------------------------------------------------------



\begin{definition}
The \defe{ultraweak}{ultraweak topology}\index{topology!ultraweak} is the weakest topology (the one with the fewest open sets) on $\oB(\hH)$ such that the functionals
\begin{equation}
	T\mapsto \sum_{n=1}^{\infty}\langle v_nT, Tw_n\rangle
\end{equation}
are continuous for every choice of sequences $(v_n)$ and $(w_n)$ such that $\sum_n\| v_n \|<\infty$ and $\sum_n\| w_n \|<\infty$.
\end{definition}

\begin{proposition}
Every ultraweakly continuous linear functional on $\oB(\hH)$ has the form
\begin{equation}
	T\mapsto\sum\langle v_n, Tv_n\rangle
\end{equation}
where $(v_n)$ is any sequence of vectors such that $\sum_n\| v_n \|<\infty$.
\end{proposition}

\begin{proof}
No proof.
\end{proof}

An operator $T\in\oB(\hH)$ is \defe{Hilbert-Schmidt class}{Hilbert-Schmidt!class operator} if
\begin{equation}
	\| T \|_{HS}^2:=\sum_{n,m}^{\infty}| \langle v_n, Tv_m\rangle  |^2<\infty
\end{equation}
for some orthonormal basis $\{ v_n \}$ of $\hH$. The operator $T$ is \defe{trace class}{trace!class operator} if
\begin{equation}
	\| T \|_1:=\sum_n\langle | T |v_n, v_n\rangle <\infty
\end{equation}
for some orthonormal basis $\{ v_n \}$ of $\hH$. Notice that, from a simple change of basis formula, the fact to put these conditions for \emph{some} basis of for \emph{every} basis are equivalent. When $T$ is a trace class operator, we define its \defe{trace}{trace!of an operator} by
\begin{equation}
	\tr(T)=\sum_n\langle Tv_n, v_n\rangle
\end{equation}
where $\{ v_n \}$ is an orthonormal basis of $\hH$. That definition is independent of the choice. We denote by $\oL^1(\hH)$\nomenclature[F]{$\oL(\hH)$}{Space of trace class operators over $\hH$} the set of trace class operators in $\hH$.

\begin{proposition}
One has
\begin{equation}
	\| ST \|_1\leq \| S \|_{\oB(\hH)}\| T \|_1
\end{equation}
whenever it makes sense. Here, $\| . \|_{\oB(\hH)}$ denotes the operator norm over $\oB(\hH)$ and $\| . \|_1$ denotes the trace.
\end{proposition}
\begin{proof}
No proof.
\end{proof}

One consequence of that proposition is that the map
\begin{equation}
	S\mapsto \tr(ST)
\end{equation}
is a bounded linear functional on $\oB(\hH)$.

\begin{proposition}
Every trace class operator is a compact linear operator.
\end{proposition}

\begin{proof}[Sketch of the proof]
First, we know that $T$ is compact if and only if $| T |$ is compact. From spectral theory, the largest spectral value is the largest element in the sum $\sum_n\langle | T |v_n, v_n\rangle $, and the second largest spectral value is the second largest element of that sum and so on. Thus the convergence of the sum implies that $| T |$ is compact.
\end{proof}

Thus every positive trace class operator read
\begin{equation}
	Tv=\sum_n\lambda_n\langle v_n, v\rangle v_n,
\end{equation}
where the numbers $\lambda_n$ are positive reals, while a general trace class operator, being a positive one followed by a partial isometry, read
\begin{equation}		\label{EqFormTraceClassGene}
	Tv=\sum_n\lambda_n\langle v_n, v\rangle w_n
\end{equation}
where $\{ v_n \}$ and $\{ w_n \}$ are orthonormal basis of $\hH$ related by the partial isometry implied in the decomposition of $T$. When $T$ is the trace class operator \eqref{EqFormTraceClassGene}, the trace reads
\begin{equation}
	\tr(ST)=\sum_n\lambda_n\langle v_n, Sw_n\rangle .
\end{equation}
Since the $\lambda_n$ are eigenvalues of a compact operator, we have $\sum_n\lambda_n<\infty$, so that one can redefine $v_n\to \sqrt{\lambda_n}v_n$ and $W_n\to \sqrt{\lambda_n}w_n$ that are now summable sequences instead of being actual orthonormal basis. Thus one has
\begin{equation}
	\tr(ST)=\sum_n\langle v_n, Sw_n\rangle ,
\end{equation}
which proves that $\tr(ST)$ is ultraweakly continuous as function of $T$. In turn, that implies that the space of ultraweakly continuous linear functions identifies with the space $\oL(\hH)$ of trace class operators by the formula
\begin{equation}
	\varphi_S(T)=\tr(ST)
\end{equation}
for $S\in\oL^1(\hH)$. Notice that general theory of trace class operators assures that $\tr(ST)$ makes sense when $S\in\oL^1(\hH)$, and we have moreover $\| \varphi_S \|=\| S \|_1$. There is also the map
\begin{equation}
\begin{aligned}
 \oB(\hH)&\to \oL^1(\hH) \\
   T&\mapsto \varphi^T
\end{aligned}
\end{equation}
where $\varphi^T(S)=\varphi(ST)$.

In the same way as $l^1(\eN)^*\simeq l^{\infty}(\eN)$, we have the following.
\begin{theorem}
The map $T\mapsto \varphi^T$ is an isometric isomorphism between $\oB(\hH)$ and $\oL^1$.
\end{theorem}

\begin{proof}
No proof.
\end{proof}
Notice that if $x\in l^{\infty}(\eN)$, the functions $\varphi_i(x)=x_i$ provide an inclusion of $l^1(\eN)$ in $l^{\infty}(\eN)^*$, the latter space being in fact much bigger.

\begin{theorem}[Hahn-Banach]\index{Hahn-Banach theorem}
If $M\subseteq\oB(\hH)$ is a closed subspace in the ultraweak topology, then
\begin{equation}
	M\simeq (M_{*})*
\end{equation}
where $M_*$ is the space of ultraweakly continuous linear functionals on $M$. Moreover the weak-$*$ topology on $M$ is equivalent to th ultraweak one.
\end{theorem}

\input{reprez_Lie_Hilbert}

\chapter{Sobolev spaces}
% This is part of (almost) Everything I know in mathematics
% Copyright (c) 2016
%   Laurent Claessens
% See the file fdl-1.3.txt for copying conditions.

\section*{Notations}
%+++++++++++++++++++

Some notations about functional spaces:

\begin{enumerate}
\item $\cdE(X)$ is the set of smooth functions on $X$ also denoted by $ C^{\infty}(X)$,
\item $\cdD(X;K)$ is the subspace of $\cdE(X)$ of functions with support in $K$,
\item $\cdD(X)$ is the union of all $\cdD(X;K)$ when $K$ runs over the compact subsets of $X$,
\item $C_{c}(X)$ compactly supported functions on $X$.
\item $ C^{\infty}_{c}(X)$ smooth compactly supported functions on $X$.
\item $\scrC_F(E)$ is the set of continuous maps from $E$ to $F$.
\item $\cdD^{(0)}_{\eR}(X)$ is the set of continuous real functions with compact support in $X$,
\item $\scrD^{(r)}_p(X;K)$ is the set of the differential $p$-forms of class $C^r$ on $K$ when $K$ is a compact subset of the manifold $X$,
\item $\cdE^{(p)}_F(A)$ is the set of $p$ times continuously differentiable functions $A\to F$; if $p$ is not mentioned it means $\infty$,
\item $\swS$, the space of smooth rapidly decreasing Schwartz functions, page \pageref{not_swS}.
\end{enumerate}

Some distribution spaces:

\begin{enumerate}
\item $M(X)$ is the space of complex Borel measures on $X$, page \pageref{defMX}
\item $M_0(X)$ is the subspace of $M(X)$ of compact supported measures, page \pageref{defMzX}.
\item $R(G)$ is the dual of $ C^{\infty}(G)$. All element of this space are not measures.
\end{enumerate}

When $f$ and $g$ are real function, we write $f=O(g)$ if there exists $C\leq\infty$ such that $f(x)\leq Cg(x)$ for all $x$.

\section{Distributions}\label{sec:Distrib}
%++++++++++++++++++++++

Matter of this section is taken from \cite{Treves,Dieu3}

Let $X$ be an open set in $\eR^n$. A \defe{distribution}{distribution} on $X$ is a linear form on $\cdD(X)$ whose restriction to $\cdD(X;K)$ is continuous for each compact set $K\subset X$. We denote it by $\cdD'(X)$\nomenclature{$\cdD'$}{distribution space}. More generally, if $\cdA$ is a space of function, we denote by $\cdA'$ the set of linear form on $\cdA$ whose are continuous on each compact.

If $T$ is a linear form on $\cdD(X)$. For $T$ to be a distribution, it is necessary and sufficient that for all sequence $(f_k)\in C^{\infty}(X)$  with $f_k\in\cdD(X,K)$ such that $f_k\to 0$ in $\cdE(X)$, the sequence $(Tf_k)$ converges to $0$ in $\eC$.

Let $T$ be a distribution for which all the restrictions to $\cdD(X;K)$ are continuous for the induced topology from $\cdD^{(r)}(X;K)$. In this case, we say that $T$ has \defe{order}{order!of a distribution}\label{pg:reforder} lesser than $r$. The \emph{order} of $T$ is the smaller such $r$. If it doesn't exist, then we say that $T$ has order infinite.

\begin{proposition}
Let $T$ be of order $\leq r$. Then $T$ is the restriction to $\cdD(X)$ of a linear form $T'$ on $\cdD^{(r)}(X)$ whose restriction to each $\cdD^{(r)}(X;K)$ is continuous. This $T'$ is unique.
\end{proposition}

I only give the beginning of the proof\quext{Il faudra la completer}.

\begin{proof}
Let $K$ be a compact in  $X$ and $K'$ a compact neighbourhood of $K$ in $X$. There exists a function $h\in C^{\infty}(X)$ such that $h=1$ in a compact neighbourhood of $K$ and $h=0$ outside $K'$. Consider a function $g\in\cdD^{(r)}(X;K)$; there exists a sequence $(f_k\in\cdD(X))\to g$ for the topology of $\cdE^{(r)}$. This is the $ C^{\infty}$ approximation of $C^{(r)}$ functions. Note that in general, $g$ don't belong to $\cdD(X;K)$.

Now, the sequence $(hf_k)$ which is contained in $\cdD(X;k')$ converges to $g$ in $\cdD^{(r)}(X;k')$. Then the closure of $\cdD(X;K')$ in $\cdE^{(r)}$ contains $\cdD^{(r)}(X;K)$ and is contained in $\cdD^{(r)}(X;K')$.
\end{proof}

%---------------------------------------------------------------------------------------------------------------------------
\subsection{Dirac distribution}
%---------------------------------------------------------------------------------------------------------------------------

The distribution $\dpt{\delta_a}{\swD(\eR^d)}{\eC}$ given by $f\to f(a)$ is the \defe{Dirac distribution}{Dirac!distribution} at $a$. It was already defined in~\ref{DEFooUSTNooYEZfPN}.

We have a two-dimensional generalisation of that.

\begin{lemmaDef}        \label{LEMooYABKooWPXIXZ}
    Let \( C\) be a curve in \( \eR^2\) and a function \( f\colon \eR^2\to \eC\) integrable on \( C\). The formula
    \begin{equation}
       \langle \delta_C^f, \phi\rangle =\int_Cf\phi.
    \end{equation}
    defines and element \( \delta_C^f\in \swD'(\eR^2) \).
\end{lemmaDef}

\begin{proof}
    Let \( \phi_n\stackrel{\swD(\eR^2)}{\longrightarrow}0\); we have
    \begin{equation}
        |\langle \delta_C^{f}, \phi_n\rangle |\leq\int_C|f\phi_n|\leq  \| \phi_n \|_{\infty}   \int_C| f |.
    \end{equation}
    Since \( f\) is integrable on \( C\), we have a majoration \( \int_C| f |\leq \alpha\) for some \( \alpha>0\) and then
    \begin{equation}
        |\langle \delta_C^{f}, \phi_n\rangle |\leq \alpha\| \phi_n \|_{\infty}.
    \end{equation}
    The definition of convergence in \( \swD(\eR^2)\) implies that \( \| \phi_n \|\to 0\).
\end{proof}

The usual Dirac distribution is obtained with \( C=\{ 0 \}\) and \( f(x)=1\) on $C$.

\subsection{Distribution defined from functions}
%-----------------------------------------------

If $\dpt{f}{X}{\eR}$ is an integrable function, we define the distribution $T_f$ by \nomenclature{$T_f$}{Distribution defined by a function}
\[
  T_j(g)=\int_Xfg.
\]
If $\mU\subset\eR^N$, is open and $K\subset\mU$ is compact, for all multi-index $\nu$, the map $f\to D^{\nu}f$ is continuous from $\cdD(\mU;K)$ to $\cdD(\mU;K)$. This leads us to define the \defe{derivative}{derivation!of a distribution} of the distribution $T\in\cdD'(\mU)$ by
\begin{equation} \label{eq:defpartialT}
  (D^{\nu}T)f=(-1)^{| \nu |}T(D^{\nu}f).
\end{equation}
In particular, if $T=T_f$, then
\begin{equation} \label{eq:defTpri}
(\partial_iT_g)f=-\int g\partial_if
                =\int(\partial_ig)f
                =T_{\partial_ig}f.
\end{equation}
This relation explains the sign in definition \eqref{eq:defpartialT}. The boundary term which should appears in the integral by part is zero because $f\in\cdD(\mU;K)$ has compact support.

\subsection{Fourier transform and Schwartz functions}
%----------------------------------------------------

We consider $\{ e_1,\ldots,e_n \}$, a basis of $\eR^N$ and $\{ e'_1,\ldots,e'_n \}$ the dual basis defined by $\scal{e'_i}{e_j}=\delta_{ij}$. If $x=\sum_ix_ie_i$ and $\xi=\sum_j\xi_je'_j$, we write $\scal{x}{\xi}=\scal{\xi}{x}=\sum_i\xi_ix_i$.

The space $\swS(\eR^N)$\nomenclature{$\swS$}{Schwartz space}\label{not_swS} of \defe{Schwartz functions}{Schwartz functions} is the space of $ C^{\infty}$ functions $\varphi$ such that for all polynomials $P$ on $\eR^N$ and all multi-index $\nu$, the quantity
\begin{equation}
  | P(x)D_x^{\nu}\varphi(x) |
\end{equation}
is bounded.  The topology is given by seminorms
\begin{equation}
  \| \varphi \|_{P,\nu}=\sup_{x\in\eR^N}| P(x)D_x^{\nu}\varphi(x) |.
\end{equation}
It is possible to prove that it is a Fréchet space in which all bounded and closed sets are compact; the topology is independent of the basis. A function in this space has special property that its integral is absolutely convergent:
\[
  \int_{\eR^N}| \varphi(x) |\leq \infty.
\]
It is proved in \cite{Kirillov} that on $\swS(\eR^n)$, the following families of seminorms are equivalent:
\begin{subequations}
\begin{align}
  p_{\alpha\beta}(f)&=\sup_{x\in\eR^n}| x^{\alpha}\partial^{\beta}f(x) |\\
 p'_{\alpha\beta}(f)&=\int_{\eR^n}| x^{\alpha}\partial^{\beta}f(s) |\,ds\\
p''_{\alpha\beta}(f)&=\left( \int_{\eR^n}| x^{\alpha}\partial^{\beta}(s) |^2\,ds \right)^{1/2}.
\end{align}
\end{subequations}

Let $\xi\in(\eR^N)^*$ and consider the function  $x\to e^{-2\pi i\scal{x}{\xi}}\varphi(x)$
which belongs to $\swS(\eR^N)$ as long as $\varphi\in\swS(\eR^N)$. The \defe{Fourier transform}{Fourier transform} of $\varphi\in\swS(\eR^N)$ is the function $\dpt{\hat\varphi}{(\eR^N)^*}{\eC}$ given by
\begin{equation}
\hat\varphi(\xi)=\int \varphi(x)e^{-2\pi i\scal{x}{\xi}}dx.
\end{equation}

Classical results are summarized in the following theorem (proof in the case of space $\swS$ is given in \cite{Treves}):

\begin{theorem}
The Fourier transform is a topological vector space isomorphism from $\swS(\eR^N)$ into $\swS(\eR^N)$ and the inverse is given by formula
\[
  \varphi(x)=\int \hat\varphi(\xi)e^{2\pi i\scal{x}{\xi}}d\xi.
\]
Equalities of Parseval and Plancherel\index{Perceval}\index{Plancherel} holds:
\begin{align}
  \int \phi\overline{\psi}=\int \hat\phi\overline{\hat\psi}\textrm{ and }
\int| \phi |^2=\int| \hat\phi |^2
\end{align}
where the bar denotes the usual complex conjugation. Left hand side integrals are taken on $\eR^N$ and right integrals over $(\eR^N)^*$.
\end{theorem}

\begin{corollary}
Fourier transform can be extended to an isometry $L^2(\eR^N)\to L^2\big( (\eR^N)^* \big)$.
\end{corollary}

\subsection{Support of a distribution}
%-------------------------------------

Let $\mU$ be an open set in $X$ and $K$, a compact in $\mU$. The map $\cdD(X;K)\to\cdD(\mU;K)$ given by $f\to f_{\mU}$ is an isomorphism whose inverse is $f\to f^{\mU}$ where $f^{\mU}$ is just the prolongation of $f$ with zero outside $\mU$. For a distribution $T$ on $X$, we consider $\dpt{T|_{\mU}}{\cdD(\mU;K)}{\eC}$,
\[
         T|_{\mU}(f)=T(f^{\mU}).
\]
This is a distribution on $\mU$ called the \defe{induced}{induced!distribution} from $T$ on $\mU$. A distribution on $\mU$ is not always the restriction of a distribution on $X$ and when it is, the prolongation is not unique in general. There exists a prolongation theorem in certain cases:

\begin{theorem}
Let $(\mU_{\lambda})_{\lambda\in L}$ be an open covering of $X$ and for each $\lambda\in L$, a distribution $T_{\lambda}$ on $\mU_{\lambda}$. We suppose that for all $\lambda,\mu\in L$, $T_{\lambda}|_{\mU_{\lambda}\cap\mU_{\mu}}=T_{\mu}|_{\mU_{\lambda}\cap\mU_{\mu}}$. Then there exists an unique distribution $T$ on $X$ such that for all $\lambda\in L$, $T|_{\mU_{\lambda}}=T_{\lambda}$.
\end{theorem}

Now if the restriction  of $T$ to each $\mU_{\lambda}$ is zero, then the restriction to the union is zero too because the null distribution answers the theorem. So the union of all the open set on which $T$ is zero is an open on which $T$ is zero. It is the largest open $V\subset X$ on which $T$ is zero. The complementary $S=\complement V$ is the \defe{support}{support of a distribution} of the distribution $T$. We note it $\Supp(T)$.

Let $x\in\Supp T$: there exists no open containing $x$ on which $T$ is zero. With other words, for all  neighbourhood $V$ of $x$, there exists a function $f\in\cdD(X)$ with support contained in $V$ with $T(f)\neq 0$.

\subsection{Duality}
%-------------------

If $L$ and $M$ are algebras, the set $\Hom(L,M)$\nomenclature{$\Hom(L,M)$}{Space of linear maps from $L$ to $M$} contains all the maps $\dpt{f}{L}{M}$ such that $f(xy)=f(x)f(y)$. We are mainly interested in $\eC$-algebras: algebras $L$ endowed with a product $\eC\times L\to L$. Then we are leads to look at $\Hom_{\eC}(L,\eC)$, the subset of $\Hom(L,\eC)$ of maps which commutes with the latter product: $A\in\Hom_{\eC}(L,\eC)$ when $A\in\Hom(L,\eC)$ and $A(zl)=zA(l)$. We denote\nomenclature{$L^*$}{Linear maps from $L$ to $\eC$}.
\begin{equation}
  L^*=\Hom_{\eC}(L,\eC).
\end{equation}
This is defined when $L$ is a $\eC$-module.

We consider now a \emph{topological} vector space: $L$ is a topological group for addition and the map $\eC\times L\to L$, $(\lambda,x)\to \lambda x$ is continuous with respect to the two variables. We denote by $L'$\nomenclature{$L'$}{Topological dual of $L$} the space of linear \emph{and continuous} maps $L\to \eC$; this is the \defe{topological dual}{dual!topological}\index{topological!dual} of $L$.

Elements of $L^*$ have no continuity condition. In the general case, $L'\subset L^*$, and in the finite dimensional case, $L'=L^*$. The space $L^*$ is the \defe{dual module}{dual!module} of $L$ while $L'$ is the \defe{topological conjugate}{topological!conjugate} space of $L$. In both cases, we speak about \defe{dual space}{dual!space} of $L$.


\begin{proposition}
Dual space $\cdE'(X)$ is the space of distribution with compact support.
\label{prop:dualCinfcompact}
\end{proposition}

We can consider the sequence $ C^{\infty}_c\subset\swS\subset C^{\infty}$ of inclusion with dense images. The transposition of this gives the continuous inclusion sequence
\[
  \cdE'\subset\swS'\subset\cdD'.
\]
We say that an element of $\swS'(\eR^N)$ is a \defe{tempered distribution}{tempered!distribution}\nomenclature{$\swS'$}{Tempered distributions}.

\begin{proposition}
A distribution on $\eR^N$ is tempered if and only if it is a finite sum of derivatives of continuous functions which are bounded at infinity by a polynomial.
 \label{prop_distr_temp_sum}
\end{proposition}

A continuous function is not necessarily derivable. By ``derivative of a continuous function'' we mean a distribution of the form $(T_f)'$ (derivative in the sense of distributions) which we write $T_{f'}$ by abuse of notation. This notation is motivated by equation \eqref{eq:defTpri}.

\begin{proof}[Proof of proposition~\ref{prop_distr_temp_sum} ]
Let us first proof that a distribution $T$ is tempered if and only if the map $\varphi\to T\varphi$ is continuous on $ C^{\infty}_c$ for the topology induced from $\swS$. Let $\dpt{\tilde T}{ C^{\infty}_c}{\eC}$ be the map equals to $T$ on $ C^{\infty}_c$ and not defined anywhere else. If $\mO$ is open in $\eC$, then $\tilde T^{-1}(\mO)= C^{\infty}_c\cap T^{-1}(\mO)$. This is open in $ C^{\infty}_c$ (for the induced topology from $\swS$) if and only $T^{-1}(\mO)$ is open in $\swS$.


Let $f$ be continuous and $T=T_{f'}$. We want $\varphi\to(T_f\varphi)'$ to be continuous on $\swS$. We can write it as
\[
  (T_f)'\varphi=-T_f(\varphi')=\int_{\eR^N}f\varphi'.
\]
The integral exists because $\varphi\in\swS$, so $\varphi'$ is decreasing at infinity more rapidly than the inverse of any polynomials, while $f$ is bounded by a polynomial.

Let us now consider, a tempered distribution $T$. We have to prove that $T=T_{f'}$ for a certain continuous function $f$ bounded by a polynomial. Since $T$ is linear, its continuity is assured by the only continuity at zero. A neighbourhood of zero in $\swS$ reads under the form
\[
  A_{P,Q,\varepsilon}=\{ \varphi\in\swS\tq \sup_{x\in\eR^N}| P(X)Q(\partial_x)\varphi(x) |<\varepsilon \}
\]
for a choice of polynomials $P,Q$ and a $\varepsilon>0$. Let $\mO$ be a neighbourhood of rayon $\varepsilon$ around $0$ in $\eC$; we have
\[
  T^{-1}(\mO)=\{ \varphi\in\swS\tq | T\varphi |\leq \varepsilon \}.
\]
In order to be an open set, we have to find two polynomials $P$ and $Q$ (depending on $\varepsilon$ but not on $\varphi$) such that
\[
  | T\varphi |\leq \sup_{x\in\eR^N}| P(x)Q(\partial_x)\varphi(x) |.
\]
The continuity of $T$ gives the existence of reals $m,h\geq 0$ and $C>0$ such that
\[
  | T\varphi |\leq \sup_{| p |\leq m}\sup_{x\in\eR^N}| (1+| x |^2)^h(\partial_x)^p\varphi(x) |.
\]
Let us pose $\varphi_h(x)=(1+| x |^2)^h\varphi(x)$. It still belongs to $   C^{\infty}_c$ and moreover, the map $\varphi\to\varphi_g$ is a bijection on $ C^{\infty}_c$. By induction on $h$, we see that
\begin{equation}
  | (\partial_x)^p\varphi(x) |\leq C_{p,h}(1+| x |^2)^{-h}\sum_{q\leq p}| (\partial_x)^q\varphi_h(x) |
\end{equation}
where $q\leq p$ means $q_1\leq p_1,\ldots, q_n\leq p_n$.
 \quext{je ne fais pas le reste de la démonstration}.

\end{proof}

The space of \defe{currents}{current} is the dual (with respect to $\eR$) space of the space of differential forms. Current is the generalization of differential forms in the same sense that distributions are a generalization of functions. An example of $k$-current is given by a $(n-k)$-form $\sigma$ by setting
\[
  C_{\sigma}(\omega)=\int_M \sigma\wedge\omega.
\]

\section{Measure, distribution and integral} \label{sec_distrib_mesure}
%----------------------------------------------

Do you know what is yellow and equivalent to the existence of non measurable functions? Answer in the footnote\footnote{The Zorn lemon!}.

Most of links between measure theory and distribution are given in \cite{Dieu2}. We will state a lot of results without proof; they can be found in this reference. We always suppose that $X$ is locally compact.

Let $\cdD^{(0)}(X;K)$ be the set of continuous functions on $X$ whose support is contained in a compact $K\subset X$. A \defe{measure}{measure} on $X$ is a linear form $\dpt{\mu}{\cdD^{(0)}(X)}{\eC}$ such that for all compact $K\subset X$, there exist a real $a_K\geq 0$ for which for all $f\in\cdD^{(0)}(X;K)$
\begin{equation}
  | \mu(f) |\leq a_K\| f \|.
\end{equation}
In this case, the restriction $\mu|_{\cdD^{\infty}}(X;K)$ is continuous for the induced topology from $\cdD^{(0)}(X;K)$. Indeed if $\mO$ is open in $\eC$, then
\begin{equation}  \label{eq:18105r1}
  \mu|_{\cdD(X;K)}^{-1}(\mO)=\cdD(X,K)\cap\mu^{-1}(\mO)
\end{equation}
but the continuity condition on $\dpt{\mu}{\cdD^{(0)}}{\eC}$ makes $\mu^{-1}(\mO)$ open in $\cdD^{(0)}$. Then expression \eqref{eq:18105r1} describes an open set in $\cdD(X;K)$ for the induced topology of $\cdD^{(0)}(X;K)$.

The measures are exactly the distribution of order zero, \emph{confer} page \pageref{pg:reforder}.

\subsection{Example: the Lebesgue measure}

Let $f\in\cdD^{(0)}(\eR)$. For all $a$, $b\in\eR$ such that $\Supp f\subset [a,b]$, the value of $\int_a^b f$ is the same and is denoted by $\int_{\eR}f$. The map $f\to\int_{\eR}f$ is a linear continuous function on $\cdD(\eR)$. This is a measure because if $f\in\cdD^{(0)}(\eR,K)$ with $K=[a,b]$, then
\[
  | \int_{\eR}f |\leq (b-a)\| f \|
\]
from the mean value theorem. We recognize the usual \defe{Lebesgue measure}{Lebesgue measure}\index{measure!Lebesgue} on~$\eR$.

The measure $\mu$ is \defe{positive}{positive!measure}\index{measure!positive} when for all $f\geq0\in\cdD^{(0)}_{\eR}$, we have $\mu(f)\geq0$. It is \defe{real}{measure!real} if $\mu(f)\in\eR$ whenever $f\in\cdD^{(0)}_{\eR}(X)$. Since any function can be decomposed into $f=f^+-f^-$, a positive measure is always real.

From now we only consider positive measures on a locally compact set.

\begin{proposition}
If $\mu$ is a linear form on $\cdD^{(0)}_{\eR}(X)$ with $\mu(f)\geq0$ when $f\geq0$, then $\mu$ is a measure.
\end{proposition}

A function $\dpt{f}{X}{\overline{ \eR }}$ is \defe{lower semicontinuous}{semicontinuous!lower} at $x_0\in X$ when for all $\alpha\in\overline{ \eR }$ such that $\alpha<f(x_0)$, there exists a neighbourhood of $x_0$ in which $\alpha< f$. It is \defe{upper semicontinuous}{semicontinuous!upper} when for all $\alpha'>f(x_0)$, we have $\alpha'>f$ on a neighbourhood.

We consider the set $\mS(X)$ of lower semi continuous functions from $X$ to $\overline{ \eR }$ which are minored by a function of $\cdD^{(0)}_{\eR}(X)$. In particular, if $f\in\mS(X)$, one can define
\begin{equation}
   \mu^*(f)=\sup_{%
\begin{subarray}{l}
g\in\cdD^{(0)}_{\eR}(X)\\
g\leq f
\end{subarray}
}\mu(g)
\end{equation}
which is  a well defined number in $\eR\cup\{ +\infty \}$


\begin{proposition}
Let $(f_n\in\mS)$ an increasing sequence and $f=\sup_nf_n$. Then

\begin{enumerate}
\item $f$ exists and $f\in\mS$
\item $\mu^*(f)=\sup_n\mu^*(f_n)=\lim_{n\to\infty}\mu^*(f_n)$
\end{enumerate}

\end{proposition}

If $\dpt{f}{X}{\overline{ \eR }}$ is \emph{any} function, a function $h\in\mS(X)$ with $h\geq f$ always exists, so we can define
\begin{equation}
  \mu^*(f)=\inf_{%
\begin{subarray}{l}
h\geq f\\
h\in\mS(X)
\end{subarray}
} \mu^*(h).
\end{equation}
This number $\mu^*(f)\in\overline{ \eR }$ is the \defe{upper integral}{integral!upper} of $f$.


\begin{proposition}
If $(\dpt{f_n}{X}{\overline{ \eR }})$ is an increasing sequence with $\mu^*(f_n)>-\infty$, then
\[
  \mu^*(\sup_nf_n)=\sup_n\mu^*(f_n)=\lim_{n\to\infty}\mu^*(f_n).
\]

\end{proposition}

\begin{proposition}
For all sequence of functions $(f_n\geq 0)$, we have
\[
  \mu^*\left( \sum_{n=1}^{\infty}f_n\right)\leq\left(\sum_{n=1}^{\infty}\mu^*(f_n) \right).
\]

\end{proposition}

Now, for a function $\dpt{f}{X}{\overline{ \eR }}$, we put
\begin{equation}
  \mu_*(f)=-\mu^*(-f)
\end{equation}
and we call it the \defe{lower integral}{integral!lower} of $f$ for the measure $\mu$. It fulfils
\[
  \mu_*(f)\leq\mu^*(f).
\]
When the equality are true, we define $\mu(A)$ for a subset $A\subset X$ by
\begin{equation}
   \mu(A)=\mu_*(1_A)=\mu_*(1_A).
\end{equation}

We consider $M(X)$\nomenclature{$M(X)$}{Set of Borel measures on $X$}\label{defMX}, the set of all the \defe{Borel measures}{Borel measures} on $X$: these are measure for which all Borel sets are measurable. When a measure $\mu$ is countably additive, we define
\begin{equation}
   | \mu |(E)=\sup \sum_{k=1}^{\infty}| \mu(E_k) |
\end{equation}
where the supremum is taken over all decomposition of $E$ in disjoints sets $E_k$. It induces a norm on $M(G)$ by
\[
  \| \mu \|=| \mu |(X).
\]
The subset  $M_0(X)$\nomenclature{$M_0(X)$}{Set of compact supported Borel measures}\label{defMzX} contains the Borel regular measures with compact support. A measure is \defe{regular}{regular!measure} when for all $B\subset X$ such that $\mu(B)$ exists, we have
\begin{equation}
\begin{split}
    \mu(B)&=\sup\{ \mu(K)\tq K\subseteq B \textrm{ is compact} \}\\
		&=\inf\{ \mu(A)\tq B\subseteq A, \textrm{ $A$ is open} \}
\end{split}
\end{equation}

\subsection{Integration on more general spaces}
%----------------------------------------------

Let $X$ be locally compact and $\mu$, a measure on $X$. We define\nomenclature{$\| \mu \|$}{Norm of a measure}
\begin{equation}
  \| \mu \|=\sup_{%
\begin{subarray}{l}
\| f \|\leq 1\\
f\in\cdD^{(0)}(X)
\end{subarray}
}
| \mu(f) |\in\eR\cup\{ +\infty \}
\end{equation}
We say that the \emph{positive measure} $\mu$ is \defe{bounded}{bounded!positive measure} when $\mu^*(X)=\mu^*(1_X)$ is a finite real.  Properties of $\| \mu \|$ are that
\begin{equation}
\| \mu \|=| \mu |^*(1_X)
\end{equation}
and that $\| \mu \|$ is finite if and only if $| \mu |$ is bounded.


\begin{proposition}
If $\mu$ is bounded, any bounded and measurable function $\dpt{f}{X}{\eR}$ is integrable and
\begin{equation} \label{eq:bornintfmu}
\Big| \int_Xf\,d\mu \Big|\leq \| f \|\mu(X)
\end{equation}

\end{proposition}
When $\mu$ is not positive, we say that it is \defe{bounded}{bounded!measure} if $\| \mu \|$ is finite. Equation \eqref{eq:bornintfmu} shows that $f\to\int_Xf\,d\mu$ is a continuous linear form on $ C^{\infty}(X)$. All continuous linear form on this space are however not of the form \eqref{eq:bornintfmu}.


\subsection{Integration of vector valued functions}
%----------------------------------------------

Let us consider $E$, a vector space of \emph{finite} dimension and a function $\dpt{f}{X}{E}$. The functions $f_i$ are defined by
\[
  f(x)=\sum_{i=1}^{\infty}f_i(x)e_i
\]
where $\{ e_i \}$ is a basis of $E$. We define
\begin{equation}
  \int f\,d\mu=\sum [\int f_i]e_i
\end{equation}
when all the integrals of the right hand side make sense. The fact for $f$ to be integrable is equivalent to the fact that $x\to\xi(f(x))$ is integrable for all $\xi\in E'$ because $\xi\circ f$ is a linear combination of all the $f_i$.

Let $I$ be a set and consider a map $x\to\Fun(I,\eC)$, $x\to f_x$. We say that this is \defe{scalar integrable}{integrable scalar} if for all $\alpha\in I$, the map $x\to f_x(\alpha)$ is integrable.

The theorem which treat with infinite dimension is the following:

\begin{theorem}
   Let $\mF$ be a Fréchet space and $\mF'$ his dual. Let $x\to f_x$ be a map $X\to\mF'$ such that for all converging sequence $(a_n\in\mF)$, there exists a function  $\dpt{g}{X}{\eR}$, $g\geq0$ such that $\mu^*(g)\leq\infty$ and $| f_x(a_n) |\leq g(x)$ for all $n$. Then there exists one and only one $\xi\in\mF'$ such that
\begin{equation}
  \xi(z)=\int_Xf_x(z)\,d\mu(x).
\end{equation}

\end{theorem}

\subsection{Weak integral}
%------------------------

Let $(X,\mu)$ be a measured space and $\dpt{f}{X}{L}$ a map from $X$ to a locally convex space $L$. We say that $f$ is \emph{weakly integrable} if for all $\chi\in L'$, the map $x\to \chi(f(x))$ is integrable for the measure $\mu$, i.e. if $\mu(\chi\circ f)$ makes sense. The \defe{weak integral}{weak integral} of $f$ for with respect to the measure $\mu$ is defined by the requirement
\begin{equation} \label{eq:chbilemirhi}
  \chi\left( \int_Xf\,d\mu \right)=\int_X(\chi\circ f)\,d\mu
\end{equation}
for all $\chi\in L'$. The weak integral $\int_Xf$ is an element of $(L')^*$. In fact the left hand side of equation \eqref{eq:chbilemirhi} should better be noted
\[
  \left( \int_Xf\,d\mu \right)(\chi)
\]
We know that $(L')^*$ can be seen as a subset of $L$. It can be shown that if $f$ is continuous and $X$ compact, then $\int_Xf\in L$.
%%%%%%%%%%%%%%%%%%%%%%%%%%
%
   \section{Distribution on groups}
%
%%%%%%%%%%%%%%%%%%%%%%%%

Most of this section comes from \cite{Kirillov}.

We immediately put the attention to the reader on the fact that $(\eR^N,+)$ is a Lie group. Let $G$ be a Lie group, we define $R(G)$\nomenclature{$R(G)$}{Dual of $ C^{\infty}(G)$} as the dual of $ C^{\infty}(G)$. This is the space of compact supported distributions on $G$. The support of $T\in R(G)$ is the smallest compact $K$ for which $T(\phi)=0$ whenever $\phi|_K^{(r)}=0$ for all $r$. When $K$ is compact in $G$, we consider $R(G,K)$, the subspace of $R(G)$ of distributions with support contained in $K$.

We focus on $R(G,\{ e \})$: distributions in this space only depends on values of test functions (and derivatives) at $e$. It should be possible to reconstruct this space from the only data of the Lie algebra $\lG$ instead of then whole group. We'll see later that it is indeed possible.

\subsection{Convolution product}
%--------------------------------

The \defe{convolution}{convolution!product of distribution} of two distributions $T_1$ and $T_2$ in $R(G)$ is given by
\begin{equation}
  (T_1\star T_2)(\phi)=T_1(\phi_{T_2})
\end{equation}
where $\dpt{\phi_{T_2}}{G}{\eC}$ is given by
\begin{equation}
   \phi_{T_2}(g)=T_2(L_g^{-1}\phi)
\end{equation}
and $(L_g\phi)(g')=\phi(g^{-1}g')$ or in a more convenient way, $\phi_{\nu}(g)=\nu(\phi(g\cdot))$.

Let us show that one retrieve the well know distribution convolution in the case of $G=(\eR^N,+)$. We consider $T_f$ and $T_g$, the distributions defined from functions $f$ and $g$ on $\eR^N$. We have $\phi_{T_g}(t)=T_g(L_t^{-1}\phi)$ where $(L_t^{-1}\phi)(x)=\phi(x+t)$. Then
\begin{equation}
  \phi_{T_g}(t)=\int g(x)(L_t^{-1}\phi)(x)\,dx
		=\int g(x)\phi(x+t)\,dx.
\end{equation}
A change of variable gives
\begin{equation}
  (T_g\star T_g)(\phi)=\int f(t)\phi_{t_g}(t)\,dt
		=\int (f\star g)(u)\phi(u)\,du
		= T_{f\star g}\phi
\end{equation}
with the usual convolution product between function.

\begin{remark}
We see that the convolution operation of functions on $\eR^N$, which has remarkable properties in experimental physics, naturally generalises to a convolution product on distribution which will give a homomorphism between these distributions and the enveloping algebra of the group. Wonderful isn't?
\end{remark}

\begin{lemma}
The map $\dpt{\psi}{\mG}{R(G,\{ e \})}$ given by $\psi(\tilde X)f=Xf$ is a homomorphism.
\end{lemma}

\begin{proof}
Let us prove that $[\psi(\tilde X)\star\psi(\tilde Y)]f=(\tilde X\tilde Yf)_e$. We have
\begin{equation}
\begin{split}
   [\psi(\tilde X)\star\psi(\tilde Y)]&=\psi(\tilde X)\big( c(\psi(\tilde Y),\phi) \big)\\
		&=\Dsdd{ c(\psi(\tilde Y),\phi)\tilde X_e(t) }{t}{0}\\
		&=\DDsdd{ (L^{-1}_{X(t)}\phi)(Y(s)) }{t}{0}{s}{0}\\
		&=\DDsdd{ \phi(X(t)Y(s))) }{t}{0}{s}{0}.
\end{split}
\end{equation}

Now, $\tilde Y\phi$ is a function from $G$ to $\eC$ on which we can apply the vector $\tilde X_e=X$. We have:
\begin{equation}
\begin{split}
(\tilde X\tilde Y)_e\phi&=\tilde X_e(\tilde Y\phi)\\
		&=\Dsdd{ (\tilde Y\phi)(\tilde X_e(t)) }{t}{0}\\
		&=\Dsdd{ \tilde Y_{X(t)}\phi }{t}{0}\\
		&=\DDsdd{ \phi\big( \tilde Y_{X(t)}(s) \big) }{t}{0}{s}{0}\\
		&=\DDsdd{ \phi\big( X(t)Y(t) \big) }{t}{0}{s}{0}.
\end{split}
\end{equation}

\end{proof}

\subsection{Representations}
%--------------------------

Let $G$ be a locally compact metrisable Lie group and $V$, an Hausdorff topological vector space on $\eC$. We say that $\dpt{U}{G}{\End V}$ is a \defe{linear continuous representation}{representation!linear continuous} of $G$ on $V$ when
\begin{enumerate}
\item $U(gh)=U(g)\circ U(h)$ for all $g$, $h\in G$,
\item for each $v\in V$, then map $g\to U(g)v$ is continuous from $G$ into $V$.
\end{enumerate}
In most of cases, we want the representation to be \defe{unitary}{unitary!representation}\index{representation!unitary}: for all $g\in G$ and for all $v$, $w\in V$,
\begin{equation}
  \scal{U(g)v}{U(g)w}=\scal{v}{w}.
\end{equation}
In particular, $U(g)^*=U(g)^{-1}$. Let $v$, $w\in V$. The function $G\to\eC$ given by
\[
  g\to \scal{U(g)v}{w}
\]
is continuous and bounded. The boundary comes from Schwartz and the fact that $U(g)$ is an isometry:
\begin{equation}
  \scal{U(g)v}{w}\leq \| U(g)v \|\| w \|
		=\| v \|\| w \|.
\end{equation}
Equation \eqref{eq:chbilemirhi} then shows that it is an integrable function and that
\begin{equation} \label{eq:Tmuvborn}
\left|    \int_G\scald{U(g)v}{w}\,d\mu(g)   \right|\leq\| \mu \|\| \scal{U(\cdot)v}{w} \|
						\leq \| \mu \|\| v \|\| w \|.
\end{equation}

\subsection{Representation on a Hilbert space}\index{Hilbert space}
%----------------------------------------------

Let $H$ be a Hilbert space and $v\in H$. The map $\dpt{\phi_v}{H}{\eC}$,
\[
  \phi_v(w)=\scal{v}{w}
\]
is linear of norm $\| w \|$ and then is bounded. This is an element of $H'$. A great theorem allows us to identify $H$ and $H'$ by $v\to\phi_v$.

\begin{theorem}[Riesz-Fisher]\index{Riesz-Fisher theorem}
   Let $H$ be a Hilbert space and $\xi\in H'$. Then there exists one and only one $v\in H$ such that
\[
  \xi(w)=\scal{v}{w}
\]
for all $w\in H$. In particular, $H'\simeq H$.
\end{theorem}

Let us now consider a continuous, linear and unitary representation $\dpt{U}{G}{\End H}$. For given measure $\mu$ and vector $v$, we consider $\dpt{T_{\mu,v}}{H}{\eC}$,
\[
  T_{\mu,v}(w)=\int_G\scal{U(g)v}{w}.
\]
Equation \eqref{eq:Tmuvborn} shows that it is continuous and Riesz-Fischer gives us a vector $u$ such that $T_{\mu,v}(w)=\scal{u}{w}$. The so defined vector $u$ is written $U(\mu)v$:
\begin{equation}
   \int_G \scal{U(g)v}{w}\,d\mu(g)=\scal{U(\mu)v}{w}
\end{equation}
is the definition of $U(\mu)\in\End H$. As notation principle, we write
\[
  U(\mu)=\int_GU(g)\,d\mu(g).
\]

\begin{proposition}

This constructions gives a morphism between algebra of bounded measures (the algebra product is the convolution) and $\End H$:
\begin{equation}
  U( \mu\star\nu)=U(\mu)\circ U(\nu)
\end{equation}

\end{proposition}


\begin{proof}
We have
\begin{equation}
\scal{U(\mu\star\nu)v}{w}=\int_G \scal{U(g)v}{w}d(\mu\star\nu)(g)
		=(\mu\star\nu)\scal{U(\cdot)v}{w}
		=\mu(f_{\nu})
\end{equation}
where $f_{\nu}(g)=\int_G\scal{U(gh)v}{w}d\nu(h)$. Then
\begin{equation}
\begin{split}
\scal{U(\mu\star\nu)v}{w}&=\int_G\int_G \scal{U(h)v}{U(g)^*w}\,d\nu(h)d\mu(g)\\
		&=\int_G\scal{U(\nu)v}{U(g)^*w}\,d\mu(g)\\
		&=\scal{U(g)\circ U(\nu)v}{w}.
\end{split}
\end{equation}
\end{proof}

\subsection{Slightly more general}\quext{Tout ceci est \'a pr\'eciser.}
%--------------------------------

Let $\mu\in M_0(G)$ and a continuous representation $\dpt{T}{G}{\End V}$ where $V$ is locally convex\quext{Ce qui implique que $\End(V)$ est locallement convexe pour la topologie forte, il semble que cela joue un r\^ole.}. We can weakly integrate $T$ on $G$ for the measure $\mu$ by setting that
\[
  \int_G T(g)\,d\mu(g)\in\left( \End(V)' \right)^*
\]
such that for all $\chi\in\End(v)'$,
\begin{equation}
\scal{\int_G T(g)\,d\mu(g)}{\chi}=\int_G\scal{\chi}{T(g)}\,d\mu(g).
\end{equation}
Since $T$ is continuous and $\mu$ has compact support, one can see that
\[
  \int_GT(g)\,d\mu(g)\in\End(V)
\]
in the sense of the usual embedding $\left( \End(v)' \right)^*\subset\End(V)$. This element is denoted by $T(\mu)$.

\subsection{Representation of \texorpdfstring{$M_0(G)$}{M0G}}
%--------------------------------------
\quext{Ceci n'est pas complètement compris non plus}

Let $\dpt{T}{G}{\End(V)}$, a representation, $\mu$ a compact supported measure on $G$. Then there exists one and only one $A\in\End(V)$ such that for all $\chi\in\End(V)'$, we have
\begin{equation}
  \chi(A)=\int_G\chi(T(g))\,d\mu(g),
\end{equation}
this $A$ is denoted by $T(\mu)$. It fulfils
\begin{equation}
  \chi(T(\mu))=\int_g\chi(T(g))\,d\mu(g).
\end{equation}
This way to define $\dpt{T}{M_0(G)}{\End(V)}$ is a representation. Indeed, $T(\mu\star\nu)$ is defined by
\begin{equation}
\chi(T(\mu\star\nu))	=(\mu\star\nu)\scal{\chi}{T(.)}
			=\int_G\int_G\scal{\chi}{T(gh)}\,d\nu(h)\,d\mu(g).
\end{equation}
Let us suppose $\chi(T(g)\circ T(h))=\chi(T(g))\chi(T(h))$; in this case Fubini theorem gives
\begin{equation}
\begin{split}
  (\mu\star\nu)\scal{\chi}{T(.)}&=\int_G\int_G\scal{\chi}{T(g)}d\mu(g)\scal{\chi}{T(h)}d\nu(h)\\
			&=\chi(T(\mu))\chi(T(\nu)).
\end{split}
\end{equation}
So for all such $\chi$, we have
\[
  \chi\big(T(\mu\star\nu)\big)=\chi\big(T(\mu)\circ T(\nu)\big).
\]
It is not sufficient to conclude that $T(\mu\star\nu)=T(\mu)\circ T(\nu)$ because $G$ is not abelian\quext{D'où le fait que je ne considère pas ceci comme bien compris.}.

%+++++++++++++++++++++++++++++++++++++++++++++++++++++++++++++++++++++++++++++++++++++++++++++++++++++++++++++++++++++++++++
\section{A boundary aware Sobolev space}
%+++++++++++++++++++++++++++++++++++++++++++++++++++++++++++++++++++++++++++++++++++++++++++++++++++++++++++++++++++++++++++

We want to define \( H_0^1(\Omega)\) as the subset of \( H^1(\Omega)\) made of the functions vanishing on \( \partial \Omega\).  But since the elements of \( H^1(\Omega)\) are class of functions, and since \( \partial\Omega\) is of zero measure, such a definition makes no sense. Instead we define
\begin{definition}      \label{DEFooFICWooBWCDyO}
    The space \( H^1_0(\Omega)\) is the closure of \( \swD(\Omega)\) in \( H^1(\Omega)\).
\end{definition}
Notice that \( \swD(\Omega)\) is included in \( H^1(\Omega)\), so that the definition makes sense.

\begin{normaltext}
    The intuitive setting of this definition is the following. If \( v\in H^1_0(\Omega)\), we have a sequence \( v_i\in \swD(\Omega)\) such that
    \begin{equation}
        \| v-v_i \|_{H^1(\Omega)}\to 0.
    \end{equation}
    If one forgets about the classes, for each \( i\) one has \( v_i(x)=0\) when \( x\in\partial\Omega\), so with the limit \( v=0\) on \( \partial\Omega\).

    Due to the class stuff, this is not the truth, but the motivation for the definition~\ref{DEFooFICWooBWCDyO}.
\end{normaltext}

\begin{lemma}[\cite{ooXRCOooCFWVg}]
    The formula
    \begin{equation}
        | u |_{1,\Omega}=\| \nabla u \|_{L^2(\Omega)}=\left( \sum_{i=1}^d\int_{\Omega}\left( \frac{ \partial u }{ \partial x_i } \right)^2 \right)^{1/2}
    \end{equation}
    is a norm\footnote{Definition~\ref{DefNorme}.} on \( H_0^1(\Omega)\).
\end{lemma}

\begin{proof}
    The condition \( | \lambda u |_{1,\Omega}=| \lambda | | u |_{1,\Omega}\) is immediate. The triangular inequality is shown in much the same way as the one for the euclidian norm, see \eqref{EQooRYNYooTzZpPz}.

    The point to be proven is that \( | u |_{1,\Omega}=0\) only if \( u=0\) in the sense of the classes. Let \( | u |_{1,\Omega}=0\); since each of the \( d\) integrals is positive we have
    \begin{equation}
        \int_{\Omega}\left( \frac{ \partial u }{ \partial x_i } \right)^2=0
    \end{equation}
    for every \( i=1,\ldots, d\). Since \( (\partial_i u)^2\geq 0\) we deduce \( \partial_iu=0\) almost everywhere. Now, from a dimensional generalization of proposition~\ref{PropLGoLtcS} we get the conclusion\quext{If you know a precise statement of the ``dimensional generalization'', let me know.}.
\end{proof}

So now we consider the metric space
\begin{equation}
    \big( H^1_0(\Omega),| . |_{1,\Omega} \big).
\end{equation}
The norm on \( H^1(\Omega)\) given by the inner product \eqref{EQooQRMKooLaMpcp} can be written
\begin{equation}
    \| u \|_{H^1(\Omega)}=\| u \|_{L^2(\Omega)}+\| \nabla u \|_{L^2(\Omega)}=\| u \|_{L^2(\Omega)}+| u |_{1,\Omega}.
\end{equation}

\begin{lemma}       \label{LEMooEVQKooYoZmbH}
    Let \( v_i\in H_0^1(\Omega)\) satisfying \( v_i\stackrel{H^1(\Omega)}{\longrightarrow}v\). Then
    \begin{enumerate}
        \item
            \( v_i\stackrel{H_0^1(\Omega)}{\longrightarrow}v\)
        \item
            \( v_i\stackrel{L^2(\Omega)}{\longrightarrow}v\)
        \item
            \( \| v_i \|_{L^2}\to \| v \|_{L^2}\)
        \item
            \( | v_i |_{1,\Omega}\to | v |_{1,\Omega}\).
    \end{enumerate}
\end{lemma}

\begin{proof}
    By hypothesis we have
    \begin{equation}
        \| v_i-v \|_{H^1}=\| v_i-v \|_{L^2}+| v_i-v |_{1,\Omega}\to 0.
    \end{equation}
    Since the two terms are positive, they separately converge to \( 0\). Thus \( \| v_i-v \|_{L^2}\to 0\) and \( | v_i-v |_{1,\Omega}\to 0\). This means \( v_i\stackrel{L^2}{\longrightarrow}v\) and \( v_i\stackrel{H_0^1}{\longrightarrow}v\).

    The two last statement about the norms are immediate consequences of the continuity of the norm.
\end{proof}

\begin{theorem}[Poincaré inequality\cite{ooXRCOooCFWVg}]        \label{THOooMIHQooYShOps}
    Let \( \Omega\) be an open part of \( \eR^d\) which is bounded in at least one direction. There exists a constant \( C\) (which can depend on \( \Omega\)) such that
    \begin{equation}
        \| v \|_{L^2(\Omega)}\leq C\| v \|_{1,\Omega}
    \end{equation}
    for every \( v\in H_0^1(\Omega)\)
\end{theorem}

\begin{proof}
Let us first consider \( v\in\swD(\Omega)\) and extend \( v\) with \( 0\) outside \( \Omega\). For the sake of notational simplicity we suppose that \( \Omega\) is bounded in the direction \( x_1\), that is \( \Omega\subset \{ x\tq x_1\in \mathopen] a , b \mathclose[ \}\).

Since \( v\) has compact support in \( \Omega\), we have \( v(a,x')=0\) for every \( x'\in \eR^{d-1}\) and we can write
    \begin{equation}
    v(x_1,x')=   \int_a^{x_1}(\partial_1v)(t,x')dt=\langle 1, \partial_1v\rangle_{L^2\big( \mathopen] a , x_1 \mathclose[ \big)},
    \end{equation}
    so that
    \begin{subequations}
        \begin{align}
            | v(x_1,x') |^2&\leq \langle 1, 1\rangle \langle \partial_1v, \partial_1v\rangle \\
            &=(x_1-a)\int_a^{x_1}| \partial_1 v|^2\\
            &\leq(b-a)\int_a^b| \partial_1v |^2.
        \end{align}
    \end{subequations}
    Let us integrate that inequality over \( x'\in \eR^{d-1}\):
    \begin{equation}
        \int_{\eR^{d-1}}| v(x_1,x') |^2dx'\leq (b-a)\int_{\eR^{d-1}}\int_a^b| \partial_1v |^2=(b-a)\int_{\eR^d}| \partial_1v |^2=(b-a)\| \partial_1v \|^2_{L^2(\eR^d)}.
    \end{equation}
    We used the Fubini theorem~\ref{ThoFubinioYLtPI}. This is allowed because \( v\) is compactly supported, so that \( | \partial_1v |\) is integrable on \( \eR^{d-1}\times \mathopen] a , b \mathclose[\).

    We integrate with respect to \( x_1\) over \( \mathopen] a , b \mathclose[\). On the left we have
        \begin{equation}
            \int_a^b\int_{\eR^{d-1}}| v(x_1,x') |^2dx'dx_1=\int_{\eR^d}| v(x) |^2dx,
        \end{equation}
        And on the right the integration is a simple multiplication by \( (b-a)\):
        \begin{equation}
            \| v \|^2_{L^2(\eR^d)}\leq (b-a)^2\| \partial_1v \|^2_{L^2(\eR^d)}.
        \end{equation}
        Now we have
        \begin{equation}
            \| v \|^2_{L^2}\leq (b-a)^2\| \partial_1v \|^2\leq C\sum_{i=1}^d\| \partial_1v \|^2=C| v |_{1,\Omega}
        \end{equation}
        where \( C\) is a majoration of \( (b-a)^2\) and \( 1\).

        The result is proved for \( v\in\swD(\Omega)\).

        Let now consider \( u\in H_0^1(\Omega)\). By definition, there exists a sequence \( v_i\in\swD(\Omega)\) such that
        \begin{equation}
            v_i\stackrel{H^1(\Omega)}{\longrightarrow}u.
        \end{equation}
        For each \( i\) we have \( \| v_i \|^2_{L^2(\eR^d)}\leq C| v_i |_{1,\Omega}\). Taking the limit and using the lemma~\ref{LEMooEVQKooYoZmbH} to enter the limits inside the norms we get the result.
\end{proof}

%+++++++++++++++++++++++++++++++++++++++++++++++++++++++++++++++++++++++++++++++++++++++++++++++++++++++++++++++++++++++++++
\section{Smooth diffeomorphisms}
%+++++++++++++++++++++++++++++++++++++++++++++++++++++++++++++++++++++++++++++++++++++++++++++++++++++++++++++++++++++++++++

We already defined the Sobolev spaces \( H^s(\eR^d)\) in definition~\ref{DEFooWEAQooAIWBwx}. Let \( U\) and \( V\) be open parts of \( \eR^d\) and consider a \(  C^{\infty}\) diffeomorphism \( \psi\colon U\to V\) whose derivatives are bounded.

For \( u\in  C^{\infty}(U)\) we write
\begin{equation}        \label{EQooHSVBooIZRxzh}
    \begin{aligned}
        Tu\colon V&\to \eC \\
        x&\mapsto (u\circ \psi^{-1})(x).
    \end{aligned}
\end{equation}

A diffeomorphism \( \psi\colon U\to V\) applies to the functions by
\begin{equation}        \label{EQooMZSEooATfSTR}
    \begin{aligned}
        \psi\colon \Fun(U)&\to \Fun(V) \\
        u&\mapsto u\circ\psi^{-1}.
    \end{aligned}
\end{equation}

\begin{propositionDef}      \label{PROPooXAOKooQSBKHg}
If \( T\in \swD'(U)\) we define \( \psi T\) by
\begin{equation}
    \langle \psi T, \varphi\rangle_V=\langle T, J\cdot(\varphi\circ \psi)\rangle_U
\end{equation}
for every \( \varphi\in\swD'(V)\).

\begin{enumerate}
    \item       \label{ITEMooXHELooYhXNRs}
        This defines a distribution \( \psi T\in \swD'(V)\).
    \item       \label{ITEMooGDCYooVDFpuy}
        The so defined map \( \psi\colon \swD'(U)\to \swD'(V)\) is bijective.
    \item       \label{ITEMooNGSJooEdRgHt}
        The so defined map \( \psi\colon \swD'(U)\to \swD'(V)\) is continuous.
\end{enumerate}

\end{propositionDef}

\begin{proof}
    We have to show that the map \( \psi T\colon \swD(V)\to \eC\) is continuous or, in other words, if \( \varphi_n\stackrel{\swD(V)}{\longrightarrow}0\) we have to show that \( \langle \psi T, \varphi_n\rangle_V\to 0\) in \( \eC\).

    In terms of the seminorms we have to prove that for every \( m\in \eN\) and every compact \( K\subset V\) we have
    \begin{equation}
        p_{K,m}\big( J\cdot(\varphi_n\circ \psi) \big)\to 0
    \end{equation}
    where \( J\) is the Jacobian of \( \psi\) and
    \begin{equation}
        p_{K,m}(f)=\sum_{| \alpha |\leq m}\| \partial^{\alpha}f \|_{K,\infty}.
    \end{equation}
    By definition of \( \psi T\) and the fact that \( T\) is continuous we have equivalence of these three facts:
    \begin{subequations}
        \begin{align}
            \langle \psi T, \varphi_n\rangle_V\stackrel{\eC}{\longrightarrow}0\\
            \langle  T, J\cdot (\varphi_n\circ \psi)  \rangle_U\stackrel{\eC}{\longrightarrow}0\\
            J\cdot(\varphi_n\circ\psi)\stackrel{\swD(U)}{\longrightarrow}0.
        \end{align}
    \end{subequations}
    We are going to prove the last one. The first step is to use Leibnitz formula and some obvious notations like \( \alpha-\beta\) where \( \alpha\) and \( \beta\) are multiindex:
    \begin{equation}
        \sum_{| \alpha |\leq m}\| \partial^{\alpha}J\cdot(\varphi_n\circ\psi) \|_{K}=\sum_{| \alpha |\leq m}\| \sum_{\beta\leq \alpha}(\partial^{\beta}J)\cdot \partial^{\alpha-\beta}(\varphi_n\circ\psi) \|_K
    \end{equation}
    Since \( \psi\) is a diffeomorphism, \( J\) is bounded and from the fact that \( K\) is compact, each of the \( \partial^{\gamma}J\) is bounded. Thus we have
    \begin{equation}
        p_{m,K}(J\cdot (\varphi_n\circ \psi))\leq \sum_{| \alpha |\leq m}\|c \sum_{\beta\leq \alpha} \partial^{\beta}(\varphi_n\circ\psi) \|_{K}
    \end{equation}
    where the constant \( c\) depends on \( K\) and \( m\).

    We show by induction on \( | \beta |\) that for every sequence \( ( \varphi_n)\) in \( \swD(V)\) with \( \varphi_n \stackrel{\swD(V)}{\longrightarrow}0\) we have \(  \| \partial^{\beta}(\varphi_n\circ\psi) \|_{K,\infty}\to 0\). Starting with \( | \beta |=0\) we have
    \begin{equation}
        \| | \varphi_n\circ\psi | \|_K=\| \varphi_n \|_{\psi(K)}.
    \end{equation}
    Since \( \psi(K)\) is compact (proposition~\ref{ThoImCompCotComp}), the latter is one of the seminorms on \( V\), so that \( \| \varphi_n \|_\psi(K)\to 0\). For the induction we compute
    \begin{subequations}
        \begin{align}
            \partial_i\big( \partial^{\beta}(\varphi_n\circ\psi) \big)&=\partial^{\beta}\big( \partial_i(\varphi_n\circ\psi) \big)\\
            &=\partial^{\beta}\left(  x\mapsto  \sum_k\frac{ \partial \varphi_n }{ \partial y_k }\big( \psi(x)\big)\frac{ \partial \psi_k }{ \partial x_i }(x) \right)\\
            &\leq c\partial^{\beta}\left( \sum_k\frac{ \partial \varphi_n }{ \partial y_k }\circ\psi \right)
        \end{align}
    \end{subequations}
    where we have done the majoration \( \frac{ \partial \psi_k }{ \partial x_i }(x)\leq c\) which is legal because the function is continuous on a compact and there is a finite number of couple \( (k,i)\), so the bound can be taken uniformly with respect to \( x\), \( k\) and \( i\) in the same time. Now the function \( \frac{ \partial \varphi_n }{ \partial y_n }\) is an element of \( \swD(V)\) and by the induction hypothesis,
    \begin{equation}
       \| \partial^{\beta}\left( \sum_k\frac{ \partial \varphi_n }{ \partial y_k }\circ\psi \right) \|_K\to 0.
    \end{equation}

    This finishes the proof of point~\ref{ITEMooXHELooYhXNRs}. We pass to the points~\ref{ITEMooGDCYooVDFpuy} and~\ref{ITEMooNGSJooEdRgHt}.
    \begin{subproof}
        \item[\( \psi\colon \swD'(U)\to \swD'(V)\) is surjective]

            Let \( T_2\in\swD'(V)\). We define \( T_1\) on \( \swD(U)\) by \( T_1=\psi^{-1} T_2\); proposition~\ref{PROPooXAOKooQSBKHg} shows that \( T_1\) is a distribution over \( \swD(U)\). Moreover \( \psi T_1=T_2\). So \( \psi\) is surjective.

        \item[\( \psi\colon \swD'(U)\to \swD'(V)\) is injective]

            Suppose \( \psi T_1=0\), so
            \begin{equation}
                \langle \psi T_1, \phi\rangle_V =\langle T_1, J\times (\phi\circ\psi)\rangle_U=0
            \end{equation}
            for every \( \phi\in \swD(V)\). Since \( J\) is itself a \(  C^{\infty}\) function, in fact every element \( \varphi\in \swD(U)\) can be written under the form \( J\times (\phi\circ\psi)\) for some \( \phi\in\swD(V)\). We deduce \( T_1=0\).

        \item[\( \psi\colon \swD'(U)\to \swD'(V)\) is continuous]

            Let \( T_n\stackrel{\swD'(U)}{\longrightarrow}O\); we have to show that \( \psi T_n\stackrel{\swD'(V)}{\longrightarrow}0\). We use the proposition~\ref{PropEUIsNhD}: for \( \phi\in\swD(V)\) we have to show that \( (\psi T_n)(\phi)\to 0\), that is:
            \begin{equation}
                \langle \psi T_n, \phi\rangle_V=\langle T_n, J\times (\phi\circ\psi)\rangle_U.
            \end{equation}
            Since \( T_n\) converges to zero in \( \swD'(U)\), the limit of the right hand side is zero.
    \end{subproof}
\end{proof}

\begin{lemma}       \label{LEMooJHFUooWdAlar}
    Let a smooth diffeomorphism \( \psi\colon U\to V\) and a smooth map \( u\colon U\to \eC\). We look at the composition
    \begin{equation}
        u\circ\psi^{-1}\colon V\to \eC.
    \end{equation}
    We denote by \( y\) the coordinates on \( V\) and by \( x\) the ones on \( U\). Let \( \alpha\) be a \( y\)-multiindex. Then
    \begin{equation}
        \partial^{\alpha}(u\circ\psi^{-1})(y)=\sum_{| I |\leq | \alpha |}c_{I \alpha}(y)(\partial^{I}u)\big( \psi^{-1}(y) \big)
    \end{equation}
    for some bounded functions \( c_{I \alpha}\).

    Written under a compact form,
    \begin{equation}
        \partial^{\alpha}(u\circ\psi^{-1})=\sum_{| I |\leq | \alpha |}c_{I \alpha}(\partial^Iu)\circ\psi^{-1}.
    \end{equation}
\end{lemma}

\begin{proof}
    To be clear: \( \beta\) is a \( x\)-mutiindex. For \( \alpha\) of length \( 1\) we have
    \begin{equation}
        \frac{ \partial  }{ \partial y_i }\big( u\circ\psi^{-1} \big)(y)=\sum_k\frac{ \partial u }{ \partial x_k }\big( \psi^{-1}(y) \big)\frac{ \partial \psi_k^{-1} }{ \partial y_i }(y).
    \end{equation}
    Then applying the Leibnitz rule we will have multi-derivatives of \( u\) with respect to \( x\), always taken at the point \( \psi^{-1}(y)\) and then derivatives of the components of \( \psi^{-1}\) with respect to \( y\) taken at the point \( y\). The derivatives on \( u\) are derivatives with respect to \( x\), and each derivative with respect to \( y\) on \( u\circ\psi^{-1}\) will create derivatives on $u$ with respect to all the components of \( x\). Thus the sum is over the \( x\)-multiindiex of size lower or equal to the number of \( y\)-derivatives on \( u\circ\psi^{-1}\).
\end{proof}

\begin{proposition}[\cite{ooRRCQooJwnUgB}]
    Let \( m\in \eN\) and a diffeomorphism \( \psi\colon U\to V\) such that all the derivatives of \( \psi\) and \( \psi^{-1}\) are bounded. We consider the map
    \begin{equation}
        \begin{aligned}
            \psi\colon \Fun(U)&\to \Fun(V) \\
            u&\mapsto u\circ\psi^{-1}.
        \end{aligned}
    \end{equation}
    \begin{enumerate}
        \item       \label{ITEMooNJZOooOrzQIT}
            We have
            \begin{equation}
                \psi\colon L^2(U)\to L^2(V)
            \end{equation}
            surjectively.
        \item
    We have the diffeomorphisms
    \begin{subequations}
        \begin{align}
            \psi\colon H^m(U)\to H^m(V)\\
            \psi\colon  C^{\infty}_0(U)\to  C^{\infty}_0(V)\\
            \psi\colon \swD'(U)\to \swD'(V).
        \end{align}
    \end{subequations}
    The last one is defined by
    \begin{equation}
        \langle \psi T, \varphi\rangle_V=\langle T, J\cdot(\varphi\circ\psi)\rangle_U.
    \end{equation}
    \end{enumerate}
\end{proposition}

\begin{proof}
    The hypothesis on \( \psi\) implies in particular that \( d\psi\) and \( d\psi^{-1}\) are bounded.
    \begin{subproof}
    \item[\( \psi\colon L^2(U)\to L^2(V)\) is surjective]
        We consider the operation \( \psi\) on the functions as explained in equation \eqref{EQooMZSEooATfSTR}. We know that \( x\mapsto | f(x) |^2\) is integrable on \( U\). Thus the change of variable theorem~\ref{THOooUMIWooZUtUSg}\ref{ITEMooAJGDooGHKnvj} the function \( | f |^2\circ\psi^{-1}\times | J |\) is integrable on \( V\). That is the function \( y\mapsto | f\big( \psi^{-1}(y) \big) |^2| J_{\psi}(y) |\). Recall that
        \begin{equation}
            J_{\psi}(a)=\det(d\psi_a),
        \end{equation}
        and that by hypothesis, this is bounded. Also by hypothesis, the inverse is bounded. If \( c\) is a majoration of \( | 1/J |\), then
        \begin{equation}
            | f\circ\psi |^2=| f\circ\psi |^2\times | J |\times | 1/J |\leq c| f\circ\psi |^2\times | J |.
        \end{equation}
        The right hand side is integrable, so we deduce that \( | f\circ\psi |^2\) is integrable. At the end of the day, the function \( f\in L^2(V)\) is the image by \( \psi\) of \( f\circ\psi^{-1}\) which belongs to \( L^2(U)\).

    \item[\( \psi\colon \swD(U)\to \swD(V)\) is bijective and continuous]
        The bijection is the fact that \( \psi\) itself is a bijection of class \(  C^{\infty}\), and that it transforms a compact into a compact. The continuity is correct for topology of the norm \( \| . \|_{\infty}\) because \( \psi\) turns out to be an isometry.

    \item[\( \psi\colon \swD'(U)\to \swD'(V)\) is bijective and continuous]
        That is the proposition~\ref{PROPooXAOKooQSBKHg}.

    \item[\( \psi\colon H^m(U)\to H^m(V)\)]

        We denote by \( y\) the coordinates on \( V\) and by \( x\) the ones on \( U\). Let \( u\in H^m(U)\), let \( \phi\in\swD(V)\) and a \( y\)-multiindex \( \alpha\) with \( | \alpha |\leq m\). By definition of the weak derivative (and the fact that \( \phi\) is compactly supported, so that it vanishes at the boundaries of \( V\)) we have
        \begin{subequations}
            \begin{align}
                \langle u\circ\psi^{-1}, \partial^{\alpha}\phi\rangle_V &=\langle u, | J_{\psi}| \partial^{\alpha}\phi \rangle_V   \label{subEQooNSBJooCbtTNI}    \\
                &=\langle u, \sum_{| I |\leq | \alpha |}c_{\alpha I}\partial^I\tilde \phi\rangle_U  \label{SUBEQooYBCCooIdpsxE}   \\
                    &=\sum_{ |I|\leq |\alpha|}\langle u, c_{\alpha I}\partial^{ I}\tilde\phi\rangle_U \\
                    &=\sum_{ |I|\leq |\alpha|}\langle \bar c_{\alpha I}u, \partial^{ I}\tilde \phi\rangle_U\\
                    &=\sum_{ |I|\leq |\alpha|}\langle \partial^{ I}(c_{\alpha I}u), \tilde \phi\rangle_U       \label{SUBEQooCSHLooTdTHvf}
            \end{align}
        \end{subequations}
        Justifications:
        \begin{itemize}
            \item
                For \eqref{subEQooNSBJooCbtTNI}: the change of variable formula under the form \eqref{EQooQKARooELPCFO}.
            \item
                For \eqref{SUBEQooYBCCooIdpsxE}, we wrote \( \tilde \phi=\phi\circ\psi\) and used the lemma~\ref{LEMooJHFUooWdAlar} with \( \partial^{\alpha}(\tilde \psi\circ\psi^{-1})\). We have included the function \( | J |\) in the coefficients \( c_{\alpha I}\).

    \item
        For \eqref{SUBEQooCSHLooTdTHvf} we redefined \( c_{\alpha I}\) in order to include \( (-1)^{|  I |}\) and the complex conjugation.

        \end{itemize}

        Up to now we have
        \begin{equation}
          \langle u\circ\psi^{-1}, \partial^{\alpha}\phi\rangle_V =  \sum_{| I |\leq | \alpha |}\langle \partial^{I}(c_{\alpha I}u), \tilde \phi\rangle_U
        \end{equation}
        where \( \tilde \phi=\phi\circ\psi\in \swD(U)\) and the coefficients (and all the derivatives) \( c_{\alpha I}\) are smooth and bounded because they are combinations of derivatives of the \( \psi\)'s components.

        We can again use the Leibnitz rule on the derivative \( \partial^{I}(c_{\alpha I} u)\). We will get all the derivatives \( \partial^Ju\) (\( J\) are sub-indices of \( I\)) with coefficients that are again derivatives of \( c_{\alpha I}\). Thus we write
        \begin{equation}
            \partial^I(c_{\alpha I}u)=\sum_{J\leq I}c'_{\alpha J}\partial^Ju,
        \end{equation}
        and since \( J\) takes any value contained in \( I\) and \( I\) every values with length lower than \( | \alpha |\) we can as well write
        \begin{equation}
             \langle u\circ\psi^{-1}, \partial^{\alpha}\phi\rangle_V =  \sum_{| I |\leq | \alpha |}\langle \partial^{I}(c_{\alpha I}u), \tilde \phi\rangle_U =\sum_{| I |\leq | \alpha |}\langle c'_{\alpha I}\partial^Iu, \tilde \phi\rangle_U.
        \end{equation}

        We have \( | I |\leq | \alpha |\leq m\), so \( \partial^Iu\in L^2(U)\). Since \( c'_{\alpha I}\) is bounded, the product \( c'_{\alpha I}\partial^Iu\) belongs to \( L^2(U)\) too. The integral change of variable formula \eqref{EQooQKARooELPCFO} reads in our case:
        \begin{equation}
            \langle f, g\rangle_U=\langle f\circ \psi^{-1}, (g\circ\psi^{-1})| J^{-1} |\rangle_V,
        \end{equation}
        with \( \tilde \phi\) in the role of \( g\):
        \begin{equation}
            \langle f, \tilde g\rangle =\langle | J^{-1} |(f\circ\psi^{-1}), \phi\rangle_V
        \end{equation}
        Now with \( \sum_{| I |\leq | \alpha |}c'_{\alpha I}\partial^Iu\) in the role of \( f\) we have \( | J^{-1} |(f\circ\psi^{-1})\in L^2(V)\) because of point~\ref{ITEMooNJZOooOrzQIT} and the fact that \( | J^{-1} |\) is bounded, so that the product makes sense and we have shown that the function
        \begin{equation}
            f'=| J^{-1} |\big( \sum_{| I |\leq | \alpha |}c'_{\alpha I}\partial^Iu \big)\circ\psi^{-1}
        \end{equation}
        is an element of \( L^2(V)\) satisfying
        \begin{equation}
              \langle u\circ\psi^{-1}, \partial^{\alpha}\phi\rangle_V =   \langle f', \phi\rangle_V
        \end{equation}
        for every \( \phi\in\swD(V)\). This condition is the definition of \(   (-1)^{\alpha} \partial^{\alpha}(u\circ\psi^{-1})\). Thus we have
        \begin{equation}
            \partial^{\alpha}(u\circ\psi^{-1})=(-1)^{\alpha}f'\in L^2(V).
        \end{equation}
        Notice that the fact that \( \partial^{\alpha}(u\circ\psi^{-1})\) exists is not a big deal: it always weakly exists. The very point we shown is that the derivative is \( L^2(V)\).

        Since \( \alpha\) is  amultiindex of any size up to \( m\) we shown that \( u\circ\psi^{-1}\in H^m(V)\).


        \item[\( \psi\colon H^m(U)\to H^m(V)\) is continuous]

            Let \( u\in H^m(U)\) and a smooth diffeomorphism \( \psi\colon U\to V\); we consider \( \psi u=u\psi^{-1}\) and we want to prove an estimation like \( \| \psi u \|_{H^m(V)}\leq C\| u \|_{H^m(U)}\). We use the proposition~\ref{PROPooVEMGooYKhMFy} with \( \lambda=1\), \( V=L^2(V)\) and \( A=\swD(V)\):
            \begin{equation}
                \| \partial^{\alpha}(\psi u) \|_{L^2(V)}=\sup\{ | \langle \partial^{\alpha}(\psi u), \phi\rangle_V |\st \phi\in\swD(V)\text{ and }\| \phi \|_{L^2(V)}\leq 1 \}.
            \end{equation}
            Using lemme~\ref{LEMooJHFUooWdAlar} we have
            \begin{equation}
                \clubsuit=\langle \partial^{\alpha}(\psi u), \phi\rangle_V=\sum_{| I |\leq | \alpha |}\langle c_{\alpha I}(\partial^Iu)\circ\psi^{-1}, \phi\rangle_V\leq C\sum_{| I |\leq | \alpha |}\langle (\partial^Iu)\circ \psi^{-1}, \phi\rangle_V.
            \end{equation}
            The first majoration is \( C=\sup_{I,x}c_{\alpha I}(x)\). In the next lines, we will often modify \( C\) in order to include other majorations. With the change of variable formula \eqref{EQooQKARooELPCFO} and including a majoration of \( | J_{\psi} |\) in the constant \( C\),
            \begin{equation}
                \clubsuit=\sum_{| I |\leq | \alpha |}\langle \partial^Iu, (\phi\circ\psi)| J_{\psi} |\rangle_U\leq C\sum_{| I |\leq | \alpha |}\langle \partial^Iu, \phi\circ\psi\rangle_U.
            \end{equation}
            Now, again using the change of variable formula (but in the reverse sense) we have
            \begin{equation}
                \| \phi\circ\psi \|^2_{L^2(u)}=\langle \phi\circ\psi, \phi\circ\psi\rangle_U=\langle \phi, \phi| J^{-1} |\rangle_V\leq K\| \phi \|^2_{L^2(V)}
            \end{equation}
            where \( K\) is a majoration of \( | J^{-1} |\). We can continue the computation:
            \begin{subequations}
                \begin{align}
                    \| \partial^{\alpha}(\psi u) \|_{L^2(V)}&=\sup\{ | \clubsuit |\st \phi\in\swD(V)\text{ and }\| \phi \|_{L^2(V)}\leq 1 \}\\
                    &\leq \sup\{ C|\sum_{| I |\leq | \alpha |}\langle \partial^Iu, \phi\circ\psi\rangle_U|\st \phi\in\swD(V)\text{ and }\| \phi \|_{L^2(V)}\leq 1 \}\\
                    &\leq C\sum_{| I |\leq | \alpha |} \sup\{ |\langle \partial^Iu, \varphi\rangle_U|\st \varphi\in\swD(U)\text{ and }\| \varphi \|_{L^2(U)}\leq K  \}\\
                    &\leq C\sum_{| I |\leq | \alpha |}\| \partial^Iu \|_{L^2(U)}.
                \end{align}
            \end{subequations}
            We have:
            \begin{equation}
                \| \partial^{\alpha}(\psi u) \|_{L^2(V)}\leq C\sum_{| I |\leq | \alpha |}\| \partial^Iu \|_{L^2(U)}\leq C\| u \|_{H^m(U)}
            \end{equation}
            because of the expression \eqref{EQooMCWMooKKTqzM} of the norm on \( H^m(U)\). Now we can conclude:
            \begin{equation}
                \| \psi u \|_{H^m(V)}=\sum_{| \alpha |\leq m}\| \partial^{\alpha}(\psi u) \|_{L^2(V)}\leq C\sum_{| \alpha |\leq m}\| u \|_{H^m(V)} = a C\| u  \|_{H^m(U)}
            \end{equation}
            where \( a\) is the number of terms in the sum.

        \item[\( \psi\colon H^m(U)\to H^m(V)\) is a diffeomorphism]

            Replacing \( \psi\) by \( \psi^{-1}\) shows that \( \psi^{-1}\colon H^m(V)\to H^m(U)\) is continuous. Don't forget to show that this is actually the inverse:
            \begin{equation}
                \psi^{-1}(\psi u)=(\psi u)\circ \psi^{-1}=u\circ\psi\circ\psi^{-1}=u.
            \end{equation}
            Notice that in the later equation, the symbol ``$\psi$'' denotes sometimes the map \( \psi\colon U\to V\) and sometimes the map \( \psi\colon \Fun(U)\to \Fun(V)\).
    \end{subproof}
\end{proof}

\begin{remark}
    The hypothesis on the fact that the derivatives of \( \psi\) are bounded is not a weak one. As an example, the map
    \begin{equation}
        \begin{aligned}
        \psi\colon \mathopen] 0 , 1 \mathclose[&\to \mathopen] 0 , 1 \mathclose[ \\
            x&\mapsto x^2
        \end{aligned}
    \end{equation}
    is a diffeomorphism with Jacobian given by \( J_{\psi}(x)=2x\), so that \( 1/J\) is not bounded on \( \mathopen] 0 , 1 \mathclose[\).
\end{remark}

%+++++++++++++++++++++++++++++++++++++++++++++++++++++++++++++++++++++++++++++++++++++++++++++++++++++++++++++++++++++++++++
\section{Restriction operator}
%+++++++++++++++++++++++++++++++++++++++++++++++++++++++++++++++++++++++++++++++++++++++++++++++++++++++++++++++++++++++++++

\begin{definition}
    Let \( j=1,\ldots, d\) and \( h\in \eR\). We define the \defe{translation operator}{translation!operator}
    \begin{equation}
        \tau_{j,h}\colon \Fun(\eR^d)\to \Fun(\eR^d)
    \end{equation}
    given by the formula
    \begin{equation}
        \tau_{j,h}(f)(x)=f(x-h e_j)
    \end{equation}
    where \( e_j\) is the canonical unit vector.
\end{definition}

The adjoint operator (definition~\ref{DEFooERIYooIIRLuy}) \( \tau^*_{j,h}\colon L^2(\eR^d)\to L^2(\eR^d)\) is
\begin{equation}
    (\tau^*_{j,h}u)(x)=u(x+he_j).
\end{equation}
Indeed we have
\begin{equation}
    \langle \tau(u), v\rangle_{L^2(\eR^d)}=\int_{\eR^d}\overline{ u(x-he_j) }v(x)dx=\int_{\eR^d}\overline{ u(y) }v(y+he_j)dy=\langle u, \tau^*_{j,h}v\rangle_{L^2(\eR^d)}
\end{equation}
by a simple change of variable \( y=x-he_j\). We immediately see that \( \tau_{j,h}\) is an unitary operator of \( L^2(\eR^d)\).

If \( \Omega\) is an open part of \( \eR^d\) we define the restriction operator \( r_{\Omega}\) by
\begin{equation}
    r_{\Omega}=u|_{\Omega}.
\end{equation}

We define
\begin{equation}
    \eR^d_{+}=\{ x\in \eR^d\st x_n>0 \},
\end{equation}
and the same for \( \eR^d_-\). We denote by \( r^{\pm}\) the restriction operator to \( \eR^d_{\pm}\). The extension operator is the extension by \( 0\) outside \( \Omega\):
\begin{equation}
    (e_{\Omega}f)(x)=\begin{cases}
        f(x)    &   \text{if } x\in \Omega\\
        0    &    \text{else. }
    \end{cases}
\end{equation}

\begin{normaltext}
Do you believe that there is a difference between \( \swD\big( \overline{ \eR^d_+ }\big)\) and \( \swD(\eR^d_+)\)? Let us see with \( d=1\). Since the support is closed, a function in \( \swD(\eR_+)\) has to be zero on \( \mathopen] -\infty , \delta \mathclose]\) with \( \delta>0\).

On the other hand, a function in \( \swD\big( \overline{ \eR^d_+ } \big)\) can be nonzero immediately after \( x=0\):
\begin{equation}
    f(x)=\begin{cases}
        0    &   \text{if } x<0\\
        x    &    \text{if  } x\in\mathopen[ 0 , 1 \mathclose]\\
        \text{something}    &    \text{if } x\in \mathopen[ 1 , 2 \mathclose]\\
        0    &   \text{if } x>2.
    \end{cases}
\end{equation}
where ``something'' stands for something smooth.
\end{normaltext}

\begin{proposition}[\cite{ooRRCQooJwnUgB}]      \label{PROPooCXYRooPTgSLX}
    Some densities.
    \begin{enumerate}
        \item
            The set \( \swD(\eR^d)\) is dense in \( H^m(\eR^d)\) for every integer \( m\geq 0\).
        \item
            The set \( r^+\swD(\eR^d)\) is dense in \( H^m(\eR^d_+)\) for every \( m\geq 0\).
    \end{enumerate}
\end{proposition}

\begin{theorem}[\cite{ooRRCQooJwnUgB}]
    There exists a linear continuous operator
    \begin{equation}
        p_{(m)}\colon H^m(\eR^d_+)\to H^m(\eR^d)
    \end{equation}
    such that
    \begin{equation}
        (r^+\circ p_{(m)})u=u
    \end{equation}
    for every \( u\in H^m(\eR^d_+)\).
\end{theorem}

\begin{proof}
    The map \( r^+\) a defined as \( r^+\colon \Fun(\eR^d)\to \Fun(\eR^d)\). Since the norms \( L^2(\eR^d_+)\) are majored by the ones \( L^2(\eR^d)\) we have
    \begin{equation}
        r^+\colon H^m(\eR^d)\to H^m(\eR^d_+).
    \end{equation}

    We initiate by defining the operator \( p_{(m)}\) on a dense subset (see proposition~\ref{PROPooCXYRooPTgSLX}).  Let \( u\in r^+\swD(\eR^d)\). More precisely we choose \( u_0\in\swD(\eR^d)\) and we write \( u=r^+u_0\). We define
    \begin{equation}
        (p_{(m)}u)(x',x_d)=\begin{cases}
            u(x',x_d)    &   \text{if } x_d> 0\\
            u_0(x',0) & \text{if }  x_d=0\\
            \sum_{k=0}^{m-1}a_ku_0(x',-\lambda_kx_d)    &    \text{if } x_d<0.
        \end{cases}
    \end{equation}
    Here are some comments ont the definition.
    \begin{itemize}
        \item
            The value of \( p_{(m)}u\) does not depend on the choice of \( u_0\) among all the elements in \( \swD(\eR^d)\) which restrict to \( u\).
        \item
     The positivity of \( \lambda_k\) makes that \( -\lambda_kx_d>0\), so that the whole makes sense.
 \item
    The numbers \( \lambda_0,\ldots,\lambda_{m-1}\) are strictly positive different reals that are arbitrarily chosen.
\item
    The coefficients \( a_k\) are the solution of the system
    \begin{equation}        \label{EQooTHYTooUovXKT}
        \sum_{k=0}^{m-1}\lambda_k^ja_k=(-1)^j
    \end{equation}
    for \( j=0,\ldots, m-1\).
    \item
        The system \eqref{EQooTHYTooUovXKT} has an unique solution because its matrix is \( A_{kj}=\lambda_k^j\) whose determinant is the non vanishing Vandermonde determinant~\ref{PropnuUvtj}.
    \end{itemize}

    Let us show that \( p_{(m)}(u)\in C^{m-1}(\eR^d)\). The derivatives of \( p_{(m)}u\) with respect to the variables \( x_1\),\ldots, \( x_{d-1}\) do not create problems. Only the ones with respect to \( x_d\) at \( x_d=0\) can be difficult. Now we show by induction that
            \begin{equation}
                (\partial_n^lp_{(m)}u)(x',0)=(\partial_n^lu_0)(x',0)
            \end{equation}
            with several steps.
    \begin{subproof}
        \item[First: \( (\partial_np_{(m)}u)(x',0)=(\partial_nu_0)(x',0)\)]

            On the one hand we have
            \begin{equation}
                \lim_{t\to 0^+} \frac{ (p_{(m)}u)(x',t)-(p_(m)u)(x',0) }{ t }=\lim_{t\to 0^+} \frac{ u_0(x',t)-u_0(x',0) }{ t }=(\partial_nu_0)(x',0).
            \end{equation}
            The last equality comes from the fact that for \( u_0\) we know that the limits with \( t\to 0^+\) and \( t\to 0^-\) are equal. On the other hand, we know that \( \sum_{k=0}^{m-1}a_k=1\), so that
            \begin{subequations}
                \begin{align}
                    \lim_{t\to 0^-} \frac{ (p_{(m)}u)(x',t)-(p_(m)u)(x',0) }{ t }&=\lim_{t\to 0^+} \frac{ \sum_{k=0}^{m-1}a_ku_0(x',-\lambda_kt)-u_0(x',0) }{ t }\\
                    &=\sum_{k=0}^{m-1}a_k\lim_{t\to 0^-} \frac{ u_0(x',-\lambda_kt)-u_0(x',0) }{ t }\\
                    &=-\sum_{k=0}^{m-1}a_k\lambda_k(\partial_n u_0)(x',0).
                \end{align}
            \end{subequations}
            We made the change of variable \( t'=-\lambda_kt\) to compute the limit.

        \item[Second: \( (\partial_n^lp_{(m)}u)(x',x_d)\) with \( x_d<0\)]

            In this case the computation is immediate:
            \begin{equation}
                (\partial_n^lp_{(m)}u)(x',x_d)=\sum_{k=0}^{m-1}a_k(-\lambda_k)^l(\partial_n^lu_0)(x',-\lambda_kx_d).
            \end{equation}

        \item[Induction]

            We are going to prove by induction over \( l\) that
            \begin{equation}        \label{EQooVJITooNRgryX}
                (\partial_n^lp_{(m)}u)(x',0)=(\partial_n^lu_0)(x',0).
            \end{equation}
            The case \( l=1\) is already done. So we compute with \( l+1\) assuming that \eqref{EQooVJITooNRgryX} holds. Once again we have to compute separately the limit with \( t\to 0^+\) and \( t\to 0^-\). We have
            \begin{subequations}
                \begin{align}
                    \lim_{t\to 0^+} \frac{ (\partial_n^lp_{(m)}u)(x',t)-(\partial_n^lp_{(m)}u)(x',0) }{ t }&=\lim_{t\to 0^+} \frac{   (\partial_n^lu_0)(x',t)-  (\partial_n^lu_0)(x',0) }{ t }\\
                &=(\partial^{l+1}_nu_0)(x',0).
                \end{align}
            \end{subequations}
            Now, the limit with negative \( t\). To be computed:
            \begin{equation}    \label{EQooFKHJooPqYWeF}
                \lim_{t\to 0^-} \frac{ (\partial^l_np_{(m)}u)(x',t)-(\partial_n^lu_0)(x',0) }{ t }.
            \end{equation}
            The numerator can be expanded as
            \begin{equation}
                (-1)^l\sum_{k=0}^{m-1}a_k\lambda_k^l(\partial_n^lu_0)(x',-\lambda_kt)-(\partial_n^lu_0)(x',0).
            \end{equation}
            Since \( (-1)^l\sum_{k=0}^{m-1}a_k\lambda_k^l=1\) we can multiply the second term by that and enter \( (\partial_n^lu_0)(x',0)\) in the sum. What we get is
            \begin{equation}
                (-1)^l\sum_{k=0}^{m-1}a_k\lambda_k^l\big( (\partial_n^lu_0)(x',-\lambda_k t)-(\partial_{n}^lu_0)(x',0) \big).
            \end{equation}
            Back to the limit \eqref{EQooFKHJooPqYWeF} we permute the limit and the sum:
            \begin{subequations}
                \begin{align}
                    \lim_{t\to 0^-} \frac{ (\partial^l_np_{(m)}u)(x',t)-(\partial_n^lu_0)(x',0) }{ t }&=(-1)^l\sum_{k=0}^{m-1}\lambda_k^l\lim_{t\to 0^-} \frac{ (\partial_n^lu_0)(x',-\lambda_kt)-(\partial_n^lu_0)(x',0) }{ t }\\
                    &=(-1)^l\sum_{k=0}^{m-1}\lambda_k^l\lim_{t\to 0^+} \frac{    (\partial_n^lu_0)(x',u)- (\partial_n^lu_0)(x',0)   }{ -u/\lambda_k }\\
                    &=(-1)^{l+1}\sum_{k=0}^{m+1}\lambda_k^{l+1}(\partial_n^{l+1}u_0)(x',0)\\
                    &=(\partial^{l+1}_nu_0)(x',0).
                \end{align}
            \end{subequations}
            The induction is finished.
    \end{subproof}
    We are now able to prove that \( p_{(m)}u\) is \( C^{m-1}(\eR^d)\). The only point is to show that \( (\partial^l_nu))\) is continuous at \( x_d=0\). The limit with \( x_d>0\) does not pone any problem. For \( x_d<0\),
    \begin{subequations}
        \begin{align}
            \lim_{x_d\to 0^-}(\partial_n^lp_{(m)}u)(x',x_d)&=\sum_{k=0}^{m-1}a_k(-\lambda_k)^l\lim_{x_d\to 0^-}(\partial_n^lu_0)(x',-\lambda_kx_d)\\
            &=\sum_{k=0}^{m-1}a_k(-\lambda_k)^l(\partial_n^lu_0)(x',0)\\
            &=(\partial_n^lu_0)(x',0)\\
            &=(\partial^l_np_{(m)}u)(x',0).
        \end{align}
    \end{subequations}
    When one tries to derive more than \( m-1\) times, we have a possible discontinuity at \( x_d=0\) since we do not have any formula for the sum \( \sum_{k=0}^{m-1}k\lambda_k^m\).

    The function \(p_{(m)}u\) is compactly supported and the derivatives up to the \( m-1\)\ieme are continuous. The \( m\)\ieme derivatives is continuous on \( x_d>0\) and \( x_d<0\). The whole makes
    \begin{equation}
        p_{(m)}u\in H^m(\eR^d).
    \end{equation}

    We still have to prove that \(p_{(m)}\colon r^+\swD(\eR^d)\to H^m(\eR^d)\) is continuous and extends to \( H^m(\eR^d_+)\). We start proving the inequality
    \begin{equation}
        \| p_{(m)}u \|_{H^m(\eR^d)}\leq C \| u \|_{H^m(\eR^d_+)}
    \end{equation}
    for some constant \( C\) that depend on the \( \lambda_i\)'s. By the definition \ref{EQooMCWMooKKTqzM} we know \( \| v \|_{H^m(\eR^d)}=\sum_{| \alpha |\leq m}\| \partial^{\alpha}v \|_{L^2(\eR^d)}\) and what happens on the hyperplane \( x_d=0\) is of no interest. Let's consider a multiindex \( \alpha\) containing \( l\) times the index \( x_d\); of \( x_d<0\) we have
    \begin{equation}
        (\partial^{\alpha}p_{(m)}u)(x',x_d)=\sum_{k=0}^{m-1}(-\lambda_k)^la_k(\partial^{\alpha}u)(x',-\lambda_kx_d).
    \end{equation}
    There is a finite number of combinations \( \lambda_k^l\) with \( k=0,\ldots, d-1\) and \( l=0,\ldots, m\) (and they are all strictly positive). We majore them all by \( K\). Taking the maximum of the numbers \( | a_k |\) and including it in the \( K\) we have
    \begin{equation}
        \big| (\partial^{\alpha}p_{(m)}u)(x',x_d) \big|\leq K \sum_{k=0}^{m-1}| (\partial^{\alpha}u)(x',-\lambda_k x_d) |.
    \end{equation}
    As far as the \( L^2(\eR^d)\) norm is concerned we have
    \begin{subequations}
        \begin{align}
            \| \partial^{\alpha}p_{(m)}u \|_{L^2(\eR^d_-)}&\leq K\sum_{k=0}^{m-1}\int_{\eR^d_-}\big| (\partial^{\alpha}u)(x',-\lambda_k x_d) \big|^2dx'\otimes dx_d\\
            &=K\sum_{k=0}^{m-1}\int_{\eR^d_+}\lambda_k\big| (\partial^{\alpha}u)(x',y) \big|^2dx'\otimes dy  \label{EQooBTCBooHejadM}\\
            &=K\sum_{k=0}^{m-1}\lambda_k\| \partial^{\alpha}u \|_{L^2(\eR^d_+)}\\
            &\leq K\| \partial^{\alpha}u \|_{L^2(\eR^d_+)}.
        \end{align}
    \end{subequations}
    We made the change of variable \(y=-\lambda_k x_d\) (this is why a \( \lambda_k\) appeared as Jacobian in \eqref{EQooBTCBooHejadM}). In the last line we included the factor \( \sum_{k=0}^{m-1}\lambda_k\) into \( K\).

    From the definition we also have
    \begin{equation}
        \| \partial^{\alpha}p_{(m)}u \|_{L^2(\eR^d_+)}=\| \partial^{\alpha}u \|_{L^2(\eR^d_+)}.
    \end{equation}
    Making the sum (and neglecting the integral over \( x_d=0\)) and redefining \(K \) as \( \max\{ 1,K \}\),
    \begin{subequations}
        \begin{align}
            \| \partial^{\alpha}p_{(m)}u \|_{L^2(\eR^d)}&\leq K \| \partial^{\alpha}u \|_{L^2(\eR^d_+)}+\| \partial^{\alpha}u \|_{L^2(\eR^d_+)}\\
            &=2K\| \partial^{\alpha}u \|_{L^2(\eR^d_+)}.
        \end{align}
    \end{subequations}
    Taking once again the sum over the \( | \alpha |\leq m\) we have the majoration
    \begin{equation}
        \| p_{(m)}u \|_{H^m(\eR^d)}\leq C \| u \|_{H^m(\eR^d_+)}.
    \end{equation}

    We have continuity of the map
    \begin{equation}
        p_{(m)}\colon \Big( r^+\swD(\eR^d),\| . \|_{H^m(\eR^d_+)} \Big)\to \Big( H^m(\eR^d),\| . \|_{H^m(\eR^d)} \Big).
    \end{equation}
    Taking advantage of the fact that the Sobolev space are complete (thus Banach) and the density of \( r^+H^m(\eR^d_+)   \) in \( H^m(\eR^d_+)\) we use the proposition~\ref{PropTTiRgAq} to build a continuous extension
    \begin{equation}
        p_{(m)}\colon \Big(  H^m(\eR^d_+)  ,\| . \|_{H^m(\eR^d_+)} \Big)\to \Big( H^m(\eR^d),\| . \|_{H^m(\eR^d)} \Big).
    \end{equation}
\end{proof}

%+++++++++++++++++++++++++++++++++++++++++++++++++++++++++++++++++++++++++++++++++++++++++++++++++++++++++++++++++++++++++++
\section{Boundaries}
%+++++++++++++++++++++++++++++++++++++++++++++++++++++++++++++++++++++++++++++++++++++++++++++++++++++++++++++++++++++++++++

We already made a small step in the world of boundaries and Sobolev space defining \( H_0^1(\Omega)\) as the part of \( H^1(\Omega)\) made from functions that vanishes on \( \partial\Omega\) (definition~\ref{DEFooFICWooBWCDyO}). This is far from being the end of the story. The main point in studying Sobolev spaces and boundaries is that the boundary has zero measure, so that the definitions of the usual functional spaces make no sense.

We need an integration theory on the boundary. Let \( a>0\) (in \( \eR\)) and define
\begin{subequations}
    \begin{align}
        Q_a&=\{ x\in \eR^{d}\st | x_j |<a,\,\forall j \}\\
        Q'_a&=\{ x'\in \eR^{d-1}\st | x'_j |<a,\,\forall j=1,\ldots, d-1 \}\\
        Q^+_a&=\{ x\in Q_a\st | x_n |>0\}.
    \end{align}
\end{subequations}

\begin{definition}          \label{DEFooCDJTooYyibCc}
    A \defe{smooth diffeomorphism}{smooth!diffeomorphism} \( \varphi\colon A\to B\) is an invertible \(  C^{\infty}\)-map with \(  C^{\infty}\) inverse.
\end{definition}

\begin{definition}[\cite{ooRRCQooJwnUgB}]
    A open set \( \Omega\subset \eR^d\) is \defe{smooth}{smooth!open in $\eR^d$} if for every \( x_0\in \partial\Omega\) there exists a neighbourhood \( U\) of \( x_0\), a \( a>0\) and a smooth diffeomorphism\footnote{Définition~\ref{DEFooCDJTooYyibCc}.} \( \varphi\colon Q_a\to U\) such that
    \begin{enumerate}
        \item
            \( \varphi(Q_a^+)=U\cap\Omega\)
        \item
            \( \varphi(Q'_a)=U\cap\partial\Omega\)
        \item
            \( \varphi(0)=x_0\).
    \end{enumerate}
\end{definition}
The restriction \( \varphi\colon Q'_a\to U\cap\partial\Omega\) will be denoted by \( \lambda\).

\begin{definition}
    A triple \( (\varphi,U,Q_a)\) is a \defe{local coordinates system}{coordinate!local} for \( \Omega\) around \( x_0\).
\end{definition}

\begin{lemma}
    Let \( (\varphi_1,U_1,Q_a)\) and \( (\varphi_2,U_2,Q_a)\) be local coordinates system around \( x_1\) and \( x_2\) in \( \partial\Omega\). The map
    \begin{equation}
        \varphi_2^{-1}\circ\varphi_1\colon \varphi_1^{-1}(U_1\cap U_2)\to \varphi_2^{-1}(U_1\cap U_2)
    \end{equation}
    is a smooth diffeomorphism such that if \( y\in Q_a\) is an element of \( \varphi_1^{-1}(U_1\cap U_2)\) with \( y_n=0\) we have
    \begin{equation}
        (\varphi_2^{-1}\circ\varphi_1)(y)_n=0.
    \end{equation}
\end{lemma}

\begin{proof}
    The fact that \( \varphi_2^{-1}\circ\varphi_1\) is a smooth diffeomorphism is true by composition, and by the careful choice of the domains.

    If \( y_n=0\), we have \( y\in Q'_a\) and \( \varphi_1(y)\in U_1\cap \partial\Omega\). In the same time, from the hypothesis, \( \varphi_1(y)\in U_1\cap U_2\), so that
    \begin{equation}
        \varphi_1(y)\in U_1\cap U_2\cap\partial\Omega.
    \end{equation}
    In particular, \( \varphi_2^{-1}\big( \varphi_1(y) \big)\in Q'_a\) and \( (\varphi_2^{-1}\circ\varphi_1)(y)_n=0\).
\end{proof}
One can restate the lemma saying that \( \varphi_2^{-1}\circ\varphi_1\) is a smooth diffeomorphism preserving the property \( y_n=0\).

\begin{normaltext}
    We suppose that \( \Omega\) is bounded, so that \( \partial\Omega\) is bounded and closed, thus compact (Borel-Lebesgue, theorem~\ref{ThoXTEooxFmdI}). If one has a local coordinate system \( (\varphi_x,U_x,Q_a)\) around each \( x\in \partial\Omega\), the union \( \bigcup_{x\in\partial \Omega}U_x\) contains \( \partial\Omega\). There exists a finite subcovering.

    We conclude that when \( \Omega\) is bounded, we have a finite family of local coordinates \( (\varphi_j,U_j,Q_a)\) (\( j=1,\ldots,\ldots, N\)). We also write \( U_0=\Omega\) so that \( \bigcup_{j=0}^NU_j\) covers \( \bar\Omega\).
\end{normaltext}

As allowed by the corollary~\ref{CORooMSWPooCxvuhm}, we consider a partition of the unity for \( \bar\Omega\) subordinated to the open sets \( \{ U_j \}_{j=0,\ldots, N}\). For each \( j=0,\ldots, N\) we have \( \psi_j\in\swD(U_j)\) and we have
\begin{equation}
    \sum_{j=0}^{N}\psi_j=1
\end{equation}
on a neighbourhood of \( \bar\Omega\).

If \( u\in Fun(\Omega)\) we have
\begin{equation}
    u=\sum_{j=0}^N\psi_ju
\end{equation}
and if \( v\in\Fun(\partial\Omega)\) we have
\begin{equation}
    b=\sum_{j=1}^N\psi_jv.
\end{equation}

The function \( \psi_ju\colon U_j\to \eC\) can be send to a function on \( Q_a^+\) by a chart because the support of \( \psi_ju\) is contained in the interior of \( U_j\) while \( \varphi_j^{-1}(U_j\cap\Omega)=Q_a^+\).

The diffeomorphism \( \varphi_j\colon Q_a\to U_j\) induces a diffeomorphism
\begin{equation}
 \lambda_j\colon Q_a'\to U_j\cap\partial \Omega
\end{equation}
This is because by definition of a chart, \( \varphi_j\colon U_j\cap\partial\Omega\to Q'_a\) is a bijection.

The following defines \( L^p_{loc}(\partial\Omega)\) with respect to a chart system. So this should be written \( L^p_{loc}(\partial\Omega,\{ \varphi_j \})\).
\begin{definition}      \label{DEFooFBVPooEeNwuU}
    Let \( p\geq 1\). The space \( L^p_{loc}(\partial \Omega)\) is the set of functions \( u\in\Fun(\partial\Omega)\) such that the map
    \begin{equation}
        u\circ\lambda_j\colon Q'_a\to \eC
    \end{equation}
    belongs to \( L^p_{loc}(Q'_a)\) for every \( j\).
\end{definition}

We prove that the set \( L^p_{loc}(\partial\Omega,\{ \varphi_j \})\) does not depend on \( \{ \varphi_j \}\).
\begin{lemma}
    The set defined in~\ref{DEFooFBVPooEeNwuU} does not depend on the choice of chart system.
\end{lemma}

\begin{proof}
    We consider two coordinates charts \( (U_j,\varphi_j,Q_a)_{j\in J}\) and \( (V_{\alpha},\psi_{\alpha},Q_aa_{\alpha\in A})\) and we have to show that when \( u\) is a \( L^p_{loc} \) function with respect to one chart system, it is \( L^p_{loc}\) for the other one. We consider the restrictions
    \begin{subequations}
        \begin{align}
            \lambda_j\colon Q'_a\to U_j\cap \partial\Omega\\
            \sigma_{\alpha}\colon Q'_a\to V_{\alpha}\cap\partial\Omega.
        \end{align}
    \end{subequations}
    So we suppose that \( u\circ\lambda_j\in L^p_{loc(Q_a')} \) for every \( j\) and we have to show that \( u\circ\sigma_{\alpha}\in L^p_{loc}(Q_a')\) for every \( \alpha\in A\) (from now on we fix some \( \alpha\in A\)). Let \( K\) be compact un \( Q_a'\) and we have to show that \( u\circ\sigma_{\alpha}\in L^p(K)\).

    Let \( L_j=U_j\cap\sigma_{\alpha}(K)\); this is a compact part of \( \eR^d\) and we have \( \sigma_{\alpha}(K)=\bigcup_{j\in J}L_j\). The latter union is however not disjoint. We have
    \begin{subequations}
        \begin{align}
            \int_K| u\circ\sigma_{\alpha} |^p&=\int_{\sigma_{\alpha}}| J_{\sigma_{\alpha}} |^p| u |^p\\
            &\leq \sum_{i\in I}\int_{L_j}\underbrace{| J_{\sigma_{\alpha}} |^p}_{\leq C}| u |^p\\
            &\leq C\sum_{i\in I}\int_{\lambda_i^{-1}(L_i)}\underbrace{| J_{\lambda_i^{-1}} |}_{\leq C'} |u\circ\lambda_i |^p
        \end{align}
    \end{subequations}
    The integrals are finites because the integration domain are compacts while the integrand is \( L^p_{loc}(Q'_a)\).
\end{proof}

%+++++++++++++++++++++++++++++++++++++++++++++++++++++++++++++++++++++++++++++++++++++++++++++++++++++++++++++++++++++++++++
\section{Older work}
%+++++++++++++++++++++++++++++++++++++++++++++++++++++++++++++++++++++++++++++++++++++++++++++++++++++++++++++++++++++++++++
\label{SECooNJLDooFcUzQv}

A lot of theory about Sobolev spaces can be found in \cite{Maslov,Taylor_PDO}. The \defe{Fourier transform}{Fourier transform} of a function $\varphi$ on $\eR^n$ is defined by formulas
\begin{subequations}
\begin{align}
  (F\varphi)(p)=\hat\varphi(p)&=\int_{\eR^N} e^{-2i\pi p\cdot x}\varphi(x)\,dx\\
	\varphi(x)&=\int_{\eR^{N}} e^{2i\pi}x\cdot p\hat\varphi(p)\,dp
\end{align}
\end{subequations}
Main properties of Fourier transform are
\begin{subequations} \label{subeq_prop_Four}
\begin{align}
(\partial_jF\varphi)(p)&=-2i\pi F(x_j\varphi)\\
	(F\partial_j\varphi)&=2i\pi p_j(F\varphi)(p)
\end{align}
\end{subequations}

The associated formula for the delta Dirac ``function'' is
\begin{equation}
  \int e^{2i\pi k\cdot x}\,dx=\delta(x).
\end{equation}

\begin{proposition}
If $\hat\swS$ denote the set of functions $\hat\varphi$ with $\varphi\in\swS$, then $\hat\swS=\swS$.
\end{proposition}

\begin{proof}
No proof
\end{proof}

We consider the \defe{Laplace operator}{Laplace operator}\nomenclature{$\Delta f$}{Laplace operator}, or Laplacian,
\begin{equation}
\Delta=\sum_{j=1}^{N}\partial_j^2
\end{equation}
When $k\in\eN$, we consider the following norm on $\swS(\eR^N)$:
\begin{equation} \label{eq_def_norm_Sob}
  \| \varphi \|_{H^k}^2=\int_{\eR^N}\overline{\varphi}(x)[1-(2\pi)^{-2}\Delta]^k\varphi(x)\,dx
\end{equation}
The \defe{Sobolev space}{Sobolev space} $H^k_2(\eR^N)$\nomenclature{$H^k_2(\eR^N)$}{Sobolev space} is the completed of $\swS$ for this norm.

\begin{proposition}
These Sobolev spaces are Hilbert spaces
\end{proposition}

\begin{proof}
No proof
\end{proof}

In order to define Sobolev spaces $H^k$ with $k<0$, we have to find a definition for $[-\Delta+1]^{-l}$. We define
\[
  \swS^{(l)}=\{ [1-(2\pi)^{-2}\Delta]^l\varphi\tq \varphi\in\swS \}
\]

\begin{lemma}
For each $\psi\in\swS^{(l)}$, there exists one and only one $\varphi\in\swS$ such that $(-\Delta+1)^l\varphi=\psi$.
\end{lemma}

This unique function $\varphi$ is naturally denoted by $(-\Delta+1)^{-l}$

\begin{proof}
We give the proof with $l=1$, the other are induction. When $\psi\in\swS^{(l)}$, existence is by definition true and only unicity is non trivial. Let $\varphi_1$ and $\varphi_2$ in $\swS$ such that
\[
  [1-(2\pi)^{-2}\Delta]\varphi_1=[1-(2\pi)^{-2}\Delta]\varphi_2,
\]
the function $f=\varphi_1-\varphi_2$ fulfils $(-\Delta+1)f=0$ and thus
\[
  \int_{\eR^N}\overline{ f }(x)[1-(2\pi)^{-2}\Delta]f(x)\,dx=0.
\]
Since $f\to 0$ at infinity rapidly, an integration by part of the term containing $\Delta$ leads, up to some constants, to
\[
  \int \overline{ f }f-\int \overline{ f }\Delta f=\int | f |^2+\int | \nabla f |^2=0.
\]
This proves that $f=0$.
\end{proof}

Now, the norm \eqref{eq_def_norm_Sob} can be used to define the Sobolev space $H^{-l}$. Elements of $H^k$ ($k<0$) which are not functions are distributions. When $m\ge0$, formulas  \eqref{subeq_prop_Four} give
\begin{equation}  \label{eq_umdpi_spi}
\left( F[1-(2\pi)^{-2}\Delta]^m\psi \right)(p)=\left( (1+p^2)^m\hat\psi \right)(p).
\end{equation}

\begin{proposition}
When $\psi\in\swS^{(m)}$, equality  \eqref{eq_umdpi_spi} holds even for $m<0$.
\end{proposition}


Let us point out that $m$ keep integer; the general real case will be treated later.

\begin{proof}
Let $m=-k<0$; from definition of the space $\swS^{(m)}$, the function $\varphi=[1-(2\pi)^{-2}\Delta]^{-k}\psi$ exists; we have
\begin{equation}
\hat\psi=\left( F[1-(2\pi)^{-2}\Delta]^k\varphi \right)(p)
	=(p^2+1)^k\hat\varphi(p),
\end{equation}
therefore $\hat\varphi(p)=(p^2+1)^{-k}\hat\psi(p)$. Replacing $\varphi$ by its definition,
\begin{equation}
\left( F[1-(2\pi)^{-2}\Delta]^{-k}\psi \right)(p)=(p^2+1)^{-k}\hat\psi(p).
\end{equation}


\end{proof}

\begin{proposition}
Let $\varphi\in\swS$. There exists a $\psi\in\swS$ such that
\[
  \varphi=[1-(2\pi)^{-2}\Delta]^l\psi
\]

\end{proposition}

\begin{proof}
Let us prove it with $l=1$; other cases are obtained by iteration. We consider the function $\psi$ defined by the condition
\[
  \hat\psi(p)=(p^2+1)^{-1}(F\varphi)(p).
\]
For this function we have $[1-(2\pi)^{-2}\Delta]\psi=\varphi$
\end{proof}

The space $\hat H^s(\eR^N)$ is the completed of $\swS$ for the norm
\begin{equation}
\| \varphi \|^2_{\hat H^s}=\int_{\eR^N}(p^2+1)^2| \varphi(p) |^2\,dx
\end{equation}
where $s$ is any positive real.


\begin{theorem}
For each $\varphi\in\swS$ and $k\in\eN$, we have
\[
  \| F\varphi \|_{\hat H^k}=\| \varphi \|_{H^k},
\]
in other words, the Fourier transform in $\swS$ is an isometry $\dpt{F}{ H^k }{ \hat H^k }$.

\end{theorem}


\begin{proof}
Using Parseval and equality \eqref{eq_umdpi_spi},
\begin{equation}
\| \varphi \|_{H^k}^2=\int_{\eR^N}\overline{ \varphi }(x)[1-(2\pi)^{-2}\Delta]^k\varphi(x)\,dx
		=\int \overline{ F\varphi(p) }(p^2+1)^k(F\varphi)(p)\,dp\\
		=\| \varphi \|_{\hat H^k}^2.
\end{equation}
\end{proof}
The space $\swS$ is dense in $H^k$ and $F$ is a bounded operator on $\swS$ (because it is isometric). Therefore it can be extended to an unique isometric homomorphism $\dpt{ F }{ H^k }{ \hat H^k }$ which is evidently called \defe{Fourier transform}{Fourier transform}.


In the same way, $F^{-1}$ is extended to an inverse of $F$ and finally,
\[
  \dpt{ F }{ H^k(\eR^N) }{ \hat H^k(\eR^N) }
\]
is  an isometric isomorphism. For each $s\in\eR$, we define
\begin{equation}
  \| \varphi \|_{H^s}:=\| F\varphi \|_{\hat H^s},
\end{equation}
and the completed of $\swS$ for this norm is the \emph{Sobolev space} $H^s(\eR^N)$.


\chapter{Analysis}
\input{136_garding}
% This is part of (almost) Everything I know in mathematics
% Copyright (c) 2013-2014,2018
%   Laurent Claessens
% See the file fdl-1.3.txt for copying conditions.

%\label{app_oscilla}

\section{Operator symbol}
%+++++++++++++++++++++++++

Operator symbol are natural framework to build oscillatory integral. Most of this theory comes from the book \cite{Dieu7} of Dieudonné.

\subsection{A case without problem}
%----------------------------------
Let $X$ be an open subset of $\eR^{N'}$ and $\dpt{A_{\nu}}{X}{M_{N''\times N'}}$ with $|\nu|\leq m$, some $\Cinf$ maps. A differential operator is a map $\dpt{P}{\cdE_{\eR}(X)^{N'}}{\cdE_{\eR}(X)^{N''}}$ of the following form:
\[
  P(u)=\sum_{|\nu|\leq m}A_{\nu}\cdot D^{_{\nu}}u,
\]
where the dot denotes a product matrix times vector. We suppose that the support of $u$ is
compact in $X$, so that it can be extended by $0$ in $\eR^n\setminus X$ in order to get a function in $\scrD_{\eR}(X)^{N'}$ that we will also denote by $u$. One can compute the Fourier transform of $u$: $\mF u_k\in\scrC(\eR^n)$, and $\mF u\in\scrC(\mR^n)^{N'}$. A main property of Fourier transform is that
\[
   D^{\nu}u=\int_{\eR^n} e^{2\pi ix\cdot\xi}(2\pi u\xi)^{\nu}(\mF u)(\xi)d\xi.
\]
If we pose
\[
  A(x,\xi)=\sum_{|\nu|\leq m}(2\pi i\xi)^{\nu}A_{\nu}(x),
\]
$P(u)$ can be written as
\[ 
    P(u)=\int_{\eR^n} e^{2\pi ix\cdot\xi}A(x,\xi)\cdot(\mF u)(\xi)d\xi.
\]
This integral makes only sense because we had carefully chosen allows the regularity conditions: as far as integral over $\xi$ is concerned, the functions $A(x,\xi)$ are polynomial with coefficients in $\cdE_{\eR}(X)^{N'N''}$ while the exponential contains a scalar product. The purpose of oscillatory integral is to generalise these two circumstances.

\subsection{A problem}
%---------------------

Let us consider the integral $\int_1^{\infty}\frac{1}{x}e^{ix}dx$. How to give a sense to that? Since $e^{ix}=-i\partial_xe^{ix}$, an integration by parts gives
\begin{equation}
 \int_1^{\infty}\frac{1}{x}e^{ix}=-i\int_1^{\infty}\frac{1}{x}\partial_x(e^{ix})
          =ie^{-i}-\int_1^{\infty}\frac{1}{x^2}e^{ix}
\end{equation}
where the last integral exists in the usual sense.

\subsection{Basic definitions}
%-----------------------------

\begin{definition}
Let $X$ be an open subset of $\eR^n$, $\dpt{a}{X\times\eR^N}{\eR}$ a $\Cinf$ function and $m$ be any real. We say that $a$ is an \defe{operator symbol}{operator!symbol} of order $m$ in $X\times\eR^N$ when

$\forall L\subset X$ compact, $\forall$ multi-indices $\alpha=(\alpha_1,\ldots,\alpha_n)$, $\beta=(\beta_1,\ldots,\beta_n)$,

$\exists$ constant $c_{\alpha,\beta,L}>0$ such that $\forall (x,\xi)\in L\times\eR^N$,
\begin{equation}\label{eq:cond_symbol}
  |D_x^{\alpha} D_{\xi}\hbeta a(x,\xi)|\leq c_{\alpha,\beta,L}(1+|\xi|)^{m-|\beta|},
\end{equation}
where $|\xi|^2:=\sum_{k=1}^{N}|\xi_k|^2$.
\end{definition}

Note that in $X\times \eR^N$, $X\subset\eR^n$ with $n$ and $N$ not necessarily equals. For short, we will often say ``symbol''\ instead of ``operator symbol''; the set of symbols of order $m$ in $X\times\eR^N$  is a vector space denoted by $\mS^m(X\times\eR^N)$\label{pg:defmS}.

For terminology issues, we say that a property of the point $(x,\xi)\in X\times\eR^N$ is \emph{true when $|\xi|$ is large} if $\forall$ compact $L\subset X$, there exists $r_L>0$ such that the property is true $\forall(x,\xi)$ with $x\in L$ and $|\xi|\geq r_L$.

\noindent This allows us to re-express the definition of a symbol. We say that $a\in\Cinf(X\times\eR^N)$ is a symbol when

$\forall$ multi-indices $\alpha,\beta$,

$(1+|\xi|)^{-m+|\beta|}D_x^{\alpha}D_{\xi}^{\beta}a(x,\xi)$ is bounded when $|\xi|$ is large.

 \noindent Indeed, the statement that $f(x,\xi)$ is such that $(1+|\xi|)^{-q}f(x,\xi)$ is bounded when $|\xi|$ is large is the existence of a constant $c$ (which depend on $L\subset X$) such that $f(x,\xi)\leq c(1+|\xi|)^q$.

\begin{proposition}
   A function in $\cdE(X\times\eR^N)$ which is zero for large $| \xi |$ is a symbol for all order.
\end{proposition}
\begin{proof}
 The assumption is: for all compact subset $L\subset X$, the restriction of $a$ to $L\times\eR^N$ have a compact support. Let us fix a compact $L$ and multi-indices $\alpha,\beta$; then on $L\times\eR^N$, $a$ and $D_x^{\alpha} D_{\xi}\hbeta a$ have a compact support. Then it is bounded by continuity. The same makes $|D_x^{\alpha} D_{\xi}\hbeta a(x,\xi)|(1+|\xi|)^{-m+|\beta|}$ bounded and then it can be majored by a constant $c_{\alpha\beta L}$.
\end{proof}

More generally, for same reason, if $a\in\mS^m(X\times\eR^N)$ and $b=a$ when $|\xi|$ is large, then $b\in\mS^m(X\times\eR^N)$.

The function
\[
   \sigma(\xi)=\us{1+\xi^2}
\]
is a symbol of order $2$. Here, there are no $x$ part and $\xi\in\eR$. The problem is to find a $c_{\beta}$ for any $\beta$. For $\beta=0$, the condition \eqref{eq:cond_symbol} becomes $\sigma(\xi)\leq c(1+\xi)^2$, which is true. For $|\beta|=1$, $c=2$ works because
\[
   \frac{2\xi}{(1+\xi^2)^2}\leq 2(1+\xi).
\]
It is clear that it will always woks because the degree of the denominator becomes bigger a bigger as $|\beta|$ grows.

This is a special case of a more general situation.

\begin{proposition}
   A complex function of $\cdE(X\times\eR^N)$ which is positively homogeneous of
    degree $m$ when $\abxi$ is large is a symbol of order $m$.
\end{proposition}\label{prop:23.16.4}
\begin{proof}
 The assumption is that $\forall$ compact $L\subset X$, $\exists r_L>0$ such that $\forall x\in L$, $\abxi\geq r_L$, and for all $\lambda\geq 1$,
 \[
   a(x,\lambda\xi)=\lambda^m a(x,\xi).
\]
Let us define $b(x,\xi)=a(x,\lambda\xi)$. We have $(D_{\xi} b)(x,\xi)=\lambda(D_{\xi} a)(x,\xi)$, then
\[
   (D_{\xi}\hbeta b)(x,\xi)=\lambda^{|\beta|}(D\hbeta_{\xi} a)(x,\lambda\xi).
\]

Since $b(x,\xi)=a(x,\lambda\xi)=\lambda^ma(x,\xi)$, we have $(D^{\beta}_{\xi}b)(x,\xi)=\lambda^m(D_{\xi}^{\beta}a)(x,\xi)$. By equalizing both expression of $(D_{\xi}\hbeta b)(x,\xi)$, we find
\[
  (D_x^{\alpha} D_{\xi}\hbeta a)(x,\lambda\xi)=\lambda^{m-|\beta|}(D_x^{\alpha} D_{\xi}\hbeta a)(x,\xi),
\]
when $x\in L$, $\abxi\geq r_L$ and $\lambda\geq 1$.

On the (compact) sphere $\abxi=r_L$, $|(D_x^{\alpha} D_{\xi}\hbeta a)(x,\xi)|$ is bounded. Let $c$ be a majoration, depending only on $\alpha,\beta$ and $L$. Any $\eta\in\eR^N$ with $|\eta|\geq r_L$ can be written as $\eta=\lambda\xi$ with $\lambda\geq 1$ and $\abxi=r_L$. Then
\begin{equation}\label{eq:majo}
   |(D_x^{\alpha} D_{\xi}\hbeta a)(x,\eta)|\leq c\lambda^{m-|\beta|}.
\end{equation}

It is cleat that $1+|\eta|\geq|\eta|\geq|\eta|/\abxi=\lambda$. Then, for $m-|\beta|>0$, the majoration \eqref{eq:majo} keep if we replace $\lambda^{m-|\beta|}$ by  $(1+|\eta|)^{m-|\beta|}$.

If $m-|\beta|<0$, the replacement of $\lambda^{m-|\beta|}$ by  $(1+|\eta|)^{m-|\beta|}$ need to change the constant $c$. More precisely, it raises the question to find a constant $k$ such that $|\eta|^{-r}\leq k(1+|\eta|)^{-r}$. It is easy to see that any
\[
   k>\left(\frac{r_L}{1+r_L}\right)^{-r}
\]
works. Finally,
\begin{equation}
   |(D_x^{\alpha} D_{\xi}\hbeta a)(x,\eta)|\leq c\lambda^{m-|\beta|}
                                      \leq ck(1+|\eta|)^{m-|\beta|}.
\end{equation}
\end{proof}

This proposition speaks about a function which is homogeneous when $|\xi|$ is large. There exist functions which are homogeneous although not symbol because of problems of continuity at $0$. For example, $a(x,\xi)=|\xi|$ when $N=1$. It is not $\Cinf$ at zero.
\begin{remark}
   A function in $\cdE(X)$ is a $\Cinf$ function on $X\times\eR^N$ which is positively homogeneous of degree zero. Then
\[
    \cdE(X)\subset\mS^0(X\times\eR^N).
\]
\end{remark}

We now give an useful result without proof.
\begin{proposition}
Let $a$ be a $\Cinf$ function which is \emph{only} defined for large $\abxi$ and such that for all $\alpha,\beta$,
\[
  (1+\abxi)^{-m+|\beta|}D_x^{\alpha} D_{\xi}\hbeta a(x,\xi)
\]
is bounded for large $\abxi$.

Then there exists a symbol $b$ of order $m$ on $X\times \eR^N$ such that $a=b$ when $\abxi$ is large.
\end{proposition}

For notational convenience, we define
\begin{equation} \label{eq:def_ablambda}
   a_{\lambda}(x,\xi)=a(x,\lambda\xi).
\end{equation}

Let us clearly compute a derivative of $a_{\lambda}$. The notation $(D_{\xi}a_{\lambda})(x_0,\xi_0)$ has to be read as ``The derivative of $a_{\lambda}$ with respect to his second argument at point $(x_0,\xi_0)$''.  If $M_{\lambda}(x,\xi)=(x,\lambda\xi)$,
\begin{equation}
\begin{split}
  (D_{\xi}a_{\lambda})(x_0,\xi_0)&=D_{\xi}(a\circ M_{\lambda})(x_0,\xi_0)\\
                                 &=(D_{\xi}a)(M_{\lambda}(x_0,\xi_0))\cdot \frac{dM_{\lambda}}{d\xi}(x_0,\xi_0)\\
                                 &=\lambda(D_{\xi}a)(x_0,\lambda\xi_0).
\end{split}
\end{equation}

\begin{theorem}\label{tho:dieu23.16.6}
   A function $a\in\cdE(X\times\eR^N)$ is a symbol of order $m$ if and only if the set of the restrictions of the functions $\lambda^{-m}a_{\lambda}$ (with $\lambda\geq 1$) to $X\times(\eR^N\setminus\{o\})$ is bounded in the sense of definition~\ref{def:bounded}   in the Fréchet space $\cdE(X\times(\eR^N\setminus\{o\}))$.
\end{theorem}

\begin{proof}
Let $a$ be a symbol of order $m$. By definition of $a_{\lambda}$,
\[
  D_x^{\alpha} D_{\xi}\hbeta a_{\lambda}(x,\xi)=\lambda^{|\beta|}(D_x^{\alpha} D_{\xi}\hbeta a)(x,\lambda\xi),
\]
but any compact in $\Xrnz$ is contained in a compact which can be written as $L\times G$ with $L$ compact in $X$ and $G=\{\xi\in\eR^N\,|\, r\leq\abxi\leq R\}$ with $0<r<R$.

Since $a$ is a symbol, we have
\[
   |\DxaDxb a(x,\lambda\xi)|\leq  c_{\alpha,\beta,L}(1+\lambda\abxi)^{m-|\beta|};
\]
if we multiply it by $(1/\lambda)^{m-|\beta|}$, we find
\begin{equation}
\begin{split}
  \lambda^{|\beta|-m}|\DxaDxb a(x,\lambda\xi)|
     &\leq c_{\alpha,\beta,L}\left(\us{\lambda}+\abxi\right)^{m-|\beta|}\\
     &\leq \begin{cases}
                   c_{\alpha,\beta,L}(1+R)^{m-|\beta|} &\text{if }|\beta|\leq m\\
		   c_{\alpha,\beta,L}r^{m-|\beta|}     &\text{if }|\beta|> m.
           \end{cases}
\end{split}
\end{equation}

Now, we want to know if $\{ \lambda^{-m}a_{\lambda}|\lambda\geq 1 \}$ is bounded in $\cdE(\Xrnz)$. The computation is
\begin{equation}
\begin{split}
    p_{sj}(\lambda^{-m}a_{\lambda})&=\sup_{ \substack{(x,\xi)\in K_j\\ \alpha,\beta\text{ st } |\alpha|+|\beta|\leq s }}|\DxaDxb(\lambda^{-m}a_{\lambda})(x,\xi)|\\
                       &=\lambda^{-m}\sup_{\ldots}\lambda^{|\beta|}(\DxaDxb a)(x,\lambda\xi)|\\
		       &=\lambda^{-m+|\beta|}\sup_{\ldots}|(\DxaDxb a)(x,\lambda\xi)|.
\end{split}
\end{equation}
Then
\begin{equation}
p_{s,m}(\lambda^{-m} a_{\lambda})\leq \begin{cases}
                   c_{\alpha,\beta,L}(1+R)^{m-|\beta|} &\text{if }|\beta|\leq m\\
		   c_{\alpha,\beta,L}r^{m-|\beta|}     &\text{if }|\beta|> m.
           \end{cases}
\end{equation}

Now, we prove the reverse sense. Let us suppose that $p_{sj}$ is bounded, \emph{i.e.}
\[
| \lambda^{-m+|\beta|}(\DxaDxb a)(x,\lambda\xi)|\leq A
\]
with $x\in L$, $r\leq\abxi\leq R$ and $\lambda\geq 1$. Then, for $x\in L$ and $\abxi\geq r$,
\begin{equation}\label{eq:2409r1}
   |(\DxaDxb a)(x,\xi)|\leq A\left(\frac{\abxi}{r}\right)^{m-|\beta|}.
\end{equation}

Since for $\abxi\geq r$,
\[
   \frac{r}{1+r}\leq\frac{\abxi}{1+\abxi}\leq 1,
\]
one can ``forget''\ the $(1/r)^{m-|\beta|}$ in \eqref{eq:2409r1}: it can be absorbed in a constant which depend on $\beta$. By the same as in the proof of proposition~\ref{prop:23.16.4}, we can also replace $\abxi^{m-|\beta|}$ by $(1+\abxi)^{m-|\beta|}$; this proves that $a$ is a symbol of order $m$.

\end{proof}

\subsection{Topology on \texorpdfstring{$\mS^m(X\times\eR^N)$}{SmXRN}}
%---------------------------------------------

We will endow the vector space $\mS^m(X\times\eR^N)$ with a locally convex topological structure.

Let $(p_k)$ be the family of seminorms defining the topology of $\cdE(X\times\eR^N)$; we consider their restriction to $\mS^m(X\times\eR^N)$. We also consider $(p'_k)$, the one which defines the topology of $\cdE(\Xrnz)$. Now we pose for $a\in\mS^m(\Xrnz)$,
\[
   q_k(a)=\sup_{\lambda\geq 1}p'_k(\lambda^{-m}a_{\lambda}).
\]
One can see that these $q_k$ are seminorms. Finally, we put on $\mS^m(X\times\eR^N)$ the topology defined by the $p_k$ \emph{and} the $q_k$.

First remark: this topology on $\mS^m$ is finer\angl than the one which is induced from $\cdE(X\times\eR^N)$. In particular, $\mS(X\times\eR^N)$ is metrisabe\angl.

\begin{lemma}
Each seminorm defining the topology of a Fréchet space is continuous.
\label{lem:q_cont}
\end{lemma}
\begin{proof}
Let $q$ be one of them. By proposition~\ref{prop:semi_norm_cont}, it is sufficient to find a neighbourhood of $0$ on which $q$ is bounded. Let us show that
\[
   B(0;q,r)=\{x\in\mS^m|q(x)<r\}
\]
works. Indeed, it is an open set by definition of the topology, we just have to prove that this contains $0$, \emph{i.e.} we have to find a $r$ such that $q(0)<r$.
\[
  q(0)=sup_{\lambda\geq 1}p'(0)=p'(0)
\]
where $p$ defines the topology of $\cdE(\Xrnz)$. In proposition~\ref{prop:topo_E}, we see that $p'(f)=sup_{\ldots}|(D^{\ldots}f)(x)|$. It is clear that this is zero when $f\equiv 0$.


\end{proof}

\begin{proposition}
For this topology, $\mS^m(X\times\eR^N)$ is a Fréchet space.
\end{proposition}
\begin{proof}
We already know that $\mS^m(X\times\eR^N)$ is locally convex space because its topology is defined by seminorms. It is also Hausdorff because its topology is finer than the one of $\cdE$ which is itself separable. Now, we consider $\cdE(X\times \eR^N)$ as an additive group, so that we can use the theory of metrisable groups developed in point~\ref{sec:metrisable_groups}. Thus we can speak of \defe{Cauchy sequences}{Cauchy sequences}: a sequence $(f_n)$
in $\cdE(X\times \eR^N)$ is Cauchy when

$\forall\,V\in\mV(0)$, $\exists\,n_0$ such that $\forall\,m,n\geq n_0$, $f_n-f_m\in V$,

\noindent where $\dpt{0}{X\times \eR^N}{\eR}$ is the null function.

From proposition~\ref{prop:E_Frechet}, we know that $\cdE(\mU)$ is a Fréchet space; in particular, it is complete for the distances which defines the Cauchy sequences (the ones whose are invariant under translations).

Let us consider a Cauchy sequence $(a^n)$ in $\mS^m(X\times\eR^N)$. It converges (in the sense of $\cdE(X\times \eR^N)$) to $b\in\cdE(X\times \eR^N)$. A Cauchy sequence is always contained in a compact set, and by lemma~\ref{lem:q_cont}, $q_k$ is continuous for any $k$. Thus for each $k$ the sequence (in $\eR$) given by $(q_k(a^n))$ is bounded (continuous function on a compact).

So, for each $j$, the set $\{p'_k(\lambda^{-m}a^n_{\lambda})\}$ is bounded when $\lambda,n\geq 1$. If we let $n$ goes to infinity, we find that $\{p'_k(\lambda^{-m}b_{\lambda}):\lambda>1\}$ is bounded, and then $b\in\mS^m(X\times\eR^N)$.

We had just proved that $b$ where the limit of $(a^n)$ in the sense of $\cdE(X\times\eR^n)$; we yet have to see that $b$ is also the limit in the sense of $\mS^m(X\times\eR^n)$. As $(a^n)$ is a Cauchy sequence, proposition~\ref{prop:Cauchy_metrisable} assures us that

$\forall\,V\in\mV(0)$, $\exists\,n_0$ such that $\forall\,m,n\geq 0$, $f_n-f_m\in V$.

\noindent In particular, a neighbourhood $V$ of $0$ is for example $\{x\in\mS^m|q_k(x)<r\}$. Then $\forall\epsilon>0$, $\exists\,n_0$ such that $i$, $j>0$ implies $q_k(a^i-a^j)<\epsilon$. So $b$ is the limit of $(a^n)$ when $n\to\infty$ in the sense of $\mS^m(X\times\eR^N)$.

\end{proof}

\begin{lemma}   \label{LemHIUsKABh}
Let $B$ be a bounded set in $\cdE(\mU)$ and $A>0$ such that $f(z)>A$ for all $f\in B$ and for all $z\in\mU$. Then the set $\{f^s\}_{f\in B}$ is bounded in $\cdE(\mU)$.
\end{lemma}

\begin{proof}
We have to show that $(D^{\nu}(f^s))(z)$ is bounded when $f$ runs over $B$ and $z$ keeps in a compact subset of $\mU$. Let's use an induction on $|\nu|$ in order to prove that
\[
   D^{\nu}(f^s)=f^{s-|\nu|}P_{\nu}\big(  (d^{\rho}f_{\rho\leq\nu})   \big)
\]
where $P_{\nu}$ is a polynomial with coefficients independent of $f$. Indeed, let $\nu=\{\nu_0,\sigma\}$ where $\nu_0$ is a multi-index and $\sigma$ a single index.
\begin{equation}
\begin{split}
  D_{\nu}(f^s)=D_{\sigma}D_{\nu_0}(f^s)&=D^{\sigma}\left( f^{s-|\nu_0|}P_{\nu_0}\big( (D^{\rho}f)_{\rho\leq\nu_0} \big) \right)\\
              &=(s-|\nu_0|)f^{s-|\nu_0|-1}(D^{\sigma}f)P_{\nu_0}\left( (D^{\rho}f)_{\rho\leq\nu_0} \right)\\
              &\quad    +f^{s-|\nu_0|}D^{\sigma}\left( P_{\nu_0}(D^{\rho}f)_{\rho\leq\nu_0} \right).
\end{split}
\end{equation}
The quantity $P_{\nu}\left( (D^{\rho}f)_{\rho\leq\nu} \right)$ remains bounded because $|f(z)|^{-1}\leq A^{-1}$ and $z$ belongs to a compact set. The function $f$ take bounded values (continuous on a compact set) and while $s-|\nu|>0$, the function $f^{s-|\nu|}$ is bounded too. If $s-|\nu|<0$, there can be a problem, but th condition $f(z)\geq A$ avoid this case.
\end{proof}

\begin{theorem} \label{tho:lenumf}
Multiplication and derivations on symbols behave rather well:
\begin{enumerate}
 \item \label{enufi}  For all multi-index $\gamma$, $\delta$, the map $a\to D^{\gamma}_xD^{\delta}_{\xi}a$ is a continuous linear map from $\mS^m$ to $\mS^{m-|\delta|}$.
 \item \label{enufii} The map  $(a,b)\to ab$ is continuous and bilinear from $\mS^{m}\times\mS^{m'}$ into $\mS^{m+m'}$.
 \item \label{enufiii} Take $a\in\mS^m$. The function $a^{-1}$ is defined and is a symbol of order $-m$ if and only if for all compact $H\subset X$ there exists a constant $c_H>0$ such that, in $H\times\eR^n$, the condition
\[
|a(x,\xi)|\geq c_H(1+|\xi|)^m
\]
holds. In this case, for all real number $s$, we have $|a|^s\in\mS^{sm}$.
\end{enumerate}
\end{theorem}

We only prove the point~\ref{enufiii}.

\begin{proof}
\subdem{Necessary condition}
Let $a\in\mS^m$ such that $a^{-1}$ is well defined as symbol of order $-m$. By the definition of a symbol,
\begin{equation}
\left| \frac{1}{a(x,\xi)}  \right|\leq c_H(1+|\xi|)^{-m}
\end{equation}
for a certain $c_H$ determined by the compact $H$ in which $x$ moves. Then $|a(x,\xi)|\geq c_H^{-1}(1+|\xi|)^m$ and the claim follows

\subdem{Sufficient condition} Let $a\in\mS^m$ and for each compact $H\subset X$, there exists a $c_H>0$ such that $\forall(x,\xi)\in H\times \eR^N$,
\[
   |a(x,\xi)|\geq c_H(1+|\xi|)^m.
\]
By point~\ref{enufii}, $|a|^2\in\mS^{2m}$ and $a^{-1}=\bar a (|a|^3)^{-1}$. Let us suppose that the theorem is proved when $a(x,\xi)>0$ on $X\times\eR^N$. So suppose that $a<0$ somewhere. In this case, $a^{-1}=\bar a(|a|^2)^{-1}$ where $|a|^2$ is positive in such a manner that our assumption makes it a symbol of order $-2m$; now by~\ref{enufii}, $a^{-1}$ is a symbol of order $-m$.

So we can restrict ourself to the case where $a(x,\xi)>0$ in $X\times\eR^N$. It is clear that $\lambda^{-ms}(a_{\lambda}^s)=(\lambda^{-m}a_{\lambda})^s$. From theorem~\ref{tho:dieu23.16.6} we just have to prove that
$\lambda^{-m}a_{\lambda})^s$ is a bounded set when $s=-1$. Lemma~\ref{LemHIUsKABh} shows it for all $s$.

\end{proof}

\begin{proposition}
Let $Y$ be an open subset of $\eR^p$,
\[
  \dpt{\psi}{Y\times\eR^P}{X}
\]
and
\[
  \dpt{\theta}{Y\times\eR^P}{\eR^N},
\]
two $C^{\infty}$ functions such that for large $|\xi|$, the function $\psi$ is positively homogeneous of degree zero while $\theta$ is of degree $1$ with respect to $\xi$.

Then for all symbol $a\in\mS^m(X\times\eR^N)$, the function
		\begin{equation}
		\begin{aligned}
			b \colon Y\times\eR^P &\to \eR\
			(x',\xi')&\mapsto a(\psi(x',\xi'),\theta(x',\xi'))
		\end{aligned}
	\end{equation}
is a symbol of $\mS^m(Y\times\eR^P)$ and the linear map $\mS^m(X\times\eR^N)\to\mS^m(Y\times\eR^P)$ given by $a\to a\circ(\psi,\theta)$ is continuous.


\end{proposition}

\begin{proof}
We pose $F=(\psi,\theta)$, so that $b=a\circ F$. Homogeneity assumptions make that when $\lambda\geq 1$ and when $|\xi|$ is large,
\[
   \lambda^{-m}b_{\lambda}=(\lambda^{-m}a_{\lambda})\circ F.
\]
Then the problem reduces to prove that if $B$ is bounded in $\cdE(X\times\eR^N_0)$, then the set $f\circ F$ is bounded when $f$ runs over $B$.
\end{proof}


\begin{lemma}
Let $m<m'\in\eR$. Then the canonical injection
\[
\dpt{\id}{\mS^m}{\mS^{m'}}
\]
is continuous.

\end{lemma}


\begin{proof}
Let $\mO$ be an open set in $\mS^{m'}$; we have to show that $\id^{-1}(\mO)=\mO\cap\mS^{m}$ is open in $\mS^m$. Let $p_k$ be the family of seminorm defining the topology on $\cdE(X\times\eR^N)$ and $p'_k$ the one of $\cdE(X\times\eR^N_0)$ and then
\[
q_k(a)=\sup_{\lambda\geq 1}p'_k(\lambda^{-m}a_{\lambda}).
\]
The topology of $\mS$ is defined from $p_k$ and $q_k$. We define the ball
\[
B(a,d,r)=\{x\in E\tq d(a,x)<r\}.
\]
The property for $\mO\subset\mS^{m'}$ to be open is that one of these ball is included in $\mO$. If a ball build with one of the $p_k$'s is included in $\mO$, there are no problems because the seminorms $p_k$ are also in the definition of the topology on $\mS^m$. Let
\[
  B(a,d_k^{m'},r)=\{x\in\mS^{m'}\tq q_k^{m'}(a-x)<r\}.
\]
The candidate ball of $\mS^m$ to be included in $\mO$ is
\[
\mO'=\{x\in\mS^m\tq q_k^m(a-x)<\overline{r}\}
\]
for a certain $\overline{r}<r$. Let us prove that $q_k^m(y)\leq q_k^{m'}(y)$. Here, $k=(n,s)$.
\begin{equation}
\begin{split}
q_k^m(y)&=\sup_{\lambda\geq 1}\sup_{\substack{x\in K_n\\|\nu|<s\|}}| D^{\nu}(\lambda^{-m}y_{\lambda})(x) |\\
        &=\sup_{\lambda\geq 1}\lambda^{m'-m}\sup_{\substack{x\in K_n\\|\nu|<s\|}}| D^{\nu}(\lambda^{-m'}y_{\lambda})(x) |\\
        &\leq\sup_{\lambda\geq 1}p'_k(\lambda^{-m'}y_{\lambda})\\
        &=q_k^{m'}(y).
\end{split}
\end{equation}

\end{proof}

\begin{lemma}
The adherence of $\cdD(X\times\eR^N)$ in $\mS^m(X\times \eR^N)$ contains $\mS^{m'}(X\times\eR^N)$ for all $m'<m$.
\label{lem:DadhS}
\end{lemma}

One can characterize the topology by another choice of seminorms as the following proposition shows.

\begin{proposition}
The topology of Fréchet space on $\mS^(X\times\eR^N)$ is defined by the seminorms
\[
   p_{K,\alpha,\beta}(a)=\sup_{x\in K,\xi\in\eR^N}(1+|\xi|)^{-m+|\beta|}| D_x^{\alpha}D_{\xi}^{\alpha}a(x,\xi) |.
\]
 \label{prop:topo_alter}
\end{proposition}

\subsection{Asymptotic expansions}
%--------------------------------------

Note that $\cdD(X\times \eR^N)\subset\mS^{-\infty}(X\times\eR^N)$ because a compact supported function is always a symbol of all order. Let us give without proof the two following results \cite{Dieu7}.


\begin{lemma}
Let $a\in\mS^{m'}(X\times\eR^N)$ a symbol whose support is contained in $H\times\eR^N$ where $H$ is a compact subset of $X$. Let $h\in\cdD(X\times\eR^N)$, a function equals to $1$ for $x\in H$ and $|\xi|\leq A$ for a certain constant $A>0$. Then for all $m>m'$,
\[
  \lim_{q\to\infty}(h_{1/q}a)=a
\]
in the sense of the convergence in $\mS^m$.
\label{lem:limha}
\end{lemma}

\begin{proposition}
The adherence of $\cdD(X\times\eR^N)$ in $\mS^m(X\times\eR^N)$ contains $S^{m'}(X\times\eR^N)$ for all $m'<m$.
\end{proposition}

The main theorem is

\begin{theorem}
Consider a strictly increasing sequence in $\eR$ $(m_j)$ with $\lim_{j\to\infty}m_j=-\infty$. Let, for each $j$, $a_j\in\mS^{m_j}(X\times\eR^N)$. Then there exists $a\in \mS^{m_0}$ such that for all $k$,
\begin{equation}  \label{eq:dev_ass}
a-\sum_{j<k}a_j\in\mS^{m_k}(X\times\eR^N).
\end{equation}
Furthermore if $a'$ shares this property with $a$, then $a-a'\in\mS^{-\infty}(X\times\eR^N)$.
\label{tho:dev_ass}
\end{theorem}


\begin{proof}
The last assertion is easy. Let us subtract equations \eqref{eq:dev_ass} for two such symbols: for all $k$,
\[
a-a'\in\mS^{m_k}(X\times\eR^N).
\]
Then $a-a'\in\mS^{X\times\eR^N}$.

Now consider a locally finite countable covering $(\mU_{\alpha})$ of open relatively compact sets and $(f_{\alpha})$, a partition of unity for this covering. Let us suppose that, for each $\alpha$, we know a symbol $b_{\alpha}$ of order $m_0$ which is zero outside $\mU_{\alpha}\times\eR^N$ and such that for all $k$,
\[
  b_{\alpha}=\sum_{i<k}f_{\alpha}a_j\in\mS^{m_k}(X\times\eR^N).
\]
In other words, let us suppose that we have the answer for the symbols $f_{\alpha}a_j$ instead of the symbols $a_j$. Then the symbol $a=\sum_{\alpha}b_{\alpha}$ answer the question for the symbols $a_j$. Indeed
\begin{equation}
\begin{split}
\sum_{\alpha}b_{\alpha}-\sum_ja_j=\sum_{\alpha}(b_{\alpha}-\sum_jf_{\alpha}a_j)
\end{split}
\end{equation}
where the sums are pointwise \emph{finite} sums because the covering $(\mU_{\alpha})$ is locally finite.

So we are left to consider $a_j$ vanishing outside a set $H_j\times\eR^N$ where $H_j$ is a compact subset of $X$. For $j\geq 1$, let us define
\begin{equation}
a_{j,q}=(1-h_{j,1/q})a_j\in\mS^{m_j}
\end{equation}
where $h_j\in\cdD(X\times\eR^N)$ fulfils $h_j(x,\xi)=1$ when $x\in H_j$ and $0\leq|\xi|\leq 1$. By $h_{i,1/q}$, we mean $(h_j)_{1/q}$ in the sense of equation \eqref{eq:def_ablambda}. Since it has a compact support, it is a symbol for all orders. From lemma~\ref{lem:limha},
\[
  \lim_{q\to\infty}(a_{j,q})=\lim_{q\to\infty}a_j-\lim_{q\to\infty}h_{j,1/q}a_j=0.
\]
Proposition~\ref{prop_suiteFk} makes that for all $r\geq 0$, the sum $\sum_{j\geq r}a_{j,q_j}$ converges in $\mS^{m_r}$. We are now going to prove that the symbol $a=\sum_{j\geq 0}a_{j,q_j}$ is the one needed by the theorem. First remark that $a_{j,q}$ has support contained in $H_j\times\eR^N$. For all $q>0$,
\[
a_j-a_{j,q}\in\cdD(X\times\eR^N)\subset\mS^{-\infty}(X\times\eR^N).
\]
Now,
\[
  \Big( a-\sum_{j<k}a_j \Big)-\Big( a-\sum_{j<k}a_{j,q_j} \Big)=\sum_{j<k}\Big( a_{j,q_j}-a_j \Big)\in\mS^{-\infty}(X\times\eR^N),
\]
then
\[
a-\sum_{j<j}a_{j,q_j}=\sum_{l\geq k}a_{l,q_l}\in\mS^{m_k}
\]
from lemma~\ref{prop_suiteFk}.

\end{proof}

When equation \eqref{eq:dev_ass} holds , we say that the sum $\sum_ja_j$ is an \defe{asymptotic expansion}{asymptotic expansion} of the symbol $a$ and we write $a\sim\sum_ja_j$. In this case, for all $k$ we have
\begin{equation} \label{eq:assa}
  a-\sum_{j<k}a_j\in\mS^{m_k},
\end{equation}
thus for any multi-index $\nu$, the following holds
\[
  D^{\nu}a\sim\sum_jD^{\nu}a_j
\]
as can be seen by derivation of equation \eqref{eq:assa} and using point~\ref{enufi} of theorem~\ref{tho:lenumf}.

\subsection{Tempered functions}
%------------------------------

A function $f\in\cdE(X\times\eR^N)$ is \defe{tempered with respect to $|\xi|$}{tempered!function} if for all compact $L\subset X$, and for all pair of multi-index $(\alpha,\beta)$, there exists constants $c(\alpha,\beta,L)$ and $p(\alpha,\beta,L)$ such that
 \[
|D_x^{\alpha}D_{\xi}f(x,\xi)|\leq c(\alpha,\beta,L)(1+|\xi|)^{p(\alpha,\beta,L)}
\]
for all $(x,\xi)\in L\times\eR^N$.


\begin{lemma}
Let $c\in\cdE(X\times\eR^N)$ be a tempered function with respect to $\xi$ such that for all compact $L\subset X$ and $q>0\in\eN$, there exists a constant $C_{qL}$ such that
\[
|c(x,\xi)|\leq C_{qL}(1+|\xi|)^{-q}
\]
 for all $(x,\xi)\in L\times\eR^N$. Then $c\in\mS^{-\infty}(X\times\eR^N)$.
\label{lem:csymbol}
\end{lemma}

The following proposition gives a link between tempered functions and symbols.

\begin{proposition}
Let $X$ be an open set in $\eR^N$ and $(m_j)\in\eR$ a strictly decreasing sequence such that $\lim_{j\to\infty}m_j=-\infty$. For each $m_j$, we consider a symbol $a_j\in\mS^{m_j}$. Let $a\in\cdE(X\times\eR^N)$, a tempered function with respect to $\xi$ and suppose that for all compact $L\subset X$, there exists a decreasing sequence $(q_k)\in\eR$ with $\lim_{k\to\infty}q_k=-\infty$ and for each $k$, a constant $C_{kL}$ such that
\begin{equation}
   |a(x,\xi)-\sum_{j<k}a_j(x,\xi)|\leq C_{kL}(1+|\xi|)^{q_k}
\end{equation}
for all $(x,\xi)\in L\times\eR^N$.

Then $a$ is a symbol of order $m_0$ and admits the asymptotic expansion
\[
a\sim\sum_ja_j.
\]

\end{proposition}

\begin{proof}
Theorem~\ref{tho:dev_ass} holds for symbols $a_j$, then there exists a symbol $b\in\mS^{m_0}$ such that $b\sim\sum_{j}a_j$. So we have to prove that $a-b\in\mS^{\infty}$ because, in this case,
\[
a-\sum_{j<k}a_j=(a-b)-(\sum_{j<k}a_j-b)\in\mS^{m_k}.
\]
On the one hand, the function $a-b$ is tempered (because a symbol is always tempered). On the other hand, for all choice of $k$, we can write
\begin{equation}
\begin{aligned}
   |(a-b)(x,\xi)|&=|a-\sum_{j<k}a_j-b+\sum_{j<k}a_j|\\
                 &\leq|a-\sum_{j<k}a_j|+|b-\sum_{j<k}a_j|\\
                 &\leq C_{kL}(1+|\xi|)^{q_k}+D_L(1+|\xi|)^{m_k} \\
                 &\leq C'_{kL}(1+|\xi|)^{q'_k}
\end{aligned}
\end{equation}
where $C'_{kL}=\max(C_{kL},D_L)$ and $q'_k=\max(q_k,m_k)$. Let us now fix $q>0$; since $q_k,m_k\to-\infty$, there exists $k$ such that $q'_k<-q$. For this one, we have
\[
|(a-b)(x,\xi)|\leq C''_{qL}(1+|\xi|)^{-q}.
\]
Lemma~\ref{lem:csymbol} concludes the proof.


\end{proof}


\section{First construction of oscillatory integrals}
%++++++++++++++++++++++++++++++++++++++++++++++++++++

\subsection{Construction}
%------------------------

Let us consider $X$ be an open set in $\eR^N$, a continuous map $\dpt{\varphi}{X\times\eR^N}{\eR}$ and a symbol $a\in\mS^{m}(X\times\eR^N)$. For any compact $L\subset X$, the integral
\[
\int_{L\times\eR^N}e^{i\varphi(x,\xi)}a(x,\xi)dxd\xi
\]
converges when $m<-N$. Indeed
\begin{equation}
   \int |e^{i\varphi}a|\leq\int|e^{i\varphi}||a|
                       \leq\int C(1+|\xi|)^m.
\end{equation}
The integral over $L$ is a constant on a compact while the one on $\eR^N$ is performed with spherical coordinates, sop we are left with $\int_{\eR^+}(1+r)^mr^{N-1}$. Under these assumptions, the form
\begin{equation}  \label{eq:formaprol}
 a\to\int_{L\times\eR^N}e^{i\varphi}a
\end{equation}
is linear. In order to see that is is continuous too, we will prove that there exists a neighbourhood of zero in $S^m(X\times\eR^N)$ in which any $a$ fulfils the majoration
\[
  |a(x,\xi)|\leq C(1+|\xi)^m
\]
for all $x(x,\xi)\in L\times\eR^N$ and a certain constant $C$. Let us use the seminorms given in proposition~\ref{prop:topo_alter}  and consider the neighbourhood
\[
\mO=\{ a\in\mS^m\tq\sup_{x\in K,\xi\in\eR^N}(1+| \xi |)^{-m}| a(x,\xi) |<\varepsilon \}
\]
for a certain compact $K\subset X$ and $\varepsilon>0$. A symbol $a$ in $\mO$ fulfils in particular
$| a(x,\xi) |<\varepsilon(1+| \xi |)^m$. Then the linear form \eqref{eq:formaprol} is bounded and thus continuous.

\begin{definition} \label{def:phase}

A function $\dpt{\varphi}{X\times\eR^N_0}{\eR}$ is a \defe{phase function}{phase!function} if

\begin{enumerate}
\item $\varphi\in C^{\infty}(X\times\eR^N_0)$,
\item the $n+N$ first derivatives $D'_j\varphi$ and $D''_k\varphi$ doesn't vanish at same point of $X\times\eR^N_0$, in other words, $\varphi$ is not singular
\item for all $(x,\xi)\in X\times\eR^N_0$ and all $\lambda>0$, the formula $\varphi(x,\lambda\xi)=\lambda\varphi(x,\xi)$ holds.
\end{enumerate}

\end{definition}
Here, $D'_j$ denotes the derivative with respect to $x_j$ and $D''_k$ the one with respect to $\xi_k$. The third point implies that the prolongation $\varphi(X\times\{  0\})=0$ in continuous in $X\times\eR^N$ but not $C^{\infty}$ as the example $\varphi(x,\xi)=| \xi |$ shows with $N=1$.

An important example of phase function is given when $N=n$ by $\varphi (x,\xi)=-2\pi \scal{x}{\xi}$; it gives rise to the Fourier transforms.

An important result will help us to define oscillatory integrals

\begin{lemma}  \label{lem:defL}
Let $\varphi$ be a phase. There exists symbols $a_k\in \mS^0$, $b_j\in\mS^{-1}$ and $c\in\mS^{-\infty}$ ($1\leq j\leq N$ and $1\leq j\leq n$) such that the differential operator
\[
  L=\sum_{k=1}^Na_kD''_k+\sum_{j=1}^nb_jD'_j+c\mtu
\]
on $X\times\eR^N$ fulfil
\[
  Le^{i\varphi}=e^{i\varphi}
\]
in $X\times\eR^N_0$. The transposed $L^t$ of $L$ is given by
\[
  L^t=-\sum_{a_k}D''_k-\sum_{j=1}^nb_jD'_j+c'\mtu
\]
where $c'=c-\sum D''_ka_k-\sum D'_jb_j\in\mS^{-1}$.

For all $u\in\cdE(X\times\eR^N)$ and $r\in\eN^+_0$,
\[
  \big( (L^t)^ru \big)(x,\xi)=\sum_{| \alpha |+| \beta |\leq r}g_{\alpha\beta}(x,\xi)D_x^{\alpha}D_{\xi}^{\beta} u (x,\xi)
\]
where each $g_{\alpha\beta}$ is a symbol of order $-r+| \beta |$ on $X\times\eR^N$, independent of $u$. In other words,
$(L^t)^r= \sum_{| \alpha |+| \beta |\leq r}g_{\alpha\beta}D_x^{\alpha}D_{\xi}^{\beta}$.

In particular $a\to (L^t)ra$ is a continuous linear map from $\mS^m$ to $\mS^{m-r}$.

\end{lemma}

We will not prove all assertions.

\begin{proof}
We consider the function
\[
  g(x,\xi)=| \xi |^2\sum_{k=1}^N\big( D''_k\varphi(x,\xi) \big)^2+\sum_{j=1}^n\big( D'_j\varphi(x,\xi)^2.
\]
Since $(D''_k\varphi)(x,\lambda\xi)=(D''_k\varphi)(x,\xi)$, the function $g$ is positively homogeneous of degree $2$ with respect to $\xi$ and from the fact the $\varphi$ has no critical points, there is always at least one no zero term the two sums, then $g(x,\xi)\neq 0$ on $X\times\eR^N_0$. Now we consider a $C^{\infty}$ function $\dpt{h}{\eR}{\eR}$ with compact support (hence a symbol of all order) and equals to $1$ in a neighbourhood of zero.

We are going to study the function $\frac{1-h(| \xi |)}{g(x,\xi)}$. In a neighbourhood of zero, the numerator is zero; then we don't mind with the values of $1/g(x,\xi)$ when $| \xi |\to 0$. Proposition~\ref{prop:23.16.4} shows that $g\mS^2(X\times\eR^N)$. On the other hand the inequality
\[
  | g(x,\xi)\geq C_H(1+| \xi |) |
\]
holds for a certain constant $C_H$ because both side are of degree two. Then point~\ref{enufi} of theorem~\ref{tho:lenumf} assures that $1/g$ is a symbol of $ \mS^{-2}$. So $\frac{1-h(| \xi |)}{g(x,\xi)}\in\mS^{-2}$. Now we claim that $Le^{i\varphi}=e^{i\varphi}$ when
\begin{equation}
\begin{aligned}
a_k(x,\xi)&=| \xi |^2\frac{1-h(| \xi |)}{ig(x,\xi)}D''_k\varphi(x,\xi)\\
b_j(x,\xi)&=\frac{1-h(| \xi |)}{ig(x,\xi)}D'_j\varphi(x,\xi)\\
c(x,\xi)&=h(| \xi |).
\end{aligned}
\end{equation}
A simple counting shows that $a_k\in\mS^0$ and $b_j\in\mS^{-2}$. The computation is easy:
\begin{equation}
\begin{split}
  (Le^{i\varphi})(x,\xi)&=\sum_ka_k(x,\xi)(D''_ke^{i\varphi})(x,\xi)\\
                          &\quad+\sum_j(x,\xi)(D'_je^{i\varphi})(x,\xi)\\
                          &\quad+c(x,\xi)e^{i\varphi(x,\xi)}\\
                        &=e^{i\varphi}\frac{1-h(| \xi |)}{ig}\Big( i\sum_k| \xi |^2 (D''_k\varphi)^2+\sum_ji(D'_j\varphi)^2 \Big)+h(| \xi |)e^{i\varphi}\\
                        &=e^{i\varphi}.
\end{split}
\end{equation}
We don't prove the other assertions.

\end{proof}

The theorem which defines the oscillatory integrals is the following

\begin{theorem}
Let $\varphi$ be a phase on $X\times\eR^N$. For all $m\in\eR$ and for all compact $H\subset X$, the $\eC$-linear form $\cdD(X\times\eR^N)\to\eC$
\begin{equation}
  a\to \int_{H\times\eR^N}e^{i\varphi(x,\xi)}a(x,\xi)dxd\xi
\end{equation}
admits an unique extension into a continuous map $S^m(X)\times\eR^N\to\eC$.

\end{theorem}

\begin{proof}\label{eskdfml}
  As far as integration is concerned, the set $X\times\{ 0 \}$ is negligible, then one can apply the formula $Le^{i\varphi}=e^{i\varphi}$ although id is only true on $X\times\eR^N_0$. From construction of $L$ and the definition of the transpose,
\[
  \scald{e^{i\varphi}}{w}=\scald{Le^{i\varphi}}{w}=\scald{e^{i\varphi}}{L^tw},
\]
then in $\cdD(X\times\eR^N)$, the studied linear form reads
\begin{equation} \label{eq:intaprol}
   a\to\int_{H\times\eR^N}e^{i\varphi}(L^t)^ra
\end{equation}
for any integer $r>0$. If one choses $r>m+N$, then $(L^t)^ra\in\mS^{m-r}$ if $a\in\mS^m$ and the integral exists and is a continuous function of $a$ from discussion below equation \eqref{eq:formaprol}. Then the integral in equation \eqref{eq:intaprol} defines a prolongation of the map $u\to\int e^{i\varphi}u$ into the whole $\mS^m$ when $r>m+N$.

Lemma~\ref{lem:DadhS} makes that any element in $\mS^mX\times\eR^N$ is adherent to $\cdD(X\times\eR^N)$, then a continuous prolongation is unique.

\end{proof}

The (prolonged) linear form
\[
  a\to\int_{H\times\eR^N}e^{i\varphi}a
\]
is continuous at all $a\in\mS^m$ for all $m\in\eR$. If $a\in\mS^{\infty}(X\times\eR^N)$, we write it by
\begin{equation}
a\to \osiint_{H\times\eR^N}e^{i\varphi(x,\xi)}a(x,\xi)dxd\xi,
\end{equation}
and it is called an \defe{oscillatory integral}{oscillatory integral}. It's value is given by formula
\[
  \osiint_{H\times\eR^N}e^{i\varphi}a=\int_{H\times\eR^N}e^{i\varphi}(L^t)^ra
\]
with $r>m+N$. Let $h\in\cdD(X\times\eR^N)$ equals to $1$ at $(x,\xi)$ if $x\in H$ and $| \xi |<R$; then lemma~\ref{lem:limha} implies $\lim_{q\to\infty}h_{1/q}a=a$ in the sense of $\mS^m$ when $a\in\mS^{m'}$ and $m>m'$. Since the oscillatory integral is continuous for each $\mS^m$, one can commute the limit and the integral:
\[
  \osiint_{H\times\\eR^N}e^{i\varphi}a=\lim_{q\to\infty}e^{i\varphi}h_{1/q}a.
\]

\subsection{Parametric oscillatory integral}
%-------------------------------------------

Let us introduce a parameter in the integral. We suppose the parameter belongs to an open set $\mU\subset\eR^N$ and we consider a symbol $a\in\mS^m(X\times\mU\times\eR^N)$. For each $z\in\mU$, the function $(x,\xi)\to a(x,z,\xi)$ is a symbol of order $m$ in $X\times\eR^N$. For all compact sets $H\subset X$ and $K\subset\mU$, there exists a constant $C_{HK}$ (independent of $z\in\mU$) such that
\begin{equation}
| D^{\alpha}_xD^{\beta}_{\xi}a(x,z,\xi) |\leq C_{HK}(1+| \xi |)^{m-| \beta |}
\end{equation}
for all $(x,z,\xi)\in H\times K\times\eR^N$. This comes from the fact that all compact in $X\times\mU$ is the product of a compact in $X$ by a compact in $\mU$.


\begin{theorem}
Let $X$ be an open set in $\eR^N$, $\mU$ an open in $\eR^q$ and $\dpt{\varphi}{X\times\mU\times\eR^N}{\eR}$ be a continuous function, $C^{\infty}$ in $X\times\mU\times\eR^N_0$ such that for all $z\in\mU$, the function $(x,\xi)\to(x,z,\xi)$ is a phase on $X\times\eR^N$. Then

\begin{enumerate}
\item for all $a\in\mS^m(X\times\mU\times\eR^N)$ and all compact $H\subset X$, the map
\[
  z\to F_a(z)=\osiint_{H\times\eR^N}e^{i\varphi(x,z,\xi)}a(x,z,\xi)dxd\xi
\]
is $C^{\infty}$ in $\mU$ and $a\to F_a$ is continuous from $\mS^m(X\times\mU\times\eR^N)$ to $\cdE(\mU)$,

\item the function $\varphi$ is a phase on $X\times\mU\times\eR^N$ and for all compact $K\subset\mU$,
\begin{equation} \label{eq:intKFa}
  \int_K F_a(z)dz=\osiint_{H\times K\times\eR^N}e^{i\varphi}a.
\end{equation}

\end{enumerate}

\end{theorem}

\begin{proof}
Let us begin by proving that $\varphi$ is a phase on $X\times\mU\times\eR^N$. From assumptions it is $C^{\infty}$ on $X\times\mU\times\eR^N_0$, the positive homogeneity comes from the fact that for each $z\in\mU$, the function $(x,\xi)\to(x,z,\xi)$ is positively homogeneous and the partial derivatives cannot vanish at same point because the one with respect to $x$ and $\xi$ yet doesn't.

Now we define $\varphi_z=\varphi(x,z,\xi)$ and the operator $L_z$ given by lemma~\ref{lem:defL}. Then for all $u\in\cdE(X\times\mU\times\eR^N)$ and for all $r>0$,
\[
 \big((L_z^t)^ru\big)(x,z,\xi)=\sum_{| \alpha |+| \beta |}g_{\alpha\beta}(x,z,\xi)D_x^{\alpha}D_{\xi}^{\beta}u(x,z,\xi)
\]
where $g_{\alpha\beta}$ is a symbol of order $-r+| \beta |$ in $X\times\mU\times\eR^N$, see proof of lemma~\ref{lem:defL}. Since $L_ze^{i\varphi_z}=e^{i\varphi_z}$, the equation
\[
  F_a(z)=\int_{H\times\eR^N}e^{i\varphi}(L_z^t)^ra
\]
holds for all $a\in\cdD(X\times\mU\times\eR^N)$. If $a$ is just a symbol, the problem is no more complicated because the definition is
\[
  \osiint_{H\times\eR^N}e^{i\varphi}a=\int_{H\times\eR^N}e^{i\varphi}(L_z^t)^a,
\]
with $r>m+N$. So we have to prove the continuity of
\[
  z\to\int_{H\times\eR^N}e^{i\varphi}(L_z^t)^ra.
\]
We do it by using proposition~\ref{prop:fdefint}. First, for all $z\in\mU$, the map $(x,\xi)\to e^{i\varphi(x,z,\xi)(L_z^t)^r}a(x,z,\xi)$ is integrable from our choice of $r$. Second, the map $z\to e^{i\varphi(x,z,\xi)}(L_z^t)^ra(x,z,\xi)$ is continuous on the whole $\mU$. The third assumption of proposition~\ref{prop:fdefint} is the true point.
\begin{equation}
\begin{split}
  | e^{i\varphi}(L^t)^ra |	&\leq | L_z^ta |\\
				&=\Big| \sum_{| \alpha |+| \beta |\leq r}g_{\alpha\beta}(x,z,\xi)D_x^{\alpha}D_{\xi}^{\beta}a(x,z,\xi)    \Big|\\
				&\leq \sum | g_{\alpha\beta}(x,z,\xi) | |D_x^{\alpha}D_{\xi}^{\beta}a(x,z,\xi) |\\
				&\leq C_{\alpha\beta}C'_{\alpha\beta}(1+| \xi |)^{m-r}
\end{split}
\end{equation}
which integrable over $H\times\eR^N$ because $r>m+N$.

Now we prove equation \eqref{eq:intKFa}. For this, we write the definition of the oscillatory integral of the right hand side and we compute:
\begin{equation}
\begin{split}
  \osiiint_{H\times K\times\eR^N}e^{i\varphi}a	&=\iiint_{H\times K\times\eR^N}e^{i\varphi}(L^t)^ra \\
						&=\iiint_{H\times K\times\eR^N}L^re^{i\varphi_z}a\\
						&=\iiint_{H\times K\times\eR^N}L_z^re^{i\varphi_z}\\
						&=\iiint_{H\times K\times\eR^N}e^{i\varphi_z(x,\xi)}(L_z^t)^ra(x,z,\xi)dxdzd\xi\\
						&=\int_K\osiint_{H\times\eR^N}e^{i\varphi_z}a\,dz\\
						&=\int_KF_a(z)dz.
\end{split}
\end{equation}
We don't give the rest of the proof.

\end{proof}

We give without proof the formula
\begin{equation}
D_z^{\alpha}F_a(z)=\osiint_{H\times\eR^N}D_z^{\alpha}(e^{i\varphi}a).
\end{equation}

Now let us consider the situation in which $z\in\mU$ is the only variable in front of $\xi$. More precisely, consider a function $\dpt{\varphi}{\mU\times\eR^N}{\eR}$ which is $C^{\infty}$ on $\mU\times\eR^N_0$ and continuous on $\mU\times\eR^N$ such that for all $z\in\mU$, the function $\xi\to\varphi(z,\xi)$ is positively homogeneous of degree $1$, without critical points, i.e. $\varphi_z$ is a phase in the sense of definition~\ref{def:phase} with $X=\emptyset$. Then for all symbol $a\in\mS^m(\mU\times\eR^N)$, the map
\[
  z\to F_a(z)=\osint_{\eR^N}e^{i\varphi}a
\]
is $C^{\infty}$ on $\mU$ and  $a\to F_a$ is continuous from $S^m(\mU\times\eR^N)$ into $\cdE(\mU)$. One can moreover commutes the integral and the derivative
\[
  D_z^{\alpha}F_a(z)=\osint_{\eR^N}D^{\alpha}_z\big( e^{i\varphi(z,\xi)}a(z,\xi) \big)d\xi
\]
The function $\varphi$ is a phase on $\mU\times\eR^N$ and for all compact part $K$ of $\mU$, one has
\[
  \int_KF_a=\osiint_{K\times\eR^N}e^{i\varphi}a.
\]

\section{Other constructions of oscillatory integral}
%+++++++++++++++++++++++++++++++++++++++++++++++++++++

\subsection{Derivation definition}
%---------------------------------

This construction is very the same as the first one, and comes from \cite{Rieffel}.

We consider a strongly continuous action $\tau$ of $V$ on $ C_u(V,A)$. Let $\svec^A(V)$ be the set of smooth vectors for this action; if $W=V\times V$, the set $\svec^A(W)$ is well defined and we choose on $V$ a measure such that the unit cube has measure $1$. Let $\dpt{\varphi}{V\times V}{\eR}$, $\varphi(u,v)=2\pi u\cdot v$; we want to give a sense to the integral
\[
  \iint_{V\times V}F(u,v)e^{i\varphi(u,v)}dudv
\]
when $F\in\svec^A(W)$. As usual, we first consider the case where $F$ is compactly supported; in this case the integral makes sense with usual definitions. An integration by part gives
\[
  \iint_{V\times V}(\partial_{k,u}F)e^{i\varphi}=-\iint_{V\times V}2\pi i v_k Fe^{i\varphi}.
\]
Taking a second derivative and making the same with the variable $v$,
\[
  \iint_{V\times V}(\partial_{sv}\partial_{rv}+\partial_{ku}\partial_{lu})Fe^{i\varphi}=-4\pi^2\iint_{V\times V}(u_su_r+v_kv_l)Fe^{i\varphi};
\]
taking $s=r$ and $k=l$, and making sum, we find
\begin{equation} \label{eq:int_Delta}
\iint_{V\times V}(\Delta F)e^{i\varphi}=-4\pi^2\iint_{V\times V} (u\cdot u+v\cdot v)Fe^{i\varphi}.
\end{equation}
Now we pose $w=(u,v)\in W$ and $Qw=u\cdot v$;  equation \eqref{eq:int_Delta} reads
\begin{equation} \label{eq:int_unmDelta}
   \int_W[(1-\Delta/4\pi^2)F](w)e^{iQw}=\int_WF(w)(1+w\cdot w)e^{iQw}.
\end{equation}
This equation only holds for compact supported functions $F$. In particular, it is also true for $F'=F/(1+w^2)$ where $w^2$ stands for $w\cdot w$. For notational convenience, we write $K(w)=(1+w^2)^{-1}$ and $M_K$ the operator of pointwise multiplication by $K(w)$. With $F'$ instead of $F$, equation \eqref{eq:int_unmDelta} gives $\int_W(1- \Delta/4\pi^2)(M_KF)e^{iQ}=\int_W Fe^{iQ}$. Making $k$ times all the work,
\begin{equation}
\int_W [(1-\Delta/4\pi^2)M_K]^kFe^{iQ}=\int_WFe^{iQ}.
\end{equation}

Now let us see how to write $[(1-\Delta/4\pi^2)K]^kF$. Direct computation shows that
\[
  \Delta(KF)=K(\Delta F-2KF+8K^3w^2F)=K\sum_{| \nu |\leq 2}B_{\nu}\partial^{\nu}F
\]
where $B_{\nu}$ are bounded functions. By induction, we see that
\[
  [(1-\Delta )K]^kF=K^k\sum_{| \nu |\leq 2k}B_{\nu}\partial^{\nu}F.
\]
So one can write
\begin{equation} \label{eq:defintosc}
\int_WFe^{iQ}=\int_WK^k\big( \sum_{| \nu |\leq 2k}B_{\nu}\partial^{\nu}F \big)e^{iQ}
\end{equation}
for all $k$ and all function $F$ with compact support. But we know that $K^k$ is integrable when $k>d$, and if $F$ is  a smooth vector for the action \eqref{eq:def_act_tau}, it is bounded on any bounded set (and all the derivatives of $F$ too). So the function $K^k\sum_{\nu}B_{\nu}\partial^{\nu}Fe^{iQ}$ is the product of an integrable function by a bounded function; it is then integrable. The conclusion is that the right hand side of equation \eqref{eq:defintosc} makes sense for all $F\in\svec^A(W)$. We take it as \emph{definition} of the left hand side.

\subsubsection{Value estimations for the integral}
%////////////////////////////////////////////////////

Let us suppose that $F$ has a compact support $S(F)$ and consider the real numbers $c_k=\max_{| \nu |\leq 2k}\nu!\| B_{\nu} \|_{\infty}$ where $\| . \|_{\infty}$ denotes the usual supremum norm.

\begin{lemma} \label{lem:born_osci_un}
For all $k>0$, there exists a constant $c_k$ such that for all $F\in\svec^A(W)$ with compact support $S(F)$ and all $j$,
\[
   \| \int_WFe^{iQ} \|_j\leq c_K\| F \|_{j,2k}\int_{S(F)}K^k
\]
\end{lemma}

\begin{proof}

From definition \eqref{eq:defintosc},
\begin{equation}
  \| \int_W(w)Fe^{iQw} \|_j \leq\int_{S(F)}\| K(w)^k \|_j\sum_{| \nu |\leq 2k} \|B_{\nu}(w)\partial^{\nu}F(w) \|,
\end{equation}
We can majore $\|B_{\nu}(w)\|_j$ by $\| B_{\nu} \|_{\infty}=\sup_{w\in W}\| B_{\nu}(w) \|_j$\quext{Faut voir si ce $\sup$ est prit aussi sur les $j$, et je crois bien que oui; en tout cas, je l'utilise} which can at his turn be majored by $c_k$; the rest of the sum  is precisely $\| F \|_{j,2k}$. Since $K(w)>0$, we can forget the norm around $K^k$.

\end{proof}

Note that when $S(F)$ goes to infinity, the integral $\int_WK^k$ goes to zero and so does the integral $\int_WFe^{iQ}$. Intuitively, it comes from the fact that when $w\to\infty$, the function $F$ doesn't makes anything\footnote{Because it is a function in $A^{\infty}$}  while function $e^{iQ}$ oscillate rapidly and provokes cancellations.

Let us suppose that $S(F)$ is contained in a ball $B(w_0,r)$ of radius $r$ around the point $w_0$ and suppose that $| w_0 |<r$. In this case, $K$ takes its bigger value when $w^2=(| w_0 |-r)^2$. But there exists a constant $c'$ (depending on $r$) such that $1+(| w_0 |-r)^2>c'(1+| w_0 |^2)$. Taking volume of $B(w_0,r)$ into account ,we can compute the following majoration
\begin{equation}
\begin{split}
\int_{B(w_0,r)}K^k(w)&\leq vol\big(B(w_0,r)\big)\left( \frac{1}{1+(| w_0 |-r)^2} \right)^k\\
                     &\leq \left( \frac{vol\big( B(w_0,r) \big)}{c'}^k \right)\left( \frac{1}{1+| w_0 |^2} \right)^k\\
                     &=cK(w_0)^k
\end{split}
\end{equation}
where $c$ is a new constant which take into account all other constants. If $| w_0 |>r$, one can redefine the constant in such a way that this majoration still holds. Using inequality of lemma~\ref{lem:born_osci_un} and absorbing once again all constant in $c$, we find the

\begin{proposition}
   For all $F\in\svec^A(W)$ with support in $B(w_0,r)$ and for all $j$, there exists a constant $c$ depending only on $k$ and $r$ such that
\begin{equation}
  \| \int_WFe^{iQ} \|_j\leq c\| F \|_{j,2k}K(w_0)^k.
\end{equation}
In particular, $c$ doesn't depends on $j$.
\end{proposition}


\subsection{Lattice way to define oscillatory integrals}
%//////////////////////////////////////////////////////////

The latter proposition holds for a compact supported function; in order to see what happens when the support becomes non compact, choose a basis of $W$ and let $L$ be the lattice of points with integer components in this basis. Consider a function $\varphi_0\in C^{\infty}_c(W)$, a smooth compact supported function on $W$ and $\Phi(w)=\sum_{p\in L}\varphi_0(w+p)$. We choose $\varphi_0$ in such a way that $\Phi$ vanishes nowhere. Let $\varphi=\varphi_0/\Phi$ and $\varphi_p(w)=\varphi(w+p)$ for $p\in L$. By construction, $\sum_{p\in L}\varphi_p=1$ and each compact in $W$ intersect only finitely many support of $\varphi_p$.

Let $F\in\svec^A(W)$; if we suppose $S(\varphi)\subset B(0,r)$, then $S(F,\varphi_p)\subset B(p,r)$ and the proposition can be used with $F\varphi_p$. By virtue of the Leibnitz rule on $F\varphi_p$, we see that $\| \partial^{\nu}(F\varphi_p) \|_j$ is domined by a linear combination of terms of the form $\| \partial^{\nu}F \|_j\| \partial^{\rho}\varphi_p \|_{\infty}$ with $| \mu+\rho |\leq | \nu |$. But for all $p\in L$ and all multi-index $\rho$, $\| \partial^{\rho}\varphi \|_{\infty}=\| \partial^{\rho}\varphi \|_{\infty}$. Then if we fix $r$ and $\varphi$, the value of $\| F\varphi_p \|_{j,2k}$ is bounded by a sum from which  $\| \partial^{\rho}\varphi \|_{\infty}$ get out while the remaining sum is $\| F \|_{j,2k}$. By redefining the constant $c_k$, we get
\begin{equation} \label{eq:pr_abs_conv}
  \| \int_WF\varphi_pe^{iQ} \|\leq c_kK(p)^k\| F \|_{j,2k}.
\end{equation}
We can find a $k$ large enough that $\sum_{p\in L}K(p)^k$. For such a $k$, formula \eqref{eq:pr_abs_conv} proves the absolute convergence of the sum $\sum_{p\in L}\int_W(F\varphi_p)e^{iQ}$.

\begin{definition} \label{def:intFeiQ}
  When $F\in\svec^A(W)$ has no compact support, we define
\begin{equation}
   \int_WFe^{iQ}=\sum_{p\in L}\int_W(F\varphi_p)e^{iQ}.
\end{equation}

\end{definition}
When $F$ has a compact support, only finitely many terms are non zero and we can permute the sum and the integral. Since for all $x$, $\sum_{p\in L}(F\varphi_p)(x)=F(X)$, we find back the usual integral. For such a $k$, we can sum equation \eqref{eq:pr_abs_conv} over $p\in L$ and redefine $c_k$ and find
\[
  \| \int_W Fe^{iQ} \|\leq c_k\| F \|_{j,2k}.
\]

\begin{proposition}   \label{prop:leqd_m}
  For all $k>d$ and all positive sequence $(r_i)\to\infty$, there exists a real sequence $(d_i)\to 0$ such that for all $F\in\svec^A(W)$ vanishing on $B(0,r_m)$, the inequality
\[
  \| \int_WFe^{iQ} \|_j\leq d_m\| F \|_{j,2k}.
\]
holds.
\end{proposition}

Stated in a more natural way, the statement is that if $F=0$ in $B(0,r)$, then there exists a constant $d$ such that $\| \int_WFe^{iQ} \|_j\leq d\| F \|_{j,2k}$. The dependence of $r$ with respect to $r$ is such that $lim_{r\to\infty}d(r)=0$. This corresponds to the intuition that, since $e^{iQ}$, oscillates more and more rapidly with distance to origin, the contribution to the integral is mainly contained in the values of $F$ near zero.

\begin{proof}
Let $E_m=\{ p\in L\tq S(\varphi_p)\nsubseteq B(0,r_m) \}$. If $p\in E_m$, then $\varphi_p$ takes non zero values outside $r_m$. In the expression $\int_WFe^{iQ}=\sum_{p\in L}\int-WF\varphi_pe^{iQ}$, we can reduce the sum because $F$ is zero on $B(o,r_m)$: the terms with $p\notin E_m$ are zero. Then, using \eqref{eq:pr_abs_conv},
\[
  \| \int_WFe^{iQ} \|_j\leq\sum_{p\in E_m}\| \int_W\varphi_pFe^{iQ} \|_j\leq\sum_{p\in E_m}c_kK(p)^k\| F \|_{j,2k}.
\]
For suitably large $k$ (i.e. $k>d$ as in the assumptions), convergence of the sequence
\[
 d_m=c_k\sum_{p\in E_m}K(p)^k
\]
can be studied by the integral $\int_{\complement B(0,r_m)}K(p)^kdp$ which converges to zero when $m\to\infty$.
\end{proof}

\subsubsection{Independence with respect to choices}
%///////////////////////////////////////////////////

We have to show that definition~\ref{def:intFeiQ} don't depend on the choice of $L$ nor $\varphi$. For this, we will find a sequence of numbers independent of $L$ and $\varphi$ which converges to $\int_WFe^{iQ}$. Let $(\psi_i)\in C^{\infty}_c$, a sequence of functions with $\psi_i=1$ on $B(0,r_i)$ for a certain sequence $(r_m\in\eR^+)\to\infty$. For each $k$, we requires that the functions $\psi_i$ are equibounded for the norm $2k$ in the sense that there exists a real $b_k$ such that $\| \psi_i \|_{2k}\leq b_k$ for all $i$. Such a sequence can be build by choosing $\psi_1$ and defining $\psi_i(w)=\psi_1(w/r_i)$.

Remark that for each $i$, the sum $F\psi_i=\sim_{p\in L}F\psi_i\varphi_p$ is pointwise a finite because $\psi_i$ has compact support; then we can permute sum and integral and write
\[
  \int_W \sum_pF\psi_i\varphi_pe^{iQ}=\sum_p\int_WF\psi_i\varphi_ie^{iQ}.
\]
This allows us to write down the following majoration
\begin{equation}
\begin{aligned}
 \| \int_WFe^{iQ}-\int_WFe^{iQ}\psi_i \|_j&=\| \sum_{p\in L}\int_WF\varphi_p(1-\psi_i)e^{iQ} \|_j&&\textrm{Usual integral}\\
                                          &=\| \int_WF(1-\psi_i)e^{iQ} \|_j                      &&\textrm{Oscillatory integral}\\
                                          & \leq d_i\| F(1-\psi_i) \|_{j,2k}\\
                  &=d_i\| F \|_{j,2k}(1+b_k).
\end{aligned}
\end{equation}
The latter converges to zero when $i\to\infty$. Then
\begin{equation}
   \lim_{m\to\infty}\int_WF\psi_me^{iQ}=\int_WFe^{iQ}
\end{equation}
where the left hand side don't depend on $L$ and $\varphi$.

One can take this equation as \emph{definition} of the right hand side.

\begin{probleme}
    Let \( f\in L^2(\eR^n)\). Does the integral
    \begin{equation}
        \int  e^{i\xi x}f(x)dx
    \end{equation}
    always exist in the sense of the oscillatory integral?

    If yes, does the value of that integral equal the value of the Fourier transform of \( f\) as defined by extension from \( L^1\cap L^2\) by~\ref{THOooJLCDooAjTvJf}?
\end{probleme}

\section{Decomposition into direct sum}
%++++++++++++++++++++++++++++++++++++++

Let $W=W_0\oplus W_1$ be a direct sum decomposition; we denote by $w=w_0+w_1$ the corresponding decomposition for the vectors $w\in W$. Let $B_i$ be a basis of $W_i$ and $L_i$ be the corresponding lattice. Since $B_0\cup B_1$ is a basis of $W$, we can consider the lattice $L_0\times L_1$ of $W$. Let $\varphi\in C^{\infty}_x$ be a positive function on $W_1$ with $\sum_{p\in L_0}\varphi_p=0$ and the same for $\varphi'$ for $W_1$. The function $\lambda(w)=\varphi(w_0)\varphi'(w_1)$ belongs to $C^{\infty}_c(W)$ and since the sums are pointwise finite,
\[
  \sum_{p\in L}\lambda_p(w)=\sum_{p\in L}\varphi(w_0+p_0)\varphi'(w_1+p_1)=1.
\]
Therefore the function $\lambda$ has the required properties to define the oscillatory integral
\begin{equation}  \label{eq:osc_int_phip}
\begin{split}
    \int_WFe^{iQ}&=\sum_{p\in L}\int_W\lambda_pe^{iQ}\\
                 &=\sum_{\substack{p_0\in L_0\\p_1\in L_1}}F(w)\varphi_{p_0}(w_0)\varphi'_{p_1}(w_1)e^{iQw}dw
\end{split}
\end{equation}
with an absolutely convergent sum, there are no ordering problem in the summation.

\begin{proposition}
Let $k>d$ and an increasing sequence $(r_i\in\eR^+)\to\infty$. Then there exists a sequence $(d_i\in\eR)\to 0$ such that, if $F\in\svec^A(V\times V)$ and $F(u,v)=0$ when $u\in B(0,r_m)$,  then
\[
  \| \iint_{V\times V}e^{i\phi(u,v)} \|_j\leq d_m\| F \|_{j,2k}
\]
Here, $\phi(u,v)=2\pi u\cdot v$ is the former $\varphi$.
\end{proposition}

\begin{proof}
Let $E_m=\{ p\tq S(\varphi_p)\nsubseteq B(o,r_m) \}$. The computation is performed in rather the same way as in proposition~\ref{prop:leqd_m}:
\begin{equation}  \label{eq:re_fppe}
\| \iint_{V\times V}Fe^{iQ} \|_j=\| \sum_{p\in E_m}\sum_q F(u,v)\lambda_{p+q}(u+v)e^{i\phi}  \|
\end{equation}
where $\lambda_{p+q}(u+v)=\varphi_p(u)\varphi'_q(v)$ because $(u,v)$ and $(p,q)$ are seen as vectors of $W$. If $\varphi$ has support contained in $B(r,0)$, the function $F\varphi_p\varphi'_q$ has support contained in the ball centered at $p+q$; the proposition
\ref{lem:born_osci_un} gives
\[
  \| \iint_{V\times V}F\varphi_p\varphi_q \|\leq c_k\| F\varphi_p\varphi'_q \|K(p+q)^k.
\]
If we suppose that the sum $W=W_0\oplus W_1$ is orthogonal, $K(p+q)=(1+p\cdot p+q\cdot q)$, a simple redefinition of $c_k$, leads to $c_k\| F\varphi_p\varphi'_q \|\leq c_k\| F \|$. So we can majore equation \eqref{eq:re_fppe} by
\[
  \sum_{p\in E_m}\sum_qc_k\| F \|_{j,2k}(1+p^2+q^2)^{-k}.
\]
Now the definition $d_m=\sum_{p\in E_m}\sum_qc_k(1+p^2+q^2)^{-k}$ gives the thesis.
\end{proof}

\begin{proposition}\
Let us consider a sequence $(\psi_i)\in C^{\infty}(W_0)$ with $\psi_i=1$ on $B(0,r_i)$ and $\| \psi_i \|_{2k}\leq b_k$ for all $i$. Then for all $F\in\svec^A(W)$, we have
\begin{equation}
\int Fe^{iQ}=\lim_{m\to\infty}\int F(w)\psi_m(w_0)e^{iQw}.
\end{equation}

\end{proposition}


\begin{proof}
We consider $\varphi$ and $\varphi'$ as in equation \eqref{eq:osc_int_phip} and we write $\varphi_p$ instead of $\varphi_p\circ\pr_0$ where $\pr_0$ is the projection into $W_0$. Then
\begin{equation}
\begin{split}
  \int_WFe^{iQ}-\int_W(\psi_m\circ\pr_0)e^{iQ}&=\sum_{p,q}\int_WF\varphi_p\varphi'_qe^{iQ}
                          -\sum_{p,q}F(\psi_m\circ\pr_0)\varphi_p\varphi'_qe^{iQ}\\
                      &=\sum_{p;q}\int_WF(1-\psi_m\pr_0)\varphi_p\varphi'_qe^{iQ}
\end{split}
\end{equation}
which converges absolutely. By setting $E_m=\{ p\in L_0\tq S(\varphi_p)\nsubseteq B(0,r_m) \}$, we can reduce the sum over $p$ to $E_m$ and majore the norm of the latter expression by $\sum_{p\in E_m}\sum_qK(p+q)^k$. It converges to $0$ when $m\to \infty$.

\end{proof}

Let us now suppose that we have functions $(\psi'_i)$ on $W_1$ with the usual properties. Writing the latter proposition with $F\psi'_n$ instead of $F$, we find that, for all $b$,
\[
  \lim_{m\to\infty}\int_WF(w)\psi_m(w_0)\psi'_n(w_1)e^{iQw}=\int_WF(w)\psi'_n(w_1)e^{iQw}
\]
and inverting $W_1$ with $W_2$ in the preceding proposition, we see that the right hand side converges to $0$ when $n\to\infty$. So we find
\begin{equation} \label{eq:lim_mn_int}
  \lim_{m,n\to\infty}\int_WF(\psi_m\circ\pr_0)(\psi'_n\circ\pr_1)e^{iQ}=\int_WFe^{iQ},
\end{equation}
and we can take the limit of $n$ before $m$ if we want.


\subsection{Integral over \texorpdfstring{$V\times V$}{VV}}
%////////////////////////////////////////

Let an orthogonal direct sum decomposition $V=V_0\oplus V_1$ The decomposition $W=V_0\times V_1\times V_0\times V_1$ is naturally orthogonal. Consider an usual function $\dpt{F}{V\times V}{A}$ such that $F\big( (v_0+v_1),(v'_0,v'_1) \big)$ don't depend on $v_0$.

Let $(\psi_i)$ and $(\psi_j)$ be sequences of functions such that equation \eqref{eq:lim_mn_int} gives
\begin{equation} \label{eq:run}
  \iint F(u,v)e^{iQ(u,v)}=\lim_{mnjk}\iint F(u_1,v)\psi_m(u_0)\psi'_n(u_1)\psi_j(v_0)\psi'_k(v_1)e^{u_0\cdot v_0+u_1\cdot v_1}.
\end{equation}
We define the \defe{Fourier transform}{fourier transform} of $\psi_m$ by
\[
   \hat\psi_m(\xi)=\int_{V_0}\psi_m(u_0)e^{2\pi i u_0\cdot\xi}du_0.
\]
The main property is the inverse transform theorem:
\[
  \psi_m(v)=\int_{V_0}\hat\psi_m(\xi)e^{-2\pi i\xi\cdot v}d\xi=\psi_m(v).
\]
%
Equation \eqref{eq:run} can be rewritten as
\newcommand{\pheqrdeuxun}{\lim_{mnjk}\iint_{V\times V}F(u_1,v)\int_V}
\newcommand{\pheqrdeuxdeux}{\lim_{mnjk}\iiint_{V\times V\times V}F(u_1,v_1+v_0)}
\begin{equation} \label{eq:rdeux}
\begin{split}
  &\pheqrdeuxun\hat\psi_m(\xi)e^{-2\pi i\xi\cdot u_0}\psi'_n(u_1)\psi_j(v_0)\psi'_k(v_1)\\
&\phantom{\pheqrdeuxun}e^{2\pi i(u_0\cdot v_0+u_1\cdot v_1)}dudud\xi\\
   &=\pheqrdeuxdeux\hat\psi_m(\xi)\psi'_n(u_1)\psi_j(v_0)\psi'_k(v_1)\\
	&\phantom{pheqrdeuxdeux}e^{2\pi i[u_0\cdot(v_1-\xi)+u_1\cdot v_1]}dvdud\xi.\\
\intertext{The integral over $u_0$ gives a Dirac delta at $\xi-v_0$}
 &=\lim_{mnjk}\iint F(u_1,v_1+v_1)\hat\psi_m(v_0)\psi'_n(u_1)\psi_j(v_0)\psi'_k(v_1)e^{2\pi i u_1\cdot v_1}dudv.
\end{split}
\end{equation}

We can suppose that the functions $\psi_m$ are chosen in such a way that $(\hat\psi_m)$ is an approximation of the delta distribution. Indeed, if we choose $\psi_1$ equals to $1$ in $B(0,r_1)$, the sequence $\psi_m\to 1$ whose Fourier transform is the Dirac delta. In this case, the limit $m\to\infty$ gives $v_0=0$ and the limit $j\to \infty$ becomes trivial. Equation \eqref{eq:rdeux} becomes
\begin{equation}
\begin{split}
\quad&\lim_{nk}\iint F(u_1,v_1)\hat\psi(v_0)\psi'_n(u_1)\psi'_k(v_1)e^{2\pi i u_1\cdot v_1}\\
     &=\iint_{V_1\times V_1}F(u_1,v_1)e^{iQ(u_1,v_1)}.
\end{split}
\end{equation}
In definitive, we had shown

\begin{proposition} \label{prop:F_pas_uz}
Let $F\in\svec^A(V\times V)$ and an orthogonal decomposition $V=V_0\oplus V_1$ such that $F(u_0+u_1,v_0+v_1)$ don't depend on $u_0$. Then
\[
  \iint_{V\times V}Fe^{iQ}=\iint_{V_1\times V_1}F(u_1,v-1)e^{iQ(u_1,v_1)}.
\]

\end{proposition}


There exists a somewhat degenerate interesting case: $V_1=\{ 0 \}$ with a measure $1$. In this case the assumption is that $F(u,v)$ don't depend on $u$ at all. In this case,
\[
  \iint_{V_1\times V_1}F(u_1,v_1)e^{2\pi i u_1\cdot v_1}=\int_{V_1}F(v_1)dv_1=F(0).
\]
The result is that if $F\in\svec^A(V\times V)$ don't depend on his first variable (i.e. $F\in\svec^A(V)$), then
\[
  \int_{V\times V}Fe^{iQ}=F(0).
\]


\subsection{Normal operator}
%--------------------------

Let $T$ be a linear operator on $V$ and a decomposition $V=\ker T\oplus V_1$. In particular, $F(Tu,v)$  doesn't depends on $u_O$ (if we denote $\ker T$ by $V_0$) and we can apply proposition~\ref{prop:F_pas_uz}:
\[
  \int F(T_u,v)e^{iQ}=\iint_{V_1\times V_1}F(Tu_1,v_1)e^{iQ(u_1,v_1)}.
\]
Note that in general, $Tu_1$ don't belong to $V_1$. We now suppose that $T$ is \defe{normal}{normal!operator} in the sense that $[T,T^t]=0$. It is invertible on $V_1$. Then when we have a normal operator in an integral, one can always suppose that it is invertible because the kernel part of the space disappears.

Let $T$ be invertible on $V$, $F\in\svec^A(V\times V)$ and an usual sequence $(\psi_m)$ of functions. We have
\begin{equation}
\begin{split}
  \iint F(Tu,v)r^{iQ(u,v)}&=\lim_{m,n\to\infty}\iint F(Tu,v)\psi_m(u)\psi_n(v)e^{iQ}\\
                          &=\lim_{m,n\to\infty}(\det T)^{-1}\iint F(u,v)\psi_m(T^{-1}u)\psi_n(v)e^{2\pi i T^{-1} u\cdot v}\\
                          &=\lim_{m,n\to\infty}\iint F(u,T^tv)\psi_m(T^{-1}u)\psi_n(T^tv)e^{iQ}\\
\intertext{but the sequences $(\psi_m\circ T^{-1})$ and $(\psi_n\circ T^t)$ have same fundamental properties as $(\psi_m)$, then taking the limit,}
&=\iint F(u,T^tv)e^{iQ}.
\end{split}
\end{equation}
If $T$ is normal but not invertible, the first integral restricts to $V_1$ and for all normal operator, the equality\quext{Je ne comprends pas pourquoi.}
\[
  \iint F(Tu,v)e^{iQ}=\iint F(u,T^tv)e^{iQ}.
\]
hold. From polar decomposition of any operator into two normal operators, we conclude that this equation holds for all linear operator on $V$.

\begin{proposition}
If $\dpt{S}{A}{C}$ is a continuous linear map from $A$ to a Fréchet space $C$, then $S\circ F\in\svec^C(W)$ and
\begin{equation}
 S( \int_WFe^{iQ} )=\int_W(S\circ F)e^{iQ}.
\end{equation}

\end{proposition}

\begin{proof}
The absolute convergence of the defining sum for the left hand side integral allows us to permute linear operator and sum. Since for each term the integral is an usual integral over a compact set, we can permute the continuous operator $S$ and the integral:
\begin{equation}
  S\sum_{p\in L}\int F\varphi_pe^{iQ}=\sum S\int F\varphi_pe^{iQ}=\sum\int (S\circ F)\varphi_pe^{iQ}.
\end{equation}


\end{proof}




%%%%%%%%%%%%%%%%%%%%%%%%%%%%
%
   \section{Pseudo-differentiable operators}
%
%%%%%%%%%%%%%%%%%%%%%%%%


\subsection{Differential operator}
%---------------------------------

Let $\dpt{ \pi }{ E }{ M }$ be a vector bundle of rank $k$ on a compact manifold of dimension $n$. We denote by $\Gamma(E)$ the $ C^{\infty}(M)$-module of $ C^{\infty}$ sections of $E$. The module structure is given by $(f\cdot\psi)(x)=f(x)\psi(x)$ where the product in the right hand side is the product $\eC\times E_x\to E_x$ which defines the vector space structure on $E_x=\pi^{-1}(x)$.

A \defe{differential operator}{differential!operator}\index{operator!differential} of rank $m$  is a linear operator $\dpt{ P }{ \Gamma(E) }{ \Gamma(E) }$ for which there exists local coordinates $(x_1,\cdots,x_n)$ on $M$ in which $P$ is written as
\begin{equation}
P=\sum_{| \alpha |\leq m}A_{\alpha}(x)(-i)^{| \alpha |}\frac{ \partial^{| \alpha |} }{ \partial x^{\alpha} }
\end{equation}
where $\alpha=(\alpha_1,\cdots,\alpha_n)$ is a multi-index with $0\leq\alpha_j\leq m$ and $| \alpha |=\sum_{j=1}^n\alpha_j$. Each $A_{\alpha}$ is a $k\times k$ matrix of $ C^{\infty}$ functions on $M$ and $A_{\alpha}\neq 0$ for at least one $\alpha$ with $| \alpha |=m$.

Let $\xi\in T^*_xM$ written under the form $\xi=\sum_j\xi_j\,dx_j$ in the previously given local coordinates. The \defe{complete symbol}{symbol!complete}\index{complete!symbol} of $P$ applied to $\xi$ is the following polynomial combination of coordinates of $\xi$:
\begin{equation}
p^P(x,\xi)=\sum_{j=0}^m p^P_{m-j}(x,\xi)
\end{equation}
where
\[
  p^P_{m-j}=\sum_{| \alpha |\leq(m-j)}A_{\alpha}(x)\xi^{\alpha}
\]
where $\xi^{\alpha}=\xi_1^{\alpha_1}\cdots\xi_n^{\alpha_n}$ and $\partial^{| \alpha |}_{\alpha}=\partial_{x_1}^{\alpha_1}\circ\cdots\circ\partial^{\alpha_n}_{x_n}$.  The \defe{principal symbol}{principal!symbol}\index{symbol!principal} of $P$ is the leading term, namely
\begin{equation}
\sigma^P(x,\xi)=p^P_m(x,\xi)=\sum_{| \alpha |=m}A_{\alpha}(x)\xi^{\alpha}.
\end{equation}
For each $\xi\in T^*_xM$, the principal symbol gives rise to a map $\dpt{ \sigma_x^P }{ E_x }{ E_x }$ because $A_{\alpha}(x)$ is a $k\times k$ matrix:
\[
  \sigma_x^P(\xi)v=\sum_{| \alpha |=m}\xi^{\alpha}A_{\alpha}(x)v.
\]

\begin{definition}  \label{DefGLpDEHy}
    The operator $P$ is \defe{elliptic}{elliptic operator}\index{operator!elliptic} if for each non zero $\xi\in T^*M$, the map $\sigma_x^P(\xi_x)$ are all invertible.
\end{definition}

If $(M,g)$ is a Riemannian manifold, the fact for $P$ to be elliptic is equivalent to be invertible on the \defe{cosphere}{cosphere}
\[
  S^*M=\{ (x,\xi)\in T^*M\tq g_x(\xi,\xi)=1 \}\subset T^*M.
\]

%---------------------------------------------------------------------------------------------------------------------------
\subsection{Laplace operator}
%---------------------------------------------------------------------------------------------------------------------------

As first example, we see the Laplace operator. Let $(M,g)$ be a Riemannian manifold and its Laplace operator $\dpt{ \Delta }{  C^{\infty}(L) }{  C^{\infty}(M) }$ where $ C^{\infty}(M)$ is seen as $\Gamma(\eC\times M)$ is
\[
  \Delta f=-g^{\mu\nu}\partial^2_{\mu\nu}f+\text{lower order terms}
\]
Its principal symbol is $\sigma^{\Delta}(x,\xi)=g^{\mu\nu}\xi_{\mu}\xi_{\nu}=\| \xi \|^2$, which is invertible of all $\xi\neq 0$. Thus Laplace operator is elliptic of order $2$.

An other example of pseudo-differential operator is the Dirac operator. We will show that it is an elliptic operator in subsection~\ref{subSecREctBOh}.

%+++++++++++++++++++++++++++++++++++++++++++++++++++++++++++++++++++++++++++++++++++++++++++++++++++++++++++++++++++++++++++
\section{Invariant differential operators on Lie groups}
%+++++++++++++++++++++++++++++++++++++++++++++++++++++++++++++++++++++++++++++++++++++++++++++++++++++++++++++++++++++++++++

This section is intended to understand the paper \cite{QuantifKhalerian}. Let $G$ be a connected Lie group and $\lG$ be its Lie algebra.

An endomorphism $P\colon C^{\infty}(G,A)\to  C^{\infty}(G,A)$ of the space of functions over $G$ with values in a vector space $A$ is said to be \defe{\hypertarget{HyperDefLeftInvar}{left invariant}}{left invariant!operator}\index{differential!operator!left invariant} if
\begin{equation}  \label{EqDefLxinvarop}
	L_{x}(P\psi)=P(L_{x}\psi)
\end{equation}
for all $x\in G$ and $\psi\in C^{\infty}(G,A)$. Here, $L$ is the regular left representation of $G$ on $ C^{\infty}(G,A)$.

\begin{lemma}
	If $X\in\lG$ and if $\tilde X$ is the associated left invariant vector field, the operator $P$ defined by
	\begin{equation}
		(P\psi)(x)=\tilde X_{x}\psi
	\end{equation}
	is left invariant.
\end{lemma}

\begin{proof}
	On the one hand
	\[
	  \big( L_{x}(P\psi) \big)(y)=(P\psi)(xy)=\tilde X_{xy}\psi,
	\]
	on the other hand,
	\begin{equation}
		\begin{aligned}[]
			P(L_{x}\psi)(y)=\tilde X_{y}(L_{x}\psi)
				=\dsdd{ (L_{x}\psi)(y e^{tX}) }{t}{0}
				=\dsdd{ \psi(xy e^{tX}) }{t}{0}
				=\tilde X_{xy}\psi.
		\end{aligned}
	\end{equation}
\end{proof}

In fact, if $\Diff^G(G)$\nomenclature{$\Diff^{G}(G)$}{space of $G$-left invariant differential operators on $G$} denotes the space of $G$-left invariant differential operators on $G$, we have a morphism
\begin{equation}		\label{EqMorphismUgDiff}
	\begin{aligned}
		\mU(\lG)&\to \Diff^{G}(G) \\
		X&\mapsto \tilde X.
	\end{aligned}
\end{equation}
The operator $P$ is \defe{left invariant}{left invariant!differential operator} when it satisfies
\begin{equation}
	(Pf)(x)=P\big( L_x^*f \big)(e).
\end{equation}
As far as notations are concerned, we denote by $(L^*_xP)$\nomenclature[F]{$L^*_xP$}{Left translated differential operator} the operator defined by
\begin{equation}
	(L^*_xP)(f)=P(L^*_xf).
\end{equation}
Every differential operator $P\colon  C^{\infty}(G)\to  C^{\infty}(G)$ can be written as a sum of products $\sum_ip_i\tilde X_i$ of functions $p_i\in C^{\infty}(G)$ by operators $\tilde X_i\in\mU(\lG)$. The operator $P$ is left invariant if and only if the functions $p_i$ are constant.

As a consequence we have the
\begin{corollary}	\label{CorUisomDiff}
	The map $\mU(\lG)\to\Diff^G(G)$ is an isomorphism.
\end{corollary}

We denote by $\biDiff^G(G)$\nomenclature{$\biDiff(G)$}{Bidifferential operators on $G$} the space of $G$-left invariant \defe{bidifferential operators}{bidifferential operator} on $G$. These are differential operators
\begin{equation}
	\begin{aligned}
		A\colon  C^{\infty}(G)\times C^{\infty}(G)&\to  C^{\infty}(G) \\
		u\otimes v&\mapsto A(u\otimes v).
	\end{aligned}
\end{equation}
Since $A$ is a differential operator, the function $A(u,v)$ only depends on the class $u\otimes v$. The left invariance means that
\begin{equation}		\label{EqDefLeftInvarbiDiff}
	L_x\big( A(u\otimes v) \big)=A\big( L_x(u\otimes v) \big)
\end{equation}
for every $x\in G$.

\begin{proposition}		\label{PropbidiffUU}
	The space of left invariant bidifferential operators $\biDiff^G(G)$ is canonically isomorphic to the tensor product $\mU(\lG)\otimes\mU(\lG)$.
\end{proposition}

\begin{proof}
	Let $A$ be a bidifferential operator and $\{ X_i \}$ be a basis of $\lG$. For every multi-index $a=(a_1,\ldots,a_n)$ we write $X^a=X^{a_1}\cdots X^{a_n}$ the corresponding element in $\mU(\lG)$. Thus there exist functions $A_{ab}\in C^{\infty}(G)$ such that
	\begin{equation}	\label{EqdefAabdiffopgp}
		A(u\otimes v)=\sum_{ab}A_{ab}(\tilde X^au)(\tilde X^bv)
	\end{equation}
	for every $u,v\in C^{\infty}(G)$.

	Now we are supposing that $A$ is left invariant and we will prove that the functions $A_{ab}(x)$ are constant. We evaluate the definition \eqref{EqDefLeftInvarbiDiff} at $g\in G$:
	\begin{equation}
		\begin{aligned}[]
			A(u\otimes v)(xg)&=A\big( L_x(u\otimes v) \big)(g)\\
					&=A\big( (L^*_xu)\otimes (L^*_xv) \big)(g)\\
					&=A_{ab}(g)\tilde X^a_g(L^*_xu)\tilde X^b_g(L^*_xv).
		\end{aligned}
	\end{equation}
	If we consider that equation at $g=e$, we have
	\begin{equation}
		A(u\otimes v)(x)=A_{ab}(e)\tilde X^a_e(L^*_xu)\tilde X^b_e(L^*_xv)=A_{ab}(e)\tilde X^a_xu\tilde X^b_xv
	\end{equation}
	where we used the invariance property \eqref{EqInvarUgField} on each of the operators $X^a\in\mU(\lG)$. Now, the expression \eqref{EqdefAabdiffopgp} evaluated at $x$ gives
	\begin{equation}
		A(u\otimes v)(x)=A_{ab}(x)(\tilde X_x^au)(\tilde X_x^bv),
	\end{equation}
	so that $A_{ab}(x)=A_{ab}(e)$ when $A$ is left invariant.
\end{proof}

\begin{proof}[Alternative proof]
	Using the same notations, we have
	\begin{equation}		\label{EaAuvDeuxDun}
		A(u\otimes v)(x)=A_{ab}(x)\big( \tilde X^a_x\otimes\tilde X^b_x \big)(u\otimes v)=A_{ab}(x)\big( \tilde X^a_xu \big)\big( \tilde X^b_xv \big).
	\end{equation}
	Using the left invariance of $A$, on the other hand, we have
	\begin{equation}		\label{EqAuvUnDun}
		A(u\otimes v)(x)=A\big( L^*_xu\otimes L^*_xv \big)(e)=A_{ab}(e)\big( \tilde X^a_eL^*_xu \big)\otimes\big( \tilde X^b_eL^*_xv \big),
	\end{equation}
	but
	\begin{equation}
		\tilde X^a_eL^*_xu=\Dsdd{ (L^*_xu)( e^{t\tilde X^a}) }{t}{0}=\Dsdd{ u\big( x e^{t\tilde X^a} \big) }{t}{0}=\tilde X^a_xu,
	\end{equation}
	thus the expression \eqref{EqAuvUnDun} becomes
	\begin{equation}
		A_{ab}(e)\big( \tilde X^a_xu \big)\big( \tilde X^a_xv \big),
	\end{equation}
	and the comparison with \eqref{EaAuvDeuxDun} shows that $A_{ab}(x)=A_{ab}(e)$, so that the coefficients $A_{ab}$ are constant.
\end{proof}

\begin{remark}  \label{REMooGIFYooTphiex}
    If $\tilde X$ is the left invariant differential operator associated with $X\in\lG$, the Leibnitz rules reads
    \begin{equation}		\label{EqXfgDeltaUnif}
        \tilde X(fg)=\widetilde{\Delta(X)}(f\otimes g)
    \end{equation}
    where \( \Delta\) is the coproduct defined in~\ref{SUBSECooTKZAooWVXXug}.
\end{remark}

Let us now consider the space $\Diff(G\times G)$ of differential operators on $G\times G$. These operators an operator in $\Diff(G\times G)$ acts on the space of functions $ C^{\infty}(G\times G)$. Such an operator reads $X\cdot Y$ with $X,Y\in\mU(\lG)$. If $f\in C^{\infty}(G\times G)$,
\begin{equation}
	(Pf)(x,y)=(X\cdot Y f)(x,y)
\end{equation}
where $X$ acts on the first variable and $Y$ acts on the second variable. The space $\Diff(G\times G)$ is then naturally isomorphic to $\mU(\lG)\otimes \mU(\lG)$ as an operator $P\in\Diff(G\times G)$ can be written as $P=X\otimes Y$. This acts on tensor product of functions by
\begin{equation}
	P(u\otimes v)=Xu\otimes Yv.
\end{equation}
Since they are both isomorphic to $\mU(\lG)\otimes\mU(\lG)$, the spaces of left invariant operators $\Diff^{G\times G}(G\times G)$ and $\biDiff^G(G)$ are isomorphic.

Let us make that isomorphism more explicit. First, an element $P\in\Diff(G\times G)$ provides the element $X\otimes Y\in\mU(\lG)\otimes\mU(\lG)$. The latter produces the bidifferential operator $A\in\biDiff(G)$ defined by $A(u\otimes v)(x)=X_xuX_xv$. Thus we define
\begin{equation}		\label{EqDefAlphaDiffbiDiff}
	\begin{aligned}
		\alpha\colon \Diff(G\times G)&\to \biDiff(G) \\
		\alpha(P)(u\otimes v)(x)&=X_xuY_xv.
	\end{aligned}
\end{equation}

\begin{lemma}
	The map \eqref{EqDefAlphaDiffbiDiff} is surjective.
\end{lemma}

\begin{proof}
	If $A\in\biDiff(G)$ is given by $A(u\otimes v)(x)=X_xuY_xv$, we have $\alpha(P_A)=A$ with $P_A\in\Diff(G\times G)$ given on $f\in C^{\infty}(G\times G)$ by
	\begin{equation}
		(P_Af)(x,y)=(XYf)(x,y)
	\end{equation}
	where $X$ acts on the first variable and $Y$ on the second.
\end{proof}

\begin{proposition}
	If we restrict to the left invariant operators, the map $\alpha\colon \Diff^{G\times G}(G\times G)\to \biDiff^G(G)$ is an isomorphism.
\end{proposition}

\begin{proof}
	First, remark that if $A\in\biDiff(G)$ is invariant, then an operator $P_A$ such that $\alpha(P_A)=A$ is also invariant because of proposition~\ref{PropbidiffUU} which states that
	\begin{equation}
		A(u\otimes v)(x)=A_{ab}(x)(\tilde X^a_x\otimes \tilde X^b_x)(u\otimes v)
	\end{equation}
	is invariant only when $A_{ab}$ are constant functions. Thus the map $\alpha$ is surjective from $\Diff^{G\times G}(G\times G)$ to $\biDiff^G(G)$.

	Now we have to prove that the map $\alpha$ is injective from the subspace of invariant operators. Let $P\in\Diff(G\times G)$ be an operator in the kernel of $\alpha$, so if $P$ reads
	\begin{equation}
		(Pf)(x,y)=c(x,y)(\tilde X\tilde Yf)(x,y),
	\end{equation}
	we have
	\begin{equation}
		(\alpha P)(u\otimes v)(x)=c(x,x)\tilde X_xu\tilde Y_xv=0
	\end{equation}
	for every $x\in G$ and every $u,v\in C^{\infty}(B)$. In that case, $c(x,x)$ has to vanish.

	What we proved is that the kernel of $\alpha$ is the set of operators with $c(x,x)=0$. In other words, the kernel is the set of operators $P$ such that
	\begin{equation}	\label{EqPfxxzero}
		(Pf)(x,x)=0
	\end{equation}
	for every function $f\in C^{\infty}(G\times G)$.

	Let us now suppose that $P$ is left invariant and compute $(Pg)(x,y)$ using the left invariance. We have
	\begin{equation}
		(Pg)(x,y)=dL_{x\times y}(Pg)(e,e)=P(L^*_{x\times y}g)(e,e)=0
	\end{equation}
	because of \eqref{EqPfxxzero} applied to the function $f=L^*_{x\times y}g$.
\end{proof}


%+++++++++++++++++++++++++++++++++++++++++++++++++++++++++++++++++++++++++++++++++++++++++++++++++++++++++++++++++++++++++++
\section{Pseudo differential operators}
%+++++++++++++++++++++++++++++++++++++++++++++++++++++++++++++++++++++++++++++++++++++++++++++++++++++++++++++++++++++++++++

\subsection{Composition}
%-----------------------

If $A$ and $B$ are pseudo-differential operators with symbols $a$ and $b$, the symbol of the composition is given by
\begin{equation}		\label{EqCompPSDSymb}
c(x,\xi)\sim \sum_{\alpha\in\eN^n} \frac{ (-i)^{| \alpha |} }{ \alpha! }\partial_{\xi}^{\alpha}a\partial_x^{\alpha}b.
\end{equation}


\subsection{Fourier transform and pseudo-differential operator}
%--------------------------------------------------------------

Let $\dpt{ \psi }{ M }{ E }$ be a local section and $P$ a differential operator. One can prove that
\begin{subequations} \label{eq:PspiFour}
\begin{align}
  (P\psi)(x)&=\frac{1}{ (2\pi)^{n/2} } \int e^{i\xi\cdot x}p(x,\xi)\hat\psi(\xi)\,d\xi\\
  \hat\psi(\xi)&= \frac{1}{ (2\pi)^{n/2} }\int e^{-i\xi\cdot y}\psi(y)\,dy
\end{align}
\end{subequations}
where $\xi\cdot x=\sum_j\xi_jx_j$ is the usual scalar product. So a pseudo-differential operator has to be written under the form
\begin{equation}
  (Au)(x)=(2\pi)^n\iint e^{i(x-y)\cdot \xi}a(x,\xi)u(y)\,dy\,d\xi
\end{equation}
and $a$ is the \defe{symbol}{symbol!of a pseudo-differential operator} of $A$.


Let us see a simple example: $P\psi=\partial_1\psi$. In this case, $p(x,\xi)=\xi_1$ and equations \eqref{eq:PspiFour} give
\[
  (P\psi)(x)=(2\pi)^{-n}\iint e^{i\xi\cdot(x-y)}\xi^1\psi(y)\,d\xi\,dy,
\]
 an integration by part gives
\[
  (2\pi)^{-n}\iint e^{i\xi\cdot(x-y)}(-\frac{ i }{ \xi_1 })\xi_1(\partial_{y_1}\psi)(y)\,d\xi\,dy.
\]
The integration over $\xi$ produce de Dirac distribution centred at $x-y$, i.e. a factor $(2\pi)^{-n}\delta(x-y)$ and the integral over $y$ leads to $(\partial_1)\psi(x)$.

To define pseudo-differential operator, we begin by only consider the trivial vector bundle over $\eR^n$ and thus functions $\dpt{ u }{ \eR^n }{ \eC^k }$.

\begin{definition}
	A \defe{pseudo-differential}{pseudo-differential operator}\index{operator!pseudo-differential} operator of order $m$ is an operator $P$ which can be written under the form
	\begin{equation}
		(Pu)(x)=(2\pi)^{-n/2}\int e^{i\xi\cdot x}p(x,\xi)\hat u(\xi)\,d\xi
	\end{equation}
	where
	\begin{equation}
		\hat u(\xi)=(2\pi)^{-n/2}\int e^{-i\xi\cdot y}u(y)\,dy
	\end{equation}
	and $p\in\mS^m$, the space defined at page \pageref{pg:defmS}.
\end{definition}

The set of all pseudo-differential operators is denoted by $\mP^m$\nomenclature{$\mP$}{Set of pseudo-differential operator}. If $p\in\mS^m$ is the symbol associated with $P\in\mP^m$, the \defe{principal symbol}{principal!symbol} of $P$ is the class $\sigma^P=[p]\in\mS^m/\mS^{m-1}$. Operators with associated symbol in $\mS^{\-\infty}$ are call \defe{smoothing operator}{smoothing operator} and we denote their space by $\mP^{-\infty}$.

A pseudo-differential operator $Q$ is a \defe{\wikipedia{en}{Parametrix}{\hypertarget{DefParametrix}{parametrix}}}{parametrix} for the pseudo-differential operator $P$ is the operators $PQ-\mtu$ and $QP-\mtu$ are compact operators.


\subsection{Example} \label{pg_exem_psdo}
%-------------------

The operator $[1-(2\pi)^{-2}\Delta]^{1/2}$ is the pseudo-differential operator on $\eR^N$ with symbol $p(\xi)=(1+\xi^2)^{1/2}$ where $\xi^2$ stands for $| \xi |^2$. Indeed if $P$ is the pseudo-differential operator with this symbol
\[
\begin{aligned}
  (Pu)(x)&=\int e^{2i\pi\xi\cdot x} (1+\xi^2)^{1/2}\hat u(\xi)\,d\xi\\
   	&=\int e^{2i\pi\xi\cdot x}\int e^{-2i\pi\xi\cdot y}[1-(2\pi)^{-2}\Delta]^{1/2}u(y)\,dy\,d\xi	&& \text{by \eqref{eq_umdpi_spi}}\\
	&=\int e^{2i\pi\xi\cdot(x-y)}\\
	&=\int\delta(x-y)[1-(2\pi)^{-2}\Delta]^{1/2}u(y)\,dy\\
	&=[1-(2\pi)^{-2}\Delta]^{1/2}u(x).
\end{aligned}
\]

\subsection{Trace operators}  \label{subsec_traceop}
%---------------------------

A pseudo-differential operator $P$ is expressed by its symbol $p$ and the formula
\[
  (Pu)(x)=\int  e^{2i\pi \xi\cdot x}p(x,\xi)\hat y(\xi)\,d\xi.
\]
Assume that $P$ can also be written under the form
\[
  (Pu)(x)=\int K(x,y)u(y)\,dy
\]
and let us equalize these two expressions for all $u$ (for example in $H^{1/2}(\eR^n)$. When it makes sense,
\begin{equation}
  K(x,y)=\int e^{2i\pi \xi\cdot (x-y)}p(x,\xi)\,d\xi.
\end{equation}
The \defe{trace}{trace!of an operator}\nomenclature[F]{$\tr P$}{Trace of the operator $P$} of $P$ is defined by
\begin{equation}
\tr P=\int K(x,y)\,dx
	=\iint p(x,\xi)\,dx\,d\xi.
\end{equation}
When the latter converges, one says that $P$ is a \emph{trace operator}.

\subsection{Asymptotic expansions}
%------------------------------------

We say that the pseudo-differential operator $A$ is \defe{classical}{classical!pseudo-differential operator} and we write $A\in\Psi^p(M)$\nomenclature[F]{$\Psi^n(M)$}{Space of classical pseudo-differential operators on $M$} if the symbols accepts the asymptotic expansion
\[
  a(x,\xi)\sim\sum_{j=0}^{\infty}a_{p-j}(x,\xi)
\]
where each of the $a_r$ is $r$-homogeneous with respect to $\xi$: $a_r(x,t\xi)=t^ra_r(x,\xi)$. Although the whole definitions are made in local coordinates, one can show that the \defe{principal symbol}{principal!symbol} is a globally defined function over the cotangent bundle $T^*M$. The operator is elliptic if $a_n(x,\xi)$ is invertible for all $\xi\neq 0$.

Notice that an element of $\Psi^p$ has an asymptotic expansion which begins with order $p$, so that the spaces $\Psi^d$ fulfil $\Psi^{p-1}\subset\Psi^p$. Now the \defe{algebra of classical pseudo-differential operators}{algebra!of classical $\Psi$DO} is the quotient
\begin{equation}		\label{EqDefmPalgOpeClass}
\mP=\Psi^{-\infty}/\Psi^{\infty}.
\end{equation}

%+++++++++++++++++++++++++++++++++++++++++++++++++++++++++++++++++++++++++++++++++++++++++++++++++++++++++++++++++++++++++++
\section{Dirac and Laplace type operators}

Let $(M,\partial M)$ be a manifold with boundary and $V$, a vector bundle over $M$. A second order differential operator $\Delta$ on $V$ is of \defe{Laplace type}{laplace type operator} if its principal symbol is the metric tensor, i.e. if it has a local expression of the form
\begin{equation}
\Delta = -(g^{ij}\partial^2_{ij}+A^k\partial_k+B)
\end{equation}
where $a\in \Gamma(TM\otimes\End(V))$ and $B\in\Gamma(\End(V))$. A first order operator $D$ on $\Gamma(V)$ is of \defe{Dirac type}{dirac type operator} if its principal symbol defines a Clifford module over $V$. Such an operator has a local expression
\begin{equation}			\label{EqFormGeneDirac}
	D=\gamma^i\partial_i-r
\end{equation}
with $\gamma\in\Gamma(TM\otimes\End(V))$ and $r\in\Gamma(\End(V))$. The condition is that
\begin{equation}
\gamma^i\gamma^j+\gamma^j\gamma^i=2g^{ij}\id|_V.
\end{equation}

\begin{proposition}
The operator $D$ is of Dirac type if and only if $D^2$ is of Laplace type.
\end{proposition}

\begin{proof}
If $D$ is of Dirac type, up to lower order terms we have
\[
  D^2=\gamma^i\gamma^j\partial^2_{ij}+\ldots=\frac{ 1 }{2}(\gamma^i\gamma^j+\gamma^j\gamma^i)\partial^2_{ij}+\ldots=-g^{ij}\partial^2_{ij}+\ldots
\]
\end{proof}
It is however not true that every Laplace type operator is the square of a Dirac operator.

\section{Wodzicki residue}
%+++++++++++++++++++++++++

Let $M$ be a $n$-dimensional manifold. In that case the term of order $(-n)$ in the asymptotic expansion
\[
  a(x,\xi)\sim\sum_{j=0}^{\infty}a_{n-j}(x,\xi)
\]
of the principal symbol of the pseudo-differential operator $A$ is specially important. Let $\mU\subset M$ be a coordinate open set on which the cotangent bundle is trivial. One can see $a_{-n}$ as a smooth function on $T^*\mU\invtible$ (the set of sections from which we remove the zero section).

We define
\[
  \alpha(x,\xi)=a_{-n}(x,\xi)d\xi_1\wedge\cdots\wedge d\xi_n\wedge dx^1\wedge\cdots\wedge dx^n.
\]
That form is invariant under dilatations $\xi\to t\xi$ because $a_{-n}(x,t\xi)=t^{-n}a_{-n}(x,\xi)$. Let us consider $R=\sum_j \xi_j\partial_{\xi_j}$, the dilatation generator.

From formula $\mL_R=\iota_Rd+d\iota_R$, we have $d\iota_R\alpha=\mL_R\alpha=0$ where $\mL$ denotes the Lie derivative \eqref{liesurforme}. If we write $dx=d^nx=dx^1\wedge\cdots\wedge dx^n$, we have
\[
  (\iota_R\alpha)(x,\xi)=a_{-n}(x,\xi)\sigma_n\wedge dx
\]
where $\sigma_{\xi}=\sum (-1)^{j-1}\xi_j\,d\xi_1\wedge\cdots\widehat{d\xi_j}\wedge\cdots\wedge d\xi_n$. For each particular $x\in M$, $\sigma_{\xi}$ is a volume form on the unit sphere $| \xi |=1$ on $T^*_xM$. We can integrate $\iota_R\alpha$ on such a sphere:
\[
  \int_{| \xi |=1}a_{-n}(x,\xi)\sigma_{\xi}.
\]
One can prove that under the coordinate change $w\to y=\phi(x)$, $\xi\to\eta=\phi'(x)^t\xi$, we have $a_{-n}(x,\xi)\to \tilde a_{-n}(y,\eta)$ and
\[
  \int_{| \eta |=1}\tilde a_{-n}(y,\eta)\sigma_{\eta}=| \det\phi'(x) |\int_{| \xi |=1}a_{-n}(x,\xi)\sigma_{\xi}.
\]
That shows that the quantity
\begin{equation}
\left( \int_{| \xi |=1}a_{-n}(x,\xi)\sigma_{\xi} \right)dx
\end{equation}
is a $1$-density over $M$ that we call $\ResW(x)A$. The \defe{Wodzicki residue}{wodzicki residue} is the integral of that over $M$:
\begin{equation}
\ResW A=\int_M\Res_W(x)A=\int_{S^*M}\iota_R\alpha=\int_{S^*M}a_{-n}(x,\xi)\sigma_{\xi}dx
\end{equation}
where $S^*M=\{ (x,\xi)\in T^*M\tq | \xi |=1 \}$. Notice that the latter integral can diverge.

\begin{proposition}
The operation $\ResW$ is a trace on the algebra of classical pseudo-differential operators. That means that $\Res_W[A,B]=0$ whenever $A$, $B$ are classical pseudo-differential operators.
\end{proposition}

\begin{proof}
If $A\in\Psi^d(M)$ and $B\in\Psi^r(M)$, then the product $AB$ belongs to the space $\Psi^{d+r}(M)$ and the commutator $P=[A,B]$,  seen as in $\Psi^{-\infty}$, has symbol (see equation \eqref{EqCompPSDSymb})
\[
  p(x,y)\sim\sum_{| \alpha |>0}\frac{ (-i)^{| \alpha |} }{ \alpha! }\big( \partial^{\alpha}_{\xi}a\partial_x^{\alpha}b-\partial^{\alpha}_{\xi}b\partial_x^{\alpha}a \big).
\]
We can suppose that $A$ and $B$ have compact support because the aim is to integrate them with a partition of unity.
\end{proof}

\begin{probleme}
	Unfinished proof
\end{probleme}

\begin{proposition}
The operation $\ResW$ is the unique trace on the algebra $\mP$ defined in equation~\ref{EqDefmPalgOpeClass}.
\end{proposition}
\begin{proof}
No proof.
\end{proof}

\section{Interpolation theory}
%------------------------------

\begin{probleme}
Il faut citer encore V\'a rrily et le second bouquin de Connes et le truc de Landi comme sources
\end{probleme}

Here is a short review of what is given in the book \cite{ConnesNCG}. Let $B_0$ and $B_1$ be two Banach algebras continuously embedded in a topological vector space. We begin to define
\[
  K(\lambda,x)=\inf\{ \| x_{0} \|_{B_0}+\lambda^{-1}\| x_1 \|_{B_1}\tq x_0+x_1=x,\, x_0\in B_0,\,x_1\in B_1 \}
\]
for all $x\in B_0+B_1$ and $\lambda\in]0,\infty[$. For a fixed $x$, we consider the function
\[
  f(\lambda)=\lambda^{\alpha} K(\lambda,x),
\]
and we define the norm of $x\in_{(\alpha,\beta)}$ by
\begin{equation}
	\| x \|_{\alpha,\beta}=\left( \int_{\eR^+_0}f(\lambda)^q \frac{ d\lambda }{ \lambda } \right)^{1/q}
\end{equation}
where $q=1/\beta$.

Now we consider the special case in with $B_0$ is the ideal of compact operators on the Hilbert space $\hH$ and $B_1=\oL^1(\hH)$, both seen as subspaces of $\oL^1(\hH)$.

\begin{proposition}
If $\alpha=1/p$ and $\beta=1/q$, and $q<0$, a compact operator $T$ belongs to $\oL^{(p,q)}(\hH)$ if and only if
\[
  \sum_{N=1}^{\infty}N^{\alpha-1}q^{-1}\sigma_N(T)^q <\infty
\]
where $\sigma_N(T)=\sup\{ \| T|_E \|_1\tq \dim E=N \}$. When $q=\infty$, we have $T\in\oL^{(p,\infty)}$ if and only if $N^{\alpha-1}\sigma_N(T)$ is a bounded sequence.
\end{proposition}



\begin{proposition}
Each space $\oL^{(p,q)}$ with $1<p<\infty$ and $1\leq q\leq\infty$ is a two-sided ideal in $\oL(\hH)$ and when $p_1<p_2$, we have
\[
\oL^{(p_1,q_1)}\subset\oL^{(p_1,q_1)}.
\]
When $p_1=p_2$, this inclusion holds if $q_1\leq q_2$.
\end{proposition}

Notice that the fact to say that the sequence $N^{(\alpha-1)}\sigma_N(t)$ is bounded is the same as to say that $\sigma_N(T)=O(N^{1-\alpha})$. This in turns is noting else than $\mu_n(T)=O(n^{-\alpha})$ and the norm that we put on $\oL^{(p,\infty)}$ is
\[
  \| T \|_{p,\infty}=\sup_{N\geq 1}\frac{1}{ N^{1-\alpha} }\sigma_N(T).
\]

\section{Trace class operators}
%++++++++++++++++++++++++++++++

\subsection{Trace}
%-----------------

We say that a bounded linear operator $A$ on the Hilbert space $\hH$ is \defe{trace class}{trace!class operator} if for a certain basis $\{ e_k \}$, the sum
\begin{equation}	\label{EqDeftRCLAss}
\sum_k\langle A^*Ae_k, e_k\rangle^{1/2}
\end{equation}
converges. In that case the sum $\sum_k\langle Ae_k, e_k\rangle $ converges absolutely (because each term in the sum \eqref{EqDeftRCLAss} is positive) and is thus independent with respect of the choice of the basis. That number is called the \defe{trace}{trace!of an operator}:
\begin{equation}\nomenclature{$\tr(A)$}{Trace of the operator $A$}
\tr(A)=\sum_k\langle Ae_k, e_k\rangle.
\end{equation}

In the space $\oL^1$ of trace class operators, we have that\nomenclature[F]{$\| T \|_1$}{Another norm for operators in $\oL^1$.}
\begin{equation}
\| T \|_1=\tr| T |
\end{equation}
is a norm which is not equal to the operator norm $\| T \|=\mu_0(T)$. More generally we have the following lemma.
\begin{lemma}
We have
\begin{equation}
    \sigma_n(T)=\sup\{ \| TP_n \|\text{ such that }P_n \text{ is a projector of rank } n \},
\end{equation}
and each $\sigma_n$ is a norm on the space $\oK(\hH)$\nomenclature[F]{$\oK(\hH)$}{The space of compact operators over the Hilbert space $\hH$} of compact operators over the Hilbert space $\hH$.
\end{lemma}

From formula \eqref{Defmuncaharacinfn}, the sequence of $\mu_k(T)$ is decreasing, so that we get the inequalities
\[
  \sigma_n(T)\leq n\mu_0(T)=n\| T \|.
\]

\begin{lemma}
We have the formula
\begin{equation}	\label{EqsigmainfRST}
\sigma_n(T)=\inf\{ \| R \|_1+n\| S \|\text{ such that }R\in \oK,S\in\oK,R+S=T \}
\end{equation}
for every $T\in\oK(\hH)$.
\end{lemma}

\begin{proof}
It $T=R+S$, we have $\sigma_n(T)\leq\sigma_n(R)+\sigma_n(S)\leq \| R \|_1+n\| S \|$. Now we have to prove that $\sigma_n(T)$ actually reaches the infimum.  We can suppose that $T$ is positive; if not, we can change every signs in \eqref{EqsigmainfRST}, and nothing is changed. So lemma~\ref{Lemmulamequ} is applicable. Let $P_n$ be the rank $n$ projector over the space spanned by the eigenvectors corresponding to the eigenvalues $\mu_0,\cdots,\mu_{n-1}$ of $T$. Then we consider $R=(T-\mu_n)P_n$ and $S=\mu_nP_n+T(1-P_n)$. Then we have $\| R \|_1=\sum_{k<n}(\mu_k-\mu_n)=\sigma_n(T)-n\mu_n$ and $\| S \|=\mu_n$. We conclude that
\[
  \sigma_n(T)=\| R \|_1+n\mu_n=\| R \|_1+n\| S \|,
\]
which concludes the proof.
\end{proof}
That formula only holds for $n\in\eN$, but we can \emph{define}
\begin{equation}
\sigma_{\lambda}=\inf\{ \| R \|_1+\lambda\| S \|\text{ such that }R \}
\end{equation}
for every $\lambda\in\eR^{+}$.

\begin{proposition}
If $0\leq\lambda\leq 1$, we have
\[
  \sigma_{\lambda}=\lambda\| T \|,
\]
and more generally if $\lambda=n+t$ with $n\in\eN$, $0\leq t\leq 1$ we have
\begin{equation}	\label{Eqsigunmoinstn}
\sigma_{\lambda}(T)=(1-T)\sigma_n(T)+t\sigma_{n+1}(T).
\end{equation}
Moreover we have that the function $\lambda\mapsto\sigma_{\lambda}(T)$ is an increasing piecewise linear concave function.
\end{proposition}
\begin{proof}
No proof.
\end{proof}
The inequality \eqref{Eqsigunmoinstn} makes that $\sigma_{\lambda}$ is a norm that satisfies in particular the triangular inequality.

\begin{proposition}
The inequality \eqref{Eqsigunmoinstn} can be reinforced into
\begin{equation}
\sigma_{\lambda}(A)+\sigma_{\mu}(B)\leq \sigma_{\lambda+\mu}(A+B)
\end{equation}
when $A$ and $B$ are positive operators. Combining with the triangular inequality, we find
\begin{equation}
\sigma_{\lambda}(A+B)\leq\sigma_{\lambda}(A)+\sigma_{\lambda}(B)\leq_\sigma{2\lambda}(A+B)
\end{equation}
under the same assumptions.
\end{proposition}

%%%%%%%%%%%%%%%%%%%%%%%%%%
%
   \section{Dixmier traces}
%
%%%%%%%%%%%%%%%%%%%%%%%%

We follow the approach given in \cite{Landi,itoNCG_Varilly}.

\subsection{Banach limit}
%------------------------

Let $l_{\infty}$\nomenclature[F]{$l_{\infty}$}{Space of complex-valued bounded sequences} be the Banach space of complex-valued bounded sequences. A \defe{Banach limit}{Banach!limit} is a linear functional $\phi\colon l_{\infty}\to \eR$ such that for every real sequences $x$ and $y$, we have
\begin{itemize}
\item $\phi(\lambda x+\mu y)=\lambda\phi(x)+\mu\lambda(y)$
\item if $x\geq 0$, then $\phi(x)\geq 0$,
\item if $S$ is the \defe{shift operator}{shift operator} $(Sx)_i=x_{i+1}$, then $\phi(x)=\phi(Sx)$,
\item if $x$ converges, then $\phi(x)=\lim x$.
\end{itemize}
Such a functional is not unique: if $\phi$ and $\varphi$ are two such Banach limits, one can find a sequence $x$ such that $\phi(x)\neq\varphi(x)$. Such an example has to be non-convergent.

\subsection{Infinitesimal operator}
%----------------------------------

Proposition~\ref{prop_comp_ini} leads us to think to compact operators as infinitesimals because it is almost zero on th major part of the space. Here is the precise definition.

\begin{definition}
Let $\alpha\in\eR^+$. An \defe{infinitesimal}{infinitesimal operator} of order $\alpha$ is a compact operator $T$ such that there exists a $C\leq\infty$ for which
\[
  \mu_n(T)\leq Cn^{-\alpha}.
\]
for all $n\geq 1$.
\end{definition}

When $T_1$ and $T_2$ are two compact operators, one can prove that
\[
  \mu_{n+m}(T_1T_2)\leq\mu_n(T_1)\mu_m(T_2)n,
\]
hence if $T_j$ is of order $\alpha_j$, $T_1T_2$ is of order $\leq\alpha_1+\alpha_2$. Moreover the infinitesimals of order $\alpha$ form a two-sided ideal (non closed) in $\opB(\hH)$ because for all $T\in\opK(\hH)$ and $B\in\opB(\hH)$, we have
\begin{equation}
\begin{split}
\mu_n(TB)&\leq\| B \|\mu_n(T)\\
\mu_n(BT)&\leq\| B \|\mu_n(T).
\end{split}
\end{equation}
For a proof, see bibliography of \cite{Landi}.

Remark that for an infinitesimal of order $1$, the characteristic values are bounded by $\mu_n(T)\leq 1/n$, so that there are no reason for such a $T$ be belongs to $\oL^1$, but the divergence of $\tr(T)$ is at most logarithmic:
\[
  \sum_{n=1}^{N-1}\mu_n(T)\leq C\ln N.
\]

\subsection{Dixmier trace}
%-------------------------

We want to build a trace which is non zero on infinitesimals of order $1$, but which vanishes on infinitesimals of lager order. The usual trace is defined, for $T\in\mathscr{L}^{1}$,  by\index{trace}
\[
  \tr T:=\sum_n\langle T\xi_n|\xi_n\rangle
\]
and is independent of the chosen orthonormal basis $\{ \xi_n \}$ of $\hH$. When $T$ is positive and compact, we define the trace by
\[
  \tr T=\sum_{n=1}^{\infty}\mu_n(T).
\]
The problem is that infinitesimals of order $1$ are not in general in $\mathscr{L}^{1}$ because on these operators, we do not have a better control that $\mu_n(T)\leq \frac{ C }{ n }$. Hence the sum can diverge. Worse: the space $\mathscr{L}^{1}$ contains infinitesimals of order lager than $1$. However we know that in the case of infinitesimals positive operators of order $1$ is at most logarithmic:
\[
  \sum_{n=0}^{N-1}\mu_n(T)\leq C\ln N.
\]
We are going to find a way to extract the coefficient of the logarithmic divergence. We denote by $\mathscr{L}^{(1,\infty)}$ the ideal of compact operators which are infinitesimals of order $1$. If $T\in\mathscr{L}^{(1,\infty)}$, we want to define the trace by
\[
  \lim_{N\to\infty}\frac{1}{ \ln N }\sum_{n=0}^{N-1}\mu_n(T).
\]
This definition has two main problems: it is not specially linear in $T$ and does not converge in general. Let the sums
\[
  \sigma_N(T)=\sum_{n=1}^{N-1}\mu_n(T)\quad{ and }\quad\gamma_N(T)=\frac{ \sigma_N(T) }{ \ln N }.
\]
One can prove that
\begin{subequations}
\begin{align}
\sigma_N(T_1+T_2)&\leq\sigma_N(T_1)+\sigma_N(T_2)\\
\sigma_{2N}(T_1+T_2)&\geq\sigma_N(T_1)+\sigma_N(T_2);
\end{align}
\end{subequations}
the second relation only holds with $T_1,T_2\geq0$. Therefore, when $T_1,T_2>0$, we have
\begin{equation} \label{eq_gammaNleq}
\begin{split}
  \gamma_N(T_1+T_2)&\leq \gamma_N(T_1)+\gamma_N(T_2)\\
		&\leq \frac{ \sigma_{2N}(T_1+T_2) }{ \ln N }\\
		&=\frac{ \gamma_{2N}(T_1+T_2) }{ \ln N }\ln 2N\\
		&\leq \gamma_{2N}(T_1+T_2)\big( 1+\frac{ \ln 2 }{ \ln N } \big)
\end{split}
\end{equation}
because for suitably large $N$,
\[
  1+\frac{ \ln 2 }{ \ln N }=\frac{ \ln N+\ln 2 }{ \ln N }\leq\frac{ \ln 2N }{ \ln N }=\ln N.
\]
If the sequence $\gamma_N$ converges, then it is linear because when $N\to\infty$, we have $1+\ln 2/\ln N\to1$; hence equalities
\[
  \gamma_N(T_1+T_2)\leq\gamma_N(T_2)+\gamma_N(T_2)\leq\gamma_{2N}(T_2+T_2)\big( 1+\frac{ \ln 2 }{ \ln N } \big).
\]
fix the limit of $\gamma_N(T_1)+\gamma_N(T_2)$ on the one of $\gamma_{2N}(T_1+T_2)$. This however does not resolve the problem of convergence of $\gamma_n$, even when it is bounded.

The trick is to not take the usual limit, but to define a linear form $\lim_{\omega}$ on the space $l^{\infty}(\eN)$ of bounded sequences and to impose to $\lim_{\omega}$ to fulfil certain conditions.

From remark on page \pageref{pg_char_inv_U}, we know that the values $\mu_n(T)$ are unitary invariant, hence the sequence $(\gamma_N)$ is also unitary invariant. This leads us to search for an unitary invariant form $\lim_{\omega}$. The following proposition allows us to only define $\lim_{\omega}$ in the positive part of $\mathscr{L}^{(1,\infty)}$.

\begin{proposition}
The space $\mathscr{L}^{(1,\infty)}$ is generated by its positive part.
\end{proposition}
\begin{proof}
No proof.
\end{proof}
Here are the conditions we impose to $\lim_{\omega}\colon l^{\infty}(\eN)\to \eN$:
\begin{enumerate}
\item it is a linear form,
\item $\lim_{\omega}(\gamma_N)\geq 0$ is $\gamma_N\geq0$,
\item $\lim_{\omega}(\gamma_N)=\lim\gamma_N$ if the usual limit exists,
\item\label{limomiii}$\lim_{\omega}(\gamma_1,\gamma_1,\gamma_2,\gamma_2,\gamma_3,\gamma_3)=\lim_{\omega}(\gamma_N)$,
\item\label{limomiv} $\lim_{\omega}(\gamma_{2N})=\lim_{\omega}(\gamma_N)$.
\end{enumerate}
The condition~\ref{limomiv} is the \emph{scale invariance}\index{scale invariance}; this property is equivalent to the property~\ref{limomiv}. Dixmier has found a lot of such form. For each of them, one has a trace
\begin{equation}
   \tr_{\omega}(T)=\lim_{\omega}\frac{1}{ \ln N }\sum_{n=0}^{N-1}\mu_n(T)
\end{equation}
for positive $T\in\mathscr{L}^{(1,\infty)}$. When $T_1$ and $T_2$ are positive, we have linearity:
\[
  \tr_{\omega}(T_1+T_2)=\tr_{\omega}(T_1)+\tr_{\omega}(T_2).
\]
Since $\mathscr{L}^{(1,\infty)}$ is generated by its positive part, the form $\tr_{\omega}$ ---which is initially only defined for positive operators $T$--- extends to the whole $\mathscr{L}^{(1,\infty)}$ with properties
\begin{enumerate}
\item $\tr_{\omega}(T)\geq0$ if $T\leq0$,
\item $\tr_{\omega}(\lambda_1T_2+\lambda_2 T_2)=\lambda_1\tr_{\omega}(T_1)+\lambda_2\tr_{\omega}(T_2)$,
\item $\tr_{\omega}(BT)=\tr_{\omega}(TB)$ for all $B\in\opB(\hH)$,
\item\label{item_tromTiv} $\tr_{\omega}(T)=0$ if $T$ is an infinitesimal of order larger than $1$.
\end{enumerate}
For a proof, see \cite{Landi}. For~\ref{item_tromTiv}, remark that the space of infinitesimals of order larger than $1$ form a two-sided ideal whose elements fulfil $n\mu_n(T)\to 0$. Then the sequence $(\gamma_B)$ converges to zero too and the Dixmier trace vanishes.


\subsection{Dixmier: second}
%---------------------------

The set of \defe{infinitesimals of order $1$}{infinitesimal!of order $1$} is the normed ideal
\begin{equation}
\oL^{1+}=\{ T\in\oK\tq \| T \|_{1+}<\infty \}
\end{equation}
where the norm $\| T \|_{1+}$\nomenclature[F]{$\| T \|_{1+}$}{Norm for the order one infinitesimals} is defined by
\begin{equation}
\| T \|_{1+}=\sup_{\lambda\geq a}\frac{ \sigma_{\lambda}(T) }{ \ln\lambda }.
\end{equation}
That idea include the trace class operators. We define\nomenclature[F]{$\oL^p$}{A functional space around the Dixmier trace}
\begin{equation}
\oL^p=\{ T\in\oK\tq \tr| T |^p<\infty \}.
\end{equation}

\begin{proposition}
On the space $\oL^p$, we have
\[
  \sigma_{\lambda}=O(\lambda^{1-1/p})
\]
and $\oL^{1+}\subset\oL^p$ when $p>1$.
\end{proposition}
Notice that, when $T\in\oL^{1+}$, the function $\lambda\mapsto\sigma_{\lambda}(T)/\ln\lambda$ is bounded and continuous on the interval $[e,\infty[$. It belongs thus to the $C^*$-algebra $C_b\big( [e,\infty[\big)$. So we can use the \defe{Ces\` aro means}{ces\` aro mean}:
\begin{equation}	\label{EqCearomaen}
\tau_{\lambda}(T)=\frac{1}{ \ln\lambda }\int_e^{\lambda}\frac{ \sigma_u(T) }{ \ln u }\frac{ du }{ u },
\end{equation}
and the function $\lambda\mapsto\tau_{\lambda}(T)$ still belongs to $C_B\big( [e,\infty[ \big)$ with $\| T \|_{1+}$ as upper bound.

\begin{proposition}
The double inequality
\[
  0\leq \tau_{\lambda}(A)+\tau_{\lambda}(B)-\tau_{\lambda}(A+B)\leq\big( \| A \|_{1+}+\| B \|_{1+} \big)\ln 2\frac{ \ln\ln\lambda }{ \ln\lambda }.
\]
holds for $A$, $B\in\oL^{1+}$.
\end{proposition}
\begin{proof}
No proof.
\end{proof}
That proves that $\tau\lambda$ becomes additive when $\lambda$ goes to infinity. We can work on that in order to make it additive. First we consider
\[
  \mB=C_b\big( [e,\infty[ \big)/C_0\big( [e,\infty[ \big),
\]
and we consider $[\tau(A)]$, the class of $\tau(A)$ (i.e. the function $\lambda\mapsto\tau_{\lambda}(A)$) with respect to that quotient.

\begin{proposition}
The map $[\tau]$ is additive, positive and homogeneous from the positive cone in $\oL^{1+}$ to $\mB$. Moreover
\[
  [\tau(UAU^{-1})]=[\tau(A)]
\]
for every unitary $U$.
\end{proposition}
That makes that $[\tau]$ extends to a linear map $[\tau]\colon \oL^{1+}\to \mB$ such that $[\tau](ST)=[\tau](TS)$ for all $T\in\oL^{1+}$ and $S$.

Now if $\omega\colon \mB\to \eC$ is any state, we define the \defe{Dixmier trace}{Dixmier trace} as
\begin{equation}
\tr_{\omega}(T)=\omega\big( [\tau](T) \big).
\end{equation}
That definition has a problem: the $C^*$-algebra $\mB$ being non separable, one cannot exhibit a state, so that the formula is in practice unusable.

\subsection{Noncommutative integral}
%-----------------------------------

Let us consider $f\in C_b\big( [e,\infty \big)$; the limit $\lim_{\lambda\to\infty}f(\lambda)$ exists if and only if $\omega(f)$ does not depend on $\omega$ because of the quotient by $C_0\big( [e,\infty[ \big)$ which makes that $\omega$ can only depend on the behaviour near infinity. We say that the operator $T\in\oL^{1+}$ is \defe{measurable}{measurable operator} if the function $\lambda\mapsto\tau_{\lambda}(T)$ converges when $\lambda\to\infty$. In that case, $\tr_{\omega}(T)$ equals that limit, and we denote by $\dashint T$ the common value of the Dixmier traces:\nomenclature[F]{$\dashint T$}{The noncommutative integral}
\begin{equation}
\dashint T=\lim_{\lambda\to\infty}\tau_{\lambda}(T)
\end{equation}
if the limit exists. That is the \defe{noncommutative integral}{noncommutative!integral} of $T$.

\begin{proposition}
    It $T$ is a compact operator and if $\sigma_n(T)/\ln n$ converges when $n\to\infty$, then the limit $\lim_{\lambda\to\infty}\tau_{\lambda}(T)$ exists and $T$ is a measurable operator in $\oL^{1+}$. 
\end{proposition}

\begin{proof}
    Indeed in that case the quantity $\sigma_u(T)/\ln u$ becomes constant when $u$ is large, so that we are left in the definition \eqref{EqCearomaen} with
\[
  \lim_{\lambda\to\infty}\frac{C}{ \ln\lambda }\int_e^{\lambda}\frac{ du }{ u }=C\lim_{\lambda\to\infty}\frac{ \ln\lambda-1 }{ \ln\lambda }.
\]
which exists.
\end{proof}

\subsection{Residues}
%-------------------

Let $M$ be a compact Riemannian spin manifold of dimension $n$. Let $T$ be a pseudo-differential operator of order $-n$ acting on the sections of a complex vector bundle $E\to M$. Its \defe{residue}{residue} is defined by
\begin{equation}
\ResW T=\frac{1}{ n(2\pi)^n }\int_{S^*M}\tr_E\sigma_{-n}(T)\,d\mu
\end{equation}
where $\sigma_{-n}\colon T^*M\to \End E$ is the principal symbol of $T$ (is as a homogeneous function of degree $-n$) and the integral is taken on the cosphere\index{cosphere}
\[
  S^*M=\{ (x,\xi)\in T^*M\tq \| \xi \|=1 \}
\]
with the measure $d\mu=dx\,d\xi$.


\chapter{Partial derivative equation}
\input{171_pde}

\chapter{Finite elements}
% This is part of (almost) Everything I know in mathematics
% Copyright (c) 2016
%   Laurent Claessens
% See the file fdl-1.3.txt for copying conditions.

%+++++++++++++++++++++++++++++++++++++++++++++++++++++++++++++++++++++++++++++++++++++++++++++++++++++++++++++++++++++++++++
\section{Lax-Milgram theorem}
%+++++++++++++++++++++++++++++++++++++++++++++++++++++++++++++++++++++++++++++++++++++++++++++++++++++++++++++++++++++++++++

\begin{definition}  \label{DEFooGFTZooUQfUdY}
    A bilinear form \( a\colon V\times V\to \eR\) is \defe{elliptic}{bilinear form!elliptic} or \defe{coercive}{bilinear form!coercive} is there exists a \( \alpha>0\) such that \( a(u,u)\geq \alpha\| u \|^2\) for every \( u\in V\).
\end{definition}

\begin{theorem}[Lax-Milgram\cite{ooRIEKooXIQYhE}]       \label{THOooFDJYooCSNnuv}
    Let \( V\) be ah Hilbert space and
    \begin{enumerate}
        \item
            a linear and bounded map \( L\colon V\to \eR\); we write \( \| L \|=C\),
        \item
            a bilinear map \( a\colon V\times V\to \eR\) is continuous; we write \( M\) a constant such that \( | a(u,v) |\leq M\| u \|\| v \|\) for all \( u,v\in V\),
        \item
            the bilinear form \( a\) is elliptic\footnote{Définition~\ref{DEFooGFTZooUQfUdY}} and we write \( \alpha\) a strictly positive constant such that \( a(u,u)\geq \alpha\| u \|^2\).
    \end{enumerate}
    Then the problem of finding \( u\in V\) such that
    \begin{equation}
        a(u,v)=L(v)
    \end{equation}
    for every \( v\in V\) has one and only one solution \( u\in V\). Moreover this solution satisfies
    \begin{equation}
        \| u \|\leq \frac{ M }{ \alpha }C.
    \end{equation}
\end{theorem}
The map \( L\) is linear and bounded; it is continuous by proposition~\ref{PROPooQZYVooYJVlBd}. The existence of \( M\) is due to the fact that \( a\) is bilinear on \( V\) and in particular linear (and continuous, then bounded) on \( V\times V\). But this is not quite obvious from the definition \eqref{DefFAJgTCE}. It is shown in \cite{ooCUHNooNYIeGt} that putting on \( V\times V\) the product topology\footnote{Définition~\ref{DefIINHooAAjTdY}.} that a sesquilinear map is continuous if and only if there exist such a constant. And since the topology of the product norm is the product topology (lemma~\ref{DefFAJgTCE}), we are safe.

%+++++++++++++++++++++++++++++++++++++++++++++++++++++++++++++++++++++++++++++++++++++++++++++++++++++++++++++++++++++++++++
\section{Variational formulation (not too rigorous)}
%+++++++++++++++++++++++++++++++++++++++++++++++++++++++++++++++++++++++++++++++++++++++++++++++++++++++++++++++++++++++++++

As mentioned in the title, we are not going to deal with existence of the derivative and the integrals that we will write down.

Let the partial derivative equation
\begin{subequations}        \label{EQooZAISooSylvFH}
        \begin{numcases}{}
            -\Delta u=f\\
            u|_{\partial \Omega}=0
        \end{numcases}
    \end{subequations}
where \( \Delta u=\sum_{j=1}^n\frac{ \partial^2 v  }{ \partial x_j }\) on the open bounded part \( \Omega\) of \( \eR^n\).

We are searching the solutions in a vector space
\begin{equation}
    V=\{ v\colon \Omega\to \eR\tq v|_{\partial \Omega}=0 \}.
\end{equation}
Our aim is to found a bilinear form \( a\colon V\times V\to \eR\) and a linear map \( L\colon V\to \eR\) such that the solutions of the original problem \eqref{EQooZAISooSylvFH} are solutions of the problem
\begin{subequations}
    \begin{numcases}{}
        u\in V\\
        a(u,v)=L(v)\,\forall v\in V
    \end{numcases}
\end{subequations}
The choice of \( V\), \( a\) and \( L\) is a \defe{variational formulation}{variational!formulation} of the differential equation\footnote{I said ``not too rigorous'' in the title, so please don't ask yourself now what space $V$ can be.}.

In order to have a variational formulation of the equation \eqref{EQooZAISooSylvFH} we multiply \( -\Delta u=f\) by a test function \( v\in V\) and we integrate over \( \Omega\):
\begin{equation}
    -\int_{\Omega}\Delta u\,v=\int_{\Omega}fv.
\end{equation}
Now if we set
\begin{subequations}
    \begin{align}
        a(u,v)&=-\int_{\Omega}(\Delta u)v   \label{SUBEQooKUNUooOtKVaP}\\
        L(v)&=\int_{\Omega}fv
    \end{align}
\end{subequations}
we have a variational formulation of our problem. A solution of a variational formulation is a \defe{weak solution}{solution!weak} of the partial derivative equation.

Does the form \eqref{SUBEQooKUNUooOtKVaP} check the hypothesis of the Lax-Milgram theorem~\ref{THOooFDJYooCSNnuv}? Obviously not because we did not defined the space \( V\), so nothing has any sense here. But we can say more: the bilinear form \( a\) is not \emph{obviously} positive. As we will see it is positive on \( V\) because of the boundary condition. We want to write is slightly differently in order to, taking into account the boundary condition, have a bilinear form that is for sure positive.

Using the integration by part of formula \eqref{EQooJLDTooIMtxEX} taking into account the fact that the boundary term vanishes we have
\begin{equation}
    \int_{\Omega}(\Delta u)v=-\int_{\Omega}\nabla u\cdot \nabla v,
\end{equation}
so that we can as well consider the variational problem
\begin{subequations}
    \begin{align}
        a(u,v)&=-\int_{\Omega}\nabla u\cdot \nabla v   \label{SUBEQooLFDKooTDKiDA}\\
        L(v)&=\int_{\Omega}fv
    \end{align}
\end{subequations}
In this case, the form \( a\) is more clearly positive defined:
\begin{equation}
    a(u,u)=\int_{\Omega}| \nabla u |^2\geq 0.
\end{equation}

Notons que cette formule pour \( a\) est symétrique et que nous n'avons pas encore démontré quoi que ce soit pour les hypothèses du théorème de Lax-Milgram. Nous espérons seulement que la forme bilinéaire \eqref{SUBEQooLFDKooTDKiDA} ait de meilleures propriétés que \eqref{SUBEQooKUNUooOtKVaP}.

The result of this sections is the following.
\begin{proposition}[Not too rigorous]
    A function \( u\in V\) is solution of the variational problem if and only if it is solution of the Poisson equation.
\end{proposition}

\begin{proof}
    The fact that the solution of the Poisson equation (including the boundary conditions) are solutions of the variational problem is what we just did.

    In the other sense we recall the equation:
    \begin{subequations}
        \begin{numcases}{}
            -\Delta u=f\\
            u|_{\partial \Omega}=0
        \end{numcases}
    \end{subequations}
    The variational problem is searching a function in \( V\), that is a function that automatically satisfy the boundary condition. If \( u\) is solution of the variational problem, then
    \begin{equation}
        \int_{\Omega}\nabla u\cdot \nabla v=\int_{\Omega}fv
    \end{equation}
    We integrate by part the left hand side:
    \begin{equation}
        \int_{\Omega}\nabla u\cdot \nabla v=-\int_{\Omega}(\Delta u)v+\underbrace{\int_{\partial\Omega}\frac{ \partial u }{ \partial n }v}_{=0},
    \end{equation}
    so that
    \begin{equation}        \label{EQooAJMDooNJTYRm}
        \int_{\Omega}(f-\Delta u)v=0
    \end{equation}
    for every \( v\in V\).

    If we really know nothing about the space \( V\), we cannot conclude that \( f-\Delta u=0\). We can however do something that will probably work if \( V\) is not too strange. If \( f-\Delta u\neq 0\) at some point \( x\in \Omega\) (suppose \( (f-\Delta u)(x)>0\) in order to fix the ideas), then \( f-\Delta u>0\) on an open set \( A\) around \( x\). If \( v\) is a positive function that vanishes outside \( A\) then, taking \( B\subset A\) on which \( v>0\),
    \begin{equation}
        \int_{\Omega}(f-\Delta u)v=\int_A(f-\Delta u)v>\int_B(f-\Delta u)v.
    \end{equation}
    The last integral is for sure strictly positive, which contradicts \eqref{EQooAJMDooNJTYRm}.
\end{proof}

%+++++++++++++++++++++++++++++++++++++++++++++++++++++++++++++++++++++++++++++++++++++++++++++++++++++++++++++++++++++++++++
\section{Galerkin's approximation}
%+++++++++++++++++++++++++++++++++++++++++++++++++++++++++++++++++++++++++++++++++++++++++++++++++++++++++++++++++++++++++++

Let us once again be not too rigorous and deal with the problem
\begin{subequations}
    \begin{numcases}{}
        -u''+u=f\\
        u(0)=u(1)=0
    \end{numcases}
\end{subequations}
on the open interval \( \mathopen] 0 , 1 \mathclose[\). The good functional space seems to be
\begin{equation}
    H_0^1\big( \mathopen] 0 , 1 \mathclose[ \big)=\{ u\in H^1\big( \mathopen] 0 , 1 \mathclose[ \big)\st u(0)=u(1)=0 \}.
\end{equation}
Of course, this definition is not rigorous because the elements in the Sobolev spaces are classes of functions and the boundary values are not defined. Let us go on and see what happens.

Let \( v\in H_0^1\). We multiply the equation by \( v\) and integrate over the interval \( I=\mathopen] 0 , 1 \mathclose[\):
\begin{equation}
    -\langle u'', v\rangle +\langle u, v\rangle =\langle f, v\rangle,
\end{equation}
and an integration by part, taking into account the fact that \( v\) vanishes at the border gives
\begin{equation}
    \langle u', v'\rangle +\langle u, v\rangle =\langle f, v\rangle,
\end{equation}

If \( u\in C^2(I)\), the two formulations are equivalent. If not, we are not sure. The point of the second formulation is that one can build a piecewise affine approximation. We divide the interval \( I\) into \( N+1\) pieces
\begin{equation}
    x_j=\frac{ j }{ N+1 }
\end{equation}
with \( j=0,\ldots, N+1\). Let \( V_N\) be the set of continuous piecewise affine functions that are vanishing on the border:
\begin{equation}
V_N=\{ v\in C^0(I)\st v|_{\mathopen] x_j , x_{j+1} \mathclose[}\text{ is linear and } v(0)=v(1)=0 \}.
\end{equation}
This is a finite dimensional vector space because the elements are determined by the values on the \( x_i\)'s. Moreover the space \( V_N\) is included in \( H^1_0(I)\).

\begin{proposition}
    There exists an unique element \( u_N\in V_N\) satisfying the equation
    \begin{equation}        \label{EQooOFLCooHmjaOM}
        \langle u_n', v'\rangle +\langle u_N, v\rangle =\langle f, v\rangle
    \end{equation}
    for all \( v\in V_N\). This solution is the \defe{Galerkin approximation}{Galerkin approximation}.
\end{proposition}

\begin{proof}
    We consider the basis \( \{ \phi_j \}_{j=1,\ldots, N}\) of \( V_N\) defined by
    \begin{equation}
        \phi_j(x_i)=\delta_{ij}.
    \end{equation}
    We are searching for \( u_N\) under the form \( u_N=\sum_{j=1}^Na_j\phi_j\). Just by computing on the point \( x_j\) we know that
    \begin{equation}
        a_j=u_N(x_j).
    \end{equation}
    At this moment, this equality does not help, but we keep it in mind. Since the equality \eqref{EQooOFLCooHmjaOM} has to hold for every \( v \in V_N\), it holds in particular for \( v=\phi_k\):
    \begin{equation}
        \langle u_N', \phi_k'\rangle +\langle u_N, \phi_k\rangle =\langle f, \phi_k\rangle
    \end{equation}
    and by linearity of the inner product,
    \begin{equation}
        \sum_la_l\big( \langle \phi'_k, \phi'_l\rangle +\langle \phi_l, \phi_k\rangle  \big)=\langle f, \phi_k\rangle .
    \end{equation}
    If we set
    \begin{subequations}
        \begin{align}
            R_{kl}&=\langle \phi_k', \phi_l'\rangle +\langle \phi_k, \phi_l\rangle \\
            b_k&=\langle f, \phi_k\rangle,
        \end{align}
    \end{subequations}
    we have to solve the linear system
    \begin{equation}
        Ra=b.
    \end{equation}
    In order to show that this system has an unique solution, we have to get some informations about the matrix \( R\). The matrix \( R\) is the matrix of the \( 2\)-form
    \begin{equation}
        R(f,g)=\langle f', g'\rangle +\langle f, g\rangle
    \end{equation}
    in the basis \( \{ \phi_j \}\) of \( V_N\).

    \begin{subproof}
        \item[\( R\) is strictly positive defined]
            We have \( R(f,f)\geq 0\) and if \( R(f,f)=0\), then \( \langle f, f\rangle =0\) and \( \langle f', f'\rangle =0\). Thus \( f=0\) almost everywhere and since elements of \( V_N\) are continuous, \( f=0\).
        \item[\( R\) is symmetric] Clear from the definition.
    \end{subproof}

    The matrix is thus invertible (in fact we do not use the symmetry to reach this conclusion) and the system has an unique solution in \( V_N\).

\end{proof}

We can compute the matrix \( R\): the elements are only some inner products and integrals. Here is a graph of \( \phi_1\) and \( \phi_2\):
\begin{center}
   \input{auto/pictures_tex/Fig_DGFSooWgbuuMoB.pstricks}
\end{center}
The affines pieces are:
\begin{subequations}
    \begin{align}
        f_1(x)&=(N+1)x\\
        f_2(x)&=-(N+1)x+2\\
        f_3(x)&=(N+1)x-1\\
        f_4(x)&=-(N+1)x+3.
    \end{align}
\end{subequations}
And we have to integrate. We make the computations:
\lstinputlisting{tex/sage/sageSnip008.sage}
returns
\VerbatimInput[tabsize=3]{tex/sage/sageSnip008.txt}
which means that
\begin{subequations}
    \begin{align}
        \langle \phi_j, \phi_j\rangle &=\frac{ 2 }{ 3(N+1) }\\
        \langle \phi_j, \phi_{j+1}\rangle&=\langle \phi_j, \phi_{j-1}\rangle =\langle f_2, f_3\rangle =\frac{1}{ 6(N+1) }\\
        \langle \phi'_j, \phi'_j\rangle &=2N+2\\
        \langle \phi'_j, \phi'_{j+1}\rangle &=-N-1
    \end{align}
\end{subequations}
The inner product \( \langle \phi_i, \phi_j\rangle \) is zero when \( | i-j |\geq 2\).

%+++++++++++++++++++++++++++++++++++++++++++++++++++++++++++++++++++++++++++++++++++++++++++++++++++++++++++++++++++++++++++
\section{Gradient on a boundary}
%+++++++++++++++++++++++++++++++++++++++++++++++++++++++++++++++++++++++++++++++++++++++++++++++++++++++++++++++++++++++++++

Let \( \Omega\) be an open subset of \( \eR^3\) and a smooth function \( f\colon \eR^3\to \eR\) that assumes positives and negatives values in \( \Omega\). Let \( \Omega_1=\Omega\cap\{ f>0 \}\), \( \Omega_2=\Omega\cap\{ f<0 \}\) and \( C=\Omega\cap\{ f=0 \}\). We have the decomposition
\begin{equation}
    \Omega=\Omega_1\cup \Omega_2\cup C
\end{equation}

A typical example is \( \Omega=B(0,1)\) subdivided into \( \Omega_1=B(0,1)\cap\{ z>0 \}\), \( \Omega_2=B(0,1)\cap \{ z<0 \}\) and \( C=\{ (x,y,0)\tq x^2+y^2<1 \}\).


Let \( u\) be a vector field defined on \( \Omega\) by
\begin{equation}
    u(x)=\begin{cases}
        u_1(x)    &   \text{if } x\in \Omega_1\\
        u_2(x)    &    \text{if } x\in \Omega_2
    \end{cases}
\end{equation}
where \( u_1\) and \( u_2\) are defined on \( \Omega\). We do not define \( u\) on \( C\) because it is of measure zero, but we consider the step function
\begin{equation}
    s=u_2-u_1
\end{equation}
that is defined on \( \Omega\). We assume that \( u\) is derivable on \( \Omega\setminus C\).

Let \( T_u\in \swD'(\Omega)\) be the distribution associated with \( u\). We compute its gradient:
\begin{equation}
    \langle \nabla\cdot T_u, \phi\rangle =-\langle T_u, \nabla\cdot u\rangle =-\int_{\Omega}u\cdot\nabla \phi=-\int_{\Omega_1}u_1\cdot \nabla \phi-\int_{\Omega_2}u_2\cdot \nabla \phi.
\end{equation}
Here we used the fact that \( \Omega=\Omega_1\cup \Omega_2\cup C\) while the integral on \( C\) is zero since we are computing a three-dimensional integral. We use the integration by part \eqref{EQooRUCKooUUrgxI}:
\begin{equation}        \label{EQooTKJDooOopGtW}
    \int_{\Omega_1}u_1\cdot \nabla\phi=\int_{\partial \Omega_1}\phi u_1\cdot n_1-\int_{\Omega_1}\phi\nabla\cdot u_1.
\end{equation}
An element of \( \partial \Omega_1\) is in particular the limit of a sequence in \( \Int(\Omega_1)\). The limit can \emph{a priori} belong in one of the following: \( \Int(\Omega)\), \( \Int(\Omega_1)\), \( \Int(\Omega_2)\), \( \partial \Omega\), \( \partial \Omega_1\), \( \partial\Omega_2\) or \( C\). Let us review them
\begin{itemize}
    \item Since we are speaking about an element of \( \partial\Omega_1\),it cannot belong to \( \Int(\Omega_1)\).
    \item Since \( \Omega_1\) and \( \Omega_2\) have no intersection, the limit of a sequence contained in \( \Omega_1\) cannot belong to \( \Int(\Omega_2)\).
    \item If the element we are speaking about belong to \( \partial \Omega_2\), then it belong to \( \partial\Omega_1\cap\partial\Omega_2=C\).
    \item Let \( a\in\partial \Omega_1\cap \Int(\Omega)\). A neighbourhood around \( a\) cannot be completely contained in \( \Omega_1\) or \( \Omega_2\), but it has to have parts in each of these two sets (because \( C\) has lower dimension). Thus it has to belong to \( \partial \Omega_1\cap\partial\Omega_2=C\).
\end{itemize}
Thus we are left with \( \partial\Omega_1\subset C\cup\partial\Omega\). Then one can write
\begin{equation}
    \int_{\partial \Omega_1}\phi u_1\cdot n_1\leq =\int_{C}\phi u_1\cdot n_1+\int_{\partial \Omega\cap\partial\Omega_1}\phi u_1\cdot n_1.
\end{equation}
But \( \phi\in \swD(\Omega)\) and \( \Omega\) is open, so \( \phi\) vanishes on the boundary of \( \Omega\), so that the second term is zero. One can thus reduce the integral over \( \partial\Omega_1\) into an integral over \( C\) in the equation \eqref{EQooTKJDooOopGtW}:
\begin{equation}
    \langle \nabla\cdot T_u, \phi\rangle =\int_{\Omega_1}\phi\nabla\cdot u_1-\int_C\phi u_1\cdot n_1+\int_{\Omega_2}\phi\nabla\cdot u_2-\int_C\phi u_2\cdot n_2.
\end{equation}
Since \( n_i\) is normal and exterior to \( \Omega_i\), we have \( n_1=-n_2\) on \( C\). By convention, we name \( n=n_1\) and \( s=u_2-u_1\). So we have
\begin{equation}
    u_1\cdot n_1+u_2\cdot n_2=u_1\cdot n-u_2\cdot n=-s\cdot n
\end{equation}
and we can write
\begin{equation}
    \langle \nabla\cdot T_u, \phi\rangle =\sum_{i=1}^2\int_{\Omega_i}\phi\nabla\cdot u_i+\int_C(s\cdot n)\phi.
\end{equation}
The whole concludes in
\begin{equation}
    \nabla\cdot T_u=T_{\nabla\cdot u}+\delta_C^{s\cdot n}
\end{equation}
where we used the Dirac ``generalization'' of~\ref{LEMooYABKooWPXIXZ}.

%+++++++++++++++++++++++++++++++++++++++++++++++++++++++++++++++++++++++++++++++++++++++++++++++++++++++++++++++++++++++++++
\section{Locally integrable functions}
%+++++++++++++++++++++++++++++++++++++++++++++++++++++++++++++++++++++++++++++++++++++++++++++++++++++++++++++++++++++++++++

\begin{definition}
    A function \( f\colon \Omega\to \eC\) is \defe{locally integrable}{locally!integrable} if \( f\in L^1(K)\) for every compact \( K\) in \( \Omega\). The set of locally integrable functions on \( \Omega\) is denoted by \( L^1_{loc}(\Omega)\)\nomenclature[Y]{\( L^1_{loc}(\Omega)\)}{locally integrable functions on \( \Omega\)}
\end{definition}

\begin{proposition}
    We have the inclusion \( L^2(\Omega)\subset L^1_{loc}(\Omega)\).
\end{proposition}

\begin{proof}
    Let \( K\) be compact in \( \Omega\) in \( \eR^d\) and \( f\in L^2(\Omega)\). We have to show \( f\in L^1(K)\). Using the Cauchy-Schwarz inequality~\ref{ThoAYfEHG} in \( L^2(K)\) we have
    \begin{equation}
        \int_K| f |=\langle f, 1\rangle_{L^2(K)} \leq \| f \|_{L^2(K)}\| 1 \|_{L^2(K)}\leq \| f \|_{L^2(\Omega)}\Vol(K)<\infty.
    \end{equation}
    Since \( f\in L^2(\Omega)\),
\end{proof}

%+++++++++++++++++++++++++++++++++++++++++++++++++++++++++++++++++++++++++++++++++++++++++++++++++++++++++++++++++++++++++++
\section{An approximation result}
%+++++++++++++++++++++++++++++++++++++++++++++++++++++++++++++++++++++++++++++++++++++++++++++++++++++++++++++++++++++++++++

\begin{theorem}[\cite{ooXRCOooCFWVg}]
    Let \( \Omega\) be an open set in \( \eR^3\) subdivided into a finite number of subdomains \( \Omega_i\). We suppose that for each \( i\), the polynomials of degree \( n\) are part of \( H^1(\Omega_i)\).

    We consider a function \( u\colon \Omega\to \Omega\) which is a polynomial of degree \( n\) on each of the subdomains \( \Omega_i\) and we suppose it to be continuous on the boundaries of \( \Omega_i\).

    \begin{enumerate}
        \item
            Then \( u\in H^1(\Omega)\)
        \item
            If we suppose that \( u\) is differentiable on each boundary, then \( u\in H^2(\Omega)\).
    \end{enumerate}
\end{theorem}

\begin{proof}
    First, the function \( u\) belongs to \( L^2(\Omega)\) because the integral of \( | u |^2\) on \( \Omega\) reduces to the integral over the interior of each \( \Omega_i\) (because the boundaries have zero measure). Since \( u\) belongs to \( L^2(\Omega_i)\) for each \( i\), we have \( u\in L^2(\Omega)\).

    Now we have to prove that the partial derivatives (in the weak sense) belong to \( L^2(\Omega)\) too. Since \( u\) is a polynomial on the interiors we can consider the function
    \begin{equation}
        f_{\alpha}(x)=\frac{ \partial u }{ \partial x_{\alpha} }
    \end{equation}
    on \( \Int(\Omega_i)\) (for each \( i\)). Here the partial derivative is not in the weak sense. We do not define the function \( f_{\alpha}\) on the boundaries. Let \( \varphi\in \swD(\Omega_i)\) and perform an integration by part:
    \begin{equation}        \label{EQooMGESooSkBybZ}
        \int_{\Omega_i}f_{\alpha}\varphi=\int_{\Omega_i}\frac{ \partial u }{ \partial x_{\alpha} }\varphi=-\int_{\Omega_i}u\frac{ \partial \varphi }{ \partial x_{\alpha} }+\int_{\partial\Omega_i}u\varphi(n_i\cdot e_{\alpha})
    \end{equation}
    where \( n_i\) is the normal vector field to \( \partial\Omega_i\). In order to make sense, the integrals over \( \Omega_i\) are in fact integrals over \( \Int(\Omega_i)\) because \( f_{\alpha}\) and \( \partial_{\alpha}u\) are only well defined on the interior.

    We sum \eqref{EQooMGESooSkBybZ} over \( i\):
    \begin{equation}        \label{EQooKACAooGlBMaQ}
        \int_{\Omega}f_{\alpha}\varphi=\sum_i\int_{\Omega_i}f_{\alpha}\varphi=-\int_{\Omega}u\frac{ \partial \varphi }{ \partial x_{\alpha} }+\sum_i\int_{\partial\Omega_i}u\varphi(n_i\cdot e_{\alpha}).
    \end{equation}
    The value of \( u\) on \( \partial \Omega_i\) is univoque since we assume that \( u\) is continuous on the boundaries.

    The set \( \bigcup_i\partial\Omega_i\) can be subdivided into two parts. Some points are in \( \partial \Omega\) and the other ones are on the intersections \( \partial\Omega_i\cap \partial\Omega_j\).

    The function \( \varphi\) vanishes on \( \partial\Omega\). And the contribution of \( \partial\Omega\) in \( \sum_i\int_{\partial\Omega_i}u\varphi (n_i\cdot e_{\alpha})\) is zero.

    The contribution of the intersections in this integral is double: the intersection \( \partial\Omega_i\cap\partial\Omega_j\) comes in the integral over \( \partial\Omega_i\) and in the one over \( \partial\Omega_j\). Since on the intersection we have \( n_i=-n_j\) (because they are outwards), the sum vanishes and the whole sum of integrals over \( \partial\Omega_i\) in \eqref{EQooKACAooGlBMaQ} disappear. Here we also use the fact that \( u\) is continuous on the intersections. We are left with
    \begin{equation}
        \int_{\Omega}f_{\alpha}\varphi=-\int_{\Omega}u\frac{ \partial \varphi }{ \partial x_{\alpha} }.
    \end{equation}
    We proved that setting
    \begin{equation}
        f_{\alpha}(x)=\begin{cases}
            \frac{ \partial u }{ \partial x_{\alpha} }    &   \text{if } x\in\Int(\Omega_i) \text{for some }i\\
            \text{whatever}    &    \text{otherwise }
        \end{cases}
    \end{equation}
    we get
    \begin{equation}
        \langle f_{\alpha}, \varphi\rangle =-\langle u, \partial_{\alpha}\varphi\rangle
    \end{equation}
    for every \( \varphi\in\swD(\Omega)\). This shows that \( f_{\alpha}\) is the weak derivative of \( u\).

    Moreover \( f_{\alpha}\) is a sum of polynomials of degree \( n-1\) on \( \Omega\) and is thus integrable. So by hypothesis \( f_{\alpha}\in H^1(\Omega)\subset L^2(\Omega)\).
\end{proof}

%+++++++++++++++++++++++++++++++++++++++++++++++++++++++++++++++++++++++++++++++++++++++++++++++++++++++++++++++++++++++++++
\section{Lax-Milgram with a boundary condition}
%+++++++++++++++++++++++++++++++++++++++++++++++++++++++++++++++++++++++++++++++++++++++++++++++++++++++++++++++++++++++++++

We want to give an example of use of the Sobolev space \( H_O^1(\Omega)\) defined in~\ref{DEFooFICWooBWCDyO}.

\begin{example}
    Let the differential equation for \( u\colon \mathopen[ 0 , 1 \mathclose]\to \eR\)
    \begin{subequations}
        \begin{numcases}{}
            -u''(x)=f(x)\\
            u(0)=u(1)=0
        \end{numcases}
    \end{subequations}
    with \( f\) in a not yet well precise functional space. The most obvious prescription for the functional spaces if to ask \( f\in L^2\) and then \( u\in H^2\), so that \( u\) has \( L^2\) second derivative. Instead of that, we will look at the variational form of the problem. We consider \( w\in\swD\big( \mathopen[ 0 , 1 \mathclose] \big)\) satisfying \( w(0)=w(1)=0\) and compute the inner product between \( v\) and the equation:
    \begin{equation}
        -\langle u'', w\rangle =\langle f, w\rangle .
    \end{equation}
    An integration by part produces the equation
    \begin{equation}        \label{EQooRTXVooYmoAJM}
        \int_0^1u'w'=\int_0^1fw.
    \end{equation}

    Since no precision is provided about the functional spaces, the equation \eqref{EQooRTXVooYmoAJM} is by no means related to initial equation. We only hope that the solution of the variational problem will be a solution of the initial problem, and that we will be able to furnish a functional setting in which the integral by part makes sense.

    Since the variational formulation only needs first derivative of \( u\) and \( v\) we are lead to consider \( H^1\big( \mathopen] 0 , 1 \mathclose[ \big)\). Let us write down the variational problem.

    Let \( f\in L^2\big( \mathopen] 0 , 1 \mathclose[ \big)\) and consider the space \( V=H_0^1\big( \mathopen] 0 , 1 \mathclose[ \big)\). We define the functionals defined by
        \begin{equation}
            \begin{aligned}
                a\colon V\times V&\to \eC \\
                (u,w)&\mapsto \int_0^1u'w'
            \end{aligned}
        \end{equation}
        and
        \begin{equation}
            \begin{aligned}
                l\colon V&\to \eC \\
                w&\mapsto \int_0^1fw.
            \end{aligned}
        \end{equation}
    We are searching for \( u\in V\) such that \( a(u,\cdot)=l\), that is such that
    \begin{equation}
        a(u,w)=l(w)
    \end{equation}
    for every \( w\in V\).

    Let us check the hypothesis of the Lax-Milgram theorem~\ref{THOooFDJYooCSNnuv}.

    \begin{subproof}
        \item[\( l\colon H_0^1\to \eR\) is continuous]

            Let \( w_i\stackrel{H_0^1}{\longrightarrow}0\). Using the Cauchy-Schwarz inequality~\ref{ThoAYfEHG} we have
            \begin{equation}
                | l(w_i) |^2=| \langle w_i, f\rangle  |^2\leq\| w_i \|_{L^2}\| f \|_{L^2}\leq C| w_i |_{1,\Omega}\| f \|_{L^2}
            \end{equation}
            where we used the Poincaré inequality of theorem~\ref{THOooMIHQooYShOps}. By hypothesis we have \( w_i\stackrel{H^1(\Omega)}{\longrightarrow}0\) and in particular \( | w_i |_{1,\Omega}\to 0\), so that
            \begin{equation}
                | l(w_i) |^2\to 0,
            \end{equation}
            and \( l\) is continuous.

        \item[\( a\colon H_0^1\times H_0^1\to \eR\) is continuous]

            By definition
            \begin{equation}
                a(u,v)=\langle u', w'\rangle_{L^2\big( \mathopen] 0 , 1 \mathclose[ \big) }.
            \end{equation}
            If \( (u_i,w_i)\stackrel{H_0^1\times H_0^1}{\longrightarrow}(0,0)\) we have
            \begin{equation}
                | a(u_i,w_i) |\leq \| u' \|_{L^2}\| w' \|_{L^2}=| u_i|_{1,\Omega}| w_i |_{1,\Omega}\to 0.
            \end{equation}
            <++>

    \end{subproof}
    <++>

\end{example}


\chapter{Chain complexes}
\input{chaines}

\chapter{Homogeneous and symmetric spaces}
\input{112_homo}
\input{113_homo}
% This is part of (almost) Everything I know in mathematics
% Copyright (c) 2013-2014,2016,2020
%   Laurent Claessens
% See the file fdl-1.3.txt for copying conditions.

%+++++++++++++++++++++++++++++++++++++++++++++++++++++++++++++++++++++++++++++++++++++++++++++++++++++++++++++++++++++++++++
\section{Symplectic symmetric spaces}
%+++++++++++++++++++++++++++++++++++++++++++++++++++++++++++++++++++++++++++++++++++++++++++++++++++++++++++++++++++++++++++

\begin{definition}
	A \defe{symplectic symmetric}{symplectic!symmetric space} space is a triple $(M,s,\omega)$ where $(M,s)$ is a symmetric space, $(M,\omega)$ is a symplectic space such that $s_x^*\omega=\omega$ for every $x\in M$.
\end{definition}

\begin{remark}
	We can weaken the symplectic condition in the definition and only ask for $\omega$ to be non degenerate because the condition $s_x^*\omega=\omega$ implies $d\omega=0$.
\end{remark}

%---------------------------------------------------------------------------------------------------------------------------
\subsection{Example}
%---------------------------------------------------------------------------------------------------------------------------

Let $G=\SL(2,\eR)$ and look at the coadjoint action $\Ad^*\colon G\to \GL(\lG^*)$. We consider the element $Z=E-F$ and the orbit
\begin{equation}
	\mO=\Ad^*(G)(Z^{\flat}).
\end{equation}
The space $\lG^*$ has the metric
\begin{equation}
	\langle X^{\flat}, Y^{\flat}\rangle =\beta(X,Y)
\end{equation}
Let us consider $\mfo=Z^{\flat}\in\lG^*$ and consider the stabilizer:
\begin{equation}
	\Stab_{\mfo}(\mO)=\{ g\in G\tq \Ad^*(g)Z^{\flat}=Z^{\flat} \}.
\end{equation}
The Lie algebra is given by
\begin{equation}
	\stab_{\mfo}(\mO)=\{ X\in\lG\tq Z^{\flat}\circ\ad(X)=0 \}.
\end{equation}
The condition of the Lie algebra reads
\begin{equation}
		0=\langle Z^{\flat}, [X,Y]\rangle
		=\beta(Z,[X,Y])
		=-\beta\big( [X,Z],Y \big)
\end{equation}
for every $Y$, which implies $[X,Z]=0$ because $\beta$ is nondegenerate. Now, in $\gsl(2,\eR)$, the only possibility is that $X$ is proportional to $Z$. Thus the Lie algebra reduces to $\eR Z$ in fact.

%---------------------------------------------------------------------------------------------------------------------------
\subsection{Algebraic setting}
%---------------------------------------------------------------------------------------------------------------------------

We want now to encode the symplectic space structure in an algebraic data. What we are going to discover is the notion of symplectic triple that will be developed in section~\ref{SubSecTripleSylple}.

Let $(G,\sigma)$ be an involutive Lie group and $H$ a closed subgroup of $G$ such that
\begin{equation}
	G_0^{\sigma}\subset H\subset G^{\sigma}.
\end{equation}
Let $\pi$ be the projection $\pi\colon G\to M=G/H$.

The symmetry on the quotient $G/H$ is given by the theorem~\ref{ThoStructSymGH}:
\begin{equation}		\label{EaSymGH}
	s_{[g]}[g']=\big[ \sigma(g^{-1}g') \big]
\end{equation}

Now, if we denote by $\sigma$ the differential $d\sigma_e$, we can decompose the Lie algebra $\mG$ into $\mG=\mH\oplus\mP$ and we have the isomorphism (see lemma~\ref{LemdpiisomMTM})
\begin{equation}
	d\pi_e|_{\mP}\colon \mP\to T_{\mfo}(M)
\end{equation}
where $\mfo=[e]$. Thus we can see the form $\omega_{\mfo}$ on $\mP$ by
\begin{equation}
	\Omega=\big( d\pi_e|_{\mP} \big)^*\omega_{\mfo}
\end{equation}
and the space $(\mP,\Omega)$ becomes a symplectic vector space.

\begin{lemma}
	We have
	\begin{enumerate}

		\item
			the space $\mK=[\mP,\mP]$ is a Lie subalgebra of $\mH$,

		\item
			the adjoint action of $\mK$ over $\mP$ preserves the symplectic form, i.e.
			\begin{equation}
				\Omega\big( [Z,X],Y \big)+\Omega\big( X,[Z,Y] \big)=0
			\end{equation}

	\end{enumerate}

\end{lemma}

\begin{proof}
	Sketch of the proof.

	Let $x_j,y_j\in\mP$. Using the Jacobi identity on the nested commutator $\big[ [x_1,y_1],[x_2,y_2] \big]$ and the facts that $[\mP,\mP]\subset\mH$ and $[\mH,\mP]\subset \mP$, we find the commutator of two elements of $[\mP,\mP]$ belongs to $[\mP,\mP]$.

	First we consider $\mG^{(M)}=\mK\oplus\mP$ and $G(M)$, the associated Lie group. Then we have
	\begin{equation}
		M\simeq G(M)/K.
	\end{equation}

	Now one can see that the group $G(M)$ is generated by the products $\{ s_{\mfo}s_x\}$ with $x\in M$. Indeed let $X\in\mP$ and look at $\exp(\mP)$ as map on $M$. Using the symmetry \eqref{EaSymGH} and the fact that $\sigma e^{X/2}= e^{-X/2}$, we have
	\begin{equation}
		s_{\exp(X/2)\cdot \mfo}\mfo= e^{X/2}\big[ \sigma[ e^{-X/2}] \big]=[ e^{X}]= e^{X}\cdot \mfo.
	\end{equation}
	If we act on an other point than $\mfo$, we have
	\begin{equation}
		\begin{aligned}[]
			s_{[ e^{X/2}]}[g]&= e^{X/2}\big[ \sigma( e^{-X/2}g) \big]\\
			&= e^{X/2}\big[  e^{X/2}\sigma(g) \big]\\
			&= e^{X}s_{\mfo}[g]
		\end{aligned}
	\end{equation}
	because $\big[ \sigma(g) \big]=s_{\mfo}[g]$.

	Now, using the lemma~\ref{LemAlgEtGroupesGenere}, the fact that the elements $ e^{X}$ with $X\in\mP$ generate $ e^{\mP}$ in $G$ implies that it also generate the elements of the form $ e^{[\mP,\mP]}$ and then the whole $G(M)$. Since the elements $ e^{\mP}$ are of the form $s_{x}s_{\mfo}$, we conclude that $G(M)$ is generated by the products $s_{x}s_{\mfo}$.

	Thus we have $g^*\omega=\omega$ for every $g\in G(M)$ because $\omega$ is preserved by all the symmetries.
\end{proof}

%---------------------------------------------------------------------------------------------------------------------------
\subsection{Symplectic triple}
%---------------------------------------------------------------------------------------------------------------------------
\label{SubSecTripleSylple}

A \defe{symplectic triple}{symplectic!triple} is the data of the triple $(\mG,\sigma,\Omega)$ where $(\mG,\sigma)$ is an involutive Lie algebra and $\Omega$ is a $\mK$-invariant nondegenerate $2$-form $\Omega\in\Lambda^2(\mP^*)$. The $\mK$ invariance means that for every $Z\in\mK$ and $X,Y\in\mP$,
\begin{equation}
	\Omega\big( [Z,X],Y \big)+\Omega\big( X,[Z,Y] \big)=0.
\end{equation}

A symplectic triple is the infinitesimal version of a symplectic symmetric space. The following more abstract version of the definition comes from \cite{StrictSolvableSym}:
\begin{definition}
	The triple $(\mG,\sigma,\Omega)$ is a \defe{symplectic triple}{symplectic!triple} when $\Omega\in\Lambda^2\mG$ and
	\begin{enumerate}
		\item  If  $\mG=\mK\oplus\mP$ is the decomposition of $\mG$ into eigenspaces of $\sigma$,  then $[\mP,\mP]=\mK$ and the adjoint representation of $\mK$ on $\mP$ is faithful. ($\mK$ is the eigenspaces with eigenvalue $+1$ of $\sigma$ while $\mP$ is the one of $-1$)

		\item The $2$-form $\Omega$ is a Chevalley $2$-cocycle for the trivial representation of $\mG$ on $\eR$.

		\item $i(\mK)\Omega=0$ and $\Omega|_{\mP\times\mP}$ is nondegenerate.
	\end{enumerate}
\end{definition}

Notice that $[\mP,\mP]\subset\mK$ is automatic from the definition of $\mP$ and $\mK$ as eigenspaces of $\sigma$; the hypothesis is the equality.

Let us now see how one build a symplectic symmetric space from the data of the symplectic triple $(\mG,\sigma,\Omega)$. First we consider $G$, the group associated with $\mG$ and $M=G/K$ with the left invariant form $\omega$ build on $\Omega$.

%---------------------------------------------------------------------------------------------------------------------------
\subsection{Example on the Heisenberg algebra}
%---------------------------------------------------------------------------------------------------------------------------

Let $\pH=V\oplus\eR E$  be the Heisenberg algebra of $(V,\Omega^0)$, and consider the derivation
\begin{equation}
	D=\id|_V\oplus(2\id)|_{\eR E}.
\end{equation}
If we consider the algebra $\mA=\eR H$, we build the semi direct product
\begin{equation}
	\mS=\mA\ltimes_D\pH
\end{equation}
with the definition $[H,x]=D(x)$ when $x\in\pH$.

An other split extension that can be done is
\begin{equation}
	\mG_0=\mA\rtimes_{\rho}(\pH\oplus\pH)
\end{equation}
with $\rho=D\oplus(-D)$. The algebra $\mG_0$ is to be endowed with a symplectic triple structure. We define $\sigma_0\colon \mG_0\to \mG_0$
\begin{equation}
	\begin{aligned}[]
		\sigma_0(x,y)&=(y,x)&\in\pH\oplus\pH\\
		\sigma_0(H)&=-H
	\end{aligned}
\end{equation}
and $(\mG_0,\sigma_0)$ is an involutive automorphism. Indeed, we have
\begin{equation}
	\sigma[H,x\oplus y]=\sigma\big( Dx\oplus(-Dy) \big)=-Dy\oplus Dx,
\end{equation}
while
\begin{equation}
	\big[ \sigma H,\sigma(x\oplus y) \big]=[-H,y\oplus x]=-Dy\oplus Dx.
\end{equation}

Let us take the notation\nomenclature[G]{$W_{\pm}$}{The set of elements of the form $(w,\pm w)$ in $W\oplus W$}
\begin{equation}		\label{EqDefNitWpm}
	W_{\pm}=\{ (w,\pm w) \}_{w\in W}.
\end{equation}

If we decompose $\mG_0=\mK_0\oplus\mP_0$, we have
\begin{equation}
	\begin{aligned}[]
		\mK_0&=\pH_+\\
		\mP_0&=\mA\oplus\pH_-.
	\end{aligned}
\end{equation}
We have $H\in\mP$ $x\oplus(-x)\in\mP$ and $x\oplus x\in\mK$.

In fact we have an identification between $\mS$ and $\mP_0$ by
\begin{equation}
	\begin{aligned}
		\mS&\to \mP_0 \\
		a+x&\mapsto a+x_-
	\end{aligned}
\end{equation}
where $x_{\pm}=\frac{ 1 }{2}(x,\pm x)$. Using the notation \eqref{EqDefNitWpm}, we write
\begin{equation}
	\begin{aligned}[]
		\mK&=\pH_{+}\\
		\mP&=\mA\oplus\pH_{-}.
	\end{aligned}
\end{equation}

Under that identification we have le following lemma.
\begin{lemma}
	We have
	\begin{equation}
		\Lambda^2(\mP_0^*)\simeq\Lambda^2(\mS^*)
	\end{equation}
	and if we define
	\begin{equation}
		\Omega^{\mS}(a+x,a'+x')=\Omega(a+x_-,a'+x'_-),
	\end{equation}
	we have $\Omega\in\Lambda^2(\mP_0^*)$ and it is $\mK_0$ invariant if and only if the two conditions
	\begin{equation}
		\begin{aligned}[]
			\Omega^{\mS}(E,\pH)&=0\\
			\Omega^{\mS}|_{V\times V}&=\frac{ 1 }{2}\Omega^{\mS}(H,E)\Omega^0.
		\end{aligned}
	\end{equation}
	hold.
\end{lemma}

\begin{proof}
	Let us write down the condition of $\mK$-invariance of the symplectic form
	\begin{equation}
		\Omega\big( [x_+,a+y_-],a'+y'_- \big)+\Omega\big( a+y_-,[x_+,a'+y'_-] \big)=0.
	\end{equation}
	If we develop $x_+$ and $y_-$, the commutator in the first term becomes
	\begin{equation}
		\big[ \frac{ 1 }{2}(x,x),a+\frac{ 1 }{2}(y,-y) \big]=-\frac{ a }{2}(Dx,-Dx)+\frac{1}{ 4 }\big( [x,y],-[x,y] \big).
	\end{equation}
	The first term is rewritten as $[x,a]_-$, while the second term is $\frac{ 1 }{2}[x,y]_-$. The sum is then $[x,a+\frac{ 1 }{2}y]_-$. Looking at the definition of $\Omega^{\mS}$, the invariance condition reads
	\begin{equation}
		\begin{aligned}[]
			0&=\Omega\big( [x,a+\frac{ 1 }{2}y],a'+y' \big)+\Omega^{\mS}\big( a+y,[x,a'+\frac{ 1 }{2}y'] \big)\\
			&=\Omega^{\mS}\big( -a Dx+\frac{ 1 }{2}\Omega_0(x,y)E,a'+y' \big)+\Omega^{\mS}\big( a+y,-a'Dx+\frac{ 1 }{2}\Omega_0(x_V,y_V')E \big).
		\end{aligned}
	\end{equation}
	If we look at that condition with $a'=0$, $a=1$, $x_V=0$ and $y=0$ and taking into account $Dx=x_V+2x_EE$ we find
	\begin{equation}
		\Omega^{\mS}(2x_EE,y')=0,
	\end{equation}
	so that $\Omega^{\mS}(E,y')$. We conclude that a necessary condition for the invariance is
	\begin{equation}
		\Omega^{\mZ}(E,\pH)=0.
	\end{equation}
	Now if we consider $y'\in V$, $x_E=0$ and $x_V\neq 0$, we find
	\begin{equation}
		\Omega^{\mS}|_{V\times V}=\frac{ 1 }{2}\Omega^{\mS}(H,E)\Omega_0.
	\end{equation}

	One can check that these two conditions insure the $\mK$-invariance of $\Omega^{\mS}$.
\end{proof}

To each non degenerate form satisfying these two conditions corresponds a symplectic triple $(\mG_0,\sigma_0,\Omega)$.

%---------------------------------------------------------------------------------------------------------------------------
\subsection{Realization as coadjoint orbit}
%---------------------------------------------------------------------------------------------------------------------------

Let $(M=G/H,s,\omega)$ be a symmetric symplectic space. We are going to study under which conditions we can realise $M$ as a coadjoint orbit, i.e. we want the two conditions
\begin{enumerate}

	\item
		there exists a $\xi_0\in\mG^*$ such that $\Stab_G(\xi_0)=H$ where $\Stab$ stands for the stabilizer for the coadjoint action of $G$. Let
		\begin{equation}
			\begin{aligned}
				\Phi\colon M&\to \Ad^*(G)\xi_0=\mO \\
				[g]&\mapsto \Ad^*(g)\xi_0
			\end{aligned}
		\end{equation}
		be the identification between $M$ and the coadjoint orbit $\mO$.
	\item
		The identification $\Phi$ fits the symplectic structures:
		\begin{equation}
			\Phi^*\omega^{\mO}=\omega
		\end{equation}
		where $\omega^{\mO}$ is the canonical symplectic structure on the coadjoint orbit given by \eqref{eq_omega_Gs}.
\end{enumerate}

\begin{definition}
	A \defe{good polarization}{polarization!good} associated to $\xi_0$ is a Lie subalgebra $\mB$ of $\mG$ which is maximal for the property $\delta\xi_0|_{\mB\times \mB}\equiv 0$ where the alternate bilinear $2$-form $\delta\xi_0$ on $\mG$ is defined by
	\begin{equation}
		\delta\xi_0=\langle \xi_0, [.,.]\rangle.
	\end{equation}
\end{definition}

If $\mB$ is a good polarization, we consider $B=\exp(\mB)$ and we have a representation $\chi\colon \mB\to \gU(1)$ given by
\begin{equation}
	\begin{aligned}
		\chi\colon \mB&\to \gU(1) \\
		\exp(y)&\mapsto  e^{i\langle \xi_0, y\rangle }.
	\end{aligned}
\end{equation}
It turns out that $\chi$ is a representation even when $\mB$ is non abelian. Indeed, if $x,y\in\mB$, we have
\begin{equation}
	\chi( e^{x} e^{y})=\chi( e^{x+y+W})= e^{i\langle \xi_0, x+y+W\rangle }= e^{i\langle \xi_0, x+y\rangle }= \chi( e^{x})\chi( e^{y})
\end{equation}
where $W$ is a combination if commutators of $x$ and $y$ (Campbell-Baker-Hausdorff) so that by definition of $\mB$, $\langle \xi_0, W\rangle =0$.

Since we are in the hypothesis (see subsection~\ref{SubSecUnitInducedPrep}), we can define the induced unitary representation
\begin{equation}
	U\colon G\to \gU(\hH_{\chi})
\end{equation}
where $\hH_{\chi}=L^2(Q,dq)$. Let $dg$ be the left invariant Haar measure on $G$.  To each $u\in L^1(G,dg)$, we make correspond an operator $U(u)$ on $\hH_{\chi}$ given by
\begin{equation}	\label{EqDefUudansHh}
	\langle U(u)\varphi, \psi\rangle =\int_Gu(g)\langle U(g)\varphi, \psi\rangle dg
\end{equation}
for every $\varphi,\psi\in\hH_{\chi}$. Let us prove that this integral exists. We have
\begin{equation}	\label{EqIntdefUgGdgvppsi}
		| \langle U(g)\varphi, \psi\rangle  |\leq\int_G| u(g) | |\langle U(g)\varphi, \psi\rangle  |dg,
\end{equation}
but the Cauchy-Schwarz inequality shows that $| \langle U(g)\varphi, \psi\rangle  |\leq| U(g)\varphi | |\psi |=| \varphi | |\psi |$, so that the integral in \eqref{EqIntdefUgGdgvppsi} is smaller than
\begin{equation}
	| \varphi | | \psi |  \int_G| u(g) |dg
\end{equation}
which exists because we supposed $u\in L^1(G,dg)$.

\begin{probleme}
	The following paragraph can be more precise.
\end{probleme}

We can rewrite the definition \eqref{EqDefUudansHh} using the measure theory given around section~\ref{sec_distrib_mesure}. Indeed the space $\opB(\hH)$ of bounded operators\footnote{For linear operators on Hilbert spaces, the fact to be bounded is equivalent to continuity.} on the Hilbert space $\hH$ is endowed with the operator norm for which $\opB(\hH)$ becomes a normed algebra ($\| AB \|_{op}\leq\| A \|_{op}\| B \|_{op}$). The unitary group $\gU(\hH)$ is a subalgebra (because it is closed for the composition), so that one can consider, for each function $u$, the function
\begin{equation}
	\begin{aligned}
		G&\to \opB(\hH) \\
		g&\mapsto u(g)U(g)
	\end{aligned}
\end{equation}
and its integral
\begin{equation}
	\int_G u(g)U(g)dg
\end{equation}
which is an element in $\opB(\hH)$. This integral is well defined in $\opB(\hH_{\chi})$ because
\begin{equation}
	\| \int_G u(g)U(g)dg \|_{op}\leq\int_G | u(g) |\cdot \| U(g) \|_{op}dg=\| u \|_{L^1}.
\end{equation}

Using the measure theory, one can prove that
\begin{equation}
	\langle U(u)\varphi, \psi\rangle =\langle  \big( \int_G u(g)U(g)dg \big)\varphi , \psi\rangle.
\end{equation}

What we build up to here is a map
\begin{equation}
	\begin{aligned}
		U\colon L^1(G,dg)&\to \opB(\hH_{\chi}) \\
		u&\mapsto U(u)
	\end{aligned}
\end{equation}
given by
\begin{equation}
	U(u)=\int_G u(g)U(g)dg.
\end{equation}
This map is linear and continuous because $\| U(u) \|_{op}\leq\| u \|_{L^1}$.

We are now going to use the symmetry on $M$ in order to descend $U$ from $L^1(G,dg)$ to $L^1(M)$. Let us take a look at the two projections from $G$:
\begin{equation}
	\xymatrix{%
	G \ar[r]^{\pi^M}\ar[d]_{\pi^Q}		&	G/H	\\
	   G/B
	   }
\end{equation}
If $X,Y\in\mH$, we recall that the definition of $\ad(X)^*$ is
\begin{equation}
	\langle \xi_0, [X,Y]\rangle =-\langle \ad(X)^*\xi_0, Y\rangle,
\end{equation}
but, since $ e^{tX}\in\Stab(\xi_0)$, we have $\Dsdd{ \Ad(\exp(tX))^*\xi_0 }{t}{0}=0$, thus
\begin{equation}
	\langle \ad(X)^*\xi_0, Y\rangle =0
\end{equation}
and we can suppose that the good polarization $\mB$ contains $\mH$. In that case we have the well defined map
\begin{equation}
	\begin{aligned}
		\tilde\pi\colon G/H&\to G/B \\
		gH&\mapsto gB
	\end{aligned}
\end{equation}
This is well defined because, since $H\subset B$, we have
\begin{equation}
	\tilde\pi(ghH)=ghB=gB
\end{equation}
for every $h\in H$.

\begin{lemma}
	The map $\tilde\pi$ is a submersion.
\end{lemma}
\begin{proof}
	No proof.
\end{proof}
The following diagram commutes:
\begin{equation}
	\xymatrix{%
	G \ar[r]^{\pi^M}\ar[d]_{\pi^Q}		&	G/H\ar[dl]^{\tilde\pi}\\
	   G/B
	   }
\end{equation}

Still two assumptions about $\sigma$:
\begin{enumerate}
	\item
		we suppose that $B$ is stable under $\sigma$,
	\item
		we suppose that $\xi_0$ is $\sigma$-invariant, that is $\xi_0(\sigma X)=\xi_0(X)$.
\end{enumerate}
The second assumption is easy to fulfill. If $\xi_0$ is not $\sigma$-invariant, we consider
\begin{equation}
	\xi_0'=\frac{ 1 }{2}(\xi_0+\sigma^*(\xi_0))
\end{equation}
instead.

Now, the symmetry
\begin{equation}
	\begin{aligned}
		\sigma_H\colon M&\to M \\
		gH&\mapsto \sigma(g)H
	\end{aligned}
\end{equation}
descends to $Q$ as
\begin{equation}
	\begin{aligned}
		\underline\sigma\colon Q&\to Q \\
		gB&\mapsto\tilde\pi\big( \sigma_H(gH) \big)=\sigma(g)B.
	\end{aligned}
\end{equation}
The so defined map $\underline\sigma$ is well defined because
\begin{equation}
	\underline\sigma(gbB)=\sigma(gb)B=\sigma(g)\sigma(b)B=\sigma(g)B=\underline\sigma(gB).
\end{equation}

\begin{lemma}
	Using the hypothesis of $\sigma$-invariance of $\xi_0$, we have that
	\begin{equation}
		\sigma^*\colon  C^{\infty}(G,\eC)^B\to  C^{\infty}(G,\eC)^B,
	\end{equation}
	the image of a $B$-equivariant function on $G$ by $\sigma^*$ is still $B$-equivariant.
\end{lemma}

\begin{proof}
	Let $\hat\varphi\in C^{\infty}(G,\eC)^B$, then we have
	\begin{equation}
		\begin{aligned}[]
			(\sigma^*\hat\varphi)(gb)&=\hat\varphi\big( \sigma(g)\sigma(b) \big)\\
			&=\chi\big( \sigma(b)^{-1} \big)\hat\varphi\big( \sigma(g) \big)\\
			&= e^{-i\langle \xi_0, \sigma\log(b)\rangle }\hat\varphi(\sigma g)\\
			&= e^{-i\langle \xi_0, \log(b)\rangle }(\sigma^*\hat\varphi)(g)	&\text{because }\xi_0(\sigma X)=\xi_0(X)\\
			&=\chi(b^{-1})(\sigma^*\hat\varphi)(g).
		\end{aligned}
	\end{equation}
\end{proof}

Since the measure $dq$ is $\sigma^*$-invariant by hypothesis, we have
\begin{equation}
	\int_Q\overline{ (\underline\sigma^*u) }(q)(\underline\sigma^*v)(q)dq=\int_Q\overline{ u(q') }v(q)\underline\sigma^*dq=\int_Q\overline{ u(q') }v(q)dq
\end{equation}
where we used the change of variable $q'=\sigma q$. A consequence is that $\underline\sigma^*$ is an involution
\begin{equation}
	\underline\sigma^*\colon L^2(Q,dq)\to L^2(Q,dq).
\end{equation}
Since, in an abstract way, we denoted $L^2(Q,dq)$ by $\hH_{\chi}$, we denote by $\Sigma$ the involution $\underline\sigma^*$ on $\hH_{\chi}$. Now we consider the function
\begin{equation}
	\begin{aligned}
		\Omega\colon G&\to \gU(\hH_{\chi}) \\
		g&\mapsto U(g)\Sigma U(g^{-1})
	\end{aligned}
\end{equation}
which is a composition of unitary maps. This is not a representation of the group $G$, but we have
\begin{equation}
	\Omega(gh)=\Omega(g)
\end{equation}
for every $h\in H$ and $g\in G$. Indeed let us compute $\widehat{\Omega(gh)\varphi}$ for $\varphi\in\cdD(Q)$. We have
\begin{equation}		\label{EqwOshvhvkl}
		\widehat{\Omega(gh)\varphi}=\hat U(gh)\sigma^*\hat U(h^{-1}g^{-1})\hat\varphi
		=\hat U(h)\hat U(h)\sigma^*\hat U(h^{-1})\hat U(g^{-1})\hat\varphi.
\end{equation}
The element $h$ only appears in the combination $\hat U(h)\sigma^*\hat U(h^{-1})$, so let us see how it acts on an equivariant function $\hat \varphi$. If we evaluate it on $g_0$ we find
\begin{equation}		\label{EqbhUsigmastargzi}
		\big( \hat U(h)\sigma^*\hat U(h^{-1})\hat\varphi \big)(g_0)=\big( \sigma^*\hat U(h^{-1})\hat\varphi \big)(h^{-1}g_0)
		=\hat\varphi\big( h\sigma(h^{-1}g_0) \big).
\end{equation}
Let us recall that we are in the context\footnote{With many notational incoherences.} of subsection~\ref{SubSecInducrepresBBGC}: the representation $U$ on $\cdD(Q)$ comes from the regular left representation $\hat U$ on $ C^{\infty}(G,\eC)^B$. Thus we have $\big( \hat U(g)\hat\varphi \big)(g_0)=\hat\varphi(g^{-1}g_0)$. Equation \eqref{EqbhUsigmastargzi} is thus equal to
\begin{equation}
	\hat\varphi\big( h\sigma(h^{-1}g_0) \big)=\hat\varphi\big( h\sigma(h^{-1})\sigma(g_0) \big)=\hat\varphi\big( \sigma(g_0) \big)=(\sigma^*\hat\varphi)(g_0),
\end{equation}
so that equation \eqref{EqwOshvhvkl} does not depend on $h$, which proves that
\begin{equation}
	\Omega(gh)=U(g)\Sigma U(g^{-1})=\Omega(g).
\end{equation}
One consequence of this circumstance is that $\Omega$ is a function which pass to the quotient $G\to G/H$. Thus we consider the map
\begin{equation}
	\Omega\colon M=G/H\to \gU(\hH_{\chi})\subset\opB(\hH).
\end{equation}

Since $M$ is a symplectic manifold, we have a natural volume form
\begin{equation}
	dx=\frac{1}{ n! }\omega^n
\end{equation}
where $n=\frac{ 1 }{2}\dim M$. This measure allows us to consider the map
\begin{equation}
	\begin{aligned}
		\Omega\colon L^1(M,dx)&\to \opB(\hH) \\
		u&\mapsto \Omega(u)
	\end{aligned}
\end{equation}
defined by
\begin{equation}
	\Omega(u)=\int_M u(x)\Omega(x)dx
\end{equation}
which is a continuous linear map. This is not a representation (even on $G$ the initial $\Omega$ was not a representation and $M$ is not a group), but it is an unitary representation of $M$ in the following sense.

\begin{definition}
	An \defe{unitary representation}{representation!of a symmetric space} is a map $\Omega\colon M\to \gU(\hH)$ of the symmetric space $M$ when it satisfies to the properties
	\begin{enumerate}
		\item
			$\Omega(x)\Omega(y)\Omega(x)=\Omega(s_xy)$
		\item
			$\Omega(x)^2=\id|_{\hH}$.

	\end{enumerate}
	for every $x,y\in M$.
\end{definition}
\section{Hermitian and symplectic spaces}
%+++++++++++++++++++++++++++++++++++++++
\label{SecHermEtSymplecticSpaces}

If $ANK$ is the Iwasawa decomposition of a Lie group\quext{c'est pas mal de dire quel genre de groupes : simple ? semi ? compact ?}, one can consider the manifold $M=G/K$. There is a natural identification
\[
   \mP=T_KM
\]
where $\mP$ comes from the Cartan decomposition $\mG=\mP\oplus\mK$. Indeed, the Iwasawa theorem says that $M\simeq AN$ so that a path in it reads $g(t)=a(t)n(t)$ with $g(0)=e$. But one has a diffeomorphism $A\times N\times K\to G$, so that $g(0)=e$ implies $a(0)=n(0)=e$. Thus Leibnitz makes $g'(0)=a'(0)+n'(0)$ and $g'(0)\in\mA\oplus\mN=\mP$.

Let us recall that when $G$ is a Lie group, and $H$ a closed connected subgroup of $G$, $G/H$ is a manifold on which $G$ acts. This structure is an \defe{homogeneous space}{homogeneous!space}. If moreover $H$ is the set of the fixed points of an involution on $G$, $G/H$ is says to be a \defe{symmetric space}{symmetric!space}.

More precisely, the involution is a $\dpt{\theta}{\mG}{\mG}$ which let fixed $\mH\subset\mG$; then $H$ is the connected Lie group whose Lie algebra is $\mH$. All this makes that the $G/K$ from Iwasawa is a symmetric space.

\begin{definition}  \label{DefCLtjFtD}
    Here, $M$ denotes a connected smooth manifold. An \defe{almost complex structure}{almost!complex structure} on $M$ is a $(1,1)$ tensor field $J$ such that $\forall X\in\cvec(M)$,
    \begin{equation}
       (J\circ J)X=-X.
    \end{equation}
    The tensor field $J$ is a \defe{complex structure}{complex structure} when moreover it satisfies the \emph{integrability condition}: $\forall X,Y\in\cvec(M)$,
    \begin{equation}  \label{DefComplStruct}
       N(X,Y):=[X,Y]+J[JX,Y]+J[X,JY]-[JX,JY]=0.
    \end{equation}
\end{definition}
We already spoke about complex structure in order to define the signed curvature of planar curve around the definition~\ref{DEFooTSJXooTIyRXf}.
%TODO : Make clearer the definitions~\ref{DefCLtjFtD},~\ref{DefSymHermMGKalg} and~\ref{DefKONtphK} that are more or less the same.

\begin{definition}		\label{DefSymHermMGKalg}
  The symmetric space $M=G/K$ is \defe{hermitian}{hermitian!symmetric space} if there exists an endomorphism $J\in\End{\mP}$, $\dpt{J}{\mP}{\mP}$ such that
\begin{subequations}
\begin{align}
  J^2&=-id_{\mP},                                           \label{eq:herm_1} \\
  B(JX,JY)&=B(X,Y)            && \forall\,X,Y\in\mP,    \label{eq:herm_2}\\
  \ad (k)\circ J&=J\circ\ad(k)&& \forall\,k\in\mK.      \label{eq:herm_3}
\end{align}
\end{subequations}
\label{def:hermitien}
\end{definition}

Since one has the identification $\mP=T_KM$, $J$ is only defined on $T_{[e]}M$. The following proposition extends the definition.

\begin{proposition} \label{prop:ext_J}
    The hermitian structure $J$ can be extended to a complex structure $\oJ$ on the whole $TM$.
\end{proposition}

\begin{proof}
 For $X\in T_{[g]}M$, we define
\begin{equation}
  \oJ(X):=dL_g\circ J\circ dL_{g^{-1}}X.
\end{equation}
where $dL$ is the differential of $\dpt{L_g}{G/K}{G/K}$, $L_g[h]=[gh]$.
From this, $\oJ^2(X)=-X$ because
\begin{equation}
  (\oJ\circ\oJ)X=( dL_g J dL_{g^{-1}} )\circ( dL_g J dL_{g^{-1}} )X
           =dL_g J^2dL_{g^{-1}}X
	   =-X.
\end{equation}
On the other hand, $J$ satisfies $\ad(k)\circ J=J\circ \ad(k)$ and we want the same for $\oJ$:
\[
  \ad(X)\circ\oJ=\oJ\circ\ad(X)
\]
for $X\in T_{[g]}M$. Note that it is true for $[g]=[e]$ because $T_{[e]}M=\mK$. Let us consider $X\in T_{[g]}M$, and let us see what is $ \big( (\ad X)\circ J \big)Y $ for a $Y\in T_{[g]}M$. Consider $x$, $y\in T_{[e]}M$ such that $X=dL_g x$ and $Y=dL_g y$. Suppose one has
\begin{equation}\label{eq:suppose}
   \ad(dL_g x)Y=dL_g\circ\ad(x)(dL_{g^{-1}}Y);
\end{equation}
then one can compute
\begin{subequations}
    \begin{align}
\big(  \ad(X)\circ\oJ \big)Y&=\ad(dL_g x)\circ dL_g\circ J\circ dL_{g^{-1}}Y\\
&=dL_g\ad(x) \circ J\circ dL_{g^{-1}}Y\label{subEqEMyROwA}\\
	                    &=dL_g\circ J\circ\ad(x)\circ dL_{g^{-1}}Y\\
			    &=(dL_g\circ J\circ dL_{g^{-1}})\circ (dL_g\circ\ad(x)\circ dL_{g^{-1}})\\
			    &=\oJ\circ\ad(X)Y.
    \end{align}
\end{subequations}
The line \eqref{subEqEMyROwA} comes from $\ad(x)\circ J=J\circ\ad(x)$ because $x\in T_{[e]}M$.

Now, we prove equation \eqref{eq:suppose} which is rewritten in a more convenient way as $[dL_g x,Y]=dL_g[x,dL_{g^{-1}}Y]$. Thus one has to see if for any $x$, $y\in T_{[e]}M$,
\[
   dL_g[x,y]=[dL_g x,dL_g y].
\]
This is true because of \cite{Helgason}, proposition 3.3, page 34.  Now we know that $\forall X\in T_{[g]}M$ we have $\oJ\circ\ad(X)=\ad(X)\circ\oJ$ and ${\oJ}\,^2X=-X$.  In order to have a complex structure, one also need to check condition \eqref{DefComplStruct}, which is true because
\begin{equation}
\begin{split}
J[JX,Y]&=-\ad Y\circ JJX=\ad(Y)X=[X,Y],\\
J[X,JY]&=J\circ\ad(X)JY=-[X,Y],\\
-[JX,JY]&=-(\ad JX\circ J)Y
        =-J(\ad JX)Y
	=-[Y,X].
\end{split}
\end{equation}

\end{proof}

\noindent If $X\in T_{[g]}M$,
\begin{equation}\nonumber
\begin{split}
  (\oJ\circ dL_h)X&=dL_{hg}\circ J\circ dL_{(hg)^{-1}}dL_h X\\
           &=dL_{hg}\circ J\circ dL_{g^{-1}} X\\
	   &=dL_h\circ dL_g\circ J\circ dL_{g^{-1}} X\\
	   &=(dL_h\circ \oJ)X.
\end{split}
\end{equation}
so we have an important property:
\begin{equation}\label{eq:J_dL}
   \oJ\circ dL_h=dL_h\circ\oJ.
\end{equation}
From now, it is clear that we will often forget the bar on $\oJ$.  In the same way that $J$ extends to $M$,

\begin{proposition}
For $X$, $Y\in T_{[g]}M$, the formula
\begin{equation}\label{eq:BdL}
  \overline{ B }(X,Y):=B(dL_{g^{-1}}X,dL_{g^{-1}}Y)
\end{equation}
defines a Riemannian metric on $M$.
\end{proposition}

\begin{proof}
One has to see that it is nondegenerate. Say that $Z\in T_{[g]}M$ is such that for any $X$,
$\overline{ B }(Z,X)=0$. Then $B(dL_{g^{-1}}Z,dL_{g^{-1}}X)=0$. But $dL_g$ is a vector space isomorphism because
$dL_g(o)=\Dsdd{L_g(X_t)}{t}{0}$ with $X_t$, a constant path at $[e]\in M$.

But since $B$ is nondegenerate, the definition \eqref{eq:BdL} says us $dL_{g^{-1}}Z=0$, and then $Z=0$.
\end{proof}

Now, one knows\quext{Cf cours de géométrie symplectique} that
\begin{equation}
  \omega^M_x(X,Y)=g_x(JX,Y)
\end{equation}
defines a $G$-invariant symplectic structure on $M$.

In order to see it, one has to show that $(M,g,J)$ is  a Kähler structure. The $G$-invariance comes from the extension of Killing form that we had chosen: $\forall X,Y\in T_{[g]}M$, $B_{[g]}(X,Y)=B(dL_{g^{-1}}X,dL_{g^{-1}}Y)$.
It is clear that
\begin{equation}
B_{[hg]}(dL_hX,dL_hY)=B_{[g]}(X,Y).
\end{equation}
From this and equation \eqref{eq:J_dL}, one can see the $G$-invariance of $\omega^M$:
\begin{equation}
\begin{split}
   \omega^M_{[hg]}\big((dL_h)_{[g]}X, (dL_h)_{[g]}Y\big)&=B_{[hg]}(JdL_hX,dL_hY)
                                                =B_{[hg]}(dL_h J X,dL_hY)\\
						&=B_{[g]}(JX,Y)
						=\omega^M_{[g]}(X,Y).
\end{split}
\end{equation}
The formulation of the $G$-invariance is
\begin{equation}
   \omega^M_{[hg]}\Big( (dL_h)_{[g]}X, (dL_h)_{[g]}Y \Big)=\omega^M_{[g]}(X,Y).
\end{equation}

\subsection{The Chevalley cohomology}
%------------------------------------

Let $\mG$ be a Lie algebra (maybe infinite dimensional) and $(V,\rho)$ a representation of $\mG$ on the vector space $V$. The \defe{Chevalley cohomology}{chevalley!cohomology} of $\mG$ associated with the representation $\rho$ is given by the following definitions:

A $p$-cochain is a map $\dpt{C}{\underbrace{\mG\times\ldots\times\mG}_{p \text{ times}}}{V}$ which is multi-linear and skew-symmetric. In particular, a $1$-cochain is a linear map $\dpt{\xi}{\mG}{V}$. In the case of the trivial representation on $\eR$, a $1$-cochain is an element of $\mG^*$. The coboundary of a $p$ cochain is the $p+1$-cochain given by
\begin{equation}
\begin{split}
(\delta C)(X_0,\ldots,X_p)=&\sum_{i=0}^{p}(-1)^i\rho(X_i)C(X_0\ldots,\hX_i,\ldots,X_p)\\
                           &+\sum_{i<j}(-1)^{i+j}C\big(  [X_i,X_j],\ldots,\hX_i,\ldots,\hX_j,\ldots,X_p \big).
\end{split}
\end{equation}
The main property is $\delta^2=0$. The others definitions are as usual: a $p$-cocycle is a $p$-cochain $C$ such that $\delta C=0$, a $p$-coboundary is a $p$-cochain which can be written as $\delta B$ for some  $(p-1)$-cochain $B$. Finally, the cohomology classes are:
\begin{equation}
    H^{p}_{(V,\rho)}=\frac{p\text{-cocycles}}{p\text{cochain}}=H^p_{\rho}(\mG,V).
\end{equation}

When one consider the trivial representation, i.e. $\rho(X)=0$, a $1$-cochain is $\xi\in\mG^*$ and
\begin{equation}  \label{EqDefcochaintrivC}
(\delta\xi)(X,Y)=-\xi([X,Y]).
\end{equation}

\begin{probleme}
Au cas où ça t'intéresserait, je te dis que le signe moins, tu ne l'as ajouté qu'en février 2007. T'étonnes pas si y'a des signes qui foirent plus bas.
\end{probleme}

Now, on the symmetric hermitian space $M=G/K$, one defines a $\Omega\in\Lambda^2(\mG^*)$ by
\begin{subequations}
\begin{align}
   \Omega(X,Y)&=B(JX,Y)&\text{for }X,Y\in\mP  \label{eq:def_Omega_1}    \\
   \Omega(\mK,\mG)&=0.                          \label{eq:def_Omega_2}
\end{align}
\end{subequations}

A great property of this definition is that $\Omega$ is a $2$-cocycle for the trivial representation of $\mG$ on $\eR$:
\[
\Omega([X,Y],Z)+\Omega([Y,Z],X)+\Omega([Z,X],Y)=0.
\]

Indeed, if $X$ ,$Y$, $Z\in\mP$, the commutators are in $\mK$, so that \eqref{eq:def_Omega_2} makes the whole null. The second case is $X$, $Y\in\mP$ and $Z\in\mK$; for this, we are led to consider the quantity $-B( [Y,Z],JX )-B([Z,X],JY)$. The first term can be transformed as:
\[
\begin{aligned}
  B([Y,Z],JX)&=-B(J[Y,Z],X)&&\text{by def. \eqref{eq:herm_2} } \\
             &=B([Z,JY],X)&&\text{by def.  \eqref{eq:herm_3}}\\
             &=-B(JY,[Z,X])&&\ad-\text{invariance of } B\\
	     &=-B([Z,X],JY).
\end{aligned}
\]
So it is zero.

\begin{lemma}[Whitehead's lemma]
If $\mG$ is  a finite dimensional semisimple  Lie algebra and $\rho$ a non trivial\quext{Ce qui n'est pas le cas ici} representation of $\mG$ on $V$, then $\forall\,q\geq 0$,
\[
      H^q(\mG,V)=0.
\]
\end{lemma}

This gives us the existence of a $\xi_0\in\mG^*$ (an Chevalley $1$-cochain) such that
\[
   \delta\xi_0=\Omega.
\]
\begin{equation}\label{eq:Z_0}
   \xi_0=B(Z_0,.).
\end{equation}

\begin{definition}
	The \defe{center}{center!of a Lie algebra}\nomenclature{$\mZ(\mG)$}{Center of a Lie algebra} of the Lie algebra $\mG$ is the set $\mZ(\mG)\subset\mG$ of elements $Z$ such that $[Z,X]=0$ for every $X\in\mG$. See also the definition of a centralizer on page \pageref{PgDefCentralisateur}.
\end{definition}

\begin{proposition}
The $Z_0$ defined in \eqref{eq:Z_0} and the $J$ of proposition~\ref{prop:ext_J} satisfy
\begin{subequations}
\begin{align}
   Z_0&\in\mZ(\mK),\\
   J&=\pm\ad(Z_0)|_{\mP}.
\end{align}
\end{subequations}


\end{proposition}

\begin{proof}
First, we see that for any $K\in\mK$, $[Z_0,K]=0$. We know from \eqref{eq:def_Omega_2} that $\forall\,G\in\mG$, $K\in\mK$, $\Omega(K,G)=0$, or
\begin{equation}
\begin{split}
  0=\delta B(Z_0,.)(K,G)=-B(Z_0,.)([K,G])
                       =B([K,Z_0],G),
\end{split}
\end{equation}
thus $[K,Z_0]=0$ because $B$ is nondegenerate. We will see below that $Z_0\in\mP$ is not possible.  On the other hand, the condition \eqref{eq:def_Omega_1} gives us
\[
  B( [X,Z_0],Y )=B(JX,Y)
\]
for any $Y\in\mP$. Thus $[X,Z_0]=JX$ and the second claim follows. Let us now see that $Z_0\in\mP$ is not possible (and so we finish the proof of the first claim). We know that $J^2X=[Z_0,[Z_0,X]]$, but for $Z_0$, $X\in\mP$, $[Z_0,X]\in\mK$ and so $J^2X=0$ which is not possible.

\end{proof}


\begin{lemma}
A symmetric space $G/K$ is hermitian if and only if $\mZ(\sK)\neq 0$.
\end{lemma}

\begin{proof}
If the space is hermitian, we just said that the $J$ can be written under the form $J=-\ad(Z_0)|_{\sP}$ for a $Z_0\in\mZ(\sK)$. For the sufficient condition, we define $J=-\ad(Z_0)$ for a certain $Z_0\in\mZ(\sK)$. As a first point for all $k\in\sK$ and $p\in\sP$,
\begin{equation}
\begin{split}
(J\circ\ad k)p=[ [k,p],Z_0]
              =-[ [p,Z_0],k]-[ [Z_0,k],p]
              =[k,[p,Z_0]]
              =(\ad k\circ J)p
\end{split}
\end{equation}
The two other points are
\begin{subequations}
\begin{align}
  J^2=(-\ad Z_0)[X,Z_0]=-[ [Z_0,X],Z_0]
\intertext{and}
  B([Z_0,X],[Z_0,Y])=-B(X,[ Z_0,[Z_0,Y]])=B(X,Y)
\end{align}
\end{subequations}
These are true if $[ [Z_0,X],Z_0]=X$\quext{Mais je ne vois pas comment obtenir \c ca. Si $\ad Z_0$ est un automorphisme de $\sP$, alors je suis d'accord.}

Let us prove that $\mM:=\ker(\ad Z_0)=0$. For remark that $(\sG,B)$ is a Riemannian space and let $W$ be the orthogonal complement of $\mM$ in $\sP$. We begin to prove that $\mM$ is $(\ad\sK)$-invariant.

If $x\in\ket Z_0$ and $k\in \sK$, then
\[
  [Z_0,[k,x]]=-[k,[x,Z_0]]-[x,[Z_0,k]]=0
\]
because $[x,Z_0]=0=[Z_0,k]$. Now if $\mM$ is $(\ad\sK)$-invariant, then $W=\mM^{\perp}$ is too because
\[
  B([k,x],m)=-B(x,[k,m])=0
\]
since $[k,x]\in\mM$. So we have the orthogonal direct sum $\sP=\mM\oplus W$. We are now going to see that $[\mM,W]=0$. Let $X,X'\in\sP$;
\begin{equation}
  B\big( [ [m,w],X],X' \big)=B\big( [m,w],[X,X'] \big)
		=B\big( w,[ [X,X'],m] \big)
		=0
\end{equation}
since $[X,X']\in\sK$ and $[ [X,X'],m]\in\mM$ from the $\sK$invariance of $\mM$. If we define $A=[m,w]$, the endomorphism $\ad(A)|_{\sP}$ is zero.

We know\quext{Il faut encore voir d'où sort ce truc.} that $[\sK,\sK]=\sP$, and then that
\[
  [A,k]=[A,[p,p']]=-[p,[p',A]]-[p',[A,p]]=0.
\]
So $[A,\sG]=0$ and $A=0$ because $\sG$ is semisimple. If we write $\sG=[\sP,\sP]\oplus\sP$, we find
\[
  \sG=\big( [\mM,\mM]\oplus\mM \big)\oplus\big( [W,W]\oplus W \big),
\]
where the two brackets commute. It furnish a decomposition of $\sG$ into ideals which impossible from the semi-simplicity assumption. We conclude that $\mM=$ and that $\ad Z_0$ is bijective on $\sP$.

\end{proof}


\begin{lemma}
Let $\sG$ a simple Lie algebra with Iwasawa decomposition $\sG=\sK\oplus\sA\oplus\sN$. We suppose that $\mZ(\sK)\neq 0$. Then $\dim\sA\geq\dim\mZ(\sN)$.
\end{lemma}

\begin{proof}
Let $\dpt{i}{\sR}{\sG}$ be the canonical projection and $\xi_0\in\sG^*$ such that $\delta\xi_0=\Omega$. Since $\sK\cap \sR=\{ 0 \}$, the radical of $\delta(i^*\xi_0)$ is trivial. Indeed, when $X\in\sR$, we have $(i^*\xi_0)X=\xi_0 X$ and equation $\delta(i^*\xi_0)(X,Y)=0$ for all $X$, $Y\in \sR$ gives $B(JX,Y)=0$ because $\sK\cap\sR=\{ 0 \}$. Then $JX=0$ and $X=0$.\quext{\c Ca demande que $B$ soit non d\'eg\'en\'er\'ee sur $\sR$, et je ne vois pas trop pourquoi ce serait vrai.}.

Let $V$ be the radical of $\Omega$ in $\sN$; if $z\in\mZ(\sN)$, then $\Omega(\sN,z)=(\delta\xi_0)(z,\sN)=\xi_0[z,\sN]=0$. Then $\mZ(\sN)\subset V$. Now let us consider the map $\dpt{\psi}{V}{\sA^*}$,
\[
  \psi(v)=\Omega(v,.)|_{\sA}.
\]
Let us prove that $\psi$ is injective. For, we consider a $v\in V$ such that $\Omega(v,\sA)=0$. Since $v\in V$, we have $[v,\sN]=0$ and then $\Omega(v,\sN)=0$. So,
\[
  0=\Omega(v,\sA\oplus\sN)=\delta(i^*\xi_0)(v,\sR),
\]
 and then $v=0$ because the radical of $i^*\xi_0$ in $\sR$ is only zero. Consequently,
\[
  \dim\sA=\dim\sA^*\geq \dim V\geq\dim\mZ(\sN)
\]
because there exists an injection from $V$ into $\sA^*$ and $\mZ(\sN)\subset V$.

\end{proof}


\begin{lemma}
Let us suppose that $\dim\sA=1$ and $\dim\sG\geq 3$. Then

\begin{enumerate}
\item The root system is $\Phi=\{ \pm\alpha,\pm 2\alpha \}$,
\item $\sN=\sG_{\alpha}\oplus\sG_{2\alpha}$ and $\sG_{2\alpha}=\mZ(\sN)$
\item $\dim\mZ(\sN)=\dim\sA=1$
\item There exists a $E\in\mZ(\sN)$ such that $[x,y]=\Omega(x,y)E$ for all $x$, $y\in\sN$. The subspaces $\sA\oplus\sN$ and $\sG_{\alpha}$ are symplectic and orthogonal in $(\sR,\Omega)$. In particular, $\sN$ is an Heisenberg algebra.
\end{enumerate}

\end{lemma}

\begin{proof}
No proof.
\end{proof}

This lemma allows us to parametrize $\sR$ as
\[
  r=aA+x+zE
\]
with $x\in\sG_{\alpha}$ and $a\in \sA$ because $\mZ(\sN)$ is spanned by the unique element $E$. Now if we consider a function $u\in C^{\infty}(\sR)$, we can define a partial Fourier transform
\[
  F(u)(a,x,\xi)=\hat u(a,x,\xi)=\int_{\mZ(\sN)}e^{-i\xi z}u(aA+x+zE)dz.
\]

\begin{theorem}
Let consider the diffeomorphism $\dpt{\phi_{\hbar}}{\sR}{\sR}$ given by
\[
   \phi_{\hbar}(a,x,\xi)=\left( a,\frac{1}{\cosh(\frac{\hbar\xi}{2})}x,\frac{\sinh(\hbar\xi)}{\hbar} \right).
\]
Then

\begin{enumerate}
\item $\phi_{\hbar}^*\swS(\sR)\subset\swS(\sR)$,
\item $(\phi^{-1}_{\hbar})^*\swS(\sR)\subset\swS'(\sR)$.
\end{enumerate}
where $\swS$ and its dual $\swS'$ are defined in section~\ref{sec:Distrib}.

\end{theorem}

We recall the notation for functions: $\varphi^*f=f\circ\varphi$.

\begin{proof}
For sake of simplicity, we forget about variable $a$, we pose $y=\hbar \xi$ and we look at the function $\dpt{\phi}{\eR^2}{\eR^2}$ given by
\[
  \phi(x,y)=(\sech(\frac{y}{2})x,\sinh(y)).
\]
Formula
\[
  \frac{\sqrt{2}}{2}(1+\sqrt{1+y^2})^{1/2}=\cosh\big( \frac{\arcsinh(y)}{2} \big),
\]
allows us to write
\begin{equation}
\phi^{-1}(x,y)=\left( \cosh\Big( \frac{\arcsinh(y)}{2} \Big)x,\arcsinh(y) \right).
\end{equation}
We pose $p_{nm}(x,y)=x^ny^m$ and we are going to study
\[
  (p_{nm}\circ\phi^{-1})(x,y).
\]
It has a polynomial grown because, for large $y$, $\sinh(\ln y)=\frac{1}{2} y$. Hence $\arcsinh(y)\simeq \ln(2 y)$. The matrix of $d\phi_{(x,y)}$ is given by
\begin{equation}
d\phi_{(x,y)}=
\begin{pmatrix}
\sech(\frac{y}{2}) & -\frac{x}{2}\tanh(\frac{y}{2})\sech(\frac{y}{2})\\
0                  &   \cosh(y)
\end{pmatrix}.
\end{equation}
Since $\phi$ is a diffeomorphism, and then is bijective,
\begin{equation}
\begin{split}
  \sup_{a\in\eR^2}| p_{nm}(a)(u\circ \phi)(a) |&=\sup_{a\in\eR^2}| p_{nm}(\phi^{-1}(a))u(a) |\\
                                               &\leq \sup_{a\in\eR^2}| P_{MN}(a)u(a) |
\end{split}
\end{equation}
for a choice of $N$, $M\in\eN$. In order to check the derivatives, we need the asymptotic behaviour of $(u\circ\phi)'(x,y)$ given components of\quext{Pour moi, la composante $a=i=2$ ne fonctionne pas parce que c'est $(\partial_2)_{\phi(x,y)}\cosh(y)$.}
\[
  \partial_a(u\circ\phi)(x,y)=\sum_i(\partial_iu)_{\phi(x,y)}(\partial_a\phi_i)(x,y).
\]
The derivatives of $\phi^*u$ are the quantities
\[
  \partial_a(u\circ\phi^{-1})(x,y)=(\partial_iu)(\phi^{-1}(x,y))\partial_a(\phi_i^{-1})(x,y).
\]
This has a polynomial behaviour. One can see recursively that the same is true for second derivatives $\partial^2_{ab}(u\circ\phi^{-1})$ and higher. This proves that $\phi^*\swS(\eR^2)\subset\swS(\eR^2)$.

In order to see that $(\phi^{-1})^*u\in\swS'$ when $u\in\swS$, we have to prove that
\begin{equation} \label{eq:r1181205}
  \int_{\mU}| x^{-N}y^{-M}(\phi^{-1})^*u(x,y)dxdy |<\infty
\end{equation}
for a choice of $N,M$. Here, $\mU$ is the complement in $\eR^2$ of a compact neighbourhood of the origin. Indeed, the fact for $f$ to belongs to $\swS'$ is the \emph{distribution} $T_f$ to belongs to $\swS'$. In other words, the condition $f\in\swS'$ is the continuity of $\varphi\to\int_X f\varphi$ when $\varphi\in\swS$. The essentially resides in the existence of the integral.

In general -- here, $f$ take the role of $\phi^*u$ -- we have
\[
  | \int_X f\varphi |\int | f\varphi |\leq \int | f p_{-N,-M} |
\]
for all $N,M\geq 0$ because $\varphi$ decrease more rapidly than any polynomial. If we find $M$ and $N$ such that $\int | fp_{-N,-M} |<\infty$, then we prove that the distribution belongs to $\swS'$. In our case more precisely, we know that $\varphi$ is smooth. Then it can be majored in any compact set. This is the reason why we write an integral over $\mU$ instead of the whole $\eR^2$.

In equation \eqref{eq:r1181205}, we perform the change of variable $a'=\phi^{-1}(a)$:
\begin{equation}
\begin{split}
\int_{\mU}| x^{-N}y^{-M}(u\circ\phi^{-1})&(a)|\,da=\int_{\mU'}| \frac{1}{p_{MN}(\phi(a'))}u(a') | |J_{\phi}(a') |\,da'\\
                        &=\int_{\mU'}\frac{| u(a) |}{\left|  \big( \frac{x}{\cosh(\frac{y}{2})} \big)^N\sinh(y)^M  \right|}
                                     \left|  \frac{\cosh(y)\cosh(\frac{y}{2})}{}  \right|da\\
                        &=\int_{\mU'}\left| \frac{1}{x^N}2^{1-M}\cosh(\frac{y}{2})^{N-M}\sinh(\frac{y}{2})^{1-M}\right|\, |u(a) |\,da.
\end{split}
\end{equation}
The latter integral is finite when $M\geq 1$ and $M>N$.\quext{Pierre trouve d'autres choses, mais sa conlusion est la même; comme s'il utilisait un autre formulaire de trigono hyperbolique que moi.}

The same kind of upper bound\quext{Traduction de «majoration»} holds for the derivatives of $u\circ\phi^{-1}$ for which we have to study the behaviour of the inverse matrix $(d\phi)^{-1}$. All this proves that $(\phi^{-1})^*\swS(\eR^2)\subset \swS(\eR^2)$.

\end{proof}

\subsection{Involutive symmetric Lie algebras}
%----------------------------------------------

\begin{definition}
An \defe{involutive Lie algebra}{involutive!Lie algebra} is a doublet $(\mG,\sigma)$ where $\mG$ is a real finite dimensional Lie algebra and $\dpt{\sigma}{\mG}{\mG}$ is an involutive automorphism of $\mG$.
\end{definition}

There are three types of triples $(\mG,\sigma,\Omega)$:
\begin{enumerate}

	\item
		the symplectic triple,\index{triple!symplectic}
	\item
		the exact triple, \index{triple!exact}
	\item
		the elementary solvable exact triple (ESET). \index{triple!elementary solvable}
\end{enumerate}
In these three types, $(\mG,\sigma)$ is an involutive Lie algebra. The following definitions can be found in \cite{StrictSolvableSym}.

Symplectic triples were already defined in section~\ref{SubSecTripleSylple}.
\begin{definition}
An \defe{exact triple}{exact triple} is a triple $(\mG,\sigma,\Omega)$ such that
\begin{enumerate}
\item $\mG\stackrel{\sigma}{=}\mK\oplus\mP$ and $[\mP,\mP]=\mK$,
\item $\Omega$ is a Chevalley $2$-coboundary such that $i(\mK)\Omega=0$ and $\Omega|_{\mP\times\mP}$ is a symplectic structure on $\mP$.
\end{enumerate}
\end{definition}
The exact triple has the following differences compared to the symplectic one:
\begin{itemize}
\item $\Omega$ is a coboundary instead as a cocycle,
\item $\Omega|_{\mP\times\mP}$ is not only nondegenerate, but also symplectic.
\end{itemize}
From definition of a coboundary, in an exact triple, there exists a $\xi\in\mG^*$ such that $\Omega=\delta\xi$.

\begin{definition}
An \defe{elementary solvable exact triple}{elementary!solvable exact triple}\index{ESET} (ESET) is an exact triple $(\mG,\sigma,\Omega)$ such that
\begin{enumerate}
\item The Lie algebra $\mG$ is a split extension of abelian algebras:
\begin{equation}   \label{EqSplitmGABab}
  \mG=\mA\oplus_{\rho}\mB.,
\end{equation}
\item the automorphism $\sigma$ preserves the vector space decomposition $\mG=\mA\oplus\mB$.
\end{enumerate}

\end{definition}

\begin{remark}
When one writes $\mG=\mA\oplus_{\pi}\mB$, one has $\pi\colon \mA\to \Der(\mB)$. This is the inverse convention of the one chosen in the article \cite{StrictSolvableSym}.
\end{remark}

 In the case of an ESET, we have $\mA\cap\mK\subset\mA\cap[\mG,\mG]$ because $\mK$ is equal to $[\mP,\mP]$ and is thus included in $[\mG,\mG]$. But $[\mG,\mG]$ can be $[a,b]$, $[a,b]$ or $[b,b]$. The two latter are zero (because $\mA$ and $\mB$ is abelian) and, by definition of the split extension, $[a,b]=\rho(a)b\in\mB$. So $\mA\cap[\mG,\mG]=0$. Therefore,
\[
  \mA\cap\mK=0;
\]
we deduce that $\mA\subset\mP$ and $\mK\subset \mB$. Since $\mK\subset\mB$, we define $\mL$ as the complement:
\[
  \mB=\mK\oplus\mL.
\]
In particular, $\mK$ and $\mL$ are abelian.

The dimension\index{dimension!of a symplectic triple} of a triple is the dimension of $\mP$ and two triples $(\mG_i,\sigma_i,\Omega_i)$ are \defe{isomorphic}{isomorphism!of symplectic!triple} if there exists a Lie algebra isomorphism $\dpt{\psi}{\mG_1}{\mG_2}$ such that $\psi\circ\sigma_1=\sigma_2\circ \psi$ and $\psi^*\Omega_2=\Omega_1$.

\subsection{Symplectic symmetric spaces and involutive Lie algebra}
%--------------------------------------------------------------------

Let $(M,\omega,s)$ be a symplectic symmetric space; we associate an involutive Lie algebra $(\lG,\sigma)$ in the following way (we omit some non trivial proofs). Let $o\in M$, $G$ the transvection group and $H$, the stabiliser of $o$ in $\Aut(M,\omega,s)$ and $K=G\cap H$. We consider the map
\begin{equation}
	\begin{aligned}
		\tilde\sigma\colon \Aut(M,\omega,s)&\to \Aut(M,\omega,s) \\
		\tilde\sigma(g)&=s_o\circ g\circ s_o.
	\end{aligned}
\end{equation}

Let $\mG$ be the Lie algebra of the group $G$ and $\dpt{\sigma}{\mG}{\mG}$ the induced involutive automorphism from $\tilde\sigma$. Now, $(\mG,\sigma)$ is an involutive Lie algebra. We have a natural projection $\dpt{\pi}{G}{M}$ because $H$ stabilises $o$, so that $K=G\cap H$ is the stabiliser of $o$ in $\Aut(M,\omega,s)$ which is transitive on $M$. Then $M=G/K$ as homogeneous spaces. One can see that $(\mG,\sigma,\pi^*(\omega_o))$ is a symplectic triple.

The precise proposition is the following.

\begin{proposition}
There exists a bijection between symplectic simply connected symmetric spaces and symplectic triples. This bijection is given up to isomorphism.
\end{proposition}

\subsection{Symmetric spaces and coadjoint orbits}
%-------------------------------------------------

We are now going to describe $(M,\omega,s)$ as coadjoint orbit on $\mG^*$. When a Lie group $G$ of symplectomorphism acts on a symplectic manifold $(M,\omega)$, we say that the action is weakly Hamiltonian\index{Hamiltonian!action} if there exists $\dpt{\mu_X}{M}{\eC}$ such that $i(X^*)\omega=d\mu_X$. If $\dpt{\mu}{\mG}{ C^{\infty}(M)}$ is a Lie algebra homomorphism, we say that the action is Hamiltonian and we usually write $\lambda$ instead of $\mu$.

\begin{proposition}
Let $(\mG,\sigma,\Omega)$ be a simple triple, $(M,\omega,s)$ the associated symmetric simply connected symplectic space and $G$, the transvection group. Then

\begin{enumerate}
\item The action of the transvection group on $M$ is Hamiltonian if and only if there exists a $\xi\in\mG^*$ with $\Omega=\delta\xi$ for the Chevalley cohomology\index{Chevalley!cohomology}.
\item In this case, $(M,\omega,s)$ is a $G$-equivariant symplectic covering of the coadjoint orbit of $\xi$ in $\mG^*$.
\end{enumerate}

\end{proposition}

\begin{definition}
A \defe{symmetric symplectic space}{symmetric!symplectic space} is a triple $(M,\omega,s)$ where
\begin{itemize}
\item $M$ is a connected smooth ($\Cinf$) manifold,
\item $\omega$ is a symplectic form on $M$,
\item $\dpt{s}{M\times M}{M}$ is a smooth map which we write with the notation $s_x(y):=s(x,y)$.
\end{itemize}
These elements must satisfy the following conditions:
 \begin{enumerate}
 \item $\forall x\in M$, $s_x$ is an involutive symplectic diffeomorphism of $(M,\omega)$ which is called the \defe{symmetry}{symmetry} at $x$,
 \item $\forall x\in M$, $x$ is an isolated fixed point of $s_x$,
 \item $\forall x,y\in M$, $s_xs_ys_x=s_{s_x(y)}$.
\end{enumerate}

\end{definition}


\begin{definition}
Two symplectic symmetric spaces $(M,\omega,s)$ and $(M',\omega',s')$  are \defe{isomorphic}{isomorphism!of symplectic!symmetric spaces} if there exists a symplectic diffeomorphism $\dpt{\varphi}{(M,\omega)}{(M',\omega')}$ such that
\begin{equation}
  \varphi\circ s_x=s'_{\varphi(x)}\circ\varphi.
\end{equation}

\end{definition}


\begin{definition}
An \defe{exact triple}{exact triple} is a triple $(\mG,\sigma,\Omega)$ such that

\begin{enumerate}
\item $(\mG,\sigma)$ is an involutive Lie algebra with $[\mP,\mP]=\mK$ if $\mG=\mK\oplus\mP$ is the decomposition of $\mG$ with respect to $\sigma$.

\item $\Omega$ is a Chevalley $2$-coboundary such that $i(\mK)\Omega=0$ and $\Omega_{\mP\times\mP}$ is symplectic.

\end{enumerate}

\end{definition}

From definition, there exists a $\xi\in\mG^*$ for which $\Omega=\delta\xi$. We can choose it in such a way that $\xi(\mP)=0$; in this case we say that $\xi\in\mK^*$ by abuse of notation. Indeed, put $\Omega=\delta\xi$ with $\xi=\xi'+\eta'$ where $\xi'\in\mK^*$ and $\eta'\in\mP^*$. If we consider $k\in\mK$ and $B=B_k+B_p\in\mG$, using $i(\mK)\Omega=0$, we find
\[
  0=\Omega(k,B)=-\xi'[k,B_k]-\eta'[k,B_p].
\]
Taking $B_k=0$, we find $\eta'[\mK,\mP]=0$ while with $B_p=0$, we find $\eta'[\mK,\mK]=0$. Moreover $\eta'[\mP,\mP]=\eta'\mK=0$. Then an acceptable $\eta'$ must satisfy $\eta'[\mG,\mG]=0$, so that
\[
  \Omega[A,B]=-\xi'[A,B]
\]
which proves that the $\eta'$ part of $\xi$ has no importance; we can choose it as zero.


Let $(\mG,\sigma)$ be an involutive Lie algebra associated with a triple $(M,\omega,s)$ with transvection group $G$. If $(\mG,\sigma,\Omega)$ is exact, $\mZ(\mG)\subset \mK$ because $[Z,p]=0$ for all $p\in\mP$ whenever $Z\in\mP\cap\mZ(\mG)$. Then $\Omega(Z,p)=0$ for all $p\in \mP$ which is not possible from non degeneracy of $\Omega$.

\subsection{Elementary solvable symmetric spaces}
%-----------------------------------------------

Let $(M,\omega,s)$ a symmetric space with associated triple $(\mG,\sigma,\Omega)$. The space $M$ is \defe{elementary solvable}{elementary!solvable!exact triple} if

\begin{enumerate}
\item $\mG$ is a split extension  (see subsection~\ref{subsec:semi_Lie}) of two abelian algebras $\mA$ and $\mB$,
\item the automorphism $\sigma$ preserves the decomposition $\mG=\mB\oplus\mA$.
\end{enumerate}
Since $\mK=[\mP,\mP]$, we have
\[
  \mA\cap\mK\subset\mA\cap[\mG,\mG]=0.
\]
Indeed, let $\dpt{\rho}{\mA}{\Der\mB}$ be the split homomorphism; the commutator on the split extension is defined by
\[
  [A,B]=\rho(A)B\in\mB.
\]
Then $[\mG,\mG]\subset\mB$. All this shows that $\mA\subset\mP$. So there exists a $\mL\subset\mP$ such that $\mB=\mK\oplus\mL$. Let us show that $\mL$ is abelian.
\[
  0=[\mB,\mB]=[\mK,\mK]+[\mK,\mL]+[\mL,\mK]+[\mL,\mL].
\]
The three first terms are in $\mP$ while the last one is included in $\mK$. The identical annihilation of the sum imposes $[\mL,\mL]=0$.

\subsection{Mid-point map}
%-------------------------

Let us now take an ESET $(\mG,\sigma,\Omega)$ and its corresponding ESSS $(M,\omega,s)$. There exists a $\xi\in\mG^*$ such that $\Omega=\delta\xi$ and we define
\begin{equation}
\begin{aligned}
 \zeta\colon \mA\times\mL&\to \eR \\
(a,l)&\mapsto \xi(\sinh(a)l)
\end{aligned}
\end{equation}
where $\sinh(a)l$ has to be understood as $\frac{ 1 }{2}( e^{\rho(a)}- e^{-\rho(a)})l$ with $\rho(a)\in\End(\mB)$ being the splitting homomorphism of \eqref{EqSplitmGABab}.
\begin{proposition}
Let $(M,\omega,s)$ be a ESSS and $\omega=\Omega=\delta\xi$ the symplectic form of the corresponding ESET. The \defe{mid-point map}{mid-point map} $M\to M$, $x\mapsto x/2$ defined by
\[
  s_{x/2}o=x
\]
is globally defined if and only if $\phi$ is a diffeomorphism.
\end{proposition}
Notice that the affirmation $\omega=\Omega=\delta\xi$ means that one has a symplectic form $\omega\colon \mA\times\mL\to \eR$,
\[
  \omega(a,l)=\Omega(a,l)=-\xi\big( [a,l] \big).
\]

\subsection{Kähler structures}
%-------------------------------

\begin{definition}  \label{DefKONtphK}
    If $M$ posses an almost complex structure\footnote{Definition~\ref{DefCLtjFtD}.} $J$ and a Riemannian metric $g$, we say that the metric is \defe{hermitian}{hermitian!metric} when
    \begin{equation}
       g(JX,JY)=g(X,Y).
    \end{equation}
\end{definition}

Notice that a symmetric space must be hermitian (definition~\ref{def:hermitien}), hence equation \eqref{eq:herm_3}, implies the integrability condition.

\begin{definition}
If one has an almost complex structure with an hermitian metric such that $\nabla J=0$, then $(M,J,g)$ is a \defe{Kähler manifold}{kähler!manifold}.
\end{definition}

\begin{remark}
$\nabla J=0$ reads $\forall X,Y\in M$,
\[
    (\nabla_XJ)(Y)=\nabla_X(JY)-J(\nabla_XY)=0.
\]
\end{remark}

\begin{remark}
By ``$\nabla$''\ we mean the Levi-Civita connection for $g$. In particular it is torsion free:
\[
   \nabla_XY-\nabla_YX=[X,Y].
\]
\end{remark}

\begin{lemma}
If $(M,g,J)$ is a Kähler manifold, then $J$ is integrable.
\end{lemma}

\begin{proof}
From the formula $\nabla_Z(JY)=(\nabla_ZJ)Y+J\nabla_ZY$ and the fact that $\nabla J=0$, we know that
\begin{equation}\label{eq:inter_1}
  \nabla_Z(JY)=J(\nabla_ZY),
\end{equation}
while the torsion-free condition for $\nabla$ gives
\begin{equation}\label{eq:inter_2}
\nabla_XY-\nabla_YX=[X,Y].
\end{equation}
With these two, we find $\nabla_Z(JY)-\nabla_Y(JZ)=J\nabla_ZY-J\nabla_YZ=J[Z,Y]$.  Writing it with $JZ$ instead of $Z$,
\[
   \nabla_{JZ}(JY)+\nabla_Y(Z)=J[JZ,Y].
\]
The anti-symmetric part of this equation gives
\[
   \nabla_{JZ}(JY)+\nabla_YZ-J[JZ,Y]-\nabla_{JY}(JZ)-\nabla_ZY+J[JY,Z].
\]
Using \eqref{eq:inter_1} and \eqref{eq:inter_2}, one finds the thesis.

\end{proof}

When $(M,g,J)$ is a Kähler manifold, one defines the \defe{Kähler $2$-form}{kähler!$2$-form} by
\begin{equation}
\omega(X,Y):=g(X,JY).
\end{equation}

\begin{proposition}
The Kähler $2$-form is a symplectic structure on $M$.
\end{proposition}

\begin{proof}
Since $g$ is nondegenerate and $JX=0$ implies $X=0$, it is clear that $\omega$ is nondegenerate. The antisymmetry of $\omega$ is because the metric is hermitian. The only point is to see that $d\omega=0$.

From \eqref{eq:d_omega_nabla} which gives $d\omega$ in terms of $\nabla\omega$, we see that we just have to prove that $\nabla\omega=0$. By definition,
\begin{align*}
(\nabla_Z\omega)(X,Y)&=Z(\omega(X,Y))-\omega(\nabla_ZX,Y)-\omega(X,\nabla_ZY)\\
                     &=Zg(X,JY)-g(\nabla_ZX,JY)-g(X,J\nabla_ZY).\\
                     &=(\nabla_Zg)(X,JY)=0
\end{align*}
because the vanishing of $\nabla J$ implies that $J(\nabla_ZY)=\nabla_Z(JY)$.
\end{proof}

\subsection{Symplectic structure on the Iwasawa component}\index{symplectic!on $R$}
%-------------------------------------

The Iwasawa theorem gives us a global diffeomorphism between $R=AN$ and $M=G/K$ by $\dpt{\varphi}{R}{G/K}$, $\varphi(an)=[an]$. But one has a symplectic form on $M$: $\omega^M_x(X,Y)=g_x(JX,Y)$. So, $R$ has also a symplectic form defined by, $\forall\,X,Y\in T_{an}R$,
\begin{equation}
\omega^R=\varphi^*\omega^M,
\end{equation}
or more explicitly:
\[
  \omega^R_{an}(X,Y)=\omega^M_{\varphi(an)}(d\varphi_{an}X,d\varphi_{an}Y).
\]

\begin{proposition}
This symplectic form is $R$-invariant under the left action; in other words,  $\forall r\in R$,
\begin{equation}
\omega_{ran}^R\Big(  (dL_r)_{an}X,(dL_r)_{an}Y  \Big)=\omega^R_{an}(X,Y).
\end{equation}

\end{proposition}

\begin{proof}
For a $r\in R$, we want to looks at
\begin{equation}
\omega^R_{ran}(dL_rX,dL_rY)=\omega^M_{[ran]}(d\varphi_{ran}dL_rX,d\varphi_{ran}dL_rY)\\
\end{equation}

But we know the invariance of $\omega^M$:
\[
  \omega^M_{[hg]}(dL_hX,dL_hY)=\omega^M_{[g]}(X,Y),
\]
Now, let us show that $d\varphi_{ran}dL_rX=dL_rd\varphi_{an}X$. For this, we consider a path which gives $X\in T_{an}R$: $X(t)\in R$, $X(0)=an$. So,
\begin{equation}
  d\varphi_{ran}(dL_r)_{an}X=\Dsdd{[rX(t)]}{t}{0}
                        =\Dsdd{L_r[X(t)]}{t}{0}
			=(dL_r)_{[an]}d\varphi_{an}X.
\end{equation}
Finally,
\begin{equation}
\begin{split}
\omega_{ran}^R(dL_rX,dL_rY)&=\omega^M_{[ran]}(d\varphi_{ran}(dL_r)_{an}X,d\varphi_{ran}(dL_r)_{an}Y)\\
                           &=\omega^M_{[ran]}( (dL_r)_{[an]}d\varphi_{an}X,(dL_r)_{[an]}d\varphi_{an}Y )\\
			   &=\omega^M_{[an]}(d\varphi_{an}X,d\varphi_{an}Y)\\
			   &=\omega^R_{an}(X,Y).
\end{split}
\end{equation}

\end{proof}


\subsection{Iwasawa coordinates}
%-------------------------------

We consider $M=G/K$, an hermitian symmetric space (irreducible of non-compact type\quext{je ne sais pas ce que \c{c}a veut dire, mais je ne sais pas non plus o\`u on l'utilise. (p. 301 d'Helgason) }). Let us consider a $Z\in\mZ(\mK)$ as before: $\delta B(Z,.)|_{\mP\times\mP}$ is a $\mK$-invariant $2$-form on $\mP$.
There are some remarkable spaces: $\mR=\mA\oplus\mN$, the Lie algebra of $R=AN$; $\mO=\Ad(G)Z\subset\mG$. We consider the following diffeomorphism:
\begin{subequations}
\begin{align}
&\dpt{\mI}{\mA\oplus\mN}{R},  &\mI(a,n)&=e^ae^n,\\
&\dpt{\varphi}{R}{M},             &\varphi(an)&=[an],\\
&\dpt{\phi}{\mR}{\mO},        &\phi(r)&=\Ad(\mI(r))Z,\\
&\dpt{\lambda}{\mO}{M},       &\lambda(\Ad(g)Z)&=[g].
\end{align}
\end{subequations}
Note that $\mR=T_eR=\mA\oplus\mN=T_r\mR$ where $\mR=T_r\mR$ is a standard identification of vector spaces. The symplectic forms on these spaces are naturally defined by
\begin{subequations}
\begin{align}
   \omega^R&={\varphi^*}^{-1}\omega^M\\
   \omega^{\mR}&=\mI^*\omega^R\\
   \omega^{\mO}&=\lambda^*\omega^R
\end{align}
\end{subequations}
%
By the way, the diffeomorphism $\mI$ is called the \defe{Iwasawa coordinates}{Iwasawa!coordinates}.

\begin{proposition}
The map $\phi$ is bijective.
\end{proposition}

\begin{proof}
For the surjective condition, we have to obtain $\Ad(ank)Z$ under the form $\Ad(\mI(A,N))Z$. For this, remark that one can find $K\in\mK$, $A\in\mA$, $N\in\mN$  such that $k=e^K$, $a=e^A$,$n=e^N$, then
\[
  \Ad(e^Ae^N)Z=\Ad(e^Ae^Ne^K)Z=\Ad(ank)Z.
\]

In order to see the injective condition, let us consider $r,r'\in\mA\oplus\mN$ such that
\[
  \Ad(\mI(r))Z=\Ad(\mI(r'))Z.
\]
Then, $\Ad(\mI(r'))^{-1}\circ\Ad(\mI(r))=\id$. This makes $\Ad(e^{-N'}e^{-A'}e^Ae^N)=id$, so that
\[
   e^{-N'}e^{-A'}=\left(e^Ae^N\right)^{-1},
\]
but $\exp$ is a diffeomorphism, then $(A,N)=(A',N')$.

\end{proof}

\begin{lemma}\label{lem:om_O_om_R}
The symplectic forms $\omega^{\mR}$ and $\omega^{\mO}$ are related by
\begin{equation}
\omega^{\mO}=(\phi^{-1})^*\omega^{\mR}.
\end{equation}
\end{lemma}

\begin{proof}
The definitions make that
\begin{equation}
  (\phi^{-1})^*\omega^{\mR}=(\phi^{-1})^*\mI^*\omega^R
                          =(\phi^{-1})^*\mI^*\varphi^*\omega^M,
\end{equation}
so that we just need to see that $\varphi\circ\mI\circ\phi^{-1}=\lambda$. This is true because for any $g\in K$,
\[
   \varphi\circ\mI\circ\phi^{-1}(\Ad(g)Z)=\varphi\circ\mI(\mI^{-1}(g))=\varphi(g)=[g].
\]

\end{proof}

Now, consider $u\in T_r\mR$, with $r=a+n\in\mA\oplus\mN$, and (just for fun) let us compute $d\phi_r(u)$. In the following computation, $u_A$ and $u_N$ denotes the unit vectors in the direction of $\mA$ and $\mN$.
\begin{equation}
\begin{split}
   d\phi_r(u)&=\Dsdd{  \Ad(e^{a+tu_A}e^{n+tu_N})Z  }{t}{0}\\
             &=\Dsdd{ \Ad(e^{tu_A})\Ad(e^{an})Z  }{t}{0}\\
             &\quad+\Ad(e^a)\Dsdd{  \Ad(e^n)\Ad(e^{-n})\Ad(e^{n+tu_N})Z  }{t}{0}\\
	     &=-(u_A^*)_{\phi(r)}\\
	     &\quad+\Ad(e^ae^n)\Dsdd{ \Ad(e^{-n})\Ad( e^{n+tu_N} )Z  }{t}{0}\\
	     &=-(u_A^*)_{\phi(r)}+\Ad(\mI(r))\Dsdd{  \Ad(e^{ CBH(-n,n+tu_N) })Z  }{t}{0},
\end{split}
\end{equation}
where $CBH$ denote the \href{http://en.wikipedia.org/wiki/Baker-Campbell-Hausdorff_formula}{Campbell-Baker-Hausdorff}\index{Campbell-Baker-Hausdorff formula} function defined by
\[
   e^xe^y=e^{CBH(x,y)}.
\]
One maybe knows the formula
\begin{equation}
\Dsdd{  CBH(-n,n+tu_N)  }{t}{0}=F(\ad(n))u_N,
\end{equation}
where $F(\ad(n))$ is defined by the expansion of
\[
F(z)=\frac{1-e^{-z}}{z}
\]
for $z\in\eC$\quext{Il faut encore aller voir dans Duitsermaat les tenants et aboutissants de ce truc.}. Finally,
\begin{equation}
d\phi_r(u)=-(u_A^*)_{\phi(r)}-\left(  \Ad(\mI(r))F(\ad(n))u_N  \right)^*_{\phi(r)}.
\end{equation}
Now, remark that $\Ad(e^a)|_{\mA}=id|_{\mA}$ because $\ad a|_{\mA}=0$ ($\mA$ is abelian) and
the famous lemma~\ref{Ad_e}.

We have \footnote{From proposition 1.1 page 5 in BM}, $\omega_x^{\mO}(X^*,Y^*)=-B(x,[X,Y])$ for $x\in\mO$, $X$, $Y\in\mG$. The lemma~\ref{lem:om_O_om_R} gives us immediately
\[
   (\mI^*\omega^R)_r(u,v)=(\phi^*\omega^{\mO})_r(u,v).
\]
\subsection{Summary of the construction}
%---------------------------------------

We pick\quext{L'existence de ce $\mK$ contre-dit ce que je dis quand une autre question à propos du type non-compact} $Z\in\mZ(\mK)$. Then one defines
\[
  J=\ad(Z)|_{\mP}
\]
and
\[
  \omega^M(X,y)=
\begin{cases}
 B(JX,Y)&\text{If }X,Y\in\mP\\
 0&\text{if }X \text{ or } Y \text{ belong to }\mK.
\end{cases}
\]
The maps $\mI$, $\varphi$, $\phi$ and $\lambda$ between spaces $R$, $\mA\oplus\mN$, $M$, $\mR$ and $\mO$ are designed to propagate the symplectic form from $\omega^M$ to $\omega^{\mR}$, $\omega^R$, $\omega^{\mO}$. The group $R$ acts on each of these spaces and in particular on $\mO$ by the adjoint action. One can prove that $\omega^{\mO}:=\lambda^*(\varphi^{-1})^*\omega^M$ is
\[
  \omega^{\mO}_x(X^*,Y^*)=-B(x,[X,Y])
\]
for all $X$, $Y\in\mR$. In the whole construction, $\sigma$ is the Cartan involution which gives the decomposition
\[
  \sigma=\id|_{\mK}\oplus(-\id)|_{\mP}.
\]
Therefore $\sigma E=-E$ because $E\in\mN\subset\mP$.

The Lie algebra $\mG$ possesses two roots: $\alpha$ and $2\alpha$, so we decompose it as\quext{Le fait d'être de type non compact est peut-être l'absence de composante $K$ pour l'Iwasawa, qu'en penses-tu ?}
\[
  \mG=\mA\oplus\mG_{\alpha}\oplus\mG_{2\alpha}.
\]
We pick $A\in\mA$, $y\in\mG_{\alpha}$ and $E\in\mG_{2\alpha}$. For example, if $B\in \mA$, $[B,y]=\alpha(A)y$.

\subsection{Continuation}
%------------------------

\begin{proposition}
Let $M=G/K$ be an hermitian irreducible symmetric space of non compact type. We suppose that $\dim\mP\geq 4$. We consider the action $\dpt{ \tau }{  \mR\times\mR  }{ \mR }$,
\[
  \tau_g(X)=\mI^{-1}(g\mI(X)).
\]
This action is Hamiltonian for the constant symplectic structure $\Omega$ on $\mR$ and the dual momentum maps are given by
\begin{subequations}
\begin{align}
\lambda_A(X)&=2\alpha(A)B(\sigma A,E)n_E&&\text{(}A\in \mA\\
\lambda_y(X)&= e^{-\alpha(a)}\Omega(n,y)&&\text{(}y\in\mG_{\alpha}\\
\lambda_E(X)&= e^{-Z\alpha(a)}B(\sigma E,E)
\end{align}
\end{subequations}
where $X=(a,n)$ and $n=n_{\alpha}+n_EE$ for the decomposition $\mN=\mG_{\alpha}\oplus\eR E$.

As a consequence, the Moyal star product is $R$-covariant.

\end{proposition}

\begin{proof}
From equation \eqref{eq_XlambdaYs}, we have to prove the identities
\[
  \{ \lambda_X,\lambda_Y \}=X^*(\lambda_Y).
\]
We begin by proving the identity
\[
  \{ \lambda_A,\lambda_y \}(L)=A^*_L(\lambda_y)
\]
where $L=(a',n')\in\mR$. In these coordinate, we suppose without loss of generality that $A=(1,0)$. As usual, we will use some abuse of notation as $\mI(L)= e^{a'} e^{n'}= e^{a'A} e^{n'}$;
\begin{equation}
\begin{split}
  A^*_L(\lambda_y)&=\Dsdd{ \lambda_y(\tau_{ e^{-tA}}L) }{t}{0}\\
		&=\Dsdd{ (\lambda_y\circ\mI ^{-1}) e^{(a'-t)A} e^{n'} }{t}{0}\\
		&=\Dsdd{ \lambda_y\big( (a'-t),n' \big) }{t}{0}\\
		&=\Dsdd{  e^{-\alpha(a'-t)}\Omega(n',y) }{t}{0}\\
		&=\Dsdd{  e^{(t-a')\alpha(A)} }{t}{0}\Omega(n',y)\\
		&=\alpha(A) e^{-\alpha(a')\Omega(n',y)}.
\end{split}
\end{equation}
In this computation, we used the fact that $\alpha(a'-t)=(a'-t)\alpha(A)$.
On the other hand,
\[
  \lambda_{[A,y]}(L)=\alpha(A)\lambda_y(L)=\alpha(A) e^{-\alpha(a')\Omega(n',y)}.
\]
This concludes the first check. The check that $\{ \lambda_A,\lambda_E \}=\lambda_{[A,E]}$ is the same, using the fact that $E\in\mG_{2\alpha}$ and thus that $[A,E]=2\alpha(A)E$. For the third, $[y,E]=0$ therefore, we have to prove that $\| \lambda_y,\lambda_E \|=0$. We have
\[
  \| \lambda_y,\lambda_E \|(L)=y^*_L(\lambda_E)=\Dsdd{ (\lambda_E\circ\mI^{-1})\big(  e^{-ty} e^{a'} e^{n'} \big) }{t}{0}.
\]
The problem is to commute $ e^{-ty}$ with $ e^{a'}$. Since the $t$ will always stands in front of $y$ and $\lambda_E$ doesn't depends on $y$, the derivative is zero\quext{Je ne crois pas que cette justification soit juste.}.
cs
\[
\begin{split}
  A^*_o&=\Dsdd{  e^{-tA}\cdot(0,0) }{t}{0}\\
	&=-A.
\end{split}
\]
Since $d\mI=\id$,
\[
\begin{split}
\omega^{\mR}(A^*,X)=&\omega^{\mR}(A,x_yy+x_EE)\\
		&=-\omega^{R}(A,x_yy+x_EE),
\end{split}
\]
but for any element in $\mA\oplus\mN$, via the identification $\mR=[\mA\oplus\mN]$ (the additive class),
\[
\begin{split}
d\lambda^{-1}A&=\Dsdd{ \lambda^{-1}[ e^{tA}] }{t}{0}\\
		&=\Dsdd{ \Ad( e^{tA})Z }{t}{0}\\
		&=-A^*.
\end{split}
\]
Thus
\begin{equation} \label{eq_omeOmO}
\begin{split}
\omega^{\mR}(A^*,X)&=-\omega^{\mO}(d\lambda^{-1}A,d\lambda^{-1}(x_yy+x_EE))\\
		&=-\omega^{\mO}(A^*,x_yy^*+x_EE^*)
\end{split}
\end{equation}
where $A^*$, $y^*$ and $E^*$ are taken in the sense of the adjoint action of $R$ on $\mO$.

Now we prove that $\lambda_A$ is well a dual momentum map. For this, we choose $X=x_AA+x_yy+x_EE\in\mR$ and we check the identity $d\lambda_AX=\omega^{\mR}(A^*,X)$ where $A^*$ stands for the given action of $R$ on $\mR$.

A question arise: at which point is taken $\omega^{\mO}$ in equation \eqref{eq_omeOmO}? Since we compute $\omega^{\mR}$ at identity, we compute $\omega^{\mO}$ at $Z$. So
 \[
\begin{split}
  \omega^{\mR}(A^*,X)&=-\omega^{\mO}_Z(A^*,x_yy^*+x_EE^*)\\
		&=-B(Z,[A,y+E])\\
		&=-\alpha(A)B(Z,y)-2\alpha(A)B(Z,E).
\end{split}
\]
Here, we have to remark that it is not zero because $\mN$ is not included in $\mP$, but is transverse.

\end{proof}
%+++++++++++++++++++++++++++++++++++++++++++++++++++++++++++++++++++++++++++++++++++++++++++++++++++++++++++++++++++++++++++
\section{Elementary normal symplectic spaces}
%+++++++++++++++++++++++++++++++++++++++++++++++++++++++++++++++++++++++++++++++++++++++++++++++++++++++++++++++++++++++++++
\label{SecElemNormSymplSpace}

This section is closely related to the Pyatetskii-Shapiro theory treated in section~\ref{SecPyateskiiShapiro}. See \cite{QuantifKhalerian} as reference.

Let $(V,\Omega)$ be a symplectic real vector space of dimension $2n$. We build the \defe{Heisenberg algebra}{heisenberg!algebra} by
\begin{equation}
	\pH=V\oplus \eR E
\end{equation}
with the relation $[v,v']=\Omega(v,v')E$. Now we consider a new element $H$ and $\lA=\eR H$ and the split extension
\begin{equation}
	0\to\pH\to\lS\to\lA\to 0
\end{equation}
where $\lS=\lA\oplus_{\rho}\pH$ and $\rho\colon \lA\to \Der(\pH)$ is given by
\begin{equation}
	\rho(H)(v\oplus tE)=[H,v\oplus tE]=v\oplus 2tE.
\end{equation}
We denote by $(a,v,t)$ an element of $\lS$, that is
\begin{equation}
	(a,v,t)=aH+ v+tE
\end{equation}
with $a,t\in\eR$ and $v\in V$. We consider the $2$-form
\begin{equation}
	\omega^{\sS}=2da\wedge dt+\Omega,
\end{equation}
the pair $(\lS,\omega^{\lS})$ is said to be a \defe{normal elementary symplectic algebra}{elementary!normal symplectic algebra}. We denote by $(\eS,\omega)$ the associated connected simply connected Lie group.

\begin{proposition}		\label{Prop2807DescSMdarboux}
	With the previous notations we have
	\begin{enumerate}

		\item
			The map
			\begin{equation}
				\begin{aligned}
					(\lS,\omega^{\lS})&\to (\eS,\omega) \\
					(a,v,t)&\mapsto  e^{aH} e^{v+tE}
				\end{aligned}
			\end{equation}
			is a global Darboux chart (in particular it is a global diffeomorphism).

			By this diffeomorphism we identify $\lS$ and $\eS$, i.e. we will denote by $(a,v,t)$ the element $ e^{aH}e^{v+tE}\in\eS$ as well as the element $aH+v+tE\in\lS$.
		\item
			Within the coordinates $(a,v,t)$ the group law is given by
			\begin{equation}
				(a,v,t)\cdot (a',v',t')=
				(a+a', e^{-a'}v+v', e^{-2a,}t+t'+\frac{ 1 }{2} e^{-a'}\Omega(v,v')).
			\end{equation}
		\item
			If we define
			\begin{equation}		\label{Eq1807StuctSymM}
				s_{(a,v,t)}(a',v',t')=
				\big(2a-a',2\cosh(a-a')v-v',2\cosh(2(a-a'))t+\Omega(v,v')\sinh(a-a')-t'\big),
			\end{equation}
			the space $\eM=(\eS,\omega,\lS)$ becomes a symplectic symmetric space.

		\item
			The structure of symplectic symmetric space  is preserved by the left translations. In other words, for every $x\in\eS$ we have $L_x\in\Aut(\eM)$. And the subgroup $\{ L_x\tq x\in\eS \}$ acts simply transitively on $\eM$.

		\item
			We have
			\begin{equation}
				\SP(V,\Omega)\subset\Aut(\eM)
			\end{equation}
			if we define $g\cdot(a,v,t)=(a,g\cdot v,t)$ for every $g\in\SP(V,\Omega)$.
	\end{enumerate}
\end{proposition}


%+++++++++++++++++++++++++++++++++++++++++++++++++++++++++++++++++++++++++++++++++++++++++++++++++++++++++++++++++++++++++++
\section{Pyatetskii-Shapiro structure theorem}
%+++++++++++++++++++++++++++++++++++++++++++++++++++++++++++++++++++++++++++++++++++++++++++++++++++++++++++++++++++++++++++
\label{SecPyateskiiShapiro}

\begin{definition}
    A \defe{normal $j$-algebra}{normal $j$-algebra} is a triple $(\lS,\alpha,j)$ where
    \begin{enumerate}

        \item
            the Lie algebra $\lS$ is solvable and such that $\ad(X)$ has only real eigenvalues for every $X\in\lS$,
        \item
            the map $j\colon \lS\to \lS$ is an endomorphism of $\lS$ such that $j^2=-1$ and
            \begin{equation}
                [X,Y]+j[jX,Y]+j[X,jY]-[jX,jY]=0
            \end{equation}
            for every $X,Y\in\lS$,
        \item
            $\alpha$ is is  a linear form on $\lS$ such that
            \begin{enumerate}
                \item
                    $\alpha\big( [jX,X] \big)>0$ if $X\neq 0$,
                \item
                    $\alpha\big( [jX,jY] \big)=\alpha\big( [X,Y] \big)$.
            \end{enumerate}

    \end{enumerate}
\end{definition}

If $\lS'$ is a subalgebra of $\lS$ which is invariant under $j$, then the triple $(\lS',\alpha|_{\lS'},j|_{\lS'})$ is a also normal $j$-algebra and is said to be a \defe{normal $j$-subalgebra}{normal!$j$-subalgebra} of $\lS$.

A normal $j$-algebra has a real inner product defined by the formula
\begin{equation}
    g(X,Y)=\alpha\big( [jX,Y] \big).
\end{equation}

If $\lG$ is an Hermitian Lie algebra\footnote{i.e. the center of its maximal compact is one dimensional.}, we can build a normal $j$-algebra out of $\lG$ in the following way. First, we choose an Iwasawa decomposition
\begin{equation}            \label{EqDecIwalGj}
    \lG=\lA\oplus\lN\oplus\lK,
\end{equation}
and we pick $\lS=\lA\oplus\lN$. Let $G=ANK$ be the group associated with the Iwasawa decomposition \eqref{EqDecIwalGj}. The manifold $M=G/K$ is an Hermitian symmetric space, and we have a global diffeomorphism
\begin{equation}
    \begin{aligned}
        R=AN&\to G/K \\
        g&\mapsto gK
    \end{aligned}
\end{equation}
which endows the group $R$ with an exact left invariant symplectic structure and a compatible complex structure, see section~\ref{SecHermEtSymplecticSpaces}. We define $\alpha$ by $\Omega_e=d\alpha$ ($\Omega$ is exact) and $j$ is the complex structure evaluated at identity.

A normal $j$-algebra build from an Hermitian symmetric space of rank $1$ (i.e. $\dim\lA=1$.) is \defe{elementary}{normal!elementary $j$-algebra}. Elementary normal $j$-algebra are well understood by the following proposition.

\begin{proposition}     \label{PropStructNormalElementaireJalg}
    An elementary normal $j$-algebra is a split extension
    \begin{equation}        \label{EqDecoEleJal}
        \lS_{el}=\lA_{1}\oplus_{\ad}\lN_1=\lA_1\oplus_{\ad}\big( V\oplus\lZ_1 \big)
    \end{equation}
    where $\lN_1$ is an Heisenberg algebra $\lN_1=V\oplus\lZ_1$ and $\lA_1$ is one dimensional. Moreover, $V$ is a symplectic vector space and one can choose $H\in \lA_1$ and $E\in\lZ_1$ in such a way that
    \begin{equation}            \label{EqRelColNormaljAlg}
        \begin{aligned}[]
            [H,v]&=v,\\
            [v,v']&=\Omega(v,v')E,\\
            [H,E]&=2E.
        \end{aligned}
    \end{equation}
\end{proposition}

Any normal $j$-algebra is build from elementary normal $j$-algebras by mean of the following lemma.
\begin{proposition}         \label{PropStructNormalJalg}
    Let $(\lS,\alpha,j)$, a normal $j$-algebra and $\lZ_1$, a one dimensional ideal of $\lS$.
    \begin{enumerate}

        \item
            There exists a vector space $V$ such that
            \begin{equation}
                \lS_1=j\lZ_1+V+\lZ_1
            \end{equation}
            is an elementary normal $j$-algebra, and such that $\lS$ is a split extension
            \begin{equation}        \label{EqDecNormale}
                \lS=\lS'\oplus_{\ad}\lS_1
            \end{equation}
            where $\lS'$ is, itself, a normal $j$-algebra.

        \item
            If $\lS_1=\lA_1\oplus_{\ad}(V\oplus\lZ_1)$, then
            \begin{equation}
                j\lZ_1+\lZ_1=\lA_1\oplus\lZ_1
            \end{equation}
            and
            \begin{equation}
                \begin{aligned}[]
                    [\lS',\lA_1\oplus\lZ_1]&=0,\\
                    [\lS',V]&\subset V.
                \end{aligned}
            \end{equation}
        \item
            Such an ideal $\lZ_1$ exists in every normal $j$-algebra.
    \end{enumerate}
\end{proposition}

Let us see what are the possibilities for $j$. If $jE=aH+b_iv_i+cE$, then
\begin{equation}
    [jE,E]=2aE.
\end{equation}
We can prove that $a\neq 0$. Indeed, if $a=0$, then $jE=cE$ and $-E=j^2E=cjE=c^2E$.

Now, we use the following Jacobi identity on $[H,[jE,v]]$ and the commutation relations, we find $b_i=0$. Now, suppose that $jH=a'H+b'_i+c'E$. In that case,
\begin{equation}
    -E=j^2E=j(aH+cE)=aa'H+ab'_iv_i+ac'E+caH+c^2E.
\end{equation}
Since $a\neq 0$, we have $b'_i=0$. So we have
\begin{equation}
    \begin{aligned}[]
        jE&=aH+cE\\
        jH&=a'H+c'E.
    \end{aligned}
\end{equation}
Expressing that $j^2E=-E$ and $j^2H=-H$, we find the following constrains on the coefficients:
\begin{equation}
    \begin{aligned}[]
        aa'+ca&=0\\
        ac'+c^2&=-1\\
        c'^2+c'a&=-1\\
        c'c+c'c&=0.
    \end{aligned}
\end{equation}
We check that $a\neq 0$, $c'\neq 0$ and $a'=-c$. The remaining relation is $c^2+c'a=-1$. Thus in the basis $\{ H,E \}$, the endomorphism $j$ reads
\begin{equation}
    j=\begin{pmatrix}
        -c  &   a   \\
        c'  &   c
    \end{pmatrix}
\end{equation}
with $\det j=1$.

\begin{lemma}
    An elementary normal $j$-algebra has no proper $j$-ideal.
\end{lemma}

\begin{proof}
    Let $\lI$ be  a $j$-ideal of the elementary normal $j$-algebra $\lS_{el}$. Let $\lS_{el}=\lA\oplus_{\ad}(V\oplus\lZ)$. We denote by $H$ and $E$ the elements of $\lA$ and $\lZ$ (which are one dimensional) who fulfill the standard relations \eqref{EqRelColNormaljAlg}. If $X=aH+b_iv_i+cE\in\lI$, then $\big[ [X,v],v \big]\in\lI$. Using the relations, we conclude that $\lZ\subset\lI$. By $j$-invariance of $\lI$, we have $j\lZ\subset\lI$. Now, the fact that $[jE,v]=av$ implies that $\lI=\lS_{el}$.
\end{proof}

The structure of a normal $j$-algebra $\lS$ is thus as follows. We have the decomposition
\begin{equation}
    \lS=\lS'\oplus_{\ad}\Big( \lA_1\oplus_{\ad}(V_1\oplus\lZ_1) \Big)
\end{equation}
where $\lS'$ is again a normal $j$-algebra. Furthermore, $\dim\lA_1=\dim\lZ_1=1$ and we can choose a basis $H\in\lA_1$, $E\in\lZ_1$ such that
\begin{equation}
    \begin{aligned}[]
        [H,v]&=v\\
        [H,E]&=2E\\
        [v,v']&=\Omega(v,v')E\\
        [\lS',V]&\subset V\\
        [\lS',\lA_1\oplus\lZ_1]&=0.
    \end{aligned}
\end{equation}
for all $v,v'\in V_1$. The algebra $V_1\oplus\lZ_1$ is an Heisenberg algebra.

The algebra $\lS'$ can be decomposed in the same way again and again up to end up with a sequence of elementary normal $j$-algebra.


\chapter{Heat kernel expansions}
\input{chaleur}

\chapter{From Clifford algebras to Dirac operator}
% This is part of (almost) Everything I know in mathematics
% Copyright (c) 2013-2014, 2019-2020
%   Laurent Claessens
% See the file fdl-1.3.txt for copying conditions.

Bibliography for Clifford algebras, spin group and related topics are \cite{memP,Michelson,Witkowski,mellor,ResEtaDiracType}. More algebraic point of view  can be found in \cite{Fult,Chevalley}. More details about ``square rooting'' second order differential operators are in \cite{Bronn}. For physical concerns, the reader should refer to \cite{Weinberg,Peskin,schwabl}.

\section{Invitation: Clifford algebra in quantum field theory}\index{quantum!field theory}
%++++++++++++++++++++++++++++++++++++++++++++++++

\label{Secqft}
\subsection{Schrödinger, Klein-Gordon and Dirac}
%----------------------------------

The origin of the Klein-Gordon equation is almost the same as the one of the Schrödinger: one replace physical functions by operators. For a free particle, the correspondence are
\begin{align*}
 \textrm{energy}&& E&\rightarrow i\hbar\dsd{}{t},\\
 \textrm{momentum}&& \overline{p}&\rightarrow -i\hbar\overline{ \nabla }.
\end{align*}
The Schrödinger equation\index{equation!Schrödinger} (which is the non relativistic quantum wave equation) comes from replacement in the non non relativistic expression of the Hamiltonian
\[
E=\frac{\overline{p}^2}{2m}\longrightarrow\left(\partial_t-\frac{i\hbar}{2m}\nabla^2\right)\psi=0,
\]
while the Klein-Gordon\index{equation!Klein-Gordon} one (which is the relativistic quantum wave equation) comes from the relativistic corresponding equation:
\[
E^2=\overline{p}^2c^2+m^2c^4\longrightarrow\left(\partial\hmu\partial_{\mu}+(\frac{mc}{\hbar})^2\right)\psi=0.
\]

This is a second order differential equation; there are however no ``law of nature''{} which forbid a first order equation. We try
\[
 i\hbar\dsd{\psi}{t}
 =\left(\frac{\hbar c}{i}\alpha^k\partial_k+\beta mc^2\right)\psi\equiv \hat{H}\psi.
\]

There are some physical constraints on the coefficients $\alpha^k$ and $\beta$. We will study one of them: we want the components of $\psi$ to satisfy the Klein-Gordon equation, so that the plane waves fulfill the fundamental relation $E^2=p^2c^2+m^2c^4$.

In order to see the implications of this constraint on the coefficients, we apply two times the operator $\hat{H}$, and we compare the result with the Klein-Gordon equation. We find:
\begin{subequations}
\begin{align}
 \alpha^i\alpha^j+\alpha^j\alpha^i&=2\delta^{ij}\mtu,\\
 \alpha^i\beta+\beta\alpha^i&=0,\\
 (\alpha^i)^2=\beta^2&=\mtu.
\end{align}
\end{subequations}
%
If we define $\gamma^0=\beta$ and $\gamma^i=\beta\alpha^i$, we find that the matrices $\gamma^{\mu}$ have to give a representation of the Clifford algebra\footnote{Don't be afraid with the extra minus sign: the quantum field theory is most written with the metric $(+,-,-,-)$ instead of $(-,+,+,+)$.}:
\begin{equation}\label{cliffphys}
	\gamma\hmu\gamma\hnu+\gamma\hnu\gamma\hmu=2\eta^{\mu\nu}\mtu.
\end{equation}
The Dirac equation reads
\[
\left(-i\gamma\hmu\partial_{\mu}+\frac{mc}{\hbar}\right)\psi=0.
\]
If we want to perform some computation with the quantum field theory, we need an explicit form for the $\gamma$'s; that's the reason why we study representations of the Clifford algebra. The \defe{Dirac operator}{dirac!operator} $\Dir$ is the operator which lies in the Dirac equation:
\begin{equation}
 \label{dirflat}\Dir=\sum_{\mu=0}^3\gamma\hmu\dsd{}{x\hmu}.
\end{equation}


\begin{definition}
Let $V$ be a (finite dimensional) vector space and $q$, a bilinear quadratic form over $V$. The \defe{Clifford algebra}{Clifford!algebra}\index{algebra!Clifford} $\Cliff(V,q)$ is the unital associative algebra generated by $V$ subject to the relation
\begin{equation}\label{501r1}
       v\cdot v=q(v)
\end{equation}
for all $v$ in $\Cliff(V,q)$. Here the dot denotes the algebra product and $q(v)$ means $q(v,v)$.
\end{definition}
Theorem proves~\ref{tho_Cliffunif} proves unicity of such an algebra, so that it makes sense.

\begin{remark}
The relation \eqref{501r1} is no more a restriction for the elements in $\Cliff(V,q)$ than a restriction on the choice of the algebra product.
\end{remark}


An explicit construction of $\Cliff(V,q)$ can be achieved in the following way. On the tensor algebra\footnote{Definition \ref{DEFooHPQXooETvEyn}.} \( T(V)\), we consider the two-sided ideal $\mI$ generated by elements of the form $v\otimes v-q(v)1$. The  \defe{Clifford algebra}{Clifford!algebra}\index{algebra!Clifford} for $(V,q)$ is given by\nomenclature[G]{$\Cliff(p,q)$}{Clifford algebra of $\eR^{1,3}$}
\begin{equation}	\label{defI}
	\Cliff(p,q):=T(V)/\mI
\end{equation}
in which product of $\Cliff(V,q)$ is naturally defined by $[a]\otimes[b]=[a\otimes b]$ if $[a]$ is the class of $a\in T(V)$.

Let us now fix some notations more adapted to what we want to do. Let $V=\eR^{p,q}$ the vector space $\eR^{p+q}$ endowed with a diagonal metric which contains $p$ plus sign and $q$ minus signs. For $v$, $w\in V$, the inner product with respect to the metric $\eta$ of $v$ by $w$ will be denoted by $\eta(v,w)$.  The norm on $V$ will be defined by $\|v\|^2=-\eta(v,v)$. It is neither positive defined, nor negative defined. The explanation of the minus sign will come soon. The Clifford algebra is the quotient $\Cliff(p,q):=T(V)/\mI$ of the tensor algebra by the two-sided ideal $\mI$ generated by elements of the form
\[
	(v\otimes w)\oplus (w\otimes v)\oplus 2\eta(v,w)1
\]
 for $v,w$ in $V$. Depending on the context, we will often use the notations $\Cliff(\eta)$ or $\Cliff(V)$ or $\Cliff(p,q)$. The algebra product is $[x]\cdot[y]=[x\otimes y]$, $x$, $y\in T(V)$.  As long as $z\in V\subset\Cliff(p,q)$, the expression $\eta(z,z)$ is meaningful. The definition of $\Cliff$ is such that $z\cdot z=-\eta(z,z)$. This leads to the somewhat surprising formula  $z^2=\|z\|^2=-\eta(z,z)$.

\subsection{First representation}
%----------------------------------

Let $(V,g)$ be a metric vector space and $\Cliff(V,g)$ its Clifford algebra. For each $v\in V$, we define the two following elements of $\End_{\eR}(\Wedge V)$:
\begin{subequations}
\begin{align}
\epsilon(v)\big( u_1\wedge\cdots\wedge u_k \big)&=v\wedge u_1\wedge\cdots\wedge u_k\\
\iota(v)\big( u_1\wedge\cdots\wedge u_k \big)&=\sum_{j=1}^k(-1)^{j-1}g(u,u_j)u_1\wedge\cdots\wedge\hat u_j\wedge\cdots\wedge u_j.
\end{align}
\end{subequations}
One has $\epsilon(v)^2=0$ and $\iota(v)^2=0$ because $v\wedge v=0$. In order to understand the latter, we wonder what are the terms with $g(v,u_i)g(v,u_j)$ are in
\[
  \iota(v)^2\big( u_1\wedge\cdots\wedge u_k \big)=\sum_{l=1}^k(-1)^{j-1}g(v,u_j)\sum_{l=1}^{k-1}(-1)^{l-1}g(v,u_l)u_1\wedge\hat u_l\wedge\hat u_j\wedge\cdots\wedge u_k.
\]
Let's suppose $i<j$. The first term comes when the first $\iota(v)$ acts on $u_j$, its sign is given by $(-1)^{j-1}(-1)^{i-1}$. The second term has the same $(-1)^{i-1}$, but in this term, $u_j$ is on the position $j-1$ because $u_i$ has disappeared.

Now we use $c(v)=\epsilon(v)+\iota(v)$ which fulfils for all $u$, $v\in V$:
\[
\begin{split}
c(v)^2&=g(v,v,)1\\
c(u)v(v)+c(v)c(u)&=2g(u,v)1.
\end{split}
\]
Therefore $c$ can be extended to a representation $c\colon \Cliff(V,g)\to \End(\Wedge V)$. If $\{ e_0,\cdots e_n \}$ is an orthonormal basis of $V$ (i.e. $g(e_i,e_j)=\eta_{ij}$); in this case the $c(e_j)$ are anticommuting and a basis of $\Cliff(V,g)$ is given by
\begin{equation}
 \{ c(e_{k_1})\cdots c(e_{k_r})\tq 1\leq k_1<\cdots<k_r\leq n \}.
\end{equation}

\subsection{Some consequences of the universal property}
%---------------------------------------------------

The map $-\id|_V$ extends to $\alpha\in\Aut\big( \Cliff(V) \big)$,
\[
  \alpha(v_1\cdots v_r)=(-1)^rv_1\cdots v_r
\]
($v_i\in V$) and provides a graduation
\[
  \Cliff(V)=\Cliff^0(V)\oplus\Cliff^1(V).
\]
The map $\tau\colon \Cliff(V)\to \Cliff(V)$ extends $\id|_V$ to an anti-homomorphism:
\begin{equation}
\tau(v_1\cdots v_r)=v_r\cdots v_1.
\end{equation}

The \defe{complexification}{complexification!of Clifford algebra} of $\Cliff(V,g)$ is
\[
  \CCliff(V,g):=\Cliff(V,g)\otimes_{\eR}\eC\simeq\Cliff(V^{\eC},g^{\eC}),
\]
the isomorphism being a $\eC$-algebra isomorphism. The $\eR$-linear operator $v\mapsto \overline{ v }$ in $V^{\eC}$ of complex conjugation extends to a $\eR$-linear automorphism $a\mapsto \overline{ a }$. We define the \defe{adjoint}{adjoint!in Clifford algebra} by
\begin{equation}
  a^*=\tau(\overline{ a })
\end{equation}

\subsection{Trace}
%---------------------

\begin{theorem}
There exists one an only one trace $\tr\colon \CCliff(V)\to \eC$ such that
\begin{enumerate}
\item $\tr(1)=1$,
\item $\tr(a)=0$ when $a$ is odd.
\end{enumerate}

\end{theorem}

\begin{proof}

Let $\{ e_1,\cdots,e_n \}$ be an orthonormal basis of $(V,g)$ and $a\in\CCliff(V)$. When one decomposes $a$ into the basis of $e_i$, one finds a lot of terms of each order. Since $\tr$ is a trace, when the $k_i$ are all different,
\[
\tr(e_{k_1}\cdots e_{k_{2r}})	=\tr(-e_{k_2}\cdots e_{k_{2r}} e_{k_1}
				=\tr(-e_{k_1}\cdots e_{k_{2r}})\\
\]
So the trace of any even element is zero. We decompose $a$ into
\[
  a=\sum_K a_K\prod_{i\in K}e_i
\]
where the sum is taken on the subsets of $\{ 1,\ldots, n \}$. A trace which fulfils the conditions must vanishes on even (but non zero) elements as well as on odd elements, so the only possible form is
\[
  \tr a=a_{\emptyset}.
\]
Notice that in order to get this precise form, we used $\tr(1)=1$ and linearity. This proves unicity and existence. Now we have to prove that this is a good definition in the sense that an other choice of basis gives the same result. So we take a new orthonormal basis
\[
  e'_j=\sum_{k=1}^nH_{jk}e_k
\]
with $H^tH=\mtu_{n\times n}$. Now we have
\[
  a=\sum_{K}^{}a_K\prod_{i\in K}e_i=\sum_{K}^{}a'_K\prod_{i\in K}e'_i,
\]
and we will prove that $a_{\emptyset}=a'_{\emptyset}$. Let's compute a lot:
\[
\begin{split}
   e_i'e_j'&=\sum_{k}^{}\sum_{l}^{}H_{ik}H_{jl}e_ke_l\\
		&=\sum_{k=l}^{}H_{ik}H_{jl}e_{k}e_{l}+\sum_{k\neq l}^{}H_{ik}H_{jl}e_{k}e_{l}\\
		&=\sum_{k}^{}H_{ik}H_{jk}1+\sum_{k\neq l}^{}H_{ik}H_{jl}e_{k}e_{l}\\
		&=(HH^t)_{ij}1+\sum_{k\neq l}^{}H_{ik}H_{jl}e_{k}e_{l}.
\end{split}
\]
The sense of this formula is that when $i\neq j$, the product $e'_{i}e'_{j}$ has no term of order zero. In other terms, as long as we only have terms of order zero, one and two, a change $e\to e'$ does not change the term of order zero. We are now going to an induction proof: we want to prove that $e'_{j_{1}}\ldots e'_{j_{2r}}e'_{l}e'_{k}$ has no scalar term assuming that no even combination has scalar terms up to $2(r-1)$. It reads
\[
    \sum_{K \text{ even}}^{}a_{K}\prod_{i\in K} e_{i}e'_{l}e'_{k},
\]
therefore we just have to look at terms of the form
\[
  e_{j_{1}}\ldots e_{j_{2r}}\Big( (HH)^t_{kl}1-\sum_{i\neq j}^{}C_{kl}^{ij}e_{i}e_{j} \Big)
\]
where the $e_{j_{l}}$ are all different. The first term cannot produce a scalar term. In order to find a scalar term in $e'_{j_{1}}\ldots e'_{j_{2r}}e_{k}e_{l}$, we begin to look at terms whose decomposition of $e'_{j_{1}}\ldots e'_{j_{2r}}$ ends by $e_{l}e_{k}$, i.e.
\[
  H_{j_{2r-2}l}H_{j_{2r-1}k}e'_{j_{1}}\ldots e'_{2r-3}e_{l}e_{k}e_{k}e_{l}.
\]
The induction assumption says that there are no scalar term in $e'_{2r-3}e_{l}e_{k}e_{k}e_{l}$.

\end{proof}
One can prove that $\CCliff(C)$ is a Hilbert space with the scalar product
\begin{equation}
\langle a |b\rangle=\tr(a^*b).
\end{equation}


Let $v\in V$ with $g(v,v)=1$ (thus in $\Cliff(V)$, we have $v^2=1$); since $v=\overline{ v }$, we have
\[
  a^*v=vv^*=v^2=1.
\]

\begin{lemma}
The maps $a\mapsto ua$ and $a\mapsto au$ are unitary if and only if $uu^*=u^*u=1$.
\end{lemma}

\begin{proof}
We pick $\lambda\in U(1)$ and $w=\lambda v\in V^{\eC}$ which fulfils $w^*w=1$. This is the most general element such that $ww^*=w^*w=1$. Now for an arbitrary $a$, $b\in\CCliff(V)$, we compute the two followings:
\[
  \langle wa|wb\rangle=\tr\big( (wa)^*wb \big)
		=\tr\big( a^*w^*wb \big)
		=\tr(a^*b)
		=\langle a|b\rangle,
\]
and
\[
  \langle aw|bw\rangle=\tr\big( w^*a^*bw \big)
		=\tr(ww^*a^*b)
		=\tr(a^*b)
		=\langle a|b\rangle.
\]
This proves that $a\mapsto wa$ and $a\mapsto aw$ are two unitary operators on the Hilbert space $\CCliff(V)$.

For the converse, we impose for all $a$, $b\in\CCliff(V)$:
\[
  \langle ua|ub\rangle=\tr(ba^*u^*u)\stackrel{!}{=}\tr(ba^*).
\]
In particular with $a^*b=1$, $\tr(u^*u)=\tr(1)=1$, thus the scalar part of $u^*u$ is $1$. So we write $u^*u=1+f$ where $f$ is non scalar, and for any $x\in\CCliff(V)$ , we have
\[
   \tr(x)=\tr(xu^*u)=\tr(x)+\tr(xf).
\]
We conclude that $\tr(xf)=0$, and therefore that $f=0$.
\end{proof}


\section{Spinor representation}
%+++++++++++++++++++++++++++++

For the spinor representation, we restrict ourself to the even case $p+q=2n$.

The aim of this subsection is to find some faithful\footnote{Definition \ref{DEFooXVMSooXDIfZV}.} representations of the complex Clifford algebra $\Cliff^{\eC}(p,q)$. In order to achieve this, we first consider $V^{\eC}$, the complex vector space of $V$ with an orthonormal basis $\{ e_1,\cdots,e_{p-1},e_p,\cdots,e_q  \}$. The metric is $\eta(e_k,e_k)=1$ and $\eta(e_{p+k},e_{p+k})=-1$ for $k=0,\cdots,p-1$. We use the following basis:
\begin{align}
f_k&=\frac{1}{2}(e_k+e_{p+k}),& g_k&=\frac{1}{2}(e_k-e_{p+k}),\\
 f_{p+s}&=\frac{1}{2}(e_{2p+2s}+ie_{2p+2s+1}),& g_{p+s}&=\frac{1}{2}(e_{2p+2s}-ie_{2p+2s})
\end{align}
for $k=0,\cdots,p-1$.
We note that $\{f_0,g_0\}$ spans a $\eC^2$-space which is $\eta$-orthogonal to the one which is spanned by $\{f_1,g_1\}$. The following two  spaces will prove to be useful:
\begin{subequations}
\begin{align}
  W           &=\Span_{\eC}\{f_0,f_1\}\simeq\eC^2,\\
 \underline{W}&=\Span_{\eC}\{g_0,g_1\}\simeq\eC^2.
\end{align}
\end{subequations}
\nomenclature{$W,\underline{W}$}{Totally isotropic subspace}
It is easy to compute the various products; among others we find
\begin{equation}
 \eta(f_0,f_0)=0,\quad
 \eta(f_1,f_0)=0,\quad
  \eta(f_1,f_1)=0;
\end{equation}
so that for any $w\in W$, we have $\scal{w}{w}=0$; for this reason, we say that $W$ is a \defe{completely isotropic}{isotropic!subspace!completely} subspace of $(V^{\eC},\eta^{\eC})$. The space $\underline{W}$ has the same property.

\begin{proposition}
We have
\begin{equation}
  \underline{W}\simeq W^*,
\end{equation}
where $W^*$ is the dual space of $W$. By $\simeq$ we mean that there exists a linear bijective map $\dpt{\psi}{\underline{W}}{W^*}$.
\end{proposition}
\begin{proof}
For each $\uw\in\uW$, we define $\dpt{\psi(\uw)}{W}{\eC}$ by
\[
   \psi(\uw)(w)=\eta(w,\uw).
\]
We first show that the map $\psi$ is injective. Let $\uw\in\uW$ be so that $\psi(\uw)=0$. Thus for all $v\in W$, we have
\begin{eqnarray}
   \label{3101r1}\psi(\uw)v=\eta(\uw,v)=0.
\end{eqnarray}
By decomposing $\uw=ag_0+bg_1$ and taking successively $v=f_0$ and $v=f_1$, we see that $a=b=0$.

The next step is to see that the map $\psi$ is surjective. We know that $dim_{\eC}\uW=dim_{\eC}W^*=2$ and that $\psi(g_0)\neq 0$. Let's prove that $\{\psi(g_0),\psi(g_1)\}$ is a basis of $W^*$. It is clear by linearity that $\{\psi(ag_0):a\in\eC\}=\Span\{\psi(g_0)\}$. The fact that $\psi$ is injective  imposes that $\psi(g_1)$ doesn't belong to $\Span\{\psi(g_0)\}$. So $\{\psi(g_0),\psi(g_1)\}$ is a two-dimensional free subset of $W^*$, and therefore is a basis of $W^*$.
\end{proof}

We turn our attention to the exterior algebra $\Lambda W=\eC\oplus W\oplus(W\wedge W)\oplus\cdots\oplus\wedge^{p+q}W$\nomenclature{$\Lambda W$}{Space of spinor representation} of $W$.
\nomenclature{$\dpt{\tilde\rho}{(\eR^{1+3})^{\eC}}{\End(\Lambda W )}$}{Spinor representation}

\begin{definition}
\index{endomorphism!of $\Lambda W$}
We define the homomorphism $\dpt{\tilde\rho}{V^{\eC}}{\End(\Lambda W)}$ by
\begin{equation}
\begin{split}
 \tilde\rho(f_i)\alpha&=f_i\wedge\alpha,\\
 \tilde\rho(g_i)\alpha&=-\iota(g_i)\alpha
\end{split}
\end{equation}
($v\in V^{\eC}$, $\alpha\in\Lambda W$) where $\iota$ denotes the interior product defined in page \pageref{pg_DefProdExt}.\
\label{defrt}
\end{definition}
  More explicitly, for all $z\in\eC$ and for all $w,w'\in W$, we have
\begin{subequations}
\begin{align}
 \tilde\rho(f_i)z&=zf_i,&\tilde\rho(g_i)z&=0,\\
 \tilde\rho(f_i)w&=f_i\wedge w,&\tilde\rho(g_i)w&=-\eta(g_i,w)1,\\
 \tilde\rho(f_i)(w\wedge w')&=0,&\tilde\rho(g_i)(w\wedge w')&=-\eta(g_i,w)w'+\eta(g_i,w')w.
\end{align}
\end{subequations}
We will see that, \emph{via} some manipulations, $\tilde\rho$ provides a faithful representation of the Clifford algebra, the \defe{spinor representation}{spinor!representation}.
\index{representation!of Clifford algebra}

\begin{remark}
By ``endomorphism of $\Lambda W$'', we mean an endomorphism for the \emph{linear} structure of $\Lambda W$. We obviously not have $\tilde\rho(x)(\alpha\wedge\beta)=\tilde\rho(x)\alpha\wedge\tilde\rho(x)\beta$.
\end{remark}

\begin{proposition}
The map $\tilde\rho$ is injective.
\end{proposition}

\begin{proof}
We have to show that $\tilde\rho(v)=0$ ($v$ in $V^{\eC}$) implies $v=0$. Any $v\in V^{\eC}$ can be written as
$v=a^if_i+b^ig_i$ with a sum over $i$. We first have that
\[
 \tilde\rho(a^if_i+b^ig_i)z=za^if_i=0,
\]
 but the $f_i$ are independents and then $a^i=0$. We can also write
\[
 \tilde\rho(b^0g_0+b^1g_1)f_1=-b^0\eta(g_0,f_1)-b^1\eta(g_1,f_1)=-\frac{b^1}{2}=0,
\]
 then $b^1=0$. The same with $f_0$ proves that $b^0=0$.
 \end{proof}

The homomorphism $\tilde\rho$ extends to the whole the tensor algebra of $V^{\eC}$ by the following definitions:
\begin{subequations}
\begin{align}
 \tilde\rho(1)              &=\id_{\Lambda W},\\
 \tilde\rho(e_k)            &=\tilde\rho(e_k),\\
 \tilde\rho(e_{k_1}\otimes\ldots\otimes e_{k_r})&=
                      \tilde\rho(e_{k_1})\circ\ldots\circ\tilde\rho(e_{k_r}).\label{eq:3101r2}
\end{align}
\end{subequations}
So we get $\dpt{\tilde\rho}{T(V^{\eC})}{\End(\Lambda W)}$.  The following proposition will allow us to descent $\tilde\rho$ to a representation of the Clifford algebra.

\begin{proposition}
The homomorphism $\tilde\rho$ maps $\mI$ to $0$: $\tilde\rho(\mI)=0$.
\end{proposition}

\begin{probleme}
This proposition is wrong: there is a double covering.

Moreover, there is a sign problem in the proof: the sign in the first lines is not the one used in the definition of the Clifford algebra.
\end{probleme}


\begin{proof}
We have to check the following:
\[\tilde\rho(v\otimes w\oplus w\otimes v-2\eta(v,w)1)=0\]
for any choice of
 $v,w$ in $\{e_0,e_1,e_2,e_3\}$.
  Here we will just check it explicitly for $v=e_0$ and $w=e_1$. The computation uses the definition \eqref{eq:3101r2}:
\begin{equation}
\begin{split}
\tilde\rho(e_0\otimes e_1\oplus e_1\otimes
             e_0-2\eta(e_0,e_1)&=\tilde\rho(e_0)\circ\tilde\rho(e_1)+\tilde\rho(e_1)\circ\tilde\rho(e_0)\\
                               &=2\left[\tilde\rho(f_0)^2-\tilde\rho(g_0)^2\right].
\end{split}
\end{equation}
It is easy to see that $\tilde\rho(f_0)^2=0$:
\begin{equation}
 \tilde\rho(f_0)^2\left[z\oplus w\oplus w_1\wedge w_2 \right]=\tilde\rho(f_0)[zf_0\oplus f_0\wedge w]
                                                   =zf_0\wedge f_0,
							=0.
\end{equation}
 The proof that $\tilde\rho(g_0)^2=0$ is almost the same:
\[
 \tilde\rho(g_0)^2\left[z\oplus w\oplus w_1\wedge w_2 \right]
 =\tilde\rho(g_0)[-\eta(g_0,w)1\oplus-\eta(g_0,w_1)w_2\oplus\eta(g_0,w_2)w_1].
\]

\end{proof}

We can now see $\tilde\rho$ as a map $\dpt{\tilde\rho}{\Cliff^{\eC}(p,q)}{\End(\Lambda W )}$. By construction, it is a homomorphism and, thus, is a representation of $\Cliff^{\eC}(p,q)$ on $\Lambda W$. For compactness, we use the notation \index{dirac!matrices}\nomenclature{$\gamma_i$}{Abstract definition of Dirac matrices}
\begin{equation}
 \label{defgamma}\gamma_a:=\sqrt{2}\tilde\rho(e_a).
\end{equation}

\begin{lemma}
The $\gamma$'s operators satisfy the following relation:
\begin{equation}\label{3101r3}
  \gamma_a\gamma_b+\gamma_b\gamma_a=-2\eta_{ab}\mtu.
\end{equation}
\label{3101l1}
\end{lemma}

\begin{proof}
We have to check this equality on any element of $\Lambda W$. If we choose
$w_1=af_0+bf_1$ and $w_2=a'f_0+b'f_1$,
 we find $w_1\wedge w_2=(ab'-ba')f_0\wedge f_1$.

For example, we will explicitly check \eqref{3101r3} with $a=b=0$, i.e. $\tilde\rho(e_0)\circ\tilde\rho(e_0)=\frac{1}{2}\id$, which proves that $\gamma_0\circ\gamma_0=\id$.
 \begin{equation}
\begin{split}
   \tilde\rho(e_0)^2[z\oplus w\oplus(ab'-ba') f_0\wedge f_1]&=\tilde\rho(f_0+g_0)^2[z\oplus w\oplus(ab'-ba') f_0\wedge f_1]\\
                                              &=\tilde\rho(f_0+g_0)\Big[zf_0\oplus f_0\wedge w\oplus-\eta(g_0,w)1\\
                                                            &\qquad-(ab'-ba')\eta(g_0,f_0)f_1\\
                                                             &\qquad+(ab'-ba')\eta(g_0,f_1)f_0\Big]\\
                                              &=\frac{1}{2}(z\oplus w\oplus(ab'-ba') f_0\wedge f_1).
\end{split}
\end{equation}
\end{proof}

\begin{lemma}
For any sequence $i_0,\ldots i_3$ of $0$ and $1$ (with at least one of them equals to $1$), we have
\begin{equation}
 \tr(\gamma_0^{i_0}\cdots \gamma^{i_{2n-1}}_{2n-1})=0.
\end{equation}
We take the convention that $\gamma_a^0=\mtu$.
\label{3101l2}
\end{lemma}

\begin{proof}
 If the number of nonzero $i_k$ is even (say $2m$), we have:
\[
	\tr(\gamma_{a_1}\ldots\gamma_{a_{2m}})=\tr(\gamma_{a_{2n}}\gamma_{a_1}\ldots\gamma_{a_{2m-1}})
\]
because the trace is invariant under cyclic permutations. But we can also permute $\gamma_{a_{2m}}$ with the $2m-1$ other $\gamma$'s.  $\tr(\gamma_{a_1}\ldots\gamma_{a_{2m}})=(-1)^{2n-1}\tr(\gamma_{a_{2m}}\gamma_{a_1}\ldots\gamma_{a_{2m-1}})$ because each permutation gives an extra minus sign (\hbox{lemma~\ref{3101l1}}). Then the trace is zero.

If the number of nonzero $i_k$ is odd (say $2m-1$). Let $i_a=0$ (we restrict ourself to the even dimensional case). We have $\tr(A)=-\eta_{aa}\tr(A\gamma_a\gamma_a)$. Using once again the cyclic invariance of the trace, $\tr(\gamma_{a_1}\ldots\gamma_{a_{2m-1}}\gamma_a\gamma_a)=\tr(\gamma_a\gamma_{a_1}\ldots\gamma_{a_{2m-1}}\gamma_a)$. But, if we permute the \emph{first} $\gamma_a$ with the $2m-1$ first $\gamma$'s, we find \hbox{$\tr(\gamma_{a_1}\ldots\gamma_{a_{2m-1}}\gamma_a\gamma_a)=-\tr(\gamma_a\gamma_{a_1}\ldots\gamma_{a_{2m-1}}\gamma_a)$ }, and the trace is zero again.
\end{proof}

\begin{proposition}
The subset
\[
	\left\{\mtu,\dga{a}{b}\,(a<b),\tga{a}{b}{c}\,(a<b<c), \cdots, \gamma_{0}\cdots\gamma_{2n} \right\}
\]
 is free in $\End(\Lambda W)$.
\end{proposition}

\begin{proof}
We consider a general linear combination of these operators:
\[
 E=\lambda\mtu+\sum_a\lambda_a\gamma_a+\sum_{a<b}\lambda_{ab}\dga{a}{b}+\cdots+
 \sum_{a<b<c<d}\lambda_{abcd}\qga{a}{b}{c}{d}.
\]
The claim is that if $E=0$, then all the coefficients $\lambda_{(\ldots)}$ must be zero. First note that $Tr(E)=0=\lambda$ by lemma~\ref{3101l2}. It is also clear that $Tr(\gamma_iE)=0=\lambda_i$. In order to see that $\lambda_{ij}=0$, we compute $Tr(\gamma_j\gamma_iE)=0=\lambda_{ij}$. And so on.
\end{proof}

How many operators does we have in this free system? Any operators in this system can be written as $\gamma_{0}^{i_{0}},\cdots\gamma^{i_{2n-1}}_{2n-1}$ with $i_k$ equal to zero or one. Thus we have $2^{2n}$ operators. On the other hand, we know that $dim_{\eC}\Lambda W=2p+2$, and then that $dim_{\eC}\End(\Lambda W)=4^2=16$. The result is that $\{ \gamma_{0}^{i_{0}},\cdots\gamma^{i_{2n-1}}_{2n-1}  \tq i_k=0\,or\,1\}$ is a basis of $\End(\Lambda W)$. In other words (if we suppose a suitable ordering), the image by $\tilde\rho$ of
\[
B=\{1,e_a,e_a\otimes e_b,e_a\otimes e_b\otimes e_c,e_a\otimes e_b\otimes e_c\otimes e_d\}
\]
 is a basis of $\End(\Lambda W)$.

If $B$ is a basis of $C^{\eC}_{(p,q)}$, then $\tilde\rho$ is bijective and thus isomorphic.  Therefore, we expect $\dpt{\tilde\rho}{C^{\eC}_{(p,q)}}{\End(\Lambda W)}$ to be a faithful representation\index{representation!of Clifford algebra}. It is not difficult to see that $B$ is indeed a basis thanks to the equivalence relation.

\subsection{Explicit representation}
%------------------------------

First, we choose a basis for $\Lambda W$:
\begin{eqnarray}\label{102r2} 1=\left(\begin{matrix}
1 \\
0 \\
0 \\
0
\end{matrix}\right),\quad
f_0=\left(\begin{matrix}
0 \\
1 \\
0 \\
0
\end{matrix}\right),\quad
f_1=\left(\begin{matrix}
0 \\
0 \\
1 \\
0
\end{matrix}\right),\quad
f_0\wedge f_1=\left(\begin{matrix}
0 \\
0 \\
0 \\
1
\end{matrix}\right).
\end{eqnarray}
 Here is the explicit computation for the matrix $\gamma_0$ in this basis. First remark that $\tilde\rho(e_0)1=f_0$, $\tilde\rho(e_0)f_0=\frac{1}{2}$, $\tilde\rho(e_0)f_1=f_0\wedge f_1$, $\tilde\rho(e_0)(f_0\wedge f_1)=\frac{1}{2} f_1$. Then
\begin{equation}
\begin{split}
 \gamma_0\left(\begin{matrix}1 \\0 \\0 \\0\end{matrix}\right)=\sqrt{2}\left(\begin{matrix}0 \\1 \\0  \\0\end{matrix}\right),\quad
 \gamma_0\left(\begin{matrix}0 \\1 \\0 \\0\end{matrix}\right)=\sqrt{2}\left(\begin{matrix}\frac{1}{2} \\0 \\0 \\0\end{matrix}\right),\\
 \gamma_0\left(\begin{matrix}0 \\0 \\1 \\0\end{matrix}\right)=\sqrt{2}\left(\begin{matrix}0 \\0 \\0 \\1\end{matrix}\right),\quad
 \gamma_0\left(\begin{matrix}1 \\0 \\0 \\0\end{matrix}\right)=\sqrt{2}\left(\begin{matrix}0 \\0 \\\frac{1}{2} \\0\end{matrix}\right).
\end{split}
\end{equation}
This allows us to write down $\gamma_0$; the same computation gives the other matrices.\index{dirac!matrices}\nomenclature{$\gamma_i$}{Explicit form of gamma matrices}
\begin{equation}
\begin{split}
\gamma_0=\sqrt{2}\begin{pmatrix}
0 & \frac{1}{2} & 0 & 0 \\
1 & 0 & 0 & 0 \\
0 & 0 & 0 & \frac{1}{2} \\
0 & 0 & 1 & 0
\end{pmatrix}, \qquad
\gamma_1=\sqrt{2}\left(\begin{matrix}
0 & -\frac{1}{2} & 0 & 0 \\
1 & 0 & 0 & 0 \\
0 & 0 & 0 & -\frac{1}{2} \\
0 & 0 & 1 & 0
\end{matrix}\right),\\
\gamma_2=\sqrt{2}\begin{pmatrix}
0 & 0 & -\frac{1}{2} & 0 \\
0 & 0 & 0 & \frac{1}{2} \\
1 & 0 & 0 & 0 \\
0 & -1 & 0 & 0
\end{pmatrix},\qquad
\gamma_3=\sqrt{2}\begin{pmatrix}
0 & 0 & -\frac{i}{2} & 0 \\
0 & 0 & 0 & \frac{i}{2} \\
-i & 0 & 0 & 0 \\
0 & i & 0 & 0
\end{pmatrix}.
\end{split}
\end{equation}
It is easy to check that these matrices satisfies \eqref{3101r3}.

Notice that, up to a suitable change of basis in $\Lambda W $, these are the usual Dirac matrices\index{dirac!matrices}. Indeed we actually solved the physical problem to find a representation of the algebra \eqref{cliffphys}.  We understand by the way why do physicists work with $4$-components spinors: the $\gamma$'s are operators on the four-dimensional space $\Lambda W$; hence the Dirac operator will naturally acts on four-components objects.

The main result of this section is an explicit faithful representation of $\CCliff(p,q)$. This allows us to write a \defe{Dirac operator}{dirac!operator!on $\protect\eR^{1,3}$} which solve (see the invitation~\ref{Secqft} and \cite{Bronn}) the problem  to find a ``square root'' of the d'Alembert operator: the differential operator $\Dir=\gamma\hmu\partial_{\mu}$ satisfies $\Dir^2=\Box$.

\subsection{A remark}  % C'est ce qui arrive quand je ne sais pas quel titre donner.
%---------------------

 Let us compare the two faithful representations
\[
\begin{split}
  c\colon \Cliff(V)&\to \End_{\eR}(\wedge V)\\
\tilde\rho\colon \CCliff&\to  \End_{\eR}(\wedge W).
\end{split}
\]
They obviously comes from the same ideas. One common point is that
\[
  c(e_1)(e_1\wedge e_2)=2\tilde\rho (e_1)(e_1\wedge e_2)=e_2,
\]
but they are different:
\[
\begin{split}
  \tilde\rho(e_3)(e_0\wedge e_2)&=0\\
c(e_3)(e_0\wedge e_2)&=e_3\wedge e_0\wedge e_1.
\end{split}
\]

\subsection{General two dimensional Clifford algebra}
%---------------------------------------------------

The Clifford algebra for the metric
\[
  g=\begin{pmatrix}
\alpha&\delta\\\delta&\beta
\end{pmatrix}
\]
is realised by matrices
\[
  \gamma_1=
\epsilon\begin{pmatrix}
\sqrt{\alpha}\\ & -\sqrt{\alpha}
\end{pmatrix},\quad
\gamma_2=
\epsilon\begin{pmatrix}
\delta/\sqrt{\alpha}& \beta-\delta^2/| \alpha |\\
1		& -\delta/\sqrt{\alpha}
\end{pmatrix}
\]
where $\epsilon=\pm 1$ is chosen in such a way that $\epsilon| \alpha |=\alpha$.

\section{Spin group}
%+++++++++++++++++++

We will not immediately go on with Dirac operators on Riemannian manifolds because we still have to build some theory about the Clifford algebra itself. In particular, we have to define the spin group which will play a central role in the definition of the Dirac operator. Almost all --and (too?) much more-- the concepts we will introduce in this section can be found in \cite{Chevalley}; a more physical oriented but useful approach can be found in \cite{Preparation}.

Let define the map $\dpt{\chi}{\Gamma(p,q)}{GL(\eR^{1,3})}$ by
\begin{equation}
                \chi(x)y=\alpha(x)\cdot y\cdot x^{-1}.
\end{equation}
\nomenclature{$\chi$}{A representation of $\Gamma(p,q)$}\nomenclature[G]{$\Gamma(p,q)$}{Clifford group}
Let
\[
 \Gamma(p,q)=\{x\in\Cliff(p,q)\tq\textrm{$x$ is invertible and }  \chi(x)y  \in V\textrm{ for all $y\in V$}\}.
\]
It should be remarked that this definition comes back to the real Clifford algebra. The Clifford algebra product gives this subset a group structure which is called the \defe{Clifford group}{Clifford!group}. Any $x\in V$ is invertible since $x\cdot x=-\eta(x,x)1$, the inverse of $x$ is given by $x^{-1}=x/\|x\|^2$.

\index{Clifford!algebra!grading of}
The subset $\Cliff(p,q)^+$ (resp. $\Cliff(p,q)^-$) of $\Cliff(p,q)$ is the image of even (resp. odd) tensors of $T(V)$ by the canonical projection $T(V)\to\Cliff(p,q)$. With these definitions, we have a natural grading of $\Cliff$:
\begin{equation}
 \label{directC}\Cliff(p,q)=\Cliff(p,q)^+\oplus\Cliff(p,q)^-,
 \end{equation}
and the subgroups
\begin{align}
\label{defgplus}
\Gamma(p,q)^+&=\Gamma(p,q)\cap\Cliff(p,q)^+,&\Gamma(p,q)^-&=\Gamma(p,q)\cap\Cliff(p,q)^-.
\end{align}\nomenclature[G]{$\Cliff(p,q)^{\pm}$}{Grading of Clifford algebra}

For $x_1,\ldots,x_n\in V$, we have $\tau(x_1\cdots x_n)=x_n\cdots x_1$. \nomenclature[G]{$\Spin(p,q)$}{Spin group of $\eR^{1,3}$} The \defe{spin group}{spin!group!on $\protect\eR^{1,3}$} is
\begin{equation}   \label{defSpinun}
 \Spin(p,q)=\{x\in\Gamma(p,q)^+\vert\tau(x)=x^{-1}\}
\end{equation}
while the \defe{spin norm}{spin!norm}\nomenclature{$\dpt{N}{\Gamma(p,q)}{\Gamma(p,q)}$}{Spin norm} is the map $\dpt{N}{\Gamma(p,q)}{\Gamma(p,q)}$ defined by
\[
 N(x)=x\tau(\alpha(x)).
\]

\begin{proposition} \label{proppourN}
The map $N$ takes values in $\eR$ and the formula
\begin{equation}
             N(x\cdot y)=N(x)N(y),
\end{equation}
holds for all $x$, $y\in\Gamma(p,q)$.
\end{proposition}

\begin{proof}
We write as usual $x\in\Gamma(p,q)$ as $x=cv_1\cdots v_r$. So,
\begin{equation}
 N(x)=cv_1\cdots v_r\tau(\alpha(cv_1\cdots v_r))
     =\me{r}c^2v_1\cdots v_r\cdot v_r\cdots v_1.
\end{equation}
The first equality comes from the fact that $\alpha(cv_1\cdots v_r)=\me{r}cv_1\cdots v_r$. Now $N(x)\in\eR$ because $v_i\cdot v_i=-\brak{v_i}{v_i}\in\eR$ for all $i$. Hence the following hold:
\begin{equation}
\begin{split}
 N(x\cdot y)&=v\cdot y\cdot\tau(\alpha(v\cdot y))\\
            &=v\cdot y\cdot\tau(\alpha(y))\cdot\tau(\alpha(v))\\
            &=v\cdot N(y)\tau(\alpha(v))\\
            &=N(y)N(x).
\end{split}
\end{equation}
This is the claim.
\end{proof}

Therefore  $N\colon \Gamma(p,q)\to \eR$ is an homomorphism.

\begin{remark}
The elements of $\Spin(p,q)$ are spin-normed at $1$. Indeed, take a $s$ in $\Spin(p,q)$. We have $N(s)=s\cdot \tau(s)=1$ because $\alpha(s)=s$ and $\tau(s)=s^{-1}$. In particular $\Spin(p,q)\cap\eR=\eZ_{2}$.
\label{rem:spin_norm_u}
\end{remark}

\subsection{Studying the group structure}
%--------------------------------

\begin{proposition}
The set $\Gamma(p,q)$ admits a Lie group structure.
\end{proposition}
\begin{proof}

During this proof, $\mu$ denotes the Clifford multiplication: $\mu(x,y)=x\cdot y$. We know that $\Cliff^{\eC}(p,q)$ is isomorphic to $\End(\Lambda W)$ in which the multiplication is a continuous map. Thus $\mu$ is continuous on $\Cliff^{\eC}(p,q)$. But $\Cliff(p,q)$ is a closed subset of $\Cliff^{\eC}(p,q)$, so $\mu$ is a continuous map in $\Cliff(p,q)$. This proves that  $\chi$ seen as a map from $\Gamma(p,q)\times V$ to $V$ is a continuous map.

The space $V$ is closed in $\Cliff(p,q)$, thus $\sigma^{-1}(V)$ is also closed. But $\sigma^{-1}(V)=\Gamma(p,q)\times\Cliff(p,q)$. So $\Gamma(p,q)$ is closed in $\Cliff(p,q)$.

Now the result is just a consequence of theorems~\ref{Helgason2.3} and~\ref{Helgason4.2}. Indeed, let us study the subset $\mI$ which appears in the definitions of the Clifford algebra. It makes no difficult to convince ourself that it is a closed subgroup of $T(V)$. The theorem~\ref{Helgason4.2} thus makes $\Cliff(p,q)=T(V)/\mI$ a Lie group. But we just say that $\Gamma(p,q)$ is closed in $\Cliff(p,q)$, and the fact that $\Gamma(p,q)$ is a subgroup of $\Cliff(p,q)$ is clear. By theorem~\ref{Helgason2.3} we conclude that there exists a Lie group structure on $\Gamma(p,q)$.
\end{proof}

\begin{lemma}
The map $\chi$ is a homomorphism, in other words $\chi$ is a representation of $\Gamma(p,q)$.
\index{representation!of $\Gamma(p,q)$}
\end{lemma}

\begin{proof}
The following computation uses the fact that $\alpha$ is a homomorphism:
\[
\begin{split}
\chi(a\cdot b)y&=\alpha(a\cdot b)\cdot y\cdot (a\cdot b)^{-1}
               =\alpha(a)\cdot\alpha(b)y\cdot b^{-1}\cdot a^{-1}\\
               &=\alpha(a)\cdot\chi(b)y\cdot a^{-1}
               =\chi(a)\chi(b)y.
\end{split}
\]
\end{proof}
Let $y\in\Gamma(p,q)^-$ and $v\in V$. Where is $y\cdot v$? First note that $(y\cdot v)^{-1}=v^{-1}\cdot y^{-1}$, so that
\begin{equation}
\begin{split}
  \alpha(y\cdot v)\cdot w\cdot(y\cdot v)^{-1}&=-\alpha(y)\cdot v\cdot w\cdot v^{-1}\cdot y^{-1}\\
                                            &=-\alpha(y)\big( 2\eta(v,w)-w\cdot v \big)\cdot v^{-1}\cdot y^{-1}\\
					    &=-2\eta(v,w)\alpha(y)\cdot v^{-1}\cdot y+\alpha(y)\cdot w\cdot y^{-1}
\end{split}
\end{equation}
which belongs to $V$ because $y\in\Gamma(p,q)$. This reasoning shows that (apart for $0$), $y\cdot v\in\Gamma(p,q)^+$ if and only if $y\in\Gamma(p,q)^-$.

\begin{lemma}
If $x\in V$ is non-isotropic (i.e. $\eta(x,x)\neq 0$), the automorphism $\chi(x)$  is the orthogonal symmetry with respect to $x^{\perp}$.
\end{lemma}

We recall that\nomenclature{$x^{\perp}$}{Space orthogonal to $x$}
\[
  x^{\perp}=\{ y\in V\tq\eta(x,y)=0  \}.
\]
We will denote by $\sigma^x$ the orthogonal symmetry with respect to $x^{\perp}$.

\begin{proof}
When the operator $\sigma^x$ acts on $y$, it just change the sign of the ``$x$-part''\ of $y$. So we can write $\sigma^x y=y-2\eta(x,y) 1_x$, where $1_x:=x/\|x\|$. It should be checked if
$\chi(x)y=\alpha(x)\cdot y\cdot x^{-1}$ is equal to $y-2\eta(x,y) 1_x$ or not. We know that $x\cdot x=\eta(x,x)1=-\|x\|$. It follows that
\[
  x\cdot y+y\cdot x=2\eta(x,y)\frac{x\cdot x}{\|x\|}.
  \]
If we multiply this at right by $x^{-1}$, using the fact that $\alpha(x)=-x$, we find
\[
-\alpha(x)\cdot y\cdot x^{-1}=-y+2\eta(x,y) 1_x,
\]
which is precisely the identity we wanted to check.
\end{proof}

The following result will help us to identify subgroups of Clifford group as isometry groups.
\begin{theorem}[Cartan-Dieudonné theorem]
\index{Cartan-Dieudonné theorem}\index{theorem!Cartan-Dieudonné}
Each $\sigma$ in $O(1,3)$ can be written as
\hbox{$\sigma=\tau_1\circ\ldots\circ\tau_m$}, where the $\tau$'s are orthogonal symmetries with respect to hyperplanes which are orthogonal to non-isotropic vectors.
\label{CartanDieu}
\end{theorem}

\begin{proposition}
\[
              \chi(\Gamma(p,q))=O(p,q).
\]
\label{prop1001t1}
\end{proposition}

\begin{proof}
In order to show that $\chi(\Gamma(p,q))\subset O(p,q)$ take $z\in V$ and $x\in\Gamma(p,q)$. Since $\alpha(x)\cdot z\cdot x^{-1}$ lies in $V$, we can write:
\[
\alpha(x)\cdot z\cdot x^{-1}=-\alpha\left(\alpha(x)\cdot z\cdot x^{-1}\right)
=-x\cdot\alpha(z)\cdot\alpha(x^{-1})=x\cdot z\cdot\alpha(x^{-1}).
\]
In order to see that $\chi(x)\in O(p,q)$, we have to prove that $\left\|\chi(x)y\right\|_{(p,q)}^2=\|y\|_{(p,q)}^2$. This is achieved by the following computation:
\begin{equation}
\begin{split}
 \left\|\chi(x)y\right\|_{(p,q)}^2&=-\left(\alpha(x)\cdot y\cdot x^{-1}\right)^2
                                  =\left(\alpha(x)\cdot y\cdot x^{-1}\right)\left(x\cdot y\cdot\alpha(x^{-1})\right)\\
                                  &=-\alpha(x)\cdot y^2\cdot\alpha(x^{-1})
                                  =\|y\|^2_{(p,q)}.
\end{split}
\end{equation}
The last step is simply the fact that $y^2\in\eR$ and therefore commutes with anything. We now know that $\chi(x)\in O(p,q)$ for all $x\in\Gamma(p,q)$. Thus $\chi(\Gamma(p,q))\subset O(p,q)$.

For the second part, let $\sigma$ be in $O(p,q)$. The Cartan-Dieudonné theorem\index{Cartan-Dieudonné theorem}\index{theorem!Cartan-Dieudonné}(theorem~\ref{CartanDieu}) says that $\sigma=\sigma^{x_1}\circ\ldots\circ\sigma^{x_r}$ for some $x_1,\ldots, x_r$ in $V$. Thus $\sigma=\chi(x_1\cdots x_r)$, \hbox{and $O(p,q)\subset\chi(\Gamma(p,q))$}.
\end{proof}

\begin{proposition}
\begin{equation}
      \ker\chi=\eR\invtible
\end{equation}
where the right hand side is the set of invertible elements of $\eR$.
\label{prop1001p1}
\end{proposition}
\nomenclature{$\cA\invtible$}{The set of invertible elements of the algebra $\cA$; for example $\eR\invtible=\eR\setminus\{ 0 \}$}

\begin{proof}
Before beginning the proof, we want to insist on the fact that $x\in \ker\chi$ does not mean that $\chi(x)y=0$ for all $y$ in $V$. The ``zero''\ of an algebra is the element $e$ which satisfies $e\cdot y=y\cdot e=y$ for all $y$ in the algebra. In other words, $x$ is in the kernel of $\chi$ if and only if $\chi(x)=\id$.

First we show that $\eR_0\subset \ker\chi$. If $x\in \eR$, then $\alpha(x)=x$. Therefore, when $x\neq 0$,
\[
\chi(x)y=\alpha(x)\cdot y\cdot x^{-1}=y,
\]
because the algebra product $\cdot$ between an element of $\Cliff(p,q)$ and a real is commutative. Note that this does not work with $x=0$.

We are now going to show that $\ker\chi\subset\eR$. Let $z\in\ker\chi$. We decompose (definitions \eqref{defgplus}) it into his odd and even part: $z=z^++z^-$, with $z^{\pm}\in\Gamma(p,q)^{\pm}$. These two can be written as $z^+=e_{j_1}\cdots e_{j_{2r}}$ and $z^-=e_{i_1}\cdots e_{i_{2r-1}}$ with no two $i_k$ or $j_k$ equals. This is almost the general form of elements in even and odd part of $\Gamma(p,q)$: the only other possibility is $z$ in $\eR$. Obviously $\alpha(z^{\pm})=\pm z^{\pm}$. Multiplying the condition $\chi(z)y=y$ at right by $(z^++z^-)$, we find \[(z^+-z^-)y=y(z^++z^-).\] Thanks to equation \eqref{directC}, we can split this condition into even and odd parts:
\begin{align}
 z^+y&=yz^+,
 &z^-y&=-yz^-.
\end{align}
The first equation with $y=e_{j_1}$ gives $e_{j_1}\cdots e_{j_{2r}}\cdot e_{j_1}=e_{j_1}e_{j_1}\cdots e_{j_{2r}}$. In the left hand side, permute the last $e_{j_1}$ from last to second position. So we find the right hand side, with an extra minus sign. This means that $z^+=0$. In the same way, the second equation gives $z^-=0$. We are left with the last possibility: $z\in\eR$.
\end{proof}

\begin{corollary}
For any $s\in\Gamma(p,q)$, there exists some non-isotropic vectors $x_1,\ldots,x_r$, and $c\in\eR$ such that $s=cx_1\cdots x_r$.
\label{602c1}
\end{corollary}

\begin{proof}
Let us take a $s\in\Gamma(p,q)$; we just saw (theorem~\ref{prop1001t1}) that $\chi(s)$ is an element of $O(p,q)$. It can be written $\chi(s)=\sigma_1\circ\ldots\circ\sigma_m$. But we had shown that $\sigma_i=\chi(x_i)$ for any $x_i$ normal to the hyperplane defining $\sigma_i$. We thus have
\[
     \chi(s)=\chi(x_1\cdots x_m),
\]
where $s$ belongs to $\Gamma(p,q)$ and is therefore invertible. This leads us to write $\id=\chi(s^{-1}\cdot x_1\cdots x_m)$. But the kernel of $\chi$ is $\eR$ (proposition~\ref{prop1001p1}); so one can find a $r\in\eR$ such that $s^{-1}\cdot x_1\cdots x_m=r$. The claim follows.
\end{proof}

\begin{lemma}
If $v\in V$,
\begin{equation}
                 \det\chi(v)=-1.
\end{equation}
\end{lemma}

\begin{proof}
We already know that $det\chi(v)=\pm 1$. To check that the right sign is plus, take the following basis of $V$: $\{v,v_i^{\perp}\}$ where $\{v_i^{\perp}\}$ is a basis of $v^{\perp}$. Calculating the action of $\chi(v)$ on this basis, we find:
\begin{equation}
\begin{split}
 \chi(v)v&=-v\cdot v\cdot v^{-1}=-v,\\
 \chi(v)v_i^{\perp}&=-v\cdot v_i^{\perp}\cdot v^{-1}
                   =v_i^{\perp}\cdot v\cdot v^{-1}
                   =v_i^{\perp}.
\end{split}
\end{equation}
In this computation\nomenclature{$\sQ$}{A subgroup of $\sG$}, we used the relation $v\cdot w=-w\cdot v-2\brak{v}{w}$ which is true for all $v$, $w$ in $V$. The action of $\chi(v)$ on this basis is thus to let unchanged three vectors and to change the sign of the fourth. This proves the claim.
\end{proof}

\begin{theorem}
\begin{equation}
                   \chi(\Gamma(p,q)^+)=\SO(p,q).
\end{equation}
\label{2102p1}
\end{theorem}

\begin{proof}
From corollary~\ref{602c1}, and definition~\ref{defgplus}, an element $s\in\Gamma(p,q)^+$ reads $s=cv_1\cdots v_{2r}$. Thus
\begin{equation}
 \det\chi(s)=\det\chi(v_1\cdots v_{2r})
            =\det\left[\chi(v_1)\ldots\chi(v_{2r})\right].
\end{equation}
 But we know that, for all $v_i$ in $V$, $det\chi(v_i)=-1$. So $\det\chi(s)=1$ and $\chi(\Gamma(p,q)^+)\subseteq \SO(p,q)$. As set,
\[
  \Gamma(p,q)=\Gamma(p,q)^+\cup\Gamma(p,q)^-,
\]
but the lemma shows that $\det\chi(\Gamma(p,q)^-)=-1$ so, from theorem~\ref{prop1001t1}, $\chi(\Gamma(p,q)^+)$ must be the whole $\SO(p,q)$.
\end{proof}


\begin{theorem}
We have the following isomorphism of groups
\[
  \Spin(p,q)=\SO_0(p,q).
\]
provided by the map $\chi$.
\end{theorem}

\begin{probleme}
	This result is wrong because of a double covering issue. The real proposition is the next one. I should try to merge the proofs.
\end{probleme}

\begin{proof}
Let $\{ e_1,\cdots,e_p,f_1,\cdots,f_p \}$ be a basis of $\eR^{p+q}$ where the $e_i$'s are time-like and the $f_j$'s are space-like.
Following the discussion at page \pageref{PgDisGeoConnSO}, we have
\[
  \SO(p,q)=\SO_0(p,q)\cup\xi \SO_0(p,q)
\]
where $\xi$ is defined as follows: $\xi e_1=-e_1$, $\xi f_1=-f_1$ and $\xi e_k=e_k$, $\xi f_k=f_k$ for $k\neq 1$. This element can be implemented as $\xi=\chi(g)$ for $g=e_1f_1$. It is easy to see that $g^{-1}=-f_1e_1$ and that $\tau(g)=f_1e_1$, so that $g\notin\Spin(p,q)$.

Is it possible to find another $h\in\Gamma(p,q)$ such that $\chi(h)=\xi$? If $\chi(a)=\chi(b)$ for $a$, $b\in\Gamma(p,q)$, then $a=rb$ for a certain $r\in\eR$. So we find that $h=g^{-1}/r$ is the general form of an element in $\Gamma(p,q)$ such that $\chi(h)=\xi$. This is an element of $\Spin(p,q)$ if and only if $\tau(h)=h^{-1}$, or $-e_1f_1/r=re_1f_1$ which has no solutions. We conclude that no element of $\Spin(p,q)$ is send on $\xi$ by $\chi$. So
\[
  \chi\big( \Spin(p,q) \big)\subset SO_0(p,q).
\]

\begin{probleme}
	I still have to prove the surjectivity of $\chi$ from $\Spin(p,q)$ to $\SO(p,q)$.
\end{probleme}

\end{proof}
\begin{theorem}
\begin{equation}	\label{EqchiSpinSO}
             \chi(\Spin(p,q))=\SO_0(p,q)
\end{equation}
where the index $0$ means the identity component.
\end{theorem}

\begin{proof}
Proposition~\ref{prop1001p1}, theorem~\ref{2102p1} and remark~\ref{rem:spin_norm_u} show that the map $\dpt{\chi}{\Spin(p,q)}{\SO(p,q)}$ is a homomorphism with $\mathbb{Z}_2$ as kernel. We begin to prove that $\dpt{\chi}{\Spin(p,q)}{\SO_0(p,q)}$ is surjective. On the one hand, elements of $\Spin(p,q)$ satisfy one more condition than the ones of $\Gamma(p,q)^+$. Thus the algebra $\Spin(p,q)$ has codimension one in $\Gamma(p,q)^+$.

On the other hand, we know that $\SO(p,q)=\SO_0(p,q)\cup h\SO_0(p,q)$ where $h$ is the matrix such that $he_i=-e_i$ for $i=0,\ldots,3$. Since $\Spin(p,q)$ has codimension one in $\Gamma(p,q)^+$, there is at most one more generator in $\chi(\Gamma(p,q)^+)$ than in $\chi(\Spin(p,q))$ (because $\chi$ is a homomorphism). In order to prove this theorem, we just need to show that elements of $\chi(\Gamma(p,q)^+)$ which do not belong to $\chi(\Spin(p,q))$ is $h$.

Is is no difficult to see that $\chi(e_0\cdot e_1\cdot e_2\cdot e_3)e_i=-e_i$ for $i=0\ldots 3$: just write
\begin{equation}
\chi(e_0\cdot e_1\cdot e_2\cdot e_3)e_i=e_0\cdot e_1\cdot e_2\cdot e_3\cdot e_i\cdot e_3^{-1}\cdot e_2^{-1}\cdot e_1^{-1}\cdot e_0^{-1}
\end{equation}
and use the commutation relations. An easy computation gives
$N(e_0\cdot e_1\cdot e_2\cdot e_3)=-1$; then this is not in $\Spin(p,q)$ by remark~\ref{rem:spin_norm_u}.
\end{proof}

We write it by the exact sequence
\begin{equation}
 \xymatrix{
    \eZ_2  \ar@{^{(}->}[r] & \Sppq \ar[r]^{\chi} & \SO_0(p,q)
  }
\end{equation}
we say that the group $\Spin(p,q)$ is a \defe{double covering}{double covering!of $\SO_{0}(p,q)$} of $\SO_0(p,q)$.

\begin{lemma}
If $\dpt{\pi}{\tX}{X}$ is a covering which satisfies

\begin{enumerate}
\item $X$ is path connected,
\item $\forall x\in X$, $\tX_x:=\pi^{-1}(x)$ is path connected in $\tX$ \emph{i.e.} for all $a$, $b\in \tX$, there exist a path in $\tX$ which joins $a$ and $b$,
\end{enumerate}
then $\tX$ is path connected.
\label{lem_cov_path_con}
\end{lemma}
\begin{proof}
If $\tx$ and $\ty$ are in $\tX$, we can suppose that $\pi(\tx)\neq\pi(\ty)$ (because if $\pi(\tx)=\pi(\ty)$, the second assumption gives the thesis). We define $x$ and $y$ as their projections: $x=\pi(\tx)$ and $y=\pi(\ty)$. Let $\gamma$ be a path such that $\gamma(0)=x$ and $\gamma(1)=y$, and $\tgamma$ be the lift of $\gamma$ in $\tX$ which contains $\tx$: $\tgamma(0)=\tx$ and $\pi(\tgamma(1))=\gamma(1)=y$. Then $\tgamma(1)$ lies in $\tX_y$. Therefore, we can consider $\gamma'$ which joins $\tgamma(1)$ and $\ty$.

So, $\gamma'\circ\tgamma$ is a path which contains $\tx$ and $\ty$.
\end{proof}


\begin{proposition}
 The group $\Spin(p,q)$ is connected.
\end{proposition}

\begin{proof}
We will prove that the covering $\dpt{\chi}{\Spin(p,q)}{\SO_0(p,q)}$ fulfils lemma~\ref{lem_cov_path_con}. We just have to show that $\Spin(p,q)$ fulfills the second assumption of the lemma. First note that $\chi(\tx)=\chi(\ty)$ implies $\chi(\tx\ty^{-1})=e$, and then $\tx=\pm\ty$ because of proposition~\ref{prop1001p1}. Since the other case is trivial, we can suppose $\tx=-\ty$.

It remains to prove that for every $g\in\Spin(p,q)$, there is a path in $\Spin(p,q)$ which joins $g$ and $-g$. The answer is given by the path $t\mapsto \gamma(t)g$ where
\[
\gamma(t)=\exp(te_1\cdot e_2)=\cos(t)(-1)+\sin(t)e_1\cdot e_2
\]
which satisfies $\gamma(0)=1$ and $\gamma(\pi)=-1$.
\end{proof}

\begin{proposition}

The homomorphism $\tilde\rho$ restricts to a homomorphism $\tilde\rho\colon \Spin(p,q)\to \GL(\Lambda^+W)$.
\end{proposition}

\begin{proof}
An element in $\Spin(p,q)$ reads $s=cv_1\cdots v_{2r}$ and its image by $\tilde\rho$ is
\[
  \tilde\rho(s)=c\tilde\rho(v_1)\circ \cdots \circ\tilde\rho(v_{2r}).
\]
When one applies $\tilde\rho(v_1)$ to an element $\alpha\in\Lambda^kW$, one obtains a linear combination of an element of $\Lambda^{k-1}W$ and one of $\Lambda^{k+1}W$. The element $\tilde\rho(s)$ being an even composition of such maps, its transforms an element of $\Lambda^+W$ into an element of $\Lambda^+W$.
\end{proof}

Notice that an element of $V$ ---no $V^{\eC}$--- is represented on $\Lambda^+W$ by complex matrices. This is not a problem. In the case of $\eR^{1,3}$, we have $\dim\Lambda^+W=2$ and thus
\[
  \tilde\rho\big( \Spin(1,3) \big)\subset \GL(2,\eC).
\]
The following is the lemma 8.5 (page 57) of \cite{Michelson}.

\begin{lemma}
Let $\rho\colon \Cl(p,q)\to \Hom_{\eC}(E,E)$ be a representation of the Clifford algebra on a vector space $E$. If $p+q\geq 2$, then for all $s\in\Spin(p-1,q)\subset \Cl(p,q) $,
\[
  \det{}_{\eC}\big( \rho(s) \big)=\pm 1.
\]

\end{lemma}
\begin{proof}
No proof.
\end{proof}

\begin{theorem}
The representation $\tilde\rho$ provides a group isomorphism
\[
  \Spin(1,3)\simeq \SL(2,\eC)
\]

\end{theorem}

\begin{proof}
In the case $p=2$, $q=3$, the lemma assures us that for each $s$ in the spin group, $\det\tilde\rho(s)=1$. Since $\Spin(1,3)$ is connected and the determinant function is continuous, we deduce that $\det\tilde\rho(s)\equiv 1$. This proves that $\tilde\rho\big( \Spin(1,3) \big)\subset \SL(2,\eC)$. The proposition~\ref{PropUssGpGenere} thus implies that
\[
  \tilde\rho\big( \Spin(1,3) \big)=\SL(2,\eC),
\]
 but from $\Cl(1,3)$, the representation $\tilde\rho$ is yet injective. \emph{A forciori}, the representation $\tilde\rho$ is injective from $\Spin(1,3)$. This finishes the proof.
\end{proof}

% This is part of (almost) Everything I know in mathematics
% Copyright (c) 2013-2014,2016,2020
%   Laurent Claessens
% See the file fdl-1.3.txt for copying conditions.

\subsection{Redefinition of \texorpdfstring{$\Spin(V)$}{Spin(V)}}
%----------------------------------------------------------------

As it, this new definition only holds when $g$ is positive defined.

\begin{probleme}
	When we work with a signature $(p,q)$, maybe we only get the connected part. To be checked.
\end{probleme}

Let us take $v$, $x\in V$ with $g(v,v)=1$. We have
\[
  -vxv^{-1}=-vxv=-2g(x,v)v+xv^2
		=x-2g(x,v)v\in V.
\]
The effect was to reverse the $v$ component of $x$; the map $x\mapsto -vxv^{-1}$ is $\sigma^v$. Now, when $\lambda\in U(1)$ and $w=\lambda v$, we also have that $x\mapsto -wxw^{-1}$ is $\sigma^v$. Now we look at $\chi(a)\colon x\mapsto \alpha(a)xa^{-1}$ with $a=w_{1}\ldots w_{r}$, a product of unitary vectors in $V^{\eC}$. Explicitly,
\[
  \chi(a)x=(-1)^{r} w_{1}\ldots w_{r}xw_{r}^{-1}\ldots w_{1}^{-1},
\]
a composition of reflexions in $V$. When $r$ is even, it is a rotation. We conclude that when $a$ is an even product of unitary vectors in $V^{\eC}$, then $\chi(a)\in \SO(V)$. Theorem~\ref{CartanDieu} states that any rotation of $V$ is a composition of reflexions. So we define\nomenclature[G]{$\Spin^{c}(V)$}{A group related to $\Spin$}
\begin{equation}
\Spin^{c}(V)=\{ w_{1}\ldots w_{2k}\tq w_{j}\in V^{\eC},\,w_{j}^*w_{j}=1 \}\subset \CCliff^{0}(V),
\end{equation}
and $\chi\colon \Spin^{c}(V)\to \SO(V)$ is a surjective group homomorphism. The inverse in $\Spin^{c}(V)$ is given by
\[
  (w_{1}\ldots w_{2k})^{-1}=w_{2k}^*\ldots w_{1}^*=\overline{ w_{2k} }\ldots\overline{ w_{1} }.
\]
In the real case, proposition~\ref{prop1001p1} says that $\ker\chi=\eR\invtible$. In the complex case we get  $\ker\chi=\eC\invtible$ and, when we look at $\ker\chi|_{\Spin^{c}(V)}$, we find
\begin{equation}
\ker\chi=U(1).
\end{equation}
Then we find the short exact sequence
\begin{equation}
\xymatrix{%
   1 \ar[r]^-{\id}&U(1) \ar[r]^-{\id}&\Spin^{c}(V) \ar[r]^-{\chi}&\SO(V)\ar[r]^-{\id}&1.
}
\end{equation}
Let $u=w_{1}\ldots w_{2k}\in\Spin^{c}(V)$ with $w_{j}=\lambda_{j}v_{j}$ and $\lambda_{j}\in V$, so $\tau(u)=w_{2k}\ldots w_{1}$ and
\[
  \tau(u)u=w_{2k}\ldots w_{1}w_{1}\ldots w_{2k}
		=\lambda_{1}^{2}\ldots \lambda_{2k}^{2}\in U(1).
\]
This proves that $\tau(u)u$ is central in $\Spin^{c}(V)$. We define the homomorphism
\begin{equation}
\begin{aligned}
\nu \colon \Spin^{c}(V)&\to U(1) \\
u&\mapsto \tau(u)u.
\end{aligned}
\end{equation}
This is a homomorphism because
\[
\begin{split}
  \nu(u_{1}u_{2})&=\tau(u_{1}u_{2})u_{1}u_{2}
		=\tau(u_{2})\underbrace{\tau(u_{1})u_{1}}_{\text{central}}u_{2}
		=\tau(u_{2})u_{2}\tau(u_{1})u_{1}\\
		&=\nu(u_{2})\nu(u_{1})
		=\nu(u_{1})\nu(u_{2}).
\end{split}
\]
The map $\nu$ naturally restricts to $U(1)$ as
\[
  \nu(\lambda)=\lambda^{2}.
\]
The combined map $(\chi,\nu)\colon \Spin^{c}(V)\to \SO(V)\times U(1)$ has kernel $\{ \pm 1 \}$. We define\nomenclature[G]{$\Spin(V)$}{The spin group}
\begin{equation}  \label{eq_defSpindeux}
\Spin(V)=\ker\nu|_{\Spin^{c}(V)}.
\end{equation}

\begin{lemma}
This group is the same as the one defined in equation \eqref{defSpinun}.
\end{lemma}

\begin{proof}
Let $u\in\Spin(V)$ (in the sense of equation \eqref{eq_defSpindeux}). The fact for $u$ to belongs to $\Spin(V)$ implies the two following:
\begin{enumerate}
\item $u\in\Spin^{c}(V)\Rightarrow u^*u=1$,
\item $u\in\ker\nu\Rightarrow \tau(u)u=1$.
\end{enumerate}
The second point says that $u^{-1}=\tau(u)$, which is a first good point to fit the first definition of $\Spin(V)$. Now we have to prove that $u\in\Gamma^{+}(V)$: $u$ must be invertible and $\chi(u)x$ must belongs to $V$ for all $x\in V$. These two points are contained in the definition of $\Spin^{c}(V)$.
\end{proof}
Let us see in the new definition how is $\chi\colon \Spin(V)\to \SO(V)$. On $\Spin^{c}(V)$, we have $\ker\chi=U(1)$, but on $\Spin(V)$ we require moreover $\tau(u)u=1$, thus an element of $\ker\chi$ in $\Spin(V)$ fulfils $\tau(\lambda)\lambda=1$, so that $\lambda=\{ \pm1 \}$. We conclude that $\ker\chi|_{\Spin(V)}=\{ \pm 1 \}$, and then that $\Spin(V)$ is a double covering of $\SO(V)$.\index{double covering!of $\SO(V)$}


\subsection{A few about Lie algebra}
%----------------------------------

\nomenclature[G]{$\spin(p,q)$}{Lie algebra of the group $\Spin(p,q)$}
\begin{proposition}
We have an isomorphism
\[
                    \spin(p,q)\simeq\so(p,q)
\]
between the Lie algebras of $\Spin(p,q)$ and $\SO(p,q)$.
\label{prop:spin_so}
\end{proposition}

\begin{proof}
Using the second part of lemma~\ref{Helgason5.1}, with the map $\dpt{\chi}{\Spin(p,q)}{\SO(p,q)}$, we find that $d\chi_e(\spin(p,q))=\so(p,q)$. Then we know (lemma~\ref{1203r1}) that
\[
	\so(p,q)=\spin(p,q)/\ker\,d\chi_e.
\]
On the other hand, the first part of the same lemma gives us that $\chi^{-1}(e)$ is a Lie subgroup of $\Spin(p,q)$ whose Lie algebra is $\ker\,d\chi_e$. But $\chi^{-1}(e)=\eZ_2$, so $\ker\,d\chi_e=\{0\}$.
\end{proof}

Let us now shortly speak about the Lie algebra of $\Gamma(p,q)^+$. A basis of $\Cliff(p,q)^+$ is \[\{1,\gamma_0\cdot\gamma_1,\gamma_0\cdot\gamma_1 ,\gamma_0\cdot\gamma_3
,\gamma_0\cdot\gamma_1\cdot\gamma_2\cdot\gamma_3  \}.\] Thanks to the anticommutation relations, we don't need $\gamma_1\cdot\gamma_2$ in the basis.

Remember that $\Gamma(p,q)^+$ is the set of the $x\in\Cliff^+(p,q)$ such that $x\cdot v\cdot\alpha(x^{-1})$ lies in $V$ for all $v\in V$. Let $x(t)$ be a path in $\Gamma(p,q)^+$ such that $x(0)=e$ and $\dot{x}(0)=X$. Differentiating the definition relation, we find
 \[
 \dot{x}\cdot v\cdot\alpha(x^{-1})|_0+x\cdot v\cdot(-)\alpha(\dot{x})|_0=X\cdot v-v\cdot X,
 \]
 therefore\nomenclature[G]{$\Lie{\Gamma(p,q)^+}$}{Algèbre de $\Gamma(p,q)^+$}
\[
  \Lie{\Gamma(p,q)^+}=\left\{X\in\Cliff^+(p,q)\textrm{ such that } X\cdot v-v\cdot X\in V,\,\forall v\in V\right\}.
\]

It is clear that $\eC$ is a subset of $\Lie{\Gamma(p,q)^+}$, and that $V$ is not. The following computation shows that $V\cdot V$ is a subset $\Lie{\Gamma(p,q)^+}$:
\[
         a\cdot b\cdot v-v\cdot a\cdot b=2\eta(v,a)b-2\eta(v,b)a.
\]
 We can also check that $V\cdot V\cdot V\cdot V\cap\Lie{\Gamma(p,q)^+}=\emptyset$. A basis of $\Lie{\Gamma(p,q)^+}$ is
\[
	\{ 1,e_{\alpha}\cdot e_{\beta}\tq \alpha<\beta \}
\]

 We know that $\ker[\dpt{\chi}{\Gamma(p,q)^+}{\SO(p,q)}]=\eR_0$. So the kernel of the restriction of $d\chi_e$ to $\Lie{\Gamma(p,q)^+}$ is the Lie algebra of $\eR_0$ (see lemma~\ref{Helgason5.1}), which is $\eR$. Therefore, a basis of $\spin(p,q)$ is
\[
	\{e_{\alpha}\cdot e_{\beta}\tq \alpha<\beta\}.
\]

\subsection{Grading \texorpdfstring{$\Lambda W$}{LW}}
%-------------------------------

We already know that $\Lambda W=\eC\oplus W\oplus\Lambda^2W$. This space can be written as \[\Lambda W =\Lambda W^+\oplus\Lambda W^-,\] with $\Lambda W^+=W$ and $\Lambda W^-=\eC\oplus\Lambda^2W$. The interest of such a decomposition lies in the definition of an action of $\Cliff^+(p,q)$ on $\Lambda W $. This action will be defined by $\dpt{\bullet}{\Cliff^+(p,q)\times\Lambda W }{\Lambda W }$,
  \[
 x\bullet\alpha=\tilde\rho(x)\alpha
 \]
for any $x$ in $\Cliff^+(p,q)$ and any $\alpha$ in $\Lambda W $ (see definition~\ref{defrt}).

\begin{proposition}
This action preserves the grading of $\Lambda W $:
\begin{equation}
\begin{split}
 \Cliff^+(p,q)\bullet\Lambda W^+&=\Lambda W^+\\
 \Cliff^+(p,q)\bullet\Lambda W^-&=\Lambda W^-.
\end{split}
\end{equation}

\end{proposition}
\begin{proof}
For $x\in\eC$, theses equalities are obvious. We have to check it for $x=e_i\cdot e_j$. Here, we will just check that $(e_1\cdot e_0)\bullet(v\wedge w)\in\Lambda W^+$. This follows from a simple computation:
\begin{equation}
\begin{split}
\tilde\rho(e_1)\tilde\rho(f_0+g_0)(v\wedge w)&=
                         \tilde\rho(f_1+g_1)\left[-\eta(g_0,v)w+\eta(g_0,w)v\right]\\
                    &=-\eta(g_0,v)f_1\wedge w+\eta(g_0,w)f_1\wedge v\\
                    &\quad+\eta(g_0,v)\eta(g_1,w)-\eta(g_0,w)\eta(g_1,v).
\end{split}
\end{equation}
\end{proof}

Since $\Spin(p,q)$ is a subgroup of $\Cliff^+(p,q)$, the following two are representations of \( \Spin(p,q)\) : $\dpt{\rho^{\pm}}{\Spin(p,q)\times\Lambda W ^{\pm}}{\Lambda W ^{\pm}}$,
\begin{equation}
\begin{split}
 \rho^-(s)w^-&=\tilde\rho(s)w^-,\\
 \rho^+(s)w^+&=\tilde\rho(s)w^+,
\end{split}
\end{equation}
for $w^{\pm}$ in $\Lambda W ^{\pm}$. This is no more than the fact that $\tilde\rho$ is reducible and that two invariant subspaces are $\Lambda W^+$ and $\Lambda W^-$.
\subsection{Clifford algebra for \texorpdfstring{$V=\eR^2$}{V=R2}}\label{cliffR2}
%----------------------------------------------

\subsubsection{General definitions}
%/////////////////////////////////

The whole construction can also be applied to $V=\eR^2$ with the Euclidean metric. This is our business now. We take the complex vector space $V^{\eC}$ and an orthonormal basis $\{e_1,e_2\}$. As before, we define
 \[
f_1=\frac{1}{2}(e_1+ie_2),\qquad g_1=\frac{1}{2}(e_1-ie_2).
\]
There are no difficulties to see that $Span(f_1)$ is a completely isotropic subspace\index{isotropic!subspace!in $\eR^{2}$} of $V^{\eC}$. Thus we define $W=\eC f_1$, $\Lambda W =\eC\oplus W$, $\Lambda W^+=\eC$, and $\Lambda W^-=W$\nomenclature{$\Lambda W^{\pm}$}{Spinor space}. The homomorphism $\dpt{\tilde\rho}{V^{\eC}}{\End(\Lambda W )}$\nomenclature{$\dpt{\tilde\rho}{(\eR^2)^{\eC}}{\End(\Lambda W )}$}{Spinor representation} in $\Lambda W $ is defined by
\begin{equation}
\begin{split}
 \tilde\rho(f_1)\alpha&=f_1\wedge\alpha,\\
 \tilde\rho(g_1)\alpha&=-i(g_1)\alpha,
\end{split}
\end{equation}
where $\alpha$ is any element of $\Lambda W $. In the basis $1=\begin{pmatrix}
1 \\
0
\end{pmatrix} $ and $f_1=\begin{pmatrix}
0 \\
1
\end{pmatrix} $, we easily find that
\[
 \tilde\rho(e_1)=\begin{pmatrix}
 0 & -\frac{1}{2} \\
 1 & 0
 \end{pmatrix},\quad\tilde\rho(e_2)=\begin{pmatrix}
 0 & -\frac{i}{2} \\
 -i & 0
 \end{pmatrix}.\]
For $c\in\eR$ we	 also have $\tilde\rho(c)f_1=cf_1$ and $\tilde\rho(c)1=c$, thus we assign the matrix $\begin{pmatrix}
c & 0 \\
0 & c
\end{pmatrix}$ to the number $c$.

As before, we define $\gamma_i=\sqrt{2}\tilde\rho(e_i)$. We immediately have $\gamma_1\gamma_2+\gamma_2\gamma_1=0$ and $\gamma_i\gamma_i=-2\mtu$, so that the $\gamma$'s satisfy the Clifford algebra for the euclidian metric.

For notational conveniences, it proves useful to make a change of basis so that we get
\begin{equation}\label{gammaR2}
\gamma_1=\begin{pmatrix}
0 & -1 \\
1 & 0
\end{pmatrix},\quad\gamma_2=-\begin{pmatrix}
0 & i \\
i & 0
\end{pmatrix}.
\end{equation}

The algebra $\Cliff(2)$\nomenclature[G]{$\Cliff(2)$}{Clifford algebra of $\eR^2$} is isomorphic to the algebra which is generated by direct sum $\Cliff(2)\simeq\eR\oplus\gamma_1\oplus\gamma_2\oplus\eR\gamma_1\gamma_2$. A general element of $\Cliff(2)$ can be written as $x\gamma_1+y\gamma_2+x'\eR+y'\gamma_1\gamma_2$. In the representation of $\tilde\rho$, a general element of $\Cliff(2)$ is therefore
\[\begin{pmatrix}
x'+iy' & x+iy \\
-x+iy & x'-iy'
\end{pmatrix},\] so that we can write the Clifford algebra of $\eR^2$ as\index{algebra!Clifford}
\[
\Cliff(2)=\left\{\begin{pmatrix}
 \alpha & \beta \\
 -\obeta & \oalpha
 \end{pmatrix}\,:\,\alpha,\beta\in\eC\right\}.
\]
The following four matrices provide a basis:
\begin{align}\label{pauli}
1&=\begin{pmatrix}
1 & 0 \\
0 & 1
\end{pmatrix}, &i&=\begin{pmatrix}
-i & 0 \\
0 & i
\end{pmatrix},&j&=\begin{pmatrix}
0 & i \\
i & 0
\end{pmatrix},&k&=\begin{pmatrix}
0 & 1 \\
-1 & 0
\end{pmatrix}.
\end{align}
We can check that these matrices satisfies the quaternionic algebra\index{quaternion!algebra}\index{algebra!quaternion}:
\begin{equation}
\begin{split}
i^2&=j^2=k^2=-1\\
ij &=-ji=k,\\
jk &=-kj=i,\\
ki &=-ik=j.
\end{split}
\end{equation}
The algebra $\Cliff(2)=\eH$\nomenclature{$\eH$}{quaternionic algebra} is represented by $\tilde\rho$ on $\eC^2$ by the \defe{Pauli matrices}{pauli matrices} $1,i,j,k$ which are given by \eqref{pauli}.

\subsubsection{The maps \texorpdfstring{$\alpha$}{a} and \texorpdfstring{$\tau$}{t}}
%///////////////////////////////////////////////

What are the matrices which represent $V$? These are $\tilde\rho(e_1)$ and $\tilde\rho(e_2)$. Thus we can write $V=\Span_{\eR}\{\gamma_1,\gamma_2\}=\Span_{\eR}\{j,k\}$, or
\[
 V=\left\{\begin{pmatrix}
 0 & \xi \\
 -\oxi & 0
 \end{pmatrix}\,:\,\xi\in\eC\right\}.
\]

As before, $\alpha$ is the unique homomorphic extension to $\Cliff(2)$ of $-\id$ on $V$. From the definitions, we get $\alpha(j)=-j$, $\alpha(k)=-k$.
The extension present no difficult. For example: $\alpha(i)=\alpha(jk)=\alpha(j)\alpha(k)=jk=i$, but $\alpha(jk)=\alpha(i)$; then $\alpha(i)=i$. The same gives $\alpha(1)=1$.

The case of $\tau$ is treated in similar way. We find: $\tau(j)=j$, $\tau(k)=k$, $\tau(i)=-i$, $\tau(1)=1$.

Now, we can find the group $\gud$. The condition for $x\in\Cliff(2)$ to be in $\gud$ is $\alpha(x)yx^{-1}$ to belongs to $V$ for all $y\in V$. We put
\[ x=\begin{pmatrix}
\alpha & \beta \\
-\obeta & \oalpha
\end{pmatrix},\qquad\alpha(x)=\begin{pmatrix}
\alpha & -\beta \\
\obeta & \oalpha
\end{pmatrix}.\]
A typical $y$ in $V$ is
\[
 y=\begin{pmatrix}
 0 & \eta \\
 -\oeta & 0
 \end{pmatrix}.
\]
A few computation gives:
\[
 \alpha(x)yx^{-1}=\us{|\alpha|^2+|\beta|^2}\begin{pmatrix}
 \alpha\eta\obeta+\beta\oeta\oalpha & \alpha\alpha\eta-\beta\beta\oeta \\
 \obeta\obeta\eta-\oalpha\oalpha\eta & \eta\alpha\obeta+\oalpha\oeta\beta
 \end{pmatrix}.
\]
If we impose it to be of the form $\begin{pmatrix}
0 & \xi \\
-\oxi & 0
\end{pmatrix} $ for all $\eta\in\eC$, we get, for all $\eta\in\eC$,
 $\real(\oalpha\beta\oeta)=0$, which implies $\oalpha\beta=0$. So we conclude:
\[
 \gud=\left\{\begin{pmatrix}
 \alpha & 0 \\
 0 & \oalpha
 \end{pmatrix}, \begin{pmatrix}
 0 & \beta \\
 -\obeta & 0
 \end{pmatrix}\,:\,\alpha,\beta\in\eC\textrm{ not both equals zero}\right\}.
\]
Be careful on a point: $\gud$ is the \emph{multiplicative} group generated by these two matrices, not the additive one.

\subsubsection{The spin group}
%////////////////////////////

It present no difficult to find that
\begin{equation}
 \gud^+=\left\{\begin{pmatrix}
 \alpha & 0 \\
 0 & \oalpha
 \end{pmatrix}\,:\,\alpha\neq 0\right\}.
\end{equation}
The \defe{spin group}{spin!group!on $\protect\eR^2$} is made of elements of $\gud^+$ which satisfy $\tau(x)=x^{-1}$. We know that
$\tau\begin{pmatrix}
\alpha & 0 \\
0 & \oalpha
\end{pmatrix} =\begin{pmatrix}
\oalpha & 0 \\
0 & \alpha
\end{pmatrix}$ and that $\begin{pmatrix}
\alpha & 0 \\
0 & \oalpha
\end{pmatrix}^{-1} =\us{\displaystyle\alpha\oalpha}\begin{pmatrix}
\oalpha & 0 \\
0 & \alpha
\end{pmatrix}$. Thus the condition \hbox{$\tau(x)=x^{-1}$} becomes $|\alpha|^2=1$. The first conclusion is that
\begin{equation}
                    \Spin(2)=U(1).
\end{equation}
A typical $s$ in $\Spin(2)$ is
\[s=e^{i\theta}=\begin{pmatrix}
e^{i\theta} & 0 \\
0 & e^{-i\theta}
\end{pmatrix}.\]

The next point is to see the action of $\Spin(2)$ on $V$.\index{action!of $\Spin(2)$ on $\eR^2$} The action of $s\in\Spin(2)$ on a vector $v\in V$ is still defined by $s\bullet v=\chi(s)v=\alpha(s)\cdot v\cdot s^{-1}$. More explicitly:
\begin{equation}
 \chi(s)v=\begin{pmatrix}
 e^{i\theta} & 0 \\
 0 & e^{-i\theta}
 \end{pmatrix} \begin{pmatrix}
 0 & z \\
 -\overline{z} & 0
 \end{pmatrix} \begin{pmatrix}
 e^{-i\theta} & 0 \\
 0 & e^{i\theta}
 \end{pmatrix}=\begin{pmatrix}
 0 & e^{2i\theta}z  \\
 -e^{-2i\theta}\overline{z} & 0
 \end{pmatrix},
\end{equation}
where the  matrix $\begin{pmatrix}
0 & z \\
\overline{z} & 0
\end{pmatrix} $ denotes the representation of the vector $v$ of $V$. This equality can be written $e^{i\theta}\cdot v=e^{2i\theta}v$. If we note $v=v_1+iv_2=\begin{pmatrix}
v_1 \\
v_2
\end{pmatrix} $, we get
\[ e^{2i\theta}\bullet v=\begin{pmatrix}
\cos 2\theta & -\sin 2\theta \\
\sin 2\theta & \cos 2\theta
\end{pmatrix}\begin{pmatrix}
v_1 \\
v_2
\end{pmatrix}. \]
Therefore, we can write
\[\chi(e^{i\theta})=\begin{pmatrix}
\cos 2\theta & -\sin 2\theta \\
\sin 2\theta & \cos 2\theta
\end{pmatrix}.\]

So $\chi$ projects $U(1)$ into $\SO(2)$ with a kernel $\eZ_2$, for this reason, we say that $U(1)$ is a \defe{double covering}{double covering!of $\SO(2)$} of $\SO(2)$. We note it
\begin{equation}
            \eZ_2\rightarrow U(1)\stackrel{\chi}{\rightarrow}\SO(2).
\end{equation}

\section{Clifford modules}  \label{susec_Cliffmodule}\index{Clifford!module}
%---------------------------

References: \cite{ResEtaDiracType,mellor}.

Let $M$ be a manifold. We denote by $\CCliff(M)$ the bundle whose fibre at $x\in M$ is the complex Clifford algebra of the metric $g_x$: $\CCliff(M)_x=\CCliff(g_x)$. We define the important map
\begin{equation}
\begin{aligned}
 \gamma\colon \Gamma(M,\CCliff(M))&\to \oB(\hH) \\
\gamma(dx^{\mu})&\mapsto \gamma^{\mu}(x)
\end{aligned}
\end{equation}
which can be extended to the whole Clifford algebra.

Let $V$ be a vector space endowed with a bilinear symmetric form. We consider $\Cliff(V)$, the corresponding Clifford algebra. A \defe{Clifford module}{Clifford!module} is a real vector space $E$ with a $\eZ_2$-graduation and a morphism
\[
  \rho_E\colon \Cliff(V)\to \End(E)
\]
of $\eZ_2$-graded vector spaces. It is defined by a linear map $\rho_E\colon V\to \End(V)$ such that
\begin{equation}
\rho_E(v)\rho_E(w)+\rho_E(w)\rho_E(v)=B(v,w)\id
\end{equation}
for every $v$, $w\in E$. The element $\rho_E(x)v$ will often be denoted by $x\cdot v$ and the operation $\rho_E$ is the \defe{Clifford multiplication}{Clifford!multiplication}. The \defe{dual module}{dual module} $E^*$ is defined by $\rho_{E^*}(x)=\rho_E(x^t)^*$, i.e.
\begin{equation}
\langle \rho_{E^*}(x)\psi,v \rangle =(-1)^{| \psi | |x |}\langle \psi, \rho_E\big( \tau(x) \big)v\rangle
\end{equation}
for every $\psi\in E^*$ and $v\in E$. Here

Let $\cA$ be a $\eZ_2$-graded subalgebra of $\Cliff(V)$ and $E_1$, a $\cA$-module. Then the space\nomenclature{$\Ind_{\cA}^{\Cliff(V)}(E_1)$}{Induced Clifford module}
\[
  E=\Ind_{\cA}^{\Cliff(V)}(E_1)=\Cliff(V)\otimes_{\cA}E_1
\]
has a structure of Clifford module, the \defe{induced module}{induced!Clifford module}. The tensor product $\otimes_{\cA}$ is the usual one modulo the subspace spanned by elements of the form
\[
  x\otimes a\cdot y-xa\otimes y
\]
for every $x$, $a\in\Cliff(V)$ and $y\in E_1$. In a similar way, if $E$ is a complex vector space we have a notion of $\CCliff(V)$-module.

Let $x\in\Cliff(V)$ be such that $x^2=1$. In that case the Clifford multiplication $\rho_E(x)$ decomposes $E$ in eigenspaces
\[
  E^{\pm}=\frac{ 1 }{2}\big( 1\pm\rho_E(x) \big)E.
\]

If $V$ is a $n$-dimensional vector space with an oriented orthonormal basis $\{ e_1,\ldots, e_n \}$, the algebra $\Cliff(V)$ has a \defe{volume element}{volume!element} $\omega=e_1e_2\ldots e_n$ which does not depend on the choice of the basis. The volume element squares to
\begin{equation}
\omega^2=(-1)^{n(n+1)/2}.
\end{equation}
In the complex case, we consider the complex vector space $V^{\eC}$ and the complex Clifford algebra $\CCliff(V)=\Cliff(V)\otimes_{\eR}\eC$, and the volume element is defined as
\begin{equation}
\omega_{\eC}=i^{[ (n+1)/2 ]}\omega.
\end{equation}
where $[x]$ is denotes the integer part of $x$. Performing a separate computation for $n$ even or odd, it is easy to see that in both case,
\begin{equation}
\omega_{\eC}^2=1.
\end{equation}
So in the complex case we always have an element in $\Cliff(V)$ which squares to $1$, and a $\CCliff(V)$-module $W$ always accepts a decomposition as $W^{\pm}=\frac{ 1 }{2}(1+\omega_{\eC})W$.

One says that a representation\index{representation!of Clifford algebra} $\rho$ of $\Cliff(V)$ on $W$ is \defe{reducible}{reducible!representation of Clifford} if there exists a splitting $W=W_1\oplus W_2$ such that $\rho(\Cliff(V))W_i\subset W_i$. If the representation is not reducible, it is said to be irreducible. Two representations $\rho_j\colon \Cliff(V)\to \End(W_j)$ are \defe{equivalent}{equivalence!representation of Clifford} if there exists a linear isomorphism $F\colon W_1\to W_2$ such that $F\circ\rho_1(x)\circ F^{-1}=\rho_2(x)$ for every $x\in\Cliff(V)$.

\begin{proposition}
The real Clifford algebra has
\[
 \begin{cases}
2&\text{if }n+1=0\mod 4\\
1&\text{otherwise}
\end{cases}
\]
inequivalent irreducible representations. The complex Clifford algebra $\CCliff(V)$ has
\[
 \begin{cases}
     2&\text{if }n \text{ is odd}\\
     1&\text{if }n \text{is even}
\end{cases}
\]
inequivalent irreducible representations.
\end{proposition}
\begin{proof}
No proof.
\end{proof}


If $M$ is a manifold, we denote by $\Cliff(M)=\Cliff(TM)$ the bundle whose fiber at $x$ is the Clifford algebras of $T_xM$. We consider an orthonormal basis $\{ e_i \}$ and if $\Sigma$ is a multi-index $\{ 1\leq\sigma_1,\ldots,\leq\sigma_t\leq m \}$, we pose $e_{\Sigma}=e_{\sigma_1}\ldots e_{\sigma_t}\in\Cliff(M)$. By convention, $e_{\emptyset}=1$. Since the elements $e_i$ are ordered, they provide an orientation:
\begin{equation}
d\vol=e_1\wedge\ldots\wedge e_m\in\Wedge^m(M).
\end{equation}
Since the map $e_{\sigma_1}\wedge\ldots\wedge e_{\sigma_t}\mapsto e_{\sigma_1\ldots e_{\sigma_t}}$ is an isomorphism between $\Cliff(M)$ and $\Wedge(M)$, we say that $d\vol\in\Cliff(M)$. Now we define
\[
  \kappa=i^{-[(m+1)/2]}d\vol,
\]
which is nothing else that the volume form normalised in such a way that $\kappa^2=1$. If $m$ is even, it anti-commutes with $TM$, and if $m$ is odd, it commutes with $TM$.

Let $V$ be a $m$-dimensional real vector space, and $\CCliff(V)$, the corresponding complex Clifford algebra.
\begin{lemma}
Every $\CCliff(V)$-module accepts an unique decomposition as sum of irreducible representations as follows
\begin{enumerate}
\item if $m=2n$, there exists one and only one irreducible $\CCliff(V)$-module $\Delta$ and $\dim(\Delta)=2n$,
\item if $m=2n+1$, we have two inequivalent irreducible modules $\Delta_{\pm}$ with $\gamma(\kappa)=\pm 1$ on $\Delta_{\pm}$ and $\dim(\Delta_{\pm})=2^n$.
\end{enumerate}
\end{lemma}
\begin{proof}
No proof.
\end{proof}

Let $V$ be a vector bundle over $M$. A structure of $\Cliff(M)$-module over $V$ is a morphism of unital algebra $\gamma\colon \Cliff(M)\to \End(V)$. When one has a basis $\{ e_i \}$ of $V$, we pose $\gamma_i=\gamma(e_i)$. The following lemma is the lemma 1.2 of \cite{ResEtaDiracType}.
\begin{lemma}			\label{LemGammaBaseConstant}
Let $V$ be a $\Cliff(V)$-module and $\{ e_i \}$, an orthonormal basis for $TM$ on a contractible open set $V$. Then there exists a local frame for $V$ such that the matrices $\gamma(e_i)$ are constant.
\end{lemma}
We also define $\gamma^i=\gamma(dx^i)=g^{ij}\gamma_j$. One easily proves that
\begin{equation}
\gamma^i\gamma^j+\gamma^j\gamma^i=-2g^{ij}
\end{equation}
where $(g^{ij})$ is the inverse matrix of $(g_{ij})$. If the endomorphisms $\gamma_i$ are constant in the basis $\{ e_i \}$, then the endomorphisms $\gamma^i$ are constant in the basis $\{ f_i=g_{ki}e_k \}$.



\section{Spin structure}	\label{sec:spin_str}
%++++++++++++++++++++++++

We consider a (pseudo-)Riemannian manifold $(M,g)$ with metric signature $(p,q)$, and $\SO(M)$, its frame bundle; it admits a $\SO(p,q)$-principal fiber bundle structure which is well defined by the metric $g$ (see~\ref{subsubsecframebundle}).

\begin{definition}
We say that $(M,g)$ is a  \defe{spin manifold}{spin!manifold} if there exists a $\Sppq$-principal bundle $P$ over $M$ and a principal bundle homomorphism $\dpt{\varphi}{P}{\SO(M)}$ which induced covering for the structure groups is $\chi$, i.e.
$\varphi(\xi\cdot s)=\varphi(\xi)\cdot\chi(s)$. A choice of $P$ and $\varphi$ is a \defe{spin structure}{spin!structure} on $M$.
\label{defvarspin}
\end{definition}
\[
\xymatrix{ \Sppq \ar@{~>}[r]	& P \ar[rr]^-{\displaystyle\varphi} \ar[rd]_{\displaystyle\pi} && \SO(M) \ar[ld]^{\displaystyle p}&\SO(p,q) \ar@{~>}[l]  \\& & M }
\]
The wavy arrows mean ``structural group of''.

\begin{remark}
When we will use the concept of spin structure in the physical oriented chapters, we will naturally use $\SLdc$ as group instead of $\Sppq$. The isomorphism $\SLdc\simeq\Sput$ is proved in \cite{Michelson}. A physical motivation of such a structure is given at page \pageref{pg_spinenphyz}.
\end{remark}

\subsection{Example: spin structure on the sphere \texorpdfstring{$S^2$}{S2}}
%----------------------------------------------------------------

It is no difficult to see that $\SO(S^2)\simeq \SO(3)$. Indeed, each element of $\SO(S^2)$ is described by three orthonormal vectors: one which point to an element $x$ of $S^2$ and two which gives a basis of $T_xS^2$. The action $\SO(3)\times S^2\to S^2$ is transitive, and the stabilizer of any element is $\SO(2)$.

We define $\dpt{\alpha}{\SO(3)/\SO(2)}{S^2}$ by $\alpha(g\SO(2))=g$. Proposition~\ref{propHelgason4.3} shows that $\alpha$ is a diffeomorphism. Then
\[
                S^2=\frac{\SO(3)}{\SO(2)}.
\]

On the other hand,  we know that
\begin{eqnarray}\label{explsu2} T_eSU(2)=su(2)=\left\{\begin{pmatrix}
ix & \xi \\
-\oxi & -ix
\end{pmatrix}\,:\,\xi\in\eC,x\in\eR\right\}.
\end{eqnarray}
 It is a classical result that $\mathfrak{su}(2)\simeq\eR^3$ not only as set but also as metric space with the identification
\[
\langle X,Y\rangle=-\frac{1}{2}\tr(XY),
\]
 for all $X$, $Y\in su(2)$. As we are in matrix groups, we know (see \cite{Lie} to get more details) that $Ad_xY=xYx^{-1}$. In our case, this gives the formula
\[
                \langle Ad(g)X,Ad(g)Y\rangle=\langle X,Y\rangle.
\]
We can now state the result for $S^2$.

 \begin{proposition}
The manifold $S^2$ with the usual metric induced from $\eR^3$ admits the following spin structure:
\begin{eqnarray}\label{spins2}
\xymatrix{ \Spin(2)\ar@{~>}[r] &SU(2) \ar[rr]^{\displaystyle \varphi=Ad} \ar[rd]_{\displaystyle U(1)}^{\displaystyle\pi} && \SO(3) \ar[ld]^{\displaystyle \SO(2)}_{\displaystyle p} \\& & S^2 },
\end{eqnarray} where the arrow
$\xymatrix{X \ar[r]^{f}_G & Y }$ means that $G$ is the kernel of the map $\dpt{f}{X}{Y}$.
\end{proposition}
\begin{proof}
 First, let us precise the concept of frame bundle for $S^2$, and how it is well described by $\SO(3)$. Let $\{e_1,e_2,e_3\}$ be the canonical basis of $\eR^3$. To $A\in \SO(3)$, we make correspond the basis $\{Ae_2,Ae_3\}$ at the point $Ae_1$ of $S^2$. The projection $\dpt{p}{\SO(3)}{S^2}$ is then defined by $p(A)=Ae_1$. It is clear that we will  define the map $\dpt{\pi}{SU(2)}{S^2}$ in the same way: $\pi(U)=p(Ad(U))$.

For the rest of the demonstration, we will use the ``$su(2)$ description''\ of $\eR^3$ given by \eqref{explsu2} with $\xi=y+iz$.

Now, let us show that $\dpt{\pi}{SU(2)}{S^2}$ is a $\Spin(2)$-principal bundle. Since we had already shown that $\Spin(2)\simeq U(1)$, we define the right action of $\Spin(2)$ on $SU(2)$ by right multiplication: $U\cdot s=Us$ with $s=\begin{pmatrix}
e^{i\theta} & 0 \\
0 & e^{-i\theta}
\end{pmatrix}$. It is clear that $\pi(Us)=\pi(U)$:
\begin{equation}
 Ad(Us)e_1=(Us)\begin{pmatrix}
 1 \\
 0 \\
 0
 \end{pmatrix}s^{-1} U^{-1}=Us
 \begin{pmatrix}
 i&0\\
 0&-i
 \end{pmatrix}s^{-1} U^{-1},
\end{equation}
because $\begin{pmatrix}
i&0\\
0&-i
\end{pmatrix}$ is the vector $e_1$ in the ``$su(2)$ description''\ of $\eR^3$.

In order for $\dpt{\pi}{SU(2)}{S^2}$ to be a $\Spin(2)$-principal bundle, we still need to show that for all $x\in S^2$,
\[
   \pi^{-1}(x)=\left\{\xi\cdot g\tq g\in\Spin(2)\,\forall\xi\in\pi^{-1}(x)\right\}.
\]
Take $A$, $B\in\pi^{-1}(x)$, i.e. $Ae_1=Be_1=x$. We need to find a $s\in\Spin(2)$ such that
\begin{eqnarray}
 \label{1603r3} A=B\cdot s.
\end{eqnarray}
The matrices $A$ and $B$ are such that
\begin{eqnarray}\label{1603r1}
 B^{-1} A\begin{pmatrix}
 i&0\\
 0&-i
         \end{pmatrix}A^{-1} B=\begin{pmatrix}
 i&0\\
 0&-i
         \end{pmatrix}.
\end{eqnarray}
This implies that $B^{-1} A\in\Spin(2)$. As $Ad$ is surjective from $SU(2)$ into $\SO(3)$, a general $C$ in $\SO(3)$ which acts on $e_1$ can be written $Ue_1U^{-1}$ for $U\in SU(2)$ such that $Ad(U)=C$. The condition \eqref{1603r1} becomes
\[
\begin{pmatrix}
\alpha&\beta\\
-\obeta&\oalpha
\end{pmatrix}
\begin{pmatrix}
i&0\\
0&-i
\end{pmatrix}
\begin{pmatrix}
\oalpha&-\beta\\
\obeta&\alpha
\end{pmatrix}=
\begin{pmatrix}
i&0\\
0&-i
\end{pmatrix},
\]
which implies $\alpha=e^{i\theta}$, $\beta=0$. Then $B^{-1} A$ belongs to $\Spin(2)$, and $s=B^{-1} A$ fulfills the condition \eqref{1603r3}.

What about the induced covering for the structural groups? The group \( \Spin(2)\) is the the structural group of $\dpt{\pi}{SU(2)}{S^2}$, while \( \SO(2)\) is the one of $\dpt{p}{\SO(3)}{S^2}$. Indeed, for each $x\in S^2$, $\SO(2)$ acts on $T_xS^2$, leaving $x$ unchanged. We have the following associations:
\[
         U\in SU(2)\stackrel{\varphi}{\longrightarrow}A\in \SO(3),
\]
the matrix $A$ being defined by $A\cdot X=UXU^{-1}$. For $s\in\Spin(2)$ we of course also have
\[
         Us\in SU(2)\stackrel{\varphi}{\longrightarrow}As\in \SO(3),
\]
with $As\cdot X=UsXs^{-1} U^{-1}$. As we act by $\Spin(2)$ on $SU(2)$, in the fibres of $\SO(3)$, the action of $\Spin(2)$ is --via $\varphi$-- the composition with $X\to sXs^{-1}$. But this is exactly $\chi(s)X$ because $\alpha(s)=s$, since $s\in\Spin(2)$.
\end{proof}

\subsection{Spinor bundle}
%--------------------------

Let us take once again the spin structure on the (pseudo-)Riemannian manifold $(M,g)$:
\[
  \xymatrix{ \Sppq \ar@{~>}[r]& P \ar[rr]^-{\displaystyle\varphi}
   \ar[rd]_{\displaystyle\pi} && \SO(M) \ar[ld]^{\displaystyle p}&\SO(p,q) \ar@{~>}[l]
   \\& &   M }
\]
with $\varphi(\xi\cdot g)=\varphi(\xi)\cdot\chi(g)$.

Let us define $S=\Lambda W $, and $\mS=P\times_{\rho}S$. Take $\dpt{\rho}{\Sppq\times\mS}{\mS}$, $\rho(g,s)=\tilde\rho(g)s$, where $\tilde\rho$ is the spinor representation of $\Sppq$ on $S$. We also have
$\dpt{\chi}{\Sppq}{\SO_0(p,q)}$, $\chi(g)v=\alpha(g)\cdot v\cdot g^{-1}$, with $\alpha(g)=g$ for $g\in\Sppq$.

The \defe{spinor bundle}{spinor!bundle} is the associated bundle
\begin{equation}
                   \mS=P\times_{\rho}S\to M
\end{equation}
A \defe{spinor field}{spinor!field} is an element of $\Gamma(\mS)$, the space of section of the spinor bundle.

On $\SO(M)$, we look at a connection $1$-form $\alpha\in\Omega^1(\SO(M),so(\eR^m))$,
and, if $T(M)$ is the tensor bundle over $M$, we define a covariant derivative $\dpt{\nabla^{\alpha}}{\cvec(M)\times T(M)}{T(M)}$ by
 \[
             \widehat{\nabla^{\alpha}_X s}=\overline{X}\hat{s},
\]
 for any $s\in T(M)$. See theorem~\ref{tho_nablaE}, and the fact that $T(M)$ can be see as an associated bundle; it is explicitly done for $\cvec(M)$ at page \pageref{equivvec}.

As seen in~\ref{subsection_levi}, an automatic property of this connection is $\nabla^{\alpha} g=0$ if $g$ is the metric of $M$. The \defe{Levi-Civita connection}{connection!Levi-Civita}\index{connection!Levi-Civita} is the unique\footnote{We will not prove unicity.} such connection which is torsion-free: $T^{\nabla^{\alpha}}=0$.


\begin{proposition}
The $1$-form $\talpha=\varphi^*\alpha\in\Omega^1(P,so(\eR^{m}))$ defines a connection on $P$. See definition~\ref{defconnform} and theorem~\ref{tho_nablaE}.
\end{proposition}

\begin{proof}
Let us denote by $R_g$ the right action of $g\in\Sppq$ on $P$ (\emph{id est} $R_g\xi=\xi\cdot g$), and by $R_u^{\SO(M)}$ the right action of $u\in\Sopq$ on $\SO(M)$.
We  have to check the usual two conditions of a connection.

\subdem{First condition}
The first one is:
\[
            (R_g^*\talpha)_{\xi}(\Sigma)=Ad(g^{-1})(\talpha_{\xi}(\Sigma)),
\]
for all $\xi\in P$, and $\Sigma\in T\bxi P$. In order to check this, we first remark that $\varphi\circ R_g=R_{\chi(g)}^{\SO(M)}\circ\varphi$. Indeed, for all $\xi\in P$, definition~\ref{defvarspin} gives us $\varphi(R_g\xi)=\varphi(\xi\cdot g)=\varphi(\xi)\cdot\chi(g)$.  With this, we can make the following computation:
\begin{equation}\label{1603r4}
\begin{aligned}
R_g^*\talpha&=R_g^*\varphi^*\alpha=(\varphi\circ   R_g)^*\alpha	=(R_{\chi(g)}^{\SO(M)}\circ\varphi)^*\alpha\\
            &=\varphi^*R_{\chi(g)}^{\SO(M)*}\alpha=\varphi^*(Ad(\chi(g)^{-1})\circ\alpha).
\end{aligned}
\end{equation}
The last equality comes from the fact that $\alpha$ is a connection $1$-form. As we are in matrix groups, we have $Ad(g)x=gxg^{-1}$, so
\begin{equation}
   [Ad(\chi(g))x]v=[\chi(g) x \chi(g)^{-1}]v
                  =\chi(g)[xg^{-1} vg]
                  =gxg^{-1}.
\end{equation}
In the first line, the product is the usual matrix product which can be seen as operator composition.

But $(Ad(g)x)v=gxg^{-1} v$. Then $Ad(g)=Ad(\chi(g))$, if we identify $\sppq\simeq\sopq$ by proposition~\ref{prop:spin_so}. Moreover, the action of $Ad$ is linear, so it commutes with $\varphi^*$. With these remarks, we can continue the computation \eqref{1603r4}:
\begin{equation}
 \varphi^*(Ad(\chi(g)^{-1})\circ\alpha)=\varphi^*(Ad(g^{-1})\circ\alpha)
                                  =Ad(g^{-1})\circ\varphi^*\alpha
                                  =Ad(g^{-1})\circ\talpha.
\end{equation}
This proves the first condition.

\subdem{Second condition}
The second one is $\talpha(A^*\bxi)=-A$ with the definition \eqref{defastar}. This is also a computation. First remark
\[
 \talpha\bxi(A\bxi^*)=(\varphi^*\alpha)\bxi(A^*\bxi)=\alpha_{\varphi(\xi)}(\varphi_{*\xi}A^*\bxi).
\]
 We compute $\varphi_{*\xi}A^*$ with lemma~\ref{lemsur5d}:
\begin{equation}
\begin{split}
 \varphi_{*\xi}A^*&=\dsdd{\varphi(\xi\cdot\exp -tA)}{t}{0} =\dsdd{(R_{\chi(\exp -tA)}^{\SO(M)}\circ\varphi)(\xi)}{t}{0}\\
              &=\dsdd{\varphi(\xi)\cdot\chi(\exp -tA)}{t}{0}=\dsdd{\varphi(\xi)\cdot\exp(-td\chi_eA)}{t}{0}=(d\chi_eA)^*_{\varphi(\xi)}.
\end{split}
\end{equation}
But $d\chi_e=\id_{so(p,q)}$, thus $\varphi_{*\xi}A^*=A^*_{\varphi(\xi)}$. The whole makes that:
\[
\talpha\bxi(A^*\bxi)=\alpha_{\varphi(\xi)}(\varphi_{*\xi}A^*\bxi)=\alpha_{\varphi(\xi)}(A^*_{\varphi(\xi)})=-A.
\]
This completes the proof.
\end{proof}

\begin{definition}
This connection $1$-form on $P$ is called the \defe{spinor connection}{spinor!connection}. It gives us a covariant derivative on any associated bundle and in particular on the spinor bundle, $\dpt{\tnab}{\cvec(M)\times\Gamma(\mS)}{\Gamma(\mS)}$.
 \label{spinconn}
\end{definition}
\nomenclature[D]{$\dpt{\tnab}{\cvec(M)\times\Gamma(\mS)}{\Gamma(\mS)}$}{Covariant derivative for the spinor connection}

\begin{proposition}
If $X$, $Y\in\cvec(M)$ are such that $X_x=Y_x$, then for all $s\in\Gamma(\mS)$,
\[
              (\tnab_Xs)(x)=(\tnab_Ys)(x).
\]
 \label{2303p1}
\end{proposition}
\begin{proof}
We just have to show that for all vector field $Z$ such that $Z_x=0$, $(\tnab_Zs)(x)=0$. Such a $Z$ can be written as $Z=fZ'$ for a function $f$ on $M$ which satisfies $f(x)=0$. We have:
\[
\tnab_Zs=\tnab_{fZ'}s=f\tnab_{Z'}s,
\]
which is obviously zero at $x$.
\end{proof}

Let $x\in M$ and $\{e_{\alpha\,x}\}$ be an orthonormal basis of $T_xM$. We can extend it to $\{e_{\alpha}\}$, a local basis field around $x$ such that $e_{\alpha}$ is a section of the frame bundle (in other words, we ask the extension to be smooth). The claim of proposition~\ref{2303p1} is that $\tnab_{e_{\alpha}}(x)$ is an element of $\mS_x$ which doesn't depend on the extension.

\section[Dirac operator]{Dirac operator\protect\quad{\Huge\Smiley}}		\label{applgamma}
%---------------------------------------------------------------------

\subsection{Preliminary definition}
%----------------------------------

Let $M$ be a $m$-dimensional (pseudo)Riemannian manifold with its spin structure
\[
\xymatrix{ \Sppq \ar@{~>}[r]& P \ar[rr]^{\displaystyle\varphi} \ar[rd]_{\displaystyle\pi} && \SO(M) \ar[ld]^{\displaystyle p}&\SO(p,q) \ar@{~>}[l]  \\& & M }
\]
where $\varphi$  satisfies $\varphi(\xi\cdot g)=\varphi(\xi)\cdot\chi(g)$.

Recall that for any vector space, one can see $\End{V}=V^*\otimes V$ with the definition $(v^*\otimes v)w=(v^*w)v$. This allows us to define an action of $\Sppq$ on $\End{S}$ by defining an action of $\Sppq$ on $S$ and $S^*$ separately. We know the action
\begin{equation}
\begin{aligned}
 \Spin(p,q)\times S&\to S \\
(g,v)&\mapsto \tilde\rho(g)v,
\end{aligned}
\end{equation}
and as action on $S^*$, we take the dual one
\begin{equation}
\begin{aligned}
 \Spin(p,q)\times S^*&\to S^* \\
 g\cdot\alpha&= \alpha\circ\tilde\rho(g^{-1})
\end{aligned}
\end{equation}
for all $g\in\Spin(p,q)$ and $\alpha\in S^*$.

Now we can make the following computation with $g\in\Sppq$, $\alpha\in S^*$ and $v\in S$, using the fact that $\tilde\rho$ is linear:
\begin{equation}
\begin{split}
[g\cdot(\alpha\otimes v)]w&=[(\alpha\circ\tilde\rho(g^{-1}))w]\tilde\rho(g)v\\
                          &=\tilde\rho\left([(\alpha\circ\tilde\rho(g^{-1}))w]g\right)v\\
                          &=\big[\tilde\rho(g)\circ(\alpha\otimes v)\circ\tilde\rho(g^{-1})\big]w.
\end{split}
\end{equation}
Then we write the action of $\Sppq$ on $\End{S}$\index{action!of $\Sppq$ on $\End{S}$} by ($A\in\End S$)
\begin{equation}
     g\cdot A=\tilde\rho(g)\circ A\circ\tilde\rho(g^{-1}).                          \label{actspin}
\end{equation}
Notice that this definition is the one required in condition \eqref{equivA}.

The tangent bundle $T_xM$ is given with a metric $g_x$. As usual, we build $S_x=\Lambda W _x$, a completely isotropic subspace of $T_xM$ with respect to the metric $g_x$, and a representation
\[
\tilde \rho_x\colon T_xM\to \End(\Lambda W_x)
\]
The first step in the definition of $\gamma(X)$ is to build $\dpt{\ha_X}{P}{\End(\Lambda W )}$ setting\footnote{See subsection~\ref{equivvec} for the definition of $\hX$.} $\ha_X(p)=\tilde\rho(\hX_{\varphi(p)})$.

\begin{lemma}
The function $\hat a$ is equivariant, i.e. it satisfies
\begin{equation}
     \ha_X(p\cdot g)=g^{-1}\cdot\ha_X(p)                             \label{equivaX}
\end{equation}
for all $g\in\Sppq$.
\end{lemma}

\begin{proof}
It is no more than a simple computation using the equivariance of $\hX$. Indeed:
\begin{equation}
\begin{split}
 \ha_X(p\cdot g)&=\tilde\rho(\hX_{\varphi(p\cdot g)})=\tilde\rho(\hX_{\varphi(p)\chi(g)})=\tilde\rho(\chi(g^{-1})\cdot\hX_{\varphi(p)})\\
		&=\tilde\rho(g^{-1}\cdot\hX_{\varphi(p)}\cdot g)=\tilde\rho(g^{-1})\circ\tilde\rho(\hX_{\varphi(p)})\circ\tilde\rho(g)\\
                &=g^{-1}\cdot\ha_X(p).
\end{split}
\end{equation}
In the fourth line, the dots mean the Clifford product, and the last equality comes from the definition of the action \eqref{actspin} of $\Sppq$ on $\End{S}$.
\end{proof}

From the discussion of section~\ref{sec_fnequiv}, the function $\dpt{\ha_X}{P}{\End{S}}$ defines a section $\dpt{a_X}{M}{\End{\mS}}$. We define $\dpt{\gamma}{\cvec(M)}{\End{\Gamma(\mS)}}$ by
\nomenclature[D]{$\dpt{\gamma}{\cvec(M)}{\End{\Gamma(\mS)}}$}{A key ingredient for Dirac operator}
\begin{equation}		\label{EqDefgammax}
                       \gamma(X)=a_X.
\end{equation}
We immediately have
\[
                     \widehat{\gamma(X)}(p)=\tilde\rho(\hX_{\varphi(p)})
\]
for any $p\in P$. If we define
\begin{equation}\label{3103r1}
  \widehat{\gamma\cdot a_X}(p)=\widehat{\gamma(X)}(p),
\end{equation}
the map $\gamma$ can be seen as an action on the section of $\mS$. Indeed, $\widehat{\gamma\cdot s}_X$ is an equivariant function:
\begin{equation}
\begin{split}
 \hat{\gamma}(p\cdot g)(\ha_X(p\cdot g))
                  &=\rho(g)^{-1}\hat{\gamma}(p)\rho(g)\rho(g^{-1})\ha_X(p)\\
                  &=\rho(g)^{-1}\hat{\gamma}(p)\ha_X(p)\\
                  &=\rho(g^{-1})\widehat{\gamma\cdot a_X}(p),
\end{split}
\end{equation}
 so that
\[
    \widehat{\gamma\cdot a_X}(p)=\rho(g^{-1})\widehat{\gamma\cdot a_X}(p).
\]

The map $\dpt{\widehat{\gamma\cdot a_X}}{P}{\End{\Lambda W }}$ defined by \eqref{3103r1} is equivariant, and thus defines a section
$\gamma\cdot a_X\in\Gamma(\mS)$, as seen in the section~\ref{sec_fnequiv}.

\subsection{Definition of Dirac}
%-------------------------------

If we consider a basis $\{e_{\alpha}\}$ of $TM$, \emph{i.e.} $m$ sections $\dpt{e_{\alpha}}{M}{TM}$ such that for all $x$ in $M$, the set $\{e_{\alpha x}\}$ is a basis of $T_xM$, we note $\gamma^{\alpha}:=\gamma(e_{\alpha})\in\End(\mS)$.

\begin{remark}
This is not always globally possible. The example of the sphere is given in subsection~\ref{subsec_DimofModule}.
\label{rem_secnoglobal}
\end{remark}

For any $s\in\Gamma(\mS)$, we consider the local\footnote{Extensions of $e_{\alpha}$ do not always globally exist, see remark~\ref{rem_secnoglobal}.} section $\psi$ of $\mS$ given by
\[
    \psi(x)=
   \sum_{\alpha\beta}g_x(e_{\alpha},e_{\beta})\gamma_x\hbeta(\tnab_{e_{\alpha}}s)(x).
\]

For each $x\in M$, take a $A_x$ in\footnote{By $\SO(g_x)$, we mean the set of all the matrix $A$ such that $A^tg_xA=g$; $A_x$ is an isometry of $(T_xM,g_x)$. In other words, we consider $A$ as a section of what we could call the ``isometry bundle''.} $\SO(g_x)$, and consider the new basis $e'_{\alpha}=A_{\alpha}^{\phantom{\alpha}\beta}e_{\beta}$. As $A$ is an isometry, $g_x(e'_{\alpha},e'_{\beta})=g_x(e_{\alpha},e_{\beta})$; and since $\tilde\rho$ is linear, $\gamma_x'^{\alpha}=\tilde\rho_x(e'_{\alpha x})=A_{\alpha}^{\phantom{\alpha}\beta}\tilde\rho(e_{\beta x})=A_{\alpha}^{\phantom{\alpha}\beta}\gamma_x\hbeta$. In the new basis, the section reads:
\begin{equation}
\begin{split}
   \psi(x)&=\sum_{\alpha\beta\eta\sigma}g_x(e_{\alpha},e_{\beta})
                A_{\beta}^{\phantom{\beta}\sigma}\gamma_x^{\sigma}
                (\tnab_{A_{\alpha}^{\phantom{\alpha}\eta}e_{\eta}}s)(x)\\
          &=\sum_{\alpha\beta\eta\sigma}(A^t)\heta_{\phantom{\eta}\alpha}
                  g_{\alpha\beta}(x)A_{\beta}^{\phantom{\beta}\sigma}
                  \gamma_x^{\sigma}(\tnab_{e_{\eta}}s)(x)\\
          &=\sum_{\eta\sigma}g_x(e_{\eta},e_{\sigma})\gamma_x^{\sigma}(\tnab_{e_{\eta}}s)(x),
\end{split}
\end{equation}
where we used the fact that $A^tgA=g$ and that all the $A_{\alpha}^{\phantom{\alpha}\beta}$ are $C^{\infty}$ functions on $M$, so that
$\tnab_{A_{\alpha}^{\phantom{\alpha}\beta}X}=A_{\alpha}^{\phantom{\alpha}\beta}\tnab_X$.  This shows that $\psi(x)$ doesn't depend on the choice of the basis, so it defines a section from the data of $s$ alone.


The \defe{Dirac operator}{dirac!operator!on $(M,g)$, a spin manifold}\nomenclature[D]{$\Dir$}{Dirac operator} $\Dir\colon \Gamma(\mS)\to \Gamma(\mS)$ acting on a spinor field is defined by
\begin{equation}\label{dirac}
(\Dir s)(x)=g_x(e_{\alpha},e_{\beta})\gamma_x\hbeta(\tnab_{e_{\alpha}}s)(x).
\end{equation}

\begin{proposition}
If the field of basis $e_{\alpha}\in\cvec(M)$ is everywhere an orthonormal basis, the Dirac operator reads
\begin{equation}
(\Dir s)(x)=g_{\alpha\beta}\gamma^{\alpha}(\tilde\nabla_{e_{\beta}}s)(x)
\end{equation}
where $\gamma^{\alpha}$ is a constant numeric matrix acting on $\Lambda W$.
\end{proposition}

\begin{proof}
The building of the Dirac operator begins by considering the vector space $T_xM$ endowed with the metric $g_x$; then the spinor representation $\tilde\rho_x\colon T_xM\to \End(\Lambda W_x)$ where $\Lambda W_x$ is build from isotropic vectors of $T_xM$ is defined. If the vector fields $e_{\alpha}\in\cvec(M)$ are everywhere orthonormal for the metric $g$, then we have the matricial equality
\begin{equation}
	\tilde\rho_x\big( (e_{\alpha})_x \big)_{ij}=\tilde\rho(v_{\alpha})_{ij}
\end{equation}
where the left hand side describe the matrix component of a linear operator acting on $\Lambda W_x$ while in the right hand side we have the matrix component of a linear operator acting on $\Lambda W$ and $v_{\alpha}$ is a basis on $\eR^n$ with respect to which the metric is the same as the metric $g_x$ in the basis $(e_{\alpha})_x$. Let $\hat{\psi}\colon P\to \Lambda W$ be an equivariant function; from definition \eqref{EqDefgammax} of $\gamma$ we have
\[
  \big( \gamma(e_{\alpha}\hat{\psi}) \big)(\xi)=(a_{\alpha}\hat{\psi})(\xi)
\]
where $a_{\alpha}(\xi)=\tilde\rho\Big( \tilde{e}_{\alpha}\big( \phi(\xi) \big) \Big)$. In this expression, $\tilde{e}_{\alpha}$ is the equivariant function associated with the vector field $e_{\alpha}\in\cvec(M)$. It is defined in subsection~\ref{equivvec} as
\begin{equation}
\begin{aligned}
 \tilde{e}_{\alpha}\colon \SO(M)&\to \eR^m \\
b&\mapsto b^{-1}\big( (e_{\alpha})_{\pi(b)} \big).
\end{aligned}
\end{equation}
So we have $\hat a_{\alpha}\colon P\to \End(\Lambda W)$ defined by
\[
  \hat a_{\alpha}(\xi)=\tilde\rho\big( \varphi(\xi)^{-1}e_{\alpha}(x) \big)
\]
with $x=\pi(\xi)$. Now if $\xi$ is any element of $\pi^{-1}(x)$, we have
\begin{align*}
\big( \gamma(e_{\alpha})\psi \big)(x)&=(a_{\alpha}\psi)(X)=\big[ \xi,\hat a_{\alpha}(\xi)\hat\psi(\xi) \big]
		=\big[ \xi,\tilde\rho\big( \varphi(\xi)^{-1}e_{\alpha}(x) \big)\hat{\psi}(\xi) \big].
\end{align*}
There exists a $g\in\Spin(p,q)$ such that $\varphi(\xi\cdot g)=\mtu$; taking this element and using equivariance of the latter expression,
\begin{align}
  \big( \gamma(e_{\alpha})\psi \big)(x)=\big[ \xi\cdot g,\tilde\rho\big( e_{\alpha}(x) \big)\hat{\psi}(\xi\cdot g) \big]
		=\big[ \xi\cdot g,\gamma^{\alpha}\hat{\psi}(\xi) \big]
		=[\xi,\gamma^{\alpha}\hat{\psi}(\xi)].
\end{align}
What we proved is that $\big( \gamma e_{\alpha}\psi \big)(x)=\gamma^{\alpha}\psi(x)$ is the sense that
\begin{equation}
	\widehat{\gamma(e_{\alpha})\psi}=\gamma^{\alpha}\hat{\psi}.
\end{equation}
Hence the Dirac operator reads
\[
  (\Dir s)(x)=g_{\alpha\beta}\gamma^{\alpha}\big( \tilde\nabla_{e_{\beta}}s \big)(x)
\]
in the sense that
\begin{equation}
\widehat{\Dir s}=g_{\alpha\beta}\gamma^{\alpha}\widehat{  \tilde\nabla_{e_{\beta}}s }.
\end{equation}

\end{proof}


An often more convenient way to write the Dirac operator is to consider an orthonormal basis (so that the metric $g$ and the matrices $\gamma$ are constant) and to consider the equivariant functions:
\[
  \widehat{\Dir\psi}=g_{\alpha\beta}\gamma^{\alpha}\widehat{\nabla_{e_{\alpha}}\psi}.
\]
This formulation is typically used when one search for Dirac operator on Lie groups. In this case, we choose left invariant vector fields generated by an orthonormal basis of the Lie algebra. The resulting field of basis is everywhere Killing-orthonormal.

Acting on a function $\dpt{f}{M}{\eR}$, it is defined by $\dpt{\Dir}{C^{\infty}(M)}{C^{\infty}(M)}$\index{dirac!operator!on functions},
\begin{equation} \label{eq_defDirac_f}
(\Dir f)(x)=g_x(e_{\alpha},e_{\beta})\gamma\hbeta_x(e_{\alpha x}\cdot f).
\end{equation}
With these definitions, one has
\[(\Dir(fs))(x)=(f\Dir s)(x)+(\Dir f)(x).\] Indeed,
\begin{equation}
\begin{split}
   (\Dir(fs))(x)&=g_{\alpha\beta}\gamma_x\hbeta(\tnab_{e_{\alpha}}fs)(s)\\
                &=g_{\alpha\beta}\Big((e_{\alpha}\cdot f)s(x)+f(x)(\tnab_{e_{\alpha}}s)(x)\Big)\\
                &=f(x)(\Dir s)(x)+g_{\alpha\beta}\gamma_x\hbeta(e_{\alpha x}\cdot f)\\
                &=(f\Dir s)(x)+(\Dir f)(x).
\end{split}
\end{equation}
With that definition, the Dirac operator becomes a derivation of the spinor bundle.


\section[Dirac operator on  \texorpdfstring{$\eR^2$}{R2}]{Example: Dirac operator on \texorpdfstring{$\eR^2$}{R2} with the euclidian metric}\label{Pg_exempleRdeux}\index{dirac!operator!on $\eR^2$}
%---------------------------------------------------

%\subsubsection{Example: tangent bundle}
%--------------------------

Since the frame bundle $B(M)$ is a principal bundle (see subsection~\ref{subsec_frbundle}), one can consider some associated bundles on it. We are now going to see that the one given by the definition representation $\dpt{\rho}{GL(n,\eR)}{GL(n,\eR)}$ on $\eR^n$ is the tangent bundle. So we study $B(M)\times_{\rho} \eR^n$. By choosing a basis on each point of $M$, we identify each $T_xM$ to $\eR^n$. An element of $B(M)\times \eR^n$ is a pair $(b,v)$ with $b=(\overline{b}_1,\ldots,\overline{b}_n)$ and $v=(v^1,\ldots,v^n)$. We can identify $v$ to the element of $T_xM$ given by $v=v^i\overline{b}_i$.

In order to build the associated bundle, we make the identifications
\[
  (b,v)\cdot g\sim(b\cdot g,g^{-1} v).
\]
Here, by $gv$ we mean the vector whose components are given by $(gv)^i=v^j\bghd{g}{j}{i}$. The tangent vector given by $(b\cdot g,g^{-1} v)$ is $(g^{-1} v)^i(b\cdot g)_i=v^j\bghd{(g^{-1})}{j}{i}\bghd{g}{i}{k}\overline{b}_k=v^k\overline{b}_k$ So the identification map $\dpt{\psi}{B(M)\times_{\rho}\eR^n}{TM}$ given by
\[
  \psi([b,v])=v^i\overline{b}_i
\]
is well defined.

\index{spin!structure!on $\eR^2$}
The following step is to consider the following spin structure:
 \[\xymatrix{
    \Spin(2)  \ar@{~>}[r]&  \eR^2\times \SO(2) \ar[r]^-{\displaystyle\varphi} & \SO(\eR^2)  & \SO(2) \ar@{~>}[l].
  }\]

We have to define the two actions and $\varphi$. One of the main result of section~\ref{cliffR2} is that $\dpt{\chi}{\Spin(2)=U(1)}{\SO(2)}$ is surjective. So, we can define the action of $\Spin(2)$ on $P$ by
\[(x,b)\cdot s=(x,\chi(s)^{-1} b).\]

On the other hand, an element $A$ in $\SO(\eR^2)$ can be written as $A=\baz{a}{x}$ where $e_i$ is the canonical basis of $T_x\eR^2$, and $a$ is a matrix of $\SO(2)$. See subsection~\ref{subsec_frbundle}. For $g\in \SO(2)$, we define
\begin{eqnarray}
 \label{r1504d2}A\cdot g=\{g^{-1} ae_i\}_x.
\end{eqnarray}
and  $\dpt{\varphi}{\eR^2\times \SO(2)}{\SO(\eR^2)}$ by
\[
\varphi(x,b)=\{be_i\}_x.
\]
The following shows that these definitions give a spin structure:
\begin{equation}
   \varphi((x,b)\cdot s)=\varphi(x,\chi(s)^{-1} b)
                    =\{\chi(s)^{-1} be_i\}_x
                    =\{be_i\}_x\cdot\chi(s)
                    =\varphi(x,b)\cdot\chi(s).
\end{equation}


\subsection{Connection on \texorpdfstring{$\SO(\eR^2)$}{SO(R2)}}\index{connection!on $\SO(\eR^2)$}
%///////////////////////////////////////

We are searching for a torsion-free connection on the simplest metric space: the euclidian $\eR^2$. Thus we will try the simplest choice of horizontal space: we want an horizontal vector to be tangent to a curve of the form $X(t)=\baz{b}{x(t)}$. For this reason, we want to define the connection $1$-form by $\omega(X)=b'(0)$. For technical reasons which will soon be apparent, we will not exactly proceed in this manner. For $X(t)=\baz{b}{x(t)}$, we define
\begin{equation}
                       \omega(X)=-(b(t)b(0)^{-1})'(0).
\end{equation}
We of course have $\omega(X)=0$ if and only if $b'(0)=0$: this choice of $\omega$ follows our first idea. In order for $\omega$ to be a connection form, we have to verify the two conditions of definition~\ref{defconnform}.

\begin{proposition}
The $1$-form defined by
\[
              \omega(X)=-(b(t)b(0)^{-1})'(0)
\]
for $X=\displaystyle\dsdd{\baz{b(t)}{x(t)}}{t}{0}$ is a connection $1$-form.
\end{proposition}

\begin{proof}
Let $A\in \SO(2)$. If $u=\baz{b}{x}$, equation \eqref{r1504d2} gives:
\[
   A^*_u=\dsdd{\baz{e^{-tA}b}{x}}{t}{0},
\]
 so that $\omega(A^*_u)=-(e^{-tA}bb^{-1})'(0)=A$. This checks the first condition. For the second, one remarks that the path in $\SO(\eR^2)$ which defines the vector $R_{g*}X$ is $(R_{g*}X)(t)=\baz{g^{-1} b(t)}{x}$. It follows that
\begin{equation}
\begin{split}
\omega(R_{g*}X)&=-(g^{-1} b(t)b(0)^{-1} g)'(0)\\
               &=-\left(\AD_{g^{-1}}(b(t)b(0)^{-1})\right)'(0)\\
               &=-Ad_{g^{-1}}(b(t)b(0)^{-1})'(0)\\
               &=Ad_{g^{-1}}\omega(X).
\end{split}
\end{equation}

\end{proof}

\begin{proposition}
The covariant derivative induced on $M$ by this connection is
\begin{equation}\label{derrcovexplicite}
                 \nabla_XY=X(Y).
\end{equation}
\end{proposition}

\begin{proof}
In this demonstration, we will use the equivariant functions defined in~\ref{equivvec}. In order to compute $(\nabla_XY)_x$, we have to use the definition of theorem~\ref{tho_nablaE}. We first have to compute the horizontal lift of $X$. It is no difficult to see that $\oX_{\baz{b}{x}}$ is given by the path
\[\oX(t)=\baz{b}{X(t)}\]
if the vector field $X$ is given by the path $X(t)$ in $M$. Indeed, it is trivial that $\omega(\oX)=0$, and
\[d\pi_*\oX=\dsdd{\pi\baz{b}{X(t)}}{t}{0}=\dsdd{X(t)}{t}{0}=X.\]

Now, we compute $(\oX\hs)(b)$ for $b=\{Se_i\}_x$. We begin using the basic definitions and notations:
\[
(\oX\hs)(b)=\oX_b\hs=\dsdd{\hs(\oX_b(t))}{t}{0}=\dsdd{\hs(\baz{S}{X(t)})}{t}{0}.
\]
We can rewrite it with $\hY$ instead of $\hs$. By construction (see \eqref{r1404e1}), if $b=\baz{S}{x}$, $\hY(b)=S^{-1}(Y_x)$. Thus
\[
(\oX\hY)(b)=\dsdd{S^{-1}(Y_{X(t)})}{t}{0},
\]
where, if $\{\oui\}$ is a basis of $\eR^m$, then $S$ is
\begin{equation}
\begin{aligned}
 S\colon\eR^m&\to T_{X(t)}M \\
 v^i\oui &\mapsto S^i_jv^j(\partial_j)_{X(t)}
\end{aligned}
\end{equation}
So if we write $Y_x=Y^i(x)\partial_i$, we have
\[
S^{-1}(Y_{X(t)})=(S^{-1})^i_jY^j(X(t))\oui
\]
and
\[
\dsdd{S^{-1}(Y_{X(t)})}{t}{0}=(S^{-1})^i_j\dsdd{Y^j(X(t))}{t}{0}\oui=(S^{-1})^i_jX(Y^j)\oui.
\]
Since $b$ is an isomorphism, we can apply $b$ on both side of $\hX(b)=b^{-1}(X_x)$, and take $\nabla_XY$ instead of $X$:
\begin{equation}
 (\nabla_XY)(x)=b\big((S^{-1})^i_jX(Y^j)\oui\big)
               =S^k_i(S^{-1})^i_jX(Y^j)(\partial_k)_x
               =X(Y^j)(\partial_j)_x
               =X(Y)_x.
\end{equation}
\end{proof}

From this and definition~\ref{deftorsion}, we immediately conclude that our connection is torsion-free. In a certain manner, one can say that our covariant derivative is the usual one.

\subsection{Construction of \texorpdfstring{$\gamma$}{g}}
%//////////////////////////////////////

Now, we construct the map $\gamma$ of subsection~\ref{applgamma}. The first step is to define $\dpt{\ha_X}{P}{\End{(\Lambda W )}}$ by
\[
\ha_X(p)=\tilde\rho(\hX_{\varphi(p)}).
\]
Here, $\Lambda W $ is the completely isotropic subspace of $(\eR^2)^{\eC}$ with euclidian metric; thus we can use the result of section~\ref{cliffR2}. In particular, we know the representation $\tilde\rho$.

To see it more explicitly, we need the expression of $\hX$. It is given in subsection~\ref{equivvec}: if $b$ is the basis $\baz{b}{x}$, $\hY(b)=b^{-1}(Y_x)$. As $\varphi(b,x)=\baz{b}{x}$, we have
\[
\ha_X(b,x)=\tilde\rho(b^{-1}(X_x)).
\]

The subsection~\ref{equivendo} explains how to explicitly get $\gamma(X)$ with the definition $\gamma(X)=a_X$. If $\psi$ is a section of $\mS$ and $\psi(x)=[\xi,v]$, the general definition gives us $(a_X\psi)(x)=[\xi,\ha_X(\xi)v]$ and in our particular case, if $\xi=(b,x)$, we get:
\begin{eqnarray}
 \label{gammaX}(\gamma(X)\psi)(x)=[\xi,\tilde\rho(b^{-1}(X_x))v].
\end{eqnarray}

\subsection{Covariant derivative on \texorpdfstring{$\protect\Gamma(\mS)$}{S}}
%///////////////////////////////////////////

Remember the spin structure of $\SO(\eR^2)$: $\varphi(x,S)=\{Se_i\}_x$. We now construct the connection on $P=\eR^2\times \SO(2)$. It is defined by the $1$-form $\tomega=\varphi^*\omega$. If $v$ is a vector of $P$, it is described by a path $v(t)=(x(t),b(t))$, then the path of $d\varphi(v)$ is $\{b(t)e_i\}_{x(t)}$ and $\tomega(v)=\omega(d\varphi(v))=-(b(t)b(0)^{-1})'(0)$.

The next step defining the Dirac operator is to find out an explicit form for the map $\dpt{\tnab}{\cvec(M)\times\Gamma(\mS)}{\Gamma(\mS)}$. A section $s\in\Gamma(\mS)$ is a map $\dpt{s}{M}{\mS=(\eR^2\times \SO(2))\times_{\rho}\Lambda W }$; it is defined by an equivariant function $\dpt{\hs}{P}{\Lambda W }$. In order to find the value of $(\tnab_Xs)(x)$ for $X\in\cvec(M)$, we use the definition
\[
 \widehat{\tnab_Xs}(\xi)=\oX\bxi(\hs)
\]
where $\oX$ is the horizontal lift in the sense of $\tomega$. For the same reason as in the proof of proposition~\ref{derrcovexplicite}, $\oX_{(b,x)}$ is given by the path $\oX(t)=(b,X(t))$ where $X(t)$ is the path which defines $X$. So we have
\[
 \widehat{\tnab_Xs}(\xi)=\oX_{(b,x)}(\hs)=\dsdd{\hs(b,X(t))}{t}{0}.
\]
Remark that $\Lambda W $ is a vector space; so for every $\alpha\in\Lambda W $, the identification $T_{\alpha}\Lambda W =\Lambda W $ is correct.

Our first form of $\tnab$ is
\[
(\tnab_Xs)(x)=\Big[\xi,\dsdd{\hs(b,X(t))}{t}{0}\Big],
\]
but we can modify this in order to get simpler expressions. Remark that we have an equivalence class, so that we can always choose the element of the class such that $\xi=(\mtu,x)$. We define $\dpt{\os}{\eR^2}{\Lambda W }$, $\os(v)=\hs(\mtu,v)$. Our second and final form for $\tnab$ is:
\begin{subequations}
 \begin{align}
 (\tnab_Xs)(x)&=\Big[(\mtu,x),\dsdd{\os(X(t))}{t}{0}\Big]\\\label{nabs}
              &=[(\mtu,x),X(\os)],
\end{align}
\end{subequations}
where $X(\os)$ is well defined because $\os$ is a map from $\eR^2$ into a vector space (namely: $\Lambda W $).

\subsection{Dirac operator on the euclidian \texorpdfstring{$\eR^2$}{R2}}
%///////////////////////////////////////////////////
\index{dirac!operator!on $\eR^2$}

We continue to write explicitly the definition \eqref{dirac}. Putting together \eqref{gammaX} and \eqref{nabs}, one finds
\begin{equation}
 \gamma^{\alpha}_x(\tnab_{e_{\beta}}s)(x)	=\gamma(e_{\alpha x})[\xi,e_{\beta}(\os)]
                                     		=[\xi,\tilde\rho(b^{-1}(e_{\alpha x}))e_{\beta}(\os)].
\end{equation}
Here, $e_{\beta}=\partial_{\beta}$ and $b=\mtu$, then
\[
 \gamma^{\alpha}_x(\tnab_{e_{\beta}}s)(x)=[(\mtu,x),\tilde\rho(e_{\alpha})\partial_{\beta}\os].
\]
Now, the Dirac operator reads
\[
 (\Dir s)(x)=[(\mtu,x),\gamma^{\alpha}\partial_{\alpha}\os].
\]

We can obtain a more compact expression by defining ``$Ys$''\ and ``$As$'' when $s\in\Gamma(\mS)$, $Y\in\cvec(\eR^2)$ and $A\in\End{\Lambda W }$. The definitions are
\begin{align*}
(Ys)(x)&=[(\mtu,x),(Y\os)(x)],\\
(As)(x)&=[(\mtu,x),A\os(x)].
\end{align*}
With these conventions, one writes:
\[
(\Dir s)(x)=\gamma^{\alpha}(\partial_{\alpha} s)(x).
\]
This justifies the expression \eqref{dirflat}: $\Dir=\gamma^{\alpha}\partial_{\alpha}$ on flat spaces. With a good choice of basis of $\Lambda W $, the matrices $\gamma^{\alpha}$ are given by \eqref{gammaR2}, and
\[
\gamma^{\alpha}\partial_{\alpha}=
\begin{pmatrix}
0 & -1 \\
1 & 0
\end{pmatrix}\partial_x-
\begin{pmatrix}
0 & i \\
i & 0
\end{pmatrix}\partial_y.
\]
If we identify $\eR^2$ with $\eC$ we have the following definitions:
\[
\partial_z=\frac{1}{2}(\partial_x-i\partial_y),\qquad\partial_{\overline{z}}=\frac{1}{2}(\partial_x+i\partial_y),\]
so that
\[\Dir=\begin{pmatrix}
0 & -\partial_{\overline{z}} \\
\partial_z & 0
\end{pmatrix}.
\]

%---------------------------------------------------------------------------------------------------------------------------
\subsection{Dirac operator as elliptic pseudo-differential operator}
%---------------------------------------------------------------------------------------------------------------------------
\label{subSecREctBOh}

Let $(M,g)$ be a spin manifold and $D$, its Dirac operator which is locally written under the form $D=\gamma(dx^{\mu})\partial_{\mu}$. So $A_{\mu}(x)=\gamma(dx^{\mu})$, and the principal symbol is
\[
  \xi^{\mu}A_{\mu}(x)=\gamma(\xi).
\]
Let us point out that $\gamma(\xi)$ is not a real number, but an endomorphism of the spinor bundle. Using relation \eqref{3101r3}, we find that
\[
  \gamma(\xi)^2=-\| \xi \|^2\id,
\]
which is invertible when $\xi\neq 0$. We conclude that Dirac is an elliptic operator\footnote{Definition~\ref{DefGLpDEHy}.}.


\section{Clifford algebras and Morita equivalence}
%++++++++++++++++++++++++++++++++++++++++++++++++

Let $\cA$ be an algebra. An algebra $\cB$ is said to be \defe{Morita equivalent}{Morita equivalence}\label{PgMoritaEq} to $\cA$ if $\cB=\End_{\cA}(\modE)$ for some finite projective module $\modE$ over $\cA$. The algebra $\cA$ is Morita equivalent to itself taking the trivial module $\modE=\cA$.

We consider a manifold $M$ of dimension $n=2m$.

\begin{probleme}
	The two following statements are imprecise.
\end{probleme}

\begin{proposition}
A module which implement a Morita equivalence between two $C^*$-algebras is finite projective.
\end{proposition}

\begin{theorem}[Serre-Swan]
If one of the two Morita equivalent is the continuous function space over a manifold $\cA=C(M)$, then the module which gives the Morita equivalence is the section of continuous sections of a vector bundle over $M$, $\modE=\Gamma(E)$.
\end{theorem}
Furthermore, if $\cA=C(M)$ and $\cB=\Gamma(\Cl(M))$, we have $\End E\simeq \Cl(M)$ as isomorphism of vector bundle. Since $\Cl M$ is of rank $2^n$, $\End E$ has same rank and $E_x$ has dimension $\sqrt{2^n}=2^m$. So it is possible to choose the Clifford action in such a way that $\Gamma(E)$ is an irreducible Clifford module.

We often look at an anti-linear map $J\colon \Gamma(E)\to \Gamma(E)$ such that for all $\psi\in\Gamma(E)$
\begin{enumerate}
\item $J(\psi f)=(J\psi)\overline{ f }$ for all $f\in C(M)$,
\item $J(a\psi)=\epsilon(a)a J\psi$ for all $a\in\Gamma^{\infty}(\Cl M)$.
\end{enumerate}
How to define $a\psi$? We consider $\cA=C(M)$, $\cB=\Gamma(\Cl M)$ and we define $\Gamma(E)$ is such a way that it implements a Morita equivalence between $\cA$ and $\cB$; hence $\Gamma(E)$ is a $C(M)$-module. From dimensional considerations, we can define on $\Gamma(E)$ a Clifford module structure, i.e. a $C(M)$-linear
\begin{equation}
  c\colon \Gamma(\Cl M)\to \End(\Gamma E),
\end{equation}
hence $a\psi$ makes sense for any $a\in\Gamma^{\infty}(\Cl M)$ and $\psi\in\Gamma(E)$ with definition
\begin{equation}
 (a\psi)(x)=\big( c(a)\psi \big)(x)
		=c(a(x))\psi(x)
\end{equation}

\begin{theorem}
Let $(M,S,J)$ be a spin manifold of dimension $n$. There exists an unique connection
\[
  \nabla^S\colon \Gamma^{\infty}(S)\to \Gamma^{\infty}(S)\otimes\Omega^1(S)
\]
such that
\begin{enumerate}
\item $\scalp{ \nabla^S\psi }{ \phi }+\scalp{ \psi }{ \nabla^S\phi }=d\scalp{ \psi }{ \phi }$,
\item $[\nabla^S,J]=0$,
\item $\nabla^S\big( c(a)\psi \big)=c\big( \nabla a \big)\psi+c(a)\nabla^S\psi$ for all $a\in\Cl(M)$ and $\psi\in\Gamma^{\infty}(S)$.
\end{enumerate}
In the latter, the action of $\Gamma^{\infty}(\Cl M)$ on $\Gamma^{\infty}(S)$ is induced from the action $c\colon \Cl(T^*_xM)\to \End S$. The $\nabla$ which acts on $a$ is the connection extended to $\Gamma^{\infty}(\Cl M)$ by virtue of Leibnitz rule $\nabla(uv)=\nabla(u)v+u\nabla(c)$.

\end{theorem}

\begin{proof}
No proof
\end{proof}


In this setting, we define
\begin{equation}
\begin{aligned}
 \hat c\colon\Gamma^{\infty}(S)\otimes\Gamma^{\infty}(\Cl M)&\to \Gamma^{\infty}(S) \\
 \psi\otimes a&\mapsto c(a)\psi.
\end{aligned}
\end{equation}
Then we define the \defe{Dirac operator}{dirac!operator} $\Dir\colon \Gamma^{\infty}(S)\to \Gamma^{\infty}(S)$,
\begin{equation}
  \Dir=-i(\hat c\circ\nabla^S).
\end{equation}

\subsection{Example: quantum field theory}
%-----------------------------------------

Let us show how does this operator gives back the usual Dirac operator of quantum field theory. Let $M$ be a manifold and with two local basis $\{ \partial_u \}$ and $\{ \partial_{\alpha} \}$ of $T_xM$. The first one is the ``natural'' basis: $g(\partial_u,\partial_v)=g_{uv}$ has no particular properties while the second one is orthonormal $g(\partial_{\alpha},\partial_{\beta})=\delta_{\alpha\beta}$. The first dual basis is defined by $dx^{\alpha}\partial_{\beta}=\delta^{\alpha}_{\beta}$.

We write $\partial_{\alpha}=e_{\alpha}^u\partial_u$ and for the dual basis, $dx^{\alpha}=e^{\alpha}_u\,dx^u$. In order these definition to be coherent, we impose $dx^{\alpha}\partial_{\beta}=\delta^{\alpha}_{\beta}$:
\begin{equation}
  dx^{\alpha}\partial_{\beta}=e_u^{\alpha}dx^u\big( e^v_{\beta}\partial_v \big)
		=e^{\alpha}_ue^v_{\beta}\delta^u_v
		=e^{\alpha}_ue^u_{\beta}.
\end{equation}
We conclude that the \defe{vielbein}{vielbein} $(e^{\alpha}_u)$ is the inverse of $(e^v_{\beta})$: $e^{\alpha}_ue^u_{\beta}=\delta^{\alpha}_{\beta}$. The vielbein are eventually complexes.



\subsection{An other definition of the Dirac operator}
%-----------------------------------------------------

Let us consider an orthonormal basis $\{ e_a \}$ of $M$, i.e. on each $x\in M$,
\[
  g_x(e_a(x),e_b(x))=\eta_{ab}.
\]
This basis is related to a ``natural'' basis $\{ \partial_{\mu} \}$ by
\begin{equation}
  e_a=e_a^{\mu}\partial_{\mu}
\end{equation}
where  $e_a^{\mu}$ is called \defe{vielbein}{vielbein} (here, they are more precisely $n$-beins). As far as metric is concerned we have
\begin{subequations}
\begin{align}
	g^{\mu\nu}&=e_a^{\mu}e_b^{\nu}\eta_{ab}\\
	\eta_{ab}&=e_a^{\mu}e_b^{\nu}g_{\mu\nu}.
\end{align}
\end{subequations}
If $\nabla$ is the covariant derivative associated with $g$, we define the coefficients $\omega_{\mu a}^b$ by
\begin{equation}
\nabla_{\mu}e_a=\omega_{\mu a}^be_b.
\end{equation}
On the other hand, $\nabla$ is related to the Christoffel symbols by
\begin{equation}
\nabla_{\mu}\partial_{\nu}=\Gamma_{\mu\nu}^{\sigma}\partial_{\sigma}.
\end{equation}
Let $\Cl(M)$ be the Clifford module whose fibre is the Clifford complex algebra $\Cl(T^*_xM)^{\eC}$. We consider $\Gamma(\Cl(M))$, the module of corresponding sections. It gives an algebra morphism
\begin{equation}
\begin{aligned}
 \gamma\colon\Gamma(\Cl(M))&\to \opB(\hH) \\
dx^{\mu}&\mapsto \gamma^{\mu}(x)=\gamma^ae_a^{\mu}
\end{aligned}
\end{equation}
which can be extended to the whole Clifford algebra. One can choose matrices $\gamma^{\mu}(x)$ and $\gamma^a$ to be hermitian; they satisfy
\begin{subequations}
\begin{align}
\gamma^{\mu}(x)\gamma^{\nu}(x)+\gamma^{\nu}(x)\gamma^{\mu}(x)&=-2g(dx^{\mu},dx^{\nu})=-2g^{\mu\nu}\\
\gamma^a\gamma^b+\gamma^b\gamma^a&=-2\eta^{ab}.
\end{align}
\end{subequations}
All this allow us to lift the Levi-Civita connection from the tangent bundle to the spinor bundle by defining
\begin{equation}
\nabla_{\mu}^S=\partial_{\mu}+\omega^S_{\mu}=\partial_{\mu}+\frac{ 1 }{2}\omega_{\mu ab}\gamma^a\gamma^b.
\end{equation}
The \defe{Dirac operator}{operator!Dirac} is then given by
\[
  \Dir=\gamma\circ\nabla
\]
and can locally be written under the form
\begin{equation}  \label{eq_Dirac_deux}
\Dir=\gamma^{\mu}(x)(\partial_{\mu}+\omega_{\mu}^S)
	=\gamma^ae_a^{\mu}(\partial_{\mu}+\omega^S_{\mu}).
\end{equation}


\chapter{Relativistic fields and group theory}
% This is part of Mes notes de mathématique
% Copyright (c) 2014, 2019-2020
%   Laurent Claessens
% See the file fdl-1.3.txt for copying conditions.

\section{Mathematical framework of field theory}
%++++++++++++++++++++++++++++++++++++++++++++++

    This is a short review; the aim is to see why the quantum theory of fields needs representations of the Poincaré group. It will be mostly physics oriented. References dealing with field theory including gauge theory and representations are \cite{Sternberg,Preparation,Naber,QMVirtanen,MQSenechal,Weinberg,WormerAngular}.

\subsection{Axioms of the (quantum) relativistic field theory}
%-------------------------------------------------------------

The quantum mechanics is based on a few number of axioms:
\begin{enumerate}\label{pg:axiomes}
\item We have a Hilbert space $\pH$. A physical state is given by a \defe{ray}{ray} in $\pH$, i.e. a set
       \[
           \rR=\{\xi\psi:|\xi|=1\}
       \]
for a certain $\psi\in\pH$ with $\scalh{\psi}{\psi}=1$. In other words, the set of physical sates is the quotient of the set of unital vectors in $\pH$ by the relation $\psi\sim\psi'$ if and only if $\psi=\xi\psi'$ for some unimodular complex number $\xi$. We denote by $\Ray\pH$\nomenclature{$\Ray\pH$}{Rays in a Hilbert space} the set of all rays in $\pH$.

\item\label{ax:vaps} The observables are represented by hermitian linear operators on $\pH$. A state $\rR$ has value $\alpha$ for the observable $A$ if $A\rR=\alpha\rR$, where the action of $A$ on the ray is obvious (and well defined because $A$ is linear).

\item If one has a system described by a state $\rR$, and if one want to measure if it is in one of the state $\rR_1,\ldots,\rR_n$ (orthogonal rays), the answer will be $\rR_i$ with probability
        \[
	    P(\rR\to\rR_i)=|\scalh{\rR}{\rR_i}|^2.
        \]
If the $\rR_n$ form a complete system, one has a theorem which states that
\[
   \sum_i P(\rR\to\rR_i)=1.
\]

\item\label{ax:reprez}  The rays of $\pH$ furnish a representation of the (identity component of) Poincaré group.
\end{enumerate}

This last point can look strange; we will see later (page \pageref{pg:poincare_act}) how it comes. It is the expression of a relativistic theory. That axiom is the reason why one make intensive use of representation theory in relativistic (quantum) field theory \ldots or maybe the intensive use of representation theory is the reason of that axiom. However, we will make an intensive use of representation theory developed in chapter~\ref{ChapThoComsGroupes}.

\subsection{Symmetries and Wigner's theorem}
%-------------------------------------------

Consider the following situation: someone observes a system in a state $\rR$, and makes measures $P(\rR\to\rR_i)$. An other person observes the same system which is, for him, in a state $\rR'$ and observes $P(\rR'\to\rR'_i)$.

If two observers are related by a transformation of the Hilbert state which induces $\rR\to\rR'$ and $\rR_i\to\rR_i'$, there are said \defe{equivalent}{equivalence!of observer} if
\begin{equation}\label{eq:sym_isom}
   P(\rR\to\rR_i)=P(\rR'\to\rR'_i).
\end{equation}

Let us say it more precisely from a mathematical point of view. A \defe{symmetry}{symmetry!of quantum system} is an invertible operator $\dpt{T}{\Ray\pH}{\Ray\pH}$ such that for any $\phi_i\in\rR_i$, $\phi_i'\in T\rR_i$ and $\phi''_i\in T^{-1}\rR_i$,
\begin{equation}\label{eq:}
  |\scalh{\phi'_1}{\phi'_2}|^2=|\scalh{\phi_1}{\phi_2}|^2=|\scalh{\phi''_1}{\phi''_2}|^2
\end{equation}


\begin{remark}
Here, neither $\rR$ nor $\rR'$ are measurable: the $P$'s only are measurable.
\end{remark}

The following can be found in  \cite{Weinberg} p.91, \cite{Sternberg} p.354.
\begin{theorem}[Wigner]\label{tho:Wigner}
Any symmetry $T$ is induced by an operator $U$ on $\pH$ such that $\psi\in\rR$ implies $U\psi\in\rR'$. This operator is either unitary and linear, either anti-unitary and antilinear.
\end{theorem}

So, the symmetry operator must satisfy
\begin{subequations}
\begin{align}
  \scalh{U\psi}{U\phi}&=\scalh{\psi}{\phi}\\
  U(\xi\psi+\eta\psi)&=\xi U\psi+\eta U\phi,
\end{align}
\end{subequations}
or
\begin{subequations}
\begin{align}
  \scalh{U\psi}{U\phi}&=\scalh{\psi}{\phi}^*\\
  U(\xi\psi+\eta\psi)&=\xi^* U\psi+\eta^* U\phi.
\end{align}
\end{subequations}
In the anti-linear case operator, we do not define $U^{\dag}$ by $\scalh{\phi}{U^{\dag}\psi}=\scalh{U\phi}{\psi}$ because the left-hand side should be anti-linear with respect to $\psi$ while the right-had should be linear. In place, for an antilinear operator $A$, we define $A^{\dag}$ by
\begin{equation}
   \scalh{\phi}{A^{\dag}\psi}=\scalh{A\phi}{\psi}^*=\scalh{\psi}{A\phi}.
\end{equation}
In this way, the definitions of unitary and anti-unitary in term of dagger are the same: $U^{\dag}=U^{-1}$.

For any transformation $\dpt{T}{\Ray \pH}{\Ray \pH}$, the Wigner's theorem provides an operator $\dpt{U(T)}{\pH}{\pH}$ which induces $T$ on $\Ray$. If the operator $T$ depends on a parameter $\theta$, the operator $U(T(\theta))$ depends on $\theta$. If $T$ depends continuously on the parameter then the family $U(T(\theta))$ only contains unitary/linear operators or only antiunitary/antilinear operators.

In physical cases, $T(\theta)$ is mostly a Poincaré transformation: $\theta=(\Lambda,p)$. But $T(\mtu,0)$ is the identity which is represented by $U(\mtu,0)=\mtu$. Then all the (connected to identity) Poincaré transformations are represented by linear and unitary operators on $\pH$.

We will follow the proof given in \cite{Weinberg}. An other form of the proof can be found in \cite{Sternberg}. The latter use a slightly different formalism in the axioms of the quantum mechanics; this is explained in appendix~\ref{app:Wigner}. It is now time to prove the theorem.

\begin{proof}[Proof of Wigner's theorem]
We consider an orthonormal basis $\{\psi_k\}$ of $\pH$ with $\psi_k\in\rR_k$, and a choice of $\psi'_k\in T\rR_k$. From this and the assumptions, we have
\[
|\scalh{\psi'_k}{\psi'_l}|^2=|\scalh{\psi_k}{\psi_l}|^2=\delta_{kl}.
\]
Then $\scalh{\psi'_k}{\psi'_k}=0$ whenever $k\neq l$ and, since $\scalh{\psi'_k}{\psi'_k}$ is real and positive, $\scalh{\psi'_k}{\psi'_k}=1$. So $\scalh{\psi'_k}{\psi'_l}=\delta_{kl}$.

The set $\psi_k'$ is also complete in $\pH$. Indeed suppose that we have a vector $\psi'\in\pH$ such that $\scalh{\psi'}{\psi'_k}=0$ for all $k$. If $\psi'\in\rR$, we consider a $\psi''\in T^{-1}\rR$ and we have
\[
   |\scalh{\psi''}{\psi_k}|^2=|\scalh{\psi'}{\psi'_k}|^2=0,
\]
which contradicts the fact that the $\psi_k$'s form a complete set. Now we have to fix a phase convention for the $\psi_k$. Since there are no canonical choice of phase, we fix with respect to an arbitrary one of the $\psi_k$, say $\psi_1$. We put
\begin{equation}
  \gamma_k=\us{\sqrt{2}}(\psi_1+\psi_k)\in\rC_k
\end{equation}
for $k\neq 1$. Any $\gamma'_k\in T\rC_k$ can be written in the basis $\{\psi'_k\}$:
\begin{equation}\label{eq:gamma_psi_k}
  \gamma'_k=\sum_l c_{kl}\psi'_l.
\end{equation}
From assumption \eqref{eq:sym_isom} and the fact that $|c_{kl}|^2=|\scalh{\gamma'_k}{\psi'_l}|^2$, we find, for $k,l\neq 1$
\[
  |c_{kl}|^2=\frac{1}{2}\delta_{kl}.
\]
We can choose the phase of $\gamma'_k$ and $\psi'_k$ in order to get $c_{kk}=c_{k1}=1/\sqrt{2}$. For this, we begin to fix $\gamma'_k$ in such a manner to get $c_{k1}=1/\sqrt{2}$ (from $|c_{k1}|=|\scalh{\gamma'_k}{\psi'_1}|$), and next we fix $\psi_k'$ for the $c_{kk}$. From now on, the so chosen $\gamma'_k$ and $\psi'_k$ are denoted by $U\gamma_k$ and $U\psi_k$.

What we  did until now is to take a basis $\{\psi_k\}$ of $\pH$ and define $\gamma_k=1/\sqrt2(\psi_1+\psi_k)$. Next we had chosen the phases of $\psi'_k\in T\rR_k$ and $\gamma'_k\in T\rC_k$ in order to have
\begin{equation}\label{eq:c_kl}
\begin{aligned}
   c_{kk}=c_{k1}&=1/\sqrt 2&&\forall k,\\
   c_{kl}&=0 &&\text{if }l\neq k,l\neq 1.
\end{aligned}
\end{equation}
This allows us to check a certain linearity for the operator $U$:
\begin{equation}
\begin{aligned}
U\left( \us{\sqrt 2}(\psi_k+\psi1)  \right)
   &=U\gamma_k\\
   &=\gamma'_k\\
   &=\us{\sqrt 2}\psi'_1+\us{\sqrt 2}\psi'_k&&\text{from \eqref{eq:gamma_psi_k} and \eqref{eq:c_kl}}\\
   &=\us{\sqrt{2}}\big(  U\psi_1+U\psi_k   \big).
\end{aligned}
\end{equation}
Now we have to build $U$ on a general vector $\psi=\sum_k\psi_k\in\mR$. Any vector $\psi'\in T\mR$ can be decomposed with respect to the basis $\{\psi'_k=U\psi_k\}$:
\begin{equation}\label{eq:dev_Upsi}
  \psi'=\sum_k C'_kU\psi_k.
\end{equation}
From the conservation of probability $|\scalh{\psi_k}{\psi}|^2=|\scalh{U\psi_k}{\psi'}|^2$ and $|\scalh{\gamma_k}{\psi}|^2=|\scalh{U\gamma_k}{\psi'}|^2$, we find
\begin{subequations}\label{eq:deux_C_k}
\begin{align}
          |C_k|^2&=|C'_k|^2, \label{eq:deux_C_k_a} \\
       |C_k+C_1|^2&=|C_k'+C'_1|^2.
\end{align}
\end{subequations}
If one writes $C_k=a_k+ib_k$, one finds $\real(C_k/C_1)=(a_ka_1+b_kb_1)/|C_1|^2$. By doing the same with $C'_k$ and using \eqref{eq:deux_C_k},
\begin{equation}\label{eq:C_k_C_1}
  \real(C_k/C_1)=\real(C'_k/C'_1).
\end{equation}
Equation \eqref{eq:deux_C_k_a} also imposes
\begin{equation}\label{eq:frac_C_k}
 |C_k/C_1|^2=|C'_k/C'_1|^2,
\end{equation}
while compatibility between \eqref{eq:frac_C_k} and \eqref{eq:C_k_C_1} requires
\begin{equation}\label{eq:C_k_C_1_im}
  \imag(C_k/C_1)=\pm\imag(C'_k/C'_1).
\end{equation}
Equations \eqref{eq:C_k_C_1} and \eqref{eq:C_k_C_1_im} show that $C_k$ and $C'_k$ must satisfy
\begin{subequations}
\begin{align}
  C_k/C_1&=C'_k/C'_1\\
\intertext{xor}
  C_k/C_1&=(C'_k/C'_1)^*.
\end{align}
\end{subequations}
For a given $\psi$ we have to show that the choice must be the same for all the $C_k$\footnote{We will show later that for a given $T$, the choice must be the same for all the $\psi$.}. Let $l\neq k$ and suppose that $C_k/C_1=C'_k/C'_1$ and $C_l/C_1=(C'_l/C'_1)^*$; we will show that in this case, one of the two ratios is real. So we can suppose $k\neq 1\neq l$. We consider the vector $\Phi=\us{\sqrt 3}(\psi_1+\psi_k+\psi_l)$,
\[
  \Phi'=\frac{\alpha}{\sqrt 3}\big( U\psi_1+U\psi_k+U\psi_l  \big)
\]
where $\alpha\in\eC$ satisfies $|\alpha|=1$. The conservation of probability $|\scalh{\Phi}{\psi}|^2=|\scalh{\Phi'}{\psi'}|^2$ gives $|C_1+C_k+C_l|^2=|C'_1+C'_k+C'_l|^2$. Since $|C_1|^2=|C'_1|^2$, we can divide the left hand side by $|C_1|^2$ and the right one by $|C'_1|^2$. We find
\[
\left|1+\frac{C_k}{C_1}+\frac{C_l}{C_1}\right|^2=\left|1+\frac{C'_k}{C'_1}+\frac{C'_l}{C'_1}\right|^2.
\]
Using the assumption $C_k/C_1=C'_k/C'_1$ and $C_l/C_1=(C'_l/C'_1)^*$, we are in a case of an equation of the form $|u+v|^2=|u+v^*|^2$ with $u$, $v\in\eC$. If we write $u=a+bi$ and $v=x+iy$, we find $b+y=\pm(b-y)$, so that it leaves the choice $y=0$ or $b=0$ which corresponds to $(C_k/C_1)\in\eR$ or $(C_l/C_1)\in\eR$. So the coefficients $C'_k$ ($k\neq 1$) in the expansion \eqref{eq:dev_Upsi} must satisfy
\begin{subequations}
\begin{align}
  C_k/C_1&=C'_k/C'_1\quad\forall k \label{eq:rap_C_a} \\
\intertext{xor}
  C_k/C_1&=(C'_k/C'_1)^*\quad\forall k  \label{eq:rap_C_b}.
\end{align}
\end{subequations}
Note that the phase of $C_1$ is not yet fixed. We naturally choice $C_1=C'_1$ or $C_1={C'_1}^*$ following the case. We define $\dpt{U}{\pH}{\pH}$ by
\begin{subequations}\label{eq:def_U}
\begin{align}
   U\left( \sum_k C_k\psi_k \right) &= \sum_kC_kU\psi_k&&\text{if \eqref{eq:rap_C_a}},
\label{eq:def_U_a}
\intertext{xor}
  U\left( \sum_k C_k\psi_k \right) &= \sum_kC_k^*U\psi_k&&\text{if \eqref{eq:rap_C_b}}.
\label{eq:def_U_b}
\end{align}
\end{subequations}
This preserves the probability because $|\scalh{\psi}{\psi_k}|^2=|C_k|^2$ while $|\scalh{U\psi}{U\psi_k}|$ is equal to $|C_k|^2$ or $|C^*_k|^2$ (which are the same) following the case \eqref{eq:def_U_a} or \eqref{eq:def_U_b}.

Now we have to prove that the choice \eqref{eq:def_U_a} or \eqref{eq:def_U_b} is fixed by the data of $T$ and must be the same for all the $\psi\in\pH$. For, let us consider two vectors $\phi=\sum A_k\psi_k$, $\varphi=\sum B_k\psi_k$ and suppose that
\[
   U\phi=\sum_k A_k U\psi_k\text{ but } U\varphi=\sum_kB^*_k U\psi_k.
\]
In order to see that it is impossible, looks at the conservation of probability $|\sum_k A_kB^*_k|^2=|\sum_k A_kB_k|^2$, then
\begin{equation}
\sum_{kl}\big(   B^*_lB_kA_lA^*_k-B^*_lB_kA^*_lA_k    \big)=\sum_{kl}B^*_lB_k\imag(A_lA^*_k)=0.
\end{equation}
Since $A_lA^*l\in\eR$, we can regroup each term $(k,l)$ with the corresponding term $(l,k)$. We get
\begin{equation}\label{eq:imim_zero}
\begin{split}
  0=\sum_{kl}\imag(A_lA^*_k)(B^*_lB_k-B^*_kB_l)
   =\sum_{kl}\imag(A^*_kA_l)\imag(B^*_kB_l).
\end{split}
\end{equation}
We can find a vector $\sum_kC_k\psi_k$ such that
\begin{subequations}\label{eq:choix_C}
\begin{align}
 \sum_{kl}\imag(C^*_kC_l)\imag(A^*_kA_l)&\neq 0  \label{eq:choix_C_a}\\
\intertext{and}
 \sum_{kl}\imag(C^*_kC_l)\imag(B^*_kB_l)&\neq 0.
\end{align}
\end{subequations}
In order to see how to find such a vector, let us show that there always exists a choice $(i,j)$ such that $B^*_iB_j$ is not real. Let us say $B_1=x+iy$ and $B_k=a_k+bi$. If $y\neq 0$, the condition $\imag(B_1^*B_k)=0$ gives $B_k=\frac{b_k}{y}B_1$. It is always possible to find a sequence $(b_k)$ which gives $1$ as norm for $\sum B_k\psi_k$; the problem is not there. The problem is that $B_k/B_1\in\eR$, so that the choice \eqref{eq:def_U}  is not a true choice. For the same reason, all the $B^*_iB_k$ can't be pure imaginary.

Now we can find the vector which satisfy \eqref{eq:choix_C}. There are several cases. If there is a pair $(k,l)$ such that $A^*_kA_l$ and $B^*_kB_l$ are both complex, we can take all $C_i$'s zero for $k\neq i\neq l$ and choose $C_k$ and $C_l$ in such a way that $C^*_kC_l$ is not real. If there is a pair $(k,l)$ with $A^*_kA_l$ complex and $B^*_kB_l$ real, we consider a pair $(m,n)$ such that $B^*_mB_n$ is complex. If $A^*_mA_n$ is complex, we take all the $C_i$'s zero except $C_m$ and $C_n$ such that $\imag(C^*_mC_n)\neq 0$. If $A^*_mA_n$ is real, we take all the $C_i$'s zero except $C_k,C_l,C_m,C_n$ which we choose in such a way that $\imag(C^*_mC_n)\neq 0$ and $\imag(C^*_kC_k)\neq 0$.

Equation \eqref{eq:choix_C_a} makes that the same choice must be made for $\sum A_k\psi_k$ and $\sum C_k\psi_k$ (if it was not the case, we would have an equation of the form of \eqref{eq:imim_zero}). For the same reason, the same choice must be made for $\sum B_k\psi_k$ and $\sum C_k\psi_k$. So we conclude that the data of $T$ fixes the choice between \eqref{eq:def_U_a} and \eqref{eq:def_U_b} and that this choice must be the same for all the vectors of $\pH$.

We have to show that the possibility \eqref{eq:def_U_a} makes $U$ linear and unitary while the possibility \eqref{eq:def_U_b} makes $U$ antilinear and antiunitary. For we consider $\psi=\sum_k A_k\psi_k$ and $\phi=\sum_k B_k\psi_k$. If \eqref{eq:def_U_a} works,
\begin{equation}
\begin{split}
  U(\alpha\psi+\beta\phi)&=U\Big(  \sum_k(\alpha A_k+\beta B_k)\psi_k     \Big)\\
                         &=\sum_k(\alpha A_k+\beta B_k)U\psi_k\\
			 &=\alpha U\psi+\beta U\phi,
\end{split}
\end{equation}
and
\begin{equation}
\scalh{U\psi}{U\phi}=\sum_{kl}A^*_kB_l\scalh{U\psi_k}{U\psi_l}=\sum_kA^*_kB_k,
\end{equation}
so that $\scalh{U\psi}{U\phi}=\scalh{\psi}{\phi}$. Thus in this case $U$ is linear and unitary. In the case where \eqref{eq:def_U_a} works, the computations are almost the same:
\begin{equation}
\begin{split}
  U(\alpha\psi+\beta\phi)&=U\Big(  \sum_k(\alpha A_k+\beta B_k)\psi_k     \Big)\\
                         &=\sum_k(\alpha^* A_k^*+\beta^* B^*_k)U\psi_k\\
			 &=\alpha^* U\psi+\beta^* U\phi,
\end{split}
\end{equation}
and
\begin{equation}
\scalh{U\psi}{U\phi}=\sum_{kl}A_kB^*_l\scalh{U\psi_k}{U\psi_l}=\sum_kA_kB^*_k,
\end{equation}
so that $\scalh{U\psi}{U\phi}=\scalh{\psi}{\phi}^*$. In this case, $U$ is antilinear and antiunitary.

\end{proof}

\subsection{Projective representations}
%-------------------------------------------------

If $T_1(\rR_n)=\rR_n'$ and $\psi_n\in\rR_n$, then $U(T_1)\psi_n\in\rR_n'$. If $T_2(\rR')=\rR''$, then $U(T_2)U(T_1)\psi_n\in\rR_n''$. But $U(T_2T_1)\psi_n$ also belongs to $\rR_n''$. Then there exists a $\phi_n(T_2,T_1)\in\eR$ such that
\[
   U(T_2)U(T_1)\psi_n=e^{i\phi_n(T_2,T_1)}U(T_2T_1)\psi_n.
\]
Note that for any $\psi\in \pH$, there exists a $\lambda\in\eR$ such that $\|\lambda\psi\|=1$. Since a real can be sent out the $U(T)$'s, for \emph{any} $\psi\in \pH$, there exists a $\phi$ which only depends on $\psi/\|\psi\|$ such that
\begin{equation}
    U(T_2)U(T_1)\psi=e^{i\phi(T_2,T_1)}U(T_2T_1)\psi
\end{equation}

\begin{proposition}
The $\phi$ doesn't depend at all on the $\psi$:
\begin{equation}
  U(T_2)U(T_1)=e^{i\phi(T_2,T_1)}U(T_2T_1).
\end{equation}
\end{proposition}

\begin{proof}
Let us consider a $\psi_A$ and a $\psi_B$ which are not proportional each other. One has a $\phi_{AB}(T_2,T_1)$ such that
\begin{equation}
\begin{split}
e^{i\phi_{AB}(T_2,T_1)}U(T_2T_1)(\psi_A+\psi_B)&=U(T_2)U(T_1)(\psi_A+\psi_B)\\
                                               &=e^{i\phi_A(T_2,T_1)}U(T_2T_1)\psi_A\\
                                               &\quad +e^{i\phi_B(T_2,T_1)}U(T_2T_1)\psi_B.
\end{split}
\end{equation}
Now, we apply $U(T_2T_1)^{-1}$ to both sides. If it is unitary, the $e^{i\phi}$ get out without problems; else is get out as $e^{-i\phi}$:
\begin{equation}
  e^{\pm i\phi_{AB}}(\psi_A+\psi_B)=e^{\pm i\phi_A}\psi_A+e^{\pm i\phi_B}\psi_B.
\end{equation}
Since $\psi_A$ and $\psi_B$ are linearly independent, the only solution is $e^{i\phi_{AB}}=e^{i\phi_A}=e^{i\phi_B}$.

\end{proof}

Since the operators $U(T)$ must only fulfil
\begin{equation}\label{eq:projectif}
   U(T_2)U(T_1)=e^{i\phi(T_2,T_1)}U(T_2,T_1),
\end{equation}
these form a \defe{projective representation}{projective!representation} of the symmetry group on the physical Hilbert space $\pH$.

\begin{remark}
In order to have some physical relevance, this demonstration supposes that a state $\psi_A+\psi_B$ exists in nature. If one can divide the particles in several ``incompatibles''\ classes labeled by $a,b$ such that $\psi_a+\psi_b$ doesn't exist, then equation \eqref{eq:projectif} is false and one has to write
\[
   U(T_2)U(T_1)\psi_a=e^{i\phi_a(T_2,T_1)}U(T_2T_1)\psi_a
\]
because we can't show that $\phi_a=\phi_b$ from the simple fact that $\psi_a+\psi_b$ doesn't exist!

For example, physicists think that there are no superposition of state of integer and semi-integer spin.
\end{remark}

\begin{remark}
If the group satisfies some requirements, one can choose $\phi=0$. From now we suppose that we are in this case: we work with ``true''\ representations.
\end{remark}


\subsection{Representations and power expansions}
%------------------------------------------------

Let $G$ be an arc connected Lie group whose elements are denoted by $T(\theta)$ with $\theta$, a continuous family of parameters (from a local chart). The multiplication law is given by a function $\dpt{f}{\eR^n\times\eR^n}{\eR^n}$:
\begin{equation}\label{eq:T_groupe}
T(\theta')T(\theta)=T\big(f(\theta',\theta)\big).
\end{equation}
If $ \theta=0$ is the coordinate of the identity,
\begin{equation}\label{eq:f_0}
f(0,\theta)=f(\theta,0)=\theta.
\end{equation}

We suppose that $G$ acts on the rays of a Hilbert space $\pH$, so that there are represented on $\pH$ by unitary operators $U\big(T(\theta)\big)$. We denote by $W$ the group of transformations of $\pH$; roughly speaking,
\[
     W=U(G).
\]
Now, we are going to cheat a little. We know that there exists a normal neighbourhood\index{normal!neighbourhood} of $e$ in $W$. In simple words, the map $\dpt{\exp}{\lW}{W}$ is a diffeomorphism between the elements of $\lW$ ``close''\ to $0$ and the ones of $W$ close to $e$. By \textit{close to}, we mean that the components of $\theta$ are small enough. If $\{it_a\}$ is a basis of $\lW$, we define
\begin{equation}\label{eq:U_expo}
   U(T(\theta))=e^{i\theta^at_a}.
\end{equation}
In other words, one considers the exponential map for a neighbourhood of identity.

The cheat is the fact that $U(T(\theta))$ is actually defined by Wigner's theorem from the data of the group $G$. So equation \eqref{eq:U_expo} should be seen as a requirement in the choice of the basis $\{t_a\}$.

\begin{remark}
The $i$ in the exponential in \eqref{eq:U_expo} and in the definition of the basis $\{it_a\}$ is a convention in order the $t_a$'s to be hermitian. Indeed, the Lie algebra of a group of unitary matrices is made of \emph{anti}hermitian matrices.
\end{remark}

With all that,
\begin{equation}\label{eq:dev_U}
   U(T(\theta))=\mtu+i\theta^at_a+\frac{1}{2}\theta^b\theta^ct_{bc}+\ldots
\end{equation}
where $t_{bc}$ is defined (among other requirements) to absorb the ``intuitive''\ minus sign in the third term.

Now we are going to explore some consequences of equation \eqref{eq:T_groupe}. Equation \eqref{eq:f_0} makes the expansion of $f$ as
\begin{equation}\label{eq:dev_f}
f^a(\theta',\theta)=\theta^a+{\theta'}^a+f^a_{bc}{\theta'}^b\theta^c+\ldots
\end{equation}
From expansions \eqref{eq:dev_f} and \eqref{eq:dev_U} of $f$ and $U(T(\theta))$, ``group structure''\ equation \eqref{eq:T_groupe} gives (at order two):
\begin{equation}\label{eq:t_ab}
   t_{bc}=-t_bt_c-if^a_{bc}t_a
\end{equation}
and nothing for the first order. Then, providing that one knows the group structure (the $f$), one knows the second order of the representation from the first one.
From equation \eqref{eq:U_expo}, one finds the value of $t_{ab}$:
\[
   e^{i\theta^at_a}=1+i\theta^at_a+\frac{1}{2}(i)^2(\theta^at_a)(\theta^bt_b),
\]
up to constant coefficients, one can choose $t_{ab}$ to be symmetric with respect to $a$ and $b$:
\[
   t_{ab}=\frac{1}{2}(t_at_b+t_bt_a).
\]
Taking this convention and computing $t_{bc}-t_{cb}$ from \eqref{eq:t_ab}, we find
\begin{equation}
  [t_a,t_b]=iC_{ab}^ct_c
\end{equation}
with $C_{ab}^c=f_{ab}^c-f_{ba}^c$.

On the other hand, one knows that if a group is abelian, its algebra is also abelian; we can see it here by considering that if $G$ is abelian, $f(\theta,\theta')=f(\theta',\theta)$, then $f_{ab}^c$ is symmetric and $[t_a,t_b]=0$. We can say more about $f$ Since the $t_a$ commute, equations \eqref{eq:T_groupe} and \eqref{eq:U_expo} make that
\begin{equation}
  e^{if(\theta,\theta')^at_a}=e^{i\theta^at_a}e^{i{\theta'}^bt_b}\\
                             =e^{i(\theta^a+{\theta'}^a)t_a},
\end{equation}
so that
\[
   f(\theta,\theta')=\theta+\theta'.
\]

\section{The symmetry group of nature}
%-------------------------------------

\subsection{Spin and double covering}\label{subsec:sym_nature}
%-----------------------------------

Some of literature carry an ambiguity in the choice of the right space-time symmetry group in the quantum field theory. A very good and deep discussion about the choice of the space-time symmetry group of nature is given in the book \cite{Naber} which will be used here. An other enlightening review can be found in \cite{ModavePoincarre}.

From a relativistic point of view, the group is the Poincaré group of all the maps $\eR^4\to\eR^4$ which leaves invariant the quantity $s^2=-t^2+x^2+y^2+z^2$.  At this point we can already make an important remark: the so defined quantity $s$ is in fact \emph{not} a relativistic invariant. Indeed if I follow a (spatially) closed path, I will measure $\Delta t\neq 0$ and $\Delta x=\Delta y=\Delta z=0$ because in \emph{my} frame, my displacement is zero. A guy who keeps at my starting point will measure (between the beginning and the end of my travel) $\Delta 't\neq 0$ and also $\Delta x=\Delta y=\Delta z=0$. If $s=s'$, then $\Delta t=\Delta t'$.

So the relativistic invariance is only local: $ds^2=ds'^2$, and as far as relativity is concerned, one can work with infinitesimal transformations only. In this case, the distinction between the \emph{groups} $L_+^{\uparrow}$ and $\SLdc$ is no relevant. Intuitively, we choose $L_+^{\uparrow}$ to be the space-time symmetry group. As we will see the difference will reveal to be crucial in relativistic field theory because $L_+^{\uparrow}$ has no half-integer spin representations.

This group naturally splits into two parts: the translations and the rotations (and boost). As far as I know, the translation part makes no difficulties. For the other one, there are some difficulties to find the \emph{minimal} group of symmetry. First, one often want to separate the space-time inversions $P$ and $T$ from the remaining: the group then becomes the homogeneous orthochrone Lorentz group $L_+^{\uparrow}$. An other often presented group is \ldots $\SLdc$. This is our choice here. The physical reason of this choice is all but immediate. As we will see during the following pages, an elementary particle is an irreducible representation of the symmetry group.

For massive particles, the relevant subgroup of $\SLdc$ reveals to be $SU(2)$. If we had chosen the most intuitive $L_+^{\uparrow}$, we would have found $\SO(3)$. There is an important difference between $SU(2)$ and $\SO(3)=SU(2)/\eZ_2$: the first one admits representations of any integer and half-integer spin while the second only posses the integer spin representations (see \ref{NORMooHWAYooPlSDOp}).

Let us now be more precise about the relation between $L_+^{\uparrow}$ and $\SLdc$. A know result is
\[
   L_+^{\uparrow}=\frac{\SLdc}{\eZ_2}.
\]
Let $\dpt{\mSpin}{\SLdc}{L_+^{\uparrow}}$ be the surjective homomorphism with kernel $\pm\mtu_{2\times 2}$ giving this relation.
We will not give a complete proof, but we will explain how $\SLdc$ acts by isometries on $\eR^4$. First, we remark that there exists a bijection between $\eR^4$ and the $2\times 2$ complex hermitian matrices:
\begin{equation}
  v=\begin{pmatrix} t+z & x-iy\\x+iy&t-z\end{pmatrix}=\begin{pmatrix}t\\x\\y\\z\end{pmatrix}.
\end{equation}
If $\lambda\in\SLdc$, the matrix $\lambda v\lambda^{\dag}$ is also hermitian and $\|v\|^2=\det v$. Thus
\begin{equation}
\begin{aligned}
 \Lambda(\lambda)  \colon \eR^4&\to \eR^4 \\
   v&\mapsto \lambda v\lambda^{\dag}
\end{aligned}
\end{equation}
is a Lorentz transformation if and only if $|\det\lambda|=1$. Moreover,
\[
     \Lambda(\lambda\lambda')=\Lambda(\lambda)\Lambda(\lambda').
\]
If $\lambda'=e^{\phi}\lambda$, then $\Lambda(\lambda')=\Lambda(\lambda)$, thus it is natural to impose $\det v=1$ and to consider $\SLdc$ instead of $L(2,\eC)$ to fit $L_+^{\uparrow}$. Now, $\Lambda(\lambda)=\Lambda(-\lambda)$, and we wish to consider $\SLdc/\eZ_2$.

I think the problem is the following: as far as the action of the ``nature group''{} on the space-time is concerned, it is sufficient to consider $L_+^{\uparrow}$. But the group which acts on the state space is wider: it must be $SL(2,\eC)$.

From now, when we say ``Poincaré group'', we mean $\SLdc\times\eR^4$ while  ``Lorentz''\ means $\SLdc$ acting on $\eR^4$ by $\Lambda(\lambda)v=\lambda v\lambda^{\dag}$.

We know by lemma \ref{lem:SO_3} a link between the representations of \( \SU(2)\) and \( \SO(3)\). A know result is the fact that the map $\mSpin$ restricts to a surjective homomorphism $\dpt{\mSpin}{SU(2)}{\SO(3)}$ with kernel $\pm\mtu$ giving the relation $\SO(3)=SU(2)/\eZ_2$. If one considers a representation $\dpt{\rho}{\SO(3)}{GL(V)}$, then $\tilde\rho=\rho\circ\mSpin$ is a representation of $SU(2)$ on $V$. So every representation of $\SO(3)$ comes from a representation of $SU(2)$.

As far as the transformation rule of a (quantum mechanical) wave function under a rotation $R\in \SO(3)$ is concerned, one can see (it is done in \cite{Naber}) that the try
\[
   \begin{pmatrix}
     \psi_1\\
     \psi_2
   \end{pmatrix}\to
T(R)
\begin{pmatrix}
     \psi_1\\
     \psi_2
   \end{pmatrix}
\]
doesn't works if $T(R)$ is a representation of $\SO(3)$ on $\eC^2$. If one allows $T$ to be a representation of $SU(2)$, then our choice ---for an electron--- should naturally be the spin one half representation $T=D^{(1/2)}$. Let us do it. The remaining problem is the following. Let's consider that in a certain frame, an electron is described by the wave function $\begin{pmatrix} \psi_1&\psi_2\end{pmatrix}$, the question is to know the wave function observed by a guy which use another frame linked to the first frame by $R\in \SO(3)$. We always have exactly two elements in $\SU(2)$ projected to $R$ by $\mSpin$; namely $\mSpin(\pm g)=R$; so how to choose between
\[
   D^{(1/2)}(g)
\begin{pmatrix}
     \psi_1\\
     \psi_2
   \end{pmatrix}
\quad\text{and}\quad
   D^{(1/2)}(-g)
\begin{pmatrix}
     \psi_1\\
     \psi_2
   \end{pmatrix}\; ?
\]
The trick is to remark that a change of frame is not the mathematical process described by a single element $R$ of $\SO(3)$, but a physical \emph{continuous} process which begins at the identity and stops at $R$. In other word, we have to ask ourself \emph{how to go from a frame to another}? Taking as example the rotations around the $x$ axis, we can look at two different path in $\SO(3)$ from $\mtu$ to $\mtu$ given by the same expression
\[
  R_1(t)=R_2(t)=
\begin{pmatrix}
1&0&0\\
0&\cos t&\sin t\\
0&-\sin t&\cos t
\end{pmatrix},
\]
but considering $t\colon 0\to 2\pi$ for $R_1$ and $t\colon 0\to 4\pi$ for $R_2$. The covering map $\dpt{\mSpin}{SU(2)}{\SO(3)}$ allows us to lift any path in $\SO(3)$ to a path in $SU(2)$ in an unique way providing a starting point. In other words, if $\mSpin(g)=R$,
\[
\begin{split}
  \exists !\, \tilde{R}(t)\in SU(2)\text{ such that }\mSpin\circ\tilde{R}=R \text{ and } \tilde{R}(0)=\mtu,\\
  \exists !\, \tilde{R}(t)\in SU(2)\text{ such that }\mSpin\circ\tilde{R}=R \text{ and } \tilde{R}(0)=-\mtu.
\end{split}
\]
The question is now: how to choose the right path among these two? The answer comes from the homotopy of $\SO(3)$: the path $R_1$ and $R_2$ belongs to two different classes.

Considering the ``change of frame''{} as a continuous process, the initial point is naturally chosen to be $\mtu$. With this choice, the lift of $R_1$ and $R_2$ are given by
\[
  g_1(t)=g_2(t)=
\begin{pmatrix}
\cos\frac{t}{2}&-i\sin\frac{t}{2}\\
-i\sin\frac{t}{2}&\cos \frac{t}{2}
\end{pmatrix}
\]
with $\dpt{t}{0}{2\pi}$ for $g_1$ and $\dpt{t}{0}{4\pi}$ for $g_2$. In $SU(2)$, the ending point of $g_1$ is $-\mtu$ while the one of $g_2$ is $\mtu$.

It is still possible to say a lot of interesting thinks about the space-time symmetry group of nature; let's just conclude saying that $SU(2)$ is more adapted to the rotations of non zero spin than $\SO(3)$. (it is  not intuitive!)
\subsection{How to implement the Poincaré group}
%----------------------------------------------
\label{pg:poincare_act}

We are not making physics here, but differential geometry and group theory; so we will not discuss the physical relevance of the Poincaré group from a ``speed of light''\ point of view. We consider the \defe{Poincaré group}{Poincaré!group} as the group of all the affine isometries of metric $\eta=diag(-1,1,1,1)$ and the \defe{Lorentz group}{Lorentz!group}\index{group!Lorentz} as the subgroup of rotations and boost.

A Poincaré transformation of $\eR^4$ is given by $(\Lambda,a)$ with $\Lambda$ a $4\times 4$ matrix and $a\in\eR^4$, a translation vector. The composition of $(\Lambda,a)$ with $(\Lambda',a')$ is given by $(\Lambda'\Lambda,\Lambda'a+a')$, the inverse is $(\Lambda^{-1},-\Lambda^{-1} a)$, the neutral is $(\mtu,0)$, and $(\det\Lambda)^2=1$.

The axiom~\ref{ax:reprez} at page \pageref{pg:axiomes} gives us a group of transformation of the rays in $\pH$ parametrised by $(\Lambda,a)$ such that
\begin{equation}
  T(\Lambda',a')T(\Lambda,a)=T(\Lambda'\Lambda,\Lambda'a+a'),
\end{equation}
$\dpt{T(\Lambda,a)}{\Ray \pH}{\Ray \pH}$. Then Wigner's theorem defines a representation of the Poincaré group on $\pH$ by unitary matrices:
\[
\psi\to U(\Lambda,a)\psi.
\]

\begin{remark}
Wigner only ensure existence of \emph{projective} representations. Here we suppose that our symmetry group (maybe slightly different that Poincaré) is such that any projective representations can be turn into a classical representation. We will therefore use the composition law
\begin{equation}\label{eq:composition_U}
  U(\Lambda',a')U(\Lambda)=U(\Lambda'\Lambda,\Lambda'a+a')
\end{equation}
instead of $U(\Lambda',a')U(\Lambda,a')=e^{i\phi(\Lambda,a,\Lambda',a')}U(\Lambda'\Lambda,\Lambda'a+a')$.
\end{remark}

By axiom, the (connected) Poincaré group acts on rays of $\pH$, and we have the representation $U$ which form a group acting on $\pH$. The Lie algebra acts also:
\begin{equation}
u\psi=\Dsdd{U(t)}{t}{0}\psi:=\Dsdd{U(t)\psi}{t}{0}.
\end{equation}
This definition is natural because $\pH$ is a vector space: it can be identified with its tangent space: $U(t)\psi$ is a path in $\pH$ and its derivative at $t=0$ is still a well defined element in $\pH$. Now recall that the operators $U$ are unitary, so that the corresponding operators $u$ are hermitian (therefore diagonalisable).

Let us consider an abelian subgroup $A$ of Poincaré with Lie algebra $\lA$. One can find a basis of $\pH$ made of common eigenvectors of a basis of $\lA$. In other words, one can find a basis of $\pH$ which simultaneously diagonalises all $\lA$. If $\{a_i\}$ is a basis of $\lA$, one can find a basis $\{\ket{\psi_{\lambda}}\}$ (here $\lambda$ labels a basis of $\pH$: it might take continuous values) such that
\begin{equation}
   a_i\ket{\psi_{\lambda}}=\lambda_i\ket{\psi_{\lambda}}.
\end{equation}

\subsection{Momentum operator}
%-----------------------------

Of course, there exists an abelian subgroup of Poincaré: the pure translations, $A=\{U(\mtu,a)\}$. A basis of the Lie algebra is given by four vectors labeled as $P^{\mu}$ and defined by
\[
P^{\mu}=\Dsdd{U(\mtu,te^{\mu})}{t}{0}
\]
 where $e^{\mu}$ is the unit vector following the direction $\mu$ (for $\mu=0$, $e^0=(1,0,0,0)$). One can consider a basis which diagonalises the $P^{\mu}$'s:
\begin{equation}
  P^{\mu}\ket{p,\sigma}=p\hmu\ket{p,\sigma}
\end{equation}
where by definition,
\begin{equation}\label{eq:def_P}
   P\hmu\ket{p,\sigma}=\Dsdd{U(\mtu,te\hmu)\ket{p,\sigma}}{t}{0}.
\end{equation}

\begin{remark}
Be careful on a point: we don't say anything about the symbol ``$p$''\ in the ket. The only property is that it labels a Hilbert space $\pH$. But nothing is already imposed to $\pH$: it must just carry a representation of the Poincaré group on its rays. In particular, it is \emph{a priori} false to say that $p$ is a ``momentum $4$-vector'' and that $p\hmu$ is a component of $p$. Naturally, our notations are adapted to think that! Maybe it is a pedagogical mistake; I don't know.
\end{remark}

This remark can be disturbing: why is generally $\ket{p,\sigma}$ called ``a state of momentum $p$''? Since $U(\mtu,a)$ is unitary, $P\hmu$ is hermitian; the $p\hmu$ are eigenvalues for an hermitian operator, so by axiom~\ref{ax:vaps} (page \pageref{pg:axiomes}) they are candidate to be physical values. But equation \eqref{eq:def_P} shows that $P\hmu$ is what a physicist should call an ``infinitesimal translation'', so that Noether suggests us to interpret the eigenvalue as momentum. We are safe!

The parameters $\sigma$ are not yet defined neither. It will come later. For the moment, we include into the definition of a \defe{one particle state}{state!one particle} that $\sigma$ takes discrete values.

 Since $U(\mtu,a)=e^{a_{\mu} P\hmu}$,
\[
   U(\mtu,a)\ket{p,\sigma}=e^{ia_{\mu} p^{\mu}}\ket{p,\sigma}.
\]
Now we are interested in the determination of $U(\Lambda,a)\ket{p,\sigma}$.

\begin{proposition}
The operators $P^{\mu}$ are subject to the ``transformation law''
\begin{equation}
   U(\Lambda,a)P^{\mu} U(\Lambda,a)^{-1}=\Lambda^{\mu}_{\nu} P^{\nu}.
\end{equation}
\end{proposition}

\begin{proof}
Since operators $U(\Lambda,a)$ are linear, they can be putted in the derivative which defines $P\hmu$. Using the composition law  \eqref{eq:composition_U} we find:
\begin{equation}
\begin{split}
   U(\Lambda,a)P\hmu U(\Lambda,a)^{-1}&=\Dsdd{ U(\Lambda,a)U(\mtu,te\hmu)U(\Lambda,a)^{-1} }{t}{0}\\
                                     &=\Dsdd{U(\mtu,t\Lambda e\hmu)}{t}{0}.
\end{split}
\end{equation}
The $\Lambda$ can be putted out of derivative; let us see it for a sum of two terms (here it is four):
\begin{equation}
\begin{split}
  \Dsdd{ U(\mtu,t(e\hmu+e\hnu)) }{t}{0}&=\Dsdd{U(\mtu,te\hmu)U(\mtu,te\hnu)}{t}{0}\\
                                      &=\Dsdd{U(\mtu,te\hmu)U(\mtu,0)}{t}{0}
				        +\Dsdd{U(\mtu,0)U(\mtu,te\hnu)}{t}{0}\\
				      &=P\hmu+P\hnu.
\end{split}
\end{equation}
Thus
\begin{equation}
    \Dsdd{U(\mtu,\Lambda\hmu_{\nu} e\hnu)}{t}{0}=\Lambda\hmu_{\nu}\Dsdd{U(\mtu,te\hnu)}{t}{0}
                                              =\Lambda\hmu_{\nu} P\hnu.
\end{equation}
\end{proof}

\subsection{Pure Lorentz transformation}
%--------------------------------------

Now we consider a pure Lorentz transformation $U(\Lambda)\equiv U(\Lambda,0)$, and we want to look at $U(\Lambda)\ket{p,\sigma}$. In order to see its decomposition into others $\ket{k,\sigma'}$, we apply a $P\hmu$:
\begin{equation}
\begin{split}
  P\hmu U(\Lambda)\ket{p,\sigma}&=U(\Lambda)\Big( U(\Lambda)^{-1} P\hmu U(\Lambda)  \Big)\ket{p,\sigma}\\
                                &=U(\Lambda) {(\Lambda^{-1})}\hmu_{\nu} P\hnu \ket{p,\sigma}\\
				&=(\Lambda^{-1})\hmu_{\nu} p\hnu U(\Lambda)\ket{p,\sigma}.
\end{split}
\end{equation}
Thus the vector $U(\Lambda)\ket{p,\sigma}\in\pH$ has $(\Lambda^{-1})\hmu_{\nu} p\hnu$ as eigenvalue for $P\hmu$. If the $p\hmu$'s are seen as components of a $4$-vector $p$, one can write
\[
    P\hmu U(\Lambda)\ket{p,\sigma}=(\Lambda p)\hmu U(\Lambda)\ket{p,\sigma};
\]
     thus we naturally write
\begin{equation}
  U(\Lambda)\ket{p,\sigma}=\sum_{\sigma'}C_{\sigma'\sigma}(\Lambda,p)\ket{\Lambda p,\sigma'}.
\end{equation}


Note that we had not yet given anything about the nature of the $p$ in the ket $\ket{p,\sigma}$ so we can \emph{define} the product $\Lambda p$ by the fact that the ket $\ket{\Lambda p,\sigma}$ has eigenvalue $(\Lambda^{-1})\hmu_{\nu} p\hnu$ for the operator $P\hmu$. So it is one of the $\ket{p',\sigma'}$.

\subsection{Rebuilding of a basis for \texorpdfstring{$\pH$}{H}}
%----------------------------------------

From general considerations about the Lorentz group (many physicists had written very better books than me about) anyone knows that the only functions of the $p\hmu$'s  which are invariant under all the Lorentz transformations are $p^2=\eta_{\mu}nu p\hmu p\hnu$ and the sign of $p^0$ when $p^2< 0$.

For any value of $p^2$ and sign of $p^0$, one consider a ``standard vector''\ $k$. For example:
\begin{subequations}
\begin{align}
   k&=(1,0,0,1)&&\text{for }p^2=0,\\
   k&=(1,0,0,0)&&\text{for }p^2<0,p^0>0\\
   k&=(-1,0,0,0)&&\text{for }p^2<0,p^0<0.
\end{align}
\end{subequations}
With this convention, $p$ can be written as $p=L(p)k$ for a suitable Lorentz transformation $L(p)$. The vector $U(L(p))\ket{k,\sigma}$ has eigenvalue $L(p)k$ for the operator $P$, thus it is a linear combination of some $\ket{p,\sigma'}$.

Now we will cheat and redefine our basis of the Hilbert space $\pH$. First, we consider a fixed $k$; in other words, we build the state space for a given particle which has given momentum $p$.  The basis vectors must be eigenvectors for the fours operators $P\hmu$. As far as we say no more, any eigenvalue is possible. Thus our basis must be labelled by at least an element $p$ of $\eR^4$ with only one constraint: the value of $p^2$ (plus eventually the sign of $p^0$). So we define the $\ket{k,\sigma}$ to be such that
\[
   P\hmu\ket{k,\sigma}=k\hmu\ket{k,\sigma}.
\]
Since we know that with this definition of $\ket{k,\sigma}$, the eigenvalue of $U(L(p))\ket{k,\sigma}$ for $P\hmu$ is $p\hmu$, we \emph{define} $\ket{p,\sigma}$ as
\begin{equation}
  \ket{p,\sigma}=N(p)U(L(p))\ket{k,\sigma}.
\end{equation}
where $N(p)$ is a normalization to be discussed later. With this construction, we have an eigenvector for any possible eigenvalue for $P\hmu$. We have to show that these vectors are linearly independent.

The set of the $\ket{p,\sigma}$ with different $p$ is free in $\pH$ because they are eigenvectors for different eigenvalue of an hermitian operator\quext{I did not checked that it is sufficient}. There are no reason to think that the set of operators $P\hmu$ is complete; in other words, it remains not clear that there exist only one way to diagonalise the all the $P\hmu$. The function of the extra label $\sigma$ is to label different linearly independent vectors with same eigenvalue for $P$.

From now, we are interested in $\ket{k,\sigma}$ and $N(p)$.

\subsection{Little group}
%-----------------------

We have:
\begin{equation}
\begin{split}
  U(\Lambda)\ket{p,\sigma}&=N(p)U(\Lambda L(p))\ket{k,\sigma}\\
                          &=N(p)U(L(\Lambda p))U\big(  L(\Lambda p)^{-1}\Lambda L(p) \big)\ket{k,\sigma},
\end{split}
\end{equation}
So we will try to understand the operation $L(\Lambda p)^{-1}\Lambda L(p)$. First remark that
\[
   U(L(\Lambda p)^{-1})\ket{\Lambda p,\sigma}=N(\Lambda p)\ket{k,\sigma},
\]
and then compute:
\begin{equation}
\begin{split}
  U(L(\Lambda p)^{-1}\Lambda L(p) )N(p)\ket{k,0}&=U(L(\Lambda p)^{-1}\Lambda)\ket{p,\sigma}\\
                                               &=U(L(\Lambda p)^{-1})\sum_{\sigma'}
					         C_{\sigma'\sigma}(\Lambda,p)\ket{\Lambda p,\sigma'}\\
					       &=\sum_{\sigma'}C_{\sigma'\sigma}(\Lambda,p)
					         N(\Lambda p)\ket{k,\sigma'}.
\end{split}
\end{equation}

The \defe{little group}{little group} is the subgroup of the Lorentz transformations which leaves the chosen standard vector $k$ invariant: $Wk=k$. For any $W$ in the little group,
\[
   U(W)\ket{k,\sigma}=\sum_{\sigma'}D_{\sigma'\sigma}(W)\ket{k,\sigma'}
\]
With this definition, the $D$'s form a representation of the little group. Indeed for any $V,W$ in the little group,
\begin{equation}
\begin{split}
  \sum_{\sigma'}D_{\sigma'\sigma}(VW)\ket{k,\sigma'}&=U(VW)\ket{k,\sigma}\\
                                    &=U(V)\sum_{\sigma''}D_{\sigma''\sigma}(W)\ket{k,\sigma''}\\
				    &=\sum_{\sigma'\sigma''}D_{\sigma'\sigma''}(V)D_{\sigma''\sigma}(W)
				      \ket{k,\sigma'}.
\end{split}
\end{equation}
Since we want the $\ket{p,\sigma}$ with different $p$ and $\sigma$ to form a basis of $\pH$, they are linearly independent, then we can get rid of the sum over the $\sigma'$ and keep the equation
\[
D_{\sigma'\sigma}(VW)=\sum_{\sigma''}D_{\sigma'\sigma''}(VW)D_{\sigma''\sigma}(VW);
\]
if we adopt a more ``matricial''\ notation,
\begin{equation}
D(VW)=D(V)D(W).
\end{equation}

We are now able to perform a step in the study of the vector $U(\Lambda)\ket{p,\sigma}$. We naturally define $W(\Lambda,p)=L(\Lambda p)^{-1}\Lambda L(p)$. This belongs to the little group\footnote{Pay attention that $L(p)$ depends implicitly on the choice of $k$.}. Then,
\begin{equation}
\begin{split}
  U(\Lambda)\ket{p,\sigma}&=N(p)U(L(\Lambda p))U(W(\Lambda,p))\ket{k,\sigma}\\
                          &=N(p)\sum_{\sigma'}D_{\sigma'\sigma}(W)U(L(\Lambda p))\ket{k,\sigma'}\\
			  &=\frac{N(p)}{N(\Lambda p)}\sum_{\sigma'}D_{\sigma'\sigma}(W(\Lambda,p))
			      \ket{\Lambda p,\sigma'}.
\end{split}
\end{equation}

But we have no constraint on the $D$'s: it must just form a representation of the little group. Consequently, we are at a point in which our axioms are no more sufficient to continue the building of quantum field theory: we will get as many theories as representations of the little group.

The physical interpretation is the following\label{pg:phyz_reprez}: each type of particle has its own representation. When we consider a Hilbert space on which $U(\Lambda)$ acts via one given representation of the little group, we consider the Hilbert space which describes the corresponding particle. Note that the little group depends on the choice of $k$, and therefore depends on the particle which is studied (massive or not).

In this sense, a particle is a representation of the Poincaré group\quext{I think that the irreducibility of a representation is related to \emph{elementary} particles.}. In particular, the nature of the index $\sigma$ can change from the one representation to the other.

\begin{remark}
As far as normalization is concerned, we will pose
\[
  N(p)=\sqrt{k^0/p^0}.
\]
There are some good reasons to take it; but it is irrelevant from our group point of view of the theory.
\end{remark}

\subsection{Positive mass}
%------------------------

This is the easy case. The choice of standard momentum is $k=\begin{pmatrix}1&0&0&0\end{pmatrix}$. One could believe that the little group is $\SO(3)$. It would be the case if we had chosen $L_+^{\uparrow}$ instead of $\SLdc$ --see point~\ref{subsec:sym_nature}. In our hermitian representation of $\eR^4$, $k=\mtu$. Then a matrix of $\SLdc$ which leaves it invariant fulfills
\[
   \lambda k\lambda^{\dag}=\lambda\lambda^{\dag}=\mtu,
\]
this is $\lambda\in SU(2)$. By the way, note that $\SO(3)=SU(2)/\eZ_2$.

The celebrated ``law of transformation''\ of a massive particle of spin $j$ (integer or half integer) under the Lorentz transformation $\Lambda$ is
\begin{equation}
  U(\Lambda)\ket{p,\sigma}=\sqrt{\frac{ (\Lambda p)^0 }{p^0}}
       \sum_{\sigma'}D^{(j)}_{\sigma'\sigma}(W(\Lambda,p))\ket{\Lambda p,\sigma'}
\end{equation}
where $\sigma$ runs from $-j$ to $j$ by step of $1$.

\subsection{Null mass}
%---------------------

In the case of a null mass, the standard vector is $k=\qvect{1}{0}{0}{1}$ and an element of the little group fulfils $Wk=k$. As the little group is part of the Lorentz group, this is an isometry, so
\begin{subequations}
\begin{align}
  \scalh{Wt}{Wk}&=\scalh{t}{k}\\
  \scalh{Wt}{Wt}&=\scalh{t}{t},
\end{align}
\end{subequations}
for any $t\in\eR^4$. Taking in particular $t=\qvect{1}{0}{0}{0}$,
\begin{subequations}
\begin{align}
  (Wt)^{\mu}k_{\mu}&=t\hmu k_{\mu}=-1\\
  (Wt)^{\mu}(Wt)_{\mu}&=t\hmu t_{\mu}=-1.
\end{align}
\end{subequations}
If we write $Wt=({a},{b},{c},{d})$,  the first relation gives $d=a-1$, so that  $Wt=({1+\xi},{\alpha},{\beta},{\xi})$, while the second one gives $\xi=(\alpha^2+\beta^2)/2$. The conclusion is that $W$ acts on $t$ as a certain Lorentz transformation $S(\alpha,\beta)$:
\begin{equation}
Wt=\begin{pmatrix}
     1+\xi\\
     \alpha\\
     \beta\\
     \xi
   \end{pmatrix}=
   \begin{pmatrix}
     1+\xi  & -\xi    & \alpha & \beta\\
     \alpha & -\alpha &   1    &   0\\
     \beta  & -\beta  &   0    &   1\\
     \xi    & (1+\xi) & \alpha & \beta.
   \end{pmatrix}
   \begin{pmatrix}
     1\\
     0\\
     0\\
     0
   \end{pmatrix}.
\end{equation}
Be careful: it doesn't means that $W=S$, but $Wt=St$. However it is an information: $S(\alpha,\beta)^{-1} W$ is a Lorentz transformation which leaves $t$ invariant. Then it is a spatial rotation. More precisely, since $W$ and $S$ conserve $\qvect{1}{0}{0}{1}$, it is a rotation around the $z$ axis: $S(\alpha,\beta)^{-1} W=R(\theta)$, and
\begin{equation}
  W(\theta,\alpha,\beta)=S(\alpha,\beta)R(\theta)
\end{equation}
is the most general element of the non massive little group.


\chapter{Relativistic fields and fiber bundle formalism}
\input{117_Fibre_QFT}

\chapter{Conformal fields theory}
\input{142_CFT}

\chapter{Banach and \texorpdfstring{$C^*$}{C*}-algebras}
% This is part of (almost) Everything I know in mathematics
% Copyright (c) 2013-2014, 2020
%   Laurent Claessens
% See the file fdl-1.3.txt for copying conditions.

The main references for this chapter are \cite{Dixmier,Landsman}. In this chapter, all algebras are over $\eC$, or $\eR$ when it is mentioned. Definition and spectral properties of Banach algebras are given in chapter~\ref{Sec_SpecBanach}.

\section{Commutative Banach algebra}
%+++++++++++++++++++++++++++++++++++

We suppose the Banach algebra $\cA$ to be commutative.

\subsection{Structure space}
%----------------------------

\begin{definition}      \label{DefStructureSpaceDel}
    The \defe{structure space}{structure!space} $\Delta(\cA)$\nomenclature[C]{$\Delta(\cA)$}{structure space if the $C^*$-algebra $\cA$} of a commutative algebra is the set of the nonzero linear maps $\dpt{\omega}{\cA}{\eC}$ such that $\forall A$, $B\in\cA$,
\[
    \omega(AB)=\omega(A)\omega(B).
\]
We say that an element of this space is a \defe{character}{character!of an algebra}, or a \defe{multiplicative}{multiplicative} map of $\cA$.
\end{definition}

\begin{proposition}
Let $\cA$ be an unital commutative Banach algebra. Then for any $\omega\in\Delta(\cA)$,
\begin{enumerate}

\item $\omega(\mtu)=1$.
\item the character $\omega$ is bounded (and then continuous from~\ref{prop:cont_born}) with norm $\|\omega\|=1$ and for all $A\in\cA$,
\begin{equation} \label{eq:omAleqnA}
  \|\omega(A)\|\leq \|A\|.
\end{equation}
\end{enumerate}
\end{proposition}


\begin{proof}
The first claim is obvious because $\omega(A)=\omega(\mtu A)=\omega(\mtu)\omega(A)$.  For the second one, we know from lemma~\ref{lem:cv_Ak} that $(A-z)$ is invertible when $|z|>\|A\|$. By
linearity,
\[
\omega(A-z)=\omega(A)-z\neq 0
\]
because $\omega$ in a homomorphism. Now remark that $A-z$ is invertible implies $|\omega(A)|\neq |z|$. Indeed let us suppose the opposite, then there exists a $\alpha\in\eR$ such that $\omega(A)=e^{i\alpha}z$, but $|e^{i\alpha}z|=|z|$. Conclusion: if $|z|>\|A\|$, then $|\omega(A)|\neq|z|$. This immediately yields $|\omega(A)|\leq\|A\|$.

From there, it is clear that $\|\omega\|=1$ because the norm is the supremum of $|\omega(A)|$ with $\|A\|=1$. Since $\omega(\mtu)=1$, $\|\omega\|\geq 1$, but what we just showed implies $\|\omega\|\leq 1$.

\end{proof}

\begin{theorem}
Let $\cA$ be an unital commutative Banach algebra. Then we have a bijection between $\Delta(\cA)$ and the set of maximal ideals in $\cA$. More precisely,

\begin{enumerate}
\item $\ker(\omega)$ is an ideal,                   \label{enuei}
\item $\omega_1=\omega_2$ if and only if $\ker\omega_1=\ker\omega_2$,   \label{enueii}
\item each maximal ideal is the kernel of an element in $\Delta(\cA)$.  \label{enueiii}
\end{enumerate}\label{tho:ideal_kernel}
\end{theorem}


\begin{proof}
\ref{enuei} Since $\omega$ is continuous, the set $\ker(\omega)$ is closed. It is also clear that is $Z\in\ker(\omega)$, then $AZ\in\ker(\omega)$ for all $Z\in\cA$ because $\omega$ is multiplicative. Then $\ker(\omega)$ is an ideal. In order to see that it is a maximal ideal, remark that $\omega(X)=0$ is a linear equation which describe a vector subspace of $\cA$ of codimension\label{pg_codimun} $1$.

\ref{enueii} In any vector space, $\ker{\omega_1}=\ker{\omega_2}$ implies that $\omega_1$ and $\omega_2$ are multiples each others. In the case of $\Delta(\cA)$, this in turn implies the equality.

\ref{enueiii} Let $\cI\neq\cA$ be a maximal ideal and $B\neq 0$ outside $\cI$. Consider
\[
\cI_B=\{BA+J\tq A\in\cA\textrm{ and }J\in\cI\}.
\]
By construction it is a left-ideal and by commutativity of $\cA$, it is an ideal. We have $\cI\subsetneq\cI_B$. Since $\cI$ is maximal, the conclusion is $\cI_B=\cA$. In particular $\cun=BA+J$ for a suitable choice of $A\in\cA$ and $J\in\cI$. For these,
\[
  \tau(\cun)=\tau(BA)=\tau(B)\tau(A),
\]
but $B$ is arbitrary. Then any element of $\cA/\cI$ is invertible and the Gelfand-Mazur theorem (corollary~\ref{cor:GelfandMazur}) concludes $\cA/\cI\simeq\eC$. Let $\dpt{\psi}{\cA/\cI}{\eC}$ be the isomorphism. We consider
        \begin{equation}
        \begin{aligned}
            \omega \colon \cA &\to \eC\
            A&\mapsto \psi(\tau(A)).
        \end{aligned}
    \end{equation}
It is clearly linear (because $\psi$ and $\tau$ are) and $\omega(A)\omega(B)=\omega(AB)$. Furthermore $\omega(B)\neq 0$ and $\omega(\cun)=1$ are two good reasons to conclude that $\omega\neq 0$. Then $\omega\in\Delta(\cA)$. It remains to be proved that $\cI=\ker\omega$. First, $\cI=\ker\tau$, then $\cI\subseteq\ker\omega$. But when $B\notin\cI$, we have $\omega(B)\neq 0$, then $\cI=\ker\omega$. This finish the proof.
\end{proof}


\begin{theorem}[Banach-Alaoglu]
If $X$ is a closed normed vector space, then the unit closed ball in the dual $X^*$ is compact for the $x^*$-topology.
 \label{tho:Banach_Alaoglu}
\end{theorem}

\begin{proposition}
When $\cA$ is an unital commutative Banach algebra, the space $\Delta(\cA)$ is compact and Hausdorff for the Gelfand topology.
\end{proposition} \label{prop:DcA_comp_Hauss}

\begin{proof}
We first prove that $\Delta(\cA)$ is closed by showing that it contains all limits of converging sequences\footnote{It is no related to \emph{complete} spaces in which any Cauchy sequence converge}. Let us take a sequence $\omega_n\to\omega$ with $\omega_n\in\Delta(\cA)$. We will show that $\omega\in\Delta(\cA)$:
\[
| \omega(AB)-\omega(A)\omega(B) | \leq| \omega(AB)-\omega_n(AB) |+| \omega_n(A)\omega_n(B)-\omega(A)\omega(B) |,
\]
but
\[
 \begin{split}
\omega_n(A)\omega_n(B)-\omega(A)\omega(B)&=[\omega_n(A)-\omega(A)]\omega_n(B)+\omega(A)[\omega_n(B)-\omega(B)]\\
        &\leq | \omega_n(A)-\omega(A) |\| B \|+\| A \| \omega_n(B)-\omega(B) |
\end{split}
\]
because $\omega_n(B)\leq \| B \|$. Taking the limit $n\to\infty$, we find
\[
 \begin{split}
| \omega(AB)-\omega(A)\omega(B) |&\leq| \omega(AB)-\omega_n(AB) |\\
        &\quad+| \omega_n(A)-\omega(A) |\| B \|\\
        &\quad+\| A \| |\omega_n(B)-\omega(B) |\to 0.
\end{split}
\]
This proves that $\omega\in\Delta(\cA)$ and therefore that $\Delta(\cA)$ is closed. Since $\| \omega \|=1$ for all $\omega$, we have $\Delta(\cA)\subset\cA_1^*$, the unit ball in $\cA^*$. Theorem~\ref{tho:Banach_Alaoglu} claims that $\cA_1^*$ is compact in the Gelfand topology. So $\Delta(\cA)$ is closed in a compact. This makes $\Delta(\cA)$ compact by lemma~\ref{lem:ferme_compact}.

Now, we check that it is also Hausdorff. If $\omega\neq\eta\in\Delta(\cA)$, there exists a $A\in\cA$ such that $\omega(A)\neq\eta(A)$. We thus consider $\mO$ and $\mO'$, two disjoints open subsets of $\eC$ around $\omega(A)$ and $\eta(A)$ respectively. With these definition, it is easy to see that $\hat A^{-1}(\mO)$ and $\hat A^{-1}(\mO')$ are disjoints neighbourhoods of $\omega$ and~$\eta$.
\end{proof}


\subsection{Topology on \texorpdfstring{$\Delta(\cA)$}{DA}}\label{subsec:topo_Delta}
%----------------------------------------------
We begin to put the \defe{$w^*$-weak topology}{$w^*$-weak topology} on $\cA^*$ which defined by the convergence notion $\omega_n\to\omega$ if and only if $\omega_n(A)\to\omega(A)$ for all $A\in\cA$.

The \defe{Gelfand topology}{Gelfand!topology} is the induced topology from $\cA^*$ on $\Delta(\cA)$. Let us define the \defe{Gelfand transform}{Gelfand!transform}
        \begin{equation}
        \begin{aligned}
            \hat A \colon \Delta(\cA) &\to \eC\\
            \hat A(\omega)&\mapsto \omega(A).
        \end{aligned}
    \end{equation}
General theory of functional analysis shows that the $w^*$-weak topology is the weakest in which all linear functional are continuous, so a basis of this topology is given by sets of the form $\{f\in\cA^*\tq f(A)\in\mathcal{O}\}$ where $\mathcal{O}$ is an open in $\eC$ and $A\in\cA$.

A basis of the Gelfand topology is the intersection of these set with $\Delta(\cA)$:
\begin{equation}
  \hat A^{-1}(\mathcal{O})=\{\omega\in\Delta(\cA)\tq \omega(A)\in\mathcal{O}\}.
\end{equation}



\begin{lemma}
An element $A\in\cA$ is invertible if and only if $\omega(A)\neq0$ for all $\omega\in\Delta(\cA)$.
\end{lemma}


\begin{proof}
Let $A$ be an invertible element in $A$ and $\omega\in\Delta(\cA)$ such that $\omega(A)=0$. Then
\[
  1=\omega(\cun)=\omega(A)\omega(A^{-1})=0.
\]

Let us take now a $A\notin G(\cA)$, then the ideal $\cI_A:=\{AB\tq B\in\cA\}$ don't contain $\cun$ and is not a proper ideal. From choice axiom, $\cI_A$ is contained in a maximal ideal $\cI$. From~\ref{enueiii} of~\ref{tho:ideal_kernel}, there exists a $\omega\in\Delta(\cA)$ whose kernel is $\cI$. In particular, $\omega(A)=0$.
\end{proof}

\begin{theorem}
Let $\cA$ be an unital commutative Banach algebra. Then
\begin{enumerate}
\item The Gelfand transform is a homomorphism $\cA\to C(\Delta(\cA))$. \label{enugi}
\item The image of $\cA$ under the Gelfand transform separates the points in $\Delta(\cA)$, see definition~\ref{def:separe}. \label{enugii}
\item \label{enugiii} The spectrum of $A\in\cA$ is
\[
   \sigma(A)=\sigma(\hat A)=\{\hat A(\omega):\omega\in\Delta(\cA)\}.
\]
\item \label{enugiv} The Gelfand transform is a \defe{contraction}{contraction}:   $\|\hat A\|_{\infty}\leq\|A\|$.
\end{enumerate}\label{tho:unital_comm}
\end{theorem}

\begin{proof}
Item~\ref{enugi} is easy: $\widehat{AB}(\omega)=\omega(AB)=\omega(A)\omega(B)=\hat A(\omega)\hB(\omega)$. Point~\ref{enugii} is immediate too: let $\omega_1\neq\omega_2\in\Delta(\cA)$. We need a $A\in\cA$ such that $\hat A(\omega_1)\neq\hat A(\omega_2)$. But the definition of the inequality $\omega_1\neq\omega_2$ is the existence of a $A\in\cA$ such that $\omega_1(A)\neq\omega_2(A)$.

For~\ref{enugiii}, recall that
\[
  \rho(A)=\{z\in\eC\tq(A-z)^{-1}\textrm{ exists}\}.
\]
From the lemma the existence of $(A-z)^{-1}$ makes that $\forall\omega\in\Delta(\cA)$, $\omega(A)\neq z$. So the complementary is
\begin{equation}
\begin{split}
 \sigma(A)&=\{z\in\eC\tq\exists\omega\in\Delta(\cA)\textrm{ such that }\omega(A)=z\}\\
          &=\{\omega(A)\tq\omega\in\Delta(\cA)\}\\
          &=\{\hat A(\omega)\tq\omega\in\Delta(\cA)\}.
\end{split}
\end{equation}

The fifth point comes from definition~\ref{def:sup_norm} and the fact that, because of the third point,  $r(A)=sup\{\hat A(\omega):\omega\in\Delta(\cA)\}$.
Therefore
\[
\|\hat A\|_{\infty}=\sup_{\omega\in\Delta(\cA)}|\hat A(\omega)|=r(A)\leq\|A\|.
\]
\end{proof}

When a Banach algebra is non unital, one can extend it to $\cA_{\cun}$ and a character $\omega\in\Delta(\cA)$ can be extended too as $\tilde{\omega}\in\Delta(\cA_{\cun})$ by
\[
  \tilde{\omega}(A+\lambda\cun)=\omega(\cun)+\lambda.
\]
The fact that it is multiplicative is a simple computation.

\begin{theorem}
For every element $A$ of a commutative Banach algebra $\cA$, we have $\Spec(A)=\Spec(\hat A)$.
\end{theorem}

\begin{proof}
We want to prove that when $\lambda\in\Spec(A)$, there exists a $\varphi$ such that $\varphi(A)=\lambda$. The ideal generated by $(A-\lambda)$ is a proper ideal which is thus contained in a maximum ideal $M$ by Zorn's lemma. This maximal ideal is closed (if not, the closure would be bigger ideal). Consider an element $x$ in the quotient $\cA/M$. Since $\Spec(x)\neq\emptyset$, the element $(x-\lambda)$ is not invertible for some $\lambda$. That provides an isomorphism $\cA/M\simeq \eC$, and we define $\varphi$ as the composition of that isomorphism by the projection of $\cA$ into $\cA/M$. For this $\varphi$, we have $\varphi(A)=\lambda$.

\begin{probleme}
Faudrait creuser pourquoi on a un isomorphisme $\cA/M\simeq\eC$.
\end{probleme}
\end{proof}

\subsection{An example}
%----------------------

Let $\cA=L^1(\eR)$ with the norm
\[
  \|f\|_1=\int_{\eR}|f(x)|dx,
\]
and the convolution product
\[
   (f\star g)(x)=\int_{\eR}f(x-y)g(y)dy.
\]
We don't take care to analysis subtleties as completion and precise convergence of integrals. For example, we will use and abuse of Fubini's theorem and often say ``for all'' when ``for almost all'' should be preferable. From the fact that $|f(x-y)g(y)|\leq|f(x-y)||g(y)|$ and $\int_{\eR}f(x-y)dx=\int_{\eR}f(x)dx$, we find that
\[
   \|f\star g\|_1\leq \|f\|_1\|g\|_1
\]
as needed to prove that $(L^1(\eR),\star)$ is a Banach algebra. This is a non unital Banach space because the unit should be the Dirac delta. From analysis, one knows that the dual space of $L^1(\eR)$ is $L^{\infty}(\eR)$ with, for $u\in L^{\infty}(\eR)$,
\[
  u(f)=\int_{\eR}f(x)u(x)dx.
\]
Since $\Delta(\cA)$ is a subset of $L^{\infty}(\eR)$, there exists, for each $\omega\in\Delta(L^1(\eR))$, a $\hat{\omega}$ such that $\omega(f)=\int_{\eR}f(x)\hat{\omega}(x)dx$. With an easy change of variable, the multiplicative condition $\omega(f\star g)=\omega(f)\omega(g)$ gives
\[
\int_{\eR^2}f(t)g(y)\hat{\omega}(t+y)dtdy=\int_{\eR^2}f(x)g(y)\hat{\omega}(x)\hat{\omega}(y)dxdy.
\]
We can conclude that $\hat{\omega}(x+y)=\hat{\omega}(x)\hat{\omega}(y)$, in such a manner that
\[
\hat{\omega}(x)=e^{ipx}
\]
for a certain $p\in\eC$. For $\hat{\omega}$ to belongs to $L^{\infty}(\eR)$, we must have $p\in\eR$. So we get a bijection $\Delta(\cA)\simeq\eR$. By this identification, we denote by $p$ the element of $\Delta(L^1(\eR))$ given by $\hat{\omega}(x)=e^{ipx}$. With theses notations, the Gelfand $\hat A(\omega)=\omega(A)$ transform reads
\begin{equation}
  \hat f(p)=\omega(f)=\int_{\eR}f(x)\hat{\omega}(x)
                     =\int_{\eR}f(x)e^{ipx}dx.
\end{equation}
This is nothing else than the Fourier transform! We know that Fourier transform changes the convolution product into the pointwise usual product of functions:
\[
\widehat{f\star g}(p)=\hat f(p)\hat g(p)=(\hat f\hat g)(p).
\]
This express the fact that the Gelfand transform is a homomorphism between $\cA$ ---i.e. the product $\star$--- and $C(\Delta(\cA))$ ---i.e. the pointwise product. It is precisely the claim~\ref{enugi} of theorem ~\ref{tho:unital_comm}.


\begin{theorem}
Let $\cA$ be a non unital commutative  Banach algebra. Then

\begin{enumerate}
\item \label{enuhi} The space $\Delta(\cA)$ is Hausdorff locally compact for the Gelfand topology,
\item \label{enuhii} $\Delta(\cA_{\cun})$ is the one point compactification of $\Delta(\cA)$,
\item \label{enuhiii} the Gelfand transformation is a homomorphism $\cA\to C_0(\Delta(\cA))$,
\item \label{enuhiv} the spectrum of $A\in\cA$ is
\[
  \Spec(A)=\sigma(A)=\{0\}\cup\{\hat A(\omega)\tq\omega\in\Delta(\cA)\}.
\]
\item \label{enuhv} The image of $\cA$ by the Gelfand transform separates points in $\Delta(\cA)$,
\item \label{enuhvi} Gelfand transform is a contraction:
\[
\|\hat A\|_{\infty}\leq\|A\|.
\]

\end{enumerate}

\end{theorem}

For one point compactification issues, see section~\ref{sec:compactific}.

\begin{proof}
\ref{enuhi} We add an unity to $\cA$ and we remark that
\[
\Delta(\cA_{\cun})=\Delta(\cA)\cup\infty
\]
where $\infty$ is defined by $\infty(A+\lambda\cun)=\lambda$. Indeed let $\psi\in\Delta(\cA)$ and let us ask ourself how to extend it to a multiplicative functional in $\varphi\in\Delta(\cA_{\cun})$. For, let $B\in\cA$ such that $\psi(A)\neq 0$ remark that multiplicative condition imposes $\varphi( (\lambda\cun)(B) )=\varphi(\lambda)\varphi(B)$ while the linearity gives $\varphi(\lambda B)=\lambda\varphi(B)$. Thus $\varphi(\lambda\cun)=\lambda$ and the unique possibility to extends $\psi$ is
\[
\varphi(A+\lambda\cun)=\varphi(A)+\lambda
\]
and we note $\infty$ the new functional
\[
\infty(A+\lambda\cun)=\lambda.
\]

Since $\cA_{\cun}$ is unital, the character space $\Delta(\cA_{\cun})$  is Hausdorff and compact for its Gelfand topology. As set
\[
  \Delta(\cA)=\Delta(\cA_{\cun})\setminus\{\infty\}.
\]
We should prove that the induced topology on $\Delta(\cA)$ from the Gelfand of $\Delta(\cA_{\cun})$ is precisely the own Gelfand topology of $\Delta(\cA)$. In this case, properties of compactification shall gives local compactness.

A basis of the topology of $\Delta(\cA_{\cun})$ is given by $\hat A^{-1}=\{\omega\in\Delta(\cA_{\cun})\tq\omega(A)\in\mO\}$. Then any open set of $\Delta(\cA)$ is open for the induced topology because
\[
\{\omega\in\Delta(\cA)\tq\omega(A)\in\mO\}=\{\eta\in\Delta(\cA_{\cun})\tq\eta(A)\in\mO\}\cap\Delta(\cA).
\]
For the converse, an open set for the induced topology is given by
\[
\begin{split}
&\{\omega\in\Delta(\cA_{\cun})\tq\exists A\in\cA,\lambda\in\eC:\omega(A+\lambda\cun)\in\mO\}\setminus\{\infty\}\\
&=\{\omega\in\Delta(\cA)\tq\exists A\in\cA,\lambda\in\eC:\omega(A)\in\mO-\lambda\}
\end{split}
\]
where $\mO-\lambda$ is as open as $\mO$. This proves~\ref{enuhi} and~\ref{enuhii}.

For~\ref{enuhiii}, the point is not to prove that Gelfand transform is a homomorphism (that is trivial), but rather that it takes values in $C_0(\Delta(\cA))$.

The complementary of a compact set $K$ in $\Delta(\cA_{\cun})$ is an open set which contains $\infty$. Since $\hat A(\infty)=0$, the values of $\hat A$ in the complementary of $K$ are as small as we want when $K$ becomes larger and larger.

In order to prove~\ref{enuhiv}, recall that, by definition, $\sigma_{\cA}(A)=\sigma_{\cA_{\cun}}(A)$. Then
\begin{subequations}
    \begin{align}
  \sigma_{\cA_{\cun}}&=\{\hat A(\omega)\tq\omega\in\Delta(\cA_{\cun})\}\\
                     &=\{\hat A(\omega)\tq \omega\in\Delta(\cA)\}\cup\hat A(\infty)\\
                     &=\{\hat A(\omega)\tq \omega\in\Delta(\cA)\}\cup\{0\}.
    \end{align}
\end{subequations}
Since~\ref{enuhv} and~\ref{enuhvi} are true for $\cA_{\cun}$, they are true for $\cA$.

\end{proof}

\section{Commutative \texorpdfstring{$C^*$}{C}-algebras}
%+++++++++++++++++++++++++++++++++++

\begin{definition}
    A \defe{$C^*$-algebra}{c-star@$C^*$-algebra} is an involutive (complex) Banach algebra such that for all $A$, $B\in\cA$,
    \begin{enumerate}
        \item $\|AB\|\leq\|A\|\|B\|$,
        \item $\|A^*A\|=\|A\|^2$.
    \end{enumerate}
    One immediately has $\|A\|^2=\|A^*A\|\leq\|A^*\|\|A\|$, then
    \begin{equation}
        \|A\|=\|A^*\|
    \end{equation}
    for all element $A$ in a $C^*$-algebra.
\end{definition}

\begin{lemma}       \label{LemFiniCSestVNa}
    Every finite dimensional $C^*$-algebra is a von~Neumann algebra.
\end{lemma}

\begin{lemma}[Stone-Weierstrass theorem]
Let $X$ be a compact and Hausdorff space. Any $C^*$-subalgebra of $C(X)$ containing $1_X$ and separating points in $X$ is exactly $C(X)$ seen as $C^*$-algebra.
\end{lemma}\label{lem:Stone_W}
Here, $1_X$ denotes the constant function $1$ on $X$.

\begin{proposition}     \label{PropcomCstarDelCeqX}
Let $X$ be a compact Hausdorff space and see $C(X)$ as a commutative $C^*$-algebra. Then $\Delta(C(X))$ is homeomorphic to $X$.\label{prop:comHauffhomeo}
\end{proposition}

\begin{proof}
For $x\in X$, one defines
        \begin{equation}
        \begin{aligned}
            \omega_x \colon C(X) &\to \eC\\
            f&\mapsto f(x).
        \end{aligned}
    \end{equation}
It is clearly non zero and multiplicative. Then $\omega_x\in\Delta(C(X))$. We denote by $ \dpt{E}{X}{\Delta(C(X))}$ the map which makes the correspondence between $x$ and $\omega_x$
\[
  E(x)f=f(x).
\]
Urysohn lemma~\ref{lem:Urysohn} applied to the compact Hausdorff space $X=\Delta(C(X))$ makes that if $x\neq y$, then there exists a function $f\in C(X)$ such that $f(x)\neq f(y)$. This proves that $E$ is injective.

From theorem~\ref{tho:ideal_kernel}, we know that
\[
\cI_x=\ker\omega_x=\{f\in C(X)\tq f(x)=0\}
\]
is an ideal in $C(X)$. Suppose that $E$ is not surjective. Then there exists some $\omega\in\Delta(C(X)))$ which don't come from a $x\in X$; for such a $\omega$, we pose
\[
\cI_{\omega}=\ker\omega=\{f\in C(X)\tq \omega(f)=0\}.
\]
 This $\cI_{\omega}$ can't contains any $\cI_x$ because they are maximal ideals. Then for all $x\in X$ , there exists a $f\in C(X)$ such that $f(x)=0$ with $f\notin\cI_{\omega}$. If $E$ is not surjective, then there exists a maximum ideal $\cI$, kernel of a character which is not in the image of $E$. In order this ideal to be included in none of the $\cI_x$, one needs that for all $x\in X$, there exists $f_x\in\cI$ such that $f_x(x)\neq 0$. Let $\mO_x$ be an open set on which $f_x\neq 0$. Since $X$ is compact, one can extract a finite subcovering $X=\bigcup_i\mO_{x_i}$. Now we build
\[
 g:=\sum_{i=1}^n|f_{x_i}|^2.
\]
This is a strictly positive function, then $1/g\in C(X)$, and then $\cI=C(X)$ and $\cI$ should be the kernel of a zero character. This is impossible, then $E$ is surjective and it is a bijection.

In order to prove that $E$ is an homeomorphism, we will use the lemmas~\ref{lem:Hausweak} and~\ref{lem:wiki}. Let $X_0$ be the space $X$ endowed with its initial topology and $X_G$ the same space with the topology induced from $E^{-1}$, i.e. that an open set in $X_G$ is always the image by $E^{-1}$ of an open set in $\Delta(C(X))$. From definition, $E$ is continuous for the topology $X_G$. We are going to prove that $X_0=X_G$. Definitions give for all $f\in C(X)$,
\[
 (\hat f\circ E)(x)=\hat f(\omega_x)=\omega_x(f)=f(x).
\]
But Gelfand topology is the weakest topology for which all $f$ are continuous. On the other hand, $f$ is continuous because it belongs to $C(X_0)$. Then the topology of $X_G$ is weaker than the one of $X_0$. Indeed, let $\mO$ be an open set for $X_G$ and let us prove that it contains an open set of $X_0$. From definition, $\mO=E^{-1}(\mO')$ for a certain open set $\mO'$ of $\Delta(C(X))$, i.e. $\mO'=\hat f^{-1}(A)$ for an open $A$ in $\eC$. The topology $X_G$ is the minimal one for which $E^{-1}\circ\hat f^{-1}(A)$ is open. But $E^{-1}\circ\hat j^{-1}=f^{-1}$, then $(E^{-1}\circ\hat f^{-1})(A)=f^{-1}(A)$ is open in $X_0$.
\end{proof}


\begin{theorem}[Gelfand theorem]    \index{Gelfand!theorem}\index{theorem!Gelfand}

For any commutative unital $C^*$-algebra $\cA$, there exists an unique (up to isomorphism) compact and Hausdorff space $X$ such that  $\cA$ is isomorphic to $C(X)$.
\label{thoGelfand}
\end{theorem}

\begin{proof}
We immediately give the answer: $X=\Delta(\cA)$ and the isomorphism is
        \begin{equation}
        \begin{aligned}
            \varphi \colon \cA &\to C(\Delta(\cA))\\
            A&\mapsto \hat A.
        \end{aligned}
    \end{equation}
We first have to prove that $\Delta(\cA)$ is compact and Hausdorff. Then it should be proved that $\varphi$ is an isometric $C^*$-algebra isomorphism and finally that this is the only possibility.

\subdem{The space $\Delta(\cA)$ is compact and Hausdorff}

Proposition~\ref{prop:DcA_comp_Hauss} gives it.

\subdem{The map $\varphi$ takes values in $C(\Delta(\cA))$}

From discussion at top of subsection~\ref{subsec:topo_Delta}, the functional $\hat A$ is continuous on $X=\Delta(\cA)$.

\subdem{The map $\varphi$ is a morphism}

Linearity of $\varphi$ is clear. Property $\varphi(AB)=\varphi(A)\varphi(B)$ comes from point~\ref{enugi} of proposition~\ref{prop:DcA_comp_Hauss}. So we are left to prove that $\varphi(A^*)=\varphi(A)^*$. It is sufficient to prove that, if $A=A^*$, then $\varphi(A)$ takes his values in $\eR$. So let $A\in\cA_{\eR}$ and write $\omega(A)=\alpha+i\beta$ with $\alpha,\beta\in\eR$. If we define $B=A-\alpha\cun$, then $\omega(B)=i\beta$ because $\omega(\cun)=1$. Furthermore $B=B^*$. Let $t\in\eR$; we have
\begin{equation}  \label{eq:rcinq}
|\omega(B+it\cun)|^2=|\omega(B)+it|^2
                    =\beta^2+2\beta t+t^2.
\end{equation}
Using formulas $|\omega(A)|\leq\|A\|$ and $\|AA^*\|=\|A\|^2$, we find
\begin{equation}
  |\omega(B+it\cun)|^2\leq\|B+it\cun\|^2
                      =\|B^2+t^2\|
                      \leq \|B\|^2+t^2.
\end{equation}
Then net result is that for all $t\in\eR$,   $\beta^2+2t\beta\leq \|B\|^2$. It is only possible when $\beta=0$. Then $\omega(A)\in\eR$ as soon as $A=A^*$.

\subdem{The map $\varphi$ is isometric}

Let us begin with $A=A^*$. So $\|A^2\|=\|A\|^2$ and $\|A^{2^m}\|=\|A\|^{2^m}$. Using proposition~\ref{prop:An_usn}, we find
\[
  r(A)=\|A\|.
\]
On the other hand the definition of the supremum norm on the Hausdorff space $\Delta(\cA)$ reads
\begin{equation} \label{eq:AinfA }
  \|\hat A\|_{\infty}=\sup_{\omega\in\Delta(\cA)}|\hat A(\omega)|=r(A)=\|A\|.
\end{equation}
Then $\varphi$ is isometric when $A=A^*$. Now, $A^*A$ is selfadjoint and $\|A^*A\|=\|A\|^2$, then
\[
\|A\|^2=\|A^*A\|=\|\widehat{A^*A}\|_{\infty}=\|{\hat A}^*\hat A\|_{\infty}=\|\hat A\|^2_{\infty}.
\]

\subdem{The map $\varphi$ is injective}

If $\varphi(A)=\varphi(B)$, then $\varphi(A-B)=0$. The only way for $\varphi$ to be an isometry is $A-B=0$.

\subdem{The map $\varphi$ is surjective}

Since $\varphi$ is an isometry, it sends a closed set into a closed set, but $\cA$ is closed because it is a Banach space. Point~\ref{enugii} of theorem~\ref{tho:unital_comm} says that $\varphi(\cA)$ separates points in $\Delta(\cA)$ and we just proved the $\varphi$ preserves the adjoint, so $\varphi(\cA)$ is a $C^*$-subalgebra of $C(\Delta(\cA))$. Finally, it is clear that $\hat{\cun}=1_X$. Lemma~\ref{lem:Stone_W} concludes $\varphi(\cA)=C(\Delta(\cA))$.

Now proposition~\ref{prop:comHauffhomeo} makes $\varphi$ and homeomorphism between $\cA$ and $\Delta(\cA)$. So the topological structure of $\cA$ is encoded in the algebraic (Banach) structure of $C(\Delta(\cA))$. So if $C(Y)\simeq\cA\simeq C(X)$ as $C^*$-algebras, then $X\simeq Y$ as topological space. This proves the unicity part and concludes the Gelfand theorem.

\end{proof}

As far as notations are concerned, let us recall that the Gelfand transform is $A\mapsto\hat A$ with
\begin{equation}
\begin{aligned}
 \hat A\colon \Delta(\cA)&\to \eC \\
   \omega&\mapsto \omega(A).
\end{aligned}
\end{equation}
One particular class of elements in $\Delta\big( C(X) \big)$ is the ones of the form $\omega_x$ for $x\in X$. These are defined by
\begin{equation}
\begin{aligned}
 \omega_x\colon C(X)&\to \eC \\
   f&\mapsto f(x).
\end{aligned}
\end{equation}
The Gelfand theorem says that every element of $\Delta \big(C(X))$ reads $\omega_x$ for a certain $x\in X$.

\begin{lemma}
Let $\cA$ be a $C^*$-algebra, and $\oB(\cA)$, the set of bounded operators on $\cA$. Then

\begin{enumerate}
\item The map
        \begin{equation}
        \begin{aligned}
            \rho \colon \cA &\to \oB(\cA)\\
            \rho(A)B&\mapsto AB
        \end{aligned}
    \end{equation}
is a diffeomorphism between $\rho(\cA)$ and $\rho(\cA)\subset\oB(\cA)$.

\item If $\cA$ has no unit, one can define a norm on $\cA_{\cun}$ by
\begin{equation} \label{eq:normCAu}
\|A+\lambda\cun\|=\|\rho(A)+\lambda\cun\|
\end{equation}
where the right hand side norm is the one in $\oB(\cA)$, see~\ref{def:normappl}. With the usual multiplication and the involution
\begin{equation}
  (A+\lambda\cun)^*=A^*+ \overline{\lambda} \cun,
\end{equation}
the set $\cA_{\cun}$ becomes an unital $C^*$-algebra.

\end{enumerate}
 \label{lem:unitariz_C}
\end{lemma}

\begin{proof}
Since $\cA$ is a $C^*$-algebra, $\|\rho(A)B\|\leq\| A \|\|B\|$, then for all $A\in \cA$, one has $\| \rho(A) \|\leq \| A \|$. On the other hand, we know that $\| A^*A \|=\| A \|^2$ and $\| A^* \|=\| A \|$, then
\[
\| A \|=\frac{\| AA^* \|}{\| A \|}=\left\|  \rho(A)\frac{A^*}{\| A \|}  \right\|\leq\| \rho(A) \|
\]
from definition of the sup norm. Then $\| \rho(A) \|=\| A \|$ and $\rho$ is an isometry and then is injective because it is linear. It is clearly a homomorphism too. The map $A+\lambda\cun\to\rho(A)+\lambda\cun$ is a $C^*$-algebra-morphism if we define\footnote{We know a definition of $*$ when we look at $\oB(H)$ where $H$ is a Hilbert space, but we are here with $\oB(\cA)$ where $\cA$ is no more than a Banach space; hence we do not have a definition of $*$.} $\rho(A)^*=\rho(A^*)$.  Since the sup norm fulfils condition \eqref{eq:normBanach}, the norm \eqref{eq:normCAu} fulfils the same. So $\cA_{\cun}$ becomes a Banach $*$-algebra and lemma~\ref{lem:STARAlC} will help us to conclude that it is a $C^*$-algebra.

The formula $\| A \|^2-\varepsilon\leq\| Av \|^2$ holds for an operator $A$ on a general Banach algebra and an arbitrary vector $v$ with norm $1$. In our present case, if $\| B \|=1$,
\begin{equation}
\begin{split}
  \| \rho(A)+\lambda\cun \|^2-\varepsilon
        &\leq \| (\rho(A)+\lambda\cun)B \|^2\\
        &    =\| AB+\lambda B \|^2\\
        &    =\| (AB+\lambda B)^*(AB+\lambda B) \|\\
        &    =\| \rho(B^*)\rho(A^*+\overline{\lambda}\cun)\rho(A+\lambda\cun)B \|\\
        &\leq \| \rho(B^*) \|\| (\rho(A)+\lambda\cun)^*(\rho(A)+\lambda\cun) \|\| B \|,
\end{split}
\end{equation}
but we also know that $\| \rho(B^*) \|=\| B^* \|=\| B \|=1$. Letting $\varepsilon\to 0$, we find $\| A \|^2\leq\| A^*A \|$ in the Banach $*$-algebra $\cA_{\cun}$.

\end{proof}

\label{pg:unit_nonunic} This lemma gives us an unitization of a $C^*$-algebra which is not the one previously given for a Banach algebra. This shows that unitization of Banach algebra is not unique. For a $C^*$-algebra, however, we have an unicity result:

\begin{proposition}
For every $C^*$-algebra without unit, there exists an unique unital $C^*$-algebra $\cA_{\cun}$ and an isometric morphism (hence injective) $\cA\to\cA_{\cun}$ such that $\cA/\cA_{\cun}=\eC$.
\label{prop_unitariz_csa}
\end{proposition}

\begin{proposition}
If $\cA$ is a commutative $C^*$-algebra, any character is hermitian.
\end{proposition}

\begin{proof}
When $\chi$ is a character, $\chi(A)\in\sigma(A)$ for all $A\in\cA$ and when $A=A^*$, we have $\sigma(A)\subset\eR$. For any $A$, we have a decomposition $A=A_1+iA_2$ and
\[
  \chi(A^*)=\chi(A_1-iA_2)=\underbrace{\chi(A_1)}_{\in\eR}-i\underbrace{\chi(A_2)}_{\in\eR}=\overline{ \chi(A) }.
\]
\end{proof}

\section{Functional calculus in unital \texorpdfstring{$C^*$}{C}-algebras}
%+++++++++++++++++++++++++++++++++++++++++++++++++++++++++++++++++++++++++

From now, the $C^*$-algebra $\cA$ is no more assumed to be commutative, but it is unital.

\begin{definition}      \label{DefElemNormal}
    An element $A$ in an involutive algebra is said \defe{normal}{normal!element of an involutive algebra} when $[A,A^*]=0$.
\end{definition}
This is a direct generalisation of the concept of normal operator in the Hilbert space setting (definition~\ref{DefFQFKZbB}).

If $\cA$ is a $C^{*}$-algebra and $A$, $B\in\cA$ we denotes by $C^*(A_1,\ldots,A_n)$ the $C^{*}$-algebra generated by the $A_i$. This is the closure of every finite products of the form $Z_1\cdots Z_k$ where each $Z_j$ is one of the $A_i$.

For any $A$ in a $C^{*}$-algebra we know that $\|A^*A\|=\|A\|^2$.
%TODO: a proof of equation \eqref{eq:ray_norme}

If $A$ is normal, then $C^*(A,\cun)$ is commutative. Indeed any element of the form $A_A\ldots A_n$ with $A_i=A$ or $A^*$ can be written under the form $A\ldots AA^*\ldots A^*$.

\begin{proposition}
\begin{equation}\label{eq:ray_norme}
\|A\|=\sqrt{ r(A^*A) }
\end{equation}
\end{proposition}

\begin{proof}
Let $A$ be in $\cA$ and consider a $z\in\rho(A)$. By definition, $(A-z)^{-1}$ exists in $\cA$; since $\varphi$ is a morphism, $\varphi(A-z)$ is also invertible: it is clear that $\varphi( (A-z)^{-1} )$ is a two-sided inverse of $\varphi(A-z)$. Hence $\rho(A)\subseteq\rho(\varphi(A))$ and thus $\sigma(\varphi(A))\subseteq\sigma(A)$. Definition (\ref{def:spectre}) of the spectral radius makes $r(\varphi(A))\leq r(A)$ and equation \eqref{eq:ray_norme} gives the thesis.
\end{proof}


\begin{theorem}
 Consider an unital $C^{*}$-algebra $\cA$ and a $A\in\cA$ such that $A^*=A$. Then
\begin{enumerate}
\item The spectrum $\sigma_{\cA}(A)$ is the same as $\sigma_{C^*(A,\mtu)}(A)$, so that one can speak about $\sigma(A)$ without ambiguities.

\item $\sigma(A)\subset\eR$.

\item \label{enukiii} $\Delta(C^*(A,\mtu))$ is homeomorphic to $\sigma(A)$ and $C^*(A,\mtu)$ is isomorphic to $C(\sigma(A))$. Under this isomorphism, the Gelfand transformed $\dpt{\hat A}{\sigma(A)}{\eR}$ is the identity $\dpt{id_{\sigma(A)}}{t}{t}$.
\end{enumerate} \label{tho:l_2.5.1}
\end{theorem}
\begin{proof}
We first consider a normal $B\in G(\cA)$, and the $C^{*}$-algebra $C^*(B,B^{-1},\mtu)$ generated by $B$, $B^{-1}$ and $\mtu$. Since $(B^{-1})^*=(B^*)^{-1}$ and $BB^*=N^*B$, $[B^{-1},{B^*}^{-1}]=0$.

Now, we are going to show that $[{B^*}^{-1},B]=0$. First remark that ${B^*}^{-1} B=(B^{-1} B^*)^{-1}$. We have to show that $B^{-1} B^*B{B^*}^{-1}=\mtu$ and
$B{B^*}^{-1} B^{-1} B^*=\mtu$. These two equalities comes from $[B,B^*]=0$ and $[B^{-1},{B^*}^{-1}]=0$. The same makes that $[B^*,B^{-1}]=0$.

The result is that $C^*(B,B^{-1},\mtu)$ is a commutative $C^{*}$-algebra So one can simply say that it is the closure of the polynomials in
$B$, $B^*$, $B^{-1}$, and ${B^*}^{-1}$.

By the Gelfand theorem, $C^*(B,B^{-1},\mtu)$ is then isomorphic to a $C(X)$ for some compact Hausdorff space $X$. Since $B$ is invertible and the Gelfand transform is an isomorphism, $\hB$ is invertible. Then $\forall\,x\in X$, $\hB(x)\neq 0$. Indeed, the $X$ is (up to an isomorphism) $\Delta(C^*(B,B^{-1},\mtu))$. If for an $\omega\in\Delta(C^*(B,B^{-1},\mtu))$, $\hB(\omega)$, then $\omega(B)=0$ and thus $\omega\equiv 0$. But in the definition of $\Delta(\cA)$, we have explicitly excluded the null form.

On the other hand let us consider $f\in C(X)$ everywhere non zero. Since (pointwise) $0<\|f\|^{-2}_{\infty}ff^*\leq 1$,
\begin{equation}\label{eq:ff}
   0\leq 1_X-\|f\|^{-2}_{\infty}ff^*<1.
\end{equation}
% TODO: retrouver où c'est fait, et référentier.
But if $\|A\|<1$, then
\[
   \sum_{k=0}^{n}A^k\to (\mtu-A)^{-1}.
\]
As far as $f$ is concerned for the sup norm, equation \eqref{eq:ff} makes $1_X-ff^*/\|f\|^2_{\infty}$ satisfy this convergence. Then
\[
   \left(
      \frac{ff^*}{\|f\|^2_{\infty}}
   \right)^{-1}
      =
   \sum_{k=0}^{\infty}
   \left(
       \mtu-\frac{ff^*}{\|f\|^2_{\infty}}
   \right)^k,
\]
so that
\begin{equation}
   \us{f}
      =
   \frac{f^*}{\|f\|^2_{\infty}}
   \sum_{k=0}^{\infty}
   \left(
       \mtu-\frac{ff^*}{\|f\|^2_{\infty}}
   \right)^k.
\end{equation}
This is true for any $f$ such that $f(x)\neq 0$ $\forall x\in X$; in particular, it is true for $\hB$. Thus $\hB^{-1}$ is a limit of polynomials in $\hB$ and $\hB^*$. By the inverse Gelfand transform (which is obviously an isomorphism), $B^{-1}$ is a limit of polynomials in $B$ and $B^*$. This is:
\begin{equation}
   C^*(B,B^{-1},\mtu)=C^*(B,\mtu).
\end{equation}

Now, we take our $A$ from the hypothesis: $A=A^*$. Clearly, $A$ is normal and $A-z$ too. If we take $z\in\rho(A)$, our work about $B$ applies to $A-z$. Then $(A-z)^{-1}$ can be written as polynomials in $(A-z)$ and $\mtu$. Thus $(A-z)^{-1}\in C^*(A-z,\mtu)$ and
$z\in\rho_{C^*(A-z,\mtu)}(A)$, but it is clear that $C^*(A-z,\mtu)=C^*(A,\mtu)$. Finally:
\[
   \rho_{\cA}(A)=\rho_{C^*(A,\mtu)}(A).
\]
The set $\sigma$ being nothing else than the complement of $\rho$, the first point of the theorem is finish.

In the course of the demonstration of the Gelfand theorem, we had shown that since $A=A^*$, $\forall\omega\in\Delta(\cA)$, $\hat A(\omega)\in\eR$. But
\[
   \sigma(A)=\{\hat A(\omega):\omega\in\Delta(\cA)\}.
\]
Then $\sigma(A)\subset\eR$.

The proof that $\hat A$ is a bijection and that it is continuous is not done here. Here we will just prove the continuity of $\hat A^{-1}$. From theorem~\ref{tho:unital_comm} and what we just did, we know that
\[
   \sigma(A)=\{\hat A(\omega):\omega\in\Delta(\cA)\}\subset\eR,
\]
but $\hat A^{-1}(z)=\omega$ when $\hat A(\omega)=z$, or $\omega(A)=z$. Then $\hat A^{-1}$ is defined on $\sigma(A)$. So from now, one can only consider $z\in\sigma(A)$ and $\hat A^{-1}(z)(A)=z$. By induction, $\hat A^{-1}(z)(A^n)=z^n$. An element in $\Delta(C^*(A,\cun))$ is completely determined by its value on $A$. Then an open set therein has the general form
\[
  \mR=\{ \omega\in\Delta(C^*(A,\cun))\tq\hat A(\omega)\in\mO \}
\]
where $\mO$ is any open set in $\eC$. From definition, $\hat A(\mR)=\mO$. So $\hat A^{-1}$ is continuous.

\end{proof}


\begin{probleme}
    There is a notational clash: what is written $\sigma(A)$ is the spectrum of $A$. I want to write it $\Spec(A)$ instead.
\end{probleme}

The following proposition is the \defe{continuous functional calculus}{continuous!functional calculus!selfadjoint in $C^*$-algebra}.
\begin{theorem}[Continuous functional calculus]     \label{ThoContFuncCalculus}
Let $A\in\cA$ be self-adjoint and $f\in C(\Spec(A))$. One can define a map $\tilde f\colon \cA\to \cA$ in such a way that when $f$ is a polynomial, $\tilde f=f$ and in other cases, it is the uniform approximation of $f$ by polynomials. This map $\tilde f$ which will be denoted by~$f$ fulfills
\begin{enumerate}
\item $\Spec(f(A))=f(\Spec(A))$,  \label{enuji}
\item $\|f(A)\|=\|f\|_{\infty}$.
\end{enumerate}\label{prop:cont_calc}
\end{theorem}

\begin{probleme}
    The proof has to be reordered.
\end{probleme}

\begin{proof}
We know from theorem~\ref{tho:l_2.5.1} that $\Delta(C^*(A,\cun))$ is homeomorphic to $\sigma(A)$ and $C^*(A,\cun)$ to $C(\sigma(A))$.

The Gelfand theorem says that if one has a commutative unital $C^{*}$-algebra then  one has an unique (up to homeomorphism) $X$ such that $\cA$ is isomorphic to $C(X)$. Moreover, this isomorphism is an isometry\quext{Is is correct?}. But we just showed that $C^*(A,\mtu)$ where isomorphic to $C(\sigma(A))$, then one has
\begin{equation}\label{eq:norm_vp_B}
  \|\varphi(B)\|=\|B\|,
\end{equation}
the first norm is taken in $C(\sigma(A))$ and the second one in $C^*(A,\mtu)$. But when $X$ is Hausdorff, we had adopted the $\|.\|_{\infty}$ norm, so that $\|\varphi(B)\|=\|f\|_{\infty}$ and equation \eqref{eq:norm_vp_B} reads:
\begin{equation}
\|f\|_{\infty}=\|f(A)\|.
\end{equation}

Now remark that $f(\sigma(A))$ is the set of values that $f$ takes on $\sigma(A)$, but we know\quext{Vas voir si on know \c{c}a vraiment} that
\[
\sigma(A)=\sigma(\hat A)=\{ \hat A(\omega):\omega\in\Delta(C^*(A,\mtu)) \}.
\]

It is now times to give a sense to $f(\hat{A})$.  Since $f$ is continuous on $\sigma(A)$, there exists a converging infinite sum such that $f(t)=\sum_{k=0}^{\infty}c_kt^k$ for any $t\in\sigma(A)$. In particular, $\forall\omega\in\Delta(C^*(A,\mtu))$, $\hat A(\omega)\in\sigma(A)$; thus
   $\sum c_k[\hat A(\omega)]^k$ converges everywhere we want. This sum will be denoted by $f(\hat{A})(\omega)$:
\begin{equation}
f(\hat A)(\omega)=\sum_{k=0}^{\infty}c_k[\hat A(\omega)]^k.
\end{equation}
In other words, $f(\hat A)(\omega)=f(\hat A(\omega))$. We have
\begin{equation}
   f(\sigma(A))=\{f(\hat A(\omega)):\omega\in\Delta( C^*(A,\mtu) )\}
               =\{ f(\hat A)(\omega):\omega\in\Delta( C^*(A,\mtu) ) \}
           =\sigma(f(\hat A)).
\end{equation}


It remains to be proved that $\sigma(f(\hat{A}))=\sigma(f(A))$. We already know that $\sigma(A)=\sigma(\hat{A})$, so we just have to prove that $f(\hat{A})=\widehat{ f(A) }$. On the one hand,
\[
 f(\hat{A})\omega=\sum c_k[\hat{A}(\omega)]^k
        =\sum c_k [\omega(A)]^k
        =f\big( \omega(A) \big),
\]
on the other hand,
\[
  \widehat{ f(A) }\omega=\omega\big( f(A) \big)=\omega\big[ \sum c_k A^k \big].
\]
On the other hand, one already know that $\sigma(A)=\sigma(A)$, thus we just have to see that
$f(\hat A)=\widehat{f(A)}$ when $f\colon \sigma(A)\to \eC$ is continuous. The problem is a permutation of $\omega$ and a limit:
\[
 f(\hat A)\omega=\sum_{k=0}^{\infty} c_k\,\omega(A^k),\quad\widehat{f(A)}\omega=\omega\left(\sum_{k=0}^{\infty} c_kA^k\right).
\]
What theorem ~\ref{tho:l_2.5.1} says is that $C^*(A,\cun)$ is isomorphic to $C(\sigma(A))$ with
\[
  \sum c_kA^K\mapsto f(x)=\sum c_kx^k.
\]
 In this isomorphism, the map $\hat{A}\colon \sigma(A)\to \eR$ corresponds to the identity map. More precisely, the isomorphism $\varphi\colon C^*(A,\cun)\to C(\sigma(A))$ is the following:
\[
  \varphi(B)(t)=a+\sum c_kt^k
\]
when $B=a+\sum c_kA^k$. We know in general that
\[
\sigma(A)=\sigma(\hat{A})=\{ f\big( \hat{A}(\omega) \big)\tq\omega\in\Delta(\cA) \}.
\]
 In the present case, we are working with $\cA=C^*(A,\cun)$, therefore
\[
  f\big( \sigma(A) \big)=\{ f\big( \hat{A}(\omega) \big)\tq \omega\in\Delta\big( C^*(A,\cun) \big) \}.
\]
We have to prove that $f\big( \hat{A}(\omega) \big)=f(\hat{A})\omega$. Since $f$ is continuous on $\sigma(A)$, the sum $f(t)=\sum c_kt^k$ converges for all $t\in\sigma(A)$. In particular for $\hat{A}(\omega)\in\sigma(A)$, the sum $\sum_{k=0}^{\infty}\big[ \hat{A}(\omega) \big]^k$ converges.

It is now time to give a sense to $f(\hat{A})$. We know from theorem~\ref{tho:l_2.5.1} that $\hat{A}\colon \Delta\big( C^*(A,\cun) \big) \to\sigma(A) $ is an isomorphism. As definition we set
 \begin{equation}
  f(\hat{A})(\omega)=\sum c_k\big[ \hat{A}(\omega) \big]^k
\end{equation}
everywhere it converges. But, since $\hat{A}(\omega)\in\sigma(A)$, it converges everywhere it is interesting for us.

By definition, $\sum_{k=0}^{\infty}=\lim_{n\to\infty}\sum_{k=0}^{n}$, but the proposition~\ref{prop:continu_cv}, which gives link between convergence and continuity, assures us that one can permute the sum and $\omega$ because it is a continuous function on $C^*(A,\mtu)$ which is by definition the closure of all polynomials in $A$:
\begin{equation}
\begin{split}
  \omega(\sum_k B^k)&=\omega(\lim_{n\to\infty}\sum_k^n B^k)
                    =\lim_{n\to\infty}\omega(\sum_k^n B^k)\\
            &=\lim_{n\to\infty}\sum_k^n\omega(B^k)
            =\sum_{k=0}^{\infty}\omega(B^k).
\end{split}
\end{equation}


We now turn our attention to the second point: $\| f(A) \|_{C^*(A,\cun)}=\| f(A) \|_{\cA}$. It uses proposition 2.26, chapter 4 of \cite{LaHarpe}.

\end{proof}

\subsection{The isomorphism \texorpdfstring{$C^*(A,\mtu)\leftrightarrow C(\sigma(A))$}{AAm AsA}}
%--------------------------------------------------------------------

By definition an element $B\in C^*(A,\mtu)$ can be written as $B=f(A)$ where $f$ is a sum of $\mtu$, $A$, $A^2$, $A^*$, $(A^*)^2$,\ldots In the setting of continuous functional calculus, we suppose that $A$ is selfadjoint, i.e. $A=A^*$, so that $f(A)$ is polynomial (eventually infinite) in $A$ with an independent term $\mtu$. The isomorphism that we consider is
\begin{equation}
\begin{aligned}
 \varphi\,:\,C^*(A,\mtu)&\to C(\sigma(A))\\
    \varphi(B)&=f\in C(\sigma(A))
\end{aligned}
\end{equation}
where $f$ is the ``definition'' function of $B$ in $C^*(A,\mtu)$.

\begin{remark}      \label{RemExpansionSqrtConCal}
    The map $\varphi$ depends on $A$. It could be better written $\varphi_A$. As an example, if $A=A^*$, the element $A^{1/2}$ is computed as follows. First, we know the \wikipedia{en}{Taylor_series}{expansion}
    \begin{equation}        \label{EqExpanSqrtt}
        \sqrt{t}=\sum_ka_kt^k.
    \end{equation}
    Then we define $\sqrt{A}=\sum_k a_kA^k$ as element of $C^*(A,\mtu)$.
\end{remark}

\begin{probleme}
    An expansion \eqref{EqExpanSqrtt} is only possible when $t$ is close to $1$. Maybe the definition of $\sqrt{A}$ has to first look at $B=\lambda A$ with $\lambda$ such that the norm of $B$ is close to $1$. Then we write $\sqrt{A}=\frac{1}{ \sqrt{\lambda} }\sqrt{B}$. The square root of $\lambda$ is well defined as a square root in $\eR^+$.
\end{probleme}


In order to show that it is actually an isomorphism, we have to show the following points:
 \begin{enumerate}
     \item
          it is linear;
      \item
         bijective;
     \item
         $\varphi(CD)=\varphi(C)\varphi(D)$;
     \item
         $\varphi(B^*)=\varphi(B)^*$.
 \end{enumerate}
 Here are the proofs.
 \begin{enumerate}
     \item
        The linearity is clear.
    \item
        Suppose $\varphi(B)=\varphi(C)$. Definition of $\varphi$ gives $B=\varphi(B)(A)$ and $C=\varphi(C)(A)$. For the surjectivity, note that $C(\sigma(A))$ is given by continuous functions whose can be uniformly approximated by polynomials; then for each $f\in C( \sigma(A))$, there corresponds a $B=\varphi(A)\in C^*(A,\mtu)$.
    \item
        Consider $C=f(A)$, $D=g(A)$; thus $CD=(fg)(A)$ and $\varphi(CD)=fg=\varphi(C)\varphi(D)$.
    \item
        The last point comes from the fact that $A=A^*$. Indeed, consider $B=f(A)=\sum_k c_kA^k$. Then
        \[
            B^*=\sum_k c_k^*(A^*)^k=\sum_k c_k^*A^k=f^*(A).
        \]
 \end{enumerate}

We have shown that $\varphi(B)=f$ when $B=f(A)$ is an isomorphism between $C^*(A,\mtu)$ and $C(\sigma(A))$ if $A$ is selfadjoint: $A=A^*$.

\begin{corollary}
For each $C^*$-algebra, there exists an unique unital $C^*$-algebra $\cA_{\cun}$ and an isometric morphism $\cA\to\cA_{\cun}$ such that $\cA_{\cun}/\cA\simeq\eC$.
\end{corollary}\label{cor_csa_unit}

\begin{proof}
We yet defined $\cA_{\cun}$ in lemma~\ref{lem:unitariz_C} and we just prove that the norm was unique. Since all elements in $\cA_{\cun}$ are given under the form $A+\lambda\cun$ with $A\in\cA$, it is obvious that $\cA_{\cun}/\cA\simeq\eC$. The canonical injection $\varphi(A)=A$ is a morphism.
\end{proof}


\begin{lemma}
If $\dpt{\varphi}{\cA}{\cB}$ is a morphism of $C^*$-algebra and if $A=A^*$, then
\[
  f(\varphi(A))=\varphi(f(A)).
\]
for all $f\in C(\sigma(A))$.
\end{lemma} \label{lem:fvpvpf}


\begin{proof}
Since $\sigma(\varphi(A))\subseteq\sigma(A)$, the function $f$ is well defined on $\varphi(A)$. If $f$ is a polynomial, the result comes from the fact that $\varphi(AB)=\varphi(A)\varphi(B)$. If $f$ is a general continuous function, it can be approximated by polynomials. Taking partial sums, $s_n=\sum_{k=1}^n\varphi(c_kA^k)$ and $v_n=\varphi(\sum_{k=1}c_kA^k)$, the linearity of \emph{finite sums} gives the result.
\end{proof}

Where in the proof did we use the assumptions? The definition of $f(A)$ when $\dpt{f}{\eR}{\eR}$ was given in~\ref{prop:cont_calc} in order to get formulas $\sigma\circ f=f\circ\sigma$ and $\| f(A) \|=\| f \|$.

\section{Positivity}
%+++++++++++++++++++

Let $\cA$ be a $C^*$-algebra. We say that $A\in\cA$ is \defe{positive}{positive!element!of a $C^*$-algebra} when
\begin{enumerate}
\item  $A=A^*$
\item  $\Spec(A)\subset\eR^+$.
\end{enumerate}
In this case, we write $A\geq 0$ or $A\in\cA^+$,
\[
  \cA^+=\{ A\in\cA_{\eR}\tq\sigma(A)\subset\eR^+ \}.
\]
A set of particular importance is the set of selfadjoint elements:\nomenclature[C]{\(\cA_{\eR}\)}{The set of selfadjoint elements in \(\cA\)}
\begin{equation}
    \cA_{\eR}=\{ A\in\cA\tq A=A^* \}.
\end{equation}
These elements have real spectrum. We will see in theorem~\ref{ThoElsPositifsBBstar} that the set of positive elements in $\cA$ is given by
\begin{equation}
    \cA^+=\{ A^2\tq A\in\cA_{\eR} \}=\{ B^*B\tq B\in\cA \}.
\end{equation}

\begin{lemma}
For all $A$ such that $A=A^*$, we have a decomposition
\[
  A=A_++A_-
\]
where $A_+,A_-\in\cA^+$ and $A_+A_-=0$. Furthermore
\[
  \| A_{\pm} \|\leq \| A \|.
\]
 \label{lem:AsAdecm}
\end{lemma}

\begin{proof}
We apply the continuous functional calculus with $f=\in_{\sigma(A)}=f_++f_-$ where
\begin{equation} \label{eq:rrdeux}
\begin{aligned}
  \id_{\sigma(A)(t)}&=\max{0,t}&&\textrm{because $\sigma(A)\subset\eR^+$}\\
  f_+(t)&=\max\{ 0,t \}\\
  f_-(t)&=\max\{ -t,0 \}.
\end{aligned}
\end{equation}
Recall that when $A=A^*$, the spectral radius is given by $r(A)=\| A \|$. Then $\| f_{\pm} \|_{\infty}\leq r(A)=\| A \|$.

Let us prove that $f_+(A)\in\cA^+$. From the continuous calculus and the fact that $f_+(A)^*=f_+(A)$, we find that $\sigma(f_+(A))\subset\eR^+$. Since $A\in\cA_{\eR}$, we know that $\sigma(A)\subset\eR$ and thus that $f_+(\sigma(A))\subset\eR^+$. From equation part~\ref{enuji} of the continuous functional calculus, theorem~\ref{prop:cont_calc}, we conclude that $\sigma(f_+(A))\subset\eR^+$ and then that $f_+(t)f_-(t)=0$.
\end{proof}

\begin{lemma} \label{lem:rtrois}
If $-C^*C\in\cA^+$ for $C\in\cA$, then $C=0$.
\end{lemma}

\begin{proof}
We can decompose $C=D+iE$ with $D$, $E\in\cA_{\eR}$; then
\begin{equation}  \label{eq:rquare}
C^*C=2D^2+2E^2-CC^*.
\end{equation}
 If $z\neq 0$ and $AB-z$ is invertible, then $BA-z^{-1}\cun$ is invertible and $(BA-z)^{-1}=B(AB-z)^{-1}A-z^{-1}\cun$. Then $\sigma(AB)\cup\{ 0 \}=\sigma(BA)\cup\{ 0 \}$ and $\sigma(C^*C)\subset\eR^-$ imply $\sigma(-CC^*)\subset\eR^+$. Now all terms of the right hand side of \eqref{eq:rquare} are in $\cA^+$ and $C^*C\in\cA^+$. Since the assumption is $-C^*C\in\cA^+$, we conclude that $\sigma(C^*C)=0$ and $C=0$.
\end{proof}


\begin{theorem}     \label{ThoElsPositifsBBstar}
The set of positive elements in $\cA$ is given by
\begin{equation}
\cA^+=\{ A^2\tq A\in\cA_{\eR} \}=\{ B^*B\tq B\in\cA \}
\end{equation}
when $\cA$ is an unital $C^*$-algebra.
\end{theorem}

\begin{proof}
If $A\in\cA^+$, one can define $\sqrt{A}\in\cA_{\eR}$ in the same way as in proposition~\ref{prop:cont_calc} with $f=\sqrt{\cdot}$. With this definition we have $(\sqrt{A})^2=A$, so that $\cA^+\subset\{ A^2\tq A\in\cA_{\eR} \}$.

Using the linearity of the involution term by term in the formula $\sqrt{A}=\sum_k c_kA^k$ shows that $\sqrt{A}\in\cA_{\eR}$ when $A=A^*$.

For the inverse inclusion, consider $A\in\cA_{\eR}$. Since $A=A^*$, we have $\sigma(A)\subset\eR$. Using formula $\sigma(f(A))=f(\sigma(A))$ with $f(t)=t^2$, we find $\sigma(A^2)=\sigma(A)^2\subset\eR^+$. The first equality is proved.

For the second equality, we begin by applying lemma~\ref{lem:AsAdecm} to $B^*B$, let $B^*B=A_+-A_-$. From equations \eqref{eq:rrdeux} we see that $A_+-A_-=-A_-$. Then $(A_-)^3=-A_-(A_+-A_-)A_-=-A_-B^*BA_-=-(BA_-)^*BA_-$. Since $AA_-$ is positive, $\sigma(A_-)\subset\eR^+$. Using the continuous calculus with $f(t)=t^3$, it proves that $(A_-)^3\geq 0$ and thus that $-(BA_-)^*BA_-\geq 0$. Lemma~\ref{lem:rtrois} shows that $BA_-=0$.

This proves that $(A_-)^3=0$. From the continuous functional calculus with $f(t)=t^{1/3}$, it proves that $A_-=0$ and then $B^*B=A_+\in\cA^+$.
\end{proof}
\begin{corollary}
When $A_1,A_2\in\cA_{\eR}$ and $B\in\cA$, if $A_1\leq A_2$, then $B^*A_1B\leq B^*A_2B$ .
\end{corollary}

\begin{proof}
The assumption is $A_2-A_1\geq 0$, but from theorem~\ref{ThoElsPositifsBBstar}, there exists $A_3\in\cA$ such that $A_2-A_1=A^*_3A_3$. The same property shows that $(A_3B)^*A_3B\geq 0$. This gives the corollary.
\end{proof}


\begin{corollary}
For all $A$, $B\in\cA$, we have
\[
  B^*A^*AB\leq \| A \|^2B^*B.
\]
 \label{cor:BeAAeB}
\end{corollary}

\begin{proof}
The inequality $-\| A \|\cun\leq A\leq \| A \|\cun$ holds when $A=A^*$. Let us write it for $A^*A$ and recall that $\| A^*A \|=\| A \|^2$ in all $C^*$-algebra. Then $A^*A\leq \| A \|^2\cun$ and by applying the previous corollary, we find $B^*A^*AB\leq\| A \|^2B^*B$
\end{proof}



An element $A\in\cA$ is a \defe{projection}{projection!in a $C^*$-algebra } if $A=A^*$ and $A^2=A$. In the case of a $C^*$-algebra of linear operators acting on a vector space, if $x$ is an eigenvector of the projection $A$ with the eigenvalue $\lambda$, then $Ax=\lambda x$ and $A^2x=\lambda^2x=\lambda x$. Thus $1$ is the only eigenvalue of a projection (or zero, which is the kernel). In particular a projection is positive and reads\label{PgProjPositif} $A=B^*B$ for some $B\in \cA$ by theorem~\ref{ThoElsPositifsBBstar}.

\begin{probleme}
    I think that the notation \(\cA_{\eR}\) stand for the elements with real spectrum. I have to check it and add to the notation index.
\end{probleme}

\begin{proposition}
An element $A\in\cA_{\eR}$ is positive if and only if the Gelfand transform $\hat A$ is pointwise positive in $C(\sigma(A))$.
\end{proposition}

\begin{proof}
\subdem{Necessary condition}
We know from theorem~\ref{tho:unital_comm},~\ref{enugiii} that
\[
  \sigma(A)=\sigma(\hat A)=\{ \hat A(\omega)\tq\omega\in\Delta(\cA) \},
\]
but $\sigma(A)\subset\eR^+$ if $A$ is positive.

\subdem{Sufficient condition}
From hypothesis, $A\in\cA_{\eR}$ and $A^*=A$. We have to see that positivity of $\hat A$ implies $\sigma(A)\subset\eR^+$. From point~\ref{enukiii} of theorem~\ref{tho:l_2.5.1}, the function $\dpt{\hat A}{\sigma(A)}{\eR}$ is identity and positive, $\sigma(A)\subset\eR^+$.

\end{proof}


\begin{proposition}     \label{PropAplusConvexCone}
    The set $\cA^+$ of positive elements of the $C^*$-algebra $\cA$ is a convex cone (see definition~\ref{DefConvexCone}).
\end{proposition}

Note that the $C^*$-algebra $\cA$ has to be commutative in order the Gelfand transform to be defined. It is supposed unital too.

\begin{proof}

    We have to check the \(3\) points of definition~\ref{DefConvexCone}.
    \begin{enumerate}
            \item

                We know that if $A=A^*$ and $f\in C(\sigma(A))$, the commutator $[\sigma,f]$ is zero; as a particular case $\sigma(tA)=t\sigma(A)$. Then for $t>0$, the element $tA$ is positive.

            \item

                The fact that $\sigma(A)\subset [0,r(A)]$ implies that for all $t\in\sigma(A)$ and for all $c\geq r(A)$,  $| c-t |\leq c$. Now we study the quantity
                \[
                  \sup_{t\in \sigma(A)}| c1_{\sigma(A)}-\hat A |.
                \]
                The function $1_{\sigma(A)}$ is $0$ or $1$ following the argument belongs to $\sigma(A)$ or not while $\hat A(t)=t$ in $\sigma(A)$. Then
                \[
                  \sup_{t\in\sigma(A)}| c1_{\sigma(A)}(t)-\hat A(t) |=\sup_{t\in\sigma(A)}| c-t |\leq c.
                \]
                This shows that
                \begin{equation} \label{eq:cunhatA}
                    \| c1_{\sigma(A)}-\hat A \|_{\infty}\leq c
                \end{equation}
                for all $c>r(A)$ and then for all $c>\| A \|$. Since $\cA$ is commutative and $A=A^*$, we know that $\| \hat A \|_{\infty}=\| A \|$ from  \eqref{eq:AinfA }. Taking the inverse Gelfand transform of equation \eqref{eq:cunhatA}, we find
                \begin{equation} \label{eq:norcin}
                    \| c\cun-A \|\leq c
                \end{equation}
                for all $c\geq\| A \|$. Be careful on a point: the inverse Gelfand transform is not taken into $\cA$, but into $C^*(A,\cun)$ which is commutative and unital and then fulfills $\| \hat A \|_{\infty}=\| A \|$.

                We know that the norm of $f(A)$ in $\cA$ and in $C^*(A,\cun)$ are the same, namely equation \eqref{eq:norcin} is a relation for the norm of $c\cun-A$ in $C^*(A,\cun)$. Until now we had proved that if $\sigma(A)\subset\eR^+$, then $\| c\cun-A \|\leq c$ for all $c\geq\| A \|$.

                Taking the inverse argument, we can say that if $\| c\cun-A \|\leq c$ for a certain $c\geq\| A \|$, then $\sigma(A)\subset\eR^+$. Indeed the Gelfand transform of the assumption gives $\| cA_{\sigma(A)}-\hat A \|_{\infty}\leq c$, i.e. $\sup_{t\in\sigma(A)}| c1_{\sigma(A)}A-\hat A |\leq c$. As $\hat A$ is identity on $\sigma(A)$, for all $t\in\sigma(A)$, we have $| c-t |\leq c$. This shows that $t>0$ for all $t\in\sigma(A)$. Thus $\sigma(A)\subset\eR^+$.

                Let us now take $A+B$ instead of $A$ and $c=\| A \|+\| B \|$. Remark that $c\geq \| A+B \|$. We have
                \[
                    \| c\cun-(A+B) \|\leq\| (\| A \|-A) \|+\| (\| B \|-B) \|
                \]
                where $\| A \|-A=r\cun-A$ with $r=\| A \|$. On the other hand, $\| r\cun-A \|\leq r$ for all $r\geq\| A \|$, then we can apply the first result to get
                \[
                    \| c\cun-(A+B) \|\leq \| A \|+\| B \|=c
                \]
                with $c\geq \| A+B \|$. Then the inverse argument gives $\sigma(A+B)\subset\eR^+$ and $A+B\in\cA^+$.

            \item

                If $A\in\cA^+\cup (-\cA^+)$. Then $\sigma(A)\subset\eR^+$ and $\sigma(A)\subset\eR^-$; we conclude that $\sigma(A)=\{  0\}$. Since $\| A \|=r(A)$, this gives $\| A \|=0$.

        \end{enumerate}
\end{proof}


\begin{proposition}
Let $E$ be a real locally convex space and $C$ a closed convex cone with top on $0$\quext{Par top je veux dire le sommet du cône je ne sais pas comment dire en anglais.} and $x\in E$, $x\notin C$. Then there exists a continuous linear function $f\colon E\to \eR$ such that
\begin{itemize}
\item $f\geq 0$ on $C$,
\item $f(x)<0$.
\end{itemize}

\end{proposition}

\begin{proof}
It is possible to find a continuous linear form $f$ and a real $\alpha$ such that $f(y)\geq\alpha$ on $C$ and $f(x)<\alpha$ (see Urysohn lemma~\ref{lem:Urysohn}). We have $0=f(0)\geq\alpha$, so $f(x)<0$. If $f(y)<0$ fora $y\in C$, we find $f(\lambda y)<\alpha$ for a large enough $\lambda$. This is absurd and we conclude that $f\geq0$ on $C$.
\end{proof}

\input{119_cstar}
% This is part of (almost) Everything I know in mathematics
% Copyright (c) 2013-2014, 2020
%   Laurent Claessens
% See the file fdl-1.3.txt for copying conditions.

\section{Representation}
%+++++++++++++++++++++++

\subsection{Representation of involutive algebra}
%-------------------------------------------------


Let $\cA$ be an involutive algebra and $H$ a Hilbert space. A \defe{representation}{representation!of involutive algebra} of $\cA$ in $H$ is a map $\pi\colon \cA\to \mL(H)$ such that
\begin{equation}
\begin{aligned}
  \pi(A+B)&=\pi(A)+\pi(B),&\pi(\lambda A)&=\lambda\pi(A),\\
\pi(AB)&=\pi(A)\circ\pi(B),&\pi(A^*)&=\pi(A)^*.
\end{aligned}
\end{equation}

A linear form $f\colon \cA\to \eR$ on the involutive algebra $\cA$ is \defe{positive}{positive!form on involutive algebra} if $f(A)\geq 0$ whenever $A>0$.

\begin{lemma}
If $\rho\colon \eM_n(\eC)\to \End(V)$ is a representation of the matrix algebra $\eM_n(\eC)$ on the finite dimensional space $V$, then there exists an isomorphism $V\to \eC^n\oplus\ldots\oplus\eC^n$ which intertwines $\rho$ to a multiple of the standard representation of the matrices on $\eC^n$.
\end{lemma}


\begin{proposition}
Let $\cA$ be an involutive algebra. We have
\begin{enumerate}
\item \label{itemi_prop_invalgrepr} If $\pi$ is a representation of $\cA$ in $H$ and if $\xi\in H$, then $A\mapsto\scal{ \pi(A)\xi }{ \xi }$ is a positive form on $\cA$.
\item  \label{itemii_prop_invalgrepr} Let $\pi$ and $\pi'$ be two representations of $\cA$ in $H$ and $H'$ respectively, and $\xi,\xi'$, two corresponding totalizing vectors. If $\scal{ \pi(A)\xi }{ \xi }=\scal{ \pi'(A)\xi' }{ \xi' }$ for all $A\in\cA$, then there exists an unique isometry $H\to H'$ which transforms $\pi$ into $\pi'$ and $\xi$ into $\xi'$, i.e.
\[
 \begin{split}
U\pi(A)U^{-1}&=\pi'(A)\\
U\xi&=\xi'
\end{split}
\]
\end{enumerate}
 \label{prop_invalgrepr}
\end{proposition}


\begin{proof}
 For the first point, it is easy:
\[
  \scal{ \pi(A^*A)\xi }{ \xi }=\scal{ \pi(A^*)\pi(A)\xi }{ \xi }=\scal{ \pi(A)\xi }{ \pi(A)\xi }=\| \pi(A)\xi \|^2\geq 0.
\]
We used property $\pi(A^*)=\pi(A)^*$.

For the second point, $\overline{ \pi(\cA)\xi }=H$ because $\xi$ is totalizing. We have
\begin{equation} \label{eq_piAxipreis}
\scal{ \pi(A)\xi }{ \pi(B)\xi }=\scal{ \pi(B^*A)\xi }{ \xi }
        =\scal{ \pi'(B^*A)\xi' }{ \xi' }
        =\scal{ \pi'(A)\xi }{ \pi'(B)\xi' },
\end{equation}
but vectors of the form $\pi(A)\xi$ are everywhere dense in $H$ (and $\pi'(A)\xi'$ in $H'$), so we have an isomorphism
\begin{equation}
\begin{aligned}
 U\colon H&\to H' \\
\pi(A)\xi&\mapsto \pi'(A)\xi'
\end{aligned}
\end{equation}
Equation \eqref{eq_piAxipreis} shows that $U$ is an isometry; it is linear because of linearity of $\pi$. The map $U$ is surjective because $\xi'$ is totalizing and injective because if $\pi'(A)\xi'=\pi'(B)\xi'$, $\pi'(A-B)=0$ which proves that $A=B$ because $\pi'$ is linear.

Now we check that this $U$ is the searched map. For all $A$, $B\in\cA$,
\begin{equation}
\begin{split}
  \big[ U\circ\pi(A) \big](\pi(B)\xi)&=U\pi(AB)\xi=\pi'(AB)\xi'\\
        &=\pi'(A)\pi'(B)\xi'=\big[ \pi'(A)\circ U \big](\pi(B)\xi),
\end{split}
\end{equation}
so $U$ transforms $\pi$ into $\pi'$. In order to prove that $U\xi=\xi'$, we will use the fact that $U$ is an isometry:
\[
 \scal{ \xi' }{ \pi'(A)\xi' }=\scal{ \xi }{ \pi(A)\xi }
        =\scal{ U\xi }{ U\pi(A)\xi }
        =\scal{ U\xi }{ \pi'(A)\xi' }.
\]

For unicity, notice that equation $U\big( \pi(A)\xi \big)=\pi'(A)\xi'$ defines $U$ on a dense subspace of $H$. Continuity finishes to fix $U$.

\end{proof}

With the same notations,  the map $A\to\scal{ \pi(A) }{ A }$ is the form \defe{associated}{associated form with a representation} with representation $\pi$ and the vector $\xi$. Let $\cB$ be an involutive subalgebra of $\mL(H)$ and $\xi$, any element in $H$. We denote by $\omega_{\xi}$ the form on $\cB$ defined by
\begin{equation}
\omega_{\xi}(A)=\scal{ A\xi }{ \xi }
\end{equation}
for all $A\in\cB$. A positive form $\eta$ on $\cB$ is a \defe{vector}{vector!form} form if there exists a $\xi\in H$ such that $\eta=\omega_{\xi}$.

\begin{proposition}
Let $\cA$ be an involutive Banach algebra with approximate unit $(u_i)$,$\pi$ a nondegenerate representation of $\cA$ in $H$, $\xi\in H$ and $f$ the positive form defined from $\pi$ and $\xi$. We have
\begin{equation}
\| f \|=\scal{ \xi }{ \xi }.
\end{equation}
\end{proposition}

\begin{proof}
Point~\ref{itemv_prop_invaddunit} of proposition ~\ref{prop_invaddunit} states that if $(u_i)_{i\in J}$ is an approximate unit in $\cA$, then
\[
  f(u_j)\to\| f \|\quad\text{and}\quad f(u_j^*u_j)\to \| f \|.
\]
Hence $\| f \|=\lim f(u_j)=\lim\scal{ \pi(u_j)\xi }{ \xi }$.

\end{proof}

\begin{proposition}
Let $\cA_{\cun}$ be the involutive algebra deduced from $\cA$ by adding an unit and $\pi$, a representation of $\cA$ in $H$. There exists one and only one way to extend $\pi$ to a representation $\pi_{(\cun)}$ of $\cA_{\cun}$ in such a way that $\pi_{(\cun)}(\cun)=\id$.
\end{proposition}
The representation $\pi_{(\cun)}$ is the \defe{canonical extension}{canonic!extension of a representation} of $\pi$. When $\pi$ is a representation of $\cA$ in $H$, the set
\begin{equation}
 K=\{ \pi(A)\xi\tq A\in\cA,\xi\in H \}
\end{equation}
is a closed vector subspace of $H$. We say that $K$ is the \defe{essential subspace}{essential subspace of a representation} of $\pi$. The representation is \defe{nondegenerate}{nondegenerate!representation}\index{representation!nondegenerate} if $K=H$.

\begin{proposition}
Let $\cA$ be an involutive Banach algebra and $\pi$, a nondegenerate representation of $\cA$ on $H$. If $(u_i)$ is an approximate unit in $\cA$, then $\pi(u_i)$ strongly converges to $\id$ (see subsection~\ref{subsec_topomL}).
\end{proposition}

\begin{proof}
We have to prove that $\| \pi(u_i)\xi-\xi \|\to 0$ for any $\xi\in H$. From non degeneracy, the set $\{ \pi(B)\xi\, , B\in \cA \}$ is total in $H$, so it is sufficient to prove the convergence for $\xi$ of the form $\pi(B)\xi$. We have
\[
  \| \pi(u_iB)-\pi(B) \|\leq\| u_iB-B \|\to 0
\]
because $u_i$ is an approximate unit and relation \eqref{eq_morleqpi}.

The fact that $\pi$ is nondegenerate makes
\[
  \{ \pi(A)\xi\tq A\in\cA,\,\xi\in H \}=H.
\]
If $\mU$ is an open set in $\mL(H)$ around $\id$, we have to prove that $\pi(u_i)\in\mU$ for all $i\geq i_0$. Open sets are taken in the sense of seminorms $s_{\xi}=\| A\xi \|$. The balls ---which are not open--- are of the form
\[
 \begin{split}
B(A;(\xi_j),(r_j))&=\{ X\in\mL(H)\tq s_{\xi_i}(X-A)<r_j\,\forall j \}\\
        &=\{ X\in\mL(H)\tq \| X\xi_j-A\xi_j \|<r_j\,\forall j \}.
\end{split}
\]
If a sequence fall into the balls $B(A;(\xi_j),(r_j))$ for a fixed $A$ and if we consider an open set around this $A$, the latter open set will contain at least one of the $B(A;(\xi_j),(r_j))$, hence the sequence will fall into this open set too. So we have to prove that for all \emph{finite} sequence $(\xi_j)$ and $r_j$ ($\xi_j\in H$ and $r_j\in\eR^+$),
\[
  \pi(u_i)\in B(\id;(\xi_j),(r_j))
\]
when $i$ is large enough.

Let $B\in\cA$ and $\xi\in H$, we have already proved that $\| \pi(u_iB)-\pi(B) \|\leq\| u_iB-B \|\to 0$. We have to prove that $\| \pi(u_i)\xi_j-\xi_j \|< r_j$. Since $\pi(C)\zeta$ is a total set, by redefinition of $\xi_j$, we can write $\xi_j$ under the form $\pi(B_j)\xi_j$. So
\[
  \| \pi(u_i)\pi(B_j)\xi_j-\pi(B_j)\xi_j \|=\| \pi(u_iB_j)\xi_j-\pi(B_j)\xi_j \|=\| [\pi(u_iB_j)- \pi(B_j)]\xi_j \|,
\]
but we know that when $A$ is a bounded operator on a Hilbert space, $\| Av \|\leq \| A \|\,\| v \|$. Thus we have
\[
\| \pi(u_i)\pi(B_j)\xi_j-\pi(B_i)\xi_j \|=\| [\pi(u_iB_j)-\pi(B_i)]\xi_j \|
        =\| \pi(u_iB_j)-\pi(B_i) \|\,\| \xi_j \|.
\]
Since the sequence $(\xi_j)$ is finite, one can bound $\| \xi_j \|$ by a certain $M$, hence
\[
  \| \pi(u_i)\pi(B_j)\xi_j-\pi(B_i)\xi_j \|\leq M\| \pi(u_iB_j)-\pi(B_i) \|.
\]
When $i$ is large, the latter is as small as we want and in particular it can become smaller than all the $r_j$ of the sequence. This proves that $\pi(u_i)$ strongly converges to $\id$ in $\mL(H)$.
\end{proof}

\begin{proposition}[Another version of GNS construction]		\label{PropGNSanother}
Let $\cA$ be an involutive Banach algebra with an approximate unit and the following elements:
\begin{itemize}
\item $\cA_{\cun}$ the involutive algebra obtained by adding an unit to $\cA$,
\item $f$ a positive continuous form on $\cA$,
\item $\tilde f$ its canonical extension to $\cA_{\cun}$,
\item $N$ the left ideal of $\cA_{\cun}$ defined by $N=\{ A\in\cA_{\cun}\tq \tilde f(A^*A)=0 \}$,
\item $\cA'_f$ the pre-Hilbert  space $\cA_{\cun}/N$,
\item $\cA_f$ the Hilbert space obtained by completion of the previous one.
\end{itemize}
For each $A\in\cA_{\cun}$, let
\begin{itemize}
\item $\pi'(A)$, the operator in $\cA_{\cun}/N$ obtained from quotient of the left multiplication by $A$ in $\cA_{\cun}$,
\item $\xi$, the canonical image of $\cun$ in $\cA'_f$.
\end{itemize}
In this setting we have
\begin{enumerate}
\item \label{itemi_prop_DixGNS}Each $\pi'(A)$ extends in one and only one way to a linear continuous representation of $\cA$ on $\cA_f$,
\item \label{itemii_prop_DixGNS}the map $A\to\pi(A)$ with $A\in\cA$ is a representation of $\cA$ in $\cA_f$,
\item \label{itemiii_prop_DixGNS}the vector $\xi$ is totalizing for $\pi(\cA)$,
\item \label{itemiv_prop_DixGNS} $\forall A\in\cA$, we have $f(A)=\scal{ \pi(A)\xi }{ \xi }$.
\end{enumerate}
\label{prop_DixGNS}
\end{proposition}

In this context, the vector $\xi$, being the representative of the identity, is sometimes called the \defe{vacuum}{vacuum!of a GNS representation} of the GNS representation.

\begin{proof}

	Let $\eta\in\cA_{\cun}/N$ and let's say $\eta=[B]$ for $B\in\cA_{\cun}$, $\pi'(A)\eta=[AB]$ and $\xi=[\cun]=\{ \cun+n\tq n\in N \}$. If $A\in N$, we have $\tilde f(A^*A)=0$, thus for all $B\in N$,
	\[
	  \big|  \tilde f\big( (BA)^*(BA) \big)   \big|=\big| \tilde f\big( A^*(B^*B)A \big) \big| \leq \| B^*B \|f(A^*A)=0.
	\]
	This proves that $N$ is an ideal. Now $\pi'(B)\xi=\pi'(B)[\cun]=[B]$, so
	\[
	\scal{ \pi'(A)\pi'(B)\xi }{ \pi'(A)\pi'(B)\xi }=\scal{ [AB] }{ [AB] }
			=\scal{ [B^*A^*AB] }{ [\cun] }
	\]
	where this product is defined by
	\begin{equation}
	  \scal{ [A]\, }{\, [B] }:=\tilde f(AB).
	\end{equation}
	So
	\[
	 \begin{split}
	 \scal{ \pi'(A)\pi'(B)\xi }{ \pi'(A)\pi'(B)\xi }&=\tilde f(B^*A^*AB)\\
			&\leq \| A^*A \|\tilde f(B^*B) \\
			&=\| A^*A \|\scal{ \pi'(Bj\xi) }{ \pi'(B)\xi }.
	\end{split}
	\]
	It gives $\| \pi'(A)[B] \|\leq\| A^*A \|\,\| [B] \|$, and finally
	\begin{equation}
	  \| \pi'(A) \|\leq\| A^*A \|,
	\end{equation}
	which proves that $\pi'$ is continuous, and therefore bounded. But a bounded operator on a part of a Hilbert space may be extended to the whole space. This finish the proof of~\ref{itemi_prop_DixGNS}.

	Now we prove that $\pi$ is a representation for the structure of involutive algebra. The algebra structure is clear. For involution,
	\[
	\scal{ \pi(A)\pi(B)\xi }{ \pi(C)\xi }	=\tilde f(C^*AB)
						=\tilde f\big( (A^*C)^*B \big)
						=\scal{ \pi(B)\xi }{ \pi(A^*)\pi(C)\xi },
	\]
	so $\pi(A)^*=\pi(A^*)$. Notice that there are no argument as ``$\pi(B)\xi$ is dense''; we just use the fact that $\pi(B)\xi=[B]$ is the most general element in $\cA_{\cun}/N$. Ok for point~\ref{itemii_prop_DixGNS}.

	On the one hand, $\cA_{\cun}$ is everywhere dense in $\cA$. On the other hand, $\cA'_f$ is everywhere dense in $\cA_f$ from the definition of a completion. But $\pi(\cA)\xi$ is the image of $\cA$ in $\cA'_f$, so $\pi(\cA)\xi=\cA'_f$ is everywhere dense in $\cA_f$. This proves that $\xi$ is totalizing for $\pi(\cA)$.  This proves~\ref{itemiii_prop_DixGNS}.

	Finally, for all $A\in\cA_{\cun}$,
	\[
	  \scal{ \pi(A)\xi }{ \xi }=\tilde f(\cun^*A\cun)=\tilde f(A).
	\]

\end{proof}
We say that the representation $\pi$ and the vector $\xi$ are defined from $f$. So we often write $\pi_f$ and $\xi_f$.

\subsection{Cyclic representations of \texorpdfstring{$C^*$}{C*}-algebra }
%-----------------------------------------------------------------------

\begin{definition}
A \defe{representation}{representation!of a $C^*$-algebra} of the $C^*$-algebra $\cA$ on a Hilbert space $\hH$ is a linear map $\dpt{\pi}{\cA}{\oB(\hH)}$ such that

\begin{enumerate}
\item $\pi(AB)=\pi(A)\circ\pi(B)$,
\item $\pi(A^*)=\pi(A)^*$
\end{enumerate}
for all $A$, $B\in\cA$.
\end{definition}
Most of time, we will denote a representation by the pair $(\pi,\hH)$. When the represented $C^*$-algebra is ambiguous, we write $(\cA,\pi,\hH)$.

\begin{lemma}       \label{Lemrepresnormpresou}
A representation $\pi$ is continuous and fulfills
\begin{equation}
\| \pi(A) \|\leq \| A \|,
\end{equation}
moreover when the representation is faithful, we have $\| \pi(A) \|=\| A \|$.
\end{lemma}

\begin{proof}
The spaces $\cA$ and $\oB(\hH)$ are $C^*$-algebra and $\pi$ is a morphism proposition~\ref{PropMDfqcUs} concludes. The second claims follows from lemma~\ref{lem:injmorpisom}.
\end{proof}

Let $(\pi_1,\hH_1)$ and $(\pi_2,\hH_2)$ be two representations. They are \defe{equivalent}{equivalence!of representation of $C^*$-algebra } when there exists an unitary isomorphism $\dpt{U}{\hH_1}{\hH_2}$ such that
\begin{equation}
  U\pi_1(A)U^*=\pi_2(A)
\end{equation}
for all $A\in\cA$.

A representation $(\cA,\pi,\hH)$ is \defe{nondegenerate}{degenerated representation of $C^*$-algebra} if $0$ is the only vector to be cancelled by all $\pi(A)$. It is \defe{cyclic}{cyclic!representation} if there exists a \defe{cyclic vector}{cyclic!vector} $\Omega\in\hH$, i.e. the closure of $\pi(\cA)\Omega=\hH$.


\begin{lemma}
If $\pi$ is an irreducible representation on $\hH$, then any non zero vector is cyclic.
\end{lemma}

\begin{proof}
Let $v\neq 0$; if it were not cyclic, then $\overline{ \pi(\cA)v}$ should be a proper invariant  subspace of $\hH$.
\end{proof}

\subsection{Primitive spectrum}
%------------------------------
For this short note about primitive spectrum, we follow \cite{Landi}.

When $\cA$ is any (not specially commutative) $BC^*$-algebra, the \defe{primitive spectrum}{primitive spectrum}\index{spectrum!primitive} of $\cA$ is the set $\Prim\cA$\nomenclature{$\Prim$}{Primitive spectrum} of kernels of $*$ irreducible representations. An element in $\Prim\cA$ is a two-sided ideal.

There exists a suitable topology on this space, the \defe{hull-kernel topology}{hull-kernel topology} or \defe{Jacobson topology}{Jacobson topology}\index{topology!hull-kernel}\index{topology!Jacobson}. This is given by means of closure. If $W\subset\Prim\cA$, then we define the closure of $W$ by
\begin{equation}
  \overline{ W }=\{ \mI\in\Prim\cA\tq \cap W\subseteq \mI \}.
\end{equation}
where $\cap W=\bigcap_{\mJ\in W}\mJ$. The inclusion $\cap W\subseteq\mI$ is an inclusion of subsets of $\cA$.

\begin{proposition}
This definition defines a topology. Namely, it fulfils the Kuratowsky axioms:

\begin{enumerate}
\item\label{enu802i} $\overline{ \emptyset }=\emptyset$,
\item \label{enu802ii} $W\subseteq\overline{ W }$,
\item \label{enu802iii} $\overline{ \overline{ W } }=\overline{ W }$,
\item \label{enu802iv} $\overline{ W_1\cup W_2 }=\overline{ W_1 }\cup\overline{ W_2 }$.
\end{enumerate}
\end{proposition}

\begin{proof}
Points~\ref{enu802i}  and~\ref{enu802ii} are trivial. For~\ref{enu802iii}, remark that $\cap W=\cap\overline{ W }$ (equality as subsets of $\cA$). Indeed, consider $A\in\cap W$. Then any $\mJ\in\overline{ W }$ contains $A$ and then $\cap\overline{ W }$ contains $A$. Now if $A\in\cap\overline{ W }$, then any $\mI$ such that $\cap W\subseteq\mI$ contains $A$ and then $A\in\cap W$.

The proof of~\ref{enu802iv} is more complicated. If $V\subset W$, then $(\cap W)\subseteq(\cap V)$ and then $\overline{ V }\subseteq\overline{ W }$. Then $\overline{ W_i }\subseteq\overline{ W_1\cup W_2 }$ for each of $i=1,2$.

The inverse inclusion is as follows. Let $\mI$ be an ideal kernel of the irreducible representation $\pi$ of $\cA$ on the Hilbert space $\hH$. Suppose that $\mI\notin \overline{ W_1 } \cup \overline{ W_2 }$. Then we will prove that $\mI\notin\overline{ W_1\cup W_2 }$. There exists $A\in W_1$ and $B\in W_2$ such that $\pi(A)\neq 0$ and $\pi(B)\neq0$. Let $\xi\in\hH$ such that $\pi(A)\xi\neq0$. Since $\pi$ is irreducible, then $\pi(A)\xi$ is cyclic. Since $\pi(B)\neq 0$, then there exists a $\psi\in\hH$ such that $\pi(B)\psi\neq0$. Cyclicity of $\pi(A)\xi$ shows that there exists a $C$ such that $\pi(C)\pi(A)\xi$ is sufficiently close to $\psi$ to satisfy
\[
  \pi(B)\big( \pi(C)\pi(A)\xi \big)\neq0.
\]
 Then $BCA\notin\ker\pi=\mI$. Since $W_i$ are ideals,
\[
  BCA\in(\cap W_1)\cap(\cap W_2)=\cap(W_1\cup W_2).
\]
Then $BCA\in W_1$ and $\cap(W_1\cup W_2)\nsubseteq\mI$. Consequently, $\mI\notin\overline{ W_1\cap W_2 }$.


\end{proof}


\subsection{GNS construction}
%----------------------------

\begin{lemma}
Let $\mfM$ be a $*$-algebra in $\oB(\hH)$, $\psi\in\hH$ and $p$, the projection into the closure of $\mfM\psi$. Then $p\in\mfM'$, i.e. $[p,A]=0$ for all $A\in\mfM$.\label{lem_preGNS}
\end{lemma}

\begin{proof}
Let $A\in\mfM$; by definition of $p$, we have $Ap\hH\subseteq p\hH$. For a $A\in\mfM$, we have 
\begin{equation}
    Ap\hH=\{ AB\psi\tq B\in\mfM \}, 
\end{equation}
but $\mfM$ is an algebra, then $AB\in\mfM$ and $Ap\hH\subseteq\mfM\psi\subseteq p\hH$. If we define $p^{\perp}=\mtu-p$, we find $p^{\perp}Ap=0$.

 Indeed $(\mtu-p)Ap=Ap-pAp$ and $Apx-pApx$ can be computed by setting $x=B\psi$ for a certain $B$ in the closure of $\mfM$ (the part of $x$ ``outside'' $\mfM$ has no importance). Then
\begin{equation}
\begin{aligned}
   ApB \psi-pAB\psi&=AB\psi-pAB\psi  &&\textrm{because $B\psi\in\mfM$}\\
        &=AB\psi-AB\psi  &&\textrm{because $AB\psi\in\mfM\psi$}\\
        &=0.
\end{aligned}
\end{equation}
It shows that $ApB\psi-pAB\psi=0$ and then that $p^{\perp}Ap=0$.

From this, we see that $Ap=pAp$. Let us now consider $A=A^*$ (for a general element in $\cA$, use the decomposition). We have $(Ap)^*=p^*A^*=pA$, but $(Ap)^*=(pAp)^*=pAp=Ap$. Then $[A,p]=0$.


\end{proof}


\begin{proposition}
Any nondegenerate representation is direct sum of cyclic representations.
\end{proposition}

\begin{proof}
We apply the lemma with $\mfM=\pi(\cA)$. The non degeneracy of $\pi$ makes $p$ non zero: $\mfM\psi$ is never zero. Now we consider the map $\dpt{\rho}{\cA}{\oB(\hH)}$, $\rho(A)=p \pi(A)$. This is a representations because
\begin{equation}
    \rho(AB)=p\pi(A)\pi(B)
        =p\pi(A)p\pi(B)
        =\rho(A)\rho(B),
\end{equation}
and
\begin{equation}
\begin{split}
\rho(A^*)&=p\pi(A^*)=\pi(A^*)p\\
        &=\pi(A)^*p=(p\pi(A))^*=\rho(A)^*.
\end{split}
\end{equation}
More precisely, $\rho$ is a representation on $p\hH$ and when $\pi(A)\in p\hH$, we have $\rho(A)=\pi(A)$. The representation $\rho$ is constructed in such a way that $\psi$ is a cyclic vector:
\[
  \rho(A)\psi=p\hH.
\]
The same construction with $\psi_2\in p^{\perp}\hH$ gives and going on gives the thesis.
\end{proof}


Let $(\cA,\pi,\hH)$ be a nondegenerate representation and $\Psi\in\hH$ a vector with norm $1$. The state $\psi$ given by formula
\[
  \psi(A)=\scal{\Psi}{\pi(A)\Psi}
\]
is the \defe{vector state}{vector!state}\index{state!vector} of $\Psi$ relative to $\pi$.

\begin{theorem}[GNS construction]       \label{ThoGNScontruction}
Let $\cA$ be an unital $C^*$-algebra  and $\dpt{\omega}{\cA}{\eC}$, a state

\begin{enumerate}
\item\label{GNSi} There exists a Hilbert space $\hH$, a representation $\dpt{\pi_{\omega}}{\cA}{\oB(\hH_{\omega})}$ and a cyclic unit vector $\Omega_{\omega}$ such that
\begin{equation}
  \omega(A)=\scal{\Omega_{\omega}}{\pi_{\omega}(A)\Omega_{\omega}}
\end{equation}
for all $A\in\cA$.
\item\label{GNSii} The triple $(\hH_{\omega},\pi_{\omega},\Omega_{\omega})$ is unique up to isomorphism in the following sense. Let $\hH$ be a Hilbert space, $\dpt{\pi}{\cA}{\oB(\cA)}$ a representation and $\Omega\in\hH$ an unit cyclic vector such that $\omega(A)=\scal{\Omega}{\pi(A)\Omega}$ for all $A\in\cA$; then there exists an unitary isomorphism $\dpt{u}{\hH_{\omega}}{\hH}$ such that $u(\Omega_{\omega})=\omega$ and
\[
  \pi(A)=u\pi_{\omega}(A)u^*
\]
for all $A\in \cA$.

\end{enumerate}
\label{tho:GNS}
\end{theorem}

\begin{proof}
From \eqref{eq:omABleq} we know that, if $A\in\mN_{\omega}$ and $B\in\cA$, then $\omega(B^*A)=0$. So we can define $\mN$ in the two equivalent ways:
\begin{equation}
    \begin{aligned}[]
        \mN_{\omega}&=\{ A\in\cA\tq \omega(A^*A)=0 \}\\
        &=\{ A\in\cA\tq \omega(B^*A)=0\textrm{ for all $B\in\cA$} \}.
    \end{aligned}
\end{equation}
From the second line, we see that $\mN_{\omega}$ is an ideal in $\cA$. The set $\mN_{\omega}$ is closed from continuity of $\omega$. We use the product defined by equation \eqref{eq:defprodetat}: $(A,B):=\omega(A^*B)$. It defines a sesquilinear form on the quotient $\cA/\mN_{\omega}$
\begin{equation}
\scal{ VA }{VB}=\omega(A^*B)
\end{equation}
where $\dpt{V}{\cA}{\cA/\mN_{\omega}}$ is the canonical projection $VA=A+\mN_{\omega}$. So $\cA/\mN_{\omega}$ is a pre-Hilbert space from which we build $\hH_{\omega}$ by completion. We define the cyclic vector $\Omega_{\omega}=V\cun\in\hH_{\omega}$.

For each $A\in\cA$, we define $\dpt{L_A}{\cA/\mN_{\omega}}{\cA/\mN_{\omega}}$ by
\begin{equation}  \label{eq:defpiomega}
L_AVB=V(AB).
\end{equation}
We have
\begin{equation}
\begin{aligned}
\| VAB \|^2&=\scal{VAB}{VAB}^2\\
        &=\omega(B^*A^*AB)^2\\
        &\leq \| A \|^4\omega(B^*B)^2&&\textrm{corollary~\ref{cor:BeAAeB}}\\
        &=\| A \|^4\| VB \|^2,
\end{aligned}
\end{equation}
then for all $\psi\in\cA/\mN_{\omega}$, $\| L_A\psi \|\leq \| A \|^2\| \psi \|$. We conclude that
\begin{equation}
  \| L_A \|\leq \| A \|^2.
\end{equation}
Then $L_A$ can be extended to a bounded operator $\pi_{\omega}(A)$ on the whole $\hH_{\omega}$. The map $\dpt{\pi_{\omega}}{\cA}{\oB(\hH_{\omega})}$ is a representation such that for all $A\in\cA$,
\begin{subequations}
\begin{align}
\omega(A)&=\scal{\Omega_{\omega}}{\pi_{\omega}(A)\Omega_{\omega}},\\
\overline{ \pi_{\omega}(A)\Omega_{\omega} }&=\overline{ \cA/\mN_{\omega} }=\hH_{\omega}.
\end{align}
\end{subequations}
It proves point~\ref{GNSi}; we now turn our attention to~\ref{GNSii}. For all $A$, $B\in\cA$, we have
\begin{equation}  \label{eq_r1903r4}
\scal{\pi_{\omega}(B)\Omega_{\omega}}{\pi_{\omega}(A)\Omega_{\omega}}=\scal{\Omega_{\omega}}{\pi_{\omega}(B^*A)\Omega_{\omega}}
        =\scal{V\cun}{V(B^*A)}
        =\omega(B^*A),
\end{equation}
but from assumptions, $\omega(A)=\scal{\Omega}{\pi(A)\Omega}$; then
\begin{equation} \label{eq:BesAOmBeA}
  \omega(B^*A)=\scal{\Omega}{\pi(B^*A)\Omega}
        =\scal{\pi(B)\Omega}{\pi(A)\Omega}.
\end{equation}
We conclude that
\begin{equation}
  \scal{\pi_{\omega}(B)\Omega_{\omega}}{\pi_{\omega}(A)\Omega_{\omega}}=
    \scal{\pi(B)\Omega}{\pi(A)\Omega}.
\end{equation}
From definition of a cyclic vector, $\pi_{\omega}(\cA)\Omega_{\omega}$ is dense in $\hH_{\omega}$ and $\pi(\cA)$ in $\hH$. This allows us to define $\dpt{u}{\hH_{\omega}}{\hH}$ by the condition
\[
  u\pi_{\omega}(A)\Omega_{\omega}=\pi(A)\Omega.
\]
It is a well defined Hilbert space isomorphism because if $\pi_{\omega}(A)\Omega_{\omega}=\pi_{\omega}(B)\Omega_{\omega}$, then equation (true for all $D\in\cA$) $\scal{\pi_{\omega}(B)\Omega_{\omega}}{\pi_{\omega}(D)\Omega_{\omega}}=\scal{\pi(B)\Omega}{\pi(D)\Omega}$ gives an equation of the form $\scal{x}{d}=\scal{y}{d}$ for all $d$ in a dense subset. This equation implies that $x=y$, or $\pi(A)\Omega=\pi(B)\Omega$. We know from general Hilbert space theory that a surjective isometry is unitary; this is the case of $u$.

\end{proof}

For later use, we mention that equation \eqref{eq:BesAOmBeA} gives
\begin{equation}  \label{eq:piomomaesm}
\| \pi_{\omega}(A)\Omega_{\omega} \|=\omega(A^*A)
\end{equation}
when $A=B$.

\begin{corollary}
Let $(\pi_i,\hH_i)$ $(i=1,2)$ be two cyclic representations with cyclic vectors $\Omega_i$. If for all $A\in\cA$,
\[
  \omega_1(A):=\scal{\Omega_1}{\pi_1(A)\Omega_1}=\scal{\Omega_2}{\pi_2(A)\Omega_2}=:\omega_2(A),
\]
then they are equivalent representations.

\end{corollary}

\begin{proof}
The representation $\pi_i$ in $\hH_i$ is cyclic and induces a GNS representation $\pi_{\omega_i}$. Since $\omega_1=\omega_2$, these two GNS representations are the same and $\pi_1$ and $\pi_2$ are thus both equivalent to the same representation.

\end{proof}

The following is just a restatement of~\ref{GNSii} of theorem~\ref{tho:GNS}.
\begin{proposition}
If $(\cA,\pi,\hH)$ is a cyclic representation, then all the GNS constructions build from a vector state are unitary equivalent to $\pi$. \label{prop:cyclequivGNS}
\end{proposition}


\subsection{Universal representation}
%-----------------------------------

\begin{theorem}
Any $C^*$-algebra accepts an isometric representation on a Hilbert space.
\end{theorem}

\begin{proof}
Let $\cA_h$ be the real Banach space build from hermitian elements of $\cA$ and choose a non zero $A\in\cA$. The element $-A^*A$ does not belong to $\cA^+$. Since $\cA^+$ is a convex closed cone, there exists a linear continuous form $f_A$ on $\cA_h$ such that $f_A(B)\geq 0$ for all $B\in\cA^+$ and $f_A(-A^*A)<0$. One can identify $f_A$ to an hermitian form on $\cA$ because $B$ is decomposed as $B=A_1+iA_2$ with $A_1,A_2\in\cA_h$; we define $f_A(B)=f_A(B_1)+f_A(B_2)$. The function $f_A$ is also a positive form on $\cA$ because on any positive element $B^*B$, we have $f_A(B^*B)>0$.

Now we look at $\pi_A$, the representation defined by $f_A$. We have $f_A(B)=\scal{ \pi_A(B)\xi }{ \xi }$, thus
\[
  f_A(A^*A)=\scal{ \pi_A(A^*)\pi_A(A)\xi }{ \xi },
\]
which is zero if $\pi_A(A)=0$. Then $\pi_A(A)\xi\neq 0$ and we conclude that $\pi_A(A)\neq 0$.

We consider $\pi$, the direct sum of all the representations $\pi_A$ for all $A\in\cA$, $A\neq 0$. First we prove that $\pi$ is injective. Indeed if $\pi(B)=0$, we have $\pi_A(B)=0$ for all $A$; in particular
\begin{equation}
  0=\scal{ \pi_A(B)\xi }{ \pi_A(B) }
        =\scal{ \pi_A(B^*B)\xi }{ \xi }
        =f_A(B^*B)
\end{equation}
by~\ref{itemiv_prop_DixGNS} of proposition~\ref{prop_DixGNS}. So if $\pi$ is not invertible, there exists a $B\neq 0$ such that for all $A$, $f_A(B^*B)=0$; this implies $\| B^*B \|=0$ and $B=0$. Contradiction.

The map $\pi$ is isometric as injective morphism: $\| \pi(A) \|=\| A \|$.
\end{proof}


The \defe{universal representation}{universal!representation} $\pi_u$ of a $C^*$-algebra $\cA$ is the direct sum of all the GNS representations $\pi_{\omega}$ with $\omega\in\etS(\cA)$. The representation space is
\[
  \hH_u=\bigoplus_{\omega\in\etS(\cA)}\hH_{\omega}.
\]


\begin{theorem}[Gelfand-Neumark]
A $C^*$-algebra is isomorphic to a subalgebra of $\oB(\hH)$ for a certain Hilbert space $\hH$.
\end{theorem}

\begin{proof}
Let's show that $\hH=\hH_u$ and the isomorphism $\pi_u$ answer the question.

\subdem{Injective}
Let $A\in\cA$ such that $\pi_u(A)=0$; from definition of the direct sum and of universal representation,  $\pi_{\omega}(A)=0$ for all $\omega\in\etS(\cA)$. Using equation \eqref{eq:piomomaesm} we find that for such a $A$, we have $0=\| \pi_{\omega}(A)\Omega_{\omega} \|^2=\omega(A^*A)^2$. It is true for all $\omega\in\etS(\cA)$. Lemma~\ref{lem:omAenomA} then shows that $\| A^*A \|=0$ and then that $A=0$ because of definition of a $C^*$-algebra.

\subdem{Surjective}
The representation $\pi_u$ is not specially surjective on $\oB(\hH_u)$, but it is surjective on the subalgebra $\pi_u(\cA)$ which is enough for the present purpose.

\subdem{Morphism}
The map $\dpt{\pi_u}{\cA}{\oB(\hH_u)}$ is a morphism because it is a representation.


\subdem{Isometry}
Lemma~\ref{lem:injmorpisom} says that an injective morphism of $C^*$-algebra is isometric.

\end{proof}

The universal representation is trivially faithful, but it is very huge. For example the smallest faithful representation of $\oB(\hH)$ is the definition representation on $\hH$.

\begin{corollary}
An operator $A$ is positive if and only if $\pi(A)\geq 0$ for all cyclic representation $\pi$.
\end{corollary}

\begin{proof}
Since $A$ is positive, then $\sigma(A)\subset\eR^+$ and $A^*=A$. In order to prove that $\pi_u(A)$ is positive, we have to show that $\sigma(\pi_u(A))\subset\sigma(A)$. Let $z\in\sigma(\pi_u(A))$: the operator $\pi_u(A)-z\mtu_u$ where $\mtu_u$ is the unit operator on $\hH_u$ is not invertible. From equation  \eqref{eq:defpiomega},  we have $\mtu_u=\pi_u(\mtu)=\sum_{\omega\in\etS(\cA)}\pi_{\omega}(\cun)$.

Let $z\notin\sigma(A)$ and let us see that $z\notin\sigma(\pi_u(A))$. From assumption on $z$, there exists a $B\in\cA$ such that $B(A-z\cun)=(A-z\cun)=\cun$. Then $\pi_u(B)$ is the inverse of $\pi_u(A)-z\pi_u(\cun)$. It proves that $z\notin\sigma(\pi_u(A))$ and so that $\pi_u(A)$ is positive.

We now prove that $\pi_{\omega}(A)$ is positive for all GNS representation $\pi_{\omega}$. Since $\pi_u$ acts separately on each space $\hH_{\omega}$, all what we said about the invertibility about $\pi_u$ can be said for each $\pi_{\omega}$.

But we know that all cyclic representation is equivalent to a GNS representation from proposition~\ref{prop:cyclequivGNS}. We just have to prove that positivity is conserved by equivalence. Suppose that it is not the case. Let $z\in\sigma(\pi_{\omega}(A))$ and $B\in\cA$ such that $B\big( \pi_{\omega}(A)-z\mtu_{\omega} \big)=\mtu_{\omega}$. Then
\[
  UBU^*\big( \pi(A)-z\mtu_{\omega} \big)=\mtu_{\omega}
\]
and then $UBU^*$ is the inverse of $\pi(A)-z\mtu_{\omega}$ and $z\notin\sigma(\pi(A))$. Thus
\[
  \sigma(\pi(A))\subset\sigma(\pi_{\omega}(A))\subset\eR^+
\]
which proves that $\pi(A)$ is positive.

We now prove the second sense of the corollary. All GNS representation is cyclic, then $\pi_u(A)$ is positive as sum of positive representation. We want to deduce that $A\geq0$. Since $\pi_u(A)$ is positive, $\pi_u(A)=\pi_u(A)^*=\pi_u(A^*)$. This implies that $A=A^*$ because $\pi_u$ is injective. The positivity of $\pi_u(A)$ gives the existence of a $B$ such that  $\pi_u(A)=\pi_u(B)^*\pi_u(B)=\pi_u(B^*B)$. The injectivity then shows that $A$ is positive.

\end{proof}

The GNS construction is done for unital algebras with states. Since there exists a notion of state on non unital algebras, ones raises the question to generalization of the GNS construction to non unital algebras.

%%%%%%%%%%%%%%%%%%%%%%%%%%
%
   \section{Spaces of matrices}
%
%%%%%%%%%%%%%%%%%%%%%%%%

Let $\cA$ be a $C^*$-algebra and $n\in\eN$. The $C^*$-algebra $\mfM^n(\cA)$\nomenclature{$\mfM^n(\cA)$}{$C^*$-algebra of matrices} is the space on $n\times n$ matrices with entries in $\cA$.The multiplication is defined by
\begin{equation}
  (MN)_{ij}=\sum_kM_{ik}N_{kj}
\end{equation}
where the product in the right hand side is the (in general noncommutative) one in $\cA$. The involution is naturally given by
\begin{equation}
  (M^*)_{ij}=M_{ij}^*.
\end{equation}
One can identify $\mfM^n(\cA)$ to $\cA\otimes\mfM^n(\eC)$ by identifying\footnote{The matrix $E_{ij}$ is the matrix full of zero except a $1$ at position $ij$.} $E_{ij}\in\mfM^n(\cA)$ to $A\otimes E_{ij}\in\cA\otimes\mfM^n(\eC)$.

The Gelfand-Neumark gives the existence of a faithful representation $\pi$ of $\cA$ on $\hH$. If one sees elements of $\hH\otimes\eC^n$ as $n$-uples $(v_1,\ldots,v_n)$ where each $v_i\in\hH$, we can define $\pi_n$ on $\hH\otimes\eC^n$ by linear extension of
\begin{equation} \label{eq:defreprezmfM}
  \big[\pi_n(M)v\big]_i:=\pi(M_{ij})v_j.
\end{equation}
The norm $\| M \|$ is defined as the norm of $\pi_n(M)$. Since $n<\infty$, $\pi_n(\mfM^n(\cA))$ is a closed $*$-algebra in $\oB(\hH\otimes\eC^n)$ and then $\mfM^n(\cA)$ is a $C^*$-algebra for this norm. Proposition~\ref{prop:unicitenormcsa} states that the norm is unique and then that the norm $\| M \|=\pi_n(M)$ is independent of the choice of $\pi$.

\begin{definition}      \label{DefComplPositive}
    A linear map $\dpt{ q }{ \cA }{ \cB }$ is \defe{completely positive}{positive!completely!map between $C^*$-algebra}\index{completely!positive} if for all $n\in\eN$, the map $\dpt{ q_n }{ \mfM^n(\cA) }{ \mfM^n(\cB) }$ defined by
    \[
      (q_n(M))_{ij}=q(M_{ij})
    \]
    is positive.
\end{definition}
As an example, a morphism $\varphi$ is always completely positive because if $a=b^*b$ in $\mfM^n(\cA)$, then $\varphi(a)=\varphi(b)^*\varphi(b)$ which is positive in $\mfM^n(\cB)$.

%%%%%%%%%%%%%%%%%%%%%%%%%%
%
   \section{Stinespring theorem}
%
%%%%%%%%%%%%%%%%%%%%%%%%

Let us state a classical result about Hilbert space

\begin{lemma}
If $K$ is closed in a Hilbert space $\hH$ and if the linear operator $\dpt{ T }{ \hH }{ \hH }$ fulfils

\begin{itemize}
\item $\scal{ Tx }{ y }=\scal{ x }{ Ty }$ for all $x$, $y\in\hH$,
\item $Ty=y$ for all $y\in K$
\item $Tz=0$ for all $z\in K^{\perp}$,
\end{itemize}
then $T$ is the projection on $K$.

\end{lemma}
This lemma allows us to check that $WW^*$ is the projection to the image of $W$. Let us prove that $W^*W$ is the projection on $K_1$. The first condition is clear. The third is satisfied by definition of $W$: if $z\in K_1^{\perp}$, then $W^*Wz=0$. For the second one, remark that if $y\in K_1$, $\scal{ W^*Wx }{ y }=\scal{ x }{ y }$ and if $y\in K_1^{\perp}$, then $\scal{ W^*Wx}{ y }=0=\scal{ x }{ y }$. In both cases $y\in K_1$ and $y\in K_1^{\perp}$, we have $\scal{ W^*Wx }{ y }=\scal{ x }{ y }$. It is sufficient to conclude that $W^*Wx=x$ because $\hH=K_1\oplus K_1^{\perp}$.

\begin{theorem} \index{Stinespring theorem}\index{theorem!Stinespring}
   Let $\dpt{ q }{ \cA }{ \cB }$ be a completely positive map between unital  $C^*$-algebra such that $q(\cun)=\cun$. We suppose that $\cB$ is given with a faithful representation $\cB\simeq\pi_{\chi}(\cB)\subseteq\oB(\hH_{\chi})$ for a certain Hilbert space $\hH_{\chi}$. Then there exists a Hilbert space $\hH^{\chi}$, a representation $\pi^{\chi}$ of $\cA$ on $\hH^{\chi}$ and a partial isometry $\dpt{ W }{ \hH_{\chi} }{ \hH^{\chi} }$ with $W^*W=\cun$ and
\begin{equation}  \label{eq:stinun}
  \pi_{\chi}(q(A))=W^*\pi^{\chi}(A)W
\end{equation}
for all $A\in\cA$.

Stated in an equivalent way, if we define $P=WW^*$ and $\tilde\hH_{\chi}=P\hH^{\chi}\subset\hH^{\chi}$, and $\dpt{ U }{ \hH_{\chi} }{ \tilde\hH_{\chi} }$ as the restriction of $W$ (in such a way that $U$ is unitary because $W$ is a partial isometry), then we have
\begin{equation} \label{eq:stindeux}
  U\pi_{\chi}(q(A))U^{-1}=P\pi^{\chi}(A)P.
\end{equation}
\label{tho:stinespring}
\end{theorem}


\begin{proof}
On $\hH_{\chi}$ we have the scalar product $\scal{.}{.}_{\chi}$ and we define the sesquilinear form $\scal{.}{.}_0^{\chi}$ on $\cA\otimes\hH_{\chi}$ by sesquilinear extension of
\[
  (A\otimes v,B\otimes w)_0^{\chi}=\scal{v}{\pi_{\chi}(q(A^*B))w}_{\chi}.
\]
 \subdem{This form is semi positive definite}
Let us compute
\begin{equation}
\sum_{ij}(A_i\otimes v_i,A_j\otimes v_j)_0^{\chi}=\sum_{ij}\scal{v_i}{ \pi(q(A_j^*A_j))v_j }_{\chi}.
\end{equation}
For this, we consider $a\in\mfM^n(\cA)$ with elements $a_{ij}=A_i^*A_j$. Let $\pi_n$ be the faithful representation \eqref{eq:defreprezmfM}  of $\mfM^n(\cA)$ on $\hH\otimes \eC^n$ defined by
\[
  [\pi_n(M)v]_i=\sum_j\pi(M_{ij})v_j\in\hH.
\]
where $\pi$ is a faithful representation of $\cA$. If $z\in\eC^n\otimes\hH$, we can define $(az)\in\eC^n\otimes\hH$ by
\[
  (az)_i=\big( \pi_n(a)z \big)_i.
\]
So we have
\begin{equation}
 \scal{ z }{ az }=\sum_{ij}\scal{ z_i }{ \pi(a_{ij})z_j }
        =\sum_{ij}\scal{ \pi(A_i)z_i }{ \pi(A_j)z_j }
        =\| Az \|^2\geq0
\end{equation}
where we use the notation $Az=\sum_i\pi(A_i)z_i\in\hH$. The conclusion is that $a\geq0$. Since $q$ is completely positive, $b$ is positive if we define $b_{ij}=q(A_i^*A_j)$. Positivity of $b$ gives rise to an element $c\in\mfM^n(\cB)$ such that $n=c^*c$ where $(c^*)_{ij}=(c_{ji})^*$. From the representation $\pi_{\chi}$ of $\cB$, we can build the faithful representation $\pi'_n$ of $\mfM^n(\cB)$ on $\eC^n\otimes\hH_{\chi}$: $[\pi'_n(b)v]_i=\sum_j\pi_{\chi}(b_{ij})v_j$ where each $v_i$ now belongs to $\hH_{\chi}$.

We are now able to prove that $\scal{ . }{ . }_0^{\chi}$ is a positive form. Indeed
\begin{equation}
\begin{split}
\sum_{ij}  ( A_i\otimes v_i , A_j \otimes v_j )_0^{\chi}
        &=\sum_{ij} \scal{ v_i }{ \pi_{\chi}\big( q(A_i^*A_j) \big)v_j }_{\chi}
        =\sum_{ij} \scal{ v_i }{ \pi_{\chi}(b_{ij})v_j }_{\chi}\\
        &=\sum_{ijk}\scal{ \pi_{\chi}(c_{ki})v_i }{ \pi_{\chi}(c_{kj})v_j }_{\chi}\geq0
\end{split}
\end{equation}
where the last equality comes from the fact that $(c^*c)_{ij}=\sum_k(c^*)_{ik}c_{kj}=\sum_k(c_{ki})^*c_{kj}$.  This proves that $( . , . )_0^{\chi}$ is positive semi definite.

\subdem{Definition of $\pi^{\chi}$.}

Now we denote by $\mN_{\chi}$ the null space of $( . , . )_0^{\chi}$. Let $\dpt{ V_{\chi} }{ \cA\otimes\hH_{\chi} }{ \cA\otimes\hH_{\chi}/\mN_{\chi} }$ be the canonical projection. We define
\begin{equation}
\scal{ V_{\chi}(A\otimes v) }{ V_{\chi}(B\otimes w) }^{\chi}:=( A\otimes v , B\otimes w )_0^{\chi}
\end{equation}
and we denote by $\hH^{\chi}$ the closure of $\cA\otimes\hH_{\chi}/\mN_{\chi}$ with respect to this scalar product. We can now define $\pi^{\chi}$, a representation of $\cA$ on $\cA\otimes\hH_{\chi}/\mN_{\chi}$ by linear extension of
\begin{equation}
  \pi^{\chi}(A)V_{\chi}(B\otimes w)=V_{\chi}(AB\otimes w)
\end{equation}
which is well defined because $\pi^{\chi}(A)\mN_{\chi}\subseteq\mN_{\chi}$.


Let us now prove that $\| \pi^{\chi}(A) \|\leq \| A \|$. From equation \eqref{cor:BeAAeB} used in $\mfM^n(\cA)$ with $B=\cun_n$,we know that
\begin{equation} \label{eq:r502061}
  0\leq A^*A\cun_n\leq \| A \|^2\cun_n.
\end{equation}
Now we consider any $B_1,\cdots,B_n\in\cA$ and we build the matrix
\[
  b=
\begin{pmatrix}
B_1&\cdots&B_n\\
0&\cdots&0\\
\vdots&\ddots&\vdots\\
0&\cdots&0
\end{pmatrix},\quad
  b^*=
\begin{pmatrix}
B_1^*  & 0      & \ldots&0\\
\vdots &\vdots  & \ddots&\vdots\\
B^*_n  &0&\ldots & 0
\end{pmatrix}.
\]
We conjugate \eqref{eq:r502061} with $b$:
\[
  0\leq b^*A^*Ab\leq \| A \|^2b^*b,
\]
but $q$ is completely positive, then it respects the inequality:
\[
  q_n(b^*A^*Ab)\leq \| A \|^2q_n(b^*b)
\]
where $\dpt{ q_n }{ \mfM^n(\cA) }{ \mfM^n(\cB) }$ is defined by $\big( q_n(M) \big)_{ij}=q(M_ij)$. The definition of complete positivity of $q$  is precisely positivity of $q_n$. We consider now the representation $\pi_{\chi}$ of $\cB$ on $\hH_{\chi}$. Since $a\geq0$, we can find a $e\in\mfM^n(\cB)$ such that $q_n(a)=e^*e$; then
\begin{equation}
\begin{split}
\sum_{ij}\scal{ v_i }{ \pi_{\chi}\big( q(a_{ij})v_j \big) }&=\sum_{ij}\scal{ v_i }{ \pi_{\chi}(e^*e)_{ij}v_j }\\
        &=\sum_{ijk}\scal{ v_i }{ \pi_{\chi}(e^*_{ki})\pi_{\chi}(e_{kj} }\\
        &=\sum_{ijk}\scal{ \pi_{\chi}(e)_{ki}v_i }{ \pi_{\chi}(e_{ij})v_j }\geq0.
\end{split}
\end{equation}

Let us consider $\Psi=\sum_iV_{\chi}B_i\otimes v_i$ and compute
\begin{equation}
\begin{split}
\| \pi^{\chi}(A)\Psi \|^2&=\sum_{ij}( AB_i\otimes v_i , AB_j\otimes v_j )_0^{\chi}\\
        &=\sum_{ij}\scal{ v_i }{ \pi_{\chi}\big( q(B_i^*A^*AB_j) \big)v_j }_{\chi}\\
        &\leq \| A \|^2\sum_{ij}\scal{ v_i }{ \pi_{\chi}\big( q(B^*_iB_j) \big)v_j }_{\chi}\\
        &=\| A \|^2\sum_{ij}\scal{ B_i\otimes v_i }{ B_j\otimes v_j }_{\chi}\\
        &=\| A \|^2( V_{\chi}\sum_i B_i\otimes v_i , V_{\chi}\sum_jB_j\otimes v_j )^{\chi}\\
        &=\| A \|^2\| \Psi \|^2.
\end{split}
\end{equation}
Then $\| \pi^{\chi}(A)\Psi \|\leq \| A \|^2\| \Psi \|$ which proves that
\begin{equation}
\| \pi^{\chi}(A) \|\leq\| A \|^2.
\end{equation}
The formula
\begin{equation}
  \pi^{\chi}(A)V_{\chi}(B\otimes w)=V_{\chi}(AB\otimes w)
\end{equation}
defines a continuous representation $\pi^{\chi}$ on $\cA\otimes\hH_{\chi}/\mN_{\chi}$ which can be extended to a continuous representation on the whole $\hH^{\chi}$. This extension fulfils $\pi^{\chi}(A^*)=\pi^{\chi}(A)^*$.

We define $\dpt{ W }{ \hH_{\chi} }{ \hH^{\chi} }$ by
\begin{equation}
  Wv=V_{\chi}\cun\otimes v.
\end{equation}

\subdem{The map $W$ is a partial isometry}

In order to prove that $W$ is a partial isometry, just compute
\begin{equation}
\begin{split}
( Wv , Ww )^{\chi}&=( V_{\chi}\cun\otimes v , V_{\chi}\cun\otimes w )^{\chi}\\
        &=( \cun\otimes v , \cun\otimes w )_0^{\chi}\\
        &=\scal{ v }{ w }_{\chi}.
\end{split}
\end{equation}

\subdem{Adjoint of $W$}

We claim that $\dpt{ W^* }{ \hH^{\chi} }{ \hH_{\chi} }$, $W^*V_{\chi} A\otimes v=\pi_{\chi}(q(A))v$ is the adjoint of $W$. Recall that the definition of the adjoint requires that
\[
  \scal{ w }{ W^*\psi }_{\chi}=( Ww , \psi )^{\chi}
\]
for all $w\in\hH_{\chi}$ and all $\psi\in\hH^{\chi}=\overline{ A\otimes\hH_{\chi}/\mB_{\chi} }$. A $\psi\in\hH^{\chi}$ can be written under the form $V_{\chi}A\otimes v$ with $A\in\cA$ and $v\in\hH_{\chi}$; then
\begin{equation}
\begin{split}
\scal{ w }{ W^*V_{\chi}A\otimes v }_{\chi}&=\scal{ w }{ \pi_{\chi}(q(A))v }_{\chi}\\
        &=( \cun\otimes w , A\otimes v )_0^{\chi}\\
        &=( V_{\chi}\cun\otimes w , V_{\chi}A\otimes v )^{\chi}\\
        &=\scal{ Ww }{ \psi }
\end{split}
\end{equation}
as expected. One can check that $W^*W=\mtu$ and that $W^*\pi^{\chi}(A)W=\pi_{\chi}(q(A))$ because
\begin{equation}
  W^*\pi^{\chi}(A)Wv=W^*\pi^{\chi}(A)V_{\chi}(\cun\otimes v)
        =W^*V_{\chi}(A\otimes v)
        =\pi_{\chi}(q(A)).
\end{equation}

\subdem{Last point: \eqref{eq:stinun}$\Rightarrow$\eqref{eq:stindeux}}

Since $W$ is a partial isometry, $P=WW^*$ is the projection on the image of $W$ and $W^*W$ ($=\cun$) is the projector on the subspace of $\hH_{\chi}$ on which $W$ is isometric; this space is $\hH_{\chi}$ itself. The $\tilde\hH_{\chi}=P\hH^{\chi}=\hH^{\chi}$ and $U=W$. Then
\begin{equation}
  U\pi_{\chi}\big( q(A) \big)U^{-1}=W\pi_{\chi}\big( q(A) \big)W^*
        =WW^*\pi^{\chi}(A)WW^*
        =P\pi^{\chi}(A)P
\end{equation}
This concludes the proof of theorem~\ref{tho:stinespring}.

\end{proof}


\begin{remark}
If on the one hand $q(\cun)$ is not $\cun$, then the construction works, but $W$ is no more a partial isometry and we have
\[
  \| W \|^2=\| q(\cun) \|,
\]
so $\hH_{\chi}$ can not be seen as a subspace of $\hH^{\chi}$ by the map $\dpt{W}{ \hH_{\chi} }{ \hH^{\chi} }$.

If on the other hand $\cA$ or $\cB$ is not unital, then works if $q$ can be extended (keeping positive) to the unitization of $\cA$ in such a way that it conserves the identity in (the unitization of ) $\cB$.
\end{remark}

\begin{proposition}
Any positive map $q\colon \cA\to \cB$ from a commutative unital $C^*$-algebra $\cA$ is completely positive.
\end{proposition}

The proof will be decomposed into several propositions. Let us begin by a remark: from theorem~\ref{thoGelfand}, we can write $\cA=C(X)$ for a certain locally compact Hausdorff space $X$. So we can identify $\mfM^n(C(X))$ with $C(X,\mfM^n(\eC))$ because to each $a\in\mfM^n(C(X))$, ($a_{ij}$ is a map $\dpt{ a_{ij} }{ X }{ \eC }$) we make correspond the map $\dpt{ \eta }{ X }{ \mfM^n(\eC) }$ defined by $\eta(x)_{ij}=a_{ij}(x)$.


\begin{proposition}
The set of finite linear combinations of elements $F$ of the form
\[
  F(x)=\sum_i f_i(x)M_i
\]
with $f_i\in C(X)$ and $M_i\in\mfM^n(\eC)$ is dense in $C(X,\mfM^n(\eC))$.
 \label{prop:lencombpart}
\end{proposition}


\begin{proof}
Let $G\in C(X,\mfM^n(\eC))$ and $\varepsilon>0$. Continuity of $G$ makes the set
\[
  \mO_x^{\varepsilon}=\{ y\in X\tq \| G(x)-G(y) \|\leq\varepsilon \}
\]
open for all $x\in X$. These set give an open covering of the compact space $X$, then we can extract a finite subcovering and build an unity partition $\varphi_i$. We define $F_l\in C\big( X,\mfM^n(\eC) \big)$ by
\begin{equation} \label{eq:Fllim}
  F_l(x)=\sum_{i=1}^l\varphi_i(x)G(x_i).
\end{equation}
We have
\begin{equation}
\| F_l(x)-G(x) \|=\| \sum_{i=1}^l\varphi(x)( G(x_i)-G(x) ) \|
        \leq \sum \varphi_i(x)\| G(x_i)-G(x) \|
        \leq \sum\varphi_i\varepsilon
    =\varepsilon
\end{equation}
Then $\| F_j-G \|=\sum_{x\in X}\| F_l(x)-G(x) \|\leq\varepsilon$ and the sequence $F_l$ converges to $G$. It proves the density.
\end{proof}


\begin{proposition}
When $\{ M_i \}$ is a basis of $\mfM^n(\eC)$ composed with positive elements, an element $F\in C(X,\mfM^n(\eC))$ of the form $F(x)=\sum_if_i(x)M_i$ is positive if and only if it each of $f_i$ is positive.
\end{proposition}

\begin{proof}
We know that $F\in C(X,\mfM^n(\eC))$ is positive when $F(x)$ is positive in $\mfM^n(\eC)$ for all $x\in X$. We have $F(x)=\sum_if_i(x)M_i$, but positivity of $F(x)$ requires $F(x)=F(x)^*$ and then $f_i(x)=f_i(x)^*$ because $M_i$ is positive.

\end{proof}

\begin{probleme}
  \cite{Landsman} states a stronger result that seems wrong to me because \( -3+7\) is positive.
\end{probleme}

\begin{proposition}
If $G\in C(X,\mfM^n(\eC))$ is positive, then there exists a sequence $F_k\geq 0$ such that $\lim_{k\to\infty}F_k=G$
\end{proposition}

\begin{proof}
Each element $F_l$ \eqref{eq:Fllim} is positive because  $G(x_i)$ is positive for all $x_i$.
\end{proof}

\begin{probleme}
    There are too much unclear thinks in my mind; I do not finish the proof.
\end{probleme}

The main result is the following proposition.

\begin{proposition}
If $\cA$ is a commutative unital $C^*$-algebra, then any positive map $\dpt{ q }{ \cA }{ \cB }$ is completely positive.
\end{proposition}


\section{Representations}
%+++++++++++++++++++++++++

As notational convention, when $H$ is a Hilbert space, we denote by $\mL(H)$\nomenclature{$\mL(H)$}{Space of continuous endomorphisms of $H$} the set of the continuous endomorphism of $H$. Topology on $\mL(H)$ is discussed in subsection~\ref{subsec_topomL}.

When $\pi$ is a representation of $\cA$ in $H$ and $\xi\in H$, the space $\overline{ \pi(\cA)\xi }$ is a closed subspace of $H$ stable under $\pi(\cA)$.

\begin{definition}
A vector $\xi\in H$ is said \defe{totalizing}{totalizing vector} for a
representation $\pi$ of $\cA$ if $\overline{ \pi(\cA)\xi }=H$.
\end{definition}

\begin{proposition}
Let $\cA$ be an involutive algebra, $H$ an hermitian space and $\pi$ a representation of $\cA$ in $H$. Then the following facts are equivalent
:
\begin{enumerate}
\item The only closed subspace in $H$ which are stable for $\pi(\cA)$
are $\{o\}$ and $H$.
\item The subset of $\mL(H)$ which commutes with $\pi(\cA)$ is reduced
to $\mC$.
\item Any non zero vector in $H$ is totalizing for $\pi$, or $\pi$ have
dimension $1$.
\end{enumerate}
 \label{prop:reprez_topo}
\end{proposition}

\begin{proof}
We begin proving that (ii) implies (iii). Let $\xi\in H$, $\xi\neq 0$.
If $\pi(\cA)\xi$ is not everywhere dense in $H$, (i) makes
$\pi(\cA)\xi=0$. Then $\eC\xi$ is stable under $\pi(\cA)$. But
$\eC\xi\neq\{o\}$, then $\eC\xi=H$. Thus $H$ has dimension $1$ and $\pi$
is the null representation.

Now, we prove that (iii) implies (i). Let $K\neq\{o\}$ be a closed
vector
subspace of $H$ stable under $\pi(\cA)$. We have to show that $K=H$. If
$\dim H=1$, it is obvious. Let us consider a non zero  $\xi\in K$. Since
$K$ is stable, $\pi(\cA)\xi\subset K$, but (iii) implies
 $\overline{ \pi(\cA)\xi }=H$. Thus $K=H$.

We turn our attention to the equivalence between (i) and (ii). First
(ii) implies (i). We consider $K$, a vector subspace of $H$ stable under
$\pi(\cA)$. We want $K=\{o\}$ or $K=H$. Let us consider
$\dpt{p_K}{H}{K}$, the orthogonal projection. Since $K$ is a vector
subspace, it makes sense to write $H\ominus K$.

Let us consider $\xi\in K$ and $\eta\in H\ominus K$. For any $A\in\cA$,
$\pi(A^*)\xi\in K$ because $\pi(A)\xi\in K$. This yields
\[
   \langle \pi(A)\eta|\xi\rangle =\langle\eta|\pi(A^*)\xi\rangle=0,
\]
then $\pi(A)\eta\in H\ominus K$. We can conclude that
\[
   [p_K,\pi(\cA)]=0
\]
because $\pi(\cA)p_K\xi=\pi(\cA)\xi=p_K\pi(\cA)\xi$, $\pi(\cA)p_K\eta=0$  and $p_K\pi(\cA)\eta=0$ from $\langle \pi(\cA)\eta|\xi\rangle =0$.

From (ii), $p_K$ is then a scalar operator: $p_K=0$ or $p_K=id$, \emph{i.e.} $K=\{o\}$ or $K=H$.

Finally, we show the implication from (i) to (ii). Let $T$ be an element of $\mL(H)$ which commute with the whole $\pi(\cA)$; we have to show that $T$ is scalar. Since it is clear that $T+T^*$ and $T-T^*$ also commute with $\pi(\cA)$, we can suppose $T=T^*$.

We had shown that a projector $p_K$ commutes with $\pi(\cA)$ if $K$ is stable under $\pi(\cA)$, closed and a sub vector space of $H$. This is the case of the eigenspaces because $(T-\lambda\mtu)v=0$, then $(T-\lambda\mtu)\pi(\cA)v=0$ because $[T,\pi(\cA)]=0$.

The spectral projector of $T$ commute with $\pi(\cA)$. Thus, the eigenspaces $H_{\lambda}$ are stable under $\pi(\cA)$, thus (by (i)) these are only $\{o\}$ and $H$. In other words $(T-\lambda\mtu)v=0$ has solutions which are on two subspaces whose projectors are $0$ and $1$. On the space on which $p_{\lambda}=1$, $T=\lambda\mtu$ and the one where $p_{\lambda}=0$ is $\{0\}$ then $T=0$.

\begin{probleme}
    We have to show that every $v$ belong to one of these two spaces. Is it because $T$ is hermitian, or do we need the compact assumption?
\end{probleme}

\end{proof}

\begin{definition}
Let $\cA$ be an involutive algebra, $H$ a hermitian space and $\pi$ a representation of $\cA$ in $H$. We say that $\pi$ is \defe{topologically irreducible}{irreducible!topologically} if it fulfils proposition~\ref{prop:reprez_topo}. The representation $\pi$ is \defe{algebraically irreducible}{irreducible!algebraically} if the only stable vector subspaces of $H$ under $\pi(\cA)$ are $\{0\}$ and $H$.
\end{definition}

When $\dim H=\infty$, the second notion is stronger because it excludes the case where one has an \emph{open} stable subspace. Point 2.8.4 in \cite{Dixmier} shows that in the case of $C^*$-algebras, a topologically irreducible representation is automatically algebraically irreducible, so one can simply speak about irreducible representations.

%%%%%%%%%%%%%%%%%%%%%%%%%
%
   \section{Pure states}
%
%%%%%%%%%%%%%%%%%%%%%%%%


A subset $C$ of a vector space is \defe{convex}{convex} if for all $\lambda\in[0,1]$ and $v$, $w\in C$, the element $\lambda v+(1-\lambda)w$ belongs to C.

An \defe{extreme point}{extreme point} of a convex set $K$ is an element $\omega\in K$ which can de written under the form $\omega=\lambda\omega_1+(1-\lambda)\omega_2$ with $\lambda\in[0,1]$ only for $\omega_1=\omega_2=\omega$.

\begin{definition}
An extreme point of the state space $\etS(\cA)$ is a \defe{pure state}{state!pure}\index{pure state} and states that are not pure are \defe{mixed states}{mixed!state}\index{state!mixed}.
\end{definition}

The set of extreme points of the convex set $K$ is denoted by $\partial_cK$ and is called the \defe{extreme boundary}{boundary!extreme} of $K$.  An extreme point in the state space $\etS(\cA)$ of a $C^*$-algebra is a \emph{pure state} and other states are \defe{mixed states}{mixed!state}\index{state!mixed}. As notation, $\partial_c(\etS(\cA))$ is denoted by $\etP(\cA)$ or simply $\etP$ when there are no ambiguity.

\subsection*{Example \texorpdfstring{$\cA=\eC\oplus\eC$}{A=C+C}}
 Points are given by $(\lambda,\mu)$. Let's consider the convex set of states $r\in[0,1]$ defined by $r(\lambda,\mu)=(1-r)\lambda-r\mu$.  Extreme points are given by $0$ and $1$: $0(\lambda, \mu)=\lambda$ and $1(\lambda,\mu)=\mu$.

\subsection*{Example: \texorpdfstring{$\cA=\mfM^2(\eC)$}{A=M2C}}
We can identify $\mfM^2(\eC)$ with its dual in the following way. A linear form $\omega$ on $\cA$ can always be written as
\[
  \omega
\begin{pmatrix}
 A_{11}&A_{12}\\
A_{21}&A_{22}
\end{pmatrix}
=\omega_{11}A_{11} +\omega_{12}A_{21}+ \omega_{21}A_{12}+ \omega_{22}A_{22}
\]
So to each form $\omega$, one can associate the matrix of $\omega_{ij}$ and the following holds:
\[
  \omega(A)=\tr(\omega A)
\]
The identification between $\mfM^2(\eC)$ and its dual is then well given by $\omega\simeq(\omega_{ij})$. Let us see in terms of this identification the set $\etS(\cA)$. The condition $\omega(A)\geq 0$ imposes to the matrix $(\omega_{ij})$ to be positive and the condition $\omega(\cun)=1$ imposes $\tr\omega=1$. Then $\etS(\cA)$ is parametrized by
\[
  \rho=\frac{ 1 }{2}
\begin{pmatrix}
1+x&y+iz\\
y-iz&1-x
\end{pmatrix}
\]
where $x$, $y$, $z\in\eR$. The pure states are such matrices $\rho$ with $x^2+y^2+z^2=1$.

If $M$ is a $*$-algebra in $\oB(\hH)$, the \defe{commutant}{commutant} $M'$ is
\[
  M'=\{ A\in\oB(\hH)\tq [A,m]=0    \forall\, m\in M \}.
\]

\begin{proposition}
The following properties are equivalent:

\begin{enumerate}
 \item \label{enumgz} The representation $\pi(\cA)$ is irreducible in $\hH$.
\item\label{enumgi} The commutant of $\pi(\cA)$ in $\oB(\hH)$ is
\[
  \pi(\cA)'=\{ \lambda\cun\tq\lambda\in\eC \}.
\]
\item \label{enumgii} $\pi(\cA)''=\oB(\hH)$.

\item \label{enumgiii} Each vector $\Omega\in\hH$ is cyclic for $\pi(\cA)$.

\end{enumerate}
 \label{prop_equiv_rep_irred}
\end{proposition}

\begin{proof}
\ref{enumgii}$\Rightarrow$\ref{enumgi}
We have to see that an operator which commutes with the whole $\oB(\hH)$, then it is a multiple of identity. If $[a,A]=0$ for all $A\in\oB(\hH)$, then it commutes in particular with an operator $A$ which leaves the basis vector $e_{\beta}$ (and only this basis vector). In this case, $ae_{\beta}=\lambda_{\beta}e_{\beta}$. We conclude that $a$ must be diagonal. Since $a$ must also commute with an operator which leaves $e_{\alpha}+e_{\beta}$ unchanged, we conclude that $\lambda_{\alpha}=\lambda_{\beta}$, so that $a=\lambda\mtu$.

\ref{enumgz}$\Rightarrow$\ref{enumgi} Will be done later.
\ref{enumgi}$\Rightarrow$\ref{enumgz} Let us suppose that $\pi(\cA)'=\eC\cun$ and that $\pi$ is irreducible; we will find out a contradiction. We have a non trivial subspace of $\sH$ stable under $\pi(\cA)$. The projection operator on this space commutes with the whole $\pi(\cA)$ although it is not a multiple of identity.

\ref{enumgz}$\Rightarrow$\ref{enumgiii} We proceed by contradiction once again.  Let $\psi\in\hH$ such that $\pi(\cA)\psi$ is not dense in $\hH$ and $P$ be the projection on the closure of $\pi(\cA)\psi$. Lemma~\ref{lem_preGNS} assures that $P\in\pi(\cA)'$. Consequently $\pi(\cA)'\neq \eC\cun$ and $\pi(\cA)$ is not irreducible by~\ref{enumgi}.

\ref{enumgiii}$\Rightarrow$\ref{enumgz} If $\psi$ belongs to a (non trivial) invariant subspace, then $\pi(\cA)\psi$ cannot be dense because it is a proper subspace of $\hH$.

\end{proof}

\begin{probleme}
    There are still unfinished points in that proof.
\end{probleme}


Let us point out the following part of the proposition:

\begin{lemma}[Schur's lemma]
The representation $\pi$ on $\cA$ is irreducible if and only if the commutant of $\pi(\cA)$ in $\oB(\hH)$ is
 \[
  \pi(\cA)'=\{ \lambda\cun \}_{\lambda\in\eC}.
\]

\end{lemma}


\begin{lemma}
Let $\hat Q$ be a bounded quadratic form on a Hilbert space $\hH$. There exists a bounded operator $Q$ on $\hH$ such that for each $\psi,\phi\in\hH$ we have
\[
\hat Q(\psi,\phi)=\scald{ \psi}{Q\phi }
\]
 and $\| Q \|\leq C$ where $C$ is the ``bounding constant'': $| \hat Q(\psi,\phi) |\leq \| \psi \|\| \phi \|$.
Moreover if
%
\begin{equation}  \label{eq_r19032}
\hat Q(\phi,\psi)=\overline{ \hat Q(\psi,\phi) }
\end{equation}
 the operator $Q$ will be selfadjoint.
\label{lem_r19031}
\end{lemma}

\begin{proof}
Let us fix a $\psi\in\hH$ and look at the map $\phi\mapsto\hat Q(\psi,\phi)$. It is a bounded form, so Riesz theorem gives the existence of a $\Omega\in\hH$ such that $\hat Q(\psi,\phi)=\scald{ \Omega }{ \phi }$. We can define $Q$ by $Q\psi=\Omega$. It is clear that is is self-adjoint if equation \eqref{eq_r19032} is satisfied.

From equalities $\hat Q(\psi,\phi)=\scald{ \Omega }{ \phi }=\scald{ Q\psi }{ \phi }$, we find
 \begin{equation}
\begin{split}
\| Q\psi \|^1&=| \scald{ Q\psi }{ Q\psi } |\\
        &=| \hat Q(Q\psi,\psi) |\\
        \leq \| Q\psi \|\| \psi \|\\
        &=\| Q \|\| \psi \|^2.
\end{split}
\end{equation}
Taking supremum on $\| \psi \|=1$, we find $\| Q \|^2\leq C\| Q \|$ and then
\[
  \| Q \|\leq C.
\]

\end{proof}

\begin{theorem}
   The GNS representation $\pi_{\omega}$ of a state $\omega\in\etS(\cA)$ is irreducible if and only if $\omega$ is pure.
\label{tho_GNS_irred_pure}
\end{theorem}

\begin{proof}
Let us begin to suppose that $\omega$ is a pure state and that $\pi_{\omega}$ is reducible. Then the projection $P$ onto the invariant subspace $K$ of $\pi(\cA)$ belongs to $\pi(\cA)'$ (see the proof of Schur's lemma). Let $\Omega_{\omega}$ be the cyclic vector of $\pi_{\omega}$.

If $P\Omega_{\omega}=0$, then for each $A\in\cA$ we have
\[
  0=\pi_{\omega}P\Omega_{\omega}=P\pi_{\omega}(A)\Omega_{\omega}.
\]
Since $\Omega_{\omega}$ is cyclic, it proves that $P=0$ which is impossible if $\pi_{\omega}$ is reducible. For the same reason, $P^{\perp}\Omega_{\omega}$ is neither not possible because it should implies that $P=\cun$.

We define the two following states on $\cA$:
\begin{subequations}
\begin{align}
  \psi(A)&=k\scald{ P\Omega_{\omega} }{ \pi(A)P\Omega_{\omega} }\\
  \psi^{\perp}(A)&=l\scald{ P^{\perp}\Omega_{\omega} }{ \pi_{\omega}(A)P^{\perp}\Omega_{\omega} }
\end{align}
\end{subequations}
From definition of a projection, we have $\scald{ Px }{ Py }=\scald{ P^*Px }{ y }=\scald{ Px }{ y }$. Taking any $k$, $\lambda=1/k$ and $l=1/(1-1/k)$, using the relation $\omega(A)=\scald{ \Omega_{\omega} }{ \pi_{\omega}(A)\Omega_{\omega} }$ we find
\[
  \omega=\lambda\psi+(1-\lambda)\psi^{\perp},
\]
which is in contradiction with the fact that $\omega$ is pure.

We now suppose that $\pi_{\omega}$ is irreducible, and that $\omega$ reads
\[
  \omega=\lambda\omega_1+(1-\lambda)\omega_2
\]
with $\lambda\in[0,1]$ and $\omega_1,\omega_2\in\etS(\cA)$. We will prove that $\omega_1$ is proportional to $\omega$, so that $\omega$ is pure. Since elements of $\etS(\cA)$ are positive and $(1-\lambda)\geq0$, the form $\omega-\lambda\omega_1=(1-\lambda)\omega_2$ is positive. Therefore for all $A\in\cA$, we have $\lambda\omega_1(A^*A)\leq\omega(A^*A)$. From equation \eqref{eq:omABleq}, we find
%
\begin{equation} \label{eq_r19031}
\begin{split}
| \lambda\omega_1(A^*B) |&\leq\lambda^2\omega_1(A^*A)\omega_1(B^*B)\\
        &\leq\omega(A^*A)\omega(B^*B).
\end{split}
\end{equation}
It allows us to define a quadratic form $\hat Q$ on $\pi_{\omega}(\cA)\Omega_{\omega}$ by
\begin{equation}
\hat Q\big( \pi_{\omega}(A)\Omega_{\omega},\pi_{\omega}(B)\Omega_{\omega} \big):=\lambda\omega_1(A^*B).
\end{equation}

\subdem{$\hat Q$ is well defined}

We have to prove that $\pi_{\omega}(A_1)=\pi_{\omega}(A_2)\Omega_{\omega}$ implies
\[
 \hat Q(\pi_{\omega}(A_1)\Omega_{\omega},\cdot)=\hat Q(\pi_{\omega}(A_2)\Omega_{\omega},\cdot).
\]
 Equation \eqref{eq:piomomaesm} gives
\[
  \| \pi_{\omega}(A)\Omega_{\omega} \|^2=\omega(A^*A)
\]
and makes that $\omega\big( (A_1-A_2)^*(A_1-A_2) \big)=0$. Thus, using \eqref{eq_r19031}, we have
\begin{equation}
\begin{split}
\Big| & \hat Q(\pi_{\omega}(A_1)\Omega_{\omega},\pi_{\omega}(B)\Omega_{\omega})-\hat Q(\pi_{\omega}(A_2)\Omega_{\omega},\pi_{\omega}(B)\Omega_{\omega})      \Big|^2\\
        &=\Big|   \lambda\omega_1(A_1^*B)-\lambda\omega_1(A_2^*B)    \Big|^2\\
        &=\Big|  \lambda\omega_1\big( (A_1-A_2)^*B \big)  \Big|^2
        \\&\leq0.
\end{split}
\end{equation}
The whole is finally zero.


\subdem{$\hat Q$ is bounded}

Equation \eqref{eq:piomomaesm} together with the equality $| \lambda\omega_1(A^*B) |^2\leq\omega(A^*A)\omega(B^*B)$ give
\begin{equation}
\begin{split}
  \Big|  \hat Q(\pi_{\omega}(A)\Omega_{\omega},\pi_{\omega}(B)\Omega_{\omega})   \Big|^2
    &=| \lambda\omega_1(A^*B) |^2\\
    &\leq\omega(A^*A)\omega(B^*B)\\
    &=\| \pi_{\omega}(A)\Omega_{\omega} \|^2\| \pi_{\omega}(B)\Omega_{\omega} \|^2.
\end{split}
\end{equation}
Therefore $| \hat Q(\psi,\phi) |^2\leq\| \psi \|\| \phi \|$ and $\hat Q$ is bounded.

The quadratic form $\hat Q$ can be continuously extended to the whole $\hH_{\omega}$. From general property $\omega(A^*)=\overline{ \omega(A) }$ (when $\omega$ is a state),
\[
  \hat Q(\phi,\psi)=\overline{ \hat Q(\psi,\phi) }.
\]
Lemma~\ref{lem_r19031} immediately applies to our $\hat Q$, so we have an operator $Q$ such that $\scald{ \psi }{ Q\phi }=\hat Q(\psi,\phi)$. Therefore
%
\begin{equation} \label{eq_1903r3}
  \scald{ \pi_{\omega}(A)\Omega_{\omega}}{Q\pi_{\omega}(B)\Omega_{\omega}}=\lambda\omega_1(A^*B)
\end{equation}
Since $\pi$ is a representation, we have for all $A$, $B\in\cA$:
 \begin{equation}
\begin{split}
  \scald{ \pi_{\omega}(A)\Omega_{\omega} }{ Q\pi_{\omega}(B)\Omega_{\omega} }
     &=\hat Q\big( \pi_{\omega}(B^*A)\Omega_{\omega},\Omega_{\omega} \big)\\
        &=\scald{ \pi_{\omega}(B^*A)\Omega_{\omega} }{ Q\Omega_{\omega} }\\
        &=\scald{ \pi_{\omega}(A)\Omega_{\omega} }{ \pi_{\omega}(B)Q\Omega_{\omega} }.
\end{split}
\end{equation}
This proves that $[Q,\pi_{\omega}(C)]=0$ for each $C\in\cA$. Therefore $Q\in\pi_{\omega}(\cA)'$ and there exists a $t\in\eR$ such that $Q=t\cun$. Using equation \eqref{eq_1903r3} and  \eqref{eq_r1903r4}, we find
%
\begin{equation}
\begin{split}
t\omega(A^*B)&=t\scald{ \pi_{\omega}(A)\Omega_{\omega} }{ \pi_{\omega}(B)\Omega_{\omega} }\\
&=\lambda\omega_1(A^*B)\\
&=\hat Q\scald{ \pi_{\omega}(A)\Omega_{\omega} }{ \pi_{\omega}(B)\Omega_{\omega} }\\
\end{split}
\end{equation}
So $t\omega(A^*B)=\lambda\omega_1(A^*B)$; thus $\omega$ and $\omega_1$ are proportional and the decomposition
%
\[
  \omega=\lambda\omega_1+(1-\lambda)\omega_2
\]
is only possible for $\omega_1=\omega_2=\omega$. This proves that $\omega$ is a pure stare.

\end{proof}


\begin{corollary}
If the representation $\big( \pi(\cA),\hH \big)$ is irreducible, then the GNS representation $\big( \pi_{\omega}(\cA),\hH_{\omega} \big)$ build from any vector state (corresponding to $\Psi\in\hH$ such that $\| \Psi \|=1$) is unitary equivalent to $(\pi(\cA),\hH)$.
 \label{cor_GNSirredst}
\end{corollary}

\begin{proof}
    Since $\pi$ is irreducible, any vector in $\hH$ is cyclic from proposition~\ref{prop_equiv_rep_irred}. So $\pi$ is cyclic and proposition~\ref{prop:cyclequivGNS} concludes.

\end{proof}

\begin{corollary}
All irreducible representation of a $C^*$-algebra is (up to an equivalence) the GNS construction from a pure state.
\end{corollary}


\begin{proof}
We know that an irreducible representation is unitary equivalent to a GNS representation, but the GNS representation will only be irreducible when $\omega$ is pure state.
\end{proof}


\begin{proposition}
A state $\omega$ is pure if and only if for each positive functional $\rho$ such that $0\leq \rho\leq\omega$, there exists a $t\in\eR^+$ such that $\rho=t\omega$.
\label{prop_pureiff}
\end{proposition}

\begin{proof}
From corollary~\ref{cor_csa_unit} and  proposition~\ref{prop_st_unit_ext} we can suppose that $\cA$ is unital because if not, the notion of positivity is defined from unitization. When $\rho=0$ or $\rho=\omega$, the result is trivial. Now we suppose that $0\neq\rho\neq\omega$.

\subdem{Direct sense}

We know that $\omega$ is pure and $0\leq\rho\leq\omega$, $0<\rho(\cun)<1$ because $\omega-\rho$ is positive. Therefore
\[
  \| \omega-\rho \|=\omega(\cun)-\rho(\cun)=1-\rho(\cun),
\]
thus $\rho(\cun)=1$ should implies $\| \omega-\rho \|=0$ and then $\omega=\rho$. The possibility $\rho(\cun)=1$ is also not possible. Thus $\rho(\cun)$ is between $0$ and $1$.

This allows us to consider the states
\[
  \frac{ \omega-\rho }{ 1-\rho(\cun) },\quad\text{and}\quad\frac{ \rho }{ \rho(\cun) }.
\]
For $\lambda=1-\rho(\cun)$,
\[
  \omega=\lambda\frac{ \omega-\rho }{ 1-\rho(\cun) }-(1-\lambda)\frac{ \rho }{ \rho(\cun) }.
\]
Since $\omega$ is pure, it implies $\frac{ \omega-\rho }{ 1-\rho(\cun) }=\frac{ \rho }{ \rho(\cun) }$. So, $\rho=\rho(\cun)\omega$.

\subdem{Inverse sense}

Let us consider a decomposition $\omega=\lambda\omega_1+(1-\lambda)\omega_2$ of $\omega$. In the proof of theorem~\ref{tho_GNS_irred_pure}, we find that $0\leq\lambda\omega_1<\omega$. Then the assumption says that $\lambda\omega_1=\omega=\omega_2$ and the normalization makes automatically $\omega_1=\omega=\omega_2$ which proves that $\omega$ is pure.

\end{proof}


\begin{lemma}
Let $\cA$ be a $C^*$-algebra and $\rho$, a positive form on $\cA$. If we pose $M=\ker(\rho)$, $N=\{ A\in\cA\tq \rho(A^*A)=0 \}$, then
\[
  N+N^*\subseteq M
\]
and if $\rho$ is pure, $N+N^*=M$.
\end{lemma}

\begin{proof}
The functional $\rho$ being positive, equation  \eqref{eq:defprodetat} holds. With $A=\cun$ we find
\begin{equation}  \label{eq_pos_Bdtho}
| \rho(B)^2 |\leq\rho(\cun)\rho(B^*B)
\end{equation}
 and thus $\rho(A^*A)=0$ implies $\rho(A)=0$.

No proof for the second part.
\end{proof}


\begin{theorem}
The space of pure states of the (commutative) $C^*$-algebra $C_0(X)$ (the space of functions which are decreasing to zero at infinity) endowed with the relative $w^*$-topology is homeomorphic to $X$.
\end{theorem}

\begin{probleme}
    The following proof is buggy and very unsure.
\end{probleme}

\begin{proof}
The $C^*$-algebra  $\cA=C_0(X)$ is not specially unital. If it is not, Gelfand theorem~\ref{thoGelfand} says that there exists a locally compact and Hausdorff space $Y$ such that $\cA$ is isomorphic to $C_0(Y)$. If $\cA=C_0(X)$ with a non compact $X$, we begin to prove that the unitization is $\cA_{\cun}=C(\tilde X)$ where $\tilde X$ is the one point compactification of $X$. From proposition~\ref{prop_unitariz_csa}, we have an unique unital $C^*$-algebra $\cA_{\cun}$  with an isometric morphism $\cA\to\cA_{\cun}$ such that $\cA_{\cun}/\cA\simeq \eC$.

 Therefore, we have to prove that $C(\tilde X)$ is a suitable $\cA_{\cun}$ and so it will be the unique unitization of $C_0(X)$. The identity map $\id\colon C_0(X)\to C(\tilde X)$ works. We have to check that $C(\tilde X)/C_0(X)\simeq\eC$. By identifying all functions of $C(\tilde X)$ which only differ by a function of $C_0(X)$, there are in fact only one function for each complex number $z$: the which is constant (or another which is $z$ at $\infty$).

We have proved that if $\cA=C_0(X)$, then $\cA_{\cun}=C(\tilde X)$. From proposition~\ref{prop_st_unit_ext} pure states on $C_0(X)$ uniquely extend to a pure state on $C(\tilde X)$. So if $X$ is non compact, we do not loss anything by considering $C(\tilde X)$ instead of $C_0(X)$. We still have to prove that taking $C(\tilde X)$ instead of $C(X)$ does not \emph{gain} anything: we must have $\etS\big( C_0(X) \big)=\etS\big( C(\tilde X) \big)$. In the case of unital $C^*$-algebras, pure states are linear functionals. The way to extend linear functionals from $\cA$ to $\cA_{\cun}$ is the same as the one to extend states.

If $X$ is compact, then $C_0(X)=C(X)$. Hence we are in both case ($X$ compact or not) reduced to prove the theorem for $C(X)$ with compact $X$.

Following proposition~\ref{prop:comHauffhomeo}, $\Delta(C(X))$ is homeomorphic to $X$ because the latter is compact and Hausdorff. We have now to prove that there exists an homeomorphism between $\Delta(C(X))$ and set of pure states on $C(X)$. We are going to prove a bijection. Let on the one hand $\omega_x\in\Delta(C(X))$ be defined by
\begin{equation}
\begin{aligned}
 \omega_x\,:\,C(X)&\to \eC \\
\omega_x(f)&= f(x),
\end{aligned}
\end{equation}
and on the other hand a functional $\rho$ such that $0\leq\rho\leq\omega_x$. We have $\ker(\omega_x)\subset\ker(\rho)$. When $f$ is positive, $0\leq\rho(f)\leq\omega_x(f)$, so for any function,
\[
  o\leq\rho(f^*f)\leq\omega_x(f^*f),
\]
but $\omega_x$ is multiplicative, therefore $\rho(f^*f)\leq\omega_x(f^*)\omega_x(f)$. If $f\in\ker(\omega_x)$, we have $\rho(f^*f)=0$. Lemma concludes $\rho(f)=0$. So if $f\in\ker\omega_x$, we have $f(x)=0$. From theorem~\ref{tho:ideal_kernel}, $\ker\omega_x$ is a maximal ideal while $\ker(\rho)$ is an ideal. So $\ker(\omega_x)$ is a maximal ideal contained in an ideal, therefore if $\rho\neq0$, $\ker(\omega_x)=\ker(\rho)$ which implies that $\rho=\lambda\omega_x$. Proposition~\ref{prop_pureiff} concludes that $\omega_x$ is pure.

\begin{probleme}
    Why is $\ker(\rho)$ an ideal?
\end{probleme}


Now we suppose that $\omega$ is pure and we take  $g\in C(X)$ such that $0\leq g\leq 1_X$ on $C(X)$, we define
\begin{equation}
\begin{aligned}
 \omega_g\,:\,C(X)&\to \eC \\
f&\mapsto \omega(fg)
\end{aligned}
\end{equation}
 So  $\omega(f)-\omega_g(f)=\omega\big( f(1-g) \big)$ and if $f$ is positive, $\omega(f)-\omega_g(f)\geq O$. So for a certain $t\in\eR^+$, we have
\[
  \omega_g=t\omega
\]
 because $0\leq\omega_g\leq\omega$ and proposition~\ref{prop_pureiff}. In particular, $\ker (\omega_g)=\ker(\omega)$, hence when $f\in \ker(\omega)$, for any $g\in C(X)$, $fg \in \ker(\omega)$ because any function in $C(X)$ is a linear combination of functions $g$ with $0\leq g\leq 1_X$. This proves that $\ker (\omega)$ is an ideal. On the other hand, $\ker (\omega)$ is a maximal ideal because the kernel of any functional on a vector space has codimension 1, see page \pageref{pg_codimun}. Theorem~\ref{tho:ideal_kernel} shows that $\omega$ is multiplicative. So $\omega\in\Delta\big( C(X) \big)$.

\end{proof}

\subsection{Existence of pure states}
%------------------------------------

It is possible for $\etS$ to do not contain pure states. It is the case when $\etS$ is a convex cone. Such a $C^*$-algebra  has no irreducible representations. We are going to prove that this case is not possible. We define the convex hull of the part $A$ of a vector space by
\[
  co(A)=\{ \lambda+(1-\lambda)w\text{ with }w\in A,\lambda\in[0,1] \}.
\]



\begin{theorem}
 A compact connected set $K$ in a locally convex vector space is the closure of the convex hull of its extreme points. In other words:
    \[
  K=\overline{ co(\partial_eK) }.
\]

\end{theorem}
\begin{proof}
No proof
\end{proof}


\begin{lemma}

 Let $\cA$ be an unital $C^*$-algebra  and $\cB$ an self-adjoint vector subspace of $\cA$ with $\cun\in\cB$. Let $F$ be the set of linear forms $g$ on $B$ such that


\begin{itemize}
\item  $g(A^*) = \overline{g(A)}$ for all $A\in\cB$,
 \item $g(A) \geq 0$ for all $A\in \cB\cap \cA^+$,
\item $g(\cun)=1$.
\end{itemize}
 Any element of $F$ can be extended into a state on $\cA$.
\label{lem_DixcBprol}
 \end{lemma}

\begin{theorem}
For all $A\in\cA$ and a $a\in\sigma(A)$, we have a pure state $\omega_a$ on $\cA$ such that $\omega_a(A)=a$. There also exists a pure state $\omega$ such that $| \omega(A) |=\| A \|$.
\label{tho_existsetat}
\end{theorem}

\begin{proof}
Let us take an intermediary result in proof of lemma~\ref{prop_st_unit_ext}:
\begin{equation}
\begin{aligned}
 \tilde\omega,:\,\eC A+\eC\cun&\to \eC \\
(\lambda A+\mu\cun)&\mapsto \lambda a+\mu
\end{aligned}
\end{equation}
We extend this state by continuity and multiplicatively to $C^*(A,\cun)$ with formulas as $\tilde\omega_a(A^n)=a^n$. We have to check that this extension is pure: $\tilde\omega_a$ is positive and belongs to $\Delta(C^*(A,\cun))$.

Positivity comes from assumption that $A\in\cA_{\eR}$. Indeed, $a\in\sigma(A)\subset\eR^+$ . So positives elements in $C^*(A,\cun)$ are even power (and completion) of $A$. So the images are even powers of  $a\in\eR$ and are therefore positives.
The fact that $\tilde\omega_a$ is multiplicative on $C^*(A,\cun)$ comes from the fact that it is the same, in expressions as
$\tilde\omega_a(xy)$, to distribute inside the $\tilde\omega_a$ and push out terms $a^n$ by linearity, or write $\tilde\omega_a(x)\tilde\omega_a(y)$ and distribute outside. So $\tilde\omega$ is a pure state in $C^*(A,\cun)$.

 We consider the set $K_a$ of extensions of $\tilde\omega_a$ which are states on $\cA$. The fact that $K_a$ is non empty comes from the lemma~\ref{lem_DixcBprol}.

Let us now prove that $K_a$ is convex. For, we take $\omega_1$ and $\omega_2$, two extensions of $\tilde\omega_a$ and we will prove that $\omega=\lambda\omega_1+(1-\lambda)\omega_2$ is an extension too. By linearity, $\omega$ is a state. It is an extension of $\tilde\omega_a$ because
\[
\begin{split}
\omega(\cun+A^2)&=\lambda\omega_1(\cun+A^2)+(A-\lambda)\omega_1(\cun+A^2)\\
        &=\lambda(1+a^2)+(1-\lambda)(1+a^2)\\
        &=1+a^2.
\end{split}
\]

 In order to prove that $K_a$ is closed, we prove that its complement is open. Let $\omega\notin K_a$. We will prove that there exists $\varepsilon$ such that for all $\eta$ with $\| \omega-\eta \|\leq\varepsilon$. Let $\lambda$ be such that $\| \lambda A \|=1$ in such a manner that $\tilde\omega_a(\lambda A)=\lambda A$. Let $\omega(A)=s$ ($\omega\notin K_a$). We have
\[
\begin{split}
\| \omega-\eta \|&=\sup\{ | \omega-\eta |\text{ st } \| B \|=1 \}\\
        &\geq | (\omega-\eta)(\lambda A) |\\
        &=| \omega(\lambda A)-\eta(\lambda A) |,
\end{split}
\]
but
\[
  | \omega(\lambda A)-\eta(\lambda A) |\leq \| \omega-\eta \|\leq\varepsilon.
\]
So $| s-\eta(\lambda A) |\leq \varepsilon$, and when $\varepsilon$ is small (for example when $\varepsilon<| s-\lambda a |$), $\eta(\lambda A)$ is close to $s$ which is different of $\lambda A$. This proves that $K_a$ is closed.

We know that $K_a$ is convex and closed. So it has at least one extreme point. Let $\omega_a$ be one of them. We are going to prove that it is also extreme in $\etS$. If not it can be decomposed as $\omega=\lambda\omega_1+(1-\lambda)\omega_2$. Taking the latter at $A$ shows that, on $C^*(A,\cun)$, the functionals $\omega$, $\omega_1$ and $\omega_2$ are equals. So $\omega_a$ cannot be an extreme point in $K_a$. This concludes the first part of the proof.


Theorem~\ref{tho:prop_sigma} says us that in a Banach algebra, $\sigma(A)$ is compact for all $A$. So there exists a $a\in\sigma(A)$ such that $r(A)=| a |$. With this $a$,
 \[
  | \omega_a(A) |=| A |=r(A)=\| A \|.
\]
because $A=A^*$.

\end{proof}

The Gelfand Neumark theorem was proved using lemma~\ref{lem:omAenomA}. Now we have at hand a refining of this lemma. Hence we can use
\[
  \pi_r=\bigoplus_{\omega\in\etP(\cA)}\pi_{\omega}
\]
instead of the universal representation. The justification of this claim is that when $A$ is such that $\omega(A^*A)=0$ for all pure states, $\| A^*A \|=0$. Indeed $A^*A$ is positive, so there exists a pure state $\omega_a$ such that $\omega_a(A^*A)=\| A^*A \|$. Then $\| A^*A \|=0$.

Now we say that two states are \defe{equivalent}{equivalence!of states}\index{state!equivalent} if their GNS representations are equivalent. Gelfand Neumark theorem  says that
\[
  \cA\simeq\pi_r(\cA):=\bigoplus_{\omega\in[\etP(\cA)]}\pi_{\omega}(\cA).
\]
where $[\etP(\cA)]$ stands for the set of equivalences classes in $\etP(\cA)$.

\begin{proposition}
Any finite dimensional $C^*$-algebra is isomorphic to a direct sum of matricial algebras.
\end{proposition}
\begin{proof}
    No proof.
\end{proof}

\section{Generalization of matrix \texorpdfstring{$C^*$}{C}-algebra}
%+++++++++++++++++++++++++++++++++++++++++++++++

If we want to generalize the $C^*$-algebra $\mfM^n(\eC)$ to infinite dimensional Hilbert spaces, we first try to use the $C^*$-algebra of bounded operators. It does not work because such a $C^*$-algebra possesses many non equivalent representations on non separable Hilbert spaces.

\subsection{Example}
%------------------

Let us consider the $C^*$-algebra $\oB(\hH)$ and its definition representation which is obviously irreducible. From GNS construction and corollary~\ref{cor_GNSirredst}, we know that all the GNS representations build from a vector state are equivalent to the definition one.

On the other hand, there exists some self-adjoint bounded operators with non empty continuous spectrum. The take $A\in\oB(\hH)$ and $a\in\sigma(A)$ such that there are no $\psi_a\in\hH$ such that $A\psi_a=a\psi_a$. Theorem~\ref{tho_existsetat} gives a pure state $\omega_a$ such that $\omega_a(A)=a$, and hence a GNS representation $\pi_a$ on $\hH$. This representation is irreducible because $\omega_a$ is pure. In $\pi_a$, we have  a cyclic vector $\Omega_a$ such that
\[
  \scal{\Omega_a}{\pi_a(A)\Omega_a}=\omega_a(A)=a.
\]

%+++++++++++++++++++++++++++++++++++++++++++++++++++++++++++++++++++++++++++++++++++++++++++++++++++++++++++++++++++++++++++
\section{Tensor product}            \label{SecTensProdCSA}
%+++++++++++++++++++++++++++++++++++++++++++++++++++++++++++++++++++++++++++++++++++++++++++++++++++++++++++++++++++++++++++

If $\cA$ and $\cB$ are $C^*$-algebra, the \defe{tensor product}{tensor product!of $C^*$-algebra} is the completion of the space generated by the finite sums of the form $\sum_{i=1}^n A_i\otimes B_i$ with $A_i\in\cA$ and $B_i\in\cB$.

As an example of the importance of the completion, consider a compact group $G$ and $\cA=C(G)$ the $C^*$-algebra of continuous functions on $G$. We can build the map $\Delta\colon \cA\to \cA\otimes \cA$ by
\begin{equation}
    \Delta(f)(x,y)=f(xy)
\end{equation}
for every $x$ and $y$ in $G$ and $f\in C(G)$. The well-definiteness of $\Delta$ is due to the fact that $C(G)\otimes C(G)\simeq C(G\times G)$ by completion. This trick is used whenever we define a coproduct on a space of functions on a group, see for example definition~\ref{DefHopfsurCG} and section~\ref{SecExtenLemK} around equation \eqref{EqCABsimeqCACB}.

If $A$ and $B$ are manifolds, we have
\begin{equation}
    C^{\infty}(A)\otimes C^{\infty}(B)\simeq C^{\infty}(A\times B)
\end{equation}
by the map
\begin{equation}        \label{EqIsoCABCACBCstar}
    \begin{aligned}
        \varphi\colon  C^{\infty}(A)\otimes C^{\infty}(B)&\to  C^{\infty}(A\times B) \\
        \sum_ia_i\otimes b_i&\mapsto \Big[ (x,y)\mapsto\sum_ia_i(x)b_i(y) \Big].
    \end{aligned}
\end{equation}
The image by $\varphi$ of the \emph{algebraic} tensor product $ C^{\infty}(A)\otimes C^{\infty}(B)$ is dense in $ C^{\infty}(A\times B)$ as for example the polynomials are contained in the image. Indeed let $f(x,y)=\sum_{ij}f_{ij}x^iy^j$. The function $f$ is the image by $\varphi$ of
\begin{equation}        \label{EqDecompffklCABCACB}
    \sum_{kl} f_{kl} a_k\otimes b_l
\end{equation}
where $a_k(x)=x^k$ and $b_k(y)=y^k$. Thus, if we consider the $C^*$-algebraic tensor product, we have the equality.

See also \cite{Delaroche} for the sequel about tensor products. Let $\cA_1$ and $\cA_2$ be $C^*$-algebra and $\cA_1\odot\cA_2$ be their algebraic tensor product. There are at least two ways to define a $C^*$-norm on the $*$-algebra $\cA_1\odot\cA_2$.

\begin{enumerate}
    \item
        The \defe{maximal norm}{norm!maximal}\index{maximal!norm} of $A\in\cA_1\odot$ is defined by
        \begin{equation}
            \| A \|_{max}=\sup_{\pi}\| \pi(A) \|
        \end{equation}
        where the supremum is taken over all the representations\footnote{i.e. all the homomorphisms $\pi\colon \cA_1\odot\cA_2\to \opB(\hH)$.} of $\cA_1\odot\cA_2$ over some Hilbert space $\hH$. The \defe{maximal tensor product}{maximal!tensor product} is the completion of $\cA_1\odot\cA_2$ for that norm.

    \item
        The \defe{minimal norm}{minimal!norm} is obtained by taking the supremum only over the representations of the for $\pi_1\otimes \pi_2$:
        \begin{equation}
            \| x \|_{min}=\sup_{\pi_1,\pi_2}\| (\pi_1\otimes\pi_2)(x) \|
        \end{equation}
        where $\pi_i$ is a representation of $\cA_i$ on an Hilbert space $\hH_i$.
\end{enumerate}

\begin{lemma}
    If $\cA_1$ and $\cA_2$ are sub-$C^*$-algebra of $\opB(\hH_1)$ and $\opB(\hH_2)$, then $\cA_1\otimes_{min}\cA_2$ is the closure of $\cA_1\odot\cA_2$ seen as subalgebra of $\opB(\hH_1\otimes\hH_2)$.
\end{lemma}

%---------------------------------------------------------------------------------------------------------------------------
\subsection{Examples: Hopf algebra of functions}
%---------------------------------------------------------------------------------------------------------------------------
%\label{SubSecHoptUnivecvgp}

\begin{definition}		\label{DefHopfsurCG}
    Let $C(G)$ be the set of continuous functions on a topological group $G$. If we denote by $1$ the function $1\colon G\to \eC$, $1(x)=1$ for all $x\in G$, the following produces a structure of Hopf algebra\footnote{Definition~\ref{DefHopfAlgebra}.} on $C(G)$:
\begin{enumerate}
		\item
			$(f\cdot g)(x)=f(x)g(x)$,
		\item
			$\eta(\lambda)=\lambda 1$, the unit,
		\item\label{ItemHopfCGiii}
			$(\Delta f)(x\otimes y)=f(xy)$,
		\item\label{ItemHopfCGiv}
			$\epsilon(f)=f(e)$,
		\item
			$S(f)(x)=f(x^{-1})$.
	\end{enumerate}
\end{definition}

The item~\ref{ItemHopfCGiii} deserves some comments. If $f\in C(G)$, the element $\Delta(f)\in C(G)\otimes C(G)$ is given by $\Delta(f)=f_{(1)}\otimes f_{(2)}$ with the requirement that
\begin{equation}
	f_{(1)}(x)f_{(2)}(y)=f(xy).
\end{equation}
Such choice is possible since by density of the functions of the form $f(x)g(y)$ in $C(G\times G)$, see section~\ref{SecTensProdCSA}.

Let us check that $(\id\otimes\epsilon)\Delta=\id$. First $(\id\otimes\epsilon)\big( f_{(1)}\otimes f_{(2)} \big)=f_{(1)}\otimes f_{(2)}(e)$. Now we use the identification between $C(G)\otimes \eC$ and $C(G)$ ($(f\otimes z)(g)=zf(g)$) in order to get
\begin{equation}
	\Big( (\id\otimes\epsilon)\big( f_{(1)}\otimes f_{(2)} \big) \Big)(g)=\big( f_{(1)}\otimes f_{(2)}(e) \big)(g)=f_{(1)}(g)f_{(2)}(e)=f(ge)=f(g).
\end{equation}

\begin{probleme}
	Unicity of $f_{(1)}$ and $f_{(2)}$ seems doubtful to me.
\end{probleme}


%+++++++++++++++++++++++++++++++++++++++++++++++++++++++++++++++++++++++++++++++++++++++++++++++++++++++++++++++++++++++++++
\section{Traces over $C^*$-algebra }
%+++++++++++++++++++++++++++++++++++++++++++++++++++++++++++++++++++++++++++++++++++++++++++++++++++++++++++++++++++++++++++
\label{SecTraceCstar}

Source: \cite{DixmierTrace}. For tracial functionals on von Neumann algebras, see section~\ref{SecTracevonNeuman}. 

\begin{definition}
    Let $\cA$ be a $C^*$-algebra. A \defe{trace}{trace!over $C^*$-algebra} on $\cA^+$ is a function $\tau\colon A^+\to \mathopen[ 0 , \infty \mathclose]$ such that
    \begin{enumerate}
        \item
            if $A$ and $B$ are in $\cA^+$, $\tau(A+B)=\tau(A)+\tau(B)$,
        \item
            If $A\in\cA^+$ and $\lambda\in\eR^+$, then $\tau(\lambda A)=\lambda\tau(A)$. Here if $\lambda=0$, we pose $0\times\infty=0$;
        \item
            If $Z\in\cA$, we have $\tau(ZZ^*)=\tau(Z^*Z)$.
    \end{enumerate}
    We say that $\tau$ is semifinite if for every $A\in\cA^+$,
    \begin{equation}
        \tau(A)=\sup\{ \tau(B)\tq B\in\cA^+,B\leq A,\tau(B)<\infty \}.
    \end{equation}
\end{definition}
We recall that for every $Z\in\cA$, we have $ZZ^*\in\cA^+$.

The following lemma is the lemma $3$ in \cite{DixmierTrace}.
\begin{lemma}       \label{LemTraceAplusextmlmn}
    Let $\cA$ be a $C^*$-algebra and $\tau$ a trace on $\cA^+$. Then the following hold.
    \begin{enumerate}
        \item
            The set
            \begin{equation}
                \mL=\{ A\in\cA\tq\tau(AA^*)<\infty \}
            \end{equation}
            is a bilateral ideal in $\cA$.
        \item
            The set $\mN=\langle \mL^2\rangle$ is the set of complex linear combinations of $\mN^+$: $\mN=\langle \mN^+\rangle.$
        \item
            We have
            \begin{equation}
                \mN^+=\{ A\in\cA^+\tq\tau(A)<\infty \}.
            \end{equation}
        \item
            There exists one and only one linear form $f$ on $\mN$ which coincides with $\tau$ on $\mN^+$.
        \item
            The linear form $f$ satisfies
            \begin{enumerate}
                \item
                    $f(A^*)=\overline{ f(A) }$;
                \item
                    $f(AB)=f(BA)$ for every $u$ and $v$ in $\mL$;
                \item
                    $f(ZA)=f(AZ)$ for every $A\in\mN$ and $Z\in \cA$.
            \end{enumerate}

    \end{enumerate}

\end{lemma}

\begin{proof}
    Let us begin by pointing out the fact that if $A\leq B$, then $\tau(A)\leq\tau(B)$ because of linearity: $\tau(B)=\tau(A)+\tau(B-A)\geq\tau(A)$ since $B-A\in\cA^+$.

    \begin{enumerate}
        \item
            If $A\in\mL$, then $A^*\in\mL$ because $\tau(A^*A)=\tau(AA^*)$ (by definition of a trace). If $A,B\in\mL$, we have
            \begin{equation}
                (A+B)(A+B)^*\leq 2(AA^*+BB^*),
            \end{equation}
            so that
            \begin{equation}
                \tau\big( (A+B)(A+B)^* \big)<2\tau(AA^*)+2\tau(BB^*)<\infty.
            \end{equation}
            This proves that $A+B\in\mL$.

            Let now $A\in\mL$ and $Z\in\cA$. Since $AZZ^*X^*\leq\| ZZ^* \|AA^*$, we have
            \begin{equation}
                \tau\big( AZ(AZ)^* \big)<\infty.
            \end{equation}
            So $\mL$ is a right ideal in $\cA$. This is also a left ideal because $ZA=(A^*Z^*)^*$, but the fact that $A^*\in\mL$ implies $A^*Z^*\in\mL$, so that $ZA\in\mL$.
        \item
            An element $X$ in $\mN$ reads
            \begin{equation}
                X=\sum_{j=1}^{n}A_jB_j^*
            \end{equation}
            with $A_j,B_j\in\mL$. The \defe{polarization}{polarization} relation reads
            \begin{equation}        \label{EqPolaXAB}
                \begin{aligned}[]
                    4X&=\sum_j(A_j+B_j)(A_j+B_j)^*\\
                    &\quad+\sum_j(A_j-B_j)(A_j-B_j)^*\\
                    &\quad+\sum_j(A_j+iB_j)(A_j+iB_j)^*\\
                    &\quad+\sum_j(A_j-iB_j)(A_j-iB_j)^*.
                \end{aligned}
            \end{equation}
            So $X$ is a linear combination of elements in $\mN^+$ (that is elements of the form $AA^*$ with $A\in\mL$). Notice that $A\in\mL$ implies $iA\in\mL$ because
            \begin{equation}
                \tau\big( iA(iA)^* \big)=-\tau(iAiA^*)=\tau(AA^*)<\infty.
            \end{equation}
            This proves that $\mN\subset\langle \mN^+\rangle$. The fact that $\langle \mN^+\rangle$ is a subset of $\mN$ is by construction.
        \item
            An element $X$ in $\langle \mN^+\rangle$  reads $X=\sum_jA_jB^*_j$ where, for each $j$, we have $A_jB_j^*\in\mN^+$, in particular $A_jB^*_j=(A_jB^*_j)^*=B_jA_j^*$. We can still write down the polarization identity, but now the last two terms of \eqref{EqPolaXAB} give $2i(B_jA_j^*-A_jB_j^*)=0$.

            Thus we have
            \begin{equation}
                \begin{aligned}[]
                    4X&=\sum_j(A_j+B_j)(A_j+B_j)^*-\sum_j(A_j-B_j)(A_j-B_j)^*\\
                    &\leq\sum_j(A_j+B_j)(A_j+B_j)^*,
                \end{aligned}
            \end{equation}
            but we already know that $\tau\big( \sum_j(A_j+B_j)(A_j+B_j)^* \big)<\infty$. Thus we have $\tau(X)<\infty$. So we proved that
            \begin{equation}
                \mN^+\subset\{ X\in\cA^+\tq\tau(X)<\infty \}.
            \end{equation}

            Let now $X\in\cA^+$ be such that $\tau(X)<\infty$. In order to prove that $X\in\mN^+$, it is sufficient to prove that $X\in\mN$. Since $X=X^*$, we can use the continuous functional calculus (see theorem~\ref{ThoContFuncCalculus} and remark~\ref{RemExpansionSqrtConCal}) in order to define $X^{1/2}$. We have
            \begin{equation}
                \tau\big( X^{1/2}(X^{1/2})^* \big)=\tau\big( X^{1/2}X^{1/2} \big)<\infty,
            \end{equation}
            so that $X^{1/2}\in\mL$.
        \item
            Since $\mN$ is generated by $\mN^{+}$, the functional $\tau$ on $\mN^+$ there is one and only one extension of $\tau$ to a linear functional $f$ on $\mN$.
        \item
            \begin{enumerate}
                \item
                    An element of $\mN$ is a linear combination of elements of $\mN^+$: $X=\sum_j\lambda_iA_i$ with $A_i\in\mN^+$. Thus using the linearity of $f$ and the properties of $\tau$, we have
                    \begin{equation}
                        \begin{aligned}[]
                            f(X^*)&=f\big( \sum_j\overline{ \lambda_j }A_j^* \big)\\
                            &=\sum_j\overline{ \lambda_j }f(A_j)\\
                            &=\sum_j\overline{ \lambda_j }\underbrace{\tau(A_j)}_{\in\eR^{+}}\\
                            &=\sum_j\overline{ \lambda_j \tau(A_j)}\\
                            &=\overline{ f(X) }.
                        \end{aligned}
                    \end{equation}
                    This is the first property we had to check.
                \item
                    If $A\in\mL$, we have $AA^*\in\mN$ and by definition of a trace,
                    \begin{equation}
                        f(AA^*)=\tau(AA^*)=\tau(A^*A)=f(A^*A).
                    \end{equation}
                    If $A$ and $B$ belong to $\mL$, we use the polarization identity:
                    \begin{equation}
                        4AB^*=(A+B)(A+B)^*-(A-B)(A-B)^*+i(A+iB)(A+iB)^*-i(A-iB)(A-iB)^*,
                    \end{equation}
                    and we do the same computation in order to get $f(AB^*)=f(B^*A)$ whenever $A$ and $B$ belong to $\mL$.
                \item
                    Let $Z\in\cA$. An element in $\mN$ reads $X=\sum_jA_jB_j$ with $A_j$ and $B_j$ in $\mL$. Since $\mL$ is an bilateral ideal in $\cA$ we have $ZA_j\in\mL$ and $B_jZ\in\mL$, thus we can make the following computation:
                    \begin{equation}
                        \begin{aligned}[]
                            f\big( Z\sum_jA_jB_j \big)&=\sum_jf\big( (ZA_j)B_j \big)\\
                            &=\sum_jf\big( B_j(ZA_j) \big)\\
                            &=\sum_jf\big( (B_jZ)A_j \big)\\
                            &=\sum_jf\big( A_j(B_jZ) \big)\\
                            &=f\big( (\sum_jA_jB_j)Z \big)\\
                            &=f(XZ).
                        \end{aligned}
                    \end{equation}
            \end{enumerate}
            This concludes the proof of the lemma.
    \end{enumerate}


\end{proof}


\chapter{Compact quantum groups}
\input{149_compact_quantum_groups}

\chapter{von Neumann algebras}
\input{121_VN_algebras}
\input{122_VN_algebras}
% This is part of (almost) Everything I know in mathematics
% Copyright (c) 2013-2014,2016, 2020
%   Laurent Claessens
% See the file fdl-1.3.txt for copying conditions.

%+++++++++++++++++++++++++++++++++++++++++++++++++++++++++++++++++++++++++++++++++++++++++++++++++++++++++++++++++++++++++++
					\section{Position of submodules}
%+++++++++++++++++++++++++++++++++++++++++++++++++++++++++++++++++++++++++++++++++++++++++++++++++++++++++++++++++++++++++++

\begin{proposition}
Let $M$ be a von~Neumann algebra (not specially with trace). Every finitely generated submodule of a finite generated projective module is projective.
\end{proposition}

\begin{proof}
Passing to the matrix algebra (lemmas~\ref{LemEprojEEEfproj} and~\ref{LemFGenEEEsingleGen}), one can assume that the module and the submodule are in fact singly generated.

A singly generated module over $ M$ has the form $E=MT\subseteq M$ for some element $T\in M$. We saw during the proof of proposition~\ref{PropMTprojpourtoutT} that $MT\simeq MP$ where $P$ is the projection onto $\overline{ \Image(T) }$. The fact that $MP$ is a projective module was already argued on page \pageref{PgMPprojModule}.
\end{proof}

\begin{proposition}			\label{PropEfgpFssmodQuotProj}
If $E$ is a finite projective module over $M$ and $F$ is a submodule of $E$, then $E/\Cl_E(F)$ is projective.
\end{proposition}

\begin{proof}
Once again, using the matrix trick, one can suppose that $E$ is singly generated by an element $e$. A left module map $\varphi\colon E\to M$ such that $\varphi(F)=0$ is determined by the value of $\varphi(e)\in M$. The element $\varphi(e)$ is an operator on $\hH$ and we define $P_{\varphi}$ as the projection onto $\overline{ \Image\big( \varphi(e) \big) }$.

Now we define the module map
\begin{equation}
\begin{aligned}
\tilde{\varphi} \colon E&\to M \\
   \tilde{\varphi}(Te)&\mapsto TP_{\varphi}.
\end{aligned}
\end{equation}
This is well defined. Indeed if $Se=0$, then $\varphi(Se)=S\varphi(e)=0$, which implies that $SP_{\varphi}=0$. For the same reason, $\ker(\varphi)=\ker(\tilde{\varphi})$. We define
\begin{equation}
	Q=\bigvee_{\varphi}P_{\varphi},
\end{equation}
and
\begin{equation}
\begin{aligned}
 \psi\colon E&\to M \\
   \psi(Te)&\mapsto TQ
\end{aligned}
\end{equation}
which is well defined because $TQ=0$ if and only if $TP\varphi =0$ for every $\varphi$, since the operator $T$ is bounded\footnote{If $T$ is not bounded, it can get bigger and bigger on the range of $\varphi_k$ when $k$ goes to infinity, so that the limit of $T\bigvee_{\varphi\in\alpha}$ is not zero when $\alpha$ gets bigger.}.
\begin{probleme}
The explanation in the footnote is unclear; it has to be expressed in terms of nets.
\end{probleme}
By construction, $\ker(\psi)=\Cl_E(F)$, while $\Image(\psi)=MQ$, so that
\begin{equation}
	E/\ker(\psi)\simeq MQ,
\end{equation}
while the latter is projective.
\end{proof}

\begin{corollary}		\label{CorEfgpFssIsom}
When $E$ is a finite projective module and $F$ a submodule, we have an isomorphism
\[
	E=\Cl_E(F)\oplus E/\Cl_E(F)
\]
as direct sum of modules.
\end{corollary}

\begin{proof}
If $\xi\in E$, the class of $\xi$ in $E/\Cl_E(F)$ is
\begin{equation}
	[\xi]=\{ \xi+\eta\tq \eta\in\Cl_E(F) \}.
\end{equation}
Let us choose a representative $[\xi]_0$ in each of the classes\quext{This uses the famous axiom; it would be possible to do otherwise, isn't?}. The following is the module isomorphism we are searching for
\begin{equation}
\begin{aligned}
 \psi\colon E&\to \Cl_E(F)\oplus E/\Cl_E(F) \\
   \xi&\mapsto  \eta\oplus [\xi]_0
\end{aligned}
\end{equation}
where $\eta$ is the unique element of $\Cl_E(F)$ such that $[\xi]_0+\eta=\xi$.
\end{proof}

When $E$ is a finite projective module over $M$, we say that $\Cl_E(0)$ is the \defe{torsion submodule}{torsion!submodule} of $E$.

Let us see an example. Let
\begin{equation}
	E=M/MT
\end{equation}
with $T\geq 0$ and $\overline{ \Image(T) }=\hH$. We claim that the torsion submodule of $E$ is $E$ itself. A module map $\varphi\colon E\to M$ is a module map $M\to M$ composed with a projection, or in other words a module map $\varphi\colon M\to M$ such that $\varphi(T)=0$. Since $\varphi$ is a module map, its fulfils
\begin{equation}
	\varphi(S)=\varphi(S\mtu)=S\varphi(\mtu)=SX
\end{equation}
for a certain $X\in M$ such that $TX=0$. Thus we have $\overline{ \Image(X) }\subset\ker(T)$, but since $T\geq 0$ (in particular $T$ is self-adjoint) and $\overline{ \Image(T) }=\hH$, we have $\ker(T)=0$, and we conclude that $X=0$, so that $\varphi=0$. The question is to know if $E=M/MT$ has a finitely generated submodule or not. Let $P$ be a projective submodule of $M/MT$; by the lifting property \eqref{EqLiftPropProjModules} we have a map $\tilde \lambda$ such that the following commutes
\begin{equation}
\xymatrix{%
 									&  M \ar@{->>}[d]^{\displaystyle\lambda}\\
   P \ar[r]\ar@{.>}[ru]							& M/MT
}
\end{equation}
where the arrow from $P$ to $M/MT$ is injective. The image of $P$ by $\tilde\lambda$ is a submodule $MX$ that has to be injectively\quext{How to say the fact to be injective in English? Does the word \emph{injectively} exist?} projected in $M/MT$, so that $MX\cap MT=\emptyset$. Notice that it is not possible when $\hH$ is finite dimensional because $T$ is invertible (from the fact that the closure of its image is the whole space), so that $MT=M$.

Now we suppose that $X$ is positive. This is done without loss of generality because from polar decomposition, for every $X\in M$, there exists a positive $Y$ such that $MX=MY$.

Since $T$ is positive, we can consider the spectral theorem~\ref{ThoSpectralTho} and the isomorphism
\begin{equation}
\begin{aligned}
 \theta\colon C^*(T,\cun)&\to C\big( \Spec(T) \big).
\end{aligned}
\end{equation}
We define the following function on $\Spec(T)$
\begin{equation}
	f_{\epsilon}=
			\begin{cases}
					0				&\text{if }x<\epsilon\\
					\frac{ 1 }{ (\theta T)(x) }	&\text{if }x\geq\epsilon
			\end{cases}
\end{equation}
and then one defines the operator $S_{\epsilon}=\theta^{-1}(f_{\epsilon})\in C^*(T,\cun)$. Let us prove that $P_{\epsilon}=S\epsilon T$ is a projection. We have $P_{\epsilon}^2=S_{\epsilon} TS_{\epsilon}T$. Take an orthonormal basis of $\hH$ of eigenvectors of $T$ and let's call $\lambda_i$ the eigenvalues: $Te_i=\lambda_ie_i$. If $\lambda_k<\epsilon$, then $S_{\epsilon}e_i=0$ and of course $S_{\epsilon}TS_{\epsilon}e_i=S_{\epsilon}e_i$. Otherwise, we have
\begin{equation}
	S_{\epsilon}TS_{\epsilon}e_i=\frac{1}{ \lambda_i }S_{\epsilon}T e_i =S_{\epsilon}e_i.
\end{equation}
That proves that $P_{\epsilon}^2=P_{\epsilon}$, and so that this is a projection. We obviously also have $P_{\epsilon}\to \mtu$ in $MT$.

Similarly one can define $Q$, the projection onto $\overline{ \Image(X) }$ and we have projections $Q_{\epsilon}\to Q$ with $Q_{\epsilon}=Y_{\epsilon} X$ for some $Y\epsilon\in M$ and $Q_{\epsilon}\in MX$. Now take $\epsilon_1$ and $\epsilon_2$ and look at the projection
\begin{equation}
	Q_{\epsilon_1}\vee P_{\epsilon_2}
\end{equation}
onto $\Image(Q_{\epsilon_1})\cap\Image(P_{\epsilon_2})$. The latter intersection is in fact $0$ because $A_{\epsilon_1}\vee P_{\epsilon_2}\in MQ_{\epsilon_1}\cap MP_{\epsilon_2}=MX\cap MT=0$. Using lemma~\ref{LemDimSupDeuxProjs}, we get
\begin{equation}
	\dim\mtu\geq \dim(Q_{\epsilon_1}\vee P_{\epsilon_2})=\dim(Q_{\epsilon_1})+\dim(P_{\epsilon_2}).
\end{equation}
Recall that the trace used to define the dimension has to be normal, so that the dimension function is continuous in such a way that taking the limit $\epsilon_1\to 0$ and $\epsilon_2\to 0$ provides the expected result
\begin{equation}
	\dim(\mtu)\geq \dim(Q)+\dim(\mtu),
\end{equation}
from which one deduce that $\dim Q=0$ and therefore $X=0$. This finish the proof that the module $E=M/MT$ with a positive $T$ and $\overline{ \Image(T) }=\hH$ has no projective submodules.

\begin{proposition}
If $E$ is a finitely generated module over $M$, then the torsion submodule does not contains non zero projective finite submodules.
\end{proposition}

\begin{proof}
Later.
\end{proof}


\begin{lemma}			\label{LemFClosEF}
If $E$ is a finitely generated submodule of a finitely generated projective module $E$, then $F$ is projective and $F\simeq \Cl_E(F)$.
\end{lemma}

\begin{proof}
We assume as usual that $E$ and $F$ are singly generated. The singly generated projective module $E$ reads $E=MP$ while the general form of a singly generated submodule is $F=MT$ for a positive $T$ which vanishes on $P^{\perp}\hH$ and $\Image(T)\subseteq\Image(P)$. We already proved that $MT\simeq MQ$ where $Q$ is the projection over $\Image(T)$.

We have $\Cl_{MP}(MT)=MQ$ because of the direct sum decomposition $MP=MQ\oplus M(P-Q)$ from which we can build an homomorphism $MP\to M$ which vanishes on $MT$, namely the projection  because $MT\subseteq MQ$.
\begin{probleme}
I do not understand one single word about the latter justification $:($
\end{probleme}
\end{proof}

\begin{corollary}		\label{Corfgfgdilleqdim}
If $F$ is a finitely generated submodule of a finitely generated projective module $E$, then $\dim(F)\leq\dim(E)$.
\end{corollary}

\begin{proof}
By lemma~\ref{LemFClosEF}, we have $\dim(F)\simeq\dim\big( \Cl_E(F) \big)$ while we know that $E=\Cl_E(F)\oplus E/\Cl_E(F)$. The latter makes that $\dim E$ is given by the trace of two projections:
\begin{equation}
	\dim E=\tr P_{\Cl_E(F)}+\tr P_{E/\Cl_E(F)}.
\end{equation}
\end{proof}

\begin{corollary}
If $E$ is a finitely projective module over a von~Neumann algebra with a trace, then $\dim(E)=\Dim(E)$.
\end{corollary}

\begin{proof}
By very definition, $\Dim(E)\geq\dim(E)$ because $E$ itself belongs to the set on which the supremum is taken in the definition \eqref{DefDimAvecGrandD} of $\Dim$. The corollary~\ref{Corfgfgdilleqdim} provides the opposite inequality.
\end{proof}

\begin{proposition}		\label{PropProjFiniDimCldim}
If $E$ is projective finitely generated and if $F\subseteq E$, then $\dim\big( \Cl_E(F) \big)=\Dim(F)$.
\end{proposition}

\begin{proof}
The case where $F$ is finitely generated is already done by lemma~\ref{LemFClosEF}. For the general case, suppose that the proposition does not hold. In this case, the lemma~\ref{LemFClosEF} yields
\begin{equation}
    \sup\{ \dim\big( \Cl_E(H) \big)\tq H \text{ is finite projective } \}<\dim\big( \Cl_E(F) \big).
\end{equation}
On the one hand, by corollary~\ref{CorEfgpFssIsom}, the module $\Cl_E(H)$ is a direct summand of $\Cl_E(F)$ which is itself (by the same result) a direct summand of $E$. On the other hand, being a finitely generated module, it reads $E=M^nP$ for some projection $P$ by proposition~\ref{PropFGPRkP}. Combining both, we have a projection $P_H\leq P$ such that $\Cl_E(H)=M^nP_H$.

Now the finitely generated projective submodules of $F$ form a directed system. For every finitely generated projective submodules $H_1$ and $H_2$ of $F$, there exists at least $H_3=H_1\oplus H_2$ which contains $H_1$ and $H_2$. Thus the set of $P_H$ is directed too and one can look at $\bigvee_HP_H$, the smalest projection whose range contains the range of all of the $P_H$.

We have $\bigvee_HP_H<P$ because if not, using normality of the trace,
\begin{equation}
	\dim\big( \bigvee_HP_H \big)=\sup_H\dim(P_H)<P.
\end{equation}
Now let $Q=P-\bigvee_HP_H$. If $F$ is some submodule of $E$, one has
\begin{equation}
\begin{split}
    F&=\bigcup\{ H\subseteq F\tq H\text{ is finitely generated} \}\\
    &=\bigcup\{ H\subseteq F\tq H\text{ is finitely generated and projective} \}\\
\end{split}
\end{equation}
because $H\subseteq F\subseteq E$ which is finitely generated projective. Such a $H$ has the form $M^nP_H$, sp $F\subseteq M^n\big( \bigvee_HP_H \big)$. Notice that this $F$ lies in the kernel of the non zero map
\begin{equation}
\begin{aligned}
 \Cl_E(F)=M^nP&\to M^nQ \\
   V&\mapsto VQ.
\end{aligned}
\end{equation}
Indeed, since $\bigvee_HP_H<P$, we have $\big( \bigvee_HP_H \big)P=\bigvee_HP_H$, so that for every $T\in M^n$ we have $T\bigvee_HP_H\big( P-\bigvee_HP_H \big)=0$. By looking at the complement of $VQ$, one has a nonzero homomorphism $ \Cl_E(F)\to M$ which vanishes on $F$. That contradicts the definition of the closure.

\end{proof}

The three point in this demonstration that use the von~Neumann algebra background (and not only general module theory) are the following.
\begin{itemize}
\item First we used normality of the trace to commute the dimension with the supremum,
\item and second, we used continuity of $\bigvee$ with respect to the dimension.
\end{itemize}

%---------------------------------------------------------------------------------------------------------------------------
					\subsection{Summary}
%---------------------------------------------------------------------------------------------------------------------------

We have two dimension functions $\dim$ and $\Dim$ such that
\begin{enumerate}
\item $\dim E=\Dim E$ whenever $E$ is a finitely generated projective module,
\item for every finitely generated module $E$ and every submodule $F$, the module $E/\Cl_E(F)$ is finitely generated and projective,
\begin{probleme}
Check if one does not need the assumption that $E$ is projective too.
\end{probleme}
\item If $F\subseteq E$ and if $E$ is a finitely generated projective module, then $\Dim(F)=\Dim\big( \Cl_E(F) \big)$.
\end{enumerate}
From now we do no more use the ``von~Neumann algebra'' assumption. Instead we suppose to have a ring $R$ and two dimensions functions satisfying these three properties.

%---------------------------------------------------------------------------------------------------------------------------
					\subsection{Properties of the dimension function}
%---------------------------------------------------------------------------------------------------------------------------

\begin{proposition}
If $E$ is the union of a directed system of submodules $E_{\alpha}$, then $\Dim(E)=\sup \Dim E_{\alpha}$.
\end{proposition}

\begin{proof}
A finitely generated projective submodule in $E$ is generated by $n$ elements, each of them being contained in some $E_{\alpha_1}$, $E_{\alpha_2},\ldots$ By definition of a directed set, the union of all the so defined $E_{\alpha_i}$ is contained in a $E_{\beta}$. Thus every finitely generated projective submodule in $E$ is of the form $E_{\beta}$.
\end{proof}

\begin{lemma}			\label{LemHinjectifHdimdim}
If one has a projective module map $\rho\colon H_1\to H_2$, then $\Dim(H_1)\geq\Dim(H_2)$.
\end{lemma}

\begin{proof}
Let $F$ be any projective module for which there exists an inclusion $\iota\colon F\to H_2$. That map can be lifted because $F$ is projective. So among all the submodules of $H_1$, there is the one which is the image of $F$ be the lifted map. That one of course contains $H_2$ itself. Thus $H_2$ is a submodule of $H_1$ and the supremum defining the dimension in $H_1$ is automatically bigger or equal to the one defining the dimension of $H_2$.
\end{proof}

\begin{proposition}
If
\begin{equation}
	\xymatrix{%
   0\ar[r] 	&E_0\ar[r]^{}	&E_1\ar[r]^{p}	&E_2 \ar[r]	&0
}
\end{equation}
is a short exact sequence of modules, then $\dim(E_1)=\dim(E_0)+\dim(E_2)$.
\end{proposition}

\begin{proof}
Using last proposition, we can assume that all of $E_0$, $E_1$ and $E_2$ are finitely generated. Indeed when a module is not finitely generated, it is still the union of the directed system of all its finitely generated submodules.

Let $F$ be a finitely generated projective submodule of $E_2$, we have the exact sequence
\begin{equation}
	\xymatrix{%
   0\ar[r] 	&E_0\ar[r]	&p^{-1}(F)\ar[r]	&F \ar[r]	&0.
}
\end{equation}
The module $F$ being projective, we have $p^{-1}(F)\simeq F\oplus E_0$. We do not know if the dimension function is additive with respect to direct sum, but by definition of a supremum, we have the inequality $\Dim(E_0)+\Dim(F)\leq \Dim\big( p^{-1}(F) \big)$, and the chain
\begin{equation}
	\Dim(E_0)+\Dim(F)\leq \Dim\big( p^{-1}(F) \big)\leq\Dim(E_1)
\end{equation}
which in turn provides the inequalities
\begin{equation}
	\Dim(E_0)+\Dim(E_2)\leq\Dim(E_1).
\end{equation}
For the reverse inequality, let $F$ be a finitely generated projective submodule of $E_1$, and consider the following exact sequence of finitely generated projective module
\begin{equation}
	\xymatrix{%
   0\ar[r] 	&\Cl_F(F\cap E_0)\ar[r]	&F\ar[r]	&F/\Cl_F(F\cap E_0) \ar[r]	&0
}
\end{equation}
The fact that $F/\Cl_F(F\cap E_0)$ is projective is proposition~\ref{PropEfgpFssmodQuotProj}. Since $F/\Cl_F(F\cap E_0)$ is at most a subset of $F$ which is finitely generated, it has to be finitely generated too. We also know by corollary~\ref{CorEfgpFssIsom} that $F$ splits into
\begin{equation}
	F\simeq \Cl_F(F\cap E_0)\oplus F/\Cl_F(F\cap E_0)
\end{equation}
which is a direct sum of finitely generated projective module, so that we can use the definition of dimension with traces (which sums up over direct sum) instead of the one with supremum. Thus we have
\begin{equation}
	\dim(E)=\dim\big( \Cl_F(F\cap E_0) \big)+\dim\big( F/\Cl_F(F\cap E_0) \big).
\end{equation}
The module $F$ being projective and finitely generated, proposition~\ref{PropProjFiniDimCldim} allows us to replace $\dim\big( \Cl_F(F\cap E_0) \big)$ by $\Dim(F\cap E_0)$ and write
\begin{equation}
	\dim(E)=\Dim(F\cap E_0)+\Dim\big( F/\Cl_F(F\cap E_0) \big)\leq \Dim(E_0)+\Dim(F/(F\cap E_0))
\end{equation}
where we also used lemma~\ref{LemHinjectifHdimdim}. Since $F/(F\cap E_0)$ is a quotient of $E_2$, we have $\Dim(F/(F\cap E_0))\leq\Dim(E_2)$, and taking the supremum over all suitable $F$, we find the result
\begin{equation}
	\dim(E_1)\leq \Dim(E_0)+\Dim(E_2).
\end{equation}
\end{proof}

\begin{proposition}
If $E$ is a finitely generated module and $F\subseteq E$, then $\Dim(F)=\Dim\big(\Cl_E(F)\big)$.
\end{proposition}
\begin{proof}
Hint: by the proposition, one can assume that $E$ is actually projective.
\end{proof}
\begin{probleme}
 That has to be completed.
\end{probleme}
%+++++++++++++++++++++++++++++++++++++++++++++++++++++++++++++++++++++++++++++++++++++++++++++++++++++++++++++++++++++++++++
					\section{Decomposition of operators and representations}
%+++++++++++++++++++++++++++++++++++++++++++++++++++++++++++++++++++++++++++++++++++++++++++++++++++++++++++++++++++++++++++

%---------------------------------------------------------------------------------------------------------------------------
					\subsection{Motivation}
%---------------------------------------------------------------------------------------------------------------------------

Let $G$ be a compact topological group and consider the space $L^2(G)$ (with respect to the Haar measure), and $M$ be the von~Neumann algebra generated by the left regular representation
\begin{equation}
	M=\{ U_g\tq g\in G \}''
\end{equation}
with $(U_gf)(f)=g(g^{-1} h)$. Each $U_g$ is an unitary operator in $H$. In this case, the commutant $M'$ turns out to be the von~Neumann algebra generated by the regular right representation.

\begin{probleme}
	The operator $R$ given by
	\begin{equation}
		\big(R(f)\big)(x)=\frac{ f(x) }{ 2 }
	\end{equation}
	commutes with all the regular left representation, but is not part of the regular right representation. Do we impose unitarity conditions?
\end{probleme}

When $\sigma\colon G\to \End(V)$ is a finite dimensional irreducible representation of the group $G$, the \defe{character}{character!of a representation} of $\sigma$ is the function $\chi_{\sigma}\colon G\to \eC$ defined by the formula
\begin{equation}
	\chi_{\sigma}(g)=\tr\big( \sigma(g) \big).
\end{equation}
The representation $\sigma$ thus defines the operator $P_{\sigma}$ on $H=L^2(G)$,
\begin{equation}
	(P_{\sigma})(f)=\frac{ \dim(\sigma) }{ \volume(G) }\int \chi_{\sigma}(g^{-1})(U_gf)\,dg.
\end{equation}

\begin{theorem}[Peter-Weyl]			\label{ThoPeterWeyl}\index{Peter-Weyl theorem}
	The operator $P_{\sigma}$ is a projection which belongs to $M\cap M'$ and for which the following hold
	\begin{enumerate}
		\item\label{ItemPeterWeyli} If $\sigma_1\nsimeq \sigma_2$, then $P_{\sigma_1}P_{\sigma_2}=0$,
		\item \label{ItemPeterWeylii} in the strong topology, $\sum_{\sigma}P_{\sigma}=\mtu$,
		\item the algebra $M\cap M'$ is generated by $\{ P_{\sigma} \}$ and each of the $P_{\sigma}$ is minimal in the center,
		\item $P_{\sigma}MP_{\sigma}$ is a factor on $H_{\sigma}$ and we have $(P_{\sigma}MP_{\sigma})'=P_{\sigma}M'P_{\sigma}$,
		\item $P_{\sigma}MP_{\sigma}$ is finite dimensional.
	\end{enumerate}
\end{theorem}

Notice that, because of point~\ref{ItemPeterWeyli} and~\ref{ItemPeterWeylii}, if $H_{\sigma}$ denotes the range of $P_{\sigma}$, then
\begin{equation}
	H=\bigoplus_{\sigma}H_{\sigma}.
\end{equation}

\begin{proof}
No proof.
\end{proof}

%---------------------------------------------------------------------------------------------------------------------------
					\subsection{Conventions and definitions}
%---------------------------------------------------------------------------------------------------------------------------

From now, all Hilbert space will be separable and measure spaces $(X,\mu)$ will be as follows. The topological space $X$ will be an Hausdorff, secondly countable metrisable space and $\mu$ will be the completion of a Borel measure which is finite on compact sets on $X$. By \defe{completion}{completion!of a measure}, we mean that all subset of a null set are measurable of measure zero.

Now let $X$ be a set. A \defe{field of Hilbert space}{field!of Hilbert space} over $X$ is a set of Hilbert spaces $H_x$ for each $x\in X$. A \defe{section}{section!of field of Hilbert space} of the field is a function $V\colon C\to \bigsqcup_{x\in X}H_x$ such that $V(x)\in H_x$. We often write $V_x$ instead of $V(x)$. The set of sections is denoted by $\Gamma\{ H_x \}$.

Let $H$ be a (separable) Hilbert space. A \defe{direct integral decomposition}{direct!integral decomposition} of $H$ is
\begin{itemize}
	\item a measure space $(X,\mu)$,
	\item a field of Hilbert spaces over $X$,
	\item a function from $H$ to the space of sections of the field that we denote by $v_x$, the value at $x$ of the section associated with $v\in H$
\end{itemize}
such that
\begin{enumerate}
	\item for every $v$ and $w$ in $H$, the function $x\mapsto\langle v_x, w_x\rangle $ is integrable and
	\begin{equation}
		\int_X\langle v_x, w_x\rangle d\mu(x)=\langle v, w\rangle ,
	\end{equation}
	\item if $u$ is any section of the field such that $x\mapsto\langle u_x, w_x\rangle $ is integrable for all $w\in H$, then $u$ is almost everywhere equal to the section associated with a vector of $H$,
	\item the set of all decompositions of vectors in $H$ is maximal.
\end{enumerate}

\begin{probleme}
I do not understand the precise signification if the maximality, cf problem~\ref{ProbMaximvE}.
\end{probleme}

Notice that the set of sections that are equal almost everywhere to sections obtained by decomposition of vectors in $H$ form a module over $L^{\infty}(X,\mu)$.


An technical fact that will be used in much of proofs is the following.

\begin{lemma}		\label{LemdensHdensHx}
If $\{ v_1,v_2,\ldots \}\subseteq H$ has a dense span in $H$, then for almost every $x\in X$, the set $\Span\{ v_{1x},v_{2x},\ldots \}$ is dense in $H_x$.
\end{lemma}

%---------------------------------------------------------------------------------------------------------------------------
					\subsection{Examples}
%---------------------------------------------------------------------------------------------------------------------------

%///////////////////////////////////////////////////////////////////////////////////////////////////////////////////////////
					\subsubsection{First example}
%///////////////////////////////////////////////////////////////////////////////////////////////////////////////////////////



Let $H_x$ with $x\in X$ be a field of Hilbert space over a countable set $X$. Then define $H=\bigoplus_{x\in X}H_x$. That Hilbert space decomposes by $(X,\mu)$ with $\mu$ being the counting measure, and to $v\in H$ corresponds a component $v_x$ in each $H_x$ that are taken as the section associated to the vector $v$.

%///////////////////////////////////////////////////////////////////////////////////////////////////////////////////////////
					\subsubsection{Second example}
%///////////////////////////////////////////////////////////////////////////////////////////////////////////////////////////


Let $(X,\mu)$ be the counting measure over the countable set $X$, and consider the Hilbert space $H=L^2(X,\mu)$. A decomposition of that Hilbert space is given as follows: first pose $H_x=\eC$ for each $x\in X$. Then, for each vector $v\in H$, we associate some choice of function $f$ in the equivalence class that defines $v$. Then we define the section associated with $v$ is $v_x=f(x)$.

Let us check that this construction fulfils the axioms of a direct integral decomposition of Hilbert space. First the function $x\mapsto \langle v_x, w_x\rangle $ has to be integrable. If $f$ is the function associated with $v$ and $g$ the one associated with $w$, we have $\langle v_x, w_x\rangle =f(x)\overline{ g(x) }$ which is integrable by Cauchy-Schwarz, and by the definition of the product in $L^2$, we have
\begin{equation}
	\langle v, w\rangle =\int_X f(x)\overline{ g(x) }\,d\mu(x).
\end{equation}
Second, suppose that $u$ is a section such that the map $x\mapsto u_x\overline{ f(x) }$ is integrable for every $f$ associated with a vector in $H$. The measured space $(X,\mu)$ being countable, $X$ is a union of finite measure sets $\{ X_i \}$ on each of them one can consider the characteristic function $1_i$. Then for every $i$, the function $x\mapsto u_x 1_i$ is measurable on $X_i$. That proves that $u$ is measurable. We conclude that it is square integrable and thus defines a function which is almost everywhere equal to the section associated with a vector $v\in L^2(X,\mu)$, namely the element of $L^2(X,\mu)$ which is the class of $u$.

%///////////////////////////////////////////////////////////////////////////////////////////////////////////////////////////
					\subsubsection{Third example}
%///////////////////////////////////////////////////////////////////////////////////////////////////////////////////////////

Let $H=L^2(X,\mu_1)\oplus L^2(X,\mu_2)$ and $\mu=\mu_1+\mu_2$ which is still a complete measure. By Radon-Nikod\'ym theorem~\ref{ThoRadonNikodym}, there exists positive measurable functions $f_1$ and $f_2$ such that $\mu_i=f_i\mu$. We can decompose $H$ as the direct integral
\begin{equation}
	H_x=l^2\Big(    \{ x \}\cap\{ f_1>0 \}\sqcup \{ x \}\cap\{ f_2>0 \}  \Big).
\end{equation}
Depending on the $x$, the set $\{ x \}\cap\{ f_1>0 \}\sqcup \{ x \}\cap\{ f_2>0 \}$ has zero, one or two elements. Now if $v=(v_1,v_2)\in H$ with $v_!=[g_1]$ and $v_2=[g_2]$, we define the section associated with $v$ as
\begin{equation}
	v_x=\big( f_1(x)^{1/2}g_1(x),f_2(x)^{1/2}g_2(x) \big).
\end{equation}
As notation, if $f_1(x)=0$, we identify this with the element $f_2(x)^{1/2}g_2(x)$ instead of the couple $(0,f_2(x)^{1/2}g_2(x))$.

%---------------------------------------------------------------------------------------------------------------------------
					\subsection{Decompositions of operators}
%---------------------------------------------------------------------------------------------------------------------------



Suppose that the separable Hilbert space $H$ is provided with a decomposition over $(X,\mu)$. A \defe{decomposition}{decomposition!of operator} of the operator $T\in \oB(H)$ is a family of bounded operators $T_x\colon H_x\to H_x$ such that for every $v\in H$, the condition
\begin{equation}		\label{EqCondTvDecompOps}
	(Tv)_x=T_xv_x
\end{equation}
holds for almost every $x\in X$.

\begin{remark}
The conditions~\ref{EqCondTvDecompOps} are in fact uncountably many conditions (one for each $v\in H$) each of them having the possibility to be wrong on a null set. One has thus to be prudent because an uncountable union of null set \emph{might} be a set of non zero measure.
\end{remark}


\begin{lemma}
The set of decomposable operators is a $*$-algebra.
\end{lemma}

\begin{proof}
If the operator $T$ is decomposable, the rule $(T^*)_x=(T_x)^*$ provides a decomposition of $T^*$.
\end{proof}

\begin{lemma}
Let $T$ be a decomposable operator. We have $T\geq 0$ if and only if $T_x\geq 0$ almost everywhere.
\end{lemma}

\begin{proof}
	In order to prove that $T$ is positive, we have to evaluate $\langle Tv, v\rangle $ from the decomposition of $T$ and $v$. We suppose that $T_x\geq 0$ for almost every $X\in X$ and we compute
	\begin{equation}
		\langle Tv, v\rangle =\int_X\langle (Tv)_x, v_x\rangle d\mu(x)=\int_X\langle T_xv_x, v_x\rangle d\mu(x)\geq 0.
	\end{equation}
	where we were allowed to change $(Tv)_x$ by $T_xv_x$ because the equality holds almost everywhere.

	For proving the contrary, let pick a dense, countable and rational vector subspace $H_0\subseteq H$, and assume that $T\geq 0$. If $v\in H$ and if $E\subseteq X$ is measurable, then by maximality of the Hilbert space decomposition there is some vector $v_E\in H$ such that
	\begin{equation}		\label{EqCondvEvx}
		(v_E)_x=
	\begin{cases}
		v_x&\text{if }x\in E\\
		0&\text{otherwise.}
	\end{cases}
	\end{equation}

	\begin{probleme}		\label{ProbMaximvE}
	I do not see why to use the maximality. Indeed, one considers the section
	\begin{equation}
		s_x=
	\begin{cases}
		v_x&\text{if }x\in E\\
		0&\text{otherwise.}
	\end{cases}
	\end{equation}
	By definition of decomposition of $v$, the integral
	\begin{equation}		\label{EqProbIntX}
		\int_X\langle v_x, w_x\rangle d\mu(x)
	\end{equation}
	exists for every $w\in H$. Now we have that
	\begin{equation}		\label{EqProbIntE}
		\int_X\langle s_x, w_x\rangle d\mu(x)= \int_E\langle v_x, w_x\rangle d\mu(x)
	\end{equation}
	whose existence is assured by existence of \eqref{EqProbIntX}, isn't?

	Existence of integral \eqref{EqProbIntE} assures the existence of a vector $v_E\in H$ such that $(v_E)_x=s_x$ almost everywhere. So we have the $v_E$ without maximality axiom. Do we really want the condition \eqref{EqCondvEvx} to hold everywhere and not only almost everywhere? Since we only use it in integrals, I think that one does not care about a violation of condition \eqref{EqCondvEvx} on a null set.
	\end{probleme}

	We have that
	\begin{equation}
		\int_E\langle T_xv_x, v_x\rangle d\mu(x)=\int_X\langle T_xv_{Ex}, v_{Ex}\rangle =\langle Tv_E, v_E\rangle \geq 0
	\end{equation}
	because $T\geq 0$ by assumption. So we proved that $x\mapsto\langle T_xv_x, v_x\rangle $ is an integrable function on $X$ for which the integral over any subset is a positive real number. Then the function is real positive almost everywhere. Thus we have
	\begin{equation}
		\langle T_xv_x, v_x\rangle \geq 0
	\end{equation}
	almost everywhere. Therefore there exists a null set $N\subseteq X$ such that $\langle T_xv_x, v_x\rangle \geq 0$ for every $v\in H_0$ whenever $x\in X\setminus N$. If $x\in X\setminus N$, then $\Span\{ v_x\tq v\in H_0 \}$ is dense in $H_x$ because of lemma~\ref{LemdensHdensHx}. Notice that $H_0$ is countable and there is one null set for each element of $H_0$ on which we are unsure of the sign of $\langle T_xv_x, v_x\rangle $.

	By enlarging the set $N$ we can assume that
	\begin{equation}
		(a_1v_1+a_2v_2)=a_1(v_1)_x+a_2(v_2)_x
	\end{equation}
	for every $x\in X\setminus N$ and $a_i\in\eQ[i]$. Indeed the equalities are an equation for each $a_i\in\eQ[i]$ and $v_i\in H_0$. That is thus a countable set of equations; thus the set where the equations are not satisfied is a countable union of null sets.

	What we have now is that every vector in $H_x$ (with $xin X\setminus N$) can be written as a limit if vectors in $H_0$, and then $\langle T_xv_x, v\rangle \geq 0$, so that $T_x\geq 0$.

\end{proof}

\begin{lemma}
We have
\begin{equation}
	\| T \|=\esssup\| T_x \|=\min\{ c\tq \| T_x \|\leq c\text{ almost everywhere} \}.
\end{equation}
\end{lemma}

\begin{proof}
	The inequality $\| T \|\leq\esssup\| T_x \|$ comes from the fact that
	\begin{equation}
		\| Tv \|=\int_X\langle T_xv_x, T_xv_x\rangle d\mu(x)
	\end{equation}
	which gives the bound. For the inverse inequality, remark that $\| T \|^2=\| T^*T \|$ and $(T^*T)_x=(T_x)^*T_x$ almost everywhere, so that one can replace $T$ by $T^*T$, or more simply we can assume that $T\geq 0$. Now consider the positive and self-adjoint operator $\| T \|\mtu-T$. The ``component'' operator is positive almost everywhere: $\| T \|\mtu=T_x\geq 0$ for almost every $x\in X$. We know that the maximum of the spectrum of $T_x$ is $\| T_x \|$, so that the number $\| TY \|-\| T_x \|$ is almost everywhere positive because it is an element of the spectrum of $\| T \|\mtu-T_x$ which is positive.
\end{proof}

%---------------------------------------------------------------------------------------------------------------------------
					\subsection{Diagonal and decomposable operators}
%---------------------------------------------------------------------------------------------------------------------------



An operator $S$ is \defe{diagonal}{diagonal operator} if it is decomposable and if $S_x$ is multiple of identity for almost every $x\in X$.

As motivation, consider the example $H=\eC^{n_1+n_2}=\eC^{n_1}\oplus\eC^{n_2}$, the measure space being two points with the counting measure. A diagonal operator looks like
\begin{equation}
	\begin{pmatrix}
	  \lambda_1\mtu	&		\\
		&	\lambda_2\mtu
	\end{pmatrix},
\end{equation}
while decomposable operators are
\begin{equation}
	\begin{pmatrix}
	 		 T_1	&		\\
			&	T_2
	\end{pmatrix}.
\end{equation}
One sees that these two sets are mutually commutant. We will see in theorem~\ref{ThoDecopDiagCommE} that this is a general fact that decomposable operators and diagonal operators are mutually commutant.

\begin{theorem}[Kaplansky density theorem]\index{Kaplansky density theorem}		\label{ThoKaplanskyDensity}
	If $A$ is a $*$-algebra of operators on $H$, then every strong limit of net in $A$ is a strong limit of a bounded net of elements in $A$. In other words, if $T=\lim_{\rightarrow}T_{\alpha}$, then there exist a net $S_{\beta}$ with $\| S_{\beta} \|\leq M$ (fixed $M$) such that $T=\lim_{\rightarrow}T_{\beta}$.
\end{theorem}

A complete proof is given in \cite{JonesVN}.

\begin{proof}[Sketch of proof]
	First, a convex subset of $\oB(\hH)$ is strongly closed if and only if it is weakly closed. So, using what is said on page \pageref{PgStarWeakRespecte} about the fact that weak topology is compatible with the involution, we know that if $T$ lies in the strong closure of $A$, and if $T=T^*$, then $T$ is the limit of a net of self-adjoint elements in $A$. That allows us to focus the proof of the theorem on self-adjoint elements of $A$.

	\begin{probleme}
	For me, the fact that this reasoning actually allows to restrict ourself to self-adjoint elements. But I suppose that it would become more clear when one understands the link between the statement here and the statement of Kaplansky's theorem in \cite{JonesVN}, page 32.
	\end{probleme}

	Consider the function $f(x)=2x/(1+x^2)$ that provides an homeomorphism from $[-1,1]$ to $[-1,1]$. Thus by continuous functional calculus, if $T=T^*$ and $\| T \|\leq 1$, we have $T=f(S)$ where $S=S^*$ and $S$ belongs to the norm closure of the algebra generated by $T$.

	Now if $T$ is the strong limit of the net $T_{\alpha}$ of self-adjoint operators, then $S$ is the strong limit of $S_{\alpha}$ with $S_{\alpha}\in A$ and $S^*_{\alpha}=S_{\alpha}$. Indeed one can prove that $S_{\alpha}\to S$ implies\footnote{In the weak topology, that implication is the continuous functional calculus, but in the strong topology (the one we are considering here), the validity of this assertion relies on the particular form of $f$.} $f(S_{\alpha})\to f(S)$.

	Since $\| f(S_{\alpha}) \|\leq 1$, thus $\| T \|\leq 1$.
\end{proof}



\begin{theorem}		\label{ThoDecopDiagCommE}
The algebra of decomposable and diagonal operators are mutually commutant.
\end{theorem}

\begin{proof}
	The fact that decomposable operators are in the commutant of diagonal operators and vice versa. The strategy to prove the theorem will be to first prove that decomposable and diagonal operators are von~Neumann algebras, and then show that every operator in the commutant of diagonal operators is decomposable. From there, the conclusion yields from the double commutant theorem.

	Suppose that $T$ is a strong limit of decomposable operators, i.e
	\begin{equation}
		T=\slim T_{\alpha}
	\end{equation}
	where $T_{\alpha}$ is decomposable and $\sup\{ | T_{\alpha} | \}<\infty$.
	 We want to show that $T$ is decomposable. For, let $H_0$ be a dense countable rational subspace of $H$. If $T$ is limit of a uniformly bounded \emph{net}, the fact that $H_0$ is countable makes that there is a \emph{sequence} of uniformly bounded decomposable operators $T_n$ such that $T_nv\to Tv$ for every $v\in H_0$.

	We have
	\begin{equation}
		\lim_{n\to\infty}\int_X \| T_{nx}v_x - (Tv)_x \|^2d\mu(x)=0
	\end{equation}
	for almost every $v\in H_0$. A general fact about $L^2$ spaces is that when a sequence goes to zero, then the functions (representative of the elements of $L^2$) themselves goes to zero on a subsequence. So for every given $v\in H_0$ there is a subsequence $\{ n_j \}$ such that
	\begin{equation}
		\| T_{n_jx}v_x-(Tv)_x \|\to 0
	\end{equation}
	for almost every $x\in X$. The subsequence might depends on the $v$, bu a diagonal argument provides a subsequence $\{ n_k \}$ of $\{ n_j \}$ such that for every $v\in H_0$,
	\begin{equation}
		\| T_{n_kx}v_x-(Tv)_x \|\to 0
	\end{equation}
	for almost every $x\in X$. These relations are a countable number of convergence almost everywhere. Then there exists a null set $N\subseteq X$ such that
	\begin{equation}		\label{EqTnkxTvxHorsN}
		\| T_{n_kx}v_x-(Tv)_x \|\to 0
	\end{equation}
	for every $v\in H_0$ and $x\in X\setminus N$.

	By enlarging again the null set $N$, we define
	\begin{equation}
		H_{0x}=\{ v_x\tq v\in H_0 \}
	\end{equation}
	for $x\in X\setminus N$. This is a dense subspace of $H_x$ by lemma~\ref{LemdensHdensHx}. Since the essential supremum of $\| T_{nx} \|$ is the norm of $T_n$, we have a ``double uniformly bounded'' relation
	\begin{equation}
		\sup_n\sup_x\| T_{nx} \|<\infty.
	\end{equation}
	Now for $x\in X\setminus N$ we define $T_x$ on $H_{0x}$ by
	\begin{equation}
		T_xv_x=\lim_{k\to\infty} T_{n_kx}v_x
	\end{equation}
	for each $v\in H_0$. That limit exists by construction.

	\begin{probleme}
		I suppose that this definition together with property \eqref{EqTnkxTvxHorsN} means that for every $w\in H_0$ and every $x\in X\setminus N$,
		\begin{equation}			\label{EqSurHzeroTwTxwx}
			(Tw)_x-T_xw_x=0.
		\end{equation}
	\end{probleme}

	It extends by continuity from $H_{0x}$ to $H_x$. The resulting operator is bounded because the sequence $T_{n_kx}$ is uniformly bounded.

	Now we enlarge once again the set $N$ by the null set appearing in the definition of the essential supremum we find that $T_x$ is bounded for every $x\in X\setminus N$. All the work make $\{ T_x \}_{x\in X}$ a candidate decomposition of $T$.

	In order to check that this actually is a decomposition of $T$, we have to prove that $(Tv)_x=T_xv_v$ for every $v\in H$ and almost every $x\in X$. We will prove that property by proving that the integral
	\begin{equation}
		\int_X\| (Tv)_x-T_xv_x \|\,d\mu(x)
	\end{equation}
	vanishes. For that, we choose $w\in H_0$ such that $\| w-v \|\leq\epsilon$ and, by virtue of \eqref{EqSurHzeroTwTxwx}, we add $T_xw_x-(Tw)_w$ in the integrand:
	\begin{equation}
		\begin{split}
			\int_X\| (Tv)_x-T_xv_x \|^2\,d\mu(x)	&= \int_X\| (Tv)_x -(Tw)_x+T_xw_x -T_xv_x \|^2\,d\mu(x)\\
								&\leq 2\int_X\| (Tv)_x-(Tw)_x\|^2\,d\mu(x)+\int_X\| T_xw_x-T_xv_x\|^2\,d\mu(x)\\
								&=2\| T(v-w) \|^2+2\| \tilde T(v-w) \|^2\\
								&=2\| T \|\,\| v-w \|^2 + 2\| \tilde T \|\,\| v-w \|^2\\
								&\leq \text{constant}\cdot \epsilon
		\end{split}
	\end{equation}
	where the operator $\tilde T$ is defined by $(\tilde Tw)_x=T_xw_x$.


	Thus the set of decomposable operators is strongly closed, so that it is a von~Neumann algebra. The same argument holds to prove that diagonal operators form a von~Neumann algebra too. We want now to prove that the commutant of diagonal operators is the set of decomposable. For that, it is sufficient to prove that every operator in the commutant of diagonals is decomposable. Since a von~Neumann algebra is generated by its projections , we have only to prove that every projection in the commutant of the diagonal is decomposable.

	Let $P$ be such a projection and enlarge $H_0$ in order to have $PH_0\subset H_0$. That remains a countable set because such an enlargement is $H'_0=H_0+PH_0$ for example. We choose a null set $N\subseteq X$ such that if $x\in X\setminus N$ then
	\begin{equation}
		(a_1v_1+a_2v_2)_x=a_1v_{1x}+a_2v_{2x}
	\end{equation}
	for every $v_1$ and $v_2$ in $H_0$ and $a_i\in\eQ[i]$. The set $H_{0x}=\{ v_x\tq v\in H_0 \}$ is dense in $H_x$. For $f\in L^{\infty}(X,\mu)$, we define the multiplication operator
	\begin{equation}
		(M_fv)_x=f(x)v_x,
	\end{equation}
	which is, by definition, a diagonal operator. Let $v\in PH_0$ and $w\in P^{\perp}H_0$, we have
	\begin{equation}
		\int_Xf(x)\langle v_x, w_x\rangle \,d\mu(x)= \int_X\langle f(x) v_x, w_x\rangle \,d\mu(x)=\langle M_fv, w\rangle .
	\end{equation}
	Since $P$ is in the commutant of diagonals, it commutes with $M_f$, so that $M_fv\in PH$ while $w\in P^{\perp}H_0$ and the latter product is thus zero: $\langle M_fv, w\rangle =0$. That proves that $x\mapsto\langle v_x, w_x\rangle $ is a function that integrates to zero when multiplicated\quext{My spelling corrector does not know that word. Bad word or bad corrector?} by any $L^{\infty}$ function. That means in turn that $\langle v_x, w_x\rangle =0$ almost everywhere. We enlarge $N$ in such a way that
	\begin{equation}		\label{EqPerpPHPperpH}
		\{ v_x\tq v\in PH_0 \}\perp\{ w_x\tq w\in P^{\perp}H_0 \}
	\end{equation}
	for every $x\in X\setminus N$. Of course, $PH_0\oplus P^{\perp}H_0=H_0$, so that the direct sum of the two spaces of \eqref{EqPerpPHPperpH} is $H_{0x}$. Taking the strong closure,
	\begin{equation}
		\overline{ \overline{ \{ v_x\tq v\in PH_0 \} } }\oplus\overline{ \overline{ \{w_x\tq w\in P^{\perp}H_0 \}} }=H_x
	\end{equation}
	where the double bar denotes the strong closure.
	%
	%	If someone knows ho to produce a double bar in a more intrinsically way that double \overline{  }, I am open to learn.
	%
	Now, for $x\in C\setminus N$, define $P_x\colon H_x\to H_x$, the projection onto $\overline{ \overline{ \{ v_x\tq v\in PH_0 \} } }$. By the same reasoning as for $T$ before, we prove that $\{ P_x \}$ decomposes $P$.
\end{proof}

\begin{lemma}		\label{LemMabelAstrDense}
If $M$ is an abelian von~Neumann algebra on a separable Hilbert space $H$, then $M$ has a strongly dense $C^*$-algebra $A$
\end{lemma}

\begin{proof}
No proof.
\end{proof}

\begin{theorem}		\label{ThoVNableHDiag}
If $M$ is an abelian von~Neumann algebra on a (as usual separable) Hilbert space $H$, then there is a decompositon of $H$ such that $M$ is the set of diagonal operators.
\end{theorem}

\begin{proof}
	Pick a $A\subseteq M$ as in the lemma~\ref{LemMabelAstrDense}, and add an unit is needed. Then we have $A\simeq C(X)$ for some metrisable compact space $X$ by the
	\href{http://en.wikipedia.org/wiki/Gelfand_isomorphism}{Gelfand theorem} (see \cite{Landsman}). We can write
	\begin{equation}
		H=\bigoplus_{k=1}^{\infty}H_k
	\end{equation}
	where $H_k$ is invariant under the action of $A$ and has a cyclic vector. Indeed, pick a vector $v_1$ in $H$ and define $H_1=Av_1$, then pick $v_2$ in the complement and continue. We have an isomorphism $H_k\simeq L^2(X,\mu_k)$ where $\mu_k$ is some probability measure by $v_k\mapsto 1$ and $Tv_k\mapsto\hat T$ where the hat denotes the Gelfand isomorphism.

	\begin{probleme}
		That gives an element of $L^2(X,\mu_k)$ for each vector inside $H_k$. But I do not see the isomorphism. For example a simple step function belongs to $L^2(X,\mu_k)$ and corresponds to which element of $H_k$?

		The map
		\begin{equation}
			\begin{aligned}
			 \psi\colon Av_1&\to L^2(X,\mu) \\
			   Tv_1&\mapsto \hat T
			\end{aligned}
			\end{equation}
		(whose image is included in $L^2(X,\mu)$ because $X$ is compact, so that every continuous function is square integrable with respect to a probability measure) is injective because two different continuous functions cannot belong to the same class in $L^2(X,\mu)$.

		For me, surjectivity is not clear (and even wrong) because there exists many elements in $L^2(X,\mu)$ who have no continuous representative. Do we have to game with limits and exploit the fact that $A$ is strongly closed?

		\end{probleme}
	So we can assume that $A=C(X)$ and $H=\bigoplus_kL^2(X,\mu)$. Let
	\begin{equation}
		\mu=\sum_{k=1}^{\infty}2^{-k}\mu_k
	\end{equation}
	which is still a probability measure, and define $H_x=l^2(\eN,\mu_x)$ where $\mu_x$ is the measure on $\eN$ defined by
	\begin{equation}
		\mu_x\big( \{ k \} \big)=\frac{ d\mu_k }{ d\mu }(x)
	\end{equation}
	by the Radon-Niked\'ym theorem. For recall, $d\mu_k/d\mu$ is the function on $X$ defined by the fact that $d\mu_k=(d\mu_k/d\mu)d\mu$ in the sense that
	\begin{equation}
		\int_Xfd\mu_k=\int_Xf\frac{ d\mu_k }{ d\mu }d\mu
	\end{equation}
	for every functions $f$ on $X$. Now for each $\varphi=(\varphi_1,\varphi_2,\ldots)\in H$, we define $\varphi_x\in H_x$ by
	\begin{equation}
	\begin{aligned}
	 \varphi_x\colon \eN&\to \eC \\
	   k&\mapsto \varphi_k(x).
	\end{aligned}
	\end{equation}
	One has to show that the so defined $\varphi_x$ is actually an element of $l^2(\eN,\mu_x)$, that is
	\begin{equation}		\label{EqSumVpNK}
		\sum_k| \varphi_k(x) |^2\frac{ d\mu_k }{ d\mu }(x)
	\end{equation}
	has to be finite. For, we have the computation
	\begin{equation}		\label{EqIntXsumnormVarPhi}
	\begin{split}
		\int_X\sum_k| \varphi_k(x) |^2\frac{ d\mu_k }{ d\mu }(x)\,d\mu(x)
			&=\sum_k\int_X| \varphi_k(x) |^2\frac{ d\mu_k }{ d\mu }(x)\,d\mu(x)\\
			& = \sum_k\int_X| \varphi_k |^2\,d\mu_k(x)\\
			& = \sum_k\| \varphi_j \|^2\\
			& = \| \varphi \|^2.
	\end{split}
	\end{equation}
	where, in the first line we permuted the sum and the integral using the monotone convergence theorem. The fact that integral \eqref{EqIntXsumnormVarPhi} is finite proves that the function \eqref{EqSumVpNK} has finite values almost everywhere, or
	\begin{equation}
		\| \varphi_x \|^2_{l^2(\eN,\mu_x)}<\infty
	\end{equation}
	almost everywhere. We redefine now $\varphi_x$ as zero for the $x$ for which the former definition gives $\| \varphi_x \|^2_{l^2(\eN,\mu_x)}$. This is a redefinition over a null set.

	One can check that this construction provides a decomposition of $H$ for which $M$ is the algebra of diagonal operators.

	\begin{probleme}
	To be done\ldots
	\end{probleme}
\end{proof}

Let $M$ be an abelian von~Neumann algebra of operators on $H$ and suppose that there are two decompositions $\{ H_x \}$ and $\{ H_y \}$ of $H$ using $M$ by theorem~\ref{ThoVNableHDiag}. Let now
\begin{equation}
	M_2=
\left\{
\begin{pmatrix}
  T	&		\\
  	&	T
\end{pmatrix}
 \right\}\tq T\in M
\end{equation}
That von~Neumann algebra decomposes $H\oplus H$ into $\{ H_x \}\oplus\{ H_y \}$. The matrix
\begin{equation}
	S=
\begin{pmatrix}
  0	&	1	\\
  1	&	0
\end{pmatrix}
\end{equation}
is decomposable because it belongs to the commutant of $M_2$. Thus it provides an unitary isomorphism of $H$ that applies $H_x$ on $H_y$.

\begin{proposition}		\label{PropNprimexxNprime}
	If $N$ lies in the commutant of $M$, there are von~Neumann algebras $N_x\subseteq\oB(H_x)$ for almost every $x$ such that
	\begin{enumerate}
		\item $N_x=\{ T_x\tq T\in N \}''$,
		\item $(N_x)'=(N')_x$ almost everywhere.
	\end{enumerate}
\end{proposition}

\begin{proof}
No proof.
\end{proof}

%---------------------------------------------------------------------------------------------------------------------------
					\subsection{Decompositions of representations}
%---------------------------------------------------------------------------------------------------------------------------



Let $A$ be a separable $C^*$-algebra and $\pi\colon A\to \oB(H)$ be a representation of $A$ on a separable Hilbert space. We do not assume that $\pi$ is faithful.

\begin{lemma}		\label{LemHDecPipixxpi}
If $H$ is provided with a decomposition $\{ H_x \}$ and if $\pi(A)$ consists of decomposable operators, then there are representations $\pi_x\colon A\to \oB(H_x)$ such that $\pi(a)_x=\pi_x(a)$.
\end{lemma}

\begin{proof}
No proof.
\end{proof}

\begin{theorem}		\label{ThoRepDecSepIrrep}
	If $\pi\colon A\to \oB(H)$ is any representation of a separable $C^*$-algebra, then there is a decomposition of $H$ such that $\pi(A)$ consists of decomposable operators. There also exists a decomposition $\{ \pi_x \}$ of $\pi$ such that each $\pi_x$ is irreducible.
\end{theorem}

That theorem says that every representation decomposes into irreducible.

\begin{proof}

	We do not prove the first part, and suppose then that $\{ \pi_x \}$ is a decomposition given by lemma~\ref{LemHDecPipixxpi}, so that $\pi_x(a)=\pi(a)_x$.

	Let $M$ be a maximal (Zorn's lemma) abelian von~Neumann subalgebra of $\big( \pi(A) \big)'$, and decompose $H$ so that $M$ is the algebra of diagonal operators. Now we consider $N$, the von~Neumann algebra generated by $M$ and $\pi(A)$. An element in $N'$ belongs to $\pi(A)'$ and commutes with $M$ which is maximal. Thus $N'=M$.

	By the property of the decomposition $\pi_x$, an element of $\pi_x(A)'$ has to commute with all $M$, so that
	\begin{equation}
		\pi_x(A)'=(N_x)'=(N')_x=M_x=\eC\mtu,
	\end{equation}
	and by Schur's lemma, the representation $\pi_x$ is irreducible almost everywhere.
\end{proof}

As an example, take $A=C^*(F_2)$ where $F_2$ is the free group with two generators and $A$ is formed by taking all the linear combination and then the closure as $C^*$-algebra. One can show that
\begin{enumerate}
	\item There are two decompositions of the regular representation $\pi$ of $A$ on $H=l^2(F_2)$ that we denote by $\{ H_x \}$ and $\{ H_y \}$. These decompositions are \emph{a priori} built on different measured spaces.
	\item All the representations in $\{ \pi_x \}\sqcup\{ \pi'_y \}$ are irreducible and mutually inequivalent.
\end{enumerate}
So the decomposition in irreducible representations is not unique at all, in contrast to the group representation case. The point is that the regular representation of $C^*(F_2)$ is a factor, while the following theorem only assures unicity of decomposition into factors.

\begin{theorem}
	Let $\pi\colon A\to \oB(H)$ be a representation of the $C^*$-algebra $A$. There is an essentially unique decomposition of $H$ such that
	\begin{enumerate}
		\item the representation $\pi(A)$ consists of decomposable operators,
		\item the representations $\pi_x$ are factors representations for almost every $x$.

		By ``essentially unique'', one means that if
		\begin{align}
			\big( X,\mu,\{ H_x \}, H\to\Gamma\{ H_x \} \big)&\text{and}&\big( X',\mu',\{ H'_x \}, H\to\Gamma\{ H'_x \} \big)
		\end{align}
		are two decompositions, there exists a map $X\to X'$ which is almost everywhere an equivalence of measurable spaces. That is it maps $\mu$ to a measure which is mutually absolutely continuous with $\mu'$, and an unitary isomorphism $H_x\to H'_x$ defined for almost every $X$ which maps $v_x$ to $v'_x$ where $v'_x$ is the decomposition of $v$ with respect to the decomposition $\{ H'_x \}$.
	\end{enumerate}
\end{theorem}

\begin{proof}
	Let us prove the existence part of the theorem For we proceed as in the proof of theorem~\ref{ThoRepDecSepIrrep}, but we take $M=\pi(A)''\cap\pi(A)'$, this is the center of the von~Neumann algebra generated by $\pi(A)$. Using proposition~\ref{PropNprimexxNprime}, we find
	\begin{equation}
		\pi_x(A)''\cap\pi_x(A)'=\big( \pi(A)''\cap\pi(A)' \big)_x=\eC\mtu.
	\end{equation}
	That shows that $\pi_x(A)$ is a factor.
\end{proof}

A $C^*$-algebra is \defe{liminal}{liminal $C^*$-algebra } if for every irreducible representation $\pi\colon A\to \oB(H)$, we have $\pi(A)=\oK(H)$, the space of compact operators on $H$.

\begin{theorem}
If $G$ is a semisimple Lie group, then $C^*(G)$ is liminal.
\end{theorem}
\begin{proof}
No proof.
\end{proof}

\begin{theorem}
If $A$ is a liminal $C^*$-algebra, then every factor representation is of type I.
\end{theorem}

\begin{theorem}
	If $\pi$ is any representation of a liminal $C^*$-algebra on a separable Hilbert space, then there is a decomposition of $H$ such that
	\begin{enumerate}
		\item the space $H_x$ carries a representation which is a factor of type I,
		\item $H_x=H_{\pi_x}\otimes L_x$ where $H_{\pi_x}$ is an irreducible representation of $A$, all mutually inequivalent and $L_x$ is some Hilbert space which says the ``multiplicity'' of the representation $\pi_x$ inside $\pi$.
	\end{enumerate}
	That decomposition is essentially unique.
\end{theorem}

\begin{proof}
No proof.
\end{proof}

\begin{lemma}
Every irreducible representation of $A_1\otimes A_2$ with both $A_i$ being liminal is a tensor product of irreducible representations of $A_1$ and $A_2$.
\end{lemma}

Now take an unimodular group $G$ such that $C^*(G)$ is liminal. One knows that the von~Neumann algebras generated by the left and right regular representations are mutually commutant. We consider the bi-regular representation $G\times G$ on $L^2(G)$. The commutant of that representation is equal to the center of the von~Neumann algebra generated by the two regular representation and is in particular abelian.

Using the lemma, the space $ L^2(G)$ decomposes into representations $H_{\pi\tau}=H_{\pi}\otimes H_{\tau}$ of $G\times G$ where $\pi$ and $\tau$ are irreducible representations. From the unimodular assumption, the symmetry $f(g)\to f(g^{-1})$ is an isometry that intertwines left and right regular representations. Notice that a function can be seen as a vector on $L^2(G)$ as well as as an operator over $L^2(G)$ by the convolution. Thus we turn $H_{\pi\tau}$ into a $*$-algebra by the multiplication $H_{\pi\tau}\otimes H_{\pi\tau}\tau H_{\pi\tau}$.

We denote by $\hat G$\nomenclature{$\hat G$}{Set of irreducible representations of the group $G$} the set of irreducible representations of $G$

\begin{theorem}
	Let $G$ be an unimodular locally compact liminal group. There is an unique measure on $\hat G$, the \defe{Plancherel measure}{plancherel measure}, denoted by $\mu$ such that for every $f\in L^1(G)\cap L^2(G)$ (as vector in $L^2(G)$ or as operator by involution),
	\begin{equation}
		\| f \|^2_{L^2(G)}=\int_{\hat G}\| \pi(f) \|^2_{HS}\,d\mu(\pi)
	\end{equation}
	where $\| . \|_{HS}$ denotes the Hilbert-Schmidt operator norm. Moreover we have
	\begin{equation}
		\pi(f)=\int_G f(g)\pi(g)\,dg.
	\end{equation}
\end{theorem}

That theorem has to be compared to the Fourier theory for abelian groups like $\eR$ or $S^1$, in particular the \href{http://en.wikipedia.org/wiki/Parseval's_identity}{Parseval equality}. This has also to be compared with the following theorem.

\begin{theorem}[Peter-Weyl]
If $G$ is a compact group, then
\begin{equation}
	\| f \|_{L^2(G)}^2=\sum_{\hat G}\frac{ \dim(\pi) }{ \volume(G) }\| \pi(f) \|^2_{HS}
\end{equation}
where the ration $\dim(\pi)/\volume(G)$ is the Plancherel measure made explicit in the compact case.
\end{theorem}


%+++++++++++++++++++++++++++++++++++++++++++++++++++++++++++++++++++++++++++++++++++++++++++++++++++++++++++++++++++++++++++
					\section{Index theory}
%+++++++++++++++++++++++++++++++++++++++++++++++++++++++++++++++++++++++++++++++++++++++++++++++++++++++++++++++++++++++++++


One speaks about index and subfactors in \cite{JonesVN,ConnesNCG,JonesSunder}.

%---------------------------------------------------------------------------------------------------------------------------
					\subsection{Introduction}
%---------------------------------------------------------------------------------------------------------------------------



Let $M$ be a type $II_1$ factor, which, thus, possesses an unique normal, tracial, normalized faithful state. We can assign a dimension to every modules over $M$ by theorem~\ref{ThoPropDimiM}. In particular, $M$ being an algebra of operators on the Hilbert space $\hH$, the space $\hH$ is a $M$-module and we can consider the number
\begin{equation}
	\dim_M(\hH).
\end{equation}
Following the cases, that can be any number in $]0,\infty[$. Indeed, taking any $v\in\hH$, we have the $M$-module map
\begin{equation}
\begin{aligned}
 \rho\colon M&\to \hH \\
   T&\mapsto Tv ,
\end{aligned}
\end{equation}
so that by lemma~\ref{LemHinjectifHdimdim}, we have
\begin{equation}
	\dim_M(\hH)\geq\dim_MM.
\end{equation}
But we saw on page \pageref{subsubsecExemDimMMMod} that $\dim_MM=\tr(\mtu)$, see definition \eqref{EqPreDefDimModuleRA} and bellow. We deduce that $\dim_M\hH\geq \tr(\mtu)$, which can be any non vanishing positive real number.

Let $N\subseteq M\subseteq \oB(\hH)$ be a subfactor, i.e. that $N$ is a factor in $\oB(\hH)$ and $N\subseteq M$. For the same reason as before, $\dim_N(\hH)\in]0,\infty[$. We consider the ratio
\begin{equation}
	[M:N]=\frac{ \dim_N(\hH) }{ \dim_M(\hH) }
\end{equation}
that is called the \defe{index}{index}\nomenclature{$[M:N]$}{The index of two modules}.

The theorem that we are intend to prove is the following.
\begin{theorem}[Jones]
If $[M:N]<4$, then $[M:N]=4\cos^2(\pi/n)$ for some $n<2$.
\end{theorem}

\begin{proof}
No proof.
\end{proof}

\begin{center}
\begin{tabular}{ccr}
	$n$		&	$4\cos^2(\pi/n)$\\
\hline
	$2$		&	$0$\\
	$3$		&	$1$\\
	$4$		&	$2$\\
	$5$		&	$2.618$		& the golden ratio\\
	$6$		&	$3$\\
	\vdots		&	\vdots
\end{tabular}
\end{center}
In the dots, we have an increasing sequence of numbers bigger than $3$ that converges to $4$.

%---------------------------------------------------------------------------------------------------------------------------
					\subsection{Example}
%---------------------------------------------------------------------------------------------------------------------------


Let $G$ be a discrete group with infinite conjugacy classes and $H$, a subgroup which has infinite conjugacy classes for its own right. Consider $M=M(G)\subseteq\oB\big(l^2(G))$ and $N=M(H)\subseteq\oB\big( l^2(H) \big)$. By simple inclusion of $H$ in $G$, we have $M(H)\subseteq\oB\big( l^2(G) \big)$ too.

As defined in subsection~\ref{sssOnePartCaseMG}, the von~Neumann algebra $M(H)$ is generated by the operators $\mU_h$ with $h\in H$. We already proved in proposition~\ref{ProplDeuxGFGP}\label{PglDeuxGFGPutiliseIci} that $l^2(G)$ is a finitely generated projective module over $M(G)$, so that one can address the question of $\dim_{M(G)}\big( l^2(G) \big)$ using the decomposition given by corollary~\ref{CorEfgpFssIsom}:
\begin{equation}
	l^2(G)=\Cl_{l^2(G)}\big( M(G) \big)\oplus l^2(G)/M(G).
\end{equation}
On the one hand, by proposition~\ref{PropDimClEgalDim}, we have $\dim\Big( \Cl_{l^2(G)}\big( M(G) \big) \Big)=\dim\big( M(G) \big)$, and on the other hand, one can prove that $l^2(G)/M(G)$ is a torsion, so that
\begin{equation}		\label{EqdimmGlDeuxGbig}
	\dim_{M(G)}\big( l^2(G) \big)=\dim_{M(G)}\big( M(G) \big)=1
\end{equation}
when a correct choice of normalisation is done.

\begin{probleme}
Is it correct that the latter dimension is in fact the trace of $\mtu$ on $M(G)$ which has to be normalised?
\end{probleme}

Let us now consider $g_1,\ldots,g_n$ be representatives of the classes of $G/H$ and look at the decomposition
\begin{equation}
	l^2(G)=l^2(Hg_1)\oplus\ldots\oplus l^2(Hg_n).
\end{equation}
Since $l^2(Hg_i)$ is a submodule of $l^2(H)$, each of $\dim_n\big( l^2(G) \big)$ lies in the same case as equation \eqref{EqdimmGlDeuxGbig}, so that
\begin{equation}
	\dim_N\big( l^2(G) \big)=n.
\end{equation}
Thus $[M:N]=n$.

One can explicitly construct examples of index for every real bigger than $4$.

Let $M\subseteq\oB(\hH)$ be a von~Neumann algebra and form $M^{(\infty)}\subseteq\oB(\hH\oplus\ldots\oplus\hH\oplus\ldots)$,
\begin{equation}
	M^{(\infty)}=\big\{  (T,T,\ldots)\tq T\in M  \big\}.
\end{equation}
Since $M^{(\infty)}$ does not contains more information than $M$ itself, we want to be able to say that they are the same von~Neumann algebra. The interest of ultraweak topology described in subsection~\ref{subSecUltraWtopol} is that if we consider both $M$ and $M^{(\infty)}$ with the ultraweak topology, then they are homeomorphic in a natural way.


%---------------------------------------------------------------------------------------------------------------------------
					\subsection{Isomorphisms of abstract von~Neumann algebras}
%---------------------------------------------------------------------------------------------------------------------------

If $M\subseteq\oB(\hH)$ is an ultraweakly closed subspace (what a von~Neumann algebra always is), then the Hahn-Banach theorem applies and in particular, every von~Neumann algebra is the dual of a space.

A positive linear map $\Phi$ between von~Neumann algebras is \defe{normal}{normal!map between von~Neumann algebras} if
\begin{equation}
	\Phi(\bigvee_{\alpha}P_{\alpha})=\bigvee_{\alpha}\Phi(P_{\alpha})
\end{equation}
for every increasing net of projections. A net of projections being said \defe{increasing}{increasing!net of projections} when $\alpha>\beta$ implies $P_{\alpha}>P_{\beta}$. An example of a normal map is the trace of a type $II_1$ factor.

\begin{theorem}		\label{ThoDixLinVNanormifffuwc}
A positive linear functional on a von~Neumann algebra is normal if and only if it us ultraweakly continuous.
\end{theorem}
A proof can be found in \cite{DixDecompBk}.

\begin{probleme}
It is the right citationm, isn't?
\end{probleme}

\begin{theorem}				\label{ThoPhiEquivuwcvna}
Let $\Phi\colon M\to \oB(\hH)$ be a $*$-homomorphism from a von~Neumann algebra into $\oB(\hH)$. The the following are equivalent:
\begin{enumerate}
\item\label{ItemPhiEquivuwcvnai} $\Phi$ is ultraweakly continuous,
\item\label{ItemPhiEquivuwcvnaii}  $\Phi$ is normal,
\item\label{ItemPhiEquivuwcvnaiii}  $\Phi(M)$ is a von~Neumann algebra.
\end{enumerate}
\end{theorem}
Notice that we do not suppose $M$ to be a von~Neumann algebra acting on the Hilbert space $\hH$.

\begin{proof}
The implication~\ref{ItemPhiEquivuwcvnaiii} $\Rightarrow$~\ref{ItemPhiEquivuwcvnaii} is proved by using the theorem~\ref{ThoDixLinVNanormifffuwc}. The implication~\ref{ItemPhiEquivuwcvnaii} $\Rightarrow$~\ref{ItemPhiEquivuwcvnai} is only the fact that ultraweak continuity is defined in terms of linear functionals.

The difficult part is to prove that~\ref{ItemPhiEquivuwcvnai} implies~\ref{ItemPhiEquivuwcvnaiii}. Since $\Phi(M)$ is a $*$-algebra, proving that it is strongly closed proves that it is a von~Neumann algebra. So, let us suppose $T$ to be in the strong closure of $\Phi(M)$. By Kaplansky density theorem~\ref{ThoKaplanskyDensity}, there exists a net $T_{\alpha}\in\Phi(M)$ with $\sup\| T_{\alpha} \|<\infty$ and
\begin{equation}
	T=\slim T_{\alpha}=\slim \Phi(S_{\alpha})
\end{equation}
for some $S_{\alpha}$. Since $\Phi$ is isometric, the set $\{ S_{\alpha} \}$ is uniformly bounded: $\| S_{\alpha} \|<\infty$. Thus, using the Banach-Alaogu theorem~\ref{ThoBanachAlaoglu}, and taking a subnet, we can assume that there is a $S\in M$ such that
\begin{equation}
	S=\uwlim S_{\alpha}.
\end{equation}
Since $\Phi$ is ultraweakly continuous by assumption, we have $\Phi(S)=\uwlim\Phi(S_{\alpha})$. Since the limit $T=\slim T_{\alpha}$ exists, the same exists in the weak topology (and is equal), thus $T=\Phi(S)$, which proves that $\Phi(M)$ is strongly closed.
\end{proof}

So if a von~Neumann algebra is algebraically realised as two different algebras of operators on two different Hilbert spaces, the two realisations are homeomorphic. For short, we say that $*$-isomorphisms of von~Neumann algebras correspond to weakly continuous maps between realisations of them.

Remark that a map $\Phi$ which fulfils the theorem~\ref{ThoPhiEquivuwcvna} is automatically isometric because $\| T \|$ is the spectral radius of $| T |$ while the spectral radius (which is a purely algebraic notion) does not change when we consider $T$ as an operator acting on one Hilbert space or an other. So, the map $\Phi$ preserves the $C^*$-algebra structure.

Let $M$ be a finite factor. We denote by $L^2(M)$\nomenclature{$L^2(M)$}{Hilbert space of the standard form for a finite factor}\label{PgLdM} the completion of $M$ in the norm derived  from the inner product
\begin{equation}
	\langle T_1, T_2\rangle =\tau(T_1T_2^*).
\end{equation}
This is the GNS construction applied to the tracial state $\tau$. The \defe{standard form}{standard!form of a von~Neumann algebra} of the finite factor $M$ is the realisation of $M$ as left multiplication operators on $L^2(M)$. The image is a von~Neumann algebra because the representation is normal and faithful.


%---------------------------------------------------------------------------------------------------------------------------
					\subsection{Index of finite subfactors}
%---------------------------------------------------------------------------------------------------------------------------

We say that the vector $v\in L^2(M)$ is \defe{bounded}{bounded!element of $L^2(M)$} by the positive real $K$ if for every projection $P\in M$, the condition
\begin{equation}
	\| Pv \|^2_{L^2(M)}\leq K\tau(P).
\end{equation}

\begin{lemma}
An element $v\in L^2(M)$ belongs to $M$ if and only if it is bounded.
\end{lemma}
\begin{proof}
No proof.
\end{proof}

We consider the map
\begin{equation}
\begin{aligned}
 J\colon L^2(M)&\to L^2(M) \\
   T&\mapsto T^*.
\end{aligned}
\end{equation}
\emph{A priori}, that is only defined on $M$, but it is an isometry because
\begin{equation}
	\| T \|=\tr(TT^*)=\tr(T^*T)=\| T^* \|.
\end{equation}
So it extends to the completion $L^2(M)$.

\begin{lemma}
We have
\begin{equation}
	M'=JMJ
\end{equation}
in the standard form.
\end{lemma}
Notice that operators in $JMJ$ are operators of right multiplication on $M$ because $(JTJ)S=JT(S^*)=J(TS^*)=ST^*$ while the elements of $M$ are left multiplications, so that it is clear that $JMJ\subseteq M'$.

\begin{proof}
No proof.
\end{proof}

Let $M_0$ be a subfactor of a finite factor $M_1$. We define the \defe{index}{index}\nomenclature{$[M_1:M_0]$}{Index of a subfactor.}
\begin{equation}
	[M_1:M_0]=\dim_{M_0}\big( L^2(M_1) \big).
\end{equation}

\begin{theorem}
One has
\begin{equation}
	[M_1:M_0]=\dim_{M_0}(M_1).
\end{equation}
and moreover
\begin{equation}
	[M_1:M_0]=\frac{ \dim_{M_0}(\hH) }{   \dim_{M_1}(\hH) }
\end{equation}
 if $M_1$ is represented on the Hilbert space $\hH$.
\end{theorem}
\begin{proof}
No proof.
\end{proof}

Suppose given $M_1$ and a finite subfactor $M_0$ and suppose $[M_1:M_0]<\infty$. Since $M_0$ is a subfactor of $M_1$, we can see $M_0'$ as subset of $\oB\big( L^2(M_1) \big)$. If $M_1'$ is not finite, then there is a part of $L^2(M_1)$ which is identified with $L^2(M_1)$ as $M_0$-module. That part has obviously the same dimension as the whole $L^2(M_1)$ (because the module structure is the same). Thus in this case, $L^2(M_1)$ contains infinitely many submodules with all the same dimension, so that $\dim_{M_0}\big( L^2(M_1) \big)=\infty$, which contradicts the assumption. We deduce that
\begin{equation}
	M_0'\subseteq\oB\big( L^2(M_1) \big)
\end{equation}
is a finite von~Neumann algebra. We consider
\begin{equation}
	P_1\colon L^2(M_1)\to L^2(M_1)
\end{equation}
be the orthogonal projection onto $L^2(M_0)\subseteq L^2(M_1)$.


\begin{lemma}		\label{LemTMunPTTPiffTMzero}
If $T\in M_1$, then $TP_1=P_1T$ if and only if $T\in M_0$.
\end{lemma}

\begin{proof}
It $T\in M_0$, we have
\begin{equation}		\label{EqTPPTPTProj}
	TP_1=P_1TP_1.
\end{equation}
Taking the adjoint of that equation, $P_1T^*=P_1T^*P_1$, and using the relation \eqref{EqTPPTPTProj} for $T^*$, we find $P_1T^*=T^*P_1$ which holds for every $T\in M_0$. Thus we have
\begin{equation}
	TP_1=P_1T
\end{equation}
for every $T\in M_0$.

For the reverse sense, equality $TP_1=P_1T$ is an equality of operators on $M_1$. Let us apply it to $\mtu\in M_1$. Since $\mtu$ belongs to $M_0$ too (every von~Neumann algebra contains the unit), $P_1(\mtu)=\mtu$ and we stay with $T=P_1T$, so that $T\in M_0$.
\end{proof}

Let now $M_2$ be a von~Neumann algebra generated by $M_1$ and $P_1$
\begin{equation}
	M_2=\{ M_1,P_1 \}''.
\end{equation}
The properties of $M_1$, $P_1$ and $M_2$ are summarized in the following theorem.
\begin{theorem}
	If $[M_1:M_0]<\infty$, then
\begin{enumerate}
\item $M_2$ is a finite factor, in particular it has a unique normal faithful trace $\tau$,
\item $[M_2:M_1]=[M_1:M_0]$,
\item $\tau(P_1)=[M_1:M_0]^{-1}$.
\end{enumerate}
\end{theorem}
\begin{proof}
No proof.
\end{proof}

The von~Neumann algebra $M_2$ was created from $M_0$ and $M_1$, while the theorem shows that $M_1$ has the same properties in $M_2$ than $M_0$ in $M_1$, so that one can redo the construction starting from $M_2$ and $M_1$ instead of $M_0$ and $M_1$ to build $M_3$ and $P_2$. The index does not change and the projection $P_2$ still has the same trace.

%---------------------------------------------------------------------------------------------------------------------------
					\subsection{Subfactors}
%---------------------------------------------------------------------------------------------------------------------------

\begin{probleme}
Je pense qu'à partir d'ici, j'ai inversé $M_0$ et $M_1$ partout jusqu'à la subsection~\ref{SubSecPropSeqMO}.
\end{probleme}

Let $M_1\subseteq M_0$ be a subfactor of the finite factor $M_0$, and $P\colon L^2(M_0)\to L^2(M_0)$ be the projection onto $L^2(M_1)$. Notice that $L^2(M_1)\subseteq L^2(M_0)$ because the trace on $M_1$ is the restriction to the one of $M_0$ from unicity. Now we consider $\langle M_0, P\rangle $\nomenclature{$\langle M_0, P\rangle $}{The von~Neumann algebra generated by $M_0$ and $P$}, the von~Neumann algebra generated by $M_0$ and $P$.

\begin{lemma}		\label{LemPMNinclu}
We have $P(M_0)\subseteq M_1$.
\end{lemma}
\begin{proof}
No proof.
\end{proof}

\begin{proposition}		\label{PropPropMappMN}
The map
\begin{equation}
\begin{aligned}
p \colon M_0&\to M_1 \\
   T&\mapsto P(T).
\end{aligned}
\end{equation}
satisfies
\begin{enumerate}
\item\label{ItemPropMappMNi} $p(T^*)=p(T)^*$,
\item\label{ItemPropMappMNii}$p(T^*T)\geq 0$,
\item\label{ItemPropMappMNiii} $p(T^*T)=0$ if and only if $T=0$,
\item\label{ItemPropMappMNiv} $p(S_1TS_2)=S_1p(T)S_2$ for every $S_1$ and $S_2$ in $M_1$.
\end{enumerate}
for every $T\in M_0$.
\end{proposition}

\begin{proof}
The point~\ref{ItemPropMappMNi} is the fact that $M_1$ is closed under the involution. The point~\ref{ItemPropMappMNii} comes form lemma~\ref{LemPMNinclu} and the positivity property in $M_1$. For the point~\ref{ItemPropMappMNiii}, remark that $\tr\big( p(T) \big)=\tr(T)$ because $p(\mtu)=\mtu$, while by unicity up to normalisation, the trace is determined by its value on~$\mtu$.
\end{proof}

A basic implication if~\ref{ItemPropMappMNi} is that
\begin{equation}
	PJ=JP.
\end{equation}


\begin{lemma}		\label{LemPTPpTPopLdeux}
If $T\in M_0$, then
\begin{equation}
	PTP=p(T)P
\end{equation}
as operators on $L^2(M_0)$.
\end{lemma}

\begin{proof}
No proof.
\end{proof}

\begin{lemma}		\label{LemNMpPNcup}
If $M_1\subseteq M_0$ is a subfactor of the finite factor $M_0$ and if $P$ is the projection onto $L^2(M_1)$ (as operator in $L^2(M_0)$), then
\begin{equation}
	M_1=\{ M_0'\cup P \}',
\end{equation}
and
\begin{equation}
	M_1'=\{ M_0'\cup P \}''
\end{equation}
as consequence.
\end{lemma}

\begin{proof}
First, every element of $\{ M_0'\cup P \}'$ has to commute with $M_0'$ and then to belongs to $M_0''=M_0$. But we proved in lemma~\ref{LemTMunPTTPiffTMzero} that when $T\in M_0$ and $TP=PT$, then $T\in M_1$. That proves that $\{ M_0'\cup P \}'=M_1$.
\end{proof}

As other consequence of lemma~\ref{LemTMunPTTPiffTMzero}, we have that, when $T\in M_1$, the operator $PT\colon S\mapsto P(TS)\in L^2(M_0)$ is equal to the operator $S\colon \mapsto TP(S)$, in other words,
\begin{equation}		\label{EqPTSeqalTPS}
	P(TS)=TP(S)
\end{equation}
for every $T\in M_1$ and $S\in M_0$.

\begin{lemma}		\label{LemMOJNJequal}
We have
\begin{equation}		\label{EqLemMPJNJequal}
	\langle M_0, P\rangle =JM_1'J,
\end{equation}
and as consequence, $\langle M_0, P\rangle $ is a finite factor.
\end{lemma}

\begin{proof}
First note that the commutant of a factor is a factor, so that $M_1'$ is a factor and $JM_1'J$ is a factor because it is algebraically isomorphic to the factor $M_1'$. Thus the lemma proves that the left hand side of \eqref{EqLemMPJNJequal} is in particular a finite factor.

Let us now prove the equality \eqref{EqLemMPJNJequal}. One needs to show that $J\langle M_0, P\rangle J=M_1'$. The commutant of the left multiplication action is the right one, while the left action of $JTJ$ is the right action of $T$, so one has $JM_0J=M_0'$. Since $JPJ=P$, it is sufficient to show that
\begin{equation}
	\langle M_0', P\rangle =M_1'.
\end{equation}
The von~Neumann algebra generated by $M_0'$ and $P$ is $\{ M_0',P \}''$ that is equal to $M_1'$ by lemma~\ref{LemNMpPNcup}.
\end{proof}
So $\langle M_0, P\rangle $ is a factor and it is finite when $M_1'$ is finite, which happens when $[M_0:M_1]<\infty$. Assuming that finiteness, we thus have a trace $\tr\colon \langle M_0, P\rangle \to \eC$ which extends the trace on $M_0$ (by uniqueness).

Notice that, as particular case of equation \eqref{EqLemMPJNJequal} when $M_1=M_0$, we have $P=\id$ and
\begin{equation}
	M_0=JM_0'J.
\end{equation}
That gives us an action of $M_0'$ on $L^2(M_0)$ by
\begin{equation}		\label{EqScdotTTSJT}
	S\cdot T=(JSJ)T
\end{equation}
when $S\in M_0'$ and $T\in M_0$.

\begin{proposition}		\label{PropdimMQhHQDIMMhH}
If $M_0$ is any finite factor acting on $\hH$ and if $Q\in M_0'$, then $Q\hH$ is invariant by $M_0$, and
\begin{equation}
	\dim_{M_0}(Q\hH)=\tr(Q)\dim_{M_0}(\hH)
\end{equation}
where $Q\hH$ is seen as module over $M_0$ (because of its invariance).
\end{proposition}

\begin{proof}
First, notice that a finite factor always has an unique trace. If $Q=0$ or $Q=\mtu$, the claim is obvious. Suppose now that $\tr(Q)=r/s$, that is a rational number.

In that case, we have
\begin{equation}		 \label{EqQhQhHHHeqMmod}
    \underbrace{Q\hH\oplus\ldots\oplus Q\hH}_{s\text{ times }} =\underbrace{\hH\oplus\ldots\oplus \hH}_{ r\text{ times}}
\end{equation}
\begin{probleme}
I can understad that in the setting of the fractional dimensions as described in subsection~\ref{SubSecRationalRealDim}, but there, we have the assumption of no minimal projection.
\end{probleme}
First, notice that the trace of the identity over $Q\hH$ is the trace of $Q$ on $\hH$. Indeed, let $\{ v_i \}$ be an Hilbertian basis of $Q\hH$ and $\{ w_k \}$ be a one of $Q^{\perp}\hH$. Then
\begin{equation}
	\tr(Q)=\sum_i\langle v_i, Qv_i\rangle +\sum_k\langle w_k, Qw_k\rangle =\sum_i\langle v_i, v_i\rangle =\tr(\mtu_{Q\hH}).
\end{equation}
Thus, by additivity,
\begin{equation}
	\tr(\mtu_{Q\hH\oplus\ldots\oplus Q\hH})=s\tr(Q)=r.
\end{equation}
On the other hand, $\tr(\mtu_{\hH\oplus\ldots\oplus\hH})=r\tr(\mtu_{\hH})=r$. So the identity over these two $M_0$-modules are the same.
\begin{probleme}
Is that enough to deduce that these are isomorphic as $M_0$-modules?
\end{probleme}
Taking the dimension of both sides of \eqref{EqQhQhHHHeqMmod}, we find $s\dim_{M_0}(Q\hH)=r\dim_{M_0}(\hH)$, which proves the statement in the case of $\tr(Q)\in\eQ$.

Now, the functions $\dim_{M_0}(Q\hH)$ and $\tr(Q)\dim_{M_0}(\hH)$ are increasing functions of $\tr(Q)$ that coincide on $\eQ$. Thus they are equal on $\eR$.

\end{proof}



\begin{proposition}		\label{ProptrPTMNtrT}
We have the formula
\begin{equation}		\label{EqClaimLemPTfracMNtrT}
	\tr(PT)=\frac{1}{ [M_0:M_1] }\tr(T)
\end{equation}
for every $T\in M_0$.
\end{proposition}


\begin{proof}
We have $\tr(PT)=\tr(PPT)=\tr(PTP)=\tr\big( p(T)P \big)$ because of lemma~\ref{LemPTPpTPopLdeux}. It is sufficient to prove the claim \eqref{EqClaimLemPTfracMNtrT} for $T\in M_1$ because when $S\in M_0$, we would have $\tr(PS)=\tr\big( p(S)P \big)=\tr(S)/[M_0:M_1]$ because $p(S)\in M_1$. Let thus $T\in M_1$ and consider the map
\begin{equation}
	T\mapsto \tr(PT)
\end{equation}
where $PT$ is considered as operator on $L^2(M_0)$. Thanks to \eqref{EqPTSeqalTPS}, this is a (normal faithful) trace on $M_0$. So that must be a multiple of the trace on $M_0$, namely there exists a $\tau\in\eC$ such that $\tr(PT)=\tau\tr(T)$. Taking that equality with $T=\mtu$ shows that $\tau=\tr(P)$. Using proposition~\ref{PropdimMQhHQDIMMhH}, we have thus
\begin{equation}
	\dim_{M_1}\big( P\cdot L^2(M_0) \big)=\tr(P)\dim_{M_0}\big( L^2(M_0) \big),
\end{equation}
but, as $M_0$-module, we have $P\cdot L^2(M_0)=L^2(M_1)$ by definition of $P$, so we have
\begin{equation}
	1=\tr(P)\cdot \dim_{M_1}\big( L^2(M_0) \big).
\end{equation}

\end{proof}


The set of expressions
\begin{equation}
	T_0+\sum_{j=1}^nT_{j_1}PT_{j_2}
\end{equation}
with $T_0$, $T_{j_1}$ and $T_{j_2}$ in $M_0$ is a $*$-algebra. Indeed, in the multiplication of two such expressions, we get monomials of the form
\begin{equation}
	T_1\underbrace{PT_2 P}_{=p(T_2)P}T_3=\underbrace{T_1p(T_2)}_{\in M_0}PT_3.
\end{equation}
By the double commutant theorem, the von~Neumann algebra $\langle M_0, P\rangle $ is the strong closure of such expressions, and
\begin{equation}
	\tr(T_1PT_2)=\tr(PT_2T_1)=\frac{1}{ [M_0:M_1] }\tr(T_1T_2)
\end{equation}
by proposition~\ref{ProptrPTMNtrT}. Thus we have an explicit formula for the trace on $\langle M_0, P\rangle $ from the trace over $M_0$.

\begin{lemma}		\label{lemPhHPNPfrac}
If $P\in M_1$, then we have
\begin{equation}
	\dim_{PM_1P}( P\hH)=\frac{1}{ \tr(P) }\dim_{M_1}(\hH),
\end{equation}
but $P\hH$ is no more a $M_1$-module.
\end{lemma}

\begin{proof}
Let $\modE$ be any $M_1$-module. Since $P\modE$ is a $PM_1P$-module, we can define $\dim^{(P)}(\modE)$ by
\begin{equation}
	\dim^{(P)}(\modE)=\dim_{PM_1P}(P\modE).
\end{equation}
That defines a dimension function on the $M_1$-modules, which is thus a multiple of the standard one. We know that, as $M_1$-module, $M_1P$ as dimension $\tr(P)$, so that the proportionality factor can be fixed on $M_1P$.

\begin{probleme}
	That proof is not complete.
\end{probleme}
\end{proof}

Let take $M_1=\eM_3(\eC)$ and $\hH=\eC^3$ as example. Each column of $a\in M_1$ is an element of $\eC^3$, so $M_1$ is three copies of $\hH$ and, as left module, we have $\hH\oplus\hH\oplus\hH=N$, and $\dim_{M_1}(\hH)=1/3$. Let
\[
	P=
\begin{pmatrix}
  1	&		&	\\
  	&	0	&	\\
 	&		& 0
\end{pmatrix}.
\]
We have $P\hH\simeq \eC$, and $PM_1P=\eC$ (the upper left element of the matrix). Thus if we have $\tr(P)=1/3$, we conclude by the lemma~\ref{lemPhHPNPfrac} that $\dim_{PM_1P}(P\hH)=1$.

\begin{lemma}
We have
\begin{equation}
	\dim_{M_1}(\hH)\dim_{M_1'}(\hH)=1
\end{equation}
if $M_1$ and $M_1'$ are finite factors.
\end{lemma}

\begin{proof}
Let us first check for $\hH=L^2(M_1)$. The action of $M_1'$ on $L^2(M_1)$ is given by \eqref{EqScdotTTSJT}, so we have $\dim_{M_1'}\big( L^2(M_1) \big)=\dim_{JM_1J}\big( L^2(M_1) \big)$, so that the action of $M_1'$ on $L^2(M_1)$ is conjugate of the one of $M_1$. That proves the lemma in the case where $\hH=L^2(M_0)$.

\begin{probleme}
Pourquoi ? Cela est fait dans la proposition~\ref{PropDimIIun}. Et je crois que du côté de cette proposition, on trouve pas mal de réponses à pas mal de questions ici.
\end{probleme}

\end{proof}

\begin{corollary}
We have
\begin{equation}
	[M_0:M_1]=[M_1':M_0']
\end{equation}
when $M_0\in\oB(\hH)$ is a factor of type $II_1$ and $M_1$ is a subfactor of $M_0$.
\end{corollary}
\begin{proof}
No proof.
\end{proof}

\begin{theorem}
We have
\begin{equation}
	[ \langle M_0, P\rangle :M_0 ]=[M_0:M_1]
\end{equation}
under the same assumptions.
\end{theorem}

\begin{proof}
If $\modE$ is any representation of $\langle M_0, P\rangle $, we have the formula
\begin{equation}	\label{EqlangleMOmodEME}
	[\langle M_0, P\rangle :M_0]=\frac{ \dim_{M_0}\modE }{ \dim_{\langle M_0, P\rangle }\modE }.
\end{equation}
We know by formula \eqref{EqDimMMprimeprodun} that
\begin{equation}
	\Big( \dim_{\langle M_0, P\rangle }\modE \Big)\cdot \Big(  \dim_{\langle M_0, P\rangle '}\modE\Big)=1.
\end{equation}
On the other hand, since $\langle M_0, P\rangle =JM_1'J$ (lemma~\ref{LemMOJNJequal}), we have
\begin{equation}
	\dim_{\langle M_0, P\rangle '}\modE=\dim_{JM_1J}\modE=\dim_{M_1}\modE=[M_0:M_1].
\end{equation}
Since formula \eqref{EqlangleMOmodEME} holds for every choice of $\modE$, we compute the right hand side in the case where $\modE=L^2(M_0)$. In that particular case, $\dim_{M_0}\modE=1$, and the claim follows.
\end{proof}

%---------------------------------------------------------------------------------------------------------------------------
					\subsection{Example of index bigger than \texorpdfstring{$4$}{4}}
%---------------------------------------------------------------------------------------------------------------------------

Consider $R$, the hyperfinite factor of type $II_1$ given by\footnote{Recall that one needs a functional in order to define the infinite tensor product.}
\begin{equation}
	R=\bigotimes_1^{\infty}\big( \eM_2(\eC),\tr \big)
\end{equation}
It turns out that, for that factor, we have $R\simeq PRP$ for every projection $P\in R$. Choose isomorphism $\alpha\colon R\to PRP$ and $\beta\colon R\to P^{\perp}RP^{\perp}$, and form the algebra
\begin{equation}
	S=\{ \alpha(T)+\beta(T)\tq T\in R \}.
\end{equation}
That algebra is algebraically isomorphic to $R$, and the isomorphism is ultraweakly continuous, so that $S$ is a von~Neumann algebra. The index of $S$ in $R$ is given by
\begin{equation}
	[R:S]=\tr(P)^{-1}+\tr(P^{\perp})^{-1}
\end{equation}
where $\tr(P)$ can take any value between $0$ and $1$, and $\tr(P^{\perp})=1-\tr(P)$. The possible values of the index are then given by the range of the function
\begin{equation}
	f(x)=\frac{1}{ x }+\frac{1}{ 1-x }
\end{equation}
when $x$ runs over $[0,1]$. One easily checks that that range is $[4,\infty]$.


%---------------------------------------------------------------------------------------------------------------------------
					\subsection{Properties of the sequence of \texorpdfstring{$M_i$}{Mi}, \texorpdfstring{$P_i$}{Pi}}		\label{SubSecPropSeqMO}
%---------------------------------------------------------------------------------------------------------------------------

Let $M_0$ be a subfactor of $M_1$. We define 
\begin{equation}
[M_1:M_0]=\dim_{M_0}\big( L^2(M_1) \big)
\end{equation}
and we consider $P_1\colon  L^2(M_1)\to L^2(M_1)$, the projection onto $L^2(M_0)$. Then we consider the new factor
\begin{equation}
	M_2=\langle M_1, P_1\rangle.
\end{equation}
For that definition, we see $M_1$ as subalgebra of $\oB\big( L^2(M_1) \big)$. Now, $M_2$ is also a subalgebra of $\oB\big( L^2(M_2) \big)$, so that, defining $P_2\colon L^2(M_2)\to L^2(M_2)$ as the projection onto $L^2(M_1)$, allows to define
\begin{equation}
	M_3=\langle M_2, P_2\rangle \subseteq\oB\big( L^2(M_2) \big).
\end{equation}
Using that construction again and again, we get a sequence
\begin{equation}
	M_0\subseteq M_1\subseteq M_2\subseteq\ldots
\end{equation}
of factors defined by $M_{n+1}=\langle M_n, P_n\rangle $ where $P_n\colon L^2(M_n)\to L^2(M_n)$ is the orthogonal projection onto $L^2(M_{n-1})$. One can prove that
\begin{equation}
	P_n(M_n)\subseteq M_{n-1}.
\end{equation}
We thus consider the maps
\begin{equation}
\begin{aligned}
 p_n\colon M_n&\to M_{n-1} \\
   p_n(T_n)&\mapsto P_n(T_n),
\end{aligned}
\end{equation}
and we have the sequence
\begin{equation}
\xymatrix{%
   P_{n-1}\in M_n \ar[r]^{p_n}	&	M_{n-1}\ar[r]^{p_{n-1}}	& M_{n-2}.
}
\end{equation}

The following is the proposition 3.3.2 in \cite{JonesSunder}.

\begin{proposition}		\label{ProppropsindexMarkov}
We have
\begin{enumerate}
\item $[M_n:M_{n+1}]=\lambda^{-1}$ does not depend on $n$,
\item $\tr(P_n)=\lambda$,
\item $\tr(P_nT_n)=\lambda\tr(T_n)$ for every $T_n\in M_n$.
\end{enumerate}
\end{proposition}

The last property is the \defe{Markov property}{Markov property}.

\begin{proof}
No proof.
\end{proof}

\begin{lemma}		\label{LemPnPllamcun}
We have
\begin{equation}
	p_n(P_{n-1})=\lambda\cun
\end{equation}
where $\lambda=\tr(P_n)$ does not depends on $n$ by the proposition~\ref{ProppropsindexMarkov}.
\end{lemma}

\begin{proof}
Since the map $(S,T)\mapsto\tr(ST)$ is a nondegenerate bilinear functional on $M_{n-1}$, it is sufficient to prove that $\tr\big( Sp_n(P_{n-1}) \big)=\lambda\tr(S)$ for every $S\in M_{n-1}$. Using point~\ref{ItemPropMappMNiv} of proposition~\ref{PropPropMappMN}, we have $Sp_n(P_{n-1})=p_n(SP_{n-1})$, so that
\begin{equation}
	\tr\big( Sp_n(P_{n-1}) \big)=\tr(SP_{n-1})=\lambda\tr(S),
\end{equation}
where the last equality is the Markov property.
\end{proof}

\begin{proposition}		\label{PropAlgPPPKoi}
	The projections $P_i$ fulfil the algebra
	\begin{subequations}		\label{SubeqPnPalgPPI}
	\begin{align}
		P_nP_{n+1}P_n&=\lambda P_n		\label{EqLoiPPun}		\\
		P_nP_{n-1}P_n&=\lambda P_n		\label{EqLoiPPdeux}		\\
		P_nP_m&=P_mP_n				\label{EqLoiPPtrois}
	\end{align}
	\end{subequations}
	when $| n-m |\geq 2$.
\end{proposition}

\begin{proof}
	Let us prove the second one. Using lemma~\ref{LemPnPllamcun}, we find
	\begin{equation}
		P_nP_{n-1}P_n=p_n(P_{n-1})P_n = \lambda P_n,
	\end{equation}
	which is the claim.
\end{proof}

\begin{corollary}
	The algebra generated by
	\begin{equation}
		\{ \mtu,P_1,\ldots, P_n\}
	\end{equation}
	is a finite dimensional $C^*$-algebra in which there exists an unique tracial state such that for every~$k$,
	\begin{equation}
		\tr(P_kT)=\lambda \tr(T)
	\end{equation}
	whenever $T$ belongs to the algebra generated by $\mS_{k-1}=\{ \mtu,P_1,\ldots,P_{k-1} \}$.
\end{corollary}

\begin{proof}[Sketch of the proof]
Let a \emph{reduced word} in $P_1$, \ldots, $P_k$ be a word which is as small as possible using the three rules \eqref{SubeqPnPalgPPI}. We prove by induction that such a reduced word contains $P_k$ at most once. For beginning, the only reduced word in $P_1$ is $P_1$ itself.

Now, let a word containing twice $P_k$. What is between two successive occurrences of $P_k$ is a word of $\mS_{k-1}$. That word is reduced, and thus contains $P_{k-1}$ only once by induction hypothesis. From rule \eqref{EqLoiPPtrois}, the operator $P_k$ commutes with $\mS_{k-2}$, and then the rule \eqref{EqLoiPPdeux} reduces the word (because $P_{k-1}$ appears only once).


Now, a general reduced word of $\mS_k$ has only one $P_k$ and thus has the form $m_1 P_k m_2$ where $m_1$ and $m_2$ are reduces words of $\mS_{k-1}$. Of course, $m_1$ them self decomposes in $n_{1} P_{k-1} n_2$ where $n_i$ are reduces words of $\mS_{k-2}$, and the process continues.

Every reduced word in $\mS_9$ look like
\begin{equation}
	(P_5P_4P_3)(P_8P_7P_6P_5P_4)(P_9P_8P_7P_6).
\end{equation}
Some comments
\begin{itemize}
\item The sub words are made of consecutive projections. For example the sub word $(P_5P_3)$ can be rearranged as the two sub words $(P_3)(P_5)$ by rule \eqref{EqLoiPPtrois}.
\item The sub words begin by $P_5$, $P_8$ and $P_9$. Notice that $5<8<9$. It will always be like that. If not, a rearrangement is possible.
\end{itemize}
Now fix a $n$ and consider the set of all the reduced words of $\mS_n$; there are only finitely many of them which will be labelled as $W_i$. We look at the matrix whose element $ij$ is given by
\begin{equation}		\label{EqMtrWWtr}
	\tr(W_i^*W_j).
\end{equation}
Each of $W_i^*W_j$ is a long string of elements of $\mS_n$ and the cyclic property of the trace allows us to do
\begin{equation}
	\tr\big( (\ldots)(\ldots)(P_nP_{n-1}\ldots) \big)=\tr\big( (P_nP_{n-1}\ldots)(\ldots)(\ldots)  \big).
\end{equation}
Then, using the Markov property of proposition~\ref{ProppropsindexMarkov}, we can extract $P_n$. That matrix is positive semidefinite because it is formed of scalar products. If not, that would means that the assumption $M_0\subseteq M_1$ is wrong.

\end{proof}

%---------------------------------------------------------------------------------------------------------------------------
					\subsection{Some interesting functions}
%---------------------------------------------------------------------------------------------------------------------------

Consider the polynomials defined by the induction
\begin{subequations}
\begin{align}
	f_0(x)&=0	\\
	f_1(x)&=1	\\
	f_n(x)&=f_{n-1}(x)-xf_{n-2}(x).
\end{align}
\end{subequations}
Since $f_n(0)=f_{n-1}(0)$, all these polynomials pass by the point $(0,1)$. By induction, one can see that
\begin{equation}
	f_n\left( \frac{1}{ 4\cos^2\theta } \right)=\frac{\sin(n\theta)}{2^{n-1}\cos^{n-1}(\theta)\sin(\theta)},
\end{equation}
so that	the smallest positive root of $f_n$ is
\begin{equation}
	\frac{1}{ 4\cos^2\left( \frac{ \pi }{ n } \right) },
\end{equation}
and $f_n$ is negative on the interval between $1/4\cos^2(\pi/n)$ and $1/4\cos^2(\pi/(n-1))$.

Suppose $\lambda^{-1}=[M:M_0]$ and $\lambda>1/4$. Now, assume that
\begin{equation}
	\lambda\neq \frac{1}{ 4\cos^{2}(\pi/n) }
\end{equation}
for any value of $n\in\eN$. In this case, we will found a negative diagonal entry of the matrix \eqref{EqMtrWWtr}, which is impossible because a semipositive definite matrix has no negative diagonal element.

Let
\begin{equation}
	Q_k=\mtu-P_1\vee\ldots\vee P_k.
\end{equation}
From lemma~\ref{LemFiniCSestVNa}, the operator $Q_k$ is in fact a projection in a finite dimensional algebra, and is thus a polynomial in the $P_i$'s.

\begin{lemma}
	We have
\begin{equation}		\label{EqQPQffQPQ}
	(Q_{k-1}P_kQ_{k-1})^2=\frac{ f_k(\lambda) }{ f_{k+1}(\lambda) }Q_{k-1}P+kQ_{k-1},
\end{equation}
and
\begin{equation}
	Q_{k+1}=Q_k-\frac{ f_{k+1}(\lambda) }{ f_k(\lambda) }Q_{k-1}Q_{k-1},
\end{equation}
if $f_{k+1}(\lambda)$, \ldots, $f_1(\lambda)\neq0$.
\end{lemma}
Notice in particular that, up to a factor, the operator of equation \eqref{EqQPQffQPQ} is a projection.

\begin{proof}
No proof.
\end{proof}

\begin{corollary}
The operators $Q_{k-1}$, $P_k$ and $Q_{k+1}$ are projection, and the product is a positive operator.
\end{corollary}

That corollary shows that
\begin{equation}
	\frac{ f_k(\lambda) }{ f_{k+1}(\lambda) }>0.
\end{equation}
But a simple study of the polynomials shows that this ratio is in fact negative. We conclude that operator satisfying the algebra of proposition \eqref{PropAlgPPPKoi} are only possible when $\lambda=1/4\cos(\pi/n)$.

\input{124_VN_algebras}

\chapter{Dirichlet forms}
\input{Dirichlet}

\chapter{K-theory}
\input{ktheory}

\chapter{BF theory}
\input{bf_theory}

\chapter{BTZ black hole from identifications}
\input{PetiteDim}
\input{oldBTZ}
\input{BTZ_3}

\chapter{BTZ black holes in anti de Sitter spaces}                  \label{ChapBHinAdS}
\input{introduction}
% This is part of (almost) Everything I know in mathematics and physics
% Copyright (c) 2013-2016, 2020
%   Laurent Claessens
% See the file fdl-1.3.txt for copying conditions.

\begin{abstract}
This chapter deals with black holes in anti de Sitter spaces. The latter are the simplest non flat solutions to Einstein's equations with constant negative cosmological constant; they are in particular pseudo-Riemannian manifolds that carry a causal structure, physically due to the finiteness of speed of light. That physical restriction is mathematically encoded by the existence of three types of geodesics: the space-, time- and light-like ones, existence which is in turn implied by the non positivity of the metric. A causal structure is introduced by defining two points as \emph{causally connected} when there exists a time- or light-like path connecting them.

 The  originality of our approach is that the $l$-dimensional space $AdS_l$ is seen as a quotient of groups $\SO(2,l-1)/\SO(1,l-1)=G/H$, and that the special causal black hole structure is described in terms of orbits of the action of a subgroup of the isometry group of the manifold.

Using symmetric spaces techniques, we show that closed orbits of the Iwasawa subgroup of $\SO(2,l-1)$ naturally define a causal black hole singularity in anti de Sitter spaces in $l \geq 3$ dimensions. In particular, we recover for $l=3$ the non-rotating massive BTZ black hole. The method presented here is very simple and in principle generalizable to any semisimple symmetric space.

The main references for this part are \cite{lcTNAdS,articleBVCS,These}.

\end{abstract}

%%%%%%%%%%%%%%%%%%%%%%%%%%
 %
    \section{Introduction}
%
%%%%%%%%%%%%%%%%%%%%%%%%

\subsection{Physics and mathematics of black holes}	\label{SubSecGeneBH}
%--------------------------------------------------

\subsubsection{Notion of Causality}
%\\\///////////////////////////////

This subsection is devoted to introduce the mathematical definition of a black hole from the intuitive physical notions of causality and maximality of the speed of light. Let us pose the origin of time and space respectively now and here. So we are at $(0,0)$. If we denote by $c$ the seed of light, we cannot reach the moon before time $\unit{340000}{\kilo\meter}/c$. More generally we cannot reach a point at spacial distance $d$ within a time inferior to $d/c$. Then the space is thus divided into three very different regions with respect to causality: the points that we can reach traveling slower than light, the points that only light can reach and points that we cannot reach at all.

It is convenient to introduce the function $s(t,x)=c^2t^2-x^2$ which basically says you which points are accepted and which points are unaccepted. The mathematical way to implement these ideas is to consider a pseudo-Riemannian manifold $(M,g)$. The \defe{norm}{pseudo-Riemannian!norm} of a vector $X\in T_xM$ is defined as $\| X \|^2=g_x(X,X)$. There are three possibilities:
\begin{itemize}\label{PgDefsGenre}
	\item if $\| X \|^2>0$, we say that $X$ is \defe{time-like}{time-like},
	\item if $\| X \|^2<0$, we say that $X$ is \defe{space-like}{space-like},
	\item if $\| X \|^2=0$, we say that $X$ is \defe{light-like}{light-like}.
\end{itemize}
A path $c\colon \eR\to M$ is time, space or light-like when its tangent vector is everywhere time, space or light-like. The manifold $M$ is \defe{time orientable}{time orientation} if it accepts an everywhere time-like vector field. A \emph{time orientation} is the choice of such a vector field. If $T$ is a time orientation on $M$, we say that a vector $X_x\in T_xM$ is \defe{future directed}{future!directed vector} if $g_x(T_x,X)>0$. From now we suppose that a choice of time orientation is possible and done.

The concept of causality is now easy to determine. If $x$ and $y$ belong to $M$, the point $x$ has a \defe{causal influence}{causal!influence} on $y$ if there exists a future directed path $c\colon [0,1]\to M$ such that $c(0)=x$ and $c(1)=y$. One has to notice that the relation \emph{has a causal influence on} is not symmetric in general, but there exist some examples in which it is symmetric.

%As example, consider the space $\eR^2$ represented on the figure~\ref{FigMink}.
As example consider the space $M=\eR^2$ endowed with the constant pseudo-Riemannian structure $g=\begin{pmatrix}c^2&0\\0&1 \end{pmatrix}$. That space is the \defe{Minkowski space}{Minkowski!space}. The relation of causality is given by the previously mentioned function $s$; this relation is \emph{never} symmetric and there exist pairs of point who have no causal effect on each other. If one takes the quotient by the relation $t\sim t+1$, we get a space in which the causality is everywhere symmetric.

\subsubsection{Notion of singularity and black hole}
%////////////////////////////////////////////////

Up to the choice of a time orientation, a pseudo-Riemannian manifold is comes with a canonical notion of causality. In order to have a black hole in our causal space we need an extra structure:~the singularity. We take here a very conservative point of view and we say that a \emph{singularity} in $M$ is any strict subset of $M$. In the literature one often add conditions on the singularity such like to be a submanifold, time-like, connected,\ldots of course most of ``real live'' singularities fulfil that kind of conditions.

The singularity defines two types of points in the space: the ones from which every time-like path intersect the singularity (from a physical point of view, these points correspond to observers who will fall in the singularity without doubt) and the points from which at least one time-like path does not intersect the singularity. We define the black hole associated with the singularity $\hS$ as
\begin{align}
  BH=\big\{ x\in M\tq \forall &\text{ future directed time-like path } c \text{ with } c(0)=x,\\
			&\exists t\geq0  \text{ such that } c(t)\in \hS \big\}.
\end{align}
The easiest example is given by defining a small line as singular in the Minkowski space as shown in figure~\ref{LabelFigEJRsWXw}.
\newcommand{\CaptionFigEJRsWXw}{The red line is the singularity and the green zone is the black hole associated with.}
\input{auto/pictures_tex/Fig_EJRsWXw.pstricks}

In order the construction to be non trivial, we ask the black hole to be bigger than the singularity (that is of course part of the black hole), but smaller that the full space.

%The basic notions needed in order to define a causal structure on a time orientable pseudo-Riemannian manifold are that of time-, light- and space-like tangent vector. A tangent vector is said to be respectively \emph{time-}, \emph{space-} or \emph{light-like} when its norm is positive, negative or null; physically, only time-like vectors are allowed to be the velocity of an observer (this is the fact that light speed cannot be attained by a massive particle), and it is only possible for massless particle (such as photons) to follow trajectories with light-like tangent vectors.

From a geometric point of view, a black hole is the data of a causal manifold $M$ together with a subset $\hS \subset M$ called \emph{singularity} such that the whole manifold is divided into two parts: the \emph{interior} and the \emph{exterior} of the black hole. A point is said to be \emph{interior} if all future light-like geodesics through the point have a non empty intersection with the singularity. A point is \emph{exterior} if it is not interior. An important subset of the space is the \emph{event horizon}: the boundary between these two subsets.

\subsection{BTZ black hole}		\index{BTZ black hole}
%------------------------------

The BTZ black hole introduced and developed by Bañados, Teitelbaum, Zannelli and Henneaux in \cite{BTZ_un,BTZ_deux} is an example of a black hole whose singularity is not motivated by metric divergences\footnote{It turns out that general relativity accepts a lot of solutions presenting metric divergences; or more precisely, there are a lot of \emph{physical situations} from which Einstein's equations lead to divergences of some metric invariant such as the curvature.}. The construction is roughly as follows. We consider the anti de Sitter space in which we pick up a Killing vector field whose sign of norm is not constant. Then we perform a \emph{discrete} quotient along the integral curves of this vector field. Of course we obtain a lot of closed geodesics. The point is that, in the region where the Killing vector field is space-like, these closed curves are space-like. That violates the physical principle of causality. For that reason, we decree that this region is singular or, equivalently, that the boundary of this region is singular. The BTZ singularity is then the loci where the chosen Killing vector field has a vanishing norm. Since discrete quotients do not affect local structures, the resulting space remains a solution of the $(2+1)$-dimensional general relativity with negative cosmological constant\footnote{For honesty, we have to warn the reader that the real world's cosmological constant has been measured very small but positive. We also have to point out that the four dimensional anti de Sitter space is a solution of general relativity \emph{without masses}. From a physical point of view, this thesis has to be seen as a toy model.}. In this context one can define pertinent notions of  \emph{mass} and \emph{angular momentum} which depend on the chosen Killing vector field.

\begin{probleme}
Il faut trouver une référence pour dire que la constante cosmologique est positive.
\end{probleme}

In the case of the \emph{non-rotating massive} BTZ black hole, the structure of the singularity and the horizon are closely related to the action of a minimal parabolic (Iwasawa) subgroup of the isometry group of anti de Sitter, see \cite{BTZB_deux,Keio}. The whole work on the BTZ black hole and the fact that it belongs to the class of causal symmetric spaces (for definitions and some examples, see \cite{HilgertOlaf}) motivate the following definition:

\begin{definition}
A \defe{causal solvable symmetric black hole}{causal!solvable symmetric black hole} is a causal symmetric space where the closed orbits of minimal parabolic subgroups of its isometry group define a black hole singularity. See section~\ref{SecCausal} for definitions of causality and singularity in the $AdS$ case.
\label{Def1}
\end{definition}

\subsection{Generalization and group setting}
%--------------------------------------------
\label{SubSecGEneBHGrop}


The original BTZ black hole was constructed in dimension three, but we will see in this chapter that, exploiting their group theoretical description, they can easily be generalized to any dimension, as pointed out in \cite{BDRS,lcTNAdS}.  Notice that higher-dimensional generalizations of the BTZ construction have been studied in the physics literature, by classifying the one-parameter isometry subgroups of $\Iso(AdS_l)=\SO(2,l-1)$, see \cite{Figueroa,AdSBH,Madden,BanadosIQxXuEh,Aminneborg,HolstPeldan}, but these approaches do not exploit the symmetric space structure of anti de Sitter.

The structure that will be described with full details in next pages may be summarized as follows. Take $G=\SO(2,l-1)$, fix a Cartan involution $\theta$ and a $\theta$-commuting involutive automorphism $\sigma$ of $G$ such that the subgroup $H$ of $G$ of the elements fixed by $\sigma$ is locally isomorphic to $\SO(1,l-1)$. The quotient space $M=G/H$ is a $l$-dimensional Lorentzian symmetric space, the {\sl anti de Sitter space-time}.  We denote by $\sG$ and $\sH$ the Lie algebras of $G$ and $H$. We have the decomposition $\sG=\sH\oplus\sQ$ into the $\pm 1$-eigenspace  of the differential at $e$ of $\sigma$ that we denote again by $\sigma$.  We also consider $\sG=\sK\oplus\sP$, the Cartan decomposition induced by $\theta$; and $\sA$, a $\sigma$-stable maximally abelian subalgebra of $\sP$. A positive system of roots is chosen  and let $\sN$ be the corresponding nilpotent subalgebra (see Iwasawa decomposition, theorem~\ref{ThoIwasawaVrai}).  Set  $\overline{\sN}=\theta(\sN)$, $\sR=\sA\oplus\sN$ and $\overline{\sR}=\sA\oplus\overline{\sN}$. Finally denote by $R=AN$ and $\overline{R}=A\overline{N}$ the corresponding analytic subgroups of $G$.  One then has

\begin{theorem}
The $l$-dimensional anti de Sitter space with $l\geq 3$, seen as the symmetric space $\SO(2,l-1)/\SO(1,l-1)$, becomes a causal solvable symmetric black hole, as defined above, when the closed orbits of the Iwasawa subgroup $R$ of $\SO(2,l-1)$ and its Cartan conjugated $\overline{ R }$ are said to be singular. There exists in particular a non empty event horizon. The group $R$ has exactly two such closed orbits.
\label{ThoLeBut}
 \end{theorem}

This chapter intends to prove this theorem, and for the sake of completeness, we also analyze in some detail in section~\ref{sec_AdSdeux} the two-dimensional case, for which the construction does not yield a black hole structure.

The black hole causal structure is thus completely determined by the action of a solvable group.  This observation gives prominence to potential embeddings of these spaces in the framework of noncommutative geometry, in defining noncommutative causal black holes (see also \cite{BDRS}) through the existence of universal deformation formulae for solvable groups actions which have been obtained in the context of WKB-quantization of symplectic symmetric spaces \cite{StrictSolvableSym,Biel-Massar-2}. These issues are investigated in chapter~\ref{ChDefoBH} and in \cite{articleBVCS}.


%---------------------------------------------------------------------------------------------------------------------------
					\subsection{Some notations}
%---------------------------------------------------------------------------------------------------------------------------

We are going to use the following notations. We denotes the \defe{free part}{free!part of a black hole} of the space $AdS_l$ by $\hF_l$; this is the subset of $AdS_l$ for which there exists a light-like direction which escapes the singularity. We denote by $BH_l$ the black hole in $AdS_l$; this is the set of points from which all the light-like geodesics intersect the singularity.



%+++++++++++++++++++++++++++++++++++++++++++++++++++++++++++++++++++++++++++++++++++++++++++++++++++++++++++++++++++++++++++
\section{Visite guidée}
%+++++++++++++++++++++++++++++++++++++++++++++++++++++++++++++++++++++++++++++++++++++++++++++++++++++++++++++++++++++++++++

%---------------------------------------------------------------------------------------------------------------------------
\subsection{En termes de BTZ}
%---------------------------------------------------------------------------------------------------------------------------

Nous travaillons dans $AdS_l=\SO(2,l-1)/\SO(1,l-1)=G/H$. Nous définissons les orbites fermées de $AN$ et $A\bar N$ (le groupe d'Iwasawa de $G$ et son conjugué par une involution de Cartan) comme \emph{singulières}.

Il a été prouvé il y a déjà bien longtemps que cette définition donne lieu à une structure de trou noir. Cette structure est par ailleurs la même, en dimension 3, que celle du trou noir BTZ connu de la physique. J'ai récemment poussé un peu plus loin et donné la structure de l'horizon en dimension $4$ en termes de celle en dimension $3$. Il se fait que (théorème~\ref{ThoEqHorQCoore})

\begin{theorem}
L'horizon de $AdS_4$ est donné par
\begin{equation}
	\hH_4=G_V\cdot \iota(\hH_3)\cup G_X\cdot\iota(\hH_3),
\end{equation}
où $\iota\colon \eR^4\to \eR^5$ est l'inclusion de $AdS_3$ dans $AdS_4$ et où les groupes $G_V$ et $G_X$ sont donnés par
\begin{equation}
	G_V=\{  e^{\alpha V}\tq\alpha\in\eR \},
\end{equation}
le vecteur $V$ étant l'élément de base de l'espace de racine $\sG_{(0,1)}$ de $\SO(2,3)$, et $X$ est l'élément de base de $\sG_{(0,-1)}$. Ces espaces de racines sont vides dans le cas de $AdS_3$.
\end{theorem}
L'inclusion $\iota$ peut également être vue comme l'inclusion du groupe $\SO(2,3)$ dans $\SO(2,4)$. La preuve est faite avec du calcul matriciel explicite très peu généralisable à d'autres espaces symétriques.

L'énoncé de ce théorème est la seule chose élégante de la section~\ref{SecHOrOrbEquation}. Le reste est du calcul matriciel. Ce théorème donne, cependant, une bonne idée de ce vers quoi on va : il semble possible que les horizons en dimension supérieure s'obtiennent par récurrence. Le groupe qui générerait la singularité en dimension $l$ serait le groupe généré par les espaces de racines $(1,0)$ et $(0,1)$ de $\SO(2,l-1)$.

Affin de trouver des preuves plus intrinsèques, on commence par bien définir les différents éléments de l'algèbre, et en particulier la base de $\sQ$ en termes des espaces de racines. Cela se passe à la section~\ref{SecRebuildStructRoot}. Je définit par exemple
\begin{equation}
	\begin{aligned}[]
		q_0&=(X_{++})_{\sQ\sK}\\
		q_2&=(X_{++})_{\sQ\sP},
	\end{aligned}
\end{equation}
et je montre que le premier est de norme (de Killing) positive et le second de norme négative, mais qu'en valeur absolue, ils ont la même norme. Je choisit donc $X_{++}$ de telle façon que $q_0$ et $q_1$ soient normés à $1$. Les vecteurs $q_0\pm q_1$ sont donc de genre lumière.

Toute une série de propriétés sont ensuite prouvées. Le but est évidement de construire, de façon intrinsèque, une base de $\sA$, $\sN$, $\sK$ et de $\sQ$ de telle façon à avoir toutes les propriétés agréables que les matrices explicites avaient.

Cette partie est destinée à être remplacée par une application du théorème de structure de Pyatetskii-Shapiro. Un petit changement de base sera toutefois indispensable parce que l'élément dont l'annulation de la norme du champ de vecteur fondamental donne la singularité n'est pas dans la base donnée par Pyatetskii-Shapiro.

Tant que l'on travaillait avec des matrices et qu'on utilisait explicitement le fait que $AdS$ était un sous-ensemble de $\eR^n$, nous utilisions la caractérisation suivante de la singularité :
\begin{equation}
	\hS\equiv t^2-y^2=0.
\end{equation}
Maintenant, il est bon d'utiliser une caractérisation de la singularité qui ne fait pas appel aux coordonnées. Une telle caractérisation existe : si $J_1$ est un élément de $\sA\cap\sH$ (qui est de dimension $1$), alors la singularité est donnée par
\begin{equation}
	\hS\equiv \| J_1^* \|=0
\end{equation}
où $J_1^*$ est le champ de vecteur fondamental de l'action de $G$ sur $G/H$ associé au vecteur $J_1$. Cette caractérisation fait qu'un point $[g]\in G/H$ est dans la singularité si et seulement si le vecteur
\begin{equation}		\label{VisiteEqprQcaract}
	\pr_{\sQ}\left( \Ad(g^{-1})J_1 \right)
\end{equation}
a une norme nulle. Ici, $\pr_{\sQ}$ est la projection sur $\sQ$.

À part une foule de petits détail encore à vérifier, il est maintenant prouvé, en utilisant la caractérisation \eqref{VisiteEqprQcaract}, qu'un point $[kan]$ est dans la singularité si et seulement s'il appartient à $[AN]$, $[A\bar N]$, $[-\mtu_{\SO(2)}AN]$ ou $[-\mtu_{\SO(2)}A\bar N]$, c'est-à-dire à une des orbites fermées de $AN$ ou de $A\bar N$.


Tout cela est dans le chapitre~\ref{ChapBHinAdS}.

%---------------------------------------------------------------------------------------------------------------------------
\subsection{En termes de généralisations}
%---------------------------------------------------------------------------------------------------------------------------

Afin de se mettre dans une perspective de généralisation, l'idée suivante est proposée.

\begin{enumerate}

	\item



On considère un espace homogène symétrique $G/H$ où $G$ a 1000 décompositions d'Iwasawa possibles.

\item
 On sait par des arguments d'hermiticité et de $\mZ(\sK)$ non nul que la composante d'Iwasawa de $G$ est une $j$-algèbre.

\item
Il y a sûrement un argument pour dire qu'il existe des choix d'Iwasawa qui font que les racines positives et les éléments correspondants de $\sA\oplus\sN$ tombent exactement dans les $A$ ,$Z$ et $V $ de la décomposition en $j$-algèbres élémentaires.

\item
On choisit cette décomposition particulière d'Iwasawa comme décomposition "de référence".

\item
 On définit la singularité sur $G/H$ par $\| H1+H2\|=0$. Ici, c'est la première fois que le $H$ apparaît dans la construction.

 \item
 On considère l'Iwasawa qui change de base dans $\sA$ pour choisir $J1=H1+H2$ et $J2=H1-H2$. Cela devrait être fait sans changer de décomposition $\sG=\sK\oplus\sP$.

\item		\label{ItemVGDern}
 On prouve que la singularité est les orbites fermées de $AN$ et $A\theta(N)$ pour cette nouvelle Iwasawa.
\end{enumerate}


Le point~\ref{ItemVGDern} est là uniquement pour montrer que l'ensemble de la construction redonne le BTZ déjà connu.

\section{Connectedness of groups and anti de Sitter spaces}
%++++++++++++++++++++++++++++++++++++++++++++++++++++++++++++++++

\label{PgDisGeoConnSO}Let us give some detail on the geometric nature of the two connected components of $\SO(p,q)$\footnote{See lemma~\ref{LemOHjzfsL}. A discussion about the physics is in \cite{Schomblond_em}.}. What is proved in \cite{HelgasonSym} is that $\SO(p,q)$ is homeomorphic to the topological product
\[
  \SO(p,q)=\SO(p,q)\cap\SU(p+q)\times \eR^{d}=\SO(p,q)\cap\SO(p+q)\times \eR^{d}
\]
for some $d\in\eN$. Hence an element of $\SO(p,q)$ reads
\[
  \begin{pmatrix}
A&0\\
0&B
\end{pmatrix}\times v
\]
where $v\in\eR^{d}$, $A\in \gO(p)$, $B\in\gO(q)$ are such that $\det A\det B=1$. The $v$ part corresponds to boost while $A$ and $B$ correspond to pure temporal and pure spatial rotations. An element of $\gO(n)$ has always determinant equals to $\pm 1$. Therefore one can decompose the rotation part as $(\det A=\det B=1)\otimes (\det A=\det B=-1)$. Both parts are connected.

Hence the first connected component contains $\mtu$ while the second one contains the element that simultaneously changes the sign of one spacial and one time direction.

\subsection{The quotient for anti de Sitter}
%--------------------------------------------

Homogeneous space considerations (see section~\ref{SecSymeStructAdS}) will naturally lead us to define the anti de Sitter space as the quotient $G/H=\SO(2,l-1)/\SO(1,l-1)$ while the black hole definition (section~\ref{SecCausal}) needs to consider Iwasawa decompositions of $G$. So we face the problem that the Iwasawa theorem~\ref{ThoIwasawaVrai} only works with connected groups. In order to prevent any problems of this type, we prove now that, if $G_0$ and $H_0$ denote the identity component of $\SO(2,l-1)$ and $\SO(1,l-1)$ respectively, then $G/H=G_0/H_0$.

The groups that are considered here have only two connected components $G_0$ and $G_1$. We can chose $i_1\in G_1\cap H$ such that $i_1^2=\mtu$. Using lemma~\ref{LemConnSpecMo}, it easy to prove that
\begin{itemize}
\item $G_0G_0=G_0$,
\item $G_0G_1=G_1$,
\item $G_1G_1=G_0$.
\end{itemize}
For the last one, take $g$ and $g'$ in $G_1$. Then consider $g_0$ and $g'_0$ in $G_0$ such that $g=g_0i_1$ and $g'=g_0'i_1$. If $g_0(t)$ and $g'_0(t)$ are path from $\mtu$ to $g_0$ and $g_0'$, then the path $g_0(t)i_1g'_0(t)i_1$ is a path from $\mtu$ to $gg'$.

\begin{proposition} \label{PropGHconn}
The map
\begin{equation}
\begin{aligned}
 \psi\colon G/H&\to G_0/H_0 \\
[g]&\mapsto \overline{ g_0 }
\end{aligned}
\end{equation}
where we define $g_0=g$ when $g\in G_0$ or $g_0=gi_1$ when $g\in G_1$ is a diffeomorphism. The classes are $[g]=\{ gh\tq h\in H \}$ and $\overline{ g }=\{ gh_0\tq h_0\in H_0 \}$.
\end{proposition}

\begin{proof}
First we prove that $\psi$ is well defined. For that we suppose that $[g]=[g']$. There are three cases:
\begin{enumerate}
\item The elements $g$ and $g'$ both belong to $G_0$. In this case, $g'=gh_0$ with $h_0\in H_0$ and $\overline{ gh }=\overline{ g }$.
\item The element $g$ belongs to $G_0$ while $g'$ belongs to $G_1$. In this case, $g'=gh$ with $h=h_0i_1$ and $h_0\in H_0$. Then $\psi[g]=\overline{ g }$ and $\psi[g']= \overline{ (gh_0i_1)_0 }=\overline{ gh_0i_1i_1 }=\overline{ gh_0 }=\overline{ g } $.
\item The case with $g$ and $g'$ in $G_1$ is similar.
\end{enumerate}

The fact that the map $\psi$ is surjective is clear. For injectivity, let $\psi[g]=\psi[g']$, i.e. there exists a $h_0$ in $H_0$ such that $g'_0=g_0h_0$. Thus we have $g'i_1^k=gi_1^lh_0$ with $k,l=0,1$ following the cases. Then $g'=gi_1^lh_0i_1^k$ in which $i_1^lh_0i_1^k$ belongs to $H$, so that $[g']=[g]$.

\end{proof}


\section{Symmetric space structure on anti de Sitter}\label{SecSymeStructAdS}
%------------------------------------------

The $l$-dimensional anti de Sitter space $AdS_l$ can be described as set of points $(u,t,x_1,\ldots,x_{l-1})\in \eR^{2,l-1}$  such that $u^2+t^2-x_1^2-\ldots-x_{l-1}^2=1$. The next few pages are devoted to describe the homogeneous and symmetric space structures on $AdS_l$ induced by the transitive an isometric action of $\SO(2,l-1)$. We suppose that the groups $\SO(2,l-1)$ and $\SO(1,l-1)$ are parametrized in such a way that the second, seen as subgroup of the first one, leaves unchanged the vector $(1,0,\ldots,0)$. In this case, proposition 4.3 of chapter II in \cite{Helgason} provides the homogeneous space isomorphism
\begin{equation}
\begin{aligned}
  \SO(2,l-1)/\SO(1,l-1)&\to AdS_l \\
[g]&\mapsto
 g\cdot
\begin{pmatrix}
1\\0\\\vdots
\end{pmatrix}
\end{aligned}
\end{equation}
where the dot denotes the usual ``matrix times vector'' action of the representative $g\in [g]$ in the defining representation of $\SO(2,l-1)$ on $\eR^{2,l-1}$. As far as notations are concerned, the classes are taken from the right:  $[g]=\{gh\tq h\in H\}$; in particular the class of the identity $e$ is denoted by $\mfo$; the groups $\SO(2,l-1)$ and $\SO(1,l-1)$ are denoted by $G$ and $H$ respectively and their Lie algebras by $\sG$ and $\sH$. Following proposition~\ref{PropGHconn}, we can in fact only consider the identity components of $G$ and $H$. We denote by $\tau$ the natural action of $G$ on $G/H$:
\begin{equation}
\begin{aligned}
 \tau\colon G\times AdS_l&\to AdS_l \\
   \tau_r[g]&= [rg]
\end{aligned}
\end{equation}

As far as dimensions are concerned, a candidate $R\subset G$ such that $R\cdot\mfo$ is open must satisfy
\begin{equation}\label{cond_dim}
                  \dim\mR\geq\dim M.
\end{equation}

The case that interest us is $G=\SOdn$ and $H=\SOun$:\nomenclature{$AdS_n$}{Anti de Sitter space}
\[
M=AdS_{n+1}=\dfrac{\SOdn}{\SOun},
\]
 so that we have to consider the action of $\SO(2,n)$ on $AdS_n$.  If $ANK$ is the Iwasawa decomposition of $\SO(2,n)$, we can consider more particularly the action of $R=AN$, and ask us if the orbit $R\cdot\mfo$ is open or not. It is easy to see that the condition \eqref{cond_dim} is satisfied. Indeed,
\[
 \dim\lG=\frac{n(n-1)}{2}+2n+1,\qquad\dim\lK=\frac{n(n-1)}{2}+1,
\]
so that $\dim(\mA\oplus\mN)=2n$, but $\dim AdS_n=n$. The Iwasawa subgroup\index{Iwasawa!group} $AN$ is a candidate for $AN\cdot\mfo$ to be open in $AdS_n$.

\begin{proposition}
The homogeneous space $AdS_l$ is reductive\index{reductive!$AdS_n$}.
\label{PropAdSreduct}
\end{proposition}

\begin{proof}
The proof relies on lemma~\ref{lem:Killing_ss_descent} and the fact that $\SO(2,n)$ is semisimple. From the Killing form of $G$, one defines
\[
   \sQ=\sH^{\perp}=\{X\in\sG:B(X,H)=0\,\forall H\in\sH\}.
\]
Let $H$, $H'\in\sH$ and $Y\in\sQ$. From $\ad$-invariance of the Killing form, we have $B([H,Y],H')=0$. Hence $(\ad(\sH)\sQ)\subset \sQ$ and the claim is proved.

\end{proof}

Matrices of $\SO(2,n)$ are $(2+n)\times(2+n)$ matrices while the $n$-dimensional anti de Sitter space is a quotient of $\SO(2,n-1)$. In order to avoid confusions, we will reserve the letter $n$ to the study of the group $\SO(2,n)$ and the letter $l$ will denote the dimension of the anti de Sitter space which will thus be $AdS_l$.

We define the involutive automorphism $\sigma=\id|_{\sH}\oplus(-\id)|_{\sQ}$.  The vector space $\sQ$ can be identified with the tangent space $T_{[e]}AdS_l$, and that identification can be extended by defining $\sQ_g=dL_g\sQ$. In this case $\dpt{d\pi}{\sQ_g}{T_{[g]}AdS_l}$ is a vector space isomorphism.\label{PgdpibaseQTgM} An homogeneous metric on $T_{[g]}AdS_l$ is defined as in subsection~\ref{SubsecKillHomo}.

Cartan decomposition of $\SO(2,l-1)$ are of crucial importance in chapter~\ref{ChapBHinAdS}, so that we want to use a Cartan involution $\theta$ such that $[\sigma,\theta]=0$ (see \cite{Loos} page 153, theorem 2.1). One can show that $X\mapsto -X^t$ has that property. The corresponding Cartan decomposition is described in subsection~\ref{SubSecCartandeuxN}.

As a consequence of relations \eqref{EqDefRedHQ},
\begin{equation}  \label{EqdpiAdpi}
d\pi\Ad(h)=\Ad(h) d\pi
\end{equation}
because, if $X\in\sQ$, $d\pi^{-1}(X)=\{ X+Y\tq Y\in\sH \}$, so $\Ad(h)Y\in\sH$ and $\Ad(h)X\in \sQ$.

\subsection{Anti de Sitter as symmetric space}\index{symmetric!space}
%--------------------------------------
\label{pg:AdS_n_syme}

We know the decomposition $\sodn=\sQ\oplus\sH$. From equation \eqref{EqDefRedHQ} one can find an involutive automorphism $\sigma$ of $\sG$ which leaves $\sH$ invariant.

There exists a neighbourhood $U$ of $0$ in $\sodn$ on which $\exp$ is diffeomorphic to a neighbourhood $V$ of $e$ in $\SOdn$. We define $\dpt{\sigma_G}{V}{V}$ by $\sigma_G(e^X)=e^{\sigma X}$. Now, this $\sigma_G$ can be extended to the whole $G$. From now we will denote by $\sigma$ this map or its differential (i.e. an abuse of notation between $\sigma$ and $d\sigma_e$).

All this make $(\SOdn,\SOun)$ a symmetric pair. Since $H=\SOun$ is connected and fixed by $\sigma$, $H=H_{\sigma}=(H_{\sigma})_0$. Thus theorem ~\ref{tho:sigma_theta} gives us a Cartan involution $\theta$ on $\sG$ such that $[\sigma,\theta]=0$ and theorem~\ref{tho:sym_homo} gives a symmetric structure to $M=G/H$. Now we understand the computations of page \pageref{pg:calcul_sigma_theta}.

\section{Open and closed orbits}
%+++++++++++++++++++++++++++++++

\subsection{Open orbits in anti de Sitter spaces}
%----------------------------------------------

Let us start by computing the closed orbits of the actions of $AN$ and $A\bar{N}$ on $AdS_l$. In order to see if $[g]\in AdS_l$ belongs to a closed orbit of $AN$, we ``compare'' the space spanned by the basis $\{d\pi dL_g q_i\}$ of $T_{[g]}AdS_l$ and the space spanned by the fundamental vectors of the action. If these two spaces are equal, then $[g]$ belongs to an open orbit (because a submanifold is open if and only if it has same dimension as the main manifold). That idea is precisely contained in the following theorem which holds for any homogeneous space $M=G/H$.

\begin{probleme}
Il faut trouver une référence pour ce théorème.
\end{probleme}

The strategy is to to check openness of the $R$-orbit of $[g]$ by checking openness of the $\AD(g^{-1})R$-orbit of $\mfo$ using the theorem~\ref{tho:pr_ouvert}.

The problem is simplified by the following remark.  We know that matrices of $K$ and $H$ are given by
\begin{equation}	\label{eq:K_H_SO}
  K\leadsto \begin{pmatrix}
                \SO(2)&   \\
		      & \SO(n)
            \end{pmatrix},\quad
  H\leadsto \begin{pmatrix}
                    1 & \\
		     & \SOun
            \end{pmatrix},
\end{equation}
so we obviously have
\[
\bigcup_{s\in \SO(2)} \tau_{AN}([s]) =\bigcup_{\substack { s\in \SO(2)\\ h\in \SO(n)}}[ANsh] =\bigcup_{k\in K} [ANk] =[G].
\]
This is nothing else than the fact that the $AN$-orbits are $AN$-invariant.
So the $K$ part of $[g]=ank$ alone fixes the orbit which contains $[g]$ and we have at most one orbit for each element in $\SO(2)$. Computations using theorem~\ref{tho:pr_ouvert} show that the $R$-orbits of $[\mu]$ with
\[
\mu=
\begin{pmatrix}
\cos\mu &\sin\mu\\
-\sin\mu&\cos\mu\\
&&\mtu
\end{pmatrix}
\]
is not open if and only if $\sin \mu=0$. We will see later that they are actually closed (page \pageref{PgTopoOrb}), so that the singularity is described as
\begin{equation}\label{Sing2}
\hS=[AN(\pm\mtu_{\SO(2)})]\bigcup[A \bar{N}(\pm\mtu_{\SO(2)})].
\end{equation}
 Because of $AN$-invariance of the $AN$-orbits, the equation of the $AN$-closed orbits can be expressed as
\begin{equation}
\sin \mu=0.
\end{equation}

Let us recall that $-\mtu_{\SO(2)}=k_{\theta}= e^{\pi q_0}$. With these notations, we have that the closed orbits of $AN$ are
\begin{equation}
	\begin{aligned}[]
		[AN]&&\text{and}&&[ANk_{\theta}]=[k_{\theta}A\bar N],
	\end{aligned}
\end{equation}
while the closed orbits of $A\bar N$ are given by
\begin{equation}
	\begin{aligned}[]
		[A\bar N]&&\text{and}&&[A\bar Nk_{\theta}]=[k_{\theta}AN].
	\end{aligned}
\end{equation}

Notice that there are some differences between the two choices of Iwasawa decompositions of equations \eqref{TabelPrem} and \eqref{TableSeconde} in the determination of open and closed orbits. In the $AN$ Iwasawa decomposition, up to matrices of $\sH$ (given in equation \eqref{eq:gene_H}), a general matrix of $\sR$ is $jJ_1+mM+lL+kJ_2$. If we note $x=m+l$,
\begin{equation} \label{eq:geneR}
\sR\leadsto
\begin{pmatrix}
0&x&k&-x\\
-x\\
k\\
-x
\end{pmatrix}
\end{equation}
and it is obvious that the matrix $q_0$ can't be obtained by combinations of such matrices. So the $R$-orbit of $\mfo$ is not open.

We can do the same computation with the Iwasawa group $\bar\sR=\sA\oplus\bar\sN$. A general element of this is of the form $jJ_1+kJ_2+nN+fF$. If we write $a=n+f$ and $b=n-f$, we get
\begin{equation}
	\begin{pmatrix}
 0	&	a	&	k	&	a	&	0\\
 -a	&	0	&	b	&	j	&	0\\
 k	&	b	&	0	&	b	&	0\\
 a	&	j	&	-b	&	0	&	0\\
0	&	0	&	0	&	0	&	0
 \end{pmatrix}.
\end{equation}
Looking at the positions of the $a$, we see that it is impossible to put the element $q_0$ under that form. We deduce that the $\bar R$-orbits of $\mfo$ are not open neither.

That situation is, however, not generic. If we use for example the other Iwasawa decomposition, the one of subsection~\ref{SubSecANbarIwa}, the result is completely different. We have
\begin{align}
  q_{0}&=\pr\left( \frac{ N+M }{ 2 } \right),
&q_{1}&=\pr H_{2},
&q_{2}&=\pr\left( N-\frac{ N+M }{ 2 } \right),
\end{align}
and other elements of $\sQ$ are projections of the matrices $V_{i}$'s.  So we see that the map $\dpt{\pr}{\iR}{\sQ}$ is surjective and \label{pg:mfo_ouvert} the orbit $R\cdot\mfo$ is open.

Here is some explicit matricial computation.
\[
   M+N=2
 \underbrace{
\begin{pmatrix}
  0 &1&0&0\\
  -1&0&0&0\\
  0 &0&0&0\\
  0 &0&0&0
\end{pmatrix}}_{\displaystyle\in\sQ}
+
\underbrace{
\begin{pmatrix}
  0&0&0&0\\
  0&0&2&0\\
  0&2&0&0\\
  0&0&0&0
\end{pmatrix}}_{\displaystyle\in\sH},
\]
thus $\pr(\frac{M+N}{2})$ is yet a part of $\sQ$. An other:
\[
 \underbrace{
\begin{pmatrix}
  0 &0&1&0\\
  0&0&0&0\\
  1 &0&0&0\\
  0 &0&0&0
\end{pmatrix}}_{\displaystyle =m_2}
= \underbrace{
\begin{pmatrix}
  0 &0  &1 &0\\
  0 &0  &0 &-1\\
  1 & 0 &0 &0\\
  0 &-1 &0 &0
\end{pmatrix}}_{\displaystyle =H_1}
+
\underbrace{
\begin{pmatrix}
  0&0&0&0\\
  0&0&0&1\\
  0&0&0&0\\
  0&1&0&0
\end{pmatrix}}_{\displaystyle =h\in\sH},
\]
so that $\pr H_1=\pr(m_2-h)=m_2$. Third,
\[
\underbrace{
\begin{pmatrix}
 0&0&0&1\\
 0\\
 0\\
 1
\end{pmatrix}}_{\displaystyle=m_3}
=
\underbrace{N-\frac{M+N}{2}}_{\displaystyle\in\iR}
-
\underbrace{%
\begin{pmatrix}
  0&0&0 &0\\
  0&0&0 &0\\
  0&0&0 &1\\
  0&0&-1&0\\
\end{pmatrix}}_{\displaystyle\in\sH},
\]
thus $\pr(N-\frac{M+N}{2})=m_3$. The last possibility in $\sQ$ is $m_i=E_{1i}+E_{i1}$ ($i\geq 5$), but
\[
  \underbrace{V_i}_{\displaystyle\in\iR}
    =\underbrace{E_{1i}+E_{i1}}_{\displaystyle =m_i}+\underbrace{E_{3i}-E_{i3}}_{\displaystyle\in\sH}.
\]

%+++++++++++++++++++++++++++++++++++++++++++++++++++++++++++++++++++++++++++++++++++++++++++++++++++++++++++++++++++++++++++
\section{Some complements}
%+++++++++++++++++++++++++++++++++++++++++++++++++++++++++++++++++++++++++++++++++++++++++++++++++++++++++++++++++++++++++++

\subsection{A first brute force computation}
%/////////////////////////////////////////////

Let us use the ``old''{} Iwasawa decomposition for a little demonstrative and inessential computation. The exponentiations from $\sA\oplus\sN$ to $AN$ is given at page \pageref{pg:exp_AN}. Remark that a matrix of $\SOun$ leave unchanged the first column of $\SOdn$:
\[
\begin{pmatrix}
  a&.&.&.\\
  b&.&.&.\\
  c&.&.&.\\
  d&.&.&.\\
\end{pmatrix}
\begin{pmatrix}
 1&0&0&0\\
 0&.&.&.\\
 0&.&.&.\\
 0&.&.&.
\end{pmatrix}=
\begin{pmatrix}
  a&.&.&.\\
  b&.&.&.\\
  c&.&.&.\\
  d&.&.&.\\
\end{pmatrix}.
\]
Thus, in order to compute the orbit $[AN]$ of $[\mtu]$, one can begin to compute a general matrix of $AN$ and impose conditions on the first column (it will not be affected by the equivalence classes). For example, in order to see if $[-\mtu]\in [\SO(2)]$ belongs to the orbit $[AN]$, we compute:\label{PgExplAN}
\begin{equation}\label{eq:gene_R}
   \begin{pmatrix}
  \cosh\xi &    0      & \sinh\xi &    0  \\
    0       & \cosh\eta &    0      & \sinh\eta \\
  \sinh\xi &    0      & \cosh\xi &    0     \\
    0       & \sinh\eta &    0      & \cosh\eta \\
     \end{pmatrix}
     \begin{pmatrix}
                 1-2ab & b+a & 2ab   & b-a \\
 		 -b-a  & 1   & b+a   &  0   \\
		 -2ab  & b+a & 1+2ab &  b-a \\
		  b-a  &  0  &  a-b  &  1
	      \end{pmatrix}
\end{equation}
and we impose the first column to be $( -1,0,0,0)$:
\begin{subequations}\label{eq:S_14}
\begin{align}
  (1-2ab)\cosh\xi&-(2ab)\sinh\xi=-1   \label{eq:S1}\\
  (-b-a)\cosh\eta&+(b-a)\sinh\eta=0    \label{eq:S2}\\
  (1-2ab)\sinh\xi&-(2ab)\cosh\xi=0     \label{eq:S3}\\
  (-b-a)\sinh\eta&+(b-a)\cosh\eta=0.  \label{eq:S4}
\end{align}
\end{subequations}
It is easy to see that these equations doesn't accept any solutions. Indeed, the sum of equations \eqref{eq:S2} and \eqref{eq:S4} gives
\[
   -2a(\cosh\eta+\sinh\eta)=0.
\]
The possibility $a=0$ gives $\cosh\xi=-1$ in \eqref{eq:S1}; but the hyperbolic cosine is always bigger than one. Then $\cosh\eta=-\sinh\eta$. In this case, \eqref{eq:S4} gives $2b\cosh\eta=0$; since $\cosh\eta=0$ is not possible, $b$ must be zero. But with $b=0$, \eqref{eq:S1} gives once again $\cosh\xi=-1$.

Now we have to see that the matrices $e^{V_i}$ and $e^{W_j}$ doesn't change the result\label{pg:influence_V_W}. Since $W_j\in\sH$, it is clear that they will not change any result. If we compute $e^{cV_5}$ for example ($V_5^3=0$), we find
\begin{equation}		\label{EqExpDeV}
   e^{cV_5}=
\begin{pmatrix}
1+c^2/2	&		.	&	 -c^2/2	&	.	&	c\\
.		&	1	&	.	&	.	&	.\\
c^2/2		&	.	&	1-c^2/2	&	.	&	c\\
.		&	.	&	.	&	1	&	.\\
c		&	.	&	-c	&	.	&	1
\end{pmatrix}
\end{equation}
By multiply it by matrices of $AN$, we find
\[
\begin{pmatrix}
 &.&.&.&0\\
 &.&.&.&0\\
 &.&.&.&0\\
 &.&.&.&0\\
 &.&.&.&1\\
\end{pmatrix}
\begin{pmatrix}
 1+c^2/2&.&.&.&.\\
 0&.&.&.&.\\
 c^2/2&.&.&.&.\\
 0&.&.&.&.\\
 c&.&.&.&.
\end{pmatrix}
=
\begin{pmatrix}
 .&.&.&.&.\\
 .&.&.&.&.\\
 .&.&.&.&.\\
 .&.&.&.&.\\
 c&.&.&.&.
\end{pmatrix}.
\]
Then $c=0$ if we want it to be equal to $\pm\mtu$. As far as the matrices $W_j$ are concerned, it is even simpler: $W_j\in\sH$; then it will not affect the classes.

All that suppose that $N$ can \emph{globally} be written under the form
\[
  e^{aM}e^{bN}\prod_{i=5}^{n}e^{c_iV_i}\prod_{j=5}^{n}e^{c_jW_j}
\]
It is locally true from lemma~\ref{lem:decomp}.


\subsection{Search for \texorpdfstring{$Z(K)$}{ZK}}
%////////////////////////////

Now we are going to find the center of $K$. From the explicit form \eqref{eq:K_H_SO}, we see that $Z(K)=\SO(2)$. Let us show it more abstractly. We consider the decomposition $\sG=\sH\oplus\sQ$ of $\sodn$ with respect to the involution $\sigma$:
\[
   \iK=\iKQ\oplus\iKH;
\]
This is an expression of the fact that $\SOdn/\SO(1,n)$ is a symmetric space. Since $\dim Z(K)=1$, it is a subset of $\iKQ$ or of $\iKH$ because $Z(K)$ is $\sigma$

On the other hand, we know\quext{Faudra un peu voir pourquoi} that a Cartan involution can be written as $\theta=\Ad(\exp Z)$ for a $Z\in \mZ(\iK)$. Then
\begin{equation}
\begin{split}
  0=[\sigma,\theta](X)&=\sigma\Ad(e^Z)X-\Ad(e^Z)\sigma X\\
                      &=\sigma e^{\ad Z}X-\Ad(e^Z)\sigma X\\
		      &=\Ad(e^{\sigma Z})\sigma X-\Ad(e^Z)X.
\end{split}
\end{equation}
Since $G$ has $\pm\mtu$ as center, this implies $e^{\sigma Z}=e^Z$. Then $e^Z\in H$. We can't however conclude that $z\in\sH$.

\subsection{The same with the ``old'' Iwasawa decomposition}
%//////////////////////////////////////////////////////////////////////

A general matrix of $\tsR$ is $mM+nN+tH_1+uH_2$; with the change of variable $x=n+m$, $y=n-m$, $a=t+u$ and $b=u-t$, it is
\begin{equation}
r=
\begin{pmatrix}
  0  & x & a  & y\\
  -x & 0 & x  & b\\
  a  & x & 0  & y\\
  y  & b & -y & 0
 \end{pmatrix}.
\end{equation}

Now we want to explicitly compute $\sR_z$ for a general matrix $z$ in the ``$\SO(2)$''\ part of $\SOdn$, i.e. for
\[
   z=
\begin{pmatrix}
  \cos u  & \sin u \\
  -\sin u & \cos u \\
    &  & 1  & 0\\
    &  & 0  & 1
 \end{pmatrix}.
\]
The result is
\begin{equation}
\begin{split}
  \Ad&(z)\sR=\\
&\begin{pmatrix}
  0      & x      & a\cos u+x\sin u & y\cos u+b\sin u\\
  -x     & 0      & x\cos u-a\sin u & b\cos u-y\sin u\\
  a\cos u+x\sin u   & x\cos u -a\sin u &   0   & y\\
  y\cos u+b\sin u  & b\cos u-y\sin u &   -y  & 0
 \end{pmatrix}.
\end{split}
\end{equation}
We are interested in the projection of these matrices with respect to $\sH$ (see equation \eqref{eq:gene_H}); then in order to see if $\sR_z=\sR$, we have to compare
\[
\begin{pmatrix}
  0      & x & a\cos u+x\sin u & y\cos u+b\sin u\\
  -x     &  &        & \\
  a\cos u+x\sin u  &  &        & \\
  y \cos u+b\sin u  &  &        &
 \end{pmatrix}
\]
with
\[
\begin{pmatrix}
  0  & x' & a'& y'\\
  -x' &  &  & \\
  a'  &  &  & \\
  y'  &  &  &
 \end{pmatrix}
\]
By working on $a,b,x$ and $y$, we can easily fit the first matrix on the second
one if $c\neq 0$. In other words, $\sR_z\neq\sR$ only if the $\SO(2)$
part of $z$ is the rotation of an angle of $\pm\pi/2$. All that we can say now is that $\cos u=0$ is  not in the $\tR$-orbit of $\mfo$.

Up to here we have not taken the matrices $V_i$ and $W_j$ into account. It is rather easy to see that they don't change anything because (taking $i=j=5$ for sake of notational simplicity)
\[
  V_i+W_i\simeq
\begin{pmatrix}
0&0&0&0&1\\
0\\
0\\
0\\
1
\end{pmatrix}
\]
but
\[
  \Ad(z)(V_i+W_i)\simeq
\begin{pmatrix}
0&0&0&0&\sin u+\cos u\\
0\\
0\\
0\\
\sin u+\cos u
\end{pmatrix}.
\]

\subsection{A non-open orbit and a precision}\label{subsec:precision}
%-------------------------------------------

We consider the situation where $[\sigma,\theta]=0$ and the Iwasawa decomposition of $\SOdn$ for which $R_H$ is a subgroup of $R$ ($R_H$ is the ``$AN$''{} of $\SOun$). We denote by $\mfo$ the identity class: $\mfo=[e]=eH$. Since $\sH=\sR_H\oplus\sK_{\sH}$,
\[
   T_{\mfo}(R\mfo)=\pr_{\sH}(\sR)=(\sR+\sH)/\sH
\]
where the sum $\sR+\sH$ is not a direct sum. But $\sR+\sH=\sR+(\sR_H\oplus\sK_{\sH})=\sR\oplus\sK_H$ because $\sK_H\subset\sK$. Thus
\[
   T_{\mfo}(R\mfo)=(\sR\oplus\sK_H)/(\sR_H\oplus\sK_H)=\sR/\sR_H.
\]
\label{pg:subt_tilde}In the case $AdS_3$, $\dim(R\mfo)=6-3=3<4$; so that the identity orbit is not open. It is important to note that it contradicts the result of page \pageref{pg:mfo_ouvert}. The reason is that the latter was obtained with the ``old''{} Iwasawa decomposition.

From now the ``old''\ decomposition will be denoted by a tilde:
\[
\lG=\tsA\oplus\tsN\oplus\sK.
\]
In this case, there exists a $k_0\in K$ such that $\Ad(k_0^{-1})\tsA=\sA$.  Moreover with this $k_0$, we have $\tA=k_0Ak_0^{-1}$ and $\tN=k_0Nk_0^{-1}$. This result can be found in \cite{Helgason} and maybe\quext{Faut que tu v\'erifies.} in \cite{Wisser}\quext{Quand tu auras refusion\'e, il faudra r\'ef\'erentier ta transcription de Wisser.}.

With these precisions, the previous computations are still relevant because
\begin{equation}
   \tR[g]=k_0Rk_0^{-1}[g],
\end{equation}
which assures that the $\tR$-orbit of $[g]$ is the $R$-orbit of $[k_0^{-1} g]$ translated by
$k_0$. Since the translation is a diffeomorphism, the openness is not affected by the
translation. So the open $\tR$-orbit of $\mfo$ stated at page
\pageref{pg:mfo_ouvert} is now the open $R$-orbit of $k_0^{-1}$.

Now let us find out this famous $k_0$ matrix. Since the orbits are defined by the elements of the center of $K$ (i.e. a bloc-diagonal matrix in $\SOdn$ with a $\SO(2)$ matrix in the upper left corner and $\mtu$ anywhere else), we guess that $k_0$ is such a matrix. Before to compute it, we take a change of basis in $\tsA$: we consider $\frac{1}{2}(H_1+H_2)$ and $\frac{1}{2}(H_2-H_1)$ instead of $H_1$ and $H_2$. If we consider $d=\begin{pmatrix} 1&0\\0&0 \end{pmatrix}$, the problem is to find a matrix $z\in \SO(2)$ such that
\[
   \begin{pmatrix}
     z&0\\
     0&\mtu
   \end{pmatrix}
   \begin{pmatrix}
     0&d\\
     d&0
   \end{pmatrix}
   \begin{pmatrix}
     z^{-1}&0\\
     0&\mtu
   \end{pmatrix}
   =
   \begin{pmatrix}
    0&0&0&0\\
    0&0&1&0\\
    0&1&0&0\\
    0&0&0&0
   \end{pmatrix}.
\]
We can easily find that the only solution is
$z=\begin{pmatrix}
     0&-1\\
     1&0
   \end{pmatrix}$.
Then
\begin{equation}
  k_0=
  \begin{pmatrix}
    0 &1&&\\
    -1&0&&\\
    &&1&0\\
    &&0&1
   \end{pmatrix}.
\end{equation}

\subsection{The same with the  ``old'' Iwasawa decomposition}
%/////////////////////////////////////////////////////////////////////

This useless point shows how to get the same conclusion with a bad choice of Iwasawa decomposition. If we consider some $z$ in the centrer of $K$ ($\SO(2)$) and $\tR_z=z\tR z^{-1}$, then the $\tR$-orbit of $\mfo$ is the $\tR$-orbit of $[z^{-1}]$ translated by $z$ and the $R$-orbit of $k_0z^{-1}$ translated by $zk_0^{-1}$:
\begin{equation}
   \tR_z=z\tR z^{-1}=zk_0 Rk_0^{-1} z^{-1}.
\end{equation}
Thus the study of the openness of the $\tR_z$-orbit of $\mfo$ is the study of the openness of the $R$-orbit of $k_0z^{-1}$. Proposition~\ref{tho:pr_ouvert} allow us to perform this study by means of the surjectivity of $\dpt{\pr_{\sH}}{\tilde{\sR}_z}{\sQ}$.

We begin by compute the matrices of $\sR_z$: $zH_1z^{-1}$, $zH_2z^{-1}$,
$zMz^{-1}$, $zNz^{-1}$, $zV_iz^{-1}$, $zW_jz^{-1}$. We make the following change of
basis in $\tsR$:
\begin{subequations}
\begin{align}
  H_1,H_2&\to \frac{1}{2}(H_1+H_2),\frac{1}{2}(H_2-H_1),\\
  M,N&\to\frac{1}{2}(M+N),\frac{1}{2}(M-N).
\end{align}
\end{subequations}
The result is that with
\begin{equation}
 z=\begin{pmatrix}S&0\\0&\mtu\end{pmatrix}\;\text{where}\; S=\begin{pmatrix}c&s\\-s&c\end{pmatrix},
\end{equation}
and $c^2+s^2=1$, we have
\begin{equation}
H_{1z}=
\begin{pmatrix}
 &   & c &0\\
 &   &-s &0\\
c & -s &  &\\
0 & 0  &  &\\
\end{pmatrix},\quad
H_{2z}=
\begin{pmatrix}
 &   &  &s\\
 &   &  &c\\
 &   &  &0\\
s & c  & 0 &0\\
\end{pmatrix};
\end{equation}
\begin{equation}
M_z=
\begin{pmatrix}
0  & 1 & s &0\\
-1 & 0 & c &0\\
s  & c & 0 &0\\
0  & 0 & 0 &0\\
\end{pmatrix},\quad
N_z=
\begin{pmatrix}
  &  &  &-c\\
  &  &  &s\\
  &  &  &-1\\
-c & s & 1 &0\\
\end{pmatrix};
\end{equation}
\begin{subequations}
\begin{align}
   V_{iz}&=c(E_{1i}+E_{i1})-s(E_{2i}+E_{i2})+E_{3i}-E_{i3},\\
   W_{jz}&=s(E_{1j}+E_{j1})+c(E_{2j}+E_{j2})+E_{4j}-E_{j4}.
\end{align}
\end{subequations}

The matrices we \emph{really} need are the projections of these one with respect to $\sH$. We denote it by  symbols with a line:
\begin{subequations}
\begin{align}
\overline{H}_{1z}&=
\begin{pmatrix}
 &   &c  &0\\
 &   &0 &0\\
c & 0  &  &\\
0 & 0  &  &\\
\end{pmatrix},
&\overline{H}_{2z}&=
\begin{pmatrix}
 &   & 0 &s\\
 &   & 0 &0\\
0 & 0  &  &\\
s & 0  &  &\\
\end{pmatrix},\\
\overline{M}_z&=
\begin{pmatrix}
0  & 1 & s &0\\
-1 & 0 & 0 &0\\
s  & 0 & 0 &0\\
0  & 0 & 0 &0\\
\end{pmatrix},
&\overline{N}_z&=
\begin{pmatrix}
  &  &  &-c\\
  &  &  &0\\
  &  &  &0\\
-c & 0 & 0 &0\\
\end{pmatrix}\\
  \overline{V}_{iz}&=c(E_{1i}+E_{i1})
 &\overline{W}_{jz}&=s(E_{1j}+E_{j1}).
\end{align}
\end{subequations}

An explicit study shows that $\dpt{\pr_{\sH}}{\tsR_z}{\sQ}$ is surjective if and only if $c\neq 0$. Then the open $R$-orbits are the ones of $k_0^{-1} z^{-1}$ with $c\neq 0$:
\begin{equation}
\begin{pmatrix}
  0 & -1 & 0 & 0\\
  1 & 0  & 0 & 0\\
  0 & 0  & 1 & 0\\
  0 & 0  & 0 & 1\\
\end{pmatrix}
\begin{pmatrix}
  c & -s & 0 & 0\\
  s & c  & 0 & 0\\
  0 & 0  & 1 & 0\\
  0 & 0  & 0 & 1\\
\end{pmatrix}
=
\begin{pmatrix}
  -s & -c & 0 & 0\\
  c  & -s & 0 & 0\\
  0  &  0 & 1 & 0\\
  0  & 0  & 0 & 1\\
\end{pmatrix}
\end{equation}
Consequently, the not open $R$-orbits are the ones of $[\pm\mtu_{\SO(2)}]$.

\subsection{Orbits as homogeneous spaces}\index{homogeneous!space}
%----------------------------------------
A homogeneous space is a space with a transitive homeomorphism group. We consider the orbit $\mO=R[z]$ ($z\in \SO(2)$) which is a homogeneous space because the action of $G$ is trivially transitive. Theorem~\ref{tho:homeo_action} assures us that $\mO$ can be written as $R/S$ where $S$ is the group which fixes some point in $\mO$.

We will firstly work out the homogeneous structure of the open $\tsR$-orbits; the matrices of $\tsR$ which leave invariant the point $[\mtu]$ of $G/H$ are the ones of the product \eqref{eq:gene_R} which satisfy some equations that are almost the same as \eqref{eq:S_14}:
\begin{subequations}
\begin{align}
  (1-2ab)\cosh\xi&-(2ab)\sinh\xi=1   \\
  (-b-a)\cosh\eta&+(b-a)\sinh\eta=0   \\
  (1-2ab)\sinh\xi&-(2ab)\cosh\xi=0    \\
  (-b-a)\sinh\eta&+(b-a)\cosh\eta=0.
\end{align}
\end{subequations}
It is easy to see that the solutions are given by $a=b=\xi=0$ and no constraint on $\eta$. The set $S$ is thus given by
\begin{equation}
S\leadsto
\begin{pmatrix}
 1 & 0 & 0 & 0 \\
 0 & \cosh\eta & 0 & \sinh\eta \\
 0 & 0 & 1 & 0 \\
 0 & \sinh\eta & 0 & \cosh\eta \\
\end{pmatrix}.
\end{equation}
which corresponds\footnote{More precisely, because of the quotient by $H$, we had derived a necessary characterization of $S$, not a sufficient one. But we will soon see that in facts this is also a sufficient condition.} to $ e^{t(H_{1}-H_{2})}$ via the change of variable \eqref{eq:chm_xi_eta}. Since this is a matrix of $H$, this leaves $[\mtu]$ unchanged. As far as the matrices $V_i$ and $W_j$ are concerned, the reasoning of page \pageref{pg:influence_V_W} still holds.

So the part of $\tsR$ which leaves $\mtu$ unchanged is $\tsR\cap H$ (this is not really amazing). Now recall that $\tsR=k_0Rk_0^{-1}$. If $\tilr\in\tsR$ fixes $\mtu$, then $k_0^{-1}\tilr k_0\in R$ fixes $k_0^{-1}$, and if $r\in R$ fixes $k_0^{-1}$, then $k_0rk_0^{-1}\in\tsR$ fixes $\mtu$, so that $k_0rk_0^{-1}\in\tR\cap H$.

Then $r\in R$ fixes $[k_0^{-1}]$ if and only if $r\in k_0^{-1}(\tsR\cap H)k_0$. On the other hand, for the closed orbits the stabilizer in $R$ is $R\cap H$: $r\in R$ fixes $[\mtu]$ if $r[\mtu]=[\mtu]$, i.e. $[r]=[\mtu]$, which needs a $h\in H$ such that $r=h$.

\section{Causality, light cone and related topics on anti de Sitter} \label{SecCausal}
%++++++++++++++++++++++++++++++++++++++++++++++++++++++++++++++++++

We particularize the general definitions of subsection~\ref{SubSecGeneBH} to the case of the anti de Sitter space. We consider the $l$-dimensional\footnote{The symbol $n$ denotes the number of space-like directions of the underlying space of the matricial group $\SO(2,n)$; this space has dimension $n+2$ while $AdS$ is a quotient by (something like) one time-like direction. In order to avoid confusions, the symbol $l$ denotes the dimension of the $AdS$ space. This is the reason for which we write $\SOdn$ and $AdS_l$. %
    So equation \eqref{eq:defAdS} is best written as \[AdS_l=\frac{\SOdn}{\SO(1,n)}.\]} anti de Sitter space
\begin{equation}    \label{eq:defAdS}
  AdS_l=\frac{ \SO(2,l-1) }{ \SO(1,l-1) }(\equiv u^2+t^2-x_1^2-\cdots-x_{l-1}^2=1).
 \end{equation}
According to proposition~\ref{PropGHconn}, we can only consider the identity component of $\SO(2,l-1)$ and $\SO(1,l-1)$ instead of full groups\footnote{Since we are about to consider Iwasawa decompositions of these groups, actually we \emph{have to} use the identity components.}. The metric that we put on $AdS_l$ is the one induced from the Killing form of $\SO(2,l-1)$ by formula \eqref{EqDefMetrHomo}. This metric has a Minkowskian signature, so that we have  natural notions of time-, space- and light-like vectors. From now we denote by $G$ and $H$ the groups $\SO(2,l-1)$ and $\SO(1,l-1)$.

An other beautiful way to see that the metric on $AdS$ as one and only one time-like direction is the following. The tangent space of $AdS$ at the point $(u,t,x_1,\cdots,x_{l-1})$ is the orthogonal complement (in $\eR^{2,l-1}$) of that vector. From the very definition of $AdS$, the given vector is time-like (its norm is $1$), so that it remains one and only one time-like vector in the tangent space.

We write \( ANK\) the Iwasawa decomposition of the connected group $\SO_0(2,l-1)$ (see the theorem \ref{ThoIwasawaVrai}). Let $A\bar N$ be the $\theta$-conjugate\footnote{Roughly speaking, it corresponds to different choices in the Iwasawa decomposition of $\SO(2,l-1)$.}group of $AN$ where $\theta$ is the Cartan involution of subsection~\ref{SubSecCartandeuxN}. We will see that the actions of $AN$ and $A\bar N$ have closed and open orbits. The closed ones are denoted by $\hS_{AN}$ and $\hS_{A\bar N}$. The following definition is motivated all previously existing work about BTZ black hole.

\begin{remark}
Here, we consider $\SOun$ as a subgroup of $\SOdn$. Thus the matrices of $\SOun$ are $(n+2)\times (n+2)$ of the form
\[
\begin{pmatrix}
   1 & 0\\
   0 & \fbox{M}
\end{pmatrix}
\]
where $M$ is a $(n+1)\times (n+1)$ matrix of the ``true''\ $\SOun$. From this, one can believe the closeness of $\SOun$ in $\SOdn$.
\end{remark}

In order to get a full definition of the black hole and its structure, we need to define and characterise the notions of light ray and light cone. These notions are of course directly issued from physics of relativity.
\begin{definition}
A \defe{light ray}{light!ray} is a geodesic whose tangent vector is everywhere light-like.
\label{lightraycone}
 \end{definition}

The \defe{causal structure}{causal!structure} of a general pseudo-Riemannian manifold $M$ is the fact that two points are said to be \emph{causally connected} when there exists a light ray which passes by both points. More precisely, we say that $x$ has a \defe{causal effect}{causal!effect} on $y$ if there exists a future oriented time-like path $c\colon [0,1]\to M$ such that $c(0)=x$ and $c(1)=y$.

A light ray trough $\mfo$ is given by a vector of $\sQ$ with vanishing norm. So let us study these vectors. Let $E_1=q_0+q_1$ and $k$, a general element of  $\SO(n)$ which reads $k= e^{K}$ with $K=a^{ij}(E_{ij}-E_{ji})$, $i,j\geq 3$ and $a^{ij}=-a^{ji}$.  If we pose $A_j=E_{1j}+E_{j1}$, we have $[K,E_1]=(2a)^{j3}A_j$ and $[K,A_k]=a^{jk}A_j$. Hence,
\[
\ad(K)^nE_1=\big((2a)^n\big)^{k3}A_k,
\]
and
\begin{equation} \label{eq:Adkeu}
\begin{split}
\Ad(k)E_1=e^{\ad K}E_1&=E_1+\sum_{n\geq 1}\big( (2a)^n\big)^{k3}A_k\\
	      &=E_1+\sum_{n=0}^{\infty}\big(  (2a)^n \big)^{k3}A_k-\delta^{j3}A_j\\
		&=E_1-E_{31}-E_{13}+\big( e^{2a}\big)^{j3}A_j\\
	      &=q_0+\sum_{j=1}^{l-1}w_jq_j
\end{split}
\end{equation}
where $w_i=\big(  e^{2a} \big)^{i3}$. Under an explicit form, we have
 \begin{equation} \label{eq:AdkE}
   \Ad(k)E_1=
\begin{pmatrix}
0&1&w_1&w_2&\ldots\\
-1\\
w_1\\
w_2\\
\vdots
\end{pmatrix}
\end{equation}
The exponential $ e^{2a}$ being an element of $\SO(n)$, the parameters $w_i$ are restricted by the condition $\sum_{k}w_k^2=1$.  Remark moreover that \emph{every} matrix of $\SO(2)$ can be written under the form $e^{2a}$ for a good choice of $a\in\sod$. The light cone is therefore given by the set of vectors of the form $(1,w_i)$ with $\|w\|^2=1$. If we consider the metric $diag(+--\cdots)$ on $\sQ$ with respect to the basis $\{q_i\}$, we have
\[
  \|\Ad(k)E_1\|^2=0.
\]
This is coherent with the intuitive notion of light cone. On the one hand \emph{every} light-like vector of $\sQ$ reads $\Ad(k)E_1$ for some $k\in\SO(n)$. On the other hand every nilpotent element of $\sQ$ is light-like because trace of nilpotent matrix is zero (using \wikipedia{en}{Engel_theorem}{Engel's theorem}). In definitive, we proved the following:

\begin{proposition}		\label{PropNormZeroEQnil}
When $E$ is any nilpotent element of $\sQ$, the set of light-like vectors of $\sQ$ is parametrized by $\lambda\Ad(k)E$ with $k\in\SO(n)$ and $\lambda\in\eR$.
\label{PropToutVectLumQ}
\end{proposition}

\begin{corollary}		\label{CorNormZeroEQnil}
An element of $\sQ$ has a vanishing norm if and only if it is nilpotent.
\end{corollary}

\begin{proof}
We know that, when $E$, is any nilpotent in $\sQ$, the set of vanishing norm vectors in $\sQ$ are given by $\{ \lambda\Ad(k)E \}$, but all these vectors are nilpotent.
\end{proof}

Let us point out the fact that only the first column of the ``direction''{} $k\in \SO(n)$ has an importance in causality issues. So the word ``directions''{} will often be used to refer to the vector $w$. It is not a particular feature of our particular matrix representation choice. Indeed the element $k$ only appears in the combination $\Ad(k)E$ which is a light-like vector in $\sQ$, i.e. $\Ad(k)E=tq'_0+\sum_i x_iq'_i$ with $t^2-\sum_i x_i^2=0$ for any orthonormal basis $\{q'_i\}$ of $\sQ$. As far as causality is concerned, a rescaling $\Ad(k)E$ to $\lambda\Ad(k)E$ has no importance, so one can choice $t=1$ and find back $\sum_i x_i^2=1$. We see that it is a natural feature that the light-like rays are parametrized by  unital vectors of $\eR^n$.

\begin{lemma}		\label{LemGeodGenreLumiere}
Let $E$ be a nilpotent element in $\sQ$, and $\pi: G \rightarrow G/H$, the canonical projection. A light ray through $[g]\in AdS_l$ has the form
\begin{equation}
   l^k_{[g]}(s)=\pi\big( ge^{-s\Ad(k)E} \big)
\end{equation}
for a certain $k\in K_H=K\cap H=\SO(n)$.
 \label{lem:AdkEcone}
\end{lemma}

\begin{proof}
General theory of symmetric spaces (see \cite{kobayashi2}, pages 230--233, particularly theorem 3.2) proves that a light ray through $\mfo=[e]$ has the form
\[
  l(s)=\pi\big( e^{sX} \big).
\]

\begin{probleme}
	Il me semble que ce qui est de cette forme, ce sont les géodésiques, et non les rayons de lumière. Relire Kobayashi-Nomizu.
\end{probleme}


In our context, we have the additional request for the tangent vector to be light-like. Proposition~\ref{PropToutVectLumQ} thus imposes $X$ to be of the form $\Ad(k)E$. That proves the claim for geodesics trough $\mfo$.

The fact that $d\tau_g$ is an nondegenerate isometry then extends the result to all points.

\end{proof}

\begin{corollary}		\label{CorNilLightQ}
If $E$ is nilpotent in $\sQ$, then $\{\Ad(k)E\}_{k\in K_H}$ is the set of light-like vectors in $T_{[\mfo]}AdS_l\simeq\sQ$. Therefore
\begin{equation}
  \exp_{\mfo}( t\Ad(k)E )=\exp(t\Ad(k)E)\cdot\mfo.
\end{equation}
is the light cone of $\mfo$ in $AdS_l$.
    Note that in this equation, the first $\exp$ is the
one defined from the $AdS_l$-connection while the second is the exponential from a Lie algebra to the Lie group. It comes from the fact that in a symmetric space, $\exp_o v=e^z\cdot\mfo$.
\end{corollary}

In order to fix ideas, we will always use the element $E_1$ as choice of nilpotent element in $\sQ$ in order to parametrize light-cone.  Since $\SO(2,l-1)$ acts on $AdS_l$ by isometries, the \defe{light cone}{light!cone} at $\pi(g)$ is given by a translation of the one at $\mfo$:
\begin{equation}	\label{eq_defcone}
  C^+_{\pi(g)}=g\cdot C_{\mfo}=\{  \pi\big( g e^{t\Ad(k)E_1}  \big)  \}_{\substack{t\in\eR^+\\ k\in K_H}}.
\end{equation}
The product being taken at left while the quotient being taken at right, one can fear a problem of well definiteness in this expression. The following proposition shows that all is right.

\begin{proposition}		\label{PropDefConeIndepRepre}
Definition \eqref{eq_defcone} is independent of the representative $g$ in the class $\pi(g)$. In other words,
\begin{equation}  \label{eq_statdefcone}
  \{ \Ad(hk)E_1 \}_{k\in K_H}=\{ \Ad(k)E_1 \}_{k\in K_H}
\end{equation}
for all $h\in H$.
\end{proposition}

\begin{proof}
The metric on $\sQ$ is the restriction of the Killing form of $\sG$ (notice that $\sQ$ has no own Killing form for the simple reason that it is not a Lie algebra). From $\Ad$-invariance, we have in particular
\[
  B\big(\Ad(h)X,\Ad(h)Y \big)=B(X,Y)
\]
for all $h\in \SO(1,l-1)$. The point is that reducibility makes $\Ad(h)X\in\sQ$ when $X\in\sQ$. The element $\Ad(hk)E_1$ in the left hand side of equation \eqref{eq_statdefcone} being zero-normed in $\sQ$, it reads $\Ad(k')E_1$ for some $k'\in K_H$. That proves the inclusion in one sense. For the second inclusion, we have to find a $k'\in K_H$ such that $\Ad(hk')E_1=\Ad(k)E_1$. Existence of such a $k'$ follows from the fact that $\Ad(h^{-1}k)E_1$ is a light-like vector of $\sQ$.
\end{proof}

\begin{remark}		\label{RemGedNonInvarChoix}
Although the \emph{set} of geodesics $\{ \pi(g e^{s\Ad(k)E_2}) \}$ is equal to the \emph{set} of geodesics $\pi(gh e^{s\Ad(k)E_1})$, each geodesic are not independent in the choice of the representative $g$: $\pi(g e^{\Ad(k)E_1})\neq\pi(gh e^{\Ad(k)E_1})$ in general.

In particular, in the setting of the anti de Sitter black hole, the property ``intersect the singularity'' for the geodesic $\pi(g e^{s\Ad(k)E_1})$ is not invariant under the choice of the representative $g$ in the class $[g]$.
\end{remark}

It is also possible to prove result of independence~\ref{PropDefConeIndepRepre} with a lot of matricial computations: let us decompose $h=a_hn_hk_h$; the part $k_h$ is just a redefinition of $k$ in equation \eqref{eq_statdefcone}, so we forget it. We begin by proving that \eqref{eq_statdefcone} holds whenever $\Ad(h)\in \SO(\sQ)$. Consider $\Ad(k')E_1=X\in\sQ$. If $\Ad(h)\in \SO(\sQ)$, then $\Ad(h^{-1})\in \SO(\sQ)$ too and we consider $Y=\Ad(h^{-1})X$ which is a vector of norm zero in $\sQ$. There exists $\bar k\in K_H$ such that $\Ad(\bar k)E_1X=Y$. Now,
\begin{equation}
\Ad(h\bar kk')E_1=\Ad(h\bar k)X
		=\Ad(h)Y
		=X.
\end{equation}
In order to prove that $\Ad(a_h)\in \SO(\sQ)$, we compute
\[
  \ad(J_1)\begin{pmatrix}
0	& z	& w_1	& w_2	& w3\\
-z\\
w_1\\
w_2\\
w_3
\end{pmatrix}
=\ad(J_1)(zq_0+w_iq_i).
\]
In the basis $\{ q_0,q_i \}$, we see that
\[
  \ad(J_1)=\begin{pmatrix}
0&0&-1&0\\
0\\-1\\0
\end{pmatrix}\in\mathfrak{so}(1,3),
\]
so $\Ad(J_1)\in \SO(\sQ)$. On the other hand, a general element of $\sN_{\sH}$ is
\[
  A=\begin{pmatrix}
\cdot\\
&\cdot& a&\cdot& v\\
&a&\cdot &-a&\cdot\\
&  \cdot& a&\cdot& v\\
&v&\cdot&-v&\cdot\\
\end{pmatrix},
\]
and simple computations shows that on $\sQ$,
\[
  \ad(A)=\begin{pmatrix}
\cdot &-a&\cdot&-v\\
-a&\cdot&-a&\cdot\\
\cdot&a&\cdot &v\\
-v&\cdot&v&\cdot
\end{pmatrix}\in\mathfrak{so}(1,3).
\]

\subsection{Time orientation}
%////////////////////////////

A \defe{time orientation}{time!orientation} on $\sQ$ is the choice of a vector $T$ such that $\scal{T}{T}>0$. When such a choice is made, a vector $v$ is \defe{future directed}{future!directed vector} when $\scal{v}{T}>0$. In our case, the choice is the intuitive one: the vector $q_0$ defines the time orientation on $\sQ$ and $v=(v^0,v^1,v^2,v^3)$ is future directed if and only if $v^0>0$. So a light-like future directed vector is always --up to a positive multiple-- of the form $(1,\overline{v})$ with $\|\overline{v}\|=1$. For this reason, the set
\begin{equation}	\label{EqTousVecLumTy}
  \{t\Ad(k)E_1\}_{%
\begin{subarray}{l}
t>0\\k\in \SO(3)
\end{subarray}
}
\end{equation}
is exactly the set of light-like future-directed vectors of $\sQ$.

We are now able to define causality as follows.  A point $[g]\in AdS_l$ belongs to the \defe{interior region}{interior!region} if for every direction $k\in K_H$, the future light ray $l^k_{[g]}$ intersects the singularity within a \emph{finite} time.  In other words, it is interior when the whole light cone ends up in the singularity.  A point which is not interior is said to be \defe{exterior}{exterior!point}. A particularly important set is the \defe{event horizon}{event horizon}, or simply \emph{horizon}, defined as the boundary of the interior. When a space contains a non trivial causal structure (i.e. when there exists a non empty horizon), we say that the definition of singularity gives rise to a \defe{black hole}{black hole}. By extension, the term ``black hole'' often refers to the set of interior points.

%///////////////////////////////////////////////////////////////////////////////////////////////////////////////////////////
\subsubsection{Singularity}
%///////////////////////////////////////////////////////////////////////////////////////////////////////////////////////////

\begin{definition}		\label{Singular}
    The \defe{singularity}{singularity} in $AdS_l$ is the set
    \[
      \hS=\text{singularity}=\hS_{AN}\cup\hS_{A\bar N},
    \]
    so that a point is \defe{singular}{singular!point in a black hole} when it belongs to a closed orbit of $AN$ or $A\bar N$. The \defe{black hole}{black hole} is defined as
    \[
      BH=\{ x\in AdS_{l} \text{ st } \forall \text{ time-like vector } k\in T_xAdS_l,\,  l^k_x\cap\mathcal{S}\neq\emptyset \}
    \]
    where $l^k_x$ is the (future directed) geodesic in the direction $k$ starting at $x$ (see equation \eqref{EqTousVecLumTy} and the discussion above).
\end{definition}

The aim of this chapter is to prove that the so-defined black hole is non trivial in the sense that the following inclusions are strict:
\begin{equation}		\label{EqhSssubBH}
    \hS\varsubsetneq BH\varsubsetneq AdS_l.
\end{equation}

\subsection{Action of \texorpdfstring{$H$}{H} and \texorpdfstring{$\Ad(\sQ)$}{AdQ}}
%///////////////////////////////

Remember that we decree closed orbits to be \emph{singular}. Now the fact for a point $\pi(g)\in AdS_l$ to be \emph{exterior} is that there exists an non empty set $\mO$ of $K_H$ such that $\forall k\in\mO$,
\[
  \pi\big( g e^{t\Ad(k)E_1}  \big)\cap\mS=\emptyset.
\]

The restriction of the Killing form to $\sQ$ reads
\begin{subequations}
\begin{align}
	B(q_0,q_0)&=\tr(q_0q_0)=-2,\\
	B(q_{i},q_{i})&=\tr(q_{i},q_{i})=2&\textrm{for $i\geq 1$}.
\end{align}
\end{subequations}
So the norm on $\sQ$ is $\| X \|=-\frac{ 1 }{2}B(X,X)$. The bi-invariance of the Killing form and the fact that the decomposition $\sG=\sQ\oplus\sH$ is reductive  imply $\| \Ad(h)X \|=\| X \|$, hence
\begin{equation}  \label{EqInclAdHSOq}
  \Ad(H)|_{\sQ}\subset\SO(\sQ).
\end{equation}
A question is to know the kernel of this inclusion: which $h\in H$ fulfill $\Ad(h)q_i=q_i$ for all $i$? The equation $Aq_iA^{-1}=q_i$ can be simplified (from a computational point of view) using the relation $A^{-1}=\eta A^t\eta$ which defines $\SO(1,n)$. It is a somewhat long but easy computation to prove that $A=\pm\mtu$ are the only two solutions in $SO(1,n)$ to the system $A(q_i\eta)A^t=q_i\eta$.

One can go further than inclusion \eqref{EqInclAdHSOq} and prove the following
\begin{proposition}
 Let $h\in H_0$ seen as a matrix acting on $\eR^{1,l-1}$ and let see $\Ad(h)$ as a matrix acting on $\sQ$. In this case we have $\Ad(h)_{ij}=h_{ij}$. In particular
\begin{equation}
   \Ad(H_0)=\SO_0(\sQ)
\end{equation}
where the index zero denotes the identity component.
\label{PropSOADHequal}
\end{proposition}

\begin{proof}
We will prove that for each unital vector $X\in\sQ$, the element $\Ad(h)X$ is a general element of norm $1$ in $\sQ$ when $h$ runs over $H_0$. Explicit matrix computation will show by the way the equality  $\Ad(h)_{ij}=h_{ij}$. The general product to be computed is
\[
\Ad(h)X=
  \begin{pmatrix}
1	&	0\\
0	&
\begin{pmatrix}
&&\\
&h^{-1}\\
&&
\end{pmatrix}
\end{pmatrix}
\begin{pmatrix}
0&-w_0&w_1&\cdots\\
w_0\\
w_1\\
\vdots
\end{pmatrix}
  \begin{pmatrix}
1	&	0\\
0	&
\begin{pmatrix}
&&\\
&h\\
&&
\end{pmatrix}
\end{pmatrix}.
\]
But we know that the result is a matrix of $\sQ$, so it is sufficient to compute the first line. If we denote by $c_i$ the columns of $h$, we find
\[
  \Ad(h)X=\sum_{i=0}^{l-1}(w\cdot c_i)q_i
\]
where the dot denotes the inner product of $\eR^{1,l-1}$. Since $\{ c_i \}$ is a general orthonormal basis of $\eR^{1,l-1}$, the latter expression is a general vector of norm $1$ in $\sQ$.
\end{proof}

\subsection{Two other characterizations of the singularity}		\label{SubSecTwoCharSing}
%++++++++++++++++++++++++++++++++++++++++++++++++++++++++

In this short section, we first give a coordinatewise characterization of the singularity (which allows some brute force computations), and then we point out that the vector field $J_1^*$ has vanishing norm on the singularity (see also proposition~\ref{PropAdSDeuxJannule}). That should make the connection with the quotient construction of the original BTZ black hole.  Notice that we do not classify all vectors from which vanishing of the norm define a singularity. The point is that one can make our black hole ``causally inextensible'' by making a discrete quotient of $AdS_l$ along the integral curves of $J^*_1$.

\begin{proposition}		\label{Proptcarrycarr}
In term of the embedding of $AdS_l$ in $\eR^{2,l-1}$, the closed orbits of $AN \subset \SO(2,l-1)$ are located at $y-t = 0$.  Similarly, the closed orbits of $A \bar{N}$ correspond to $y+t=0$. In other words, the equation
\begin{equation} \label{tcarrycarr}
t^2-y^2=0
 \end{equation}
describes the singularity $\hS=\hS_{AN}\cup\hS_{A\bar{N}}$.

More precisely, a point belongs to a closed orbit of $AN$ if and only if $t-y=0$ and to a closed orbit of $A\bar N$ if and only if $t+y=0$.
\end{proposition}

\begin{proof}
The different fundamental vector fields of the $AN$ action can be computed with the matricial relation $X^*_{[g]}=-Xg\cdot\mfo$. For example, in $AdS_3$,
\[
\begin{split}
   M^*_{[g]}&=
\begin{pmatrix}
0&-1&0&1\\
1&0&-1&0\\
0&-1&0&1\\
1&0&-1&0
\end{pmatrix}
\begin{pmatrix}
u\\t\\x\\y
\end{pmatrix}
=
\begin{pmatrix}
-t+y\\u-x\\-t+y\\u-x
\end{pmatrix}\\
&=(y-t)\partial_u+(u-x)\partial_t+(y-t)\partial_x+(u-x)\partial_y.
\end{split}
\]
Full results are
\begin{subequations}\label{Gen}
\begin{align}
J_1^*&=-y\partial_t-t\partial_y							\label{EqNormeJun}\\
J_2^*&=-x\partial_u-u\partial_x                                                      \label{eq:Jds}\\
M^*  &=(y-t)\partial_u+(u-x)\partial_t+(y-t)\partial_x+(u-x)\partial_y\\
L^*  &=(y-t)\partial_u+(u+x)\partial_t+(t-y)\partial_x+(u+x)\partial_y\\
W_i^*&=-x_i\partial_t-x_i\partial_y+(y-t)\partial_i\\
V_j^*&=-x_j\partial_u-x_j\partial_x+(x-u)\partial_j,
\label{eq:Vjs}
\end{align}
\end{subequations}
with $i,j=3,\ldots,l-1$.
First consider points satisfying $t-y=0$. It is clear that, at these points, the $l$ vectors $J_1^*$, $M^*$, $L^*$ and $W_i^*$ only span the direction $\partial_t+\partial_y$. Thus, there are at most $l-1$ linearly independent vectors amongst the $2(l-1)$ vectors \eqref{Gen}. We conclude that a point satisfying $t-y=0$ belongs to a closed orbit of $AN$.

Now we show that a point with $t-y\neq 0$ belongs to an open orbit of $AN$. It is easy to see that $J_1^*$, $M^*$ and $L^*$ are three linearly independent vectors. The vectors $V_i^*$ gives us $l-3$ more. Then they span a $l$-dimensional space.

The same can be done with the closed orbits of $A\bar{N}$. We have
\begin{subequations}
\begin{align}
	N^*	&=	-(y+t)\partial_u+(u-x)\partial_t-(y+t)\partial_x+(x-u)\partial_y\\
	F^*	&=	-(y+t)\partial_u+(x+y)\partial_t+(y+t)\partial_x-(x+u)\partial_y\\
	X^*_i	&=	-x_i\partial_u+x_i\partial_x-(x+u)\partial_i\\
	Y_j^*	&=	z\partial_t-z\partial_y+(y+t)\partial_i.
\end{align}
\end{subequations}
When $t+y=0$, the vectors $J_1^*$, $N^*$, $F^*$ and $Y_j^*$ only span the direction $\partial_t-\partial_y$. On the other hand, if $t+y\neq 0$, we look at the vectors $J_1^*$, $N^*$ and $F^*$. The vector $J_1^*$ is linearly independent of $N^*$ and $F^*$ because is does not contain a $\partial_u$ component. Now, the vector $N^*$ contains a component $\partial_u+\partial_x$ while $F^*$ contains $\partial_u-\partial_x$. We conclude that the vectors $J_1^*$, $N^*$ and $F^*$ span three linearly independent vectors. Thus a point with $t+y\neq 0$ belongs to an open orbit of $A\bar N$.

The result is that a point belongs to a closed orbit of $A\bar{N}$ if and only if $t+y=0$.
\end{proof}
This shows that in the three dimensional case, our black hole reduces to the previously existing one.

The following corollary shows that a discrete quotient of $AdS_l$ along the orbits of $J_1^*$ gives a direct higher-dimensional generalization of the non-rotating BTZ black hole\footnote{It was also proved for \( AdS_2\) in proposition~\ref{PropAdSDeuxJannule}.}.
\begin{corollary} \label{CorJannsingul}
The singularity coincides with the set of points in $AdS_l$ where $\| J_1^* \|^2 = 0$ for the metric induced from the ambient space $\eR^{2,l-1}$.
\end{corollary}

\begin{proof}
    The expression \eqref{EqNormeJun} shows that the norm of $J_1^* $ is $y^2-t^2$ which vanishes on the singularity.
\end{proof}

In the three-dimensional case, it was shown in \cite{BTZ_deux,BTZB_un} that the non-rotating BTZ black hole singularity is precisely given by equation \eqref{tcarrycarr}. Hence, the following is a particular case of theorem~\ref{ThoLeBut}:

\begin{corollary}
    The non-rotating BTZ black hole is a causal symmetric solvable black hole.
\end{corollary}

\subsection{A criterion with the tangent spaces}\label{subsec:R_z}
%-----------------------------------------------

Since $G$ acts transitively on $G/H$, the tangent spaces of $G/H$ at different points are not really independents: it is possible to guess global structure from consideration about tangent spaces. If $\mO$ denotes the orbit of $[z]$ under $R$, we have
\[
   T_{[z]}\mO=\Span\{X^*_{[z]}\tq X\in\sR\}.
\]
\begin{probleme}
C'est le lemme~\ref{lem:equiv_1} pris en un autre point. Il faut trouver un argument pour voir que c'est correct.
\end{probleme}

We can work out the structure of the fundamentals vector fields\index{fundamental!vector field}:
 \begin{equation}
  X^*_{[z]}=\Dsdd{ \pi(e^{-tX}z) }{t}{0}
	   =(d\pi)_z(dR_z)_e\Dsdd{e^{-tX}}{t}{0}
	   =-(d\pi)_z\utX_z
\end{equation}
where $\utX$ denotes the right invariant vector field of $X\in\sG$.

\begin{probleme}
Il y a presque certainement une faute de notation entre le tilde au dessus et celui en dessous.
\end{probleme}

If $\tau$ is the action of $G$ on $G/H$, we can try to bring the expression of $X^*_{[z]}$ in $T_{[e]}\mO$ in the following sense:
\begin{equation}
\begin{split}
(d\tau_{z^{-1}})_{[z]}X^*_{[z]}&=(d\tau_{z^{-1}})_{\pi(z)}(d\pi)_z\utX_z
                              =d(\tau_{z^{-1}}\circ \pi)_z\utX_z\\
			      &=\Dsdd{ \pi( z^{-1} e^{-tX}z ) }{t}{0}
			      =(d\pi)_e\Ad(z^{-1})X.
\end{split}
\end{equation}
Now we define the space\nomenclature{$\sR_z$}{Trick to compute open orbits}
\begin{equation}
\sR_z=(d\pi)_e\Ad(z^{-1})\sR,
\end{equation}
and we can state a necessary condition for two points to belongs to the same orbit.

\begin{proposition}
If the elements $z$ and $z'$ of $G$ are related by $r\in R$ (i.e. $z'=rz$), then $\sR_{z'}=\sR_z$.
\end{proposition}

\begin{proof}
If is just a computation. Let $z'=rz$; we have
\begin{equation}
\begin{split}
\sR_{z'}=(d\pi)_e\Ad(z'\,\!^{-1})\sR
        =(d\pi)_e\Ad(z^{-1} r^{-1})\sR
	=(d\pi)_e\Ad(z^{-1})\Ad(r^{-1})\sR
	=\sR_z
\end{split}
\end{equation}
because $\Ad(r^{-1})\sR=\sR$.
\end{proof}

Taking the general form \eqref{eq:geneR} of an element in $\sR$, we compute
\begin{equation}
\begin{split}
\Ad(z)\sR&=
\begin{pmatrix}
\cos\mu & \sin\mu\\
-\sin\mu & \cos\mu\\
&&\mtu
\end{pmatrix}
\begin{pmatrix}
0&x&k&-x\\
-x&0&0&j\\
k&0&0&0\\
-x&j&0&0
\end{pmatrix}
\begin{pmatrix}
\cos\mu & -\sin\mu\\
\sin\mu & \cos\mu\\
&&\mtu
\end{pmatrix}
\\
&\simeq
\begin{pmatrix}
0& x&k\cos\mu &-x\cos\mu+j\sin\mu\\
-x\\
k\cos\mu\\
-c\cos\mu+j\sin\mu
\end{pmatrix}
\end{split}
\end{equation}
where $\simeq$ stand for ``equals up to a matrix of $\sH$''. If $\cos\mu=0$, then $\sR_z\neq\sR$. This shows that
\begin{equation}
\begin{pmatrix}
0&1\\
-1&0\\
&&1\\
&&&1
\end{pmatrix}
\text{ and }
\begin{pmatrix}
0&-1\\
1&0\\
&&1\\
&&&1
\end{pmatrix}
\end{equation}
does not belong to the orbit of $\mfo$. We also see that $\cos \mu=-1$ is either not in the orbit of $\mfo$.

\subsubsection{Search for open orbits}
%/////////////////////////////////////

We consider the group $R_z=\AD(z)R=zRz^{-1}$ for some $z\in \SO(2)$. The openness of the $R_z$-orbit of $\mfo$ is the same as the one of the $R$-orbit of $[z^{-1}]$. Matrices of $\sR_z=z\sR z^{-1}$ are easy to find. Here are the projections on $\sQ$:
\begin{subequations}
\begin{align}
\pr_{\sQ} M_z&=
\begin{pmatrix}
0&1&\sin\mu&-\cos\mu\\
-1\\
\sin\mu\\
-\cos\mu
\end{pmatrix}
&\pr_{\sQ} L_z&=
\begin{pmatrix}
0&1&-\sin\mu&-\cos\mu\\
-1\\-\sin\mu\\
-\cos\mu
\end{pmatrix}\\
 \pr_{\sQ} {J_1}_z&=
\begin{pmatrix}
0&0&0&\sin\mu&\\
0\\
0\\
\sin\mu
\end{pmatrix}
&\pr_{\sQ} {J_2}_z&=
\begin{pmatrix}
0&0&\cos\mu&0\\
0\\
\cos\mu\\
0
\end{pmatrix}\\
 \pr_{\sQ} {W_i}_z&=\sin\mu(E_{i1}+E_{1i})&\pr_{\sQ} {V_i}_z&=\cos\mu(E_{i1}+E_{1i})
\end{align}
\end{subequations}
When $\sin\mu$ and $\cos\mu$ are non-zero, we have
\begin{subequations}
\begin{align}
q_0&=\frac{1}{2}\big(M_z+L_z+\frac{\cos\mu}{\sin\mu}{J_1}_z\big)&q_1&=\us{\cos\mu}{J_2}_z\\
q_2&=\us{\sin\mu}{J_1}_z&q_i&=\us{\sin\mu}{W_i}_z=\us{\cos\mu}{V_i}_z
\end{align}
\end{subequations}
 So when $\sin\mu=0$, the element $q_{0}$ does not belong to $\pr_{\sQ}\sR_{z}$. Hence the $R_z$-orbit of $\mfo$ is non open if and only if $\sin \mu=0$.

\subsection{Orbits  and topology}
%--------------------------------
\label{PgTopoOrb}

Let  $D^{\pm}=AN\SO(n)\SO(2)^{\pm}$ where $\SO(2)^{\pm}$ are the subgroups of $\SO(2)\subset \SO(2,n)$ with strictly positive (negative) cosine. We see $\SO(2)$ and $\SO(n)$ as subgroups of $\SO(2,n)$ in the way indicated by equation \eqref{eq:K_H_SO}. Notice that the parts $\SO(2)$ and $\SO(n)$ are commuting and that $\SO(n)\subset H$. The notation $-\mtu_{\SO(2)}$ refers to the element of $\SO(2,n)$ which the identity as $AN$-component and $-\mtu$ as $\SO(2)$-component.

A continuous path from $[D^+]$ to $[D^-]$ must pass trough an element of the form $[AN\mtu_{\SO(2)}]$. We saw that the $AN$-orbit of such an element is not open while the $AN$-orbit of an element of $[D^+]$ is open. So we deduce that an orbit passing trough $[D^+]$ does not intersect $[D^-]$.

The set $[D^+]$ is connected in $G/H$ and $D^+$  being open in $G$, the set $[D^+]=\pi(D^+)$ is also open in $G/H$ from the definition of the topology (see theorem~\ref{tho:struc_anal}). Now, the orbits of $AN$ in $[D^+]$ (who are all open) furnish an open partition of $[D^+]$. Such a partition is impossible for an open connected set. We deduce that $[D^+]$ is only one orbit of $AN$ in $G/H$. The same can be done with $[D^-]$.

We are left with the sets $[AN]$ and $[AN(-\mtu_{\SO(2)})]$ whose union is closed because we just saw that the complement is open. Now we prove that these two sets are disjoint, in such a way that they have to be separately closed. Existence of an intersection point between $[AN]$ and $[AN(-\mtu_{\SO(2)})]$ would lead to the existence of a $h\in H$ such that $an\mtu_{\SO(2)}=(-\mtu_{\SO(2)})h$, or
\[
  h=(-\mtu_{\SO(2)})an,
\]
that is a non trivial $K$-component to $h$ in the decomposition $KAN$, but the only $K$-component in $H$ is $\SO(n)$. Hence such a $h$ does not exist and $R[\mtu]\cap R[-\mtu_{\SO(2)}]=\emptyset$.

The conclusion is that the Iwasawa group $AN$ has only four orbits:
\begin{align}
[D^+],&&[D^-],&&[AN\mtu_{\SO(2)}],&&[AN(-\mtu_{\SO(2)})].
\end{align}
The two first are open and the other two are closed. Remark\label{PgNoticeKpassung} that an element of $[K]$ does not belong to a closed orbit of $AN$ or $A\bar N$.


\subsection{The volume form method}    \label{subsecVolumeForm}
%-----------------------------------

Let us give an alternative to proposition~\ref{tho:pr_ouvert} to study the openness of an $AN$-orbit. We explain the method for $\hS_{AN}$, but the same with trivial adaptations is true for $\hS_{A\bar{N}}$.

If $x\in M$ belongs to $\hS_{AN}$, the tangent space of its $AN$-orbit has lower dimension than the tangent space of $M$.  In this case the volume spanned by the fundamental vectors at $x$ is zero.  The idea is to build the volume form $\nu_x$ of $T_xM$ and then apply it on a basis of the fundamental fields.  If the result is zero, then $x$ belongs to the $\hS_{AN}$.  More precisely, the action is given by
		\begin{equation}
		\begin{aligned}
			\tau \colon AN\times M &\to M\
			(an,[g])&\mapsto [ang].
		\end{aligned}
	\end{equation}
If $X\in\sA\oplus\sN$ and $[g]\in M$, then
\begin{equation}
  X^*_{[g]}=-d(\pi\circ R_g)X.
\end{equation}
As mentioned in corollary~\ref{Cordpiietwii}, if $\{q_i\}$ is a basis of $\sQ$ then a basis of $T_{[g]}M$ is given by $\{d\pi dL_gq_i\}$. We define
\[
\nu=q_0^{\flat}\wedge q_1^{\flat}\wedge \ldots \wedge
q_{l-1}^{\flat}
\]
where $q_{i[g]}^{\flat}=B_{[g]}(d\pi dL_g q_i,\cdot)$. The condition for $[g]$ to belongs to $\hS_{AN}$ reads
\begin{equation}\label{eq:nusurN}
\nu_{[g]}(N_1^*{}_{[g]},N_2^*{}_{[g]},\ldots,N_l^*{}_{[g]})=0
\end{equation}
for every choices of $N_j$ in a basis of $\sA\oplus\sN$. It corresponds to the vanishing of $l \times l$ determinants. Our purpose is now to compute the products
\[
\begin{split}
  B_{[g]}(d\pi dL_g q_i,N^*_j{}_{[h]})	&=-B_g(\pr dL_g q_i,\pr dR_g N_j)\\
					&=-B_g(dL_g q_i,dR_g N_j)\\
					&=-B_e(q_i,\Ad(g^{-1})N_j).
\end{split}
\]
where $\pr\colon T_{g}M\to dL_{g}\sQ$ is the projection. The step from the first to the second line is as follows. First, $\pr dL_gq_i=dL_gq_i$ by definition. For the second, let us write $dR_g X=dL_g X_h+dL_g X_q$ with $X_h\in\sH$ and $X_q\in\sQ$. From equations \eqref{EqDefRedHQ}, we see that $B(\sQ,\sH)=0$, so $B(dL_g q_i,dL_g X_h+dL_g X_q)=B(dL_g q_i,dL_g X_q)$. Remark that one cannot do it computing $\|J_i^*\|$.

We consider the quantity
\[
\Delta_{ij}([g])=B(q_i,\Ad(g^{-1})N_j)
\]
where $N_j$ runs over a basis of $\sA\oplus\sN$ and $q_i$ a one of $\sQ$. Our problem of light cone (see explanations in section~\ref{SecCausal}) leads us to compute
 \begin{equation} \label{eq:elemtr}
\Delta_{ij}(\pi(ge^{-tk\cdot E}))=B(\Ad(e^{-tk\cdot
E})q_i,\Ad(g^{-1})N_j)
\end{equation}
where $k\cdot E$ is a notation for $\Ad(k)E$.

A way to proceed is, following proposition~\ref{prop:enuc},  to express all our elements of $\so(2,n)$ in the root space decomposition
\[
\sG=\sG_{(0,0)}\bigoplus_{\lambda\in\Sigma}\sG_{\lambda}.
\]
The purpose of that resides in the fact that the Killing form $B(X,Y)$ is easier to compute when $X$ and $Y$ belongs to some root spaces.

An important computational remark is the fact that $E$ is nilpotent, so $\Ad(k)E$ also is and $\Ad(e^{-t\Ad(k)E})X=e^{-t\ad(k)E}X$ only gives second order expressions with respect to $t$. These computations are nevertheless heavy, but can fortunately be circumvented by a simple counting of dimensions, as we describe in proposition~\ref{Proptcarrycarr}.

Let us make some computations now. In a first time, we restrict ourself to elements in $K$: we put $g=e^{uR}$ with
\[
R=
\begin{pmatrix}
0&1\\
-1&0\\
&&0\\
&&&0
\end{pmatrix}\in\sK.
\]
On the other hand, an useful way to express $k\cdot E_1$ is the following (cf. equations \eqref{eq:Adkeu} ):
\[
\Ad(k)E_1=
\begin{pmatrix}
0&1&w_1&w_2&w_3\\
-1\\
w_1\\
w_2\\
w_3
\end{pmatrix}
=q_0+w_1q_1+w_2q_2+w_3q_3\in \sQ.
\]
It should be noted that by choosing $k$, all the vectors $\begin{pmatrix}w_1&w_2&w_3\end{pmatrix}$ with $\|w\|^2=1$ are possible.

Let us begin by systematically computing the elements $[k\cdot E_1,q_i]$ and $[k\cdot E_1,[k\cdot E_1,q_i]]$; the others $\ad(k\cdot E_1)^nq_i$ are zero because one can see that $\ad(E_1)^3X_{\alpha}=0$ for all $X_{\alpha}$ in the root spaces. All computations can be performed by decomposing the $q_i$'s in the root space basis and using the known commutations relations between root spaces. The way we choose here is to directly use the huge formula
\begin{equation}
\begin{split}
[k\cdot E_1,[k\cdot E_1]]&=w_1^2[q_1,[q_1,q_i]]\\
                         &\quad +w_1w_2\big(  [q_1,[q_2,q_i]]+[q_2,[q_1,q_i]]  \big)\\
                         &\quad +w_1w_3\big(  [q_1,[q_3,q_i]]+[q_3,[q_1,q_i]]  \big)\\
                         &\quad +w_2^2[q_2,[q_2,q_i]]\\
                         &\quad +w_2w_3\big(  [q_2,[q_3,q_i]]+[q_3,[q_2,q_i]]  \big)\\
                         &\quad +w_3^2[q_3,[q_3,q_i]]\\
\end{split}
\end{equation}
and use the commutations relations between the $q_i$'s. The results are
\begin{equation}
\begin{split}
\ad(k\cdot E_1)q_0 &=\frac{w_1}{4}(M+N-L-F)+w_2J_1+\frac{w_3}{2}(W-Y)\\
\ad(k\cdot E_1)q_1 &=\frac{1}{4}(L+F-M-N)+\frac{w_2}{4}(F+M-L-M)-\frac{w_3}{2}(V+X)\\
\ad(k\cdot E_1)q_2 &=J_1+\frac{w_1}{4}(L+N-F-M)-\frac{w_3}{2}(W+Y)\\
\ad(k\cdot E_1)q_3 &=\frac{1}{2}(Y-W)+\frac{w_1}{2}(V+X)+\frac{w_2}{2}(W+Y)\\
\end{split}
\end{equation}
and
\begin{equation}
\begin{split}
\ad(k\cdot E_1)^2q_0 &=k\cdot E_1\\
\ad(k\cdot E_1)^2q_1 &=-w_1q_0+(w_2^2+w_3^2-1)q_1-w_1w_2q_2-w_1w_3q_3\\
\ad(k\cdot E_1)^2q_2 &=-w_2q_0-w_1w_2q_1+(w_1^2+w_3^2-1)q_2-w_2w_3q_3\\
\ad(k\cdot E_1)^2q_3 &=-w_3q_0-w_1w_3q_1-w_2w_3q_2+(w_1^2+w_2^2-1)q_3\\
\end{split}
\end{equation}
It is rather easy to check that $\ad(k\cdot E_1)^3q_i=0$ by virtue of $\|w\|^2=0$. All these expressions have to be extended in the basis of the root spaces.
\begin{equation}
\begin{split}
\ad(k\cdot E_1)^2q_0 &=\frac{1}{4}(M+N+L+F)+\frac{w_2}{4}(N+F-M-L)\\
                     &\quad +\frac{w_3}{2}(V-X)+w_1q_1\\
\ad(k\cdot E_1)^2q_1 &=-\frac{w_1}{4}(M+N+L+F)+ \frac{w_1w_2}{4}(M+L-N-F)\\
                     &\quad +\frac{w_1w_3}{2}(X-V) +(w_2^2+w_3^2-1)q_1\\
\ad(k\cdot E_1)^2q_2 &=-\frac{w_2}{4}(M+N+L+F)+\frac{w_1^2+w_3^2-1}{4}(N+F-M-L)\\
                     &\quad +\frac{w_2w_3}{2} (X-V)-w_1w_2q_1\\
\ad(k\cdot E_1)^2q_3 &= -\frac{w_3}{4}(M+N+L+F) +\frac{w_2w_3}{4}(M+L-N-F)\\
                     &\quad +\frac{w_1^2+w_2^2-1}{2}(V-X)-w_1w_3q_1
\end{split}
\end{equation}


\subsubsection{The column of \texorpdfstring{$V$}{V}}
%///////////////////////////////

An explicit computation shows that
\begin{equation}
\begin{split}
\Ad(e^{uR})V&=
\begin{pmatrix}
&&&&\cos u\\
&&&&-\sin u\\
&&&&1\\
&&&&0\\
\cos u&-\sin u&-1&0&0
\end{pmatrix}\\
  &=\frac{1}{2}(1-\cos u)X+\frac{1}{2}(\sin u) Y\\
  &\quad+\frac{1}{2}(1+\cos u)V-\frac{1}{2}(\sin u) W.
\end{split}
\end{equation}

\begin{remark}
Because of the invert in \eqref{eq:elemtr}, we are looking at the destiny of the point $[e^{-uR}]$, not the one of $[e^{uR}]$.
\end{remark}

Thanks to the properties of the root space decomposition, we know that the only non zero Killing form containing $X,Y,V,W$ are $B(W,Y)$ and $B(V,X)$. So in the expression
\[
\Ad(k\cdot E_1)q_0=q_0+\frac{tw_1}{4}(N+M+L+F)+tw_2J_1+\frac{tw_3}{2}(W-Y)+\frac{t^2w_3}{2}(V-X),
\]
we can forget the three first terms when we compute $\Delta_{q_0,V}$. The result is
\begin{equation}
\boxed{\Delta_{q_0,V}=B(W,Y)\frac{tw_3}{2}\sin u-B(V,X)\frac{t^2w_3}{4}\cos u}
\end{equation}
In the same way,
\begin{equation}
\boxed{\Delta_{q_1,V}=-B(V,X)\left( \frac{tw_3}{2}+\frac{t^2w_2w_3}{4} \right)},
\end{equation}
\begin{equation}
 \boxed{ \Delta_{q_2,V}=B(X,V)\frac{t^2w_2w_3}{4}\cos u },
\end{equation}
\begin{equation}
\boxed{\Delta_{q_3,V}=-B(V,X)\frac{1}{2}\big(  \cos u-tw_1+\frac{t^2}{2}(w_1^2+w_2^2-1)\cos u  \big)-B(W,Y)\frac{t}{2}\sin u}
\end{equation}
Remark that the only term in this column which doesn't vanishes when $t=0$ contains $\cos u$.

\subsubsection{The column of \texorpdfstring{$J_1$}{J1}}
%//////////////////////////////////

\begin{probleme}
C'est justement un de ceux que tu soup\c connes de ne servir \`a rien.
\end{probleme}

A direct computation shows that
\begin{equation}
\begin{split}
\Ad(e^{uR})J_1&=\sin(u) q_2+\cos(u) J_1\\
              &=\us{4}\sin(u)(N+F-M-L)+\cos(u) J_1.
\end{split}
\end{equation}
We only have to consider the non zero Killing form $B(J_1,J_1)$, $B(W,Y)$, $B(V,X)$, $B(N,L)$, $B(M,F)$.
\begin{equation}
\boxed{\Delta_{q_0,J_1}=6t^2w_2\sin u+6tw_2\cos u}
\end{equation}

\subsubsection{The column of \texorpdfstring{$J_2$}{J2}}
%//////////////////////////////////

For the computation of $\Ad(e^{uR})J_2$, we recall that $R=q_0$ and $J_2=q_1$. It is easy to see that $[q_0,q_1]=\us{4}(L+F-M-N)$ and $[q_0,[q_0,q_1]]=-q_1$, so that the exponential series looks good and gives
\[
  \Ad(e^{uR})q_1= \cos(u)q_1+\frac{\sin u}{4}(L+F-M-N).
\]
A lot of computation gives
\begin{equation}
\boxed{\Delta_{q_0,J_2}=3t^2w_1\cos u-6tw_1\sin u}
\end{equation}



\begin{equation}
\boxed{\Delta_{q_1,J_2}=-3t^2w_1\cos u+6t\sin u+6\cos u}
\end{equation}



\begin{equation}
\boxed{\Delta_{q_2,J_2}=-3t^2w_1w_2\cos u}
\end{equation}


\begin{equation}
\boxed{\Delta_{q_3,J_2}=-3t^2w_1w_3\cos u}
\end{equation}


\subsubsection{The column of \texorpdfstring{$M$}{M}}
%///////////////////////////////

The first computation is
\[
  \Ad(e^{uR})M=\frac{1-\cos u}{2}(F-M)+\sin(u)(q_1+J_1)+M.
\]

\begin{equation}
\boxed{\Delta_{q_0,M}=6(1+w_1\sin u)+6t(w_1(1-\cos u)+w_2\sin u)+3t^2(1+w_2\cos u).
}
\end{equation}

\begin{align}
\Delta_{q_1,M}=B
\Big(&
  q_1+\frac{t}{4}(F-M)+\frac{tw_2}{4}(F-M)\\
          &+\frac{t^2}{2}
      \big[
            -\frac{w_1}{4}(F+M)+\frac{w_1w_2}{4}(M-F)-w_1^2q_1
      \big],\\
  &\frac{1}{2}(1-\cos u)(F-M)+\sin u(q_1+J_1)+M
\Big).
\end{align}
Collecting the terms and using the following relations,
\begin{subequations}
\begin{align}
B(M+F,F-M)&=0&B(M-F,F-M)&=3\cdot 16\\
B(F-M,M)&=3\cdot 8&B(F+M,M)&=3\cdot 8.
\end{align}
\end{subequations}
we find
\begin{equation}
\boxed{\Delta_{q_1,M}=6\sin u-6t(2-\cos u)(1+w_2)+3t^2(w_1+w_1w_2\cos u-w_2^2\sin u)}
\end{equation}

\begin{equation}
\boxed{%
\begin{aligned}
\Delta_{q_2,M}=-6(2-\cos u)&+6t(\sin u+w_1)\\&+3t^2\big(
-w_2+\frac{w_2^2}{2}(1-\cos u)-w_1w_2\sin u
\big)
\end{aligned}
}
\end{equation}

\begin{equation}
\boxed
{
  \Delta_{q_3,M}=3t^2w_3(1+w_2(2-\cos u)-w_1\sin u).
}
\end{equation}

\subsubsection{Existence for \texorpdfstring{$AdS_3$}{AdS3}}
%////////////////////////////////////

From computer computations, the (non identically zero) volume determinants are given by
\begin{subequations}
\begin{align}
  &-32(tw_2+t\cos u-\sin u)^2\big(\cos u+t(\sin u-w_1)\big)\\
  &-32(tw_2+t\cos u-\sin u)^9\\
\begin{split}
16t^2w_3\Big(&w_2(w_1-\sin u)+\cos u(w_1-\sin u)-w_2\cos u\\
	&-\cos^2u+\sin u(-w_1+\sin u)
\Big)+16tw_3\cos u\sin u
\end{split}\\
&16tw_3
\big(
 -\cos u+t(w_1-\sin u)
\big)
(tw_2+t\cos u-\sin u)
\end{align}
\end{subequations}


One can deduce the existence of an horizon. Indeed the vanishing of all the determinants for a point in $[\SO(2)]$ with respect to the $AN$ singularity only requires
\begin{subequations} \label{eq:annul_trois}
\begin{align}
t_{AN}=\frac{\sin u}{\cos u-\sin k}
\intertext{while the same for $A \overline{N}$ requires}
   t_{A\overline{N}}=\frac{\sin u}{\sin k+\cos u}
\end{align}
\end{subequations}
The (class of the) point $u$ belongs to the black hole if for all $k\in \SO(2)$, $t_{AN}>0$ or $t_{A\overline{N}}>0$. In this case, all light-like geodesic from the point $u$ fall into the hole after a positive time. There are two possibilities:
\begin{subequations}
\begin{align}
\begin{split}
\sin u<0\\
\cos u<0
\end{split}\\
\intertext{or}
\begin{split}
\sin u>0\\
\cos u>0
\end{split}
\end{align}
\end{subequations}
Let us insist to the fact that the points $u=0$ and $u=\pi$ are not in the horizon although they separate black points and free points. These two points belongs to the singularity. In fact the spaces $\sin u\geq0$ and $\sin u \leq0$ are two completely separated spaces.

So in the space $\sin u\geq 0$, the point $u=\pi/2$ is part of the horizon. This proves the existence of an horizon and gives one point of it. The determination of the horizon is not likely easy.

\subsubsection{Existence for \texorpdfstring{$AdS_4$}{AdS4}}
%///////////////////////////////////

One can parametrize $\Ad(k)E_1$ as
\begin{equation}
\Ad(k)E_1=
\begin{pmatrix}
0&1&w_1&w_2&w_3\\
-1\\
w_1\\
w_2\\
w_3
\end{pmatrix}.
\end{equation}
The volume forms for the $AN$ and $A \overline{N}$ orbits are respectively annihilated by
\begin{equation}
t_{AN}=\frac{\sin u}{\cos u+w_2}, \text{ and } t_{A \overline{N}}=\frac{\sin u}{\cos u-w_2}.
\end{equation}
These are the same as \eqref{eq:annul_trois}. Once again the doomed part of the space is given by
\begin{subequations}
\begin{align}
\begin{split}
\sin u<0\\
\cos u<0
\end{split}\\
\intertext{or}
\begin{split}  \label{eq:possdeux}
\sin u>0\\
\cos u>0
\end{split}
\end{align}
\end{subequations}
For example in the case \eqref{eq:possdeux}, the directions with $\cos u<w_2<-\cos u$ escape the singularity.
\section{Existence of a non trivial horizon}		\label{SecExistenceHor}
%++++++++++++++++++++++++++++++++++++++++++++

We are now able to prove that definition~\ref{Singular} provides a non empty horizon satisfying condition \eqref{EqhSssubBH}.  First we  consider points of the form $\SO(2)\cdot\mfo$, which are parametrized by an angle $\mu$. By lemma~\ref{LemGeodGenreLumiere}, up to the choice of this parametrization, a light-like geodesic trough $\mu$ is given by
 \begin{equation}
   K\cdot \mbox{e}^{-s\Ad(k)E_1}\cdot\mfo
\end{equation}
with $k\in \SO(l-1)$ and  $s\in\eR$. Using the isomorphism $[g]\mapsto g\cdot \mfo$ between $G/H$ and $AdS_l$, we find
\begin{equation}		\label{EqhohnCondHOrExpl}
  l^k_{[u]}(s)= \pi\big( u e^{s\Ad(k)E_1} \big)=
\begin{pmatrix}
\cos\mu&\sin\mu\\
-\sin\mu&\cos\mu\\
&&1\\
&&&1\\
&&&&1\\
&&&&&\ddots
\end{pmatrix}
 e^{s\Ad(k)E_1}
\begin{pmatrix}
1\\0\\0\\0\\0\\\vdots
\end{pmatrix}
=
\begin{pmatrix}
u_{k}(s)\\t_{k}(s)\\x_{k}(s)\\y_{k}(s)\\z_{k}(s)\\\vdots
\end{pmatrix}
\end{equation}
According to proposition~\ref{Proptcarrycarr}, this geodesic reaches the singularity in the future if $t_{k}(s)^{2}-y_{k}(s)^{2}=0$ for a certain positive $s$. Since $\Ad(k)E_1$ is nilpotent, the computation of $ e^{s\Ad(k)E_1}$ is simple and we only need the first column because it only acts on the first basis vector. A short computation shows that
\begin{equation}  \label{EqGedCompo}
  l_{[\mu]}^{k}(s)=
\begin{pmatrix}
\cos\mu-s\sin\mu\\
-\sin\mu-s\cos\mu\\
sw_{1}\\
sw_{2}\\
\vdots
\end{pmatrix}.
\end{equation}

We used the computation
\[
  e^{s\Ad(k)E_1}=\mtu+s
\begin{pmatrix}
0&1&w_1&w_2&w_3&\cdots\\
-1\\w_1\\w_2\\w_3\\\vdots
\end{pmatrix}
+\frac{s^2}{2}
\begin{pmatrix}
0&0&0&0&0&\cdots\\
0&-1&-w_1&-w_2&-w_3&\cdots\\
0&w_1&w_1w_1&w_1w_2&w_1w_3&\cdots\\
0&w_2&w_2w_1&w_2w_2&w_2w_3&\cdots\\
\vdots&\vdots&\vdots&\vdots&\vdots
\end{pmatrix}
+\cdots
\]
Notice that the sum if finite because $E_1$ is nilpotent. However, the first power of $E_1$ which vanishes depends on the dimension.

We conclude that the geodesic reaches $\hS_{AN}$ and $\hS_{A\bar{N}}$ for values $s_{AN}$ and $s_{A\bar{N}}$ of the affine parameter, given by
\begin{align}   \label{eq:tempssingul}
 s_{AN}&= \frac{\sin\mu}{\cos\mu - w_2}&s_{A\bar{N}}&= \frac{\sin\mu}{\cos\mu + w_2}
\end{align}
where $w_{2}$ is the second component of the first column of $k$, see equation \eqref{eq:AdkE}; in particular $-1\leq w_2 \leq 1$.

Since the part $\sin \mu =0$ is precisely  $\hS_{AN}$, we may restrict ourselves to the open connected domain of $AdS_l$ given by $\sin \mu > 0$. More precisely, $\sin\mu=0$ is the equation of $\hS_{AN}$ in the $ANK$ decomposition. In the same way, $\hS_{A\bar{N}}$ is given by $\sin\mu'=0$ in the $A\bar{N}K$ decomposition.  In order to escape the singularity, the point $[\mu]$ needs both $s_{AN}$ and $s_{A\bar{N}}$ to be strictly positive.  It is only possible to find directions (i.e. a parameter $w_2$) which respects this condition when $\cos \mu>0$.  So the point
\begin{equation}  \label{EqUnPtHoriz}
u\equiv \cos\mu=0
\end{equation}
is one point of the horizon. Theorem~\ref{ThoLeBut} is now proved. Remark that the two-dimensional case here appears as degenerate. Therefore, it is treated later in section~\ref{SecAdS2}, where we show that \emph{no black hole arises from this construction in $AdS_2$}.

The following proposition contains some physical intuition about the nature of the horizon.

\begin{proposition}
A light-like geodesic which escapes the singularity (i.e. which does not intersect $\hS$) and which passes trough a point of the horizon is contained in the horizon.
\end{proposition}

\begin{proof}
Let $x=[g]$ be a point of the horizon and $\pi(ge^{tAd(k)E_1})$, a light-like geodesic escaping the singularity. Near from $x$, there exists a point $y=[g']$ in the black hole. From definition of a black hole, for all $k\in \SO(3)$ and $t_{0}\in\eR^{+}$, points of the form  $\pi(g'e^{t_0Ad(k)E_1})$ also belong to the black hole. From continuity, in each neighbourhood of $\pi(ge^{t_0Ad(k)E_1})$, there is such a $\pi(g'e^{t_0Ad(k)E_1})$. This proves that $\pi(ge^{t_0Ad(k)E_1})$ belongs to the closure of the black hole. But it is not in the interior of the black hole because (by assumption) the given geodesic escapes the singularity, so every point of the form $\pi\big( g e^{t_0\Ad(k)E_1} \big)$ belongs to the horizon.
\end{proof}

\begin{proposition}		\label{PropTNFerme}
The set $BH_l\setminus\hS_l$ is open.
\end{proposition}

\begin{proof}
A point $v\in AdS_l$ belongs to $BH_l\setminus\hS_l$ if and only if all  the solutions in $s$ of the equation
\begin{equation}
	(T\pm Y)\big( v e^{s\Ad(k)E_1} \big)\in\hS_l
\end{equation}
are strictly positive (and non infinite). The \emph{strict} is due to the fact that we excluded $\hS_l$ itself. Let $s_{\pm(v,k)}$ be these solutions for the point $v\in AdS_l$ and the direction $k\in S^l$. Let now consider $v_0\in BH_l\setminus\hS_l$. The function $s_{\pm}(v_0,.)\colon S^l\to \eR$ is a continuous function on the compact set $S^l$, thus its image is a compact subset of $\eR_0^+$, because the function reach its extrema.

The function $v\mapsto s_{\pm}(v,k)$ is also continuous, so that, if $\epsilon$ is small enough, and if $v\in B(v_0,\epsilon)$, the image of $s_{\pm}(v,.)$ is still a compact subset of $\eR_0^+$. That means that, from the point $v$, every light-like geodesic intersect the singularity within a finite strictly positive time, this is the fact that $v\in BH_l\setminus\hS_l$.
\end{proof}

\begin{corollary}		\label{CorTNFermeHorEchape}
The set of free points in $AdS_l$ is closed and the points on the horizon do have at least one direction which escape the singularity.
\end{corollary}

Let us consider the point of the horizon that we know (the one given by \eqref{EqUnPtHoriz}), and see how can that point hope to escape the singularity.  Equations \eqref{eq:tempssingul} which give the time needed to fall into the singularity become
\begin{align}
  t_{AN}&=\frac{1}{w_{2}}&t_{A \bar{N}}&=-\frac{1}{w_{2}}.
\end{align}
So for every $w_{2}\neq 0$, this point reaches the singularity within a finite time. Taking the direction $w_{2}=0$ the point is able to reject his fall to infinity. This agrees to physical intuition which is that the horizon corresponds to points that fall into the singularity within an infinite time.

Up to a reparametrization of $\SO(n)$, the safe directions are given by (equation \eqref{eq:AdkE} with $w_2=0$)
\[
   \Ad(k)E_1=
\begin{pmatrix}
0&1&\cos a&0&\sin a\\
-1\\
\cos a\\
0\\
\sin a
\end{pmatrix}.
\]
A direct  computation of equation \eqref{EqGedCompo}  shows that the points of the horizon that are joined by this way are given by
$
\begin{pmatrix}
-1\\
0\\
\cos a\\
0\\
\sin a
\end{pmatrix}.
$

% This is part of (almost) Everything I know in mathematics and physics
% Copyright (c) 2013-2014,2018, 2020
%   Laurent Claessens
% See the file fdl-1.3.txt for copying conditions.

\section{Characterization by angles in \texorpdfstring{$SO(l-1)$}{SOl-1}}
%++++++++++++++++++++++++++++++++++++++++++++++++++++++++++++++++++++++++++

Let $D[g]$ be the set of light-like directions (vectors in $\SO(n)$) for which the point $[g]$ falls into $\hS_{AN}$. Similarly, the set $\overline{D}[g]$ is the one of directions which fall into $\hS_{A \bar{N}}$. One can express $\overline{ D }$ in terms of $D$:
\[
\begin{split}
\overline{ D }[g]&=\{ k\in\SO(n)\tq\exists t\text{ for which }\pi\big( g e^{t\Ad(k)E_1} \big)\in\hS_{A\bar N} \}\\
		&=\{ k\in\SO(n)\tq\exists t\text{ for which }\pi\big( \theta(g)\theta( e^{tAd(k)E_1}) \big)\in\hS_{AN}\}\\
		&=\{ k\in\SO(n)\tq \pi(k)\in D\big( \theta[g] \big)\}\\
		&=\{ k\in\SO(n)\tq k\in\big( D(\theta[g]) \big)_{\theta}\},
\end{split}
\]
So
\begin{equation} \label{eq:DbarD}
\overline{D}[g]=(D\theta[g])_{\theta}
\end{equation}
where by definition, $k_{\theta}=Jk$ with $J$ being defined by $\theta=\Ad(J)$ ($\theta$ is the Cartan involution). It is easy to see that $\theta$ changes the sign of the spacial part of $k$, i.e. changes $w_i\to -w_i$.

\begin{probleme}
C'est la même chose qu'un autre problème que de voir l'involution de Cartan comme un automorphisme interne.
\label{propCrtadeux}
\end{probleme}

 A main property of $k_{\theta}$ is
\[
	\theta(\Ad(k)E_1)=\Ad(k_{\theta})E_1.
\]
Since $k_{\theta}$ only appears in the expression $\Ad(k)E_1$, that property is actually a sufficient characterization of $k_{\theta}$ for our purpose. In particular, $k_{\theta\theta}\neq k$, but $\Ad(k_{\theta\theta})E_1=\Ad(k)E_1$.

How to express the condition $g\in\hH$ in terms of $D[g]$? The condition to belong to the black hole is $D[g]\cup \overline{D}[g]=\SO(n)$. If the complementary of $D[g]\cup \overline{D}[g]$ has an interior (i.e. if it contains an open subset), then by continuity the complementary $D[g']\cup \overline{D}[g']$ has also an interior for all $[g']$ near from $[g]$. In this case, $[g]$ cannot belong to the horizon. So a characterization of $\hH$ is the fact that the boundary of $D[g]$ and $\overline{D}[g]$ coincide. Equation \eqref{eq:DbarD} expresses this condition under the form
\begin{equation}
  \Fr D[g]=\Fr \big( D(\theta[g])\big)_{\theta},
 \end{equation}
from which one immediately deduces that $\hH$ is $\theta$-invariant.

We have an expression of $D[\mu]$ for $\mu\in \SO(2)$ by examining equations \eqref{eq:tempssingul}. The set $D[\mu]$ is the set of $w_2\in [-1,1]$ such that $\cos \mu+w_2>0$:
\begin{equation}
  D[\mu]=]-\cos \mu,1[.
\end{equation}
So in order for $\mu$ to belong to $\hH$, the point $[\mu]$ must satisfy
\[
\overline{D}[\mu]=D[\theta \mu]_{\theta}=]-1,-\cos \mu[.
\]
Consequently, if $\mu'$ is the $K$-component of $\theta \mu$ in the $ANK$ decomposition, we impose $]-\cos \mu',1[=D[\theta \mu]\stackrel{!}{=}]-\cos \mu',1[$\,, and we can describe the horizon by
\begin{equation} \label{eq:caractcous}
\cos \mu=-\cos \mu'
\end{equation}
where $\mu'$ is the $K$-component of $\mu$ in the $A\bar{N}K$ decomposition.


\subsection{Another (useless) characterisation}
%----------------------------------------------

A way to express our characterization \eqref{eq:caractcous} is $ank=a'\overline{n}k'$ with $k'=e^{i\pi}k^{-1}$. We know\quext{Mais faudra lire Helgason hein.} that $NA\bar{N}$ is dense in $G$. Let $k_0\in \SO(2)$ and $m=k_0^2e^{i\pi}$. We define $n,n'\in N$, $a\in A$ such that $m=n^{-1} a\theta(n')$.\quext{Il faudra voir si le coup de la densit\'e fait quelque chose dans cette histoire}. For this $n$, the point $[k_0n]$ belongs to the horizon because
\begin{equation}
nk_0=a\theta(n')m^{-1} k_0
    =a\theta(n')e^{-i\pi}k_0^{-1}.
\end{equation}
Then this $nk$ reads in decomposition $A\bar{N}K$ with $k'=e^{-i\pi}k^{-1}$. Then (almost) all element in $\SO(2)$ give rise to an element in $\hH$.


%+++++++++++++++++++++++++++++++++++++++++++++++++++++++++++++++++++++++++++++++++++++++++++++++++++++++++++++++++++++++++++
					\section{Characterisation as orbit of group (by the equation)}
%+++++++++++++++++++++++++++++++++++++++++++++++++++++++++++++++++++++++++++++++++++++++++++++++++++++++++++++++++++++++++++
\label{SecHOrOrbEquation}

This section proves that, if we embed $AdS_3$ in $AdS_4$, one can express the horizon in $AdS_4$ as the result of the action of a one dimensional group on the horizon of $AdS_3$ (seen in $AdS_4$), theorem~\ref{ThoEqHorQCoore}.

%---------------------------------------------------------------------------------------------------------------------------
					\subsection{The old three dimensional case}
%---------------------------------------------------------------------------------------------------------------------------

As mentioned in \cite{Keio}, the singularity of the three dimensional black hole in $AdS_3$ (seen as the group $\SL(2,\eR)$) accepts a nice description as lateral classes of $AN$ and $A\bar N$. That description is recalled in the proposition~\ref{PropLatClassANSLdeuxR}. We want here to provide a similar description for the dimensional generalization $AdS_l=\SO(2,l-1)/\SO(1,l-1)$.

Let us first make a simple remark. A lateral class in the description of proposition~\ref{PropLatClassANSLdeuxR} is not guaranteed to be a lateral class in the description $AdS=G/H$. Moreover the ``$AN$'' of equation  \eqref{EqHorClassLatdeux} is not the ``$AN$'' of $\SO(2,2)$, but the one of $\SL(2,\eR)$. The results from the description $AdS_3=\SL(2,\eR)$ cannot be that simply translated into results in the description of $AdS_3=\SO(2,2)/\SO(1,2)$.

Let us begin by finding a group description of the horizon in $AdS_3$ in the description $AdS_3=\SO(2,2)/SO(1,2)$. The matricial expression of $ANJ$ in $AdS_3=\SL(2,\eR)$ is
\begin{equation}		\label{EqProSLJANexp}
\begin{pmatrix}
	e^a	&	le^a	\\
	0	&	 e^{-a}
\end{pmatrix}
\begin{pmatrix}
	0	&	1	\\
	-1	&	0
\end{pmatrix}
=
\begin{pmatrix}
	-le^a	&	e^a	\\
	- e^{-a}	&	0
\end{pmatrix}
\end{equation}
The part of the hyperboloid described by these matrices is obtained by equating \eqref{EqProSLJANexp} with the matrix
\begin{equation}		\label{EqIdentMatriSLAdS}
	\begin{pmatrix}
	u+x	&	y+t	\\
	y-t	&	u-x
\end{pmatrix}.
\end{equation}
The result is the vectors of the form
\begin{equation}		\label{EqVectoPotementSingAN}
\psi\big( Z(G)ANJ \big)\leadsto
	\begin{pmatrix}
	u	\\
	t	\\
	x	\\
	y
\end{pmatrix}=
\pm
\begin{pmatrix}
	-\frac{ 1 }{2}e^al	\\
	\cosh(a)	\\
	-\frac{ 1 }{2}e^al	\\
	\sinh(a)
\end{pmatrix}
=
\pm
\begin{pmatrix}
	\alpha	\\
	\cosh(a)	\\
	\alpha	\\
	\sinh(a)
\end{pmatrix}
=\pm r_{AN}
\end{equation}
with $\alpha$, $a\in\eR$. This is a (almost\footnote{We did not compute the $A\bar N$ part of the horizon in $\SL(2,\eR)$.}) general vector of $AdS_3$ with $u^2-x^2=0$, which is coherent with the description \eqref{BTZSingHor}.

The same computation, using \eqref{EqGeneANbarSLdeuxR}, shows that the other part of the horizon in $AdS_3$ is given by
\begin{equation}		\label{EqVectoPotementSingANbar}
\psi\big( Z(G)A\bar NJ\big)
=
\pm\psi
\begin{pmatrix}
	0	&	e^a	\\
	- e^{-a}	&	l e^{-a}
\end{pmatrix}
\leadsto
\begin{pmatrix}
	u	\\
	t	\\
	x	\\
	y
\end{pmatrix}=
\pm
\begin{pmatrix}
	\frac{1}{ 2 } e^{-a}l	\\
	\cosh(a)	\\
	-\frac{1}{ 2 } e^{-a}l	\\
	\sinh(a)
\end{pmatrix}
=
\pm
\begin{pmatrix}
	\alpha	\\
	\cosh(a)	\\
	-\alpha	\\
	\sinh(a)
\end{pmatrix}
=\pm
r_{A\bar N}
\end{equation}
where $a$ and $\alpha$ are running over $\eR$.

From the equations \eqref{EqVectoPotementSingAN} and \eqref{EqVectoPotementSingANbar}, we are able to express the horizon in $AdS_3$ as union of lateral classes of the element
\begin{equation}
	b=\begin{pmatrix}
		0	\\
		1	\\
		0	\\
		0
	\end{pmatrix}
\end{equation}
because $G_{ X_{(-1,1)},J_1}\cdot b =G_{ X_{(1,1)},J_1}\cdot b$ and $G_{ J_1,X_{(1,-1)},J_1}\cdot b=G_{ J_1,X_{(-1,-1)} ,J_1}\cdot b$. We can express the horizon $\hH_3$ in the following way:
\begin{equation}
	\begin{aligned}[]
		\hH_3	&=\pm G_{ X_{(-1,1)},J_1}\cdot b\cup \pm G_{ X_{(1,-1)},J_1}\cdot b  \\
			&=\pm G_{ \{J_1,X_{(1,1)}\}}\cdot b\cup \pm G_{ \{J_1,X_{(-1,-1)}\}}\cdot b,
	\end{aligned}
\end{equation}
and the two other combinations. Here, $G_{X,Y}$ is the group generated by $X$ and $Y$.

%---------------------------------------------------------------------------------------------------------------------------
\subsection{Characterization by induction on the dimension}
%---------------------------------------------------------------------------------------------------------------------------

From a computational point of view, it reveals to be more or less impossible to directly check that \eqref{EqVectoPotementSingAN} belongs to the singularity using the method of equation \eqref{EqhohnCondHOrExpl}, not even in dimension $4$. Here is the strategy to compute the horizon in higher dimension:
\begin{enumerate}
\item
The map $\psi\colon \SL(2,\eR)\to AdS_3$ given by \eqref{EqIdentMatriSLAdS} is an isometry which maps the singularity into the singularity. Thus it has to map the horizon to the horizon. If $\hH_{\SL(2,\eR)}$ denotes the horizon in $\SL(2,\eR)$, then the set $\psi\big( \hH_{\SL(2,\eR)} \big)$ is the horizon in $AdS_3=\SO(2,2)/SO(2,1)$.

\item
We consider the inclusion $\iota\colon \SO(2,n)\to \SO(2,n+1)$ given by $g\mapsto\begin{pmatrix}
	g	&	0	\\
	0	&	1
\end{pmatrix}$ and its differential $d\iota\colon \so(2,n)\to \so(2,n+1)$, $X\mapsto\begin{pmatrix}
	X	&	0	\\
	0	&	0
\end{pmatrix}$. Now, we are going to build the horizons of $AdS_l$ by induction over $l$, starting on $l=3$.
\end{enumerate}

We denote by $\hH_l$ and $\hS_l$ the horizon and the singularity in $AdS_l$. The structure of the algebras (equations \eqref{EqLeANEnDimAlg} and \eqref{EqTableSOIwa}) show immediately that
\begin{equation}
	(\sA\oplus\sN)_{\so(2,n+1)}=\Span\left\{   d\iota(\sA\oplus\sN)_{\so(2,n)},V_{n+2},W_{n+2}  \right\},
\end{equation}
so that the structure of one dimension is defined from the structure of the previous one by adding the two new vectors $V$ and $W$. The same holds for $\sA\oplus\bar\sN$.


Now, the work is to find what is \emph{added} to the horizon when one passes from one dimension to the higher one. From that point of view, the matrix $V_i$ has a wonderful property: $ e^{V}$ does not change the $t$ and $y$ component of the vector on which it acts. Thus we have the following.
\begin{lemma}		\label{LemHorpigeVDdeux}
We have
\begin{equation}
	\pi(g e^{-s\Ad(k)E_1})\in\hS
\end{equation}
if and only if
\begin{equation}
	 \pi(e^{V}g e^{-s\Ad(k)E_1})\in\hS.
\end{equation}
The same holds replacing $V$ by $X$.
\end{lemma}

\begin{proof}
The exponential of the matrix $V_5$ is given in equation \eqref{EqExpDeV}. The second and fourth column being the identity, $e^V1_t=1_y$ and $e^V1_y=1_y$. Thus the characterisation $t^2-y^2=0$ of the singularity is satisfied for one point $x\in AdS$ if and only if it is satisfied by the point $e^Vx$.
\end{proof}


We consider the following points in the horizon:
\begin{equation}		\label{EqPartewWrAN}
	\begin{aligned}[]
	r(a,\alpha,w)&= e^{wW}r_{AN}=
\frac{ 1 }{2}
\begin{pmatrix}
	2\alpha	\\
	e^{-a}w^2+2\cosh(a)\\
	2\alpha	\\
	e^{-a}w^2+2\sinh(a)	\\
	2 e^{-a}w
\end{pmatrix},\\
	\bar r(a,\alpha,w)&= e^{wW}r_{A\bar N}=
\frac{ 1 }{2}
\begin{pmatrix}
	2\alpha	\\
	 e^{-a}w^2+2\cosh(a) \\
	-2\alpha	\\
	e^{-a}w^2+2\sinh(a)	\\
	 2e^{-a}w
\end{pmatrix}.
	\end{aligned}
\end{equation}

The tangent vectors of that surface are given by
\begin{equation}
	\begin{aligned}[]
		(\partial_ar)(a,\alpha,w)&=
\begin{pmatrix}
	0	\\
	\frac{ - e^{-a}w^2+2\sinh(a) }{2}	\\
	0	\\
	\frac{ - e^{-a}w^2+2\cosh(a) }{2}	\\
	- e^{-a}w
\end{pmatrix}
,&
		(\partial_{\alpha}r)(a,\alpha,w)&=
\begin{pmatrix}
	1	\\
	0\\
	1	\\
	0\\
	0
\end{pmatrix}
,&
		(\partial_wr)(a,\alpha,w)&=
\begin{pmatrix}
	0	\\
	 e^{-a}w\\
	0	\\
	 e^{-a}w\\
	e^{-a}
\end{pmatrix}
	\end{aligned}.
\end{equation}
Notice that these three vectors are nowhere vanishing. It is immediate that the vector $\partial_{\alpha}r$ is linearly independent of $\partial_{a}r$ and of $\partial_wr$. It is also immediately apparent that $\partial_ar=-w\partial_wr$ is the worse possible situation. It is, however, not possible because it would imply that
\begin{equation}
	\begin{aligned}[]
		-w^2 e^{-a}&=\frac{ - e^{-a}w^2+2\sinh(a) }{2}&\text{and}&&-w^2 e^{-a}&=\frac{ - e^{-a}w^2+2\cosh(a) }{2},
	\end{aligned}
\end{equation}
which is only possible when $\cosh(a)=\sinh(a)$, in other words: never. Thus, the part of $AdS_4$ described by \eqref{EqPartewWrAN} has dimension $3$.


\begin{proposition}
We have
\begin{equation}
	 G_W\cdot\iota(\hH_3)=
	\{ r(a,\alpha,w)\cup\bar r(a,\alpha,w) \}_{a,\alpha,w\in\eR}.
\end{equation}
\end{proposition}

\begin{proof}
The facts that $G_W\cdot\iota(\hH_3)=\{ r(a,\alpha,w)\cup\bar r(a,\alpha,w) \}_{a,\alpha,w\in\eR}$ and that all the elements of that set are subject to $u^2-x^2=0$ are by construction.

We still have to prove that $\{ u^2-x^2=0\}\subseteq G_W\cdot\iota(\hH_3)$.

Let $v=(y,t,x,y,z)$ be a vector which satisfies $u^2-x^2=0$. Following the signs of $u$ and $t$, we are searching $v$ under the form $\pm r(\alpha,a,w)$ or $\pm \bar r(\alpha,a,w)$. In any case, the value of $u$ and $x$ fix $\alpha$ and we are left with the condition
\begin{equation}
\pm\frac{ 1 }{2}
	\begin{pmatrix}
	e^{-a}w^2+2\cosh(a)	\\
	e^{-a}w^2+2\sinh(a)	\\
	2 e^{-a}w
\end{pmatrix}
=
\begin{pmatrix}
	t	\\
	y	\\
	z
\end{pmatrix}
\end{equation}
with $t^2-y^2-z^2=1$.  If $t-y>0$, we choose the sign $+$ and the value of $t-y$ fixes $a$ because $t-y= e^{-a}$. In that situation, $w$ is given by $w= e^{a}\big(2y-2\sinh(a)\big)$. If $t-y<0$, we have $t-y=- e^{-a}$ and the same argument holds.

\end{proof}

In the sequel, we will use the following notations:
\begin{equation}
	\begin{aligned}[]
		G_W&=\{  e^{wW}\tq w\in\eR \}\\
		G_V&=\{  e^{\alpha V}\tq \alpha\in\eR \}\\
		G_X&=\{  e^{\beta X}\tq \beta\in\eR \}\\
		G_Y&=\{  e^{y Y}\tq y\in\eR \}
	\end{aligned}
\end{equation}
These are one parameter subgroups of $SO(2,3)$.

\begin{proposition}		\label{PropInclusionsTroisQuatreWVXY}
If $v\in AdS_4$ satisfies $u-t\neq 0$, then $v= e^{wW}v'$ for a certain $v'\in AdS_3$. In other words,
\begin{equation}
	\{ y-t\neq 0 \}_4\subset G_W\cdot\iota(AdS_3).
\end{equation}
In particular, every points outside the singularity $\hS_4$ are obtained by action of $G_W$ on a point of $AdS_3$. We also have
\begin{equation}
	\begin{aligned}[]
		\{ x-u\neq 0 \}_4&\subseteq G_V\cdot\iota(AdS_3)\\
		\{ x+u\neq 0 \}_4&\subseteq G_X\cdot\iota(AdS_3).
	\end{aligned}
\end{equation}
\end{proposition}

\begin{proof}
We have
\begin{equation}
	 e^{wW}
\begin{pmatrix}
	u'	\\
	t'	\\
	x'	\\
	y'	\\
	0
\end{pmatrix}=
\begin{pmatrix}
	u'	\\
	\left( 1+\frac{ w^2 }{2} \right)t'-\frac{ w^2 }{2}y'	\\
	x'	\\
	\frac{ w^2 }{2}t'+\left( 1-\frac{ w^2 }{2} \right)y'	\\
	w(t'-y')
\end{pmatrix}
=
\begin{pmatrix}
	u	\\
	t	\\
	x	\\
	y	\\
	z
\end{pmatrix}
\end{equation}
when
\begin{equation}
	\begin{aligned}[]
		u'&=u,& t'&=\frac{ z^2+2ty-2t^2 }{ 2(y-t) },&x'&=x,&y'&=\frac{ z^2-2ty+2y^2 }{ 2(y-t) },&w&=-\frac{ z }{ y-t }.
	\end{aligned}
\end{equation}
In the same way, the equation
\begin{equation}
	 e^{\alpha V}\begin{pmatrix}
	u'	\\
	t'	\\
	x'	\\
	y'	\\
	0
\end{pmatrix}=
	 \begin{pmatrix}
	u	\\
	t\\
	x	\\
	y\\
	z
\end{pmatrix}
\end{equation}
is solved by
\begin{equation}
	\begin{aligned}[]
		u'&=\frac{ z^2+2ux-2u^2 }{ 2(x-u) },&x'&=\frac{ z^2-2ux+2x^2 }{ 2(x-u) },&\alpha&=-\frac{ z }{ x-u }.
	\end{aligned}
\end{equation}
Thus, $\{ x-u\neq 0 \}_4\subseteq G_V\cdot\iota(AdS_3)$. And, finally, the equation
\begin{equation}
	 e^{\beta X}\begin{pmatrix}
	u'	\\
	t'	\\
	x'	\\
	y'	\\
	0
\end{pmatrix}=
	 \begin{pmatrix}
	u	\\
	t	\\
	x	\\
	y	\\
	z
\end{pmatrix}
\end{equation}
is solved by
\begin{equation}
	\begin{aligned}[]
		u'&=\frac{ z^2-2ux-2u^2 }{ 2(x+u) },&x'&=\frac{ z^2+2ux+2x^2 }{ 2(x+u) },&\beta&=-\frac{ z }{ x+u }.
	\end{aligned}
\end{equation}
Thus, $\{ x+u\neq 0 \}_4\subseteq G_X\cdot\iota(AdS_3)$.

\end{proof}

One interest of that proposition resides in the fact that every element of $AdS_4$ outside the singularity is the image of an element of $AdS_3$ by $G_W$.


\begin{proposition}		\label{PropSingQTiV}
We have
\begin{equation}
	\hS_4=G_V\cdot\iota(\hS_3)
\end{equation}
where $G_V=\{  e^{vV}\tq v\in\eR \}$ is the group generated by $V$.
\end{proposition}

\begin{proof}
A point of $\iota(\hS_3)$ is of the form
$	\begin{pmatrix}
	u	\\
	\alpha	\\
	x	\\
	\epsilon\alpha	\\
	0
\end{pmatrix}
$, while an element of $\hS_4$ is of the form
$
	\begin{pmatrix}
	u'	\\
	\alpha	\\
	x'	\\
	\epsilon\alpha	\\
	z'
\end{pmatrix}
$ where $\epsilon=\pm 1$. So we have to solve the equation
\begin{equation}
	 e^{vV}
\begin{pmatrix}
	u	\\
	\alpha	\\
	x	\\
	\epsilon\alpha	\\
	0
\end{pmatrix}=
\begin{pmatrix}
	u\left( \frac{ v^2 }{ 2 }+1 \right)+\frac{ v^2 }{2}x	\\
	\alpha	\\
	\left( 1-\frac{ v^2 }{2} \right)x+\frac{ v^2 }{2}u	\\
	\epsilon\alpha	\\
	v(u-x)
\end{pmatrix}
\stackrel{!}{=}
\begin{pmatrix}
	u'	\\
	\alpha	\\
	x'	\\
	\epsilon\alpha	\\
	z'
\end{pmatrix}
\end{equation}
with respect to $v$, $u$ and $x$. A solution is given by
\begin{equation}
	\begin{aligned}[]
		u&=\frac{ z'^2+2u'x'-2u'^2 }{ 2(x'-u') },&x&=\frac{ z'^2-2u'x'+2x'^2 }{ 2(x'-u') },&v&=-\frac{ z' }{ x'-u' }.
	\end{aligned}
\end{equation}
The condition $u'^2-x'^2-z'^2=1$ imposes $x'\neq u'$, so that that solution always makes sense: a point of $\hS_4$ is always obtained as the result of the action of an element of $G_V$ on an element of~$\hS_3$.

Since the operator $ e^{vV}$ does not touch the variables $t$ and $y$, it is obvious that $G_V\cdot \hS_3\subseteq\hS_4$.
\end{proof}

\begin{lemma}		\label{LemTNTroisIneq}
In $AdS_3$, the black hole is given by $u^2-x^2>0$
\end{lemma}

\begin{proof}
The black hole is the set of point from which every light ray intersect the singularity. The boundary of that set is given by the horizon (this is the definition of the horizon), and we already proved that $\hH_3\equiv u^2-x^2=0$. Thus the black hole is $u^2-x^2>0$, or $u^2-x^2<0$. Since the singularity (which is part of the black hole) is given by $t^2-y^2=0$, the singularity satisfies $u^2-x^2=1$, and is thus in the part $u^2-x^2>0$.
\end{proof}


Let $TN[g]$ be the subset of $\{ \Ad(k)E_1 \}_{k\in \SO(3)}$ of elements for which there exists a $s>0$ such that
\begin{equation}
	\pi(g e^{s\Ad(k)E_1})\in\hS.
\end{equation}
In other words, $TN[g]$ is the set of directions along which $[g]$ falls in the singularity. If the complementary $TN[g]^c$ has a non empty interior, the by continuity, the complementary $TN[g']$ will have an interior as well for every $[g']$ close enough from $[g]$. In that case, $[g]$ does not belongs to the horizon. So a point belongs to the horizon when the set of safe direction has no interior.


\begin{lemma}
We have
\begin{equation}
	G_V\cdot\iota(\hH_3)\equiv u^2-x^2-z^2=0,
\end{equation}
so that it is the good candidate to be the horizon.
\end{lemma}

\begin{proof}
An element of $\iota(\hH_3)$ has the form
$r=\begin{pmatrix}
	u'	\\
	t'	\\
	x'	\\
	\pm\sqrt{t'^2-1}	\\
	0
\end{pmatrix}$,
so that we have to solve the equation
\begin{equation}
	 e^{vV}r=\begin{pmatrix}
	\left( \frac{ v^2 }{ 2 }+1 \right)u'-\frac{ v^2 }{ 2 }x'	\\
	t'	\\
	\left( 1-\frac{ v^2 }{ 2 } \right)x'+\frac{ v^2 }{ 2 }u'	\\
	\pm\sqrt{t'^2-1}	\\
	v(u'-x')
\end{pmatrix}
=
\begin{pmatrix}
	u	\\
	t	\\
	x	\\
	\pm\sqrt{t^2-1}	\\
	z
\end{pmatrix}.
\end{equation}
The solution is
\begin{equation}
	\begin{aligned}[]
		u'&=\frac{ z^2+2ux-2u^2 }{ 2(x-u) },&x'&=\frac{ z^2-2ux+2x^2 }{ 2(x-u) },&v&=-\frac{ z }{ x-u }.
	\end{aligned}
\end{equation}
Since $u^2-x^2-z^2=1$, we have $x-u\neq 0$, so that these solutions always make sense.
\end{proof}

\begin{lemma}
If $[g]=\begin{pmatrix}
	u	\\
	t	\\
	x	\\
	y	\\
	z
\end{pmatrix}\in AdS_4$ with $u$ and $x$ not both vanishing, then
\begin{equation}
	[g]\in G_V\cdot\iota(AdS_3)\cup G_X\cdot \iota(AdS_3).
\end{equation}
Notice that the union is not disjoint.
\end{lemma}

\begin{proof}
The proof is a simple computation. Following proposition~\ref{PropInclusionsTroisQuatreWVXY}, we have $\{ x-u\neq 0 \}_4\subseteq G_V\cdot\iota(AdS_3)$ and $\{ x+u\neq 0 \}_4\subseteq G_X\cdot\iota(AdS_3)$.

So the only part of $AdS_4$ which is not included in $G_V\cdot \iota(AdS_3)\cup G_X\cdot\iota(AdS_3)$ is the part where $x+u=x-u=0$.
\end{proof}

Now, we want to study the horizon, that means the boundary of $BH_4$. If $v\in\partial\big(\Adh(BH_4) \big)$, there exists, in any neighbourhood of $v$, an element $\bar v$ and a direction following which the geodesic from $\bar v$ escapes the singularity.

Up to now, we studied the way $AdS_3$ embed in $AdS_4$. In particular, we proved that the horizon of $AdS_3$ is included in the horizon of $AdS_4$. We can propagate the results by $G_V$ and $G_X$ because, given a $v\in AdS_3$, the existence of a $\alpha$ such that $ e^{\alpha V}v\in\iota(AdS_3)$ or $ e^{\alpha X}v\in\iota(AdS_3)$ is related to the fact that $u^2-x^2\neq 0$, while that condition holds in a neighbourhood of $v$.

%klklklmkmlkklmmlkkmlkmùlmklkll
%+++++++++++++++++++++++++++++++++++++++++++++++++++++++++++++++++++++++++++++++++++++++++++++++++++++++++++++++++++++++++++
\section{Organization of the next few pages}
%+++++++++++++++++++++++++++++++++++++++++++++++++++++++++++++++++++++++++++++++++++++++++++++++++++++++++++++++++++++++++++

\begin{abstract}
	This paper is a sequel of \emph{Solvable symmetric black hole in anti de Sitter spaces} \cite{lcTNAdS}. In the latter, we described the BTZ black hole in every dimension by defining the singularity as the closed orbits of the Iwasawa subgroup of $\SO(2,n)$. In this article, we study the horizon of the black hole and we show that it is expressed as lateral classes of one point of the space. The computation is given in the four-dimensional case, but it makes no doubt that it can be generalized to any dimension.

	The main idea is to define an ``inclusion map'' from $AdS_3$ into $AdS_4$ and to show that all the relevant structure pass trough the inclusion. We prove, for example, that the inclusion of the three dimensional horizon into $AdS_4$ belongs to the four dimensional horizon: $\iota(\hH_3)\subseteq\hH_4$ and then we deduce the expression of the horizon in $AdS_4$.
\end{abstract}

In section~\ref{SecOldResults}, we describe some old results about BTZ black hole.

In subsection~\ref{SubSecHorInThreeDimensionOld}, we recall how we proved the existence of the black hole structure in \cite{lcTNAdS} and how the horizon was described in the three dimensional case in \cite{Keio}. We adapt the latter result in our homogeneous space setting.

The subsection~\ref{subSecTopoHor} gives some topological remarks about the black hole and the horizon. We point out that there are some light-like geodesics that are intersecting the singularity \emph{and then} the free part later in the future. We explain why that circumstance is very different from the situation of the most famous black holes in physics like the Schwarzschild's one.

Section~\ref{SecNewWithMatrices} is devoted to the proof of our main result: the horizon of the BTZ black hole in $AdS_4$ is given by
\begin{equation}
	\hH_4=G_{X_{0+}}\cdot \iota(\hH_3)\cup G_{X_{0-}}\iota(\hH_3).
\end{equation}
where $\iota$ is the inclusion of $AdS_3$ in $AdS_4$ and $\hH_3$ is the horizon of the BTZ black hole in $AdS_3$.

%+++++++++++++++++++++++++++++++++++++++++++++++++++++++++++++++++++++++++++++++++++++++++++++++++++++++++++++++++++++++++++
\section{Some old results}
%+++++++++++++++++++++++++++++++++++++++++++++++++++++++++++++++++++++++++++++++++++++++++++++++++++++++++++++++++++++++++++
\label{SecOldResults}

From the results of section~\ref{SecExistenceHor}, we know that a non trivial horizon exists. However, the question of the structure of the horizon was not yet addressed. This is what we are going to do now.

%---------------------------------------------------------------------------------------------------------------------------
\subsection{Horizon in the three dimensional case}
%---------------------------------------------------------------------------------------------------------------------------
\label{SubSecHorInThreeDimensionOld}

The structure of the horizon of $AdS_3$ was described in \cite{Keio} in the setting of $AdS_3=\SL(2,\eR)$. Our first job is to translate that result into the language of quotient of groups. This is done by the identification
\begin{equation}
	\begin{aligned}
		\psi\colon \SL(2,\eR)&\to AdS_3 \\
		\begin{pmatrix}
			u+x	&	y+t	\\
			y-t	&	u-x
		\end{pmatrix}&\mapsto \begin{pmatrix}
			u	\\
			t	\\
			x	\\
			y
		\end{pmatrix}.
	\end{aligned}
\end{equation}
We see that the points of the horizon are given by
\begin{equation}			\label{EqHOrAdSTroisVecteur}
	\begin{aligned}[]
		\pm\begin{pmatrix}
			\alpha	\\
			\cosh(a)	\\
			\alpha	\\
			\sinh(a)
		\end{pmatrix}&&\text{and}&&\pm\begin{pmatrix}
			\alpha	\\
			\cosh(a)	\\
			-\alpha	\\
			\sinh(a)
		\end{pmatrix},
	\end{aligned}
\end{equation}
which correspond to the points $(u,t,x,y)$ such that $u^2-x^2=0$. One should notice that these points can be expressed as lateral classes of the point $b=(0,1,0,0)$~:
\begin{equation}
	\hH_3=\pm G_{\{ J_1,X_{++} \}}b\cup\pm G_{\{ J_1,X_{--} \}}b
\end{equation}
where $G_{\{ X,Y \}}$ is the group of elements of the form $\exp(aX+bY)$. Notice that $G_{\{ J_1,X_{++} \}}b=G_{\{ J_1,X_{-+} \}}b$ and $G_{\{ J_1,X_{--} \}}b=G_{\{ J_1,X_{+-} \}}b$. For example,
\begin{equation}
	e^{aJ_2} e^{\alpha X_{++}}b=\begin{pmatrix}
		\alpha	\\
		\cosh(a)	\\
		\alpha	\\
		\sinh(a)
	\end{pmatrix}.
\end{equation}
We are now intended to extend that result and express the horizon in $AdS_4$ as lateral classes of the horizon in $AdS_3$. Before to complete that work, we have to make a few remarks about the topology.

%---------------------------------------------------------------------------------------------------------------------------
\subsection{Topology and horizon}
%---------------------------------------------------------------------------------------------------------------------------
\label{subSecTopoHor}

The definition given in the previous sections produces a paradox. Let $x\in AdS$ and $l(s)$ be a light like geodesic trough $x$ which only intersects the singularity in past. We suppose that $l(0)=x$ and that $s_0<0$ is the biggest value of $s$ such that $l(s_0)\in \hS$. Thus, all points of the form $l(s)$ with $s_0<s<0$ are free. That form a sequence of free points which converges to the singularity, and then $l(s_0)$ belongs to the horizon.

This is however not possible in $AdS_3$ because the equation of the singularity is $t^2-y^2=0$ while the equation of the horizon is $u^2-x^2=0$. These two parts are really separated.

The situation here is really different from the situation in the Schwarzschild's case. In the latter the singularity is well inside the horizon, and there are no geodesics reaching the infinity which have intersected the singularity in the past.

In our case, however, such geodesics do exist. The reason of such a difference resides in the fact that the causal structure (geodesics) are defined by the metric while, in our BTZ black hole, the singularity is not defined from metric considerations. There are thus no reasons to expect some compatibility relations like the fact to have a non naked singularity.

In order to correctly define the horizon, we have to introduce the space $BTZ=AdS\setminus\hS$ which in endowed with the induced topology. Then we define
\begin{equation}
	BH=\{ v\in BTZ\tq\forall k\in \SO(n),\, l_v^k(s)\in\hS\text{ has a solution with }s>0 \}.
\end{equation}
Let us point out that the singularity itself is not part of the black hole, because it is not even part of $BTZ$. We define the free part of $BTZ$ as the set of points from which there exists a light-like geodesics which does not intersects the singularity in the future:
\begin{equation}
	\hF=\{ v\in BTZ\tq\exists k\in \SO(n),\, l_v^k(s)\in\hS\Rightarrow s<0 \}.
\end{equation}
The first definition makes that the black hole part is open by continuity and compactness of $\SO(n)$: the minimum and the maximum of time to reach the singularity from one point of the black hole are both strictly positive numbers, and then can be maintained strictly positive in a neighborhood of the point.

\begin{proposition}		\label{PropBHouvertLibreFerme}
	The set of points in the black hole is open and set of free points is closed. In particular, the horizon is contained in the free set.
\end{proposition}

\begin{proof}
	The first point is the remark above. Now, the free part is closed in $BTZ$ as complementary of an open set.
\end{proof}

The following theorem says that if the set of directions escaping the singularity from a point in $BTZ$ has an interior, then that point does not lies in the horizon.
\begin{proposition}		\label{PropvFOsvghorvec}
	A point $v\in\hF_l$ such that there is an open set $\mO\subset S^{l-1}$ of directions for which $l^{w}_v(s)\in\hS$ has no solutions for $s\in\eR^+_0$ belongs to $\Int(\hF)$.
\end{proposition}

\begin{proof}
	Using the matricial representation \eqref{eq:AdkE}, we see that a point $v=[g]$ belongs to the singularity if the vector
	\begin{equation}
		g\cdot \begin{pmatrix}
			1	\\
			-s	\\
			s\bar w
		\end{pmatrix}
	\end{equation}
	satisfies $t^2-y^2=0$. That equation is a second order polynomial in $s$ whose coefficients cannot be a constant for an open set with respect to $\bar w\in S^{l-1}$. From the assumptions, all the roots of that polynomial belong to $\eC\setminus\eR^+_0$. The latter being open, the roots of $l_{v'}^w(s)\in\hS$ are still in $\eC\setminus\eR^+_0$ when $v'$ runs over a small enough open set around $v$.

	We conclude that $v$ is in the interior of the free zone rather than on the horizon.
\end{proof}

An important characterisation of the horizon, pointed out in \cite{Keio}, is the following.
\begin{theorem}		\label{ThoHorIntDansS}
	A point belongs to the horizon if and only if the set of light-like directions for which the geodesics does not intersects the singularity has no interior in $S^{l-1}$.
\end{theorem}


%+++++++++++++++++++++++++++++++++++++++++++++++++++++++++++++++++++++++++++++++++++++++++++++++++++++++++++++++++++++++++++
\section{The horizon of the BTZ black hole}
%+++++++++++++++++++++++++++++++++++++++++++++++++++++++++++++++++++++++++++++++++++++++++++++++++++++++++++++++++++++++++++
\label{SecNewWithMatrices}

In this section, we show, that the horizon of the horizon of $AdS_4$ can be obtained using the action of a very simple group on the horizon of $AdS_3$, which is, itself, the orbit of one point under a known group. The result opens the possibility of describing the horizon in $AdS_l$ by induction on the dimension, and the possibility to compute the group which generates the horizon.  We define the inclusion map
\begin{equation}
	\begin{aligned}
		\iota\colon AdS_3&\to AdS_4 \\
		\begin{pmatrix}
			u	\\
			t	\\
			x	\\
			y
		\end{pmatrix}&\mapsto \begin{pmatrix}
			u	\\
			t	\\
			x	\\
			y	\\
			0
		\end{pmatrix}.
	\end{aligned}
\end{equation}
At the matrix level, it corresponds to add a line and a column of zeros. We will denote by $\hF_l$ the free part of $AdS_l$. By definition, if $v\in\hF_l$, there exists a light like geodesic trough $v$ which does not intersect the singularity in the future. We also denote by $BH_l$ the set of elements of $AdS_l$ from which all the light-like geodesics intersect the singularity in the future.

Notice that $BH_l$ is open while $\hF_l$ is closed, as explained in proposition~\ref{PropBHouvertLibreFerme}.

\begin{lemma}		\label{LemOouversttq}
	Let $v\in AdS_4$ and $g\in \SO(2,3)$ be a representative of $v$. If the set
	\begin{equation}
		\{ \begin{pmatrix}
			w_1	\\
			w_2
		\end{pmatrix}\in S^2\tq
		\pi g\begin{pmatrix}
			1	\\
			-s	\\
			s\bar w	\\
			0
		\end{pmatrix}\cap\hS_4=\emptyset\text{ with }s>0
				\}
	\end{equation}
	has an interior in $S^1$, then the set
	\begin{equation}
		\{
		\begin{pmatrix}
			w_1	\\
			w_2	\\
			w_3
		\end{pmatrix}\in S^2\tq
		\pi g\begin{pmatrix}
			1	\\
			-s	\\
			s\bar w
		\end{pmatrix}\cap\hS_4=\emptyset\text{ with }s>0
		\}
	\end{equation}
	has an interior in $S^2$.
\end{lemma}

\begin{proof}
The matrix $g$ in $\SO(2,3)$ representing the point $v$ has the form
\begin{equation}
	g=\begin{pmatrix}
 u	&	.	&	.	&	.	&	.\\
 t	&	a	&	b	&	c	&	d\\
 x	&	.	&	.	&	.	&	.\\
 y	&	a'	&	b'	&	c'	&	d'\\
z	&	.	&	.	&	.	&	.
 \end{pmatrix}
\end{equation}
where the numbers $a,b,c,d,a',b',c',d'$ are not uniquely determined. We choose the representative in such a way to have $b\neq \pm b'$, which is always possible.

The assumption is that there exists an open set (with respect to $(w_1,w_2)\in S^1$) around $(w_1,w_2,0)$ such that the path
\begin{equation}		\label{EqPathgexpUTXYZ}
	\pi(g e^{s\Ad\left( k \right)E_1)})=
	\begin{pmatrix}
		U	\\
		T	\\
		X	\\
		Y	\\
		Z
	\end{pmatrix}=
	\begin{pmatrix}
 u	&	.	&	.	&	.	&	.\\
 t	&	a	&	b	&	c	&	d\\
 x	&	.	&	.	&	.	&	.\\
 y	&	a'	&	b'	&	c'	&	d'\\
z	&	.	&	.	&	.	&	.
 \end{pmatrix}
 \begin{pmatrix}
	 1	\\
	 -s	\\
	 sw_1	\\
	 sw_2	\\
	 0
 \end{pmatrix}
\end{equation}
does not intersects the singularity in the future. In other words, we have $T\pm Y=0$ only with $s\leq 0$. Let
\begin{equation}
	\begin{aligned}[]
		T(w_1,w_2)&=t+s(bw_1+cw_2-a)\\
		Y(w_1,w_2)&=y+s(b'w_1+c'w_2-a')\\
		A_+(w_1,w_2)&=(b+b')w_1+(c+c')w_2-(a+a')\\
		A_-(w_1,w_2)&=(b-b')w_1+(c-c')w_2-(a-a').
	\end{aligned}
\end{equation}
We also denote by $\sigma_{\pm}$ the sign of $t\pm y$.

A simple computation shows that $T+Y=0$ when
\begin{equation}
	s=s_+=-\frac{ t+y }{ A_+(w_1,w_2) },
\end{equation}
and $T-Y=0$ when
\begin{equation}
	s=s_-=-\frac{ t-y }{ A_-(w_1,w_2) },
\end{equation}
The assumption is that the direction $(w_1,w_2,0)$ (and an open set in $S^1$ with respect to $(w_1,w_2)$) escapes the singularity, so that for every $(w_1',w_2')$ in a neighborhood of $(w_1,w_2)$, we have
\begin{equation}
	\begin{aligned}[]
		\sigma_{\pm}A_{\pm}(w_1',w_2')\geq 0,
	\end{aligned}
\end{equation}
which assures that the values of $s$ which annihilate $T+Y$ and $T-Y$ are negative or non existing. Since we choose $b\neq \pm b'$, the functions $A_{\pm}$ are nowhere constant, so we can find a direction $(w_1,w_2)$ such that $\sigma_{\pm}A_{\pm}(w_1,w_2)>0$. Notice that, by continuity, there exists a neighbourhood of $(w_1,w_2)$ in $S^1$ which escapes the singularity.

We are now studying what happens when one looks at a neighbourhood of $(w_1,w_2,0)$ in $S^3$. The path \eqref{EqPathgexpUTXYZ} is replaced by
\begin{equation}
	\pi(g e^{s\Ad(k)E_1})=
	\begin{pmatrix}
 u	&	.	&	.	&	.	&	.\\
 t	&	a	&	b	&	c	&	d\\
 x	&	.	&	.	&	.	&	.\\
 y	&	a'	&	b'	&	c'	&	d'\\
z	&	.	&	.	&	.	&	.
 \end{pmatrix}
\begin{pmatrix}
	1	\\
	-s	\\
	s(w_1+\epsilon_1)	\\
	s(w_2+\epsilon_2)	\\
	\epsilon_3
\end{pmatrix},
\end{equation}
and we consider
\begin{equation}
	\begin{aligned}[]
		T(w_1,w_2,\bar\epsilon)&=t+s\big( b(w_1+\epsilon_1)+c(w_2+\epsilon_2)+d\epsilon_3-a \big)\\
		Y(w_1,w_2,\bar\epsilon)&=y+s\big( b'(w_1+\epsilon_1)+c'(w_2+\epsilon_2)+d'\epsilon_3-a' \big)
	\end{aligned}
\end{equation}
where $\bar\epsilon$ stands for $\epsilon_1$, $\epsilon_2$ and $\epsilon_3$. The same computations as before shows that $T+Y=0$ when
\begin{equation}
	s=s_+=-\frac{ t+y }{ A_+(w_1,w_2)+(b+b')\epsilon_1+(c+c')\epsilon_2+(d+d')\epsilon_3 },
\end{equation}
Since $\sigma_+A(w_1,w_2)>0$, there exists a $\delta$ such that $s_+$ remains negative for every choice of $\bar\epsilon<\delta$. The same holds with $T-Y$ which is zero when
\begin{equation}
	s=s_-=-\frac{ t-y }{ A_-(w_1,w_2)+(b-b')\epsilon_1+(c-c')\epsilon_2 +(d-d')\epsilon_3 }.
\end{equation}
Since $\sigma_-A_-(w_1,w_2)>0$, one can find a $\delta>0$ such that $\bar\epsilon<\delta$ implies that this fraction remains negative.

Thus, there exists a neighbourhood of $(w_1,w_2,0)$ in $S^2$ of directions escaping the singularity from the point $v$.
\end{proof}

\begin{lemma}		\label{LemIntTroisQueatr}
	With the notations defined before, we have
	\begin{equation}
		\iota\big( \Int(\hF_3) \big)\subseteq \Int\big( \hF_4 \big)
	\end{equation}
	where $\Int$ stands for the interior. In other words,
	\begin{equation}
		\Adh(BH_4)\cap\iota(AdS_3)\subset\iota\big( \Adh(BH_3) \big).
	\end{equation}
\end{lemma}

\begin{proof}

	Let $v=\iota(v')\notin\iota\big( \Adh(BH_3) \big)$, we also consider $g'$ a representative of $v'$ and $g=\iota(g')$, which is a representative of $v$. The element $v'$ is in the interior of the free zone: there exists an open set of directions which do not intersect the singularity of $AdS_3$ by theorem~\ref{ThoHorIntDansS}. In other words, the set
\begin{equation}		\label{EqwwswswUn}
	\{ \begin{pmatrix}
	w_1	\\
	w_2
\end{pmatrix}\in S^1\tq
\pi g'\begin{pmatrix}
	1	\\
	-s	\\
	sw_1	\\
	sw_2
\end{pmatrix}\cap\hS_3 =\emptyset\}
\end{equation}
contains an open set of $S^1$. On the other hand, the $z$-component of the latter vector is obviously zero because $g=\iota(g')$ has the form
\begin{equation}
	g=\begin{pmatrix}
 .	&	.	&	.	&	.	&	0\\
 .	&	.	&	.	&	.	&	0\\
 .	&	.	&	.	&	.	&	0\\
 .	&	.	&	.	&	.	&	0\\
0	&	0	&	0	&	0	&	1
 \end{pmatrix},
\end{equation}
thus equation \eqref{EqwwswswUn} can be ``extended'' and there exists an open set in $S^1$ such that
\begin{equation}
	\pi g\begin{pmatrix}
		1	\\
		-s	\\
		sw_1	\\
		sw_2	\\
		0
	\end{pmatrix}\cap\iota(\hS_3)=\emptyset.
\end{equation}
Now, lemma~\ref{LemOouversttq} shows that the set
\begin{equation}
	\{
		\begin{pmatrix}
			w_1	\\
			w_2	\\
			w_3
		\end{pmatrix}\in S^2\tq
		\pi g\begin{pmatrix}
			1	\\
			-s	\\
			sw_1	\\
			sw_2	\\
			sw_3
		\end{pmatrix}\cap\hS_4=\emptyset
	\}
\end{equation}
contains an open subset of $S^2$. That means that $\pi(g)=v$ belongs to the interior of $\hF_4$.
\end{proof}

\begin{proposition}		\label{PropFqTroisFt}
We have $\hF_4\cap\iota(AdS_3)\subset \iota(\hF_3)$.
\end{proposition}

\begin{proof}
Let $v\in\hF_4\cap\iota(AdS_3)$. With the same notations as above, we have
\begin{equation}		\label{EqRepresSOiotag}
	\iota(g')=
\begin{pmatrix}
 u	&	.	&	.	&	.	&	0\\
 t	&	a	&	b	&	c	&	0\\
 x	&	.	&	.	&	.	&	0\\
 y	&	a'	&	b'	&	c'	&	0\\
0	&	0	&	0	&	0	&	1
 \end{pmatrix}
\end{equation}
The assumption is that, for every representative $g'$ of $v'$, there exists a direction $(w_1,w_2,w_3)\in S^2$ such that the path
\begin{equation}		\label{EqGedgpudt}
	\pi   \iota(g')\begin{pmatrix}
	1	\\
	-s	\\
	sw_1	\\
	sw_2	\\
	sw_3
\end{pmatrix}
\end{equation}
only intersects the singularity fore negative values of $s$. The values of $s$ that annihilate $t^2-y^2$ in the geodesic \eqref{EqGedgpudt} are
\begin{equation}
	\begin{aligned}[]
		s_+	&=-\frac{ t+y }{ -(a+a')+(b+b')w_1+(c+c')w_2 }\\
		s_-	&=-\frac{ t-y }{ -(a-a')+(b-b')w_1+(c-c')w_2 },
	\end{aligned}
\end{equation}
and these two values are either negative either non existing (vanishing denominator).

The work is now to find a direction $(w'_1,w'_2)\in S^1$ such that the geodesic
\begin{equation}
	\pi\big( g'\begin{pmatrix}
	1	\\
	-s	\\
	sw'_1	\\
	sw'_2
\end{pmatrix} \big)
\end{equation}
does not intersect the singularity. The values of $s$ for which the latter geodesics intersects the singularity are
\begin{equation}
	\begin{aligned}[]
		s'_+	&=-\frac{ t+y }{ -(a+a')+(b+b')w'_1+(c+c')w'_2 }\\
		s'_-	&=-\frac{ t-y }{ -(a-a')+(b-b')w'_1+(c-c')w'_2 }.
	\end{aligned}
\end{equation}
If $w_3=0$, the proposition is true because one can choose $(w'_1,w'_2)=(w_1,w_2)$. If $w_3\neq 0$, the vector $(w_1,w_2)$ does not belong to $S^1$, and we have to find something else.

Let us consider the following two cases.
\begin{enumerate}
\item
there exists a representative \eqref{EqRepresSOiotag} with $a=a'=0$,
\item
there exists a representative \eqref{EqRepresSOiotag} with $c=c'=0$.
\end{enumerate}
In the first case, we have
\begin{equation}		\label{EqDenoAAnnulerspm}
	s'_{\pm}=-\frac{ t\pm y }{ (b\pm b')w'_1+(c\pm c')w'_2 },
\end{equation}
and we can choose $(w'_1,w'_2)=N(w_1,w_2)$ with $N\in\eR$ fixed in such a way that $(w'_1,w'_2)\in S^1$. Thus we have $s'_{\pm}=\frac{1}{ N }s_{\pm}$ and it is sufficient to choose $N>0$ in order to leave the denominators of \eqref{EqDenoAAnnulerspm} of the right sign or zero.

In the second case, we have
\begin{equation}
	s'_{\pm}=-\frac{ t\pm y }{ -(a\pm a')+(b\pm b')w'_1 },
\end{equation}
thus one has to choose $w'_1=w_1$ and $w'_2=\sqrt{1-w_1^2}$.

Let us now discuss the values of $u$, $t$, $x$ and $y$ for which the first or the second cases are enforced. In order to be in the first case, we need to build a matrix of $\SO(2,2)$ of the form
\begin{equation}
	g'=\begin{pmatrix}
 u	&	\alpha	&	.	&	.	\\
 t	&	0	&	.	&	.	\\
 x	&	\beta	&	.	&	.	\\
 y	&	0	&	.	&	.
 \end{pmatrix}.
\end{equation}
That requires $\alpha^2-\beta^2=1$ and $u\alpha-x\beta=0$, while, for the second case, we need to build a matrix of $\SO(2,2)$ of the form
\begin{equation}
	g'=\begin{pmatrix}
 u	&	.	&	\alpha	&	.	\\
 t	&	.	&	0	&	.	\\
 x	&	.	&	\beta	&	.	\\
 y	&	.	&	0	&	.
 \end{pmatrix}.
\end{equation}
That requires $\alpha^2-\beta^2=-1$ and $u\alpha-x\beta=0$.

In both cases, we have $\beta=\frac{ u }{ x }\alpha$ and $\alpha^2-\beta^2=\alpha^2\left( 1-\frac{ u^2 }{ x^2 } \right)$. If $| u |>| x |$, we can solve $\alpha^2-\beta^2=-1$, and if $| u |<| x |$, then we can solve $\alpha^2-\beta^2=1$.

The last possible situation is $u=\pm x$. A point of $AdS_3$ in that situation belongs to the horizon by equation \eqref{EqHOrAdSTroisVecteur}, while one knows that point of horizon do have some directions which escape the singularity by corollary~\ref{PropBHouvertLibreFerme}. Notice that in the latter situation, we do not use the assumption that $\iota(v')$ is free in $AdS_4$.
\end{proof}

\begin{corollary}		\label{CorBHBHHHHH}
	We have $\iota(BH_3)\subset BH_4$ and $\iota(\hH_3)\subset \hH_4$.
\end{corollary}

\begin{proof}
	If $\iota(v)\notin BH_4$, we have $\iota(v)\in \hF_4\cap\iota(AdS_3)\subset\iota(\hF_3)$, which is not possible if $v\in BH_3$.

	For the second part, we consider $v\in\hH_3\subset\hF_3$ (proposition~\ref{PropBHouvertLibreFerme}). There is a direction $\begin{pmatrix}
		w_1	\\
		w_2
	\end{pmatrix}\in S^1$ which escapes the singularity from $v$ in $AdS_3$. Of course, the direction $\begin{pmatrix}
		w_1	\\
		w_2	\\
		0
	\end{pmatrix}\in S^2$ escapes the singularity from $\iota(v)$ in $AdS_4$. Thus $\iota(v)\in\hF_4$.

	In every neighborhood of $v$, there exists a $\bar v\in BH_3$, and thus $\iota(\bar v)\in BH_4$. In other words, in every neighborhood of $\iota(v)$, there is that $\iota(\bar v)$ which belongs to $BH_4$. That proves that $\iota(v)$ belongs to $\hH_4$.
\end{proof}

\begin{lemma}		\label{LemHinteridansH}
	We have $\hH_4\cap\iota(AdS_3)\subset\iota(\hH_3)$.
\end{lemma}

\begin{proof}
	Let $v\in\hH_4\cap\iota(AdS_3)$. Since $\hH_4\subset\hF_4$, we have $v\in\hF_4\cap\iota(AdS_3)\subset\iota(\hF_3)$ (proposition~\ref{PropFqTroisFt}), and then there exists a $v'\in\hF_3$ such that $v=\iota(v')$. Now, we have to prove that $v'\in\hH_3$. If $v'$ belongs to the interior of $\hF_3$, lemma~\ref{LemIntTroisQueatr} implies that
	\begin{equation}
		v=\iota(v')\in\iota\big( \Int(\hF_3) \big)\subset\Int(\hF_4),
	\end{equation}
	which disagrees with the fact that $v\in\hH_4$.
\end{proof}

\begin{lemma}		\label{LemPresqueHOrQadp}
Let $v\in\hH_4$ such that $u$ and $x$ are not both vanishing. In that case, $v\in G_V\cdot \iota(\hH_3)\cup G_X\cdot\iota(\hH_3)$.
\end{lemma}

\begin{proof}
The assumption on $u$ and $x$ make that $v\in G_V\cdot(AdS_3)\cup G_X\iota(AdS_3)$. In order to fix ideas, let us suppose that $v= e^{\alpha V}\iota(v')$ with $v'\in AdS_3$. Since the set of directions $(w_1,w_2,w_3)\in S^2$ which save the points $v$, $ e^{\alpha V}v$ and $ e^{\beta X}v$ are the same, the assumption that $v\in\hH_4$ implies that $\iota(v')\in \hH_4$, which in turn proves that $v'\in \hH_3$ by lemma~\ref{LemHinteridansH}. Thus $v\in G_V\cdot\iota(\hH_3)$.

The same being true with $X$ instead of $V$, the lemma is proved.
\end{proof}

\begin{proposition}		\label{PropovHhnonXYzero}
	Let $v'=(u',t',x',y',z')\in\hH_4$ with $u'$ and $x'$ not both vanishing. Then
	\begin{equation}
		v'\in G_{X_{0+}}\cdot \iota(\hH_3)\cup G_{X_{0-}}\cdot \iota(\hH_3).
	\end{equation}
\end{proposition}

\begin{proof}
	As a first step, we want to solve the equation
	\begin{equation}
		e^{\alpha X_{0+}}\begin{pmatrix}
			u	\\
			t	\\
			x	\\
			y	\\
			0
		\end{pmatrix}=
		\begin{pmatrix}
			\frac{ \alpha^2(u-x) }{2}+u	\\
			t	\\
			\frac{ \alpha^2(u-x) }{2}+x	\\
			y	\\
			-\alpha(x-u)
		\end{pmatrix}=\begin{pmatrix}
			u'	\\
			t'	\\
			x'	\\
			y'	\\
			z'
		\end{pmatrix}
	\end{equation}
	with respect to $u$, $t$, $x$, $y$ and $\alpha$. The result is $t=t'$, $y=y'$ and
	\begin{equation}
		\begin{aligned}[]
			\alpha&=\frac{ z' }{ u'-x' },&u&=u'-\frac{ z'^2 }{ 2(u'-x') },&x&=\frac{ z'^2 }{ 2(u'-x') }-x'.
		\end{aligned}
	\end{equation}
	We conclude that, as long as $u'-x'\neq 0$, the point $v'$ belongs to $G_{X_{0+}}\cdot\iota(AdS_3)$. The same computation shows that $v'\in G_{X_{0-}}\cdot\iota(AdS_3)$ as long as $x'+u'\neq 0$. Let us observe that the actions of the matrices $ e^{\alpha X_{0+}}$ and $ e^{\beta X_{0-}}$ do not change the $t$ and $y$ component of a vector in $\eR^{2,l-1}$, so that the set of directions for which $v$ falls in the singularity is exactly the same as the set of directions for which $ e^{\alpha X_{0+}}v$ and $ e^{\beta X_{0-}}v$ fall in the singularity.

	Now, let us suppose that $v= e^{\alpha X_{0+}}\iota(v')\in\hH_4$ with $v'\in AdS_3$. We want to prove that $\iota(v')\in\hH_4$ (i.e. there is an element in the black hole in each neighbourhood of $\iota(v')$) because in that case, lemma~\ref{LemHinteridansH} would conclude that $v'\in\hH_3$.

	Let $\mO$ be a neighbourhood of $\iota(v')$. The set $ e^{\alpha X_{0+}}\mO$ is a neighborhood of $v$, and thus there exists an element $\bar v\in e^{\alpha X_{0+}}\mO\cap BH_4$. Now the element $ e^{-\alpha X_{0+}}\bar v$ belongs to $\mO\cap BH_4$, so that $\iota(v')$ belongs to $\hH_4$.
\end{proof}

\begin{lemma}		\label{LemPasLEsDerniersAQ}\label{Lemuxznonsing}
The points of $AdS_4$ of the form $v=\begin{pmatrix}
	0	\\
	t	\\
	0	\\
	y	\\
	z
\end{pmatrix}$ do not belong to the horizon.
\end{lemma}

\begin{proof}

Since the horizon is $A$-invariant, we can reduce the lemma to the case of any element of the form $ e^{\eta J_1}v$. We have
\begin{equation}
	 e^{\eta J_1}
\begin{pmatrix}
	0	\\
	t	\\
	0	\\
	y	\\
	z
\end{pmatrix}=
\begin{pmatrix}
 1	&	0		&	0	&	0		&	0\\
 0	&	\cosh(\eta)	&	0	&	\sinh(\eta)	&	0\\
 0	&	0		&	1	&	0		&	0\\
 0	&	\sinh(\eta)	&	0	&	\cosh(\eta)	&	0\\
 0	&	0		&	0	&	0		&	1
 \end{pmatrix}
\begin{pmatrix}
	0	\\
	t	\\
	0	\\
	y	\\
	z
\end{pmatrix}=
\begin{pmatrix}
	0				\\
	\cosh(\eta)t+\sinh(\eta)y	\\
	0				\\
	\sinh(\eta)t+\cosh(\eta)y	\\
	z
\end{pmatrix}
\end{equation}
We annihilate the $y$ component by choosing $\eta=\ln\left( \frac{ t-y }{ t+y } \right)$. Notice that $t^2-y^2>0$, thus we have $| t |>| y |$ and the expression in the logarithm is always positive.

A representative of $(0,t,0,0,z)$ in $\SO(2,2)$ is easy to find, and the geodesic in the direction $\bar w\in S^2$ is given by
\begin{equation}
	\begin{pmatrix}
 0	&	1	&	0	&	0	&	0\\
 t	&	0	&	0	&	0	&	-z\\
 0	&	0	&	1	&	0	&	0\\
 0	&	0	&	0	&	1	&	0\\
z	&	0	&	0	&	0	&	-t
 \end{pmatrix}
\begin{pmatrix}
	1	\\
	-s	\\
	sw_1	\\
	sw_2	\\
	sw_3
\end{pmatrix}=
\begin{pmatrix}
	.	\\
	t-szw_3	\\
	.	\\
	sw_2	\\
	.
\end{pmatrix}.
\end{equation}
It belongs to the singularity when $s$ takes one of the values
\begin{equation}
	s_{\pm}=\frac{ t }{ w_3z\pm w_2 }.
\end{equation}
As long as $|w_2|<|w_3z|$, the two values $s_{\pm}$ have the same sign, which can be decided by making $w_3$ positive or negative. That provides an open set in $S^2$ of directions which escape the singularity, so that $v\notin\hH_4$.
\end{proof}


\begin{theorem}			\label{ThoHorQuatreInclusionHorTrois}\label{ThoEqHorQCoore}
	The horizon of $AdS_4$ is given by
	\begin{equation}		\label{EqEqHOrGVGXQuatr}
		\hH_4=G_{X_{0+}}\cdot \iota(\hH_3)\cup G_{X_{0-}}\iota(\hH_3).
	\end{equation}
	i.e. an union of lateral classes of the horizon of $AdS_3$ by one dimensional subgroups of $N$ and $\bar N$.

	The equation in the ambient $\eR^5$ is $\hH_4\equiv u^2-x^2-z^2=0$.
\end{theorem}

\begin{proof}
	We begin by the direct inclusion. If $v=(u,t,x,y,z)\in\hH_4$ with $u\neq 0$ or $x\neq 0$, we proved in proposition~\ref{PropovHhnonXYzero} that $v$ has the form \eqref{EqEqHOrGVGXQuatr}. Now, if $u=x=0$, the lemma~\ref{Lemuxznonsing} shows that $v$ does not belongs to the horizon.

	For the reverse inclusion, we know that elements of $\iota(\hH_3)$ belong to $\hH_4$ by corollary~\ref{CorBHBHHHHH}. If $v$ belong to $\hH_4$, then $ e^{\alpha X_{0+}}v$ and $ e^{\beta X_{0-}}v$ also belong to the horizon.
\end{proof}

%+++++++++++++++++++++++++++++++++++++++++++++++++++++++++++++++++++++++++++++++++++++++++++++++++++++++++++++++++++++++++++
\section{Conclusion}
%+++++++++++++++++++++++++++++++++++++++++++++++++++++++++++++++++++++++++++++++++++++++++++++++++++++++++++++++++++++++++++

The horizon of the BTZ black hole in $AdS_3$ was already expressed in \cite{Keio} as lateral classes of one point under the action of the Iwasawa component of the isometry group of $AdS_3$.

We proved that the simple inclusion map $\iota\colon AdS_3\to AdS_4$ transports the causal structure (free zone, black hole, horizon) from $AdS_3$ to $AdS_4$. We studied in particular the way the horizon changes when ones jumps from dimension $3$ to dimension $4$ and we obtained that the horizon in $AdS_4$ is expressed as lateral classes of the inclusion of the horizon of $AdS_3$ in $AdS_4$. In the same time, we obtained a simple equation for the horizon seen as a subset of $\eR^5$.

Although the results are quite satisfying, the method used here to prove them is quite unsatisfactory because we didn't used all the wealth structure of $\so(2,3)$ and of its reductive decompositions $\sG=\sH\oplus\sQ=\sK\oplus\sK$. We plan, in a future work, to get a much deeper understanding of the structure of $\sG$ and $\sQ$, in such a way to provide simpler proofs, in the same time as a dimensional generalization of the result of theorem~\ref{ThoHorQuatreInclusionHorTrois}. We would also like to define a class of homogeneous spaces $G/H$ which accept a BTZ-like black hole.


%+++++++++++++++++++++++++++++++++++++++++++++++++++++++++++++++++++++++++++++++++++++++++++++++++++++++++++++++++++++++++++
\section{The algebras without matrices}
%+++++++++++++++++++++++++++++++++++++++++++++++++++++++++++++++++++++++++++++++++++++++++++++++++++++++++++++++++++++++++++
\label{SecRebuildStructRoot}

We have two decompositions
\begin{equation}
	\begin{aligned}[]
		\sG&=\sK\stackrel{\theta}{=}\sP\\
		\sG&=\sH\stackrel{\sigma}{=}\sQ
	\end{aligned}
\end{equation}
of $\sG=\so(2,n)$. From there, we will build the basis elements of $\sA$, $\sN$, $\bar\sN$ with all the properties we used so far. The explicit matrices \eqref{EqGeueuleVWXY} and \eqref{EqGeneRedQ} consist in a concrete realisation of what we are about to do.

%---------------------------------------------------------------------------------------------------------------------------
\subsection{The structure theorem by Pyatetskii-Shapiro}
%---------------------------------------------------------------------------------------------------------------------------

We are going to use the  Pyatetskii-Shapiro's decompositions of normal $j$-algebra \eqref{EqDecNormale} and \eqref{EqDecoEleJal}.

\begin{lemma}
	We have
	\begin{equation}
		\| (X_{\alpha\beta})_{\sK} \|=\| (X_{\alpha\beta})_{\sP} \|.
	\end{equation}
\end{lemma}

\begin{proof}
	We use the invariance of the Killing form:
	\begin{equation}
		\begin{aligned}[]
			B\big( (X_{\alpha\beta})_{\sK},(X_{\alpha\beta})_{\sK} \big)&=\frac{1}{ \alpha }B\big( (X_{\alpha\beta})_{\sK},\ad(J_1)(X_{\alpha\beta})_{\sP} \big)\\
			&=-\frac{1}{ \alpha }B\big( \ad(J_1)(X_{\alpha\beta})_{\sK},(X_{\alpha\beta})_{\sP} \big)\\
				&=-B\big( (X_{\alpha\beta})_{\sP},(X_{\alpha\beta})_{\sP} \big).
		\end{aligned}
	\end{equation}
\end{proof}

\begin{lemma}
	we have
	\begin{equation}
		\| (X_{\alpha\beta})_{\sK} \|=\| (X_{\alpha,-\beta})_{\sK} \|.
	\end{equation}
\end{lemma}

\begin{proof}
	First, remark that $X_{\alpha,-\beta}=\sigma X_{\alpha\beta}$. We also know that $[\pr_{\sK},\sigma]=0$ because $[\sigma,\theta]=0$. The conclusion now comes from the fact that $\sigma$ is an isometry.
\end{proof}

%+++++++++++++++++++++++++++++++++++++++++++++++++++++++++++++++++++++++++++++++++++++++++++++++++++++++++++++++++++++++++++
\section{Characterisation of the horizon (vanishing norm)}
%+++++++++++++++++++++++++++++++++++++++++++++++++++++++++++++++++++++++++++++++++++++++++++++++++++++++++++++++++++++++++++
\label{SecVanNormChar}

In order to get the theorem~\ref{ThoEqHorQCoore}, we used the equation of the singularity, $\hS\equiv t^2-y^2=0$, which was proved in proposition~\ref{Proptcarrycarr}. But subsection~\ref{SubSecTwoCharSing} provides an other characterisation of the singularity, namely the loci of points $[g]$ such that $\| J_1^* \|=0$.

\section{Conclusions and perspectives}		\label{SecConcPerspAd}
%++++++++++++++++++++++++++++++++++++

Higher-dimensional generalizations of the BTZ construction have been studied in the physics' literature, by classifying the one-parameter subgroups of $\Iso(AdS_l)=\SO(2,l-1)$, see \cite{Figueroa,AdSBH,Madden,BanadosIQxXuEh,Aminneborg,HolstPeldan}.  Nevertheless, the approach we adopt here is conceptually different. We first reinterpret the non-rotating BTZ black hole solution using symmetric spaces techniques and present an alternative way to express its singularity.  We saw the latter as the union of the closed orbits of Iwasawa subgroups of the isometry group.  As shown, this construction extends straightforwardly to higher dimensional cases, allowing to build a non trivial black hole on anti de Sitter spaces of arbitrary dimension $l\geq 3$.  From this point of view, all anti de Sitter spaces of dimension $l\geq 3$ appear on an equal footing.

A natural question arising from this analysis is the following: \emph{given a semisimple symmetric space, when does the set of closed orbits of the Iwasawa subgroups of the isometry group, seen as singularity, define a non-trivial causal structure?} We answered this question in the case of anti de Sitter spaces, using techniques allowing in principle for generalization to any semisimple symmetric space.

We also proved that performing a discrete quotient along the orbits of $J_1$ makes the resulting space causally inextensible (closed space-like curves appear in the singular part of the space), but we did not address  questions like: are there other vector fields defining singularities (in the three dimensional case, we know that the answer is positive)? Can we identify a mass and an angular momentum from these hypothetic vectors? Are \emph{all} BTZ black holes obtainable in this way in higher dimensions?

% This is part of (almost) Everything I know in mathematics and physics
% Copyright (c) 2013-2014,2016, 2020
%   Laurent Claessens
% See the file fdl-1.3.txt for copying conditions.

%+++++++++++++++++++++++++++++++++++++++++++++++++++++++++++++++++++++++++++++++++++++++++++++++++++++++++++++++++++++++++++
\section{BTZ from the structure of \texorpdfstring{$ \so(2,n)$}{so(2,n)}}
%+++++++++++++++++++++++++++++++++++++++++++++++++++++++++++++++++++++++++++++++++++++++++++++++++++++++++++++++++++++++++++

\begin{abstract}
	In this section, we study the relevant structure of the algebra $\so(2,n)$ which makes the BTZ black hole possible in the anti de Sitter space $AdS=SO(2,n)/SO(1,n)$. We pay a particular attention to the reductive Lie algebra structures of $\so(2,n)$ and we study how this structure evolves when one increases the dimension.

	We define the singularity as the closed orbits of the Iwasawa subgroup of the isometry group of anti de Sitter, but we insist on an alternative (closely related to the original conception of the BTZ black hole) way to describe the singularity as the loci where the norm of fundamental vector field vanishes. We provide a manageable Lie algebra oriented formula which describes the singularity and we use it in order to derive the existence of a black hole and to give a geometric description of the horizon.

\end{abstract}


\section{Introduction}
\label{LONGSecSumStructExist}

\subsection{Anti de Sitter space and the BTZ black hole}

The anti de Sitter space (hereafter abbreviated by $AdS$ or $AdS_l$) is a static solution to the Einstein's equations that describes a universe without mass. It is widely studied in different context in mathematics as well as in physics.

The BTZ black hole in $AdS$, initially introduced in \cite{BTZ_un,BTZ_deux} and then described and extended in various ways \cite{HolstPeldan,Aminneborg,Madden}, is an example of black hole structure which does not derives from a metric singularity, but from a causal issue.

The point of view we consider here grown from the papers \cite{BTZB_deux,Keio} in the case of $AdS_3$ and insists on the homogeneous space structure and the action of Iwasawa groups, in particular the singularity was described by looking at the closed orbits of the action of the Iwasawa group of $SO(2,2)$ on $AdS_3$. A dimensional generalization was then performed in \cite{lcTNAdS} (see also \cite{These} and the references therein for for a longer review).

One of the motivation in going that way is to embed the study of BTZ black hole into the noncommutative geometry and singleton physics \cite{BTZ_WZW,articleBVCS} since the deformation and the black hole are defined in a compatible way. Indeed the deformation is performed using the action of the same group as the groups which produce the singularity.

%
\subsection{The way we describe the BTZ black hole}
%

The anti de Sitter space $AdS_l$ is the surface in $\eR^{l+1}$ described by the equation
\begin{equation}
	AdS_l\equiv t^2+u^2-x_1^2-\ldots-x_{l-1}^2=1.
\end{equation}
This space can be investigated in the framework of homogeneous spaces since it reads as the following quotient of groups:
\begin{equation}
	AdS_l=\frac{ \SO(2,l-1) }{ \SO(1,l-1) }=G/H.
\end{equation}
We denote by $\sG=\so(2,l-1)$ and $\sH=\so(1,l-1)$ the Lie algebras and by $\pi$ the projection $G\to G/H$. The class of $g$ will be written $[g]$ or $\pi(g)$. We choose an involutive automorphism $\sigma\colon \sG\to \sG$ which fixes elements of $\sH$, and we call $\sQ$ the eigenspace of eigenvalue $-1$ of $\sigma$. Thus we have the reductive decomposition
\begin{equation}		\label{LONGEqIntroRedDecompHQLieAlg}
	\sG=\sH\oplus\sQ.
\end{equation}
The compact part of $\SO(2,l-1)$ decomposes into $K=\SO(2)\times\SO(l-1)$. We denote by $\sK$ the Lie algebra of $K$. Let $\theta$ be a Cartan involution which commutes with $\sigma$, and consider the corresponding Cartan decomposition
\begin{equation}        \label{LONGEqIntroRedDecompKPLieAlg}
	\sG=\sK\oplus\sP,
\end{equation}
where $\sK$ is the $+1$ eigenspace of $\theta$ and $\sP$ is the $-1$ eigenspace. A maximal abelian algebra $\sA$ in $\sP$ has dimension two and one can choose a basis $\{ J_1,J_2 \}$ of $\sA$ in such a way that $J_1\in\sH$ and $J_2\in\sQ$.

Now we consider an Iwasawa decomposition
\begin{equation}
	\sG=\sK\oplus\sA\oplus\sN,
\end{equation}
where $\sK$ and $\sA$ are the compact and abelian parts while $\sN$ is the nilpotent part corresponding to a choice of positivity on the set of roots. We denote by $\sR$ the Iwasawa component $\sR=\sA\oplus\sN$. We are also going to use the algebra $\bar\sN=\theta\sN$ and the corresponding Iwasawa component $\bar\sR=\sA\oplus\bar\sN$.

The Iwasawa groups $R=AN$ and $\bar R=A\bar N$ are naturally acting on anti de Sitter by $r[g]=[rg]$. It turns out that each of these two action has exactly two closed orbits, regardless to the dimension we are looking at. The first one is the orbit of the identity and the second one is the orbit of $[k_{\theta}]$ where $k_{\theta}$ is the element which generates the Cartan involution at the group level: $k_{\theta}gk_{\theta}^{-1}=\theta(g)$ ($g\in G$). In a suitable choice of matrix representation, the element $k_{\theta}$ is the block-diagonal element which is $-\mtu$ on $\SO(2)$ and $\mtu$ on $\SO(l-1)$. The $A\bar N$-orbits of $\mtu$ and $k_{\theta}$ are also closed. Moreover we have
\begin{equation}
	\begin{aligned}[]
		[A\bar N k_{\theta}]&=[k_{\theta}AN]\\
		[AN k_{\theta}]&=[k_{\theta}A\bar N]
	\end{aligned}
\end{equation}
because $A$ is invariant under the adjoint action of $k_{\theta}$ and, by definition, $\theta(N)=\bar N$. We define as \defe{singular}{Singular point} the points of the closed orbits of $AN$ and $A\bar N$ in $AdS$.

The Killing form of $\SO(2,l-1)$ induces a Lorentzian metric on $AdS$. The sign of the squared norm of a vector thus divides the vectors into three classes:
\begin{equation}
	\begin{aligned}[]
		\| X \|^2&>0&\rightarrow&&\text{time like,}\\
		\| X \|^2&<0&\rightarrow&&\text{space like,}\\
		\| X \|^2&=0&\rightarrow&&\text{light like.}
	\end{aligned}
\end{equation}
A geodesic is time (resp. space, light) like if its tangent vector is time like (resp. space, light).

If $E_1$ is a nilpotent element in $\sQ$, then every nilpotent in $\sQ$ are given by $\{ \Ad(k)E_1 \}_{k\in\SO(l-1)}$. These elements are also all the light like vectors at the base point. A light like geodesic trough the point $\pi(g)$ in the direction $\Ad(k)E_1$ is given by
\begin{equation}
	\pi(g e^{s\Ad(k)E_1}).
\end{equation}
One says that  points with $s>0$ are in the \defe{future}{Future} of $\pi(g)$ while points with $s<0$ are in the \defe{past}{Past} of $\pi(g)$.

We say that a point in $AdS_l$ belongs to the \defe{black hole}{Black hole} if every light like geodesics trough that point intersect the singularity in the future. We call \defe{horizon}{Horizon} the boundary of the set of points in the black hole. One says that there is a (non trivial) black hole structure when the horizon is non empty or, equivalently, when there are some points in the black hole, and some outside.

All these properties can be easily checked using the matrices given in \cite{These,lcTNAdS}. In this optic, I wrote a program using Sage\cite{Sage} which checks all the properties that are shown in this paper. It will be published soon.

As far as notations are concerned, we denote by $X_{\alpha\beta}$ the basis of $\sN$ and $\bar\sN$ corresponding to our choice of Iwasawa decomposition. That basis is chosen in such a way that $\ad(J_1)X_{\alpha\beta}=\alpha X_{\alpha\beta}$ and $\ad(J_2)X_{\alpha\beta}=\beta X_{\alpha\beta}$.

%
\subsection{Organization of the paper}
%

The main goal of this paper is to reorganize all this structure in a coherent way. Then we use it efficiently in order to define the singularity of the BTZ black hole, to prove that one has a genuine black hole in every dimension, and to determine the horizon.

In section~\ref{LONGSecProgressRidMatrices}, we list the commutators of $\so(2,n)$ with respect to its root spaces and we organize them in such a way to get a clear idea about the evolution of the structure when the dimension increases. We prove that, when one passes from $\so(2,n)$ to $\so(2,n+1)$, one gets four new vectors in the root spaces and that these are Killing-orthogonal to the vectors existing in $\so(2,n)$ (this is the ``dimensional slice'' described in subsection~\ref{LONGSubSecDimensionalSlices}).

We give in subsection~\ref{LONGSubSecReductiveDecompQ} an original way to describe the space $\sQ$ without reference to $\sH$. The space $\sQ$ is usually described as a complementary of $\sH$. Here we show that it can be described by means of the root spaces and the Cartan involution $\theta$. The space $\sH$ is then described as $\sH=[\sQ,\sQ]$. In some sense, we describe the quotient space $AdS=G/H$ directly by its tangent space $\sQ$ without passing trough the definition of $H$. Of course, the knowledge of $\sH$ will be of crucial importance later.


The subsection~\ref{LONGSubSecPropRedDecompQH} is devoted to the proof of many properties of the decompositions $\sG=\sH\oplus\sQ$ and $\sG=\sK\oplus\sP$.


The first important result is the proposition~\ref{LONGXUnALaTwistingSuperCool} that shows that the elements of $\sQ$ are $\ad$-conjugate to each others: there exist elements of the adjoint group which are intertwining the elements of $\sQ$.

Using that property, we compute the norm of these elements and we identify the nilpotent vectors in $\sQ$ (these are the light-like vectors). In the same time, we prove that the space $G/H$ is Lorentzian and we provide an orthogonal basis of $\sQ$.

The second central result is the fact that nilpotent elements in $\sQ$ are of order two: if $E\in\sQ$ is nilpotent, then $\ad(E)^3=0$. That result will be used in a crucial way in the proof of the black hole existence, as well as in the study of its properties.

In section~\ref{LONGSecBlacHole}, we define and study the structure of the BTZ black hole in the anti de Sitter space. First we identify the closed orbits of the Iwasawa group (theorem~\ref{LONGThoOrbitesOuverttes}) and we define them as singular. In a second time, we provide an alternative description the singularity: theorem~\ref{LONGThosSequivJzero} shows that the singularity can be described as the loci of points at which a fundamental vector field has vanishing norm. We also provide in lemma~\ref{LONGLemExpressionCoolNormJUn} a convenient way to compute that norm on arbitrary point of the space.

We prove, in section~\ref{LONGSubSecExistenceTrouNoir}, that our definition of singularity gives rise to a genuine black hole in the sense that there exists points from which some geodesics escape the singularity in the future and there exists some points from which all the geodesics are intersecting the singularity in the future.

In section~\ref{LONGSecHorizonSansMatrices}, we provide a geometric description of the horizon (theorem~\ref{LONGThoCausalPasseParR}). We show that, if we see $AdS_l$ as a subset of $AdS_{l+1}$, the space $AdS_{l+1}$ is generated by the action on $AdS_l$ by a one parameter group which leaves the singularity invariant. This action thus leaves invariant the whole causal structure and we are able to express the horizon in any dimension from the well known horizon in $AdS_3$.
%
\section{Structure of the algebra}
%
\label{LONGSecProgressRidMatrices}

%
\subsection{The Iwasawa component}
%

Our study of $AdS_l=\SO(2,l-1)/\SO(1,l-1)$ will be based on the properties of the algebra $\sG=\so(2,l-1)$ endowed with a Cartan involution $\theta$ and an Iwasawa decomposition $\sG=\sA\oplus\sN\oplus\sK$. In this section we want to underline the most relevant facts for our purpose. The part we are mainly interested in is the Iwasawa component $\sA\oplus\sN$ where
\begin{subequations}		\label{LONGEqLeANEnDimAlg}
\begin{align}
	\sN&=\{ X^{k}_{+0},X^{k}_{0+},X_{++},X_{+-} \}\\
	\sA&=\{ J_1, J_2\},
\end{align}
\end{subequations}
where $k$ runs %
\footnote{The ``new'' vectors which appear in $AdS_l$ with respect to $AdS_{l-1}$ are $X_{0\pm}^{l-1}$ and $X_{\pm 0}^{l-1}$. Such an element appears for the first time in $AdS_4$ and is not present when one study $AdS_3$.} %
from $3$ to $l-1$. The commutator table is
\begin{subequations}  \label{LONGEqTableSOIwa}
	\begin{align}
		[X_{0+}^{k},X_{+0}^{k'}]&=\delta_{kk'}X_{++}		&[X_{0+}^{k},X_{+-}]&=2X_{+0}^{k}\\
		[ J_1,X_{+0}^{k}]&=X_{+0}^{k}				&[ J_2,X_{0+}^{k}]&=X_{0+}^{k}\\
		[ J_1,X_{+-}]&=X_{+-}					&[ J_2,X_{+-}]&=-X_{+-}\\
		[ J_1,X_{++}]&=X_{++}					&[ J_2,X_{++}]&=X_{++}.
	\end{align}
\end{subequations}
We see that the Iwasawa algebra belongs to the class of $j$-algebras whose Pyatetskii-Shapiro decomposition is
\begin{equation}
	\sA\oplus\sN=(\sA_1\oplus_{\ad}\sZ_1)\oplus_{\ad}\big( \sA_2\oplus_{\ad}(V\oplus \sZ_2) \big),
\end{equation}
with
\begin{subequations}
	\begin{align}
		\sA_1&=\langle H_1\rangle		&\sA_2&=\langle H_2\rangle\\
		\sZ_1&=\langle X_{+-}\rangle		&\sZ_2&=\langle X_{++}\rangle\\
		&					&V&=\langle X_{0+}^{k},X_{+0}^{k}\rangle_{k\geq 3}
	\end{align}
\end{subequations}
where
\begin{equation}
	\begin{aligned}[]
		H_1&=J_1-J_2\\
		H_2&=J_1+J_2.
	\end{aligned}
\end{equation}

The general commutators of such an algebra are
\begin{subequations}		\label{LONGsubEqsGenPySO}
\begin{align}
	[H_1,X_{+-}]		&=2X_{+-}	&	[H_2,X_{0+}^{k}]	&=X_{0+}^{k}				&[H_1,V]	&\subset V	\\
				&		&	[H_2,X_{+0}^{k}]	&=X_{+0}^{k}				&[X_{+-},V]	&\subset V	\\
				&		&	[H_2,X_{++}]		&=2X_{++}								\\
				&		&	[X_{0+}^k,X_{+0}^l]	&=\Omega(X_{0+}^k,X_{+0}^l)X_{++}
\end{align}
\end{subequations}
In the case of $\so(2,n)$, we have the following more precise relations:
\begin{subequations}		\label{LONGSubEqsPlusPresPySO}
	\begin{align}
		[H_1,X^k_{0+}]&=-X^k_{0+}\\
		[X_{+-},X^k_{0+}]&=-2X^k_{+0}
	\end{align}
\end{subequations}
and the link between $\sN$ and $\tilde\sN$ is given by
\begin{subequations}		\label{LONGSubEqsThethaPySO}
	\begin{align}
		[\theta X_{+0}^{k},X_{++}]	&=2X_{0+}^{k}\\
		[\theta X_{0+}^{k},X^k_{0+}]	&=2J_2\\
		[\theta X_{++},X_{++}]		&=4H_2=4(J_1+J_2)\\
		[\theta X_{++},X^{k}_{0+}]	&=2X_{-0}^{k}\\
		[\theta X_{+-},X_{+-}]		&=4H_1=4(J_1-J_2)\\
		[\theta X_{+-},X^{k}_{+0}]	&=2X_{0+}^{k}.
	\end{align}
\end{subequations}
The relations between the higher dimensional root spaces are
\begin{equation}
	\begin{aligned}[]
		[X_{0+}^i,X_{-0}^j]&=-\delta_{ij}X_{-+}\\
		[X_{0+}^i,X_{+0}^j]&=\delta_{ij}X_{++}\\
		[X_{+0}^i,X_{0+}^j]&=-\delta_{ij}X_{++}\\
		[X_{+0}^i,X_{0-}^j]&=\delta_{ij}X_{+-}.
	\end{aligned}
\end{equation}
%
%
The space $\mZ_{\sK}(\sA)$ of the elements of $\sK$ which commute with all the elements of $\sA$ is given by the elements $r_{ij}=\frac{ 1 }{2}[X_{0+}^i,X_{0+}^j]$ for every $i,j\geq 3$, $i\neq j$ realise the $\so(l-1)$ algebra. We will say more about them in subsection~\ref{LONGSubSec_Thecompactpart}.


We deduce the following relations that will prove useful later
\begin{equation}
	\begin{aligned}[]
		[\theta X_{0+}^k,X_{++}]&=-2X^k_{+0}\\
		[\theta X_{+0}^k,X_{+-}]&=-2X^k_{0-}\\
		[\theta X_{++},X_{+0}^k]&=-2X^k_{0-}\\
		[X_{-+},X^k_{0-}]&=-2X^k_{-0}.
	\end{aligned}
\end{equation}


%
%
\subsection{The compact part}
%
\label{LONGSubSec_Thecompactpart}

The compact part of $\so(2,l-1)$, the algebra $\so(2)\oplus\so(l-1)$ is well known. What is interesting from our point of view is to write the commutation relations between the elements of $\so(l-1)$ and the roots.

We define the following elements that are non vanishing:
\begin{equation}
	[X_{0+}^i,X_{0-}^j]=[X_{0-}^i,X_{0+}^j]=2r_{ij}.
\end{equation}
One immediately has $\theta r_{ij}=r_{ij}$, so that $r_{ij}\in\sK$. We also have $r_{ij}\in\sG_0$ so that $r_{ij}\in\mZ_{\sK}(\sA)$ and they act on the root spaces. The action is given by
\begin{subequations}
	\begin{align}		\label{LONGEqComsRRN}
        [r_{ij},X_{+0}^k]&=0		&\text{if }i,j,k \text{ are different}\\
        [r_{ij},X_{0+}^k]&=0		&\text{if } i, j,k \text{ are different}\\
        [r_{ij},X_{+0}^j]&=X_{+0}^i	&\text{if }i\neq j\\
        [r_{ij},X_{0+}^j]&=-X_{0+}^i	&\text{if }i\neq j\\
		[r_{ij},X_{\pm\pm}]&=0.			\label{LONGsubeqrXpmpm}
	\end{align}
\end{subequations}
The elements $r_{ij}$ satisfy the algebra of $\so(n)$.

\begin{remark}
	If $\sigma$ is an involutive automorphism which commutes with $\theta$ and such that $\sigma J_1=J_1$, $\sigma J_2=-J_2$, then one has $\sigma r_{ij}=r_{ij}$. We will see later that this fact makes $r_{ij}\in\sH$.
\end{remark}

We know\footnote{See \cite{Berndt} for example.} that $\sG=\sG_0\oplus\sN\oplus\bar\sN$ where $\sG_0=\sA\oplus\mZ_{\sK}(\sA)$. Let us perform a dimension count in order to be sure that the vectors $r_{ij}$ generate $\mZ_{\sK}(\sA)$. When we are working with $AdS_l$, we have
\begin{equation}
	\begin{aligned}[]
		\dim(\sA)&=2\\
		\dim(\tilde\sN_2)&=4\\
		\dim(\bigoplus_{k}\tilde\sN_k)&=4(l-3)\\
		\dim(\langle r_{ij}\rangle)&=\frac{ 1 }{2}(l-4)(l-3).
	\end{aligned}
\end{equation}
The last line comes from the fact that we have the elements $r_{34}$, $r_{35}$,\ldots,$r_{45}$,\ldots The first such element appears in $AdS_5$. Making the sum, we obtain $\frac{ l(l+1) }{ 2 }$, which is the dimension of $\so(2,l-1)$. Thus we have
\begin{equation}
	\sG=\mZ_{\sK}(\sA)\oplus\sA\oplus\tilde\sN_2\oplus\bigoplus_{k}\tilde\sN_k
\end{equation}
where $\mZ_{\sK}(\sA)$ is generated by the elements $r_{ij}$.
%

%
\subsection{Dimensional slices}
%
\label{LONGSubSecDimensionalSlices}

Since the set $\mZ_{\sK}(\sA)$ of elements in $\sK$ that commute with elements of $\sA$ is a part of $\sH$ (see later), they will have almost no importance in the remaining\footnote{We will however need them in the computation of the coefficients \eqref{LONGEqCoefsabcBE}.}. The most important part of $\sG$ is
\begin{equation}		\label{LONGEqDecomDimQlipm}
    \sA\oplus\sN\oplus\bar\sN=\underbrace{\langle J_1,J_2,X_{\pm,\pm}\rangle}_{\text{for every dimension}}\oplus\underbrace{\langle X_{0\pm}^{4},X_{\pm 0}^{4}\rangle}_{\text{for }\so(2,\geq 3)}\oplus\ldots\oplus\underbrace{\langle X_{0\pm}^l,X_{\pm 0}^{l}\rangle}_{\text{for } \so(2, l-1)}.
\end{equation}
We use the following notations in order to make more clear how does the algebra evolve when one increases the dimension:
\begin{equation}
	\begin{aligned}[]
		\sN_2&=\langle X_{+-},X_{++}\rangle,		&\sN_k&=\langle X^k_{0+},X_{+0}^k\rangle	\\
		\bar\sN_2&=\langle X_{-+},X_{--} \rangle, 	&\bar\sN_k&=\langle X_{0-}^k,X_{-0}^k\rangle\\
		\tilde\sN_2&=\langle \sN_2,\bar\sN_2\rangle,	&\tilde\sN_k&=\langle \sN_k,\bar\sN_k\rangle
	\end{aligned}
\end{equation}
for $k\geq 3$. The relations are
\begin{equation}		\label{LONGEqsCommWithtsNDeuxkA}
	\begin{aligned}[]
		[\tilde\sN_2,\tilde\sN_2]&\subseteq\sA
		&[\tilde\sN_2,\tilde\sN_k]&\subseteq\tilde\sN_k\\
		[\tilde\sN_k,\tilde\sN_{k}]&\subseteq\sA\oplus\tilde\sN_2
		&[\tilde\sN_k,\tilde\sN_{k'}]&\subset \mZ_{\sK}(\sA) \\
	\end{aligned}
\end{equation}

As a consequence of the splitting and the commutation relations, we have many Killing-orthogonal subspaces in $\so(2,l-1)$:
\begin{subequations}
	\begin{align}
		\tilde\sN_k\perp\tilde\sN_{k'}		\label{LONGEqtsnkperprtsnkp}	\\
		\sA\perp\tilde\sN_2			\label{LONGEqAperpNTrois}		\\
		\sA\perp\tilde\sN_k.	\label{LONGEqAperpNk}				\\
		\tilde\sN_2\perp\tilde\sN_k	\label{LONGEqNTtroisperpNk}
	\end{align}
\end{subequations}

In order to check there relations, look at the trace of action of $\ad(X)\circ\ad(Y)$ on the various spaces:
\begin{equation}
	\ad(\sA)\circ\ad(\tilde\sN_2) \colon
	\begin{cases}
		\sA\to\tilde\sN_2\to\tilde\sN_2\\
		\tilde\sN_2\to\sA\to 0\\
		\tilde\sN_k\to\tilde\sN_k\to\tilde\sN_k\\
		\mZ_{\sK}(\sA)\to\tilde\sN_2\to\tilde\sN_2,
	\end{cases}
\end{equation}
and
\begin{equation}
	\ad(\sA)\circ\ad(\tilde\sN_k)\colon
	\begin{cases}
		\sA\to\tilde\sN_k\to\tilde\sN_k\\
		\tilde\sN_2\to\tilde\sN_k\to\tilde\sN_k\\
		\tilde\sN_k\to\sA\oplus\tilde\sN_2\to\tilde\sN_2\\
		\langle r_{kl}\rangle\to\tilde\sN_l\to\tilde\sN_l
	\end{cases}
\end{equation}
and
\begin{equation}
	\ad(\tilde\sN_2)\circ\ad(\tilde\sN_k)\colon
	\begin{cases}
		\sA\to\tilde\sN_k\to\tilde\sN_k\\
		\tilde\sN_2\to\tilde\sN_k\to\tilde\sN_k\\
		\tilde\sN_k\to\sA\oplus\tilde\sN_2\to\tilde\sN_2\oplus\sA\\
		\langle r_{kl}\rangle\to\tilde\sN_l\to\tilde\sN_l\oplus\sA
	\end{cases}
\end{equation}
and
\begin{equation}
	\ad(J_1)^2|_{\tilde\sN_2}=\ad(J_2)^2|_{\tilde\sN_2}=\id|_{\tilde\sN_2}.
\end{equation}
%
As a consequence,
\begin{equation}		\label{LONGEqDecompGPourKillOrtho}
	\sG=\mZ_{\sK}(\sA)\oplus\sA\oplus\tilde\sN_2\bigoplus_{k\geq 3}\tilde\sN_k
\end{equation}
is a Killing-orthogonal decomposition of $\so(2,l-1)$.

%
\subsection{Reductive decomposition}
%
\label{LONGSubSecReductiveDecompQ}

Related to the decompositions \eqref{LONGEqIntroRedDecompHQLieAlg} and \eqref{LONGEqIntroRedDecompKPLieAlg} we introduce the following notations: $X_{\sK}$, $X_{\sP}$ will denote the $\sK$ and $\sP$-components of the vector $X$ with respect to the decomposition $\sG=\sK\oplus\sP$. We define $X_{\sH}$ and $X_{\sQ}$ similarly with respect to the decomposition $\sG=\sH\oplus\sQ$. In the same way we define $X_{\sA}$, $X_{\sZ(\sK)}$ and so on by the Iwasawa decomposition
\begin{equation}
    \sG=\sA\oplus\sN\oplus\sK=\sA\oplus\sN\oplus\sZ(\sK)\oplus\so(l-1).
\end{equation}

Let $\sQ$ be the following vector subspace of $\sG$:
\begin{equation}		\label{LONGEqDecQEspacesCools}	%
	\sQ=\big\langle \mZ(\sK),J_2,[\mZ(\sK),J_1],(X_{0+}^k)_\sP\big\rangle_{k\geq 3}.
\end{equation}
Then we choose a subalgebra $\sH$ of $\sG$ which, as vector space, is a complementary of $\sQ$. In that choice, we require that there exists an involutive automorphism $\sigma\colon \sG\to \sG$ such that
\begin{equation}
	\sigma=(\id)_{\sH}\oplus(-\id)_{\sQ}.
\end{equation}
In that case the decomposition $\sG=\sH\oplus\sQ$ is reductive, i.e. $[\sQ,\sQ]\subset\sH$ and $[\sQ,\sH]\subset\sQ$.




From definition \eqref{LONGEqDecQEspacesCools}, it is immediately apparent that one has a basis of $\sQ$ made of elements in $\sK$ and $\sP$, so that one immediately has
\begin{equation}
	[\sigma,\theta]=0.
\end{equation}

If $X\in\sG$, the projections are given by
\begin{equation}		\label{LONGEqProjHQPKsigmatheta}
	\begin{aligned}[]
		X_{\sH}&=\frac{ 1 }{2}(X+\sigma X),		&X_{\sK}&=\frac{ 1 }{2}(X+\theta X),\\
		X_{\sQ}&=\frac{ 1 }{2}(X-\sigma X),		&X_{\sP}&=\frac{ 1 }{2}(X-\theta X).
	\end{aligned}
\end{equation}
In particular $\theta\sH\subset\sH$ since $\theta$ and $\sigma$ commute.

We introduce the following elements of $\sQ$:
\begin{equation}			\label{LONGAlignPRedDefQQ}
	\begin{aligned}[]
		q_0&=(X_{++})_{\sK\sQ}\\
		q_1&=J_2\\
		q_2&=-[J_1,q_0]=-(X_{++})_{\sP\sQ}\\
		q_k&=(X^k_{0+})_{\sP}.
	\end{aligned}
\end{equation}
In order to see that $[J_1,q_0]=(X_{++})_{\sP\sQ}$, just write the $\sP\sQ$-component of the equality $[J_1,X_{++}]=X_{++}$. We will prove later that this is a basis and that each of these elements correspond to one of the spaces listed in \eqref{LONGEqDecQEspacesCools}.

Since $X_{\sP}=(\theta X-X)/2$, we have
%
\begin{equation}
	[q_i,q_j]=-\frac{1}{ 4 }\big( [X_{0+}^i,X_{0-}^j]+[X_{0-}^i,X_{0+}^j] \big)=r_{ij}.
\end{equation}%
%
%

From equations \eqref{LONGEqAperpNTrois} and \eqref{LONGEqAperpNk}, we have $q_1\perp q_2$ and $q_2\perp q_k$. Using the other perpendicularity relations $\sK\perp\sP$ and \eqref{LONGEqtsnkperprtsnkp}, \eqref{LONGEqAperpNTrois}, \eqref{LONGEqAperpNk},
%
%
we see that the set $\{ q_i \}_{0\leq i\leq l-1}$ is orthogonal.

The space $\sH$ is defined as generated by the elements
\begin{equation}			\label{LONGAlignPremDefHH}
	\begin{aligned}[]
		J_1&		&r_k&=[J_2,q_k]			\\
		p_1&=[q_0,q_1]	&p_k&=[q_0,q_k]			\\
		s_1&=[J_1,p_1]	&s_k&=[J_1,p_k].
	\end{aligned}
\end{equation}
Elements \eqref{LONGAlignPRedDefQQ} and \eqref{LONGAlignPremDefHH} will be studied in great details later.%

%
\subsubsection{Remark on the compact part}
%
\label{LONGSubSubSecRemCompPart}

Elements of $\sK$ are elements of the form $X+\theta X$. A part of the elements inside $\mZ_{\sK}(\sA)$, these elements are of two kinds:
\begin{subequations}
	\begin{align}
		X_{++}+X_{--}\\
		X_{+-}+X_{-+}
	\end{align}
\end{subequations}
on the one hand, and
\begin{subequations}
	\begin{align}
		X^k_{0+}+X^k_{0-}\\
		X^k_{+0}+X^k_{-0}
	\end{align}
\end{subequations}
on the other hand. The first two are commuting, so that $\mZ(\sK)$ is two dimensional when one studies $AdS_3$. That correspond to the well known fact that the compact part of $\so(2,2)$ is $\so(2)\oplus\so(2)$ which is abelian. These elements, however, do not commute with the two other. For example, the combination
\begin{equation}
	\big( X_{++}+X_{--}\big)-\big(X_{+-}+X_{-+}\big)
\end{equation}
does not commute with the elements of the second type. Now, one checks that the combination
\begin{equation}		\label{LONGEqZKChoixblabl}
	\big( X_{++}+X_{--}\big)+\big( X_{+-}+X_{-+})
\end{equation}
commutes with all the other, so that it is the generator of $\mZ(\sK)$ for $AdS_{\geq 4}$. This corresponds to the fact that the compact part of $\so(2,n)$ is $\so(2)\oplus\so(n)$. In other terms,
\begin{equation}		\label{LONGEqDecompZKenDeuxSuivantDim}
	\mZ(\sK)=\langle X_{++}+X_{--}+X_{+-}+X_{-+}\rangle\oplus\underbrace{\langle X_{++}+X_{--}-X_{+-}-X_{-+}\rangle}_{\text{only for }AdS_3}.
\end{equation}


Notice that, for $AdS_{\geq 4}$, we can define $q_0=(X_{++})_{\mZ(\sK)}$ as $\sK=\so(2)\oplus\so(l-2)$ for $AdS_l$. The case of $AdS_3$ is particular because $\mZ(\sK)$ is of dimension two and we have to set by hand what part of $\mZ(\sK)$ belongs to $\sQ$ (the other part belongs to $\sH$). From what is said around equation \eqref{LONGEqZKChoixblabl}, we know that $q_0$ is a multiple of $X_{++}+X_{--}+X_{+-}+X_{-+}$.

Dimension counting shows that $\dim\sQ=l$ and general theory of homogeneous spaces shows that $\sQ$ has to be seen as the tangent space of the manifold $G/H$.

%
\subsection{Properties of the reductive decompositions}
%
\label{LONGSubSecPropRedDecompQH}

We are considering the two reductive decompositions $\sG=\sH\oplus\sQ=\sK\oplus\sP$ in the same time as the root space decomposition \eqref{LONGEqDecompGPourKillOrtho}. We are now giving some properties of them.


We know that $\sK\cap\sQ=\langle q_0\rangle$ belongs to $\tilde\sN_2$. As a consequence, the elements $X^k_{\alpha 0}$ and $X^k_{0\alpha}$ have no $\sK\sQ$-components and
\begin{equation}
	\begin{aligned}[]
		[J_2,\tilde\sN_k]_{\sP\sH}&=0\\
		[J_2,\tilde\sN_k]_{\sP\sQ}&=0\\
		\pr_{\sP\sQ}X_{\alpha 0}^k&=0\\
		\pr_{\sP\sH}X_{0\alpha }^k&=0.
	\end{aligned}
\end{equation}
Since $X^k_{\alpha 0}$ and $X^k_{0\alpha}$ are not  eigenvectors of $\theta$, they have a non vanishing $\sP$-component. We deduce that
\begin{equation}				\label{LONGEqXalphazeroaduPH}
	\begin{aligned}[]
		\pr_{\sP\sH}X^k_{\alpha 0}\neq 0\\
		\pr_{\sP\sQ}X^k_{0 \alpha }\neq 0.
	\end{aligned}
\end{equation}

As a consequence of compatibility between $\theta$ and $\sigma$, we have
\begin{equation}
	\begin{aligned}[]
		[J_1,(X_{\alpha\beta})_{\sH}]&=\alpha (X_{\alpha\beta})_{\sH}\\
		[J_1,(X_{\alpha\beta})_{\sQ}]&=\beta (X_{\alpha\beta})_{\sQ}
	\end{aligned}
\end{equation}
and
\begin{equation}		\label{LONGSubEqsJdeuxXalphaneta}
	\begin{aligned}[]
		[J_2,(X_{\alpha\beta})_{\sH}]&=\beta (X_{\alpha\beta})_{\sQ}\\
		[J_2,(X_{\alpha\beta})_{\sQ}]&=\alpha (X_{\alpha\beta})_{\sH}.
	\end{aligned}
\end{equation}
So $X_{\sQ}$ itself is an eigenvector of $\ad(J_1)$. In the same way, we prove that
\begin{equation}		\label{LONGEqJUnXabPK}
	\begin{aligned}[]
		[J_1,(X_{\alpha\beta})_{\sP}]=\alpha(X_{\alpha\beta})_{\sK}\\
		[J_1,(X_{\alpha\beta})_{\sK}]=\alpha(X_{\alpha\beta})_{\sP}
	\end{aligned}
\end{equation}
because $J_1\in\sP$.

\begin{corollary}		\label{LONGCorHPHKQPQKXuu}
	The vector $X_{++}$ has non vanishing components in $\sH\cap\sP$, $\sH\cap\sK$, $\sQ\cap\sP$ and $\sQ\cap\sK$.
\end{corollary}

\begin{proof}
	Since $\ad(J_2)$ inverts the $\sH$ and $\sQ$-components of $X_{++}$ (equation \eqref{LONGSubEqsJdeuxXalphaneta}), they must be both non zero. In the same way $\ad(J_1)$ inverts the components $\sP$ and $\sK$ of vectors of $\sH$ and $\sQ$ (equations \eqref{LONGEqJUnXabPK}).
\end{proof}

\begin{lemma}		\label{LONGLEmDesZPP}
	We have	$(X^{k}_{0+})_{\sK\sQ}=(X^{k}_{0+})_{\sP\sH}=0$ and consequently, $(X^k_{0+})_{\sP}=(X^k_{0+})_{\sQ}$.
\end{lemma}

\begin{proof}
	Consider the decomposition of the equality $[J_1,X^{k}_{0+}]=0$ into components $\sP\sQ$, $\sP\sH$, $\sK\sQ$, $\sK\sH$. Since $J_1\in\sP\cap\sH$, the $\sK\sH$ and $\sP\sQ$ components are
	\begin{subequations}
		\begin{align}
			[J_1,(X^{k}_{0+})_{\sP\sH}]&=0\\
			[J_1,(X^{k}_{0+})_{\sK\sQ}]&=0.
		\end{align}
	\end{subequations}
	In the same way, using the fact that $J_2\in\sP\cap\sQ$, we have
	\begin{subequations}	\label{LONGEqDeuxJUDzpKq}
		\begin{align}
			[J_2,(X_{0+}^{k})_{\sP\sH}]&=(X_{0+}^{k})_{\sK\sQ}\\
			[J_2,(X^{k}_{0+})_{\sK\sQ}]&=(X^{k}_{0+})_{\sP\sH}.		\label{LONGsubEqDeuxJUDzpKqb}
		\end{align}
	\end{subequations}
	Since $\dim(\sK\cap\sQ)=1$, the component $(X^{k}_{0+})_{\sK\sQ}$ has to be a multiple of $(X_{++})_{\sK\sQ}$. Thus we have
	\begin{equation}
		0=[J_1,(X^{k}_{0+})_{\sK\sQ}]=\lambda [J_1,(X_{++})_{\sK\sQ}]=\lambda (X_{++})_{\sP\sQ},
	\end{equation}
	but $(X_{++})_{\sP\sQ}\neq 0$, thus $\lambda=0$ and we conclude that $(X_{0+}^{k})_{\sK\sQ}=0$. Now, equation \eqref{LONGsubEqDeuxJUDzpKqb} shows that $(X_{0+}^{k})_{\sP\sH}=0$.
\end{proof}

\begin{lemma}				\label{LONGLemSigmaXzpBien}
	We have $\sigma X^k_{0+}=X^k_{0-}$.
\end{lemma}

\begin{proof}
	Since we know that $\sigma\sG_{\alpha\beta}\subset\sG_{\alpha,-\beta}$, the work is to fix the sign in
	\begin{equation}		\label{LONGEqsigmaXzppmthetaXzp}
		\sigma X^k_{0+}=\pm X^k_{0-}=\pm\theta X^k_{0+}.
	\end{equation}
	Lemma~\ref{LONGLEmDesZPP} states that $(X_{0+}^k)_{\sP}=(X_{0+}^k)_{\sQ}$. Thus the $\sQ$-component of $\theta X^k_{0+}$ is $-(X^k_{0+})_{\sQ}$, which is also equal to the $\sQ$-component of $\sigma(X^k_{0+})$. That  fixes the choice of sign in equation \eqref{LONGEqsigmaXzppmthetaXzp}.
\end{proof}

The following is an immediate corollary of lemma~\ref{LONGLemSigmaXzpBien} and the fact that $\theta$ fixes $\sP$ and $\sK$ while $\sigma$ fixes $\sH$ and $\sQ$.

\begin{corollary}		\label{LONGCorXzpHQPKXzm}
	We have
	\begin{subequations}
		\begin{align}
			(X^k_{0+})_{\sH}	&=(X^k_{0-})_{\sH}\\
			(X^k_{0+})_{\sQ}	&=-(X^k_{0-})_{\sQ}\\
			(X^k_{0+})_{\sP}	&=-(X^k_{0-})_{\sP}\\
			(X^k_{0+})_{\sK}	&=(X^k_{0-})_{\sK}.
		\end{align}
	\end{subequations}
\end{corollary}

\begin{proof}
	Since $\sigma$ acts as the identity on $\sH$ and changes the sign on $\sQ$, we have
	\begin{equation}
		\sigma X^k_{0+}=\sigma\big(  (X^k_{0+})_{\sH}+(X^k_{0+})_{\sQ}  \big)=(X^k_{0+})_{\sH}-(X^k_{0+})_{\sQ},
	\end{equation}
	but lemma~\ref{LONGLemSigmaXzpBien} states that $\sigma X^k_{0+}=X^k_{0-}=(X^k_{0-})_{\sH}+(X^k_{0-})_{\sQ}$. Equating the $\sH$ and $\sQ$-component of these two expressions of $\sigma X^k_{0+}$ brings the two first equalities.

	The two other are proven the same way. We know that $\theta X^k_{0+}=X^k_{0-}$, but
	\begin{equation}
		\theta X^k_{0+}=\theta\big( (X^k_{0+})_{\sP}+(X^k_{0+})_{\sK} \big)=-(X^k_{0+})_{\sP}+(X^k_{0+})_{\sK}.
	\end{equation}
	The two last relations follow.
\end{proof}

%
\subsubsection{An interesting basis of \texorpdfstring{$\sQ$}{Q}}
%
\label{LONGSubSubSecInterestingBasisQ}

Being the tangent space of $AdS$, the space $\sQ$ is of a particular importance. Let us now have a closer look at the vectors that we already mentioned in equations \eqref{LONGAlignPRedDefQQ}:
\begin{subequations}				\label{LONGEqBasQQzi}
	\begin{align}
		q_0&=(X_{++})_{\sK\sQ}\\
		q_1&=J_2\\
		q_2&=-[J_1,q_0]=-(X_{++})_{\sP\sQ}\\
		q_k&=(X^k_{0+})_{\sQ}		&\text{lemma~\ref{LONGLEmDesZPP}}.
	\end{align}
\end{subequations}

By lemma~\ref{LONGLEmDesZPP}, and the discussion about $\mZ(\sK)$ (equation \eqref{LONGEqDecompZKenDeuxSuivantDim}), we can express the elements $q_i$ without explicit references to $\sQ$ itself and each element corresponds to a particular space (once again, the choice of $q_0$ is not that simple in $AdS_3$):

%
\begin{subequations}		\label{LONGEqBasQQziPlusMieux}%
	\begin{align}
		q_0&=(X_{++})_{\mZ(\sK)}&\in \sK\cap\sQ\cap\tilde\sN_2					\\
		q_1&=J_2&\in\sQ\cap\sA\\
		q_2&=-[J_1,q_0]	&\in\sP\cap\sQ\cap\tilde\sN_2	\label{LONGsubEqJUnQZQDeux}		\\
		q_k&=(X^k_{0+})_{\sP}&\in\sP\cap\sQ\cap\tilde\sN_k.
	\end{align}
\end{subequations}
These elements correspond to the expression \eqref{LONGEqDecQEspacesCools}.
%
The compact part isomorphic to $\so(2,l-1)$ is then generated by the elements%

%
\begin{remark}

The space $\mZ(\sK)$ is given by the structure of the compact part of $\so(2,n)$, the elements $(X_{0+}^k)_{\sP}$ are defined from the root space structure of $\so(2,n)$ and the Cartan involution. The elements $J_1$ and $J_2$ are a basis of $\sA$. However, we need to know $\sH$ in order to distinguish $J_1$ from $J_2$ that are respectively generators of $\sA_{\sH}$ and $\sA_{\sQ}$.

Thus the basis
\eqref{LONGEqBasQQziPlusMieux} %
 %
is given in a way almost independent of the choice of $\sH$.
\end{remark}
%
%
\begin{corollary}		\label{LONGCorQdansPetK}
	We have $q_0\in\sK$ and $q_i\in\sP$ if $i\neq 0$. The set $\{ q_0,q_1,\ldots,q_l \}$ is a basis of $\sQ$. Moreover we have $\sQ\cap\tilde\sN_k=\langle q_k\rangle$.
\end{corollary}
%
\begin{proof}
	The first claim is a direct consequence of the expressions \eqref{LONGEqBasQQziPlusMieux}. Linear independence is a direct consequence of the spaces to which each vector belongs. A dimensional counting shows that it is a basis of $\sQ$.
\end{proof}

%
\subsubsection{Magic intertwining elements}
%

The vectors $q_i$ have the property to be intertwined by some elements of $\sH$. Namely, the adjoint action of the elements
%
\begin{subequations}\label{LONGAlignDefMagicIntert}
	\begin{align}
		X_1=p_1&=-[J_2,q_0]	&\in\sP\cap\sH\cap\tilde\sN_2\\
		X_2=s_1&=[J_1,X_1]	&\in\sK\cap\sH\cap\tilde\sN_2\\
		X_k=-r_k&=-[J_2,q_k]	&\in\sK\cap\sH\cap\tilde\sN_k				\label{LONGEqDefXkCommeComm}
	\end{align}
\end{subequations}%
%
intertwines the $q_i$'s in the sense of the following proposition.
%
\begin{proposition}[Intertwining properties]			\label{LONGXUnALaTwistingSuperCool}
	The elements defined by equation \eqref{LONGAlignDefMagicIntert} satisfy
	\begin{multicols}{2}
		\begin{subequations}				\label{LONGEqCalculBBBJUnUnNirme}
			\begin{align}
				\ad(J_1)q_0&=-q_2		\label{LONGEqCalculBBBJUnUnNirmeA}\\
				\ad(J_1)q_2&=-q_0.		\label{LONGEqCalculBBBJUnUnNirmeB}
			\end{align}
		\end{subequations}
		\begin{subequations}				\label{LONGEqSubEqbXUnqZero}
			\begin{align}
				\ad(X_1)q_1&=q_0		\label{LONGSubEqbXZeroqUn}\\
				\ad(X_1)q_0&=q_1,		\label{LONGSubEqbXUnqZero}
			\end{align}
		\end{subequations}
		\begin{subequations}
			\begin{align}
				\ad(X_2)q_2&=q_1		\label{LONGSubEqXdeuxQdeuxa}\\
				\ad(X_2)q_1&=-q_2		\label{LONGSubEqXdeuxQun}
			\end{align}
		\end{subequations}
		\begin{subequations}				\label{LONGEqSubEqbXkqZero}
			\begin{align}
				\ad(X_k)q_k&=-q_1.		\label{LONGSubEqbXIMoinsqZero}\\
				\ad(X_k)q_1&=q_k		\label{LONGSubEqbXkQunQk}
			\end{align}
		\end{subequations}
	\end{multicols}
\end{proposition}


\begin{proof}


	Equation \eqref{LONGEqCalculBBBJUnUnNirmeA} is by definition while equation \eqref{LONGEqCalculBBBJUnUnNirmeB} follows from the first one and the fact that $\ad(J_1)^2$ acts as the identity on $\tilde\sN_2$.

	The equality \eqref{LONGSubEqbXZeroqUn} is a direct consequence of the fact that $\ad(J_2)^2$ is the identity on $\tilde\sN_2$, so that
	\begin{equation}
		[X_1,q_1]=-\big[ [J_2,q_0],q_1 \big]=\ad(J_2)^2q_0=q_0.
	\end{equation}

	For the relation \eqref{LONGSubEqbXUnqZero}, first remark that, since $q_0=(X_{++})_{\sK\sQ}$, we have
	\begin{equation}
		X_1=-(X_{++})_{\sP\sH}
	\end{equation}
	and we have to compute
	\begin{equation}
		[X_1,q_0]=-\big[ (X_{++})_{\sP\sH},(X_{++})_{\sK\sQ} \big]
	\end{equation}
	Using the projections \eqref{LONGEqProjHQPKsigmatheta}, we have
	\begin{equation}
		\begin{aligned}[]
			(X_{++})_{\sP\sH}&=\frac{1}{ 4 }(X_{++}+\sigma X_{++}-\theta X_{++}-\sigma\theta X_{++})\\
			(X_{++})_{\sK\sQ}&=\frac{1}{ 4 }(X_{++}-\sigma X_{++}+\theta X_{++}-\sigma\theta X_{++})
		\end{aligned}
	\end{equation}
	We compute the commutator taking into account the facts that $\sigma$ is an automorphism and that, for example, $[X_{++},\sigma X_{++}]=0$ because $\sigma X_{++}\in\sG_{(+-)}$. What we find is
	\begin{equation}
		\big[ (X_{++})_{\sP\sH},(X_{++})_{\sK\sP} \big]=\frac{1}{ 4 }\frac{ 1 }{2}\Big( [X_{++},\theta X_{++}]-\sigma [X_{++},\theta X_{++}] \Big)=\frac{1}{ 4 }[X_{++},\theta X_{++}]_{\sQ}.
	\end{equation}
	Since $[X_{++},X_{--}]=-4(J_1+J_2)$, we have $[X_1,q_0]=J_2=q_1$ as expected.


	For equation \eqref{LONGSubEqXdeuxQdeuxa} we use the Jacobi relation and the relation \eqref{LONGEqCalculBBBJUnUnNirmeB}.
	\begin{equation}
		\begin{aligned}[]
			[q_2,X_2]&=\big[ q_2,[J_1,p_1] \big]\\
				&=-\big[ J_1,[p_1,q_2] \big]-\big[ p_1,[q_2,J_1] \big]\\
				&=-[p_1,q_0]\\
				&=-q_1
		\end{aligned}
	\end{equation}
	For equation \eqref{LONGSubEqXdeuxQun}, we use the definition of $X_2$, the Jacobi identity and the facts that $[p_1,J_2]=q_0$ and $[J_1,q_0]=-q_2$.

	We pass now to the fourth pair of intertwining relations. By definition, $q_k=(X^k_{0+})_{\sP}$, but taking into account the fact that $J_2\in\sP$ we can decompose the relation $[J_2,X_{0+}]=X_{0+}$ into
	\begin{subequations}
		\begin{align}
			[J_2,(X_{0+})_{\sP}]&=(X_{0+})_{\sK}		\label{LONGEqJdeuxXzpsPsK}\\
			[J_2,(X_{0+})_{\sK}]&=(X_{0+})_{\sP}.
		\end{align}
	\end{subequations}
	Equation \eqref{LONGEqJdeuxXzpsPsK} tolds us that
	\begin{equation}
		X_k=-(X_{0+})_{\sK}.
	\end{equation}
	Now we have to compute $[X_k,q_k]=-\big[ (X_{0+})_{\sK},(X_{0+})_{\sP} \big]$. We know that $[X_{0+},X_{0-}]=-2J_2\in\sP$. Thus corollary~\ref{LONGCorXzpHQPKXzm} brings
	\begin{equation}
		-2J_2=\big[ (X_{0+})_{\sK},(X_{0-})_{\sP} \big]+\big[ (X_{0+})_{\sP},(X_{0-})_{\sK} \big]=-2\big[ (X_{0+})_{\sK},(X_{0+})_{\sP} \big]=2[X_k,q_k],
	\end{equation}
	and the result follows.

	For equation \eqref{LONGSubEqbXkQunQk}, we have to compute $[X_k,q_1]=[J_2,(X_{+0}^k)_{\sK}]$. The $\sP$-component of $[J_2,X_{0+}^k]=X_{0+}^k$ is exactly
	\begin{equation}
		[J_2,(X_{0+}^k)_{\sK}]=(X_{0+}^k)_{\sP}=q_k.
	\end{equation}

\end{proof}

These intertwining relations will be widely used in computing the norm of the vectors $q_i$ in proposition~\ref{LONGPropBaseQOrtho} as well as in some other occasions.

Let us now give a few words about the existence and unicity of these elements. The fact that there exists an element $X_1$ such that $\ad(J_2)X_1=q_0$ comes from the decomposition \eqref{LONGEqsDecopmQXpmpm} and the fact that each $X_{\pm\pm}$ is an eigenvector of $\ad(J_2)$. It is thus sufficient to adapt the signs in order to manage a combination of $X_{++}$, $X_{+-}$, $X_{-+}$ and $X_{--}$ on which the adjoint action of $J_2$ creates $q_0$. However, the fact that this element has in the same time the ``symmetric'' property $\ad(X_1)q_0=q_1$ could seem a miracle. See theorem~\ref{LONGThoAdSqIouZero}.

\begin{lemma}
	An element $X_1$ such that $\ad(X_1)q_1=q_0$ can be chosen in $\sP\cap\sH\cap\tilde\sN_2$. Moreover, this choice is unique up to normalisation.
\end{lemma}

\begin{proof}
	Unicity is nothing else than the fact that $\dim(\sP\cap\sH\cap\tilde\sN_2)=1$. Indeed, since $\sG=\sA\oplus\tilde\sN\oplus\mZ_{\sK}(\sA)$ and $\sA\subset\sP$, we have $\sK\subset\tilde\sN$. Dimension counting shows that $\dim(\tilde\sN_2\cap\sH)=2$ (because $\dim(\tilde\sN_2)=4$ and $q_0,q_2\in\sQ\cap\tilde\sN_2$). As we are looking in $\tilde\sN_2$, we are limited to elements in $\so(2,2)$ (not the higher dimensional slices), so that we can consider $\sK=\so(2)\oplus\so(2)$. One of these two $\so(2)$ factors belongs to $\sH$, so that $\dim(\sK\cap\sH\cap\tilde\sN_2)=1$  and finally $\dim(\sP\cap\sH\cap\tilde\sN_2)=1$.

	Let now $X_1$ be such that $[X_1,q_1]=q_0$. If $X_1$ has a component in $\sQ$, that component has to commute with $q_1$ (if not, the commutator $[X_1,q_1]$ would have a $\sH$-component). So we can redefine $X_1$ in order to have $X_1\in\sH$.

	In the same way, a $\sA$-component has to be $J_1$ (because $J_2\in\sQ$) which commutes with $q_1$. We redefine $X_1$ in order to remove its $J_1$-component. We remove a component in $\tilde\sN_k$ because $[\tilde\sN_2,\tilde\sN_k]\subset\tilde\sN_k$, and a $\sK$-component can also be removed since its commutator with $q_1$ would produce a $\sP$-component. We showed that $X_1\in\sP\cap\sH\cap\tilde\sN_2$.
\end{proof}

\begin{lemma}		\label{LONGLemChoixDeXk}
	An element $X_k$ such that $\ad(X_k)q_1=q_k$ can be chosen in $\sK\cap\sH\cap\tilde\sN_k$.
\end{lemma}

\begin{proof}
	The proof is elementary in tree steps using the fact that $q_1\in\sP\cap\sQ\cap\sA$:
	\begin{enumerate}
		\item
			A $\sP$-component can be annihilated because $[\sP,\sP]\subset\sK$ while $q_k\in\sP$,
		\item
			a $\sQ$-component can be annihilated because $[\sQ,\sQ]\subset\sH$ while $q_k\in\sQ$,
		\item
			if $k'\neq k$, a $\tilde\sN_{k'}$-component can be annihilated because $[\tilde\sN_{k'},\sA]\subset\tilde\sN_{k'}$ while $q_k\in\tilde\sN_k$.
	\end{enumerate}
\end{proof}


%
\subsubsection{Killing form and orthogonality}
%




We \emph{define} the norm of an element in $\sG=\so(2,n)$ as
\begin{equation}	\label{LONGEqDefNormeKillingSix}
	\| X \|=-\frac{ 1 }{2n}B(X,X).
\end{equation}
Notice that $q_0$ belongs to the compact part of $\sG$, so that its Killing norm is negative, so that $\| q_0 \|$ is positive.


\begin{proposition}		\label{LONGPropBaseQOrtho}
	We have $\| q_0 \|=1$ and $\| q_i \|=-1$ ($i\neq 0$). As a consequence, the space $G/H$ is Lorentzian.
\end{proposition}


\begin{proof}
	We begin by computing the norm of $q_1=J_2$. The Killing form $B(J_2,J_2)=\tr\big( \ad(J_2)\circ\ad(J_2) \big)$ is the easiest to compute in the basis $\mZ_{\sK}(\sA)\oplus\sA\oplus\sN\oplus\bar\sN$ of eigenvectors of $J_2$. If we look at the matrix of $\ad(J_2)\circ\ad(J_2)$, we have one $1$ at each of the positions of $X_{++}$, $X_{+-}$, $X_{-+}$ and $X_{--}$. Moreover, for each higher dimensional slice, we get additional $2$ because of $X_{0+}^k$ and $X_{+0}^k$. When one looks at $\so(2,n)$ we have $n-2$ higher dimensional slices, so that
	\begin{equation}
		B(J_2,J_2)=4+2(n-2)=2n.
	\end{equation}

	The result is that $B(q_1,q_1)=6$, so that $\| q_1 \|=-1$.

	We are going to propagate that result to other elements of the basis, using the``magic'' intertwining elements $X_1$, $X_k$ and $J_1$.

	Using left invariance of the Killing form, we find
	\begin{equation}		\label{LONGEqCalculBBBqZeroqUnNirme}
		B(q_0,q_0)=B\big( q_0,-\ad(J_1)q_2 \big)=B\big( \ad(J_1)q_0,q_2 \big)=-B(q_2,q_2),
	\end{equation}
	so that $\| q_0 \| =-\| q_2 \|$.

	Now, the same computation with $X_1$ and $X_k$ instead of $J_1$ show that $\| q_0 \|=-\| q_1 \|$ and $\| q_1 \|=\| q_k \|$.
\end{proof}



\begin{remark}		\label{LONGRemBProdScal}
	Using the fact that the basis $\{ q_i\}$ is orthonormal, we can decompose an element of $\sQ$ by the Killing form. One only has to be careful on the sign: if $X=aq_0+\sum_{i>0}b_iq_i$, we have
	\begin{equation}		\label{LONGEqsabKillProjComp}
		\begin{aligned}[]
			a&=B(X,q_0)\\
			b_i&=-B(X,q_i).
		\end{aligned}
	\end{equation}
\end{remark}


\begin{remark}	\label{LONGRemOrdreNilpotentQ}
	As a consequence, a light like direction reads, up to normalization, $E(w)=q_0+\sum_{i=1}^{l-1}w_iq_i$ with $w\in S^{l-2}$.
\end{remark}

%
\subsubsection{Other properties}
%


\begin{lemma}		\label{LONGLemXZUAHetQ}
	We have $\sigma X_{\alpha\beta}\in\sG_{(\alpha,-\beta)}$. In particular, $X^k_{0+}$ has non vanishing components in $\sH$ and in $\sQ$.
\end{lemma}

\begin{proof}
	If one applies $\sigma$ to the equality $[J_2,X_{\alpha\beta}]=\beta X_{\alpha\beta}$, we see that $\sigma X_{\alpha\beta}$ is an eigenvector of $\ad(J_2)$ with eigenvalue $-\beta$. The same with $\ad(J_1)$ shows that $\sigma X_{\alpha\beta}$ has $+1$ as eigenvalue. Thus $\sigma X_{\alpha\beta}\in\sG_{(\alpha,-\beta)}$.

	In particular, $\sigma X^k_{0+}\neq \pm X^k_{0+}$ so that it does not belongs to $\sH$ nor to $\sQ$.
\end{proof}

Notice that, as corollary, we have
\begin{equation}
	\sigma X_{\alpha,\beta}=\pm X_{\alpha,-\beta}.
\end{equation}


\begin{lemma}				\label{LONGLemSigmaXppEgalXPm}
	We have $(X_{++})_{\sQ}=(X_{+-})_{\sQ}$ or, equivalently, $\sigma X_{++}=-X_{+-}$.
\end{lemma}

\begin{proof}
	Since $q_1=J_2\in\sA$ and $q_k\in\tilde\sN_k$, the $\sQ$-component of $X_{++}$ and $X_{+-}$ are only made of $q_0$ and $q_2$.
	We are	going to prove the following three equalities.
	\begin{enumerate}

		\item\label{LONGItemBpmqDeux}
			$B(X_{+-},q_2)=B(X_{+-},q_0)$
		\item
			$B(X_{++},q_2)=B(X_{++},q_0)$
		\item
			$B(X_{++},q_0)=B(X_{+-},q_0)$
	\end{enumerate}

	The first point is proved using the fact that $q_2=[q_0,J_1]$ and the $\ad$-invariance of the Killing form:
	\begin{equation}
		B(X_{+-},q_2)=-B\big( X_{+-},\ad(J_1)q_0 \big)=B\big( \ad(J_1)X_{+-},q_0 \big)=B(X_{+-},q_0).
	\end{equation}
	One checks the second point in the same way. For the third equality, we know from decomposition \eqref{LONGEqDecompZKenDeuxSuivantDim} that $q_0$ is a multiple of $X_{++}+X_{--}+X_{+-}+X_{-+}$. If the multiple is $\lambda$, $B(X_{++},q_0)=\lambda B(X_{++},X_{--})$ and $B(X_{+-},q_0)=\lambda B(X_{+-},X_{-+})$. Thus we have to prove that the traces of the operators
	\begin{equation}
		\begin{aligned}[]
			\gamma_1&=\ad(X_{++})\circ\ad(X_{--})\\
			\gamma_2&=\ad(X_{+-})\circ\ad(X_{-+})
		\end{aligned}
	\end{equation}
	are the same. That trace is straightforward to compute on the natural basis of $\sG=\mZ_{\sK}(\sA)\oplus\sA\oplus\sN\oplus\bar\sN$.
	%
	The only elements on which $\ad(X_{--})$ is not zero are $\sA$, $X^k_{0+}$, $X^k_{+0}$ and $X^k_{++}$, while for $\ad(X_{-+})$, the only non vanishing elements are $\sA$, $X^k_{0-}$, $X^k_{+0}$ and $X_{+-}$. From equation \eqref{LONGsubeqrXpmpm}, we have $\gamma_1(r_{ij})=\gamma_2(r_{ij})=0$. Using the commutation relations, we find
	\begin{subequations}
		\begin{align}
			\gamma_1J_1&=[X_{++},X_{--}]=-4(J_1+J_2) \\
			\gamma_1J_2&=[X_{++},X_{--}]=-4(J_1+J_2) \\
			\gamma_1X^k_{0+}&=2[X_{++},X^k_{-0}]=-4X^k_{0+} \\
			\gamma_1X^k_{+0}&=-2[X_{++},X^k_{0-}]=-4X^k_{+0} \\
			\gamma_1X_{++}&=[X_{++},4(J_1+J_2)]=-8X_{++}.
		\end{align}
	\end{subequations}
	Thus $\tr(\gamma_1)=-24$. The same computations bring
	\begin{subequations}
		\begin{align}
			\gamma_2J_1&=[X_{+-},X_{-+}]=-4(J_1-J_2)\\
			\gamma_2J_2&=[X_{+-},X_{-+}]=4(J_1-J_2)\\
			\gamma_2X^k_{0-}&=-2[X_{+-},X^k_{-0}]=-4X^k_{0-}\\
			\gamma_2X_{+-}&=[X_{+-},4(J_1-J_2)]=-8X_{+-}\\
			\gamma_2X^k_{+0}&=2[X_{+-},X^k_{0+}]=-2X^k_{+0},
		\end{align}
	\end{subequations}
	and $\tr(\gamma_2)=-24$. Thus we have
	\begin{equation}
		\pr_{\mZ(\sK)}(X_{++})=\pr_{\mZ(\sK)}(X_{+-}).
	\end{equation}
	%
\end{proof}

Notice that the lemma is trivial if we consider that $X_{++}-X_{+-}$ belongs to $\sH$ by definition of $\sH$. From a $AdS$ point of view, we define $AdS=G/H$ and we have to define $H$, so from that point of view, lemma~\ref{LONGLemSigmaXppEgalXPm} is by definition. However, the direction we have in mind is to use the more generic tools as possible. From that point of view, the fact to set $\mZ(\sK)\subset\sQ$ is more intrinsic than to set $X_{++}-X_{+-}\in\sH$.


\begin{proposition}		\label{LONGPropXmpXppqq}
	We have $(X_{++})_{\sQ}=(X_{+-})_{\sQ}=q_0-q_2$.
\end{proposition}

\begin{proof}
	Using the remark~\ref{LONGRemBProdScal}, the three Killing forms computed in the proof of lemma~\ref{LONGLemSigmaXppEgalXPm} are expressed under the form
	\begin{subequations}
		\begin{align}
			(X_{+-})_{q_0}&=-(X_{+-})_{q_2}\\
			(X_{++})_{q_0}&=-(X_{++})_{q_2}\\
			(X_{++})_{q_0}&=(X_{+-})_{q_2}.
		\end{align}
	\end{subequations}
	Consequently, we have $(X_{++})_{\sQ}=\lambda(q_0-q_2)$ and $(X_{+-})_{\sQ}=\lambda(q_0-q_2)$ for a constant $\lambda$ to be fixed. It is fixed to be $1$ by the facts that, by definition, $q_0=(X_{++})_{\sK\sQ}$ and $q_2\in\sP$.
\end{proof}

\begin{lemma}		\label{LONGLemComJDeuxQ}
	We have
	\begin{equation}
		\begin{aligned}[]
			[J_2,q_0]&=(X_{++})_{\sH\sP}\neq 0\\
			[J_2,q_1]&=0\\
			[J_2,q_2]&=(X_{++})_{\sH\sK}\neq 0\\
			[J_2,q_k]&=(X^k_{0+})_{\sH}\neq 0\\
			[J_1,p_1]&=-(X_{++})_{\sK\sH}\neq 0
		\end{aligned}
	\end{equation}
	where $k\geq 3$.
\end{lemma}

\begin{proof}
	Using the fact that $J_2\in\sQ\cap\sP$ and that $X_{++}$ has non vanishing components ``everywhere'' (corollary~\ref{LONGCorHPHKQPQKXuu}), we have
	\begin{equation}
		\begin{aligned}[]
			[J_2,q_0]&=[J_2,(X_{++})_{\sK\sQ}]=(X_{++})_{\sP\sH}\neq 0\\
			[J_2,q_2]&=[J_2,(X_{++})_{\sP\sQ}]=(X_{++})_{\sK\sH}\neq 0\\
			[J_1,p_1]&=[J_1,(X_{++})_{\sP\sH}]=-(X_{++})_{\sK\sH}\neq 0\\
			[J_2,q_k]&=[J_2,(X_{0+}^k)_{\sQ}]=(X^k_{0+})_{\sH}\neq 0		&\text{lemma~\ref{LONGLemXZUAHetQ}}\\
		\end{aligned}
	\end{equation}
\end{proof}

\begin{lemma}		\label{LONGLemNonHXaz}
	If $\alpha\neq 0$, then $X_{\alpha 0}^k\in\sH$.
\end{lemma}

\begin{proof}
	The element $\pr_{\sQ}X^k_{\alpha 0}$ is a combination of $q_i$. Since $\ad(J_2)\pr_{\sQ}X^k_{\alpha 0}=0$, we must have $(X^k_{\alpha 0})_{\sQ}=\lambda J_2$ by lemma~\ref{LONGLemComJDeuxQ}. Using the fact that $J_1\in\sH$, the $\sQ$-component of the equality $[J_1,X^k_{\alpha 0}]=\alpha X^k_{\alpha 0}$ becomes
	\begin{equation}
		[J_1,\lambda J_2]=\alpha\lambda J_2.
	\end{equation}
	The left-hand side is obviously zero, so that $\lambda=0$ which proves that $X^k_{\alpha 0}\in\sH$.
\end{proof}

Applying successively the projections \eqref{LONGEqProjHQPKsigmatheta}, and lemma~\ref{LONGLemSigmaXppEgalXPm}, we write the basis elements of $\sQ$ in the decomposition $\sG=\sG_0\oplus\sN\oplus\bar\sN$:
\begin{subequations}			\label{LONGEqsDecopmQXpmpm}
	\begin{align}
		q_0&=\frac{1}{ 4 }(X_{++}+X_{+-}+X_{-+}+X_{--}),		\\
		q_1&=J_2,							\\
		q_2&=\frac{1}{ 4 }(-X_{++}-X_{+-}+X_{-+}+X_{--}),		\\
		q_k&=\frac{ 1 }{2}(X^{k}_{0+}-X^{k}_{0-})
	\end{align}
\end{subequations}
with $k\geq 3$. Notice that none of them has component in $\mZ_{\sK(\sA)}$.

These decompositions allow us to compute the commutators $[q_i,q_j]$ and $[q_i,J_p]$. Instead of listing here every commutation relations, we will only write the ones we use when we need them.


\begin{lemma}		\label{LONGLemQzQdeuxJun}
	We have $[q_0,q_2]=-J_1$.
\end{lemma}

\begin{proof}
	The proof is exactly the same as the one of equation \eqref{LONGSubEqbXUnqZero} in lemma~\ref{LONGXUnALaTwistingSuperCool}. Here we use
	\begin{equation}
		(X_{++})_{\sP\sQ}=\frac{1}{ 4 }\big( X_{++}-\sigma X_{++}-\theta X_{++}+\sigma\theta X_{++} \big)
	\end{equation}
	and we find
	\begin{equation}
		[q_0,q_2]=-\big[ (X_{++})_{\sK \sQ},(X_{++})_{\sP\sQ} \big]=-\frac{1}{ 4 }[\theta X_{++},X_{++}]_{\sH}=-J_1.
	\end{equation}
\end{proof}

\begin{lemma}
We have
	\begin{equation}		\label{LONGEqXunQdeuxcommutent}
		[X_1,q_2]=[X_1,q_k]=0
	\end{equation}
	for $k\geq 3$.
\end{lemma}
\begin{proof}
	The proof is elementary:
	\begin{equation}
		\begin{aligned}[]
			[X_1,q_2]&\in[\sP\cap\sH\cap\tilde\sN_2,\sP\cap\sQ\cap\tilde\sN_2]\subset\sK\cap\sQ\cap\sA=\{ 0 \}\\
			[X_1,q_k]&\in[\sP\cap\sH\cap\tilde\sN_2,\sP\cap\sQ\cap\tilde\sN_k]\subset\sK\cap\sQ\cap\tilde\sN_k=\{ 0 \}.
		\end{aligned}
	\end{equation}
\end{proof}

The following is a first step in the proof of theorem~\ref{LONGThoAdESqqq}.
\begin{corollary}			\label{LONGCorAdQUncarreqi}
	We have $\ad(J_1)|_{\tilde\sN_2}^2=\ad(J_2)|_{\tilde\sN_2}^2=\id$ and $\ad(q_1)^2q_i=q_i$.
\end{corollary}

\begin{proof}
	The action of $\ad(q_1)^2$ is to change two times the sign of the components $X_{\alpha -}$. Thus $\ad(q_1)^2=\id$ on $\tilde\sN_2$. The result is now proved for $i=0,1,2$. For the higher dimensions, we use the fact that $J_2=q_1$ and we find
	\begin{equation}
		q_k=[X_k,q_1]=-\big[ [q_1,X_k],q_1 \big]=\ad(q_1)^2q_k
	\end{equation}
	as claimed.

	Moreover, the elements of $\tilde\sN_2$ are build of elements of the form $X_{\pm\pm}$, so that $\ad(J_1)^2$ changes at most twice the sign.

\end{proof}

\begin{lemma}		\label{LONGLemXkqzerozero}\label{LONGLemXkJunzero}
	We have
	\begin{subequations}
		\begin{align}
			[X_k,q_0]=[X_k,J_1]=[X_k,q_2]=0\\
			[J_1,q_k]=0.		\label{LONGEqJUnqkzero}
		\end{align}
	\end{subequations}
\end{lemma}

\begin{proof}
	The first claim is proved in a very standard way:
	\begin{equation}
		[X_k,q_0]\in[\sK\cap\sH\cap\tilde\sN_k,\sK\cap\sQ\cap\tilde\sN_2]\subset\sK\cap\sQ\cap\tilde\sN_k=\{ 0 \}.
	\end{equation}

	For the second commutator, we use the Jacobi identity and the definition $X_k=-[J_2,q_k]$:
	\begin{equation}		\label{LONGEqJ1J2qkzero}
		\big[ J_1,[J_2,q_k] \big]=-\big[ J_2,[q_k,J_1] \big]-\big[ q_k,\underbrace{[J_1,J_2]}_{=0} \big],
	\end{equation}
	while
	\begin{equation}
		[q_k,J_1]\in[\sP\cap\sQ\cap\tilde\sN_k,\sP\cap\sH\cap\sA]\subset\sK\cap\sQ\cap\tilde\sN_k=\{ 0 \}.
	\end{equation}
	That proves \eqref{LONGEqJUnqkzero} in the same time.

	For the third commutator, remark that, since $q_2=[q_0,J_1]$, we have
	\begin{equation}
		[X_k,q_2]=-\big[ q_0,[J_1,X_k] \big]-\big[ J_1,[X_k,q_0] \big].
	\end{equation}
	which is zero by the two first claims.
\end{proof}

\begin{proposition}			\label{LONGEtOrdreDeux}
	If $E$ is nilpotent in $\sQ$, then $\ad(E)^3=0$.
\end{proposition}

\begin{proof}
	If $E_1$ is a nilpotent element of $\sQ$, then every nilpotent elements in $\sQ$ are of the form $\lambda\Ad(k)E_1$ for some $k\in K$ and $\lambda\in\eR$\cite{These}. It is then sufficient to prove that one of them is of order two. The element
	\begin{equation}		\label{LONGEqDecqzmoinsqDeux}
		q_0-q_2=\frac{ 1 }{2}(X_{++}+X_{+-}),
	\end{equation}
	is obviously of order two because the eigenvalue for $\ad(J_1)$ increases by one unit at each iteration of $\ad(q_0-q_2)$.
\end{proof}

\begin{lemma}
	We have $[J_1,q_k]=0$.
\end{lemma}

\begin{proof}
	The proof is standard:
	\begin{equation}
		[J_1,q_k]\in[\sH\cap\sP\cap\sA,\sQ\cap\sP\cap\tilde\sN_k]\subset\sQ\cap\sK\cap\tilde\sN_k=\{ 0 \}.
	\end{equation}
\end{proof}

The following theorem, which relies on the preceding lemmas, will be central in computing the Killing form which appears in the characterization of theorem~\ref{LONGThosSequivJzero}.
%
\begin{theorem}			\label{LONGThoAdESqqq}
	We have
	\begin{equation}
		\ad(q_i)^2q_j=q_j
	\end{equation}
	if $i\neq j$ and $i\neq 0$. If $i=0$, we have
	\begin{equation}
		\ad(q_0)^2q_j=-q_j.
	\end{equation}
\end{theorem}


\begin{proof}
	The case $i=1$ is already done in corollary~\ref{LONGCorAdQUncarreqi}.

	We propagate that result to the other $\ad(q_i)^2$ with the intertwining elements $J_1$, $X_1$ and $X_k$.

	Let us compute $\ad(q_0)^2q_i=\ad\big([X_1,q_1]\big)^2q_i$ using twice the Jacobi identity and the properties of $X_1$ (in order to be more readable, we write $XY$ for $[X,Y]$)
	\begin{equation}		\label{LONGEqAdqZsqqi}
		\begin{aligned}[]
			\ad(q_0)^2q_i	&=	(X_1q_1)\Big( (X_1q_1)q_i \Big)\\
					&=	-(X_1q_1)\Big( (q_1q_I)X_1+(q_iX_1)q_1 \Big)\\
					&=(q_1q_i)\big( X_1(X_1q_1) \big)+(q_iX_1)\big( q_1(X_1q_1) \big)\\
					&\quad + X_1\big( (X_1q_1)(q_1q_i) \big)+q_1\big( (X_1q_1)(q_iX_1) \big)\\
					&=(q_1q_i)q_1-\ad(X_1)^2q_i+X_1\big( q_0(q_1q_i) \big)+q_1\big( q_0(q_iX_1) \big).
		\end{aligned}
	\end{equation}

	If $i=1$, the only non vanishing term is $-\ad(X_1)^2q_1=-q_1$.  Thus $\ad(q_0)^2q_1=-q_1$.

	If $i=2$, the relation \eqref{LONGEqXunQdeuxcommutent} annihilates the second and fourth terms while $[q_1,q_2]$ commutes with $q_0$ because $q_0\in\mZ(\sK)$. We are thus left with the term $-q_2$, and $\ad(q_0)^2q_2=-q_2$.

	If $i=k\geq 3$, we find
	\begin{equation}
		\ad(q_0)^2q_k=-\ad(q_1)^2q_k-\ad(X_1)^2q_k+X_1\big( q_0(q_1q_k) \big)+q_1\big( q_0(q_kX_1) \big).
	\end{equation}
	Since $[q_1,q_k]\in\sK$, it commutes with $q_0$. Using the fact that $[X_1,q_k]=0$, we get $\ad(q_0)^2q_k=-q_k$.

	Let us perform the same computations  as in \eqref{LONGEqAdqZsqqi} with $q_k$ ($k\geq 3$) instead of $q_0$ and $X_k$ (equations \eqref{LONGEqSubEqbXkqZero}) instead of $X_1$. What we get is
	\begin{equation}
		\ad(q_k)^2q_i=\ad(q_1)^2q_i-\ad(X_k)^2q_i+X_k\big( q_k(q_1q_i) \big)+q_1\big( q_k(q_iX_k) \big).
	\end{equation}

	If we set $i=0$, taking into account the commutator $[X_k,q_0]=0$, we have
	\begin{equation}		\label{LONGEqkdeuxqzerointer}
		\ad(q_k)^2q_0=\ad(q_1)^2q_0+X_k\big( q_k(q_1q_0) \big).
	\end{equation}
	As already proved, the first term is $q_0$. Now,
	\begin{equation}
		\big[ q_k,[q_1,q_0] \big]\in\sK\cap\sQ\cap\tilde\sN_k=\{ 0 \},
	\end{equation}
	so that the second term in \eqref{LONGEqkdeuxqzerointer} is zero. Thus we proved that $\ad(q_k)^2q_0=q_0$.

	If we set $i=1$, taking into account the relations \eqref{LONGEqSubEqbXkqZero}, we find
	\begin{equation}
		\ad(q_k)^2q_1=-\ad(X_k)^2q_1+q_1\big( q_k(q_1X_k) \big)=q_1.
	\end{equation}

	If we set $i=2$ and using the fact that $[X_k,q_2]=0$, we find
	\begin{equation}
		\ad(q_k)^2q_2=q_2-q_k\big( X_k(q_1q_2) \big).
	\end{equation}
	Using once again the Jacobi identity inside the big parenthesis, we find $2q_2-\ad(q_k)^2q_2$. This proves that $\ad(q_k)^2q_2=q_2$.


	We turn now our attention to $\ad(q_2)^2q_i$. We perform the same computation, using the intertwining property \eqref{LONGEqCalculBBBJUnUnNirme} of $J_1$. What we get is
	\begin{equation}		\label{LONGEqadqqqDeuxIGene}
		\ad(q_2)^2q_i=(J_1q_i)(q_0q_2)-\ad(q_0)^2q_i+q_0\big( q_2(J_1q_i) \big)+J_1\big( q_2(q_iq_0) \big).
	\end{equation}

	If we set $i=1$, we use the already proved property $\ad(q_0)^2q_1=-q_1$, and we obtain
	\begin{equation}
		\ad(q_2)^2q_1=(J_1q_1)(q_0q_2)+q_1+q_0\big( q_2(J_1q_1) \big)+J_1\big( q_2(q_1q_0) \big).
	\end{equation}
	We claim that all of these terms are zero except of $q_1$. First, $\big[ q_2,[q_1,k_k] \big]\in\big[ \tilde\sN_2,[\sA,\tilde\sN_2] \big]\subset\sA$.	Thus the last term vanishes .The commutator $[J_1,q_1]$ vanishes because $q_1=J_2$. We are done with $\ad(q_2)^2q_1=q_1$.

	If we set $i=k$ ($k\geq 3$) in \eqref{LONGEqadqqqDeuxIGene}, we use $\ad(q_0)^2q_k=-q_k$ and what we find is
	\begin{equation}		\label{LONGEqadqqqdeuxkk}
		\ad(q_2)^2q_k=(J_1q_k)(q_0q_2)+q_k+q_0\big( q_2(J_1q_k) \big)+J_1\big( q_2(q_kq_0) \big).
	\end{equation}
	We already know that $[J_1,q_k]=0$. We have $\big[ q_2,[q_k,q_0] \big]=0$ because
	\begin{equation}
		\begin{aligned}[]
			\big[ q_2,[q_k,q_0] \big]&\in\big[ \sP\cap\sQ\cap\tilde\sN_2,[\sP\cap\sQ\cap\tilde\sN_k,\sK\cap\sQ\cap\tilde\sN_2] \big]\\
						&\subset[\sP\cap\sQ\cap\tilde\sN_2,\sP\cap\sH\cap\tilde\sN_k]\\
						&\subset\sK\cap\sQ\cap\tilde\sN_k=\{ 0 \}.
		\end{aligned}
	\end{equation}
	The remaining terms in \eqref{LONGEqadqqqdeuxkk} are $\ad(q_2)^2q_k=q_k$.

	In order to compute $\ad(q_2)^2q_0$, we write $q_0=\ad(X_1)q_1$. Using twice the Jacobi identity, we get
	\begin{equation}
		\ad(q_2)^2q_0=X_1\big( (q_1q_2)q_2 \big)+q_1\big( (X_1q_2)q_2 \big)+(q_1q_2)(X_1q_2)+(X_1q_2)(q_2q_1).
	\end{equation}
	Using the fact that $[X_1,q_2]=0$, we are left with
	\begin{equation}
		\ad(q_2)^2q_0=X_1\big( \ad(q_2)^2q_1 \big)=[X_1,q_1]=q_0
	\end{equation}
	as desired.
\end{proof}


%
\subsection{A convenient basis for the root spaces and computations}
%
\label{LONGSubSecMOreConvBasisBlbla}
%

The most natural basis of $\tilde\sN_2$ is
\begin{equation}
	\tilde\sN_2=\langle X_{++},X_{+-},X_{-+},X_{--}\rangle,
\end{equation}
but the multiple commutators of these elements with $q_0$ reveal to require some work.

We provide in this section an other basis for $\tilde\sN$ that corresponds to the decomposition $\sK\oplus\sP$. Since $q_0$ is central in $\sK$, the exponential $e^{xq_0}X$ is trivial when $X\in\sK$ and, since $q_0\in\sK$, the commutator $[q_0,X]$ remains in $\sP$ when $X\in\sP$.

Here is the new basis:
%
\begin{equation}		\label{LONGEqSuperBaseeB}
	\eB=\{J_1,J_2, q_0,q_2,p_1,s_1,q_k,p_k,r_k,s_k \}_{k=3,\ldots,l-1}.
\end{equation}
%
where
\begin{subequations}		\label{LONGSubEqsBaseSuperTsNTrois}\label{LONGEqDecompqprskXX}
	\begin{align}
		q_0&		&&=\frac{1}{ 4 }(X_{++}+X_{+-}+X_{-+}+X_{--})	&\in\sK\cap\sQ\cap\tilde\sN_2\\
		q_2&=[q_0,J_1]	&&=\frac{1}{ 4 }(-X_{++}-X_{+-}+X_{-+}+X_{--})	&\in\sP\cap\sQ\cap\tilde\sN_2\\
		p_1&=[q_0,q_1]	&&=\frac{1}{ 4 }(-X_{++}+X_{+-}-X_{-+}+X_{--})	&\in\sP\cap\sH\cap\tilde\sN_2	\label{LONGEqqzqupuDef}\\
		s_1&=[J_1,p_1]	&&=\frac{1}{ 4 }(-X_{++}+X_{+-}+X_{-+}-X_{--})	&\in\sK\cap\sH\cap\tilde\sN_2\\
		q_k&		&&=\frac{ 1 }{2}(X_{0+}^k-X_{0-}^k)	&\in\sP\cap\sQ\cap\tilde\sN_k	\\
		p_k&=[q_0,q_k]	&&=\frac{ 1 }{ 2 }(X_{-0}^k-X_{+0}^k)	&\in\sP\cap\sH\cap\tilde\sN_k	\\
		r_k&=[J_2,q_k]	&&=\frac{ 1 }{2}(X^k_{0+}+X^k_{0-})	&\in\sK\cap\sH\cap\tilde\sN_k	\\
		s_k&=[J_1,p_k]	&&=-\frac{ 1 }{2}(X^k_{-0}+X^k_{+0})	&\in\sK\cap\sH\cap\tilde\sN_k	\\
		r_{ij}&=[q_i,q_j]&&					&\in\sK\cap\sH\cap\sG_0.
	\end{align}
and
	\begin{align}
		J_1&\in\sP\cap\sH\cap\sA\\
		q_1=J_2&\in\sP\cap\sQ\cap\sA
	\end{align}
\end{subequations}
Notice that the elements $p_1$ and $s_1$ are non vanishing by lemma~\ref{LONGLemComJDeuxQ}. We have
\begin{subequations}
	\begin{align}
		\sP\cap\sQ\cap\tilde\sN_2&=\langle q_2\rangle	&	\sK\cap\sQ\cap\tilde\sN_2&=\langle q_0\rangle\\
		\sP\cap\sQ\cap\tilde\sN_k&=\langle q_k\rangle	&	\sK\cap\sQ\cap\tilde\sN_k&=\emptyset\\
		\sP\cap\sQ\cap\sA&=\langle J_2\rangle	&	\sK\cap\sQ\cap\sA&= \emptyset\\
		\sP\cap\sH\cap\tilde\sN_2&=\langle p_1\rangle	&	\sK\cap\sH\cap\tilde\sN_2&=\langle s_1\rangle\\
		\sP\cap\sH\cap\tilde\sN_k&=\langle p_k\rangle	&	\sK\cap\sH\cap\tilde\sN_k&=\langle r_k,s_k\rangle\\
		\sP\cap\sH\cap\sA&=\langle J_1\rangle	&	\sK\cap\sH\cap\sA&=\emptyset
	\end{align}
\end{subequations}

The decomposition of $\tilde\sN_2$ into $\sK\oplus\sP$ is
\begin{equation}		\label{LONGEqDecomptsNTroisKP}
	\tilde\sN_2=\langle q_0,s_1\rangle\oplus \langle q_1,p_1\rangle.
\end{equation}


The decomposition of $\tilde\sN_k$ into $\sK\oplus\sP$ is
\begin{equation}		\label{LONGEqDecomptsNkKP}
	\tilde\sN_k =\langle r_k,s_k,r_{ij}\rangle_{k,i,j\geq 3}\oplus \langle q_k,p_k\rangle.
\end{equation}


We are now going to compute all the Killing form and commutators in this basis.
\begin{proposition}		\label{LONGPropBprsk}
	We have
	\begin{equation}
		\begin{aligned}[]
			B(p_k,p_k)&=-B(q_0,q_0)\\
			B(r_k,r_k)&=B(q_0,q_0)\\
			B(s_k,s_k)&=B(q_0,q_0),
		\end{aligned}
	\end{equation}
	and then
	\begin{equation}	\label{LONGeqNormInHigherDimensionalSlices}
		-\| p_k \|^2=\| r_k \|^2=\| s_k \|^2=1.
	\end{equation}
\end{proposition}

\begin{proof}
	Using the definitions \eqref{LONGEqDecompqprskXX} and the theorem~\ref{LONGThoAdESqqq}, we have
	\begin{subequations}		\label{LONGSubEqsBpprrssk}
	\begin{equation}
			B(p_k,p_k)=B\big( \ad(q_k)q_0,\ad(q_k)q_0 \big)
					=-B\big( \ad(q_k)^2q_0,q_0 \big)
					=-B(q_0,q_0),
	\end{equation}
	and
	\begin{equation}
		B(r_k,r_k)=B\big( \ad(q_1)q_k,\ad(q_1)q_k \big)=-B(q_k,q_k)=B(q_0,q_0).
	\end{equation}
	and
	\begin{equation}
		B(s_k,s_k)=-B\big( \ad(J_1)^2p_k,p_k \big)=-B(p_k,p_k)=B(q_0,q_0).
	\end{equation}
	\end{subequations}
\end{proof}
Notice that we are not surprised by the positivity of the norms of $r_k$ and $s_k$ because they belong to the compact part of the algebra.

\begin{proposition}
	The Killing norm in the space $\mZ_{\sK}(\sA)$ are given by
	\begin{equation}
		B(r_{ij},r_{kl})=\begin{cases}
			B(q_0,q_0)	&	\text{if }\{ i,j \}=\{ k,l \}\\
			0	&	 \text{otherwise}.
		\end{cases}
	\end{equation}
\end{proposition}

\begin{proof}
	First, we have
	\begin{equation}
		\begin{aligned}[]
			B(r_{ij},r_{ij})&=B\big( [q_i,q_j],[q_i,q_j] \big)\\
					&=-B\big( \ad(q_i)^2q_j,q_j \big)\\
					&=-B(q_j,q_j)\\
					&=B(q_0,q_0).
		\end{aligned}
	\end{equation}
	For the mixed case we have
	\begin{equation}
			B(r_{ij},r_{ik})=-B\big( \ad(q_i)^2q_j,q_k \big)=0.
	\end{equation}
	We suppose now that $i$, $j$, $k$ and $l$ are four different numbers. The action of $\ad(r_{kl})$ on $\sA$ is zero because $r_{kl}\in\mZ(\sA)$. From \eqref{LONGEqComsRRN}, the action of $\ad(r_{ij})\circ\ad(r_{kl})$ on $\sN$ is zero. Since the elements $r_{ij}$ satisfy the algebra of $\so(n)$, we have $\ad(r_{ij})\circ\ad(r_{kl})r_{mn}=0$ when $i$, $j$, $k$ and $l$ are four different numbers. Finally we have
	\begin{equation}
		B(r_{ij},r_{kl})=0.
	\end{equation}
\end{proof}

\begin{lemma}		\label{LONGLempunjdeuxqzero}
	We have $[p_1,J_2]=q_0$.
\end{lemma}

\begin{proof}
	The lemma comes from theorem~\ref{LONGThoAdESqqq} because
	\begin{equation}
		[p_1,J_2]=-[q_1,p_1]=\ad(q_1)^2q_0=q_0.
	\end{equation}
\end{proof}

\begin{lemma}				 \label{LONGlemJDeuxqDeuxsUn}
	We have $s_1=[J_2,q_2]$.
\end{lemma}

\begin{proof}
	We use the definition $p_1=[q_0,q_1]$ and the Jacobi identity:
	\begin{equation}
		[J_1,p_1]=\big[ J_1,[q_0,q_1] \big]=-\big[ q_0,[q_1,J_1] \big]-\big[ q_1,[J_1,q_0] \big].
	\end{equation}
	The first terms vanishes because $q_1\in\sA$ while $[J_1,q_0]=-q_2$ by definition.
\end{proof}

\begin{proposition}		\label{LONGPropBJpsun}
	We have
	\begin{equation}
		\begin{aligned}[]
			B(J_1,J_1)&=-B(q_0,q_0)\\
			B(p_1,p_1)&=-B(q_0,q_0)\\
			B(s_1,s_1)&=B(q_0,q_0),
		\end{aligned}
	\end{equation}
	and then
	\begin{equation}
		-\| J_1 \|^2=-\| p_1 \|^2=\| s_1 \|^2=1.
	\end{equation}
\end{proposition}

\begin{proof}
	Using $J_1=[q_0,q_2]$ (lemma~\ref{LONGLemQzQdeuxJun}) we find
	\begin{equation}
		B(J_1,J_1)=B\big( \ad(q_2)q_0,\ad(q_2)q_0 \big)=-B\big( \ad(q_2)^2q_0,q_0 \big)=-B(q_0,q_0).
	\end{equation}
	In much the same way, using the definition of $p_1$ and $s_1=[q_1,q_2]$ (lemma~\ref{LONGlemJDeuxqDeuxsUn}), we find $B(p_1,p_1)=B(q_1,q_1)$ and $B(s_1,s_1)=-B(q_2,q_2)$.

\end{proof}

\begin{lemma}		\label{LONGLemJunrkzero}
	We have $[J_1,r_k]=0$.
\end{lemma}

\begin{proof}
	Using the definition of $r_k$ and the Jacobi identity,
	\begin{equation}
		[J_1,r_k]= \big[ J_1,[J_2,q_k] \big]=-\big[ J_2,[q_k,J_1] \big]-\big[ q_k,[J_1,J_2] \big]=0
	\end{equation}
	because of equation \eqref{LONGEqJUnqkzero} and the fact that $\sA$ is abelian.
\end{proof}

Now, the Killing norms of the basis $\eB$ can be computed.
\begin{theorem}		\label{LONGThoBaisXXorthoigher}
	The basis
	\begin{equation}
		\eB=\{ q_0,q_2,p_1,s_1,q_k,p_k,r_k,s_k,J_1,J_2 \}.
	\end{equation}
	given by the definitions \eqref{LONGEqSuperBaseeB}  is orthonormal and
	\begin{equation}
		\begin{aligned}[]
			\| J_1 \|^2=\| q_1 \|^2=\| q_2 \|^2=\| p_1 \|^2=\| q_k \|^2=\| p_k \|^2&=-1\\
			\| r_{ij} \|=\| q_0 \|^2=\| s_1 \|^2=\| r_k \|^2=\| s_k \|^2&=1
			\end{aligned}
	\end{equation}
\end{theorem}

\begin{proof}
	The norms are given by the propositions~\ref{LONGPropBprsk},~\ref{LONGPropBJpsun}, and~\ref{LONGPropBaseQOrtho}.

	For the orthogonality, we know that $\sP\perp\sK$, $\sQ\perp\sH$ (from general theory) as well as $\tilde\sN_2\perp\tilde\sN_k$, $\tilde\sN_2\perp\sA$ and $\tilde\sN_k\perp\sA$ (equations \eqref{LONGEqtsnkperprtsnkp}, \eqref{LONGEqAperpNTrois}, \eqref{LONGEqAperpNk} and \eqref{LONGEqNTtroisperpNk}). Thus, among the elements of the basis $\eB$, the two only ones that could not orthogonal are $r_k$ and $s_k$. However, we have $B(r_k,s_k)=0$ because
	\begin{equation}
		B(r_k,s_k)=B\big( r_k,\ad(J_1)p_k \big)=-B\big( \ad(J_1)r_k,p_k \big)=0
	\end{equation}
	by lemma~\ref{LONGLemJunrkzero}.

\end{proof}


It turns out that we are able to compute all the commutators using the following techniques
\begin{enumerate}

	\item
		the orthonormality of the basis, theorem~\ref{LONGThoBaisXXorthoigher} among with the $\ad$-invariance of the Killing form
	\item
		the Jacobi identity
	\item
		the theorem~\ref{LONGThoAdESqqq} and the commutators of lemmas~\ref{LONGLempunjdeuxqzero},~\ref{LONGlemJDeuxqDeuxsUn} and~\ref{LONGLemJunrkzero}.

\end{enumerate}
Computing all the commutators that way is quite long and very few interesting. The interesting point is that it is possible. You can immediately jump to subsection~\ref{LONGSubSecSomeExpo}.

\begin{enumerate}
	\item$\ad(q_0)J_1=q_2$. Definition.
	\item$\ad(q_0)J_2=p_1$. Definition.
	\item$\ad(p_1)J_1=-s_1$. Definition.
	\item$\ad(q_0)q_k=p_k$\label{LONGItemComqzpk}. Definition.
	\item$\ad(q_k)J_2=-r_k$. Definition.
	\item$\ad(p_k)J_1=-s_k$\label{LONGItemCompkJun}. Definition.
	\item$\ad(p_1)J_2=q_0$\label{LONGItemCompunJdeux}. Lemma~\ref{LONGLempunjdeuxqzero}.
	\item$\ad(q_2)J_2=-s_1$. Lemma~\ref{LONGlemJDeuxqDeuxsUn}.
	\item$\ad(J_1)r_k=0$\label{LONGItemComJunrk}. Lemma~\ref{LONGLemJunrkzero}.

	\item$\ad(q_0)p_1=-J_2$\label{LONGItemComqzpun}. We have $\ad(q_0)p_1=\ad(q_0)^2q_1=-q_1$.
	\item$\ad(q_0)s_1=0$. By the usual techniques, we get $[q_0,s_1]\in\sK\cap\sQ\cap\sA=\{ 0 \}$. The following few are obtained in the same way.
	\item$\ad(q_0)r_k=0$\label{LONGItemComqzrk}.
	\item$\ad(q_0)s_k=0$.
	\item$\ad(q_2)p_1=0$.
	\item$\ad(q_k)J_1=0$.
	\item$\ad(q_2)p_k=0$.
	\item$\ad(q_2)p_k=0$.
	\item$\ad(p_1)q_k=0$\label{LONGItemCompunqk}.
	\item$\ad(J_2)p_k=0$.
	\item$\ad(q_2)s_1=-J_2$. We know that $[q_2,s_1]\in\sP\cap\sQ\cap\sA=\langle J_2\rangle$. Thus $[q_0,s_1]$ is a multiple of $J_2$. The coefficient is given by$B\big( [q_2,s_1],J_2 \big)/B(J_2,J_2)$. Using the $\ad$-invariance of the Killing form, we are left to compute $B(s_1,\ad(q_2)J_2)$. Lemma~\ref{LONGlemJDeuxqDeuxsUn} shows then that the coefficient we are searching if $B(s_1,s_1)/B(J_2,J_2)=1$.
	\item\label{LONGItemComqdeuxqk}$\ad(q_2)q_k=-s_k$. The spaces show that $[q_2,q_k]\in\sK\cap\sH\cap\tilde\sN_k=\langle r_k,s_k\rangle$. Thus we have to check the two possible components. First, $B\big( [q_2,q_k],r_k \big)=-B\big( q_2,\ad(q_k)^2J_2 \big)=0$ by theorem~\ref{LONGThoAdESqqq}. For the second, we use the definition of $s_k$ and the Jacobi identity:
		\begin{equation}
			\begin{aligned}[]
				B\big( [q_2,q_k],s_k \big)&=B\big( [q_2,q_k],\ad(J_1)[q_0,q_k] \big)\\
				&=B\big( [q_2,q_k],-\ad(q_0)\underbrace{[q_k,J_1]}_{=0}-\ad(q_k)\underbrace{[J_1,q_0]}_{=-q_2} \big)\\
				&=-B\big( \ad(q_2)q_k,\ad(q_2)q_k \big)\\
				&=B\big( q_k,\ad(q_2)^2q_k \big)\\
				&=B(q_k,q_k).
			\end{aligned}
		\end{equation}
		Finally, what we have is
		\begin{equation}
			[q_2,q_k]=\frac{ B(q_2,q_2) }{ B(s_k,s_k) }s_k=-s_k.
		\end{equation}
	\item$\ad(q_2)r_k=0$. From the spaces, $[q_2,r_k]\in\langle q_k\rangle$, but
		\begin{equation}
			B\big( \ad(q_2)r_k,q_k \big)=-B\big( r_k,[q_2,q_k] \big)=B(r_k,s_k)=0.
		\end{equation}
	\item$\ad(q_2)s_k=-q_k$. From the spaces, $[q_2,s_k]\in\sP\cap\sQ\cap\tilde\sN_k=\langle q_k\rangle$, so we compute
		\begin{equation}
			B\big( [q_2,s_k],q_k \big)=-B(s_k,[q_2,s_k])=B(s_k,s_k),
		\end{equation}
		and
		\begin{equation}
			[q_2,s_k]=\frac{ B(s_k,s_k) }{ B(q_k,q_k) }q_k=-q_k.
		\end{equation}
	\item$\ad(q_2)r_k=0$. From the spaces, $[q_k,q_2]\in\sP\cap\sQ\cap\tilde\sN_k=\langle q_k\rangle$. Thus we compute
		\begin{equation}
			B\big( [q_2,r_k],q_k \big)=-B\big( r_k,[q_2,q_k] \big)=B(r_k,s_k)=0
		\end{equation}
		where we used the item~\ref{LONGItemComqdeuxqk}.
	\item\label{LONGItemComjunpun}$\ad(J_1)p_1=s_1$. From the spaces, $[J_1,p_1]\in\langle s_1\rangle$. We have
		\begin{equation}
			B\big( [J_1,p_1],s_1 \big)=-B\big( p_1,[J_1,s_1] \big)=-B\big( p_1,\ad(J_1)^2p_1 \big)=-B(p_1,p_1),
		\end{equation}
		thus
		\begin{equation}
			[J_1,p_1]=-\frac{ B(p_1,p_1) }{ B(s_1,s_1) }s_1=s_1.
		\end{equation}
	\item\label{LONGItemComskqk}$\ad(s_k)q_k=-q_2$. From the spaces, $[s_k,q_k]\in\langle J_2,q_2\rangle$. Thus we have two Killing forms to compute. The first is
		\begin{equation}
			B\big( [s_k,q_k],J_2 \big)=B\big( s_k,[q_k,J_2] \big)=B(s_k,r_k)=0
		\end{equation}
		were we used the definition of $r_k$. The second is
		\begin{equation}
			B\big( [s_k,q_k],q_2 \big)=B\big( s_k,[q_k,q_2] \big)=B(s_k,s_k).
		\end{equation}
		where we used item~\ref{LONGItemComqdeuxqk}. Thus we have
		\begin{equation}
			[s_k,q_k]=\frac{ B(s_k,s_k) }{ B(q_2,q_2) }q_2=-q_2
		\end{equation}
	\item$\ad(s_k)J_2=0$. From the spaces, $[s_k,J_2]\in\langle q_k\rangle$, but
		\begin{equation}
			B\big( \ad(s_k)J_2,q_k \big)=-B\big( J_2,[s_k,q_k] \big)=-B(J_2,q_2)=0
		\end{equation}
		where we used item~\ref{LONGItemComskqk}.
	\item$\ad(q_2)s_k=-q_k$. From the spaces, $[q_2,s_k]\in\langle q_k\rangle$. We have
		\begin{equation}
			B\big( \ad(q_2)s_k,q_k \big)=-B\big( s_k,\ad(q_2)q_k \big)=B(s_k,s_k)
		\end{equation}
		where we used item~\ref{LONGItemComqdeuxqk}. Thus
		\begin{equation}
			[q_2,s_k]=\frac{ B(s_k,s_k) }{ B(q_k,q_k) }q_k=-q_k.
		\end{equation}
	\item$\ad(p_1)s_1=-J_1$. From the spaces, $[p_1,s_1]\in\langle J_1\rangle$. We have
		\begin{equation}
			B\big( [p_1,s_1],J_1 \big)=-B\big( s_1,[p_1,J_1] \big)=B(s_1,s_1),
		\end{equation}
		where we used item~\ref{LONGItemComjunpun}. Thus,
		\begin{equation}
			[p_1,s_1]=\frac{ B(s_1,s_1) }{ B(J_1,J_1) }J_1=-J_1.
		\end{equation}
	\item$\ad(p_1)r_k=p_k$\label{LONGItemCompunrk}. We use the definition of $r_k$ and Jacobi:
		\begin{equation}
			[p_1,r_k]=\big[ p_1,[J_2,q_k] \big]=-\big[ J_2,\underbrace{[q_k,p_1]}_{=0} \big]-\big[ q_k,\underbrace{[p_1,J_2]}_{=q_0} \big]=-[q_k,q_0]=p_k.
		\end{equation}
		where we used items~\ref{LONGItemCompunqk},~\ref{LONGItemCompunJdeux} and~\ref{LONGItemComqzpk}.
	\item$\ad(q_k)J_2=-r_k$\label{LONGItemComkJdeux}. From the spaces, $[q_k,J_2]\in\langle r_k,s_k\rangle$. First, we have
		\begin{equation}
			B\big( [q_k,q_2],s_k \big)=-B\big( J_2,\underbrace{[q_k,s_k]}_{=-q_2} \big)=0
		\end{equation}
		where we used item~\ref{LONGItemComskqk}. Secondly we have
		\begin{equation}
			B\big( [q_k,J_2],r_k \big)=B\big( q_k,\underbrace{[J_2,r_k]}_{\ad(J_2)^2q_k} \big)=B(q_k,q_k)
		\end{equation}
		Thus
		\begin{equation}
			[q_k,J_2]=\frac{ B(q_k,q_k) }{ B(r_k,r_k) }r_k=-r_k.
		\end{equation}
	\item$\ad(p_1)p_k=r_k$\label{LONGItemCompunpk}. We use the definition of $p_k$ and the Jacobi identity:
		\begin{equation}
			\begin{aligned}[]
				[p_1,p_k]&=-\big[ q_0,\underbrace{[q_k,p_1]}_{=0} \big]-\big[ q_k,\underbrace{[p_1,q_0]}_{=J_2} \big]\\
					&=-[q_k,J_2]\\
					&=r_k
			\end{aligned}
		\end{equation}
		where we used items~\ref{LONGItemCompunqk},~\ref{LONGItemComqzpun} and~\ref{LONGItemComkJdeux}.
	\item$\ad(p_1)s_k=0$. From the spaces, $[p_1,s_k]\in\langle p_k\rangle$. We have
		\begin{equation}
			B\big( [p_1,s_k],p_k \big)=-B\big( s_k,\underbrace{[p_1,p_k]}_{=0} \big)=0
		\end{equation}
		where we used the item~\ref{LONGItemCompunpk}.
	\item$\ad(s_1)q_k=0$. From the spaces, $[s_1,q_k]\in\langle q_k\rangle$. We have
		\begin{equation}
			B\big( [s_1,q_k],q_k \big)=B\big( s_1,[q_k,q_k] \big)=0.
		\end{equation}
	\item$\ad(s_1)p_k=0$\label{LONGItemComsunpk}. From the spaces, $[s_1,p_k]\in\langle p_k\rangle$. We have
		\begin{equation}
			B\big( [s_1,p_k],p_k \big)=B\big( s_1,[p_k,p_k] \big)=0.
		\end{equation}
	\item$\ad(s_1)r_k=s_k$. From the spaces, $[s_1,q_k]\in\langle r_k,s_k\rangle$. Using Jacobi,
		\begin{equation}
			\begin{aligned}[]
				[s_1,r_k]&=\big[ [J_1,p_1],r_k \big]\\
				&=-\big[ \underbrace{[p_1,r_k]}_{=p_k},J_1 \big]-\big[ \underbrace{[r_k,J_1]}_{=0},p_1 \big]\\
					&=-[p_k,J_1]\\
					&=s_k
			\end{aligned}
		\end{equation}
		where we used items~\ref{LONGItemCompunrk},~\ref{LONGItemComJunrk} and the definition of $s_k$.
	\item$\ad(s_1)J_1=-p_1$\label{LONGItemComsunJun}. We have $[s_1,J_1]=-\ad(J_1)^2p_1=-p_1$ by theorem~\ref{LONGThoAdESqqq}.
	\item$\ad(s_1)s_k=-r_k$\label{LONGItemComsunsk}. We use the definition of $s_k$ and Jacobi:
		\begin{equation}
			\begin{aligned}[]
				[s_1,s_k]&=\big[ s_1,[J_1,p_k] \big]\\
				&=-\big[ J_1,\underbrace{[p_k,s_1]}_{=0} \big]-\big[ p_k,\underbrace{[s_1,J_1]}_{=-p_1} \big]\\
					&=[p_k,p_1]\\
					&=-r_k
			\end{aligned}
		\end{equation}
		where we used items~\ref{LONGItemComsunpk},~\ref{LONGItemComsunJun} and~\ref{LONGItemCompunpk}.
	\item$\ad(J_2)s_k=0$. From the spaces, $[J_2,s_k]\in\langle q_k\rangle$. We have
		\begin{equation}
			B\big( [J_2,s_k],q_k \big)=B\big( J_2,\underbrace{[s_k,q_k]}_{=-q_2} \big)=0
		\end{equation}
		where we used item~\ref{LONGItemComskqk}.
	\item$\ad(q_k)p_k=-q_0$. Using the definition of $p_k$ and the theorem~\ref{LONGThoAdESqqq},
		\begin{equation}
			[q_k,p_k]=\big[ q_k[q_0,q_k] \big]=-\ad(q_k)^2q_0=-q_0.
		\end{equation}
	\item$\ad(q_k)r_k=-J_2$\label{LONGItemComqkrk}. Using the definition of $r_k$ and theorem~\ref{LONGThoAdESqqq}, we have
		\begin{equation}
			\big[ q_k,[J_2,q_k] \big]=-\ad(q_k)^2J_2=-J_2.
		\end{equation}
	\item$\ad(p_k)r_k=-p_1$. Using the definition of $r_k$ and Jacobi,
		\begin{equation}
			\begin{aligned}[]
				[p_k,r_k]&=-\big[ pk,[q_0,q_k] \big]\\
				&=\big[ q_0,\underbrace{[q_k,r_k]}_{=-J_2} \big]+\big[ q_k,\underbrace{[r_k,q_0]}_{=0} \big]\\
					&=[J_2,q_0]\\
					&=-p_1
			\end{aligned}
		\end{equation}
		where we used the items~\ref{LONGItemComqkrk},~\ref{LONGItemComqzrk} and the definition of $p_1$.
	\item$\ad(p_k)s_k=-J_1$. The spaces show that $[p_k,s_k]\in\langle J_1,p_1\rangle$. We have
		\begin{equation}
			B\big( [p_k,s_k],J_1 \big)=-B\big( s_k,\underbrace{[p_k,s_1]}_{=-s_k} \big)=B(s_k,s_k)
		\end{equation}
		and
		\begin{equation}
			B\big( [p_k,s_k],p_1 \big)=-B\big( s_k,\underbrace{[p_k,p_1]}_{=-r_k} \big)=0
		\end{equation}
		where we used the items~\ref{LONGItemCompkJun} and~\ref{LONGItemCompunpk}. Thus
		\begin{equation}
			[p_k,s_k]=\frac{ B(s_k,s_k) }{ B(J_1,J_1) }J_1=-J_1.
		\end{equation}
	\item$\ad(r_k)s_k=s_1$. From the spaces, $[r_k,s_k]\in\langle s_1\rangle$. We have
		\begin{equation}
			B\big( [r_k,s_k],s_1 \big)=B\big( r_k,\underbrace{[s_k,s_1]}_{=r_k} \big)=B(r_k,r_k)
		\end{equation}
		where we used item~\ref{LONGItemComsunsk}. Thus
		\begin{equation}
			[r_k,s_k]=\frac{ B(r_k,r_k) }{ B(s_1,s_1) }s_1=s_1.
		\end{equation}
	\item$\ad(J_1)q_2=-q_0$. Using the definition of $q_2$ and theorem~\ref{LONGThoAdESqqq},
		\begin{equation}
			[J_1,q_2]=-\ad(J_1)^2q_0=-q_0.
		\end{equation}
	\item$\ad(J_1)s_k=p_k$. From the spaces, $[J_1,s_k]\in\langle p_k\rangle$. We have
		\begin{equation}
			B\big( [J_1,s_k],p_k \big)=-B\big( s_k,\underbrace{[J_1,p_k]}_{=s_k} \big)=-B(s_k,s_k)
		\end{equation}
		where we used the definition of $s_k$.
	\item$\ad(J_2)p_1=-q_0$\label{LONGItemComJdeuxpun}. Using the definition of $p_1$ and theorem~\ref{LONGThoAdESqqq}, $[J_2,p_1]=-\ad(q_1)^2q_0=-q_0$.
	\item$\ad(J_1)s_1=q_2$. We use the definition of $s_1$ and Jacobi:
		\begin{equation}
			\begin{aligned}[]
				[J_2,s_1]&=\big[ J_2,[J_1,p_1] \big]\\
				&=-\big[ J_1,\underbrace{[p_1,J_2]}_{=q_0} \big]-\big[ p_1,\underbrace{[J_2,q_1]}_{=0} \big]\\
				&=-[J_1,q_0]\\
				&=q_2
			\end{aligned}
		\end{equation}
		where we used item~\ref{LONGItemComJdeuxpun} and the definition of $q_2$.
	\item$\ad(J_2)r_k=q_k$. Using the definition of $r_k$ and theorem~\ref{LONGThoAdESqqq}, $[J_2,r_k]=\ad(q_1)^2q_k=q_k$.
\end{enumerate}

\begin{proposition}
	We have $\sH=[\sQ,\sQ]$.
\end{proposition}

\begin{proof}
	The inclusion $[\sQ,\sQ]\subset\sH$ is by construction. Now every elements in the basis \eqref{LONGAlignPremDefHH} can be expressed in terms of commutators in $\sQ$ because
	\begin{subequations}
		\begin{align}
			J_1&=[q_0,q_2]\\
			s_k&=[q_k,q_2]\\
			s_1&=[J_2,q_2]	\label{LONGEqJdqdsu}
		\end{align}
	\end{subequations}
\end{proof}

%
\subsection{Properties of the basis}
%


The basis $\eB$ is motivated by the fact that $\ad(q_0)^2q_k=-q_k$, so that $ e^{\ad(xq_0)}$ is easy to compute on $q_k$ and $p_k$. Moreover, $r_k$ and $s_k$ belong to $\sK$, so that $[q_0,r_k]=[q_0,s_k]=0$. The drawback of that decomposition is that the basis elements do not belong to $\sN$ or $\bar\sN$ while it will be useful to have basis elements in $\sN$ and $\bar\sN$, among other for theorem~\ref{LONGThoOrbitesOuverttes}.

At a certain point, we are going to compute the exponentials $ e^{\ad(xq_0)}X$ when $X$ runs over $\sN$ and $\bar\sN$. We are going to extensively use the commutation relations listed in \eqref{LONGsubEqsGenPySO}, \eqref{LONGSubEqsPlusPresPySO} and \eqref{LONGSubEqsThethaPySO}. A particular attention will be devoted to the projection over $\sQ$ which will be central in determining the open and closed orbits of $AN$ in $G/H$.

\begin{lemma}		\label{LONGLemDecomptsNDanseB}
	The decomposition of $\tilde\sN_2$ with respect to $\eB$ is
	\begin{subequations}		\label{LONGSubeqsDecompXqps}
		\begin{align}
			X_{++}&=q_0-q_2-p_1-s_1\\
			X_{+-}&=q_0-q_2+p_1+s_1
		\end{align}
	\end{subequations}
	and the decomposition of $\tilde\sN_k$ with respect to $\eB$ is
	\begin{subequations}
		\begin{align}
			X_{0+}^k & = q_k+r_k	&    X_{0-}^k&=-q_k+r_k		\label{LONGsubEqqrkXzpegal}	\\
			X_{+0}^k & = -p_k-s_k	&	X_{-0}^k & = p_k-s_k
		\end{align}
	\end{subequations}

\end{lemma}

\begin{proof}


	Using known commutator and the fact that $[\ad(J_1),\ad(J_2)]=0$ on $\tilde\sN_2$, we find the following commutators:
	\begin{subequations}
		\begin{align}
			[J_1,q_0]&=-q_2	&&	&[J_2,q_0]&=-p_1			\label{LONGsubEqJunqzJdeuxqzmoinspun}\\
			[J_1,q_2]&=-q_0	&&	&[J_2,q_2]&=s_1\label{LONGsubEqJunqzJdeuxqzmoinspdeux}\\
			[J_1,p_1]&=s_1	&&	&[J_2,p_1]&=-q_0\\
			[J_1,s_1]&=p_1	&&	&[J_2,s_1]&=q_2.
		\end{align}
	\end{subequations}
	From these properties, we deduce that $q_0-q_2-p_1-s_1$ is proportional to $X_{++}$. Since, by definition, $q_0$ is the $\sK\sQ$-component of $X_{++}$, the proportionality factor is $1$. We also know that  $X_{+-}$ is proportional to $q_0-q_2+p_1+s1$. Since $q_0-q_2=(X_{++})_{\sQ}=(X_{+-})_{\sQ}$ (proposition~\ref{LONGPropXmpXppqq}), the proportionality coefficient is $1$. The relations \eqref{LONGSubeqsDecompXqps} are now proved.

	For the elements of $\tilde\sN_k$, the commutation relations give
	\begin{subequations}
		\begin{align}
			[J_1,q_k+r_k]&=0		&[J_1,p_k-s_k]&=s_k-p_k\\
			[J_2,q_k+r_k]&=q_k+r_k		&[J_2,p_k-s_k]&=0
		\end{align}
	\end{subequations}
	so that
	\begin{equation}
		\begin{aligned}[]
			q_k+r_k&\propto X_{0+}^k\in\sN\\
			s_k+p_k&\propto X^k_{+0}\in\sN\\
			p_k-s_k&\propto X^k_{-0}\in\bar\sN
		\end{aligned}
	\end{equation}

	We have $r_k=[J_2,q_k]\in\sK\cap\sH$, so that the $\sP$-component of $q_k+r_k$ is $q_k$. But $q_k=(X_{0+}^k)_{\sP}$ is the $\sP$-component of $X_{0+}^k$. The proportionality between $q_k+r_k$ and $X_{0+}^k$ together with the equality of their $\sP$-component provide the equality \eqref{LONGsubEqqrkXzpegal}.

	For the two other, let us suppose that
	\begin{subequations}
		\begin{align}
			X_{+0}^k&=a(p_k+s_k)\\
			X_{-0}^k&=b(p_k-s_k).
		\end{align}
	\end{subequations}
	In this case, we have
	\begin{equation}
			(X_{+0}^k)_{\sP}=\frac{ 1 }{2}(X_{+0}^k-\theta X_{+0}^k)
				=\frac{ 1 }{2}\big( (a-b)p_k+(a+b)s_k \big),
	\end{equation}
	so that $a=-b$ because $s_k\in \sK$. Now let us look at the $\sK\sQ$-component of the equality $[X_{+0}^k,X_{0+}^k]=-X_{++}$ taking into account the fact that $X_{+0}^k\in\sH$ and $(X_{0+}^k)_{\sK\sQ}=0$. What we have is 
    \begin{equation}
        \big[ (X_{+0}^k)_{\sP\sH},(X_{0+}^k)_{\sP\sQ} \big]=-q_0,
    \end{equation}
    but $(X_{0+}^k)_{\sP}=q_k$ and $(X_{+0}^k)_{\sP}=ap_k$, so that $[ap_k,q_k]=-q_0$. If we replace $p_k$ by its definition $[q_0,q_k]$, we get
	\begin{equation}
		a\big[ [q_0,q_k],q_k \big]=a\ad(q_k)^2q_0=-q_0,
	\end{equation}
	so that $a=-1$.

	The last point comes from $X_{0-}^k=\theta X_{0+}^k$.

\end{proof}
Notice that this result was already obvious from the decompositions given in \eqref{LONGEqDecompqprskXX}.


%
\subsection{Some exponentials}
%
\label{LONGSubSecSomeExpo}

It will be important to compute the element $ e^{\ad(xq_0)}X$ when $X$ runs over the vectors of $\eB$.
%
The action of $e^{xq_0}$ on $\sA$ is
\begin{subequations}			\label{LONGSubEqsAdxqzJJ}
	\begin{align}
		e^{\ad(xq_0)}J_1=\cos(x)J_1+\sin(x)q_2\\
		e^{\ad(xq_0)}J_2=\sin(x)p_1+\cos(x)q_1,
	\end{align}
\end{subequations}
on $\tilde\sN_2$ we have
\begin{subequations}					\label{LONGSubEqsexpxzqpsdzuu}
	\begin{align}
		e^{\ad(xq_0)}q_2&=\cos(x)q_2-\sin(x)J_1\\
		e^{\ad(xq_0)}q_0&=q_0\\
		e^{\ad(xq_0)}p_1&=\cos(x)p_1-\sin(x)q_1 		\label{LONGEqAdqzpUn}\\
		e^{\ad(xq_0)}s_1&=s_1,
	\end{align}
\end{subequations}
and the action on $\tilde\sN_k$ is
\begin{subequations}		\label{LONGEqExpAdqkpk}
	\begin{align}
		e^{\ad(xq_0)}q_k&=\cos(x)q_k+\sin(x)p_k\\
		e^{\ad(xq_0)}p_k&=\cos(x)p_k-\sin(x)q_k\\
		e^{\ad(xq_0)}p_k&=\cos(x)p_k-\sin(x)q_k.
	\end{align}
\end{subequations}

Combining with lemma~\ref{LONGLemDecomptsNDanseB},
\begin{subequations}			\label{LONGEqExpoQzSurNk}
	\begin{align}
		e^{\ad(xq_0)}X_{0+}^k=e^{\ad(xq_0)}(q_k+r_k)&=r_k+\cos(x)q_k+\sin(x)p_k		\label{LONGEqExpoQzSurNka}\\
		e^{\ad(xq_0)}X_{+0}^k=-e^{\ad(xq_0)}(s_k+p_k)&=-s_k-\cos(x)p_k+\sin(x)q_k.\label{LONGEqExpoQzSurNkb}
	\end{align}
\end{subequations}
The same way,
\begin{subequations}			\label{LONGEqExpoQzN}
	\begin{align}
		e^{\ad(xq_0)}X_{++}&=q_0+\sin(x)q_1-\cos(x)q_2+\sin(x)J_1-\cos(x)p_1\\
		e^{\ad(xq_0)}X_{+-}&=q_0-\sin(x)q_1-\cos(x)q_2+\sin(x)J_1+\cos(x)p_1.
	\end{align}
\end{subequations}
The projections on $\sQ$ of all these combinations are immediate.

%
\subsection{Classification of the basis by the spaces}
%

The basis \eqref{LONGEqDecompqprskXX} allows to generalize the theorem~\ref{LONGThoAdESqqq}.

By very definition, we have $\ad(\sP)^2\sP\subset\sP$, $\ad(\sP)^2\sK\subset\sK$ and the same for the couple $(\sH,\sQ)$. The relations \eqref{LONGEqsCommWithtsNDeuxkA} say that the same is true with the triple $(\tilde\sN_2,\tilde\sN_k,\sA)$, i.e. $\ad(\sA)^2\tilde\sN_k\subset\tilde\sN_k$ and $\tilde\sN_k(\tilde\sN_2)\subset\tilde\sN_2$.

\begin{theorem}		\label{LONGThoAdSqIouZero}
	The basis $\eB$ has the property to be stable under the commutators: $[X,Y]\in\{0,\pm\eB\}$ when $X,Y\in\eB$. Moreover, we have
	\begin{equation}		\label{LONGEqadXsqYzYmY}
		\ad(X)^2Y=\begin{cases}
			0	&	\text{if }\ad(X)Y=0\\
			Y	&	\text{if }X\in\sP\\
			-Y	&	 \text{if }X\in\sK.
		\end{cases}
	\end{equation}
\end{theorem}

\begin{proof}
	This theorem can be immediately checked using the commutators. It is however instructive to see that equation \eqref{LONGEqadXsqYzYmY} can be checked from few considerations. First, remark that, if we look at the decompositions $\sG=\sQ\oplus\sH=\sP\oplus\sK=\mZ_{\sK}(\sA)\oplus\sA\oplus\tilde\sN_2\oplus\tilde\sN_k$, the commutation relations \eqref{LONGEqsCommWithtsNDeuxkA}, we find
	\begin{equation}
		\ad(\tilde\sN_2)\circ\ad(\tilde\sN_2)\colon
		\begin{cases}
			\tilde\sN_2\to\sA\to\tilde\sN_2\\
			\tilde\sN_k\to\tilde\sN_k\to\tilde\sN_k\\
			\sA\to\tilde\sN_2\to\sA,
		\end{cases}
	\end{equation}
	\begin{equation}
		\ad(\tilde\sN_k)\circ\ad(\tilde\sN_k)\colon
		\begin{cases}
			\tilde\sN_2\to\tilde\sN_k\to\sA\oplus\tilde\sN_2\\
			\tilde\sN_k\to(\sA\oplus\tilde\sN_2)\to\sA\oplus\tilde\sN_k\\
			\sA\to\tilde\sN_k\to\sA\oplus\tilde\sN_2,
		\end{cases}
	\end{equation}
	\begin{equation}
		\ad(\sA)\circ\ad(\sA)\colon
		\begin{cases}
			\tilde\sN_2\to\tilde\sN_2\to\tilde\sN_2\\
			\tilde\sN_k\to\tilde\sN_k\to\tilde\sN_k\\
			\sA\to 0.
		\end{cases}
	\end{equation}
	Thus we have $\ad(X)^2Y=\lambda Y$ whenever we are not in the cases $(X,Y)\in(\tilde\sN_k,\tilde\sN_2)$ and $(X,Y)\in(\tilde\sN_k,\tilde\sN_k)$. We should check that these cases cannot bring a $\sA$-component. We should also check that, since $\sK\cap\sH\cap\tilde\sN_k$, is two dimensional, $\ad(X)^2r_k$ has no $s_k$-component as well as $\ad(X)^2s_k$ has no $r_k$-component.

	Let us suppose that the spaces fit. We have $\ad(X)^2Y=\lambda Y$ and we still have to check the values of $\lambda$. Since $\eB$ is orthonormal, we have
	\begin{equation}
		\lambda=\frac{ B\big( \ad(X)^2Y,Y \big) }{ B(Y,Y) }=-\frac{ B\big( \ad(X)Y,\ad(X)Y \big) }{ B(Y,Y) }.
	\end{equation}
	First, remark that this is zero if and only if $\ad(X)Y=0$. Now, there are $4$ possibilities following that $X,Y$ belong to $\sP$ or $\sK$ because we know that $B|_{\sK}<0$ and $B|_{\sP}>0$. The result is that $\lambda$ is positive when $X\in\sP$ and negative when $X\in\sK$. Now, the fact that $\eB$ is stable under the commutators implies that $\ad(X)^2Y\in\{ 0,\pm \eB \}$, so that $\lambda\in\{ 0,1,-1 \}$.
\end{proof}

\input{128_BTZ}
\input{129_DiracAdS}
% This is part of (almost) Everything I know in mathematics and physics
% Copyright (c) 2013-2014, 2020
%   Laurent Claessens
% See the file fdl-1.3.txt for copying conditions.

\section{Dirac operator on \texorpdfstring{$AdS_{3}$}{AdS3}}  \label{PgDiracAdSTrois}
%--------------------------------------------------------------

The definition is
\[
  AdS_{3}=\frac{ \SO(2,2) }{ \SO(1,2) },
\]
and the group which acts is the $AN$ of $\SO(2,2)$. The Lie algebra is given by
\[
\begin{split}
  \sA&=\{ J_{1},J_{2} \}\\
  \sN&=\{ M,L \}
\end{split}
\]
which has dimension $4$. So there is a stabiliser. One can prove that for the open orbit of $u=\begin{pmatrix}
0&1\\-1&0
\end{pmatrix}$, the stabiliser is $\{  e^{aJ_{2}} \}$, i.e.
\begin{equation}
  [ e^{aJ_{2}}u]=[u].
\end{equation}
For the spin group, we find
\[
  \Spin(2,1)\simeq SL_2^*(\eR),
\]
the group of $2\times 2$ matrices with determinant equals to $\pm 1$ (cf \cite{Michelson}). Let us recall that the isomorphism $AdS_{3}\simeq SL(2,\eR)$ is given by
\[ 
  SL(2,\eR)=\begin{pmatrix}
t+x&y-u\\
y+u&t-x
\end{pmatrix}
\]
with $u^{2}+t^{2}-x^{2}-y^{2}=1$. For sake of simplicity, we denote $SL(2,\eR)$ by $G$. It is explained in \cite{Clement} that the map
\begin{equation}
\begin{aligned}
 \psi\colon (G\times G)\times AdS_{3}&\to AdS_3 \\
(g_{1},g_{2})x&= g_{1}xg_{2}^{-1}
\end{aligned}
\end{equation}
provides a local isomorphism $G\times G\simeq O(2,2)$. Moreover we have locally:
\[
  \frac{ G\times G }{ \eZ_{2} }\simeq \SO(2,2).
\]
At each point $x\in AdS_3$, we have an isomorphism
\[
  \SO(2,2)_{x}\simeq \SO(2,1)
\]
where $\SO(2,2)_{x}$ is the stabiliser of $x$ in $\SO(2,2)$. So we define the isomorphism
\[
  \chi_{x}\colon \Spin(2,1)\to \SO(2,2)_{x}
\]
which is a double covering. If $d\psi\colon \mG\oplus\mG\to \mathfrak{so}(2,2)$ is the isomorphism of \cite{Clement}, we define $\psi\colon G\times G\to \SO(2,2)$ by
\[
  \psi( e^{X})= e^{d\psi X},
\]
which is a good definition because the exponential is surjective on $G\times G$. For each $x\in AdS_3$, we consider the isomorphism
\[
  \phi_{x}\colon \SO(2,1)\to \SO(2,2)_{x}
\]
such that $\phi_{x}\big( \SO(2,1) \big)=\SO(2,2)_{x}$.

We define $\chi(s)_i\colon \Spin(2,1)\to \SO(2,2)$ by
\[
  \chi(s)=\chi(s)_1v\chi(s)_2.
\]
The choice of $\chi(s)_i$ is not unique. So we define the action of $\Spin(2,1)$ on $G\times G$ by
\begin{equation}
(g,h)\cdot s=\big( \chi(s)_1g,\chi(s)_2^{-1}h \big).
\end{equation}
Therefore we have
\[
\begin{split}
  \psi\big( (g,h)\cdot s \big)x&=\chi(s)_1gxh^{-1}\chi(s)_2\\
        &=\chi(s)\big( gxh^{-1} \big)\\
        &=\chi(s)\big(\psi(g,h)x\big).
\end{split}
\]
\subsection{Spin structure on \texorpdfstring{$AdS_3$}{AdS3}}
%+++++++++++++++++++++++++++++++++++

From previous considerations, the first choice should be
\[
  P=\frac{ AN }{ S }\times\Spin(2,1),
\]
but it is easy to remark that $\sR'=\{ J_{1},M,L \}$ is a Lie algebra. So we use the corresponding Lie group $R$ instead of the homogeneous space $AN/S$ (these two are isomorphic). Thus the choice is
\begin{equation}
P=R'\times\Spin(2,1),
\end{equation}
with the projection $\pi\colon P\to AdS_3$,
\[
  \pi\big( r',s \big)=\left[ r'\begin{pmatrix}
0&1\\-1&0
\end{pmatrix} \right]
\]
 We consider
\begin{equation}
\begin{aligned}
 \theta\colon R'&\to \mU=Ro \\
r'&\mapsto ro=\left[ r'\begin{pmatrix}
0&1\\-1&0
\end{pmatrix} \right].
\end{aligned}
\end{equation}
The projection $\pi\colon P\to \mU$ reads $\pi=\theta\circ\pr_{1}$,
\[
  \pi\big( r',s \big)=[ro].
\]
This definition works because for all $a$, there exists a $h\in H$ such that
\[
   e^{aJ_{2}}\begin{pmatrix}
0&1\\-1&0
\end{pmatrix}=
\begin{pmatrix}
0&1\\-1&0
\end{pmatrix}h,
\]
from construction of $S$. Then we look at the following:
\[
\xymatrix{%
   \Spin(2,1) \ar@{~>}[r]&R'\times\Spin(2,1)\ar[rr]^{\displaystyle\varphi}\ar[rd]_{\displaystyle\pi}&&\SO\big( \mU \big)\ar[ld]&\SO(2,1)\ar@{~>}[l]   \\
  &&\mU
}
\]
The action of $\Spin(2,1)$ on $P$ is
\[
  \big( r',s' \big)\cdot s=\big( r',s's \big).
\]
In order to define $\varphi$, we consider the isomorphism
\[
  \phi_{x}\colon \SO(2,1)\to \SO(2,2)_{x}
\]
between $\SO(2,1)$ and the stabiliser of $x$ in $\SO(2,2)$. This extends to an automorphism
\[
  \phi_{x}\colon \SO(2,2)\to \SO(2,2),
\]

\begin{probleme}
    I'm not sure of that extension, but we do not use it here.
\end{probleme}


and we define the action of $\SO(2,1)$ on $\SO\big( AdS_3 \big)$ by
\begin{equation}
\{ b_{i} \}_{x}\cdot g=\{ \phi_{x}(g)b_{i} \}_{x}.
\end{equation}
Then we define
\begin{equation}
  \varphi\big( r',s \big)=\{ \phi_{\pi r'}\big( \chi(s) \big)b_{i} \}_{\pi r'}
\end{equation}
 if $\{ b_{i} \}$ is a reference basis at $\pi[r]$. So this construction implies the choice of a section of $\SO\big( AdS_3 \big)$. Now, using the fact that both $\phi_{x}$ and $\chi$ are morphisms, we find
\begin{equation}
\begin{split}
\varphi\big( (r',s)\cdot s' \big)&=\left\{ \phi_{\pi r'}\big( \chi(ss') \big) \right\}_{\pi r'}\\
            &=\left\{ \phi_{\pi r'}\big( \chi(s) \big)b_{i} \right\}\cdot\chi(s)\\
            &=\varphi(r',s)\cdot\chi(s').
\end{split}
\end{equation}
This proves that the construction gives a spin structure.

\subsection{Connection on the spinor bundle}
%-----------------------------------------------

A left invariant vector on $\mU$ is of the form
\[
  \tilde X_{xo}=\Dsdd{ x e^{tX}o }{t}{0}
        =\Dsdd{ \pi\big( x e^{tX},s \big) }{t}{0}
\]
for any $s\in\Spin(2,1)$. On $AdS_3$ (in fact on $\mU$) we consider the left invariant vector field
\begin{equation}
X^{\sharp}_{[x]}=\Dsdd{ [x e^{tX}] }{t}{0}
\end{equation}
which leads us to consider the following field on $P$:
\begin{equation}
\xi_{X}\big( r',s \big)=\Dsdd{ r' e^{tX},s }{t}{0}\in T_{( r',s )}P.
\end{equation}
This defines a field which projects to the left invariant field on $\mU$:
\begin{equation}   \label{eq_xiXprojXsharp}
d\pi\xi_{X}(r',s)=X^{\sharp}_{r'}.
\end{equation}


\begin{lemma}
On the general vector
\begin{equation}   \label{eq_gebevectSig}
  \Sigma=\Dsdd{ r'(t),s(t) }{t}{0},
\end{equation}
the formula
\begin{equation}
\alpha_{(r',s_{0})}\Sigma=-\Dsdd{ s_{0}^{-1}s(t) }{t}{0}\in\spin(2,1)
\end{equation}
where $s_{0}=s(0)$ defines a connection form.

\end{lemma}

\begin{proof}
First let $A\in\spin(2,1)$ and
\[
  A^*_{\xi}=\Dsdd{ \xi\cdot e^{-tA} }{t}{0}.
\]
We have
\[
   \alpha\big( A^*_{(r'),s_{0}} \big)=\alpha \Dsdd{ (r',s_{0})\cdot e^{-tA}  }{t}{0}
        =\alpha\Dsdd{ (r',s_{0} e^{-tA}) }{t}{0}
        =-\Dsdd{ s_{0}^{-1}s_{0} e^{-tA}} {t}{0}
        =A.
\]
Now we take back the vector $\Sigma$ of equation \eqref{eq_gebevectSig}, an element $a\in\Spin(2,1)$ and we compute
\[
\begin{split}
  (dR_{a}\alpha)\Sigma&=\alpha\Dsdd{ \big( r'(t),s(t) \big)\cdot a }{t}{0}\\
        &=\alpha\Dsdd{ \big( r'(t),s(t)a \big) }{t}{0}\\
        &=-\Dsdd{ a^{-1}s(0)^{-1}s(t)a }{t}{0}\\
        &=-\Ad(a^{-1})\Dsdd{ s(0)^{-1}s(t) }{t}{0}\\
        &=\Ad(a^{-1})\alpha(\Sigma).
\end{split}
\]

\end{proof}
Thus that is a connection. This is however not the spin connection. Let $\beta$ be the Levi-Civita connection on the frame bundle $\SO(AdS_3)$. If
\[
  \Sigma=\Dsdd{ r'(t),s(t) }{t}{0},
\]
we have
\begin{equation} \label{eQbetadphiSigma}
  \beta d\phi\Sigma=\left. \phi_{r'}\big( \chi(s_{0}) \big)^{-1}\Dsdd{ \phi_{r'(t)}\big( \chi(s_{t}) \big) }{t}{0}\right|_{\sH}.
\end{equation}
If we note $\phi_{r'(t)}\big( \chi(s_{t}) \big)=\phi\big( r'(t),\chi(s_{f}) \big)$, the derivative in \eqref{eQbetadphiSigma} with respect to $t$ reads
\begin{equation}
\Dsdd{ \phi\big( r'(t),\chi(s_{0}) \big) }{t}{0}+\Dsdd{ \phi\big( r',\chi(s_{t}) \big) }{t}{0}.
\end{equation}
The second term of $\beta d\varphi\Sigma$ is
\begin{align*}
\left. \Dsdd{ \phi_{r'}(\chi(s_{0}))^{-1}\phi_{r'}(\chi(s_{t})) }{t}{0}\right|_{\sH}
        &=\left. \Dsdd{ \phi_{r'}\big( \chi(s_{0}^{-1}s_{t}) \big) }{t}{0}\right|_{\sH}\\
        &=\left. d\phi_{r'}d\chi(s_{0}^{-1}s'(0))\right|_{\sH}.
\end{align*}
From all that we want to define
\begin{equation}
   \alpha^{S}_{(r',s_{0})}\Sigma=\left.d\phi d\chi\big(s_{0}^{-1}s'(0)\big)\right|_{\sH}+\left.\phi_{r'}\big( \chi(s_{0}) \big)^{-1}\Dsdd{ \phi_{r'(t)}\chi(s_{0}) }{t}{0}\right|_{\sH},
\end{equation}
and we would not have $\alpha^{S}(\xi_{X})=0$.


\subsection{Horizontal lift}
%------------------------------

Since the spin component of the path of $\xi_{X}$ is constant, we have $\alpha(\xi_{X})=0$, so equation  \eqref{eq_xiXprojXsharp} says that
\begin{equation}
\overline{ X^{\sharp} }=\xi_{X}.
\end{equation}
Let us recall that an equivariant function (which defined a section of an associated bundle) is
\begin{equation}
\begin{aligned}
 \hat{\psi}\colon P&\to V \\
\hat{\psi}(\xi\cdot g)&= \rho(g^{-1})\hat{\psi}(\xi).
\end{aligned}
\end{equation}
General definition of an equivariant derivative (theorem~\ref{tho_dercovassoequiv}) leads to
\[
  \widehat{    \nabla_{X^{\sharp}}\psi    }=\overline{ X^{\sharp} }\cdot\hat{\psi}=\xi_{X} \cdot \hat{\psi}.
\]
In our setting, the equivariance of $\hat{\psi}$ reads, for all $a\in\Spin(2,1)$,
\[
  \hat{\psi}\big( ([r],s)\cdot a \big)=\hat{\psi}\big( [r],sa \big)\stackrel{!}{=}\rho(a^{-1})\hat{\psi}\big( [r],s \big).
\]
We check the equivariance of $\widehat{\nabla_{X^{\sharp}}\psi}$ by the following computation:
\[
\begin{split}
\widehat{\nabla_{X^{\sharp}}\psi  }\big( ([r],s)\cdot a \big)&=\widehat{\nabla_{X^{\sharp}}\psi}( [r],sa )\\
        &=(\xi_{X}\cdot \hat{\psi})([r],sa)\\
        &=\Dsdd{ \hat{\psi}\big( [r e^{tX}],sa \big) }{t}{0}\\
        &=\Dsdd{ \rho(a^{-1})\hat{\psi}\big( [r e^{tX}],s \big) }{t}{0}\\
        &=\rho(a^{-1})(\xi_{X}\cdot \hat{\psi})\big( [r],s \big)\\
        &=\rho(a^{-1})\widehat{  \nabla_{X^{\sharp}}\psi  }\big( [r],s \big).
\end{split}
\]

We define $\tilde{\psi}\colon AN/S\to \Lambda W$ by
\[
  \tilde{\psi}([r])=\hat{\psi}( [r],e ),
\]
so that
\begin{equation}
\hat{\psi}([r],s)=\rho(s^{-1})\tilde{\psi}([r]).
\end{equation}
We can conclude
\[
\begin{split}
\widetilde{ \nabla_{X^{\sharp}}\psi  }([r])&=\widehat{\nabla_{X^{\sharp}}\psi}([r],e)\\
        &=\xi_{X}\hat{\psi}([r],e)\\
        &=\Dsdd{ \hat{\psi}\big( [r e^{tX}],e \big) }{t}{0}\\
        &=\Dsdd{ \tilde{\psi}\big( [r e^{tX}] \big) }{t}{0}\\
        &=\tilde X_{[r]}\tilde{\psi}([r]).
\end{split}
\]
So
\begin{equation}
\widetilde{\nabla_{X^{\sharp}}\psi}=\tilde X_{[r]}\tilde{\psi}.
\end{equation}

\subsection{Spin structure on \texorpdfstring{$AdS_3$}{AdS3} }
%+++++++++++++++++++++++++++++++++++++++++++++++++++++++++

\subsubsection{Spin structure on the whole \texorpdfstring{$AdS_3$}{AdS3} }

\begin{probleme}
    The following seems to contradict what I find in Michelson-Donaldson
\end{probleme}
The central fact is that
\[
  \Spin(2,1)\simeq\Delta\simeq\SL(2,\eR)
\]
where $\Delta=\{ (g,g)\tq g\in\SL(2,\eR) \}\subset G_0$. We take as notations: $G_{0}=\SL(2,\eR)$ and $\overline{G}=G_0\times G_{0}$.

\begin{lemma}
We have the following homogeneous space isomorphism:
\[
  \overline{G}/\Delta\simeq\SL(2,\eR).
\]
\end{lemma}

\begin{proof}

We have an action $\overline{G}\times AdS_3\to AdS_3$,
\begin{equation} \label{EqActghgxh}
  (g,h)x=gxh^{-1}
\end{equation}
where $x\in\SL(2,\eR)$ is seen as in $AdS_3$ by the usual isomorphism. Moreover we consider the isomorphism
\begin{align}
\overline{G}/\Delta&\simeq\SL(2,\eR)\\
[x_1,x_2]&\mapsto x_1x_2^{-1}
\end{align}
which is well defined because $[x_1g,x_2g]\mapsto x_1gg^{-1}x_2^{-1}=x_1x_2^{-1}$. In particular, $[g,g]\mapsto e\in\SL(2,\eR)$. So $\overline{G}$ acts on $\SL(2,\eR)$ and the elements which fix $e$ are the one of $\Delta$. It proves the lemma.
\end{proof}

We are going to take the following structure:
\begin{equation}  \label{EqScSpinAdS}
  \xymatrix{%
   \overline{G} \ar[rr]^{\displaystyle\varphi}\ar[dr]_{\displaystyle\pi}    &   &   \overline{G}/\eZ_{2}\ar[ld]\\
                        & M
}
\end{equation}
where $M$ is $G_0$ seen as $M=\overline{G}/\Delta\simeq \SL(2,\eR)\simeq AdS_3$, and the projection $\pi\colon \overline{G}\to M$ is given by $\pi(g,h)=gh^{-1}$. The action of $\Delta\simeq \Spin(2,1)$ on $\overline{G}$ is given by formula $(xg,g)\cdot (a,a)=(xga,ga)$. First, let us prove the following.
\begin{proposition}
The frame bundle over $AdS_3$ can be seen as
\[
  \SO(AdS_3)\simeq \overline{G}/\eZ_{2}
\]
where $\overline{G}=\SL(2,\eR)\times\SL(2,\eR)$.
\end{proposition}

\begin{proof}
In the fiber bundle $\pi\colon \overline{G}\to M$, the fibre over $x\in\SL(2,\eR)$ is the set of $(g,h)$ such that $gh^{-1}=x$, or
\[
  \overline{G}_{x}=\{ (xg,g) \}\subset \overline{G}.
\]
We will give a surjective map $\overline{G}_{x}\to\SO(M)_{x}$, the fibre of the frame bundle over $x\in AdS_3$. For this, we see a basis of $AdS_3$ as an isometric map $b\colon \sG_0\to T_{x}M$ where $\sG_{0}=\mathfrak{sl}(2,\eR)$, and we define
\begin{equation}
\begin{aligned}
 \psi_{x}\colon \overline{G}_{x}&\to \SO(M)_{x} \\
\psi_{x}(xg,g)(X)&=(dL_{x})_{e}\big( \Ad(g^{-1})X \big)
\end{aligned}
\end{equation}
for all $X\in\sG_0$. Let us study the kernel of this map, i.e. elements such that $\psi(xg_1,g_1)=\psi(xg_2,g_2)$. It needs, for all $X\in\mathfrak{sl}(2,\eR)$,
\[
  \Ad(g_1^{-1})X=\Ad(g_2^{-1})X,
\]
but we know that the requirement $\Ad(g)X=X$ is the fact the $g$ is in the center of the group. In our case, it results that $g_2^{-1}g_1=\pm\id$, so
\[
  \psi(xg_1,g_1)=\psi(\pm xg_1,\pm g_1)
\]
where the same $\pm$ has to be taken in both appearances of the right hand side. Now we put all the $\psi_{x}$ together to get $\psi\colon \overline{G}\to \SO(M)$. Once again we look in which cases $\psi(g_1,h_1)=\psi(g_2,h_2)$. We put this condition under the form
\[
  \psi(g_1h_1^{-1}h_1,h_1)=\psi(g_2h_2^{-1}h_2,h_2)
\]
which immediately gives $h_1=\pm h_2$. But on the other hand the base point of $\psi(g_{i}h_{i}^{-1},h_{i})$ is $g_{i}h_{i}^{-1}$, so that the condition also ask $g_1h_1^{-1}=g_2h_2^{-1}$ which in turn gives $g_1=\pm g_2$ with the same $\pm$ as in $h_1=\pm h_2$. We conclude that $\eZ_{2}$ is the problem for the inverse of $\psi$. This proves the proposition.
\end{proof}
We will usually use the same notation, $\psi$, to denote the map from $\overline{G}$ and the one from $\overline{G}/\eZ_{2}$. The following lemma will prove useful to study the actions of the structure groups in the picture \eqref{EqScSpinAdS}.

\begin{lemma}
The map
\begin{equation}
\begin{aligned}
 \SL(2,\eR)&\to \SO_{0}(1,2)\\
   g&\mapsto\Ad(g).
\end{aligned}
\end{equation}
is a double covering.
\end{lemma}
\begin{proof}
No proof.
\end{proof}
The action of $a\in\SO_{0}(1,2)$ on $(xg,g)\in\overline{G}/\eZ_{2}$ is defined by
\begin{equation}
\psi\big( (xg,g)\cdot a \big)=(dL_{x})_{e}\Ad(a^{-1}g^{-1}).
\end{equation}
On the other hand, let us see how does $(a,a)\in\Delta\simeq \Spin(2,1)$ acts on $\overline{G}$ and how does it reflects on the $\psi$ level. Since $(xg,g)\cdot (a,a)=(xga,ga)$, we have
\[
  \psi\big( [xg,g]\cdot a \big)=\psi\big( (xg,g)\cdot(a,a) \big),
\]
and then
\[
  \varphi\big( (xg,g)\cdot a \big)=(xg,g)\cdot(a,a).
\]
This proves that our structure is a spin structure.

\subsubsection{Reduction to one open orbit}
%----------------------------------------

We will use this isomorphism between $AdS_3$ and $\SL(2,\eR)$:
\[
  \begin{pmatrix}
u\\t\\x\\y
\end{pmatrix}\mapsto
\begin{pmatrix}
u+x&y-t\\y+t&u-x
\end{pmatrix}.
\]
Then the famous point $[u]=\begin{pmatrix}
0&1\\-1&0
\end{pmatrix}\in AdS_3$ corresponds to the element $J:=\begin{pmatrix}
0&1\\-1&0
\end{pmatrix}\in \SL(2,\eR)$. This is our base point of the open orbit. We could also take
\[
  k_{0}=\frac{ \sqrt 2 }{ 2 }\begin{pmatrix}
1&1\\-1&1
\end{pmatrix}\in K_{0}
\]
where $K_{0}$ is the ``$K$'' of $\SL(2,\eR)$.
\begin{probleme}
    I think that $J$ is also a complex structure. To be checked.
\end{probleme}
We have $J=k_{0}^{2}$ and following the action \eqref{EqActghgxh}, we have $J=(k_{0},k_{0}^{-1})e$. The subgroup $\overline{R}\subset\overline{G}$ acts on $AdS_3$, and we want to know the stabilizer of $J$. The condition is $(r,r')\cdot J=J$, or
\[
  r=\AD(J)r',
\]
but $\AD(J)=\theta$ (the Cartan involution). So an element $(r,r')\in\overline{R}$ stabilises $J$ if it is of the form $(r,\theta r)$, thus
\[
  \mfs=\text{Lie algebra of the stabiliser of }J=\{ (X,\theta X)\tq X\in\sR_{0} \}\cap\sR,
\]
where the intersection with $\sR$ is important because $\theta$ can send out of $\sR_{0}$. Note that when $X$ has a $\sN$ component, then $\theta X$ has a $\overline{ \sN }$ component, so $(X,\theta X)\in(\sA\oplus\sN,-\sA\oplus\overline{ \sN })$ where the minus sign comes from the fact that $\theta(\sA)=-\sA$. Then $X$ cannot have a $\sN$ component and finally,
\[
  \mfs=\eR (H,-H)\in\sQ.
\]
The group $R'$ is
\begin{equation}
R'= e^{\sR'}=\{ (an,an')\tq n,n'\in N_{0} \}
\end{equation}
because $\sR'$ is $\sR$ minus the stabiliser, i.e. $\sR'=\eR(H,H)\oplus\sN$. We have the identification $r'\mapsto r'\cdot J$ between $R'$ and the open orbit $\mU$. As usual, the action is $(g,h)\cdot x=gxh^{-1}$ if $r'=(g,h)$. Notice in particular that $R'\neq R_{0}'\times R_{0}'$.

Up to now we studied the fiber $\overline{G}\to M$; we are now able to restrict it to $\overline{G}|_{\mU}\to\mU$ and to establish an isomorphism with the trivial bundle $R'\times G_0\to R'$. The fiber over $x\in\mU$ is
\[
  \overline{G}_{x}=\{ (xg,g) \}.
\]
We define the isomorphism as follows:
\begin{equation}
\begin{aligned}
 \tau\colon R'\times G_0&\to \overline{G}|_{\mU} \\
(r',g)&\mapsto (r'\cdot Jg,g)
\end{aligned}
\end{equation}
and we have the following picture:
\[
  \xymatrix{%
   R'\times G_0 \ar[r]^-{\displaystyle\tau}\ar[d]   &   \overline{G}|_{\mU}\ar[d]^{\displaystyle\pi}\\
   R' \ar@{.>}[r]^{\displaystyle\tau}       &   \mU
}
\]
in which are defines by
\[
  \xymatrix{%
   (r',g) \ar[r]^-{\displaystyle\tau}\ar[d] &   (r'\cdot Jg,g)\ar[d]^{\displaystyle\pi}\\
   r' \ar@{.>}[r]^{\displaystyle\tau}       &   r'\cdot J
}
\]
where the dotted line denotes the induced map from $\tau$, which is denoted by the same symbol. The map $\tau\colon R'\to \mU$ is just the restriction of the original $\tau$ to $g=e$. Notice that this $\tau$ provides a diffeomorphism of the basis spaces $R'$ and $\mU$.

\subsubsection{Spin connection}
%---------------------------

The spin connection on $\overline{G}|_{\mU}$ is given by
\begin{equation}    \label{EqDefConnAdS3}
  \alpha^{S}_{(g,h)}\Sigma=\left[ dL_{(g,h)^{-1}}\Sigma \right]_{\sH},
\end{equation}
or
\begin{equation}
\alpha^{S}_{(g,h)}=\pr_{\sH}\circ\big( dL_{(g,h)^{-1}} \big)_{(g,h)}.
\end{equation}
Notice that when we write $\sH$, we think about $\Delta$: the group by which quotient  $\overline{G}$ in order to get $\SL(2,\eR)\simeq AdS_3$.
Our task now is to transfer this connection to $R'\times G_0$ by defining $\alpha'=\tau^*\alpha^{S}$. If $\Sigma\in T_{(r',g)}(R'\times G_0)$, we define
\begin{equation}
\alpha'_{(r',g)}\Sigma=\alpha^{S}(d\tau\Sigma).
\end{equation}
Let us take $X\in\sG_0$ and $0\in\sR'$ and let us compute $d\tau(0,X)$. More precisely, we consider
\begin{align*}
d\tau(0\oplus \tilde X_{g})_{(r',g)}&=d\tau\Dsdd{ r',g e^{tX} }{t}{0}\\
        &=\Dsdd{ r'\cdot Jg e^{tX},g e^{tX} }{t}{0}\\
        &=\big( \tilde X_{(r'\cdot Jg)},\tilde X_{g} \big).
\end{align*}
The next step is to compute $d\tau\Sigma$ in the case where $\Sigma=(\utilde Y\oplus -1)_{(r',g)}$ with $Y\in\sR'\subset\sR_{0}\oplus\sR_{0}$. We have
\begin{align}
d\tau\Sigma&=\Dsdd{ \tau\big(  e^{tY}r',g \big) }{t}{0}\\
    &=\Dsdd{ ( e^{tY}r'\cdot J)g,g }{t}{0}
\end{align}
where, if $r'=(r_{1},r_{2})$, we consider $Y=\big((\utilde Y_{1})_{r_{1}},(\utilde Y_{2})_{r_{2}}\big)  $. This appears to be difficult to be computed. This reflects the fact that the connection should be complicated in the trivial bundle $R'\times G_0$.

But there are no fate. We remember that $\tau$ furnish a diffeomorphism between the basis spaces, so one can consider the bundle
\[
  \xymatrix{%
   \overline{G}|_{\mU} \ar[d]^{\displaystyle\tau^{-1}\circ\pi}\\
   R'
}
\]
Vectors of $\sH$ are of the form $(X,X)$ with $X\in\mathfrak{sl}(2,\eR)$, thus $A\in T_{(xg,g)}\overline{G}|_{\mU}$ fulfils $\alpha^{S}(A)=0$ if and only if
\[
  dL_{(xg,g)^{-1}}(A)=(X,-X)
\]
for a certain $X\in \mathfrak{sl}(2,\eR)$. All this makes that the horizontal space over $(xg,g)$ is given by
\begin{equation}
\horsp(xg,g)=\big\{ (\tilde X_{xg},-\tilde X_{g})\tq X\in \sG_0=\mathfrak{sl}(2,\eR) \big\}.
\end{equation}
The strategy now is to project that on $R'$ and express Dirac operator in terms of the result. Let us make this simple computation:
\begin{align*}
d\pi(\tilde X_{xg},\tilde X_{g})&=\Dsdd{ \pi\big( xg e^{tX},g e^{-tX} \big) }{t}{0}\\
        &=\Dsdd{ xg e^{tX} e^{tX}g^{-1} }{t}{0}\\
        &=\Dsdd{ x e^{2t\Ad(g)X} }{t}{0}\\
        &=2(dL_{x})_{e}\Ad(g)X.
\end{align*}
This result has to be brought from $\mU$ to $R'$ by $\tau^{-1}$. Now we take a $\tilde Y\in\cvec(R')$ and we want to know which is the corresponding $X$, i.e. the $X\in\mathfrak{sl}(2,\eR)$ such that
\[
  d\tau^{-1}d\pi(\tilde X_{xg},-\tilde X_{g})=\tilde Y.
\]
From the previous computation, $\tilde Y=2d\tau^{-1}dL_{x}\Ad(g)X$, so
\begin{equation}  \label{EqXfracAdY}
  X=\frac{ 1 }{2}\Ad(g^{-1})dL_{x^{-1}}d\tau\tilde Y.
\end{equation}
We now precise our idea:
\begin{equation}   \label{EqtildeYrunrdeux}
  \tilde Y_{(r_{1},r_2)}=\big(    (\tilde Y_{1})_{r_1},(\tilde Y_{2})_{r_2}   \big)=\Dsdd{ r_1 e^{tY_{1}},r_2 e^{tY_{2}} }{t}{0}
\end{equation}
for $Y_{i}\in\sR'_{0}$ and $r_1$, $r_2\in R_{0}$. In this case, the ``$x$'' in equation \eqref{EqXfracAdY} is $(r'\cdot J)^{-1}$. Let us begin by taking $s'\in R'$ and compute $L_{(r'\cdot J)^{-1}}\tau(s')$. Remember that $r'\cdot J=r_1Jr_2^{-1}$ from the general action \eqref{EqActghgxh}, so if $r'=(r_1,r_2)$,
\begin{align*}
  dL_{(r'\cdot J)^{-1}}\tau(s')&=(r'\cdot J)^{-1}s_1 Js_2^{-1}\\
        &=(r_1Jr_2^{-1})^{-1}s_1Js_2^{-1}\\
        &=-r_2Jr_2^{-1}s_1Js_2^{-1}.
\end{align*}
Now, we apply that result on computation of \eqref{EqXfracAdY} with \eqref{EqtildeYrunrdeux}:
\begin{align*}
dL_{(r'\cdot J)^{-1}}d\tau\tilde Y&=\Dsdd{ -r_2Jr_1^{-1}r_1 e^{tY_{1}}J e^{-tY_{2}}r_2^{-1} }{t}{0}\\
        &=\Dsdd{ \AD(r_2)\big( -J e^{tY_{1}}J e^{-tY_{2}} \big) }{t}{0}\\
        &=\Dsdd{ \AD(r_2) e^{-tY_{2}} }{t}{0}+\Ad(r_2)\Ad(J)Y_{1}\\
        &=-\Ad(r_2)Y_{2}+\Ad(r_2)\theta(Y_{1}),
\end{align*}
and finally,
\begin{equation}  \label{EqValeurXAdtheta}
\begin{aligned}
X&=\frac{ 1 }{2}\Ad(g^{-1})dL_{(r'\cdot J)^{-1}}d\tau\tilde Y\\
        &=\frac{ 1 }{2}\Ad(g^{-1})\big[ \Ad(r_2)\theta(Y_{1})-\Ad(r_2)Y_{2} \big].
\end{aligned}
\end{equation}
For this $X$, the horizontal lift of $\tilde Y\in\cvec(R')$ is $(X,-X)\in T\overline{G}|_{\mU}$.

\subsection{Left invariance of Dirac}
%------------------------------------

Sections of the spin bundle over the open orbit $\mU$ are given by equivariant functions $\hat{\psi}\colon \overline{G}|_{\mU}\to \eR^{2}$. The action of $\Delta\simeq\Spin(2,1)$ on $\overline{G}$ is
\[
  (g,h)\cdot(a,a)=(ga,ha).
\]
We define $\tilde{\psi}$ by
\begin{equation}
\tilde{\psi}(g)=\hat{\psi}(g,e)
\end{equation}
for $g\in\mU$. We get back the original $\hat{\psi}$ by formula
\begin{equation}
\hat{\psi}(g,h)=\rho(h,h)^{-1}\tilde{\psi}(gh^{-1}).
\end{equation}
Our intention is now to compute $\widehat{\nabla_{Z}\psi}(\xi)=\overline{ Z }_{\xi}(\hat{\psi})$ with $\xi=(xg,g)\in\overline{G}|_{\mU}$ (hence $x\in\mU$) and $Z\in\cvec(R')$. For instance we choose a left invariant $Z=\tilde Y=(\tilde Y_{1},\tilde Y_{2})$ for $Y_{1}$, $Y_2\in\sR_{0}'$. Recall that $\tilde Y$ is given by equation \eqref{EqtildeYrunrdeux}. From definition of the covariant derivative associated with the connection,
\[
  \widehat{\nabla_{\tilde Y}\psi}(\xi)=\overline{ \tilde Y }_{\xi}(\hat{\psi})=\overline{ \tilde Y }_{(xg,g)}(\hat{\psi})
\]
where $\overline{ \tilde Y }_{(xg,g)}$ is an horizontal vector at $(xg,g)$ whose projection is $\tilde Y$. From our previous work,
\[
  \overline{ \tilde Y }_{xg,g}=(\tilde X_{xg},-\tilde X_{g})
\]
with $X=\frac{ 1 }{2}\Ad(g^{-1})\big( \Ad(r_2)\theta Y_{1}-\Ad(r_2)Y_{2} \big)$. Let us understand the link between $(r_{1},r_2)$ and $g,x$. The vector $(\tilde X_{xg},\tilde X_{g})$ actually projects to a vector at $\tau^{-1}\circ\pi(xg,g)=\tau^{-1}(x)$. The fact that $x\in\mU$ guarantees existence and uniqueness of $(r_1,r_2)\in R'$ such that $r_1 Jr_2^{-1}=x$. We have
\[
\begin{split}
\protect\widetilde{\nabla_{\protect\tilde Y}\psi}(x)&=\widehat{\nabla_{\tilde{Y}}\psi}(x,e)\\
        &=(\tilde X_{x},-\tilde X_{e})\hat{\psi}\\
        &=\dsdd{ \hat{\psi}\Big( x e^{tX}, e^{-tX} \Big) }{t}{0}\\
        &=\dsdd{ \rho( e^{tX}, e^{tX})\tilde{\psi}(x e^{2tX}) }{t}{0}.
\end{split}
\]
The first term of the derivation (the one with $t=0$ in the $\rho$) gives $2\tilde X_{x}\tilde{\psi}$. This is left invariant.
The second is
\[
     \dsdd{ \rho( e^{tX}, e^{tX})\tilde{\psi}(x) }{t}{0}.
\]
We want to test the condition \eqref{EqDefLxinvarop} on this term. Let us pose
\[
  (E\tilde{\psi})(x)=(\tilde X_{x},\tilde X_{e})\hat{\psi}=\dsdd{ \rho( e^{tX}, e^{tX})\tilde{\psi}(x) }{t}{0}
\]
with $X$ given by equation \eqref{EqValeurXAdtheta}. On the one hand,
\begin{subequations}
\begin{equation}
L_{y}(E\tilde{\psi})(x)=(E\tilde{\psi})(yx)
        =\dsdd{ \rho( e^{tX_{a}}, e^{tX_{a}})\tilde{\psi}(yx) }{t}{0}
\end{equation}
with
\begin{equation}
X_{a}=\frac{ 1 }{2}\Big( \Ad(r_2)\theta Y_{1}-\Ad(r_2) \Big)Y_{2}
\end{equation}
\end{subequations}
where $(r_1,r_2)$ is given by $yx$. On the other hand,
\begin{subequations}
\begin{equation}
E(L_{y}\tilde{\psi})(x)=\dsdd{ \rho( e^{tX_{b}}, e^{tX_{b}})\tilde{\psi}(yx) }{t}{0}
\end{equation}
with
\begin{equation}
X_{b}=\frac{ 1 }{2}\big( \Ad(s_2)\theta Y_{1}-\Ad(s_2)Y_{2} \big)
\end{equation}
\end{subequations}
where $(s_1,s_2)$ is given by $x$.

The problem is that the choice of $y$ is arbitrary, so that $X_{a}$ and $X_{b}$ could be too different. Ok. That's the proof that Dirac is not invariant. Here is the proof that Dirac is invariant.

Following equation \eqref{EqDefConnAdS3}, the spin connection form is
\[
  \alpha^{S}_{(g,h)}\Sigma=\left( dL_{(g,h)^{-1}}\Sigma \right)_{\sH}.
\]
If $L_{(x,y)}$ is the left translation by $(x,y)$ we have
\[
  \big( L^{*}_{(x,y)}\alpha \big)_{(g,h)}\Sigma=\alpha_{(xg,yh)}\big( dL_{(x,y)}\Sigma \big)
        =\left( dL_{(g,h)^{-1}}\Sigma \right)_{\sH}.
\]
Thus we have $L^*_{(x,y)}\alpha^{S}=\alpha^{S}$. Now we consider the formula $\widehat{\nabla_{\tilde Y}\psi}(\xi)=(\tilde X_{xg},-\tilde X_{g})\hat{\psi}$, and we will check that
\begin{equation}
 \big( L_{\eta}\widehat{\nabla_{Z}\psi} \big)(\xi)=\widehat{  \nabla_{Z}(L_{\eta}\psi)     }(\xi).
\end{equation}
with $\xi=(xg,g)$ and $\eta=(a,b)$. On the one hand,
\begin{align*}
  \big( L_{(a,b)}\widehat{\nabla_{Z}\psi} \big)(xg,g)&=\widehat{\nabla_{Z}\psi}(axg,bg)\\
        &=\widehat{\nabla_{Z}\psi}(axgg^{-1}b^{-1}bg,bg)\\
        &=\big( \tilde X_{(axb^{-1})bg},-\tilde X_{bg} \big)\hat{\psi}.
\end{align*}
On the other hand,
\begin{align*}
  \widehat{  \nabla_{Z}(L_{(a,b)}\psi)   }(xg,g)&=(\tilde X_{xg},-\tilde X_{g})\widehat{L_{(a,b)}\psi}\\
        &=\dsdd{ \widehat{L_{(a,b)}\psi}\big( xg e^{tX},g e^{-tX} \big) }{t}{0}\\
        &=\dsdd{ \hat{\psi}\big( axg e^{tX},bg e^{-tX} \big) }{t}{0}\\
        &= (\tilde X_{axg},-\tilde X_{bg}) \hat{\psi}\\
        &=( \tilde X_{(axb^{-1})bg},-\tilde X_{bg} )\hat{\psi}.
\end{align*}

\section{Spin structure and Dirac operator on \texorpdfstring{$AdS_{l}$}{AdSl}}\label{SecDirADs}
%+++++++++++++++++++++++++++++++++++++++++++++++++++++++++++

Construction of the frame bundle is a straightforward adaptation of theorem 2.2 (chapter II) in \cite{AnnikFranc}, while connection issues are adapted from proposition 1.3 (chapter III).  According proposition~\ref{PropGHconn}, notations $G$ and $H$ stand for the identity components of $\SO(2,l-1)$ and $\SO(1,l-1)$.

\subsection{Frame bundle and spin structure}
%--------------------------------------------

An element of the frame bundle is a map from $\sQ$ to $T(G/H)$ of the form\footnote{See~\ref{SubSechoappahomsp} for notations.} $d\mu_g\circ A$ where $g\in G$ and $A\in \SO_0(\sQ)$. By proposition~\ref{PropSOADHequal}, there exists a $h\in H$ for which $A=\Ad(h)$ for every $A\in\SO_0(\sQ)$ so we have
\[
 d\mu_g\circ A=d\pi\circ dL_g\circ\Ad(h)=d\pi\circ dL_g \circ dL_h\circ dR_h=d\pi\circ dL_g \circ dL_h=d\mu_{gh}
\]
hence in fact every element in the frame bundle reads $d\mu_g$ for some $g\in G$. We conclude that the fibre $B_{[g]}$ over $[g]$ is made of maps of the form $d\tau_k$ with $k\in[g]$. The action of $H$ on the frame bundle is given by
\[
  (d\mu_g)\cdot h=d\mu_g\circ\Ad(h).
\]

\begin{proposition}
The map
\begin{equation}
\begin{aligned}
 \beta\colon G&\to B \\
g&\mapsto d\mu_g
\end{aligned}
\end{equation}
is a principal bundle isomorphism between the frame bundle and the principal bundle
\begin{equation}        \label{PrincHGGH}
\xymatrix{%
    G\ar[d]^{\pi}& H\ar@{~>}[l]     \\
                    G/H
 }
\end{equation}
where $\pi$ is the natural projection, the action of $H$ is the right one and the wavy line means ``acts on''.
\end{proposition}

\begin{proof}
Surjectivity of $\beta$ is clear. For injectivity, suppose $d\mu_g=d\mu_{g'}$. In order for the two target spaces to be equal, one needs $g'=gh$ for a $h\in H$. Now we have, for all $q_j\in\sQ$,
\begin{equation}
  d\mu_gq_j=d\mu_{gh}q_j=d\pi dR_{h^{-1}}dL_gdL_hq_j
        =d\pi dL_g\big( \Ad(h)q_j \big),
\end{equation}
but $d\pi$ is an isomorphism from $\sQ_g$, so we deduce that $q_j=\Ad(h)q_j$. Since we are using the connected component of $\SO(\sQ)$, that implies that $h=e$, and thus that $g=g'$. The following proves that $\beta$ is a morphism:
\[
  \beta(gh)=d\pi dL_g dL_h=d\pi dL_g dL_h dR_{h^{-1}}=d\pi dL_g \Ad(h)=\beta(g)\cdot h.
\]
\end{proof}

The following lemma provides  a convenient way to express the tangent bundle over $G/H$ as an associated bundle to the principal bundle \eqref{PrincHGGH}. We denote by $G\times_{\rho} \sQ$ the quotient of $G\times\sQ$ by the equivalence relation $(g,X)\sim(gh,\Ad(h^{-1})X)$ for all $h\in H$.

\begin{lemma}
The map
\begin{equation}
\begin{aligned}
 \beta\colon G\times_{\rho}\sQ&\to TM \\
[g,X]&\mapsto d\tau_gd\pi X
\end{aligned}
\end{equation}
with $\rho(h)X=\Ad(h^{-1})X$ is diffeomorphic.
\label{LemBazHGGH}
\end{lemma}

\begin{proof}

 In order to check that $\beta$ is well defined, first compute
\[
  \beta[gh,\Ad(h^{-1})X]=d\tau_{gh}d\pi\Ad(h^{-1})X=d\pi dL_{gh}\Ad(h^{-1})X,
\]
and then using the fact that $d\pi dR_h=d\pi$, the latter line reduces to $d\pi dL_gX=\beta(g,X)$. For injectivity, let $\beta[g,X]=\beta[g',X']$. In order for these two to be vectors on the same point, there must exists a $h\in H$ such that $g'=gh$. The equality becomes $d\pi dL_g dL_h X'=d\pi dL_gX$. Commuting $d\pi$ with $dL_g$ and using the fact that $d\tau_g$ is an isomorphism, we are left with the condition $d\pi dL_h X'=d\pi X$.

An element of $\sG/\sH$ is an equivalence class which contains exactly one element of $\sQ$. In the right hand side of the condition, this element is $X$ while the element of $\sQ$ in the class $d\pi dL_h X$ is $\Ad(h)X'$. Equating these two elements, we find the condition $X'=\Ad(h^{-1})X$, which proves that $[g,X]=[g',X']$ and concludes the proof of the injectivity of $\beta$.
\end{proof}

The following proposition will prove useful in order to identity the spin structure over $AdS_4$.

\begin{proposition}
If $G$ is a connected Lie group and if $Z$ is the center of $G$, then
\begin{enumerate}
\item $\Ad_G$ is an analytic homomorphism from $G$ to $\Int(G)$, with kernel $Z$,
\item the map $[g]\to\Ad_G(g)$ is an analytic isomorphism from $G/Z$ to $\Int(\lG)$ (the class $[g]$ is taken with respect to $Z$).
\end{enumerate}
\end{proposition}
On the one hand that proposition together with the fact that $Z\big( \SP(2,\eR) \big)=\eZ_2$ proves that the quotient $\SP(2,\eR)/\eZ_2$ is isomorphic to $\Int\big(\gsp(2,\eR)\big)$. On the other hand one knows that $\SO_0(2,3)$ has no center, so that $\SO_0(2,3)\simeq\Int(\so(2,3))$. But the subsection~\ref{SubSecIsosp} provides an isomorphism between $\so(2,3)$ and $\gsp(2,\eR)$. Thus we have
\begin{equation}
\SP(2,\eR)/\eZ_2\simeq \SO_0(2,3).
\end{equation}
We denote by $\varphi\colon \SP(2,\eR)\to \SO_0(2,3)$ the corresponding homomorphism with kernel $\eZ_2$. In particular the restriction $\varphi|_{\SL(2,\eC)}$ is a double covering of $\SO_0(1,3)$. But $\chi$ is the same kind of double covering, so universality of $\SL(2,\eR)$ on $\SO_0(1,3)$ provides an automorphism $f\colon \SL(2,\eC)\to \SL(2,\eC)$ such that $\varphi=\chi\circ f$. The spin structure to be considered on $AdS_4$ is
\[
\xymatrix{%
   \Spin(1,3) \ar@{~>}[r]       &\SP(2,\eR)  \ar[rd] \ar[rr]^{\displaystyle\varphi} &&  \SO_0(2,3)\ar[ld]&\SO_0(1,3)\ar@{~>}[l]\\
   &&   AdS_4
}
\]
where the action of $\Spin(1,3)$ on $\SP(2,\eR)$ is given by $a\cdot s=af^{-1}(s)$ where we identified $\Spin(1,3)$ with $\SL(2,\eC)$ as subgroup of $\SP(2,\eR)$. One immediately has $\varphi(a\cdot s)=\varphi(a)\chi(s)$.


\subsection{Connection}
%----------------------

There are a lot of ways to express a vector field $X\colon G/H\to T(G/H)$. From the identification $T(G/H)=G\times_{\rho}\sQ$, one has $X\colon G/H\to G\times_{\rho}\sQ$. As section of an associated bundle, $X$ can be expressed by an equivariant function $\hat{X}\colon G\to \sQ$ such that $X_{[g]}=[g,\hat{X}(g)]$. The $H$-equivariance of $\hat X$ means that $\hat{X}(gh)=\Ad(h^{-1})\hat{X}(g)$.  Let $X\in\sG$ and consider the function
\begin{equation}        \label{EqDefhatAcol}
\begin{aligned}
 \hat A_X\colon G&\to \sQ \\
g&\mapsto \big( \Ad(g^{-1})X \big)_{\sQ}
\end{aligned}
\end{equation}
which is equivariant because the decomposition $\sG=\sH\oplus\sQ$ is reductive. The corresponding vector field is
\[
  A_X[g]=\big[ g,\big( \Ad(g^{-1})X \big)_{\sQ} \big];
\]
or
\[
  A_X[g]=d\tau_gd\pi\big( \Ad(g^{-1})X \big)_{\sQ}=d\pi dL_g\big( \Ad(g^{-1})X \big)
\]
because $d\pi X_{\sQ}=d\pi X$. It is easy to check that the form
\[
  \omega_g(X)=-\big( dL_{g^{-1}}X \big)_{\sH}
\]
is a connection form on the principal bundle \eqref{PrincHGGH}.  We are going to determine the associated covariant derivative of this connection on the tangent space, and prove that it is torsion free. The horizontal lift of $A_X[g]$ is
\begin{equation}    \label{EqovlAprQhor}
  \overline{ A }_X(g)=dL_g\big( \Ad(g^{-1})X \big)_{\sQ}=\Dsdd{ g e^{t\pr_{\sQ}\Ad(g^{-1})X} }{t}{0}.
\end{equation}
The equivariant function associated with the covariant derivative of $A_Y$ in the direction of $A_X$ is given by $(\overline{ A }_X)_g\hat A_Y$. Using expressions  \eqref{EqDefhatAcol} and  \eqref{EqovlAprQhor} of $\hat A_Y(g)$ and $\overline{ A }_X(g)$, we have
\[
\begin{split}
  (\bar A_X)_g\hat A_Y  &=\Dsdd{ \hat A_Y\big( g e^{t\pr_{\sQ}\Ad(g^{-1})X} \big)_{\sQ} }{t}{0}\\
            &=\Dsdd{ \left( \Ad\big(  e^{-t\pr_{\sQ}\Ad(g^{-1})X}g^{-1}  \big)Y \right)_{\sQ} }{t}{0}\\
            &=\Big( \ad\big( -\pr_{\sQ}\Ad(g^{-1})X \big)\Ad(g^{-1})Y \Big)_{\sQ}\\
            &=-\left[ \big( \Ad(g^{-1})X \big)_{\sQ},\Ad(g^{-1})Y  \right]_{\sQ}.
\end{split}
\]
This commutator is an expression of the form $[Z_{\sQ},Z'_{\sQ}+Z'_{\sH}]_{\sQ}$. Using reducibility we find
\begin{equation}
  (\overline{ A }_X)_g\hat A_Y=-\Big[ \big( \Ad(g^{-1})X \big)_{\sQ},\big( \Ad(g^{-1})Y \big)_{\sH} \Big].
\end{equation}
The commutator produces
\[
  (\overline{ A }_X)_g\hat A_Y-(\overline{ A }_Y)g\hat A_X=-\hat A_{[X,Y]}(g),
\]
which by construction the equivariant function associated with the vector field $\nabla_{A_X}A_Y-\nabla_{A_Y}A_X$; so on the one hand we have
\[
(\nabla_{A_X}A_Y-\nabla_{A_Y}A_X)[g]=-d\tau_gd\pi\hat A_{[X,Y]}(g)
        =-d\tau_gd\pi\big( \Ad(g^{-1})[X,Y] \big)_{\sQ}
        =-d\pi dR_g[X,Y].
\]
On the other hand,
\[
  [A_X,A_Y][g]=d\pi[dR_g X,dR_gY]=-d\pi dR_g[X,Y],
\]
which proves that the connection is torsion free.

We are now going to study the horizontal vector fields on $\SP(2,\eR)$ with this connection and the homomorphism $h^{-1}$ of equation \eqref{Eqdefhspsl}. We have to study for which elements $\Sigma_a\in \SP(2,\eR)$ the expression
\begin{equation}        \label{EqAtrouverdhemu}
  \omega_a(\Sigma_a)=\omega_{h^{-1}(a)}\big( (dh^{-1})_a\Sigma_a \big)=-\Big( dL_{h^{-1}(a)^{-1}}dh^{-1}\Sigma_a \Big)_{\sH}
\end{equation}
vanishes. Every such element can of course be written under the form $\Sigma_a=dL_a\psi X$ for some $X\in \so(2,3)$. So we are lead to consider the expression
\begin{equation}        \label{Eqdhemuconide}
  (dh^{-1})_a(dL_a)_e\psi X.
\end{equation}
It is easy to deal with that expression in the case of $a=e$:
\[
  (dh^{-1})_e\psi(X)=\psi^{-1}\psi X=X.
\]
In particular, if $\Sigma\in\mT$, then $dh^{-1}\Sigma\in\sQ$ and when $\Sigma\in\mI$, we have $dh^{-1}\Sigma\in\sH$. This result propagates to other elements $a\in \SP(2,\eR)$ using the general result
\[
  df\circ dL_g=\big( dl_{f(g)} \big)\circ df
\]
which holds for any group homomorphism $f$. Using that property with $h^{-1}$ on the point $a\in \SP(2,\eR)$, we find $(dh^{-1})_a\circ(dL_a)_e=\big( dL_{h^{-1}(a)} \big)\circ (dh^{-1})_e$, and the expression \eqref{EqAtrouverdhemu} becomes
\[
  \omega_a(dL_a\psi X)=\Big( dL_{\big(h^{-1}(a)\big)^{-1}}dh^{-1} dL_a\psi X \Big)_{\sH}=X_{\sH}.
\]
It is zero if and only if $X\in\sQ$, so that the horizontal vectors on $a$ are exactly the ones of $dL_a\psi\sQ=\mT_a$.

\subsection{Dirac operator}
%--------------------------

When $\hat{s}\colon \SP(2,\eR) \to \Lambda W$ is the equivariant function associated with a spinor, the Dirac operator reads
\begin{equation}        \label{EqDiracAdsquatre}
\widehat{Ds}(a)=g_{ij}\gamma^j\widehat{\nabla_{t_i}s}(a)
        =g_{ij}\gamma^j\overline{ t }_i(a)\hat{s}
        =g_{ij}\gamma^j\tilde t_i(a)\hat{s}
\end{equation}
where the metric $g$ is the usual four-dimensional Minkowskian metric and the matrices $\gamma$ are the associated $4\times 4$ Dirac matrices. The elements $\tilde t_i(a)=dL_at_i=dL_a\psi(q_i)$ span the natural basis of $\mT_a$, see appendix~\ref{SubSecRedspT}. The matrices $\gamma^i$ are the usual $4\times 4$ Dirac matrices for the $4$-dimensional Minkowskian metric.

\begin{probleme}
Et il serait aussi pas mal de préciser un peu une fois tout de suite ce qu'est l'espace $\Lambda W$.
\end{probleme}

One can find a change of basis which express the Dirac operator in terms of vectors on $R_1$. For that, let $\{ X_i \}$ be a basis of $\sR_1$. We have
\[
  X^*_i[u]=\Dsdd{ [ e^{-tX_i}u] }{t}{0}=-d\pi dR_uX_i
\]
that is necessarily decomposable by corollary~\ref{Cordpiietwii} as combinations of vectors of the form $d\pi dL_uq_i$ because $[u]$ belongs to an open orbit of the action of $R_1$. That defines a matrix $B$ by
\[
  d\pi dL_uq_i=B_{ij}d\pi dR_uX_j,
\]
and then a vector $Y\in \sH$ by
\begin{equation}
q_i=B_{ij}\Ad(u^{-1})X_j+Y.
\end{equation}
Now we have $\tilde t_i(a)=dL_a\psi\big( \Ad(u^{-1})B_{ij}X_j+Y \big)$. We can go further using the fact that 
\begin{equation}
    \psi\Big( \Ad\big( h^{-1}(a) \big)X \Big)=\Ad(a)\psi(X)
\end{equation}
for every $a\in\SP(2,\eR)$ and $X\in\sG$. Defining the vectors $s_i=\Ad\big( h(^{-1}) \big)\psi X_i$ we find
\begin{equation}
\tilde t_i(a)=B_{ij}\tilde s_i(a)+\widetilde{\psi(Y)}(a).
\end{equation}

\subsection{Frame bundle}\index{frame!bundle!on $AdS_{l}$}
%------------------------

Construction of the frame bundle and the spin structure is a straightforward adaptation of theorem 2.2 (chapter???) in \cite{AnnikFranc}, while Dirac operator and connection issues are adapted from proposition 1.3 (chapter III)

A \defe{basis}{basis} of a $m$ dimensional vector space $V$ is a free and generating part; it only has the structure of a set. A frame of the vector space $V$ is a nondegenerate map $b\colon \eR^{m}\to V$. Let us give an example in three dimensions the difference. If $\{ v_{1},v_{2},v_{3} \}$ is a basis of $V$, of course $\{ v_{2},v_{1},v_{3} \}$ is the same basis. Order has no importance. But if $\{ e_{1},e_{2},e_{3} \}$ is the canonical basis of $\eR^{3}$, the \emph{frames} $b(e_{1})=v_1$, $b(e_{2})=v_2$, $b(e_{3})=v_{3}$ and $c(e_{1})=v_2$, $c(e_{2})=v_1$, $c(e_{3})=v_{3}$ are not the same.

Now we consider $AdS_l=G/H=\SO(2,l-1)/\SO(1,l-1)$, the Lie algebra $\sG$ has a reductive homogeneous space decomposition $\sG=\sQ\oplus\sH$ and we consider the canonical projection $\pi\colon G\to AdS_l$.

Let the map (see relation \eqref{EqInclAdHSOq})
\begin{equation}
\begin{aligned}
 \alpha\colon H&\to \SO(\sQ) \\
h&\mapsto \Ad(h)|_{\sQ}.
\end{aligned}
\end{equation}
We consider, on $G\times\SO(\sQ)$, the equivalence relation $(g,A)\sim(g',A')$ if and only if there exists $h\in H$ such that $g'=gh$ and $A'=\alpha(h^{-1})A$. We denote by $G\times_{\alpha}\SO(\sQ)$ the set of equivalence classes. Now we have a principal bundle
\begin{equation}   \label{EqPrincPreB}
\xymatrix{%
   \SO(\sQ) \ar@{~>}[r]     &   G\times_{\alpha}\SO(\sQ)\ar[d]^{p}\\
                &      G/H
}
\end{equation}
where $p[g,A]=[g]$ and the action is given by $[g,A]\cdot B=[g,A]$. The fact that the projection fulfils $p\big( [g,A]\cdot B \big)=p[g,A]$ is evident, and the fact that the action is well defined is a simple computation:
if $[g',A']=[g,A]$, we have a $h\in H$ such that
\[
  [g',A']\cdot B=[g',A'B]
        =[gh,\alpha(h^{-1})AB]
        =[g,AB]
        =[g,A]\cdot B.
\]

\begin{proposition}
Let $\tau(g)\colon AdS_l\to AdS_l$ be the action of $g\in G$ on $AdS_l$: $\tau(g)[g']=[gg']$, and $B$ be the frame bundle. We also consider the map $\sigma\colon \eR^{1,l-1}\to \sQ$ the isometry which sends the canonical basis of $\eR^{1,l-1}$ to the usual basis $\{ q_0,q_{1},\ldots,q_{l-1} \}$ of $\sQ$. The map
\begin{equation}
\begin{aligned}
 \beta\colon G\times_{\alpha} \SO(\sQ)&\to B \\
[g,A]&\mapsto d\tau(g)_{\mfo}A\circ\sigma
\end{aligned}
\end{equation}
provides a principal bundle isomorphism between the principal bundle  \eqref{EqPrincPreB} and the frame bundle over $AdS_l$.

\end{proposition}

By abuse of notation, we will not always write the $\sigma$.
\begin{proof}
We have to prove first that the map $\beta\colon G\times\SO(\sQ)\to B$ respects the classes. For that, consider $(g,A)\sim(g',A')$ and remark that
\[
\begin{split}
\beta(gh,\alpha(h^{-1}))&=d\tau(gh)_{\mfo}\alpha(h^{-1})A
        =d\tau(g)d\tau(h)d\pi \Ad(h^{-1})d\pi^{-1}A\\
        &=d\tau(g)d\tau(h)d\pi\Ad(h^{-1})d\pi^{-1} A
        =d\tau(g)d\pi dR_{h}d\pi^{-1}A\\
        &=d\tau(g)_{\mfo} A
        =\beta(g,A).
\end{split}
\]
where we used equation \eqref{EqdpiAdpi} and the fact that $\pi\circ L_{g}=\tau(g)\circ\pi$. The frame bundle is
\begin{equation}   \label{EqPrincB}
\xymatrix{%
   \SO(1,l-1) \ar@{~>}[r]   &   B \ar[d]^{p}\\
                &      G/H
}
\end{equation}
where the fibre $B_{[g]}$ in $B$ over $[g]$ is the set of isometric maps $\eR^{1,l-1}\to T_{[g]}(AdS_l)$. So an element of $B$ is of the form $\big( [g],\tilde f\circ\sigma \big)$ where $g\in G$ and $\tilde f\colon \sQ\to T_{[g]}(AdS_l)$ contains the main information while $\sigma$ is the previously explained isometry. The action of $h\in\SO(1,l-1)$ on $\big( [g],\tilde f\circ\sigma \big)$ is defined by means of any fixed isomorphism $\varphi_{0}\colon \SO(1,l-1)\to \SO(\sQ)$ by
\begin{equation}
  \big( [g],\tilde f\circ\sigma \big)\cdot h=\big( [g],\tilde f\circ\varphi_{0}(h)\circ\sigma \big).
\end{equation}
The map $\beta$ is a morphism of principal bundle because
\[
\beta[g,A]\cdot\varphi_{0}^{-1}(B)  =\big( [g],d\tau(g)A\circ\sigma \big)\cdot \varphi_{0}^{-1}(B)
                    =\big( [g],d\tau(g)A\circ B\circ \sigma \big)
                    =\beta\big( [g,A]\cdot B \big).
\]
It remains to be proved that $\beta$ is a bijection. Surjectivity is natural: since $d\tau(g)$ is an isometry, $d\tau(g)A$ runs over the whole $\SO\big(T_{[g]}(AdS_l)\big)$ when $A$ runs over $\SO(\sQ)$. Injectivity is as follows; let's suppose $\beta[g,A]=\beta[g',A']$. It is immediate that in this case, $\exists h\in H$ such that $g'=gh$. Using the fact that $d\pi\circ dR_{h^{-1}}\circ d\pi^{-1}=\id$ and $d\tau(h)d\pi=d\pi dL_{h}$, we have
\[
d\tau(g)_{\mfo}A=d\tau(g)d\tau(h)A'
        =d\tau(g)d\tau(h)d\pi dR_{h^{-1}}d\pi^{-1}A'
        =d\pi\Ad(h)d\pi^{-1}A'
        =\alpha(h)A'.
\]

\end{proof}
From now on, we identify $G\times_{\alpha}\SO(\sQ)$ with the frame bundle over $AdS_l$.

\subsection{Spin structure}
%--------------------------

We consider the principal bundle
\begin{equation}
\xymatrix{%
   \Spin(1,l-1) \ar@{~>}[r]     &   G\times_{\tilde\alpha}\Spin(1,l-1)\ar[d]^{p}\\
                &      G/H
}
\end{equation}
where $\times_{\tilde\alpha}$ is the following equivalence relation on $G\times\Spin(1,l-1)$. We say that $(g,s)\sim(g',s')$ if and only if there exists a $h\in H$ such that
\begin{enumerate}
\item $g'=gh$,
\item $\chi(s')=\Ad(h^{-1})\chi(s)$.
\end{enumerate}
Notice that the second condition implies that $\Ad(h)\in \SO_0(\sQ)$. It is easy to prove that the given structure is well defined and is a principal bundle.  Now we consider the spin structure as follows:
\begin{equation}
\xymatrix{%
   \Spin(1,l-1) \ar@{~>}[r]&G\times_{\tilde\alpha}\Spin(1,l-1)\ar[dr]\ar[rr]^-{\displaystyle\varphi}&&G\times_{\alpha} \SO(\sQ)\ar[dl]&\SO(\sQ)\ar@{~>}[l]  \\
                &      &G/H
}
\end{equation}
where $\varphi[g,s]=[g,\chi(s)]$. It is well defined since when $[g,s]=[g',s']$, there exists a $h\in H$ with $\chi(s')=\Ad(h^{-1})\chi(s)$ such that  $\varphi[g',s']=\varphi[gh,s']=[gh,\chi(s')]=[gh,\Ad(h^{-1})\chi(s)]=[g,\chi(s)]=\varphi[g,s]$.

\subsection{Reduction of the structural group}
%---------------------------------------------

The case of $AdS_l$ can be seen in the setting of subsection~\ref{subsecCanConCovDer}. Let us show now that the bundle
\begin{equation}   \label{EqPrincHzGM}
\xymatrix{%
   H_0 \ar@{~>}[r]      &   G\ar[d]^{\pi}\\
                &   G/H
 }
\end{equation}
is a reduction to $H_0$ (the identity component of $\SO(\sQ)$) of
\begin{equation}
\xymatrix{%
   G \ar@{~>}[r]        &   r(G)\ar[d]^{\pi}\\
                &   G/H.
 }
\end{equation}
 Indeed, $u\colon G\to r(G)$ given by $u(g)=r(g)$ provides the reduction homomorphism: $r(gh)X=d\pi dL_{gh}X$ while $\big( r(g)\cdot h \big)X$ is the same.

\begin{lemma}
The tangent space $T(G/H)$ is an associated bundle of $r(G)$ trough the identification
\begin{equation}
\begin{aligned}
 \beta'\colon r(G)\times_{\rho}\sQ&\to T(G/H) \\
  [r(g),X]&\mapsto r(g)X
\end{aligned}
\end{equation}
where $\rho(h)X=\Ad(h)X$, so that the quotient is given by $[g,X]=[gh,\Ad(h^{-1})X]$.
\end{lemma}

\begin{proof}
The proof is entirely similar to the one of lemma~\ref{LemBazHGGH}.
\end{proof}

\input{conclThese}

\chapter{General non commutative geometry}
% This is part of (almost) Everything I know in mathematics
% Copyright (c) 2013-2014, 2020
%   Laurent Claessens
% See the file fdl-1.3.txt for copying conditions.

My main references for noncommutative geometry are \cite{Landi,ConnesNCG,ConnesMotives,itoNCG_Varilly}.

%%%%%%%%%%%%%%%%%%%%%%%%%%
%
   \section{Non commutative differential forms}
%
%%%%%%%%%%%%%%%%%%%%%%%%

\subsection{Universal differential forms}
%----------------------------------------

Let $\cA$ be an associative unital algebra on $\eC$. We are going to define step by step the \defe{universal algebra}{universal!algebra} of differential forms \nomenclature{$\Omega\cA$}{Universal algebra of $\cA$}
\[
  \Omega\cA=\bigoplus_p\Omega^p\cA.
\]
Existence the universal algebra will be proved by explicitly construction later. For unicity, we will prove an universality property.

First of all, $\Omega^0\cA=\cA$. Next, $\Omega^1\cA$ is the left $\cA$-module generated by the symbols $\delta a$ with $a\in\cA$ and relations
\begin{subequations}
\begin{align}
\delta(ab)&=(\delta a)b+a\delta b				\label{seq_deltaabi}\\
\delta(\alpha a+\beta b)&=\alpha\delta a+\beta\delta b
\end{align}
\end{subequations}
for all $a$, $b\in\cA$ and $\alpha,\beta\in\eC$. A general element in $\Omega^1\cA$ is of the form $\sum_ia_i\delta b_i$, with $a_i$ and $b_i$ in $\cA$.

Equation \eqref{seq_deltaabi} with $a=1$ gives $(\delta 1)b=0$ for all $b\in\cA$, hence $\delta(1)=0$ and $\delta(\eC)=0$. So $\Omega^1\cA$ is a left $\cA$-module; we can give a structure of right $\cA$-module by defining
\[
  (\sum_ia_i\delta b_i)c=\sum_ia_i(\delta b_i)c,
\]
but equation \eqref{seq_deltaabi} gives $(\delta a)c=\delta(ac)-a\delta c$, therefore if $\omega=\sum_ia_i\delta b_i$,
\begin{equation}
\omega c=\sum_i a_i\delta(b_ic)-\sum a_ib_i\delta c.
\end{equation}
The rule \eqref{seq_deltaabi} is Leibnitz for the map $\delta\colon \cA\to \Omega^1\cA$, so we see $\delta$ as a derivation of $\cA$ with values in the bimodule $\Omega^1\cA$.


\subsubsection{Universal properties}
%///////////////////////////////////

The following proposition gives an universal property of $\Omega^1\cA$; in a certain sense, it is unique.

\begin{proposition}
Let $\modM$ be a $\cA$-bimodule and $\Delta\colon \cA\to \modM$ a derivation, i.e.
\[
  \Delta(ab)=(\Delta a)b+a\Delta b.
\]
There exists one and only one bimodule morphism $\rho_{\Delta}\colon \Omega^1\cA\to \modM$ such that $\Delta=\rho_{\Delta}\circ\delta$, i.e. the following diagram commutes:
\[
  \xymatrix{%
   \Omega^1\cA 		&	\\
   \cA \ar[r]_{\Delta}\ar[u]^{\delta}	&	\modM\ar@{.>}[ul]_{\rho_{\Delta}}
}
\]
\label{prop_modMununique}
\end{proposition}

\begin{proof}
First, remark that any bimodule morphism $\rho\colon \Omega^1\cA\to \modM$ makes $\rho\circ\delta$ a derivation with values in $\modM$. Indeed
\[
  (\rho\circ\delta)(ab)=\rho\big( (\delta a)b+a\delta b \big)
			=(\rho\circ\delta)(a)b+a(\rho\circ\delta)b
\]
Let us now prove the inverse: let $\Delta\colon \cA\to \modM$ be a derivation and $\rho_{\Delta}\colon \Omega^1\cA\to \modM$ be such that $\Delta=\rho_{\Delta}\circ\delta$. We want to prove unicity of this derivation. First, definition of $\Delta$ makes
\begin{equation} \label{eq_rhpoDelta}
\rho_{\Delta}(\delta a)=\Delta a.
\end{equation}
This completely defines $\rho_{\Delta}$ from $\Delta$ because $\delta$ generates the whole $\Omega^1\cA$ as left $\cA$-module. Indeed the only way to extends $\rho_{\Delta}$ from \eqref{eq_rhpoDelta} as a morphism on $\Omega^1\cA$ is
\[
  \rho_{\Delta}\big( \sum_ia_i\delta b_i \big)=\sum_ia_i\Delta b_i.
\]
Now it is sufficient to prove that \eqref{eq_rhpoDelta} is a bimodule morphism:
\[
\begin{split}
\rho_{\Delta}\big( f(\sum a_i\delta b_i)g \big)&=\rho_{\Delta}\big( \sum_i fa_i[\delta(b_ig)-b_i\delta g] \big)\\
		&=\sum fa_i[\Delta(b_ig)-b_i\Delta g]\\
		&=\sum fa_i(\Delta b_i)g\\
		&=f\big( \sum ai\Delta b_i \big)g.
\end{split}
\]

\end{proof}

The space $\Omega^p\cA$ is defined by
\[
    \Omega^p\cA:=\underbrace{\Omega^1\cA\ldots\Omega^1\cA}_{p \text{times}}
\]
with multiplication rule
\begin{equation}
  (a_0\delta a_1)(b_0\delta b_1):=a_0(\delta a_1)b_0\delta b_1
		=a_0\delta(a_1b_0)\delta b_1-a_0a_1\delta b_0\delta b_1.
\end{equation}
This rule serves to show how to write elements of $\Omega^2\cA$ under the form $a\delta b_1\delta b_2$. Elements of $\Omega^p\cA$ are linear combination of elements of the form
\[
  \omega=a_0\delta a_1\delta a_2\ldots \delta a_p
\]
with $a_k\in\cA$, and the product $\Omega^p\cA\times\Omega^q\cA\to\Omega^{p+q}\cA$ is a juxtaposition and a rearrangement:
\[
\begin{split}
  (a_1\delta a_1\ldots\delta a_p)&(a_{p+1}\delta a_{p+2}\ldots\delta a_{p+q})\\
		&:=a_0\delta a_1\ldots (\delta a_p)a_{p+1}\delta a_{p+2}\ldots\delta a_{p+q}\\
	&=a_0\delta a_1\ldots \big[ \delta(a_pa_{p+1})-a_p(\delta a_{p+1}) \big]\delta a_{p+2}\ldots\delta a_{p+q}\\
	&=a_0\delta a_1\ldots\delta(a_pa_{p+1})\delta a_{p+2}\ldots\delta a_{p+q}\\
	&\quad -a_0\delta a_1\ldots\big[ \delta(a_{p-1}a_p)-a_{p-1}\delta a_p \big]\delta a_{p+1}\ldots\delta a_{p+q}.
\end{split}
\]
With $p=q=3$ for example, we have
\[
\begin{split}
  (a_0\delta a_1\delta a_2)(a_3\delta a_4\delta a_5)&=a_0\delta a_1(\delta a_2)a_3\delta a_4\delta a_5\\
		&=a_0\delta a_1\big[ \delta(a_2 a_4)-a_2\delta a_3 \big]\delta a_4\delta a_5\\
		&=a_0\delta a_1\delta(a_2 a_3)\delta a_4\delta a_5\\
		&\quad -a_0\big[ \delta(a_1 a_2)-a_1\delta a_2 \big]\delta a_3\ldots\delta a_5\\
		&=a_0\delta a_1\delta(a_2a_3)\delta a_4\delta a_5\\
		&\quad-a_0\delta(a_1a_2)\delta a_3\ldots\delta a_5\\
		&\quad+a_0a_1\delta a_2\ldots\delta a_5,
\end{split}
\]
and in general,
\begin{equation}    \label{Eq_decmProdConnForDelta}
\begin{split}
(a_0\delta a_1\ldots \delta a_p)&(a_{p+1}\delta a_{p+2}\ldots\delta a_{p+q})\\&=(-1)^pa_0a_1\delta a_2\ldots\delta a_{p+q}\\
	&\quad+\sum_{i=1}^p(-1)^{p-i}a_0\delta a_1\ldots\delta a_{i-1}\delta(a_ia_{i+1})\delta a_{i+2}\ldots\delta a_{p+q}.
\end{split}
\end{equation}
So $\Omega\cA$ is a left $\cA$-module. We turn it into $\cA$-bimodule by defining
\begin{equation}
\begin{split}
(a_0\delta a_1\ldots\delta a_p)b&=a_0\delta a_1\ldots(\delta a_p)b\\
		&=(-1)^pa_0a_1\delta a_2\ldots \delta a_p b\\
		&\quad+\sum_{i=1}^{p-1}(-1)^{p-i}a_0 \delta a_1\ldots \delta a_{i-1}\delta(a_ia_{i+1})\delta_{i+1}\ldots\delta a_p\delta b\\
		&\quad+a_0\delta a_1\ldots\delta a_{p-1}\delta(a_pb).
\end{split}
\end{equation}

Now we put a differential algebra structure on the $\cA$-bimodule $\Omega\cA$ by extending $\delta$ to
\begin{equation}
\begin{aligned}
 \delta\colon\Omega^p\cA&\to \Omega^{p+1}\cA \\
\delta(a_0\delta a_1\ldots\delta a_p)&:= \delta a_0\delta a_1\ldots \delta a_p.
\end{aligned}
\end{equation}
One checks that
\[
  \delta^2=0
\]
and
\begin{equation}
  \delta(\omega_1\omega_2)=(\delta\omega_1)\omega_2+(-1)^p\omega_1\delta\omega_2
\end{equation}
for any $\omega_1\in\Omega^p\cA$ and $\omega_2\in\Omega\cA$.

If $\modE$ is a $\cA$-module, an element $\alpha\in\End_{\cA}(\modE)\otimes\Omega^1\cA$ acts on an element of $\modE\otimes_{\cA}\Omega^p\cA$ in the following way. If $\alpha=\sum_i\big( A_i\otimes_{\cA}a^i_0\delta a^i_1 \big)$, we define
\begin{equation}		\label{EqActallphaEOAp}
\alpha\big( \sum_j\xi_j\otimes_{\cA}\omega_j \big)=\sum_{ij}\big( A_i\xi_j\otimes_{\cA}a^i_0\delta a^i_1\omega_j \big)\in\modE\otimes_{\cA}\Omega^{p+1}\cA
\end{equation}
where $A_i\in\End_{\cA}(\modE)$, $a^i_j\in\cA$, $\xi_j\in\modE$ and $\omega_j\in\Omega^p\cA$. The element $\alpha$ acts in particular on $\modE$ \emph{via} the identification $\xi\in\modE\leftrightarrow\xi\otimes_{\cA}1\in\modE\otimes_{\cA}\Omega^0\cA$. In this case the result is denoted by $\alpha(\xi)$.

The following proposition gives the same type of result as proposition ~\ref{prop_modMununique}.

\begin{proposition}
Let $(\Gamma,\Delta)$ a differential graded algebra and $\rho\colon \cA\to \Gamma^0$, a morphism of unital algebras.
There exists one and only one extension of $\rho$ into a differential graded algebra morphism $\tilde\rho\colon \Omega\cA\to \Gamma$ with $\tilde\rho\circ\delta=\Delta\circ\tilde\rho$:
\begin{equation}   \label{eq_diagrhotrho}
  \xymatrix{%
   \Gamma^p \ar[r]^{\Delta}\ar@{.>}[d]_{\tilde\rho}	&	\Gamma^{p+1}\ar[d]^{\tilde\rho}\\
   \Omega^p\cA \ar[r]_{\delta}				&	\Omega^{p+1}\cA
}
\end{equation}
\end{proposition}

\begin{proof}
The map $\rho\colon \cA\to \Gamma^0$ being given, we define
\begin{equation}
\begin{aligned}
 \tilde\rho\colon\Omega^p\cA&\to \Gamma^p \\
\tilde\rho(a_0\delta a_1\ldots\delta a_p)&:= \rho(a_0)\Delta(\rho (a_1))\ldots\Delta( \rho(a_p)).
\end{aligned}
\end{equation}
It well defines $\tilde\rho$ since $\Omega^p\cA$ is generated by $a_0\delta a_1\ldots\delta a_p$. It sends products on products; for example with $p=2$,
\[
\begin{split}
  \tilde\rho\big[ (a_0\delta a_1)(b_0\delta b_1) \big]&=\rho(a_0)\Delta\rho(a_1b_0)\Delta\rho b_1\\
		&\quad-\rho(a_0a_1)\Delta\rho b_0\Delta\rho b_1\\
		&=\big[ \rho(a_0)\Delta\rho(a_1) \big]\big[ \rho(b_0)\Delta\rho b_1 \big].
\end{split}
\]
Commutativity of diagram \eqref{eq_diagrhotrho} is as follows:
\[
\begin{split}
(\tilde\rho\circ\Delta)(a_0\delta a_1\ldots\delta a_p)&=\tilde\rho(\delta a_0\ldots\delta a_p)\\
		&=\Delta\rho a_0\ldots\Delta\rho a_p\\
		&=\Delta\big( \rho(a_0)\Delta\rho a_1\ldots\Delta\rho a_p \big)\\
		&=(\Delta\circ\tilde\rho)(a_0\delta a_1\ldots\delta a_p).
\end{split}
\]

\end{proof}

\subsubsection{Cohomology}
%/////////////////////////

From definition,
\[
  \delta(a_0\delta a_1\ldots\delta a_p)=\delta a_0\ldots\delta a_p.
\]
If $\omega=a_0\delta a_1\ldots\delta a_p$, the only way to have $\delta\omega=0$ is $a_0=1$, i.e. $\omega=\delta(a_1\delta a_2\ldots\delta a_p)$. Then
\[
  H^p(\Omega\cA)=0
\]
when $p\neq 0$ because any closed form is exact. In the case $p=0$,
\[
  H^0(\Omega\cA)=\eC
\]
because $\delta(z)=0$ for all $z\in\eC$ while $z$ is not exact.

\subsubsection{Isomorphisms}
%//////////////////////////

\begin{lemma}
As bimodule, $\Omega^1(\cA)$ is isomorphic to $\ker m$ where $m\colon \cA\otimes_{\eC}\cA\to \cA$ is the multiplication map. The isomorphism is given by
\begin{equation}   \label{EqDEfphirov}
\begin{aligned}
 \varphi\colon \ker m&\to \Omega^1(\cA) \\
\sum_{j}a_i\otimes b_i&\mapsto \sum_{j}a_idb_i.
\end{aligned}
\end{equation}

\end{lemma}

\begin{proof}
First remark that $\ker m$ is generated by elements of the form $1\otimes_{\eC}a-a\otimes_{\eC}1$. Indeed if $\sum_{j}a_jb_j=m\big( \sum_{i}a_j\otimes b_j \big)=0$, we have $\sum_{j}a_j\otimes_{\eC}b_j=\sum_{j}a_j(1\otimes_{\eC} b_j-b_j\otimes_{\eC}b_j)$.

Now consider the map
\begin{equation}
\begin{aligned}
 \Delta\colon \cA&\to \ker m \\
\Delta a&= 1\otimes_{\eC}a-a\otimes_{\eC}1
\end{aligned}
\end{equation}
This map satisfies $\Delta(ab)=(\Delta a)b+a\Delta b$. Now we prove that the equation \eqref{EqDEfphirov} is surjective: $\sum_{j}a_jdb_j=\varphi\big( \sum_{j}a_j(1\otimes b_j-b_j\otimes 1) \big)$ where indeed $m\big( \sum_{j}a_j(1\otimes b_j-b_j\otimes 1) \big)=0$.

The injectivity is evident because $\varphi\big( \sum_{j}a_j\otimes b_j \big)=\varphi\big( \sum_{j}a_j'\otimes b_j' \big)$ implies $\sum_{j}a_jdb_j=\sum_{j}a_j'db_j'$.

The fact that $\varphi$ provides an isomorphism of bimodule is
\[
	\varphi\big( c(a_i\otimes b_i)y \big)=\varphi\big( (xa_i)\otimes(b_i) \big)
		=xa_id(b_iy)
		=xa_i(db_i)y+x\underbrace{a_ib_i}_{=0}dy
		=x\varphi(a_i\otimes b_i)y.
\]
\end{proof}

\subsubsection{Involution}
%/////////////////////////

If the algebra $\cA$ has an involution (which is often the complex conjugation when $\cA$ is a function algebra), the universal algebra $\Omega\cA$ becomes an involutive algebra with the definition
\begin{align*}
(\delta a)^*&=-\delta(a^*),\\
(a_0\delta a_1\cdots\delta a_p)^*&=(\delta a_p)^*\cdots(\delta a_1)^*a_0^*.
\end{align*}


\subsection{Connes differential forms}
%+++++++++++++++++++++++++++++++++++

\begin{proposition}
The formula
\[
  \pi(a^0\delta a^1\cdots \delta a^{n})=a^0[D,a^1]\cdots[D,a^{n}]
\]
defines a $*$-representation of the reduced universal algebra on $\hH$.
\end{proposition}
\begin{proof}
No proof.
\end{proof}

\begin{proposition}
Let $J_0=\ker\pi$ and $J_0^{(k)}=\{ \omega\in\Omega^k(\cA)\tq \pi(\omega)=0 \}$. In this case, $J:=J_0+\delta J_0$ is a graded differential two-sided ideal of $\Omega^*(\cA)$.
\end{proposition}

\begin{proof}
The fact that $J$ is differential comes from the fact that $d^2=0$, but one has to remark that $J_0$ is not differential by itself because there exists some \defe{junk form}{junk form} $\omega$ such that $\pi(\omega)=0$ and $\pi(d\omega)\neq 0$. In order to see that $J$ is a left ideal, consider $\omega\in J^{(k)}$:
\[
  \omega=\omega_1+\delta\omega_2
\]
with $\omega_1\in J_0\cap\Omega^k$ and $\omega_2\in J_0\cap\Omega^{k-1}$. If $\omega'\in\Omega^l$, we have
\[
  \omega\omega'=\big( \omega_1\omega'+(-1)^k\omega_2\delta\omega' \big)+\delta(\omega_2\omega')\in J^{(k+l)}.
\]
The right side is proven in the same way.
\end{proof}

Since $J$ is an ideal, one can define
\[
  \Omega^*_D(\cA)=\Omega^*(\cA)/J.
\]
One can prove that for all $k\in\eN$, $\Omega^k_D(\cA)\simeq \pi(\Omega^k(\cA))/\pi\big( d(J_0\cap\Omega^{k-1}) \big)$. We consider the product
\begin{equation}
	\langle T_1,\,T_2\rangle=\tr_{\omega}(T_2^*T_1| D |^-d)
\end{equation}
on $\pi(\Omega^k)$, and we define $\hH_k$ as the completion of $\pi(\Omega^k)$ for this product. This provides a Hilbert space in which $\pi\big( d(J_0\cap\Omega^{k-1}) \big)$ is a subspace. We denote by $P$ the projection parallel to this space.

\begin{proposition}
For all $\omega_i\in\pi\big( \Omega^k(\cA) \big)$,
\[
  \langle P\omega_1,\,\omega_2\rangle=\langle P\omega_1,\,P\omega_2\rangle.
\]

\end{proposition}
\begin{proof}
No proof.
\end{proof}
We denote by $\Lambda^k$ the completion of $\Omega^k_D$ with respect to this inner product. In fact, the proposition shows that the inner product can be used on $\pi(\Omega^k)$.



Let $(\cA,\hH,D)$ be a spectral triple, and $\Omega\cA$, the universal algebra of $\cA$. We consider the map
\begin{equation}	\label{EqDEfpirep}
\begin{aligned}
 \pi\colon\Omega\cA&\to \oB(\hH) \\
a_0\delta a_1\ldots\delta a_p&\mapsto a_0\circ[D,a_1]\circ\ldots\circ [D,a_p]
\end{aligned}
\end{equation}
for $a_i\in\cA$. Since the operations $\delta$ and $D$ both are derivations, one can expect that $\pi$ will be a homomorphism.

\begin{proposition}
The operation $\pi$ is a homomorphism.
\end{proposition}
\begin{proof}
We will just check it in the case of $1$-forms. First, remark that
\[
 \begin{split}
(a_0\delta a_1)(b_0\delta b_1)&=a_0\big( \delta(a_1b_0)-a_1\delta b_0 \big)\delta b_1\\
		&=a_0\delta(a_1b_0)\delta b_1-a_0a_1\delta b_0\delta b_1.
\end{split}
\]
Thus we have
\[
 \begin{split}
\pi\big( (a_0\delta a_1)(b_0\delta b_1) \big)&=a_0[D,a_1b_0][D,b_1]-a_0a_1[D,b_0][d,b_1]\\
		&=a_0[D,a_1]b_0[D,b_1]\\
  		&=\pi(a_0\delta a_1)\pi(b_0\delta b_1)
\end{split}
\]

\end{proof}

\subsubsection{Junk forms}
%/////////////////////////

One cannot define $\pi(\Omega\cA)$ as differential forms because there exists some $\omega\in\Omega\cA$ such that $\pi(\omega)=0$ and $\pi(\delta\omega)\neq 0$. Such a form is said to be \defe{junk}{junk form}. We define
\begin{equation}  \label{eq_defJConnes}
  J_0^p=\{ \omega\in\Omega^p\cA\tq \pi(\omega)=0 \}
\end{equation}
 and $J=J_0+\delta J_0$. Then we define the \defe{Connes differential forms}{Connes differential form} as\nomenclature{$\Omega_D\cA$}{Connes differential forms}
\begin{equation}
\Omega_D\cA=\Omega\cA/J.
\end{equation}

\begin{proposition}
Let $J_0=\bigoplus_pJ_0^p$, the two-sided graded ideal of $\Omega\cA$ generated by \eqref{eq_defJConnes}.  The set $J=J_0/\delta J$ is a two-sided graded ideal of $\Omega\cA$.
\end{proposition}

\begin{proof}
We begin by proving that $J$ is a differential algebra for the same $\delta$ as for $\cA$; the fact that $\delta^2=0$ is not a question. We have to prove that $\delta$ is internal in $J$. A general element of $J$ is $\omega=\alpha+\delta\beta$ with $\pi(\alpha)=\pi(\beta)=0$. In this case, $\delta\omega=\delta\alpha$ is of the same form. Now we want to prove that $J$ is a two-sided ideal. Let $\omega=\omega_1+\delta\omega_2\in J^p$ with $\omega_1\in J_0^p$ and $\omega_2\in J_0^{p-1}$ and consider $\eta\in\Omega^q\cA$. We have
\[
  \omega\eta=\omega_1\eta+(\delta\omega_2)\eta
		=\omega_1\eta+\delta(\omega_2\eta)-(-1)^{p-1}\omega_2\delta\eta.
\]
The first term fulfil $\pi(\omega_1\eta)=\pi(\omega_1)\pi(\eta)=0$, the second one is the $\delta$ of something whose $\pi$ is zero  while the $\pi$ of the third one is zero. So $\omega\eta\in J$. In the same manner, we conclude that $\eta\omega\in J$ too.
\end{proof}

\begin{lemma}
We have
\begin{equation}
  \Omega\cA/J\simeq \pi(\Omega\cA)/\pi(\delta J_0).
\end{equation}
\label{lem_OCAisomppiOA}
\end{lemma}

\begin{proof}
On element of $\Omega\cA$ is of the form $[\omega]\sim [\omega+\omega_1+\delta\omega_2]$ with $\pi(\omega_1)=\pi(\omega_2)=0$. We define
\begin{equation}
\begin{aligned}
 \psi\colon \Omega\cA/J&\to \pi(\Omega\cA)/\pi(\delta J_0) \\
[\omega]&\mapsto [\pi(\omega)]_B
\end{aligned}
\end{equation}
where the class $[\cdots]_B$ is modulo $\pi(\delta J_0)$, in other words when $\pi(\omega)=0$, we have $[A]_B=[A+\pi(\delta\omega)]$.  The map $\psi$ is well defined because, when $\pi(\omega_1)=\pi(\omega_2)=0$,
\[
\psi[\omega+\omega_1+\delta\omega_2]=[\pi(\omega+\omega_1+\delta\omega_2)]_B
		=[\pi(\omega)]_B+[\pi(\delta\omega_2)]_B
		=[\pi(\omega)]
		=\psi[\omega].
\]

First suppose that $\psi[\omega]=0$, i.e $\psi[\omega]=[\pi(\delta\eta)]_B$ with $\pi(\eta)=0$. Then $\omega=\delta\eta+\sigma$ with $\sigma=0$. This is the definition of $[\omega]=0$. This proves that $\psi$ is injective. For surjectivity, consider $[\omega]_B\in\pi(\Omega\cA)/\pi(\delta J_0)$ and $\pi(\omega)$ a representative in $\pi(\Omega\cA)$. For this $\omega$, we have $\psi[\omega]=[\omega]_B$.
\end{proof}

\subsubsection{Grading the Connes differential forms}
%////////////////////////////////////////////////////

\begin{lemma}			\label{LemOmpmdDp}
The isomorphism of lemma~\ref{lem_OCAisomppiOA} induces a grading
\begin{equation} \label{eq_OpDAsimeqOpA}
  \Omega^p_D\cA\simeq \Omega^p\cA/J^p,
\end{equation}
and the differential
\begin{equation}
\begin{aligned}
 d\colon \Omega^p_D\cA&\to \Omega^{p+1}_D\cA \\
[\omega]&\mapsto [\delta\omega]
\end{aligned}
\end{equation}
is well defined.
  \label{lem_isomgraODA}
\end{lemma}

\begin{proof}
 The well definiteness of the differential is because when $\pi(\omega_1)=\pi(\omega_2)$,
\[
d[\omega+\omega_1+\delta\omega_2]=[\delta\omega+\delta\omega_1+\delta^2\omega_2]
		=[\delta\omega].
\]
The isomorphism \eqref{eq_OpDAsimeqOpA} is given by   $\psi[\omega^p]=[\pi(\omega^p)+\pi(\delta\eta)]$ with $\pi(\eta)=0$ when $[\omega^p]\in\Omega^p\cA/J^p$, i.e. when $\omega^p\in\Omega^p\cA$.
\end{proof}

\subsubsection{\texorpdfstring{$0$}{t}-forms}
%///////////////////////

We have $J^0=J\cap\Omega^0\cA=J\cap\cA$ and $J^0=\{ a\in\cA\tq \pi(a)=0 \}$. As operators on $\hH$, $J^0=\{ 0 \}$. Therefore $\Omega^0_D\cA=\Omega^0\cA=\cA$.

Let us now briefly study the spaces of low degree forms.

\subsubsection{\texorpdfstring{$1$}{1}-forms}
%//////////////////////

From the isomorphism of lemma~\ref{lem_isomgraODA}, we begin to study $\delta J_0^0$
\[
  J_0^0=\{ \omega\in\cA\tq\pi(\omega)=0 \}.
\]
The crucial point is that $\pi(a)=0$ implies $a=0$ when $a\in\cA$ (is is not true for any $\omega\in\Omega\cA$!), so
\begin{equation}
\Omega^1\cA=\pi(\Omega^1\cA)
\end{equation}
and Connes $1$-forms are of the form
\[
  \omega_1=\sum_ja_0^j[D,a_1^j]
\]
with $a_i^j\in\cA$.

\subsubsection{Example on the canonical triple}
%///////////////////////////////////////////

%\begin{probleme}
%À ce propos, je te préviens que tu as un fichier nomé Dirac.dvi quelque part dans ta littérature qui explique la formule
%\[
%  [D,f]=c(df)
%\]
%quand $c$ est une action de Clifford sur un fibré vectoriel.
%\end{probleme}

Now we will use the $\gamma$ defined in subsection~\ref{susec_Cliffmodule}. When we consider the canonical triple $(\cA,\hH,D)$ on a manifold $M$, $\cA\subset\Fun(M)$. We know that $\cA$ acts on $\hH$ by $(f\psi)(x)=f(x)\psi(x)$, so that $[D,f]\psi=(\gamma^{\mu}\partial_{\mu}f)\psi$. So we say that
\[
  [D,f]=\gamma^{\mu}\partial_{\mu}f=\gamma(df).
\]
The Dirac operator act on functions as follows (see equation \eqref{eq_defDirac_f}):
\[
  (Df)(x)=g_{\alpha\beta}(x)\gamma^{\beta}_x(e_{\alpha}\cdot f),
\]
this definition is intended to get a Leibnitz rule for $D(f\psi)$. We have:
\[
[D,f]\psi(x)=D(f\psi)(x)-f(x)D\psi(x)\\
		=(Df)(x)\psi(x),
\]
so $[D,f]\psi=(Df)\psi$ and as operator on $\hH$, $[D,f]$ is the multiplicative operator by $Df$. When we consider a local orthonormal basis $e_{\alpha}$, we have
\[
(Df)(x)=g_{\alpha\beta}(x)\gamma_x^{\beta}(e_{\alpha}\cdot f)
		=\gamma^{\mu}\partial_{\mu}f(x).
\]
From all that we conclude that
\[
  [D,f]=\gamma^{\mu}\partial_{\mu}f,
\]
and therefore that, on the canonical triple,
\begin{equation}
\pi(\delta f)=[D,f]=\gamma^{\mu}\partial_{\mu}f.
\end{equation}
But $\gamma(df)=\partial_{\mu}f\gamma(dx^{\mu})=\gamma^{\mu}\partial_{\mu}f=\pi(\delta f)$. We will soon define the differential $d$ of $\Omega_D\cA$, so from now we denote by $d_M$ the usual differential of $M$ and we write $\pi(\delta f)=\gamma(d_Mf)$ and finally,
\begin{equation}
\pi(f_0\delta f_1\ldots \delta f_p)=f_0\gamma(d_Mf_1)\ldots\gamma(d_Mf_p).
\end{equation}
Note that $d_M\in\Gamma(M,\Cliff(M))$ and, since $\gamma$ is a morphism,
\begin{equation} \label{EqpildotsdeltagamdM}
  \pi(f_0\delta f_1\ldots\delta f_p)=f_0\gamma(d_Mf_1\cdot\ldots\cdot d_Mf_p)
\end{equation}
where $\cdot$ denotes the Clifford product.


\subsubsection{Differential \texorpdfstring{$0$}{0}-forms}
%-------------------------------------

\begin{lemma}
\[
\gamma(\Lambda^1(M))\simeq\Omega^1_D\cA
\]
\end{lemma}

\begin{proof}

A general $1$-form has the form $\sum_j f_0^jd_Mf_1^j$. Since $\gamma\colon \Gamma(M,\Cliff(M))\to \oB(\hH)$, we claim that the isomorphism is given by
\[
  \psi\big( f_0[D,f_1] \big)=\gamma\big( f_0d_Mf_1 \big).
\]
Surjectivity poses no problems because $f_0d_Mf_1$ is the general form of an element of $\Lambda^1(M)$. Now suppose that $\psi(f_0[D,f_1])=\gamma(f_0d_Mf_1)=0$. From linearity of $\gamma$,
\[
  f_0\gamma^{\mu}\partial_{\mu}f_1=0.
\]
At each point, either $f_0=0$ or $[D,f_1]=0$, so globally $f_0[D,f_1]=0$.
\end{proof}

\subsubsection{Differential \texorpdfstring{$1$}{1}-forms}
%-------------------------------------

Let $f\in\cA$ and
  $\alpha=\frac{ 1 }{2}\big( f\delta f-(\delta f)f \big)$.
We have
$(f\delta f)(x,y,z)=f(x)(\delta f)(y,z)
		=f(x)\big( f(y)-f(z) \big)$,
while
\[
  (\delta f)f(x,y,z)=\big( f(x)-f(y) \big)f(z).
\]

\begin{probleme}
	When we will speak about two points spaces, we will see that $(f\delta g)(x,y)=f(x)\big( g(y)-g(x) \big)$, and more or less the same for $(\delta f)g$. Thus what is done here is wrong and the correct result is
	\[
		2\alpha(x,y)=2f(x)f(y)-f(x)^{2}-f(y)^{2}.
	\]
	This does not change the conclusion, but it asks for a precise definition of $f\delta g$. Notice that, $f$ being a zero-form, and $\delta g$ a $1$-form, the product should be a $1$-form, and not a $2$-form.
\end{probleme}

This proves that $\alpha\neq 0$. The following computation uses the fact that the Leibnitz rule for $\delta$ is graded
\[
\delta\alpha=\frac{ 1 }{2}\big( \delta f\delta f+f\delta^2 f-(\delta^2 f)f+\delta f\delta f \big)
		=\delta f\delta f,
\]
so, on the one hand,
\[
\pi(\delta f)=\gamma^{\mu}\partial_{\mu}f\gamma^{\nu}\partial_{\nu}f
		=\frac{ 1 }{2}(\gamma^{\mu}\gamma^{\nu}+\gamma^{\nu}\gamma^{\mu})\partial_{\mu}f\partial_{\nu}f
		=-g^{\mu\nu}\partial_{\mu}f\partial_{\nu}f \mtu_{2^{[N/2]}}
\]
where $\mtu_{2^{[N/2]}}$ is the unit in the Clifford algebra of $\eR^n$. On the other hand,
\[
  \pi(\alpha)=\frac{ 1 }{2}\big( f\gamma^{\mu}\partial_{\mu}f-(\gamma^{\mu}\partial_{\mu}f)f \big)=0.
\]
This proves that $\alpha$ is a junk $1$-form.

\begin{probleme}
	It is also said that this is the general form of a junk, but I didn't succeed to prove it.
\end{probleme}

\subsubsection{Differential \texorpdfstring{$2$}{2}-forms}
%-----------------------------------

Let now take $f_{1}$, $f_{2}\in \cA$ and look at
\[
\begin{split}
  \gamma\big( d_{M}f_{1}\cdot d_{M}f_{2} \big)&=\gamma^{\mu}\gamma^{\nu}\partial_{\mu}f_{1j\partial_{\nu}}f_{2}\\
		&=\frac{1}{2}\big( \gamma^{\mu}\gamma^{\nu}+\gamma^{\mu}\gamma^{\nu}-\gamma^{\nu}\gamma^{\mu}+\gamma^{\nu}\gamma^{\mu} \big)\partial_{\mu}f_{1}\partial_{\nu}f_{2}\\
		&=\frac{1}{2}\big( \gamma^{\mu}\gamma^{\nu}+\gamma^{\nu}\gamma^{\mu} \big) \partial_{\mu}f_{1}\partial_{\nu}f_{2}
		 +\frac{1}{2}\big( \gamma^{\mu}\gamma^{\nu}-\gamma^{\nu}\gamma^{\mu} \big) \partial_{\mu}f_{1}\partial_{\nu}f_{2}.
\end{split}
\]
The first term is
\[
  -g^{\mu\nu}\mtu\partial_{\mu}f_{1}\partial_{\nu}f_{2}=-g(dx^{\mu},dx^{\nu})\partial_{\mu}f_{1}\partial_{\nu}f_{2}\mtu=-g(d_{M}f_{1},d_{M}f_{2})\mtu.
\]
For the second term, first recall that
\[
  d_{M}f_{1}\wedge d_{M}f_{2}=d_{M}f_1\otimes d_{M}f_2-d_Mf_2\otimes d_Mf_1
\]
where the $\otimes$ is, up to equivalence class, the product in Clifford. So the first term is
\[
  \frac{1}{2}\gamma( d_Mf_1\cdot d_Mf_2-d_Mf_2\cdot d_Mf_1 )=\gamma( d_Mf_1\wedge d_Mf_2 ).
\]
Finally we have
\begin{equation} \label{EqGamgGam}
\gamma(d_Mf_1\cdot d_Mf_2)=-g(d_Mf_1,d_Mf_2)\mtu+\gamma(d_Mf_1\wedge d_Mf_2).
\end{equation}
On the other hand, a general element of $\wedge^{2}(M)$ (the skew-symmetric differential $2$-forms on $M$) is
\[
  \sum_{j}^{}f_0^{j}\,d_Mf_1^{j}\wedge f_2^{j}
\]
with $f_0^{j}$, $f_1^{j}$, $f_2^{j}\in \cA$.

\begin{lemma}
\[
  \Omega_{D}^{2}\cA\simeq \gamma\big( \wedge^{2}(M) \big)
\]

\end{lemma}

\begin{proof}
We use the isomorphism
\[
  \Omega_D^2\cA\simeq\pi\big( \Omega^2\cA \big)/\pi\big( \delta(J_{0}\cap\Omega^1\cA) \big).
\]
where the elements of $J_{0}\cap\Omega^1\cA$ are of the form $\alpha_{f}=\frac{1}{2}\big( f\delta f-(\delta f)f \big)$. A general element of $\Omega_D^2\cA$ is a class of (sum of)
\[
  \pi(f_0\delta f_1\delta f_2),
\]
so from equation \eqref{EqpildotsdeltagamdM}, the idea is to define the candidate isomorphism $\psi$ by
\begin{equation}
\psi\Big( \big[ \gamma(f_0d_M f_1\cdot d_Mf_2) \big] \Big)=\gamma\big( d_Mf_1\wedge d_Mf_2 \big),
\end{equation}
and its linear extension. Let us compute $\psi[0]$ or $\psi[\pi\delta(\alpha_{f})]$. We have $\delta\alpha_{f}=\delta f\delta f$, so
\[
  \pi\delta(\alpha_{f})=\gamma(d_M f\cdot d_Mf).
\]
Thus
 \[
   \psi\big( [\delta\pi\alpha_{f}] \big)=\psi\Big( \big[ \gamma(d_Mf\cdot d_Mf) \big] \Big)
		=\gamma(d_Mf\wedge d_Mf)=0.
\]
We conclude that $\psi$ is well defined and injective. Surjectivity is clear.
\end{proof}

One can also prove the following generalization.

\begin{lemma}
\begin{equation}
\Omega^p_D\cA\simeq\wedge^{p}(M).
\end{equation}
\end{lemma}
\begin{proof}
No proof.
\end{proof}

\subsection{Example: two points space}		\label{SubSecTripleDeuxPoints}
%--------------------------------------

Let $Y=\{ 1,2 \}$, a space containing only two points. The space of continuous functions is $\cA=\eC\oplus\eC$ and a continuous function is of the form $f=(f_1,f_2)$ with $f_i=f(i)\in\eC$. We can build an even spectral triple of dimension zero $(\cA,\hH,D,\Gamma)$ as follows. Let $\hH_1$ and $\hH_2$ be two finite dimensional Hilbert space and $\hH=\hH_1\oplus\hH_2$. We define the action of $f\in\cA$ on $\psi\in\hH$ by
\[
  f\begin{pmatrix}
\psi_1\\\psi_2
\end{pmatrix}=
\begin{pmatrix}
f_1\psi_1\\f_2\psi_2
\end{pmatrix}
\]
if $\psi_i\in\hH_i$. This operator $f$ is clearly bounded on $\hH$. Let $M\colon \hH_1\to \hH_2$ be a linear map and define the operator $D$ as
\[
  D=\begin{pmatrix}
0& M^*\\
M&0
\end{pmatrix}.
\]
We want $[D,f]$ to be bounded, but
\[
  [D,f]\begin{pmatrix}
\psi_1\\\psi_2
\end{pmatrix}
=
D\begin{pmatrix}
f_1\psi_1\\f_2\psi_2
\end{pmatrix}-
\begin{pmatrix}
f_1(D\psi_1)\\
f_2(D\psi)_2
\end{pmatrix}.
\]
The component $(D\psi_1)_1$ does not affect the commutator; it is the reason why we had chosen an anti-diagonal operator $D$.

As parity map $\Gamma\colon \hH\to \hH$, we choose
\[
  \Gamma=\begin{pmatrix}
\id|_{\hH_1}\\ &-\id|_{\hH_2}
\end{pmatrix}.
\]
Now consider $f\in\cA$ and compute the commutator
\[
\begin{split}
  [D,f]\begin{pmatrix}
\psi_1\\\psi_2
\end{pmatrix}
&=
\begin{pmatrix}
f_2 M^*\psi_2\\f_1M\psi_1
\end{pmatrix}
-
\begin{pmatrix}
f_1M^*\psi_2\\f_2M\psi_1
\end{pmatrix}\\
&=(f_2-f_1)\begin{pmatrix}
M^*\psi_2\\-M\psi_1
\end{pmatrix}\\
&=(f_1-f_2)\begin{pmatrix}
0&-M^*\\M&0
\end{pmatrix}
\begin{pmatrix}
\psi_1\\\psi_2
\end{pmatrix}.
\end{split}
\]
So
\[
  \| [D,f] \|=| f_1-f_2 |\lambda
\]
where $\lambda$ is the larger eigenvalue of $\sqrt{MM^*}$. Hence the noncommutative distance between $1$ and $2$ is
\[
  d(1,2)=\sup\{ | f_1-f_2 |\tq \| [D,f] \|\leq 1 \}=\frac{1}{ \lambda }.
\]
As real structure, one can take
\[
  J\begin{pmatrix}
\psi_1\\\psi_2
\end{pmatrix}=
\begin{pmatrix}
\overline{ \psi_2 }\\\overline{ \psi_1 }
\end{pmatrix}.
\]
Let $Y=\{ 1,2 \}$, its triple $(\cA,\hH,D)$ with $\cA=\eC\oplus\eC$. We are going to study $\Omega^1\cA$. We define $\delta f$ as being the map
\begin{equation}
	(\delta f)(x,y)=f(x)-f(y).
\end{equation}
The space $\Omega^1\cA$ is a left $\cA$-module by the definition
\begin{subequations}
\begin{equation}
(f\delta g)(x,y)=f(x)\delta g(x,y).          \label{SubEqfdeltagxya}
\end{equation}
Now the Leibnitz rule imposes the following structure of right module:
\begin{equation}
	(\delta f)g(x,y)=(\delta f)(x,y)g(y).   \label{SubEqfdeltagxyb}
\end{equation}
\end{subequations}
Indeed $(\delta f)g=\delta(fg)-f\delta g$, so that
\begin{align*}
(\delta f)g(x,y)&=\delta(fg)(x,y)-f(x)(\delta g)(x,y)\\
		&=f(x)g(x)-f(y)g(y)-f(x)g(x)+f(x)g(y)\\
		&=\big( f(x)-f(y) \big)g(y)\\
		&=(\delta f)(x,y)g(y).
\end{align*}

\begin{probleme}
	I do not understand why things are like that, but if we look at an usual $1$-form on $\eR^N$, we need two vectors in order to get a number. For the $1$-form $\omega$ we need $x$ and $X$ in order to get $\omega_x(X)\in\eR$. As far as the multiplication by a function is concerned we write
\[
  (f\omega)_{x}(X)=f(x)\omega_{x}(X).
\]
Thus we have something like $(f\omega)(x,X)=f(x)\omega(x,X)$. This is more or less the philosophy of \eqref{SubEqfdeltagxya}. For \eqref{SubEqfdeltagxyb}, the fact that $g$ takes the $y$ instead of the $x$ is difficult to understand.
\end{probleme}
It gives
\begin{subequations}
\begin{align}
  (f\delta g)(x,y)&=f(x)\big( g(y)-g(x) \big)\\
 (\delta f)g(x,y)&=\big( f(y)-f(x) \big)g(y),
\end{align}
\end{subequations}
and thus
\begin{align*}
\delta(fg)(x,y)&=(\delta f)g(x,y)+f(\delta g)(x,y)\\
		&=f(y)g(y)-f(x)g(x)\\
		&=(fg)(y)-(fg)(x),
\end{align*}
which is coherent.

The $1$-forms are functions of two variables which are zero on the diagonal. In the case of our two point space, they takes non zero values only at $(1,2)$ and $(2,1)$, so a basis of $\Omega^1\cA$ is given by $\omega$ and $\eta$ with
\begin{align}
\omega(1,2)&=1 &\eta(1,2)&=0\\
\omega(2,1)&=0 &\eta(2,1)&=1.
\end{align}
Such a basis can be constructed by defining $e(0)=0$, $e(1)=1$ and considering $e\delta e$ and $(1-e)\delta(1-e)$.

\subsection{Example: manifold}
%---------------------------------

Let $M$ be a compact spin Riemannian manifold and $\cA$ the algebra of (continuous or more) functions on $M$. We also consider $D$, the Dirac operator on $\hH$, the space of the square integrable spinors over $M$. The algebra $\cA$ acts on $\hH$ by multiplication.

Let $C$ be the vector bundle over $M$ whose fibre are given by $C_x=\Cliff^{\eC}(T^*_xM)$. It is possible to define the notion of bounded measurable section of $C$. Let $\rho\colon M\to C$ one of them.

\begin{probleme}
	What is a bounded measurable section of $C$?
\end{probleme}

Since $f\in\cA$ is a function on $M$, the element $df$ is a section of $T^*M$ and can, up to the quotient \eqref{defI}, be seen as a section of $C$. When $df$ is seen in this way, it is denoted by $d_cf$ and we have
\[
  \pi(f^0df^1)=f^0[D,f^1]=i^{-1}\gamma(f^0d_cf^1).
\]

\begin{probleme}
It is true that
\[
  [D,f]=c(df)
\]
when $c$ is a Clifford action on a vector bundle. I should try to understand it better.
\end{probleme}


\section{Fredholm modules}
%-------------------------

Most of theory here and related topics is taken from \cite{ConnesNCG,Landi}.

If $X$ and $Y$ are Banach spaces, an operator $T\colon X\to Y$ is a \defe{Fredholm operator}{Fredholm!operator} if there exists a bounded linear operator $S\colon Y\to X$ such that the operators
\begin{subequations}
	\begin{align}
		\id_X-ST\\
		\id_X-TS
	\end{align}
\end{subequations}
are compact.

%---------------------------------------------------------------------------------------------------------------------------
\subsection{Introductory example}
%---------------------------------------------------------------------------------------------------------------------------

Let $M$ be a compact manifold and $\cA=C(M)$ the $C^*$-algebra of continuous functions on $M$. We consider $E^{\pm}$, two Hermitian complex vector bundles on $M$ and an elliptic pseudo-differential operator of order $0$, $P\colon  C^{\infty}(M,E^+)\to  C^{\infty}(M,E^-)$. Such an operator can be extended to an operator
\begin{equation}
	P\colon L^2(M,E^+)\to L^2(M,E^-)
\end{equation}
which has a \hyperlink{DefParametrix}{parametrix} $Q$. Consequently, $P$ is a Fredholm operator (in fact, \wikipedia{en}{Fredholm_operator}{wikipedia} says that all elliptic operators can be extended to Fredholm operator.)

The algebra $C(M)$ is naturally represented on $L^2(M,E^{\pm})$ by $\pi^{\pm}(f)(x)\xi=f(x)\xi(x)$ (pointwise multiplication of $\xi$ by $f$). Let us now consider the Hilbert space
\begin{equation}
	\hH=\hH^+\oplus\hH^-=L^2(M,E^{+})\oplus L^2(M,E^{-})
\end{equation}
and its $\eZ_2$-graduation
\begin{equation}
	\gamma=\begin{pmatrix}
		1	&	0	\\
		0	&	-1
	\end{pmatrix}.
\end{equation}
We represent $C(M)$ on $\hH$ by
\begin{equation}
	\pi(f)=\begin{pmatrix}
		\pi^+(f)	&	0	\\
		0	&	\pi^-(f)
	\end{pmatrix},
\end{equation}
and we pose
\begin{equation}
	F=\begin{pmatrix}
		0	&	Q	\\
		P	&	0
	\end{pmatrix}.
\end{equation}
In this case, the operators $[F,\pi(f)]$ and $F^2-\mtu$ are compact for every $f\in  C^{\infty}(M)$ because the operators
\begin{subequations}
	\begin{align}
		\pi^-P-P\pi^+\\
		\pi^+Q-Q\pi^-
	\end{align}
\end{subequations}
are compact\quext{I do not know why.}.

%---------------------------------------------------------------------------------------------------------------------------
\subsection{Definition}
%---------------------------------------------------------------------------------------------------------------------------

Let $\cA$ be an involutive algebra over $\eC$. A \defe{odd Fredholm module}{Fredholm!module!odd} over $\cA$ is
 \begin{enumerate}
\item an involutive representation $\pi$ of $\cA$ on an Hilbert space $\hH$,
\item an operator $F\colon \hH\to \hH$ such which satisfies
\begin{itemize}
\item $F=F^*$, $F^2=\mtu$,
\item $[F,\pi(a)]$ is a compact operator for each $a\in\cA$
\end{itemize}

\end{enumerate}

An \defe{even Fredholm modules}{Fredholm!module!even} is an odd Fredholm module with a $\eZ/2$ grading $\gamma\colon \hH\to \hH$ such that
\begin{itemize}
\item $\gamma=\gamma^*$, $\gamma^2=\mtu$,
\item $[\gamma,\pi(a)]=0$ for all $a\in\cA$,
\item $\gamma F=-F\gamma$.
\end{itemize}
We will almost always write $a\xi$ instead of $\pi(a)\xi$ when the underlying representation is clear. The Fredholm module $(\hH,F)$ is said to be \defe{$p$-summable}{$p$-summable Fredholm module} when $\forall a\in\cA$,
\[
  [F,a]\in\oL^p(\hH).
\]
One says that the Fredholm module $(\hH,F)$ is $\theta$-summable when $[F,a]\in J^{1/2}$ for all $a\in\cA$. The set $J^{1/2}$ is the two-sided ideal of compact operators $T$ such that
\[
  \mu_n(T)=O\big( (\ln n)^{-1/2} \big).
\]

\subsection{Cycle associated with Fredholm module}
%-----------------------------------------------

Let $(\hH,F)$ be a Fredholm module. We are going to build a cycle in the sense of section~\ref{SecCyclicHomol} associated with $(\hH,F)$. For the graded algebra $\Omega=\oplus_k\Omega^k$, we begin by $\Omega^0=\cA$ and for $k>0$, we define $\Omega^k$ as the vector space spanned by operators of the form
\[
  a^0[F,a^1]\cdots[F,a^k]
\]
with $a^{j}\in\cA$. Except from $a^0$, this is a product of $k$ elements of $\oL^{n+1}$ which belongs to $\oL^{(n+1)/k}$ by equation \eqref{EqPropLLLsvn}. The fact that $\oL^q$ is an ideal makes that
\[
  \Omega^k\subset\oL^{(n+1)/k}(\hH).
\]
The product in $\Omega$ is defined as the usual operator product.

\begin{lemma}
If $\omega\in\Omega^k$ and $\omega'\in\Omega^{k'}$, $\omega\omega'\in\Omega^{k+k'}$.
\end{lemma}

\begin{proof}
The fact that $[F,.]$ is a derivation on $\cA$ makes that
\[
\begin{split}
  a^0[F,a^1]\cdots[F,a^{k}]a^{k+1}&=\sum_{j=1}^{k}(-1)^{k-j}a^0[F,a^1]\cdots[F,a^{j}a^{j+1}]\cdots[F,a^{k+1}]\\
					&\quad+(-1)^ka^0a^1[F,a^2]\cdots[F,a^{k+1}].
\end{split}
\]
This is the same computation as in equation \eqref{Eq_decmProdConnForDelta}. Since each term of this sum has the form $r^0[F,r^1]\cdots[F,r^k]$, thus the product $\omega\omega'$ reads
\[
  \underbrace{a^0[F,a^1]\cdots[F,a^{k}]b^0}_{\textrm{Sum of }a^0[F,r^1]\cdots[F,r^k]}[F,b^1]\cdots[F,b^k]
\]
which belongs to $\Omega^{k+k'}$.
\end{proof}
From here we have a graded algebra $\Omega^*$ with a product $\Omega^j\times\Omega^{k'}\to\Omega^{k+k'}$. As differential, we choose
\begin{equation}  \label{EqFreddDefbel}
d\omega=F\omega-(-1)^k\omega F=[F,a^0][F,a^1]\cdots[F,a^{k}].
\end{equation}
The second equality can be checked by virtue of $F[F,a]=-[F,a]F$. This differential is a \defe{graded differential}{graded differential}, i.e.
\begin{equation}
  d(\omega_1\omega_2)=(d\omega_1)\omega_2+(-1)^{k_1}\omega_1d\omega_2
\end{equation}
for all $\omega_1\in\Omega^{k_1}$. Indeed,
\[
\begin{split}
d(\omega_1\omega_2)&=F\omega_1\omega_2-(-1)^{k_1+k_2}\omega_1\omega_2 F\\
		&=F\omega_1\omega_2-(-1)^{k_1+k_2}\omega_1b^0[F,b^1]\cdots[F,b^{k_2}]F\\
		&=F\omega_1\omega_2-(-1)^{k_1}\omega_1\big( -[F,b^0]+Fb^0 \big)[F,b^1]\cdots[F,b^{k_2}]\\
		&=F\omega_1\omega_2+(-1)^{k_1}\omega_1d\omega_2-(-1)^{k_1}\omega_1 F\omega_2\\
		&=(d\omega_1)\omega_2+(-1)^{k_1}\omega_1 d\omega_2.
\end{split}
\]
We also check that $d^2=0$ in the following way:
\begin{equation}
	\begin{aligned}[]
		d^2\omega&=d(F\omega-(-1)^k\omega F)\\
		&=F\big( F\omega-(-1)^k\omega F \big)+(-1)^{k}\big( F\omega-(-1)^k\omega F \big)\\
		&=\omega-(-1)^kF\omega F+(-1)^k F\omega F-\omega\\
		&=0
	\end{aligned}
\end{equation}
where we used the fact that $F^2=\mtu$.

The pair $(\Omega^*,d)$ is a graded differential algebra. We have to find a graded closed trace $\tr_s\colon \Omega^n\to \eC$. Let $T$ be an operator on $\hH$ such that $FT+TF\in\oL^1(\hH)$. We begin to define
\[
  \tr'(T)=\frac{ 1 }{2}\tr\big( F(FT+TF) \big).
\]
When $T\in\oL^1$, it makes sense to distribute the $F$ in the trace, in such a way that we obtain $\tr'(T)=\tr(T)$. In this case, we have
  $\tr'(T)=\frac{ 1 }{2}\tr\big( F(FT+TF) \big)=\frac{ 1 }{2}\tr(T+FTF)=\tr(T)$.
Now de define the trace $\tr_s\colon \Omega^n\to \eC$,
\[
  \tr_s\omega=
\begin{cases}
    \tr'(\omega)&\text{if }n \text{ is odd}\\
    \tr'(\gamma\omega)&\text{if }n \text{is even}.
\end{cases}
\]
It makes sense because when $\omega\in\Omega^n$, it fulfills $F\omega+\omega F=d\omega\in\Omega^{n+1}\subset\oL^{(n+1)/(n+1)}=\oL^1$, so that we can use the usual trace. By the way, remark that the trace $\tr_s$ reads $\frac{ 1 }{2}\tr(Fd\omega)$ when $n$ is odd and $\frac{ 1 }{2}\tr(F\gamma d\omega)$ when $n$ is even.

\begin{proposition}
	The triple $(\Omega,d,\tr_s)$ is a $n$-dimensional cycle over  $\cA$ (see definition~\ref{DefCycleCoh}).
\end{proposition}

\begin{proof}
Most of the work is already done; it just remains to prove that $\tr_s$ is a graded closed trace. First, we know that $d^2=0$, so the fact that $\tr_s$ only depends to $d\omega$ gives $\tr_s(d\omega)=0$. The form is thus closed.

Now we have to prove that it is a graded trace. If $\omega\in\Omega^k$ and $\omega'\in\Omega^{k'}$ with $k+k'=n$ (let us assume $n$ odd), we have
\[
  \tr_s(\omega\omega')=\frac{ 1 }{2}\tr\big( F(d\omega)\omega'+(-1)^{k}F\omega d\omega' \big)=\frac{ 1 }{2}\tr\big( (-1)^{k+1}(d\omega)F\omega'+(-1)^k(F\omega)d\omega' \big),
\]
but the usual trace has the property that, when $T_j\in\oL^j$ with $\frac{1}{ p_1 }+\frac{1}{ p_2 }=1$, $\tr(T_1T_2)=\tr(T_2T_1)$. So we have $\tr(F\omega d\omega')=\tr(d\omega'F\omega)$ because $F\omega\in\oL^{(n+1)/  k }$ and $d\omega'\in\Omega^{k'+1}\subset\oL^{(n+1)/(k'+1)}$. Thus we have
\[
  \tr_s(\omega\omega')=\frac{ 1 }{2}\big( (-1)^{k+1}d\omega F\omega'+(-1)^kd\omega' F\omega \big).
\]
This expression is symmetric or anti-symmetric with repsect to the inversion $\omega\leftrightarrow\omega'$ following that $(-1)^{kk'}$ equals $1$ or $-1$. We conclude that (at least when $n$ is odd) t
\[
  \tr_s(\omega\omega')=(-1)^{kk'}\tr_s(\omega'\omega).
\]
\end{proof}

The \defe{character}{character!of a cycle} is the cyclic cocycle
\begin{equation}
\tau_n(a^0,\cdots,a^{n})=
\begin{cases}
    \tr'\big( a^0[F,a^1]\cdots[F,a^{n}] \big)&\text{if }n \text{is odd}\\
    \tr'\big( \gamma a^0[F,a^1]\cdots[F,a^{n}] \big)&\text{if }n \text{is aven}.
\end{cases}
\end{equation}

%+++++++++++++++++++++++++++++++++++++++++++++++++++++++++++++++++++++++++++++++++++++++++++++++++++++++++++++++++++++++++++
\section{Hochschild cohomology}
%+++++++++++++++++++++++++++++++++++++++++++++++++++++++++++++++++++++++++++++++++++++++++++++++++++++++++++++++++++++++++++

\begin{proposition}
Let $(\hH,F)$ be a Fredholm module $(n+1)$-summable over $\cA$. We suppose that this module has the same parity as $n$. Then the characters $\tau_{n+2q}$ satisfy
\begin{equation}
\tau_{m+2}=-\frac{ 2 }{ m+2 }S\tau_m\in HC^{m+2}(\cA)
\end{equation}
when $m=n+2q$, with $q\geq 0$.
\end{proposition}

\begin{proof}
In order to see that $\tau\in C^n_{\lambda}(\cA)$, we have to check that the equality $\tau_n(a^1,\cdots,a^{n},a^0)=(-1)^n\tau_n(a^0,\cdots,a^{n})$ holds for all $a^i\in\cA$. We have
\[
\tau_n(a^0,\cdots,a^{n})=\tr'(a^0da^1\cdots da^{n})=\tr'\Big( \big( d(a^0a^1)-da^0a^1 \big)da^2\cdots da^{n} \Big)
\]
in which the first term vanishes because $d^2=0$. We go on commuting $da^0$ and we finally get $(-1)^n\tau_n(a^1,\cdots,a^n,a^0)$.

\begin{probleme}
	This proof is not finished.
\end{probleme}

\end{proof}
%+++++++++++++++++++++++++++++++++++++++++++++++++++++++++++++++++++++++++++++++++++++++++++++++++++++++++++++++++++++++++++
\section{Fredholm module and conformal structure}
%+++++++++++++++++++++++++++++++++++++++++++++++++++++++++++++++++++++++++++++++++++++++++++++++++++++++++++++++++++++++++++

Let $V$ be a compact, oriented even dimensional manifold endowed with a \defe{conformal structure}{conformal structure}. That is an equivalence class of metrics where $g\sim h$ if and only if there exists a positive smooth function $\lambda$ such that $g=\lambda^2 h$.

Let $\hH_0=L^2\big( V,\Wedge^n_{\eC}(T^*V) \big)$ with the product \eqref{EqProdWedgeHOfge}. This becomes a $ C^{\infty}(V)$-module when we define
\begin{equation}
	(f\omega)(p)=f(p)\omega(p)
\end{equation}
for every $\omega\in\hH_0$, $f\in C^{\infty}(V)$ and $p\in V$. We can extend the graduation \eqref{EqGradWedge} to the Hilbert space $\hH_0$ by
\begin{equation}
	(\gamma\omega)(p)=\gamma\big( \omega(p) \big).
\end{equation}


Consider $\hH_0=L^2\big( V,\Wedge^n_{\eC}(T^*V) \big)$ be the space of square integrable sections of the bundle $\Wedge^n_{\eC}(T^*V)$ for the product

We can consider the complex space $\wedge^n_{\eC}E$ and the operator $\gamma\colon \Wedge^nE\to \Wedge^nE$,
\begin{equation}
	\gamma=(-1)^{n(n-1)/2}i^n *.
\end{equation}
This operator squares to $\mtu$, so that it creates a $\eZ/2$-graduation of $\Wedge^n_{\eC}E$.

We have
\begin{equation}
	\frac{ 1+\gamma }{2}d(*\alpha^{n+1})=\frac{ 1 }{2}\Big( d(*\alpha^{n+1})+(-1)^s\delta\alpha^{n+1} \Big)
\end{equation}
where $s$ is a sign. We used the fact that $\delta=-*d*$. Thus the elements of $\hH_0$ of the form $\frac{ 1+\gamma }{2}d\alpha$ are orthogonal to the harmonic forms.


%+++++++++++++++++++++++++++++++++++++++++++++++++++++++++++++++++++++++++++++++++++++++++++++++++++++++++++++++++++++++++++
\section{Fredholm modules and $K$-cycles}
%+++++++++++++++++++++++++++++++++++++++++++++++++++++++++++++++++++++++++++++++++++++++++++++++++++++++++++++++++++++++++++

\begin{definition}
	A \defe{K-cycle}{K-cycle} $(\hH,D)$ over an involutive algebra $(\cA,*)$ is
	\begin{enumerate}
		\item a $*$-representation of $\cA$ on $\hH$,
		\item a selfadjoint non bounded operator $D$ with compact resolvent and such that $[D,a]$ is bounded for each $a\in\cA$.
	\end{enumerate}
\end{definition}

\begin{remark}
	The condition ``compact resolvent'' means that the operators $(D-\lambda\mtu)^{-1}$ are compact for every $\lambda$ in $\eC\setminus\Spec(D)$ (lemma~\ref{LemResLcmpResLLcmp}). In particular, the kernel of $D$ is finite dimensional (corollary~\ref{CorRezcomkerfin}).
\end{remark}

From a K-cycle on $\cA$ we canonically build a Fredholm module $(\hH',F)$, the Fredholm module \defe{associated}{Fredholm!module!associated with a K-cycle} with the K-cycle $(\hH,D)$, in the following way.
\begin{enumerate}
	\item
		$\hH'=\hH\oplus\ker(D)=\ker(D)^{\perp}\oplus\ker(D)\oplus\ker(D)$;
	\item
		$a(\xi,\eta)=(a\xi,0)$ for every $\xi\in\hH$ and $\eta\in\ker(D)$;
	\item
		$F=\Sign(D)\oplus F_1$.
\end{enumerate}
The definition of $F$ deserves some comments. First, $\Sign(D)$ is the sign of $D$, that is the partial isometry in the polar decomposition $D=V| D |$ of $D$. That acts on $\ker(D)^{\perp}$. The operator $F_1$ is the operator which exchanges the two copies of $\ker(D)$. More explicitly, if $\xi\in\hH$ and $\eta\in\ker(D)$,
\begin{equation}
	F(\xi,\eta)=F(\xi_1+\xi_0,\eta)=(V\xi_1+\eta,\xi_0)
\end{equation}
where $\xi=\xi_1+\xi_0$ is the decomposition of $\xi$ with respect to $\hH=\ker(D)\oplus\ker(D)^{\perp}$. In order to prove that $F^2=\mtu$, we have to show that $V^2=\mtu$ and that $V\xi_1\in\ker(D)^{\perp}$.

Since the part $\Sign(D)$ only acts on $\ker(D)^{\perp}$, we can see this operator as in equation \eqref{EqPolarSSKerSign}, that is the sign of the operator $D$ restricted to the space $\ker(D)^{\perp}$. This is an operator on $\ker(D)^{\perp}$, so that $V\xi_1\in\ker(D)^{\perp}$ and $V^2=\id_{\ker(D)^{\perp}}$.

%+++++++++++++++++++++++++++++++++++++++++++++++++++++++++++++++++++++++++++++++++++++++++++++++++++++++++++++++++++++++++++
\section{Spectral triple}
%+++++++++++++++++++++++++++++++++++++++++++++++++++++++++++++++++++++++++++++++++++++++++++++++++++++++++++++++++++++++++++

\subsection{General spectral triple}
%-----------------------------------
A \defe{spectral triple}{spectral!triple} is a triple $(\cA,\hH,D)$ where
\begin{itemize}
\item $\hH$ is a Hilbert space,
\item $\cA$ is an involutive algebra of bounded operators on $\hH$ ,
\item $D$ is a self-adjoint ($D=D^*$) operator on $\hH$ such that
\begin{enumerate}
\item the resolvent $(D-\lambda)^{-1}$, $\lambda\notin\eR$ is a compact operator on $\hH$,
\item\label{item_DaDcirii} $[D,a]:=D\circ a-a\circ D$ is a bounded operator for all $a$.
\end{enumerate}
\end{itemize}
In general condition~\ref{item_DaDcirii} can only be imposed on a dense subalgebra of $\cA$.

\begin{definition}	\label{DefDimSpec}
The \defe{dimension spectrum}{dimension!of a spectral triple!spectrum} of the spectral triple $(\cA,\hH,D)$ is the set $\Pi$ of complex numbers $z$ such that $\real(z)\geq 0$ and $z$ is a singularity of the analytic function $\zeta_b(z)$ for $b\in\mB$ with positive real part. Here $\mB$ is the operator algebra generated by $\delta^k(a)$ and $\delta^k[D,a]$ with $a\in \cA$ and $\delta T=[| D |,T]$. When $b\in\mB$, the function $\zeta$ is given by
\[
  \zeta_b(z)=\tr(b| D |^{-z})
\]
which is well defined when $\real(z)>m$ where $m$ is the crude dimension of the triple.
\end{definition}

We say that the dimension of the triple is \defe{simple}{simple!dimension of a spectral triple}\index{dimension!of a spectral triple!simple} id the poles of the functions $\zeta_b$ are at most simple.

We say that the triple is \defe{even}{even spectral triple} if there exists an operator $\Gamma$ on $\hH$ such that
\begin{enumerate}
\item $\Gamma=\Gamma^*$,
\item $\Gamma^2=1$,
\item $[\Gamma,D]=0$ and $[\Gamma,a]=0$ for all $a\in\cA$.
\end{enumerate}
If the triple is not even, it is \defe{odd}{odd spectral triple}. The triple $(\cA,\hH,D)$ is of \defe{dimension}{dimension!of a spectral triple} $n>0$ if $| D |^{-1}$ is an infinitesimal of order $1/n$, in other words, if $| D |-n$ is infinitesimal of order $1$. A $n$-dimensional spectral triple is sometimes said to be $n$-summable. A \defe{real structure}{real!structure!on a spectral triple} on the spectral triple $(\cA,\hH,D)$ is an antilinear isometry $J\colon \hH\to \hH$ such that
\begin{align*}
J^2&=\epsilon(n)\mtu	&[a,b^0]&=0\\
JD&=\epsilon'(n)DJ	&\big[ [D,a],b^0 \big]&=0\\
J\Gamma&=(i)^n\Gamma J
\end{align*}
where $b^0=Jb^*J^*$ and $\Gamma$ is the $\eZ_2$ graduation if the triple is even; if the triple is odd, then the corresponding condition is removed. The functions $\epsilon$ and $\epsilon'$ are periodic with period $8$ and
\[
\begin{split}
\epsilon(n)	&=(1,1,-1,-1,-1,-1,1,1)\\
\epsilon'(n)	&=(1,-1,1,1,1,-1,1,1).
\end{split}
\]

Notice that as direct consequence of the properties, we also have $\big[ [D,b^0] \big]=0$. When we consider a real spectral triple, we can endow $\hH$ with a bimodule structure over $\cA$ by
\[
  a\xi b=\pi(a)J\pi(b^*)J^*\xi.
\]
The left module structure is the usual one while the right is well defined because $J^*J=\cun$, so that
\begin{align*}
\xi(an)=Jb^*a^*J^*\xi=Jb^*J^*Ja^*J^*\xi=(\xi a)b.
\end{align*}

Two spectral triples $(\cA_i,\hH_i,\pi_i,D_i)$ are \defe{equivalent}{equivalence!of spectral triple} when there exists an unitary operator $U\colon \hH_1\to \hH_2$ such that $U\pi_1(a)U^*=\pi_2(a)$ for every $a\in\cA$ and $UD_1U^*=D_2$. If the triple is even or real, we ask moreover $U\Gamma_1U^*=\Gamma_2$ and $UJ+1U^*=J_2$.

\subsection{Commutative real triple}
%------------------------------------

When $\cA$ is commutative, the right action of $a$ is equivalent to the left action of $Ja^*J^*$ in the sense that
\[
  \xi(ab)=(Ja^*J^*)(Jb^*J^*)\xi
\]

\subsection{Analysis on a spectral triple}
%-----------------------------------------

The following is a direct computation.
\begin{lemma}
The operator $[D,\cdot\,]$ is a derivation of $\cA$.
\end{lemma}

\begin{lemma}
\begin{equation}
  [D,a]^*=-[D,a]
\end{equation}

\end{lemma}
\begin{proof}
It is nothing else than the fact that $D=D^*$:
\[
  [D,a]^*=(D\circ a)^*-(a\circ D)^*=a^*D-Da^*=[a,D]=-[D,a].
\]

\end{proof}

From definition of the spectral triple, the operator $(D-z\mtu)$ exists for all non real $z$, so the spectrum (definition~\ref{def:spectre}) of $D$ is real:
\[
  \sigma(D)\subset \eR.
\]

\begin{probleme}
	The following statements need more theory about operators with compact resolvent.
\end{probleme}

The spectrum of $D$ is discrete and the elements $\{ \lambda_n \}$ are eigenvalues of finite multiplicity. Moreover characteristic values $\mu_n\big( (D-1)^{-1} \big)\to 0$ when $n\to\infty$ and so $| \lambda_n |=\mu_n(| D |)\to0$.

When $[D,a]$ is bounded we say that $a\in\cA$ is \defe{Lipschitz}{Lipschitz}. Let $\delta$ be the derivation on $\opB(\hH)$ defined (on a dense subspace) by
\[
  \delta(T)=[| D |,T].
\]
This generates a one parameter group of automorphism of $\opB(\hH)$ defined by
\begin{equation}
\alpha_s(T)= e^{is| D |}T e^{-is| D |}.
\end{equation}
We say that $a\in\cA$ is \defe{smooth}{smooth!in spectral triple} and we write $a\in C^{\infty}$ if the map
\[
  s\to\alpha_s(a)
\]
is smooth. The element $a$ is of class $C^k$ when $s\to\alpha_s(a)$ is $C^k$.

\begin{proposition}
An element $a\in\cA$ is smooth if and only if $a\in\bigcap_{n\in\eN}\dom(\delta^n)$.
\end{proposition}

\begin{proof}
If $a$ is smooth, the existence of the derivative of $s\to  e^{is| D |}a e^{-is| D |}$ makes that $a\in\dom\delta$ because the derivative of this map is precisely $\delta$. A few computation shows that the second derivative of this map is $\delta^2$.

If, on the other hand, $a\in\bigcap_{n\in\eN}\dom(\delta^n)$, we have existence of all the derivatives of $s\to\alpha_s(a)$ and continuity is given by derivability
\begin{probleme}
	Is that justification correct?
\end{probleme}

\end{proof}

\subsection{Spectral triple over a manifold}
%-------------------------------------------

Let $(M,g)$ be a Riemannian spin manifold of dimension $n$. The \defe{canonical triple}{canonic!spectral triple} on $M$ is
\begin{enumerate}
\item $\cA= C^{\infty}(M)$,
\item $\hH=L^2(M,S)$, the bundle of square integrable spinors on $M$.
\item $D$ is the Dirac operator associated with the Levi-Civita connection of $g$.
\end{enumerate}
The rank of the spinor bundle is $2^{[n/2]}$ and the scalar product on $\hH$ is given by
\[
  (\psi,\phi)=\int \overline{ \psi(x) }\phi(x)\,d\mu(g)
\]
where the bar denotes the complex conjugation, $d\mu(g)$ is the measure associated with $g$ and $\overline{ \phi(x) }\phi(x)$ is the usual product in $\eC^{2[n/2]}$. The space $ C^{\infty}(M)$ is an algebra of operators on $\hH$ by simple multiplication:
\begin{equation}
(f\psi)(x):=f(x)p\psi(x)
\end{equation}
for all $f\in C^{\infty}(M)=\cA$ and $\psi\in\hH$.

Most of the interest in spectral triples over manifold comes from the following theorem.

\begin{theorem}
Let $(\cA,\hH,D)$ be the canonical triple on a manifold $M$. Then
\begin{enumerate}
\item $M$ is the structure space of the algebra $\overline{ \cA }=C(M)$, the norm closure of $\cA= C^{\infty}(M)$,
\item the geodesic distance between $p$, $q\in M$ is given by
\begin{equation} \label{eq_defdtriple}
  d(p,g)=\sup\{ | f(p)-f(q) |\tq f\in\cA\text{ and }\| [D,f] \|\leq 1 \},
\end{equation}
\item the Riemannian measure on $M$ is given by
\begin{equation}
\int_M f=c(n)\tr_{\omega}(f| D |^{-n})
\end{equation}
where $c(n)=2^{n-[n/2]-1}\pi^{n/2}n\Gamma(\frac{ n }{2})$.
\end{enumerate}

\end{theorem}

See \cite{Landi} for more complete proof and reference for even more complete proof.

\begin{proof}
The algebra $\overline{ \cA }$ is an unital commutative $C^*$-algebra. So Gelfand theorem~\ref{thoGelfand} says that $\overline{ \cA }\simeq C\big( \Delta(\overline{ \cA }) \big)$, but by definition $\overline{\cA}=C(M)$, hence $M=\Delta(\overline{ \cA })$. This proves the first point.

For the second point, we use the form \eqref{eq_Dirac_deux} of Dirac operator, so
 \begin{equation}
\begin{split}
[D,f]\psi(x)&=D(f\psi)(x)-f(x)D\psi(x)\\
	&=\gamma^{\mu}(x)\big( (\partial_{\mu}f)\psi+f\partial_{\mu}\psi+f\omega_{\mu}^S\psi \big)\\
	&\quad -f(x)\gamma^{\mu}(x)\big( \partial_{\mu}\psi+\omega_{\mu}^S\psi \big)\\
	&=(\gamma^{\mu}\partial_{\mu}f)\psi.
\end{split}
\end{equation}
Hence, as multiplicative operator on $\hH$, we have $[D,f]=\gamma^{\mu}\partial_{\mu}f=\gamma(df)$ for all $f\in\cA$. The norm of this operator is
\[
  \| [D,f] \|=\sup | (\gamma^{\mu}\partial_{\mu}f)(\gamma^{\nu}\partial_{\nu}f)^* |^{\frac{ 1 }{2}}.
\]
One can prove (cf \cite{Landi} for a reference) that this is the Lipschitz norm of $f$:
\[
  \| f \|_{Lip}=\sup_{x\neq y}\frac{ | f(x)-f(y) | }{ d_{\gamma}(x,y) }
\]
where $d_{\gamma}$ is the usual geodesic distance that we want to prove to be equals to $d$. The condition $\| [D,f]\leq 1 \|$ in the definition  \eqref{eq_defdtriple} retrains us to only looks at $f$ such that
\[
  \sup_{x\neq y}\frac{ | f(x)-f(y) | }{ d_{\gamma}(x,y) }\leq 1.
\]
If we fix $x=p$ and $y=q$, this condition becomes $| f(p)-f(q) |\leq d_{\gamma}(p,q)$; hence $d(p,q)\leq d_{\gamma}(p,q)$.

We have to work out the inverse inequality. For, we fix a point $q$ and consider $f_{\gamma,q}=d_{\gamma}(x,q)$. This function fulfils $\| [D,f_{\gamma,q}] \| \leq 1$ and using this function as lower bound for the supremum which defines $d(p,q_0)$, we find
\[
  d(p,q)\geq | f_{\gamma,q}(p)-f_{\gamma,q}(q) |=d_{\gamma}(p,q)
\]
because $f_{\gamma,q}=d_{\gamma}(p,q)$ and $f_{\gamma,q}(q)=0$.

\end{proof}

\begin{probleme}
	The use of Gelfand's theorem at the beginning of the proof requires $M$ to be compact?
\end{probleme}

\subsection{Distance over general triple}
%----------------------------------------

We will now show that formula \eqref{eq_defdtriple} generalises to a distance formula between states, see definition~\ref{DefStateUnital}. The \defe{distance}{distance on spectral triple} is the distance on $\etS(\overline{ \cA })$ given by
\begin{equation}
d(\omega,\chi)=\sup_{a\in\cA}\{ | \omega(a)-\chi(a) |\tq \| [D,a] \|\leq 1 \}.
\end{equation}

When $(\cA,\hH,D)$ is a triple of dimension $n$, we define the \defe{integral}{integral!on a spectral triple} of $a\in \cA$ by
\begin{equation}
\int a:=\frac{1}{ V }\tr_{\omega}(a| D |^{-1})
\end{equation}
where $V$ is a constant defined by $\mu_j\leq V j^{-1}$ when $j\to\infty$. Here, $(\mu_j)$ is the sequence of characteristic values of $| D |^{-1}$. Why does $| D |^{-n}$ appears? The operator $a$ is just bounded on $\hH$, hence the trace $\tr_{\omega}a$ makes no sense. The multiplication by $| D |^{-n}$ gives rise to an infinitesimal of order $1$ and Dixmier trace makes sense. On the other hand, the integral is normalised in the following sense:
\[
  \int\mtu=\frac{1}{ V }\tr_{\omega}| D |^{-n}
		=\frac{1}{ V }\lim_{N\to\infty}\sum_{j=1}^{N-1}\mu_j\big( | D |^{-n} \big)
		\leq\lim_{N\to\infty}\sum_{j=1}^{N-1}\frac{1}{ j }
		=1
\]
because $\frac{1}{ V }\mu_j(| D |^{-n})\leq j^{-1}$.

\begin{probleme}
	This only proves that $\int\mtu\leq1$, but the second line is probably
\[
  \frac{1}{ V }\lim_{\omega}\sum_{j=1}^{N-1}\mu_j(| D |^{-n}),
\]
and we should define $\lim_{\omega}$ in such a way that
\[
  \frac{1}{ V }\lim_{N\to\infty}\sum_{j=1}^{N-1}Vj^{-1}.
\]
After that, we still have to define $\lim_{\omega(V)}$ and prove that $\int a$ does not depend on its choice, on the choice of $V$ and of $\lim_{\omega(V)}$.
\end{probleme}

\subsection{Real triple}
%-----------------------

Let $(\cA,\hH,D)$ be a spectral triple of dimension $n$. A \defe{real structure}{real!structure!on a spectral triple} is an anti-linear isometry $J\colon \hH\to \hH$ such that
\begin{enumerate}
\item $J^2=\epsilon(n)\id$,
\item $JD=\epsilon'(n)DJ$,
\item $G\Gamma=i^n\Gamma J$ if $n$ is even with the grading $\Gamma$,
\item $[a,b^0]=0$,
\item $[ [D,a],b^0 ]=0$ where $b^0=Jb^*J^*$.
\end{enumerate}
The functions $\epsilon$ and $\epsilon'$ are defined modulo $8$:
\[
\begin{split}
\epsilon(n)&=1,1,-1,-1,-1,-1,1,1\\
\epsilon'(n)&=1,-1,1,1,1,-1,1,1.
\end{split}
\]

\subsection{Example: compact manifold}
%-------------------------------------

Let $M$ be a compact spin Riemannian manifold. We consider the triple $(\cA,\hH,D)$ where
\begin{itemize}
\item $\hH$ is the Hilbert space of $L^2$ spinors on $M$,
\item $D$ is the Dirac operator on $\hH$,
\item $\cA$ is the abelian algebra of bounded measurable functions on $M$ with the multiplicative action on $\hH$.
\end{itemize}
Via the representation, space $C(M)$ of continuous functions on $M$ is seen as a subspace of the space of linear operators on $\hH$. Hence, up to a closure, we have $M=\Delta(\cA)$ and Gel'fand theorem~\ref{thoGelfand} says that the compact topological space structure of $M$ is given by $\cA$. A point of $M$ is associated with an element of $\Delta(\cA)$, i.e. a homomorphism $\cA\to\eC$.

\begin{proposition}
Let $a\in\cA$. The operator $[D,a]$ is
\begin{enumerate}
\item defined on a dense subspace of $\hH$,
\item bounded if and only if $a$ is almost everywhere equals to a Lipschitz function.
\end{enumerate}

\end{proposition}
\begin{proof}
No proof.
\end{proof}

A function $f$ on the manifold $M$ is \defe{Lipschitz}{lipschitz!function} if for all $p$, $q\in M$,
\[
  | f(p)-f(q) |\leq C d(p,q)
\]
where $C$ is a constant and $d$ denotes the geodesic distance on $M$. Any Lipschitz function is continuous and the space of Lipschitz functions is dense in $C(M)$. Then $C(M)$ is the closure of $\cA$ in $\oL(\hH)$.

\input{132_noncommutative_geometry}


\chapter{Deformations: formal aspects}          \label{ChapDefo}
% This is part of (almost) Everything I know in mathematics
% Copyright (c) 2013-2014, 2020
%   Laurent Claessens
% See the file fdl-1.3.txt for copying conditions.

%+++++++++++++++++++++++++++++++++++++++++++++++++++++++++++++++++++++++++++++++++++++++++++++++++++++++++++++++++++++++++++
\section{Twists of module (co)algebras}
%+++++++++++++++++++++++++++++++++++++++++++++++++++++++++++++++++++++++++++++++++++++++++++++++++++++++++++++++++++++++++++
\label{SecTheoryTwist}

We are going to follow \cite{GiaquintoZhangTwist} and use the notions of subsection~\ref{subSecModulebialgebra}.
\begin{definition}	\label{DefTwist}
	An element $F\in B\otimes B$ is a \defe{twisting element}{twist} based on $B$ if it satisfies the two following conditions
	\begin{enumerate}

		\item\label{ItemTwistUn}
			$(\epsilon_B\otimes \id)F=(\id\otimes \epsilon_B)F=1\otimes 1$
		\item\label{ItemTwistDeux}
			$\big[ (\Delta_B\otimes\id)(F) \big](F\otimes 1)=\big[ (\id\otimes\Delta_B)(F) \big](1\otimes F)$.

	\end{enumerate}
\end{definition}
If $F$ is invertible, we have
\begin{equation}		\label{EqDelidFinve}
	\big[ (\Delta_B\otimes\id)F\big]^{-1}=(\Delta_B\otimes\id)F^{-1}.
\end{equation}
In order to see that, first notice that the third component in the tensor product is $F_2^{-1}$ in both sides of \eqref{EqDelidFinve} while the two first components are given by $\Delta_B(F)^{-1}$ in the left hand side and by $\Delta_B(F^{-1})$ in the right hand side. Using the coalgebra properties, we have
\begin{equation}
	\Delta_B(F)\Delta_B(F^{-1})=\Delta_B(FF^{-1})=\Delta_B(1)=1\otimes 1.
\end{equation}
Thus if we take the inverse of the property~\ref{ItemTwistDeux} in the definition of a twist, we get
\begin{equation}		\label{EqFemuVaLeLem}
	(F^{-1}\otimes 1)\big[ (\Delta_B\otimes\id)F^{-1} \big]=(1\otimes F^{-1})\big[ (\id\otimes\Delta_B)F^{-1} \big]^{-1}
\end{equation}

%---------------------------------------------------------------------------------------------------------------------------
\subsection{Twisting the module algebras}
%---------------------------------------------------------------------------------------------------------------------------

Let $\eA$ be a $B$-module algebra with its multiplication $\mu_{\eA}$. If $F$ is a twist based on $B$, we can define the new multiplication
\begin{equation}
	\begin{aligned}
		\mu_F=\mu_{\eA}\circ F_l\colon \eA\otimes\eA&\to \eA \\
		a\otimes a'&\mapsto \mu_{\eA}(ba\otimes b'a')
	\end{aligned}
a\end{equation}
if $F=b\otimes b'$. In the same way, if $C$ is a $B$-module coalgebra with comultiplication $\Delta_C\colon C\to C\otimes C$, we can deform it by
\begin{equation}
	\Delta_F=F_r\circ\Delta_C\colon C\to C\otimes C.
\end{equation}

\begin{theorem}		\label{ThoTwistAlgEtCoalg}
	Let $B$ be a bialgebra.
	\begin{enumerate}

		\item
			If $\eA$ is a left $B$-module algebra, then $\eA_F=(\mu_{\eA}\circ F_l,1_{\eA})$ is an associative algebra on $\eK$.
		\item
			If $C$ is a $B$-module coalgebra, then $C_F=(F_r\circ\Delta_C,\epsilon_C)$ is a coassociative $\eK$-coalgebra.

	\end{enumerate}
\end{theorem}

\begin{proof}
	Coassociativity of $\eA_F$ means that for every $a,b,c$ in $\eA$,
	\begin{equation}
		(\mu_{\eA}\circ F_l)\big( a\otimes(\mu_{\eA}\circ F_l)(b\otimes c) \big)=(\mu_{\eA}\circ F_l)\big( (\mu_{\eA}\circ F_l)(a\otimes b)\otimes c \big).
	\end{equation}
	In other terms,
	\begin{equation}
		(\mu_{\eA}\circ F_l)\circ\big( \id\otimes(\mu_{\eA}\circ F_l) \big)=(\mu_{\eA}\circ F_l)\circ(\mu_{\eA}\circ F_l)\otimes\id
	\end{equation}
	as map from $\eA\otimes\eA\otimes\eA$ to $\eA$. If we use the decomposition
	\begin{equation}
		\id\otimes(\mu_{\eA}\otimes F_l)=(\id\otimes\mu_{\eA})\circ(\id\otimes F_l),
	\end{equation}
	we see that we have to prove the commutativity of the diagram
	\begin{equation}
		\xymatrix{%
		\eA\otimes\eA\otimes\eA \ar[r]^{F_l\otimes\id}\ar[d]_{\id\otimes F_l}	&	\eA\otimes\eA\otimes\eA \ar[r]^{\mu_{\eA}\circ\id}	&	\eA\otimes\eA\ar[d]^{F_l}\\
		\eA\otimes\eA\otimes\eA	\ar[d]_{\id\otimes F_l}			&								& \eA\otimes\eA\ar[d]^{\mu_{\eA}}\\
		\eA\otimes\eA\ar[r]_{F_l}	& \eA\otimes\eA\ar[r]_{\mu_{\eA}}							&				 \eA
		   }
	\end{equation}
	We refer to \cite{GiaquintoZhangTwist} for the remaining of the proof.
\end{proof}

\begin{lemma}		\label{LemRemakAecP}
	If $P\in B\otimes B$ satisfies
	\begin{equation}
		(P\otimes 1)\big[ (\Delta_B\otimes\id)P \big]=(1\otimes P)\big[ (\id\otimes\Delta_B)(P) \big],
	\end{equation}
	then $P$ twists the left $B$-modules coalgebras.
\end{lemma}
\begin{proof}
		No proof.
\end{proof}
Notice that, by equation \eqref{EqFemuVaLeLem}, $F^{-1}$ satisfies lemma~\ref{LemRemakAecP} when it is invertible.


%---------------------------------------------------------------------------------------------------------------------------
\subsection{Twisting the bialgebra itself}
%---------------------------------------------------------------------------------------------------------------------------

Let $F$ be an invertible twist base on the bialgebra $B$. We can deform $B$ by
\begin{equation}
	 \Delta_B'=F^{-1}_l\circ F_r\circ\Delta_B.
\end{equation}

\begin{theorem}
	The structure
	\begin{equation}
		B_F=(\mu_B,\Delta_B',1_B,\epsilon_B)
	\end{equation}
	is a $\eK$-bialgebra.
\end{theorem}

\begin{proof}
	The second point of theorem~\ref{ThoTwistAlgEtCoalg} makes $F_r\circ\Delta_B$ coassociative since $F$ is a twist. In order to check that this is still a $B$-module coalgebra, we rewrite the diagram \eqref{EqDiagModCoAlg} with $B$ instead of $C$ and $\Delta_B(b)$ by $\Delta_B(b)F$:
\begin{equation}
	\xymatrix{%
	B\otimes B 		&	B\ar[l]_{F_r\circ\Delta}\\
	B\otimes B \ar[u]^{F_r\circ\Delta(b)_r}	&	B\ar[l]_{F_r\circ\Delta}\ar[u]_{b_r}
	   }
\end{equation}
	We see that the result of this diagram is the one of the non twisted one multiplied by $F$. For example, the vertical left arrow is the mapping
	\begin{equation}
		c\otimes c'\mapsto(c\otimes c')\cdot\Delta(b)F.
	\end{equation}
	Thus the coalgebra $B$ endowed with $\Delta=F_r\circ\Delta_B$ is still a $B$-module coalgebra. Now, using the lemma~\ref{LemRemakAecP} and the fact that $F^{-1}$ satisfies that lemma, the structure $\Delta_{B}'=F_l^{-1}\circ F_r\circ\Delta_B$ is coassociative. We check that this structure is compatible with the multiplication:
	\begin{equation}
		\begin{aligned}[]
			\Delta'_B(bb')&=F^{-1}\Delta_B(bb')F\\
					&=F^{-1}\Delta_B(b)\Delta_B(b')F\\
					&=F^{-1}\Delta_B(b)FF^{-1}\Delta_B(b')F\\
					&=\Delta_B'(b)\Delta_B'(b').
		\end{aligned}
	\end{equation}
\end{proof}

%---------------------------------------------------------------------------------------------------------------------------
\subsection{Twisting the left module algebra}
%---------------------------------------------------------------------------------------------------------------------------

\begin{theorem}
	Let $\eA$ be a left $B$-module algebra and $F$, an invertible twist based on $B$. Then
	\begin{equation}
		A_F=(\eA,\mu_{\eA}\circ F_l,1_{\eA})
	\end{equation}
	is a left $B$-module algebra.
\end{theorem}

\begin{proof}
	We already proved that the product in $\eA_F$ is associative. Let us now prove that $\eA_F$ is a $B_F$-module. The action of $B_F$ on $\eA_F$ is unchanged: $b\cdot a$ if $b\in B_F$ and $a\in\eA_F$. The first axiom of definition~\ref{DefBModuleAlgebra}, $b\cdot A_{\eA}=\epsilon(b)\cdot 1_{\eA}$, remains because we didn't change $1_{\eA}$ neither $\epsilon$. In the present case, the second condition reads
	\begin{equation}		\label{EqDiagCommdeThobmodalgtwist}
		\xymatrix{%
		\eA\otimes\eA \ar[r]^{\mu_{\eA}\circ F_l}\ar[d]_{\big( F^{-1}\Delta_B(b)F \big)_l}		&	\eA\ar[d]^{b_l}\\
		\eA\otimes\eA \ar[r]_{\mu_{\eA}\circ F_l}	&	\eA
		   }
	\end{equation}
	Let us consider $a\otimes a'\in\eA\otimes\eA$ and follow its evolution when we apply $b_l\circ F_l\circ\mu_{\eA}$.
	\begin{equation}
		\begin{aligned}[]
		a\otimes a'\stackrel{F_l}{\to} F_1a\otimes F_2a'\stackrel{\Delta_B(b)}{\to}\sum b_{(1)}F_1a\otimes b_{(2)}F_2a'\stackrel{F^{-1}}{\to}\sum F^{-1}_1b_{(1)}F_1a\otimes F_2^{-1} b_{(2)}F_2a'\\
		\stackrel{F_l}{\to} \sum b_{(1)}F_1a\otimes b_{(2)}F_1a'\stackrel{\mu_{\eA}}{\to}\mu_{\eA}\big( \stackrel{F_l}{\to} \sum b_{(1)}F_1a\otimes b_{(2)}F_1a' \big).
		\end{aligned}
	\end{equation}
	If we follow the other arrows, we find
	\begin{equation}
			a\otimes a'\stackrel{F_l}{\to}F_1a\otimes F_2a'\stackrel{\mu_{\eA}}{\to}\mu_{\eA}\big( F_1a\otimes F_2 a' \big)\stackrel{b_l}{\to}b\mu_{\eA}\big( F_1a\otimes F_2a' \big).
	\end{equation}
	Now, the commutativity of the untwisted diagram states that
	\begin{equation}
		b\mu_{\eA}(a\otimes a)=\mu_{\eA}\big( \sum b_{(1)}a\otimes b_{(2)}a' \big).
	\end{equation}
	Thus the commutativity of diagram \eqref{EqDiagCommdeThobmodalgtwist} is nothing else than the commutativity of the untwisted diagram taken with $F_1a\otimes F_2a'$ instead of $a\otimes a'$.
\end{proof}

%+++++++++++++++++++++++++++++++++++++++++++++++++++++++++++++++++++++++++++++++++++++++++++++++++++++++++++++++++++++++++++
\section{Twist and star product}
%+++++++++++++++++++++++++++++++++++++++++++++++++++++++++++++++++++++++++++++++++++++++++++++++++++++++++++++++++++++++++++

Let $\eB$ be a connected Lie group with Lie algebra $\lB$ and universal enveloping algebra $\mU(\lB)$.

\begin{definition}
	If $\Delta$ and $\epsilon$ are the usual co-product and co-unit\footnote{See definition~\ref{DefHopfsurCG}.} on $\mU(\lB)$, then a \defe{twist based on}{twist!based on $\mU(\lB)$} based on $\mU(\lB)$ is an element $F\in\mU(\lB)\dcr{h}\otimes\mU(\lB)\dcr{h}$ such that
	\begin{subequations}
		\begin{align}
			(\epsilon\otimes\id)F=1\otimes 1=(\id\otimes \epsilon)F					\label{subEqHpfun}	\\
			\big[ (\Delta\otimes\id)(F) \big](F\otimes 1)=\big[ (\id\otimes\Delta)(F) \big](1\otimes F).	\label{subEqHpfcocs}
		\end{align}
	\end{subequations}
\end{definition}
Remark in that definition the difference between ``$\id$'' which is the identity on $\mU(\lB)$ and ``$1$'' which is the constant function, and then the identity on $ C^{\infty}(\eB)$.

\begin{definition}
	A formal \defe{universal deformation formula}{universal!deformation formula} based on $\mU(\lG)$ is a twisting element $F$ based on $\mU(\lB)\dcr{h}$ which reads
	\begin{equation}
		F=1\otimes 1+hF_1+h^2F_2+\cdots+h^nF_n+\ldots
	\end{equation}
	where $F_i\in\big( \mU(\lB)\otimes\mU(\lB) \big)\dcr{h}$.
\end{definition}

\begin{theorem}
	The data of a formal universal deformation formula on $\mU(\lB)\dcr{h}$ is equivalent to the data of a left invariant star product $\star$ on $\eB$.

	The correspondence is as follows. Let $\star=\sum_k h^kC_k$ where $C_k$ are left invariant bidifferential operators on $\eB$. So
	\begin{equation}
		F=\sum_kh^kC_k
	\end{equation}
	is an element of $\mU(\lB)\dcr{h}\otimes \mU(\lB)\dcr{h}$.

	In that setting, $F$ is a Drinfel'd twist and every Drinfel'd twists are produced in that way\index{Drinfel'd twist}.
\end{theorem}

\begin{proof}
	Let us make the correspondence more explicit. From proposition~\ref{PropbidiffUU}, we have 
    \begin{equation}
        \biDiff^{\eB}(\eB)\simeq\mU(\lB)\otimes\mU(\lB), 
    \end{equation}
    the elements $C_k$ can be seen in $\mU(\lB)\otimes\mU(\lB)$ and we define the $F_k$ by $C_k=F_k^L$ where $T^L\in\biDiff^{\eB}(\eB)$ is the left invariant operator associated with $T\in\mU(\lB)\otimes\mU(\lB)$. Then one defines
	\begin{equation}
		F=\sum_kt^kF_k\in\big( \mU(\lB)\otimes\mU(\lB) \big)\dcr{t}.
	\end{equation}
	What we have to proof is that the so defined $F$ is a twist based on $\mU(\lB)$. We are going to prove that the associativity of $\star$ is equivalent to the condition~\ref{ItemTwistDeux} while the condition $f\star 1=1\star f=f$ is equivalent to the condition~\ref{ItemTwistUn} in definition~\ref{DefTwist}.

	Consider $T\in\mU(\lB)\otimes\mU(\lB)$ as $T=\sum_i X_i\otimes Y_i$ where $X_i,Y_i\in\mU(\lB)$ and the sum is finite. Associativity of the product means that
	\begin{equation}
		\tilde T\circ(\tilde T\otimes \id)=\tilde T\circ(\id\otimes \tilde T).
	\end{equation}
	In our computations we are going to use the following rules:
	\begin{enumerate}

		\item
			$(X\otimes Y)(f\otimes g)=Xf\otimes Yg$,
		\item	 \label{ItemOptimRuleDeux}
			$\widetilde{(X\otimes Y)}(f\otimes g)=(\tilde Xg)(\tilde Yg)\in  C^{\infty}(\eB)$,
		\item
			$(\tilde X\otimes \tilde Y)(f\otimes g)=\tilde Xf\otimes \tilde Yg\in C^{\infty}(\eB)\otimes C^{\infty}(\eB)$
		\item
			$\widetilde{(X\cdot Y)}f=\tilde X(\tilde Yf)$ where the dot denotes the product in $\mU(\lB)$.

	\end{enumerate}
	with, as usual, the definition $(\tilde Xf)(x)=\tilde X_xf\in \eR$. As far as the product in $\mU(\lB)$ is concerned, we have
	\begin{equation}
		\begin{aligned}[]
			\big[ (X_1\otimes Y_1\otimes Z_1)\cdot (X_2\otimes Y_2\otimes Z_2) \big]^{\expotilde}(f\otimes g\otimes h)&=
			\big[ X_1\cdot X_2\otimes Y_1\cdot Y_2\otimes Z_1\cdot Z_2 \big]^{\expotilde}(f\otimes g\otimes h)\\
			&=(X_1X_2)^{\expotilde}(f)(Y_1Y_2)^{\expotilde}(g)(Z_1Z_2)^{\expotilde}(h)
		\end{aligned}
	\end{equation}
	where the dot $\cdot$ denotes the product in $\mU(\lB)$.

	Using these rules we have
	\begin{equation}		\label{EqUneCyclTcondass}
		\begin{aligned}[]
			\tilde T\circ(\tilde T\otimes 1)(f\otimes g\otimes h)&=\sum_{ij}(\tilde X_i\otimes\tilde Y_j)\circ
							\Big( (\tilde X_i\otimes\tilde Y_j)\otimes 1 \Big)(f\otimes g\otimes h)\\
			&=\sum_{ij}(\tilde X_i\otimes\tilde Y_i)\Big( (\tilde X_j\otimes\tilde Y_j)(f\otimes g)\otimes h \Big)\\
			&=\sum_{ij}(\tilde X_i\otimes\tilde Y_i)\Big( (\tilde X_jf)(\tilde Y_jg)\otimes h \Big)\\
			&=\sum_{ij}\tilde X_i\Big( (\tilde X_jf)(\tilde Y_jg) \Big)\tilde Y_ih.
		\end{aligned}
	\end{equation}
	Using the fact that $\Delta$ is the usual coproduct on $\mU(\lB)$ and the formula \eqref{EqXfgDeltaUnif}, the last line equals
	\begin{equation}	\label{EqExpijPqsDex}
		\begin{aligned}[]
			\sum_{ij}\widetilde{\Delta(x_i)}\big( (\tilde X_jf)\otimes(\tilde Y_jg) \big)\tilde Y_ih&=\sum_{ij}\big( \widetilde{\Delta(X_i)}\otimes \tilde Y_i \big)\Big( (\tilde X_jf\otimes\tilde Y_jg)\otimes h \Big)\\
			&=\sum_{ij}\Big[ (\Delta\otimes \id)(X_i\otimes Y_i)\Big]^{\expotilde}   \Big( (\tilde X_jf\otimes\tilde Y_jg)\otimes h \Big)\\
			&=\sum_j\big[ (\Delta\otimes \id)T \big]^{\expotilde}\circ\big[ X_j\otimes Y_j\otimes 1 \big]^{\expotilde}(f\otimes g\otimes h)\\
			&=\big[ (\Delta\otimes \id)T \big]^{\expotilde}\circ\big[ T\otimes 1 \big]^{\expotilde}(f\otimes g\otimes h)\\
			&=\big[ (\Delta\otimes\id)(T)\cdot(T\otimes 1) \big]^{\expotilde}(f\otimes g\otimes h).
		\end{aligned}
	\end{equation}
	Equating the last line with the left hand side of \eqref{EqUneCyclTcondass}, we get
	\begin{equation}
		\tilde T\circ(\tilde T\otimes 1)=\big[ (\Delta\otimes I)\cdot (T\otimes 1) \big]^{\expotilde}
	\end{equation}
	Doing the same for each power of $t$, we get the same equation for $F$ instead of $T$. The same way, we also get
	\begin{equation}
		F^L\circ(I\otimes F^L)=\big[ (I\otimes\Delta)(F)\cdot(1\otimes F) \big]^L,
	\end{equation}
	so that the associativity of $\star$ is equivalent to the Hopf cocycle condition \eqref{subEqHpfcocs}.

	Now, we prove that the fact that $1$ is an unit for the star product is equivalent to the condition \eqref{subEqHpfun}. Associativity means that for every function $f$ we have
	\begin{equation}
		F^L(f\otimes 1)=F^L(1\otimes f)=f.
	\end{equation}
	Using the counit on $\mU(\lB)$ (given by item~\ref{ItemCounitUg} on page \pageref{ItemCounitUg}), we have
	\begin{equation}
		\begin{aligned}[]
			T^L(f\otimes 1)&=\sum_i(X_i^Lf)(Y_i^L1)\\
			&=\sum_i(X_i^Lf)\epsilon(Y_i)\\
			&=(I\otimes \epsilon)(T)^Lf.
		\end{aligned}
	\end{equation}
	By the same computation,
	\begin{equation}
		T^L(1\otimes f)=(\epsilon\otimes I)(T)^Lf.
	\end{equation}
	Thus the equality $1\star f=f\star 1$ is equivalent to $(I\otimes\epsilon)T=(\epsilon\otimes I)(T)$. Now if we want $\sum_i\epsilon(X_i)Y_i^Lf$ to be equal to $f$ for every $f$, we need $Y_i=1$ whenever $X_i=1$. Thus $T$ has to be of the form
	\begin{equation}
		T=1\otimes 1+\sum_iX_i\otimes Y_i
	\end{equation}
	where none of the $X_i$ and $Y_i$ are $1$. In other words, the only term containing $1$ is the term $1\otimes 1$.
\end{proof}

%+++++++++++++++++++++++++++++++++++++++++++++++++++++++++++++++++++++++++++++++++++++++++++++++++++++++++++++++++++++++++++
\section{Formal Extension lemma}		\label{SecExtenLemK}
%+++++++++++++++++++++++++++++++++++++++++++++++++++++++++++++++++++++++++++++++++++++++++++++++++++++++++++++++++++++++++++

The \emph{extension lemme} was already presented in \cite{These,articleBVCS}. A non-formal version is given in section~\ref{SecExtLem}; here we follow the presentation of \cite{QuantifKhalerian} and we give here more details from the formal twist point of view.

Let $\eB_1$ and $\eB_2$ be two Lie groups and let us consider the direct product $\eB=\eB_1\times_R\eB_2$. In particular, $\eB\simeq\eB_1\times\eB_2$ as manifold, $\eB_1$ is normal in $\eB$ and $\eB_1\cap\eB_2=\{ e \}$. The extension map $R$ is given by
\begin{equation}
	R_x(y)=xyx^{-1}
\end{equation}
for every $x\in\eB_1$ and $y\in\eB_2$.

Let now $\lB$, $\lB_1$ and $\lB_2$ be the respective Lie algebras and consider
\begin{equation}
	\rho\colon \lB_2\to \Der(\lB_1)
\end{equation}
be the differential of $R$. At the Lie algebra level we have $\rho_X(Y)=[X,Y]$, but it can be extended to the universal enveloping algebra by the formula
\begin{equation}
	\rho_X(Y)=X^{(1)}\cdot Y\cdot S(X^{(2)})
\end{equation}
where $S$ is the antipode in $\mU(\lB_2)$ and $\Delta(X)=X^{(1)}\otimes X^{(2)}$.
\begin{probleme}
	Could be great to have a confirmation of that.
\end{probleme}


We have an action
\begin{equation}
	\begin{aligned}
		r\colon \eB_2&\to \Aut(\lB_1) \\
		r_h(X)&\mapsto \Dsdd{ R_h( e^{tX}) }{t}{0}
	\end{aligned}
\end{equation}
for every $h\in\eB_2$ and $X\in\lB_1$. This action extends to $\mU(\lB_1)$.

\begin{lemma}
	The action $R$ can be retrieved from $r$ by the formula
	\begin{equation}
		R_h( e^{X})= e^{r_h(X)}.
	\end{equation}
\end{lemma}

\begin{proof}
	By definition, $ e^{r_h(X)}= e^{(dR_h)_eX}=R_h( e^{X})$.
\end{proof}

The map $R_h$ can be extended to $\mU(\lB_1)$ by
\begin{equation}
	r_h(X\cdot Y)=\DDsdd{ R_h( e^{tX})R_h( e^{sY}) }{t}{0}{s}{0}=\DDsdd{ h e^{tX} e^{sY}h^{-1} }{t}{0}{s}{0}
\end{equation}

Let now denote by $\mU(\lB_1)^{\eB_2}$\nomenclature[G]{$\mU(\lB_1)^{\eB_2}$}{The elements of $\mU(\lB_1)$ that are fixed by the action of $\eB_2$.} the set of the elements of $\mU(\lB_2)$ that are fixed by the action $r$, that is the elements $X\in\mU(\lB_1)$ such that $r_h(X)=X$ for every $h\in\eB_2$.

\begin{lemma}
	Let $X\in\mU(\lB_1)$ such that $r_h(X)=X$ and $Y\in\lB_2$. Thus $X\cdot Y=Y\cdot X$.
\end{lemma}

\begin{proof}
	The condition $r_h(X)=X$ means that $\Dsdd{ h e^{tX}h^{-1} }{t}{0}=X$ for every $h\in\eB_2$. Let us take a path $Y(s)$ in $\eB_2$ and derive the equation $r_{Y(s)}X=X$ with respect to $s$:
	\begin{equation}
			0=\DDsdd{ Y(s) e^{tX}Y(s)^{-1} }{t}{0}{s}{0}=\DDsdd{ \AD\big( Y(s) \big) e^{tX} }{t}{0}{s}{0}=[Y,X]
	\end{equation}
\end{proof}

\begin{lemma}
	We have $\Diff^{\eB_1\times\eB_1,\eB_2\times\eB_2}(\eB_1\times\eB_1)\simeq\biDiff^{\eB_1,\eB_2}(\eB_1)$.
\end{lemma}

\begin{proof}
	For the proof we show that both of the two sets can be identified with $\mU(\lB_1)^{\lB_2}\otimes\mU(\lB_1)^{\lB_2}$.

	The $\eB_2\times\eB_2$-invariance of an operator $P\in\Diff^{\eB_1\times\eB_1}(\eB_1\times\eB_1)$ means that for every $u\in C^{\infty}(\eB_1\times\eB_1)$ and $(b_2,b'_2)\in\eB_2\times\eB_2$ we have
	\begin{equation}
		(Pu)(b_2b_1,b'_2b'_1)=P\big( L_{(b_2,b'_2)}u \big)(b_1,b'_1).
	\end{equation}
	Thus for each $t,s\in\eR$, $Z_2,Z'_2\in\lB_2$ we have
	\begin{equation}
		(Pu)\big(  e^{tZ_2}b_1, e^{sZ'_2}b'_1 \big)=P\big( L_{( e^{tZ_2}, e^{sZ'_2})}u \big)(b_1,b'_1)
	\end{equation}
	If $P$ corresponds to $X\otimes Y\in\mU(\lB_1)\otimes\mU(\lB_1)$, the derivative with respect to $t$ and $s$ yields
	\begin{equation}
		\big[ (Z_2\otimes Z'_2)\cdot(X\otimes Y)u \big](b_1,b_2)=\big[ (X\otimes Y)\cdot (Z_2\otimes Z'_2)u \big](b_1,b_2),
	\end{equation}
	that is the fact that $Z_2$ and $Z'_2$ respectively commute with $X$ and $Y$. Thus the elements $X$ and $Y$ have to belong to $\mU(\lB_1)^{\lB_2}$.

	Let now $c\in\biDiff^{\eB_1}(\eB_1)$ be given by
	\begin{equation}
		c(u\otimes v)(b_1)=\sum_{ab}\tilde X_{b_1}^a(u)\tilde X_{b_1}^b(v)
	\end{equation}
	with $X^a,X^b\in\mU(\lB_1)$. Now we want to impose it to be $\eB_2$-invariant for the action $R_y(x)=yxy^{-1}$. So
	\begin{equation}
		(R_y^*c)(u\otimes v)=c\big( R^*_yu\otimes R^*_yv \big),
	\end{equation}
	or more explicitly
	\begin{equation}		\label{EqCuvexplinv}
		c(u\otimes v)(yxy^{-1})=\sum_{ab}\big( \tilde X^a_{yxy^{-1}}u \big)\big( \tilde X^b_{yxy^{-1}}v \big)
		\stackrel{!}{=}\sum_{ab}\tilde X^a_x(R^*_yu)\tilde X^b_x(R^*_yv).
	\end{equation}
	On the one hand
	\begin{equation}
		\begin{aligned}[]
			\tilde X_x^a(R^*_yu)&=\Dsdd{ (R^*_yu)(x e^{tX}) }{t}{0}\\
			&=\Dsdd{ u\big( yx e^{tX}y^{-1} \big) }{t}{0},
		\end{aligned}
	\end{equation}
	and on the other hand
	\begin{equation}
		\tilde X^a_{yxy^{-1}}u=\Dsdd{ u\big( yxy^{-1} e^{tX} \big) }{t}{0}.
	\end{equation}
	If we consider the relation \eqref{EqCuvexplinv} at $x=e$ we find the condition $ e^{tX}y=y e^{tX}$ for every $y$. Taking $y= e^{sY}$ and taking the derivative with respect to $t$ and $s$ we find the relation $[X,Y]=0$.
\end{proof}

Consider the extension
\begin{equation}
	\lB=\lB_2\times_{\rho}\lB_1
\end{equation}
and the associated group extension
\begin{equation}
	\eB=\eB_2\times_R\eR_1
\end{equation}
with $R\colon \eB_2\to \Aut(\eB_1)$.

When $\cA$ and $\cB$ are $C^*$-algebra, the elements of $\cA\otimes \cB$ are limits of sums of the form $\sum_ia_i\otimes b_i$ with $a_i\in\cA$ and $b_i\in\cB$. In the case with $\cA= C^{\infty}(A)$ and $\cB= C^{\infty}(B)$, we have
\begin{equation}		\label{EqCABsimeqCACB}
	C^{\infty}(A)\otimes C^{\infty}(B)\simeq C^{\infty}(A\times B)
\end{equation}
from section~\ref{SecTensProdCSA}.

The product on $\eB$ is given by
\begin{equation}
	(b_2,b_1)\cdot (b_2',b_1')=\big( b_2b_2',b_1\cdot R(b_2)b'1 \big).
\end{equation}
At the level of the regular left representation, we have
\begin{equation}	\label{Eq1507Lbdb1LoL}
	L^*_{(b_2,b_1)}=L^*_{b_2}\otimes L^*_{b_1}\circ R(b_2).
\end{equation}
For sake of compactness we write
\begin{equation}
	A=L_{b_1}\circ R(b_2).
\end{equation}
The identification \eqref{EqCABsimeqCACB} allows us to give a sense to the action of $L^*_{(b_2,b_1)}$ on elements like $(a\otimes u)\otimes(b\otimes v)$. First, the expression $(a\otimes u)\otimes(b\otimes v)$ is associated to $(a\cdot b)\otimes(u\cdot v)$ and \eqref{Eq1507Lbdb1LoL} allows us to write
\begin{equation}
	\begin{aligned}[]
		L^*_{(b_2,b_1)}\big( a\cdot b\otimes u\cdot v \big)&=L^*_{b_2}(a\cdot b)\otimes \big( L_{b_1}\circ R(b_2) \big)^*(u\cdot v)\\
		&=(L^*_{b_2}a)\cdot(L^*_{b_2}b)\otimes (A^*u)\cdot (A^*v)\\
		&=\big( L^*_{b_2}a\otimes A^*u \big)\otimes\big( L^*_{b_2}b\otimes A^*v \big),
	\end{aligned}
\end{equation}
where the last equality is the identification \eqref{EqCABsimeqCACB} in the reverse sense. Thus we \emph{define} that $L^*_{(b_2,b_1)}$ acts on $\Big(  C^{\infty}(\eB_2)\otimes C^{\infty}(\eB_1) \Big)\otimes\Big(  C^{\infty}(\eB_2)\otimes C^{\infty}(\eB_1) \Big)$ as
\begin{equation}		\label{Eq1507LsurBBBB}
	L^*_{(b_2,b_1)}(a\otimes u)\otimes(b\otimes v)=\big( L^*_{b_2}a\otimes A^*u \big)\otimes\big( L^*_{b_2}b\otimes A^*v \big)
\end{equation}
where $A=L_{b_1}\circ R(b_2)$.

Consider the multiplications $\mu^1$ on $ C^{\infty}(\eB_1)$, $\mu^2$ in $ C^{\infty}(\eB_2)$ and $\mu^{12}$ on $ C^{\infty}(\eB_2\times \eB_1)$.

\begin{lemma}		\label{Lem1607mualalmu}
	We have $\mu^{21}\circ(L_{b_2}\otimes A)^*\otimes(L_{b_2}\otimes A)^*=(L_{b_2}\otimes A)^*\circ\mu^{21}$.
\end{lemma}

\begin{proof}
	The result comes essentially that $A$ and $L^*_{b_2}$ commute with the pointwise product of functions:
	\begin{equation}
		\begin{aligned}[]
			A^*(u\cdot v)&=A^*u\cdot A^*v\\
			L^*_{b_2}(a\cdot b)&=(L^*_{b_2}a)\cdot(L^*_{b_2}b).
		\end{aligned}
	\end{equation}
	We have
	\begin{equation}
		\begin{aligned}[]
			(L_{b_2}\otimes A)^*\mu^{21}(a\otimes u)\otimes(b\otimes v)
			&=(L_{b_2}\otimes A)^*(a\cdot b\otimes u\cdot v)\\
			&=\mu^{21}(L^*_{b_2}a\otimes A^*u)\otimes(L^*_{b_2}b\otimes A^*v)\\
			&=\mu^{21}\circ\Big( (L_{b_2j\otimes A})^*\otimes(L_{b_2}\otimes A)^* \Big)(a\otimes u)\otimes(b\otimes v).
		\end{aligned}
	\end{equation}
\end{proof}

The same kind of result.
\begin{lemma}		\label{Lem1607LmumuLotimes}
	We have
	\begin{equation}
		L^*_{(b_2,b_1)}\circ\mu^{21}=\mu^{21}\circ\big( L^*_{(b_2,b_1)}\otimes L^*_{(b_2,b_1)} \big).
	\end{equation}
\end{lemma}

\begin{proof}
	If $f\in C^{\infty}(\eB_2)$ and $g\in C^{\infty}(\eB_1)$, seen as functions on $€B_2\times\eB_1$ which depend of only one variable, we have
	\begin{equation}
		\begin{aligned}[]
			L^*_{(b_2,b_1)}\circ\mu^{21}(f\otimes g)(b'_2,b'_1)&=L^*_{(b_2,b_1)}(f\cdot g)(b'_2,b'_1)\\
			&=(L^*_{(b_2,b_1)}f)(b'_2,b'_1)(L^*_{(b_2,b_1)}g)(b_2',b_1')\\
			&=\mu^{21}\big( L^*_{(b_2,b_1)}\otimes L^*_{(b_2,b_1)} g \big)(b'_2,b'_1)\\
			&=\mu^{21}\circ(L^*_{(b_2,b_1)}\otimes L^*_{(b_2,b_1)})(f\otimes g)(b'_2,b'_1).
		\end{aligned}
	\end{equation}
\end{proof}

A bidifferential operator on $ C^{\infty}(\eB_1)$ reads
\begin{equation}
	\tilde P=\mu^1\circ\sum_i \tilde P'_i\otimes \tilde P''_i
\end{equation}
with $P_i',P''_i\in\mU(\lB_1)$. From $\tilde P$ we define a bidifferential operator on $ C^{\infty}(\eB_2)\otimes C^{\infty}(\eB_1)$ by
\begin{equation}
	\tilde P^{(1)}=\mu^{21}\circ(\id\otimes \tilde P')\otimes(\id\otimes \tilde P'').
\end{equation}
That operator acts on (sums of) elements of the type $(u_1\otimes v_1)\otimes(u_2\otimes v_2)$ with $u_i\in C^{\infty}(\eB_2)$ and $v_i\in C^{\infty}(\eB_1)$ with
\begin{equation}
	(\id\otimes \tilde P'')(u_2\otimes v_2)=u_2\otimes \tilde P''v_2\in C^{\infty}(\eB_2)\otimes C^{\infty}(\eB_1).
\end{equation}
Thus
\begin{equation}
	P^{(1)}(u_1\otimes v_1)\otimes(u_2\otimes v_2)=\mu^{21}(u_1\otimes \tilde Pv_1)\otimes(u_2\otimes \tilde P''v_2)=u_1u_2\otimes \tilde P'v_1\tilde P''v_2.
\end{equation}

In the same way, to every bidifferential operator $\tilde L$ we associate the bidifferential operator $\tilde L^{(2)}$ on $ C^{\infty}(\eB_2\otimes\eB_1)$ defined by
\begin{equation}
	\tilde L^{(2)}=\mu^{21}\circ(\tilde L'\otimes\id)\otimes(\tilde L''\otimes\id).
\end{equation}
where a summation is understood.

The following proposition (as everything here) comes from \cite{QuantifKhalerian}.

\begin{proposition}		\label{PropPBBPtildeComm}
	Let $P\in\mU(\lB_1)\otimes\mU(\lB_1)$ and $\tilde P$ the associated left invariant bidifferential operator on $\eB_1$.
	\begin{enumerate}

		\item\label{ItemPropPBBPtildeCommi}
			For every $g\in\eB_2$ we have $R(g)\circ\tilde P=\tilde P\circ R(g)\otimes R(g)$ if and only if $P\in\big( \mU(\lB_1)\otimes\mU(\lB_1) \big)^{\lB_2}$. The latter subalgebra of $\mU(\lB_1)\otimes\mU(\lB_1)$ being formed by the elements $X_1\otimes Y_1$ such that $\rho(Z)(X_1\otimes Y_1)=0$ for every $Z\in\lB_2$.

		\item\label{ItemPropPBBPtildeCommii}
			For every $P\in\mU(\lB_1)^{\lB_2}\otimes\mU(\lB_1)^{\lB_2}$, the associated left invariant operator $P^{(1)}$ on $ C^{\infty}(\eB_2\times \eB_1)$ is $\eB$-invariant. In other words we have
			\begin{equation}
				L^*_{b_2,b_1}\circ\tilde P^{(1)}=\tilde P^{(1)}\circ L^*_{b_2,b_1}
			\end{equation}
			in the sense of the identification \eqref{EqCABsimeqCACB}.
		\item\label{ItemPropPBBPtildeCommiii}
			If $\tilde L$ is a left invariant bidifferential operator on $ C^{\infty}(\eB_2)$, the associated bidifferential operator $\tilde L^{(2)}$ on $ C^{\infty}(\eB)$ is $\eB$-invariant, that is
			\begin{equation}
				L^*_{(b_2,b_1)}\circ \tilde L^{(2)}=\tilde L^{(2)}\circ L^*_{(b_2,b_1)}.
			\end{equation}
	\end{enumerate}

\end{proposition}

\begin{proof}
	Let $P=X\otimes Y$ with $X,Y\in\lB_1$. First we have $R(g^{-1})\otimes R(g^{-1})=R(g^{-1})X\otimes R(g^{-1})Y$, but by definition we have
	\begin{equation}
		R(g)X= e^{\rho(Z)}X
	\end{equation}
	and $ e^{\rho(Z)}X=X$ if and only if $\rho(Z)X=0$.
	\begin{probleme}
		Because $ e^{tZ}X$ is a polynomial in $t$ which can only be equal to $X$ when the coefficients of all the powers of $t$ are zero??
	\end{probleme}

	Let us now prove that
	\begin{equation}
		R(g)\circ\tilde P=\big[ R(g^{-1})\otimes R(g^{-1})P \big]^{\expotilde}\circ R(g)\otimes R(g),
	\end{equation}
	which is equivalent to the property~\ref{ItemPropPBBPtildeCommi} because of the property we just mentioned. We begin with $P=X_1\otimes X_2\in\lB_1\otimes \lB_2$. We write $X=X_1\otimes X_2$.

	Using the definitions,
	\begin{equation}		\label{EqRgxtXunRgpT}
		\begin{aligned}[]
			\big( R(g)\circ \tilde P \big)(u\otimes v)(x)&=\tilde X(u\otimes v)\big( R(g)x \big)\\
			&=\Dsdd{ u\big( R(g)(x) e^{tX_1} \big) }{t}{0}\Dsdd{ v\big( R(g)(x) e^{tX_2} \big) }{t}{0}.
		\end{aligned}
	\end{equation}
	Since $R_g\in\Aut(\eB_1)$, we have $(R_gx)y=R_g(xR_{g^{-1}}y)$, so that the right hand side of \eqref{EqRgxtXunRgpT} becomes
	\begin{equation}	\label{EqDsdduRgXungu}
		\begin{aligned}[]
			\Dsdd{ u\big( R_g(xR_{g^{-1}} e^{tX_1}) \big) }{t}{0}&\Dsdd{ \big(   R_g(xR_{g^{-1}} e^{tX_2})    \big) }{t}{0}\\
			&=\Dsdd{ (R^*_gu)\big( xR_{g^{-1}} e^{tX_1}\big)}{t}{0}\Dsdd{ (R^*_gv)\big( xR_{g^{-1}} e^{tX_2} \big) }{t}{0}.
		\end{aligned}
	\end{equation}
	For sake of shortness, we write $R_gY=\Dsdd{ R_g e^{tY} }{t}{0}$, so that we have
	\begin{equation}
			\Dsdd{ w\big( xR_{g^{-1}} e^{tX} \big) }{t}{0}=(\widetilde{ R_{g^{-1}}X })_x(w)
			=\widetilde{ (R_{g^{-1}}X) }(w)(x),
	\end{equation}
	and \eqref{EqDsdduRgXungu} becomes
	\begin{equation}
		\begin{aligned}[]
			(\widetilde{ R_{g^{-1}}X_1 })(R_g^*u)(x)(\widetilde{ R_{g^{-1}}X_2 })(R_g^*v)(x)&=\Big[ (R_{g^{-1}}\otimes R_{g^{-1}})(X_1\otimes X_2) \Big]^{\expotilde}\big( R^*_gu\otimes R^*_gv \big)(x)\\
			&=\big[ (R_{g^{-1}}\otimes R_{g^{-1}})X \big]^{\expotilde}\circ(R_g\otimes R_g)(u\otimes v)(x).
		\end{aligned}
	\end{equation}
	This concludes the proof of the point~\ref{ItemPropPBBPtildeCommi}.

	For point~\ref{ItemPropPBBPtildeCommii}, we take $a_i,b_j\in C^{\infty}(\eB_2)$ and $u_i,v_j\in C^{\infty}(\eB_1)$ and we apply $L^*_{(b_2,b_1)}\circ \tilde P^{(1)}$ to the combination $\big( \sum_ia_i\otimes u_i \big)\otimes\big( \sum_j b_j\otimes v_j \big)$. By linearity, we restrict ourself to only one term and we denote by a single dot the pointwise function product:
	\begin{equation}	\label{EqLnnmuLBPP1507}
		\begin{aligned}[]
			\big( L^*_{(b_2,b_1)}\circ\tilde P^{(1)} \big)&\big( (a\otimes u)\otimes (b\otimes v) \big)=\\
			&L^*_{(b_2,b_1)}\circ\mu^{21}\big( (a\otimes \tilde P'u)\otimes(b\otimes \tilde P''v) \big)\\
			&=L^*_{(b_2,b_1)}\big( a\cdot b\otimes \tilde P'u\cdot\tilde P''v \big)\\
			&=L^*_{b_2}(a\cdot b)\otimes \big( L_{b_1}\circ R(b_2) \big)^*(\tilde P'u\cdot \tilde P''v)\\
			&=(L^*_{b_2}a)\cdot(L^*_{b_2}b)\otimes \big( L_{b_1}\circ R(b_2) \big)^*\tilde P'u\cdot \big( L_{b_1}\circ R(b_2) \big)^*\tilde P''v\\
			&=\mu^{12}\Big[ (L^*_{b_2}a)\otimes\big( L_{b_1}\circ R(b_2) \big)^*\tilde P'u \Big]\otimes\Big[ L^*_{b_2}b\otimes(L_{b_1}\circ R(b_2))^*\tilde P''v \Big].
		\end{aligned}
	\end{equation}
	What lies in the first bracket is
	\begin{equation}
		\Big( L^*_{b_2}\otimes\big( L_{b_1}\circ R(b_2) \big)^* \Big)(a\otimes \tilde P'u)=L_{(b_2,b_1)}^*(a\otimes \tilde P'u).
	\end{equation}
	Thus the last line of \eqref{EqLnnmuLBPP1507} provides
	\begin{equation}
		\begin{aligned}[]
			\big( L^*_{(b_2,b_1)}\circ\tilde P^{(1)} \big)&\big( (a\otimes u)\otimes (b\otimes v) \big)\\
			&=\mu^{12}\circ\big( L^*_{(b_2,b_1)}\otimes L^*_{(b_2,b_1)} \big)(a\otimes \tilde P'u)\otimes(b\otimes \tilde P''v)\\
			&=\mu^{12}\circ\big( L^*_{(b_2,b_1)}\otimes L^*_{(b_2,b_1)} \big)\circ\big( (\id\otimes\tilde P')\otimes(\id\otimes\tilde P'') \big)(a\otimes u)\otimes(b\otimes v)
		\end{aligned}
	\end{equation}
	Dropping the explicit reference to the functions $a$, $b$, $u$ and $v$ and using the formula \eqref{Eq1507Lbdb1LoL}, we have
	\begin{equation}
		\begin{aligned}[]
			\big( L^*_{(b_2,b_1)}\circ\tilde P^{(1)} \big)&=
			\mu^{21}\circ\Big[ \id\circ L^*_{b_2}\otimes\tilde P'\circ A \Big]\otimes\Big[  \id\circ L^*_{b_2}\otimes\tilde P'\circ A  \Big]\\
			&=\mu^{21}\circ\Big[(\id\otimes\tilde \tilde P')\circ\big( L^*_{b_2}\otimes A^* \big)]\otimes\Big[  (\id\otimes\tilde P'')\circ\big( L^*_{b_2}\otimes A^* \big)\Big]\\
			&=\mu^{21}\circ(\id\otimes \tilde P')\otimes(\id\otimes\tilde P'')\circ(L^*_{b_2}\otimes A)^*\otimes(L^*_{b_2}\otimes A)^*\\
			&=\tilde P^{(1)}\circ L^*_{(b_2,b_1)}
		\end{aligned}
	\end{equation}
	where we used the expression \eqref{Eq1507LsurBBBB}. This concludes the proof of~\ref{ItemPropPBBPtildeCommii}.

	For~\ref{ItemPropPBBPtildeCommiii}, the hypothesis of invariance is that for every $a,b\in C^{\infty}(\eB_2)$,
	\begin{equation}
		\tilde L(L^*_{b_2}a\otimes L^*_{b_2}b)=(L^*_{b_2}\circ\tilde L)(a\otimes b).
	\end{equation}
	From linearity and the density of $ C^{\infty}(\eB_2)\otimes C^{\infty}(\eB_1)$ in $ C^{\infty}(\eB_2\times\eB_1)$, it is sufficient to study
	\begin{equation}
		L^*_{(b_2,b_1)}\tilde L^{(2)}\Big( (a\otimes u)\otimes(b\otimes v) \Big).
	\end{equation}
	We have
	\begin{equation}
		\begin{aligned}[]
			L^*_{(b_2,b_1)}\tilde L^{(2)}\Big( (a\otimes u)\otimes(b\otimes v) \Big)&=L^*_{(b_2,b_1)}\circ\mu^{21}(\tilde L'a\otimes u)\otimes (\tilde L''b\otimes v)\\
			&=L^*_{(b_2,b_1)}(\tilde L'a\cdot\tilde L''b)\otimes u\cdot v\\
			&=L^*_{b_2}(\tilde P'a\cdot\tilde L''b)\otimes A^*(u\cdot v)\\
			&=\mu^{21}\Big[ L^*_{b_2}\tilde P'a\otimes A^*u \Big]\otimes\Big[ L^*_{b_2}\tilde L''b\otimes A^*v \Big]\\
			&=\mu^{21}\circ\big( (L_{b_2}\otimes A)^*\otimes(L_{b_2}\otimes A)^* \big)\\
			&\quad	\circ\big( (\tilde L'\otimes\id)\otimes(\tilde L''\otimes\id) \big)(a\otimes u)\otimes(b\otimes v)
		\end{aligned}
	\end{equation}
	The result now comes from the fact that
	\begin{equation}
		\mu^{21}\circ(L_{b_2}\otimes A)^*\otimes(L_{b_2}\otimes A)^*=(L_{b_2}\otimes A)^*\circ\mu^{21}.
	\end{equation}
	This is lemma~\ref{Lem1607mualalmu}.
\end{proof}

\begin{theorem}[\defe{formal extension lemma}{formal!extension lemma}]
	Let $\star^j=\sum_kh^kC_k^j$ be left invariant products on $ C^{\infty}(\eB_j)\dcr{h}$, and suppose that $\star^2$ is invariant under $R(g)$ for every $g\in \eB_2$. Then we define the product $\star$ on $ C^{\infty}(\eB)\dcr{h}$ by
	\begin{equation}
		\star=\star^2\otimes\star^2=\sum_kh^kC_k
	\end{equation}
	with
	\begin{equation}
		C_k(a\otimes u,b\otimes v)=\sum_{k_1+k_2=k}C_{k_1}^2(a,b)C^1_{k_2}(u,v).
	\end{equation}
	This product is left invariant under the action of $\eB$.
\end{theorem}

\begin{proof}
	What we have to study is
	\begin{equation}
		L^*_{(b_2,b_1)}\circ C_k\big( (a\otimes u),(b\otimes v) \big)=\sum_{k_1+k_2=k}L^*_{(b_2,b_1)}C^2_{k_1}(a,b)C^1_{k_2}(u,v),
	\end{equation}
	and by linearity, we can restrict ourself to study only one term in the sum:
	\begin{equation}		\label{Eq1607LCCLCLC}
		\begin{aligned}[]
			L^*_{(b_2,b_1)}C_k^2(a,b)C_l^1(u,v)&=L^*_{(b_2,b_1)}\circ\mu^{21}C^2_k(a,b)\otimes C^1_l(u,v)\\
			&=\mu^{21}\Big[ L^*_{(b_2,b_1)}C^2_k(a,b) \Big]\otimes\Big[ L^*_{(b_2,b_1)}C_l^1(u,v) \Big].
		\end{aligned}
	\end{equation}
	because here, $\mu^{21}$ has to be understood as in lemma~\ref{Lem1607LmumuLotimes}, in particular it commutes with $L^*_{(b_2,b_1)}$. We are going to study separately the content of the two brackets. For the first one we have
	\begin{equation}
		L^*_{(b_2,b_1)}C^2_k(a,b)(b'_2,b'_1)=C^2_k(a,b)(b_2b'_2,b_1R_{b_2}b'_1),
	\end{equation}
	but $a$ and $b$ do not depend on the $\eB_1$ component of the variable while $C^2$ is $\eB_2$-invariant, thus
	\begin{equation}	\label{Eq1607BrackUnMu}
		C^2_k(a,b)(b_2b'_2,b_1R_{b_2}b'_1)=C^2_k(a,b)(b_2b'_2)
		=C^2_k(L^*_{b_2}a,L^*_{b_2}b).
	\end{equation}
	For the second bracket of \eqref{Eq1607LCCLCLC} we have
	\begin{equation}		\label{1607BrackDeuxMu}
		\begin{aligned}[]
			L^*_{(b_2,b_1)}C^1_l(u,v)(b'_2,b'_2)&=C^1_l(u,v)\big( b_2b'_2,b_1R(b_2)b'_1 \big)\\
			&=(L^*_{b_1}C^1_l)(u,v)\big( .,R(b_2)b'_1 \big)\\
			&=C^1_l\big( L^*_{b_1}u,L^*_{b_1}v \big)(.,R(b_2)b'_1)\\
			&=C^1_l\big( R(b_2)^*\circ L_{b_1}^*u,A^*v \big)
		\end{aligned}
	\end{equation}
	Putting \eqref{Eq1607BrackUnMu} and \eqref{1607BrackDeuxMu} into the brackets of \eqref{Eq1607LCCLCLC} we find
	\begin{equation}
		\begin{aligned}[]
			L^*_{(b_2,b_1)}C^2_k(a,b)C^1_l(u,v)&=\mu^{21} C^2(L^*_{b_2}a,L^*_{b_2}b)\otimes C^1_l(A^*u,A^*v)\\
			&=\mu^{21}\circ(C^2_k\otimes C^1_l)(L^*_{b_2}a\otimes L^*_{b_2}b)\otimes(A^*u\otimes A^*v)\\
		\end{aligned}
	\end{equation}
	If we make the sum $\sum_{k_1+k_2=k}$ of the latter equation, we find
	\begin{equation}
		\sum_{k_1+k_2=k}C_k\Big( (L^*_{b_2}a\otimes A^*u),(L^*_{b_2}b\otimes A^*v) \Big)=\sum_{k_1+k_2=k}C_k\Big( L^*_{(b_2,b_1)}(a\otimes u),L^*_{(b_2,b_1)}(b\otimes v) \Big).
	\end{equation}
\end{proof}

This was the extension lemma from the point of view of the formal star product. From the point of view of the Drinfel'd twist, we have the following.

\begin{proposition}
	Let $F^{(j)}$ be formal twists based on $\mU(\lB_j)\dcr{h}$ with
	\begin{equation}
		F^{(1)}\in\Big( \mU(\lB_1)\dcr{h}\otimes\mU(\lB_1)\dcr{h} \Big)^{\lB_2}.
	\end{equation}
	Then we define $F\in\mU(\lB)\dcr{h}\otimes\mU(\lB)\dcr{h}$ by
	\begin{equation}
		F=F^{(2)}\otimes F^{(1)}.
	\end{equation}
	This is a Drinfel'd twist based on $\mU(\lB)\dcr{h}$.
\end{proposition}
The article about deformation philosophy by Flato is \cite{FlatoDeforView}.


\chapter{Deformations: non-formal aspects}
\input{twists_general}
\input{qansl}
\input{KG}

\chapter{WKB quantization}
% This is part of (almost) Everything I know in mathematics
% Copyright (c) 2013-2014, 2020
%   Laurent Claessens
% See the file fdl-1.3.txt for copying conditions.

\begin{abstract}
Deformation is a main theme of research in the present work. We begin here to describe WKB quantization and a general method to guess deformations of function algebras. The role of Darboux charts and momentum maps appears clearly. A careful example is given by the deformation of $\SL(2,\eR)$.

We prove a useful result (from \cite{articleBVCS}), the extension lemma, which allows to deform a split extension when one knows a deformation of the two components of the extension. The kernel is simply the product of the two kernels.

Then we see the principle of deformation by action of group: when a Lie group is deformable, one can find a deformation of any manifold on which the group acts. Universal formulas exist in some cases. This is why deformations of groups are studied. An application of that extension lemma to the Iwasawa subgroup of $\SO(2,n)$ is given in chapter~\ref{ChapNoteDev}.

\end{abstract}

\section{WKB quantization}\label{subsec:WKB}
%--------------------------------

More details can be found in the article \cite{lcBBM}. A manifold $M$ is given with its usual commutative and associative algebra $(C^{\infty}(M),\cdot)$ of smooth functions. A \defe{deformation}{deformation}, or a \emph{quantization}\index{quantization}\footnote{In fact, we make a difference between these two words. A \emph{deformation} is only the fact to find a new product from an old one; the new product depends on a parameter and has to reduce to the old one when the parameter goes to zero. A \emph{quantization} is a deformation in which the first order term (whatever it means) of the new product contains the symplectic structure as in condition \eqref{EqExigSymplePremOrd} below.}, of $M$ is the data of a new product $\star^M_{\hbar}$ on a functional space over $M$. 

Let $G$ be a Lie group acting on a manifold $M$. We consider $\Fun(M,\eC)$\nomenclature{$\Fun(M)$}{Functions on the manifold $M$}, the space of all the maps from $M$ to $\eC$, without any regularity conditions. The \defe{regular left representation}{regular!representation}\index{representation!regular left} of $G$ on $M$ is the representation of $G$ on $\Fun(M)$ given by
\begin{equation}
   [ L^*_g(a) ](h)=a(gh)
\end{equation}
for all $a\in\Fun(M)$, $g$, $h\in G$.

A $G$-invariant WKB quantization of $M$ is a product on a space of functions $A^{M}$ on $M$ of the form
\[
  (u\star^M_{\hbar}v)(x)=\int_{M\times M} a_{\hbar}(x_1,x_2,x) e^{\frac{ i }{ \hbar }S(x_1,x_2,x)} u(x_1)v(x_2)\,dx_1\,dx_2
\]
for which we require, among other conditions, (see complete definition~\ref{DefWKBCompl})
\begin{itemize}
\item $A^M\subset \Fun(M)$ is invariant under the regular left representation of $G$ and contains at least the smooth compactly supported functions,
\item the pair $(A^M,\star^M_{\hbar})$ is an associative algebra,
\item the functions $a_{\hbar}$ and $S$ are invariant under the left regular action of $G$,
\item $\forall\, x\in M$ and $\forall\,u,v\in A^M$ the product accepts an asymptotic expansion compatible with the symplectic structure in the following sense:
\[
  (u\ast_{\hbar} v)(x)\sim u(x)v(x)+\frac{ \hbar }{ i }c_{1}(u,v)(x)+o(\hbar^{2})
\]
where $c_{1}$ satisfies $c_{1}(u,v)-c_{1}(v,u)=2\{ u,v \}$.
\end{itemize}
The main property of this product is its $G$-invariance:
\[
  L_g(u\star^M_{\hbar}v)=(L_gu)\star^M_{\hbar}(L_gv).
\]

\subsection{Definitions and general setting}
%------------------------------------------

Let $(M,\omega,\nabla)$ be an affine symplectic manifold, i.e. a $2n$-dimensional symplectic manifold $(M,\omega)$ endowed with a torsion-free connection $\nabla$ such that $\nabla\omega=0$. The \defe{automorphism}{automorphism!of affine symplectic manifold} group $\Aut(M,\omega,\nabla)$\nomenclature[G]{$\Aut(M,\omega,\nabla)$}{Automorphism group of an affine symplectic manifold} is defined as
\[
  \Aut(M,\omega,\nabla)=\gpAff(\nabla)\cap\gpSymp(\omega)
\]
where $\gpAff(\nabla)$ is the group of affine transformations of the affine manifold $(M,\nabla)$ and $\gpSymp(\omega)$ is the group of symplectomorphisms of $(M,\omega)$.


\begin{probleme}
Non mais; où intervient $\nabla$ dans cette définition ? D'après Pierre, il est contenu dans les troisième ordre, mais il faudrait une référence.
\label{ProbNablades}
\end{probleme}

Let $R$ be a subgroup of $\Aut(M,\omega,\nabla)$. The following definition of a $R$-invariant WKB quantization can be found in \cite{StrictSolvableSym}.
\begin{definition}
A $R$-invariant \defe{WKB quantization}{WKB quantization} of $(M,\omega,\nabla)$ is the data of a product
\begin{equation}    \label{EqFormeWKBdel}
(u\star_{\theta}v)(x)=\frac{1}{ \theta^{2n} }\int_{M\times M} a_{\theta}(x,y,z) e^{\frac{ i }{ \theta }S(x,y,z)}u(y)v(z)\,dy\,dz
\end{equation}
(where $dy\,dz$ is the Liouville measure $\omega^{n}/n!$) with the following constrains:
\begin{enumerate}
\item For each $\theta$, we have a space $A_{\theta}$ containing the space $ C^{\infty}_{c}(M)$ of compactly supported smooth functions. The product $\star_{\theta}$ extends to $A_{\theta}$ in such a way that $(A_{\theta},\star_{\theta})$ becomes a one-parameter family of associative $*$-algebras.
\item The product $\ast_{0}$ on $A_{0}$ is the usual pointwise product  and $(A_{0},\ast_{0})$ is a Poisson subalgebra of $ C^{\infty}(M)$ for the induced Poisson structure from the symplectic form $\omega$.
\item $\forall \theta\geq 0$, the space $A_{\theta}$ is a $*$-vector subspace of $ C^{\infty}(M)$ such that \[
   C^{\infty}_{c}(M)\subset A_{0}\subset A_{\theta}
\]
where the involution $*$ on $  C^{\infty}(M)$ is the usual complex conjugation.
\item $S$ is a real valued smooth function $S\colon M\times M\times M\to \eR$ such that for all $x_{0}\in M$, the function $S(x_0,.,.)\in C^{\infty}(M\times M)$ has a nondegenerate critical point at $(x_0,x_0)$.
\item The functions $a_{\theta}$ are positive real-valued:
\[
  a_{\theta}\colon M\times M\times M\to \eR^{+}.
\]
\item The functions $S$ and $a_{\theta}$ are invariant under the diagonal action of $R$ on $M\times M\times M$.
\item $\forall\, x\in M$ and $\forall\,u,v\in C^{\infty}_{c}(M)$ with support in a suitably small neighbourhood of $x$, a stationary phase method yields the extension
\begin{equation}
  (u\star_{\theta} v)(x)\sim u(x)v(x)+\frac{ \theta }{ i }c_{1}(u,v)(x)+o(\theta^{2})
\end{equation}
where $c_{1}$ satisfies
\begin{equation}        \label{EqExigSymplePremOrd}
  c_{1}(u,v)-c_{1}(v,u)=2\{ u,v \}.
\end{equation}

\end{enumerate}
\label{DefWKBCompl}
\end{definition}

We emphasize the fact that the functional space $A^M$ is stable under $\star_{\theta}$: this is a \emph{strict} quantization in contrast to a \emph{formal} star product which only stabilises the space of formal power series of $\theta$.

An example of WKB quantization is the Weyl product which is nothing but an integral reformulation of the Moyal star product:
\[
  (f\star^W_{\hbar}g)(x)=\frac{1}{ \hbar^{2n} }\int_{\mathbb{R}^{2n}\times\mathbb{R}^{2n}}  e^{\frac{ 2i }{ \hbar }S^0(x,y,z)}f(y)g(z)\,dy\,dz
\]
where $S^0(x,y,z)=\Omega(x,y)+\Omega(y,z)+\Omega(z,x)$, and $\Omega$ denotes the usual symplectic form on $\mathbb{R}^{2n}$.

\begin{probleme}
Ce serait bien d'avoir une référence pour cette affirmation. Et aussi de savoir s'il faut un $1/\hbar^{2n}$ devant l'intégrale.
\label{ProbWeylMoy}
\end{probleme}

The function $K=a_{\theta} e^{\frac{ i }{ \theta }S}$ is the \defe{kernel}{kernel!for a WKB quantization} of the product $\star_{\theta}$. The \defe{associativity}{associativity!of a WKB quantization} of $\star_{\theta}$ on the functional space $A_{\theta}$ is the fact that the equality
\[
  \big( (u\star_{\theta}v)\star_{\theta}r \big)(x)=\big( u\star_{\theta}(v\star_{\theta}r) \big)(x)
\]
holds for every $u$, $v$, $r\in A_{\theta}$ and $x\in M$.  That condition translates under an integral form to the following relation
\begin{equation}\label{EqCondAssoc}
\begin{split}
&\int_{M\times M}K(x,y,z)\left[ \int_{M\times M}K(y,t,s)u(t)v(s)\mu_M(t,s) \right] r(z)\mu_M(y,z)\\
&=\int_{M\times M}K(x,y,z)u(y)\left[ \int_{M\times M}K(z,t,s)v(t)r(s)\mu_M(t,s) \right]\mu_M(y,z)
\end{split}
\end{equation}
where $\mu_M(y,z)=\mu_M(y)\mu_M(z)$ is the Liouville measure on $M$. Performing formal manipulations (such as a Fubini theorem), one can express this condition as
\begin{equation}        \label{EqAssosssens}
\int_{M}K(x,y,t)K(t,p,q)\mu(t)=\int_{M}K(x,\tau,q)K(\tau,y,p)\,\mu(\tau).
\end{equation}
That form is easier to handle and to check, but it is meaningless in general.

The fact to have a \defe{left invariant kernel}{left!invariant!kernel} on a group $G$ means that the kernel $K\colon G\times G\times G\to \eC$ has the property $L_g^*K=K$, or
 \begin{equation}
K(gh_{1},gh_2,gh_{3})=K(h_1,h_2,h_{3})
\end{equation}
for every $g\in G$.  The following lemma allows us to use group isomorphisms to push forward a kernel from a group to another.
\begin{lemma}
Let $G_{1}$ and $G_{2}$ be two symplectic Lie groups and $K_1$, a left invariant kernel on $G_{1}$ which provides an associative product on the functional space $A_1$. Let $\phi\colon G_{2}\to G_{1}$ be a symplectic Lie group isomorphism. Then the kernel $K_2=\phi^*K_1$ is invariant and gives rise to an associative product on $A_2=\phi^*A_1$.
\label{LemKerINvarIsom}
\end{lemma}

\begin{proof}
By definition,
\[
(\phi^*K_1)(h_1,h_2,h_3)=K_1\big( \phi(h_1),\phi(h_2),\phi(h_3)\big).
\]
Therefore, using the left invariance of $K_1$, we have
\[
\begin{split}
    L_{g}^*\phi^*K_2=(\phi\circ L_{g})^*K_2=(L_{g}\circ\phi)^*K_2=\phi^*L_{\phi(g)}^*K_1=\phi^*K_1.
\end{split}
\]
That proves left invariance of $\phi^*K_1$ on $G_{2}$.  Now we prove the associativity of $K_2$, this is to check condition  \eqref{EqCondAssoc}. We have
\[
\begin{split}
%A&=\\
 &\int_{G_2\times G_2}K_2(x,y,z)\Bigg[ \int_{G_2\times G_2}K_2(y,t,s)(\phi^*u)(t)(\phi^*v)(s)\mu_2(t,s) \Bigg]\\
&\qquad(\phi^*r)(z)\mu_2(y,z)\\
=&\int_{G_2\times G_2}K_1(\phi x,\phi y,\phi z)\Bigg[  \int_{G_2\times G_2}K_1(\phi y,\phi t,\phi s)u(\phi t)v(\phi s)\mu_2(t,s)  \Bigg]\\
&\qquad r(\phi z)\mu_2(y,z).
\end{split}
\]
We perform in this integral the change of variables $\tau_y=\phi y$, $\tau_t=\phi t$, $\tau_z=\phi z$ and $\tau_s=\phi s$. This does not affect the measure because  $\phi$ is a symplectomorphism and $\mu_i$ are the Liouville measures on $G_i$, so that for example,  $\mu_2(t)=\mu_2(\phi^{-1}\tau_t)=\mu_1(\tau_t)$. The previous integral becomes
\[
\begin{split}
  &\int_{G_1\times G_1}K_1(\phi x,\tau_y,\tau_z)\Bigg[  \int_{G_1\times G_1}K_1(\tau_y,\tau_t,\tau_s)u(\tau_t)v(\tau_s)\mu_1(\tau_t,\tau_s)  \Bigg]\\
&\qquad r(\tau_z)\mu_1(\tau_y,\tau_z).
\end{split}
\]
Using now the associativity of $K_1$ on $G_1$ and performing the inverse change of variables, we find
\[
\begin{split}
\int_{G_2\times G_2}K_2(x,y,z)(\phi^*u)(y)\Bigg[ \int_{G_2\times G_2}K_2(z,t,s)(\phi^*v)(t)(\phi^*r)(s)&\mu_2(t,s)   \Bigg]\\
                    &\mu_2(y,z),
\end{split}
\]
which proves the associativity of $K_2$ on $\phi^*A_1$.

Notice that condition \eqref{EqAssosssens} can be checked in much the same way.

\end{proof}

%TODO : C'est quoi l'erreur dont on parle ici ?
The proposition~\ref{ProperrProdInvarDiffeo} gives an improved form of this lemma. Unfortunately, this generalization revealed to be a mistake. It is worth noticing that lemma~\ref{LemKerINvarIsom} needs a group isomorphism while one often only has a Lie algebra isomorphism. Due to Campbell-Backer-Hausdorff formula, it may be very difficult to find a group isomorphism from an algebra one. An example of this difficulty is in subsection ~\ref{SubSecRemetreMu}.


\begin{remark}
Most of the time, the symplectic condition \eqref{EqExigSymplePremOrd} does not have to be checked because we just define the symplectic form $\omega_2$ on $G_2$ as $\omega_2=\phi^*\omega_1$ where $\omega_1$ is the symplectic form on $G_1$.
\end{remark}

\begin{definition}
When $\alpha\colon G\times A\to A$ is an action of a Lie group $G$ on a vector space $A$, one says that the element $a\in A$ is a \defe{differentiable vector}{differentiable!vector} of $\alpha$ if the map $g\mapsto\alpha_g(a)$ is a differentiable map from $G$ into $A$.
\end{definition}

We are now interested in the regular left representation $L\colon R\times A_{\theta}\to  A_{\theta}$ defined as usual by $\big( L_r(u) \big)(x)=u(r\cdot x)$. A function $u\in A_{\theta}$ is a differentiable vector of $L$ when the map
\begin{equation}
\begin{aligned}
 \alpha_u\colon R&\to  A_{\theta} \\
r&\mapsto L_r(u)
\end{aligned}
\end{equation}
is differentiable. The differential of $\alpha_u$ is what we will denote by $dL$ in the next few pages: $dL(X)u=(d\alpha_u)_eX$. By definition,
\[
  (d\alpha_u)_eX=\Dsdd{ \alpha_u( e^{tX}) }{t}{0}=\Dsdd{ L_{ e^{tX}}(u) }{t}{0},
\]
and the element $(d\alpha_u)_eX\in A_{\theta}$ applied to $x\in M$ is
\begin{equation}
 \big( dL(X)u \big)(x) =\Big( (d\alpha_u)X \Big)(x)=\Dsdd{ L_{ e^{tX}}(u)x }{t}{0}=\Dsdd{ u( e^{tX}\cdot x) }{t}{0}.
\end{equation}
We denote by $ A_{\theta}^{\infty}$ the space of differentiable vectors of the representation $L$.

If one particularises to  $ A_{\theta}\subset  C^{\infty}(R)$ (the manifold $M$ being $R$ itself), the vector fields of $R$ naturally act on $ A_{\theta}$. In particular, if $u\colon R\to \eC$ and $X\in \mR$ we have
\[
  \big( X^*(u) \big)(r)=X^*_r(u)=\Dsdd{ u\big(  e^{-tX}r \big) }{t}{0}=\big( dL(-X)u \big)(r),
\]
so that
\begin{equation}
   dL(X)=-X^*
\end{equation}
holds on the space of differentiable vectors $ A_{\theta}^{\infty}$.

\begin{definition}
A formal star product $\dpt{\ast_G}{\Cinf(M)\dcr{\nu}\times \Cinf(M)\dcr{\nu}}{\Cinf(M)\dcr{\nu}}$ is said to be \defe{$\mG$-covariant}{covariant!star product} if for all $X$, $Y\in\mG$,
\begin{equation}
[\lambda_X,\lambda_Y]_{\ast_G}=2\nu\{\lambda_X,\lambda_Y\}
\end{equation}
where $[\lambda_{X},\lambda_{Y}]_{\ast_{G}}:=\lambda_X\ast_G\lambda_Y-\lambda_Y\ast_G\lambda_X$. In other words the start product is $\mG$-covariant when the expected terms of higher order in the right hand side are zero.
\end{definition}

A crucial use of $\mG$-covariance will be done in proposition~\ref{Proprhonureprez} in order to build a map $\rho_{\nu}$ that fulfils the following proposition (instead of $dL$ itself).
\begin{proposition}
In the setting of definition~\ref{DefWKBCompl}, the map $dL$ is a representation  by derivation of $\mR$  on~$ A_{\theta}^{\infty}$.
\label{prop:dL_reprez}
\end{proposition}

\begin{proof}
We will not pay attention on the domain $A_{\theta}$. Its definition will come later.  First, we prove that $\dpt{dL}{\mR}{\End{ A_{\theta}^{\infty}}}$ is a representation. Indeed,
\begin{equation}
\begin{split}
  dL([X,Y])u=\Dsddp{L^*_{\exp(-t[X,Y])}u}{t}{0}
            &=\Dsddp{ [ L^*_{\exp(-tX)},L^*_{\exp(-tY)}  ] u}{t}{0}\\
        &=\big[dL(X),dL(Y)\big]u.
\end{split}
\end{equation}
Next, $L_R$-invariance of $\ast_{\theta}$ yields
\[
  \big( L^*_{\exp -tX}u \big)\ast_{\theta}\big( L^*_{\exp -tX}v \big)=L^*_{\exp -tX}(u\ast_{\theta} v).
\]
If we derive this equality with respect to $t$ at $t=0$, we find
\[
   dL(X)u\ast_{\theta} v+u\ast_{\theta} dL(X)v=dL(X)(u\ast_{\theta} v).
\]
\end{proof}

\subsection{Deformation of Iwasawa subgroups}   %\label{Sec:mG-covariance}
%-------------------------------------------

The motivation in deforming (or quantizing) groups resides in the method of deformation by group action (section~\ref{SecDefAction}) which states that if one can deform a group, one can write a formula for a deformed product on any manifold on which the group acts.

Let first describe the next few steps in the construction of WKB quantizations of groups. Let $G$ be a semisimple Lie group with its Iwasawa decomposition $G=ANK$. The group $R=AN$ is solvable and can be seen as the homogeneous space $R=G/K$. We consider the canonical multiplicative action $\dpt{\tau}{G\times R}{R}$ which we restrict to $\dpt{\tau}{R\times R}{R}$. We are interested in a $R$-invariant quantization of $R$. Here is a summary of the notations that will be used.

\begin{itemize}
\item $\stM$ is the Moyal star product on $\eR^n$ endowed with its canonical symplectic form,
\item $\star^R_{\theta}$ is the product we are searching for. It has to be defined at least on $ C_c^{\infty}(R)$ and should be extended to $ C^{\infty}(R)$,
\item $A^R\subset\Fun(R,\eC)$ must contain $ C^{\infty}_c(R)$. The purpose is $(A^R,\star^R_{\theta})$ to be an associative algebra and $A^R$ to be invariant under the left regular representation of $R$,
\item $\eA_{\nu}= C^{\infty}(R)[ [\nu]]$ is an intermediary space which serves to guess $\star^R_{\theta}$ and perform formal manipulations with $\rho_{\nu}$ and $dL$,
\item $\ast_M^R$ is the pull-back of Moyal to $\eA_{\nu}$. It serves to formal manipulations in order to guess the twist that defines $\ast^R_{\nu}$,
\item $\ast^R_{\nu}$ is the product on $\eA_{\nu}$. The problem of determining that product is formal. When this problem is solved, we have to prove that in a well chosen $A^R$, taking $\ast_{\nu}^R\to\star^R_{\theta}$ yields a solution to the problem. As previously noticed, in order to make sense, one has to apply $dL$ on the subspace $\eA_{\nu}^{\infty}$ of differentiable vector of the regular left representation. We will however not take care of this issue in the formal manipulations.
\end{itemize}

The main steps are the following:
{\renewcommand{\theenumi}{\arabic{enumi}.}
\begin{enumerate}
\item In the case of a WKB product we saw in proposition~\ref{prop:dL_reprez} that $dL$ is a representation of $\mR$ on $\eA_{\nu}^{\infty}$. Hence we will try to build a formal product for which $dL$ is a representation by derivation. From this point of view, the manipulation with $\rho_{\nu}$ is only a trick designed to guess a product formula.

\item We suppose that the group $R$ ---the one that we are trying to quantize--- has a symplectic structure $\omega$ and we consider $\phi\colon \eR^{2n}\to R$, a Darboux chart; i.e. $\omega=\phi^*\Omega$ where $\Omega$ is the canonic symplectic form on $\eR^{2n}$.

\item We suppose that the left action of $R$ on itself is strongly hamiltonian and we denote by $\lambda_X$ the momentum maps. We suppose that the Moyal product is $\mG$-covariant\footnote{In fact, we only need the $\mR$-covariance.}.

\item We pose $\rho_{\nu}(X)=\frac{1}{ 2\nu }\ad_{\ast_M^R}(\lambda_X)$. The $\mR$-covariance of $\ast_M^R$ is used in order to prove that $\rho_{\nu}$ is a  representation by derivations of $\mR$ on $(\eA_{\nu},\ast_M^R)$.
\item If one can find an intertwining operator between $dL$ and $\rho_{\nu}$ (i.e. if they are equivalent representations), we define $\ast_{\nu}^R$ as the pull-back of $\ast_M^R$ by this intertwining operator. In this case, we prove that $dL$ is a representation by derivations of the product $\ast_{\nu}^R$.
\end{enumerate}

}       % Fin du groupe qui fait que localement les enumerate soient en chiffres arabes au lieu de romains.
        % C'est quand même un petit hack latex que je ne sais pas s'il est très propre.
    It is time to read appendix~\ref{app:Moyal} about the Moyal star-product.

We try now to find a formal product $\ast_{\nu}^R$ on $\eA_{\nu}^{\infty}$ such that $dL$ is a representation by derivations. For this purpose we suppose $R$ to accept a symplectic structure $\omega$ and $\phi\colon \eR^{2n}\to R$ to be a Darboux chart, i.e. $\omega=\phi^*\Omega$ where $\Omega$ denotes the canonical symplectic form on $\eR^{2n}$. Then we bring the Moyal product of $\eR^{2n}$ to $R$ by the usual formula
\begin{equation}
    (u\ast_M^R v)=(u\circ\phi\ast_M v\circ\phi)\circ\phi^{-1}.
\end{equation}
We suppose that product to be $\mG$-covariant\footnote{Only the $\mR$-covariance will be actually used.}:
\begin{equation}
  [\lambda_X,\lambda_Y]_{\ast_M^R}=2\nu \{ \lambda_X,\lambda_Y \}_R.
\end{equation}
 Now we consider the left action of $R$ on itself and we suppose that this is an Hamiltonian action for the symplectic structure $\omega=\phi^*\Omega$ with dual momentum maps $\dpt{\lambda_X}{R}{\eC}$. We define, for each $X\in\mR$, a linear map, $\dpt{\rho_{\nu}(X)}{\eA_{\nu}}{\eA_{\nu}}$ by
\begin{equation}
\begin{aligned}
 \rho_{\nu}\colon \mR&\to \End{\eA_{\nu}} \\
X&\mapsto\us{2\nu}\ad_{\ast_M^R}(\lambda_X)
\end{aligned}
\end{equation}
Notice that the formal series of $[\lambda_X,u]_{\ast_M^R}$ begins with order one, so the division by $\nu$ make sense in the space of formal series.  The main interest of $\rho_{\nu}$ is to be as we want $dL$ to be. So it will be used to guess how to twist the product in order to make $dL$ work as $\rho_{\nu}$.

\begin{proposition}
The map $\rho_{\nu}$ is a representation of $\mR$ on $\eA_{\nu}$, and $\rho_{\nu}(X)$ is a derivation of $(\eA_{\nu},\ast_M^R)$ for each $X\in\mR$.
\label{Proprhonureprez}
\end{proposition}

\begin{proof}
The proof  that $\rho_{\nu}$ is a representation is only to check that the relation $[\rho_{\nu}(X),\rho_{\nu}(Y)]f=\rho_{\nu}([X,Y])f$ holds for any $X$, $Y\in\mR$ and $f\in\eA_{\nu}$. Using the $\mG$-covariance and the Jacobi identity,
\begin{equation}
\begin{split}
  \rho_{\nu}([X,Y])f&=\us{4\nu^2}\ad_{\ast_M^R}(2\nu\lambda_{[X,Y]})f
            =\us{4\nu^2}\ad_{\ast_M^R}([\lambda_X,\lambda_Y]_{\ast_M^R})f\\
        &=\us{4\nu^2}[[\lambda_X,\lambda_Y]_{\ast_M^R},f]_{\ast_M^R}\\
            &=\frac{1}{ 4\nu^2 }(\ad_{\ast_M^R}\lambda_X\circ\ad_{\ast_M^R}\lambda_Y-\ad_{\ast_M^R}\lambda_Y\circ\ad_{\ast_M^R}\lambda_X)f\\
        &=[\rho_{\nu}(X),\rho_{\nu}(Y)]f.
\end{split}
\end{equation}
It remains to check that $\rho_{\nu}(X)(u\ast_M^R v)=\rho_{\nu}(X)u\ast_M^R v+u\ast_M^R\rho_{\nu}(X)v$ for every $X\in\mR$. This is once again just a computation.
\begin{equation}
\begin{split}
   \rho_{\nu}(X)u\ast_M^R v+u\ast_M^R\rho_{\nu}(X)v&=\us{2\nu}(\lambda_X\ast_M^R u-u\ast_M^R\lambda_X)\ast_M^R v\\
                           &\quad+\us{2\nu}u\ast_M^R(\lambda_X\ast_M^R v-v\ast_M^R\lambda_X)\\
                           &=\us{2\nu}\ad_{\ast_M^R}\lambda_X(u\ast_M^R v).
\end{split}
\end{equation}
\end{proof}
Notice that the $\mG$-covariance of $\ast_M^R$ was used to prove that $\rho_{\nu}$ is a representation.  Now, if we could show that $\rho_{\nu}=dL$, then the answer to our deformation problem would be $A_{\theta}=\eA^{\infty}_{\nu}$ and $\stt=\ast_M^R$. But instead of that we have $\rho_{\nu}=dL+o(\nu)$ because
\begin{equation}
\begin{split}
  \rho_{\nu}(X)u&=\us{2\nu}[\lambda_X,u]_{\ast_M^R}
         =\us{2\nu} 2\nu\{\lambda_X,u\}+o(\nu)
     =X^*(u)+o(\nu)\\
    &=-dL(X)u+o(\nu)
\end{split}
\end{equation}
where the notion of fundamental field $X^*$ is taken for the regular left representation (which is Hamiltonian). That shows that $\rho_{\nu}$ is something like a deformation of $dL$. As a consequence, one has $dL(X)=X_{\lambda_X}$, or
 \begin{equation}\label{eq:dL_et_Poisson}
 dL(x)u=X_{\lambda_X}(u)=\{\lambda_X,u\}
 \end{equation}
(see subsection~\ref{app:ham_act}).

Since $\rho_{\nu}$ is not $dL$, the hope is to see if $\rho_{\nu}$ and $dL$ should be \emph{equivalent} representations. As next proposition shows, the fact to find an equivalence between $\rho_{\nu}$ and $dL$ actually solves the problem to find a product for which $dL$ is a representation by derivation.
\begin{proposition}
Let $\dpt{\mT}{\eA_{\nu}}{\eA_{\nu}}$ be an intertwining operator between $dL$ and $\rho_{\nu}$:
\begin{equation}\label{eq:TrnT}
   \mT\rho_{\nu}(X) \mT^{-1}=dL(X).
\end{equation}
If we define the star product $\ast_{\nu}^R$ by
\begin{equation}    \label{Eq_candprodANSL}
   u\ast_{\nu}^R v=\mT_{\nu}(\mT_{\nu}^{-1} u\ast^{R}_M \mT_{\nu}^{-1} v),
\end{equation}
$dL$ becomes a derivation of $\ast_{\nu}^R$.
\label{prop:def_stn}
\end{proposition}

\begin{proof}
If we develop the expression of $dL(X)(u\ast^R_M v)$, we find $\mT\rho_{\nu}(X)(\mT^{-1} u\ast^R_M \mT^{-1} v)$, using the fact that $\rho_{\nu}$ is a derivation of $\ast^R_M$, one easily finds $dL(X)u\ast^R_M v+u\ast_{\nu}^R dL(X)v$.
\end{proof}

\section{Deformation of \texorpdfstring{$\SL(2,\eR)$}{SL2R}}        \label{sec:unifsl}
%++++++++++++++++++++++++++++++++++++++++++++++++++++++++++

\begin{abstract}
This section shows in some detail an instructive example of deformation of an Iwasawa subgroup: the Iwasawa subgroup of $\SL(2,\eR)$.
In this section we will use the parametrization \eqref{EqParmalSL} of $\SL(2,\eR)$, as well as the notations $G=\SL(2,\eR)$ and $\mG=\sldr$. Here are the main steps that will be performed:
{
\renewcommand{\theenumi}{\arabic{enumi}.}

\begin{enumerate}
\item The Iwasawa component $R=AN=G/K$ provides a double covering onto $\mO=\Ad(G)Z$ where $Z$ is any element of $\mK$ (which is one dimensional). The adjoint orbit $\mO$ being endowed with a canonical symplectic form described in subsection~\ref{sub:coadjoint}, we consider on $R$ the corresponding symplectic structure.

\item The map $(a,l)\mapsto \Ad( e^{aH} e^{lE})Z$ turns out to be a global Darboux chart and induces the diffeomorphism
\[
  R\simeq \mO\simeq \eR^2.
\]
Under these identifications, the adjoint action of $R$ on $\mO$ becomes the simple multiplication of $R$ in itself, which is strongly hamiltonian.
\item  The Moyal product is $\gsl(2,\eR)$-covariant for the action of $\SL(2,\eR)$ on $\eR^2$.

\item We explicitly build the intertwining operator between $\rho_{\nu}$ and $dL$ and we write down a product (see proposition~\ref{prop:def_stn}).

\item A theorem is stated in which we list the properties of the so constructed product.

\end{enumerate}

}               % Fin d'un groupe pour faire localement numéroter en chiffres arabes.
\end{abstract}

\subsection{Actions and Symplectic structure}
%--------------------------------------------

For our purpose, we consider the $\Ad^*$-invariant $2$-form $\xi_0\in\mG^*$ and $\widetilde{\mO}=\Ad^*(G)\xi_0$. As seen in~\ref{sub:coadjoint}, the orbit $\widetilde{\mO}$ is a symplectic manifold with
 \begin{equation}  \label{EqAStrucSympCoAdj}
  \widetilde{\omega}_{\xi}(X^*,Y^*)=\langle \xi,[X,Y]\rangle
\end{equation}
 and the dual momentum maps are $\lambda_X(\xi)=\langle\xi,X\rangle$.

%TODO: Check if these are actually the momentum maps.

In the present framework, we can work with adjoint orbits instead of the coadjoint ones because the group $G=\SL(2,R)$ is semisimple. Indeed, in this case, the Killing form\index{killing!form} $\dpt{B}{\mG\times\mG}{\eR}$ gives a $\Ad(G)$-equivariant isomorphism between $\mG$ and $\mG^*$. In order to see that, recall that a basic property of the Killing form is
\begin{equation}\label{eq:B_ad_invar}
   B\big( (\ad X)Y,Z \big)=-B\big(Y, (\ad X)Z \big),
\end{equation}
and when the group is semisimple, $B$ is nondegenerate. The isomorphism is given by
$\dpt{B'}{\mG}{\mG^*}$, $B'(X)Y=B(X,Y)$. The fact that $B$ is nondegenerate makes $B'$ an isomorphism, and the property \eqref{eq:B_ad_invar} gives the $\ad(G)$-equivariance of $B'$:
\[
   B'\big( (\ad X)Y \big)Z=-B'(Y)\big( (\ad X)Z\big).
\]

Here, in contrast with the case studied in~\ref{sub:coadjoint}, we are working with adjoint orbits (and not the \underline{co}adjoint orbits), so the subalgebra to be studied is no more $\widetilde{\mO}$ but
\[
    \mO=\Ad(G)Z,
\]
where $Z$ is the generator of $\mK$ and the symplectic form is not exactly \eqref{eq_omega_Gs}, but
\begin{equation}\label{eq:omega_G}
  \omega_X(A^*,B^*)=B(X,[A,B]).
\end{equation}
The action of $G$ on $\mO$ is $g\cdot X=\Ad(g)X$. The corresponding notion of fundamental field is given by
\[
   X^*_{\phi(a,l)}=\Dsdd{ \Ad(e^{-tX})\phi(a,l) }{t}{0}.
\]
The Iwasawa theorem~\ref{ThoIwasawaVrai} claims that $G/K=AN$ and that we have global diffeomorphism $\mA\oplus\mN\to AN$, $(a,n)\to e^ae^n$; $\mA\to A$, $a\to e^a$; $\mN\to N$, $n\to e^n$. We define $\mR=\mA\oplus\mN$ and the global diffeomorphism
\begin{equation}    \label{EqDefphiaHlEZ}
\begin{aligned}
 \phi\colon \mA\oplus\mN&\to \mO \\
 aH+lE&\mapsto \Ad( e^{aH} e^{lE})Z.
\end{aligned}
\end{equation}
 That map can also be seen as
\begin{equation}
\begin{aligned}
 \phi\colon\eR^2&\to \mO \\
(a,l)&\mapsto\Ad(e^{aH}e^{lE})Z.
\end{aligned}
\end{equation}
 In this way, we identify $\mA\oplus\mN$ and $\eR^2$ as two dimensional space.
\begin{proposition}
As homogeneous space, there is a double covering
\begin{equation}
\begin{aligned}
 \psi\colon G/K&\to \mO \\
[g]&\mapsto \Ad(g)Z.
\end{aligned}
\end{equation}

\end{proposition}
\begin{proof}
The map $\psi$ is well defined and injective (up to the double covering) because the stabilizer of $\mK$ is $K$ from theorem~\ref{tho:Stab_K}. The surjective condition is clear. The \emph{double} covering is expressed by the fact that $\psi([g])=\psi([g'])$ if and only if $g=\pm g'$.
\end{proof}

The symplectic $2$-form $\omega$ on $\mO$ induces a symplectic form
\[
  \Omega=\phi^*\omega
\]
 on $\mA\oplus\mN\simeq\eR^2$.

\begin{proposition}
   The $2$-form $\phi^*\omega$ is constant and its value is
\[
              \Omega:=\phi^*\omega=-2B(F,E)da\wedge dl=\beta da\wedge dl;
\]
in other words, $\phi$ is a $\emph{global}$ Darboux chart for $\mO$.
\label{prop:Omega}
\end{proposition}

\begin{proof}
We have to compute
\[
   \Omega_{(a,l)}(\partial_a,\partial_l)=\omega_{\phi(a,l)}\big( (d\phi)_{(a,l)}\partial_a,(d\phi)_{(a,l)}\partial_l\big).
\]
First, we show that $d\phi(\partial_a)=-H^*_{\phi}$:
\[
\begin{split}
   d\phi_{(a,l)}\partial_a&=\Dsdd{\phi(a+t,l)}{t}{0}
                        =\Dsdd{\Ad(e^{(a+t)H}e^{lE})Z}{t}{0}\\
                        &=\Dsdd{ \Ad(e^{tH}e^{aH}e^{lE})Z  }{t}{0}
                =\Dsdd{ \Ad(e^{tH})\phi(a,l) }{t}{0}\\
                &=-H^*_{\phi(a,l)}.
\end{split}
\]
In the same way, we find $d\phi(\partial_l)=\big( \Ad(e^{aH})E\big)^*_{\phi}$:
\[
 \begin{aligned}
   d\phi_{(a,l)}\partial_l&=\Dsdd{ \Ad(e^{aH}e^{l+tE})Z }{t}{0}
                    =\Dsdd{ \Ad(e^{aH}e^{tE} e^{-aH}e^{aH}e^{lE} )Z }{t}{0}\\
            &=\Dsdd{ \Ad(e^{aH}e^{tE}e^{-aH})\phi(a,l) }{t}{0}
            =\Dsdd{ \Ad( e^{t\Ad(e^{aH})E} )\phi(a,l) }{t}{0}\\
            &=-\big( \Ad(e^{aH})E\big)^*_{\phi(a,l)}.
\end{aligned}
\]
Using formula \eqref{eq:omega_G} for the symplectic form,
\begin{equation}
\begin{split}
  \Omega_{(a,l)}(\partial_a,\partial_l)&=B\big( \phi(a,l),[-H,-\Ad(e^{aH})E]\big) \\
                           &=B\big( \Ad(e^{aH})\Ad(e^{lE})Z,\Ad(e^{aH})[H,E]\big)\\
               &=2B\big( Z,\Ad(e^{-lE})E \big)\\
               &=2B(Z,E).
\end{split}
\end{equation}
Defining $\beta=-2B(E,F)$ we write it as
\begin{equation}
  \Omega=\phi^*\omega=-2B(F,E)da\wedge dl=\beta da\wedge dl.
\end{equation}
\end{proof}
So, as symplectic manifold, $(\mO,\omega)$ is nothing but $(\eR^2,da\wedge dl)$, the diffeomorphism being $\phi$. The symplectic structure $\Omega$ induces a Poisson structure $P$ given by equation \eqref{eq:def_Poisson}. In the present case, it reads
\begin{align}
(\Omega_{ij})&=\beta\begin{pmatrix}
0 & 1 \\
-1 & 0
\end{pmatrix}
&(P)&=\beta^{-1}\begin{pmatrix}
0 & -1 \\
1 & 0
\end{pmatrix}
\end{align}
 and
\begin{equation}\label{eq:Poisson}
  \{f,g\}=\beta^{-1}(\partial_lf\partial_ag-\partial_af\partial_lg).
\end{equation}

The action of $G$ on $\mO$ can be turned into an action on $\eR^2$ using the chart $\phi$. It is done by defining $\dpt{\tau}{G\times\eR^2}{\eR^2}$,
\begin{equation}
   \tau=\phi^{-1}\circ \Ad\circ\phi,
\end{equation}
or $\tau_g(a,l)=\phi^{-1}\big( \Ad(g)\phi(a,l)\big)$.  The notion of fundamental field\index{fundamental!vector field!on $\eR^2$} at $x=(a,l)\in\eR^2$ is thus given by
\begin{equation}
  X^*_x=\Dsdd{e^{-tX}\cdot x}{t}{0}
       =\Dsdd{ \phi^{-1}\big( \Ad(e^{-tX})\phi(a,l)  \big) }{t}{0},
\end{equation}
for which we will often use the path representation
\[
   X^*_x(t)=\phi^{-1}\big( \Ad(e^{-tX})\phi(a,l)  \big).
\]
From $\Ad$-invariance of $\omega$,
\[
   \tau^*\Omega=\tau^*\phi^*\omega
               =(\phi\circ\phi^{-1}\circ \Ad\circ\phi)^*\omega
           =\phi^*(\Ad)^*\omega
           =\phi^*\omega
           =\Omega.
\]
Thus the symplectic form is $G$-invariant:
\begin{equation}     \label{eq:tau_s_Omega}
  \tau^*\Omega=\Omega,
\end{equation}
That implies in particular that $\tau$ satisfies theorem~\ref{tho:equiv_Poisson}.

\begin{proposition}
The action $\tau$ of $G$ on the symplectic space $(\eR^2,\Omega)$ is Hamiltonian and the dual momentum maps $\dpt{\lambda'_X}{\eR^2}{\eR}$ are given by (cf .\ref{def:app_mom_mom_duale})
\begin{equation}
  \lambda'_X(a,l)=-B\big(X,\phi(a,l)\big)
\end{equation}
for each $X\in\mG$.
\label{prop:lambda_X}
\end{proposition}

\begin{proof}
We have first to check the identity $i(X^*)\Omega=i(X^*)(\phi^*\omega)=d\lambda'_X$. Let us apply both sides on the vector\footnote{Existence comes from lemma~\ref{LemFundSpansTan}.} $A^*_x$, with $A\in\mG$ and $x=(a,l)\in\eR^2$. On the one hand
\[
  i(X^*_x)\Omega_x(A^*_x)=\omega_{\phi(x)}\big(   d\phi_xX^*_x,d\phi_xA^*_x   \big),
\]
but
\begin{equation}
  d\phi_xX^*_x=\Dsdd{\phi(X^*_x(t))}{t}{0}
              =\Dsdd{ \Ad(e^{-tX})\phi(aH,lE) }{t}{0}
          =-X^*_{\phi(a,l)}.
\end{equation}
The same being true for $A$,
\[
  i(X^*_x)\Omega_x(A^*_x)=\omega_{\phi(x)}(X^*_{\phi(x)},A^*_{\phi(x)})=B(\phi(x),[X,A]).
\]
On the other hand,
\begin{equation}
\begin{aligned}
    (d\lambda'_X)_x(A^*_x)&=\Dsdd{ (\lambda'_X\circ\phi^{-1})\Big(   \Ad(e^{tA})\phi(a,l)   \Big) }{t}{0}\\
                         &=\Dsdd{  B\Big(X,\Ad(e^{tA})\phi(a,l) \Big)  }{t}{0} \\
                         &=B\Big(  \Dsdd{\Ad(e^{tA})\phi(x)}{t}{0},X   \Big)    & B\text{ is linear }\\
             &=B\Big(  (\ad A)\phi(x),X   \Big)\\
             &=-B\big(\phi(x),(\ad A)X\big) &   B\text{ is }\Ad-\text{invariant}   \\
             &=B(\phi(x),[X,A]).
\end{aligned}
\end{equation}
That proves that $i(X^*)\Omega=d\lambda'_X$.  The second part of the proof is to see that condition \eqref{eq:hamil} holds.  Using the fact that $X_{\lambda'_Y}=Y^*$, we find
\[
\begin{split}
  \{ \lambda'_X,\lambda'_Y \}(a,l)&=-\Omega(X_{\lambda'_X},X_{\lambda'_Y})
        =-\Omega_{(a,l)}(X^*,Y^*)\\
        &=-\omega_{\phi(a,l)}(X^*,Y^*)
        =-B([X,Y],\phi(a,l))\\
        &=\lambda'_{[X,Y]}(a,l)
\end{split}
\]
where the star refers to the action on $\mO$. Explicit computations of Poisson bracket between $\lambda'_X$'s at page \pageref{pg:explic_com_lamb} will confirm that result.

\end{proof}

We are now able to furnish explicit formulas for $\lambda'_H$, $\lambda'_E$ and $\lambda'_F$ by virtue of the latter proposition.  The first computation is:
\begin{equation}
\begin{aligned}
  \lambda'_H(a,l)&=-B(H,\Ad(e^{lE})Z)
                =-B( \Ad(e^{-lE})H,Z )\\
        &=-B(H+[-lE,H]+\ldots,Z)
        =-B(H,Z)+B([-lE,H],Z)\\
        &=-2lB(E,F),
\end{aligned}
\end{equation}
so
\begin{equation}   \label{EqlamHal}
  \lambda'_H(a,l)=-\beta l.
\end{equation}
Second,
\begin{equation}
  \lambda'_E(a,l)=-B(\Ad(e^{-aH})E,\Ad(e^{lE}))
        =-e^{-2a}B(\Ad(e^{-lE})E,Z)
        =-\frac{\beta}{2}e^{-2a}.
\end{equation}
Then,
\begin{equation}  \label{EqlamEal}
\lambda'_E(a,l)=-\frac{\beta}{2}e^{-2a}.
\end{equation}
The last one is
\begin{equation}
\begin{aligned}
\lambda'_F(a,l)&=-B\big(  \Ad(e^{lE})Z,e^{-aH}F  \big)
              =-e^{2a}B\big(Z,  \Ad(e^{-lE})F   \big)\\
          &=-e^{2a}B\big(Z, F-l[E,F] +\frac{l^2}{2} [E,[E,F]]+\ldots  \big)\\
          &=-e^{2a}\left[     B(Z,F)-lB(Z,H)-\frac{l^2}{2}B(Z,2E)        \right]\\
          &=-e^{2a}\big(  B(Z,F)+l^2B(F,E)   \big)\\
          &=-e^{2a}\big(  -\frac{\beta}{2}-l^2\frac{\beta}{2}   \big)
          =e^{2a}\frac{\beta}{2}(l^2+1).
\end{aligned}
\end{equation}
Finally,
\begin{equation}  \label{EqlamFal}
\lambda'_F(a,l)=\frac{\beta}{2}e^{2a}(l^2+1).
\end{equation}
Using formula \eqref{EqPoisson} for the Poisson bracket, one can check that the required relations \eqref{eq:hamil} are satisfied:
\begin{subequations}  \label{pg:explic_com_lamb}
\begin{align}
  \{\lambda'_H,\lambda'_E\}&=2\lambda'_E\\
  \{\lambda'_H,\lambda'_F\}&=-2\lambda'_F\\
  \{\lambda'_E,\lambda'_F\}&=\lambda'_H.
\end{align}
\end{subequations}
This confirms the fact that our action of $\SL(2,\eR)$ on $AN$ is Hamiltonian.

Using the global diffeomorphism \eqref{EqDefphiaHlEZ}, and the map
\begin{equation}
\begin{aligned}
 j\colon AN&\to \mO \\
r&\mapsto \Ad(r)Z
\end{aligned}
\end{equation}
we identify
\[
   R\simeq\mO\simeq\eR^2.
\]
The action of $R$ on itself induced from the adjoint action of $R$ on $\mO$ is
\[
  r\cdot s=j^{-1}\big( r\cdot j(s) \big)=j^{-1}\big( \Ad(rs)Z \big)=rs.
\]
It is the left multiplicative action required in definition~\ref{DefWKBCompl}. The Lie group $R$ is endowed with the symplectic form
\[
\omega^R=j^*{\phi^{-1}}^*\Omega.
\]
The notion of fundamental vector for the action of $R$ on itself is given by
\begin{equation}
  X^*_r=\Dsdd{e^{-tX}\cdot r}{t}{0}
        =\Dsdd{ j^{-1}\big( e^{-tX}\cdot j(r) \big) }{t}{0}
        =dj^{-1} X^*_{j(r)},
\end{equation}
but we know that
\[
  e^{-tX}\cdot j(r)=\Ad(e^{-tX r})Z=[\phi\circ \tau(e^{-tX}r)\circ \phi^{-1}]Z,
\]
 then
\[
X^*_r=dj^{-1}\circ d\phi X^*_{r\cdot \phi^{-1}(Z)}.
\]
If $r=e^{aH}e^{lE}$, then $r\cdot \phi^{-1}(Z)=(a,l)$ and
\begin{equation}
   X^*_r=(dj^{-1}\circ d\phi) X^*_{(a,l)}
\end{equation}
where the fundamental field of the right hand side is taken in the sense of the action of $R$ on $\eR^2$.

The following proposition shows that the explicit form of $\lambda$ and $\lambda'$ are the same up to natural identifications.

\begin{proposition}
The left multiplicative action of $R$ on itself is Hamiltonian and the dual momentum maps are given by  $\lambda_X\colon R\to \eC$,
\begin{equation}
\lambda_X= \lambda'_X\circ\phi^{-1}\circ j.
\end{equation}
for each $X\in\mR$.
\label{PropMomslR}
\end{proposition}

\begin{proof}
Once again, the proof is just a verification of the two properties of a momentum map. The first one is
\begin{equation}
\begin{split}
  i(X^*_r)\omega^R Y&=\omega^R_r(dj^{-1} d\phi X^*_{(a,l)},Y)
        =\Omega_{(\phi^{-1}\circ j)(r)}\big( X^*_{(a,l)},d\phi^{-1} dj_r Y \big)\\
        &=(\lambda'_X\circ d\phi^{-1}\circ dj)Y
        =d\lambda_X Y.
\end{split}
\end{equation}
For the second condition, we consider $r=e^{aH}e^{lE}$ and
\begin{equation}
\begin{split}
  \{ \lambda_X,\lambda_Y \}(r)&=X^*_r(\lambda_Y)
        =(dj^{-1} d\phi X^*_{(a,l)})(\lambda'_Y\circ \phi^{-1}\circ j)\\
        &=X^*_{(a,l)}(\lambda'_Y)
        =\lambda'_{[X,Y]}(a,l)\in\eC
\end{split}
\end{equation}
while
\[
  \lambda_{[X,Y]}(r)=\lambda'_{[X,Y]}\circ\phi^{-1}\circ j(r)=\lambda'_{[X,Y]}(a,l).
\]
\end{proof}

\subsection{Guessing the star product}
%--------------------------------------

The Moyal star product is invariant under the action of $\eR^2$ on itself $L_xy=x+y$ in the sense that if we pose $(L_y^*f)(x)=f(x+y)$ it is clear that
\begin{equation}
\begin{split}
    (L_s^*f\ast_M L_s^*g)(x)&=
    \exp\left[{\displaystyle\frac{\nu}{2}P^{ij}(\partial_{y^i}\wedge\partial_{z^j})}\right]f(y+s)g(z+s)|_{y=z=x}\\
                        &=L^*_s(f\ast_M g)(x).
\end{split}
\end{equation}
We are however not interested by that action on $\eR^2$. The action which we look at is the one of $\SL(2,R)$.

\begin{proposition}
The product $\ast_M$ is $\gsl(2,\eR)$-invariant at order $0$ and $1$.
\end{proposition}

\begin{proof}
The invariance at order zero is given with some concise notations by
\[
 (gu)(gv)(x)=u(gx)v(gx)=(uv)(gx),
\]
The action $\tau_g$ of an element $g\in G$ satisfies $\tau_g^*\Omega=\Omega$ (equation \eqref{eq:tau_s_Omega}), so  theorem~\ref{tho:equiv_Poisson} gives $\{u\circ\tau_g,v\circ\tau_g\}=\{u,v\}\circ\tau_g$.  Since Poisson bracket\index{Poisson structure!and Moyal product} is the first term of the Moyal product, at first order
\[
  \tau_g^*(u\ast_M v)=\tau_g^*u\ast_M\tau_g^*v.
\]


\end{proof}

\begin{proposition}
   The product $\ast_M$ is $\sldr$-covariant for the homomorphism given by proposition~\ref{prop:lambda_X} or equivalently by equations \eqref{EqlamHal}, \eqref{EqlamEal}, and \eqref{EqlamFal}.
\end{proposition}

\begin{proof}
The Moyal star product can be written as
\[
   u\ast_M v=\sum \frac{\nu^k}{k!}P_k(u,v)
\]
with $P_k(u,v)=\Omega^{IJ}\partial_Iu\partial_Jv$ where $I$ and $J$ are summed over $k$-uple of $0$ and $1$, including a sum over $k$ itself ($x^0=a$, $x^1=l$). For a given $I$, there is only one $J$ such that $\Omega^{IJ}\neq 0$. There are $\binom{k}{m}$ multi-indices $I$ providing the term $\partial_I=\partial_0^m\partial_1^n$ with $n+m=k$. For each of them, $\Omega^{IJ}=\me{n}$. Therefore
\begin{equation}\label{eq:P_k}
  P_k(u,v)=\sum_{m=0}^k\me{k-m}\binom{k}{m}\partial_0^m\partial_1^nu\,\partial_0^n\partial_1^mv.
\end{equation}
For example,
\[
  P_1(u,v)=-\partial_1u\partial_0v+\partial_0u\partial_1v=\{u,v\}.
\]
If $k$ is even, the expression \eqref{eq:P_k} is symmetric with respect of $u$ and $v$, so that these terms will not contribute in the computation of the commutators $[u,v]_{\ast_M}$. We are left with
\begin{equation}\label{eq:comm_lambda_X}
   [\lambda'_X,u]_{\ast_M}
         =2\sum_{k=0}^{\infty}\frac{\nu^{2k+1}}{(2k+1)!}P_{2k+1}(\lambda'_X,u).
\end{equation}

First we compute $[\lambda'_H,u]_{\ast_M}$:
\begin{equation}
   P_{2k+1}(\lambda'_H,u)=\delta_{k0}(-\partial_1\lambda'_H\partial_0 u+\partial_0\lambda'_H\partial_1 u)=\delta_{k0}\{\lambda',u\},
\end{equation}
thus
\begin{equation}
\begin{split}
  [\lambda'_H,u]_{\ast_M}=2\nu P_1(\lambda'_H,u)
                        =2\nu\{\lambda'_H,u\}
            =2\nu\beta\partial_au.
\end{split}
\end{equation}
By the way, we point out the relation
\[
\ad_{\ast_M}\lambda'_H=2\nu\beta\partial_a.
\]

Now, we turn our attention to the commutator $[\lambda'_E,u]_{\ast_M}$:
\begin{equation}
\begin{split}
  P_{2k+1}(\lambda'_E,u)
      &=-\sum_{n=0}^{k}\me{m}\binom{2k+1}{m}\binom{2k+1}{m}(\partial_0^m\partial_1^n\lambda'_E)\,(\partial_0^n\partial_1^mu)\\
      &=\partial_a^{2k+1}\big( -\frac{\beta}{2}e^{-2a} \big)\partial_l^{2k+1}u
    =\beta 2^{2k}e^{-2a}\partial_l^{2k+1}u,
\end{split}
\end {equation}
thus
\begin{equation}  \label{eq:comm_lambda_E}
\begin{split}
  [\lambda'_E,u]_{\stM}&=2\sum_{k=0}^{\infty}\frac{\nu^{2k+1}}{(2k+1)!}\beta
                            2^{2k}e^{-2a}\partial_l^{2k+1}u
                 =\beta e^{-2a}\sinh(2\nu\partial_l)u,
\end{split}
\end{equation}
so that
\[
     \ad_{\stM}\lambda'_E=\beta e^{-2a}\sinh(2\nu\partial_l).
\]
Last we check $[\lambda'_E,\lambda'_F]_{\stM}=2\nu\{\lambda'_E,\lambda'_F\}$. When $u=0$, the only non vanishing term in the sum \eqref{eq:comm_lambda_E} is $k=0$. Since $\partial^{3}_{l}\lambda'_{F}=0$,
\[
   [\lambda'_E,\lambda'_F]_{\stM}=2\nu\beta e^{-2a}\partial_l\lambda'_F,
\]
but
\begin{equation}
 2\nu\{\lambda'_E,\lambda'_F\}
             =2\nu(\partial_a\lambda'_E\partial_l\lambda'_F-\partial_l\lambda'_E\partial_a\lambda'_F)
         =2\nu\beta e^{-2a}\partial_l\lambda'_F.
\end{equation}
\end{proof}

Before to go on, let us compute the operator $\ad_{\stM}\lambda'_F$ in order to complete our collection. We take once again the formula \eqref{eq:P_k}, with $\lambda'_F$ and $u$:
\begin{equation}
   P_{2k+1}(\lambda'_F,u)=-\sum_{m=0}^{2k+1}\me{m}\binom{2k+1}{m}\partial_a^m\partial_l^n\lambda'_F
                                                                \partial_a^n\partial_l^m u.
\end{equation}
It is clear that $\lambda'_F$ can be derived  only two times with respect of $l$ and as much as we want with respect of $a$. Then possible $n$ are $n=0,1,2$, whose corresponding $m$ are $2k-1$, $2k$, and $2k+1$. Some computations lead to
\begin{equation}
\begin{split}
    P_{2k+1}(\lambda'_F,u)&=-k(2k+1)\beta 2^{2k-1}e^{2a}\partial_a^2\partial_l^{2k-1}u\\
                    &\quad +(2k+1)\beta 2^{2k}l\partial_a\partial_l^{2k}u\\
                    &\quad -\beta 2^{2k}(1+l^2)e^{2a}\partial_l^{2k+1}u.
\end{split}
\end{equation}
Replacing into the series \eqref{eq:comm_lambda_X}, we find
\begin{align*}
\begin{split}
[\lambda'_F,u]_{\stM}&=e^{2a}\Big\{
   \sum\frac{\nu^{2k+1}}{(2k+1)!}\beta (-k)(2k+1)\frac{2^{2k+1}}{2}
                     \partial_a^2\partial_l^{2k-1}u\\
 &\qquad+\sum\frac{\nu^{2k+1}}{(2k+1)!}(2k+1)2^{2k+1}l\partial_a\partial_l^{2k}u\\
 &\qquad\beta\sum\frac{\nu^{2k+1}}{(2k+1)!}2^{2k+1}(1+l^2)\partial_l^{2k+1}u\Big\}
\end{split}\\
\begin{split}
 &=-\beta e^{2a}\partial_a^2\sum_{k=1}^{\infty}\frac{(2\nu)^{2k}}{(2k)!}k\nu\partial_l^{2k-1}\\
 &\quad+2\beta\nu e^{2a}\partial_a\circ\cosh(2\nu\partial_l)\\
 &\quad-\beta e^{2a}(1+l^2)\sinh(2\nu\partial_l).
\end{split}
\end{align*}
Finally,
\begin{equation}
\begin{split}
   \ad_{\stM}\lambda'_F&=-\nu^2\beta e^{2a}\partial_a^2\circ\sinh(2\nu\partial_l)\\
                     &\quad+2\nu\beta e^{2a}l\partial_a\circ\cosh(2\nu\partial_l)\\
             &\quad-e^{2a}(1+l^2)\sinh(2\nu\partial_l).
\end{split}
\end{equation}

\begin{corollary}
The star product $\ast_M^R$ on $R$ defined for $u,v\in C^{\infty}(R)$ by
\begin{equation}
  (u\ast_M^R v)(r)=( u\circ T^{-1}\stM v\circ T^{-1} )T(r)
\end{equation}
where $T=\phi^{-1}\circ j$ is covariant for the functions $\lambda$ of proposition~\ref{PropMomslR}.

\end{corollary}
Remark that from general theory of star products, the so-defined $\ast_R$ is a formal star product on $R$.

\begin{proof}
From definition of $\ast_M^R$, on the one hand
\[
  (\lambda_X\ast_M^R \lambda_Y)(r)-X\leftrightarrow Y=(\lambda'_X\stM\lambda'_Y)T(r)-X\leftrightarrow Y
=2\nu\{  \lambda'_X,\lambda'_Y \}_{\eR^2}T(r),
\]
while on the other hand, $\omega^R=T^*\Omega$, so that point~\ref{ite_equivii} of theorem~\ref{tho:equiv_Poisson}  gives
\[
  \{ \lambda'_X,\lambda'_Y \}_{\eR^2}\circ T=\{ \lambda'_X\circ T,\lambda'_Y\circ T \}_R=\{ \lambda_X,\lambda_Y \}_R.
\]

\end{proof}

All that makes the theory developed earlier (in particular proposition~\ref{prop:def_stn}) valid here. So we pose
\begin{equation}
\begin{aligned}
 {\rho_{\nu}}\colon\sR &\to {\End\big(  C^{\infty}(R)[ [\nu]] \big)} \\
  X &\mapsto {\frac{1}{2\nu}\ad_{\ast_M^R}(\lambda_X);}
\end{aligned}
\end{equation}
using the explicit expressions of $ad_{\stM}(\lambda'_X)$, we find
\begin{align}
  \rho_{\nu}(H)&=\beta\partial_a,   &\rho_{\nu}(E)&=\frac{\beta}{2\nu}e^{-2a}\sinh(2\nu\partial_l).
\end{align}
Using \eqref{eq:dL_et_Poisson} with $\lambda_H=-\beta l$, it is clear that $dL(H)=-\beta\{l,u\}=\beta\partial_a u$. Therefore
\begin{equation}
   \rho_{\nu}(H)=dL(H),
\end{equation}
but the requested identity $\rho_{\nu}(E)=dL(E)$ will not hold. The problem is that $dL(X)=X_{\lambda_X}$ is a vector field, while $\rho_{\nu}(E)$ comes with (infinitely) multiple derivatives, hence this is not a vector field. Conclusion: the operator $\mT$ of equation \eqref{eq:TrnT} must not act on the variable~$a$.

First we consider a partial Fourier transform $\mF$\nomenclature[F]{$\mF$}{Fourier transform}\index{partial!Fourier transform}:
\begin{equation}
 (\mF u)(a,\alpha)=\hu(a,\alpha):=\us{\sqrt{2\pi}}\int e^{-i\alpha l}u(a,l)dl,
\end{equation}
the inverse being given by
\begin{equation}
 (\mF^{-1}\hu)(a,l)=\us{\sqrt{2\pi}}\int e^{il\alpha}\hu(a,\alpha)d\alpha.
\end{equation}
It is clear that $\mF\rho_{\nu}(H)\mF^{-1}=\rho_{\nu}(H)$, but $\mF\rho_{\nu}(E)\mF^{-1}=\frac{\beta}{2\nu}e^{-2a}\sinh(2i\nu\alpha)$. Indeed, if we define $\hv(a,\alpha)=\sinh(2i\nu\alpha)\hu(a,\alpha)$,
\begin{equation}
\begin{split}
 (\rho_{\nu}(E)\mF^{-1}\hu)(a,l)&=\frac{\beta}{2\nu}e^{-2a}\us{\sqrt{2\pi}}\sinh(2\nu\partial_l)\int e^{il\alpha}\hu(a,\alpha)d\alpha\\
                        &=\frac{\beta}{2\nu}e^{-2a}\us{\sqrt{2\pi}}\int e^{il\alpha} \sinh(2i\alpha\nu)\hu(a,\alpha)d\alpha\\
            &=\frac{\beta}{2\nu}e^{-2a}(\mF^{-1}\hv)(a,l).
\end{split}
\end{equation}
This is nothing but the fact that the Fourier transform turns a derivation into a multiplication.

As can be seen on an asymptotic expansion, the deformation $\nu$ parameter is necessarily purely imaginary, then we can here pose $\nu=i\theta$ with $\theta\in\eR$, so that
\begin{equation}\label{eq:FrnEF}
   \mF\rho_{\nu}(E)\mF^{-1}=\frac{\beta i}{2\theta}e^{-2a}\sinh(2\alpha\theta).
\end{equation}
Using \eqref{eq:dL_et_Poisson}, we find
\begin{equation}\label{eq:dLE}
   dL(E)=\beta e^{-2a}\partial_l.
\end{equation}
 Comparing it with the expression of $\mF\rho_{\nu}(E)\mF^{-1}$, we see that (up to constant factor) we have to act in such a way that $\sinh(2\alpha\theta)$ is converted into a derivation. This is done by a Fourier transform. We pose $\xi=\sinh(2\theta\alpha)$ and
\[
  \tilde f (a,\xi)=\us{\sqrt{2\pi}}\int e^{i\xi p}f(a,p)dp.
\]
As usual,
\[
   \widetilde{\partial_{\alpha}f}=-i\xi\tilde f .
\]
This suggests us to consider the change of variable
\[
  \phi_{\theta}(a,\alpha)=(a,\us{2\theta}\sinh(2\theta\alpha)),
\]
and finally,
\begin{equation}
   \mT_{\theta}:=\mF^{-1}\circ\phi_{\theta}^*\circ\mF,
\end{equation}
where $\phi_{\theta}^*$ is defined by $(\phi_{\theta}^*u)(a,\alpha)=u(a,\us{2\theta}\sinh(2\theta\alpha))$. The result of our construction is the following which proves that we are in the situation of proposition~\ref{prop:def_stn}.
\begin{theorem}
\[
   \mT_{\theta}\circ\rho_{\nu}(E)\circ \mT_{\theta}^{-1}=\beta e^{-2a}\partial_l=dL(E).
\]
\end{theorem}
\begin{proof}
Notice that $(\phi_{\theta}^*\mF u)(a,\alpha)=\hu(a,\sinh(2\theta\alpha))$, and then define $\hv(a,\alpha)=\hu(a,\sinh(2\theta\alpha))$; equation \eqref{eq:FrnEF} is
\[
   (\mF\rho_{\nu}(E)\mF^{-1}\hv)(a,\alpha)=\frac{\beta i}{2\theta}e^{-2a}\sinh(2\theta\alpha)\hv(a,\alpha).
\]
Applying $(\phi^*)^{-1}$ on the right hand side, we find $\frac{\beta i}{2\theta}e^{-2a}2\theta\alpha\hu(a,\alpha)$.  This allows us to compute
\[
\begin{split}
  (\mT_{\theta}\rho_{\nu}(E)\mT_{\theta}^{-1})u(a,l)&=\beta ie^{-2a}\mF^{-1}(\alpha\hu)(a,l)\\
                                        &=\beta ie^{-2a}\us{\sqrt{2\pi}}\int \hu(a,\alpha)(-i)\partial_l e^{il\alpha}d\alpha\\
                    &=\beta e^{-2a}(\partial_lu)(a,l).
\end{split}
\]
\end{proof}

\subsection{Formula for the product}
%-----------------------------------

The fact the $\mT_{\theta}$ intertwines $\rho_{\nu}$ and $dL$ makes that the candidate to be a product on the $AN$ of $\SL(2,\eR)$ can be computed using formula \eqref{Eq_candprodANSL}. Computations are rather long and done in the articles \cite{StrictSolvableSym} and \cite{Biel-Massar} (see particularly point 4), so we will not give them here. We will also not precise the functional space of convergence for the resulting product.

\begin{probleme}
Il faut voir si ça entrelace effectivement ces deux cocos, ou bien juste $\rho_{\nu}(E)$ et $dL(E)$.
\label{Probintertw}
\end{probleme}

We are searching for a function $\dpt{\rho}{R}{\eR}$ such that
\begin{equation}\label{eq:1s15}
  \int_Rf(r)\rho(r)dr=\int_Rf(br)\rho(r)dr,
\end{equation}
where $dr=da\wedge dl$. We consider the multiplication map $\dpt{\phi}{R\times R}{R}$, $\phi(r,r')=rr'$, and its partial derivatives $A^i_j=\dsd{\phi^i}{r^j}$.  We will perform the change of variable $ s =br$ in \eqref{eq:1s15}. So, $ s^i=\phi^i(b,r)$, and if we pose , we have $dr^k=(A^{-1})^k_id s^i$. We find
\[
  \int_Rf(r)\rho(r)dr=\int_Rf( s)\rho(b^{-1} s)\det(A^{-1})d s.
\]
In the right-hand side, we rename $ s$ to $r$ and we impose the equality to be correct for all functions $\dpt{f}{R}{\eC}$. Then for all $r$, $k\in R$,
\[
   \rho(r)=\rho(k^{-1} r)\det(A^{-1}).
\]
 In order to compute the value of $\det(A)$, we have to write $\big(dL_{(a,b)}\big)_{(a',l')}$ in a matrix form. If we consider $v=(a'(t),l'(t))$, we have
\begin{equation}
\big(dL_{(a,b)}\big)_{(a',l')}v=\frac{d}{dt}\begin{pmatrix}
e^{a+a'(t)} & e^{a+a'(t)}(l'(t)+e^{-2a'(t)l}) \\
0 & e^{-a-a'(t)}
\end{pmatrix}_{t=0},
\end{equation}
but we know that $v$ can be written as
\[
  v=\begin{pmatrix}
\frac{da'}{dt}e^{a'} & \frac{da'}{dt}e^{a'}l'+e^{a'}\frac{dl'}{dt} \\
0 & -\frac{da'}{dt}e^{a'}
\end{pmatrix}
\]
Now, if we want to write $\big(dL_{(a,b)}\big)_{(a',l')}$ as $\begin{pmatrix}A&B\\C&D\end{pmatrix}$, we obtain: $A=e^a$, $C=0$, $D=e^{-a}$, and thus $\det(A)=1$.

So we can use $\rho(r)=1$ for all $r\in R$, and the integral of $\dpt{f}{R}{\eC}$ over $R$ can be written as
\[
\int_Rf(r)dr
\]
with $dr=da\wedge dl$.

In the parametrization
\[
  (a,l)=\begin{pmatrix}
 e^{a}      &  e^{a}l\\
0       &    e^{-a}
\end{pmatrix},
\]
of $R=AN$ the form $da\wedge dl$ is a left invariant measure, so the integral of the function $f\colon R\to \eR$ on $R$ is given by
\[
  \int_R f=\int_{\eR^2} f(a,l)da\,dl.
\]
Remark that $da\,dl$ is the Liouville measure\index{Liouville measure!for $\SL(2,\eR)$} by proposition~\ref{prop:Omega}. It is important for definition~\ref{DefWKBCompl}.

We consider a subset $\eA\subset\Fun(R)$, and we define the product $\star^{R}_{\theta}$ on $\eA$ by
\begin{equation}\label{eq:star_R}
\begin{split}
(a\star^{R}t b)(a_0,l_0)
=\int_{R\times R}K^R_{\theta}\big((a_0,l_0)&,(a_1,l_1),(a_2,l_2)\big)\\
                                           &a(a_1,l_1)b(a_2,l_2)da_1dl_1da_2dl_2.
\end{split}
\end{equation}
where
\[
K_{\theta}^A(g_0,g_1,g_2)=\us{\theta^2}\mA^R(g_0,g_1,g_2)e^{i\theta \mS^R(g_0,g_1,g_2)}
\]
 with
\begin{subequations}
\begin{align}
  \mA^R(g_0,g_1,g_2)&=\bigoplus_{0,1,2}\cosh(a_1-a_2)\\
  \mS^R(g_0,g_1,g_2)&=\bigoplus_{0,1,2}\sinh(2(a_0-a_1))l_2.
\end{align}
\end{subequations}
Here, the symbol $\bigoplus_{0,1,2}$ stands for a cyclic sum over the indices $0,1,2$.
These functions are invariant\index{invariant!function on a group} under the $\SL(2,R)$ left action: when $x$, $y$, $z$, $g\in\SL(2,R)$,
\begin{equation}
 \mA^R(gx,gy,gz)=\mA^R(x,y,z)\quad\text{and}\quad \mS^R(gx,gy,gz)=\mS^R(x,y,z).
\end{equation}
 The first equality is clear; let us show the second. If $x=(a_x,l_x)$ (the same for $y$ and $z$) and $g=(a,l)$, the computation of $\mS^R(gx,gy,gz)$ is the one of $\mS^R(x,y,z)$ with the replacements
\begin{subequations}
\begin{align}
a_x&\rightarrow a_x+l\\
l_x&\rightarrow l_x+e^{-2a_x}l,
\end{align}
\end{subequations}
and the same for $y$ and $z$. Using the formula $\sinh t=\frac{1}{2}(e^{t}-e^{-t})$, one finds the right cancellations.

\begin{probleme}
Il faut trouver l'article où ce résultat se trouve, et citer le théorème exact.
\label{ProbEnonSLdef}
\end{probleme}

\subsubsection{Remark on (formal) star product}\label{subsec:rem_on_sp}
%/////////////////////////////////////////////

Since proposition~\ref{prop:def_stn}, we know that the star product to be used is
\[
    (u\stt v)=T_{\theta}(T_{\theta}^{-1} u\stM T_{\theta}^{-1} v)
\]
with $T_{\theta}=\mF^{-1}\circ\phi^*_{\theta}\circ\mF$ and $\phi_{\theta}(a,\alpha)=(a,\us{2\theta}\sinh(2\alpha\theta))$. One easily finds that
\begin{equation}
\begin{split}
   (T^{-1}_{\theta} u)(a,l)&=(\mF^{-1}\circ\phi^*_{\theta}\circ\mF u)(a,l)\\
                        &=\us{2\pi}\int e^{\displaystyle i\alpha l}d\alpha\int
                          e^{\displaystyle-im\us{2\theta}\sinh(2\alpha\theta)}u(a,m)dm\\
			&=\us{2\pi}\int
			  e^{\displaystyle i[\alpha l-\us{2\pi}\sinh(2\theta\alpha)l_1] }
			  u(a,l_1)dl_1d\alpha.
\end{split}
\end{equation}

From this expression, we have to find an integral expression for the product $\stt$ and see that it is a WKB product. In order to do it, we have to write the Moyal star product into an integral form. So we have to develop a theory in which one can write oscillatory integrals and power expansions which give the usual Moyal expansion \eqref{eq:Moyal}. This is our subject of concern now.

An other way to state the same problem is to consider $(V,\Omega)$, a symplectic vector space of dimension $2n$. The \defe{Weyl product}{Weyl!product} of $u$, $v\in\Cinf_c(V)$ at $x\in V$ is defined by
\[
  (u\stW v)(x)=\theta^{-2n}\int_{V\times V}e^{\frac{2i}{\theta}S^0(x,y,z)}u(y)v(z)dyz
\]
where $S^0(x,y,z)=\Omega(x,y)+\Omega(y,z)+\Omega(z,x)$. With the
 definition  $\nu=\theta/2i$, it can be asymptotically developed by
\[
  u\stWt v\sim uv+\nu\{u,v\}+\sum_{k=2}^{\infty}\frac{\nu^k}{k!}\Omega^{IJ}\partial_Iu\partial_Jv.
\]
How can we give a sense to this expansion?

One can  find a definition of an asymptotic expansion for oscillating integrals in \cite{Dieu7} under the form
\[
   I_{\lambda}=\int e^{(i/\lambda)S(x)}\phi(x)\sim\sum_n\lambda^nc_n.
\]
It can be shown that such a  expansion used on \eqref{eq:star_R} gives rise of a formal star product:
\begin{equation}    \label{EqDevFedFor}
(a\star^{R} b)(g)\sim a(g)b(g)+\frac{\theta}{2i}\{a,b\}(g)+o(\theta^2).
\end{equation}

\section{Non formal Extension lemma}        \label{SecExtLem}
%----------------------------------

Let $(\mfs_i,\Omega_i)_{i=1,2}$ be  symplectic Lie algebras and $(S_i,\omega_i)$ the respective Lie groups with left invariant symplectic forms: $(\omega_i)_g=(L_g)^*\Omega_i$.
We suppose to know a homomorphism  $\dpt{\rho}{\mfs_1}{\Der(\mfs_2)\cap\mfsp(\Omega_2)}$ and a Darboux chart $\phi_i\colon \mathfrak{s}_i\to S_i$ for each of the two symplectic Lie groups. Our first purpose is to build a Darboux chart on the split extension
\[
   \mfs:=\mfs_1\oplus_{\rho}\mfs_2.
\]

\begin{remark}
Most of the time we are in the inverse situation: we have an algebra $\mathfrak{s}$ which turn out to be a split extension $\mathfrak{s}_{1}\oplus_{\ad}\mathfrak{s}_{2}$ for which we have to check that $\ad(\mathfrak{s}_{1})$ is a symplectic action of $\mathfrak{s}_{1}$ on $(\mathfrak{s}_{2},\Omega_{2})$. See the example of section~\ref{SecUnifSOdn}.
\end{remark}

\begin{proposition}
In this setting, the map $\dpt{\phi}{\mfs}{S}$,
\begin{equation}
  \phi(X_1,X_2)=\phi_2(X_2)\phi_1(X_1)
\end{equation}
is a Darboux chart.
\label{prop:Darboux}
\end{proposition}

\begin{proof}
An element $X\in T_{\phi^{-1}(g)}(\mfs_1\oplus\mfs_2)=\mathfrak{s}_1\oplus\mathfrak{s}_2$ is denoted by $X=(X_1,X_2)$ with $X_i\in\mathfrak{s}_i$, and the symplectic form on $\mfs_1\oplus\mfs_2$ is given by
\begin{equation}        \label{eq:Omega}
   \Omega\big( (A_1,A_2),(B_1,B_2) \big)=\Omega_1(A_1,B_1)+\Omega_2(A_2,B_2)
\end{equation}
where we identify  $\mfs_i$ and $T_e\mfs_i$.  Let $A$ and $B$ belongs to $T_{\phi^{-1}(g)}(\mathfrak{s}_1\oplus\mathfrak{s}_2)$. We have to show that the quantity
\begin{equation}\label{eq:omega_g_et_omega_e}
\begin{split}
\omega_g\Big(   (d\phi)_{\phi^{-1}(g)}A&,(d\phi)_{\phi^{-1}(g)}B      \Big)\\
                                &=\omega_e\Big(  (dL_{g^{-1}})_g (d\phi)_{\phi^{-1}(g)}A,(dL_{g^{-1}})_g(d\phi)_{\phi^{-1}(g)}B      \Big)
\end{split}
\end{equation}
does not depend on $g$.

The vector $A$ is represented by a path $A(t)=( A_1(t),A_2(t) )$ with $A_i(t)\in\mfs_i$. In order to characterise that path, we want first to know precisely what is $A_i(0)$. Since $A\in T_{\phi^{-1}(g)}\mfs$, the path must fulfil $\phi( A_1(0),A_2(0) )=g$, or
\begin{equation}
  \phi_2(A_2(0))\phi_1(A_1(0))=g.
\end{equation}
We denote $A_i(0)=G_i\in\mfs_i$ and $\phi_i(G_i)=g_i$. The relation between $g_1$ and $g_2$ is $g_2g_1=g$.  In particular, it is wrong to say ``$A_1(0)=\phi^{-1}(g)$, thus $\phi_1(A_1(0))=g$''. This point being clear,
\begin{equation}
  (d\phi)_{\phi^{-1}(g)}A=\Dsdd{\phi(A(t))}{t}{0}=\Dsdd{ \phi_2(A_2(t))\phi_1(A_1(t)) }{t}{0}.
\end{equation}
If one particularises to the case $A\in\mfs_2$, that is $A_1(t)=cst=G_1$,
\begin{equation}
  (d\phi)_{\phi^{-1}(g)}A=(dR_{g_1})_{g_2}(d\phi_2)_{G_2}A_2.
\end{equation}

Since $g=g_1g_2$, we have $L_{g^{-1}}=L_{g_1^{-1}}\circ L_{g_2^{-1}}$, and the first argument of $\omega_e$ in equation \eqref{eq:omega_g_et_omega_e} is
\[
   (dL_{g_1^{-1}})_{g_1}(dL_{g_2^{-1}})_{g_2g_1}(dR_{g_1})_{g_2}(d\phi_2)_{G_2}A_2.
\]
If we write that in terms of the derivative of the path $A_2(t)$, what we get in the derivative is
\begin{equation}
g_1^{-1} g_2^{-1} \phi_2(A_2(t))g_1=\AD_{g_1^{-1}}\Big( g_2^{-1}\phi_2(A_2(t)) \Big).
\end{equation}
Since $g^{-1}_2\phi\big( A_2(0) \big)=g^{-1}_2\phi_2(G_2)=e$, the derivative of that term is
\begin{equation}
  (dL_{g^{-1}})_g(d\phi)_{\phi^{-1}(g)}A=\Ad_{g_1^{-1}}\Big(  (dL_{g_2^{-1}})_{g_2}(d\phi_2)_{G_2}A  \Big)
\end{equation}
with some abuse between $A\in\mfs$ and $A_2\in\mfs_2$.  Doing the same computation with $B\in\mfs_1$ (so that $B_2(t)=cst=G_2$), we find
\begin{equation}
   (d\phi)_{\phi^{-1}(g)}B=\Dsdd{ g_2\phi_1(B_1(t)) }{t}{0}
                         =(dL_{g_2})_{g_1}(d\phi_1)_{G_1}B_1,
\end{equation}
 and what appears in $\omega_e$ reads
\begin{equation}
  (dL_{g^{-1}})_g(dL_{g_2})_{g_1}(d\phi_1)_{G_1}B=(dL_{g_1^{-1}})_{g_1}(d\phi_1)_{G_1}B.
\end{equation}
Finally, for $A\in\mfs_2$ and $B\in\mfs_1$,
\begin{equation}\label{eq:gros_omega_e}
\begin{split}
\omega_g\big( (d\phi)_{\phi^{-1}(g)}A&,(d\phi)_{\phi^{-1}(g)}B \big)\\
                                    &= \omega_e\Big(
       \Ad_{g_1^{-1}}\big[  (dL_{g_2^{-1}})_{g_2}(d\phi_2)_{G_2}A  \big],
       (dL_{g_1^{-1}})_{g_1}(d\phi_1)_{G_1}B
         \Big).
\end{split}
\end{equation}
The first argument belongs to $T\mfs_2$ (because $g_2\in\mfs_2$) while the second belongs to $T\mfs_1$. Hence definition \eqref{eq:Omega} makes the right hand side vanishing.

If we want to compute equation \eqref{eq:gros_omega_e} with $A$, $B\in\mfs_2$,
\begin{equation}
\begin{split}
\omega_e\Big(
           \Ad_{g_1^{-1}}\big[&  (dL_{g_2^{-1}})_{g_2}(d\phi_2)_{G_2}A  \big],
       \Ad_{g_1^{-1}}\big[  (dL_{g_2^{-1}})_{g_2}(d\phi_2)_{G_2}B  \big]
        \Big)\\
  &=\Omega(\ldots)=\underbrace{\Omega_1(\ldots)}_{=0}+\Omega_2(\ldots)\\
  &=\Big( \Ad_{g_1^{-1}}^*\Omega_2  \Big)
     \Big(
        (dL_{g_2^{-1}})_{g_2}(d\phi_2)_{G_2}A,\ldots B
     \Big)
\end{split}
\end{equation}
At this point, notice that $\Ad^*_{g_1}\Omega_2=\Omega_2$. Indeed the exponential $\exp\colon \mathfrak{s}_1\to S_1$ being surjective, there exists a $X_1\in\mathfrak{s}_1$ such that $\Ad(g_1)= e^{\ad(X_1)}$. Now, $\ad(X_1)\in\gsp(\Omega_2)$ by assumption, so that $\Ad(g_1)\in\SP(\Omega_2)$. The previous expression becomes
\begin{equation}
\begin{split}
  \Omega_2( (dL_{g_2^{-1}})_{g_2}(d\phi_2)_{G_2}A,\ldots B )
  &=(\omega_2)_{g_2}( (d\phi_2)_{g_2}A,\ldots B )\\
  &=(\phi_2^*\omega_2)_{G_2}(A,B)\\
  &=\Omega_2(A,B).
\end{split}
\end{equation}
The last line is the fact that $\phi_2$ is a Darboux chart: $ \phi_2^*\omega_2=\Omega_2$. The case with $A$, $B\in\mfs_1$ yields to compute
\[
   \omega_e\Big(
                   (dL_{g_1^{-1}})_{g_1}(d\phi_1)_{G_1}A,  (dL_{g_1^{-1}})_{g_1}(d\phi_1)_{G_1}B
           \Big).
\]
It is done by the same way as the previous cases.
\end{proof}

A direct computation shows the following \defe{extension lemma}{extension!lemma}.
\begin{lemma}[Extension lemma] \label{EXT}
Let $K_i\in \Fun(S_i^3) $ be a left invariant three point kernel on $S_i$ ($i=1,2$).  Assume that $K_2\otimes1\in \Fun(S^3)$ is invariant under conjugation by elements of $S_1$.  Then $K:=K_1\otimes K_2\in\Fun(S^3)$ is left invariant (under $S$).
\end{lemma}

\begin{proof}
Every element of $S$ can be written under the form $g_1g_2$ with $g_i\in S_i$. The multiplication is given by $(g_1g_2)(a_1a_2)=(g_1a_1)(g_2a_2)$. Using this rule, the definition of the tensor product, and the left invariance of both $K_i$,
\[
\begin{split}
\big( L_{g_1g_2}(K_1\otimes K_2) \big)&(a_1a_2,b_1b_2,c_1c_2)\\
            &=(K_1\otimes K_2)\big( (g_1g_2)(a_1a_2),(g_1g_2)(b_1b_2),(g_1,g_2)(c_1c_2) \big)\\
            &=K_1(g_1a_1,g_1b_1,g_1c_1)K_2(g_2a_2,g_2b_2,g_2c_2)\\
            &=K_1(a_1,b_1,c_1)K_2(a_2,b_2,c_2)\\
            &=(K_1\otimes K_2)(a_1a_2,b_1b_2,c_1c_2).
\end{split}
\]

\end{proof}

This lemma shows that if one has kernels on $S_1$ and $S_2$ satisfying the above hypotheses, their tensor product provides a kernel for an associative left invariant kernel on $S=S_1\otimes_{\rho} S_2$.  Proposition~\ref{prop:Darboux} allows us to hope that the product on $S$ will satisfy the same kind of symplectic compatibility as the products on $S_i$; in particular when the latter were constructed using Darboux chart in the same way as the product described in section ~\ref{sec:unifsl}.


\section{Deformation of \texorpdfstring{$\SU(1,n)$}{SU1n}}   \label{SecDefSURme}
%+++++++++++++++++++++++++++++++++++++++++++++++++++++++++

Before going on with the construction of a deformation of one dimensional split extensions of Heisenberg algebras, we have to recall a result on deformation in $\SU(1,n)$. The product on the extension of Heisenberg algebra will be nothing else than a transport of this one.

The article \cite{Biel-Massar} provides a formal universal deformation formula for the actions of the Iwasawa component $\SUR_0:=\SUA_0\SUN_0$ of $\SU(1,n)$ under an oscillatory integral form.  It turns out (see \cite{lcBBM}) that this deformation formula is in fact non-formal for proper actions on topological spaces.

Here is the precise result. The Iwasawa decomposition of $\SU(1,n)$ induces the identification $\SUR_0=\SU(1,n)/U(n)$. The group $\SUR_0$ is endowed with a (family of) left invariant symplectic structure(s)\footnote{This is done using the hermitian symmetric structure, cf proposition 1.1 in \cite{Biel-Massar}.} $\omega$.  If we denote by $\sR_0=\sA_0\oplus\sN_0$ the Lie algebra of $\SUR_0$, the map
\begin{equation}  \label{DARBOUX}
\begin{aligned}
 \phi_0\colon \sR_0&\to \SUR_0 \\
(a,n)&\mapsto \exp(a)\exp(n)
\end{aligned}
\end{equation}
reveals to be a global Darboux chart for $(\SUR_0,\omega)$.  The nilpotent component appears to accept a decomposition $\sN_0=V\times\eR Z$ in which the Lie bracket reads
\[
[(x,z)\,,\,(x',z')]=\Omega_V(x,x')\,Z;
\]
the full Iwasawa component is now parametrized by $\sR_0=\{(a,v,z)\,|\,,a,z\in\eR;x\in V\}$. The interest of this situation resides in the fact that the algebra $\sR_0$ turns out to be a one dimensional split extension of an Heisenberg algebra; namely,
\[
\sR_0=\mF(\mtu,0,2).
\]
The deformation result is the following.

\begin{theorem} \label{ThoDefHeizsansB}
For every non-zero $\theta\in\eR$, there exists a Fréchet function space $\swE_\theta$ satisfying the inclusions $C^\infty_c(\SUR_0)\subset\swE_\theta\subset C^\infty(\SUR_0)$, such that, defining for all $u,v\in C^\infty_c(\SUR_0)$
\begin{equation}  \label{PRODUCT}
\begin{split}
(u\star_\theta v)(a_0,x_0,z_0)
        :=\frac{1}{\theta^{\dim \SUR_0}} \int_{ \SUR_0\times \SUR_0}& \cosh(2(a_1-a_2))\\
        &[\cosh(a_2- a_0)\cosh(a_0-a_1)\,]^{\dim \SUR_0-2}\\
&\times \exp \Big( \frac{2i}{\theta}\varphi(r_0,r_1,r_2)\Big)\\
        &\times u(a_1,x_1,z_1)\,v(a_2,x_2,z_2)\, da\,dx\,dz;
\end{split}
\end{equation}
where
\[
\begin{split}
  \varphi(r_0,r_1,r_2)=&S_V\big(\cosh(a_1-a_2)x_0, \cosh(a_2-a_0)x_1, \cosh(a_0-a_1)x_2\big)\\
            &-\bigoplus_{0,1,2}\sinh(2(a_0-a_1))z_2
\end{split}
\]
with $S_V(x_0,x_1,x_2):=\Omega_V(x_0,x_1)+\Omega_V(x_1,x_2)+\Omega_V(x_2,x_0)$ is the phase for the Weyl product on $C^\infty_c(V)$ and $\bigoplus_{0,1,2}$ stands for cyclic summation, one has:

\begin{enumerate}

\item\label{tBMi}
     $u\star_\theta v$ is smooth and the map $ C^\infty_c(\SUR_0)\times C^\infty_c(\SUR_0) \to C^\infty(\SUR_0)$ extends to an associative product on $\swE_\theta$. The pair $(\swE_\theta,\star_\theta)$ is a (pre-$C^\star$) Fréchet algebra.

\item\label{tBMii}
     In coordinates $(a,x,z)$ the group multiplication law reads
\[
    L_{(a,x,z)}(a',x',z')=\left(a+a',e^{-a'}x+x',e^{-2a'}z+z'+\frac{1}{2}\Omega_V(x,x')e^{-a'}
\right).
\]
The phase and amplitude occurring in formula \eqref{PRODUCT} are both invariant under the left action $L:\SUR_0\times \SUR_0\to \SUR_0$.

\item\label{tBMiii}
     Formula \eqref{PRODUCT} admits a formal asymptotic expansion of the form:
 \begin{equation*}
    u\star_\theta v\sim \,uv\,+\,\frac{\theta}{2i}\{u,v\}\,+O(\theta^2);
\end{equation*}
where $\{\,,\,\}$ denotes the symplectic Poisson bracket on $C^\infty(\SUR_0)$ associated with $\omega$.  The full series yields an associative formal star product on $(\SUR_0,\omega)$ denoted by $\tilde{\star}_\theta$.
 \end{enumerate}
\end{theorem}

The setting and~\ref{tBMi} and~\ref{tBMii} may be found in \cite{Biel-Massar}, while~\ref{tBMiii} is a straightforward adaptation  to $\SUR_0$ of \cite{lcBBM}.


%%%%%%%%%%%%%%%%%%%%%%%%%%
%
   \section{One dimensional split extensions of Heisenberg algebras} \label{SecExtHeiz}
%
%%%%%%%%%%%%%%%%%%%%%%%%

\subsection{Introduction}
%------------------------

The one dimensional extensions of Heisenberg algebras are classified by triples $(\BX,\mu,d)$. The quantization in the case $(\id,0,\mu)$  reveals to be a particular case of the one studied in \cite{Biel-Massar} (see also section~\ref{SecDefSURme}),%
while quantization of other extensions can be found using symmetries of the kernel. Here we are reporting results of \cite{articleBVCS} and most of proofs (in particular the trick of subsection~\ref{subsecTrick} %
which allows to extend the known product to every one dimensional split extensions) are due to Y. Voglaire. It is to be published in his future PhD thesis.

The kernel of the quantization of \cite{Biel-Massar} will be denoted by $K$.  Then we will give a way to twist $K$ in order to obtain a kernel $K'$ on any extension of the form $(\BX,0,2)$. Quantizations of other extensions can be obtained by composing with Lie group isomorphisms. The kernel for an arbitrary extension is denoted by $K_{0}(\BX,\mu,d)$, or simply $K_{0}$ when there are no possible ambiguity.

When we will deal with the anti de Sitter situation, our starting point will be this $K_{0}$ that we will have to adapt to another symplectic form that $\delta E^*$ invoking lemma~\ref{LemJumpCoadOrb}.

\subsection{General definitions}
%--------------------------------

Let $\pH_{n}=V\oplus\eR E$ be the \hyperlink{HyperHeisenberg}{Heisenberg algebra} of dimension $2n+1$, with a natural symplectic structure defined from the Heisenberg algebra structure:
 \[
[v,w]=\Omega(v,w)E
\]
for all $v$, $w\in V$. Now we consider a one dimensional algebra $\mA=\eR A$ generated by an element $A$, and we build the split extension of $\pH_{n}$ by $\mA$:
\begin{equation}
\mF(\rho)=\mA\oplus_{\rho}\pH_{n}
\end{equation}
where the split homomorphism is an action by derivation $\rho\colon \mA\to \Der(\pH_{n})$. The so obtained algebra is what we call a \defe{one dimensional extension of Heisenberg algebra}{extension!of Heisenberg algebra}. Let us study the possibilities for $\rho(A)$. From linearity, its general form is
\[
 \rho(A)(v,z)=\rho(A)(v,0)+\rho(A)(0,z)
    =(\BX v,\mu(v))+(zv_{0},2dz)
\]
with $\BX \in\End(V)$, $\mu\in V^*$, $v_{0}\in V$ and $d\in\eR$. Since $\eR E=[\pH_{n},\pH_{n}]$, the fact that $\rho(A)$ is a derivation of $\pH_{n}$, implies that $v_{0}=0$ because
\begin{equation}
   \rho(A)\eR E=\rho(A)[\pH_{n},\pH_{n}]
        =[\rho(A)\pH_{n},\pH_{n}]+[\pH_{n},\rho(A)\pH_{n}]\subset \eR E.
\end{equation}
Thus we have
\begin{equation}    \label{EqrhoBmudz}
\rho(A)(v,z)=(\BX v,\mu(v)+2dz).
\end{equation}
All that is summarized on the short exact sequence
\[
    \xymatrix{ 0\ar[r]&\pH_{n}\ar[r]^{i} & \mR \ar@<2pt>[r]^{r} & \mA \ar@<2pt>[l]^{i}\ar[r]&0 }
\]
where the two $i$ are injections and $r$ is the projection. Of course $r\circ i=\id$.
From commutation relations in $\pH_{n}$, we easily find
\[
  [(v,z),(v',z')]=[v,v']=\Omega(v,v')E.
\]
Applying $\rho(A)$ to this equality, and using the fact that this is a derivation, we find
\[
  \Omega(\BX ,v')E+\Omega(v,\BX ')E=\rho(A)\Omega(v,v')E=2d\Omega(v,v')E
\]
which can be rewritten as
\begin{equation}
\Omega\big( (\BX -d\,\mtu)v,v' \big)+\Omega\big( v,(\BX -d\,\mtu)v' \big)=0.
\end{equation}
In conclusion, the endomorphism $\rho(A)$ is given by a triple $(\BX ,\mu,2d)$ with $(\BX -d\,\mtu)\in\gsp(V,\Omega)$, $\mu\in V^*$ and $d\in\eR$. Using this result, we write the general commutator on $\mR=\mA\oplus_{\rho}\pH_{n}$ under the form
\begin{equation}  \label{EqGeneExtHeizCom}
\big[ (a,v,z),(a',v',z') \big]=\big( 0,\BX (av'-a'v),\mu(av'-a'v)+2d(az'-a'z)+\Omega(v,v') \big)
\end{equation}
where we adopted the notation
\begin{equation}
(a,v,z)=aA+v+zE.
\end{equation}

\subsection{Symplectic structure}
%---------------------------------

The following proposition gives a symplectic structure on $\mF$.

\begin{proposition}
The algebra $(\BX,\mu,d)$ endowed with
\begin{equation}
 \Omega^{\mF}=-\delta E^*=E^*([.,.])
 \end{equation}
where the star denotes the Chevalley cocycle defined by \eqref{EqDefChevCoycl} is symplectic if and only if $d\neq 0$.
\label{PropSymplestarEG}
\end{proposition}

\begin{proof}
It is evident that $\Omega^{\mF}$ is closed because it is exact. For non-degeneracy, we compute
\[
\begin{split}
\Omega^{\mF}&=E^*[.,.]=a\mu(v')-a'\mu(v)+2d(az'-a'z)+\Omega(v,v')\\
    &=\begin{pmatrix}
0   &   \mu &   2d\\
-\mu^{t}&   \Omega  &   0\\
-2d &   0   &   0
\end{pmatrix}
\end{split}
\]
whose determinant is $\det\Omega^{\mF}=-4d^{2}\det\Omega$ which is non vanishing if and only if $d\neq 0$.

\end{proof}

This symplectic algebra  is denoted by $\mF_{\Omega}(\BX ,\mu,d)$, or simply $\mF$ when there are no possible confusions.

Since we are only interested in symplectic algebras, we suppose $d\neq 0$ and we look at extensions of type $(d\BX ,d\mu,2d)$ with $\BX -d\mtu\in\gsp(V,\Omega)$. The bracket is given by
\begin{equation}   \label{EqCommGeneF}
\big[ (a,v,z),(a',v',z') \big]=\big( 0,d\BX (a'v-a'v),d\mu(av'-a'v)+2d(az'-a'z)+\Omega(v,v') \big).
\end{equation}

Then we consider $F$, the corresponding group and the left invariant symplectic form
\[
  \omega_{g}(X_{g},Y_{g})=\Omega(dL_{g^{-1}}X_{g},dL_{g^{-1}}Y_{g}).
\]
By construction, $L_{g}^*\omega=\omega$. One can prove that the Iwasawa coordinates
\begin{equation}
\begin{aligned}
 I\colon \mF&\to F \\
(a,n)&\mapsto  e^{aA} e^{n}
\end{aligned}
\end{equation}
is not a Darboux chart, but one can find a twist.

\begin{proposition}
The chart \begin{equation}
\begin{aligned}
 \tilde I\colon \mF&\to F \\
\tilde I(a,v,z)&=  e^{aA} e^{v} e^{\big(z+\frac{1}{ 4 }\Omega(v\BX )\big)E}
\end{aligned}
\end{equation}
is Darboux for the symplectic group $(F,\omega)$. In other words,
\[
  \tilde I^*\omega=\Omega.
\]
\end{proposition}
\begin{proof}
No proof.
\begin{probleme}
    The proof of this is in the Yannick's master thesis.
\label{ProbMemYan}
\end{probleme}

\end{proof}

This chart provides an action of $F$ on $\mF$ by
\[
  \tau_{g}(X)=(\tilde I^{-1}\circ L_{g}\circ \tilde I)(X)
\]
for $X\in\mF$, $g\in F$.\begin{proposition}
The action $\tau$ is symplectic and strongly hamiltonian. The momentum maps are
\begin{subequations}\label{EqAppMomHamActSPHm}
\begin{align}
  \lambda_{A}(X)&=d\mu(v)+2dz,\\
\lambda_{w}(X)&=\mu\left( \frac{  e^{-2da}- e^{da\BX } }{ d(2-\BX ) }w \right)-\Omega(v, e^{-da\BX }w),\\
\lambda_{E}(X)&= e^{-2da}.
\end{align}
\end{subequations}
\end{proposition}

\begin{proof}
The action $\tau$ is symplectic because $I^*\omega=\Omega$ and $L_{g}^*\omega=\omega$.
\end{proof}

\begin{proposition}
The Moyal product on $(\mF,\Omega^{\mF})$ is $\mF$-covariant.
\end{proposition}

\begin{proof}
Using the expansion \eqref{EqDevStatrMoyal} of the Moyal product and the explicit form \eqref{EqAppMomHamActSPHm} of the momentum maps, we remark that the higher order terms in $\lambda_{X}\ast_{M}\lambda_{Y}$ are vanishing.
\end{proof}

\subsection{Isomorphisms}  \label{SubsecIsomsdX}
%------------------------

The extension obtained by the derivation $D=(\BX ,\mu,d)$ is \emph{a priori} not the same as the one obtained by $D'=(\BX',\mu',d')$. Two extensions are isomorphic when there exists a linear bijection $dL\colon \mF_{D}\to \mF_{D'}$ such that\footnote{the reason why we write $dL$ instead of $L$ comes from the fact that we will be interested in the corresponding group isomorphism later.}
\begin{subequations}
\begin{align}
dL\big( [X,Y]_{D'} \big)&=\big[ dL(X),dL(Y) \big]_{D'}\\
(dL)^*\Omega^{D'}&=\Omega^{D}.
\end{align}
\end{subequations}
We find the following isomorphisms:
\begin{subequations}   \label{SubEqsIsommud}
\begin{itemize}
\item $\mF(d\BX ,d\mu,2d)\simeq \mF(\BX ,\mu,d)$ by
\begin{equation}
   dL(a,v,z)=(da,v,z),
\end{equation}
\item $\mF(\BX ,\mu,2)\simeq\mF(\BX ,0,2) $ by
\begin{equation}
   dL(a,v,z)=(a,v+au,z),
\end{equation}
where $u$ is the vector of $V$ satisfying $i(u)\Omega=\mu$,
\item $\mF( \BX ,0,2)\simeq\mF(\BX ',0,2)$ by
\begin{equation}
   dL(a,v,z)=(a,M(v),z)
\end{equation}
where $M\in\SP(V,\Omega)$ fulfills $M\BX M^{-1}=\BX'$ or, equivalently,
\[
M(\BX-\mtu)M^{-1}=\BX'-\mtu.
\]
\end{itemize}
\end{subequations}
The third isomorphism only gives the equivalence between $\BX-\mtu$ and $\BX'-\mtu$ when they belongs to the same orbit of the adjoint action of $\SP(V,\Omega)$. In particular, there are no isomorphisms between the identity and anything else.

 Let us prove the second isomorphism. If $D=(\BX ,\mu,2)$, $D'=(\BX ,0,2)$, $Y=(a,v,z)\in\mF$ and $Y'=(a',v',z')\in \mF$, we have
\begin{equation}   \label{EqPrIsoDDtilde}
\begin{split}
   dL\big(  [Y,Y']_{D}  \big)&=dL\big(   0,\BX (av'-a'v),\mu(av'-a'v)+2(az'-a'z)+\Omega(v,v')     \big)\\
        &=\big(   0,\BX (av'-a'v),\mu(av'-a'v) +2(az'-a'z)+\Omega(v,v')  \big),
\end{split}
\end{equation}
while
\[
  \big[ dL(Y),dL(Y')   \big]_{D'}=\Big(   0,\BX \big(  a(v'+a'u)-a'(v+au)   \big),2(az'+a'z)+\Omega(v+au,v'+a'u)    \Big),
\]
but
\[
   \Omega(v+au,v'+a'u)=\Omega(v,v')+\mu(av'-a'v),
\]
thus we find the same as in equations \eqref{EqPrIsoDDtilde}.

The theorem~\ref{ThoDefHeizsansB} together with the isomorphisms given in~\ref{SubsecIsomsdX} only provide a product on extensions of type $(d\mtu,0,2d)$. But we saw that the extensions $(\BX ,0,2d)$ with $\BX \neq\mtu$ are different. Hence the generalization of this result to other extensions is not straightforward. We address now this question.

\subsection{Extensions with non trivial \texorpdfstring{$\protect\BX $}{X}} \label{subsecTrick}
%-------------------------------------------------------------------------

The group $F(\mtu,0,2)$ is provided with a kernel $K\colon F\times F\times F\to \eC$ by theorem~\ref{ThoDefHeizsansB}.  The symplectic group $\SP(V,\Omega)$ acts on $F$ by
\begin{equation}
\begin{aligned}
 \Phi\colon \SP(V,\Omega)\times F&\to F \\
(M,I(a,v,z))&\mapsto \Phi_{M}(I(a,v,z)):=I(a,M(v),z)
\end{aligned}
\end{equation}
where
\begin{equation}
\begin{aligned}
 I\colon \mF&\to F \\
(a,n)&\mapsto  e^{aA} e^{n}
\end{aligned}
\end{equation}
is the Iwasawa coordinate on $F$.

\begin{proposition}
The kernel $K$ is invariant under this action: $\Phi^*_{M}K=K$.
 \label{PropkernelinvarSp}
\end{proposition}

\begin{proof}
We are looking on the kernel in expression \eqref{PRODUCT}. The amplitude of $K$, i.e. all what lies outside the exponential, and the cyclic sum in the phase only depend on the $a_{i}$'s. So $\Phi_{M}$ does not act on them. As far as $S_{0}$ is concerned, up to coefficients which only depend on the $a_{i}$'s, it is a sum of elements of the form $\Omega(Mv_{i},Mv_{j})=\Omega(v_{i},v_{j})$.
\end{proof}

Let $\BX $ be a matrix such that $\bar{\BX }=\BX -\mtu\in\gsp(V,\Omega)$ and $\mF'=\mF'(\BX,0,2)$. We consider $\mS$, the one dimensional subalgebra of $\gsp(V,\Omega)$ generated by $\bar{\BX}$ and we define
\begin{equation}
  \mG=\mS\oplus_{\rho}\mF
\end{equation}
with
\[
  \rho(\bar{\BX })(a,v,z)=(0,\bar{\BX }v,0).
\]
We denote by $G$ and $S$ the corresponding groups. We have in particular $F\simeq G/S$. An element of $\mG$ has the form
\begin{equation}    \label{EqDefkavzG}
  (k\bar\BX,a,v,z)=k\bar\BX+aA+v+zE.
\end{equation}


\begin{proposition}
The group $F'$ is a subgroup of $G$.
\end{proposition}
\begin{proof}
We will prove that $\mF'$ is isomorphic to a subalgebra of $\mG$, namely, the subalgebra $\mL\subset \mS\oplus_{\rho}\mF$,
\[
  \mL=\eR(A+\bar\BX)\oplus_{\sigma}(V+\eR E)
\]
where $\sigma$ is the splitting homomorphism \eqref{EqrhoBmudz} of $\mF$, which in the present case reads $\sigma(A+\bar\BX)(0,v,z)=(0,Xv,2z)$. In other words, the algebra $\mL$ is made of elements of the form \eqref{EqDefkavzG} with $k=a$.  The isomorphism is
\begin{equation}
\begin{aligned}
\phi  \colon \mL&\to \mF'(\BX,0,2) \\
a(A+\bar\BX)+v+zE&\mapsto aA+v+zE.
\end{aligned}
\end{equation}
Indeed, using formula \eqref{EqCommGeneF} with $d=1$ and $\mu=0$, we find
\[
\begin{split}
\Big[ \phi\big( a(A+\bar{\BX })+v+zE \big)&,\phi\big( a'(A+\bar{\BX })+v'+z'E \big) \Big]_{(\BX,0,2)}\\
        &=\big[ aA+v+zE,a'A,v'+z'E \big]\\
        &=\BX (av'-a'v)+\big(2(az'-a'z)+\Omega(v,v')\big)E\\
    &=\phi\big( X(av'-a'v),2(az'-a'z+\Omega(v,v'))  \big)\\
&=\phi\big[ a(A+\bar{\BX })+v+zE,a'(A+\bar{\BX })+v'+z'E \big].
\end{split}
\]
\end{proof}

From now on, we identify $\mF'$ with $\mL$ by the isomorphism $\phi$ which will no longer be explicitly written.  Image of $F'$ in $G$ by the isomorphism are elements of the form
\[
  g'= e^{a(A+\bar{\BX })} e^{v+zE}.
\]
Since the elements $ e^{A}$ and $ e^{\BX }$ commute in $G$, we can decompose an element $\phi^{-1}(g')$ as
\[
  \underbrace{e^{a\bar{\BX }}}_{\in S}\underbrace{e^{aA} e^{v+zE}}_{\in F}.
\]
The element $a(A+\bar{\BX })+v+zE$ seen in $\mS\oplus_{\rho}\mF$ will be denoted by $(a,v,z)$ as well
\[
  (a,v,z)=\phi^{-1}(aA+v+zE).
\]
We consider the following coordinate on $F'$:
\begin{equation}
\begin{aligned}
 J\colon \mF'&\to F' \\
(a,v,z)&\mapsto  e^{a(A+\bar{\BX })} e^{v+zE}.
\end{aligned}
\end{equation}

\begin{proposition}
The group $F'$ is diffeomorphic to the homogeneous space $F\simeq G/S$.
\end{proposition}

\begin{proof}
We will prove that $F'$ acts simply transitively on $G/S$. Let us look at
\begin{equation}        \label{EqgpJSF}
  g'=J(a,v,z)= \underbrace{e^{a(A+\bar{\BX })}}_{g'_{S}} \underbrace{e^{v+zE}}_{g'_{F}}.
\end{equation}
Noticing that $ e^{a\bar{\BX }}[e]=[ e^{a\bar{\BX }}]=[e]$ we find
\[
\begin{split}
g'[e]= e^{a\bar{\BX }} e^{aA} e^{v+zE} e^{-a\bar{\BX }} e^{a\bar{\BX }}[e]
        =\AD( e^{a\bar{\BX }})\big(  e^{aA} e^{v+zE} \big)[e].
\end{split}
\]
In $\mG=\mS\oplus_{\rho}\mF$, by definition of $\rho$, we successively have
\begin{subequations}
\begin{align}
\ad(a\bar{\BX })(a,v,z)&=(0,a\bar{\BX }v,0)\\
\Ad( e^{a\bar{\BX }})(a,v,z)&=(a, e^{e\bar{\BX }}v,z)\\
\AD( e^{a\bar{\BX }})\big(  e^{aA} e^{v+zE} \big)&= e^{aA} e^{ e^{a\bar{\BX }}v+zE},
\end{align}
\end{subequations}
thus $g'[e]= e^{aA} e^{ e^{a\bar{\BX }}v+zE}[e]=[I(a, e^{a\bar{\BX }}v,z)]$. So, in order to get the element $[I(a,v,z)]\in G/S$, we have to act on $[e]$ with the element $g'=J(a, e^{-a\bar{\BX }}v,z)$. All that proves that the map
\begin{equation}
\begin{aligned}
 H\colon F'&\to G/S \\
(a,v,z)&\mapsto \big[ I(a, e^{a\bar{\BX }}v,z) \big]
\end{aligned}
\end{equation}
is a diffeomorphism.
\end{proof}

The work done up to now provides a diffeomorphism
\begin{equation}
\begin{aligned}
 \varphi\colon F'&\to F \\
\varphi\big( J(a,v,z) \big)&=I(a, e^{a\bar{\BX }}v,z)
\end{aligned}
\end{equation}
which has suitable properties listed in the proposition below.

\begin{proposition}
This map $\varphi\colon F'\to F$ has the following properties:
\begin{enumerate}
\item if $g'=J(a,v,z)=g'_{S}g'_{F}$ in the sense of decomposition \eqref{EqgpJSF},
\begin{equation}
\varphi\circ L_{g'}=\AD(g'_{S})\circ L_{g'_{F}}\circ\varphi=\Phi_{ e^{a\bar{\BX }}}\circ L_{g'_{F}}\circ\varphi=\Phi_{g'_{S}}\circ L_{g'_{F}}\circ\varphi,
\end{equation}
\item the differential fulfils
\begin{equation}
d(\varphi\circ J)_{(0,0,0)}=dI_{(0,0,0)},
\end{equation}
\item if $\omega$ is the left invariant symplectic form on $F$ and $\omega'$ the one on $F'$, we have
\[
  \varphi^*\omega=\omega',
\]
in other words, $\varphi$ is a symplectomorphism.

\end{enumerate}

\end{proposition}
\begin{proof}

The first point is a computation:
\[
\begin{split}
\varphi\big( L_{g'_Sg'_F}(g_sg_E) \big)&=\varphi\big( g'_Sg_S\AD(g_S^{-1})(g'_F)g_F \big)\\
        &=\AD(g'_Sg_S)\big( \AD(g_S^{-1})(g'_F)g_F \big)\\
        &=\AD(g'_S)\big( g'_F\AD(g_S)(g_F) \big)\\
        &=\big( \AD(g'_S)\circ L_{g'_F} \big)\big( \varphi(g_Sg_F) \big).
\end{split}
\]
When $g'=g'_Sg'_F=J(a,v,z)$, we have $g'_S=\exp(a\BX)$ and $g'_F=I(a,v,z)$, so the result is given by
\[
  \AD(g'_S)(g'_F)= e^{ \Ad(a\BX) }I(a,v,z)=I(a, e^{a\BX}v,z)=\Phi_{ e^{a\BX}}I(a,v,z).
\]
That concludes the proof of the first point.  For the second statement, we have $(\varphi\circ J)(a,v,z)=\Phi_{ e^{a\BX}}I(a,v,z)$, so
\begin{equation}
\begin{split}
    d(\varphi \circ J)_{(0,0,0)}(Y_a,Y_v,Y_z)&=\Dsdd{  \Phi_{ e^{tY_a}}I(tY_a,tY_v,tY_z)   }{t}{0}\\
        &=\Dsdd{ I\big( tY_a, e^{tY_a\BX}tY_v,tY_z \big) }{t}{0}\\
        &=dI_{(0,0,0)}(Y_a,Y_v,Y_z).
\end{split}
\end{equation}
For the third point, we denote by $e$ and $e'$ the neutral of $F$ and $F'$. On the one hand,
\[
  (\varphi^*\omega)_{g'}=\omega_{\varphi(g')}\circ d\varphi_{g'}=\omega_e\circ d\big( L_{\varphi(g')^{-1}}\circ\varphi \big)_{g'};
\]
on the other hand, $\omega'_{g'}=\omega'_{e'}\circ d\big(L_{(g')^{-1}})_{g'}$. Hence, in order to have $\varphi^*\omega=\omega'$, it is necessary that
\[
  \omega'_{e'}\circ dJ_{(0,0,0)}=\omega_e\circ d\big(  L_{\varphi(g')^{-1}}\circ\varphi\circ L_{g'}  \big)_{e'}\circ dJ_{(0,0,0)}.
\]
But, for $g'=g'_Sg'_F$, we have
\[
\begin{split}
L_{\varphi(g')^{-1}}\circ\varphi\circ L_{g'}(g)&=\varphi(g')^{-1}\varphi(g'g)\\
        &=\varphi(g')^{-1}\AD(g'_S)\big( g'_F\varphi(g) \big)\\
        &=\AD\big( (g'_F)^{-1}\big) \AD(g'_S)\big( g'_F\varphi(g) \big)\\
        &=\big( \AD(g'_S)\circ\varphi \big)(g).
\end{split}
\]
The first property yields
\[
  d\big( L_{\varphi(g')^{-1}}\circ\varphi\circ J \big)_{(0,0,0)}=\Ad(g'_S)\circ dI_{(0,0,0)}=d(\Phi_{ e^{a\BX}})_{e}\circ dI_{(0,0,0)}.
\]
Since $\omega_e$ is invariant under $\Phi_{ e^{a\BX}}$, it remains to be proved that $\omega'_{e'}\circ dJ_{(0,0,0)}=\omega_e\circ dI_{(0,0,0)}$. This is true because, in these coordinates, both sides applied on vectors $(Y_a,Y_v,Y_z)$ and $(Z_a,Z_v,Z_z)$ give
\[
  2(Y_aZ_z-Z_aY_z)+\Omega(Y_v,Z_v),
\]
so $\varphi$ is a symplectomorphism.

\end{proof}

Now, if $K$ is the kernel on $F$, we define the kernel on $F'$ by
\begin{equation}\label{EqKerRprime}
\begin{aligned}
 K'\colon F'\times F'\times F'&\to \eC\\
K'&=\varphi^*K.
\end{aligned}
\end{equation}


\begin{theorem}
The kernel $K'$ is\index{kernel!for extension $(\BX ,0,2)$ of Heisenberg}
\begin{itemize}
\item left invariant under $F'$,
\item associative on $F'$.
\end{itemize}
 \label{ThoDefoHeizAvecB}
\end{theorem}
\begin{proof}
For left invariance, let $g'=J(a,v,z)$. We have
\[
  L_{g'}^*K'=\big( \varphi\circ L_{J(a,v,z)} \big)^*K=\big( \Phi_{ e^{a\BX}}\circ L_{g'_F}\circ\varphi \big)^*K=\varphi^*L_{g'_F}^*\Phi_{ e^{a\BX}}^*K=K',
\]
because of left invariance of $K$ under $F$ and its invariance under $\Phi$. Associativity can be checked in much the same way as in lemma~\ref{LemKerINvarIsom}.
\end{proof}

\subsection{Get the \texorpdfstring{$\mu$}{u} back}  \label{SubSecRemetreMu}
%--------------------------------------------------

Equation \eqref{EqKerRprime} provides a kernel on $F'$, the group generated by the algebra
\[
  \mF'=\eR A\oplus_{(\BX ,0,2)}\pH_{n}.
\]
Now we try to find a kernel for the group corresponding to the algebra
\[
  \mF_{1}=\eR A\oplus_{(d\BX ,d\mu,2d)}\pH_{n}.
\]
The algebra isomorphism is know from equations \eqref{SubEqsIsommud}, it is
\begin{equation}
\begin{aligned}
 dL\colon \eR A\oplus_{(d\BX ,d\mu,2d)}\pH_{n}&\to \eR A\oplus_{(\BX ,0,2)}\pH_{n} \\
dL(aA+v+zE)&=daA+(v+au)+zE
\end{aligned}
\end{equation}
where $u$ is characterised by
\[
  i(u)\Omega=\mu.
\]
The problem is to find the group isomorphism $L$. Since $L$ must be an isomorphism, we have
\[
L( e^{aA} e^{v+zE})=L( e^{aA})L( e^{v+zE})
        = e^{daA+au} e^{v+zE}
\]
that we want to write under the form $ e^{a'A} e^{v'+z'E}$. So we write
\[
   e^{daA+au}= e^{daA}\underbrace{e^{-daA} e^{daA+au}}_{\text{to be traeated}}.
\]

Let $F=F(\mtu,0,2)$ and $F'=F'(\BX,0,2)$. By proposition~\ref{PropkernelinvarSp},  the kernel $K$ on $F$ is invariant under $\SP(V,\Omega)$, i.e. $\Phi^*_{M}K=K$ for all $M\in\SP(V,\Omega)$. The action of $\SP(V,\Omega)$ on $F$ is given by
\[
  \Phi_{M}\big( I(a,v,z) \big)=I(a,Mv,z).
\]
Define the map $\Phi'_M\colon F'\to F'$,
\begin{equation}   \label{EqDefPhiprimeM}
\Phi_{M}'\big( J(a,v,z) \big)=J\big( a, e^{-a\bar{\BX }}M e^{a\bar{\BX }}v,z \big)
\end{equation}
which fulfils
\[
  \Phi_{M}\circ\varphi=\varphi\circ\Phi_{M}'.
\]
Thus, using the $\SP(V,\Omega)$-invariance of $K$, we have
\[
\phi_{M}'{}^*K'=(\varphi\circ\phi_{M}')^*K=(\phi_{M}\circ\phi)^*K=\varphi^*K=K'.
\]
This proves that $K'$ is also invariant under $\SP(V,\Omega)$ too.

\subsection{Jump from one kernel to another}
%-------------------------------------------

We know the deformation of $\SU(1,n)$ described in section~\ref{SecDefSURme}.

We have a kernel for the extensions $F_{\delta E^*}(d\mtu,0,2d)$ and $F_{\delta E^*}(\BX ,0,2)$. We can consider the isomorphism $L\colon F(\BX ,0,2)\to F(d\BX ,d\mu,2d)$ which is the lift of
\begin{equation}
\begin{aligned}
 dL\colon \mF(\BX ,0,2)&\to \mF(d\BX ,d\mu,2d) \\
  (a,v,z)&\mapsto (da,v+au,z).
\end{aligned}
\end{equation}
If $K'$ is a kernel on $F_{\delta E^*}(d\BX ,d\mu,2d)$, then
\[
  K_{0}=L^*K'
\]
is a kernel on $F_{\delta E^*}(\BX ,0,2)$.

An action $\Phi_{0}(M)\colon F(d\BX ,d\mu,2d)\to F(d\BX ,d\mu,2d)$ is given by
\begin{equation}
  \Phi_{0}(M)=L^{-1}\circ\Phi'(M)\circ L
\end{equation}
where $\Phi'(M)\colon F(\BX ,0,2)\to F(\BX ,0,2)$ is given by equation \eqref{EqDefPhiprimeM}. By lemma~\ref{LemKerINvarIsom}, the kernel $K_{0}$ is left invariant under the action of $F$ and invariant under the following action of $\SP(V,\Omega)$:
\[
  \Phi_{0}(M)^*K_{0}=K_{0}.
\]

\begin{lemma} \label{LemJumpCoadOrb}
Let $\delta\eta^*$ and $\delta\xi^*$ be two exact forms on $\mF$ such that $\xi^*$ and $\eta^*$ belong to the same coadjoint orbit\index{coadjoint!orbit}: there exists a $g\in F$ such that
\begin{equation}
\xi^*\circ\Ad(g)=\eta^*.
\end{equation}
A solution of the problem to find an automorphism $\sigma\colon F\to F$ such that
\begin{equation} \label{EqHypLemSigmaAD}
\delta\eta^*_{\sigma(h)}\big( d\sigma_{h}X_{h},d\sigma_{h}Y_{h} \big)=\delta\xi^*_{h}(X_{h},Y_{h})
\end{equation}
for all $h\in F$ and $X_{h}$, $Y_{h}\in T_{h}F$ is given by $\sigma=\AD(g^{-1})$.
\end{lemma}

\begin{proof}
Transported to the identity, the condition \eqref{EqHypLemSigmaAD} becomes:
\[
\begin{split}
\delta\eta^*\big( dL_{\sigma(h)^{-1}}d\sigma_{h}X_{h},&dL_{\sigma(h)^{-1}}d\sigma_{h}Y_{h}  \big)\stackrel{!}{=}\delta\xi^*\big( dL_{h^{-1}}X_{h},dL_{h^{-1}}Y_{h} \big)\\
        &=\delta\eta^*\big( \Ad(g^{-1})dL_{h^{-1}}X_{h},\Ad(g^{-1})dL_{h^{-1}}Y_{h}  \big).
  \end{split}
\]
 If $X_{h}=\dsdd{ X_{h}(t)}{t}{0}$, we are searching for a $\sigma$ such that
\[
  \Dsdd{ \sigma(h)^{-1}\sigma\big( X_{h}(t) \big) }{t}{0}=\Dsdd{ \AD(g^{-1})\big( h^{-1}X_{h}(t) \big) }{t}{0}.
\]
Since $\sigma$ is a group isomorphism, $\sigma(h)^{-1}=\sigma(h^{-1})$ and the constraint on $\sigma$ becomes
\[
  \sigma\big(h^{-1}X_{h}(t)\big)=g^{-1}\big( h^{-1}X_{h}(t) \big)g.
\]
A solution is therefore
\begin{equation}
\sigma=\AD(g^{-1}).
\end{equation}

\end{proof}


\chapter{Deformation of anti de Sitter spaces}   \label{ChDefoBH}
\input{deformation_AdS4}

\chapter{Two notes for further developments}        \label{ChapNoteDev}
\input{axplusb}
\section{Deformation of \texorpdfstring{$\SOdn$}{SO2n}}   \label{SecUnifSOdn}
%+++++++++++++++++++++++++++++++++++++++++++++++++++++++++++++++++

\subsection{Applying the extension lemma with old Iwasawa}
%----------------------------------------

One purpose of this section is to prove that
\[
  \SO(2,n)\stackrel{\sA\oplus\sN}{=}SU(1,1)\oplus SU(1,n-1).
\]
For that purpose, we will decompose the Iwasawa algebra of $\so(2,n)$ as a (symplectic) direct sum and we will compare the root spaces decomposition of each of the two parts with the ones of $\gsu(1,1)$ and $\gsu(1,n)$. Hence prescription on $\mathfrak{s}_1$ is to be two dimensional and to contains one and only one element of $\sA$. This condition will impose a change of variable in the ``new'' Iwasawa.

The first point is to decompose the Lie algebra $\sR:=\mA\oplus\mN$ (from the Iwasawa decomposition of
 $\sodn$) into two parts $\mfs_1$ and $\mfs_2$ such that, as Lie algebras,
\[
  \sR=\mfs_1\oplus_{\rho}\mfs_2
\]
where $\rho$ is the adjoint action in $\so(2,n)$. Recall that
\[
   \mA\oplus\mN=\{H_1,H_2,M,N,V_i,W_j\}.
\]
From symplectic considerations which will appear further, we want $\mfs_1$ and $\mfs_2$ to be even dimensional.
One can easily remark that
\decompss{H_1,N}{H_2,M,V_i,W_j}{}
works. What one has to check is that $\mfs_1$ and $\mfs_2$ are closed under $\ad$: $[s,t]\in\mfs_i$ if $s$, $t\in\mfs_i$, and that $\mfs_1$ acts on $\mfs_2$: $[A,s]\in\mfs_2$ if $A\in\mfs_1$ and $s\in\mfs_2$. This is a rather strong constraint.

Let us now explore systematically the possibilities. In a first time, we will not pay attention to the symplectic part. The question is to explicitly find all the possibilities of $\mfs_i$ such that $\mA\oplus\mN$ is semi-direct product of $\mfs_1$ and $\mfs_2$.

As notational convention, when one write $E=\{W_i\}$, we mean that \emph{all} the $W_i$ are in the set $E$. If we want to say that \emph{one particular} $W_a$ is in $E$, we use the indices $a,b,c\ldots$.

If $H_1\in\mfs_2$, one can only finds $H_2$ and $M$ in $\mfs_1$ because in
\decompss{W_i,N,V_i,\ldots}{H_1,\ldots}{,}
the Lie algebra $\mfs_1$ doesn't acts on $\mfs_2$. So with $H_1\in\mfs_2$, the only possibility is
\decompss{H_2,M}{H_1,V_i,W_i,N}{.}
But $\mfs_2$ is not closed for $\ad$. First conclusion: $H_1\in\mfs_1$.

Let us consider the case $H_2\in\mfs_2$. In this case, one can only put $H_1$ and $N$ in $\mfs_1$: $H_2\in\mfs_2$ implies
\decompss{H_1,N}{H_2,M,V_i,W_j}{.}
From now, the question becomes ``who can belong to $\mfs_1$ in the same time as $H_1$?``

The first step is to show that $M$ can't. Let us consider $H_1$, $M\in\mfs_1$. Since $[V_i,W_j]=\delta_{ij}M$, in order for $\mfs_2$ to be closed for $\ad$, one has to put some $V_a$ and (or) $W_b$ in $\mfs_1$. If $W_a\in\mfs_1$, $V_a$ must also belongs to $\mfs_1$ because $\mfs_1$ must acts on $\mfs_2$. For the same reason, $V_a\in\mfs_A$ implies $W_a\in\mfs_1$. Thus, $M\in\mfs_1$ imply at least
 \[
    \{H_1,M,V_i,W_j\}\subset \mfs_1.
 \]
Now, it is also clear that $N\in\mfs_2$ is not possible because of the action: $[W_i,N]=-2V_i$. We are left with
\decompss{H_1,M,N,V_i,W_j}{H_2}{,}
but we want even dimensional spaces. Conclusion: $H_1$, $M\in\mfs_1$ is not possible.

Our second point is to show that $H_1,V_a\in\mfs_1$ is also not possible. For, let us consider that one actually has $H_1,V_a\in\mfs_1$, ant let us explore the consequences. It is clear that $W_a$ and $N$ can't be in $\mfs_2$ in the same time. If $W_a\in\mfs_1$, $M$ must also be in $\mfs_1$ in order to close under $\ad$. But we just see that it was impossible. On the other hand, if $N$ belongs to $\mfs_1$, since $[N,W_a]=2V_a\in\mfs_1$, $W_a$ can't belongs to $\mfs_2$. Then we are left with the precedent case: $W_a\in\mfs_1$. We conclude that $H_1,V_a\in\mfs_1$ is not possible.
From now, one sees that the only possibility with $H_1$, $N\in\mfs_1$ is
\decompss{H_1,N}{H_2,M,V_i,W_j}{.}

The two last cases to explore are $H_2$ or (and)  $W_a$ in $\mfs_1$ with $H_1$. If $W_a\in\mfs_1$, then $H_2\in\mfs_1$ because of the action of $\mfs_1$ on $\mfs_2$ and $[W_a,H_2]=-W_a\in\mfs_1$. On the other hand we had yet seen that $V_i\in\mfs_2$. So we are left with
\decompss{H_1,H_2,W_a,W_b}{M,N,W_{\neq a,b},V_i}{,}
and more generally, one can take in $\mfs_1$ any even number of $W_i$.

One checks that the latest possibility (which is a special case of the previous) works:
\decompss{H_1,H_2}{M,N,W_i,V_i}{.}

The decomposition of $\mA\oplus\mN$ as semi-direct product of $\mfs_1$ and $\mfs_2$ are:
\decompss{H_1,H_2,\underbrace{W_a,\ldots,W_b}_{\text{even or zero}}}{M,N,W_{\ldots},V_i}{,}
and
\decompss{H_1,N}{H_2,M,V_i,W_j}{.}


\subsubsection*{The symplectic conditions}
%////////////////////////////////////////

Let us now turn our attention to the symplectic matter. There are two symplectic constraint on the choice of the $\mfs_i$. The first one is that each one must be a \defe{symplectic Lie algebra}{symplectic!Lie algebra}: if $\Omega_i$ is the symplectic $2$-form on $\mfs_i$, then for any $x$, $y$, $z\in\mfs_i$,
\begin{equation}\label{eq:symple_Lie}
\Omega_i([x,y],z)+\Omega_i([y,z],x)+\Omega_i([z,x],y)=0.
\end{equation}

The second one is the fact that the action of $\mfs_1$ on $\mfs_2$ must be symplectic in the sense that
$\forall X\in\mfs_1,\, \ad X\in\mfsp (\Omega_2)$. So one has to check that
\begin{equation}
   \Omega_2\big(  \Ad_s A,\Ad_s B   \big)=\Omega_2(A,B)
\end{equation}
for any $s\in S_1$ and $A$, $B\in\mfs_2$. This fact was crucially used in the proof of proposition~\ref{prop:Darboux}.

We first check that
\decompss{H_1,H_2,W_a,\ldots,W_b}{M,N,W_{\ldots},V_i}{}
doesn't works.
Indeed, one must have
\[
   \Oexp{H_1}{M}{N}=\Omega_2(M,N),
\]
but $[H_1,M]=0$, $[H_1,N]=2N$ and $\exp(\ad H_1)=id+[H_1,.]+\ldots$, then
\[
  \Oexp{H_1}{M}{N}=\Omega_2(M,e^2N),
\]
so that $\Omega_2(M,N)=0$.
Note that this is more general: $\Omega_2(M,s)=0$ for all $s\in\mfs_2$ such that $[H_1,s]=\alpha s$ because
\[
   \Omega_2(M,s)\stackrel{!}{=}\Omega_2(e^{\ad H_1}M,e^{\ad H_1}s)=\Omega_2(M,e^{\alpha}s).
\]
This imposes
\begin{equation}
\begin{split}
   \Omega_2(M,V_i)&=0\\
   \Omega_2(M,N)&=0\\
   \Omega_2(M,W_i)&=0.
\end{split}
\end{equation}
Thus $\Omega_2$ is degenerate. Now, we check that the second works:
\decompss{H_1,N}{H_2,M,V_i,W_j}{.}
The condition \eqref{eq:symple_Lie} gives
\begin{equation}
\begin{split}
   \Omega_2(V,M)&=0\\
   \Omega_2(W,M)&=0\\
   2\Omega_2(V,W)+\Omega_2(M,H_2)&=0.
\end{split}
\end{equation}
When $s\in\mfs_2$ is such that $[H_1,s]=\alpha s$, $\Omega_2(H_2,s)=0$. Now the matrix of $\Omega_2$ looks like
\begin{equation}
\Omega_2=\left(
\begin{array}{c|c|c|c|c}
 & H_2 & M & V_i & W_j \\
 \hline
H_2 & 0 &  & 0 & 0 \\
\hline
M &  & 0 & 0 & 0 \\
\hline
V_i & 0 & 0 & 0 &  \\
\hline
W_j & 0 & 0 &  & 0
\end{array}
\right)
\end{equation}
We have to check that the last entries keep free.
\begin{equation} \begin{split}
   \Omega_2(M,H_2)&\stackrel{!}{=}\Oexp{H_1}{M}{H_2}\\
                  &\stackrel{!}{=}\Oexp{N}{M}{H_2}.
\end{split}
\end{equation}
Since $[H_1,M]=[H_1,H_2]=[N,M]=[N,H_2]=0$, these two conditions are fulfilled. On the other hand,
\begin{equation}
\begin{split}
   \Omega_2(V_a,W_b)&\stackrel{!}{=}\Oexp{H_1}{V_a}{W_b}\\
                    &=\Omega_2(eV_A,e^{-1} W_b)\\
		    &=\Omega_2(V_a,W_b)\\
		    &\stackrel{!}{=}\Oexp{N}{V_a}{W_b}\\
		    &=\Omega_2(V_a,W_b+2V_b).
\end{split}
\end{equation}
Thus $\Omega_2(V_a,W_b)$ has no constraints and $\Omega_2(V_a,V_b)=0$. Finally, it is easy to see that
\begin{equation}
\begin{split}
\Omega_2(W_a,W_b)&\stackrel{!}{=}\Oexp{H_1}{W_a}{W_b}\\
                 &=e^{-2}\Omega_2(W_a,W_b),
\end{split}
\end{equation}
so that $\Omega_2(W_a,W_b)=0$. Now, one can write the possible form for $\Omega_2$ as
\begin{equation}
\Omega_2=\left(
\begin{array}{c|c|c|c|c|c|c}
 & H_2 & M & V_1 & V_2 & W_1 & W_2 \\
 \hline
H_2 & 0 & a & 0 & 0 & 0 & 0 \\
\hline
M & -a & 0 & 0 & 0 & 0 & 0 \\
\hline
V_1 & 0 & 0 & 0 & 0 & a/2 & a/2 \\
\hline
V_2 & 0 & 0 & 0 & 0 & a/2 & a/2 \\
\hline
W_1 & 0 & 0 & -a/2 & -a/2 & 0 & 0  \\
\hline
W_2 & 0 & 0 & -a/2 & -a/2 & 0 & 0  \\
\end{array}
\right)
\end{equation}



\subsection{Decomposition as split extension}
%------------------------------------------

When we try to decompose $\sA\oplus\sN$ as symplectic direct sum with a strict respect to chosen basis matrices, there are only two possibilities:
\begin{equation}
 \begin{split}
  \mfs_1&=\{ J_1,J_2,\overbrace{V_a,\ldots,V_b}^{\text{even}}  \}\\
\mfs_2&=\{ L,M,W_i,V_{\ldots}  \}
\end{split}
\end{equation}
and
\begin{equation}
 \begin{split}
  \mfs_1&=\{ J_2,\overbrace{V_a,\ldots,V_b}^{\text{odd}}  \}\\
\mfs_2&=\{ J_1,W_j,M,L,V_{\ldots}  \}  .
\end{split}
\end{equation}
In the first possibility, $\mathfrak{s}_{1}$ is not symplectic (because it is abelian); while the second one only works in low dimensional cases: there must not be any $V_a$. This fact leads us to consider  the change of basis \eqref{EqChmHJ} in $\sA$: $H_1=J_1-J_2$ and $H_2=J_1+J_2$.

If $H_1\in\mathfrak{s}_2$, then $L,V_i,W_i\in\mathfrak{s}_2$ because $\mathfrak{s}_1$ must act on $\mathfrak{s}_2$. Hence $M\in\mathfrak{s}_2$ and $H_2$ remains alone in $\mathfrak{s}_1$. That proves that $H_1\in\mathfrak{s}_1$. If we suppose that $H_2\in\mathfrak{s}_2$, we find
\begin{equation}  \label{eq_HLss}
 \begin{split}
  \mfs_1&=\{ H_1,L  \}\\
\mfs_2&=\{ H_2,V_i,W_j,M  \}.
\end{split}
\end{equation}
The case $H_1,H_2\in\mathfrak{s}_1$ leads to
\begin{equation}   \label{Eq_HHVaMLW}
 \begin{split}
	\mfs_1&=\{ H_1,H_2,\overbrace{V_a,\ldots V_b}^{\text{even}}  \}\\
	\mfs_2&=\{ M,L,W_i,V_{\text{others}}  \}.
\end{split}
\end{equation}
The symplectic condition excludes the second decomposition. Indeed for each $s$ such that $[H_1,s]=\alpha s$ (i.e. $s=V_i,W_j,L$), we have
\[
  \Omega_2\big(  e^{ad H_1}M, e^{\ad H_1}s \big)= e^{\alpha}\Omega_2(M,s)\stackrel{!}{=}\Omega_2(M,s).
\]
Hence $\Omega_2(M,s)=0$. This proves that the decomposition \eqref{Eq_HHVaMLW} imposes the symplectic form $\Omega_2$ to be degenerate. We are left with decomposition \eqref{eq_HLss}.

Root space decomposition of $SU(1,n)$ can be found on pages 314--315 of \cite{Knapp}: it has $\dim\sA=1$, $\dim\sG_{2f}=1$ and $\dim\sG_f=2(n-1)$. In $\mathfrak{s}_2$, we have $V_i\in\sG_1$, $W_j\in\sG_1$, $M\in\sG_2$, and when we look at $AdS_l=\SO(2,l-1)/\SO(1,l-1)$, we have $l-3$ matrices $V_i$ and $W_j$. Therefore $\mathfrak{s}_2$ is nothing else than the $\sA\oplus\sN$ of $\mathfrak{su}(1,l-2)$ (recall $l\geq3$). The analysis shows that $\mathfrak{s}_1$ is the $\sA\oplus\sN$ of $\mathfrak{su}(1,1)$.

\subsection{Conclusion and perspectives}
%----------------------------------------

For our $AdS_l$ black hole, the algebra of the group which defines the singularity is the split extension
\[
  (\sA\oplus\sN)_{\so(2,l-1)}=(\sA\oplus\sN)_{\mathfrak{su}(1,1)}\oplus_{\ad}(\sA\oplus\sN)_{\mathfrak{su}(1,l-2)}.
\]
A deformation of the corresponding groups is given in \cite{Biel-Massar}. The extension lemma \ref{EXT} yields an oscillatory integral universal deformation formula for proper actions of the Iwasawa subgroup of $\SO(2,l-1)$. That remark provides an alternative way to deform the black hole to the one presented in section~\ref{SecGpStructOuvertOrb}.

The availability of a quantization of $AdS_l$ by action of $AN$ is an opportunity to embed our black hole toy model in the framework of noncommutative geometry. Indeed, the quantization of $AdS_l$ is the data of the anti de Sitter manifold and the action of the group $AN$; that is precisely the data which defines the black hole of chapter~\ref{ChapBHinAdS}. So we would be able to ``see'' the causal issue from the data of the deformed spectral triple. Remark that a causal structure (in the physical meaning of the term) is a special property of \emph{pseudo}-Riemannian manifolds for which spectral geometry does not exist yet.

An important remaining problem with that method is the fact that the extension lemma does not assure the existence of a stable functional space for the new product. So there is still a lot of analytic work to be done.


\chapter{Gravitation and noncommutative geometry}
\input{NCgrav}

\chapter{Levy Processes and such}
Source: \cite{UweLevy}.

%+++++++++++++++++++++++++++++++++++++++++++++++++++++++++++++++++++++++++++++++++++++++++++++++++++++++++++++++++++++++++++
\section{Lévy process}
%+++++++++++++++++++++++++++++++++++++++++++++++++++++++++++++++++++++++++++++++++++++++++++++++++++++++++++++++++++++++++++

Let $G$ be a semigroup and $(\Omega,\mF,\eP)$ a probability space. Here $\mF$ is a $\sigma$-algebra and $\eP$ is a probability measure on $\Omega$.
\begin{definition}
    A \defe{Lévy process}{Lévy!process} is, for each $s$ and $t$ such that $0\leq s\leq t<\infty$, a map $X_{st}\colon (\Omega,\mF,\eP)\to (G,\mB)$ where $\mB$ is the Borel $\sigma$-algebra on $G$ such that
    \begin{enumerate}
        \item
            $\eP(X_{ss}=e)=1$,
        \item
            $X_{ss}=\lim_{t\searrow s}X_{st}$
        \item
            $X_{st}X_{tu}=X_{su}$,
        \item
            if $s_1\leq t_1\leq s_2\leq t_2\leq \ldots\leq t_n$, then the random variables $X_{s_1t_1}$, $X_{s_2t_2}$, \ldots, $X_{s_nt_n}$ are independent.
        \item
            $X_{s+h,t+h}=X_{st}$ in the sense that
            \begin{equation}
                \eP_{X_{st}}=\eP_{X_{s+h,t+h}}
            \end{equation}
            where $\eP_X(A)=\eP\big( X^{-1}(A) \big)$.
    \end{enumerate}
    If $G$ is a group, we add the condition $X_{st}=X_s^{-1}X_t$.
\end{definition}

The map $X\colon \Omega\to G$ gives rise to the map
\begin{equation}
    \begin{aligned}
        j_X\colon L^{\infty}(G,B)&\to L^{\infty}(\Omega,\mF,\eP) \\
        j_X(f)&=f\circ X
    \end{aligned}
\end{equation}
Now we can look at the properties of a Lévy process on $j$ instead of $X$. Firstly, $j_X$ is a $*$-algebra homomorphism such that $j_X(1)=1$.

%+++++++++++++++++++++++++++++++++++++++++++++++++++++++++++++++++++++++++++++++++++++++++++++++++++++++++++++++++++++++++++
\section{Quantum probability space and stochastic processes}
%+++++++++++++++++++++++++++++++++++++++++++++++++++++++++++++++++++++++++++++++++++++++++++++++++++++++++++++++++++++++++++

\begin{definition}
    A \defe{quantum probability space}{quantum!probability space} is a pair $(A,\Phi)$ is an unital $*$-algebra $A$ and a state $\Phi$ on $A$. A \defe{quantum random variable}{quantum!random variable} $j$ over a quantum probability space $(A,\Phi)$ on a $*$-algebra $B$ is an homomorphism $j\colon B\to A$ of $*$-algebras.
\end{definition}

\begin{definition}
A \defe{quantum stochastic process}{quantum!stochastic process} is a family of quantum random variables $(j_j)_{t\in I}$. Its \defe{marginal distribution}{marginal!distribution}\index{distribution!marginal} is the set of maps
\begin{equation}
    \begin{aligned}
        \varphi_j\colon B&\to \eC \\
        \varphi_j&=\Phi\circ j_t.
    \end{aligned}
\end{equation}
\end{definition}

\begin{definition}
An \defe{operator process}{operator!process}\index{process!operator} if a family of elements $(X_t)_{t\in I}$ of a quantum probability space ($X_t\in A$).
\end{definition}

From an operator process and a choice of $b\in B$, we can define a stochastic process on $\eC\langle b,b^*\rangle$ by defining
\begin{equation}
    j_t(b)=X_t
\end{equation}
and extend it to $\eC\langle b,b^*\rangle$ as $*$-homomorphism.

On the other hand, if $\{ j_t\colon B\to A \}_{t\in I}$ is a stochastic process, we define an operator process by choosing $x\in B$ and defining
\begin{equation}
    X_t=j_t(x).
\end{equation}

\begin{definition}
    We say that maps $f_i\colon B\to A$ are \defe{tensor, or boson independent}{independent!with respect to a state} with respect to $\Phi$ if
    \begin{enumerate}
        \item
            $\Phi\big( f_1(b_1)\cdots f_n(b_n) \big)=\Phi\big( f_1(b_1) \big)\ldots\Phi\big( f_n(b_n) \big)$ for every $b_i\in B$,
        \item
            the images commute: $[f_k(b_k),f_l(b_l)]=0$ for every $k\neq l$.
    \end{enumerate}
\end{definition}

If $j_1,j_2\colon B\to A$ are maps from $B$ to an algebra $A$, we define the \defe{convolution}{convolution} by
\begin{equation}
    j_1*j_2=m_A\circ(j_1\otimes j_2)\circ\Delta.
\end{equation}

Notice that, if $j_1$ and $j_2$ are homomorphisms, it is not guaranteed in general that the convolution $j_1*j_2$ is an homomorphism. In the case of independent variables, however, it is true: if $j_1$ and $j_2$ are independent random variables, then $j_2*j_2$ is still a random variable because
\begin{equation}
    \begin{aligned}[]
        (j_1*j_2)(ab)&=m_A\circ(j_1\otimes j_2)\big( a_{(1)}b_{(1)}\otimes a_{(2)}b_{(2)} \big)\\
        &=j_1(a_{(1)})j_1(b_{(1)})j_2(a_{(2)})j_2(b_{(2)})\\
        &=j_1(a_{(1)})j_2(a_{(2)})j_1(b_{(1)})j_2(b_{(2)})\\
        &=(j_1*j_2)(a)(j_1*j_2)(b).
    \end{aligned}
\end{equation}
where $a_{(1)}b_{(1)}$ stands for $(ab)_{(1)}$.

%+++++++++++++++++++++++++++++++++++++++++++++++++++++++++++++++++++++++++++++++++++++++++++++++++++++++++++++++++++++++++++
\section{Lévy process}
%+++++++++++++++++++++++++++++++++++++++++++++++++++++++++++++++++++++++++++++++++++++++++++++++++++++++++++++++++++++++++++

\begin{definition}
    Let $(B,\Delta,\epsilon)$ be a bialgebra. A \defe{Lévy process}{Lévy!process} on $(B,\Delta)$ over the quantum probability space $(A,\Phi)$ is a quantum stochastic process $(j_{st})_{0\leq s\leq t}$ which satisfies the following requirements
    \begin{description}
        \item[Increment property] We have
            \begin{subequations}
                \begin{align}
                    j_{rs}*j_{st}&=j_{rt}       &\text{for all }0\leq r\leq s\leq t\\
                    j_{tt}(b)&=\epsilon(b)1_A   &\text{for all }0\leq t,b\in B
                \end{align}
            \end{subequations}

        \item[Independence of increments]
            for every $n\in\eN$ and $0\leq s_1\leq t_1\leq s_2\leq\ldots\leq s_n\leq t_n$, the maps $j_{s_1t_1}$, \ldots $j_{s_nt_n}$ are independent with respect to $\Phi$.
        \item[Stationarity of the increments]
            the distribution $\varphi_{st}=j_{st}\circ j_{st}$ depends only on the difference $t-s$, in other words we have $\varphi_{s+h,t+h}=\varphi_{st}$ for every $0\leq s\leq t$ and $h\geq 0$.
        \item[Weak continuity]
            $\lim_{t\searrow s}j_{st}(a)=j_{ss}(a)$ for every $a\in A$.
    \end{description}
\end{definition}

One say that the process $j_{st}\colon B\to (A,\Phi)$ and $k_{st}\colon B\to (A',\Phi')$ are \defe{equivalent}{equivalence!of Lévy process} if
\begin{equation}
    \Phi\big( j_{s_1t_1}(b_1)\ldots j_{s_nt_n}(b_n) \big)=\Phi'\big( k_{s_1t_1}(b_1)\ldots k_{s_nt_n}(b_n) \big),
\end{equation}
that is if all the expectation values that we could compute are equal.

If $(j_{st})$ is a Lévy process, we define
\begin{equation}
    \varphi_{t-s}=\Phi\circ j_{st}
\end{equation}
for $0\leq s\leq t$. This is well defined from the stationarity of increments. We can as well write $\varphi_t=\Phi\circ j_{0t}$.

%+++++++++++++++++++++++++++++++++++++++++++++++++++++++++++++++++++++++++++++++++++++++++++++++++++++++++++++++++++++++++++
\section{Schürmann triple}
%+++++++++++++++++++++++++++++++++++++++++++++++++++++++++++++++++++++++++++++++++++++++++++++++++++++++++++++++++++++++++++

\begin{proposition}
    Let $B$ be an involutive coalgebra and $(\varphi_t)_{t\geq 0}$, a convolution semigroup of linear functionals on $B$. If we define
    \begin{equation}
        L=\lim_{t\searrow 0} \frac{1}{ t }(\varphi_t-\epsilon),
    \end{equation}
    then the following are equivalents:
    \begin{enumerate}
        \item
            the maps $\varphi_{t}$ are states on $B$;
        \item
            the map $L\colon B\to \eC$ satisfies $L(1_B)=0$ and is hermitian and conditionally positive.
    \end{enumerate}
\end{proposition}

\begin{lemma}
    If we define $\varphi_t=\varphi_{0t}=\Phi\circ j_{0t}$, we get a convolution semigroup.
\end{lemma}

\begin{proof}
    The first condition comes from
    \begin{equation}
        \varphi_0(b)=\Phi\big( j_{00}(b) \big)=\Phi\big( \epsilon(b)1_{A} \big)=\epsilon(b)\Phi(1_{A})=\epsilon(b)
    \end{equation}
    since $\Phi(1_{A})=1$ because $\Phi$ is a state on $A$.

    Using the fact that $j_{0,s+t}=j_{0s}*j_{s,s+t}$, we have
    \begin{equation}        \label{EqvpsptaUn}
        \varphi_{s+t}(a)=\varphi_{0,s+t}(a)=\Phi\big( (j_{0s}*j_{s,s+t})(a) \big).
    \end{equation}
    But,
    \begin{equation}
        (j_{0s}*j_{s,s+t})(a)=m\circ(j_{0s}\otimes j_{s,s+t})(a_{(1)}\otimes a_{(2)})=j_{0s}(a_{(1)})j_{s,s+t}(a_{(2)}).
    \end{equation}
    Thus, since the intervals $0,s$ and $s,s+t$ are independent, the equation \eqref{EqvpsptaUn} becomes
    \begin{equation}
        \begin{aligned}[]
            \varphi_{s+t}(a)&=\Phi\big( j_{0s}(a_{(1)})j_{s,s+t}(a_{(2)}) \big)\\
            &=\Phi\big( j_{0s}(a_{(1)}) \big)\Phi\big( j_{s,s+t}(a_{2}) \big)\\
            &=\varphi_s(a_{(1)})\varphi_t(a_{(2)})\\
            &=(\varphi_s\otimes\varphi_t)\Delta a\\
            &=(\varphi_s*\varphi_t)(a).
        \end{aligned}
    \end{equation}
    It remains to be justified that $\Phi\big( j_{s,s+t}(a) \big)=\varphi_t(a)$. This comes from the fact that $\varphi_{s+h,t+h}=\varphi_{st}$, so that $\varphi_{s,s+t}=\varphi_{0+s,s+t}=\varphi_{0,t}$.
\end{proof}

A linear functional $\omega$ on a bialgebra $B$ is \defe{conditionally positive}{positive!conditionally}\index{conditionally positive} if $\omega(a^*a)\geq 0$ for every $a$ such that $\epsilon(a)=0$. The functional is \defe{Hermitian}{Hermitian!functional} if $\omega(a^*)=\overline{ \omega(a) }$.

In our case, we know that the functionals $\varphi_t$ are in fact states. We define
\begin{equation}
    L=\lim_{t\searrow 0} \frac{1}{ t }(\varphi_t-\epsilon).
\end{equation}
This is conditionally positive because
\begin{equation}
    L(a^*a)=\lim_{t\searrow 0} \frac{1}{ t }\big( \varphi_t(a^*a)-\epsilon(a^*a) \big)=\lim_{t\searrow 0} \frac{ \varphi_t(a^*a) }{ t }\geq 0
\end{equation}
because $\varphi_t$ is a state and thus positive.

\begin{definition}
    Let $B$ be an unital $*$-algebra with an unital hermitian character $\epsilon\colon B\to \eC$. A \defe{Schürmann triple}{Schürmann triple} on $(B,\epsilon)$ is a triple $(\rho,\eta,L)$ with
    \begin{enumerate}
        \item
            $\rho\colon B\to \aL(D)$ is a $*$-representation of $B$ on a pre-Hilbert space $D$.
        \item
            The map $\eta\colon B\to D$ is linear and
            \begin{equation}
                \eta(ab)=\rho(a)\eta(b)+\eta(a)\epsilon(b),
            \end{equation}
            such a map is called a $\rho$-$\epsilon$-1-\defe{cocycle}{cocycle!of a Schürmann triple}.
        \item
            The map $L\colon B\to \eC$ is an hermitian linear functional whose $\epsilon-\epsilon-2-$coboundary is the map
            \begin{equation}
                (a,b)\mapsto-\langle \eta(a^*), \eta(b)\rangle ,
            \end{equation}
            that means that
            \begin{equation}        \label{EqCondFiffAssetaL}
                -\langle \eta(a^*), \eta(b)\rangle =(\partial L)(a,b)=\epsilon(a)L(b)-L(ab)+L(a)\epsilon(b)
            \end{equation}
            for every $a,b\in B$.
    \end{enumerate}
\end{definition}

\begin{corollary}       \label{ItemPropCorSchr}
    A Schürmann triple has the following immediate properties.
    \begin{enumerate}
        \item
            $\eta(1)=0$.
        \item
            $L(1)=0$.
        \item       \label{ItemPropCorSchriii}
            The condition \eqref{EqCondFiffAssetaL} is equivalent to ask
            \begin{equation}        \label{EqConsSimplAssetaL}
                L(ab)=-\langle \eta(a^*), \eta(b)\rangle
            \end{equation}
            for every $a,b\in K_1$.
    \end{enumerate}

\end{corollary}

\begin{proof}
    \begin{enumerate}
        \item
            Using the cocycle property and the fact that $\rho(1)$ is the identity,
            \begin{equation}
                \eta(1\cdot 1)=\rho(1)\eta(1)+\eta(1)\epsilon(1)=2\eta(1).
            \end{equation}
        \item
            Now, writing the compatibility relation \eqref{EqCondFiffAssetaL} with $a=b=1$ and taking into account $\eta(1)=0$ we have
            \begin{equation}
                0=\epsilon(1)L(1)-L(1\cdot 1)+L(1)\epsilon(1)=L(1).
            \end{equation}
        \item
            If $a$ and $b$ belong to $K_1$, the relation \eqref{EqCondFiffAssetaL} reduces to \eqref{EqConsSimplAssetaL}. Now suppose that the condition \eqref{EqConsSimplAssetaL} holds for every $a$ and $b$ in $K_1$. Taking any $a,b\in B$ we consider $a-\epsilon(a)1$ and $b-\epsilon(b)1$ that are elements of $K_1$. Taking into account the fact that $L(1)=0$, we have
            \begin{equation}
                L(ab)-\epsilon(b)L(a)-\epsilon(a)L(b)=\langle \eta(a^*), \eta(b)\rangle .
            \end{equation}
    \end{enumerate}
\end{proof}


Let $(\pi,\eta,L)$ be a Schürmann triple on the compact quantum group $\mA_q$ (see section~\ref{SecGeneratorsonSUQn}). The map $\eta\colon \mA_q\to \oL(D) $ satisfies the cocycle condition
\begin{equation}
    \eta(ab)=\pi(a)\eta(b)+\eta(a)\epsilon(b).
\end{equation}

\begin{proposition}     \label{PropCocycleDeteretavjnmu}
    The cocycle $\eta$ is determined by its values on the elements $v_j^*$ with $j=1,\ldots,n-1$.
\end{proposition}

\begin{proof}
    Since $\epsilon(1)=1$, we have
    \begin{equation}
        \eta(1)=\pi(1)\eta(1)+\eta(1)\epsilon(1)=\big( \pi(1)+1 \big)\eta(1).
    \end{equation}
    Since the representation is nondegenerate we have $\pi(1)=\id$. The only possibility is thus $\eta(1)=0$.

    If $a$ and $b$ belong to $K_1$ we have $\eta(ab)=\pi(a)\eta(b)$. Since each element in $\mA_q$ is a polynomial in $u_{ij}$ and $u_{ij}^*$, we only have to fix the value of $\eta$ on these elements.

    First, $\eta(u_{jj}^*)=\eta(v_j^*)$ because $\eta(1)=0$. Using lemma~\ref{lestmunijdiff}, taking the adjoint,
    \begin{equation}
        q\eta(u_{ll}^*u_{ij}^*)=\eta(u_{ij}^*u_{ll}^*).
    \end{equation}
    Using the cocycle property and the fact that $\epsilon(u_{ll}^*)=1$,
    \begin{equation}
        \big( q\pi(u_{ll}^*)-\id \big)\eta(u_{ij}^*)=\pi(u_{ij}^*)\eta(u_{ll}^*).
    \end{equation}
    The operator $q\pi(u_{li}^*)-\id$ is invertible because of lemma~\ref{Propqpiideinve} and $q<1$. Thus, when $i\neq j$ and $i,j\leq n-1$,
    \begin{equation}
        \eta(u_{ij}^*)=\big( q\pi(u_{li})^*-\id \big)^{-1}\pi(u_{ij}^*)\eta(u_{ll}^*).
    \end{equation}

    Let us now compute $\eta(v_j)$ for $j=1,\ldots,n-1$. Taking the relation \eqref{Equustrunsuqn},
    \begin{equation}
        \begin{aligned}[]
            0=\eta(1)&=\sum_{p=1}^n\eta(u_{jp}u_{jp}^*)\\
            &=\eta(u_{jj}u_{jj}^*)+\sum_{p\neq j}\eta(u_{jp}u_{jp}^*)\\
            &=\pi(u_{jj})\eta(u_{jj}^*)+\eta(u_{jj})\underbrace{\epsilon(u_{jj}^*)}_{=1}+\sum_{p\neq j}\eta(u_{jp}u_{jp}^*),
        \end{aligned}
    \end{equation}
    so that
    \begin{equation}        \label{Eqetaujjpiu}
        \eta(v_j)=\eta(u_{jj})=-\pi(u_{jj})\eta(u_{jj}^*)-\sum_p\pi(u_{jj})\eta(u_{jp}^*).
    \end{equation}
    Thus the vectors $\eta(u_{jj})$ and $\eta(v_j)$ are fixed for $j\leq n-1$.

    We are going to compute $\eta(u_{ij})$ with $i\neq j$ and $i,j\leq n-1$ using the lemma~\ref{lestmaxjdiff}. We pose $k=\max(i,j)$. Since $\epsilon(u_{kk})=1$ and $\epsilon(u_{ij})=0$, the relation $\eta(u_{ij}u_{kk})=q\eta(u_{kk}u_{ij})$ leads to
    \begin{equation}
        \pi(u_{ij})\eta(u_{kk})=q\pi(u_{kk})\eta(u_{ij}).
    \end{equation}
    What we obtain is
    \begin{equation}
        \eta(u_{ij})=-\big( \id-q\pi(u_{kk}) \big)^{-1}\pi(u_{ij})\eta(u_{kk}).
    \end{equation}

    In order to compute the value of $\eta(v_n^*)$, we start from the definition \eqref{EqDefInvolutionSSUqn} of the involution. Taking into account the fact that in our case $D=1$,
    \begin{equation}
        u_{nn}^*=D^{nn}=\sum_{\sigma\colon \{ 1\ldots n-1 \}\to \{ 1,\ldots,n-1 \}}(-1)^{| \sigma |}u_{1\sigma(1)}\ldots u_{n-1,\sigma(n-1)}.
    \end{equation}
    Taking $\eta$ and using the cocycle property we have
    \begin{equation}
        \begin{aligned}[]
            \eta(v_n^*)&=\eta(u_{nn}^*)\\
            &=\sum_{\sigma(n-1)\neq n-1}(-1)^{| \sigma |}\pi\big( u_{1\sigma(1)}\ldots u_{n-2,\sigma(n-1)} \big)\eta(u_{n-1,\sigma(n-1)})\\
            &\quad+\sum_{\sigma(n-1)=n-1}(-1)^{| \sigma |}\pi\big( u_{1\sigma(1)}\ldots u_{n-2,\sigma(n-2)} \big)\eta(u_{n-1,n-1})\\
            &\quad+\sum_{\sigma(n-1)=n-1}(-1)^{| \sigma |}\eta\big( u_{1\sigma(1)}\ldots u_{n-2,\sigma(n-2)} \big)\underbrace{\epsilon(u_{n-1,n-1})}_{=1}
        \end{aligned}
    \end{equation}
    Applying many times the cocycle the term $\eta\big( u_{1\sigma(1)}\ldots u_{n-2,\sigma(n-2)} \big)$ decomposes into basic elements containing $\eta(u_{i\sigma(i)})$. This fixes $\eta(v_n^*)$.

    Taking the equation \eqref{Eqetaujjpiu} with $j=n$ we have
    \begin{equation}
        \eta(v_n)=\eta(u_{nn})=-\pi(u_{nn})\eta(u_{nn}^*)-\sum_{p=1}^n\pi(u_{nn})\eta(u_{np}^*).
    \end{equation}
    That fixes $\eta(v_n)$. Now we consider the previously proved equation
    \begin{equation}
        \eta(u_{ij})=\big( q\pi(u_{kk})-\id \big)^{-1}\pi(u_{ij})\eta(v_k)
    \end{equation}
    with $i\neq j$ and $n=\max(i,j)$. Then
    \begin{equation}
        \eta(u_{ij})=\big( q\pi(u_{nn})-\id \big)^{-1}\pi(u_{ij})\eta(v_n).
    \end{equation}
    This fixes $\eta(u_{in})$ and $\eta(u_{nj})$.

\end{proof}


%+++++++++++++++++++++++++++++++++++++++++++++++++++++++++++++++++++++++++++++++++++++++++++++++++++++++++++++++++++++++++++
\section{A representation}
%+++++++++++++++++++++++++++++++++++++++++++++++++++++++++++++++++++++++++++++++++++++++++++++++++++++++++++++++++++++++++++

\begin{proposition}     \label{PropReprezThetasuqn}
    The following is a one dimensional representation of $\SU_q(n)$ if $\sum_k\theta_k=1$:
    \begin{equation}
        \pi(u_{jk})= e^{i\theta_j}\delta_{jk}
    \end{equation}
\end{proposition}

\begin{proof}
    First we have
    \begin{equation}
        \pi(D)=\sum_{\sigma\in S(n)}(-q)^{| \sigma |} e^{i\theta_1}\delta_{1\sigma(1)}\ldots e^{i\theta_n}\delta_{n\sigma(n)}= e^{i(\theta_1+\cdots+\theta_n)}=1.
    \end{equation}
    Equations \eqref{SUBEquuijcondiv} all reduce to $0=0$ because of the Kronecker delta's and the condition on the indices.
\end{proof}

Since the condition $D=1$ imposes the sum of the $\theta$'s to be zero, we will write the representation $\pi$ by
\begin{equation}
    \epsilon_{\theta_1\ldots\theta_{n-1}}(u_{ij})= e^{i\theta_j}\delta_{jk}.
\end{equation}
Notice that the counit $\epsilon$ is $\epsilon_{0,\ldots,0}$. We define
\begin{equation}
    \epsilon'_k=\left.\frac{ \partial  }{ \partial \theta_k }\epsilon_{\theta_1\ldots\theta_{n-1}}\right|_{\theta_k=0}.
\end{equation}

As an example, in $\SU_q(3)$ we have
\begin{equation}
    \epsilon_{\theta_1\theta_2}=
    \begin{pmatrix}
        e^{i\theta_1}    &   0    &   0    \\
        0    &    e^{i\theta_2}    &   0    \\
        0    &   0    &    e^{-i(\theta_1+\theta_2)}
    \end{pmatrix}
\end{equation}
in the sense that $\epsilon_{\theta_1\theta_2}(u_{ik})=\big( \epsilon_{\theta_1\theta_2} \big)_{ik}$. Thus we have
\begin{equation}
    \begin{aligned}[]
        \epsilon'_{\theta_1}(u_{ik})&=\Dsdd{ \epsilon_{\theta_1\theta_2}(u_{ik}) }{\theta_1}{0}\\
        &=\left.\frac{ d }{ d\theta_1 }
        \begin{pmatrix}
            e^{i\theta_1}    &   0    &   0    \\
            0    &    e^{i\theta_2}    &   0    \\
            0    &   0    &    e^{-i(\theta_1+\theta_2)}
        \end{pmatrix}_{ik}\right|_{\theta_1=0}\\
        &=\begin{pmatrix}
            i    &   0    &   0    \\
            0    &   0    &   0    \\
            0    &   0    &   -i
        \end{pmatrix}.
    \end{aligned}
\end{equation}
Higher order derivative are defined the same way:
\begin{equation}
    \epsilon''_{kl}=\left.\frac{ \partial  }{ \partial \theta_k }\frac{ \partial  }{ \partial \theta_l }\epsilon_{\theta_1\ldots\theta_{n-1}}\right|_{\substack{\theta_k=0\\\theta_l=0}}.
\end{equation}

%---------------------------------------------------------------------------------------------------------------------------
\subsection{Ideals}
%---------------------------------------------------------------------------------------------------------------------------

Let $\mA_q$ be the Hopf $C^*$-subalgebra of $\SU_q(n)$ generated by $\{ 1,u_{ij} \}$, and a Schürmann triple $(\pi,\eta,L)$ where $\pi$ is a representation of $\mA_q$ on $\opB(\hH)$. We define the ideal
\begin{equation}
    K_1=\ker(\epsilon)\subset\mA_q.
\end{equation}
The ideal $K_1$ is generated by the elements of the form $u_{ij}$ ($i\neq j$) and $u_{ii}-1$. Then we define
\begin{equation}
    K_2=\Span\{ ab\tq a,b\in K_1 \}
\end{equation}
and more generally\nomenclature[Q]{$K_m$}{Ideal in $\SU_q(n)$}
\begin{equation}
    K_n=\Span\{ a_1\ldots a_n\tq a_i\in K_1 \}.
\end{equation}
We also define
\begin{equation}
    K_{\infty}=\bigcap_{m\geq 1}K_m.
\end{equation}


Each of these $K_m$ is a two-sided ideal since
\begin{equation}
    ba_1\ldots a_m=(ba_1)a_2\ldots a_m
\end{equation}
belongs to $K_m$ if $a_1\ldots a_m$ belongs to $K_m$.

We define
\begin{equation}
    \begin{aligned}[]
        v_j&=u_{jj}-1\\
        d_j&=\frac{ v_j-v_j^* }{2i}
    \end{aligned}
\end{equation}
Since $\epsilon(1)=1$, the unit does not belong to $K_1$, but the elements $u_{ij}$, $u_{ij}^*$, $v_j$ and $d_j$ belong to $K_1$ when $i\neq j$.

\begin{lemma}       \label{Lemuijvkkuij}
    If $k=\max(i,j)$ we have
    \begin{equation}
        u_{ij}v_k-qv_ku_{ij}=(q-1)u_{ij}.
    \end{equation}

\end{lemma}

\begin{proof}
    This is a computation:
    \begin{equation}
        \begin{aligned}[]
            u_{ij}v_k&=u_{ij}u_{kk}-u_{ij}\\
            &=qu_{kk}u_{ij}-u_{ij}      &\text{because }k=\max(i,j)\\
            &=(qu_{kk}-1)u_{ij}\\
            &=(qu_{kk}-q)u_{ij}+(q-1)u_{ij}\\
            &=qv_ku_{ij}+(q-1)u_{ij}.
        \end{aligned}
    \end{equation}
    The proof is completed.
\end{proof}

\begin{lemma}       \label{Lesvjvjvuuuujjiiv}
    We have
    \begin{equation}
        v_jv_j^*=u_{jj}u_{jj}^*-v_j-v_j^*-1.
    \end{equation}

\end{lemma}

\begin{proof}
    This is a computation:
    \begin{equation}
        \begin{aligned}[]
            v_jv_j^*&=(u_{jj}-1)(u_{jj}^*-1)\\
            &=u_{jj}u_{jj}^*-u_{jj}-u_{jj}^*+1\\
            &=u_{jj}u_{jj}^^**-v_j-u_{jj}^*+1-1\\
            &=u_{jj}u_{jj}^*-v_j-v_j^*-1.
        \end{aligned}
    \end{equation}

\end{proof}


\begin{proposition}\label{PropuudansKKiii}
    We have
    \begin{enumerate}
        \item
            $u_{ij}\in K_{\infty}$ if $i\neq j$;
        \item       \label{ItemuudansKKii}
            $1-u_{jj}u_{jj}^*\in K_{\infty}$;
        \item\label{ItemuudansKKiii}
            $u_{ii}u_{jj}-u_{jj}u_{ii}\in K_{\infty}$;
        \item\label{ItemuudansKKiv}
            $u_{ii}u_{jj}^*-u_{jj}^*u_{ii}\in K_{\infty}$;
        \item
            $v_j+v_j^*\in K_2$;
        \item
            $\sum_k v_k\in K_2$;
        \item       \label{ItemPropuudansKKiiiavii}
            $\sum_{k=1}^nd_k\in K_2$.
    \end{enumerate}

\end{proposition}

\begin{proof}
    \begin{enumerate}
        \item
            Since $\epsilon(u_{ij})=\delta_{ij}$, we have $u_{ij}\in K_1$ when $i\neq j$

            Now, using lemma~\ref{Lemuijvkkuij} we have
            \begin{equation}
                u_{ij}=(q-1)^{-1}(u_{ij}v_k-qv_ku_{ij}).
            \end{equation}
        \item
            Since $\epsilon(u_{jj})=1$ we have $1-u_{jj}u_{jj}^*\in K_1$. Using formula \eqref{Equustrunsuqn}, we have
            \begin{equation}
                \begin{aligned}[]
                    1-u_{jj}u_{jj}^*&=\delta_{jj}-u_{jj}u_{jj}^*\\
                    &=\sum_ku_{jk}u_{jk}^*-u_{jj}u_{jj}^*\\
                    &=\sum_{k\neq j}u_{jk}u_{jk}^*.
                    \end{aligned}
            \end{equation}
            Since each of the terms in that sum belong to $K_{\infty}$, we have the result.
        \item
            The combination $u_{ii}u_{jj}-u_{jj}u_{ii}$ belongs to $\ker(\epsilon)=K_1$. Let us suppose $i<j$ (if not, consider change the sign). Using the relation \eqref{subEquuijcondiv}, we have
            \begin{equation}
                u_{ii}u_{jj}-u_{jj}u_{ii}=(q-q^{-1})u_{uj}u_{ji}\in K_{\infty}.
            \end{equation}
        \item
            Let's begin with $i=j$. We have
            \begin{equation}
                u_{ii}u_{ii}^*-u_{ii}^*u_{ii}=u_{ii}u_{ii}^*-1+1-u_{ii}^*u_{ii}\in K_{\infty}
            \end{equation}
            where we used the point~\ref{ItemuudansKKii}.

            If $i\neq j$, we use the relation \eqref{eqREflsuusikl} which says that $u_{ii}u_{jj}^*=u_{jj}^*u_{ii}$, so that the combination we are looking at is zero.
        \item
            Lemma~\ref{Lesvjvjvuuuujjiiv} shows that
            \begin{equation}        \label{Eqvvuiiujjvv}
                v_j+v_j^*=u_{jj}u_{jj}^*-1-v_jv_j^*
            \end{equation}
            The fact that $v_j\in K_1$ and item~\ref{ItemuudansKKii} show that the right hand side belong to $K_2$.
        \item
            Let us decompose the sum defining the determinant:
            \begin{equation}
                1=\sum_{\sigma\in S_n}(-q)^{| \sigma |}u_{1\sigma(1)}\ldots u_{n\sigma(n)}=u_{11}\ldots u_{nn}+\sum_{\sigma\neq \id}(-q)^{| \sigma |}u_{1\sigma(1)}\ldots u_{n\sigma(n)}.
            \end{equation}
            The last sum belongs to $K_{\infty}$ since there is at least one $k$ with $k\neq \sigma(k)$. Thus, replacing $u_{jj}$ by $v_j+1$,  we have
            \begin{equation}        \label{Eqinteunvuvuuvnldots}
                1=(v_1+1)\ldots (v_n+1)+k_{\infty}.
            \end{equation}
            The product can be written under the form
            \begin{equation}
                (v_1+1)\ldots(v_n+1)=\sum_{p=1}^n\sum_{i_1<\ldots<i_p}v_{i_1}\ldots v_{i_p}+1.
            \end{equation}
            In the latter sum, a part of the term $p=1$, each term belongs to $K_2$, so equation \eqref{Eqinteunvuvuuvnldots} reads
            \begin{equation}
                1=\sum_{i_1}v_{i_1}+k_2+k_{\infty},
            \end{equation}
            and we conclude that $\sum_iv_i=k_2+k_{\infty}\in K_2$.
        \item
            We know that $v_1^*+\cdots +v_n^*\in K_2$, so that
            \begin{equation}
                d_1+\cdots+ d_n=\frac{1}{ 2i }\big( (v_1+\ldots v_n)-(v_1^*+\cdots +v_n^*) \big)\in K_2.
            \end{equation}
    \end{enumerate}

\end{proof}

\begin{proposition}
    We have
    \begin{equation}
        \mA_q/K_{\infty}\simeq C(\eT^{n-1})
    \end{equation}

\end{proposition}

\begin{proof}
    Since $u_{ij}$ belongs to $K_{\infty}$ when $i\neq j$, the algebra $\mA_q/K_{\infty}$ is generated by the elements $u_{jj}$ and $u_{jj}^*$ as well as $1$. That algebra is commutative from points~\ref{ItemuudansKKiii} and~\ref{ItemuudansKKiv} of proposition~\ref{PropuudansKKiii}.

    We consider the following map\footnote{Here $\Delta(\mA_q/K_{\infty})$ stands for the structure space of $\mA_q/K_{\infty}$, see definition~\ref{DefStructureSpaceDel}.}:
    \begin{equation}
        \begin{aligned}
            \varphi\colon \eT^{n-1}&\to \Delta(\mA_q/K_{\infty}) \\
            t_1,\ldots t_{n-1}&\mapsto \omega_{t_1,\ldots t_{n-1}}
        \end{aligned}
    \end{equation}
    where $\omega_t\colon \mA_q/K_{\infty}\to \eC$ ($t\in\eT^{n-1}$) is defined by
    \begin{equation}
        \omega_t(u_{kk})=\begin{cases}
            e^{it_k}    &   \text{if }k\neq n\\
            e^{-i(t_1+\cdots+t_{n-1})}  &    \text{if }k=n
        \end{cases}
    \end{equation}
    and $\omega_t(1)=1$. Notice that the value of $\omega_t$ on $u_{nn}$ is imposed by the fact that $\omega_t(1-u_{11}\ldots u_{nn})=0$ and the fact that $\omega_t$ has to be multiplicative.

    The map $\varphi$ is injective because the value of $\varphi(t_1,\ldots,t_{n-1})$ on the element $u_{kk}$ fixes the values of $t_k$.

    It is also surjective because $1-u_{jj}u_{jj}^*$ belongs to $K_{\infty}$; thus each $\omega$ in $\Delta(\mA_q/K_{\infty})$ satisfies $\omega(u_{kk}u_{kk}^*)=1$ and
    \begin{equation}
        \omega(u_{kk})\omega(u_{kk})^*=1,
    \end{equation}
    so that there exists a unique $t_k\in\mathopen[ 0 , 2\pi [$ such that $\omega(u_{kk})= e^{it_k}$.

    Now we conclude the proof using the Gelfand theorem~\ref{thoGelfand} which states that for every $C^*$-algebra $\cA$, we have $\cA\simeq C\big( \Delta(\cA) \big)$. Indeed, we just proved that
    \begin{equation}
        \eT^{n-1}\simeq\Delta(\mA_q/K_{\infty}).
    \end{equation}
    Thus we have
    \begin{equation}
        C(\eT^{n-1})\simeq C\big( \Delta(\mA_q/K_{\infty}) \big)\simeq\mA_q/K_{\infty}.
    \end{equation}
\end{proof}

%---------------------------------------------------------------------------------------------------------------------------
\subsection{Decomposition}
%---------------------------------------------------------------------------------------------------------------------------

\begin{proposition}
    Every element $x\in\cA_q$ decomposes into
    \begin{equation}
        x=c_01+\sum_{j=1}^{n-1}c_jd_j+\sum_{1\leq i\leq j\leq n}x_{ij}v_iv_j^*+k_3
    \end{equation}
    for some $c_0,c_j,c_{ij}\in \eC$ and $k_3\in K_3$.
\end{proposition}

\begin{proof}
    First, $x$ is a complex polynomial in the variables $u_{ij}$ and $u_{ij}^*$. If we replace $u_{jj}$ by $v_j+1$, we have a polynomial
    \begin{equation}
        x=p(v_j,v_j^*,u_{ij},u_{ij}^*)
    \end{equation}
    with $i\neq j$. We know that $u_{ij}\in K_{\infty}$ (proposition~\ref{PropuudansKKiii}), so that, modulo elements of $K_3$, we can neglect the terms with $u_{ij}$ ($i\neq j$). It remains a polynomial in the variables $v_j$ and $v_j^*$. The terms of order $m$ belong to $K_m$, so that we only have to consider the first two terms. What we have is then a polynomial of degree $2$ in $v_j$ and $v_j^*$:
    \begin{equation}        \label{EqxSumvvdgammaUn}
        x=c_01+\sum_{j=1}^n(\alpha_jv_j+\beta_jv_j^*)+\sum_{i,j=1}^n\sum_{\epsilon_1,\epsilon_2\in\{ 1,* \}}\gamma_{ij}^{\epsilon_1,\epsilon_2}v_i^{\epsilon_1}v_j^{\epsilon_2}+k_3.
    \end{equation}
    Here the sum over $\epsilon_i$ means that we have the terms $v_iv_j$, $v_iv_j^*$, $v_i^*v_j$ and $v_i^*v_j^*$.

    The first degree terms are of the form
    \begin{equation}
        \alpha_j v_j+\beta_jv^*_j=\tilde c(\alpha,\beta)\frac{ v_j+v_j^* }{ 2 }+c(\alpha,\beta)\frac{ v_j-v_j^* }{ 2i }.
    \end{equation}
    Since
    \begin{equation}
        d_n=d_1+\cdots+d_{n-1}+k_2,
    \end{equation}
    up to changing the coefficients of the $d_k$'s ($j\leq n-1$) and in the higher order terms, we can reduce the sum to $\sum_{j=1}^{n-1}d_j$.


    So let's say that
    \begin{equation}
        \sum_{j=1}^n c_jd_j\in K_1/K_2.
    \end{equation}
    From element~\ref{ItemPropuudansKKiiiavii} of proposition~\ref{PropuudansKKiii}, we know that there exists an element $k_2\in K_2$ such that
    \begin{equation}
        d_n=k_2-d_1-\ldots -d_{n-1},
    \end{equation}
    so that in $K_1/K_2$ we have $d_n=-\sum_{j=1}^{n-1}d_j$ and a basis of $K_1/K_2$ is
    \begin{equation}
        B_1=\{ d_1,\ldots,d_{j-1} \}.
    \end{equation}
    \begin{probleme}
        Ok, this is a generating part for $K_1/K_2$. Why is it free? In what sense?
    \end{probleme}

    The decomposition \eqref{EqxSumvvdgammaUn} reads now
    \begin{equation}        \label{EqxSumvvdgammaDeux}
        x=c_01+\sum_{j=1}^{n-1}c_jd_j+\sum_{j=1}^n\tilde c_j(v_j+v_j^*)+\sum_{i,j=1}^n\sum_{\epsilon_1,\epsilon_2}\gamma_{ij}^{\epsilon_1,\epsilon_2}v_i^{\epsilon_1}v_j^{\epsilon_2}+k_3.
    \end{equation}
    Using relation \eqref{Eqvvuiiujjvv},
    \begin{equation}
        v_j+v_j^*=v_jv_j^*+\underbrace{1-u_{jj}u_{jj}^*}_{\in K_{\infty}},
    \end{equation}
    We can replace $\sum_{j=1}^n\tilde c_j(c_j+v_j^*)$ by $\sum_{j=1}^n\tilde c_jv_jv_j^*$.

    Let us prove that modulo elements in $K_3$, the elements
    \begin{equation}
        \begin{aligned}[]
            v_iv_j,&&v_i^*v_j,&&v_i^*v_j^*
        \end{aligned}
    \end{equation}
    can be written as combinations of $v_iv_j^*$. Since $v_i+v_i^*\in K_2$, we have
    \begin{equation}
        (v_i+v_i^*)v_j\in K_3,
    \end{equation}
    so that $v_iv_j=v_i^*v_j+k_3$. Making the same with $(v_i+v_i^*)v_j^*$ and $v_i(v_j+v_j^*)$, we have
    \begin{equation}
        \sum_{\epsilon_1,\epsilon_2}\gamma_{ij}^{\epsilon_1,\epsilon_2}v_i^{\epsilon_1}v_j^{\epsilon_2}=\sum_{i,j=1}^nc_{ij}v_iv_j^*+k_3.
    \end{equation}
    We can continue the simplification because $v_iv_j^*=v_j^*v_i$ in $K_2/K_3$. Indeed
    \begin{equation}
        v_iv_j^*-v_j^*v_i=(u_{ii}-1)(u_{jj}^*-1)-(u_{jj}^*-1)(u_{ii}-1)=u_{ii}u_{jj}^*-u_{jj}^*u_{ii}\in K_{\infty}.
    \end{equation}
    Thus we have for example:
    \begin{equation}
        v_2v_1^*=v_1^*v_2=v_1v_2^*,
    \end{equation}
    and (up to redefinition of $c_{ij}$),
    \begin{equation}
        \sum_{i,j=1}^nc_{ij}v_iv_j^*=\sum_{1\leq i\leq j\leq n}c_{ij}v_iv_j^*+k_3.
    \end{equation}
    Since $v^*_1+\cdots+v^*_n\in K_2$, we have
    \begin{equation}
        v_iv_1^*+\cdots+v_iv^*_{n-1}+v_iv^*_n\in K_3,
    \end{equation}
    so that the term $v_iv^*_n$ is a combination of the terms $v_iv_j$ with $j<n$. Finally the sum \eqref{EqxSumvvdgammaDeux} reduces to
    \begin{equation}
        x=c_01+\sum_{j=1}^{n-1}c_jd_j+\sum_{1\leq i\leq j\leq n-1}c_{ij}v_iv_j^*+k_3.
    \end{equation}
\end{proof}



    The left hand side belongs to $K_1$ while the first term in the right hand side belongs to $K_2$ (see proposition~\ref{PropuudansKKiii}).

    \begin{probleme}
        Why does it prove that $d_j\in K_1/K_2$?
        \begin{enumerate}
            \item
                $K_m$ is not a vector space. So when we write $K_i/K_{i+1}$, do we mean the \emph{set difference} or the class with respect to
                \begin{equation}
                    x\sim x+k_{i+1}
                \end{equation}
                or
                \begin{equation}
                    x\sim xk_{i+1}\quad ?
                \end{equation}
            \item
                Why $K_1/K_2$ could not be empty?
            \item
                Why $\alpha v_j+\beta v_j^*$ does \emph{not} belong to $K_2$?
        \end{enumerate}

    \end{probleme}

The decomposition leads us to decompose a basis of $\suqA_q$ into
\begin{equation}
    B=\{ 1 \}\cup B_1\cup B_2\cup B_3
\end{equation}
where
\begin{equation}
    \begin{aligned}[]
        B_1&=\{ d_i\tq 1\leq i\leq n-1 \}\\
        B_2&=\{ v_iv_j^*\tq 1\leq i\leq j\leq n-1 \}
    \end{aligned}
\end{equation}
and $B_3$ is a basis of whatever remains. We have $B_1\subset K_1$, $B_2\subset K_2$ and $B_3$ can be chosen as a subset of $K_3$.


We have the homomorphisms $\epsilon_{\theta}\colon \SU_q(n)\to \eC$ and we define the derivatives
\begin{equation}
    \begin{aligned}[]
        \epsilon'_k=\frac{ \partial  }{ \partial \theta_k }\epsilon_{\theta}|_{\theta_k=0}\\
        \epsilon''_{kl}=\frac{ \partial  }{ \partial \theta_l }\frac{ \partial  }{ \partial \theta_k }\epsilon_{\theta}|_{\theta_k=\theta_l=0}
    \end{aligned}
\end{equation}

\begin{proposition}
    The functionals $\epsilon$, $\epsilon'_{k}$ and $\epsilon''_{kl}$ with $1\leq k\leq j\leq n$ separate the points $1,d_1,\ldots,d_{n-1}$, $v_iv_j^*$ with $1\leq i\leq j\leq n-1$.
\end{proposition}

\begin{proof}
    We have to prove that for each $x$ in the enumerated elements, there exist a functional $\omega$ in the list such that $\omega(x)\neq 0$ and $\omega(y)\neq 0$ for $y\neq x$ where $x,y\in\{ 1,d_i,v_iv_j^* \}_{1\leq i\leq j\leq n-1}$.

    Let us begin with $x=1$. For him, the functional is $\omega=\epsilon$. We have $\epsilon(1)=1$ and $\epsilon(d_i)=\epsilon(v_iv_j^*)=0$. Let us pass to $x=d_j$. We prove that $\omega=\epsilon'_k$ with $k\neq j$ works. We have
    \begin{equation}
        \epsilon_{\theta}(d_j)=\frac{1}{ 2i }\big( \epsilon_{\theta}(u_{jj})-\epsilon_{\theta}(u_{jj}^*) \big)=\frac{1}{ 2i }\big(  e^{i\theta_j}- e^{-i\theta_j} \big).
    \end{equation}
    If $k\neq j$, we have $\epsilon'_k(d_j)=0$ while, if $j=k$, we have
    \begin{equation}
        \epsilon'_j(d_j)=\frac{1}{ 2i }\frac{ \partial  }{ \partial \theta_j }\big(  e^{i\theta_j}- e^{-i\theta_j} \big)_{\theta_j=0}=1.
    \end{equation}
    In summary,
    \begin{equation}
        \epsilon'(d_j)=\delta_{kj}.
    \end{equation}
    In order to see that $\epsilon'_k$ separates $d_k$, we still have to prove that $\epsilon'_k(v_iv_j^*)=0$ for every $i$ and $j$.

    \begin{probleme}
        It seems to me that the functional we are looking at are $\epsilon$ and its derivatives. We are not working with arbitrary $\theta_1,\ldots,\theta_{n-1}$. For arbitrary $\theta$, we have
        \begin{equation}
            \epsilon'_k(ab)=\frac{ \partial  }{ \partial \theta_k }\big( \epsilon_{\theta}(a)\epsilon_{\theta}(b) \big)=\epsilon'_k(a)\epsilon_{\theta}(b)+\epsilon_{\theta}(a)\epsilon'_k(b).
        \end{equation}
    \end{probleme}
\end{proof}

We consider the functionals
\begin{equation}
    a\mapsto\epsilon(\theta,a)=\epsilon_{\theta}(a)
\end{equation}
and the derivatives
\begin{equation}
    \begin{aligned}[]
        \epsilon'_l(\theta,a)&=\frac{ \partial  }{ \partial \theta_l }\big( \epsilon(\theta,a) \big)\\
        \epsilon''_{kl}(\theta,a)&=\frac{ \partial  }{ \partial \theta_k }\frac{ \partial  }{ \partial \theta_l }\big( \epsilon(\theta,a) \big)\\
    \end{aligned}
\end{equation}
We have the Leibnitz rules
\begin{equation}
    \begin{aligned}[]
        \epsilon'_l(\theta,ab)&=\frac{ \partial  }{ \partial \theta_l }\big( \epsilon(\theta,a)\epsilon(\theta,b) \big)\\
        &=\epsilon'_l(\theta,a)\epsilon(\theta,b)+\epsilon(\theta,a)\epsilon_l'(\theta,b),
    \end{aligned}
\end{equation}
and
\begin{equation}        \label{EqLeibnitzepsppkl}
    \begin{aligned}[]
        \epsilon''_{kl}(\theta,ab)&=\epsilon''_{kl}(\theta,a)\epsilon(\theta,b)+\epsilon'_l(\theta,a)\epsilon'_k(\theta,b)\\
        &\quad+\epsilon'_k(\theta,a)\epsilon'_l(\theta,b)+\epsilon(\theta,a)\epsilon''_{kl}(\theta,b).
    \end{aligned}
\end{equation}
We are interested in computing them on the basics elements $1$, $d_j$, $v_iv_j^*$ that are generating $\SU_q(n)$ modulo $K_3$. Computations show that
\begin{equation}
    \begin{aligned}[]
        \epsilon(\theta,1)&=1\\
        \epsilon(\theta,d_j)&=\frac{ 1 }{2}\big(  e^{i\theta_j}- e^{-i\theta_j} \big)&&\epsilon(0,d_j)=0\\
        \epsilon(\theta,v_j)&= e^{i\theta_j}-1&&\epsilon(0,v_j)=0\\
        \epsilon(\theta,v_iv_j^*)&=( e^{i\theta_i}-1)( e^{-i\theta_j}-1)&&\epsilon(0,v_iv_j^*)=0.
    \end{aligned}
\end{equation}
Taking the derivative,
\begin{equation}
    \begin{aligned}[]
        \epsilon'_l(\theta,1)&=0\\
        \epsilon'_l(\theta,d_j)&=\frac{ 1 }{2}i\delta_{lj}( e^{i\theta_l}+ e^{-i\theta_l})                             &&\epsilon'_l(0,d_j)=i\delta_{lj}\\
        \epsilon'_l(\theta,v_j)&=i\delta_{kl} e^{i\theta_j}                                                           &&\epsilon'_l(0,v_j)=i\delta_{lj}\\
        \epsilon'_l(\theta,v_iv_j^*)&=\delta_{ki} e^{i\theta_i}( e^{-i\theta_j}-1)+( e^{i\theta_i}-1)\delta_{kj} e^{-i\theta_j}     &&\epsilon'_l(0,v_iv_j^*)=0.
    \end{aligned}
\end{equation}

\begin{probleme}
    Pourquoi dans $\epsilon'_l(\theta,d_j)$, y'a pas un $i$ qui descend à cause de la dérivée ?
\end{probleme}

Taking once again the derivative,
\begin{equation}
    \begin{aligned}[]
        \epsilon''_{kl}(\theta,1)&=0\\
        \epsilon''_{kl}(\theta,d_j)&=\frac{ 1 }{2}\delta_{lj}\delta_{kl}( e^{i\theta_l}- e^{-i\theta_j})&&\epsilon''_{kl}(0,d_j)=0\\
        \epsilon''_{kl}(\theta,v_j)&=\delta_{lj}\delta_{kj} e^{i\theta_j}&&\epsilon''_{kl}(0,v_j)=\delta_{lj}\delta_{kj}
    \end{aligned}
\end{equation}
and
\begin{subequations}
    \begin{align}
        \epsilon''_{kl}(\theta,v_iv_j^*)&
        =-\delta_{ki}\delta_{li} e^{i\theta_i}( e^{-i\theta_j}-1)
        +\delta_{li}\delta_{kj} e^{i\theta_i} e^{-i\theta_j}\\
        &\quad-\delta_{kj}\delta_{lj} e^{-i\theta_j} ( e^{i\theta_i}-1)
        +\delta_{lj}\delta_{ki} e^{-i\theta_j} e^{i\theta_i}        \label{SubEqepsppsurvvs}
    \end{align}
\end{subequations}


From these results, we deduce the following proposition.
\begin{proposition}
    The functional family
    \begin{equation}
        \begin{aligned}[]
            a&\mapsto\epsilon(a)\\
            a&\mapsto\epsilon'_l(a)\\
            a&\mapsto\epsilon''_{kl}(a)\\
        \end{aligned}
    \end{equation}
    with $1\leq k\leq l\leq n-1$ separates the points $1$, $d_j$, $v_jv_j^*$ with $1\leq i\leq j\leq n-1$.
\end{proposition}

\begin{proof}
    For each point $x\in\{ 1,d_j,v_iv_j^* \}$, we have to find a functional $\omega$ in the given family such that $\omega(y)\neq 0$ if and only if $y=x$.

    For $x=1$, the functional $\epsilon$ makes the work:
    \begin{equation}
        \begin{aligned}[]
            \epsilon(1)\neq 0,&&\epsilon(d_j)=0,&&\epsilon(v_iv_j^*)=0.
        \end{aligned}
    \end{equation}
    The functional $\epsilon'_l$ separates $d_l$, indeed
    \begin{equation}
        \begin{aligned}[]
            \epsilon'_l(0,1)=0,&&\epsilon'_l(0,d_j)=\delta_{lj},&&\epsilon'_l(0,v_iv_j^*)=0.
        \end{aligned}
    \end{equation}
    And finally the functional $\epsilon''_{kl}$ separates $v_kv_l^*$ because
    \begin{equation}
        \begin{aligned}[]
            \epsilon''_kl(0,1)=0,&&\epsilon''_kl(0,d_j)=0,&&\epsilon''_kl(0,v_iv_j^*)=\delta_{li}\delta_{kj}+\delta_{ki}\delta_{kj}.
        \end{aligned}
    \end{equation}
    The last is nonzero if and only if $k=i$ and $l=j$.
\end{proof}


\begin{proposition}     \label{PropDecompxczepsApp}
    The decomposition
    \begin{equation}
        x=c_01+\sum_{j=1}^{n-1}c_jd_j+\sum_{1\leq i\leq j\leq n-1}c_{ij}v_iv_j^*+k_3
    \end{equation}
    is unique and $c_0=\epsilon(x)$, $c_j=\epsilon'_j(x)$ and $c_{ij}=\epsilon''_{ij}(x)$.
\end{proposition}

\begin{proof}
    Applying $\epsilon$ to $x$ we get $\epsilon(x)=c_0$. We also have
    \begin{equation}
        \epsilon'_l(x)=\sum_{j=1}^{n-1}c_j\underbrace{\epsilon'_l(d_j)}_{\delta_{lj}}=c_l
    \end{equation}
    and
    \begin{equation}
        \epsilon''_{kl}(x)=\sum_{i\leq i\leq j\leq n-1}c_{ij}\delta_{lj}\delta_{kj}=c_{kl}.
    \end{equation}
    Then the element $k_3$ is unique as the difference
    \begin{equation}
        k_3=x-c_01-\sum c_jd_j-\sum c_{ij}v_iv_j^*.
    \end{equation}
\end{proof}

We define the map
\begin{equation}
    \begin{aligned}
        P\colon \mA_q&\to K_2 \ \\
        x&\mapsto x-\epsilon(x)1-\sum_{j=1}^{n-1}\epsilon'_j(x)d_j.
    \end{aligned}
\end{equation}
This map satisfies because
\begin{equation}
    \begin{aligned}[]
        \epsilon\big( P(x) \big)&=\epsilon(x)-\epsilon(x)=0\\
        \epsilon'_k\big( P(x) \big)&=\epsilon'_k(x)-\sum\epsilon'_j(x)\epsilon'_k(d_j)=\epsilon'_k(x)-\epsilon'_k(x)=0,
    \end{aligned}
\end{equation}
so that
\begin{equation}
    P\big( P(x) \big)=P(x).
\end{equation}



%+++++++++++++++++++++++++++++++++++++++++++++++++++++++++++++++++++++++++++++++++++++++++++++++++++++++++++++++++++++++++++
\section{Gaussian process}
%+++++++++++++++++++++++++++++++++++++++++++++++++++++++++++++++++++++++++++++++++++++++++++++++++++++++++++++++++++++++++++

We follow \cite{UweLevy}.

\begin{proposition}
    If $L$ is conditionally positive and hermitian, then the following conditions are equivalent:
    \begin{enumerate}
        \item
            $\eta=0$
        \item
            $L|_{K_2}=0$
        \item
            $L$ is a $\epsilon$-derivation, that means
            \begin{equation}
                L(ab)=\epsilon(a)L(b)+L(a)\epsilon(b)
            \end{equation}
            for every $ab,\in\mA$.
        \item
            The states $\varphi_t$ are homomorphisms: $\varphi_t(ab)=\varphi_t(a)\varphi_t(b)$ for every $a,b\in\mA$ and $t\geq 0$.
    \end{enumerate}
\end{proposition}
A Schürmann triple with such a $L$ is a \defe{drift}{drift!process}.

\begin{proposition}     \label{PropProcessusGaussien}
    Let $L$ be an hermitian conditionally positive linear functional on $\mA$. The following conditions are then equivalent:
    \begin{enumerate}
        \item
            $L|_{K_3}=0$;
        \item
            $\eta|_{K_2}=0$;
        \item
            $L(b^*b)=0$ for every $b\in K_2$;
        \item
            $\eta(ab)=\epsilon(a)\eta(b)+\eta(a)\epsilon(b)$ for every $a,b\in\mA$;
        \item
            $\pi(a)=\epsilon(a)1$ for every $a\in\mA$.
    \end{enumerate}
\end{proposition}
We say that the triple $(\pi,\eta,L)$ with a $L$ satisfying the above conditions describes a \defe{Gaussian}{Gaussian!Schürmann triple}.

\begin{proposition}
    A generator of a Gaussian process satisfies
    \begin{equation}
        L=\sum\alpha_k\epsilon'_k+\sum B_{ij}\epsilon''_{ij}
    \end{equation}
    where the $\epsilon_k'$ and $\epsilon''_{ij}$ are the derivatives of the representation defined in proposition~\ref{PropReprezThetasuqn}.
\end{proposition}

We suppose now that $\pi$ is not Gaussian (that is $\pi$ is not $\epsilon 1$), but
\begin{equation}
    \pi=\pi_1\oplus\pi_2
\end{equation}
where $\pi_1$ is Gaussian.

\begin{probleme}
    Il est dit que l'espace sur lequel $\pi$ agit est
    \begin{equation}
        H_1=\bigcap_{\alpha\in K_1}\ker\pi(a).
    \end{equation}
    Cependant par définition de $K_1$, $\epsilon(a)=0$ dès que $a\in K_1$, donc $\pi_1(a)=0$.

    Comment ça marche ?
\end{probleme}

%---------------------------------------------------------------------------------------------------------------------------
\subsection{Poisson process}
%---------------------------------------------------------------------------------------------------------------------------

\begin{proposition}     \label{PropDefPoissonnL}
    Let $L\colon \mA\to \eC$ be a linear hermitian conditionally positive functional. Then the following conditions are equivalent:
    \begin{enumerate}
        \item
            there exists a state $\varphi\colon \mA\to \eC$ and a real $\lambda>0$ such that
            \begin{equation}
                L(b)=\lambda\big( \varphi(b)-\epsilon(b) \big)
            \end{equation}
            for every $b\in\mA$;
        \item
            there exists a Schürmann triple $(\pi,\eta,L)$ such that the cocycle is trivial, that is there exists a $h\in H$ such that
            \begin{equation}
                \eta(b)=\big( \pi(b)-\epsilon(b) \big)h.
            \end{equation}
    \end{enumerate}
\end{proposition}
A generator $L$ satisfying these properties is a \defe{Poisson generator}{Poisson!generator}. The map $\eta$ is the \defe{coboundary}{coboundary} of $h$.

\begin{probleme}
    Il faut expliquer cette histoire d'opérateurs $\mO$.
\end{probleme}

%---------------------------------------------------------------------------------------------------------------------------
\subsection{Gaussian cocycle}
%---------------------------------------------------------------------------------------------------------------------------

\begin{definition}
    Let $\pi$ be a representation of $\mA$ on an Hilbert space $H$. We define
    \begin{equation}
        H_{\epsilon}=\bigcap_{a\in K_1}\ker\pi(a).
    \end{equation}
    The restriction of $\pi$ on $H_{\epsilon}$ is the \defe{Gaussian part}{Gaussian!part of a representation} of $\pi$.
\end{definition}
In order to see that this definition makes sense, we have to prove that $\pi(x)f\in H_{\epsilon}$ whenever $x\in\mA$ and $f\in H_{\epsilon}$. If $a\in K_1$, we have $\pi(a)\pi(x)f=\pi(ax)f=0$ because $K_1$ is an ideal.

\begin{lemma}       \label{LempiepsHepsUni}
    The representation respects the decomposition \(H=H_{\epsilon}\oplus H_{\epsilon}^{\perp}\). Namely,
    \begin{enumerate}
        \item
            The representation acts as the counit on its Gaussian part:
            \begin{equation}
                \pi(a)|_{H_{\epsilon}}=\epsilon(a)\id|_{H_{\epsilon}}.
            \end{equation}
        \item
            We have \(\pi(\suqA_q)H_{\epsilon}^{\perp}\subset H_{\epsilon}^{\perp}\).
    \end{enumerate}
\end{lemma}

\begin{proof}
    \begin{enumerate}
        \item
            Let $f\in H_{\epsilon}$. We use proposition~\ref{PropDecompxczepsApp} taking into account the fact that $\pi(d_j)f=\pi(v_j^*)f=\pi(k_3)f=0$. It remains $\pi(x)f=c_0f=\epsilon(x)f$.
        \item
            Let \(h\in H_{\epsilon}^{\perp}\). For every \(v\in H_{\epsilon}\) we have \(\langle v, h\rangle =0\). If \(x\in\suqA_q\), we have
            \begin{equation}
                \langle v, \pi(x)h\rangle =\langle \pi(x^*)v, h\rangle =0
            \end{equation}
            since \(\pi(x^*)v=\epsilon(x^*)v\in H_{\epsilon}\).
    \end{enumerate}
\end{proof}

\begin{proposition}
    Let $w=\sum_{j=1}^nv_j^*v_j$. Then
    \begin{equation}
        H_{\epsilon}=\ker\pi(w)=\bigcap_{j=1}^n\ker\pi(v_j).
    \end{equation}
\end{proposition}

\begin{proof}
    Let $f\in H_{\epsilon}$. Since $\pi(v_j)f=0$ for every $j$, taking the sum we have $\pi(w)f=0$. Thus \( H_{\epsilon}\subset\ker\pi(w)\). Let now $f\in\ker\pi(w)$. We have
    \begin{equation}
        \begin{aligned}[]
            0&=\langle f, \pi(w)f\rangle \\
            &=\sum_{j=1}^n\langle \pi(v_j)f, \pi(v_j)f\rangle \\
            &=\sum_j\| \pi(v_j)f \|^2.
        \end{aligned}
    \end{equation}
    Thus for every $j$ we have $\| \pi(v_j)f \|=0$ and $f\in\ker\pi(v_j)$. That proves the inclusion
    \begin{equation}        \label{Eqkerpiwinclucapkervj}
        \ker\pi(w)\subset\bigcap_{j=1}^n\ker\pi(v_j).
    \end{equation}

    Let $f\in\ker\pi(v_j)$ and \( i<j\). We have $\pi(u_{jj})f=f$ and from \eqref{subEquuijcondii} we know that
    \begin{equation}
        \pi(u_{ij})f=\pi(u_{ij})\pi(u_{jj})f=q\pi(u_{jj})\pi(u_{ij})f,
    \end{equation}
    in other words,
    \begin{equation}        \label{EQooCPTHooZsrqeq}
        \big[ \id-q\pi(u_{jj}) \big]\pi(u_{ij})f=0.
    \end{equation}
    From corollary \ref{CorOpOdquijInverti}, the operator $\id-q\pi(u_{jj})$ is invertible. Applying the inverse on the equation \eqref{EQooCPTHooZsrqeq} brings $\pi(u_{ij})f=0$ and then $f\in\bigcap_{i<j}\ker\pi(u_{ij})$. Thus we have
    \begin{equation}        \label{Eqketpivjsubipyjketpiuij}
        \ker\pi(v_j)\subset\bigcap_{i<j}\ker\pi(u_{ij}).
    \end{equation}
    Combining with \eqref{Eqkerpiwinclucapkervj}, we have
    \begin{equation}        \label{EqIntSubSintkerkera}
        \ker\pi(w)\subset\bigcap_{1\leq i<j\leq n}\ker\pi(u_{ij}).
    \end{equation}

    Let us now prove that
    \begin{equation}
        \bigcap_{j=1}^n\ker\pi(v_j)\subset\bigcap_{1\leq j<i\leq n}\ker\pi(u_{ij}).
    \end{equation}
    For that consider $f\in H$ such that $\pi(v_j)f=0$ for every $j=1,\cdots,n$ and apply $\pi$ to the relation \eqref{Equustrunsuqn}. For each $i$ we have
    \begin{equation}
        \id=\sum_{s=1}^n\pi(u_{si}^*)\pi(u_{si}).
    \end{equation}
    Applying to $f$ and cutting the sum in three parts,
    \begin{equation}
        f=\sum_{s<i}\pi(u_{si}^*)\pi(u_{si})f+\pi(u_{ii}^*)\pi(u_{ii})f+\sum_{s>i}\pi(u_{si}^*)\pi(u_{si})f.
    \end{equation}
    The inclusion \eqref{Eqketpivjsubipyjketpiuij} shows that $\pi(u_{si})f=0$ when $s<i$; in the same time $\pi(u_{ii})f=f$. We are thus left with
    \begin{equation}
        0=\sum_{s>i}\pi(u_{si}^*)\pi(u_{si})f
    \end{equation}
    That implies in particular that
    \begin{equation}
        0=\sum_{s>i}\langle f, \pi(u_{si})^*\pi(u_{si})f\rangle =\sum_{s>i}\| \pi(u_{si})f \|,
    \end{equation}
    and that $\pi(u_{si})f=0$. What we just proved is that
    \begin{equation}
        \bigcap_{j=1}^n\ker\pi(v_j)\subset\bigcap_{1\leq j<i\leq n}\ker\pi(u_{ij}).
    \end{equation}
    Combining with \eqref{EqIntSubSintkerkera}, we have
    \begin{equation}
        \bigcap_{j=1}^n\ker\pi(v_j)\subset\bigcap_{i\neq j}\ker\pi(u_{ij}).
    \end{equation}
    Since $K_1$ is made of $u_{ij}$ ($i\neq j$) and $v_j$, we also have
    \begin{equation}        \label{EqInclkerpivjHesp}
        \bigcap_{j=1}^n\ker\pi(v_j)\subset H_{\epsilon}
    \end{equation}
    and then $\ker \pi(w)\subset H_{\epsilon}$ because of \eqref{Eqkerpiwinclucapkervj}. The inclusion \eqref{EqInclkerpivjHesp} also shows that the intersection $\bigcap_{j=1}^n\ker\pi(v_j)$ is equal to $H_{\epsilon}$. This concludes the proof.
\end{proof}

\begin{proposition}
    Let $h_1,\ldots,h_n\in H$. If $(\pi,\eta,L)$ is a Gaussian triple and if $\eta$ satisfies $\eta(v_j^*)=h_j$ for $j=1,\ldots,n-1$, then $\eta$ reads
    \begin{equation}        \label{EqDefEtaCocyGaussU}
        \begin{aligned}
            \eta\colon \suqA_q&\to H \\
            a&\mapsto \sum_{k=1}^{n-1}i\epsilon_k'(a)h_k.
        \end{aligned}
    \end{equation}
\end{proposition}

\begin{proof}

    The first point to be shown is that the given $\eta$ is a Gaussian $\pi$-$\epsilon$-1-cocycle. For that we have
    \begin{equation}
        \begin{aligned}[]
            \eta(ab)&=\sum_{k=1}^{n-1}i\Big( \epsilon_k'(a)\epsilon(b)+\epsilon(a)\epsilon_k'(b) \Big)h_k\\
            &=\eta(a)\epsilon(b)+\epsilon(a)\eta(b).
        \end{aligned}
    \end{equation}
    Since $\pi$ is part of a Gaussian triple, it satisfies $\epsilon(a)=\pi(a)$. Thus we obtain the cocycle condition $\eta(ab)=\pi(a)\eta(b)+\eta(a)\epsilon(b)$. The fact for a cocycle to be Gaussian is $\eta|_{K_2}=0$ (proposition~\ref{PropProcessusGaussien}). Here, since $\epsilon'_k(ab)=\epsilon'_k(a)\epsilon(b)+\epsilon(a)\epsilon'_k(b)$, we have
    \begin{equation}
        \eta(ab)=0
    \end{equation}
    whenever $a$ and $b$ belong to $K_1$. We proved that the map $\eta$ of equation \eqref{EqDefEtaCocyGaussU} is a Gaussian cocycle.

    Since $v_j=u_{jj}-1$, we have
    \begin{equation}
        \begin{aligned}[]
            \epsilon'_k(0,v_j^*&)=\frac{ \partial  }{ \partial \theta_k }\big[ \epsilon(\theta,v_j^*) \big]_{\theta=0}\\
            &=\frac{ \partial  }{ \partial \theta_k }\big[ \epsilon(\theta,u_{jj}^*)-\epsilon(\theta,1) \big]_{\theta=0}\\
            &=\frac{ \partial  }{ \partial \theta_k }\big[  e^{-i\theta_j} \big]\\
            &=-i\delta_{kj},
        \end{aligned}
    \end{equation}
    so that
    \begin{equation}
        \eta(v_j^*)=\sum_{k=1}^{n-1}i(-i)\delta_{kj}h_k=h_j.
    \end{equation}
    Since proposition~\ref{PropCocycleDeteretavjnmu} states that a cocycle is determined by its values on the elements $v_j^*$ ($j=1,\ldots n-1$), the formula \eqref{EqDefEtaCocyGaussU} determines the unique cocycle such that $\eta(v_j^*)=h_j$.
\end{proof}


\begin{proposition}     \label{PropGaussCocycle}
    Every Gaussian cocycle reads
    \begin{equation}
        \eta(a)=\sum_{k=1}^{n-1}i\epsilon_k'(a)h_k
    \end{equation}
    with $h_k=\eta(v_j^*)$.
\end{proposition}

\begin{proof}
    For each $x\in\suqA_q$ we have the decomposition
    \begin{equation}
        x=\epsilon(x)1+\sum_{j=1}^{n-1}\epsilon_j'(x)d_j+k_2.
    \end{equation}
    Applying $\eta$ and taking into account $\eta(1)=\eta(k_2)=0$ (the second equality is because $\eta$ is Gaussian), we have
    \begin{equation}
        \eta(x)=\sum_{k=1}^{n-1}\epsilon_k'(x)\eta(d_k).
    \end{equation}
    It remains to be proven that $\eta(d_j)=i\eta(v_j^*)$. We have
    \begin{equation}
        \begin{aligned}[]
            \eta(v_j^*)=\eta(u_{jj}^*)&=\sum_k^{n-1}\epsilon_k'(u_{jj}^*)\eta(d_k)\\
            &=-\sum_{k=1}^{n-1}i\delta_{jk}\eta(d_k)\\
            &=-i\eta(d_j).
        \end{aligned}
    \end{equation}
    \begin{probleme}
        Ici, Anna fait un truc plus compliqué.
    \end{probleme}

\end{proof}

%---------------------------------------------------------------------------------------------------------------------------
\subsection{Gaussian generator}
%---------------------------------------------------------------------------------------------------------------------------

Following proposition~\ref{PropDefPoissonnL}, a Gaussian generator has to vanish on $K_3$ and being hermitian, conditionally positive.

Let us begin to investigate linear functional $\psi$ vanishing on $K_3$. Since an element of $\suqA_q$ decomposes into
\begin{equation}
    x=\epsilon(x)1+\sum_{j=1}^{n-1}\epsilon_j'(x)d_j+\sum_{1\leq i\leq j\leq n-1}\epsilon_{ij}'(x)v_iv_j^*+k_3,
\end{equation}
the functional $\psi$ is determined by its values on the elements $1$, $d_j$ ($j=1,\ldots, n-1$) and $v_iv_j^*$ ($1\leq i\leq j\leq n-1$). But the functionals $\epsilon$, $\epsilon_k'$ and $\epsilon_{kl}''$ are separating these points, so
\begin{equation}
    \psi=\alpha_0\epsilon+\sum_{k=1}^{n-1}\alpha_k\epsilon_k'+\sum_{k,l}^{n-1}\beta_{kl}\epsilon_{kl}''.
\end{equation}
Since $\epsilon_{kl}''$ is symmetric with respect to $k,l$, we can choose $\beta_{kl}=\beta_lk$.

\begin{proposition}     \label{Propproppsidefposhermconsdpos}
    Let $\psi$ be of the form
    \begin{equation}        \label{Eqproppsidefposhermconsdpos}
        \psi=\alpha_0\epsilon+\sum_{k=1}^{n-1}\alpha_k\epsilon_k'+\sum_{k,l}^{n-1}\beta_{kl}\epsilon_{kl}''
    \end{equation}
    and denote by $B$ the matrix made of the numbers $\beta_{ij}$. Then
    \begin{enumerate}
        \item
            $\psi(1)=0$ if and only if $\alpha_0=0$.
        \item
            The functional $\psi$ is hermitian if and only if $\alpha_i\in\eR$ and $\beta_{ij}\in\eR$.

            \begin{probleme}
                Et $\alpha_0$ aussi doit être réel pour que $\psi$ soit hermitienne ?
            \end{probleme}

        \item
            If the matrix $B$ is positive defined, the functional $\psi$ is conditionally positive.
        \item
            If $\psi$ is conditionally positive and hermitian, then $B$ is positive defined.

    \end{enumerate}


\end{proposition}

\begin{proof}
    \begin{enumerate}
        \item
            Since $\epsilon_k'(1)=\epsilon_{kl}''(1)=0$ and $\epsilon(1)=1$, we have $\psi(1)=\alpha_0$.
        \item
            Let us begin with the elements of $B_1$. If $a$ belongs to the span of $B_1$,
            \begin{equation}
                \psi(a^*)=\sum_{k=1}^{n-1}\alpha_k\epsilon_k'(a^*)=\sum_{k=1}^{n-1}\alpha_k\overline{ \epsilon_k'(a) },
            \end{equation}
            so $\psi(a^*)=\overline{ \psi(a) }$ if and only if $\alpha_k\in\eR$.

            \begin{probleme}
                Il me semble qu'il faut aussi prendre $\alpha_0\in\eR$.
            \end{probleme}

            For elements in $B_2$, we use the formula \eqref{SubEqepsppsurvvs}:
            \begin{equation}
                \begin{aligned}[]
                    \psi\big( (v_kv_l^*)^* \big)&=\sum_{i,j=1}^{n-1}B_{ij}\epsilon_{ij}''(v_lv_k^*)\\
                    &=\sum_{i,j}B_{ij}(\delta_{jl}\delta_{ik}+\delta_{jk}\delta_{il})\\
                    &=B_{kl}+B_{lk}\\
                    &=2B_{kl}
                \end{aligned}
            \end{equation}
            because $B$ is symmetric. On the other hand we have similarly
            \begin{equation}
                \psi(v_kv_l^*)=2B_{kl},
            \end{equation}
            so that $B_{kl}$ has to be real when $\psi$ is Hermitian.

        \item
            Let $a\in K_1$. Using Leibnitz rule we have
            \begin{equation}
                \begin{aligned}[]
                    \epsilon(a^*a)&=0\\
                    \epsilon_i'(a^*a)&=0\\
                    \epsilon_{ij}''(a^*a)&=\epsilon_j'(a^*)\epsilon'_i(a)+\epsilon'_i(a^*)\epsilon'_i(a).
                \end{aligned}
            \end{equation}
            Thus
            \begin{equation}        \label{eqpsiaasBij}
                \psi(a^*a)=2\sum_{ij}B_{ij}\epsilon'_i(a^*)\epsilon'_j(a)=2\sum_{ij}B_{ij}\overline{ \epsilon_i'(a) }\epsilon'_j(a).
            \end{equation}
            Let us notice that $\overline{ \epsilon_i'(a) }\epsilon'_j(a)$ is a general product $x_ix_j$ with $x\in\eR^{n-1}$ because, if $a=\sum_kx_kd_k$,
            \begin{equation}
                \epsilon_i(a)=\sum_kx_k\epsilon'_i(d_k)=ix_i,
            \end{equation}
            so that $\overline{ \epsilon'_i(a) }\epsilon'_j(a)=(-i)ix_ix_j=x_ix_j$. By definition if $B$ is positive defined, the right hand side of equation \eqref{eqpsiaasBij} is positive and $\psi$ is conditionally positive.

        \item
            If $\psi$ is Hermitian, we already see that $B_{ij}$ are real. Now positivity of
            \begin{equation}
                \sum_{ij}B_{ij}\overline{ \epsilon'_i(a) }\epsilon'_j(a)
            \end{equation}
            shows that $B$ is positive defined.

            \begin{probleme}
                Que signifie une matrice «définie positive» ?
            \end{probleme}

    \end{enumerate}

\end{proof}

\begin{corollary}
    If $\psi$ is a Gaussian generator, there exist reals numbers $\alpha_1,\cdots,\alpha_{n-1}$ and a symmetric positive defined matrix $B$ such that
    \begin{equation}
        \psi=\sum_{k=1}^{n-1}\alpha_k\epsilon'_k+\sum_{i,j=1}^{n-1}B_{ij}\epsilon_{ij}''.
    \end{equation}
\end{corollary}

\begin{proof}
    By definition, a Gaussian generator is Hermitian, conditionally positive and vanishes on $K_3$. We already know that it is of the form
    \begin{equation}        \label{EqFormgenpsiGauss}
        \psi=\alpha_0\epsilon+\sum_{k=1}^{n-1}\alpha_k\epsilon'_k+\sum_{k,l}^{n-1}B_{kl}\epsilon_{kl}''.
    \end{equation}
    If $\psi$ is Hermitian, $\alpha_i=0$ and $B_{ij}\in \eR$. If $\psi$ is Hermitian and conditionally positive, then $B$ is positive definite.

    \begin{probleme}
        Pourquoi $\alpha_0=0$ ? C'est-à-dire : pourquoi $\psi(1)=0$ ?
    \end{probleme}

\end{proof}

\begin{proposition}
    consider the Gaussian cocycle (proposition~\ref{PropGaussCocycle})
    \begin{equation}
        \eta(a)=\sum_{k=1}^{n-1}i\epsilon_k'(a)h_k.
    \end{equation}
    \begin{enumerate}
        \item
            If the numbers $\langle h_k, h_l\rangle $ are real, then the formula
            \begin{equation}
                \psi=\frac{ 1 }{2}\sum_{k,l=1}^{n-1}\langle h_k, h_l\rangle \epsilon_{kl}''
            \end{equation}
            defines a conditionally positive Hermitian functional associated with the cocycle $\eta$.

        \item
            If one of $\langle h_k, h_l\rangle $ is not real, then there are no Hermitian functional associated with $\eta$.
    \end{enumerate}

\end{proposition}

\begin{proof}
    \begin{enumerate}
        \item
            We denote $\alpha_{kl}=\langle h_k, h_l\rangle $. If these numbers are real, we have $\alpha_{kl}=\alpha_{lk}$ since $\langle h_k, h_l\rangle =\overline{ \langle h_l, h_k\rangle  }$. From corollary~\ref{ItemPropCorSchr}\ref{ItemPropCorSchriii}, the functional $\psi$ is associated with $\eta$ if
            \begin{equation}
                -\langle \eta(a^*), \eta(b)\rangle =\epsilon(a)\psi(b)-\psi(ab)+\psi(a)\epsilon(b).
            \end{equation}
            Using the Leibnitz rule \eqref{EqLeibnitzepsppkl} (reduced by the fact that $a$ and $b$ belong to $K_1$), we have
            \begin{equation}
                \begin{aligned}[]
                    \psi(ab)&=\frac{ 1 }{2}\sum_{kl}\alpha_{kl}\epsilon_{kl}''(ab)\\
                    &=\frac{ 1 }{2}\alpha_{kl}\big( \epsilon'_l(a)\epsilon'_k(b)+\epsilon'_k(a)\epsilon'_l(b) \big)\\
                    &=\sum_{kl}\langle h_k, h_l\rangle \epsilon'_k(a)\epsilon'_l(b)\\
                    &=\sum_{kl}\langle \overline{ \epsilon'_k(a)h_k }, \epsilon'_l(b)h_l\rangle \\
                    &=\langle \eta(a^*), \eta(b)\rangle .
                \end{aligned}
            \end{equation}
            The functional $\psi$ is hermitian because it is of form \eqref{Eqproppsidefposhermconsdpos} with $\alpha_i=0$ and $\alpha_{kl}\in\eR$. Moreover the matrix $B_{kl}=\langle h_k, h_l\rangle $ is symmetric and real and defines a Gaussian cocycle, so that the functional $\psi$ is Hermitian.

        \item
            The number $\langle h_k, h_l\rangle $ decomposes in real and imaginary parts as
            \begin{equation}
                \langle h_k, h_l\rangle =\beta_{kl}+id_{kl}
            \end{equation}
            with
            \begin{equation}
                \begin{aligned}[]
                    \beta_{kl}&=\frac{ 1 }{2}\big( \langle h_k, h_l\rangle +\langle h_l, h_k\rangle  \big)\\
                    d_{kl}&=\frac{ -i }{2}\big( \langle h_k, h_l\rangle -\langle h_l, h_k\rangle  \big)\\
                \end{aligned}
            \end{equation}
            If $\psi$ is Hermitian, for $a$ and $b$ in $K_1$ we have
            \begin{equation}
                \begin{aligned}[]
                    \psi(ab)&=\langle \eta(a^*), \eta(b)\rangle \\
                    &=\sum_{kl}\langle \epsilon'_k(a^*)h_k, \epsilon'_l(b)h_l\rangle \\
                    &=\sum_{kl}\overline{ \epsilon'_k(a^*) }\epsilon'_l(b)\langle h_k, h_l\rangle \\
                    &=\sum_{kl}\epsilon'_k(a) \epsilon'_l(b)\langle h_k, h_l\rangle
                \end{aligned}
            \end{equation}
            Computing that way $\psi(v_iv_j)$ and $\psi(v_j^*v_i^*)$ and taking into account the formula $\epsilon'_k(v_i)=i\delta_{ki}$, we found that $\psi$ is only Hermitian if
            \begin{equation}
                \langle h_i, h_j\rangle =\langle h_j, h_i\rangle ,
            \end{equation}
            which means that $\langle h_i, h_j\rangle $ is real.
    \end{enumerate}
\end{proof}

Let \(\pi\) be the representation of \(\suqA_q\) on \(H\) and
\begin{equation}
    H=H_{\epsilon}\oplus H_{\epsilon}^{\perp}
\end{equation}
be the decomposition of \(H\) into Gaussian and non-Gaussian parts. The Schürmann triple is completed by \(\psi\colon \suqA_q\to \eC\) and \(\eta\colon \suqA_q\to H\). We consider the corresponding decomposition of \(\eta\):
\begin{equation}
    \eta=\eta^{\epsilon}\oplus\eta^{\perp}
\end{equation}
where \(\eta^{\epsilon}=\pr_{H_{\epsilon}}\circ\eta\) and \(\eta^{\perp}=\pr_{H_{\epsilon}^{\perp}}\circ\eta\). Let us denote by \(h^{\epsilon}\) the projection of \(h\) on \(H_{\epsilon}\). We want to see under which conditions \(\eta\) is a cocycle associated with a generator.

A Gaussian generator reads
\begin{equation}
    \eta(a)=\sum_{k=1}^{n-1}i\epsilon'_k(a)h_k
\end{equation}
where \(h_i=\eta(v_i^*)\). A condition to have a generator is to have \(\langle h_i, h_j\rangle \in\eR\). Now, the Gaussian part of the cocycle reads
\begin{equation}
    \eta^{\epsilon}(a)=\sum_ki\epsilon'_k(a)\pr_{\epsilon}h_k.
\end{equation}
It will accept a generator if and only if
\begin{equation}
    \langle \pr_{\epsilon}h_i, \pr_{\epsilon}h_j\rangle
\end{equation}
is real.

On the other hand, we can check that \(\eta^{\epsilon}\) is a cocycle for the representation \(\pi\), i.e.
\begin{equation}
    \eta^{\epsilon}(ab)=\pi(a)\eta^{\epsilon}(b)+\eta^{\epsilon}(a)\epsilon(b),
\end{equation}
or
\begin{equation}
    \pr_{\epsilon}\eta(ab)=\pi(a)\pr_{\epsilon}\eta(b)+\pr_{\epsilon}\eta(a)\epsilon(b).
\end{equation}
From lemma~\ref{LempiepsHepsUni} we have \(\pi(a)\eta^{\epsilon}(b)=\epsilon(a)\eta^{\epsilon}(b)\). We write the second term under the form
\begin{equation}
    \pr_{\epsilon}\Big( \pi(a)\big( \eta^{\epsilon}(b)+\eta^{\perp}(b) \big) \Big)=\epsilon(a)\eta^{\epsilon}(b)+\pr_{\epsilon}\pi(a)\eta^{\perp}.
\end{equation}
The last term vanishes by lemma~\ref{LempiepsHepsUni}.


\chapter{Complements}
\input{app_Wigner}
\input{app_Maxima}

\input{erreurs}
\input{faq}



% SCRIPT MARK -- MATLAB

\emptyInputPath
\addInputPath{tex/matlab}

    \selectlanguage{french}
\part{Matlab}
\input{133_matlab}


% SCRIPT MARK -- EXERCICES

\emptyInputPath
\addInputPath{tex/exercices}
\part{Exercices}

\input{TD_SVT}
% This is part of Mes notes de mathématique
% Copyright (c) 2011-2012,2014, 2020
%   Laurent Claessens
% See the file fdl-1.3.txt for copying conditions.

\chapter{Géométrie analytique (Besançon)}
% This is part of Mes notes de mathématique
% Copyright (c) 2011-2012,2014, 2019-2020
%   Laurent Claessens
% See the file fdl-1.3.txt for copying conditions.

%+++++++++++++++++++++++++++++++++++++++++++++++++++++++++++++++++++++++++++++++++++++++++++++++++++++++++++++++++++++++++++
\section{Espaces vectoriels normés}
%+++++++++++++++++++++++++++++++++++++++++++++++++++++++++++++++++++++++++++++++++++++++++++++++++++++++++++++++++++++++++++
%---------------------------------------------------------------------------------------------------------------------------
\subsection{Normes}
%---------------------------------------------------------------------------------------------------------------------------

\Exo{EspVectoNorme0001}
\Exo{GeomAnal-0040}
\Exo{GeomAnal-0041}
\Exo{GeomAnal-0042}
\Exo{GeomAnal-0043}
\Exo{GeomAnal-0044}

%---------------------------------------------------------------------------------------------------------------------------
\subsection{Topologie}
%---------------------------------------------------------------------------------------------------------------------------

\Exo{EspVectoNorme0002}
\Exo{EspVectoNorme0004}
\Exo{EspVectoNorme0005}
\Exo{EspVectoNorme0006}
\Exo{EspVectoNorme0007}
\Exo{EspVectoNorme0009}


% Pour des raisons de compatibilité avec «Mes notes de mathématique», la section «Système de coordonnées» est supprimée
% mars 2012
%+++++++++++++++++++++++++++++++++++++++++++++++++++++++++++++++++++++++++++++++++++++++++++++++++++++++++++++++++++++++++++
%\section{Systèmes de coordonnées}
%+++++++++++++++++++++++++++++++++++++++++++++++++++++++++++++++++++++++++++++++++++++++++++++++++++++++++++++++++++++++++++

\Exo{OutilsMath-0002}
\Exo{OutilsMath-0003}   % la figure TriangleRectange est ici
\Exo{OutilsMath-0005} 
\Exo{OutilsMath-0006}
\Exo{GeomAnal-0034}

%+++++++++++++++++++++++++++++++++++++++++++++++++++++++++++++++++++++++++++++++++++++++++++++++++++++++++++++++++++++++++++
\section{Courbes et surfaces}
%+++++++++++++++++++++++++++++++++++++++++++++++++++++++++++++++++++++++++++++++++++++++++++++++++++++++++++++++++++++++++++
\input{listeExoCourbesSurfaces}

\Exo{GeomAnal-0038}    % TODO : comprendre pourquoi cet exercice a été éliminé

%+++++++++++++++++++++++++++++++++++++++++++++++++++++++++++++++++++++++++++++++++++++++++++++++++++++++++++++++++++++++++++
\section{Limite et continuité}
%+++++++++++++++++++++++++++++++++++++++++++++++++++++++++++++++++++++++++++++++++++++++++++++++++++++++++++++++++++++++++++
\input{listeExoLimiteContinue}

 \Exo{GeomAnal-0035}   % TODO : voir pourquoi on n'avait pas mis cet exercice.

%+++++++++++++++++++++++++++++++++++++++++++++++++++++++++++++++++++++++++++++++++++++++++++++++++++++++++++++++++++++++++++
\section{Calcul différentiel}
%+++++++++++++++++++++++++++++++++++++++++++++++++++++++++++++++++++++++++++++++++++++++++++++++++++++++++++++++++++++++++++
\input{listeExoCalculDifferentiel}


%+++++++++++++++++++++++++++++++++++++++++++++++++++++++++++++++++++++++++++++++++++++++++++++++++++++++++++++++++++++++++++
\section{Intégrales multiples}
%+++++++++++++++++++++++++++++++++++++++++++++++++++++++++++++++++++++++++++++++++++++++++++++++++++++++++++++++++++++++++++
\input{listeExoIntegralesMultiples}

%+++++++++++++++++++++++++++++++++++++++++++++++++++++++++++++++++++++++++++++++++++++++++++++++++++++++++++++++++++++++++++
\section{Autres exercices}
%+++++++++++++++++++++++++++++++++++++++++++++++++++++++++++++++++++++++++++++++++++++++++++++++++++++++++++++++++++++++++++

Cette section contient entre autres des exercices donnés à des devoirs, interrogations et DS.

\Exo{devoir1-0004}
\Exo{devoir1-0005}
\Exo{devoir1-0006}


\Exo{devoir2-0002}
\Exo{devoir2-0003}
\Exo{devoir2-0004}
\Exo{devoir2-0005}
\Exo{devoir2-0006}
\Exo{devoir2-0007}
\Exo{devoir2-0008}
\Exo{devoir2-0009}


\Exo{CourbesSurfaces0004}
\Exo{devoir3-0002}
\Exo{CourbesSurfaces0009}
\Exo{devoir3-0004}



\Exo{controlecontinu0001}
\Exo{controlecontinu0002}
\Exo{controlecontinu0003}
\Exo{controlecontinu0004}
\Exo{controlecontinu0005}
\Exo{controlecontinu0006}
\Exo{controlecontinu0007}
\Exo{controlecontinu0008}
\Exo{controlecontinu0009}
\Exo{controlecontinu0010}
\Exo{controlecontinu0011}
\Exo{controlecontinu0012}
\Exo{controlecontinu0013}


\Exo{GeomAnal-0015}
\Exo{GeomAnal-0016}
\Exo{GeomAnal-0018}
\Exo{GeomAnal-0021}
\Exo{GeomAnal-0023}
\Exo{GeomAnal-0027}


\Exo{GeomAnal-0017}
\Exo{GeomAnal-0020}
\Exo{GeomAnal-0022}
\Exo{GeomAnal-0024}
\Exo{GeomAnal-0025}
\Exo{GeomAnal-0026}


\Exo{DS2011-0001}
\Exo{DS2011-0002}
\Exo{DS2011-0003}
\Exo{DS2011-0004}


\Exo{GeomAnal-0045}
\Exo{GeomAnal-0046}
\Exo{GeomAnal-0047}
\Exo{GeomAnal-0048}
\Exo{GeomAnal-0049}

%+++++++++++++++++++++++++++++++++++++++++++++++++++++++++++++++++++++++++++++++++++++++++++++++++++++++++++++++++++++++++++
\section{Exercices pour aller plus loin}
%+++++++++++++++++++++++++++++++++++++++++++++++++++++++++++++++++++++++++++++++++++++++++++++++++++++++++++++++++++++++++++


\Exo{GeomAnal-0001}    % position 31124
\Exo{GeomAnal-0002}    % position 23657
\Exo{GeomAnal-0003}    % position 28183
\Exo{GeomAnal-0004}    % position 55702
\Exo{GeomAnal-0005}
\Exo{GeomAnal-0006}
\Exo{GeomAnal-0007}
\Exo{GeomAnal-0008}    % position 25804
\Exo{GeomAnal-0009}    % position 26329
\Exo{GeomAnal-0010}
\Exo{GeomAnal-0011}
\Exo{GeomAnal-0012}
\Exo{GeomAnal-0013}
\Exo{GeomAnal-0014}


\Exo{CalculDifferentiel0005}
\Exo{CalculDifferentiel0014}
\Exo{CalculDifferentiel0018}



\chapter{Exercices d'analyse numérique (Louvain-la-Neuve)}
\input{ExosAnalNum}


\chapter{Pour des ingénieurs (Louvain-la-Neuve)}
\input{listeLineaire}
\input{mars2010}

\chapter{Pour les géographes (Bruxelles)}
\input{janvier2009}

\chapter{Pour les pharmaciens (Bruxelles)}
\input{pharma}



%TODO : il faudra réhabiliter ces exercices.
%\chapter{Correction des exercices de rappel}    % Correspond à CdI1
%Ces corrections correspondent à des énoncés que je ne suis pas certain d'encore avoir.
%\input{corr_complex-deriv-integ}


\input{209_exo_outils_math}

% SCRIPT MARK -- CdI

\chapter{Exercices de calcul différentiel et intégral 1}
% This is part of the Exercices et corrigés de mathématique générale.
% Copyright (C) 2009-2010,2015-2016, 2019-2020
%   Laurent Claessens
% See the file fdl-1.3.txt for copying conditions.

%+++++++++++++++++++++++++++++++++++++++++++++++++++++++++++++++++++++++++++++++++++++++++++++++++++++++++++++++++++++++++++
\section{Prérequis}
%+++++++++++++++++++++++++++++++++++++++++++++++++++++++++++++++++++++++++++++++++++++++++++++++++++++++++++++++++++++++++++

Cette section contient quelques exercices du type de ce qui est plus ou moins censé être connu à l'entrée de l'université dans diverses sections scientifiques\footnote{Ils proviennent surtout d'un cours pour ingénieur de Louvain-la-Neuve.}.

\Exo{INGE1114-0006}
\Exo{INGE1114-0008}
\Exo{INGE1114-0009}

\Exo{INGE1114-0010}
\Exo{INGE1114-0011}
\Exo{INGE1114-0012}
%\Exo{INGE1114-0013}
%\Exo{INGE1114-0014}
%\Exo{INGE1114-0015}
\Exo{INGE1114-0016}
\Exo{INGE1114-0017}
\Exo{INGE1114-0018}
%\Exo{INGE1114-0019}
%\Exo{INGE1114-0020}


\Exo{INGE11140023}
\Exo{INGE11140024}
\Exo{INGE11140025}
\Exo{INGE11140027}

%+++++++++++++++++++++++++++++++++++++++++++++++++++++++++++++++++++++++++++++++++++++++++++++++++++++++++++++++++++++++++++
\section{Limites et continuité}
%+++++++++++++++++++++++++++++++++++++++++++++++++++++++++++++++++++++++++++++++++++++++++++++++++++++++++++++++++++++++++++

\Exo{INGE11140028}
\Exo{INGE11140029}
\Exo{INGE11140030}
\Exo{INGE11140031}
\Exo{INGE11140032}

%+++++++++++++++++++++++++++++++++++++++++++++++++++++++++++++++++++++++++++++++++++++++++++++++++++++++++++++++++++++++++++
\section{Suites numériques}
%+++++++++++++++++++++++++++++++++++++++++++++++++++++++++++++++++++++++++++++++++++++++++++++++++++++++++++++++++++++++++++

\Exo{INGE11140033}
\Exo{INGE11140034}
\Exo{INGE11140035}
\Exo{INGE11140036}
\Exo{INGE11140037}
% This is part of the Exercices et corrigés de mathématique générale.
% Copyright (C) 2009-2011
%   Laurent Claessens
% See the file fdl-1.3.txt for copying conditions.
%+++++++++++++++++++++++++++++++++++++++++++++++++++++++++++++++++++++++++++++++++++++++++++++++++++++++++++++++++++++++++++
					\section{Limites}
%+++++++++++++++++++++++++++++++++++++++++++++++++++++++++++++++++++++++++++++++++++++++++++++++++++++++++++++++++++++++++++

\Exo{General0010}
\Exo{General0011}
\Exo{0013}
\Exo{0017}
\Exo{0016}
\Exo{0024}

%+++++++++++++++++++++++++++++++++++++++++++++++++++++++++++++++++++++++++++++++++++++++++++++++++++++++++++++++++++++++++++
					\section{Dérivées et optimisation}
%+++++++++++++++++++++++++++++++++++++++++++++++++++++++++++++++++++++++++++++++++++++++++++++++++++++++++++++++++++++++++++

\Exo{General0012}
\Exo{General0013}
\Exo{General0014}
\Exo{General0015}
\Exo{General0016}

%+++++++++++++++++++++++++++++++++++++++++++++++++++++++++++++++++++++++++++++++++++++++++++++++++++++++++++++++++++++++++++
					\section{Primitives et intégration}
%+++++++++++++++++++++++++++++++++++++++++++++++++++++++++++++++++++++++++++++++++++++++++++++++++++++++++++++++++++++++++++

\Exo{General0017}
\Exo{General0018}
\Exo{General0019}
\Exo{General0020}
\Exo{General0021}
\Exo{General0022}
\Exo{General0023}
\Exo{General0024}
\Exo{General0025}
\Exo{General0026}
\Exo{General0027}

%---------------------------------------------------------------------------------------------------------------------------
					\subsection{Longueur d'un arc de courbe}
%---------------------------------------------------------------------------------------------------------------------------

\Exo{Inter0012}
\Exo{Inter0013}

%---------------------------------------------------------------------------------------------------------------------------
					\subsection{Aire d'une surface de révolution}
%---------------------------------------------------------------------------------------------------------------------------

\Exo{Inter0015}
\Exo{Inter0014}
\Exo{Inter0016}

% This is part of the Exercices et corrigés de mathématique générale.
% Copyright (C) 2009
%   Laurent Claessens
% See the file fdl-1.3.txt for copying conditions.
%+++++++++++++++++++++++++++++++++++++++++++++++++++++++++++++++++++++++++++++++++++++++++++++++++++++++++++++++++++++++++++
					\section{Équations différentielles}
%+++++++++++++++++++++++++++++++++++++++++++++++++++++++++++++++++++++++++++++++++++++++++++++++++++++++++++++++++++++++++++

%---------------------------------------------------------------------------------------------------------------------------
					\subsection{Équations à variables séparées}
%---------------------------------------------------------------------------------------------------------------------------

\Exo{EquaDiff0001}

%---------------------------------------------------------------------------------------------------------------------------
					\subsection{Équations homogènes}
%---------------------------------------------------------------------------------------------------------------------------

\Exo{EquaDiff0002}

%---------------------------------------------------------------------------------------------------------------------------
					\subsection{Équations linéaires}
%---------------------------------------------------------------------------------------------------------------------------

\Exo{EquaDiff0003}

%---------------------------------------------------------------------------------------------------------------------------
					\subsection{Problèmes divers}
%---------------------------------------------------------------------------------------------------------------------------

\Exo{EquaDiff0004}
\Exo{EquaDiff0005}
\Exo{EquaDiff0006}
\Exo{EquaDiff0007}
\Exo{EquaDiff0008}
\Exo{EquaDiff0009}

%---------------------------------------------------------------------------------------------------------------------------
					\subsection{Équations différentielles du second ordre}
%---------------------------------------------------------------------------------------------------------------------------

\Exo{EquaDiff0010}
\Exo{EquaDiff0011}
\Exo{EquaDiff0012}


\Exo{EquaDiff0013}
\Exo{EquaDiff0015}
\Exo{EquaDiff0014}
\Exo{EquaDiff0016}


% This is part of the Exercices et corrigés de mathématique générale.
% Copyright (C) 2009-2010
%   Laurent Claessens
% See the file fdl-1.3.txt for copying conditions.
%+++++++++++++++++++++++++++++++++++++++++++++++++++++++++++++++++++++++++++++++++++++++++++++++++++++++++++++++++++++++++++
					\section{Fonctions de deux variables réelles}
%+++++++++++++++++++++++++++++++++++++++++++++++++++++++++++++++++++++++++++++++++++++++++++++++++++++++++++++++++++++++++++

%---------------------------------------------------------------------------------------------------------------------------
					\subsection{Tracer}
%---------------------------------------------------------------------------------------------------------------------------

\Exo{FoncDeuxVar0001}

%---------------------------------------------------------------------------------------------------------------------------
\subsection{Limites à deux variables}
%---------------------------------------------------------------------------------------------------------------------------

\Exo{FoncDeuxVar0010}
\Exo{FoncDeuxVar0011}
\Exo{FoncDeuxVar0012}
\Exo{FoncDeuxVar0013}
\Exo{FoncDeuxVar0014}
\Exo{FoncDeuxVar0015}

\Exo{FoncDeuxVar0016}
\Exo{FoncDeuxVar0018}

%---------------------------------------------------------------------------------------------------------------------------
\subsection{Dérivées partielles, différentielles totales}
%---------------------------------------------------------------------------------------------------------------------------
\Exo{FoncDeuxVar0002}
\Exo{FoncDeuxVar0003}

%---------------------------------------------------------------------------------------------------------------------------
\subsection{Différentiabilité, accroissements finis}
%---------------------------------------------------------------------------------------------------------------------------

\Exo{FoncDeuxVar0019}
\Exo{Maximisation-0001}
\Exo{FoncDeuxVar0026}
\Exo{FoncDeuxVar0021}
\Exo{FoncDeuxVar0022}
\Exo{FoncDeuxVar0023}
\Exo{DerrivePartielle-0000}
\Exo{DerrivePartielle-0001}
\Exo{FoncDeuxVar0025}

%---------------------------------------------------------------------------------------------------------------------------
\subsection{Plan tangent}
%---------------------------------------------------------------------------------------------------------------------------

\Exo{FoncDeuxVar0027}
\Exo{DerrivePartielle-0002}

%---------------------------------------------------------------------------------------------------------------------------
\subsection{Dérivées de fonctions composées}
%---------------------------------------------------------------------------------------------------------------------------

\Exo{DerrivePartielle-0003}
\Exo{FoncDeuxVar0017}
\Exo{DerrivePartielle-0004}
\Exo{DerrivePartielle-0005}
\Exo{FoncDeuxVar0030}
\Exo{FoncDeuxVar0024}
\Exo{FoncDeuxVar0020}
\Exo{DerrivePartielle-0006}

%---------------------------------------------------------------------------------------------------------------------------
\subsection{Dérivées de fonctions implicites}
%---------------------------------------------------------------------------------------------------------------------------
\Exo{FoncDeuxVar0004}
\Exo{FoncDeuxVar0005}
\Exo{FoncDeuxVar0006}
\Exo{FoncDeuxVar0007}

%---------------------------------------------------------------------------------------------------------------------------
\subsection{Extrema}
%---------------------------------------------------------------------------------------------------------------------------

\Exo{FoncDeuxVar0008}
\Exo{FoncDeuxVar0009}
\Exo{FoncDeuxVar0029}
\Exo{FoncDeuxVar0028}
\Exo{DerrivePartielle-0007}
\Exo{Maximisation-0002}
\Exo{DerrivePartielle-0008}
\Exo{DerrivePartielle-0009}
\Exo{DerrivePartielle-0010}
\Exo{Maximisation-0000}

% This is part of the Exercices et corrigés de mathématique générale.
% Copyright (C) 2009-2011,2014, 2019
%   Laurent Claessens
% See the file fdl-1.3.txt for copying conditions.
Lorsque nous demandons d'étudier une fonction, nous demandons les éléments suivants : domaine de définition, croissance, extrémums, points d'inflexion, asymptote et dessiner le graphe.


\Exo{III-1}
\Exo{III-2}
\Exo{III-3}
\Exo{III-4}
\Exo{III-5}
\Exo{TP40001}
\Exo{TP40002}
\Exo{TP40003}
\Exo{TP40004}
\Exo{TP40005}
\Exo{TP50001}
\Exo{TP50002}
\Exo{TP50003}
\Exo{TP50004}

%---------------------------------------------------------------------------------------------------------------------------
\subsection{Quelques fautes usuelles}
%---------------------------------------------------------------------------------------------------------------------------

Pour l'exercice~\ref{exoTP40001}, les fautes les plus souvent commises sont
\begin{enumerate}

	\item
		$f'= e^{2x}$ implique $f=\frac{1}{ 2 } e^{x}$. Cela n'est pas vrai. La dérivée de $ e^{2x}$ est $2 e^{2x}$. Le $2$ reste dans l'exponentielle.

	\item
		Lorsqu'on intègre par partie, il faut aussi mettre les bornes pour le morceau qui n'est pas dans la nouvelle intégrale :
		\begin{equation}
			\int_a^b fg'=[fg]_a^b-\int_a^bf'g.
		\end{equation}
\end{enumerate}

Pour l'exercice~\ref{exoTP40002}, les fautes les plus souvent commises sont
\begin{enumerate}

	\item
		Lorsqu'on a trouvé la solution générale $y_k(x)$ qui dépend du paramètre $k$ (ou $C$), il faut encore trouver la valeur du paramètre $k$ telle que $y_k(\pi)=0$.

\end{enumerate}

Pour l'exercice~\ref{exoTP40003}, les fautes les plus souvent commises sont
\begin{enumerate}

	\item
		Ne pas oublier que $e^0=1$.
\end{enumerate}


% This is part of Exercices et corrigés de CdI-1
% Copyright (c) 2011,2014
%   Laurent Claessens
% See the file fdl-1.3.txt for copying conditions.

\section{Intégrales de surface, Stokes et Green}


\setcounter{CountExercice}{0}


\noindent{\bf Exercice 6}\\

{\bf $(a)$ La suite $[k\rightarrow \f{1}{k}]$ est convergente.}\\

\noindent Nous allons montrer que cette suite converge vers $0$. Il faut donc prouver la chose suivante:
   \begin{equation}\label{eqn1}\forall \epsilon >0 \hspace{0,3cm} \exists K_\epsilon \in \eN \hspace{0,3cm} {\rm tq}  \hspace{0,3cm}  \forall k\geq K_\epsilon, \hspace{0,3cm}  |x_k-x|<\epsilon\end{equation}
{Remarque}: On pourrait également montrer que cette suite est {\it de Cauchy} pour prouver qu'elle est convergente sans devoir déterminer sa limite.\\

\noindent Pour prouver que (\ref{eqn1}) s'applique bien à la suite des $\f{1}{k}$ il nous faut montrer que

   \begin{equation}\label{eqn2}\forall \epsilon >0 \hspace{0,3cm} \exists K_\epsilon \in \eN \hspace{0,3cm} {\rm tq}  \hspace{0,3cm}  \forall k\geq K_\epsilon,  \hspace{0,3cm} \f{1}{k}<\epsilon\end{equation}

   \noindent Ceci est une conséquence immédiate de l'exercice précédent. On peut également le montrer de la manière suivante: à $\epsilon$ positif donné, si nous arrivons à déterminer l'indice $K_\epsilon$ de \eqref{eqn2} tel que $\forall k\geq K_\epsilon,  \hspace{0,3cm} \f{1}{k}<\epsilon$, il est clair que la suite satisfait à la définition. Or, $\f{1}{k} < \epsilon \leftrightarrow \f{1}{\epsilon} <k$. Donc si nous prenons $K_\epsilon := \ulcorner  1/\epsilon \urcorner+1$, on a bien que $\forall k\geq K_\epsilon$, $\f{1}{k}<\epsilon$, ce qui est ce qu'il fallait démontrer.

\vspace{1cm}
{\bf $(b)$ La suite $(1, \f{1}{2}, -\f{1}{3},  \f{1}{4}, -\f{1}{5}, \ldots )$ est convergente.}\\

\noindent On remarque que cette suite tend vers zéro. (Il suffit de voir que le numérateur est borné et que le dénominateur  tend vers l'infini). Si on l'écrit  sous la forme standard, on obtient:
              \[x_1 = 1, x_k = \f{(-1)^k}{k} \hspace{0.3cm} \forall k\geq 2\]
Donc, ce que nous voulons voir est que $x_k \longrightarrow_{k\rightarrow  \infty} 0$, i.e.:
   \begin{equation}\label{eqn3}\forall \epsilon >0 \hspace{0,3cm} \exists K_\epsilon \in \eN \hspace{0,3cm} {\rm tq}
       \hspace{0,3cm}  \forall k\geq K_\epsilon,  \hspace{0,3cm} |\f{(-1)^k}{k}|<\epsilon\end{equation}

\noindent Étant donné que $|(-1)^k| = 1 \, \forall k$, il est clair que l'équation (\ref{eqn3}) est la même que l'équation (\ref{eqn2}), et donc que l'on peut affirmer que pour tout $\epsilon > 0$, il suffit de prendre $K\geq \f{1}{\epsilon}$ et la condition est satisfaite.

\noindent{\bf Exercice 7}\\

Ici il est demandé de prouver de nouvelles règles de calcul en repartant de la définition de la convergence vers l'infini:
\begin{equation}
 \label{eqnconvinfGene} x_k \longrightarrow \infty \hspace{0.3cm} {\rm si} \hspace{0.3cm}  \forall M > 0 \hspace{0.3cm} \exists K_M \in \eN \hspace{0.3cm} {\rm tq} \hspace{0.3cm} \forall k \geq K_M, x_k \geq M \end{equation}
{\bf (a) $ \lim(x_k+y_k) = +\infty$.}\\

\noindent On veut voir  la chose suivante:
\begin{equation}\label{eqnconvinfCasA}
   \forall M > 0 \hspace{0.3cm} \exists K_M \in \eN \hspace{0.3cm} {\rm tq} \hspace{0.3cm} \forall k \geq K_M, x_k + y_k \geq M
  \end{equation}

\noindent Soit $M> 0$. Comme $x_k$ et $y_k$ convergent à l'infini, on sait que
\[\left\{\begin{array}{c}
         \exists K^x_M \in \eN \hspace{0.3cm} {\rm tq} \hspace{0.3cm} \forall k \geq K^x_M, x_k \geq \f{M}{2}\\
        \exists K^y_M \in \eN \hspace{0.3cm} {\rm tq} \hspace{0.3cm} \forall k \geq K^y_M, y_k \geq \f{M}{2},
\end{array}\right.\]
et donc il suffit de prendre $K_M = \max(K_M^x, K_M^y)$ dans (\ref{eqnconvinfCasA}) pour s'assurer que la définition est satisfaite.


\vspace{0.5cm}
\noindent{\bf (b) $ \lim(x_ky_k) = +\infty$.}\\

\noindent On veut voir la chose suivante:
\begin{equation}
 \label{eqnconvinfprod}  \forall M > 0 \hspace{0.3cm} \exists K_M \in \eN \hspace{0.3cm} {\rm tq} \hspace{0.3cm} \forall k \geq K_M, x_k  y_k \geq M \end{equation}

\noindent Soit $M> 0$. Comme $x_k$ et $y_k$ convergent à l'infini, on sait que
\[\left\{\begin{array}{c}
         \exists K^x_M \in \eN \hspace{0.3cm} {\rm tq} \hspace{0.3cm} \forall k \geq K^x_M, x_k \geq \sqrt M\\
        \exists K^y_M \in \eN \hspace{0.3cm} {\rm tq} \hspace{0.3cm} \forall k \geq K^y_M, y_k \geq \sqrt M,
\end{array}\right.\]

\noindent et donc il suffit  de prendre  $K_M = \max(K_M^x, K_M^y)$ dans (\ref{eqnconvinfprod}) pour s'assurer que la définition est satisfaite.

\vspace{0.5cm}
\noindent{\bf (d) Soit $z_k$ une suite tendant vers un réel $a$ strictement positif. Prouvez que $\lim(x_k  z_k) = +\infty$.}\\

Le but de l'exercice est toujours le même, c'est-à-dire de prouver que
\begin{equation}		\label{eqnconvinfz}
  \forall M > 0 \hspace{0.3cm} \exists K_M \in \eN \hspace{0.3cm} {\rm tq} \hspace{0.3cm} \forall k \geq K_M, \;x_k  z_k \geq M
\end{equation}

\noindent Soit $M>0$. On sait  que:

\begin{equation}
\label{eqn12}\left\{\begin{array}{l}
        \forall \tilde{M} >0 \;\exists K^x_{\tilde{M}} \in \eN \hspace{0.3cm} {\rm tq} \hspace{0.3cm} \forall k \geq K^x_{\tilde{M}},\; x_k \geq  \tilde{M} \\
       \forall \epsilon >0\;\exists K^z_\epsilon \in \eN \hspace{0.3cm} {\rm tq} \hspace{0.3cm} \forall k \geq K^z_\epsilon,\; |z_k-a| <\epsilon,
\end{array}\right.\end{equation}

\noindent Prenons un $\epsilon$ tel que $a-\epsilon>0$. Par la deuxième partie de (\ref{eqn12}) on voit qu'il existe un indice $ K^z_\epsilon$ tel que $ \forall k \geq K^z_\epsilon,\; z_k > a-\epsilon >0$.

\noindent Prenons un $\tilde{M}$ tel que $M= \tilde{M}(a-\epsilon)$. Par la première partie de (\ref{eqn12}) on voit qu'il existe un indice $ K^x_{\tilde{M}} $ tel que $\forall k \geq K^x_{\tilde{M}},\; x_k \geq  \tilde{M} $.


\noindent et donc il suffit  de prendre  $K_M = \max(K_{\tilde{M}}^x, K^z_\epsilon)$ dans (\ref{eqnconvinfz}) pour avoir que
\[ \forall k \geq K_M, \;x_k  z_k \geq \tilde{M}(a-\epsilon)=M.\]


\noindent{\bf Exercice 8}\\

\noindent Une suite $x_k$ est bornée si $\exists N>0$ tel que $\forall k$, $|x_k| < N$.

\noindent On veut voir que $\f{x_k}{y_k}\longrightarrow 0$, i.e.

\begin{equation}
\label{eqnconvborne}  \forall  \epsilon > 0 \hspace{0.3cm} \exists K_\epsilon \in \eN \hspace{0.3cm} {\rm tq} \hspace{0.3cm} \forall k \geq K_\epsilon, \; |\f{x_k}{y_k}| < \epsilon \end{equation}

\noindent Soit $\epsilon >0$. Comme la suite $x_k$ est bornée, on a $|\f{x_k}{y_k}|<\f{N}{|y_k|}\; \forall k$. On utilise maintenant le fait que $y_k \longrightarrow \infty$. Prenons $M=\f{N}{\epsilon}$. On peut écrire que $\exists K_M$ tel que $\forall k \geq K_M, \; y_k \geq M=\f{N}{\epsilon}$, et donc si dans (\ref{eqnconvborne}) on prend $K_\epsilon= K_M$ on a:\[\forall k \geq K_\epsilon,\; \; |\f{x_k}{y_k}|<\f{N}{|y_k|}<\f{N}{N/\epsilon}=\epsilon.\]



\noindent{\bf Exercice 9}\\

\noindent Pour cet exercice, on peut utiliser les règles de calcul. Il faut faire attention que ces règles ne s'appliquent que si toutes les limites existent!

\vspace{0.5cm}
\noindent{ (a)} $x_k = \f{k+2}{k}\cos(k\pi)$\\

\noindent On voit que cette suite va dans deux directions différentes, $+1$ et $-1$ à cause du facteur $\cos(k\pi)=(-1)^k$. Elle ne converge donc pas. Pour le prouver, on peut prendre deux suites partielles de la suite $x_k$ qui convergent vers des limites différentes.

\noindent Choisissons \[\left\{ \begin{array}{rcl} y_k &= x_{2k}&= \f{(2k)+2}{2k}(-1)^{2k}\\
 							  z_k &= x_{2k+1} &= \f{(2k+1)+2}{2k+1} (-1)^{2k+1}\end{array}\right.\]

\noindent Comme $x_k =\f{k+1}{k}= 1+\f{1}{k}$	et que $\f{1}{k} \rightarrow  0$, nous pouvons appliquer les règles de calcul et en déduire que $x_k \rightarrow  1$. On fait la même chose pour $y_k$.


\vspace{0.5cm}
\noindent{ (c)} $x_k = \f{k^3+k+1}{5k^3+2}$\\

\noindent Nous avons que \[\forall k, \;\;\;\;x_k =\; (\f{k^3}{k^3})\f{1+\f{1}{k} +\f{1}{k^3}}{5+\f{2}{k^3}}=\;\f{1+\f{1}{k} +\f{1}{k^3}}{5+\f{2}{k^3}} \]
Comme \[\forall k \geq 1\;\; \f{1}{k^3} \; \leq \;\f{1}{k^2}\; \leq \; \f{1}{k}\] et comme $\f{1}{k}\rightarrow 0$, nous pouvons appliquer la règle de l'étau pour voir que \[\f{1}{k^3} \rightarrow 0 \; \; \; {\rm et } \;\; \;\f{1}{k^2} \rightarrow 0.\]
En appliquant les règles de calcul à la suite $x_k$ transformée, on voit donc que $x_k \rightarrow  \f{1}{5}$.

\vspace{0.5cm}
\noindent{ (d)} $x_k = \f{k+(-1)^k}{k-(-1)^k}$\\

\noindent On peut le voir par exemple par la règle de l'étau:
\[\forall k \geq 0, \;\;\; \f{k-1}{k+1} \leq \f{k+(-1)^k}{k-(-1)^k} \leq \f{k+1}{k-1}. \]
Or, comme les deux suites qui bornent la suite $x_k$ convergent toutes les deux vers $1$, il est clair que $x_k$ converge aussi vers $1$.


\vspace{0.5cm}
\noindent{ (d)} $x_k = x_{k-1}^2\;+\;1,\, x_1=1$\\

\noindent Suite définie par récurrence. Ses premiers éléments sont \[1, \; 2, \; 5, \;  26, \; 677, \; \ldots\]
Toute  limite admissible réelle finie $l$  de cette suite doit satisfaire à \[l=l^2+1\] ce qui implique qu'elle ne peut avoir de limite réelle finie. En regardant ses premiers éléments, on remarque immédiatement qu'elle semble converger à l'infini. Nous allons le prouver en utilisant la définition.

\noindent Soit $M> 0$. On a \[x_k \geq k \, \forall k.\] En effet (par récurrence sur $k$): il est clair que $x_1 \geq 1$. Supposons que $x_k \geq k$. Ceci implique t-il que $x_{k+1}\geq k+1$? Par définition des $x_k$, $x_{k+1} = x_k^2+1$. Par l'hypothèse de récurrence, on a donc $x_{k+1}\geq (k)^2 +1\geq k+1$ ce qui prouve le résultat. Comme la suite $y_k=k$ converge à l'infini, il en est de même pour la suite $x_k$.



\section{Continuité de fonctions réelles}


\begin{center}
\LARGE \bf
Travaux Personnels
\end{center}

\begin{bf}
\begin{center}
BAC2 en sciences mathématiques et physiques
\end{center}
\end{bf}

{\bf Exercice 1.} Calculer les limites suivantes

\b
a) $\displaystyle \lim_{n \to \infty} \left( 1+ \frac{2}{n-4} \right)^n$

\medskip
b) 
$\displaystyle \lim_{n \to \infty}
         \left( 1+ \frac 1n \right)^{\sqrt{n}}$

\medskip
c)  $\displaystyle \lim_{x \to \infty}
    \left( 1+ \frac \alpha x \right)^x$

\medskip
d) 
$\displaystyle \lim_{x \to 0} \frac{\log \left( 1+ \alpha x \right)}{x}$


\medskip
e) 
$\displaystyle \lim_{x \to \infty}
\frac{a_0+a_1x + \dots +a_nx^n}{b_0+b_1x + \dots +b_mx^m}$
\quad où\, $a_j, b_j \in \eC$ \,et\, $n,m \ge 0$

\medskip
f) 
$\displaystyle \lim_{x \to 0} \frac{\sqrt{1-\cos x}}{x}$




{\bf Exercice 2.} Prouver que

\medskip
a)
$\displaystyle \lim_{x \to \infty} x^{\frac 1x} = \lim_{x \to 0^+} x^x = 1$

\medskip
b)
$\displaystyle \lim_{x \to \infty} \frac{x^{\ln x}}{{\mathrm e}^x} =0$
\quad
càd ${\mathrm e}^x$ croit plus vite que $x^{\ln x}$


{\bf Exercice 3.} Prouver que
$$
\cosh 2x \,=\, \cosh^2 x + \sinh^2 x,
\qquad
\sinh 2x \,=\, 2 \sinh x \cosh x
$$


{\bf Exercice 4.} Prouver que

a)
$1 + \cos z + \cos 2z + \dots + \cos nz = \displaystyle \cos \frac{nz}{2} \cdot \frac{\sin (n+1)z/2}{\sin z/2}$

b)
$1 + \sin z + \sin 2z + \dots + \sin nz = \displaystyle \sin \frac{nz}{2} \cdot \frac{\sin (n+1)z/2}{\sin z/2}$

{\it Aide:}\;
$\displaystyle \sum_{k=0}^n
\euler^{\sii kz}
=
\frac{1-\euler^{\sii (n+1)z}}{1-\euler^{\sii z}}
= \euler^{\sii nz/2} \cdot \frac{\euler^{\sii (n+1)z/2} - \euler^{-\sii (n+1)z/2}}{
\euler^{\sii z/2}-\euler^{-\sii z/2}}
$

Rappelons qu'une fonction $f \colon \mathbb{C} \supset D \to \eC$ est {\bf uniformément continue} si pour tout $\epsilon >0$ il existe un \( \delta>0\) tel que
$$
|x-y| < \delta \,\Longrightarrow\, |f(x)-f(y)| < \epsilon
\quad \text{ pour tout }\, x,y \in D.
$$
Prouver que la fonction $f \colon \eR \to \eR$, $x \mapsto x^2$ est continue, mais n'est pas uniformément continue.


\section{Intégrales, longueur de courbes, EDO's linéaires}


\exerNico 
Soient $n,m \in \eN \cup \{0\}$.
Calculer
$$
\int_0^1 x^n (1-x)^m \,dx
\quad \text{ et } \quad
\int_{-1}^1 (1+x)^n (1-x)^m \,dx
$$

{\bf Solution:}
Posons $I_{n,m} := \int_0^1 x^n (1-x)^m \,dx$.
Intégration par partie donne
la formule récursive
$$
I_{n,m} \,=\, \frac {m}{n+1} I_{n+1,m-1}.
$$
Avec $I_{n+m,0} = \frac{1}{n+m+1}$ nous obtenons
$$
I_{n,m} \,=\, \frac{n!\,m!}{(n+m+1)!}
$$
La substitution $x := 2t-1$ fournit
$$
\int_{-1}^1 (1+x)^n (1-x)^m \,dx
\,=\, 2^{n+m+1} \int_0^1 t^n (1-t)^m \,dt \,=\,  2^{n+m+1}
\cdot I_{n,m}.
$$




\exerNico 
Soient $a,b >0$.
Calculer
$$
\int_0^{\pi /2} \displaystyle \frac{d \varphi}{a^2 \sin^2 \varphi + b^2 \cos^2 \varphi}
$$

{\bf Solution:}
$$
\,=\, \int_0^{\pi /2} \frac{1 / \cos^2 \varphi}{a^2 \tan^2 \varphi+b^2} d\varphi \,=\, \int_0^\infty \frac {dt}{a^2t^2 + b^2} \,=\, \frac{\pi}{2ab}
$$


\exerNico 
Calculer la longueur de l'arc de la parabole $y = x^2,\;x \in [0,b]$.

\medskip
{\bf Solution:}
$$
s \,=\, \int_0^b \sqrt{1+4x^2} \,dx \,=\, \frac b 2 \sqrt{1+4b^2}+ \frac 14 \ln \left(2b+ \sqrt{1+4b^2} \right)
$$


\exerNico 
La {\bf parabole de Neil} $\nu$ est la courbe définie par $\nu (t) = (t^2,t^3)$, pour  $t \in \eR$.

\medskip
a)
Esquisser la parabole de Neil.

\medskip
b)
Quelle est la signification du paramètre $t$?

\medskip
{\bf Solution:} $t = \tan \alpha$

\medskip
c)
Calculer la longueur de l'arc
$\left\{ \nu (t) \mid t \in [0,\tau] \right\}$.


\medskip
{\bf Solution:}
$$
s \,=\, \int_0^\tau \sqrt{4 t^2+9t^4} \,d\tau \,=\, \frac{8}{27} \left( \left(1+ \frac 94 \tau^2\right)^{3/2}-1 \right)
$$



\exerNico
La {\bf hélice} $\gamma$ de pas $2 \pi h$ est la courbe dans $\eR^3$ définie par
$$
\gamma(t) \,=\, \left( r \cos t , r \sin t , h t \right)  .
$$


\medskip
a)
Esquisser la hélice.

\medskip
b)
Expliquer le mot ``pas''.


\medskip
c)
Calculer la longueur de l'arc sur la hélice si on fait un tour.

\medskip
{\bf Solution:}
$\int_0^{2\pi} \sqrt{r^2+h^2} \,dt \,=\, 2 \pi \sqrt{r^2+h^2}$


\bigskip
\exerNico
Calculer un système fondamental réel pour

\medskip
a) $y^{(4)}-y = 0$,

\medskip
b) $y^{(4)} +4y'' +4y = 0$,

\medskip
c) $y^{(4)} -2y^{(3)} +5y'' = 0$.


\bigskip
{\bf Solution:}

\medskip
a) ${\rm e}^x, {\rm e}^{-x}, \cos x, \sin x$

\medskip
b) $\cos \sqrt{2} x, x \cos \sqrt{2}x, \sin \sqrt{2}x, x \sin \sqrt{2}x$

\medskip
c)
$1, x, {\rm e}^x \cos 2x, {\rm e}^x \sin 2x$



\bigskip
\exerNico
Déterminer une solution particulière de l'équation
$y''+y=q$ pour

\medskip
a) $q = x^3$,

\medskip
b) $q = \sinh x$,

\medskip
c) $q = 1/\sin x$.


\bigskip
{\bf Solution:}

\medskip
a) $x^3 - 6 x$

\medskip
b) $\frac 12 \sinh x$

\medskip
c) $\sin x \cdot \ln |\sin x| - x \cos x$


\bigskip
\exerNico
L'équation différentielle $m \ddot y = mg - k\dot y$
décrit la chute d'un corps soumit
à la gravitation si la friction est proportionnelle à la vitesse (``un homme tombant de l'avion'').

\medskip
Calculer la solution avec $y(0) =0, \dot y(0) = 0$.
Calculer la ``vitesse finale'' $v_\infty = \displaystyle \lim_{t \to \infty} \dot y (t)$.



\bigskip
{\bf Solution:}

\medskip
L'équation homogène $\ddot y + k/m \cdot y = 0$
possède les solutions $c_1+c_2 {\rm e}^{-k/m \cdot t}$,
où $c_1, c_2 \in \eR$.

L'équation inhomogène $\ddot y + k/m \cdot y = g$
possède comme solution particulière une fonction lineaire, càd
$y_p = (mg/k)t)$.
En tenant compte des conditions initiales nous obtenons
$$
y(t) \,=\, \frac{mg}{k} \left( t-\frac mk (1-{\rm e}^{-k/m \cdot t})\right).
$$
En particulier, $v_\infty = mg/k$.






\bigskip
\exerNico
Regardons l'ensemble des solutions de l'équation différentielle $P({\rm D})y =0$.

Montrer l'équivalence entre les propositions suivantes :
\begin{enumerate}

\item
Pour toute solution $y$ on a $\displaystyle \lim_{t \to \infty} y(t) = 0$

\item
Pour toute racine $z$ du polynôme caractéristique on a ${\rm Re}\, z <0$.

\end{enumerate}
Dans ce cas, l'équation différentielle est appelé  \defe{asymptotiquement stable}{asymptotiquement stable}.

\bigskip
{\bf Solution:}
On a
$\displaystyle \lim_{t \to \infty} y(t) = 0$ pour toute solution $y$ ssi c'est vrai pour tout élément d'un système fondamental.
On a $\displaystyle \lim_{t \to \infty} t^k {\rm e}^{\gl t}=0$ ssi ${\rm Re }\,\gl <0$,
d'où l'affirmation suit.






\section{Calcul de limites}

\exerNico Déterminez si les limites suivantes existent et dans
l'affirmative calculez les en utilisant, s'il y a lieu, la règle de
l'Hospital ou la règle de l'étau.
\begin{enumerate}
\item $  \lim_{x \rightarrow  +\infty} \frac{x+1}{x^2+2} $
\item $  \lim_{x \rightarrow  +\infty} \frac{\sin(x)}{x} $
\item $  \lim_{x \rightarrow  0} \frac{\sin(x)}{x} $
\item $  \lim_{x \rightarrow  +\infty}  \frac{x ^n}{e ^x} $
\item $  \lim_{x \rightarrow  +\infty} (1 + \frac{a}{x})^x $
\item $  \lim_{x \rightarrow  0} (\frac{1}{\sin(x)} - \frac{1}{x} )$
\item $  \lim_{x \rightarrow  +\infty} \cos( 2 \pi x) $
\item $  \lim_{x \rightarrow  +\infty} \frac{1}{\sin(x)+2}(x) +\ln(x)\cos(x) $
\item $  \lim_{x \rightarrow  +\infty} \frac{ \ln(x)(\sin(x) +2)}{x} $
\item $  \lim_{x \rightarrow  +\infty} x ^\frac{1}{x} $
\end{enumerate}

\exerNico Déterminez si les limites suivantes existent et dans
l'affirmative calculez-les.
\begin{enumerate}
\item $  \lim_{x \rightarrow  0} x \sin(\frac{1}{x}) $
\item $  \lim_{x \rightarrow  0} \frac{\sin(\sin(x))}{x} $
\item $  \lim_{x \rightarrow  +\infty} (\ln(x))^\frac{1}{1 - \ln(x)}$
\end{enumerate}

\exerNico Calculez les limites suivantes:
\begin{enumerate}
\item $  \lim_{x \rightarrow  +\infty} \frac{\ln(x)}{x ^a} $
\item $  \lim_{x \rightarrow  +\infty} \frac{\ln(x)^a}{x ^b} $
\item $  \lim_{x \rightarrow  +\infty} a ^x $
\item $  \lim_{x \rightarrow  +\infty} a ^\frac{1}{x} $
\end{enumerate}
où $a$ et $b$ sont des réels positifs.
%

%

\exerNico Déterminez, pour chacune des suites suivantes, si elle converge
et dans l'affirmative calculez sa limite.
\begin{enumerate}
\item $  k \rightarrow  \cos( 2 \pi k) $
\item $  k \rightarrow  \cos(\frac{\pi}{3} k) $
\item $  k \rightarrow  k(a ^\frac{1}{k} -1 ) $
\end{enumerate}
où $a$ est une réel.\\



\exerNico Calculez  les limites suivantes si elles existent.
\begin{enumerate}
\item $  \lim_{x \rightarrow  +\infty} \cos x $
\item $  \lim_{x \rightarrow  \pm \infty }\sqrt{2x^4+3}-x^2 $

\end{enumerate}

\exerNico Déterminez si la limite de chacune des suites suivantes
existe et dans l'affirmative calculez la.
\begin{enumerate}
\item $  \lim_{k \rightarrow  +\infty }(\frac{a k +1}{k})^k $
\item $  \lim_{k \rightarrow  +\infty}\frac{1}{\sin(\frac{\pi}{6}k)+1}(k) + \ln(k)\cos(\frac{\pi}{5}k)$
\item $  \lim_{k \rightarrow  +\infty} \frac{\ln(k)(\sin(\frac{\pi}{3}k) +1)}{k} $
\item $  \lim_{k \rightarrow  +\infty } \sqrt[3k]{k} (1 +
\frac{1}{3k})^{3k} $
\end{enumerate}
où $a$ est un réel.

\section{Dérivabilité}



\exerNico Déterminez l'ensemble des points où les fonctions suivantes
sont continues et celui où elles sont dérivables. Prouvez soigneusement
vos résultats.
\begin{enumerate}
\item $ x \rightarrow x]$
\item $ x \rightarrow |x| $
\item $ x \rightarrow
	\left\{ \begin{array}{ll}
	\frac{1}{x} & \mbox{si } x \not= 0 \\
	0 & \mbox{sinon}
	\end{array} \right. $
\item $ x \rightarrow x^2  $
\end{enumerate}




\exerNico Étudiez la dérivabilité et la continuité
de la dérivée de chacune des fonctions suivantes:
\begin{enumerate}
\item $ x \rightarrow
\left\{ \begin{array}{ll}
0 & \mbox{si } x \not= 0 \\
1 & \mbox{sinon}
\end{array} \right.$
%
\item $ x \rightarrow
\left\{ \begin{array}{ll}
\sin(\frac{1}{x}) & \mbox{si } x \not= 0 \\
0 & \mbox{sinon}
\end{array} \right.$
%
\item $ x \rightarrow
\left\{ \begin{array}{ll}
x \sin(\frac{1}{x}) & \mbox{si } x \not= 0 \\
0 & \mbox{sinon}
\end{array} \right.$
%
\item $ x \rightarrow
\left\{ \begin{array}{ll}
x^2 \sin(\frac{1}{x}) & \mbox{si } x \not= 0 \\
0 & \mbox{sinon}
\end{array} \right.$
\end{enumerate}

Le but de cet exercice est aussi d'exhiber des exemples illustrant les
différents types de comportements possibles, relativement à la
continuité et la dérivabilité, d'une fonction en un point.

\exerNico Étudiez la dérivabilité et la continuité
de la dérivée de chacune des fonctions suivantes:
\begin{enumerate}
\item $ x \rightarrow
\left\{ \begin{array}{ll}
\frac{2x+a}{1+e^{\frac{1}{x}}} & \mbox{si } x \not= 0 \\
0 & \mbox{sinon}
\end{array} \right.$
%
\item $ x \rightarrow
\left\{ \begin{array}{ll}
\frac{\sin(x)}{x} & \mbox{si } x \not= 0 \\
1 & \mbox{sinon}
\end{array} \right.$
%
\item $ x \rightarrow
\left\{ \begin{array}{ll}
e^{\frac{-1}{x}} & \mbox{si } x > 0 \\
0 & \mbox{sinon}
\end{array} \right.$
%
\item $ [-\frac{1}{2}, \frac{1}{2}] \rightarrow \eR: x \rightarrow
\left\{ \begin{array}{ll}
(\frac{\sin(2x)}{x})^{x+1} & \mbox{si } x \not= 0 \\
1 & \mbox{sinon}
\end{array} \right.$
\end{enumerate}
où $a$ et $b$ sont des réels.


 \exerNico Considérons la fonction
$$f:\mathbb{R}\rightarrow\mathbb{R}:x\mapsto f(x)=\left\{
\begin{array}{ll}
x&\text{si }x\text{ est rationnel}\\
0&\text{si }x\text{ est irrationnel}
\end{array}
\right.$$

Vérifiez que $f$ est continue en $0$ mais n'est ni dérivable à  gauche ni dérivable à droite en
$0$.

\exerNico
\begin{enumerate}
\item Soit $(X,d)$ un espace métrique et $f \colon (X,d) \to \eR$ une application continue.
Montrer que l'ensemble $$\left\{ x \mid f(x) = 0 \right\}$$ est fermé.

\item Soit $f \colon \eR \to \eR$ une application continue.
Montrer que l'ensemble
$$
\{ x \in \eR \mid f(x) = x\}
$$
des points fixes de $f$ est fermé.

\end{enumerate}

\exerNico  Soit $A$ un sous-ensemble de l'espace métrique $(X,d)$.
Montrer que la fonction
$$
\dist_A \colon X \to \eR,
\quad x \mapsto \inf_{a \in A} d(a,x)
$$
est continue.


\exerNico  Soient $(X,d_X)$, $(Y,d_Y)$ deux espaces métriques.
Une application $f \colon X \to Y$ est {\bf Lipschitzienne}
s'il existe une constante $L \ge 0$ telle que
$$
d_Y \bigl( f(x), f(x') \bigr) \,\le\, L \,d_X (x,x')
\quad \text{ pour tout } x,x' \in X.
$$
Dans ce cas, on dit que $f$ est {\bf $L$-Lipschitzienne}.


\begin{enumerate}
\item
Montrer qu'une application Lipschitzienne est continue.
\item Montrer qu'une application $f \colon \eR \to \eR$, $x \mapsto ax+b$
est Lipschitzienne.
Quelle est la plus petite constante $L$ qui convienne?

\item Montrer que les fonctions $z \mapsto |z|$,
$z \mapsto \overline z$,
$z \mapsto {\rm Re\,} z$ et $z \mapsto {\rm Im\,} z$
de $\eC$ dans $\eR$ sont Lipschitziennes.
Quelle sont les plus petites constantes $L$ qui conviennent?
\item Montrer que la fonction $\dist_A \colon X \to \eR$ de l'Exercice~13 est Lipschitzienne.

\end{enumerate}

% This is part of Exercices et corrigés de CdI-1
% Copyright (c) 2011,2014
%   Laurent Claessens
% See the file fdl-1.3.txt for copying conditions.




\section{Intégration}

\exerNico
Soient $n,m \in \eN \cup \{0\}$.
Calculer
$$
\int_0^1 x^n (1-x)^m \,dx
\quad \text{ et } \quad
\int_{-1}^1 (1+x)^n (1-x)^m \,dx
$$



\exerNico
Soient $a,b >0$.
Calculer
$$
\int_0^{\pi /2} \displaystyle \frac{d \varphi}{a^2 \sin^2 \varphi + b^2 \cos^2 \varphi}
$$


\exerNico
Calculer la longueur de l'arc de la parabole $y = x^2,\;x \in [0,b]$.

\exerNico
La {\bf parabole de Neil} $\nu$ est la courbe définie par
$\nu (t) = (t^2,t^3), \, t \in \eR^n$.
\begin{enumerate}
\item Esquisser la parabole de Neil.

\item Quelle est la signification du paramètre $t$?

\item Calculer la longueur de l'arc
$\left\{ \nu (t) \mid t \in [0,\tau] \right\}$.
\end{enumerate}

\exerNico
Une {\bf hélice} $\gamma$ de pas $2 \pi h$ est une courbe dans $(\eR^n)^3$ définie par
$$
\gamma (t) \,=\, \left( r \cos t , r \sin t , h t \right)  .
$$

\begin{enumerate}
\item Esquisser $\gamma$ et expliquer le mot ``pas''.

\item Calculer la longueur de l'arc sur la hélice si on fait un tour.
\end{enumerate}

\exerNico Calculez la longueur des arcs de courbe suivants:
\begin{enumerate}
\item $y= \ln(1-x^2)  \hspace{3.5cm} 0\leq x\leq \f{1}{2}$
\item  $y= x^{3/2}  \hspace{4.57cm} 0\leq x\leq 5$
\item $y = 1-\ln(\cos x) \hspace{3cm} 0\leq x \leq \f{\pi}{4}$
\item l'arc de cubique déterminé par $y=x^3+x^2+x+1$ avec $0\leq x \leq 1$.
\end{enumerate}

% This is part of Exercices et corrigés de CdI-1
% Copyright (c) 2011,2013-2014
%   Laurent Claessens
% See the file fdl-1.3.txt for copying conditions.

%+++++++++++++++++++++++++++++++++++++++++++++++++++++++++++++++++++++++++++++++++++++++++++++++++++++++++++++++++++++++++++
\section{Quelques corrections}
%+++++++++++++++++++++++++++++++++++++++++++++++++++++++++++++++++++++++++++++++++++++++++++++++++++++++++++++++++++++++++++

Ce qui suit sont des corrections d'exercices donnée sur les feuilles distribuées au début de l'année.

\noindent 31.
\begin{enumerate}
\item $df_{(1,1)}$ et $dg_{(\sqrt2,\frac{\pi}{4})}$\\
    \[\begin{array}{l}\frac{ \partial f }{ \partial x }(x,y) = \frac{1}{y}\ln(\frac{x}{y})e^{\frac{x}{y}}+\frac{1}{x}e^{\frac{x}{y}}\\
            \frac{ \partial f }{ \partial x }(1,1)=e\\
            \frac{ \partial f }{ \partial y }(x,y)=-\frac{x}{y^2}\ln(\frac{x}{y})e^{\frac{x}{y}}-xe^{\frac{x}{y}}\\
        \frac{ \partial f }{ \partial x }(1,1)=e\end{array}\]

 \noindent Par les règles de calcul, $f$ est différentiable en $(1,1)$. La différentielle $df_{(1,1)}$ est donc représentée dans les bases canoniques de $\eR^2$ et $\eR$ par la matrice jacobienne (ici gradient):\[df_{(1,1)}=(e \;\; -e)\]

 \[\begin{array}{lclllllcl}\frac{ \partial g_1 }{ \partial r }(r,\theta) &=&\cos(\theta)& & & & \frac{ \partial g_1 }{ \partial \theta }(r,\theta)   & =&-r\sin(\theta)\\
         \frac{ \partial g_1 }{ \partial r }(\sqrt2, \frac{\pi}{4})&=&\frac{\sqrt2}{2}& & &&\frac{ \partial g_1 }{ \partial \theta }(\sqrt2, \frac{\pi}{4})& =&-1 \\
         \frac{ \partial g_2 }{ \partial r }(r,\theta) &=&\sin(\theta)&  && &\frac{ \partial g_2 }{ \partial \theta }(r,\theta)  &=&r\cos(\theta) \\
     \frac{ \partial g_2 }{ \partial r }(\sqrt2, \frac{\pi}{4})&=&-\frac{\sqrt2}{2}&& & &\frac{ \partial g_2 }{ \partial \theta }(\sqrt2, \frac{\pi}{4})& = &1\end{array}\]

La fonction $g$ est également différentiable en $(\sqrt2, \frac{\pi}{4})$ et sa matrice Jacobienne est:
\[dg_{(\sqrt2, \frac{\pi}{4})}=\left(\begin{array}{cc} \frac{\sqrt2}{2} & -1\\
    \frac{\sqrt2}{2}&1\end{array}\right)\]


\item $\tilde{f} \;=\;e^{\cos(\theta)}\ln(\cos(\theta))$.
\item On voit d'abord que $g(\sqrt2, \frac{\pi}{4})\;=\;(1,1)$. Donc
    \[\begin{array}{cccc} d\tilde{f}_{(\sqrt2, \frac{\pi}{4})} & = & df_{g(\sqrt2, \frac{\pi}{4})}\circ dg_{(\sqrt2, \frac{\pi}{4})}\\
        & =& df_{(1,1)}\circ dg_{(\sqrt2, \frac{\pi}{4})} \end{array}\]
                                et  la matrice jacobienne de la différentielle de la composée est donc:\[d\tilde{f}_{(\sqrt2, \frac{\pi}{4})}=(e\;\;-e)\left(\begin{array}{cc} \frac{\sqrt2}{2} & -1\\
                                    \frac{\sqrt2}{2}&1\end{array}\right)=(0\;\;-2e)\]



\end{enumerate}


\noindent 32.
\begin{enumerate}
    \item $\frac{ \partial g }{ \partial u } = e^v\frac{ \partial f }{ \partial x }(\star,\star)+2uv\frac{ \partial f }{ \partial y }(\star,\star)$
    \item $\frac{ \partial g }{ \partial v } = ue^v\frac{ \partial f }{ \partial x }(\star,\star)+(1+u^2)\frac{ \partial f }{ \partial y }(\star,\star)$
\end{enumerate}
où $(\star,\star) = (ue^v,v(1+u^2))$.

\vspace{1cm}

\noindent 33. \\

\noindent Soit $g:\eR^2\rightarrow \eR:(x,y)\rightarrow  f(x^2-y^2)$. Dérivées partielles de:\[(x,y)\rightarrow  y(\partial_xg)(x,y)+x(\partial_yg)(x,y)?\]
Nommons cette fonction $h$.
\begin{enumerate}
\item $\partial_xg(x,y) = 2xf'(x^2-y^2)$
\item$\partial_yg(x,y) = -2yf'(x^2-y^2)$
\end{enumerate}
et donc $h(x,y) = 0 \, \forall (x,y)\in \eR^2$.

\vspace{1cm}


\noindent 34. \\

\noindent $h(t)=f(t,g(t^2))$.\\

\begin{enumerate}
    \item $h'(t)=\frac{ \partial f }{ \partial x }(\star,\star)+\frac{ \partial f }{ \partial y }(\star,\star)2tg'(t^2)$
    \item $ \begin{array}{rl} h''(t)=     &   \frac{ \partial^2f }{ \partial x }(\star,\star)+4tg'(t^2)\frac{ \partial^2f }{ \partial x\partial y }(\star,\star)+4t^2(g'(t^2))^2\frac{ \partial^2f }{ \partial y^2 }(\star,\star) \\
        & +[2g'(t^2)+4t^2g''(t^2)]\frac{ \partial f }{ \partial y }(\star,\star)\end{array}$

\end{enumerate}
où $(\star,\star) = (t,g(t^2))$.

\vspace{1cm}

\noindent 35.
\[h:\eR^2\rightarrow \eR:(u,v)\rightarrow  f(g(ue^v),g(v)(1+u^2))^{g(u+v)}\]

\noindent Comme toujours il vaut mieux faire ce genre d'exercices prudemment. Renommons donc les diverses composantes de cette fonction.\\

\noindent Soit $l(u,v)=f(g(ue^v),g(v)(1+u^2))$. On a alors:
\begin{enumerate}
    \item $\frac{ \partial l }{ \partial u }(u,v) = \frac{ \partial f }{ \partial x }(\star,\star)g'(ue^v)e^v + \frac{ \partial f }{ \partial y }(\star,\star)g(v)2u$
    \item $\frac{ \partial l }{ \partial v }(u,v) = \frac{ \partial f }{ \partial x }(\star,\star)g'(ue^v)ue^v+\frac{ \partial f }{ \partial y }(\star,\star)g'(v)(1+u^2)$
\end{enumerate}
o\`{u} $(\star,\star)=(g(ue^v),g(v)(1+u^2))$.\\

\noindent Alors $h(u,v)=l(u,v)^{g(u+v)} = e^{g(u+v)\ln(l(u,v))}$ qui est facile à dériver:

\begin{enumerate}
    \item $\frac{ \partial h }{ \partial u } = [g'(u+v)\ln(l(u,v))+\frac{g(u+v)}{l(u,v)}\frac{ \partial l }{ \partial u }(u,v)] l(u,v)^{g(u+v)}$
    \item $\frac{ \partial h }{ \partial v } = [g'(u+v)\ln(l(u,v))+\frac{g(u+v)}{l(u,v)}\frac{ \partial l }{ \partial v }(u,v)] l(u,v)^{g(u+v)}$
\end{enumerate}



\noindent 26.
\begin{enumerate}
\item $(x,y)\rightarrow  3x^2+x^3y+x$.\\
Combinaison linéaire de fonctions continues et différentiables sur $\eR^2$ (Exercice: prouver rigoureusement que les polynômes sont bien des fonctions continues et différentiables sur $\eR^2$).


\item \(  (x,y)\rightarrow\begin{cases}
        e    &   \text{si } xy\neq 0\\
        e^{x+y}    &    \text{sinon}
    \end{cases}\)
N.B.: Il est toujours utile de se représenter le domaine de chacune des fonctions.

\noindent La première remarque est que cette fonction est clairement continue et différentiable en tout point hors de $\{xy=0\}$ (fonction constante). Sur $\{xy=0\}$?
\begin{enumerate}
\item Continuité:\\
Prenons un point dans $\{xy=0\}$, par exemple le point $(a,0)$ (Remarquez que le cas $(0,b)$ est réglé par symétrie). Pour voir si la fonction est continue en ce point il faut voir si \[\lim_{(x,y)\rightarrow (0,0)}f(x,y)=f(0,0)=e^a.\] Si on prend deux manières différentes d'aller vers $(a,0)$ ($y=0$ puis $x=a$) on voit que si $a \neq1$ la fonction ne peut pas être continue. Et en $(1,0)$? Si on $(x,y)\rightarrow (1,0)$ avec d'abord $y=0$ puis $y\neq0$ on aura regardé toutes les manières de tendre vers $(1,0)$. Or dans les deux cas les limites valent $e = f(1,0)$, ce qui prouve que la fonction est continue en $(1,0)$ (et $(0,1)$ par symétrie).

\item Différentiabilité:\\
Comme la fonction est discontinue en tout point $(a,0)$ et $(0,b)$ avec $a\neq1$ et $b\neq1$ elle est aussi non différentiable en chacun de ces points. Il reste donc les points $(1,0)$ et $(0,1)$. Comme toujours, nous regardons d'abord les dérivées directionnelles en $(1,0)$:
\[\frac{ \partial f }{ \partial u }(1,0) \;=\;\lim_{t\rightarrow 0}\frac{f(1+tu_1,tu_2)-e}{t}\]
Il y a deux possibilités: $u_2=0$ et donc $u=(\pm1.0)$ ou$u_2\neq0$ (pourquoi ne regarde-t-on que ces deux cas?).

\begin{enumerate}
\item si $u\neq(\pm1,0)$.\\
    $\frac{ \partial f }{ \partial u }(1,0) \;=\;\lim_{t\rightarrow 0}\frac{e-e}{t}\;=\;0$.
\item si $u=(\pm1,0)$, i.e. si $u=(1,0) = e_1$\\
    $\frac{ \partial f }{ \partial u }(1,0) = \frac{ \partial f }{ \partial x }(1,0)=\lim_{t\rightarrow 0}\frac{f(1+t,0)-e}{t}=\lim_{t\rightarrow 0}\frac{e^{1+t}-e}{t} =^H0$.
\end{enumerate}
\end{enumerate}
\underline{Conclusion}:\\
Si $f$ était différentiable en $(1,0)$, on aurait que sa différentielle prendrait la forme suivante:
\[\begin{array}{cc} df_{(1,0)}u& = \frac{ \partial f }{ \partial x }(1,0)u_1+\frac{ \partial f }{ \partial y }(1,0)u_2\\
    & = eu_1\;\;\forall u\in\eR^2 \end{array} \]
Sa différentielle satisferait également à:
\[  df_{(1,0)}u = \frac{ \partial f }{ \partial u }(1,0) = 0 \;\; \forall u \neq (\pm1,0) \in \eR^2\]
Les deux propriétés étant contradictoires, la fonction $f$ ne peut être différentiable en $(1,0)$ (ni en $(0,1)$ par symétrie).

\item \( \rightarrow  \begin{cases}
        \frac{ x }{ y }    &   \text{si } y\neq 0\\
        0    &    \text{sinon}
    \end{cases}\)

Continue et différentiable sur $\eR-\{y=0\}$. Sur l'axe $y=0$ elle n'est pas continue.
\item \( \rightarrow  \begin{cases}
        x+ay    &   \text{si } x>0\\
        x    &    \text{sinon}
    \end{cases}\)
Si $a=0$ fonction continue et différentiable sur $\eR^2$. Si $a\neq0$, fonction continue et différentiable partout en dehors de l'axe $x=0$. Sur cet axe, elle est discontinue en tout point sauf en $(0,0)$ où elle est continue. Mais elle n'est pas différentiable en $(0,0)$ car toutes ses dérivées directionnelles  n'y sont pas définies.
\item \(  \rightarrow\begin{cases}
        \frac{ xy^5 }{ x^6+y^6 }    &   \text{si } x\neq y\\
        0    &    \text{sinon}
    \end{cases}\)
Fonction continue et différentiable partout en dehors de la droite $x=y$.  La fonction est discontinue en chacun des points de cette droite.

\end{enumerate}

30.
\begin{enumerate}
\item $(u,v)\rightarrow  u^3+12u^2v-5v^3$\\
\begin{enumerate}
    \item $\frac{ \partial f }{ \partial u } = 3u^2+24uv$
    \item $\frac{ \partial f }{ \partial v } = 12u^2-15v^2$
\end{enumerate}
\item $(u,v)\rightarrow  f(u^2)\ln(v)$\\
\begin{enumerate}
    \item $\frac{ \partial f }{ \partial u } = 2uf'(u^2)\ln(v)$
    \item $\frac{ \partial f }{ \partial v } = \frac{f(u^2)}{v}$
\end{enumerate}
\item $(x,y)\rightarrow \tan(x+y^2)$\\
\begin{enumerate}
    \item $\frac{ \partial f }{ \partial x } =\frac{1}{cos^2(x+y^2)}$
    \item $\frac{ \partial f }{ \partial v } = \frac{2y}{cos^2(x+y^2)}$
\end{enumerate}
\item $(r,\theta)\rightarrow  r^\theta$
\begin{enumerate}
    \item $\frac{ \partial f }{ \partial r } =\theta r^{\theta-1}$
    \item $\frac{ \partial f }{ \partial \theta }<++> =\ln(r)r^\theta$
\end{enumerate}
\item $(x,y)\rightarrow (x+3)e^x$
\begin{enumerate}
    \item $\frac{ \partial f }{ \partial x } =e^x(x+4)$
    \item $\frac{ \partial f }{ \partial y } =0$
\end{enumerate}
\item $(u,v)\rightarrow  \ln(f(uv)) $\\

\begin{enumerate}
    \item $\frac{ \partial f }{ \partial u } = \frac{vf'(uv)}{f(uv)}$
    \item $\frac{ \partial f }{ \partial v } = \frac{uf'(uv)}{f(uv)}$
\end{enumerate}\pagebreak
\end{enumerate}

\noindent 32.
\begin{enumerate}
    \item $\frac{ \partial g }{ \partial u } = e^v\frac{ \partial f }{ \partial x }(\star,\star)+2uv\frac{ \partial f }{ \partial y }(\star,\star)$
    \item $\frac{ \partial g }{ \partial v } = ue^v\frac{ \partial f }{ \partial x }(\star,\star)+(1+u^2)\frac{ \partial f }{ \partial y }(\star,\star)$
\end{enumerate}
o\`{u} $\frac{ \partial f }{ \partial x }(\star,\star) = \frac{ \partial f }{ \partial x }(ue^v,v(1+u^2))$ et $\frac{ \partial f }{ \partial y }(\star,\star) = \frac{ \partial f }{ \partial y }(ue^v,v(1+u^2))$.

\noindent 34. $h(t)=f(t,g(t^2))$.\\

\begin{enumerate}
    \item $h'(t)=\frac{ \partial f }{ \partial x }(\star,\star)+\frac{ \partial f }{ \partial y }(\star,\star)2tg'(t^2)$
    \item $ \begin{array}{rl} h''(t)=     &   \frac{ \partial^2f }{ \partial x^2 }(\star,\star)+4tg'(t^2)\frac{ \partial^2f }{ \partial x\partial y }(\star,\star)+4t^2(g'(t^2))^2\frac{ \partial^2f }{ \partial y^2 }(\star,\star) \\
        & +[2g'(t^2)+4t^2g''(t^2)]\frac{ \partial f }{ \partial y }(\star,\star)\end{array}$

\end{enumerate}
où $(\star,\star) = (t,g(t^2))$.




 \section{Intégration}
 \subsection{Série A}
 Exercice 11
 \begin{enumerate}
   \exr $\int \frac{x^3+3x+1}{x} d x = \frac{x^3}3 + 3x + \ln(x)$%
   \exr $\int x^2d x = \frac{x^3}3$%
   \exr $\int 3(x^2+1)^2 d x = \int 3 x^4 + 6 x^2 + 3 d x = \frac 35
   x^5 + 2 x^3 + 3x$%
   \exr $\int (3x^2 - 6x)^3 (x-1) d x = \frac1{12} (3x^2 - 6x)^4$
 \end{enumerate}

 Exercice 12
 \begin{enumerate}
   \exr $\int \sin^2(x^2+1) \cos(x^2+1) x d x = \frac16
   \sin(x^2+1)^3$%
   \exr $\int \tan(x) d x = -\ln\abs{\cos(x)}$%
   \exr $\int \frac{1}{(2+\sqrt{x})\sqrt x} d x= 2 \ln(2+\sqrt{x})$%
   \exr $\int \frac{\ln(x)}{x(1- \ln^2(x)} d x = \frac12
   \ln\abs{1-\ln^2(x)}$%
 \end{enumerate}



   Travaux perso 2 ---------------

   1. Soient deux réels $x$ et $y$ vérifiant $0 < x < y$. On veut montrer
   que pour tout naturel $k \geq 2$, on a
   \[0 < \sqrt[k]{y} - \sqrt[k]{x} < \sqrt[k]{y-x}.\]

   La première inégalité vient de l'inégalité $x < y$ élevée à la
   puissance $\frac1k$.

   On peut ré-écrire la deuxième, sachant que $x > 0$, en divisant par
   $\sqrt[k]{x}$ pour obtenir
   \begin{equation}
    \sqrt[k]{\frac yx} - 1 - \sqrt[k]{\frac yx-1} < 0 \quad \text{ avec }\frac xy > 1
   \end{equation}
     ce qui s'écrit encore $f(t) < 0$ en posant
   $f(t) \pardef \sqrt[k]t - \sqrt[k]{t-1} - 1$. On peut alors étudier
   la fonction $f$. Étant donné que $f(1) = 0$, il suffirait que $f$
   soit strictement décroissante sur $]1;\infty[$ pour qu'on ait
   l'inégalité voulue, à savoir $f(t) < 0$ dès que $t > 1$.

   Pour le montrer, on voit que
   \[f^\prime(t) = \frac 1k \left(t^{\frac{1-k}k} -
     ({t-1})^{\frac{1-k}k}\right)\] d'où on tire les équivalences
   suivantes
   \begin{align}
     & & f^\prime(t) < 0\\
     &\ssi& t^{\frac{1-k}k} < ({t-1})^{\frac{1-k}k}\\
     &\ssi& t^{1-k} < ({t-1})^{1-k}\\
     &\ssi& t > t-1\\
     &\ssi& 0 > -1
   \end{align}
   où la dernière inégalité est manifestement vraie, ce qui prouve la
   première inégalité et achève l'exercice.

   2.


 \paragraph{Exercice 1}
 \begin{enumerate}
 \item Par exemple, $B(x,r)$ avec $x \in \eR^n$ et $r > 0$.

 \item On utilise la densité de $\eQ$ dans $\eR$ pour voir que $B(q,r)$
   ($q \in \eQ^n$ et $r > 0$) est également une base.

   On observe ensuite que seuls les $r$ « petits » sont utiles,
   donc on se restreint aux boules de la forme $B(q,1/n)$ ($q \in
   \eQ^n$ et $n \in \eN_0$). Cet ensemble de boules est une base
   dénombrable\marginpar{Pourquoi ?} de la topologie usuelle sur
   $\eR^n$.
 \end{enumerate}


\input{suiteCorr}

\chapter{Exercices de calcul différentiel et intégral 2}
% This is part of Exercices et corrigés de CdI-1
% Copyright (c) 2011, 2019, 2020
%   Laurent Claessens
% See the file fdl-1.3.txt for copying conditions.

%+++++++++++++++++++++++++++++++++++++++++++++++++++++
\section{Supremum, maximum}

\Exo{0001}
\Exo{0002}
\Exo{0003}
\Exo{00035}


\Exo{0004}
\Exo{0005}


%++++++++++++++++++++++++++++++++++++++++++++++++++++
\section{Suites}

\Exo{0006}
\Exo{0007}
\Exo{0008}
\Exo{0009}
\Exo{0010}
\Exo{0011}
\Exo{0012}

\Exo{0015}
\Exo{0018}
\Exo{0019}
\Exo{0020}


\subsection{Suites définies par récurrence}

\Exo{0021}
\Exo{0022}


\section{Calcul de limites}
\label{SecCalcLimFHtQNu}

\Exo{0023}
\Exo{0025}
\Exo{0026}
\Exo{0027}

\subsection{Limites à deux variables}

\Exo{0028}
\Exo{0029}
\Exo{0030}

\Exo{LimSup0001}

%+++++++++++++++++++++++++++++++++++++++++++++++++++++++++++++++++++++++++++++++++++++++++++++++++++++++++++++++++++++++++++
					\section{Limite et continuité}
%+++++++++++++++++++++++++++++++++++++++++++++++++++++++++++++++++++++++++++++++++++++++++++++++++++++++++++++++++++++++++++

\Exo{continueSupl1}
\Exo{continueSupl2}


\Exo{0031}
\Exo{0032}
\Exo{0033}
\Exo{0034}
\Exo{0035}


\Exo{0036}
\Exo{reserve0001}
\Exo{0037}
\Exo{0038}
\Exo{0039}
\Exo{0040}

\Exo{continueSup0003}
\Exo{continueSup0004}
\Exo{continueSup0005}


%+++++++++++++++++++++++++++++++++++++++++++++++++++++++++++++++++++++++++++++++++++++++++++++++++++++++++++++++++++++++++++
					\section{Dérivées partielles et différentiabilité}
%+++++++++++++++++++++++++++++++++++++++++++++++++++++++++++++++++++++++++++++++++++++++++++++++++++++++++++++++++++++++++++


\Exo{0041}
\Exo{0042}
\Exo{0043}
\Exo{0044}
\Exo{0045}
\Exo{0046}
\Exo{0047}
\Exo{0048}
\Exo{0049}
\Exo{0050}
\Exo{0051}
\Exo{0052}
\Exo{0053}
\Exo{0054}
\Exo{0055}
\Exo{0056}
\Exo{0057}
\Exo{0058}
\Exo{0059}
\Exo{0060}
\Exo{0061}

%+++++++++++++++++++++++++++++++++++++++++++++++++++++++++++++++++++++++++++++++++++++++++++++++++++++++++++++++++++++++++++
					\section{Séries et séries de puissances}
%+++++++++++++++++++++++++++++++++++++++++++++++++++++++++++++++++++++++++++++++++++++++++++++++++++++++++++++++++++++++++++

\Exo{0062}
\Exo{0063}
\Exo{0064}
\Exo{0065}
\Exo{0066}
\Exo{0067}


%+++++++++++++++++++++++++++++++++++++++++++++++++++++++++++++++++++++++++++++++++++++++++++++++++++++++++++++++++++++++++++
					\section{Exercices de topologie}
%+++++++++++++++++++++++++++++++++++++++++++++++++++++++++++++++++++++++++++++++++++++++++++++++++++++++++++++++++++++++++++

Si $A_n$ est une suite d'ensemble, le symbole
\begin{equation}
	\bigcap_{n=1}^{\infty}A_n
\end{equation}
désigne l'ensemble des éléments qui sont dans $A_n$ pour tout $n\in\eN$. Remarquez que l'infini \emph{n'est pas} un élément de $\eN$ ! L'intersection se fait donc de $n=1$ à l'infini; l'infini non compris.

Prenons comme exemple le cas du point~\ref{ItemF0072} de l'exercice~\ref{exo0072}. Étant donné que $A_n=\mathopen]-\frac{1}{ n },\frac{1}{ n }\mathclose[$, on pourrait croire que $A_{\infty}=\mathopen]0,0\mathclose[=\emptyset$, et que par conséquent, l'intersection $\bigcap_{n=1}^{\infty}$ est vide.

%---------------------------------------------------------------------------------------------------------------------------
					\subsection{Exercices ultra basiques}
%---------------------------------------------------------------------------------------------------------------------------


\Exo{0071}
\Exo{0072}
\Exo{0073}
\Exo{0074}
\Exo{0075}
\Exo{0076}
\Exo{0077}
\Exo{0078}
\Exo{0079}
\Exo{0080}

%---------------------------------------------------------------------------------------------------------------------------
					\subsection{Exercices simplement basiques}
%---------------------------------------------------------------------------------------------------------------------------

Les exercices qui suivent ne seront en principe pas vus aux séances (faute de temps, plus que faute d'envie), mais ils sont certainement très intéressants à regarder pour celles et ceux qui désirent en savoir un peu plus sur la topologie.

\Exo{0081}
\Exo{0082}
\Exo{0083}
\Exo{0084}
\Exo{0085}
\Exo{0086}
\Exo{0087}
\Exo{0088}
\Exo{0089}


%+++++++++++++++++++++++++++++++++++++++++++++++++++++++++++++++++++++++++++++++++++++++++++++++++++++++++++++++++++++++++++
					\section{Fonctions d'une variable réelle (suite)}
%+++++++++++++++++++++++++++++++++++++++++++++++++++++++++++++++++++++++++++++++++++++++++++++++++++++++++++++++++++++++++++

\Exo{0090}
\Exo{0091}
\Exo{0092}
\Exo{0093}
\Exo{0094}
\Exo{0095}
\Exo{0096}
\Exo{0097}
\Exo{0098}
\Exo{0099}
\Exo{0100}


%+++++++++++++++++++++++++++++++++++++++++++++++++++++++++++++++++++++++++++++++++++++++++++++++++++++++++++++++++++++++++++
					\section{Développements de Taylor et Maclaurin}
%+++++++++++++++++++++++++++++++++++++++++++++++++++++++++++++++++++++++++++++++++++++++++++++++++++++++++++++++++++++++++++

\Exo{Devel0001}
\Exo{Devel0002}
\Exo{Devel0003}
\Exo{Devel0004}

\Exo{Devel0009}

\Exo{Devel0005}
\Exo{Devel0006}
\Exo{Devel0007}
\Exo{Devel0008}


\Exo{reserve0002}

%+++++++++++++++++++++++++++++++++++++++++++++++++++++++++++++++++++++++++++++++++++++++++++++++++++++++++++++++++++++++++++
					\section{Optimisation sans contraintes}
%+++++++++++++++++++++++++++++++++++++++++++++++++++++++++++++++++++++++++++++++++++++++++++++++++++++++++++++++++++++++++++

\Exo{OptimSS0001}
\Exo{OptimSS0002}
\Exo{OptimSS0003}
\Exo{OptimSS0004}
\Exo{OptimSS0005}
\Exo{OptimSS0006}

%+++++++++++++++++++++++++++++++++++++++++++++++++++++++++++++++++++++++++++++++++++++++++++++++++++++++++++++++++++++++++++
					\section{Équations différentielles}
%+++++++++++++++++++++++++++++++++++++++++++++++++++++++++++++++++++++++++++++++++++++++++++++++++++++++++++++++++++++++++++


%---------------------------------------------------------------------------------------------------------------------------
					\subsection{Équations différentielles du premier ordre}
%---------------------------------------------------------------------------------------------------------------------------

\Exo{EqsDiff0001}
\Exo{EqsDiff0002}
\Exo{EqsDiff0003}
\Exo{EqsDiff0004}
\Exo{EqsDiff0005}

%---------------------------------------------------------------------------------------------------------------------------
					\subsection{Équations différentielles du second ordre}
%---------------------------------------------------------------------------------------------------------------------------

\Exo{EqsDiff0006}
\Exo{EqsDiff0007}
\Exo{EqsDiff0008}
\Exo{EqsDiff0009}

%---------------------------------------------------------------------------------------------------------------------------
					\subsection{Équations différentielles : modélisation}
%---------------------------------------------------------------------------------------------------------------------------

\Exo{EqsDiff0010}
\Exo{EqsDiff0011}
\Exo{EqsDiff0012}

\Exo{EqsDiff0013}
\Exo{EqsDiff0014}
\Exo{EqsDiff0015}
\Exo{EqsDiff0016}



%+++++++++++++++++++++++++++++++++++++++++++++++++++++++++++++++++++++++++++++++++++++++++++++++++++++++++++++++++++++++++++
					\section{Intégrales multiples}
%+++++++++++++++++++++++++++++++++++++++++++++++++++++++++++++++++++++++++++++++++++++++++++++++++++++++++++++++++++++++++++

%%%%%%%%%%%%%%%%%%%%%%%%%
%
% Tous les exercices de cette section ont été repris dans OutilsMath le 3 avril 2011.
%
%%%%%%%%%%%%%%%%%%%%%%%%

\Exo{IntMult0001}
\Exo{IntMult0002}

Calculer le volume ou la surface d'un domaine revient à intégrer la fonction constante $1$ sur le domaine. Si nous effectuons un changement de variables, le jacobien intervient toutefois.

\Exo{IntMult0003}
\Exo{IntMult0004}
\Exo{IntMult0005}
\Exo{IntMult0006}
\Exo{IntMult0007}
\Exo{IntMult0008}
\Exo{IntMult0009}
% Il n'y a plus de IntMult0010 parce qu'il traitait de l'intégrale gausienne et a été
% intégré aux notes d'agrégation. 4 août 2012.

\Exo{IntMult0011}
\Exo{IntMult0012}
\Exo{IntMult0013}


%+++++++++++++++++++++++++++++++++++++++++++++++++++++++++++++++++++++++++++++++++++++++++++++++++++++++++++++++++++++++++++
					\section{Théorème de la fonction implicite}
%+++++++++++++++++++++++++++++++++++++++++++++++++++++++++++++++++++++++++++++++++++++++++++++++++++++++++++++++++++++++++++


\Exo{Implicite0001}
\Exo{Implicite0002}
\Exo{Implicite0003}
\Exo{Implicite0004}
\Exo{Implicite0005}
\Exo{Implicite0006}
\Exo{Implicite0007}
\Exo{Implicite0008}
\Exo{Implicite0009}


%+++++++++++++++++++++++++++++++++++++++++++++++++++++++++++++++++++++++++++++++++++++++++++++++++++++++++++++++++++++++++++
\section{Variétés et extrémums liés}
%+++++++++++++++++++++++++++++++++++++++++++++++++++++++++++++++++++++++++++++++++++++++++++++++++++++++++++++++++++++++++++

\Exo{Variete0001}
\Exo{Variete0002}
\Exo{Variete0003}
\Exo{Variete0004}
\Exo{Variete0005}

%+++++++++++++++++++++++++++++++++++++++++++++++++++++++++++++++++++++++++++++++++++++++++++++++++++++++++++++++++++++++++++
\section{Intégrales curvilignes}
%+++++++++++++++++++++++++++++++++++++++++++++++++++++++++++++++++++++++++++++++++++++++++++++++++++++++++++++++++++++++++++


\Exo{Variete0006}
\Exo{Variete0007}
\Exo{Variete0008}
\Exo{Variete0009}


\Exo{Variete0010}       % repris en version très allégée dans OutilsMath. Dans OutilsMath, il y aura un corrigé
\Exo{Variete0011}


%+++++++++++++++++++++++++++++++++++++++++++++++++++++++++++++++++++++++++++++++++++++++++++++++++++++++++++++++++++++++++++
\section{Intégrales de surface, Stokes et Green}
%+++++++++++++++++++++++++++++++++++++++++++++++++++++++++++++++++++++++++++++++++++++++++++++++++++++++++++++++++++++++++++

\Exo{Variete0012}
\Exo{Variete0013}
\Exo{Variete0014}
\Exo{Variete0015}
\Exo{Variete0016}
\Exo{Variete0017}

\Exo{Variete0018}
\Exo{Variete0019}
\Exo{Variete0020}

%+++++++++++++++++++++++++++++++++++++++++++++++++++++++++++++++++++++++++++++++++++++++++++++++++++++++++++++++++++++++++++
\section{Autres}
%+++++++++++++++++++++++++++++++++++++++++++++++++++++++++++++++++++++++++++++++++++++++++++++++++++++++++++++++++++++++++++

% Ces trois exercices étaient du travail personnel de CdI1 à l'ULB en 2008-2009.
\Exo{TP20090001}
\Exo{TP20090002}
\Exo{TP20090003}

\begin{center}
	Bonnes vacances !
\end{center}

\input{listeExo_cv}
\input{listeExo_eqdiff}

\chapter{Exercices pour analyse CTU}
\input{151_exercices_analyseCTU}

\ifbool{isFrido}{}
{
    \corrChapitre{Corrigés systématiques}
}

% SCRIPT MARK -- FINAL


\emptyInputPath
\addInputPath{tex/front_back_matter}
\input{134_choses_finales}
\end{document}

% SCRIPT MARK -- AFTER DOCUMENT
% Ma réserve d'exercices est dans le fichier réserve.tex
