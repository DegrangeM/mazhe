% This is part of Mes notes de mathématique
% Copyright (c) 2008-2019
%   Laurent Claessens, Carlotta Donadello
% See the file fdl-1.3.txt for copying conditions.

%+++++++++++++++++++++++++++++++++++++++++++++++++++++++++++++++++++++++++++++++++++++++++++++++++++++++++++++++++++++++++++
\section{Topologie et distance}
%+++++++++++++++++++++++++++++++++++++++++++++++++++++++++++++++++++++++++++++++++++++++++++++++++++++++++++++++++++++++++++


\begin{lemma}   \label{LemDUJXooWsnmpL}
    Soient \( (X_1,d_1)\) et \( (X_2,d_2)\) des espaces métriques séparables. Alors \( X_1\times X_2\) admet une base dénombrable de topologie constituée de produits de boules de \( X_1\) par des boules de \( X_2\). Plus précisément si $A_i$ est dénombrable et dense dans \( X_i\) alors l'ensemble des produits
    \begin{equation}
        \big\{ B(y_1,r_1)\times B(y_2,r_2)\big\}_{\substack{y_i\in A_i\\r_i\in \eQ^+}}
    \end{equation}
    est une base de topologie pour \( X_1\times X_2\).
\end{lemma}

\begin{proof}
    Soit \( \mO\) un ouvert de \( X_1\times X_2\) et \( (x_1,x_2)\in \mO\). Par définition de la topologie produit\footnote{Définition~\ref{DefIINHooAAjTdY}.}, il existe \( r_1,r_2\in \eQ^+\) tels que \( B(x_1,r_1)\times B(x_2,r_2)\subset\mO\). Les parties \( A_i\) étant denses, il existe \( y_i\in B(x_i,r_i/2)\cap A_i\). Avec ces choix nous avons $x_i\in B(y_i,\frac{ r_i }{2})$. Nous avons donc
    \begin{equation}
        (x_1,x_2)\in B(y_1,\frac{ r_1 }{ 2 })\times B(y_2,\frac{ r_2 }{2}).
    \end{equation}
    Il est facile de voir que \( B(y_i,r_i/2)\subset B(x_i,r_i)\). En effet si \( z_i\in B(y_i,r_i/2)\) alors
    \begin{equation}
        d_i(z_i,x_i)\leq d(z_i,y_i)+d(y_i,x_i)\leq \frac{ r_i }{2}+\frac{ r_i }{2}=r_i.
    \end{equation}
    Au final,
    \begin{equation}
        (x_1,x_2)\in B(y_1,\frac{ r_1 }{ 2 })\times B(y_2,\frac{ r_2 }{2})\subset \mO.
    \end{equation}
\end{proof}


\begin{definition}
    Si \( (X,d_X)\) et \( (Y,d_Y)\) sont des espaces métriques, une \defe{isométrie}{isométrie d'espaces métriques} est une application bijective \( f\colon X\to Y\) telle que pour tout \( x,y\in X\) nous ayons
    \begin{equation}        \label{EQooVUOXooKJntMN}
        d_Y\big( f(x),f(y) \big)=d_X(x,y).
    \end{equation}
\end{definition}

\begin{remark}
    Une application vérifiant \eqref{EQooVUOXooKJntMN} est automatiquement injective. En pratique, il ne faut donc vérifier que la surjectivité.
\end{remark}

\begin{example}[Manque de surjectivité]
    Si \( X=\mathopen[ 0 , \infty \mathclose[\) et \( f(x)=x+1\) alors \( f\) vérifie \eqref{EQooVUOXooKJntMN} pour la distance \( d(x,y)=| x-y |\), mais n'est pas surjective.
\end{example}

\begin{propositionDef}[Groupe des isométries]
    Si \( (X,d)\) est un espace métrique,
    \begin{enumerate}
        \item
            l'ensemble des isométries de \( X\), noté \( \Isom(X)\)\nomenclature[Y]{$\Isom(X)$}{Le groupe des isométries de \( X\)} est un groupe pour la composition\index{isométrie!groupe}\index{groupe!des isométries!espace métrique}.
        \item
            Ce groupe agit fidèlement\footnote{Si vous ne savez pas ce que c'est, alors vous avez zappé la définition~\ref{DefuyYJRh}.} sur \( X\).
    \end{enumerate}
\end{propositionDef}
\begin{proposition}\label{PropLYMgVMJ}
    Une isométrie entre deux espaces métriques est continue.
\end{proposition}

\begin{proof}
    Soient \( f\colon X\to Y\) une application isométrique et \( \mO\) un ouvert de \( Y\). Soit \( a\in f^{-1}(\mO)\); si \( d(a,b)<r\), alors \( d\big( f(a),f(b) \big)<r\) et donc \( b\in f^{-1}\big( B(f(a),r) \big)\). Donc autour de chaque point de \( f^{-1}(\mO)\) nous pouvons trouver une boule ouverte contenue dans \( f^{-1}(\mO)\), ce qui prouve que \( f^{-1}(\mO)\) est ouvert.
\end{proof}

\begin{example}
    Si \( X\) est un ensemble, nous pouvons écrire la \defe{distance discrète}{distance discrète} :
    \begin{equation}
        d(x,y)=\begin{cases}
            0    &   \text{si } x=y\\
            1    &    \text{si } x\neq y\text{.}
        \end{cases}
    \end{equation}
    La topologie résultante est la topologie discrète, côtoyée dans l'exemple~\ref{DefTopologieDiscrete}\footnote{Vérifiez-le tout de même!}.

    Pour cette métrique, le groupe des isométries est le groupe symétrique de \( X\), c'est-à-dire le groupe de toutes les bijections de \( X\) sur lui-même.
\end{example}

\subsubsection{Distance point-ensemble}
%////////////////////////

\begin{definition}
	Si $A$ est une partie de l'espace métrique $(X,d)$ et si $x\in X$, nous disons que la \defe{distance}{distance!point et ensemble} entre $A$ et $x$ est le nombre
	\begin{equation}		\label{EqdefDistaA}
		d(x,A)=\inf_{a\in A}d(x,a).
	\end{equation}
\end{definition}
%The result is on the figure~\ref{LabelFigDistanceEnsemble}
\newcommand{\CaptionFigDistanceEnsemble}{La distance entre $x$ et $A$ est donnée par la distance entre $x$ et $p$. Les distances entre $x$ et les autres points de $A$ sont plus grandes que $d(x,p)$.}
\input{auto/pictures_tex/Fig_DistanceEnsemble.pstricks}

%---------------------------------------------------------------------------------------------------------------------------
\subsection{Suites et espaces métriques}
%---------------------------------------------------------------------------------------------------------------------------

%TODO : il y a un contre-exemple à faire à la page http://www.les-mathematiques.net/phorum/read.php?14,787368,787582

\begin{proposition}[Caractérisation séquentielle de la limite\cite{MonCerveau}]     \label{PROPooJYOOooZWocoq}
    Soient deux espaces métriques \( X\) et \( Y\) ainsi qu'une fonction \( f\colon X\to Y\). Soit \( a\in X\) et \( \ell\in Y\). On a
    \begin{equation}\label{EqLimooJYOOooZWocoqG}
        \lim_{x\to a} f(x)=\ell,
    \end{equation}
    si et seulement si, pour toute suite \( (x_k) \) telle que \( x_k \to a \), on a
    \begin{equation}\label{EqLimooJYOOooZWocoqS}
        \lim f(x_k)=\ell.
    \end{equation}
    Par ailleurs, l'une des deux limites existe si et seulement si l'autre existe.
\end{proposition}

\begin{proof}
  Le sens direct est la proposition~\ref{fContEstSeqCont}. Pour la réciproque, nous passons par la contraposée. C'est-à-dire que nous supposons que \( \ell\) n'est pas une limite de \( f\) pour \( x\to a\). Il existe un \( \epsilon\) tel que pour tout \( \delta\), il existe un \( x\) vérifiant \( d_X(x;a) <\delta\) et \( d_Y(f(x);\ell) >\epsilon\).

  Nous construisons à présent une suite de la manière suivante. Pour \( \delta=\frac{1}{ n }\) nous considérons \( x_n\) tel que \( d_X( x_n; a) <\delta\) et \( d_Y(f(x_n);\ell) > \epsilon \). Cette suite converge vers \( a\), mais la suite \( f(x_n)\) ne converge manifestement pas vers \( \ell\) : elle ne rentre jamais dans la boule \( B(\ell,\epsilon)\).
\end{proof}

Une fonction continue est séquentiellement continue. Dans les espaces métriques la proposition suivante montre que la réciproque est également vraie et la continuité est équivalente à la continuité séquentielle. Cela n'est cependant pas vrai pour n'importe quel espace topologique.

\begin{corollary}[Caractérisation séquentielle de la continuité en un point\cite{MonCerveau}]  \label{ItemWJHIooMdugfu}
    Si \( X\) et \( Y\) sont des espaces métriques, alors une fonction \( f\colon X\to Y\) est continue en un point si et seulement si elle est séquentiellement continue en ce point.
\end{corollary}

\begin{proof}
  Paraphrasons la preuve précédente. Nous supposons que \( X\) et \( Y\) sont métriques. Si \( f\) n'est pas continue en \( a\), il existe \( \epsilon>0\) tel que pour tout \( \delta>0\), il existe \( x\) tel que \( \| x-a \|\leq\delta\) et \( \| f(x)-f(a) \|>\epsilon\). Nous considérons un tel \( \epsilon\) et pour chaque \( n\geq1\in \eN\) nous considérons un \( x_n\) correspondant à \( \delta=\frac{1}{ n }\). Cela nous donne une suite \( x_n\to a\) dans \( X\) mais \( \| f(x_n) -f(a)\|\) reste plus grand que \( \epsilon\). Cela montre que \( f\) n'est pas non plus séquentiellement continue.
\end{proof}

Les espaces métriques ont une propriété importante que la \wikipedia{fr}{Espace_séquentiel}{fermeture séquentielle} est équivalente à la fermeture.
\begin{proposition}[Caractérisation séquentielle d'un fermé]    \label{PropLFBXIjt}
    Soient \( X\) un espace métrique et \( F\subset X\). L'ensemble \( F\) est fermé si et seulement si toute suite contenue dans \( F\) et convergeant dans \( X\) converge vers un élément de \( F\).
\end{proposition}
\index{fermeture séquentielle}
\index{séquentiellement fermé}

\begin{proof}
   Une suite contenue dans un fermé ne peut converger que vers un élément de ce fermé: c'était la proposition \ref{PROPooBBNSooCjrtRb}. Le point le plus important est donc l'autre sens: si toute suite d'éléments de \( F \) converge dans \( F \) alors \( F \) est fermé.
    
   Par contraposée, supposons que \( X\setminus F\) ne soit pas ouvert. Alors il existe \( x\in X\setminus F\) pour lequel tout voisinage intersecte \( F\). En prenant \( x_k\in B(x,\frac{1}{ k })\), nous construisons une suite contenue dans \( F\), convergeant vers \( x\) qui n'est pas dans \( F \).
\end{proof}


\begin{lemma}		\label{LemLimAbarA}
  Soit $X$ un espace métrique, et soit $(x_n)$ une suite convergente contenue dans un ensemble $A\subset X$. Alors la limite $x_n$ appartient à $\bar A$.
\end{lemma}
Ce lemme est précisément la version «espace métrique» du corolaire \ref{CorLimAbarA}; mais, donnons-en une preuve tout de même.
\begin{proof}
	Supposons que nous ayons une partie $A$ de $X$, et une suite $(x_n)$ dont la limite $\ell$ se trouve hors de $\bar A$. Dans ce cas, il existe un $r>0$ tel que\footnote{Une autre manière de dire la même chose : si $\ell\notin\bar A$, alors $d(\ell,A)>0$.} $B(\ell,r)\cap A=\emptyset$. Si tous les éléments $x_n$ de la suite sont dans $A$, il n'y en a donc aucun tel que $d(x_n,\ell)<r$. Cela contredit la notion de convergence $x_n\to \ell$.
\end{proof}

\begin{corollary}		\label{CorAdhEstLim}
  Soit $X$ un espace métrique, $A \subset X$ et $a \in \bar A$. Alors il existe une suite d'éléments dans $A$ qui converge vers $a$.
\end{corollary}

\begin{proof}
  Si $a\in A$, alors nous pouvons prendre la suite constante $x_n=a$. Si $a$ n'est pas dans $A$, alors $a$ est dans $\partial A$, et pour tout $n$, il existe un point de $A$ dans la boule $B(a,\frac{1}{ n })$. Si nous nommons $x_n$ ce point, la suite ainsi construite est une suite contenue dans $A$ et qui converge vers $a$ (ce dernier point est laissé à la sagacité du lecteur ou de la lectrice).
\end{proof}

En termes savants, ce corolaire signifie que la fermeture $\bar A$ est composé de $A$ plus de toutes les limites de toutes les suites contenues dans $A$.

\begin{proposition}[Caractérisation séquentielle de la continuité\cite{MonCerveau}]     \label{PropXIAQSXr}
    Soient \( X\) et \( Y\) deux espaces métriques. Une application \( f\colon X\to Y\) est continue sur \( X\) si et seulement si elle est séquentiellement continue sur \( X\).
\end{proposition}

\begin{proof}
    Le sens direct est déjà prouvé dans la proposition \ref{fContEstSeqCont}. Nous nous concentrons donc sur la réciproque.

    Soit \( \mO\) un ouvert de \( Y\); nous allons voir que le complémentaire de \( f^{-1}(\mO)\) est fermé dans \( E\). Pour cela nous considérons une suite convergente \( x_k\stackrel{E}{\longrightarrow} x\) avec \( x_k\in\complement f^{-1}(\mO)\) pour tout \( k\). Nous allons montrer que \( x\in \complement f^{-1}(\mO)\) et la caractérisation séquentielle\footnote{Proposition~\ref{PropLFBXIjt}.} de la fermeture conclura que \( \complement f^{-1}(\mO)\) est fermé.

    Pour tout \( k\), nous avons \( f(x_k)\in\complement \mO\), mais \( \mO\) est ouvert et \( f(x_k)\stackrel{Y}{\longrightarrow}f(x)\) parce que \( f\) est séquentiellement continue. Par conséquent \( f(x)\in\complement \mO\) et \( x\in\complement f^{-1}(\mO)\).
\end{proof}

%TODO : il y a ici trois théorèmes sur la continuité séquentielle. Il faut sans doute les fusionner.

\begin{proposition} \label{PropCJGIooZNpnGF}
    Si \( X\) et \( Y\) sont deux espaces métriques et \( f,g\colon X\to Y\) sont deux fonctions continues égales sur une partie dense de \( X\) alors \( f=g\).
\end{proposition}
\index{fonction!continue!égales}

\begin{proof}
    Les fonctions \( f\) et \( g\) sont séquentiellement continues (proposition~\ref{PropFnContParSuite}, ou proposition \ref{ItemWJHIooMdugfu}). Soient \( A\) un ensemble dense dans \( X\) sur lequel \( f\) et \( g\) sont égales, et \( x\notin A\). Vu que \( A\) est dense, il existe une suite \( a_n\) dans \( A\) telle que \( a_n\to x\). La séquentielle continuité de \( f\) et \( g\) donnent
    \begin{subequations}
        \begin{align}
            f(a_n)\to f(x)\\
            g(a_n)\to g(x),
        \end{align}
    \end{subequations}
    mais pour tout \( n\), \( f(a_n)=g(a_n)\). Par unicité de la limite\footnote{Proposition~\ref{PropFObayrf}.} dans \( Y\), \( f(x)=g(x)\).
\end{proof}

%--------------------------------------------------------------------------------------------------------------------------- 
\subsection{Espace métrisable}
%---------------------------------------------------------------------------------------------------------------------------

\begin{definition}[Espace vectoriel topologique métrisable\cite{ooOFEPooVFgTXm}]
    Un espace topologique est \defe{métrisable}{métrisable!espace vectoriel topologique} si il existe une distance compatible avec la topologie.
\end{definition}
\index{espace!vectoriel topologique!métrisable}

\begin{proposition}[\cite{ooCGEHooVTyTuY}]      \label{PROPooXWBTooCvGLOj}
    Soit un espace topologique métrisable \( X\).
    \begin{enumerate}
        \item   \label{ITEMooOXVRooBsKwuq}
            Tout fermé de \( X\) est une intersection dénombrable d'ouverts.
        \item
            Tout ouvert de \( X\) est une union dénombrable de fermés.
    \end{enumerate}
\end{proposition}

\begin{proof}
    Soit une métrique \( d\) compatible avec la topologie de \( X\) et un fermé \( A\). Nous posons
    \begin{equation}
        V_n=\{ x\in X\tq d(x,A)<\frac{1}{ n } \}.
    \end{equation}
    Et juste pour faire simple nous notons \( V_0=X\).
    \begin{subproof}
        \item[Les parties \( V_n\) sont ouvertes]
            Soit \( x\in V_n\). Trouvons un voisinage de \( x\) contenu dans \( V_n\) afin de pouvoir encore invoquer le théorème~\ref{ThoPartieOUvpartouv}. D'abord, vu que \( x\in V_n\), il existe \( a\in A\) tel que \( d(x,a)<\frac{ 1 }{ n }\) (ici les inégalités strictes sont importantes).

            Soient \( \epsilon>0\) que nous fixerons plus bas, et \( y\in B(x,\epsilon)\). L'inégalité triangulaire donne
            \begin{equation}
                d(y,a)\leq d(y,x)+d(x,a)<\epsilon+\frac{1}{ n }.
            \end{equation}
            Nous pouvons donc choisir \( \epsilon\) de telle sorte que \( d(y,a)<1/n\). Avec ce \( \epsilon\), nous avons, pour tout \( y\in B(x,\epsilon)\) :
            \begin{equation}
                d(y,A)\leq d(y,a)<\frac{1}{ n }
            \end{equation}
            et donc \( y\in V_n\).
        \item[\( A\) est l'intersection des \( V_n\)]
            Nous avons évidemment \( A\subset V_n\) pour tout \( n\). Et d'autre part, si \( a\in\bigcap_{n\in \eN} V_n\) alors \( d(a,A)<\frac{1}{ n }\) pour tout \( n\). Cela implique \( d(a,A)=0\), et donc \( a\in A\) par le lemme \ref{LEMooAIARooQADaxM}.
        \end{subproof}

        Ceci démontre le point \ref{ITEMooOXVRooBsKwuq}.

    En ce qui concerne la seconde partie, nous appliquons la première partie au complémentaire. Si \( \mO\) est ouvert, \( \mO^c\) est fermé et
    \begin{equation}
        \mO^c=\bigcap_{n\in \eN}V_n,
    \end{equation}
    ce qui donne immédiatement
    \begin{equation}
        \mO=\bigcup_{n\in \eN}V_n^c
    \end{equation}
    où les \( V_n^c\) sont fermés.
\end{proof}

\begin{corollary}       \label{CORooTWFYooCNMieM}
    Si \( X\) est un espace topologique métrisable, alors \( X\) accepte une base dénombrable de topologie autour de chaque point.
\end{corollary}

\begin{proof}
    Il s'agit seulement de remarquer que les singletons sont fermés et d'appliquer la proposition~\ref{PROPooXWBTooCvGLOj}.
\end{proof}

%+++++++++++++++++++++++++++++++++++++++++++++++++++++++++++++++++++++++++++++++++++++++++++++++++++++++++++++++++++++++++++
\section{Suites de Cauchy, métrique et espaces complets}
%+++++++++++++++++++++++++++++++++++++++++++++++++++++++++++++++++++++++++++++++++++++++++++++++++++++++++++++++++++++++++++

%---------------------------------------------------------------------------------------------------------------------------
\subsection{Généralités}
%---------------------------------------------------------------------------------------------------------------------------

\begin{definition}[Suite de \( \tau\)-Cauchy, espace vectoriel topologique\cite{TQSWRiz,ooMKWJooLSkGfh}]   \label{DefZSnlbPc}
    Soit \( E\) un espace vectoriel topologique. Une suite \( (x_k)\) dans \( E\) est une \defe{suite \( \tau\)-Cauchy}{suite!de Cauchy} si pour tout voisinage \( \mU\) de \( 0\) il existe \( N\in \eN\) tel que \( x_k-x_l\in\mU\) pour tout \( k,l\geq N\).
\end{definition}

\begin{definition}[Espace \( \tau\)-complet]      \label{DEFooVQDBooNxprFU}
    Nous disons qu'une partie \( A\) d'un espace vectoriel topologique est \defe{\( \tau\)-complet}{complet!espace topologique} si toute suite \(  \tau\)-Cauchy d'éléments de \( A\) converge\footnote{Définition~\ref{DefXSnbhZX}.} vers un élément de \( A\).
\end{definition}

\begin{definition}[Suite de Cauchy, espace métrique]      \label{DEFooXOYSooSPTRTn}
    Une suite \( (a_k)\) dans un espace métrique \( (V,d)\) est \defe{de Cauchy}{suite!de Cauchy} si pour tout \( \epsilon\in \eR\), il existe \( N\) tel que si \( n,m\geq N\) alors \( d(a_n,a_m)<\epsilon\).
\end{definition}

Notons qu'ici, même si l'espace \( V\) n'a rien à voir avec \( \eR\), nous prenons \( \epsilon\) dans \( \eR\) et la distance à valeurs dans \( \eR\). Cela semble une évidence, mais il faut se rendre compte que \( \eR\) commence à prendre une place centrale dans nos constructions. Ce n'était pas le cas du temps où nous parlions de suites de Cauchy et de complétude dans des corps totalement ordonnés (définitions~\ref{DefKCGBooLRNdJf}). Dans ce contexte, le \( \epsilon\) était pris dans le corps lui-même.

\begin{definition}[Métrique complète]       \label{DEFooHBAVooKmqerL}
    Soit \( (E,d)\) un espace métrique. Nous disons que la métrique \( d\) est \defe{complète}{complet!métrique} si toute suite de Cauchy dans \( (E,d)\) converge dans \( E\).
\end{definition}

\begin{normaltext}
    Ces définitions méritent quelques remarques.
    \begin{enumerate}
        \item
            Dans le cas des espaces vectoriels topologiques, nous définissons les notions de suite \( \tau\)-Cauchy et d'espace topologique \( \tau\)-complet. Nous ajoutons le préfixe \( \tau\) pour indiquer que ce sont des notions topologiques.
        \item
            Dans le cas des espaces métriques, nous définissons la notion de \emph{métrique} complète. C'est bien la métrique qui est complète, et non l'espace. En effet nous allons voir dans l'exemple \ref{EXooNMNVooXyJSDm} que le même espace topologique peut accepter plusieurs distances différentes (donnant la même topologie) donnant lieu à des suites de Cauchy différentes.
        \item
            Si un espace vectoriel a une topologie issue d'une distance, rien ne dit que ses suites \( \tau\)-Cauchy et ses suites de Cauchy sont les mêmes. Ce sont deux notions à priori séparées. Si \( V\) est un espace vectoriel topologique que l'on peut munir de deux distances \( d_1, d_2\) donnant toutes deux la topologie, dire que \( V\) est \( \tau\)-complet, dire que \( d_1\) est complète et dire que \( d_2\) est complète sont trois choses différentes. Même si les trois topologies sont identiques.
        \item
            Nous allons bien entendu voir que dans de larges gammes d'exemples, les notions de suite de Cauchy et \( \tau\)-Cauchy coincident.
    \end{enumerate}
\end{normaltext}

\begin{definition}  \label{DefVKuyYpQ}
    Un \defe{espace de Banach}{espace!Banach}\index{Banach!espace} est un espace vectoriel normé complet\footnote{Définition \ref{DEFooHBAVooKmqerL}.} pour la topologie de la norme.
\end{definition}

\begin{example}[La complétude n'est pas une propriété topologique\cite{ooSCDYooWutzzr}]     \label{EXooNMNVooXyJSDm}
    Le fait pour un espace d'être complet n'est pas une propriété topologique, mais une propriété métrique. Plus exactement, il existe des espaces topologiques isomorphes, mais dont l'un est complet et l'autre non.

    Nous considérons la distance suivante sur \( \eN\) :
    \begin{equation}
        d_1(x,y)=\Bigl| \frac{1}{ x }-\frac{1}{ y } \Bigr|.
    \end{equation}
    Pour vérifier que cette formule définit bien une distance (définition~\ref{DefMVNVFsX}), le seul point non immédiat est l'inégalité triangulaire :
    \begin{equation}
        d_1(x,y)=\Bigl| \frac{1}{ x }-\frac{1}{ y } \Bigr|\leq\Bigl| \frac{1}{ x }-\frac{1}{ z } \Bigr|+\Bigl| \frac{1}{ z }-\frac{1}{ y } \Bigr|=d_1(x,z)+d_1(z,y).
    \end{equation}

    Au niveau de la topologie induite par cette distance, c'est la topologie discrète. En effet, soit \( x\in \eN\) et \( \epsilon>0\); nous voulons déterminer la boule \( B(x,\epsilon)\) en résolvant l'équation
    \begin{equation}
        \Bigl| \frac{1}{ x }-\frac{1}{ y } \Bigr|<\epsilon
    \end{equation}
    pour \( y\in \eN\). Nous trouvons que $\frac 1 y > \frac 1 x  - \epsilon$ et $\frac 1 y < \frac 1 x + \epsilon$, soit
    \begin{subequations}
        \begin{numcases}{}
            y > \frac 1 {\frac 1 x  + \epsilon}\\
            y < \frac 1 {\frac 1 x - \epsilon}.
        \end{numcases}
    \end{subequations}
    Si \( \epsilon \) est assez petit, la seule solution entière est \( y=x\). Les ouverts sont donc toutes les parties parce que tous les singletons sont ouverts.

    L'espace topologique associé à \( (\eN,d_1)\) est donc la topologie discrète\footnote{Celle dont toutes les parties sont des ouverts.}.

    Si nous considérons par contre la distance usuelle sur \( \eN\), à savoir \( d(x,y)=| x-y |\), nous obtenons encore la topologie discrète. Nous avons donc un isomorphisme d'espaces topologiques
    \begin{equation}
        (\eN,d)\simeq (\eN,d_1).
    \end{equation}
    Nous pouvons même donner un isomorphisme explicite : \( f(n)=n\).

    La suite \( (x_n)=n\) est une suite de Cauchy dans \( (\eN,d_1)\) parce que si \( \epsilon>0\) est donné, il suffit de prendre \( N\) assez grand pour avoir \( \frac{1}{ N }<\epsilon\) (possible par le lemme~\ref{LemooHLHTooTyCZYL}) nous avons, pour \( n,m>N\) :
    \begin{equation}
        \Bigl| \frac{1}{ n }-\frac{1}{ m } \Bigr|<\frac{1}{ n }<\frac{1}{ N }<\epsilon.
    \end{equation}
    Or cette suite ne converge pas. Soit en effet un candidat limite \( k\). Calculons
    \begin{equation}
        d_1(x_n,k)= \Bigl| \frac{1}{ n }-\frac{1}{ k } \Bigr |\to \frac{1}{ k }\neq 0.
    \end{equation}
    L'espace \( (\eN,d_1)\) n'est pas complet.

    Notons que cette suite n'est pas de Cauchy dans \( (\eN,d)\).

    En résumé :
    \begin{enumerate}
        \item
            Les espaces topologiques \( (\eN,d)\) et \( (\eN,d_1)\) sont isomorphes.
        \item
            Ils ont les mêmes notions de suites convergentes : une suite convergente pour l'un est convergente pour l'autre.
        \item
            Ils n'ont pas les mêmes notions de suites de Cauchy.
        \item
            Dans \(  (\eN,d_1)  \), il existe des suites de Cauchy qui ne convergent pas (pas complet).
        \item
            L'espace \( (\eN,d)\) est complet, mais \( (\eN,d_1)\) n'est pas complet.
        \item
            Le fait pour un espace topologique métrique d'être complet n'est pas intrinsèque à sa topologie : la complétude est une propriété de la distance. La complétude est une propriété de la métrique, et non de la topologie qui s'en suit.
    \end{enumerate}
\end{example}

%---------------------------------------------------------------------------------------------------------------------------
\subsection{Espace topologique métrique}
%---------------------------------------------------------------------------------------------------------------------------

Dans les espaces vectoriels topologiques métriques, il n'y a pas d'ambiguïté.
\begin{proposition}[Caractérisations avec la distance \( d \)]     \label{PropooUEEOooLeIImr}
    Soit \( (E,d)\) un espace vectoriel topologique métrique.
    \begin{enumerate}
        \item   \label{ItemooROYMooAQCXnj}
            Une suite \( (x_n)\) dans \( E\) est convergente\footnote{Définition~\ref{DefXSnbhZX}.} vers \( x\) si et seulement si pour tout \( \epsilon\in \eR\) il existe \( N_{\epsilon}\) tel que pour tout \( n\geq N_{\epsilon}\) nous avons \( d(x_n,x)\leq \epsilon\).
        \item
            Une suite \( (x_n)\) dans \( E\) est de Cauchy\footnote{Définition~\ref{DefZSnlbPc}.} si pour tout \( \epsilon\in \eR\), il existe un \( N_{\epsilon}\) tel que si \( p,q\geq N_{\epsilon}\), nous avons \( d(x_p,x_q)\leq \epsilon\).
    \end{enumerate}
\end{proposition}

\begin{proof}
   En ce qui concerne la convergence :
    \begin{subproof}
        \item[Sens direct]

            Nous supposons que \( x_k\to x\) dans \( E\). Soit \( \epsilon>0\); vu que \( B(x,\epsilon)\) est un ouvert contenant \( x\), il existe un \( N_{\epsilon}>0 \) tel que \( k>N_{\epsilon}\) implique \( x_k\in B(x,\epsilon)\). Cela signifie \( d(x,x_k)\leq \epsilon\).

        \item[Réciproque]

            Nous supposons que pour tout \( \epsilon>0\), il existe \( N_{\epsilon}>0\) tel que si \( k>N_{\epsilon}\) alors \( x_k\in B(x,\epsilon)\). Soit un ouvert \( \mO\) autour de \( x\). Nous sommes dans un espace métrique; ergo la topologie est donné par le théorème~\ref{ThoORdLYUu} et en particulier la liste des ouverts est donnée par \eqref{EqGDVVooDZfwSf}. Il existe donc une boule \( B(x,\epsilon)\) incluse à \( \mO\). Pour tout \( k>N_{\epsilon}\) nous avons alors \( x_k\in B(x,\epsilon)\subset\mO\).
    \end{subproof}
    En ce qui concerne les suites de Cauchy :
    \begin{subproof}
    \item[Sens direct]
        Si \( (x_n)\) est une suite de Cauchy et si \( \epsilon>0\) est donné, alors \( B(0,\epsilon)\) est un voisinage de \( 0\) et il existe \( N_{\epsilon}\) tel que si \( p,q\geq N_{\epsilon}\) alors \( x_p-x_q\in B(0,\epsilon)\). Posons \( u=x_p-x_q\); en utilisant l'invariance par translation (lemme~\ref{LEMooWGBJooYTDYIK}\ref{ITEMooLITDooPeReOk}) nous avons
        \begin{equation}
            d(u,0)=d(x_p-x_q,0)=d(x_p,x_q).
        \end{equation}
        Par conséquent \( d(x_p,x_q)\leq \epsilon\).
    \item[Réciproque]
        Soit \( \mO\) un voisinage de \( 0\). Il existe \( \epsilon\) tel que \( B(0,\epsilon)\subset \mO\). Par hypothèse il existe \( N_{\epsilon}\) tel que \( d(x_p,x_q)\leq \epsilon\) dès que \( p,q\geq N_{\epsilon}\). En utilisant encore l'invariance par translation nous avons
        \begin{equation}
            d(x_p,x_q)=d(x_p-x_q,0),
        \end{equation}
        et comme cela est plus petit que \( \epsilon\), nous avons \( x_p-x_q\in B(0,\epsilon)\subset\mO\).
    \end{subproof}
\end{proof}

\begin{proposition}[\cite{IRWFPQB}]     \label{PROPooZZNWooHghltd}
    Toute suite convergente dans un espace métrique est de Cauchy.
\end{proposition}

\begin{proof}
    Nous utilisons les caractérisations de la proposition~\ref{PropooUEEOooLeIImr} des suites convergentes et de Cauchy.

    Soit un espace métrique \( (X,d)\) et \( x_n\to\ell\) une suite convergente. Si \( \epsilon>0\), la proposition~\ref{PropooUEEOooLeIImr}\ref{ItemooROYMooAQCXnj}, dit qu'il existe \( N\) tel que pour tout \( n>N\) nous ayons \( d(x_n,\ell)<\epsilon\). Par conséquent si \( n,m>N\) alors
    \begin{equation}
        d(x_n,x_m)\leq d(x_m,\ell)+d(l,x_m)\leq 2\epsilon.
    \end{equation}
    Cela prouve que \( (x_n)\) est une suite de Cauchy.
\end{proof}

%+++++++++++++++++++++++++++++++++++++++++++++++++++++++++++++++++++++++++++++++++++++++++++++++++++++++++++++++++++++++++++
\section{Topologie et espace vectoriel}
%+++++++++++++++++++++++++++++++++++++++++++++++++++++++++++++++++++++++++++++++++++++++++++++++++++++++++++++++++++++++++++

%---------------------------------------------------------------------------------------------------------------------------
\subsection{Espace vectoriel topologique}
%---------------------------------------------------------------------------------------------------------------------------

\begin{definition}\label{DefEVTopologique}
  Un espace vectoriel \( V\) sur le corps \( \eK\) muni d'une topologie est un \defe{espace vectoriel topologique}{espace vectoriel!topologique} si
    \begin{enumerate}
        \item
            la somme de deux vecteurs est une application continue\footnote{Naturellement, l'espace \(V \times V \) est muni de la topologie produit.} \( V\times V\to V \); et
        \item
            la multiplication par un scalaire est une application continue\footnote{Naturellement, l'espace \(\eK \times V \) est muni (lui aussi) de la topologie produit.} \( \eK\times V\to V\).
    \end{enumerate}
\end{definition}
On le redit quand même: le corps\footnote{Définition~\ref{DefTMNooKXHUd}} lui-même doit avoir sa topologie. Dans la grande majorité des cas, ce corps est \( \eR\) ou \( \eC\) muni de la topologie usuelle.

Mine de rien, le fait que les deux opérations usuelles soient continues a de belles conséquences sur la topologie de l'espace\dots

\begin{proposition}[\cite{ooMKWJooLSkGfh}]
  Pour \(x \in V \) et \(\lambda \in \eK, \ \lambda \neq 0 \) fixés, les fonctions \( T_x \) et \( M_\lambda \) définies par:
  \begin{align}
    T_x:&V \to V & &\text{et}&M_\lambda:&V \to V\\
    & y \mapsto x+y & & & &y \mapsto \lambda y
  \end{align}
sont des homéomorphismes de \(V \) dans \(V \).
\end{proposition}

\begin{proof}
  Ce sont des bijections continues, dont les inverses sont respectivement \( T_{-x} \) et \( M_{1/\lambda} \).
\end{proof}

\begin{corollary}[Invariance de la topologie~\cite{ooMKWJooLSkGfh}]\label{PropInvarianceTopologie}
  Toute base de voisinage de \( 0 \) se transporte en tout point de l'espace vectoriel topologique.
\end{corollary}

\begin{lemma}[\cite{ooMKWJooLSkGfh}]\label{PropSommeTopologique}
  Soit \( V \) un espace vectoriel topologique, et \( W \) un voisinage de \( 0 \). Il existe \( U \) un voisinage de \( 0 \), symétrique\footnote{C'est-à-dire que, pour tout \( x \in V \), on a \( x \in U \) si et seulement si \( -x \in U \).}, tel que \( U + U = W \).
\end{lemma}

\begin{proof}
  Par continuité de l'addition et par la définition de la topologie produit, il existe \(U_1 \) et \(U_2 \) tels que \( U_1 + U_2 \subset W \). En posant \( U = U_1 \cap U_2 \cap (-U_1) \cap (-U_2) \), on a un sous-ensemble symétrique de \( U_1\) et \(U_2\), si bien que \( U + U = W \).
\end{proof}

\begin{definition}      \label{DEFooGTOZooRcvGHg}
    Une distance \( d\) sur un espace vectoriel topologique \( V\) est dite \defe{compatible}{distance!compatible} avec la topologie si la topologie induite\footnote{Définition~\ref{ThoORdLYUu}.} de \( d\) est celle de \( V\).

    Une distance \( d\) sur un espace vectoriel \( V\) est dite \defe{invariante}{distance!invariante} si pour tout \( x,y,u\in V\) nous avons
    \begin{equation}
        d(x+u,y+u)=d(x,y).
    \end{equation}
\end{definition}
Notons que lorsque nous parlons d'une distance compatible avec un espace vectoriel topologique, nous parlons de compatibilité avec la topologie, pas avec la structure vectorielle.

\begin{theorem}[\cite{ooMKWJooLSkGfh}]      \label{THOooAGBXooZnvQLK}
    Si $V$ est un espace vectoriel topologique possédant en tout point une base de topologie dénombrable, alors il existe une distance \( d\) sur \( V\) telle que
    \begin{enumerate}
        \item
            \( d\) est compatible avec la topologie de \( V\),
        \item
            \( d\) est invariante par translation.
    \end{enumerate}
\end{theorem}

\begin{proof}
Grâce à la proposition~\ref{PropInvarianceTopologie}, on peut tout ramener en \(0 \) puis faire les transports en tous les points de l'espace. Mieux: grâce à la proposition~\ref{PropSommeTopologique} (appliquée deux fois de suite), on peut créer une base de voisinage \( (U_n) \) de \( 0 \) telle que pour tout \(n \in \eN\),
\begin{equation}\label{EqBaseTopoMetriquePf1}
  U_{n+1} + U_{n+1} + U_{n+1} + U_{n+1} \subset U_n.
\end{equation}
Pour tous entiers naturels \(n\) et \(k\), on obtient alors
\begin{equation}\label{EqBaseTopoMetriquePf2}
  U_{n+1} + U_{n+2} + \cdots  + U_{n+(k-1)} + U_{n+k} \subset  U_{n+1} + U_{n+1} \subset U_n.
\end{equation}

On construit à présent, pour tout \( n \in \eN \), l'ensemble
\begin{equation}
  D_n = \Bigl\{\sum_{i=1}^n \frac {c_i} {2^i} \tq \forall i = 1,\cdots, n, c_i \in \{0,1\}\Bigr\},
\end{equation}
et \( D = \cup_{n>0} D_n\). Ensuite, définissons \(\phi \) sur \(D \cup \mathopen[1,+ \infty\mathclose[ \) et à valeurs dans les parties de \( V \):
\begin{equation}
  \phi (r) =
  \begin{cases}
    V & \text{si } r \geq 1;\\
    c_1 U_1 + \cdots + c_n U_n & \text{si } r \in D_n.
  \end{cases}
\end{equation}
Quelques remarques sur cette fonction.
\begin{enumerate}
  \item \emph{\(\phi(r) + \phi(s) \subset \phi(r+s)\) :} Si déjà \(r + s \geq 1 \), c'est clair. Sinon, on se place dans \( D_n \) avec le \(n \) qui va bien -- de telle sorte que \(r,\ s\) et \(r+s\) soient dedans. Notons:
    \begin{gather}
      r = \sum_{i=1}^n \frac {r_i}{2^i};\\
      s = \sum_{i=1}^n \frac {s_i}{2^i};\\
      r+s = \sum_{i=1}^n \frac {t_i}{2^i}.
    \end{gather}
    Deux cas se produisent. Si pour tout \(i\), \( t_i = r_i + s_i\), alors
    \begin{equation}
      \phi(r+s) = \sum_i t_i U_i = \sum_i r_i U_i + \sum_i s_i U_i = \phi(r) + \phi(s);
    \end{equation}
    l'égalité a lieu car \(r_i\) et \(s_i\) ne peuvent jamais valoir \(1\) en même temps.

    Sinon, posons \(k \) le plus petit entier tel que \( t_k \neq r_k + s_k\). Alors, nécessairement, \( r_k = 0,\ s_k = 0\) et \( t_k = 1\). Il s'ensuit, grâce à \eqref{EqBaseTopoMetriquePf1} et \eqref{EqBaseTopoMetriquePf2}, que
    \begin{gather}
      \phi(r) = \sum_{i=1}^{k-1} r_i V_i + \sum_{i=k+1}^n r_i V_i \subset \sum_{i=1}^{k-1} r_i V_i + V_{k+1}+ V_{k+1};\\
      \phi(s) = \sum_{i=1}^{k-1} s_i V_i + \sum_{i=k+1}^n s_i V_i \subset \sum_{i=1}^{k-1} s_i V_i + V_{k+1}+ V_{k+1};\text{ d'où}\\
      \phi(r)+\phi(s) = \sum_{i=1}^{k-1} r_i V_i + \sum_{i=1}^{k-1} s_i V_i + V_{k+1}+ V_{k+1} + V_{k+1}+ V_{k+1} = \sum_{i=1}^{k-1} t_i V_i + V_k \subset \phi(r+s).
    \end{gather}
  \item \emph{\(0  \in \phi(r)\) pour tout \(r\):} en effet, \(\phi(r)\) n'est jamais vide, c'est toujours un voisinage de \(0\).
  \item \label{PhiEstTotalementOrdonne} \emph{si \(r < s \) alors \( \phi(r) \subset \phi(s) \):} il suffit d'écrire
    \begin{equation}
      \phi(r) \subset \phi(r) + \phi(s-r) \subset \phi(s).
    \end{equation}
\end{enumerate}
Enfin, on définit
\begin{equation}\label{EqDefDistanceCompatible}
d(x,y) = \inf\{r \in \mathopen[0;1\mathclose] \tq y - x \in \phi(r)\}.
\end{equation}
Il suffit alors de voir que \(d\) convient. De par sa définition, il est clair qu'elle est invariante par translation; reste à voir que c'est bien une distance, et qu'elle est compatible avec la topologie.
\begin{subproof}
\item [$d(x,x) = 0$] Oui, car \(0\) est dans \(\phi(r)\), pour tout \(r \), puisque les \( U_i \) sont des voisinages de \(0\).
\item[$d(x,y) = d(y,x)$] Oui, car tous les voisinages considérés sont symétriques: pour tout \(i\) et tout \(x \in V\), on a \(x \in U_i\) si et seulement si  \(-x \in U_i\).
\item[$d(x,z) \leq d(x,y) + d(y,z)$] Soit \(\epsilon > 0 \). Par définition des distances comme infimums, et grâce au corolaire~\ref{CorDensiteDyadiques}, il existe \(r\) et \(s\) dans \( D \) tels que:
  \begin{equation}
    d(x,y) < r < d(x,y) + \frac \epsilon 2 \quad\text{ et }\quad d(y,z) < s < d(y,z) + \frac \epsilon 2 .
  \end{equation}
  Comme \(d(x,y) + d(y,z) <  r + s\), et par la remarque~\ref{PhiEstTotalementOrdonne} sur \(\phi\), on a \(y - x \in \phi(r)\) et \(z - y \in \phi(s)\); donc
  \begin{equation}
    (y - x) + (z - y) = z - x \in \phi(r) + \phi(s) \subset \phi(r+s)
  \end{equation}
  Ainsi, pour tout \(\epsilon > 0 \), on a
  \begin{equation}
    d(x,z) \leq r+s < d(x,y) + d(y,z) + \epsilon.
  \end{equation}

\item[Compatibilité avec la topologie] Si \(d(0,y) < r \), alors \( y  \in \phi(r) \); en particulier pour \( r = 1/2^k \), on a \(y \in \phi(r) = V_k\). D'où, pour tout \( n \in \eN,\ B(0, 1/2^n) \subset V_n \).
\end{subproof}
\end{proof}

\begin{proposition}     \label{PROPooPRLBooGtsRjr}
    Un espace vectoriel topologique\footnote{Définition \ref{DefEVTopologique}.} est métrisable si et seulement si il possède en tout point une base dénombrable de topologie.
\end{proposition}

\begin{proof}
    Il s'agit seulement de mettre bout à bout les corolaires~\ref{CORooTWFYooCNMieM} et théorème~\ref{THOooAGBXooZnvQLK}.
\end{proof}

%---------------------------------------------------------------------------------------------------------------------------
\subsection{Équivalence entre Cauchy et \texorpdfstring{$\tau-$}{tau-}Cauchy}
%---------------------------------------------------------------------------------------------------------------------------

\begin{lemma}       \label{LEMooIAHSooFkXjvr}
    Soit un espace vectoriel topologique\footnote{Définition~\ref{DefEVTopologique}.} \( V\) et une distance \( d\colon V\times V\to \eR^+\) compatible\footnote{Définition~\ref{DEFooGTOZooRcvGHg}.} avec la topologie de \( V\). Si \( d\) est invariante\footnote{Définition~\ref{DEFooGTOZooRcvGHg}.}, alors les suites de Cauchy pour \( d \) et les suites \( \tau\)-Cauchy sont les mêmes.
\end{lemma}

\begin{proof}
    Nous avons deux implications à prouver. 
    \begin{subproof}
    \item[Cauchy pour \( d\) implique \( \tau\)-Cauchy]
        Soit \( (x_n)\), une suite de Cauchy dans \( V\) pour \( d\), et un voisinage \( U\) de \( 0\). Vu que \( d\) est compatible avec la topologie de \( V\), il existe une boule ouverte \( B(0,\epsilon)\) incluse à \( U\). Soit \( N>0\) tel que \( m,n>N\) implique \( d(x_n,x_m)<\epsilon\). Par invariance de la métrique, nous avons aussi
        \begin{equation}
            d(0,x_m-x_n)<\epsilon,
        \end{equation}
        c'est-à-dire \( x_m-x_n\in B(0,\epsilon)\subset U\). La suite \( (x_n)\) est donc \( \tau\)-Cauchy.
    \item[\( \tau\)-Cauchy implique Cauchy pour \( d\)]
        Soit $(x_n)$, une suite \( \tau\)-Cauchy dans \( V\) et \( \epsilon>0\). Vu que \( B(0,\epsilon)\) est un voisinage de \( 0\) dans \( V\), il existe \( N\) tel que \( m,n>N\) implique \( x_n-x_m\in B(0,\epsilon)\). Cela signifie que \( d(0,x_n-x_m)<\epsilon\) et toujours par invariance, que \( d(x_n,x_m)<\epsilon\).
    \end{subproof}
\end{proof}

Tout ceci nous mène à donner une large classe d'espaces vectoriels topologiques sur lesquelles les notions de suites de Cauchy pour une distance et \( \tau\)-Cauchy coïncident.

\begin{theoremDef}     \label{THOooGQZSooAmQolf}
    Soit \( V\) un espace vectoriel topologique métrisable\footnote{i.e. admet une base dénombrable de topologie, voir la proposition~\ref{PROPooPRLBooGtsRjr}}, alors il admet une métrique \( d\) compatible avec la topologie telle que une suite dans \( V\) est de Cauchy pour \( d\) si et seulement si elle est \( \tau\)-Cauchy.

    Une \defe{suite de Cauchy}{Cauchy!suite} dans un espace vectoriel métrique \( (E,d)\) est une suite \( \tau\)-Cauchy ou de Cauchy pour \( d \).
\end{theoremDef}

\begin{proof}
    Soit \( d\) une métrique sur \( V\) satisfaisant au théorème~\ref{THOooAGBXooZnvQLK}. Vu qu'elle est invariante par translation, les suites \( d\)-Cauchy sont exactement les suites \( \tau\)-Cauchy par le lemme~\ref{LEMooIAHSooFkXjvr}.
\end{proof}

\begin{remark}  \label{REMooUFQYooUVCCjs}
    Même si \( V\) est métrisable, si on choisit la métrique n'importe comment, on ne peut rien espérer.
\end{remark}

\begin{normaltext}
    Sur les espaces vectoriels topologiques métrisables, nous pouvons donc parler de suite de Cauchy sans préciser si nous parlons de \( \tau\)-Cauchy ou de \( d\)-Cauchy, parce que nous sous-entendons avoir choisi une métrique non seulement compatible avec la topologie, mais également invariante par translation.

    Il reste cependant à traiter le cas d'un espace vectoriel topologique non métrisable. Dans ce cas, il n'y a pas de métrique, et la question de l'équivalence des définitions ne se pose pas.
\end{normaltext}

Le théorème suivant donne la complétude de \( \eR\) et le critère de Cauchy pour les définitions métriques et topologiques usuelles. Lorsqu'on dit que \( \eR\) est complet, le plus souvent nous parlons de ce théorème, et non de~\ref{THOooUFVJooYJlieh} qui en est un lemme indispensable mais qui parle de notions différentes, bien que très liées.
\begin{theorem}[Complétude de \( \eR\), critère de Cauchy\cite{RWWJooJdjxEK}]       \label{THOooNULFooYUqQYo}
    Nous avons :
    \begin{enumerate}
        \item
            L'espace métrique \( (\eR,d)\) est complet (définition~\ref{DEFooHBAVooKmqerL}).
        \item       \label{ITEMooUUFCooIVtGgz}
            Une suite dans \( \eR\) est convergente (définition~\ref{DefXSnbhZX}) si et seulement si elle est de Cauchy (définition~\ref{THOooGQZSooAmQolf}).
    \end{enumerate}
\end{theorem}
\index{complet!$\eR$!espace métrique}
\index{critère!de Cauchy}

\begin{proof}
    Tout ce théorème se base sur le fait que la définition de suite de Cauchy dans \( (\eR,d)\) et de suite convergente dans \( (\eR,d)\) coïncident avec les définitions correspondantes dans \( \eR\) vu comme simple corps ordonné (définitions~\ref{DefKCGBooLRNdJf}).

    Donc si \( (x_n)\) est de Cauchy dans \( (\eR,d)\), elle est de Cauchy dans le corps ordonné \( (\eR,\leq)\). Donc le théorème~\ref{THOooUFVJooYJlieh} nous dit que \( (x_n)\) est convergente dans \( (\eR,\leq)\). Et donc convergente dans \( (\eR,d)\).

    Toutes les autres affirmations se prouvent de la même manière.
\end{proof}

Si vous n'êtes pas sûr ou si vous ne voulez pas étudier les notations de convergence et de suites de Cauchy dans les corps, vous pouvez simplement recopier la démonstration du théorème~\ref{THOooUFVJooYJlieh} en remplaçant partout \( \eQ\) par \( \eR\), et aussi en remplaçant les \( | x-y |\) par \( d(x,y)\).

\begin{normaltext}
    Nous pouvons également mettre une structure d'espace métrique sur \( \eC\) en posant
    \begin{equation}
        d(z,z')=| z-z' |.
    \end{equation}
\end{normaltext}

\begin{proposition}
    L'espace métrique \( (\eC,d)\) est complet.
\end{proposition}

\begin{proof}
    Commençons par nous rendre compte que pour tout \( z\in \eC\) nous avons \( | \real(z) |\leq | z |\). C'est bon ? Vous vous en êtes rendu compte ? Ok. Continuons.

    Soit une suite de Cauchy \( (z_k)\) dans \( \eC\) et \( \epsilon>0\). Si \( x_k=\real(z_j)\), nous avons
    \begin{equation}
        | x_k-x_l |=| \real(z_k-z_l) |\leq | z_k-z_l |.
    \end{equation}
    Vu que \( (z_k)\) est de Cauchy, il existe un \( N\) tel que si \( k,l\geq N\),
    \begin{equation}
        | x_k-x_l |\leq | z_k-z_l |\leq \epsilon.
    \end{equation}

    Donc la suite des parties réelles converge par la complétude de \( (\eR,d)\) du théorème~\ref{THOooNULFooYUqQYo}. Notez que le \( d\) ici n'est pas tout à fait le même, et que la démonstration fonctionne parce que la distance prise sur \( \eR\) est la restriction à \( \eR\) de la distance prise sur \( \eC\). Notons \( x\) la limite de \( (x_k)\).

    De la même manière la suite des parties imaginaires \( y_k=\imag(z_k)\) converge vers un réel que nous notons \( y\). Avec tout cela, la suite \( z_k\) converge dans \( \eC\) vers \( x+iy\). En effet pour \( \epsilon\) donné et pour un \( k\) suffisament grand,
    \begin{equation}
        | z_k-(x+iy) |=\big| \real(z_k)-x+i(\imag(z_k)-y) \big|\leq | x_k-x |+| y_k-y |\leq \epsilon.
    \end{equation}
\end{proof}

%+++++++++++++++++++++++++++++++++++++++++++++++++++++++++++++++++++++++++++++++++++++++++++++++++++++++++++++++++++++++++++ 
\section{Norme; espace vectoriel normé}
%+++++++++++++++++++++++++++++++++++++++++++++++++++++++++++++++++++++++++++++++++++++++++++++++++++++++++++++++++++++++++++
\label{SECooWKJNooKOqpsx}

La valeur absolue est essentielle pour introduire les notions de limite et de continuité pour les fonctions d'une variable. Par exemple nous verrons dans la proposition \ref{PROPooVNGEooPwbxXP} que la fonction \( f\colon \eR\to \eR\) est continue en \( a\) si et seulement si pour tout $\varepsilon > 0$, il existe un $\delta > 0$ tel que
  \begin{equation}
    | x-a |\leq\delta \Rightarrow | f(x)-f(a) |\leq \varepsilon.
  \end{equation}
La quantité $| x-a |$ donne la «distance» entre $x$ et $a$; la définition de la continuité signifie que pour tout $\varepsilon$, il existe un $\delta$ tel que si $a$ et $x$ sont au plus à la distance $\delta$ l'un de l'autre, alors $f(x)$ et $f(a)$ ne seront éloignés au plus d'une distance $\varepsilon$.

La valeur absolue, dans $\eR$, nous sert donc à mesurer des distances entre les nombres. Les principales propriétés de la valeur absolue sont :
\begin{enumerate}

	\item
		$| x |=0$ implique $x=0$,
	\item
		$| \lambda x |=| \lambda | |x |$,
	\item
		$| x+y |\leq | x |+| y |$

\end{enumerate}
pour tout $x,y\in\eR$ et $\lambda\in\eR$.

Afin de donner une notion de limite pour les fonctions de plusieurs variables, nous devons trouver un moyen de définir les notions de «taille» d'un vecteur et de distance entre deux points de $\eR^n$, avec $n>1$. La notion de «taille» doit satisfaire propriétés analogues à celles de la valeur absolue.

La première notion de «taille» pour un vecteur de $\eR^2$ que nous vient à l'esprit est la longueur du segment entre l'origine et l'extrémité libre du vecteur. Cela peut être calculée à l'aide du théorème de Pythagore :
\begin{equation}
  \textrm{taille de } (a,b) = \sqrt{a^2+b^2}.
\end{equation}
Nous pouvons introduire une notion de distance entre les éléments de $\eR^2$ de façon similaire :
\begin{equation}
	d\big((a_x,a_y),(b_x,b_y)\big)=\sqrt{  (a_x-b_x)^2+(a_y-b_y)^2  }.
\end{equation}
Cette définition a l'air raisonnable; est-elle mathématiquement correcte ? Peut-elle jouer le rôle de la valeur absolue dans $\eR^2$ ? Est-elle la seule définition possibles de «taille» et distance en $\eR^2$ ?

Nous voulons formaliser les notions de «taille» et de distance dans $\eR^n$, et plus généralement dans un espace vectoriel $V$ de dimension finie. Pour cela nous nous inspirons des propriétés de la valeur absolue.



\subsubsection{Critère de Cauchy}
%///////////////////////////////

\begin{lemma}
    Une suite de Cauchy\footnote{Définition \ref{DEFooXOYSooSPTRTn}.} dans un espace vectoriel normé admettant une sous-suite convergente est elle-même convergente vers la même limite.
\end{lemma}

\begin{proof}
    Soit \( (a_n)\) une suite de Cauchy dans un espace vectoriel normé \( E\) et \( \ell\) la limite d'une sous-suite de \( (a_n)\). Soit \( \epsilon>0\) et \( N\in \eN\) tel que \( \| a_m-a_p \|<\epsilon\) dès que \( m,p\geq N\). Nous allons montrer que si \( k>N\) alors \( \| a_k-\ell \|<2\epsilon\). Pour cela nous considérons un \( n>N\) tel que \( \| a_n-\ell \|\leq \epsilon\) et nous calculons
    \begin{equation}
        \| a_k-\ell \|\leq \| a_k-a_n \|+\| a_n-\ell \|\leq 2\epsilon.
    \end{equation}
\end{proof}

Dans le cas des espaces de dimension finie, le fait d'être complet est automatique, comme le montre la proposition suivante.
\begin{proposition}     \label{PROPooGJDTooXOoYfw}
    Soit \( \big( E,\| . \| \big)\) un espace vectoriel normé de dimension finie sur un corps \( \eK\) qui est complet\footnote{La définition est~\ref{DefKCGBooLRNdJf}, mais si vous n'avez pas envie de vous embarquer trop loin, dites juste «toutes les suites de Cauchy convergent». Typiquement c'est \( \eR\) ou \( \eC\).}. Alors \( E\) est complet\footnote{Définition~\ref{DEFooHBAVooKmqerL}.}.
\end{proposition}
Pour rappel, la complétude de l'espace métrique \( \eR\) est la proposition~\ref{PROPooTFVOooFoSHPg}.

\begin{proof}
    Nous considérons une suite de Cauchy \( (f_n)\) dans \( E\) et si \( \{ e_{\alpha} \} \) est une base orthonormée de \( E\) nous définissons les coefficients \( f_n=\sum_{\alpha}a_{n\alpha}e_{\alpha} \). La somme sur \( \alpha\) est finie par hypothèse sur la dimension de \( E\).

    Nous avons
    \begin{equation}
        \| f_n-f_m \|=\| \sum_{\alpha}(a_{n\alpha}-a_{m\alpha})e_{\alpha} \|=\sum_{\alpha}| a_{n\alpha}-a_{m\alpha} |^2.
    \end{equation}
    Pour tout \( \epsilon\), il existe \( N\) tel que si \( m,n>N\) alors \( | a_{n\alpha}-a_{m\alpha} |<\sqrt{ \epsilon }\). Autrement dit, pour chaque \( \alpha\), la suite \( (a_{n\alpha})_{\alpha\in \eN}\) est de Cauchy dans \( \eK\) et converge donc dans \( \eK\). Soit \( a_{\alpha}\) la limite et définissons \( f=\sum_{\alpha}a_{\alpha}e_{\alpha}\). Nous avons alors
    \begin{equation}
        \| f_n-f \|=\| \sum_{\alpha}(a_{n\alpha}-a_{\alpha})e_{\alpha} \|,
    \end{equation}
    dont la limite \( n\to \infty\) est bien zéro. Donc la suite \( (f_n)\) converge vers \( f\in E\). L'espace \( E\) est alors complet.
\end{proof}




\begin{proposition}		\label{PropContinueCompactBorne}
	Soient $V$ et $W$ deux espaces vectoriels normés. Soient $K$ une partie compacte de $V$ et $f\colon K\to W$ une fonction continue. Alors l'image $f(K)$ est compacte dans $W$.
\end{proposition}
Ce résultat est démontré dans un cadre plus général par le théorème~\ref{ThoImCompCotComp}.

\begin{proof}
	Nous allons prouver que $f(K)$ est fermée et bornée.
    \begin{subproof}
		\item[$f(K)$ est fermé] Nous allons prouver que si $(y_n)$ est une suite convergente contenue dans $f(K)$, alors la limite est également contenue dans $f(K)$. Dans ce cas, nous aurons que l'adhérence de $f(K)$ est contenue dans $f(K)$ et donc que $f(K)$ est fermé. Pour chaque $n\in\eN$, le vecteur $y_n$ appartient à $f(K)$ et donc il existe un $x_n\in K$ tel que $f(x_n)=y_n$. La suite $(x_n)$ ainsi construite est une suite dans le fermé $K$ et possède donc une sous-suite convergente (proposition~\ref{THOooRDYOooJHLfGq}). Notons $(x'_n)$ cette sous-suite convergente, et $a$ sa limite : $\lim(x'_n)=a\in K$. Le fait que la limite soit dans $K$ provient du fait que $K$ est fermé.

			Nous pouvons considérer la suite $f(x'_n)$ dans $W$. Cela est une sous-suite de la suite $(y_n)$, et nous avons $\lim f(x'_n)=a$ parce que $f$ est continue. Par conséquent nous avons
			\begin{equation}
				f(a)=\lim f(x'_n)=\lim y_n.
			\end{equation}
			Cela prouve que la limite de $(y_n)$ est dans $f(K)$ et par conséquent que $f(K)$ est fermé.

		\item[$f(K)$ est borné]
			Si $f(K)$ n'est pas borné, nous pouvons trouver une suite $(x_n)$ dans $K$ telle que
			\begin{equation}		\label{EqfxnWgeqn}
				\| f(x_n) \|_W>n
			\end{equation}
			Mais par ailleurs, l'ensemble $K$ étant compact (et donc fermé), nous avons une sous-suite $(x'_n)$ qui converge dans $K$. Disons $\lim(x'_n)=a\in K$.

			Par la continuité de $f$ nous avons alors $f(a)=\lim f(x'_n)$, et donc
			\begin{equation}
				| f(a) |=\lim | f(x'_n) |.
			\end{equation}
			La suite $f(x'_n)$ est alors une suite bornée, ce qui n'est pas possible au vu de la condition \eqref{EqfxnWgeqn} imposée à la suite de départ $(x_n)$.
    \end{subproof}
\end{proof}

\begin{corollary}	\label{CorFnContinueCompactBorne}
	Si $f\colon K\to \eR$ est une application continue où $K$ est une partie compacte d'un espace vectoriel normé, alors \( f(K)\) est borné.
\end{corollary}

\begin{proof}
	En effet, la proposition~\ref{PropContinueCompactBorne} montre que $f(K)$ est compact et donc borné.
\end{proof}

% TODO: regarder ceci à propos des compacts.
% En particulier, si on recouvre $A$ par l'ensemble des boules
% $B(x,1)$ où $x$ parcourt $A$ (de sorte que tout point de $A$ est
% dans « sa » boule, et donc la réunion des boules recouvre bien
% $A$), on doit pouvoir en tirer un recouvrement fini, c'est-à-dire
% des boules $B(x_1,1), B(x_2,1), \ldots, B(x_k,1)$ (avec $k$ un
% naturel) dont la réunion contient $A$.

% Il me semble que c'est le coup qu'il ne faut vérifier le sous-recouvrement que pour des recouvrements composés d'ouverts issus d'une base donnée de la topologie.
