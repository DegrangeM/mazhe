% This is part of Mes notes de mathématique
% Copyright (c) 2011-2020
%   Laurent Claessens
% See the file fdl-1.3.txt for copying conditions.

%+++++++++++++++++++++++++++++++++++++++++++++++++++++++++++++++++++++++++++++++++++++++++++++++++++++++++++++++++++++++++++
\section{Généralités}
%+++++++++++++++++++++++++++++++++++++++++++++++++++++++++++++++++++++++++++++++++++++++++++++++++++++++++++++++++++++++++++

\begin{normaltext}      \label{NORMooGPWRooIKJqqw}
    Nous trouvons parfois le terme \defe{anneau à division}{anneau!à division}. Cela provient du fait que dans beaucoup de cas on considère uniquement des corps commutatifs; donc on voudrait une façon de parler d'un anneau dont tous les éléments non nuls sont inversibles. Dans ce cadre on dit :
    \begin{itemize}
        \item Un anneau à division est un anneau dont tous les éléments non nuls sont inversibles,
        \item Un corps est un anneau à division commutatif.
    \end{itemize}
    Pour prendre un exemple de cette différence, le théorème de Wedderburn~\ref{ThoMncIWA} est énoncé ici sous les termes «Tout corps fini est commutatif». Sous-entendu : la commutativité ne fait pas partie de la définition d'un corps. Par contre dans \cite{KXjFWKA} il est énoncé sous les termes «Tout anneau à division fini est un corps». Chez lui, un corps est toujours commutatif et un anneau à division est ce que nous appelons ici un corps.
\end{normaltext}

%---------------------------------------------------------------------------------------------------------------------------
\subsection{Corps ordonnés}
%---------------------------------------------------------------------------------------------------------------------------

Nous avons vu la définition de corps totalement ordonné en~\ref{DefKCGBooLRNdJf}.

\begin{definition}[\cite{ooTKEHooQuaFuD}]
    Un corps est \defe{formellement réel}{corps!formellement réel} si \( -1\) n'est pas une somme de carrés.
\end{definition}

\begin{proposition}
    Un corps totalement ordonné est formellement réel.
\end{proposition}

\begin{proof}
    Soit un corps totalement ordonné \( (\eK,\leq)\) et \( a\in \eK\) alors \( a^2\geq 0\). En effet si \( a\geq 0\) alors \( a^2=a\times a\geq 0\) directement par la définition~\ref{DefKCGBooLRNdJf}\ref{CONDooBYYDooElXgPO}. Si \( a\leq 0\) alors \( -a\geq 0\) et
    \begin{equation}
        a^2=(-a)^2\geq 0.
    \end{equation}
    Vu que \( -1<0\), il ne peut donc pas être écrit comme un carré. A fortiori comme somme de carrés.
\end{proof}

%---------------------------------------------------------------------------------------------------------------------------
\subsection{Automorphismes de \texorpdfstring{$ \eR$}{R} et \texorpdfstring{$ \eC$}{C}}
%---------------------------------------------------------------------------------------------------------------------------

\begin{proposition}[\cite{ooEKUSooDDDWuT,MonCerveau}]     \label{PROPooLLPMooIVanaO}
    L'identité est l'unique automorphisme du corps \( \eR\).
\end{proposition}

\begin{proof}
    Soit un automorphisme \( \sigma\colon \eR\to \eR\). Comme pour tout automorphisme,
    \begin{equation}
        \sigma(a)=\sigma(1a)=\sigma(1)\sigma(a).
    \end{equation}
    Donc \( \sigma(1)=1\).

    \begin{subproof}
    \item[Identité sur les rationnels]
    De plus
    \begin{equation}
        \sigma(n)=\sigma(1+\ldots +1)=\sigma(1)+\ldots +\sigma(1)=n,
    \end{equation}
    et
    \begin{equation}
        \sigma\left( \frac{1}{ n } \right)+\ldots +\sigma\left( \frac{1}{ n } \right)=\sigma\left( \frac{1}{ n }+\ldots +\frac{1}{ n } \right)=\sigma(1)=1.
    \end{equation}
    Donc \( \sigma(1/n)=1/n\).

    Nous en déduisons que pour tout \( q\in \eQ\), \( \sigma(q)=q\). Cela ne suffit pas pour déduire \( \sigma(x)=x\) pour tout \( x\in \eR\) parce que rien n'indique que \( \sigma\) soit continue.
        \item[Positive sur les positifs]

            Si \( x>0\) alors \( \sigma(x)=\sigma(\sqrt{ x })^2>0\).

        \item[Croissance]

            Si \( x>y\) alors \( x-y>0\) et \( \sigma(x-y)>0\). Cela donne \( \sigma(x)>\sigma(y)\).

        \item[Identité sur les réels]

            Soit un irrationnel \( x\in \eR\) et une suite \( (q_i)\) dans \( \eQ\) qui converge de façon croissante vers \( x\). Soit \( \epsilon>0\) dans \( \eQ\). Il existe \( N\) tel que si \( i>N\) alors \( q_i+\epsilon>x\); en appliquant \( \sigma\) à cette inégalité et en se souvenant que \( \sigma\) est l'identité sur \( \eQ\),
            \begin{equation}
                q_i+\epsilon>\sigma(x).
            \end{equation}
            Mais de plus, \( q_i<x\) donne \( \sigma(q_i)<\sigma(x)\), c'est-à-dire \( q_i<\sigma(x)\). En regroupant ces deux inégalités,
            \begin{equation}        \label{EQooLZOUooPhUNTI}
                q_i<\sigma(x)<q_i+\epsilon
            \end{equation}
            pour tout \( \epsilon>0\) dans \( \eQ\) et \( i>N\). Ce \( \epsilon\) étant fixé nous pouvons prendre la limite des inégalités \eqref{EQooLZOUooPhUNTI} :
            \begin{equation}
                x\leq \sigma(x)\leq x+\epsilon.
            \end{equation}
            Cela étant valable pour tout \( \epsilon>0\) dans \( \eQ\), nous avons bien \( x=\sigma(x)\).
    \end{subproof}
\end{proof}

\begin{remark}      \label{REMooGHEDooOYYUPk}
    Certains\cite{ooEKUSooDDDWuT} pensent que l'énoncé de cette proposition, ne parlant que de \emph{corps} \( \eR\) n'autorise pas l'utilisation d'autre structure réelle que celle de corps. Du coup il faut reconstruire la notion d'ordre à partir seulement du langage des corps. Par exemple en disant que \( a>b\) si et seulement si il existe \( k\) tel que \( a=b+k^2\).

    On peut s'en sortir en donnant l'énoncé suivant : «Si \( \eK\) est un corps isomorphe (en tant que corps) à \( \eR\) alors son unique automorphisme est l'identité». Cela se démontre immédiatement en disant que si \( f\) est un automorphisme de \( \eK\) et si \( \phi\) est un isomorphisme \( \eK\to \eR\) alors \( \phi\circ f\circ \phi^{-1}\) est un automorphisme de \( \eR\). Donc il est l'identité et \( f\) l'est également.

    Attention cependant à prouver que \( \phi^{-1}\) est un morphisme. En effet en posant \( \phi^{-1}(x)=a\) et \( \phi^{-1}(y)=b\) nous avons
    \begin{equation}
        \phi\big( \phi^{-1}(x)+\phi^{-1}(y) \big)=x+y
    \end{equation}
    parce que \( \phi\) est un morphisme. D'autre part,
    \begin{equation}
        \phi\big( \phi^{-1}(x)+\phi^{-1}(y) \big)=\phi(a+b).
    \end{equation}
    Donc
    \begin{equation}
        \phi^{-1}(x+y)=\phi^{-1}\big( \phi(a)+\phi(b) \big)=\phi^{-1}\big( \phi(a+b) \big)=a+b=\phi^{-1}(x)+\phi^{-1}(y).
    \end{equation}
\end{remark}

\begin{proposition}     \label{PROPooEATMooIPPrRV}
    Un automorphisme du corps \( \eC\) qui fixe \( \eR\) est soit l'identité soit la conjugaison complexe\footnote{Par «fixer \( \eR\)» nous entendons que \( \sigma(\eR)=\eR\), pas spécialement que \( \sigma(x)=x\) pour tout \( x\in \eR\)}.
\end{proposition}

\begin{proof}
    Soit un automorphisme \( \sigma\) vérifiant la condition de fixer \( \eR\). Alors la restriction de \( \sigma\) à \( \eR\) est un automorphisme de \( \eR\) et y est donc l'identité par la proposition~\ref{PROPooLLPMooIVanaO}.

    En ce qui concerne les imaginaires purs,
    \begin{equation}
        -1=\sigma(-1)=\sigma(ii)=\sigma(i)^2.
    \end{equation}
    Donc \( \sigma(i)\) est un élément de \( \eC\) vérifiant \( \sigma(i)^2=-1\). C'est-à-dire \( \sigma(i)=\pm i\).

    Si \( \sigma(i)=i\) alors \( \sigma=\id\). Si \( \sigma(i)=-i\) alors \( \sigma\) est la conjugaison complexe.
\end{proof}

%---------------------------------------------------------------------------------------------------------------------------
\subsection{Corps premier}
%---------------------------------------------------------------------------------------------------------------------------
\label{subseccorpspremhBlYIv}

\begin{definition}
    Un corps est \defe{premier}{corps!premier}\index{premier!corps} s'il est son seul sous corps. Le \defe{sous corps premier}{premier!sous corps} d'un corps est l'intersection de tous ses sous corps.
\end{definition}

\begin{lemma}
    Un corps premier est commutatif.
\end{lemma}

\begin{proof}
    Le centre d'un corps est certainement un sous corps. Par conséquent un corps premier doit être contenu dans son propre centre, c'est-à-dire être commutatif.
\end{proof}

\begin{definition}  \label{DefXIHLooBAcqYH}
Soit \( p\) un nombre premier. Nous notons \( \eF_p=\eZ_p=\eZ/p\eZ\)\nomenclature[A]{\( \eF_p\)}{lorsque \( p\) est premier}.
\end{definition}

Nous verrons plus loin (section~\ref{SecCorpsFinizkAcbS}) comment nous pouvons définir \( \eF_{p^n}\) lorsque \( p\) est premier, ainsi que l'unicité d'un tel corps.

Nous avons par exemple
\begin{equation}
    \eF_2=\eZ/2\eZ=\{ 0,1 \}
\end{equation}
avec la loi \( 2=0\).

Notons que \( \eF_p\) est un corps possédant \( p\) éléments. L'ensemble \( \eF_p^*\) est un groupe d'ordre \( p-1\).

\begin{lemma}
    Les corps \( \eQ\) et \( \eF_p\) (avec \( p\) premier) sont premiers.
\end{lemma}

\begin{proof}
    Tout sous corps de \( \eQ\) doit contenir \( 1\), et par conséquent \( \eZ\). Devant également contenir tous les inverses, il contient \( \eQ\).

    Tout sous corps de \(\eF_p \) doit contenir \( 1\) et donc \( \eF_p\) en entier. Par ailleurs nous savons de la proposition~\ref{PropzhFgNJ} que \( \eF_p\) est un corps lorsque \( p\) est premier.
\end{proof}

\begin{proposition}
    Soit \( \eK\) un corps de caractéristique \( p\) et \( \eP\) son sous corps premier. Si \( p=0\) alors \( \eP=\eQ\). Si \( p>0\), alors \( \eP=\eF_p\).
\end{proposition}

\begin{proof}
    Notons d'abord que la caractéristique d'un corps est toujours soit 0 soit un nombre premier, parce qu'un corps est en particulier un anneau intègre (proposition~\ref{LemCaractIntergernbrcartpre}).

    Étant donné que \( 1\) est dans tout sous corps, nous devons avoir \( \eZ 1\subseteq \eP\). Si \( p=0\), alors \( \eZ 1\simeq \eZ\), et nous avons
    \begin{equation}
        \eZ 1_{\eA}\subset \eP\subset \eK.
    \end{equation}
    Pour chaque \( (n,m)\in \eZ 1_{\eA}\times (\eZ 1_{\eA})^*\) l'élément \( nm^{-1}\in \eK\) est dans \( \eP\) parce que \( \eP\) est un corps. Nous en déduisons que le corps des fractions de \( \eZ\) est contenu dans \( \eP\) par conséquent \( \eP=\eQ\) (théorème~\ref{ThogbhWgo}).

    Si par contre la caractéristique de \( \eK\) est \( p\neq 0\), nous avons \( \eZ 1_{\eA}\simeq\eZ/p\eZ=\eF_p\) par le lemme~\ref{LemHmDaYH}. L'ensemble \( \eF_p\) étant un corps, c'est le corps premier de \( \eK\).
\end{proof}

\begin{proposition}     \label{PropqPPrgJ}
    Soit \( \eK\) un corps et \( \eP\) son sous-corps premier. Si \( \sigma\in\Aut(\eK)\) alors \( \sigma|_{\eP}=\id\), c'est-à-dire que $\sigma(x)=x$ pour tout \( x\in \eP\).
\end{proposition}

%---------------------------------------------------------------------------------------------------------------------------
\subsection{Petit théorème de Fermat}
%---------------------------------------------------------------------------------------------------------------------------

\begin{theorem}[Petit théorème de Fermat]       \label{ThoOPQOiO}   \index{théorème!petit de Fermat}\index{petit théorème de Fermat}
    Soit \( p\) un nombre premier. Si \( x\in \eF_p\) alors \( x^p=x\). Si \( x\in(\eF_p)^*\), alors \( x^{p-1}=1\).

    En particulier si \( x\in \eF_p^*\) alors \( x^{-1}=x^{p-2}\).
\end{theorem}

\begin{proof}
    Étant donné que \( \eF_p\) est un corps commutatif et que \( p\) est premier, la proposition~\ref{Propqrrdem} nous indique que \( \sigma(x)=x^p\) est un automorphisme. La proposition~\ref{PropqPPrgJ} nous indique alors que
    \begin{equation}
        a^p=a.
    \end{equation}
    Si \( a\) est inversible alors \( a^{p-1}=a^pa^{-1}=1\).
\end{proof}

\begin{remark}      \label{RemCoSnxh}
    Une autre façon d'énoncer le petit théorème de Fermat~\ref{ThoOPQOiO} est que si \( p\) est premier et si \( a\) est premier avec \( p\), alors \( a^{p-1}\in[1]_p\). Le nombre \( a\) n'est pas premier avec \( p\) uniquement lorsque \( a\) est multiple de \( p\). Dans ce cas c'est \( a=0\) dans \( \eF_p\) et donc \( a^{p-1}=0\).
\end{remark}

\begin{example}
    Soit \( \eK=\eF_{29}\). Le nombre \( 29\) étant premier, \( \eK\) est un corps premier. C'est le corps des entiers modulo \( 29\). Nous avons donc
    \begin{equation}
            -142=-113=-84=-55=-26=3=32=61=90=119.
    \end{equation}
    Le petit théorème de Fermat nous permet aussi de calculer des exposants et des inverses. En effet, puisque \( 1=x^{28}\) pour tout \( x \in \eF_{29}^* \), nous avons \( x^{-1}=x^{27}\), et par suite, pour tout entier \( a \),
    \begin{equation}
        x^{-a}=(x^a)^{27}=x^{27a}.
    \end{equation}
    Le nombre \( 27 a\) peut être grand par rapport à \( 29\). Mais en réutilisant le fait que \( 1=x^{28}\), on obtient
    \begin{equation}
        x^{-a}=x^{[27a]_{28}}.
    \end{equation}
    Cette expression doit être comprise comme disant que pour tout \( k\in [27a]_{28}\) nous avons \( x^{-a}=x^{k}\).

    Chose à retenir : dans les exposants nous calculons modulo \( 28\).
\end{example}

%+++++++++++++++++++++++++++++++++++++++++++++++++++++++++++++++++++++++++++++++++++++++++++++++++++++++++++++++++++++++++++
\section{Théorème des deux carrés}
%+++++++++++++++++++++++++++++++++++++++++++++++++++++++++++++++++++++++++++++++++++++++++++++++++++++++++++++++++++++++++++

\begin{proposition} \label{PropleGdaT}
    Soit \( p\) un nombre premier et \( P\) un élément de \( \eF_p[X]\). L'anneau \( \eF_p[X]/(P)\) est intègre si et seulement si \( P\) est irréductible dans \( \eF_p[X]\).
\end{proposition}

\begin{proof}
    Supposons que \( P\) soit réductible dans \( \eF_p[X]\), c'est-à-dire qu'il existe \( Q,R\in \eF_p[X]\) tels que \( P=QR\). Dans ce cas, \( \bar Q\) est diviseur de zéro dans \( \eF_p[X]/(P)\) parce que \( \bar Q\bar R=0\).

    Nous supposons maintenant que \( \eF_p[X]/(P)\) ne soit pas intègre : il existe des polynômes \( R,Q\in \eF_p[X]\) tels que \( \bar Q\bar R=0\). Dans ce cas le polynôme \( QR\) est le produit de \( P\) par un polynôme : \( QR=PA\). Tous les facteurs irréductibles de \( A \) étant soit dans \( Q\) soit dans \( R\), il est possible de modifier un peu \( Q\) et \( R\) pour obtenir \( QR=P\), ce qui signifie que \( P\) n'est pas irréductible.
\end{proof}

%---------------------------------------------------------------------------------------------------------------------------
\subsection{Un peu de structure dans \texorpdfstring{$ \eZ[i]$}{Zi}}
%---------------------------------------------------------------------------------------------------------------------------

\begin{lemma}   \label{LemSCAlICY}
     L'application
     \begin{equation}
         \begin{aligned}
             N\colon \eZ[i]&\to \eN \\
             a+bi&\mapsto a^2+b^2
         \end{aligned}
     \end{equation}
     est un stathme euclidien pour \( \eZ[i]\).
\end{lemma}
\index{stathme!sur \( \eZ[i]\)}

\begin{proof}
    Soient \( t,z\in \eZ[i]\setminus\{ 0 \}\) dont le quotient s'écrit
    \begin{equation}
        \frac{ z }{ t }=x+iy
    \end{equation}
    dans \( \eC\). Nous considérons \( q=a+bi\) où \( a\) et \( b\) sont les entiers les plus proches de \( x\) et \( y\). S'il y a \emph{ex aequo}, on prend au hasard\footnote{Dans l'exemple~\ref{ExwqlCwvV}, nous prenions toujours l'inférieur parce que le stathme tenait compte de la positivité.}. Alors nous avons
    \begin{equation}
        | \frac{ z }{ t }-q |\leq \frac{ | 1+i | }{ 2 }=\frac{ \sqrt{2} }{2}<1.
    \end{equation}
    On pose \( r=z-qt\) qui est bien un élément de \( \eZ[i]\). De plus
    \begin{equation}
        | r |=| z-qt |=| t | |\frac{ z }{ t }-q |<| t |,
    \end{equation}
    c'est-à-dire que \( | r |^2<| t |^2\) et donc \( N(r)<N(t)\).
\end{proof}
Étant donné que \( \eZ[i]\) est euclidien, il est principal (proposition~\ref{Propkllxnv}).

\begin{lemma}   \label{LemBMEIiiV}
    Les éléments inversibles de \( \eZ[i]\) sont \( \{ \pm 1,\pm i \}\).
\end{lemma}

\begin{proof}

    Déterminons les éléments inversibles de \( \eZ[i]\). Si \( z\in \eZ[i]^*\), alors il existe \( z'\in \eZ[i]^*\) tel que \( zz'=1\). Dans ce cas nous aurions
    \begin{equation}
        1=N(zz')=N(z)N(z'),
    \end{equation}
    ce qui est uniquement possible avec \( N(z)=N(z')=1\), c'est-à-dire \( z=\pm 1\) ou \( z=\pm i\). Nous avons donc
    \begin{equation}
        \eZ[i]^*=\{ \pm 1,\pm i \}.
    \end{equation}
\end{proof}

\begin{definition}[\cite{ooHZAVooDDUQce}]      \label{DEFooUCSHooJqGuVB}
    Un \defe{monoïde}{monoïde} est un triple \( (E,*,e)\) où \( E\) est un ensemble, \( e\) est un élément de \( E\) et \( *\colon E\times E\to E\) est une loi de composition telle que pour tout \( x,y\in E\),
    \begin{enumerate}
        \item
            \( x*(y*z)=(x*y)*z\) (associativité)
        \item
            \( e*x=x*e=x\) (\( e\) est un neutre).
    \end{enumerate}
\end{definition}

Nous notons \( \Sigma=\{ a^2+b^2\tq a,b\in \eN \}\).
\begin{lemma}   \label{LemIBDPzMB}
    L'ensemble \( \Sigma=  \{ a^2+b^2\tq a,b\in \eN \}  \) est un sous-monoïde\footnote{Définition \ref{DEFooUCSHooJqGuVB}.} de \( \eN\).
\end{lemma}

\begin{proof}
    Il suffit de prouver que si \( m,n\in \Sigma\), alors le produit \( mn\) est également dans \( \Sigma\). Si \( N\) est le stathme euclidien sur \( \eZ[i]\), alors  \( n\in \Sigma\) si et seulement s'il existe \( z\in \eZ[i]\) tel que \( N(z)=n\). Si \( z,z'\in \eZ[i]\), alors \( zz'\in \eZ[i]\) et de plus
    \begin{equation}
        N(zz')=N(z)N(z')=nm.
    \end{equation}
    Donc \( nm\) est l'image de \( zz'\) par \( N\), ce qui prouve que \( nm\in \Sigma\).
\end{proof}

\begin{theorem}[Théorème des deux carrés, version faible]   \label{ThospaAEI}
    Un nombre premier est somme de deux carrés si et seulement si \( p=2\) ou \( p\in[1]_4\).
\end{theorem}
\index{anneau!principal}
\index{nombre!premier}
\index{théorème!des deux carrés!version faible}

\begin{remark}
    Il n'est pas dit que les nombres dans \( [1]_4\) sont premiers (\( 9=8+1\) ne l'est pas par exemple). Le théorème signifie que (à part \( 2\)), si un nombre premier est dans \( [1]_4\) alors il est somme de deux carrés, et inversement, si un nombre premier est somme de deux carrés, il est dans \( [1]_4\).
\end{remark}

\begin{proof}
    Soit \( p\) un nombre premier dans \( \Sigma\). Si \( a=2k\), alors \( a^2=4k^2\) et \( a^2=0\mod 4\). Si au contraire \( a\) est impair, \( a=2k+1\) et \( a^2=4k^2+1+4k=1\mod 4\). La même chose est valable pour \( b\). Par conséquent, \( a^2+b^2\) est automatiquement \( [0]_4\), \( [1]_4\) ou \( [2]_4\). Évidemment les nombres de la forme \( 0\mod 4\) ne sont pas premiers; parmi les \( 2\mod 4\), seul \( p=2\) est premier (et vaut \( 1^2+1^2\)).

    Nous avons démontré que les seuls premiers de la forme \( a^2+b^2\) sont \( p=2\) et les \( p=1\mod 4\). Il reste à faire le contraire : démontrer que si un nombre premier \( p\) vaut \( 1\mod 4\), alors il est premier. Nous considérons l'anneau
    \begin{equation}
        \eZ[i]=\{ a+bi\tq a,b\in \eZ \}.
    \end{equation}
    puis l'application
    \begin{equation}
        \begin{aligned}
            N\colon \eZ[i]&\to \eN \\
            a+bi&\mapsto a^2+b^2.
        \end{aligned}
    \end{equation}
    Un peu de calcul dans \( \eC\) montre que pour tout \( z,z'\in \eZ[i]\), \( N(zz')=N(z)N(z')\).


    Nous savons que les éléments inversibles de \( \eZ[i]\) sont \( \pm 1\) et \( \pm i\) (lemme~\ref{LemBMEIiiV}).

    Le lemme~\ref{LemSCAlICY} montre que \( \eZ[i]\) est un anneau euclidien parce que \( N\) est un stathme. L'anneau \( \eZ[i]\) étant euclidien, il est principal (proposition~\ref{Propkllxnv}).



    Pour la suite, nous allons d'abord montrer que \( p\in\Sigma\) si et seulement si \( p\) n'est pas irréductible dans \( \eZ[i]\), puis nous allons voir quels sont les irréductibles de \( \eZ[i]\).

    Soit \( p\), un nombre premier dans \( \Sigma\). Si \( p=a^2+b^2\), alors nous avons \( p=(a+ib)(a-bi)\), mais étant donné que \( p\) est premier, nous avons \( a\neq 0\) et \( b\neq 0\). Du coup \( p\) n'est pas inversible dans \( \eZ[i]\), mais il peut être écrit comme le produit de deux non inversibles. Le nombre \( p\) est donc non irréductible dans \( \eZ[i]\).

    Dans l'autre sens, nous supposons que \( p\) est un nombre premier non irréductible dans \( \eZ[i]\). Nous avons alors \( p=zz'\) avec ni \( z\) ni \( z'\) dans \( \{ \pm 1,\pm i \}\). En appliquant \( N\) nous avons
    \begin{equation}
        p^2=N(p)=N(z)N(z').
    \end{equation}
    Vu que \( p\) est premier, cela est uniquement possible avec \( N(z)=N(z')=p\) (avoir \( N(z)=1\) est impossible parce que cela dirait que \( z\) est inversible). Si \( z=a+ib\), alors \( p=N(z)=a^2+b^2\), et donc \( p\in \Sigma\).

    Nous savons déjà que \( \eZ[i]\) est un anneau principal et n'est pas un corps; la proposition~\ref{PropomqcGe} s'applique donc et \( p\) sera non irréductible si et seulement si l'idéal \( (p)\) sera non premier. Le fait que \( (p)\) soit un idéal non premier implique que le quotient \( \eZ[i]/(p)\) est non intègre (c'est la définition d'un idéal premier). Nous cherchons donc les nombres premiers pour lesquels le quotient \( \eZ[i]/(p)\) n'est pas intègre.

    Nous commençons par écrire le quotient \( \eZ[i]/(p)\) sous d'autres formes. D'abord en remarquant que si \( I\) et \( J\) sont deux idéaux, on a \( (\eA/I)/J\simeq (\eA/J)/I\), du coup, en tenant compte du fait que \( \eZ[i]=\eZ[X]/(X^2+1)\), nous avons
    \begin{equation}
        \eZ[i]/(p)=(\eZ[X]/(p))/(X^2+1)=\eF_p[X]/(X^2+1).
    \end{equation}
    Nous avons donc équivalence des propositions suivantes :
    \begin{subequations}
        \begin{align}
            p\in\Sigma\\
            \eF_p[X]/(X^2+1)\text{ n'est pas intègre}\\
            X^2+1\text{ n'est pas irréductible dans } \eF_p \label{EqZkdrvh}\\
             X^2+1\text{ admet une racine dans } \eF_p\\
            -1\in (\eF_p^*)^2\\
            \exists y\in \eF_p^*\tq y^2=-1.
        \end{align}
    \end{subequations}
    Le point \eqref{EqZkdrvh} vient de la proposition~\ref{PropleGdaT}. Maintenant nous utilisons le fait que \( p\) soit un premier impair (parce que le cas de \( p=2\) est déjà complètement traité), donc \( (p-1)/2\in \eN\) et nous avons, pour le \( y\) de la dernière ligne,
    \begin{equation}
        (-1)^{(p-1)/2}=(y^2)^{(p-1)/2}=y^{p-1}=1
    \end{equation}
    parce que dans \( \eF_p\) nous avons \( y^{(p-1)}=1\) par le petit théorème de Fermat (théorème~\ref{ThoOPQOiO}). Du coup \( p\) doit vérifier
    \begin{equation}
        1=(-1)^{(p-1)/2},
    \end{equation}
    c'est-à-dire \( \frac{ p-1 }{2}=0\mod 2\) ou encore \( p=1\mod 4\).
\end{proof}

\begin{theorem}[Théorème des deux carrés\cite{KXjFWKA}]
    Soit \( n\geq 2\) un nombre dont nous notons
    \begin{equation}    \label{EqBMHTzCT}
        n=\prod_{p\in\pP}p^{v_p(n)}
    \end{equation}
    où \( \pP\) est l'ensemble des nombres premiers. Alors \( n\in \Sigma\) si et seulement si pour tout \( p\in\pP\cap[3]_4\), nous avons \( v_p(n)\in [0]_2\) (c'est-à-dire \( v_p(n)\) est pair).
\end{theorem}
\index{théorème!des deux carrés}
\index{nombre!premier!théorème des deux carrés}
\index{anneau!principal!utilisation}
%TODO : il y a un lien entre le théorème des deux carrés et les triplets pytagoritiens http://fr.wikipedia.org/wiki/Triplet_pythagoricien

\begin{proof}
    \begin{subproof}
    \item[Condition suffisante.]

        Le produit \eqref{EqBMHTzCT} est évidemment un produit fini que nous pouvons alors regrouper en quatre parties : \( \pP\cap[0]_4\), \( \pP\cap[1]_4\), \( \pP\cap[2]_4\) et \( \pP\cap[3]_4\).

        \begin{itemize}
            \item Il n'y a pas de nombres premiers dans \( [0]_4\).
            \item Les nombres premiers de \( [1]_4\) sont dans \( \Sigma\). Le produit d'éléments de \( \Sigma\) étant dans \( \Sigma\), nous avons
                \begin{equation}
                    \prod_{p\in\pP\cap[1]_4}p^{v_p(n)}\in \Sigma.
                \end{equation}
            \item
                Le seul nombre premier dans \( [2]_4\) est \( 2\). C'est un élément de \( \Sigma\).
            \item
                Le produit
                \begin{equation}
                    \prod_{p\in\pP\cap[3]_4}p^{v_p(n)}
                \end{equation}
                est par hypothèse un produit de carrés (\( v_p(n)\) est pair), et est donc un carré.
        \end{itemize}
        Au final le produit \( \prod_{p\in\pP}p^{v_p(n)}\) est un produit d'un carré par un élément de \( \Sigma\), ce qui est encore un élément de \( \Sigma\).

        Pour cette partie, nous avons utilisé et réutilisé le lemme~\ref{LemIBDPzMB}.

    \item[Condition nécessaire.]

        Soit \( p\), un nombre premier. Nous voulons montrer que
        \begin{equation}
            \{ v_p(n)\tq n\in \Sigma \}\subset [2]_2.
        \end{equation}
        Pour montrer cela nous allons procéder par récurrence sur les ensembles
        \begin{equation}
            E_k=\{ v_p(n)\tq n\in \Sigma \}\cap\{ 0,\ldots, k \}.
        \end{equation}
        Il est évident que les éléments de \( E_0\) sont pairs, vu qu'il n'y a que zéro, qui est pair.

        Supposons que \( E_k\subset[0]_2\), et montrons que \( E_{k+1}\subset[0]_2\). Soit un élément de \( E_{k+1}\), c'est-à-dire \( v_p(n)\leq k+1\) avec \( n=a^2+b^2\). Si \( v_p(n)=0\) alors l'affaire est réglée; sinon c'est que \( p\) divise \( n\). Mais dans \( \eZ[i]\) nous avons
        \begin{equation}
            n=a^2+b^2=(a+bi)(a-bi)
        \end{equation}
        Vu que \( \eZ[i]\) est principal, le lemme de Gauss~\ref{LemSdnZNX} nous dit que si \( p\) divise \( n\), alors il doit diviser soit \( a+bi\), soit \( a-bi\) (et du coup en fait les deux). Nous avons alors \( p\divides a\) et \( p\divides b\) en même temps. Du coup
        \begin{equation}
            p^2\divides a^2+b^2=n.
        \end{equation}
        Posons \( a=pa'\) et \( b=pb'\) avec \( a',b'\in \eN\). Nous avons
        \begin{equation}
            \frac{ n }{ p^2 }=\frac{ p^2a'^2+p^2b'^2 }{ p^2 }=a'^2+b'^2\in \Sigma.
        \end{equation}
        Mais par construction,
        \begin{equation}
            v_p\left( \frac{ n }{ p^2 } \right)=v_p(n)-2<k.
        \end{equation}
        Donc \( v_p(\frac{ n }{ p^2 })\) est pair et du coup \( v_p(n)\) doit également être pair.

    \end{subproof}
\end{proof}

%---------------------------------------------------------------------------------------------------------------------------
\subsection{Résultats chinois}
%---------------------------------------------------------------------------------------------------------------------------

Nous allons maintenant parler du système d'équations
\begin{subequations}
    \begin{numcases}{}
        x=a_1\mod p\\
        x=a_2\mod q
    \end{numcases}
\end{subequations}
avec \( a_1\), \( a_2\) donnés dans \( \eZ\) et \( p,q\) des entiers premiers entre eux. Le lemme chinois nous donne la liste des solutions ainsi qu'une manière de les construire. Le théorème chinois en sera une espèce de corolaire qui établira l'isomorphisme d'anneaux \( \eZ/pq\eZ\simeq \eZ/p\eZ\times \eZ/q\eZ\). Voir \href{http://www.les-mathematiques.net/b/a/d/node10.php}{les-mathematiques.net}.

\begin{lemma}[Lemme chinois \cite{CongrDuchSyl}]        \label{LemCtUeGA}
    Soient \( n_1,n_2\) deux entiers premiers entre eux. Soient \( a_1,a_2\in \eZ\). Les solutions du système
    \begin{subequations}        \label{SysVwvLKv}
        \begin{numcases}{}
            x=a_1\mod n_1\\
            x=a_2\mod n_2
        \end{numcases}
    \end{subequations}
    pour \( x\in \eZ/n_1n_2\eZ\) sont données de la façon suivante. Soient \( u_1,u_2\) deux entiers qui satisfont la relation de Bézout\footnote{voir le théorème~\ref{ThoBuNjam}}
    \begin{equation}        \label{EqWcucUG}
        u_1n_1+u_2n_2=1,
    \end{equation}
    et
    \begin{equation}        \label{EqHGchlQ}
        a=\big( a_1u_2n_2+a_2 u_1n_1 \big)\mod(n_1).
    \end{equation}
    Alors \( x=a\mod(n_1n_2)\).
\end{lemma}

\begin{proof}
    Vérifions que le \( x\) donné par \(x=a\mod(n_1n_2)\) est bien une solution. D'abord
    \begin{subequations}
        \begin{align}
            a\mod n_2&=a_1u_2n_2\mod n_1\\
            &=a_1(1-u_1n_1)\mod n_1\\
            &=a_1\mod n_1
        \end{align}
    \end{subequations}
    où nous avons utilisé l'identité de Bézout \eqref{EqWcucUG}. La vérification de \( a\mod n_2=a_2\mod n_2\) est la même.

    Soit maintenant \( x\in \eZ/n_1n_2\eZ\) une solution du système \eqref{SysVwvLKv} et \( a\) donné par la formule \eqref{EqHGchlQ}. Alors
    \begin{subequations}
        \begin{align}
            (x-a)\mod n_1&=\Big( a_1-(a_1n_2u_2+a_2u_1n_1) \Big)\mod n_1\\
            &=a_1-a_1u_2n_2\mod n_1\\
            &=0,
        \end{align}
    \end{subequations}
    donc \( (x-a)\mod n_1=0\), ce qui signifie que \( x-a\) est divisible par \( n_1\). De la même façon, \( (x-a)\mod n_2=0\) et \( x-a\) est divisible par \( n_2\). Nous savons maintenant que \( x-a\) est divisible par \( n_1\) et \( n_2\). Étant donné que \( n_1\) et \( n_2\) sont premiers entre eux, nous en déduisons que \( x-a\) est divisible par \( n_1n_2\), ou encore que \( x=a\mod n_1n_2\).
\end{proof}

\begin{theorem}[Théorème chinois]
    Soient \( p,q\) deux naturels premiers entre eux. Si \( p,q\geq 2\) alors l'application
    \begin{equation}
        \begin{aligned}
            \phi\colon \eZ/pq\eZ&\to \eZ/p\eZ\times \eZ/q\eZ \\
            [x]_{pq}&\mapsto \big( [x]_p,[x]_q \big)
        \end{aligned}
    \end{equation}
    est un isomorphisme d'anneaux.
\end{theorem}

\begin{proof}
    Nous devons prouver que l'application \( \phi\) respecte la somme, le produit et qu'elle est bijective. En ce qui concerne la somme,
    \begin{subequations}
        \begin{align}
            \phi([q]_{pq}+[y]_{pq})&=            \phi\big( (x+y)\mod pq \big)\\
            &=\big( [x+y]_{p},[x+y]_q \big)\\
            &=\big( [x]_p+[y]_p,[x]_q+[y]_q \big)\\
            &=\big( [x]_p,[x]_q \big)+\big( [y]_p,[y]_q \big)\\
            &=\phi(x)+\phi(y).
        \end{align}
    \end{subequations}
    En ce qui concerne le produit, c'est le même jeu : nous obtenons
    \begin{equation}
        \phi\big( [xy]_{pq} \big)=\phi([x]_{pq}])\phi([y]_{pq})
    \end{equation}
    en utilisant le fait que \( [xy]_{p}=[x]_p[y]_p\).

    Montrons maintenant que \( \phi\) est surjective. Soient \( y_1,y_2\in \eZ\) et \( x\in \eZ\). Demander
    \begin{equation}
        \phi([x]_{pq})=\big( [y_1]_p,[y_2]_q \big)
    \end{equation}
    revient à demander que \( [x]_p=[y_1]_p\) et \( [x]_q=[y_2]_q\), c'est-à-dire que \( x\) résolve le système
    \begin{subequations}
        \begin{numcases}{}
            x=y_1\mod p\\
            x=y_2\mod q.
        \end{numcases}
    \end{subequations}
    Le lemme chinois~\ref{LemCtUeGA} nous assure qu'une solution existe.

    En ce qui concerne l'injectivité, nous supposons que \( x\) et \( y\) soient deux entiers tels que
    \begin{equation}
        \phi([x]_{pq})=\phi([y]_{pq}).
    \end{equation}
    Nous en déduisons le système
    \begin{subequations}
        \begin{numcases}{}
            x\mod p=y\mod p\\
            x\mod q=y\mod q
        \end{numcases}
    \end{subequations}
    c'est-à-dire qu'il existe des entiers \( k\) et \( l\) tels que \( x=y+kp\) et \( x=y+lq\) ou encore tels que
    \begin{equation}
        kp+lq=0.
    \end{equation}
    Étant donné que \( p\) et \( q\) sont premiers entre eux, la seule possibilité est \( k=l=0\), c'est-à-dire \( x=y\).
\end{proof}

\begin{theorem}[Théorème chinois\cite{MonCerveau}]   \label{THOooVIGQooUhwBLS}
    Soit \( A\) un anneau commutatif. Soient \( n\geq 2\), des éléments \( x_1,\ldots,x_n\) dans \( A\) et des idéaux \( I_1,\ldots,I_n\) tels que \( I_i+I_j=A\) pour tout \( i\neq j\).

    Alors il existe un \( x\in A\) tel que \( x-x_i\in I_i\) pour tout \( 1\leq i\leq n\).
\end{theorem}
\index{théorème!chinois}

\begin{proof}
    En plusieurs points.
    \begin{subproof}
        \item[Définition de \( J_i\)]
            
            Pour \( i\in\{ 1,\ldots,n \}\) nous notons \( J_i\) le produit \( J_i=\prod_{k\neq i}I_k\). Étant donné que chaque \( I_i\) est un idéal, nous avons \( I_k\subset J_i\) lorsque \( i\neq k\).

        \item[Un produit qui vaut \( 1\)]
            Soit \( i\) fixé. Pour tout \( j\neq i\), puisque \( I_i+I_j=A\), nous pouvons trouver \( a_j\in I_i\) et \( b_j\in I_j\) tels que \( a_j+b_j=1\). Nous avons alors
            \begin{equation}
                1=\prod_{j\neq i}(a_j+b_j).
            \end{equation}
        \item[\( I_i+J_i=A\)]
            En effet, soit \( k\neq i\). Nous avons \( J_i\subset  I_k\) et donc $I_i+J_i\subset I_i+I_k=A$.
             
         \item[Des \( \alpha\) et des \( \beta\)]

            Pour rappel \( i\) est toujours fixé. Vu que \( I_i+J_i=A\), nous donc prendre \( \alpha_i\in I_i\) et \( \beta_i\in J_i\) tels que
            \begin{equation}
                \alpha_i+\beta_i = 1 = \prod_{j\neq i}(a_j+b_j).
            \end{equation}
        \item[Défixer \( i\)]


         \item[La suite]
            Nous considérons alors l'élément \( x=\beta_1x_1+\cdots+\beta_nx_n\). Il vient alors
            \begin{subequations}
                \begin{align}
                    x-x_1&=(\beta_1-1)x_1+\beta_2x_2+\cdots+\beta_nx_n\\
                    &=-\alpha_1x_1+\beta_2x_2+\cdots+\beta_nx_n.
                \end{align}
            \end{subequations}
            Mais \( \alpha_1\in I_1\) et tous les autres termes sont dans les \( J_i\) avec \( i\neq 1\), donc aussi dans \( I_1 \) par définition des \( J_i \). (Par exemple,  \( \beta_2\in J_2\subset I_1\).  Nous en déduisons \( x - x_1 \in I_1\).

            L'argument que nous venons de donner pour justifier que \(x - x_1 \in I_1 \) peut être généralisé à tous les indices. En effet, soit \( k \) un indice quelconque; nous avons
             \begin{subequations}
                \begin{align}
                    x-x_k&=(\beta_k-1)x_k+\sum_{i \neq k} \beta_ix_i\\
                    &=-\alpha_kx_k+\sum_{i \neq k} \beta_ix_i;
                \end{align}
            \end{subequations}
            et \( \alpha_k\in I_k\) et pour tout \( i \neq k \), \( \beta_i\in J_i\subset I_k\)); donc \( x - x_k \in I_k\).
    \end{subproof}
\end{proof}

\begin{remark}
    Ce théorème chinois est bien une généralisation du lemme chinois~\ref{LemCtUeGA}. En effet, l'élément \( x\) dont il est question est solution du problème \( x=x_i\mod I_i\). L'hypothèse \( I_i+I_j=A\) n'est pas nouvelle non plus étant donné que si \( p\) et \( q\) sont des entiers premiers entre eux nous avons \( p\eZ+q\eZ=\eZ\) par le corolaire~\ref{CorgEMtLj}.
\end{remark}

%+++++++++++++++++++++++++++++++++++++++++++++++++++++++++++++++++++++++++++++++++++++++++++++++++++++++++++++++++++++++++++
\section{Polynômes à coefficients dans un corps}
%+++++++++++++++++++++++++++++++++++++++++++++++++++++++++++++++++++++++++++++++++++++++++++++++++++++++++++++++++++++++++++
\label{SECooFYOGooQHitgE}

Nous supposons que \( \eK\) est un corps commutatif, et nous étudions l'anneau \( \eK[X]\), défini en~\ref{DEFooFYZRooMikwEL}.

\begin{proposition}     \label{PropqGZXvr}
    L'anneau \( \eK[X]\) des polynômes sur un corps commutatif \( \eK\) est factoriel.
\end{proposition}
%TODO : une preuve.

Le théorème suivant est un cas particulier pour \( \eK[X]\) du théorème chinois~\ref{ThofPXwiM}.
\begin{theorem}[Théorème chinois]\index{théorème!chinois!anneau des polynômes}
    Si \( P\) et \( Q\) sont deux polynômes premiers entre eux, alors nous avons l'isomorphisme
    \begin{equation}
        \eK[X]/(P,Q)\simeq\eK[X]/(P)\times \eK[X]/(Q).
    \end{equation}
\end{theorem}
% TODO : s'assurer que c'est bien un cas particulier du théorème chinois de plus haut.


%---------------------------------------------------------------------------------------------------------------------------
\subsection{Irréductibilité}
%---------------------------------------------------------------------------------------------------------------------------

\begin{definition}[\cite{ooJJSGooBXOPGF}]      \label{DefIrredfIqydS}
    Un polynôme à coefficients dans un anneau commutatif est irréductible si il
\begin{enumerate}
        \item
            n'est pas inversible,
        \item
            n'est pas le produit de deux non inversibles.
    \end{enumerate}
\end{definition}

    Un polynôme est irréductible dans \( \eK[X]\) au sens de la définition~\ref{DeirredBDhQfA} si et seulement s'il est irréductible au sens de la définition~\ref{DefIrredfIqydS} parce que seules les constantes (non nulles) sont inversibles dans \( \eK[X]\).

\begin{example}
    Si un polynôme \( P\in \eZ[X]\) n'a que des racines complexes, ça ne l'empêche pas d'être réductible sur \( \eZ\). La réductibilité ne signifie pas qu'on peut mettre des racines en évidence. Par exemple le polynôme \( P=X^4+3X^2+2\) est réductible sur \( \eZ\) en
    \begin{equation}
        P=(X^2+1)(X^2+2),
    \end{equation}
    mais n'a pas de racines dans \( \eZ\). Par contre, il est vrai que si on veut réduire plus, il faut sortir de \( \eZ\).

\end{example}

\begin{definition}  \label{DefCPLSooQaHJKQ}
    Nous disons que \( P\in\eK[X]\) est \defe{scindé}{polynôme scindé} sur \(\eK\) s'il est produit dans \(\eK[X]\) de polynômes de degré~\( 1\).
\end{definition}
Note : les constantes ne sont donc pas des polynômes scindés.

\begin{proposition}[\wikipedia{fr}{Critère_d'Eisenstein}{Critère d'Eisenstein}]
    Soit le polynôme \( P=\sum_{k=0}^n a_nX^n\) dans \( \eZ[X]\). Nous supposons avoir un nombre premier \( p\) tel que
    \begin{enumerate}
        \item
            \( p\) divise tous les \( a_0,\ldots, a_{n-1}\),
        \item
            \( p\) ne divise pas \( a_n\),
        \item
            \( p^2\) ne divise pas \( a_0\).
    \end{enumerate}
    Alors \( P\) est irréductible dans \( \eQ[X]\).

    Si de plus \( P\) est primitif au sens du \( \pgcd\) (définition~\ref{DEFooAIYGooRAEfHU}) alors \( P\) est irréductible dans \( \eZ[X]\).
\end{proposition}

\begin{proof}
    Nous considérons \( \bar P\) le polynôme réduit modulo \( p\), c'est-à-dire \( \bar P\in \eF_p[X]\). Étant donné que par hypothèse tous les coefficients sont multiples de \( p\) sauf \( a_n\), nous avons \( \bar P=cX^n\). Supposons par l'absurde que \( P=QR\) avec \( Q,R\in \eQ[X]\). Alors le lemme de Gauss (\ref{LemSdnZNX}) impose \( P,Q\in \eZ[X]\).

    Nous avons aussi, au niveau des réductions modulo \( p\) que $\bar Q\bar R=\bar P$. Or \( \bar P\) est un monôme, donc \( \bar Q\) et \( \bar R\) doivent également l'être. Donc \( \bar Q=dX^k\) et \( \bar R=eX^{n-k}\) et en particulier \( \bar Q(0)=\bar R(0)=0\), c'est-à-dire que \( Q(0)\) et \( R(0)\) sont divisibles par \( p\). Cela impliquerait que \( a_0=Q(0)R(0)\) soit divisible par \( p^2\), ce qui est exclu par les hypothèses. Donc \( P\) est irréductible.

    Supposons de surcroît que \( P\) est primitif au sens du \( \pgcd\). Il est donc irréductible et primitif sur \( \eQ[X]\) et le corolaire~\ref{CORooZCSOooHQVAOV} nous dit alors que \( P\) est irréductible sur \( \eZ[X]\).
\end{proof}

\begin{example}
    Soit le polynôme \( P=3X^4+15 X^2+10\). Pour faire fonctionner le critère d'Eisenstein il nous faut un nombre premier \( p\) divisant \( 15\) et \( 10\), mais pas \( 3\) et dont le carré ne divise pas \( 10\). C'est vite vu que \( p=5\) fait l'affaire. Le polynôme \( P\) est donc irréductible sur \( \eQ[X]\).
\end{example}

%---------------------------------------------------------------------------------------------------------------------------
\subsection{Idéaux}
%---------------------------------------------------------------------------------------------------------------------------

Soit \( P\in \eK[X]\) un polynôme. Nous notons \( (P)\) l'idéal engendré par \( P\) :
\begin{equation}        \label{EqDefxMkDtW}
    (P)=\{ PR\tq R\in\eK[X] \}.
\end{equation}

\begin{lemma}
    Nous avons
    \begin{enumerate}
        \item
            \( (P)\subset (Q)\) si et seulement si \( Q\) divise \( P\),
        \item
            \( (P)=(Q)\) si et seulement si \( P\) et \( Q\) sont multiples (non nuls) l'un de l'autre.
    \end{enumerate}
\end{lemma}

\begin{proof}
    Si \( (P)\subset (Q)\), en particulier \( P\in(Q)\) et il existe \( R\in\eK[X]\) tel que \( P=QR\), ce qui signifie que \( Q\) divise \( P\).

    Si les idéaux de \( P\) et de \( Q\) sont identiques, l'un divise l'autre et l'autre divise l'un. Ils sont donc multiples l'un de l'autre.
\end{proof}

\begin{theorem}     \label{ThoCCHkoU}
    Soit \( \eK\) un corps commutatif.
    \begin{enumerate}
        \item       \label{ITEMooLZWMooDRsRwW}
            L'anneau \( \eK[X]\) est euclidien et principal.
        \item
            Si \( I\) est un idéal dans \( \eK[X]\) et si \( P \in I\) est de degré minimal, alors \( (P)=I\).
        \item   \label{ITEMooASHKooZqkiCH}
            De plus si \( I\neq \{  0\}\), il existe un unique polynôme unitaire \( \mu\) tel que \( I=(\mu)\).
    \end{enumerate}
\end{theorem}

\begin{proof}
    Le point~\ref{ITEMooLZWMooDRsRwW} a déjà été démontré dans le lemme~\ref{LEMooIDSKooQfkeKp} via le fait que \( \eK[X]\) est euclidien. Nous allons cependant donner ici une preuve directe que tous les idéaux de \( \eK[X]\) sont principaux. Si \( I=\{ 0 \}\), le résultat est évident. Nous supposons donc \( I\) non nul. Soit \( P\) de degré minimum parmi les éléments de \( I\). Évidemment \( (P)\subset I\). Nous allons démontrer qu'en réalité \( (P)=I\).

    Soit \( P'\in I\). Par le théorème~\ref{ThodivEuclPsFexf} de la division euclidienne\footnote{Ici \( \eK\) est un corps et donc l'hypothèse d'inversibilité est automatiquement vérifiée.}, il existe \( Q\) et \( R\) dans \( \eK[X]\) tels que \( P'=PQ+R\) avec \( \deg(R)<\deg(P)\). Étant donné que \( R=P'-PQ\) nous avons \( R\in I\) et par conséquent \( R=0\) parce que \( P\) a été choisi de degré minimum dans \( I\). Nous avons donc \( P'=PQ\) et \( I\subset (P)\).

    L'existence d'un polynôme unitaire qui génère \( I\) est obtenu en choisissant \( \mu =P/a_n\) où \( a_n\) est le coefficient du terme de plus haut degré. L'unicité d'un tel polynôme est obtenu par le fait que si \( \mu \) et \( \mu' \) génèrent le même idéal, alors ils sont multiples l'un de l'autre, or puisqu'ils sont unitaires, ils sont égaux.
\end{proof}
Nous voyons que n'importe quel polynôme de degré minimum dans un idéal génère l'idéal. Une importante conséquence du théorème~\ref{ThoCCHkoU} que nous verrons plus bas est que tout polynôme annulateur d'un endomorphisme est divisé par le polynôme minimal (proposition~\ref{PropAnnncEcCxj}).

\begin{corollary}       \label{CorsLGiEN}
    Si \( \eK\) est un corps et si \( P\) est un polynôme irréductible, alors l'ensemble \( \eL=\eK[X]/(P)\) est un corps. De plus \( \eL\) est un espace vectoriel de dimension \( \deg(P)\).
\end{corollary}

\begin{proof}
    En effet \( \eK[X]\) est un anneau principal par le théorème~\ref{ThoCCHkoU}, par conséquent la proposition~\ref{PropoTMMXCx}\ref{ITEMooKPJQooWuPZXS} déduit que \( \eK[X]/(P)\) est un corps.

    Une base de \( \eL\) est donnée par les projections de \( 1,X,X^2,\ldots, X^{n-1}\). En effet ces éléments forment une famille libre parce que si \( \sum_{k=0}^{n-1}a_k\bar X^n=0\) alors un représentant de cette classe doit être de la forme \( SP\) dans \( \eK[X]\), c'est-à-dire
    \begin{equation}
        \sum_{k=0}^{n-1}a_kX^k=SP,
    \end{equation}
    ce qui n'est possible que si \( S=0\) et \( a_k=0\). D'autre part c'est un système générateur. En effet si \( P=X^n+Q\) avec \( \deg(Q)=n-1\) alors
    \begin{equation}
        \bar X^{n+l}=\bar X^n\bar X^l=(\bar P-\bar Q)\bar X^l=\bar Q\bar X^l.
    \end{equation}
    Nous avons donc exprimé \( \bar X^{n+l}\) comme une somme de termes de degré \( n+l-1\). Par récurrence nous pouvons exprimer tout \( \bar X^{n+l}\) comme combinaison d'éléments de degré plus petit que \( n\).
\end{proof}

\begin{normaltext}
    Ce corolaire prendra une nouvelle jeunesse lorsque nous parlerons de polynômes d'endomorphismes, en particulier la proposition~\ref{PropooCFZDooROVlaA} va donner des précisions.
\end{normaltext}

\begin{lemma}[\cite{ooUHHUooONXDDl}]        \label{LEMooGRIMooPxCXAZ}
    Soit un isomorphisme de corps \( \tau\colon \eK\to \eK'\). Alors
    \begin{enumerate}
        \item
            L'application étendue
            \begin{equation}
                \begin{aligned}
                    \tau\colon \eK[X]&\to \eK'[X] \\
                    \sum_ia_iX^i&\mapsto \sum_i\tau(a_i)X^i
                \end{aligned}
            \end{equation}
            est un isomorphisme d'anneaux;
        \item
            pour tout \( P\in \eK[X]\), le passage au quotient
            \begin{equation}
                \begin{aligned}
                    \phi_{\tau}\colon \eK[X]/(P)&\to \eK'[X]/\big( \tau(P) \big) \\
                    \bar Q&\mapsto \overline{ \tau(Q) }
                \end{aligned}
            \end{equation}
            est un isomorphisme d'anneaux (et d'abord est bien définie).
    \end{enumerate}
\end{lemma}

\begin{proof}
    Nous n'allons pas faire explicitement toutes les vérifications, mais tout de même les principales. Montrons que \( \tau\) respecte le produit entre \( \eK[X]\) et \( \eK'[X]\). Nous rappelons que ce produit est défini par a formule \eqref{EQooTNCSooKklisb}. En notant \( P_i\) les coefficients de \( P\) et \( Q_i\) ceux de \( Q\) et en remarquant que la définition de \( \tau\) est essentiellement que \( \tau(P)_i=\tau(P_i)\), nous avons :
    \begin{subequations}
        \begin{align}
            \tau(PQ)&=\tau\Bigl( \sum_k\bigl(\sum_{l=0}^kP_lQ_{k-l}\bigr)X^k \Bigr)\\
            &=\sum_k X^k \sum_{l=0}^k\tau(P_lQ_{k-l})\\
            &=\sum_k X^k \sum_{l=0}^k\tau(P_l)\tau(Q_{k-l})\\
            &=\sum_k X^k \sum_{l=0}^k\tau(P)_l\tau(Q)_{k-l}\\
            &=\sum_{i} \bigl(\tau(P)_iX^i\bigr)\sum_j\bigl(\tau(Q)_j X^j\bigr)\\
            &=\tau(P)\tau(Q).
        \end{align}
    \end{subequations}

    Passons à l'isomorphisme d'anneaux donné par \( \phi_{\tau}\).
    \begin{subproof}
        \item[Bien définie]

            Si \( \bar Q_1=\bar Q_2\) alors \( Q_2=Q_1+RP\) pour un certain \( R\in \eK[X]\). Dans ce cas,
            \begin{subequations}
                \begin{align}
                    \phi_{\tau}(Q_2)=\overline{ \tau(Q_2) }&=\overline{ \tau(Q_1)+\tau(RP) }\\
                    &=\overline{ \tau(Q_1)+\tau(R)\tau(P) }\\
                    &=\overline{ \tau(Q_1) }.
                \end{align}
            \end{subequations}
            Ok  pour bien définie.

        \item[Injection]

            Si \( \phi_{\tau}(\bar Q_1)=\phi_{\tau}(\bar Q_2)\) alors \( \overline{ \tau(Q_1) }=\overline{ \tau(Q_2) }\), ce qui signifie que
            \begin{equation}
                \tau(Q_1)=\tau(Q_2)+R\tau(P)
            \end{equation}
            pour un certain \( R\in \eK'[X]\). Vu que \( \tau\colon \eK[X]\to \eK'[X]\) est un isomorphisme, nous pouvons y appliquer \( \tau^{-1}\) pour trouver :
            \begin{equation}
                Q_1=Q_2+\tau^{-1}(R)P,
            \end{equation}
            ce qui signifie que \( \bar Q_1=\bar Q_2\).

        \item[Surjection]

            Un élément de \( \eK'[X]/\big( \tau(P) \big)\) est de la forme \( \bar Q\) avec \( Q\in \eK'[X]\). Cela est l'image par \( \phi_{\tau}\) de l'élément \( \overline{ \tau^{-1}(Q) }\in \eK[X]/(P)\).

        \item[Morphisme]

            Nous vous laissons vérifier que l'application \( \phi_{\tau}\) est un morphisme d'anneaux.

    \end{subproof}
\end{proof}

%---------------------------------------------------------------------------------------------------------------------------
\subsection{Bézout}
%---------------------------------------------------------------------------------------------------------------------------

\begin{theorem}[Bézout] \label{ThoBezoutOuGmLB}
    Les polynômes \( P_1,\ldots,P_n\) dans \( \eK[X]\) sont étrangers entre eux si et seulement s'il existe des polynômes \( Q_1,\ldots,Q_n\in\eK[X]\) tels que
    \begin{equation}
        P_1Q_1+\cdots+P_nQ_n=1.
    \end{equation}
\end{theorem}
\index{Bézout!polynômes}
\index{théorème!Bézout!polynômes}

Deux polynômes \( P\) et \( Q\) ne sont donc pas premiers entre eux s'il existe des polynômes \( x\) et \( y\) tels que l'identité de Bézout soit vérifiée :
\begin{equation}    \label{EqkbbzAi}
    xP+yQ=0;
\end{equation}
cette dernière pourra être écrite en termes de la matrice de Sylvester, voir sous-section~\ref{subsecSQBJfr}.

\begin{lemma}       \label{LemuALZHn}
    Soient \( (P_i)_{i=1,\ldots,n}\in \eK[X]\) des polynômes étrangers deux à deux. Alors les polynômes \begin{equation} Q_i=\prod_{j\neq i}P_j \end{equation}
    sont étrangers entre eux\footnote{Et non seulement deux à deux.}.
\end{lemma}

\begin{lemma}[\cite{SQxrsoL}]   \label{LemzwkYdn}
    Soit \( \eK\) un corps commutatif et \( \eA\subset \eK\) un sous
    anneau de \( \eK\).  Alors \( \eA[X] \), vu comme idéal de \( \eK[X]
    \), est un idéal premier.

    En d'autres termes, si \( \phi\in \eK[X]\), et s'il existe \( Q\in \eK[X]\) unitaire tel que \( \phi Q\in \eA[X]\), alors \( \phi\in \eA[X]\).
\end{lemma}

%---------------------------------------------------------------------------------------------------------------------------
\subsection{Lemme et théorème de Gauss}
%---------------------------------------------------------------------------------------------------------------------------

\begin{theorem}[Théorème de Gauss]  \label{ThoLLgIsig}
    Soient \( P,Q,R\in \eK[X]\) tels que \( P\) soit premier avec \( Q\) et divise \( QR\). Alors \( P\) divise \( R\).
\end{theorem}
\index{théorème!Gauss!polynômes}

\begin{proof}
    Étant donné que \( P\) est premier avec \( Q\), le théorème de Bézout\footnote{théorème~\ref{ThoBezoutOuGmLB}.} nous donne \( U,V\in \eK[X]\) tels que \( PU+QV=1\). De plus il existe un polynôme \( S\) tel que \( PS=QR\). En multipliant l'identité de Bézout par \( R\), nous obtenons
    \begin{equation}
        R=PUR+QVR=PUR+VPS=P(UR+VS),
    \end{equation}
    ce qui signifie que \( P\) divise \( R\).
\end{proof}

Le lemme suivant est une généralisation du lemme de Gauss dans \( \eZ\) (lemme~\ref{LemSdnZNX}).
\begin{lemma}[Lemme de Gauss\cite{fJhCTE}]       \label{LemEfdkZw}   \index{lemme!Gauss!polynômes}\index{Gauss!lemme!polynômes}
    Soient les polynômes unitaires \( P,Q\in \eQ[X]\). Si \( PQ\in\eZ[X]\), alors \( P\) et \( Q\) sont tous deux dans \( \eZ[X]\).
\end{lemma}

\begin{proof}
    Soit \( a>0\) le plus petit entier tel que \( aP\in\eZ[X]\) (c'est le PPCM des dénominateurs) et de la même façon \( b>0\) le plus petit entier tel que \( bQ\in \eZ[X]\). On pose \( P_1=aP\) et \( Q_1=bQ\).

    Si \( ab=1\), alors \( a=b=1\) et nous avons tout de suite \( P,Q\in \eZ[X]\). Nous supposons donc \( ab>1\) et nous considérons \( p\), un diviseur premier de \( ab\). Ensuite nous considérons la projection
    \begin{equation}
        \pi_p\colon \eZ[X]\to (\eZ/p\eZ)[X].
    \end{equation}
    Par définition \( abPQ=P_1Q_1\in \eZ[X]\); en prenant la projection,
    \begin{equation}
        \pi_p(P_1)\pi_p(Q_1)=\pi_p(P_1Q_1)=\pi_P(ab)\pi_p(PQ)=0
    \end{equation}
    parce que \( \pi_p(ab)=0\). Étant donné que \( (\eZ/p\eZ)[X]\) est intègre (théorème~\ref{ThoBUEDrJ}), nous avons soit \( \pi_p(P_1)=0\) soit \( \pi_p(Q_1)=0\). Supposons pour fixer les idées que \( \pi_p(P_1)=0\). Alors \( P_1=pP_2\) pour un certain \( P_2\in \eZ[X]\). Par ailleurs \( P\) est unitaire et \( P_1=aP\), donc le coefficient de plus haut degré de \( P_1\) est \( a\), et nous concluons que \( p\) divise \( a\).

    Mettons \( a=pa'\). Dans ce cas, \( pa'P=P_1=pP_2\), et donc \( a'P=P_2\in \eZ[X]\). Cela contredit la minimalité de \( a\).
\end{proof}

%---------------------------------------------------------------------------------------------------------------------------
\subsection{Polynômes sur un corps et pgcd}
%---------------------------------------------------------------------------------------------------------------------------

Nous savons qu'un corps est un anneau intègre (lemme~\ref{LemAnnCorpsnonInterdivzer}). De plus l'ensemble des polynômes sur un anneau intègre est lui-même un anneau intègre (théorème~\ref{ThoBUEDrJ}). Donc la notion de pgcd à utiliser dans le cas de \( \eK[X]\) est celle de la définition~\ref{DefrYwbct}.

\begin{lemma}[Unicité du pgcd à inversibles près]      \label{LEMooXISOooNAMeVX}
    Soit un corps commutatif \( \eK\) et \( S\subset \eK[X]\). Si \( \delta_1\) et \( \delta_2\) sont des pgcd\footnote{Définition \ref{DefrYwbct}.} de \( S\), alors \( \delta_1=k\delta_2\) avec \( k\in \eK\).
\end{lemma}

\begin{proof}
    Nous savons que \( \delta_1\) est un pgcd de \( S\), mais que \( \delta_2\) divise \( S\). Donc \( \delta_2\divides \delta_1\). De la même manière, \( \delta_1\divides \delta_2\). Il existe donc \( A,B\in \eK[X]\) tels que \( \delta_1=A\delta_2\) et \( \delta_2=B\delta_1\). En substituant,
    \begin{equation}
        \delta_1=AB\delta_1.
    \end{equation}
    Mais \( \eK[X]\) possède la propriété de simplification par la proposition~\ref{DEFooTAOPooWDPYmd}\ref{ITEMooQNTFooSRrVPK}. Donc \( AB=1\). Cela signifie entre autres que \( A\) et \( B\) sont des inversibles de \( \eK[X]\).

    Or les seuls inversibles dans \( \eK[X]\) sont les éléments de \( \eK\); si vous en doutez, pensez que le degré de \( AB\) est supérieur ou égal à celui de \( A\).
\end{proof}

\begin{normaltext}
    En général, lorsque nous dirons «le» pgcd d'un ensemble, nous parlerons du pgcd unitaire, qui existe et est bien défini par le lemme~\ref{LEMooXISOooNAMeVX}.
\end{normaltext}


\begin{lemma}[\cite{ooSGHGooAJzIjz}]        \label{LEMooIAGMooHUQtUs}
    Soit un corps commutatif \( \eK\), deux polynômes quelconques \( A,B\in \eK[X]\) et un polynôme unitaire \( G\).

    Nous avons \( G=\pgcd(A,B)\) si et seulement si les deux conditions suivantes sont satisfaites :
    \begin{enumerate}
        \item
            Il existe \( U,V\in \eK[X]\) tels que \( AU+BV=G\),
        \item
            \( G\) divise \( A\) et \( B\).
    \end{enumerate}
\end{lemma}

\begin{proof}
    Une implication dans chaque sens.

    \begin{subproof}
        \item[\( \Rightarrow\)]

        Si $G$ est le pgcd de $A$ et $B$, il est clair que $G|A$ et $G|B$.  Il reste donc à montrer l'existence des polynômes $U$ et $V$ vérifiant $AU+BV=G$. Vu que \( G\) divise \( A\) et \( B\), il existe des polynômes $A_1,B_1$ tels que $A=GA_1$ et $B=GB_1$.

        Nous montrons que les polynômes $A_1$ et $B_1$ sont premiers entre eux. S'ils ont un diviseur commun $D$, alors $GD$ est un diviseur commun à $A$ et $B$.  Or, $G$ est le pgcd de $A$ et $B$ donc $GD|G$ ; $D$ ne peut être qu'un polynôme constant (c'est-à-dire un élément de \( \eK\)). Mais comme \( G\) est unitaire, le coefficient du terme de plus haut degré de \( GD\) doit être \( 1\). Donc \( D=1\).  L'élément \( 1\) est l'unique diviseur commun de \( A_1\) et \( B_1\); donc $A_1$ et $B_1$ sont donc bien premiers entre eux.

        D'après le théorème de Bézout~\ref{ThoBezoutOuGmLB}, il existe donc $U$ et $V$ tels que $A_1U+B_1V=1$. En multipliant par $G$, nous obtenons l'égalité voulue : $AU+BV=G$.

        \item[\( \Leftarrow\)]

        Si $G$ vérifie les deux conditions, montrons que $G$ est le pgcd de $A$ et $B$. Nous savons déjà (par hypothèse) que $G$ divise $A$ et $B$, il reste à montrer que tous les diviseurs commun à $A$ et $B$ divisent aussi $G$. Soit donc $D$ un diviseur commun à $A$ et $B$ : il existe $A_1$ et $B_1$ tels que $A=DA_1$ et $B=DB_1$. Nous savons que $G=AU+BV$ donc $G=D(A_1U+B_1V)$, et $D|G$.

        Par définition, $G$ est bien le pgcd de $A$ et $B$.
        \end{subproof}
\end{proof}
Notons qu'en supprimant la condition d'unitarité de \( G\), le résultat tient presque : il suffit de remplacer partout «le pgcd» par «un pgcd».

\begin{lemma}[\cite{ooSGHGooAJzIjz}]       \label{LEMooGNAMooXRpgBn}
Soient deux polynômes $A,B$ premiers entre eux. Si le polynôme \( P\) est divisible par $A$ et par $B$ alors $P$ est divisible par $AB$.
\end{lemma}

\begin{proof}
    Vu que \( A\divides P\), il existe \( Q_1\in \eK[X]\) tel que \( P=AQ_1\). Mais \( B\) divise \( P=AQ_1\) alors que \( B\) est premier avec \( A\); donc d'après le théorème de Gauss~\ref{ThoLLgIsig} : $B|Q_1$.

    Il existe donc $Q_2\in \eK[X]$ tel que $Q_1=BQ_2$. On a donc $P=ABQ_2$ : $P$ est bien divisible par $AB$.
\end{proof}

% TODO : voir qui référentie ce lemme et mettre la référence vers le point correct.
\begin{lemma}[\cite{ooSGHGooAJzIjz}]   \label{LemUELTuwK}
    Quelques propriétés du PGCD\footnote{Définition \ref{DefrYwbct}.} dans les polynômes. Soient des polynômes \( P,Q,R\in \eK[X]\).
    \begin{enumerate}
        \item       \label{ITEMooBPOZooYeFGjl}
            Nous avons l'égalité\footnote{Notez l'analogie avec le lemme~\ref{LemiVqita}.}
            \begin{equation}
                \pgcd(P,PQ+R)=\pgcd(P,R).
            \end{equation}
        \item       \label{ITEMooUVGRooNSGDZn}
            Si \( Q \) et \( R\) sont premiers entre eux,
            \begin{equation}
                \pgcd(P,QR)=\pgcd(P,Q)\pgcd(P,R)
            \end{equation}
        \item       \label{ITEMooYXAHooXibkgV}
            Si \( P\) et \( Q\) sont premiers entre eux,
            \begin{equation}
                \pgcd(P,QR)=\pgcd(P,R)
            \end{equation}
    \end{enumerate}
\end{lemma}
\index{pgcd!polynômes}

\begin{proof}
    Dans la suite si \( A\) et \( B\) sont des polynômes, nous dirons «les diviseurs de \( \{ A,B \}\)» pour parler des diviseurs communs de \( A\) et \( B\).

    \begin{enumerate}
        \item[\ref{ITEMooBPOZooYeFGjl}]

    Nous montrons que \( \{ P,PQ+R \}\) a les mêmes diviseurs que \( \{ P,R \}\).

    D'une part, si \( A\divides\{ P,PQ+R \}\), alors il existe des polynômes \( B_1\) et \( B_2\) tels que \( P=AB_1\) et \( PQ+R=AB_2\). Donc
            \begin{equation}
                R=AB_2-PQ=AB_2-AB_1Q=A(B_2-B_1Q),
            \end{equation}
            et nous concluons que \( A\) divise \( R\).

            D'autre part, si \( A\divides\{ P,R \}\) alors il existe des polynômes \( B_1\) et \( B_2\) tels que \( P=AB_1\) et \( R=AB_2\). Donc
            \begin{equation}
                PQ+R=AB_1Q+AB_2=A(B_1Q+B_2),
            \end{equation}
            et \( A\) divise \( PQ+R\).

            Conclusion : les paires \( \{ P,PQ+R \}\) et \( \{ P,R \}\) ont même ensemble de diviseurs, et donc même \( \pgcd\).

        \item[\ref{ITEMooUVGRooNSGDZn}]

            Nous avons trois polynômes $P,Q,R$ et nous savons que $Q$ et $R$ sont premiers entre eux. Nous notons : $G_1=\pgcd(P,Q)$ et $G_2=\pgcd(P,R)$.  Il faut montrer que $G_1G_2$ est le pgcd de $P$ et $QR$; pour cela nous allons utiliser le lemme~\ref{LEMooIAGMooHUQtUs}.

            \begin{subproof}
                \item[\( \exists U,V\) tels que $G_1G_2=PU+QRV$ ]

                    Vu que $G_1=\pgcd(P,Q)$, il existe $U_1$ et $V_1$ tels que $G_1=PU_1+QV_1$ (lemme~\ref{LEMooIAGMooHUQtUs}).
On a de même : $G_2=PU_2+RV_2$. En prenant le produit :
                    \begin{equation}
                        G_1G_2=(PU_1+QV_1)(PU_2+RV_2)=P(PU_1U_2+RU_1V_2+QV_1V_2)+QR(V_1V_2).
                    \end{equation}
                    Donc c'est bon pour ce point.

                \item[\( G_1\) et \( G_2\) sont premiers entre eux]

                    Si $D$ est un diviseur commun à $G_1$ et $G_2$, alors $D$ divise $Q$ et $R$ qui sont premiers entre eux ; $D$ ne peut être qu'un polynôme constant. Tous les diviseurs communs de \( G_1\) et \( G_2\) sont dans \( \eK\). Mais le \( \pgcd\) est par définition un diviseur commun unitaire, donc \( \pgcd(G_1,G_2)=1\). Cela signifie que \( G_1\) et \( G_2\) sont premiers entre eux (définition~\ref{DefZHRXooNeWIcB}).

                \item[\( G_1G_2\divides QR\)]
                    En effet : $G_1|Q$ et $G_2|R$ donc $G_1G_2|QR$.
                \item[\( G_1G_2\divides P\)]
                    Le polynôme $P$ est divisible par $G_1$ et par $G_2$, et de plus $G_1$ et $G_2$ sont premiers entre eux. Donc le lemme~\ref{LEMooGNAMooXRpgBn} conclu que \( P\) est divisible par \( G_1G_2\).

            \end{subproof}

\item[\ref{ITEMooYXAHooXibkgV}]

Supposons d'abord que \( A\in \eK[X]\) divise \( P\) et \( QR\). Le théorème de Bézout~\ref{ThoBezoutOuGmLB} assure l'existence de polynômes $U$ et $V$ tels que $PU+QV=1$. Ensuite l'hypothèse de division nous donne des polynômes \( B_1\) et \( B_2\) tels que $P=AB_1$ et $QR=AB_2$.  Nous avons :
            \begin{equation}
                    1=PU+QV=AB_1U+QV.
            \end{equation}
            Cela prouve que \( A\) est premier avec $Q$ grâce encore à Bézout, mais dans l'autre sens. Donc \( A\) est premier avec \( Q\) et \( A\divides QR\). Donc \( A|R\) par le théorème de Gauss~\ref{ThoLLgIsig}.

    Dans l'autre sens, si $A|R$ alors on a évidemment : $A|QR$.

    Les diviseurs de $\{P,QR\}$ sont exactement les diviseurs de $\{P,R\}$. En conséquence, nous concluons que les paires $\{P,QR\}$ et $\{P,R\}$ ont le même $\pgcd$.

    \end{enumerate}

\end{proof}

