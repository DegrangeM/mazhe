% This is part of Exercices et corrigés de CdI-1
% Copyright (c) 2011,2015, 2019-2020
%   Laurent Claessens
% See the file fdl-1.3.txt for copying conditions.

\begin{corrige}{Implicite0008}

	Un point $(x,y,z)$ est dans $M$ lorsque
	\begin{equation}
		\left\{
		\begin{array}{ll}
			x+y+z=0\\
			x^2+y^2+z^2=1.
		\end{array}
		\right.
	\end{equation}
	La seconde équation est celle d'une sphère de rayon $1$, donc $M$ est inclus dans la sphère et est donc bornée. D'autre part, tant la première équation (un plan dans $\eR^3$) que la seconde définissent des ensembles fermés. En tant qu'intersection de fermés, l'ensemble $M$ est fermé. Maintenant que nous savons que $M$ est ferme et borné, nous savons que $M$ est compact.

	Maintenant nous posons
	\begin{equation}
		\begin{aligned}
			F\colon \eR^3&\to \eR^2 \\
			(x,y,z)&\mapsto (x+y+z,x^2+y^2+z^2). 
		\end{aligned}
	\end{equation}

	Les relations de définition pour $X(y)$ et $Z(y)$ sont
	\begin{equation}			\label{EqHuitEqsqDefXZ}
		\left\{
		\begin{array}{ll}
			X(y)+y+Z(y)=0\\
			X(y)^2+y^2+z(y)^2-1=0.
		\end{array}
		\right.
	\end{equation}
	Étant donné que
	\begin{equation}
		\begin{vmatrix}
			\frac{ \partial F_1 }{ \partial x }	&	\frac{ \partial F_1 }{ \partial z }	\\ 
			\frac{ \partial F_2 }{ \partial x }	&	\frac{ \partial F_2 }{ \partial z }	
		\end{vmatrix}
		=
		\begin{vmatrix}
			1	&	1	\\ 
			2x	&	2z	
		\end{vmatrix}
		=
		2(z-x).
	\end{equation}
	Si $y=0$, les équations deviennent
	\begin{equation}
		\left\{
		\begin{array}{ll}
			x+z=0\\
			x^2+y^2=1,
		\end{array}
		\right.
	\end{equation}
	de façon que $z-x\neq 0$. Donc le condition \( F\big( X(y), y, Z(y) \big)=0\) définit bien les fonctions \( X\) et \( Z\) au voisinage de $y=0$ et de $(x,z)=(0,0)$.

	L'approximation de $X(y)$ cherchée est évidemment Taylor, c'est-à-dire
	\begin{equation}
		X(0)+y(\partial_yX)(0).
	\end{equation}
	Pour trouver $X(0)$, nous devons résoudre le système
	\begin{equation}
		\left\{
		\begin{array}{ll}
			X(0)+Z(0)=0\\
			X(0)^2+Z(0)^2=1,
		\end{array}
		\right.
	\end{equation}
	donc les solutions sont $X(0)=\pm 1/2$ et $Z(0)=-X(0)$. Nous avons donc le choix de travailler autour de deux points différents :
	\begin{equation}		\label{EqPossXYzeroUn}
		(x,y,z)=(\frac{1}{ 2 },0,-\frac{1}{ 2 })
	\end{equation}
	ou bien
	\begin{equation}		\label{EqPossXYzerodeux}
		(x,y,z)=(-\frac{1}{ 2 },0,\frac{1}{ 2 }).
	\end{equation}
	Nous trouvons $X'(0)$ en dérivant les deux équations de définition \eqref{EqHuitEqsqDefXZ} par rapport à $y$ et en résolvant le système par rapport à $X'(0)$. Le système à résoudre est
	\begin{equation}
		\left\{
		\begin{array}{ll}
			X'(y)+1+Z'(y)=0\\
			2X(y)X'(y)+2y+2Z(y)Z'(y)=0.
		\end{array}
		\right.
	\end{equation}
	La résolution donne
	\begin{equation}
		X'(y)=\frac{ -2y+2Z(y) }{ 2X(y)-2Z(y) }.
	\end{equation}
	Étant donné que nous connaissons les valeurs de $X(0)$ et $Z(0)$ données par \eqref{EqPossXYzeroUn}, nous trouvons
	\begin{equation}
		X'(0)=-\frac{1}{ 2 }.
	\end{equation}
	Dans ce cas, l'approximation est
	\begin{equation}
		X(y)\sim \frac{ 1 }{2}-\frac{ 1 }{2}y.
	\end{equation}
	Si nous avions choisit de travailler avec la possibilité \eqref{EqPossXYzerodeux}, alors nous aurions obtenu
	\begin{equation}
		X(y)\sim -\frac{ 1 }{2}-\frac{ 1 }{2}y.
	\end{equation}

\end{corrige}
