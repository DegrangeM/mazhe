% This is part of Exercices et corrigés de CdI-1
% Copyright (c) 2010-2011, 2020
%   Laurent Claessens
% See the file fdl-1.3.txt for copying conditions.

\begin{corrige}{reserve0002}

	Nous nous mettons dans une boule $B$ centrée en $(t,t)$ de rayon suffisamment petit pour que tous les $x$ et $y$ soient dans la même période de la fonction sinus. Pour chaque $(x,y)$, il existe un $\xi\in\mathopen[ x , y \mathclose]$ tel que
	\begin{equation}
		\sin(x)-\sin(y)=\cos(\xi)(x-y).
	\end{equation}
    Un tel \( \xi\) existant pour chaque couple \( (x,y)\) nous pouvons considérer une fonction $\xi$ définie sur $B$ telle que $\sin(x)-\sin(y)=\cos\big(\xi(x,y)\big)(x-y)$. Certes, il existe plusieurs telles fonctions (en particulier n'importe quelle valeur convient en $\xi(t,t)$), mais nous en considérons une. Nous avons donc
	\begin{equation}
		\frac{ \sin(x)-\sin(y) }{ x-y }=\cos\big( \xi(x,y) \big).
	\end{equation}
	
	Étant donné que $\xi(x,y)\in\mathopen[ x , y \mathclose]$ lorsque $x\neq y$, nous avons la limite
	\begin{equation}
		\lim_{(x,y)\to(t,t)}\xi(x,y)=t.
	\end{equation}
	En effet, soient $\epsilon<0$ et $\delta$ tels que pour tout $(x,y)\in B\big( (t,t),\delta \big)$, nous ayons $| x-y |<\epsilon$. Nous avons alors
	\begin{equation}
		\big| \xi(x,y)-x \big|<| y-x |<\epsilon.
	\end{equation}
	Notez que nous avons établi une limite pour $\xi$ en ayant aucune idée de sa continuité ! Finalement nous avons
	\begin{equation}
		\lim_{(x,y)\to(t,t)}f(x,y)=\lim_{(x,y)\to(t,t)}\cos\big( \xi(x,y) \big)=\cos\big( \lim_{(x,y)\to(t,t)}\xi(x,y) \big)=\cos(t).
	\end{equation}

	La fonction
	\begin{equation}
		f(x,y)=\frac{ \sin(x)-\sin(y) }{ x-y }
	\end{equation}
	peut donc être continument prolongée par $\cos(x)$ sur la ligne $x=y$.
	
\end{corrige}
