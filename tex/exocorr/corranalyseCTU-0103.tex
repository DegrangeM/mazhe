% This is part of Analyse Starter CTU
% Copyright (c) 2015, 2020
%   Laurent Claessens,Carlotta Donadello
% See the file fdl-1.3.txt for copying conditions.

\begin{corrige}{analyseCTU-0103}

  \begin{enumerate}
  \item Soit $f:\,\eR\to\eR$ une fonction paire.
      \begin{enumerate}
          \item
            $f_1 (x) = f(-x)$  est paire parce que  $f_1 (-x) = f(-(-x)) = f(-x) = f_1(x)$.
        \item
            $f_2 (x) = f(x)-1$ est paire parce que $f_2 (-x) = f(-x)-1 =f(x)-1 = f_2(x)$.
        \item
            $f_3 (x)= 
            \begin{cases}
              f(x) &\text{si } x>0, \\
              f(-x)& \text{si } x\leq 0,
            \end{cases}$ est paire parce que  $f_3(x) = f(|x|)$, donc  $f_3(-x) = f(|-x|)= f(|x|) = f_3 (x)$
        \item
            $f_5(x) = \sqrt{(f(x))^2}$ est paire parce que $f_5(x) = |f(x)|$, donc  $f_5(-x) = |f(-x)| = |f(x)| = f_5(x)$.
        \item
            $f_6(x)= |f(x)| + f(x)$ est paire comme somme de deux fonctions paires.
        \item
            $f_7(x) = xf(x)$ est impaire comme produit d'une fonction paire et une impaire : $f_7(-x) = -xf(-x) = -xf(x) = -f_7(x)$.
        \item
            $f_8(x) = g(f(x))$ avec $g$ impaire. Le tout est paire parce que $f_8(-x) = g(f(-x))= g(f(x))=f_8(x)$.
      \end{enumerate}
\item  Soit $f:\,\eR\to\eR$ une fonction impaire.
      \begin{enumerate}
          \item
            $f_1 (x) = f(-x)$ est impaire et  $f_1 (-x) = f(-(-x)) = -f(-x) = -f_1(x)$
          \item
            $f_2 (x) = f(x)-1 $ n'est  ni paire ni impaire et  $f_2 (-x) = f(-x)-1 =-f(x)-1$. Elle est paire si et seulement si \( f(x)=0\) pour tout \( x\).
          \item
            $f_3 (x)= 
            \begin{cases}
              f(x) &\text{si } x>0, \\
              f(-x)& \text{si } x\leq 0,
            \end{cases}$ est paire parce que  $f_3(x) = f(|x|)$. Donc $f_3(-x) = f(|-x|)= f(|x|) = f_3 (x)$
          \item
            $f_5(x) = \sqrt{(f(x))^2}$ est paire parce que $f_5(x) = |f(x)|$, donc $f_5(-x) = |f(-x)| = |-f(x)| = f_5(x)$
          \item
            $f_6(x)= |f(x)| + f(x)$ n'est ni paire ni impaire parce que $f_6(-x) = |f(x)| - f(x) \neq f_6(x)$ et $-f_6(x)$. \( f_6\) est paire si et seulement si \( f(x)=0\) pour tout \( x\).
          \item
            $f_7(x) = xf(x)$ est paire parce qu'elle est le produit de deux fonctions impaires : 
            $f_7(-x) = -xf(-x) = xf(x) = f_7(x)$.
          \item
            $f_8(x) = g(f(x))$ où \( g\) est impaire. Le tout est impair parce que  
            $f_8(-x) = g(-f(x))= -g(f(x))=-f_8(x)$
      \end{enumerate}
  \end{enumerate} 
\end{corrige}
