% This is part of Mes notes de mathématique
% Copyright (c) 2011-2017, 2019
%   Laurent Claessens
% See the file fdl-1.3.txt for copying conditions.

% TODO : mettre à jour pour suivre les recommandations
%http://openclassrooms.com/courses/guide-des-bonnes-pratiques-en-latex


\usepackage{etex}
\usepackage{latexsym}
\usepackage{amsfonts}
\usepackage{amsmath}
\usepackage{amsthm}
\usepackage{amssymb}
\usepackage{mathrsfs}
\usepackage{mathabx}           % For \divides et \widehat.
\usepackage{bbm}

\usepackage{enumitem}
\setlist[enumerate,1]{label=(\arabic*),ref=(\arabic*)}
\setlist[enumerate,2]{label=(\arabic{enumi}\alph{enumii}), ref=(\arabic{enumi}\alph{enumii})}


\usepackage{wrapfig}
\usepackage{framed}
\let\Sun\undefined
\let\Moon\undefined
\let\Venus\undefined
\let\Mars\undefined
\let\Jupiter\undefined
\let\Saturn\undefined
\let\Uranus\undefined
\let\Mercury\undefined
\let\Venus\undefined
\let\Mars\undefined
\let\Jupiter\undefined
\let\Saturn\undefined
\let\Uranus\undefined
\let\Neptune\undefined
\let\Pluto\undefined
\let\Earth\undefined
\let\Aries\undefined
\let\Taurus\undefined
\let\Gemini\undefined
\let\Leo\undefined
\let\Libra\undefined
\let\Scorpio\undefined
\usepackage{marvosym}       % marvosym redefines the previous symbols
\usepackage{tikz}           % Configuration at 1829426939
\usepackage{calc}
\usetikzlibrary{calc}
\usetikzlibrary{patterns}
\usepackage{color}
\usepackage{graphicx}                   % Pour l'inclusion d'image en pfd.


% Increase the size to the box containing the section number in the TOC. If not, ``10.20.12`` is too long
\usepackage[subfigure]{tocloft}
\renewcommand\cftsubsecnumwidth{1.5cm}
\renewcommand\cftsecnumwidth{1cm}



\usepackage{subfigure}

\usepackage{fancyvrb}
\usepackage{stmaryrd}       % Pour le \obslash
\usepackage{xstring}        % Utilisé pour les références vers wikipédia
\usepackage{cases}
\usepackage{lscape}         % pour l'environnement landscape, utilisé dans la correction corr0076.tex
\usepackage{multicol}
%\usepackage{xspace}
\usepackage[normalem]{ulem}		% Pour le barré, commande \sout
\usepackage[all]{xy}
\let\second\undefined      % le paquet mathabx définit \second
\let\degree\undefined       % le paquet mathabx définit \degree

\usepackage[cdot,thinqspace,amssymb]{SIunits}    %   1410612643

\usepackage{textcomp}
\usepackage{lmodern}
\usepackage[a4paper,margin=2cm,left=2.6cm]{geometry}

\usepackage{hyperref}
\usepackage{makeidx}
%\usepackage{minitoc}
\usepackage[nottoc]{tocbibind}      % Biblio inclue  dans la table des matières.
\usepackage[numbers]{natbib}        % le champ URL dans le fichier bibtex
\usepackage[refpage]{nomencl}       % Some configuration at   1338719836
\usepackage{array}

\usepackage{scalerel,stackengine}   % reallywidehat

% We do not use 'exocorr' for Frido.
%\ifbool{isFrido}{}{\usepackage[fr]{exocorr} }

\input{exocorr}


\usepackage[english,french]{babel}
\providecommand\frenchbsetup[1]{}
\providecommand\frenchsetup[1]{\frenchbsetup{#1}}
\frenchsetup{
  og=«,fg=»
}


% The macro ``pdftitle'' is changed by 'pytex' and so depends on 
% the context.
\makeatletter
%\@ifundefined{hypersetup}{}{%
  \hypersetup{%
    pdfauthor={Claessens, Laurent and al.},
    pdftitle={\pdftitle},
    pdfsubject={giulietta},
    pdfkeywords={agrégation, frido, giulietta},
    pdfmenubar=true,
    colorlinks=true,
    pdfpagemode=UseNone,
    pdfstartview=FitH,
  }%
%}
\makeatother

\usepackage{listingsutf8}   % Has to be called after babel


% Some  'configuration' for tikz       1829426939
\newcounter{defHatch}
\newcounter{defPattern}
\setcounter{defHatch}{0}
\setcounter{defPattern}{0}
\newcommand{\utilde}[1]{\underline{#1}}

\input{configuration}

%%%%%%%%%%%%%%%%%%%%%%%%%%
%
%   Trucs mathématiques
%
%%%%%%%%%%%%%%%%%%%%%%%%

% ENSEMBLES DE NOMBRES
\newcommand{\eA}{\mathbbm{A}}
\newcommand{\eB}{\mathbbm{B}}
\newcommand{\eC}{\mathbbm{C}}
\newcommand{\eD}{\mathbbm{D}}
\newcommand{\eE}{\mathbbm{E}}
\newcommand{\eF}{\mathbbm{F}}
\newcommand{\eG}{\mathbbm{G}}
\newcommand{\eH}{\mathbbm{H}}
\newcommand{\eK}{\mathbbm{K}}
\newcommand{\eL}{\mathbbm{L}}
\newcommand{\eM}{\mathbbm{M}}
\newcommand{\eN}{\mathbbm{N}}
\newcommand{\eP}{\mathbbm{P}}
\newcommand{\eQ}{\mathbbm{Q}}
\newcommand{\eR}{\mathbbm{R}}
\newcommand{\eS}{\mathbbm{S}}
\newcommand{\eT}{\mathbbm{S}}
\newcommand{\eZ}{\mathbbm{Z}}


% ENSEMBLES de fonctions
\newcommand{\aL}{\mathcal{L}}       % Les applications linéaires
\newcommand{\cL}{L}       % Les applications linéaires continues
\newcommand{\aC}{\mathcal{C}}       % Les fonctions C^1, C^2 etc
\newcommand{\swS}{\mathscr{S}}          % L'ensemble des fonctions Schwartz
\newcommand{\swD}{\mathscr{D}}          % L'ensemble des fonctions Cinfinie à support compact.
\newcommand{\swE}{\mathscr{E}}          % L'espace des fonctions qu'on peut déformer (le grand epsilon)
                                  % Les espaces de distributions correspondants sont les mêmes avec un prime.
\newcommand{\comC}{\mathcal{C}}       % Le commutant d'un endomorphisme.

\DeclareMathOperator{\Pol}{Pol}
\DeclareMathOperator{\Poly}{\mathcal{P}}        % Space of polynomials
\newcommand{\sdS}{\mathcal{S}}      % L'ensemble des subdivisions d'un intervalle.
\newcommand{\TF}{\mathcal{F}} %% Transformée de Fourier.
\newcommand{\mtu}{\mathbbm{1}}              % La matrice unité
\newcommand{\caract}{\mathbbm{1}}    % Characteristic function of a set
\newcommand{\catC}{\mathscr{C}}     % \catX is for the categories
\newcommand{\catD}{\mathscr{D}}
\newcommand{\catM}{\mathscr{M}}
\newcommand{\oB}{\mathfrak{B}}          % The space of bounded operators
\newcommand{\oK}{\mathcal{K}}           % L'espace des opérateurs compacts
\newcommand{\oL}{\mathscr{L}}           % Le L pour l'idéal de Schatten-von Neumann. Cela est aussi l'ensemble des opérateurs linéaires sur des espaces vectoriels.
\newcommand{\oP}{\mathscr{P}}           % Le P est pour l'ensemble des projections dans une algèbre de VN.

\newcommand{\euler}{\mbox{\rm e}}
\newcommand{\dist}{\operatorname {dist}}
\newcommand{\sii}{\mbox{\rm \scriptsize i}}

% LES NEWCOMMAND UN PEU ACTIFS

\newcommand*{\conclusion}{\emph{Conclusion~:~}}

% DECLARE MATH OPERATORS
\DeclareMathOperator{\fl}{fl}
\DeclareMathOperator{\SimplePrec}{sp}
\DeclareMathOperator{\NaN}{NaN}
\DeclareMathOperator{\supp}{supp}
\DeclareMathOperator{\Iso}{Iso}
\DeclareMathOperator{\Isom}{Isom}       % The group of isometries
\DeclareMathOperator{\Aut}{Aut}
\DeclareMathOperator{\Ob}{Ob}           % The ``set'' of object of a category
\DeclareMathOperator{\val}{val}     % valuation d'un polynôme
\DeclareMathOperator{\res}{res}     % Le résultant de deux polynômes
\DeclareMathOperator{\Inv}{Inv}     % L'application inverse
\DeclareMathOperator{\SP}{SP}
\DeclareMathOperator{\Conf}{Conf}
\DeclareMathOperator{\gsl}{\mathfrak{sl}}
\DeclareMathOperator{\go}{\mathfrak{o}}
\DeclareMathOperator{\gsu}{\mathfrak{su}}
\DeclareMathOperator{\gsp}{\mathfrak{sp}}
\DeclareMathOperator{\so}{\mathfrak{so}}
\DeclareMathOperator{\Spin}{Spin}
\DeclareMathOperator{\mSpin}{Spin}      % La commande \mSpin dénote l'application Spin qui va de SL(2,C) vers L^+ flèche.
\DeclareMathOperator{\spin}{\mathfrak{spin}}
\DeclareMathOperator{\Cl}{Cl}
\DeclareMathOperator{\Cliff}{Cl}
\DeclareMathOperator{\CCliff}{\Cliff^{\eC}}     % Changement de notation par rapport à avant.
\DeclareMathOperator{\volume}{vol}
\DeclareMathOperator{\esssup}{ess-\sup}
\DeclareMathOperator{\gpAff}{Aff}
\DeclareMathOperator{\Vect}{Vect}
\DeclareMathOperator{\eae}{eae}         % espace affine engendré
\DeclareMathOperator{\gpSymp}{Symp}
\DeclareMathOperator{\horsp}{hor}
\DeclareMathOperator{\Dim}{Dim}
\DeclareMathOperator{\Harm}{Harm}       % The space of harmonic forms
\DeclareMathOperator{\Sign}{Sign}       % The sign function.
\DeclareMathOperator{\Rank}{Rank}
\DeclareMathOperator{\Res}{Res}
\DeclareMathOperator{\ResW}{\Res_W}
\DeclareMathOperator{\sgrad}{sgrad}     % symplectic gradient
\DeclareMathOperator{\stab}{\mathfrak{Stab}}
\DeclareMathOperator{\ad}{ad}
\DeclareMathOperator{\Ad}{Ad}
\DeclareMathOperator{\AD}{\textbf{Ad}}
\DeclareMathOperator{\Der}{\texttt{Der}}
\DeclareMathOperator{\Inn}{Inn}
\DeclareMathOperator{\Out}{Out}
\DeclareMathOperator{\Diff}{Diff}
\DeclareMathOperator{\biDiff}{bi-Diff}
\DeclareMathOperator{\Hol}{Hol}
\DeclareMathOperator{\Ray}{Ray}
\DeclareMathOperator{\mfsp}{\mathfrak{sp}}
\DeclareMathOperator{\Fr}{Fr}
\DeclareMathOperator{\Rad}{Rad}
\DeclareMathOperator{\niv}{Level}
\DeclareMathOperator{\Supp}{Supp}
\DeclareMathOperator{\sech}{sech}
\DeclareMathOperator{\Prim}{Prim}
\DeclareMathOperator{\Trans}{Trans}
\DeclareMathOperator{\Verm}{Verm}       % Pour le module de Verma
\DeclareMathOperator{\Irr}{Irr}
\DeclareMathOperator{\vol}{Vol}
\DeclareMathOperator{\Op}{Op}           % Le truc de la quantification de Weyl
\DeclareMathOperator{\rDi}{Di}
\DeclareMathOperator{\rRac}{Rac}
\DeclareMathOperator{\Maxpprod}{\texttt{pprod}}
\DeclareMathOperator{\Maxproj}{\texttt{proj}}
\DeclareMathOperator{\Maxcom}{\texttt{com}}
\DeclareMathOperator{\Maxcombis}{\texttt{combi6}}
\DeclareMathOperator{\Maxtables}{\texttt{table6}}
\DeclareMathOperator{\Maxtablesc}{\texttt{table6c}}
\DeclareMathOperator{\Maxdecomps}{\texttt{decomp6}} % Ces commandes sont pour Maxima.
\DeclareMathOperator{\Maxdecompsc}{\texttt{decomp6c}}
\DeclareMathOperator{\Maxtableqc}{\texttt{table4c}}
\DeclareMathOperator{\Maxtableq}{\texttt{table4}}
\DeclareMathOperator{\Maxdecompq}{\texttt{decomp4}}
\DeclareMathOperator{\Maxdecompqc}{\texttt{decomp4c}}
\DeclareMathOperator{\Maxomega}{\texttt{omega}}
\DeclareMathOperator{\Maxsymple}{\texttt{symple}}
\DeclareMathOperator{\Maxcycle}{\texttt{cycle}}
\DeclareMathOperator{\Maxsolve}{\texttt{solve}}
\DeclareMathOperator{\Maxdelxistar}{\texttt{delxistar}}
\DeclareMathOperator{\Maxxistar}{\texttt{xistar}}


\DeclareMathOperator{\signe}{sgn}
\DeclareMathOperator{\Vol}{Vol}
\DeclareMathOperator{\Int}{Int}     % Intérieur d'un ensemble.
\DeclareMathOperator{\Adh}{Adh}     % Adhérence d'un ensemble.
\DeclareMathOperator{\Ind}{Ind}     % l'indice d'un chemin en analyse complexe
\DeclareMathOperator{\Turn}{Turn} % Le nombre de tours d'une courbe fermée
\DeclareMathOperator{\IR}{Ind}       % indice de rotation
\DeclareMathOperator{\Cond}{Cond}   % conditionnement d'une matrice
\DeclareMathOperator{\Diam}{Diam}
\DeclareMathOperator{\id}{Id}
\DeclareMathOperator{\Graph}{Graph}
\DeclareMathOperator{\Conv}{\mathcal{C}onv}
\DeclareMathOperator{\pr}{\texttt{proj}}
\DeclareMathOperator{\dom}{dom}
\DeclareMathOperator{\Graphe}{Gr}
\DeclareMathOperator{\Spec}{Spec}   % spectre d'un opérateur
\DeclareMathOperator{\arctg}{arctg}
\DeclareMathOperator{\cotg}{cotg}
\DeclareMathOperator{\cosec}{cosec}
\DeclareMathOperator{\arcsinh}{arcsinh}
\DeclareMathOperator{\sinc}{sinc}   % Le sinus cardinal
\DeclareMathOperator{\PGL}{PGL}   % le groupe projectif
\DeclareMathOperator{\SO}{SO}
\DeclareMathOperator{\SL}{SL}
\DeclareMathOperator{\PSL}{PSL}   % Le groupe modulaire SL(2,Z)/Z2
\DeclareMathOperator{\gS}{S}        % le groupe des matrices symétriques et aussi les nombres complexes de norme 1.
\DeclareMathOperator{\SU}{SU}
\DeclareMathOperator{\su}{\mathfrak{su}}
\DeclareMathOperator{\gU}{U}
\DeclareMathOperator{\gu}{\mathfrak{u}}
\DeclareMathOperator{\gO}{O}            % On mets un g devant les groupes dont le nom est juste une lettre, ou est ambigu.
\DeclareMathOperator{\diag}{diag}
\DeclareMathOperator{\Hom}{Hom}
\DeclareMathOperator{\Domain}{Dom}      % Domaine of an operator
\DeclareMathOperator{\Domaine}{Dom}
\DeclareMathOperator{\Dom}{Domaine}

\DeclareMathOperator{\real}{Re}        % Real and imaginary part of a complex number.
\DeclareMathOperator{\imag}{Im}        % These names are from Sage.

\DeclareMathOperator{\Image}{Image}        % ... avec \Image qui donne l'image d'une fonction ou d'un opérateur.
\DeclareMathOperator{\rang}{rang}
\DeclareMathOperator{\Kernel}{Ker}
\DeclareMathOperator{\Span}{Span}
\DeclareMathOperator{\End}{End}     % L'ensemble des endomorphismes
\DeclareMathOperator{\Cyl}{Cyl}
\DeclareMathOperator{\tr}{Tr}       % la trace
\DeclareMathOperator{\Tr}{Tr}       % la trace
\DeclareMathOperator{\trace}{Tr}       % la trace
\renewcommand{\det}{\mathop{\mathrm{det}}\nolimits}           % le déterminant
\DeclareMathOperator{\Majorant}{Maj}
\DeclareMathOperator{\codim}{codim} % pour la codimension.
\DeclareMathOperator{\diam}{diam} % le diamètre d'un ensemble.
\DeclareMathOperator{\Var}{Var}     % Variance d'une variable aléatoire.
\DeclareMathOperator{\Fun}{Fun}     % Ensemble des applications d'un ensemble vers l'autre.
\DeclareMathOperator{\Cov}{Cov}     % la covariance.
\DeclareMathOperator{\gr}{gr}     % le groupe engendré
\DeclareMathOperator{\pgcd}{pgcd}
\DeclareMathOperator{\ppcm}{ppcm}
\DeclareMathOperator{\Frob}{Frob}
\DeclareMathOperator{\Card}{Card}       % Le cardinal d'un ensemble.
\DeclareMathOperator{\Stab}{Stab}       % Le stabilisateur d'un point sous l'action d'un groupe.
\DeclareMathOperator{\Frac}{Frac}       % le corps des fractions d'un anneau
\DeclareMathOperator{\Aff}{Aff}         %  l'espace affine engendré


\newcommand{\dpt}[3]{#1\colon #2\to #3}
\newcommand{\subdem}[1]{\par\noindent {\it #1.} }       % TODO : use subproof instead

\newcommand{\slim}{\mathrm{s\lim}}
\newcommand{\uwlim}{\mathrm{uw\lim}}

\newcommand{\sod}{\mathfrak{so}(2)}
\newcommand{\SOdn}{\SO(2,n)}
\newcommand{\sodn}{  {\mathfrak{so}}(2,n)   }
\newcommand{\soun}{\mathfrak{so}(1,n)}
\newcommand{\SLdc}{\SL(2,\eC)}
\newcommand{\sldr}{\mathfrak{sl}(2,\eR)}
\newcommand{\SOun}{\SO(1,n)}
\newcommand{\gud}{\Gamma_{(2)}}
\newcommand{\Sput}{\Spin(1,3)}
\newcommand{\Sppq}{\Spin(p,q)}
\newcommand{\sppq}{\mathfrak{spin}(p,q)}
\newcommand{\Sopq}{\SO(p,q)}
\newcommand{\sopq}{\mathfrak{so}(p,q)}
\newcommand{\gl}{\mathfrak{gl}}
\newcommand{\Fix}{\operatorname{Fix}}
\newcommand{\cat}{\operatorname{cat}}

\newcommand{\ecarts}{ecarts}
\newcommand{\angl}{\quext{Anglais ?}}
\newcommand{\intr}[1]{\mathaccent 23 {#1}}
\newcommand{\UU}{\intr{U}}

\newcommand{\donc}{\Rightarrow}

\newcommand{\ecart}{ecart} % quand tu trouveras une meilleure traduction...

%--------- Alphabets math

\newcommand{\mfa}{\mathfrak{a}}
\newcommand{\mfb}{\mathfrak{b}}
\newcommand{\mfg}{\mathfrak{g}}
\newcommand{\mfs}{\mathfrak{s}}
\newcommand{\mfM}{\mathfrak{M}}

\newcommand{\scrC}{\mathscr{C}}
\newcommand{\scrD}{\mathscr{D}}         %Demande le paquetage mathrsfs.
\newcommand{\scrE}{\mathscr{E}}
\newcommand{\scrM}{\mathscr{M}}
\newcommand{\scrS}{\mathscr{S}}
\newcommand{\hS}{\mathscr{S}}           % C'est lui qui donne la singularité
\newcommand{\hF}{\mathscr{F}}           % \hF donne la partie libre de l'espace

\newcommand{\cA}{\mathfrak{A}}          % Pour les C^* algebres; comme ça je peux choisir.
\newcommand{\cB}{\mathcal{B}}           % Le mathfrak{B} est pour l'ensemble des operateurs bornés.
\newcommand{\cun}{\mtu}             % L'unite dans les $C^*$-algèbres.
\newcommand{\cI}{\mathfrak{I}}

\newcommand{\Cinf}{C^{\infty}}
\newcommand{\vnM}{\mathfrak{M}}         % Le M des algèbres de von Neumann
\newcommand{\hodge}{\star}         % the Hodge dual


%\newcommand{\bmodE}{\mathcal{\bar E}}      % Pour le module conjugué
\newcommand{\modM}{\mathfrak{M}}
\newcommand{\modN}{\mathfrak{N}}


\newcommand{\pH}{\mathscr{H}}           % L'espace de Hilbert pour la physique
\newcommand{\rR}{\mathcal{R}}           % Les \r? sont les lettres pour les rayons de l'espace de Hilbert.
\newcommand{\rC}{\mathcal{C}}

\newcommand{\cdA}{\mathscr{A}}
\newcommand{\cdE}{\mathscr{E}}          % Les ensembles de fonctions continuement d\'erivables
\newcommand{\cdD}{\mathscr{D}}          % L'ensemble des fonctions à support compact.

%-------Overline,underline, hat,tilde
\newcommand{\uw}{\underline{w}}
\newcommand{\uv}{\underline{v}}
\newcommand{\uW}{\underline{W}}
\newcommand{\uvH}{\underline{H}}
\newcommand{\ovH}{\overline{H}}
\newcommand{\ovN}{\overline{N}}
\newcommand{\ovx}{\overline{x}}
\newcommand{\ovy}{\overline{y}}
\newcommand{\ovj}{\overline{j}}
\newcommand{\ova}{\overline{a}}
\newcommand{\os}{\overline{s}}
\newcommand{\oJ}{\overline{J}}
\newcommand{\oX}{\overline{X}}
\newcommand{\oY}{\overline{Y}}
\newcommand{\ovR}{\overline{R}}
\newcommand{\olG}{\overline{\lG}}
\newcommand{\olR}{\overline{\lR}}
\newcommand{\olS}{\overline{\lS}}

\newcommand{\uG}{\underline{G}}
\newcommand{\uX}{\underline{X}}
\newcommand{\uAN}{\underline{AN}}

\newcommand{\oui}{\overline{1}_i}
\newcommand{\oalpha}{\overline{\alpha}}
\newcommand{\obeta}{\overline{\beta}}
\newcommand{\oeta}{\overline{\eta}}
\newcommand{\oxi}{\overline{\xi}}
\newcommand{\ogamma}{\overline{\gamma}}
\newcommand{\odelta}{\overline{\delta}}
\newcommand{\tA}{\widetilde{A}}
\newcommand{\tK}{\widetilde{K}}
\newcommand{\tN}{\widetilde{N}}
\newcommand{\tR}{\widetilde{R}}
\newcommand{\tilr}{\widetilde{r}}
\newcommand{\tx}{\tilde{x}}
\newcommand{\ty}{\tilde{y}}
\newcommand{\tE}{\tilde{E}}
\newcommand{\tF}{\tilde{F}}
\newcommand{\tH}{\tilde{H}}
\newcommand{\tX}{\tilde{X}}
\newcommand{\tY}{\tilde{Y}}
\newcommand{\utX}{\underline{X}}                %Il faut encore trouver comment souligner avec un tilde.
\newcommand{\utE}{\underline{E}}
\newcommand{\utH}{\underline{H}}
\newcommand{\talpha}{\tilde{\alpha}}
\newcommand{\tomega}{\tilde{\omega}}
\newcommand{\tgamma}{\tilde{\gamma}}
\newcommand{\tnab}{\widetilde{\nabla}}
\newcommand{\expotilde}{\widetilde{\hphantom{X}}}       % Waiting to know how to create a suitable tilde in twists_general.tex

\newcommand{\hpsi}{\hat{\psi}}
\newcommand{\ha}{\hat{a}}
\newcommand{\hg}{\hat{g}}
\newcommand{\hs}{\hat{s}}
\newcommand{\hu}{\hat{u}}
\newcommand{\hv}{\hat{v}}
\newcommand{\hw}{\hat{w}}
\newcommand{\hx}{\hat{x}}
\newcommand{\hy}{\hat{y}}
\newcommand{\hB}{\hat{B}}
\newcommand{\hX}{\hat{X}}
\newcommand{\hY}{\hat{Y}}


%https://tex.stackexchange.com/questions/100574/really-wide-hat-symbol
\stackMath
\newcommand\reallywidehat[1]{%
\savestack{\tmpbox}{\stretchto{%
  \scaleto{%
    \scalerel*[\widthof{\ensuremath{#1}}]{\kern-.6pt\bigwedge\kern-.6pt}%
    {\rule[-\textheight/2]{1ex}{\textheight}}%WIDTH-LIMITED BIG WEDGE
  }{\textheight}% 
}{0.5ex}}%
\stackon[1pt]{#1}{\tmpbox}%
}

%\Overline,underline, hat,tilde----------

%------Produits star
\newcommand{\stG}{\star^{G}}
\newcommand{\stW}{\star^{W}}
\newcommand{\stWt}{\star^{W}_{\theta}}
\newcommand{\stWh}{\star^{W}_{\hbar}}
\newcommand{\stX}{\star^{X}}
\newcommand{\stM}{\ast_M}
\newcommand{\stt}{\star_{\theta}}
%\newcommand{\st}{\ast}
%\Produits star-------------
\newcommand{\us}[1]{\frac{1}{#1}}
\newcommand{\dsd}[2]{\frac{\partial #1}{\partial #2}}
\newcommand{\me}[1]{(-1)^{#1}}
\newcommand{\xdp}[2]{#1\to #2}
\newcommand{\brak}[2]{\langle #1,#2\rangle}
\newcommand{\dsdd}[3]{\left.\frac{d}{d#2}#1\right|_{#2=#3}}
\newcommand{\Dsddb}[4]{\frac{d}{d#2}\Big[#1\Big]_{#3=#4}}
\newcommand{\Dsdd}[3]{ \Dsddb{#1}{#2}{#2}{#3}   }
\newcommand{\Dsddc}[3]{\frac{d}{d#2}\Big(#1\Big)_{#2=#3}}
\newcommand{\Dsddp}[3]{\frac{d}{d#2}\Big(#1\Big)_{#2=#3}}
\newcommand{\dDsdd}[5]{\frac{d}{d#2}\frac{d}{d#4}
           \Big[#1\Big]_{ \begin{subarray}{l}#4=#5\\#2=#3\end{subarray} }}

\newcommand{\DDsdd}[5]{\frac{d}{d#2}\frac{d}{d#4}
           \Big[#1\Big]_{ \begin{subarray}{l}#4=#5\\#2=#3\end{subarray} }}

\newcommand{\mfo}{\vartheta}

\newcommand{\hbeta}{^{\beta}}
\newcommand{\hkappa}{^{\kappa}}
\newcommand{\bgamma}{_{\gamma}}
\newcommand{\bdelta}{_{\delta}}
\newcommand{\hmu}{^{\mu}}
\newcommand{\hnu}{^{\nu}}
\newcommand{\bab}{_{\alpha\beta}}

\newcommand{\heta}{^{\eta}}
\newcommand{\bxi}{_{\xi}}
\newcommand{\hsigma}{^{\sigma}}

%Extension de l'alphabet grec------------


\newcommand{\lA}{\mathfrak{a}}
\newcommand{\lB}{\mathfrak{b}}
\newcommand{\lF}{\mathfrak{f}}
\newcommand{\lG}{\mathfrak{g}}      % Pour les algèbres de Lie en général; comme ça je peux choisir,
\newcommand{\lH}{\mathfrak{h}}      % mais le mal du \mG est déjà loin !
\newcommand{\lI}{\mathfrak{i}}
\newcommand{\lJ}{\mathfrak{j}}
\newcommand{\lK}{\mathfrak{k}}
\newcommand{\lL}{\mathfrak{L}}
\newcommand{\lM}{\mathfrak{m}}
\newcommand{\lN}{\mathfrak{n}}
\newcommand{\lP}{\mathfrak{p}}
\newcommand{\lQ}{\mathfrak{q}}
\newcommand{\lR}{\mathfrak{r}}
\newcommand{\lS}{\mathfrak{s}}
\newcommand{\lU}{\mathfrak{u}}
\newcommand{\lX}{\mathfrak{x}}
\newcommand{\lZ}{\mathfrak{z}}

\newcommand{\lW}{\mathcal{W}}

\newcommand{\iA}{\mathcal{A}}
\newcommand{\iK}{\mathcal{K}}
\newcommand{\iN}{\mathcal{N}}           % Pour les éléments de décomposition d'Iwasawa
\newcommand{\iP}{\mathcal{P}}
\newcommand{\iR}{\mathcal{R}}
\newcommand{\iAH}{\mathcal{A_H}}
\newcommand{\iKH}{\mathcal{K_H}}
\newcommand{\iKQ}{\iK_{\sQ}}
\newcommand{\iNH}{\mathcal{N_H}}
\newcommand{\iPH}{\mathcal{P_H}}
\newcommand{\iRH}{\mathcal{R_H}}

\newcommand{\curR}{\mathrm{R}}      % La courbure scalaire d'un triple spectral

\newcommand{\SUR}{\mathrm{R}}
\newcommand{\SUA}{\mathrm{A}}       % les SUx sont pour les parties de SU(1,n).
\newcommand{\SUN}{\mathrm{N}}

\newcommand{\suqA}{\mathcal{A}}   % The algebra of SU_q(n)

\newcommand{\sA}{\mathcal{A}}
\newcommand{\sG}{\mathcal{G}}
\newcommand{\sH}{\mathcal{H}}           % Pour les morceaux de SO(2,n) et SO(1,n)
\newcommand{\sK}{\mathcal{K}}           % En fait, les morceaux de AdS_l par Iwasawa vont aussi êres notes avec des \sX
\newcommand{\sN}{\mathcal{N}}
\newcommand{\sP}{\mathcal{P}}
\newcommand{\sQ}{\mathcal{Q}}
\newcommand{\sR}{\mathcal{R}}
\newcommand{\sS}{\mathcal{S}}
\newcommand{\sZ}{\mathcal{Z}}

\newcommand{\etS}{\mathcal{S}}      % L'ensemble des etats sur une $C^*$-algebre.
\newcommand{\etP}{\mathcal{P}}      % Les états purs

\newcommand{\tsA}{\widetilde{\mathcal{A}}}
\newcommand{\tsN}{\widetilde{\mathcal{N}}}
\newcommand{\tsR}{\widetilde{\mathcal{R}}}

\newcommand{\ovf}{\overline{ f }}
\newcommand{\cvec}{\mathfrak{X}}
\newcommand{\Wedge}{\bigwedge}
\newcommand{\LogOu}{\vee}
\newcommand{\LogEt}{\wedge}
\newcommand{\cuppr}{\sharp}     % En attendant de trouver mieux.
\newcommand{\svec}{\mathcal{B}}     % Les vecteurs C^{\infty} d'une action
\newcommand{\Dir}{\mathcal{D}}

\newcommand{\yG}{\mathcal{G}}  % L'algèbre de Lie dans le truc sur YM

%---------Constructions d'un besoin passager

\newcommand{\nomscript}[1]{\emph{#1}}
\newcommand{\dtau}{\partial_{\tau}}
\newcommand{\du}{\partial_{u}}
\newcommand{\dphi}{\partial_{\phi}}
\newcommand{\frZ}[2]{   \frac{2(#1,#2)}{(#1,#1)}     }
\newcommand{\heC}{^{\eC}}
\newcommand{\beC}{_{\eC}}
\newcommand{\beR}{_{\eR}}
\newcommand{\heR}{^{\eR}}
\newcommand{\blF}{_{\lF}}
\newcommand{\etalH}{\eta_{\lH}}
\newcommand{\lHeR}{\lH_{\eR}}
\newcommand{\lGeR}{\lG_{\eR}}
\newcommand{\lFeC}{\lF^{\eC}}
\newcommand{\lGeC}{\lG^{\eC}}
\newcommand{\lHeC}{\lH^{\eC}}
\newcommand{\lbha}{\beta^{\alpha}}
\newcommand{\lbba}{\beta_{\alpha}}
\newcommand{\aba}{a_{\alpha}}
\newcommand{\abb}{a_{\beta}}
\newcommand{\abg}{a_{\gamma}}
\newcommand{\abd}{a_{\delta}}
\newcommand{\abmb}{a_{-\beta}}
\newcommand{\abmg}{a_{-\gamma}}
\newcommand{\abmd}{a_{-\delta}}
\newcommand{\abab}{a_{\alpha+\beta}}
\newcommand{\abma}{a_{-\alpha}}
\newcommand{\abmr}{a_{-\rho}}
\newcommand{\abr}{a_{\rho}}
\newcommand{\abbp}{a_{\beta'}}
\newcommand{\xbg}{x_{\gamma}}
\newcommand{\xbd}{x_{\delta}}
\newcommand{\hbb}{h_{\beta}}
\newcommand{\xbma}{x_{-\alpha}}
\newcommand{\xbb}{x_{\beta}}
\newcommand{\xbmb}{x_{-\beta}}
\newcommand{\xbmab}{x_{\alpha-\beta}}
\newcommand{\xbmamb}{x_{-\alpha-\beta}}
\newcommand{\rmg}[1]{ \big( \rho(#1)-\gamma(#1) \big) }
\newcommand{\lRlR}{[\lR,\lR]}
\newcommand{\cloi}{\overline{i}}
\newcommand{\cloj}{\overline{j}}
\newcommand{\cloip}{\overline{i'}}
\newcommand{\dD}{\scrD}

\newcommand{\AutA}{\Aut(\lA)}
\newcommand{\IntA}{\Int(\lA)}
\newcommand{\AutB}{\Aut(\lB)}
\newcommand{\IntB}{\Int(\lB)}
\newcommand{\RM}{\pr_{\sQ}\sR}
\newcommand{\Oexp}[3]{ \Omega_2\Big(  e^{\ad#1}#2,e^{\ad#1}#3  \Big)  }
\newcommand{\Xrnz}{X\times(\eR^N\setminus\{o\})}
\newcommand{\DxaDxb}{D_x^{\alpha} D\bxi\hbeta}
\newcommand{\abxi}{|\xi|}
\newcommand{\baz}[2]{\{#1e_i\}_{#2}}
\newcommand{\decompss}[3]{%
\begin{equation}
\begin{split}
\mfs_1&=\{#1\}\\
\mfs_2&=\{#2\}#3
\end{split}
\end{equation}
}
\newcommand{\delE}[2]{\delta_{#1}E_{#2}}
\newcommand{\dcr}[1]{[[#1]]}
\newcommand{\dga}[2]{\gamma_{#1}\gamma_{#2}}
\newcommand{\tga}[3]{\gamma_{#1}\gamma_{#2}\gamma_{#3}}
\newcommand{\qga}[4]{\gamma_{#1}\gamma_{#2}\gamma_{#3}\gamma_{#4}}
\newcommand{\rhoM}{\rho^M}
\newcommand{\hperp}{^{\perp}}
   % Mes produits scalaires                 % Je crois que je vais unifier sous \braket pour le produit < x | y > et sous \scal pour < x , y >.
\newcommand{\braket}[2]{ \langle #1|#2\rangle }
\newcommand{\ket}[1]{ | #1\rangle }
\newcommand{\bra}[1]{ \langle #1| }
\newcommand{\scalp}[2]{  (#1|#2) }
\newcommand{\scald}[2]{ \scal{#1}{#2} }
\newcommand{\scalh}[2]{ \braket{#1}{#2} }
\newcommand{\ketbra}[2]{|#1\rangle\,\langle #2|}


\newcommand{\dptvb}[3]{#1\stackrel{#2}{\longrightarrow}#3}
\newcommand{\ovv}{\overline{v}}
\newcommand{\ovX}{\overline{X}}
\newcommand{\ovS}{\overline{S}}

\newcommand{\bghd}[3]{#1_{#2}^{\phantom{#2}#3}}

% These commands were \mathbf instead of overline but I have the ``Too many math alphabets used in version normal`` error.
\newcommand{\bE}{\overline{ E }}
\newcommand{\bA}{\overline{ A }}
\newcommand{\bB}{\overline{ B }}
\newcommand{\BX}{\overline{ X }}

\newcommand{\gab}{g_{\alpha\beta}}
\newcommand{\sbeta}{\sigma_{\beta}}
\newcommand{\salpha}{\sigma_{\alpha}}

\newcommand{\quextproj}{\quext{In project\ldots}}
%\newcommand{\tb}{\tilde{b}}
\newcommand{\gamsai}{\gamma_{\alpha j}}
\newcommand{\bsa}{{}_{(\alpha)}{}}
\newcommand{\gamaj}{\gamma_{\alpha j}}
\newcommand{\gamai}{\gamma_{\alpha i}}
\newcommand{\psisa}{\psi\bsa}
\newcommand{\opK}{\mathfrak{K}}     % Compact operators
\newcommand{\opB}{\mathfrak{B}}     % Bounded operators
\newcommand{\invtible}{^{\times}}   % Cette commande est en attendant de trouver un symbole plus spécifique à mettre sur les ensembles pour désigner leur partir inversible.
\newcommand{\osint}{\widetilde{\int}}
\newcommand{\Lie}[1]{\mathfrak{Lie}(#1)}
\newcommand{\qvect}[4]{(#1,#2,#3,#4)}
\newcommand{\osiint}{\widetilde{\iint}}
\newcommand{\osiiint}{\widetilde{\iiint}}
% La commande suivante est tirée de symbols-letter.pdf pour écrire une intégrale avec une barre dedans
\def\Xint#1{\mathchoice
   {\XXint\displaystyle\textstyle{#1}}%
   {\XXint\textstyle\scriptstyle{#1}}%
   {\XXint\scriptstyle\scriptscriptstyle{#1}}%
   {\XXint\scriptscriptstyle\scriptscriptstyle{#1}}%
   \!\int}
\def\XXint#1#2#3{{\setbox0=\hbox{$#1{#2#3}{\int}$}
     \vcenter{\hbox{$#2#3$}}\kern-.5\wd0}}
\newcommand{\ddashint}{\Xint=}
\newcommand{\dashint}{\Xint-}


% Le premier argument est optionnel, c'est pour ajouter un [math.QA] par exemple pour la nouvelle numérotation de arXiv. Comme tu le vois, la valeur par défaut est vide.
\newcommand{\arxiv}[2][]{%
\newline
\ifthenelse{\equal{#1}{}}{%                     Tester si un argument optionnel est passé ou non.
    \href{http://arxiv.org/abs/#2}{{\tt arXiv:#2}}%     Si tu ne mets pas ce %, il y a un problème d'espace.
            }
            {%
    \href{http://arxiv.org/abs/#2}{{\tt arXiv:#2}[#1]}%     Si tu ne mets pas ce %, il y a un problème d'espace.
}%                                  Ce %-ci aussi est indispensable pour un espace à éviter avant le . ajouté par bibtex.
}               % Fin de la commande \arxiv


% -- L'environement suivant est taxé de la classe article.cls, sauf que j'ai enlevé la possibilité que ce soit sur une page de titre.
\newcommand\abstractname{Abstract}
\makeatletter
  \newenvironment{abstract}{%
      \if@twocolumn
        \section*{\abstractname}%
      \else
        \small
        \begin{center}%
          {\bfseries \abstractname\vspace{-.5em}\vspace{\z@}}%
        \end{center}%
        \quotation
      \fi}
      {\if@twocolumn\else\endquotation\fi}
\makeatother

\newcommand{\PB}[2]{\left\{#1,#2\right\}}
                    % Les opérateurs définis pour Maxima
\newcommand{\LoL}{\mathscr{L}}      % Lorentz group
\newcommand{\LoP}{\mathscr{P}}      % Poincaré group
\newcommand{\f}{\frac}


%%%%%%%%%%%%%%%%%%%%%%%%%%
%
%   Les théorèmes et choses attenantes
%
%%%%%%%%%%%%%%%%%%%%%%%%



\setcounter{tocdepth}{2}        % Profondeur de la table des matières
\setcounter{secnumdepth}{3}     % Profondeur dans le texte

\newcounter{numtho}
\newcounter{numprob}
\newcounter{numTheme}           % For the thematic index, see InternalLinks

\makeatletter
\@addtoreset{numtho}{chapter}
\@addtoreset{equation}{chapter}
%\ifbool{isFrido}{}
%{
%    \@addtoreset{CountExercice}{chapter}
%}

\makeatother

\newlength{\EnvSpace}
\setlength{\EnvSpace}{9pt}      % C'est la distance que je veux mettre avant et après les théorèmes, remarques, \ldots

\usepackage[inline]{showlabels}
\newtheoremstyle{MyTheorems}%
        {\EnvSpace}{\EnvSpace}%
        {\itshape}%
        {}%
        {\bfseries}{.}%
        {\newline}%
        {}%
\newtheoremstyle{MyExamples}%
        {\EnvSpace}{\EnvSpace}%
        {}%
        {}%
        {\bfseries}{.}%
        {\newline}%
        {}%
\newtheoremstyle{MyRemarks}%
        {\EnvSpace}{\EnvSpace}%
        {}%
        {}%
        {\bfseries}{.}%
        {\newline}%
        {}%


\newcounter{numloiphyz}

\newcounter{CounterExample}
\renewcommand{\theCounterExample}{\thechapter.\arabic{CounterExample}}
\renewcommand{\thenumtho}{\thechapter.\arabic{numtho}}


% The 'example' environment is by William Babonnaud.
\theoremstyle{MyExamples}
    \newtheorem{example}[numtho]{Exemple}
\DeclareRobustCommand{\trig}{%
   \ifmmode \quad\hbox{\triangle}
   \else
      \leavevmode\unskip\penalty9999 \hbox{}\nobreak\hfill
      \quad\hbox{$\triangle$}
   \fi
}

\let\exold\example
\let\endexold\endexample
\renewenvironment{example}{\pushQED{\trig}\exold}{\popQED\endexold}


\newenvironment{Aretenir}{\refstepcounter{numtho}\begin{oframed}\noindent{\bf À retenir \thenumtho}\newline}{\end{oframed}\vspace{\EnvSpace}}

\theoremstyle{MyRemarks}    \newtheorem{remark}[numtho]{Remarque}

                \newtheorem{amusement}[numtho]{Amusement}
                \newtheorem{erreur}[numtho]{Error}
                \newtheorem{normaltext}[numtho]{}

\theoremstyle{MyTheorems}
            \newtheorem{lemma}[numtho]{Lemme}
            \newtheorem{corollary}[numtho]{Corolaire}
            \newtheorem{theorem}[numtho]{Théorème}
            \newtheorem{definition}[numtho]{Définition}
            \newtheorem{proposition}[numtho]{Proposition}
            \newtheorem{theoremDef}[numtho]{Théorème-définition}
            \newtheorem{corollaryDef}[numtho]{Corolaire-définition}
            \newtheorem{propositionDef}[numtho]{Proposition-définition}
            \newtheorem{lemmaDef}[numtho]{Lemme-définition}
			\newtheorem{loiphyz}[numloiphyz]{Loi numéro}

            %\newtheorem{exo}[CountExercice]{Exercice}       % C'est provisoire, pour Chafaï

% La numérotation des équations change dans les corrigés
\renewcommand{\theequation}{\thechapter.\arabic{equation}}

% This counter is defined in SystemeCorr.sty.
%\ifbool{isFrido}{}{
%    \renewcommand{\theCountExercice}{\arabic{CountExercice}}       
%}


\newcommand{\defe}[2]{\textbf{#1}\index{#2}}

%\renewcommand{\theenumi}{(\arabic{enumi})}
%\renewcommand{\labelenumi}{\theenumi}
%\renewcommand{\theenumii}{\arabic{enumi}\alph{enumii}}
%\renewcommand{\labelenumii}{(\theenumii)}



%%%%%%%%%%%%%%%%%%%%%%%%%%
%
%   Les macros qui font des choses
%
%%%%%%%%%%%%%%%%%%%%%%%%

\newcommand{\mA}{\mathcal{A}}
\newcommand{\mB}{\mathcal{B}}
\newcommand{\mC}{\mathcal{C}}
\newcommand{\mCC}{\mathcal{CC}}
\newcommand{\mD}{\mathcal{D}}
\newcommand{\mE}{\mathcal{E}}
\newcommand{\mF}{\mathcal{F}}
\newcommand{\mG}{\mathcal{G}}
\newcommand{\mH}{\mathcal{H}}
\newcommand{\mI}{\mathcal{I}}
\newcommand{\mJ}{\mathcal{J}}
\newcommand{\mK}{\mathcal{K}}
\newcommand{\mL}{\mathcal{L}}
\newcommand{\mM}{\mathcal{M}}
\newcommand{\mN}{\mathcal{N}}
\newcommand{\mO}{\mathcal{O}}
\newcommand{\mP}{\mathcal{P}}
\newcommand{\mQ}{\mathcal{Q}}
\newcommand{\mR}{\mathcal{R}}
\newcommand{\mS}{\mathcal{S}}
\newcommand{\mT}{\mathcal{T}}
\newcommand{\mU}{\mathcal{U}}
\newcommand{\mV}{\mathcal{V}}
\newcommand{\mW}{\mathcal{W}}
\newcommand{\mZ}{\mathcal{Z}}


\newcommand{\scal}[2]{ \langle #1,#2\rangle }
\newcommand{\vect}[1]{\overrightarrow{#1}}

\newcommand{\tq}{\text{ tel que }}          %TODO : il y a un paquet qui doit être changé en \st
\newcommand{\st}{\text{ such that }}
\newcommand{\tqs}{\text{ tels que }}
\newcommand{\quext}[1]{\footnote{\textsf{#1}}}
\newcommand{\info}[1]{\texttt{#1}}

\newcommand{\normal}{\lhd}  % Cette notation n'est plus censée être utilisée.

\newcommand{\Borelien}{\mathcal{B}\text{or}}       % Les boréliens
\newcommand{\Lebesgue}{\mathcal{L}\text{eb}}       % La tribu de Lebesgue
\newcommand{\Baire}{\mathcal{B}\text{a}}       % La tribu de Baire
\newcommand{\tribA}{\mathcal{A}}            % Une tribu A
\newcommand{\tribB}{\mathcal{B}}
\newcommand{\tribC}{\mathcal{C}}
\newcommand{\tribD}{\mathcal{D}}
\newcommand{\tribE}{\mathcal{E}}            % Une tribu E
\newcommand{\tribF}{\mathcal{F}}            % Une tribu F
\newcommand{\tribM}{\mathcal{M}}            % Une tribu M
\newcommand{\tribN}{\mathcal{N}}
\newcommand{\tribT}{\mathcal{T}}

\newcommand{\affE}{\mathcal{E}}            % Un espace affine E
\newcommand{\affF}{\mathcal{F}}
\newcommand{\affG}{\mathcal{G}}

\newcommand{\statS}{\mathcal{S}}            % Un modèle statistique
\newcommand{\partP}{\mathcal{P}}            % L'ensemble des parties d'un ensemble
\newcommand{\pP}{\mathcal{P}}            % L'ensemble des nombres premiers

\newcommand{\polyP}{\mathcal{P}}            % L'ensemble des polynômes

\newcommand{\dB}{\mathscr{B}}       % la distribution de Bernoulli
\newcommand{\dE}{\mathscr{E}}       % la distribution exponentielle
\newcommand{\dirE}{\mathcal{E}}           % Dirichlet form
\newcommand{\dG}{\mathscr{G}}       % la distribution géométrique.
\newcommand{\dM}{\mathscr{M}}       % la distribution multinomiale
\newcommand{\dN}{\mathscr{N}}       % la distribution normale.
\newcommand{\dP}{\mathscr{P}}       % la distribution de Poisson.
\newcommand{\dT}{\mathscr{T}}       % la distribution de Student
\newcommand{\dU}{\mathscr{U}}       % la distribution uniforme

\newcommand{\hL}{\mathscr{L}}
%\newcommand{\cL}{\hL}           % Pour la partie Chafai

\newcommand{\modE}{\mathcal{E}}         % Le E des modules
\newcommand{\modF}{\mathcal{F}}         % Le F des modules
\newcommand{\hH}{\mathscr{H}}           % Le H des espaces de Hilbert

\newcommand{\ellE}{\mathcal{E}}         % Le E des ellipsoïde
\newcommand{\ellF}{\mathcal{F}}         % Le F des ellipsoïde

% Configuration for nomencl        1338719836

\makenomenclature
\renewcommand{\nomname}{Liste des notations}
%
% La syntaxe est facile, par exemple
%       $\SL(2,\eR)$\nomenclature[G]{$\SL(2,\eR)$}{Le groupe de matrices deux par deux de déterminant 1.}
\renewcommand{\nomgroup}[1]{%
    \ifthenelse{\equal{#1}{A}}{\item[\textbf{Algèbre}]}{}%
    \ifthenelse{\equal{#1}{B}}{\item[\textbf{Ensembles de matrices}]}{}%
    \ifthenelse{\equal{#1}{G}}{\item[\textbf{Géométrie}]}{}%
    \ifthenelse{\equal{#1}{M}}{\item[\textbf{Chaînes de Markov}]}{}%
    \ifthenelse{\equal{#1}{R}}{\item[\textbf{Théorie des groupes}]}{}%
    \ifthenelse{\equal{#1}{P}}{\item[\textbf{Probabilités et statistique}]}{}%
    \ifthenelse{\equal{#1}{Y}}{\item[\textbf{Analyse}]}{}%
    \ifthenelse{\equal{#1}{T}}{\item[\textbf{Topologie et théorie des ensembles}]}{}%
}
% Note pour moi-même : si cette liste est changée, il faut changer mon raccourcis dans Vim.
\newcommand*{\Sp}{\textup{Sp}}
\newcommand*{\GL}{\textup{GL}}      % Le groupe linéaire; je crois qu'ontologiquement c'est mieux que le DeclareMathOperator


\newcommand{\subprooflabel}[1]{\underline{\bf #1}}

\newenvironment{subproof}{\let\Olddescriptionlabel\descriptionlabel\let\descriptionlabel\subprooflabel \begin{description}}{\end{description}\let\descriptionlabel\Olddescriptionlabel}


% See also the file 'src_front_back_matter/157_thematique.tex' in which we
% open and close the intermediate file 'theme.toc'
\newcommand{\InternalLinks}[1]
{
    \refstepcounter{numTheme} \paragraph{Thème \arabic{numTheme} : #1} \label{THTOC\arabic{numTheme}}
    \immediate\write\themetoc{%
        \unexpanded{\ref}{THTOC\arabic{numTheme}} %
        : \unexpanded{#1\\}%
    }
}


\newenvironment{probleme}{\refstepcounter{numprob}\tiny\fbox{\bf Problèmes et choses à faire}\\}{\normalsize}

\newcommand{\TextePourISBN}{
    \begin{center}
ISBN : Y-Y-YYYYYYY-Y-Y
    \end{center}
}


% If one change something here, one should also change
% in python/generic.tex
\newcommand{\LicenceFDL}{
\begin{center}

            \includegraphics[width=1cm]{pictures_bitmap/gfdl-logo-small.png}

Copyright 2011-2019

Permission is granted to copy, distribute and/or modify this document under the terms of the \href{http://www.gnu.org/licenses/fdl-1.3.html}{GNU Free Documentation License}, Version 1.3 or any later version published by the Free Software Foundation; with no Invariant Sections, no Front-Cover Texts, and no Back-Cover Texts. A copy of the license is included in the chapter entitled ``GNU Free Documentation~License''.

\end{center}
}

\newcommand{\LogoEtLicence}{
\notbool{isFrido}{
\LicenceFDL
}
{
    %\LicenceCC
    \LicenceFDL
    \TextePourISBN
}
}


% FIN DE MES CHOSES %%%%%%%

\newcounter{exoNico}
\setcounter{exoNico}{1}
\newcommand{\exerNico}{\stepcounter{exoNico}{\bf Exercice }\arabic{exoNico}. }



\newtheorem{theo}{Th{\'e}or{\`e}me}[section]
\newtheorem{defn}{D{\'e}finition}
\newtheorem{prop}{Proposition}     % redef encore dans Chafaï
%\newtheorem{lem}{Lemme}[section]


%\newtheorem{rem}{Remarque}[section]
%\newcommand{\R}{\mathbb{R}}
%\newcommand{\Rn}{\eR}
%\newcommand{\Nn}{\eN}
\newcommand{\dem}{\textbf{D{\'e}monstration.}}
\newcommand{\vc}[1]{\boldsymbol{#1}}
\newcommand{\p}{\textrm{P}}
%\newcommand{\e}{\textrm{E}}
\newcommand{\mbt}{arbre binaire markovien}
\newcommand{\mbts}{arbres binaires markoviens}

%\newcommand{\ea}{\end{array}}


%%%%%%%%%%%%%%%%%%%%%%%%%%%%%%%%%%%%%%
%
% les petis yeux
%
%%%%%%%%%%%%%%%%%%%%%%%%%%%%%%%%%%%%%%%%%%%%%

\newcommand{\coolexo}{$\circledast\circledast$}
\newcommand{\boringexo}{$\circleddash\circleddash$}
\newcommand{\minsyndical}{$\odot\odot$}
\newcommand{\mortelexo}{$\obslash\oslash$}



%%%%%%%%%%%%%% TRUCS DE PIERRE %%%%%%%%%%%%%%%%%%%%%%


% Le paquet array est là pour faire fonctionner l'environement arrowcases dans les trucs de Pierre.

% À régler par l'utilisateur
\newlength{\arrowsep}\setlength{\arrowsep}{3pt}
\newlength{\arrowlength}\setlength{\arrowlength}{1cm}

\newenvironment{arrowcases}%
{\begin{cases}}
{\end{cases}}



\makeatletter %% \limite[condition]x x_0
\newcommand*{\limite}[3][\@empty]{\lim_{\substack{#2\rightarrow#3\\#1}}}
\makeatother

\newcommand*\sev{<} %

\let\ssi\iff
\newcommand*{\ideal}[1]{\{#1\}}
\newcommand*{\fleche}[1]{\stackrel{#1}\longrightarrow}

%\setcounter{CountExercice}{0}

\newcommand{\Acplx}{A_\cdot}
\newcommand{\Bcplx}{B_\cdot}
\newcommand{\toisom}{\fleche\simeq}
\newcommand{\D}{\partial}
\newcommand{\lied}{\mathcal L}
\newcommand*{\nom}[1]{\textsc{#1}}
\newcommand*{\inner}{\imath}
\newcommand*{\newexo}{}
\newcommand*{\principe}{}
\newcommand*{\etape}{}
\newcommand*{\preuve}{}
\newcommand*{\exr}{\item}

\newcommand*{\crochets}[1]{\Bigl[ #1 \Bigr]}
\newcommand*{\llbrack}[1]{\left\lbrack #1 \right\lbrack}
\newcommand*{\rlbrack}[1]{\left\rbrack #1 \right\lbrack}
\newcommand*{\lrbrack}[1]{\left\lbrack #1 \right\rbrack}
\newcommand*{\rrbrack}[1]{\left\rbrack #1 \right\rbrack}


\newcommand*{\alg}[1]{\mathcal{#1}} % Algèbre
\newcommand*{\TT}{\ens T}% Tore !
\newcommand*{\topologie}{\mathscr{T}}
\newcommand*{\Topologie}{\textcursive{T}}
\newcommand{\LL}{\text{\textup{L}}} %% Espace de Lebesgue droit
\newcommand{\Ll}{\mathcal{L}} %% Lebesgue ronde
\newcommand{\sigmaalgebre}[1]{\mathcal{#1}} %% Une sigma algèbre...
\DeclareMathOperator{\SymMatrix}{Sym}
\DeclareMathOperator{\ASymMatrix}{ASym}
\newcommand{\Sym}{\SymMatrix}
\newcommand{\ASym}{\ASymMatrix}
%\newcommand{\transpose}[1]{{\vphantom{#1}}^{\mathit t}{\/#1}}
\newcommand*{\dprime}{{\prime\prime}}

%% Maths : Symboles divers
\newcommand{\surj}{\vers}
\newcommand{\isom}{\simeq}
\newcommand*{\Tau}{\alg T}
\newcommand{\cdv}{\mathfrak{X}} % Champs de vecteurs



\newcommand*{\abs}[1]{\left\vert#1\right\vert} % Valeur absolue.
\newcommand*{\module}[1]{\left\vert#1\right\vert} % Valeur absolue.
\newcommand*{\norme}[1]{\left\Vert#1\right\Vert} % norme
\newcommand*{\ordre}[1]{\left\vert#1\right\vert} % L'ordre d'un élément.
\newcommand*{\scalprod}[2]{\left\langle #1,#2\right\rangle}
\let\dual\ast

\newcommand*{\pardef}{\stackrel{\text{def}}{=}} % Par définition.
\newcommand*{\iffdefn}{\stackrel{\text{def}}{\iff}} % Par définition.
\newcommand*{\Defn}[1]{\emph{#1}} %
\newcommand*{\tensor}{\otimes}
\newcommand*{\pder}[2]{\frac{\partial #1}{\partial #2}}

\DeclareRobustCommand{\sfrac}[3][\mathrm]{\hspace{0.1em}%
  \raisebox{0.4ex}{$#1{\scriptstyle
#2}$}\hspace{-0.1em}/\hspace{-0.07em}%
  \mbox{$#1{\scriptstyle #3}$}}



% The following uses xstring in order to replace specific characters by %xx codes.
% See the table http://www.utf8-chartable.de/
\providecommand{\MakeUTFPerCent}[1]{% 
   \StrSubstitute{#1}({\%28}[\result]% 
   \expandafter\StrSubstitute\expandafter{\result}){\%29}[\result]% 
   \expandafter\StrSubstitute\expandafter{\result}{à}{\%C3\%A0}[\result]% 
   \expandafter\StrSubstitute\expandafter{\result}{â}{\%C3\%A2}[\result]% 
   \expandafter\StrSubstitute\expandafter{\result}{ç}{\%C3\%A7}[\result]% 
   \expandafter\StrSubstitute\expandafter{\result}{è}{\%C3\%A8}[\result]% 
   \expandafter\StrSubstitute\expandafter{\result}{é}{\%C3\%A9}[\result]% 
   \expandafter\StrSubstitute\expandafter{\result}{ê}{\%C3\%AA}[\result]% 
   \expandafter\StrSubstitute\expandafter{\result}{ù}{\%C3\%B9}[\result]% 
   \expandafter\StrSubstitute\expandafter{\result}{û}{\%C3\%BB}[\result]% 
   \expandafter\StrSubstitute\expandafter{\result}{ô}{\%C3\%B4}[\result]% 
   \expandafter\StrSubstitute\expandafter{\result}_{\_}[\result]% 
} 

   %\expandafter\StrSubstitute\expandafter{\result}{#}{\%23}[\result]%       If you want a # in the URL you still have to write \# in the source.

%------------------------
% Links to wikipedia.
%------------------------
% Typical use is
% \wikipedia{fr}{Norme_(mathématiques)}{Norme}
% It creates the link \href to the right page on wikipedia, replacing special characters by their respective %xx codes.
\providecommand{\wikipedia}[3]{% 
   \saveexpandmode\noexpandarg 
   \MakeUTFPerCent{#2}% 
   \restoreexpandmode 
   \href{http://#1.wikipedia.org/wiki/\result}{#3}% 
} 

\providecommand{\wikiversity}[3]{% 
   \saveexpandmode\noexpandarg 
   \MakeUTFPerCent{#2}% 
   \restoreexpandmode 
   \href{http://#1.wikiversity.org/wiki/\result}{#3}% 
} 


% 1410612643  :   L'option amssymb sert à éviter un conflit avec la commande \square de amssymb. Note que cette dernière n'est plus accessible.
%The change from SIunits to siunitx causes a too many math alphabet error.
% ! LaTeX Error: Too many math alphabets used in version normal.
% Thus we stick on the old SIunits

